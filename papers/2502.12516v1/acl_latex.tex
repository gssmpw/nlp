% This must be in the first 5 lines to tell arXiv to use pdfLaTeX, which is strongly recommended.
\pdfoutput=1
% In particular, the hyperref package requires pdfLaTeX in order to break URLs across lines.

\documentclass[11pt]{article}

% Change "review" to "final" to generate the final (sometimes called camera-ready) version.
% Change to "preprint" to generate a non-anonymous version with page numbers.
\usepackage[preprint]{acl}

% Standard package includes
\usepackage{times}
\usepackage{latexsym}

% For proper rendering and hyphenation of words containing Latin characters (including in bib files)
\usepackage[T1]{fontenc}
% For Vietnamese characters
% \usepackage[T5]{fontenc}
% See https://www.latex-project.org/help/documentation/encguide.pdf for other character sets

% This assumes your files are encoded as UTF8
\usepackage[utf8]{inputenc}

% This is not strictly necessary, and may be commented out,
% but it will improve the layout of the manuscript,
% and will typically save some space.
\usepackage{microtype}

% This is also not strictly necessary, and may be commented out.
% However, it will improve the aesthetics of text in
% the typewriter font.
\usepackage{inconsolata}

%Including images in your LaTeX document requires adding
%additional package(s)
\usepackage{graphicx}

\usepackage{tabularx}
\usepackage{tabularray}

\usepackage[colorinlistoftodos,textsize=tiny,disable]{todonotes}

\usepackage{amsmath}

\usepackage{enumitem}

\usepackage{listings}

\usepackage{tcolorbox}
\usepackage{caption}

% If the title and author information does not fit in the area allocated, uncomment the following
%
%\setlength\titlebox{<dim>}
%
% and set <dim> to something 5cm or larger.

\title{Can LLMs Extract Frame-Semantic Arguments?}

% Author information can be set in various styles:
% For several authors from the same institution:
\author{Jacob Devasier, Rishabh Mediratta, Chengkai Li \\
        University of Texas at Arlington \\ cli@uta.edu}
% if the names do not fit well on one line use
%         Author 1 \\ {\bf Author 2} \\ ... \\ {\bf Author n} \\
% For authors from different institutions:
% \author{Author 1 \\ Address line \\  ... \\ Address line
%         \And  ... \And
%         Author n \\ Address line \\ ... \\ Address line}
% To start a separate ``row'' of authors use \AND, as in
% \author{Author 1 \\ Address line \\  ... \\ Address line
%         \AND
%         Author 2 \\ Address line \\ ... \\ Address line \And
%         Author 3 \\ Address line \\ ... \\ Address line}

% \author{Jacob Devasier \\
%   Affiliation / Address line 1 \\
%   Affiliation / Address line 2 \\
%   Affiliation / Address line 3 \\
%   \texttt{email@domain} \\\And
%   Rishabh Mediratta  \\
%   Affiliation / Address line 1 \\
%   Affiliation / Address line 2 \\
%   Affiliation / Address line 3 \\
%   \texttt{email@domain} \\}

%\author{
%  \textbf{First Author\textsuperscript{1}},
%  \textbf{Second Author\textsuperscript{1,2}},
%  \textbf{Third T. Author\textsuperscript{1}},
%  \textbf{Fourth Author\textsuperscript{1}},
%\\
%  \textbf{Fifth Author\textsuperscript{1,2}},
%  \textbf{Sixth Author\textsuperscript{1}},
%  \textbf{Seventh Author\textsuperscript{1}},
%  \textbf{Eighth Author \textsuperscript{1,2,3,4}},
%\\
%  \textbf{Ninth Author\textsuperscript{1}},
%  \textbf{Tenth Author\textsuperscript{1}},
%  \textbf{Eleventh E. Author\textsuperscript{1,2,3,4,5}},
%  \textbf{Twelfth Author\textsuperscript{1}},
%\\
%  \textbf{Thirteenth Author\textsuperscript{3}},
%  \textbf{Fourteenth F. Author\textsuperscript{2,4}},
%  \textbf{Fifteenth Author\textsuperscript{1}},
%  \textbf{Sixteenth Author\textsuperscript{1}},
%\\
%  \textbf{Seventeenth S. Author\textsuperscript{4,5}},
%  \textbf{Eighteenth Author\textsuperscript{3,4}},
%  \textbf{Nineteenth N. Author\textsuperscript{2,5}},
%  \textbf{Twentieth Author\textsuperscript{1}}
%\\
%\\
%  \textsuperscript{1}Affiliation 1,
%  \textsuperscript{2}Affiliation 2,
%  \textsuperscript{3}Affiliation 3,
%  \textsuperscript{4}Affiliation 4,
%  \textsuperscript{5}Affiliation 5
%\\
%  \small{
%    \textbf{Correspondence:} \href{mailto:email@domain}{email@domain}
%  }
%}

\begin{document}
\maketitle
\begin{abstract}
Frame-semantic parsing is a critical task in natural language understanding, yet the ability of large language models (LLMs) to extract frame-semantic arguments remains underexplored. This paper presents a comprehensive evaluation of LLMs on frame-semantic argument identification, analyzing the impact of input representation formats, model architectures, and generalization to unseen and out-of-domain samples. Our experiments, spanning models from 0.5B to 78B parameters, reveal that JSON-based representations significantly enhance performance, and while larger models generally perform better, smaller models can achieve competitive results through fine-tuning. We also introduce a novel approach to frame identification leveraging predicted frame elements, achieving state-of-the-art performance on ambiguous targets. Despite strong generalization capabilities, our analysis finds that LLMs still struggle with out-of-domain data. 
\end{abstract}

%!TEX root = gcn.tex
\section{Introduction}
Graphs, representing structural data and topology, are widely used across various domains, such as social networks and merchandising transactions.
Graph convolutional networks (GCN)~\cite{iclr/KipfW17} have significantly enhanced model training on these interconnected nodes.
However, these graphs often contain sensitive information that should not be leaked to untrusted parties.
For example, companies may analyze sensitive demographic and behavioral data about users for applications ranging from targeted advertising to personalized medicine.
Given the data-centric nature and analytical power of GCN training, addressing these privacy concerns is imperative.

Secure multi-party computation (MPC)~\cite{crypto/ChaumDG87,crypto/ChenC06,eurocrypt/CiampiRSW22} is a critical tool for privacy-preserving machine learning, enabling mutually distrustful parties to collaboratively train models with privacy protection over inputs and (intermediate) computations.
While research advances (\eg,~\cite{ccs/RatheeRKCGRS20,uss/NgC21,sp21/TanKTW,uss/WatsonWP22,icml/Keller022,ccs/ABY318,folkerts2023redsec}) support secure training on convolutional neural networks (CNNs) efficiently, private GCN training with MPC over graphs remains challenging.

Graph convolutional layers in GCNs involve multiplications with a (normalized) adjacency matrix containing $\numedge$ non-zero values in a $\numnode \times \numnode$ matrix for a graph with $\numnode$ nodes and $\numedge$ edges.
The graphs are typically sparse but large.
One could use the standard Beaver-triple-based protocol to securely perform these sparse matrix multiplications by treating graph convolution as ordinary dense matrix multiplication.
However, this approach incurs $O(\numnode^2)$ communication and memory costs due to computations on irrelevant nodes.
%
Integrating existing cryptographic advances, the initial effort of SecGNN~\cite{tsc/WangZJ23,nips/RanXLWQW23} requires heavy communication or computational overhead.
Recently, CoGNN~\cite{ccs/ZouLSLXX24} optimizes the overhead in terms of  horizontal data partitioning, proposing a semi-honest secure framework.
Research for secure GCN over vertical data  remains nascent.

Current MPC studies, for GCN or not, have primarily targeted settings where participants own different data samples, \ie, horizontally partitioned data~\cite{ccs/ZouLSLXX24}.
MPC specialized for scenarios where parties hold different types of features~\cite{tkde/LiuKZPHYOZY24,icml/CastigliaZ0KBP23,nips/Wang0ZLWL23} is rare.
This paper studies $2$-party secure GCN training for these vertical partition cases, where one party holds private graph topology (\eg, edges) while the other owns private node features.
For instance, LinkedIn holds private social relationships between users, while banks own users' private bank statements.
Such real-world graph structures underpin the relevance of our focus.
To our knowledge, no prior work tackles secure GCN training in this context, which is crucial for cross-silo collaboration.


To realize secure GCN over vertically split data, we tailor MPC protocols for sparse graph convolution, which fundamentally involves sparse (adjacency) matrix multiplication.
Recent studies have begun exploring MPC protocols for sparse matrix multiplication (SMM).
ROOM~\cite{ccs/SchoppmannG0P19}, a seminal work on SMM, requires foreknowledge of sparsity types: whether the input matrices are row-sparse or column-sparse.
Unfortunately, GCN typically trains on graphs with arbitrary sparsity, where nodes have varying degrees and no specific sparsity constraints.
Moreover, the adjacency matrix in GCN often contains a self-loop operation represented by adding the identity matrix, which is neither row- nor column-sparse.
Araki~\etal~\cite{ccs/Araki0OPRT21} avoid this limitation in their scalable, secure graph analysis work, yet it does not cover vertical partition.

% and related primitives
To bridge this gap, we propose a secure sparse matrix multiplication protocol, \osmm, achieving \emph{accurate, efficient, and secure GCN training over vertical data} for the first time.

\subsection{New Techniques for Sparse Matrices}
The cost of evaluating a GCN layer is dominated by SMM in the form of $\adjmat\feamat$, where $\adjmat$ is a sparse adjacency matrix of a (directed) graph $\graph$ and $\feamat$ is a dense matrix of node features.
For unrelated nodes, which often constitute a substantial portion, the element-wise products $0\cdot x$ are always zero.
Our efficient MPC design 
avoids unnecessary secure computation over unrelated nodes by focusing on computing non-zero results while concealing the sparse topology.
We achieve this~by:
1) decomposing the sparse matrix $\adjmat$ into a product of matrices (\S\ref{sec::sgc}), including permutation and binary diagonal matrices, that can \emph{faithfully} represent the original graph topology;
2) devising specialized protocols (\S\ref{sec::smm_protocol}) for efficiently multiplying the structured matrices while hiding sparsity topology.


 
\subsubsection{Sparse Matrix Decomposition}
We decompose adjacency matrix $\adjmat$ of $\graph$ into two bipartite graphs: one represented by sparse matrix $\adjout$, linking the out-degree nodes to edges, the other 
by sparse matrix $\adjin$,
linking edges to in-degree nodes.

%\ie, we decompose $\adjmat$ into $\adjout \adjin$, where $\adjout$ and $\adjin$ are sparse matrices representing these connections.
%linking out-degree nodes to edges and edges to in-degree nodes of $\graph$, respectively.

We then permute the columns of $\adjout$ and the rows of $\adjin$ so that the permuted matrices $\adjout'$ and $\adjin'$ have non-zero positions with \emph{monotonically non-decreasing} row and column indices.
A permutation $\sigma$ is used to preserve the edge topology, leading to an initial decomposition of $\adjmat = \adjout'\sigma \adjin'$.
This is further refined into a sequence of \emph{linear transformations}, 
which can be efficiently computed by our MPC protocols for 
\emph{oblivious permutation}
%($\Pi_{\ssp}$) 
and \emph{oblivious selection-multiplication}.
% ($\Pi_\SM$)
\iffalse
Our approach leverages bipartite graph representation and the monotonicity of non-zero positions to decompose a general sparse matrix into linear transformations, enhancing the efficiency of our MPC protocols.
\fi
Our decomposition approach is not limited to GCNs but also general~SMM 
by 
%simply 
treating them 
as adjacency matrices.
%of a graph.
%Since any sparse matrix can be viewed 

%allowing the same technique to be applied.

 
\subsubsection{New Protocols for Linear Transformations}
\emph{Oblivious permutation} (OP) is a two-party protocol taking a private permutation $\sigma$ and a private vector $\xvec$ from the two parties, respectively, and generating a secret share $\l\sigma \xvec\r$ between them.
Our OP protocol employs correlated randomnesses generated in an input-independent offline phase to mask $\sigma$ and $\xvec$ for secure computations on intermediate results, requiring only $1$ round in the online phase (\cf, $\ge 2$ in previous works~\cite{ccs/AsharovHIKNPTT22, ccs/Araki0OPRT21}).

Another crucial two-party protocol in our work is \emph{oblivious selection-multiplication} (OSM).
It takes a private bit~$s$ from a party and secret share $\l x\r$ of an arithmetic number~$x$ owned by the two parties as input and generates secret share $\l sx\r$.
%between them.
%Like our OP protocol, o
Our $1$-round OSM protocol also uses pre-computed randomnesses to mask $s$ and $x$.
%for secure computations.
Compared to the Beaver-triple-based~\cite{crypto/Beaver91a} and oblivious-transfer (OT)-based approaches~\cite{pkc/Tzeng02}, our protocol saves ${\sim}50\%$ of online communication while having the same offline communication and round complexities.

By decomposing the sparse matrix into linear transformations and applying our specialized protocols, our \osmm protocol
%($\prosmm$) 
reduces the complexity of evaluating $\numnode \times \numnode$ sparse matrices with $\numedge$ non-zero values from $O(\numnode^2)$ to $O(\numedge)$.

%(\S\ref{sec::secgcn})
\subsection{\cgnn: Secure GCN made Efficient}
Supported by our new sparsity techniques, we build \cgnn, 
a two-party computation (2PC) framework for GCN inference and training over vertical
%ly split
data.
Our contributions include:

1) We are the first to explore sparsity over vertically split, secret-shared data in MPC, enabling decompositions of sparse matrices with arbitrary sparsity and isolating computations that can be performed in plaintext without sacrificing privacy.

2) We propose two efficient $2$PC primitives for OP and OSM, both optimally single-round.
Combined with our sparse matrix decomposition approach, our \osmm protocol ($\prosmm$) achieves constant-round communication costs of $O(\numedge)$, reducing memory requirements and avoiding out-of-memory errors for large matrices.
In practice, it saves $99\%+$ communication
%(Table~\ref{table:comm_smm}) 
and reduces ${\sim}72\%$ memory usage over large $(5000\times5000)$ matrices compared with using Beaver triples.
%(Table~\ref{table:mem_smm_sparse}) ${\sim}16\%$-

3) We build an end-to-end secure GCN framework for inference and training over vertically split data, maintaining accuracy on par with plaintext computations.
We will open-source our evaluation code for research and deployment.

To evaluate the performance of $\cgnn$, we conducted extensive experiments over three standard graph datasets (Cora~\cite{aim/SenNBGGE08}, Citeseer~\cite{dl/GilesBL98}, and Pubmed~\cite{ijcnlp/DernoncourtL17}),
reporting communication, memory usage, accuracy, and running time under varying network conditions, along with an ablation study with or without \osmm.
Below, we highlight our key achievements.

\textit{Communication (\S\ref{sec::comm_compare_gcn}).}
$\cgnn$ saves communication by $50$-$80\%$.
(\cf,~CoGNN~\cite{ccs/KotiKPG24}, OblivGNN~\cite{uss/XuL0AYY24}).

\textit{Memory usage (\S\ref{sec::smmmemory}).}
\cgnn alleviates out-of-memory problems of using %the standard 
Beaver-triples~\cite{crypto/Beaver91a} for large datasets.

\textit{Accuracy (\S\ref{sec::acc_compare_gcn}).}
$\cgnn$ achieves inference and training accuracy comparable to plaintext counterparts.
%training accuracy $\{76\%$, $65.1\%$, $75.2\%\}$ comparable to $\{75.7\%$, $65.4\%$, $74.5\%\}$ in plaintext.

{\textit{Computational efficiency (\S\ref{sec::time_net}).}} 
%If the network is worse in bandwidth and better in latency, $\cgnn$ shows more benefits.
$\cgnn$ is faster by $6$-$45\%$ in inference and $28$-$95\%$ in training across various networks and excels in narrow-bandwidth and low-latency~ones.

{\textit{Impact of \osmm (\S\ref{sec:ablation}).}}
Our \osmm protocol shows a $10$-$42\times$ speed-up for $5000\times 5000$ matrices and saves $10$-2$1\%$ memory for ``small'' datasets and up to $90\%$+ for larger ones.

\section{Background}
\label{sec:background}


\subsection{Code Review Automation}
Code review is a widely adopted practice among software developers where a reviewer examines changes submitted in a pull request \cite{hong2022commentfinder, ben2024improving, siow2020core}. If the pull request is not approved, the reviewer must describe the issues or improvements required, providing constructive feedback and identifying potential issues. This step involves review commment generation, which play a key role in the review process by generating review comments for a given code difference. These comments can be descriptive, offering detailed explanations of the issues, or actionable, suggesting specific solutions to address the problems identified \cite{ben2024improving}.


Various approaches have been explored to automate the code review comments process  \cite{tufano2023automating, tufano2024code, yang2024survey}. 
Early efforts centered on knowledge-based systems, which are designed to detect common issues in code. Although these traditional tools provide some support to programmers, they often fall short in addressing complex scenarios encountered during code reviews \cite{dehaerne2022code}. More recently, with advancements in deep learning, researchers have shifted their focus toward using large-language models to enhance the effectiveness of code issue detection and code review comment generation.

\subsection{Knowledge-based Code Review Comments Automation}

Knowledge-based systems (KBS) are software applications designed to emulate human expertise in specific domains by using a collection of rules, logic, and expert knowledge. KBS often consist of facts, rules, an explanation facility, and knowledge acquisition. In the context of software development, these systems are used to analyze the source code, identifying issues such as coding standard violations, bugs, and inefficiencies~\cite{singh2017evaluating, delaitre2015evaluating, ayewah2008using, habchi2018adopting}. By applying a vast set of predefined rules and best practices, they provide automated feedback and recommendations to developers. Tools such as FindBugs \cite{findBugs}, PMD \cite{pmd}, Checkstyle \cite{checkstyle}, and SonarQube \cite{sonarqube} are prominent examples of knowledge-based systems in code analysis, often referred to as static analyzers. These tools have been utilized since the early 1960s, initially to optimize compiler operations, and have since expanded to include debugging tools and software development frameworks \cite{stefanovic2020static, beller2016analyzing}.



\subsection{LLMs-based Code Review Comments Automation}
As the field of machine learning in software engineering evolves, researchers are increasingly leveraging machine learning (ML) and deep learning (DL) techniques to automate the generation of review comments \cite{li2022automating, tufano2022using, balachandran2013reducing, siow2020core, li2022auger, hong2022commentfinder}. Large language models (LLMs) are large-scale Transformer models, which are distinguished by their large number of parameters and extensive pre-training on diverse datasets.  Recently, LLMs have made substantial progress and have been applied across a broad spectrum of domains. Within the software engineering field, LLMs can be categorized into two main types: unified language models and code-specific models, each serving distinct purposes \cite{lu2023llama}.

Code-specific LLMs, such as CodeGen \cite{nijkamp2022codegen}, StarCoder \cite{li2023starcoder} and CodeLlama \cite{roziere2023code} are optimized to excel in code comprehension, code generation, and other programming-related tasks. These specialized models are increasingly utilized in code review activities to detect potential issues, suggest improvements, and automate review comments \cite{yang2024survey, lu2023llama}. 




\subsection{Retrieval-Augmented Generation}
Retrieval-Augmented Generation (RAG) is a general paradigm that enhances LLMs outputs by including relevant information retrieved from external databases into the input prompt \cite{gao2023retrieval}. Traditional LLMs generate responses based solely on the extensive data used in pre-training, which can result in limitations, especially when it comes to domain-specific, time-sensitive, or highly specialized information. RAG addresses these limitations by dynamically retrieving pertinent external knowledge, expanding the model's informational scope and allowing it to generate responses that are more accurate, up-to-date, and contextually relevant \cite{arslan2024business}. 

To build an effective end-to-end RAG pipeline, the system must first establish a comprehensive knowledge base. It requires a retrieval model that captures the semantic meaning of presented data, ensuring relevant information is retrieved. Finally, a capable LLM integrates this retrieved knowledge to generate accurate and coherent results \cite{ibtasham2024towards}.




\subsection{LLM as a Judge Mechanism}

LLM evaluators, often referred to as LLM-as-a-Judge, have gained significant attention due to their ability to align closely with human evaluators' judgments \cite{zhu2023judgelm, shi2024judging}. Their adaptability and scalability make them highly suitable for handling an increasing volume of evaluative tasks. 

Recent studies have shown that certain LLMs, such as Llama-3 70B and GPT-4 Turbo, exhibit strong alignment with human evaluators, making them promising candidates for automated judgment tasks \cite{thakur2024judging}

To enable such evaluations, a proper benchmarking system should be set up with specific components: \emph{prompt design}, which clearly instructs the LLM to evaluate based on a given metric, such as accuracy, relevance, or coherence; \emph{response presentation}, guiding the LLM to present its verdicts in a structured format; and \emph{scoring}, enabling the LLM to assign a score according to a predefined scale \cite{ibtasham2024towards}. Additionally, this evaluation system can be enriched with the ability to explain reasoning behind verdicts, which is a significant advantage of LLM-based evaluation \cite{zheng2023judging}. The LLM can outline the criteria it used to reach its judgment, offering deeper insights into its decision-making process.





\begin{figure*}[ht]
    \centering
    \includegraphics[width=0.95\textwidth]{quantitative_1_2.jpg}
    \vspace{-5mm}
    \caption{Quantitative comparisons for Scenario 1 and Scenario 2. The simulation setup is illustrated on the left. We compare the prediction accuracy using ADE, and the efficiency and safety in resulting robot plans using Detour and Minimum Distance.}
    % AToM achieves the best accuracy and results in improving efficiency while maintaining safety.}
    \vspace{-5mm}
    \label{fig:quantitative_12}
\end{figure*}

\section{Methodology}
\textbf{Problem Definition.} 
Let the state and control of an agent be $\boldsymbol{x} \in \mathds{R}^{n}$ and $\boldsymbol{u} \in \mathds{R}^{m}$, where $n, m$ are state and control dimensions. 
We use subscripts $H, R, HR$ to represent human, robot, and their joint system, and superscript $k$ to represent timesteps.
The dynamics can be described by $\boldsymbol{x}^{k+1} = \Dynamics(\boldsymbol{x}^{k}, \boldsymbol{u}^{k})$.
The trajectories from timestep $i$ to timestep $j$ can be represented by $\boldsymbol{X}^{i:j} = [\boldsymbol{x}^{i}, \dots, \boldsymbol{x}^{j}]$.
In a long-term interaction setting, we assume that the goal positions and environmental obstacles are known as $\boldsymbol{g}_{HR}$ and $\mathcal{O}$, respectively.
At each timestep $t$, given the observed historical trajectories $\boldsymbol{X}_{HR}^{0:t}$ and known information $\boldsymbol{g}_{HR}$ and $\mathcal{O}$, our goal is to predict future human trajectories $\hat{\boldsymbol{X}}_{H}^{t+1:t+T_{f}}$ with prediction horizon $T_{f}$, which the robot can use for collision-free predictive planning. 
For simplicity, we will omit time superscripts when no confusion is aroused. Our method is planner-agnostic and the robot planning algorithm is outside the scope of this work. 
% We omit the time superscripts for predicted human trajectories $\hat{\boldsymbol{X}}_{H}$ in the rest of the paper for simplicity. Our method is planner-agnostic and the robot planning algorithm is outside the scope of this work. 

% Let the state of the human and the robot be $\boldsymbol{x}_{H} \in \mathds{R}^{n_{H}}$ and $\boldsymbol{x}_{R} \in \mathds{R}^{n_{R}}$, where $n_{H}, n_{R}$ are state dimensions. 
% Let their joint state be $\boldsymbol{x}_{HR} \in \mathds{R}^{n_{H} + n_{R}}$.
% Let the control of the human and the robot be $\boldsymbol{u}_{H} \in \mathds{R}^{m_{H}}$ and $\boldsymbol{u}_{R} \in \mathds{R}^{m_{R}}$, where $m_{H}, m_{R}$ are control dimensions. 
% The dynamics can be described by $\dot{\boldsymbol{x}}_{H} = \Dynamics_{H}(\boldsymbol{x}_{H}, \boldsymbol{u}_{H})$ and $\dot{\boldsymbol{x}}_{R} = \Dynamics_{R}(\boldsymbol{x}_{R}, \boldsymbol{u}_{R})$. 
% In a long-term interaction setting, we assume the goal positions are known for both agents as $g_{H}$ and $g_{R}$. 
% The human and robot share the knowledge of all agents' historical trajectories up to the current step $\boldsymbol{Z}_{HR} = [\boldsymbol{x}_{HR}^{0}, \dots, \boldsymbol{x}_{HR}^{t}]$, as well as the environmental obstacles $\mathcal{O}$. 
% At each step $t$, our goal is to predict future human trajectories $\hat{\boldsymbol{X}}_{H} = [\hat{\boldsymbol{x}}_{H}^{t+1}, \dots, \hat{\boldsymbol{x}}_{H}^{t+T_{pred}}]$ with prediction horizon $T_{pred}$, which the robot can use for collision-free predictive planning. Our method is planner-agnostic and the robot planning algorithm is outside the scope of this work. 

\textbf{Overview.} 
We first present the original nested ToM formulation in robot predictive planning. We then propose our reformulated adaptive ToM (AToM) which detaches the human prediction model from the recursive structure. The human model consists of a game-theoretic solver and a parameter update mechanism, which captures dynamic human behaviours in long-term interactions.

\subsection{Original ToM in Robot Planning}
Following prior works mentioned in Sec. \ref{sec:ToM}, we formulate the original ToM in robot predictive planning task and identify the potential issues with this formulation.

The robot plans with a finite horizon $T_{p}$, where at each timestep t the robot optimises its controls $\boldsymbol{U}_{R} = [\boldsymbol{u}_{R}^{t+1}, \dots, \boldsymbol{u}_{R}^{t+T_{p}}]$ to minimize the cumulative cost:
\begin{equation}
\label{eq:robot_planning}
\begin{aligned}
    \min &\; \Cost_{R}(\boldsymbol{x}_{R}^{t+1:t+T_{p}}, \boldsymbol{x}_{R}^{t}, \hat{\boldsymbol{X}}_{H}, \boldsymbol{g}_{R}, \mathcal{O}), \\
    \text{s.t.} \quad &\boldsymbol{x}_{R}^{k+1} = \Dynamics_{R}(\boldsymbol{x}_{R}^{k}, \boldsymbol{u}_{R}^{k}), \\
                      &\Constraint_{R}(\boldsymbol{x}_{R}^{k}, \boldsymbol{u}_{R}^{k}, \boldsymbol{x}_{H}^{k}, \mathcal{O}) \leq 0, \\
                      &k = t+1, \dots, t+T_{p},
\end{aligned}
\end{equation}
where $\Cost_{R}$ is the cost function that captures the robot's performance goals, 
$\Constraint_{R}$ contains constraints such as collision-avoidance, 
and predicted human trajectories $\hat{\boldsymbol{X}}_{H}$ can be obtained from human prediction models such as the Social Force or prediction neural networks, as described in Sec. \ref{seq:traj_prediction}.

% Human predictions $\hat{X}_{H}$ can be obtained from historical observations and other additional information such as human goal and environmental obstacles: ({\color{blue} can introduce the parameter here and remove equation \eqref{eq:humanpred2}})
% \begin{equation}
% \label{eq:human_pred}
%     \tilde{X}_{H} = f_{pred} (X_{HR}, g_{H}, \mathcal{O}),
% \end{equation}
% where $f_{pred}$ is the prediction method such as the Social Force model or neural networks. 

To obtain $\hat{\boldsymbol{X}}_{H}$ using a ToM model, predicted human actions need to be solved from predicted robot trajectories:
\begin{equation}
\label{eq:original_ToM}
\begin{aligned}
    \boldsymbol{\hat{U}}_{H} = \argmin_{\boldsymbol{u}_H^{t+1:t+T_{f}}} &\Cost_{H}(\boldsymbol{x}_{H}^{t+1:t+T_{f}}, \boldsymbol{x}_{HR}^{t}, \hat{\boldsymbol{X}}_{R}, \boldsymbol{g}_{H}, \mathcal{O}), \\
    \text{s.t.} \quad &\boldsymbol{x}_{H}^{k+1} = \Dynamics_{H}(\boldsymbol{x}_{H}^{k}, \boldsymbol{u}_{H}^{k}), \\
                      &\boldsymbol{\hat{x}}_{R}^{k+1} = \Dynamics_{R}(\boldsymbol{\hat{x}}_{R}^{k}, \boldsymbol{u}_{R}^{k}), \\
                      &\Constraint_{H}(\boldsymbol{x}_{H}^{k}, \boldsymbol{u}_{H}^{k}, \boldsymbol{x}_{R}^{k}, \mathcal{O}) \leq 0, \\
                      &k = t+1, \dots, t+T_{f}. \\
\end{aligned}
\end{equation}
% In the original ToM formulation, future human actions are optimised similarly to robot predictive planning:
% \begin{equation}
% \label{eq:original_ToM}
% \begin{aligned}
%     \min_{u_H^{t+1:t+T_{pred}}} &\sum_{i=t+1}^{t+T_{pred}} J_{H}(x_{H}^{i}, g_{H}, \tilde{X}_{R}, \mathcal{O}), \\
%     \text{s.t.} \quad &x_{H}^{i+1} = x_{H}^{i} + f_{H}(x_{H}^{i}, u_{H}^{i}), \\
%                       &x_{R}^{i+1} = x_{R}^{i} + f_{R}(x_{R}^{i}, u_{R}^{i}), \\
%                       &C(x_{H}^{i}, u_{H}^{i}, x_{R}^{i}) \leq 0, \\
%                       &i = t+1, \dots, t+T_{pred},
% \end{aligned}
% \end{equation}
Note that the predicted robot trajectories are obtained from the robot plans $\boldsymbol{U}_{R}$ solved from Problem \eqref{eq:robot_planning}. 
If we substitute Eq. \eqref{eq:original_ToM} and human dynamics $\Dynamics_{H}$ back to the cost in Problem \eqref{eq:robot_planning}, it results in a recursive optimisation problem for the robot. 
Solving such optimisation is tractable for simple settings as discussed in Sec. \ref{sec:ToM}, but complex and computationally expensive for robot planning with continuous and infinite action space. 
Therefore, we propose a reformulation of the original ToM to detach the human prediction model from the nested optimisation problem.

% \begin{figure}
%     \centering
%     \includegraphics[width=\linewidth]{quantitative_3.jpg}
%     \vspace{-8mm}
%     \caption{Quantitative comparison between AToM and SF in Scenario 3. We compare the number of steps the robot takes to reach the goal which reflects the efficiency. We mark the rounds where collisions happens using red crosses.}
%     % AToM enables the robot to pass through the doorway safely and efficiently in all rounds of experiments.}
%     \vspace{-5mm}
%     \label{fig:quantitative_3}
% \end{figure}

\subsection{Reformulation and Adaptive ToM (AToM)}
Instead of using the true robot plan as human-predicted robot actions, we argue that the human maintains an internal model of the navigation problem and they predict the optimal actions for all agents based on a dynamic belief about each agent's behavioural pattern. 
We formulate the human internal model as a multi-player general-sum differential game with finite horizon $T_{f}$. 
In a navigation scenario, the strategy for player $i$ is a control sequence $\boldsymbol{U}_{i} = [\boldsymbol{u}_{i}^{t+1}, \dots, \boldsymbol{u}_{i}^{t+T_{f}}]$. 
Each player seeks to minimize its cost while respecting constraints and dynamics:
\begin{equation}
\label{eq:game_cost}
\begin{aligned}
    \min_{U} \quad &\sum_{k=t+1}^{t+T_{f}} 
    (\boldsymbol{x}_{i}^{k} - \boldsymbol{g})^\top Q (\boldsymbol{x}_{i}^{k} - \boldsymbol{g}) 
    + \boldsymbol{u}_{i}^{k}{}^\top R \boldsymbol{u}_{i}^{k} \\
    +  &w_{s}(\sum_{n \neq i} \max (0, \|\boldsymbol{x}^{k}_{i} - \boldsymbol{x}^{k}_{n}\|_{2} - d_{s}))) \\
    + &w_{o}(\max (0, D(\boldsymbol{x}^{k}_{i}, \mathcal{O}) - d_{o}))\\
    \text{s.t.} \quad &\boldsymbol{x}_{min} \leq \boldsymbol{x}_{i}^{k} \leq \boldsymbol{x}_{max}, \\
                &\boldsymbol{u}_{min} \leq \boldsymbol{u}_{i}^{k} \leq \boldsymbol{u}_{max}, \\
                &\boldsymbol{x}_{i}^{k+1} = \Dynamics_{i}(\boldsymbol{x}_{i}^{k}, \boldsymbol{u}_{i}^{k}), \\
                &k = t+1, \dots, t+T_{f}, \\
\end{aligned}
\end{equation}
% \begin{equation}
% \label{eq:game_cost}
% \begin{aligned}
%     \min_{U} \quad & \Cost_{goal}(\boldsymbol{X}, \boldsymbol{g}, \theta_{1}) + \Cost_{speed}(\boldsymbol{X}, \theta_{2}) \\
%                    &+ \Cost_{obs}(\boldsymbol{X}, \mathcal{O}, \theta_{3}) + \Cost_{social}(\boldsymbol{X}_{joint}, \theta_{4}), \\
%     \text{s.t.} \quad &\Constraint(\boldsymbol{X}_{joint}, \boldsymbol{U}, \mathcal{O}, \theta_{5}) \leq 0, \\
%                 &\boldsymbol{x}^{k+1} = \Dynamics(\boldsymbol{x}^{k}, \boldsymbol{u}^{k}), \\
%                 &k = t+1, \dots, t+T_{f}, \\
% \end{aligned}
% \end{equation}
% where the 4 cost components are $\Cost_{goal}$ for goal reaching, $\Cost_{speed}$ for speed regulation, $\Cost_{obs}$ for obstacle avoidance, and $\Cost_{social}$ for social distancing. 
% $\boldsymbol{\theta} = [\theta_{1}, \theta_{2}, \theta_{3}, \theta_{4}, \theta_{5}]^\top$ are the parameters from the costs and constraints 
% such as the preferred speed in $\Cost_{speed}$, and preferred distance from neighbours in $\Cost_{social}$. 
where $Q \geq 0$ and $R > 0$ are the weights for the state and control, $w_{s}$ and $w_{o}$ are the weights for social and obstacle avoidance, $D$ calculates the closest distance to obstacle $\mathcal{O}$, 
$d_{s}$ and $d_{o}$ are the preferred social and obstacle distances, $\boldsymbol{x}_{min}$, $\boldsymbol{x}_{max}$, $\boldsymbol{u}_{min}$, and $\boldsymbol{u}_{max}$ are the state and control limits.
We represent these behavioural parameters for all players using $\boldsymbol{\theta}$.
% $\boldsymbol{\theta} = [w_{s}, d, w_{o}, d_{o}, \boldsymbol{x}_{min}, \boldsymbol{x}_{max}, \boldsymbol{u}_{min}, \boldsymbol{u}_{max}]^\top$ are the parameters from the costs and constraints such as the weights, preferred distance from other agents and obstacles, and limits for the states and controls.
% These parameters model the behaviours of each player in a game-theoretic setting.

\begin{figure}
    \centering
    \includegraphics[width=\linewidth]{quantitative_3.jpg}
    \vspace{-8mm}
    \caption{Quantitative comparison between AToM and SF in Scenario 3. We compare the number of steps the robot takes to reach the goal which reflects the efficiency. SF leads to collisions in 7 rounds, which we highlighted using red crosses.}
    % We mark the rounds where collisions happen using red crosses. AToM enables the robot to pass through the doorway efficiently and safely in all rounds of experiments.}
    \vspace{-5mm}
    \label{fig:quantitative_3}
\end{figure}


We then find a Nash equilibrium solution $\tilde{\boldsymbol{U}}_{1}, \dots, \tilde{\boldsymbol{U}}_{n}$ for $n$ players with each player's cost defined in Problem \eqref{eq:game_cost}.
At a Nash equilibrium point, no player $i$ can further decrease its cost by unilaterally changing its strategy $\tilde{\boldsymbol{U}}_{i}$ \cite{facchinei2010generalized}. 
Solving this generalized Nash equilibrium problem can be done using existing dynamic game solvers. In this work, we use ILQSolver \cite{fridovich2020efficient} which iteratively solves linear-quadratic approximations of the original differential game in multi-player settings. 
We consider it an algorithmic module $\GameSolver$ parameterized by $\boldsymbol{\theta}$ that takes as input the current joint state of the system, the known goal positions and environmental obstacles, and returns an open-loop joint strategy that satisfies global Nash equilibrium:
\begin{equation}
\label{eq:our_ToM}
    \tilde{\boldsymbol{U}}_{H}, \tilde{\boldsymbol{U}}_{R} = \GameSolver(\boldsymbol{x}_{HR}^{t}, \boldsymbol{g}_{HR}, \mathcal{O}, \hat{\boldsymbol{\theta}}),
\end{equation}
where $\tilde{\boldsymbol{U}}_{H}$ is the predicted human action and equivalent to $\boldsymbol{\hat{U}}_{H}$, 
$\tilde{\boldsymbol{U}}_{R}$ is the human-predicted robot action, 
and $\hat{\boldsymbol{\theta}}$ is the human-predicted behavioural parameters. 
With this reformulation, we can now easily obtain $\hat{\boldsymbol{X}}_{H}$ from Eq. \eqref{eq:our_ToM} and human dynamics $\Dynamics_{H}$, and therefore solve Problem \eqref{eq:robot_planning} without any recursive structure.
% With this reformulation, we can now substitute Eq. \eqref{eq:our_ToM} and human dynamics back to the robot predictive planning problem in Eq. \eqref{eq:robot_planning} without any recursive structure.

Existing game-theoretic robot planners use similar formulations to solve for the optimal robot plans using known cost functions for each agent \cite{tian2022safety, le2021lucidgames, hu2023emergent}. 
A fundamental difference in this work is that our human internal model $\GameSolver$ is parameterized by $\hat{\boldsymbol{\theta}}$, which represents a dynamic human belief instead of the true agent parameters.

After obtaining predicted human trajectories $\hat{\boldsymbol{X}}_{H}$, the downstream robot planner can now perform predictive planning and execute the actual robot plans. 
As we do not assume any leader-follower structure, the human simultaneously performs her/his actions. 
Up to this point, we have obtained four sets of trajectories: 
1) Predicted Human $\hat{\boldsymbol{X}}_{H}$, 
2) Human-predicted Robot $\hat{\boldsymbol{X}}_{R}$, which can be obtained from $\tilde{\boldsymbol{U}}_{R}$
3) Observed Human $\boldsymbol{X}_{H}$, 
4) Observed Robot $\boldsymbol{X}_{R}$. 
These predicted-observed pairs can be used to correct the estimated behavioural parameters $\hat{\boldsymbol{\theta}}$ in the human internal model.

We use Unscented Kalman Filter (UKF) \cite{wan2000unscented} as the update mechanism. We allow $\hat{\boldsymbol{\theta}}$ to evolve using a random walk as the process model $\ProcessModel$. The measurement states are the agents' trajectories and the measurement model $\MeasurementModel$ is therefore the game solver $\GameSolver$ combined with agent dynamics.
\begin{equation}
\begin{aligned}
    \boldsymbol{\theta}^{t+1} &= \ProcessModel(\boldsymbol{\theta}^{t}, \delta^{t}) = \boldsymbol{\theta}^{t} + \delta^{t}, \quad \delta^t \sim \mathcal{N}(0, Q_t), \\
    \boldsymbol{x}^{t+1} &= \MeasurementModel(\boldsymbol{x}^{t}, \boldsymbol{\theta}^{t}) \\
            &= \Dynamics(\boldsymbol{x}^{t}, \GameSolver(\boldsymbol{x}^{t}, \boldsymbol{\theta}^{t})) + \epsilon^{t}, \quad \epsilon^t \sim \mathcal{N}(0, R),
\end{aligned}
\end{equation}
where $Q_{t}$ and $R$ are the covariance for the process model and measurement model.
In this way, our adaptive ToM human model can be adjusted dynamically to improve its prediction accuracy, and to reflect how humans update their beliefs on others. The complete procedure is detailed in Algorithm \ref{alg:our_method}, where $\Sigma$ is the covariance of $\theta$ estimation.
At the observation step, the robot executes the planned action from the predictive planner and the true human motion is observed.

% \vspace{-1mm}
\begin{algorithm}
\caption{Predict-Observe-Update Procedures with AToM}
\label{alg:our_method}
\begin{algorithmic}[1]
\State \textbf{Inputs:} $\boldsymbol{x}_{HR}^{t}, \hat{\boldsymbol{\theta}}^{t}, \Sigma^{t}, \boldsymbol{g}_{HR}, \mathcal{O}$
\State \textbf{for} $k = t, t+1, \dots$ \textbf{do}
    \State \quad $\tilde{\boldsymbol{U}}_{HR} \gets \GameSolver(\boldsymbol{x}_{HR}^{k}, \boldsymbol{g}_{HR}, \mathcal{O}, \hat{\boldsymbol{\theta}}^{k})$ 
    \State \quad $\hat{\boldsymbol{X}}_{HR} \gets \Dynamics_{HR}(\boldsymbol{x}_{HR}^{k}, \tilde{\boldsymbol{U}}_{HR})$\Comment{Predict}
    \State \quad $\boldsymbol{x}_{R}^{k+1} \gets \text{RobotPlanner}(\boldsymbol{x}_{R}^{k}, \hat{\boldsymbol{X}}_{H}, \boldsymbol{g}_{R}, \mathcal{O})$
    % \State \quad $\boldsymbol{x}_{H}^{k+1} \gets \text{Human}$ \Comment{Observation}
    \State \quad observe human state $\boldsymbol{x}_{H}^{k+1}$ \Comment{Observe}
    \State \quad $\hat{\boldsymbol{\theta}}^{k+1}, \Sigma^{k+1} \gets $
    \Statex \quad $\text{UKF}(\boldsymbol{x}_{HR}^{k+1}, \hat{\boldsymbol{x}}_{HR}^{k+1}, \hat{\boldsymbol{\theta}}^{k}, \Sigma^{k}, \ProcessModel, \MeasurementModel)$ \Comment{Update}
\State \textbf{end for}
\end{algorithmic}
\end{algorithm}
% \vspace{-2mm}

% \begin{equation}
% \begin{aligned}
%     \theta^{t+1} &= \ProcessModel(\theta^{t}, \delta^{t}) \\
%                  &= \theta^{t} + \delta^{t}, \quad \delta^t \sim \mathcal{N}(0, Q_t), \\
%     X^{t+1} &= \MeasurementModel(X^{t}, \theta^{t}) \\
%             &= X^{t} + F(X^{t}, GameSolver_{\theta^{t}}(X^{t})).
% \end{aligned}
% \end{equation}




\section{Experiments}
\seclabel{experiments}
Our experiments are designed to test a) the extent to which open loop execution is an issue for precise mobile manipulation tasks, b) how effective are blind proprioceptive correction techniques, c) do object detectors and point trackers perform reliably enough in wrist camera images for reliable control, d) is occlusion by the end-effector an issue and how effectively can it be mitigated through the use of video in-painting models, and e) how does our proposed \name methodology compare to large-scale imitation learning? 


\subsection{Tasks and Experimental Setup}
We work with the Stretch RE2 robot. Stretch RE2 is a commodity mobile manipulator with a 5DOF arm mounted on top of a non-holomonic base. We upgrade the robot to use the Dex Wrist 3, which has an eye-in-hand RGB-D camera (Intel D405). 
We consider 3 task families for a total
of 6 different tasks: a) holding a knob to pull open a cabinet or drawer, b) holding a
handle to pull open a cabinet, and c) pushing on objects (light buttons, books
in a book shelf, and light switches). Our focus is on generalization. {\it
Therefore, we exclusively test on previously unseen instances, not used during
development in any way.} 
\figref{tasks} shows the instances that we test on. 

All tasks involve some precise manipulation, followed by execution of a motion
primitive. {\bf For the pushing tasks}, the precise motion is to get the
end-effector exactly at the indicated point and the motion primitive is to push
in the direction perpendicular to the surface and retract the end-effector 
upon contact. The robot is positioned such
that the target position is within the field of view of the wrist camera. A user
selects the point of pushing via a mouse click on the wrist camera image. The
goal is to push at the indicated location. Success is determined by whether the
push results in the desired outcome (light turns on / off or book gets pushed in). 
The original rubber gripper bends upon contact, we use a rigid known tool
that sticks out a bit. We take the geometry of the tool into account while servoing.

{\bf For the opening articulated object tasks}, the precise manipulation is grasping the
knob / handle, while the motion primitive is the whole-body motion that opens
the cupboard. Computing and executing this full body motion is difficult. We
adopt the modular approach to opening articulated objects (MOSART) from Gupta \etal~\cite{gupta2024opening} and invoke it
after the gripper has been placed around the knob / handle. The whole tasks 
starts out with the robot about 1.5m way from the target object, with the 
target object in view
from robot's head mounted camera. We use MOSART to compute articulation
parameters and convey the robot to a pre-grasp
location with the target handle in view of the wrist camera. At this point,
\name (or baseline) is used to center the gripper around the knob / handle, 
before resuming MOSART: extending the gripper till contact, close the gripper, and play rest of the predicted motion plan. Success is 
determined by whether the cabinet opens by more than $60^\circ$
or the drawer is pulled out by more than $24cm$, similar to the criteria used in \cite{gupta2024opening}.


For the precise manipulation part, all baselines consume the current and
previous RGB-D images from the wrist camera and output full body motor
commands.

% % Please add the following required packages to your document preamble:
% % \usepackage{graphicx}
% \begin{table*}[!ht]
% \centering
% \caption{}
% \label{tab:my-table}
% \resizebox{\textwidth}{!}{%
% \begin{tabular}{lcccccc}
% \toprule
%  & \multicolumn{2}{c}{ours} & \multicolumn{2}{c}{Gurobi} & \multicolumn{2}{c}{MOSEK} \\
%  & \multicolumn{1}{l}{time (s)} & \multicolumn{1}{l}{optimality gap (\%)} & \multicolumn{1}{l}{time (s)} & \multicolumn{1}{l}{optimality gap (\%)} & \multicolumn{1}{l}{time (s)} & \multicolumn{1}{l}{optimality gap (\%)} \\ \hline
% \begin{tabular}[c]{@{}l@{}}Linear Regression\\ Synthetic \\ (n=16000, p=16000)\end{tabular} & 57 & 0.0 & 3351 & - & 2148 & - \\ \hline
% \begin{tabular}[c]{@{}l@{}}Linear Regression\\ Cancer Drug Response\\ (n=822, p=2300)\end{tabular} & 47 & 0.0 & 1800 & 0.31 & 212 & 0.0 \\ \hline
% \begin{tabular}[c]{@{}l@{}}Logistic Regression\\ Synthetic\\ (n=16000, p=16000)\end{tabular} & 271 & 0.0 & N/A & N/A & 1800 & - \\ \hline
% \begin{tabular}[c]{@{}l@{}}Logistic Regression\\ Dorothea\\ (n=1150, p=91598)\end{tabular} & 62 & 0.0 & N/A & N/A & 600 & 0.0 \\
% \bottomrule
% \end{tabular}%
% }
% \end{table*}

% Please add the following required packages to your document preamble:
% \usepackage{multirow}
% \usepackage{graphicx}
\begin{table*}[]
\centering
\caption{Certifying optimality on large-scale and real-world datasets.}
\vspace{2mm}
\label{tab:my-table}
\resizebox{\textwidth}{!}{%
\begin{tabular}{llcccccc}
\toprule
 &  & \multicolumn{2}{c}{ours} & \multicolumn{2}{c}{Gurobi} & \multicolumn{2}{c}{MOSEK} \\
 &  & time (s) & opt. gap (\%) & time (s) & opt. gap (\%) & time (s) & opt. gap (\%) \\ \hline
\multirow{2}{*}{Linear Regression} & \begin{tabular}[c]{@{}l@{}}synthetic ($k=10, M=2$)\\ (n=16k, p=16k, seed=0)\end{tabular} & 79 & 0.0 & 1800 & - & 1915 & - \\ \cline{2-8}
 & \begin{tabular}[c]{@{}l@{}}Cancer Drug Response ($k=5, M=5$)\\ (n=822, p=2300)\end{tabular} & 41 & 0.0 & 1800 & 0.89 & 188 & 0.0 \\ \hline
\multirow{2}{*}{Logistic Regression} & \begin{tabular}[c]{@{}l@{}}Synthetic ($k=10, M=2$)\\ (n=16k, p=16k, seed=0)\end{tabular} & 626 & 0.0 & N/A & N/A & 2446 & - \\ \cline{2-8}
 & \begin{tabular}[c]{@{}l@{}}DOROTHEA ($k=15, M=2$)\\ (n=1150, p=91598)\end{tabular} & 91 & 0.0 & N/A & N/A & 634 & 0.0 \\
 \bottomrule
\end{tabular}%
}
% \vspace{-3mm}
\end{table*}

\begin{figure*}
\insertW{1.0}{figures/figure_6_cropped_brighten.pdf}
\caption{{\bf Comparison of \name with the open loop (eye-in-hand) baseline} for opening a cabinet with a knob. Slight errors in getting to the target cause the end-effector to slip off, leading to failure for the baseline, where as our method is able to successfully complete the task.}
\figlabel{rollout}
\end{figure*}

\begin{table}
\setlength{\tabcolsep}{8pt}
  \centering
  \resizebox{\linewidth}{!}{
  \begin{tabular}{lcccg}
  \toprule
                              & \multicolumn{2}{c}{\bf Knobs} & \bf Handle & \bf \multirow{2}{*}{\bf Total} \\
                              \cmidrule(lr){2-3} \cmidrule(lr){4-4}
                              & \bf Cabinets & \bf Drawer & \bf Cabinets & \\
  \midrule
  RUM~\cite{etukuru2024robot}  & 0/3    & 1/4         & 1/3         & 2/10 \\
  \name (Ours) & 2/3    & 2/4         & 3/3     &  7/10 \\
  \bottomrule
  \end{tabular}}
  \caption{Comparison of \name \vs RUM~\cite{etukuru2024robot}, a recent large-scale end-to-end imitation learning method trained on 1200 demos for opening cabinets and 525 demos for opening drawers across 40 different environments. Our evaluation spans objects from three environments across two buildings.}
  \tablelabel{rum}
\end{table}

\subsection{Baselines}
We compare against three other methods for the precise manipulation part of
these tasks. 
\subsubsection{Open Loop (Eye-in-Hand)} To assess the precision requirements of
the tasks and to set it in context with the manipulation capabilities of the
robot platform, this baseline uses open loop execution starting from estimates
for the 3D target position from the first wrist camera image.
\subsubsection{MOSART~\cite{gupta2024opening}}
The recent modular system for opening cabinets and drawers~\cite{gupta2024opening}
reports impressive performance with open-loop control (using the head camera from 1.5m away), combined with proprioception-based feedback to 
compensate for errors in perception and control when interacting with handles. 
We test if such correction is also sufficient for interacting with knobs. Note 
that such correction is not possible for the smaller buttons and pliable books.

\subsubsection{\name (no inpainting)} To understand how much of an issue
occlusion due to the end-effector is during manipulation, we ablate the use of
inpainting. %

\subsubsection{Robot Utility Models (RUM)~\cite{etukuru2024robot}}
For the opening articulated object tasks, we also compare to Robot Utility Models (RUM), 
a closed-loop imitation learning method recently proposed by Etukuru et al. \cite{etukuru2024robot}.
RUM is trained on a substantial dataset comprising expert demonstrations, including 
1,200 instances of cabinet opening and 525 of drawer opening, gathered from roughly 
40 different environments.
This dataset stands as the most extensive imitation 
learning dataset for articulated object manipulation to date, establishing RUM as a 
strong baseline for our evaluation.

Similar to our method, we use MOSART to compute articulation
parameters and convey the robot to a pre-grasp location
with the target handle in view of the wrist camera.
One of the assumptions of RUM is a good view of the handle.
To benefit RUM, we try out three different heights of the wrist camera,
and \textit{report the best result for RUM.}

\begin{figure*}
\insertW{1.0}{figures/figure_9_cropped_brighten.pdf}
\caption{{\bf \name \vs open loop (eye-in-hand) baseline for pushing on user-clicked points}. Slight errors in getting to the target cause failure, where as \name successfully turns the lights off. Note the quality of CoTracker's track ({\color{blue} blue dot}).}
\figlabel{rollout_v2}
\end{figure*}

\begin{figure*}
\insertW{1.0}{figures/figure_5_v2_cropped_brighten.pdf}
\caption{{\bf Comparison of \name with and without inpainting}. Erroneous detection without inpainting causes execution to fail, where as with inpainting the target is correctly detected leading to a successful grasp and a successful execution.}
\figlabel{rollouts2}
\end{figure*}


\subsection{Results}
\tableref{results} presents results from our experiments. 
Our training-free approach \name successfully 
solves over 85\% of task instances that we test on.
As noted, all these
tests were conducted on unseen object instances in unseen
environments that were not used for development in any way. We discuss our key
experimental findings below.

\subsubsection{Closing the loop is necessary for these precise tasks} 
While the proprioception-based strategies proposed in MOSART~\cite{gupta2024opening}
work out for handles, they are inadequate for targets like knobs and just
don't work for tasks like pushing buttons. Using estimates from the wrist
camera is better, but open loop execution still fails for knobs and pushing
buttons. 

\subsubsection{Vision models work reasonably well even on wrist camera images}
Inpainting works well on wrist camera images (see \figref{occlusion} and \figref{inpainting}).
Closing the loop using feedback from vision detectors and point trackers on
wrist camera images also work well, particularly when we use in-painted images.
See some examples detections and point tracks in \figref{rollout} and \figref{rollout_v2}. 
Detic~\cite{zhou2022detecting} was able to reliably detect the knobs and
handles and CoTracker~\cite{karaev2023cotracker} was able to successfully track
the point of interaction letting us solve 24/28 task instances.

\subsubsection{Erroneous detections without inpainting hamper performance on 
handles and our end-effector out-painting strategy effectively mitigates it} 
As shown in \figref{rollouts2}, presence of the end-effector caused the object
detector to miss fire leading to failed execution. Our out painting approach
mitigates this issue leading to a higher success rate than the 
approach without out-painting. Interestingly, CoTracker~\cite{karaev2023cotracker} is quite robust
to occlusion (possibly because it tracks multiple points) and doesn't benefit
from in-painting. 


\subsubsection{Closed-loop imitation learning struggles on novel objects}
As presented in \tableref{rum}, \name significantly outperforms RUM in a paired evaluation on unseen objects across three novel environments. A common failure mode of RUM is its inability to grasp the object's handle, even when it approaches it closely.
Another failure mode we observe is RUM misidentifying keyholes or cabinet edges as handles, also resulting in failed grasp attempts.
These result demonstrate that a modular approach that leverages the broad generalization capabilities of vision foundation models is able to generalize much better than an end-to-end imitation learning approach trained on 1000+ demonstrations, which must learn all aspects of the task from scratch.



% \section{Discussion}

% \begin{figure}
  \centering
  \includegraphics[width=\linewidth]{figures/per_frame_boxplot.png}
  
  \caption{\label{fig:frame-boxplot} Comparison of the distribution of F1 scores across all frames for each model.}
\end{figure}
% \subsection{Model Performance}

% \subsubsection{Out-of-Domain Performance}


% \begin{table}
    \centering
    \begin{tabularx}{\linewidth}{Xcccc}
        \hline
        \textbf{Model} & \textbf{All} & \textbf{Amb} \\ 
        \hline
        % Qwen 2.5-7B     & 0.755 & 0.665 & 0.707 & 0.547 \\ % no candidates @ fe
        % Qwen 2.5-7B     & 0.668 & 0.665 & 0.666 & 0.500 \\ % cand @ fe 
        % Phi-4           & 0.798 & 0.717 & 0.756 & 0.607 \\ % no candidates @ fe
        % Phi-4           & 0.719 & 0.717 & 0.718 & 0.560 \\ % cand @ fe
        % Qwen 2.5-7B     & 91.76 & 90.95 \\ % cand @ fe 
        Phi-4                           & 0.375 & 0.262 \\ % Not finetuned
        % $\text{Phi-4}_{cand}$ w/o LF    & 0.927 & 0.918 \\ % Finetuned on candidates
        $\text{Phi-4}_{cand}$ w/o LF    & 0.882 & \textbf{0.862} \\ % Finetuned on candidates
        $\text{Phi-4}_{cand}$ w/ LF     & 0.894 & \textbf{0.862} \\ % Finetuned on candidates
        % $\text{Phi-4}_{cand}$ w/ LF     & \textbf{0.931} & \textbf{0.918} \\ % Finetuned on candidates
        \hline
        KAF-SPA             & 0.912 & 0.776 \\
        KGFI                & 0.924 & 0.844 \\
        CoFFTEA             & \textbf{0.926} & 0.850 \\
        \hline
    \end{tabularx}
    \caption{Results on frame identification using frame element predictions.}
    \label{tab:candidate_frame}
\end{table}
% \subsection{Frame Identification}
% Previous work~\cite{devasier-etal-2024-robust} explored the possibility of filtering candidate targets produced by matching potential lexical units using a frame identification model. To build upon this idea towards a single-step frame-semantic parsing method, we explore the potential of frame elements being used to filter out candidate targets. In this approach, no ground-truth frame inputs are given. This also removes the bias from the model assuming the input always has at least one frame element.

% We represent the LLM instructions using the JSON-exist representation as it performed the best in Table~\ref{tab:representation_performance}. We used Phi-4 for this experiment as it had a very high performance-to-size ratio, as shown in Table~\ref{tab:candidate_frame}. \todo{should run this on qwen-72b} We found that directly using the model performed poorly, likely due to bias in the model learning that each input contains the given frame. To address this, we fine-tuned the LLM using candidates from the training set and found a significant improvement in performance. \todo{add candidates examples}

% Performance on par with CoFFTEA, the previous-best frame identification system.
% Maybe qwen 72b will perform better.
\section{Conclusion }
This paper introduces the Latent Radiance Field (LRF), which to our knowledge, is the first work to construct radiance field representations directly in the 2D latent space for 3D reconstruction. We present a novel framework for incorporating 3D awareness into 2D representation learning, featuring a correspondence-aware autoencoding method and a VAE-Radiance Field (VAE-RF) alignment strategy to bridge the domain gap between the 2D latent space and the natural 3D space, thereby significantly enhancing the visual quality of our LRF.
Future work will focus on incorporating our method with more compact 3D representations, efficient NVS, few-shot NVS in latent space, as well as exploring its application with potential 3D latent diffusion models.


\vspace{0.2cm}
\section*{Limitations}
Several methodological constraints impacted the scope and comprehensiveness of our analysis. Due to the substantial computational costs associated with fine-tuning large language models, we were unable to explore fine-tuning on certain high-performing models like GPT-4o and GPT-4. These models may have achieved even stronger results than those demonstrated in our current analysis.

Our experimental design relied on sequential parameter optimization to manage computational requirements. While this approach was practical, it introduces the possibility that certain combinations of parameters could yield unexpected results. For instance, XML representations might potentially outperform JSON embeddings when paired with 14B parameter models or applied to frame identification tasks. However, exploring these combinations was beyond the computational resources available for this study.

The scope of our research was limited to the English FrameNet dataset. As a result, our findings may not generalize to other languages or semantic frameworks. Cross-lingual validation would be necessary to establish the broader applicability of our approaches.

In the context of frame identification for ambiguous targets, our current method of handling multiple frames with predicted frame elements requires refinement. The randomized prediction approach can lead to inconsistent outputs. Additionally, our implementation used a fixed random seed of 0 for reproducibility, but we did not explore the potential impact of different random seeds on accuracy metrics.

Finally, our benchmark correlation analysis considered only model size as a confounding variable. This simplified approach may not account for other significant factors that could influence the relationship between benchmark performance and frame-semantic parsing capabilities. A more comprehensive analysis of confounding variables would provide deeper insights into these relationships.

% Bibliography entries for the entire Anthology, followed by custom entries
%\bibliography{anthology,custom}
% Custom bibliography entries only
\bibliography{custom,anthology}



\appendix

\subsection{Lloyd-Max Algorithm}
\label{subsec:Lloyd-Max}
For a given quantization bitwidth $B$ and an operand $\bm{X}$, the Lloyd-Max algorithm finds $2^B$ quantization levels $\{\hat{x}_i\}_{i=1}^{2^B}$ such that quantizing $\bm{X}$ by rounding each scalar in $\bm{X}$ to the nearest quantization level minimizes the quantization MSE. 

The algorithm starts with an initial guess of quantization levels and then iteratively computes quantization thresholds $\{\tau_i\}_{i=1}^{2^B-1}$ and updates quantization levels $\{\hat{x}_i\}_{i=1}^{2^B}$. Specifically, at iteration $n$, thresholds are set to the midpoints of the previous iteration's levels:
\begin{align*}
    \tau_i^{(n)}=\frac{\hat{x}_i^{(n-1)}+\hat{x}_{i+1}^{(n-1)}}2 \text{ for } i=1\ldots 2^B-1
\end{align*}
Subsequently, the quantization levels are re-computed as conditional means of the data regions defined by the new thresholds:
\begin{align*}
    \hat{x}_i^{(n)}=\mathbb{E}\left[ \bm{X} \big| \bm{X}\in [\tau_{i-1}^{(n)},\tau_i^{(n)}] \right] \text{ for } i=1\ldots 2^B
\end{align*}
where to satisfy boundary conditions we have $\tau_0=-\infty$ and $\tau_{2^B}=\infty$. The algorithm iterates the above steps until convergence.

Figure \ref{fig:lm_quant} compares the quantization levels of a $7$-bit floating point (E3M3) quantizer (left) to a $7$-bit Lloyd-Max quantizer (right) when quantizing a layer of weights from the GPT3-126M model at a per-tensor granularity. As shown, the Lloyd-Max quantizer achieves substantially lower quantization MSE. Further, Table \ref{tab:FP7_vs_LM7} shows the superior perplexity achieved by Lloyd-Max quantizers for bitwidths of $7$, $6$ and $5$. The difference between the quantizers is clear at 5 bits, where per-tensor FP quantization incurs a drastic and unacceptable increase in perplexity, while Lloyd-Max quantization incurs a much smaller increase. Nevertheless, we note that even the optimal Lloyd-Max quantizer incurs a notable ($\sim 1.5$) increase in perplexity due to the coarse granularity of quantization. 

\begin{figure}[h]
  \centering
  \includegraphics[width=0.7\linewidth]{sections/figures/LM7_FP7.pdf}
  \caption{\small Quantization levels and the corresponding quantization MSE of Floating Point (left) vs Lloyd-Max (right) Quantizers for a layer of weights in the GPT3-126M model.}
  \label{fig:lm_quant}
\end{figure}

\begin{table}[h]\scriptsize
\begin{center}
\caption{\label{tab:FP7_vs_LM7} \small Comparing perplexity (lower is better) achieved by floating point quantizers and Lloyd-Max quantizers on a GPT3-126M model for the Wikitext-103 dataset.}
\begin{tabular}{c|cc|c}
\hline
 \multirow{2}{*}{\textbf{Bitwidth}} & \multicolumn{2}{|c|}{\textbf{Floating-Point Quantizer}} & \textbf{Lloyd-Max Quantizer} \\
 & Best Format & Wikitext-103 Perplexity & Wikitext-103 Perplexity \\
\hline
7 & E3M3 & 18.32 & 18.27 \\
6 & E3M2 & 19.07 & 18.51 \\
5 & E4M0 & 43.89 & 19.71 \\
\hline
\end{tabular}
\end{center}
\end{table}

\subsection{Proof of Local Optimality of LO-BCQ}
\label{subsec:lobcq_opt_proof}
For a given block $\bm{b}_j$, the quantization MSE during LO-BCQ can be empirically evaluated as $\frac{1}{L_b}\lVert \bm{b}_j- \bm{\hat{b}}_j\rVert^2_2$ where $\bm{\hat{b}}_j$ is computed from equation (\ref{eq:clustered_quantization_definition}) as $C_{f(\bm{b}_j)}(\bm{b}_j)$. Further, for a given block cluster $\mathcal{B}_i$, we compute the quantization MSE as $\frac{1}{|\mathcal{B}_{i}|}\sum_{\bm{b} \in \mathcal{B}_{i}} \frac{1}{L_b}\lVert \bm{b}- C_i^{(n)}(\bm{b})\rVert^2_2$. Therefore, at the end of iteration $n$, we evaluate the overall quantization MSE $J^{(n)}$ for a given operand $\bm{X}$ composed of $N_c$ block clusters as:
\begin{align*}
    \label{eq:mse_iter_n}
    J^{(n)} = \frac{1}{N_c} \sum_{i=1}^{N_c} \frac{1}{|\mathcal{B}_{i}^{(n)}|}\sum_{\bm{v} \in \mathcal{B}_{i}^{(n)}} \frac{1}{L_b}\lVert \bm{b}- B_i^{(n)}(\bm{b})\rVert^2_2
\end{align*}

At the end of iteration $n$, the codebooks are updated from $\mathcal{C}^{(n-1)}$ to $\mathcal{C}^{(n)}$. However, the mapping of a given vector $\bm{b}_j$ to quantizers $\mathcal{C}^{(n)}$ remains as  $f^{(n)}(\bm{b}_j)$. At the next iteration, during the vector clustering step, $f^{(n+1)}(\bm{b}_j)$ finds new mapping of $\bm{b}_j$ to updated codebooks $\mathcal{C}^{(n)}$ such that the quantization MSE over the candidate codebooks is minimized. Therefore, we obtain the following result for $\bm{b}_j$:
\begin{align*}
\frac{1}{L_b}\lVert \bm{b}_j - C_{f^{(n+1)}(\bm{b}_j)}^{(n)}(\bm{b}_j)\rVert^2_2 \le \frac{1}{L_b}\lVert \bm{b}_j - C_{f^{(n)}(\bm{b}_j)}^{(n)}(\bm{b}_j)\rVert^2_2
\end{align*}

That is, quantizing $\bm{b}_j$ at the end of the block clustering step of iteration $n+1$ results in lower quantization MSE compared to quantizing at the end of iteration $n$. Since this is true for all $\bm{b} \in \bm{X}$, we assert the following:
\begin{equation}
\begin{split}
\label{eq:mse_ineq_1}
    \tilde{J}^{(n+1)} &= \frac{1}{N_c} \sum_{i=1}^{N_c} \frac{1}{|\mathcal{B}_{i}^{(n+1)}|}\sum_{\bm{b} \in \mathcal{B}_{i}^{(n+1)}} \frac{1}{L_b}\lVert \bm{b} - C_i^{(n)}(b)\rVert^2_2 \le J^{(n)}
\end{split}
\end{equation}
where $\tilde{J}^{(n+1)}$ is the the quantization MSE after the vector clustering step at iteration $n+1$.

Next, during the codebook update step (\ref{eq:quantizers_update}) at iteration $n+1$, the per-cluster codebooks $\mathcal{C}^{(n)}$ are updated to $\mathcal{C}^{(n+1)}$ by invoking the Lloyd-Max algorithm \citep{Lloyd}. We know that for any given value distribution, the Lloyd-Max algorithm minimizes the quantization MSE. Therefore, for a given vector cluster $\mathcal{B}_i$ we obtain the following result:

\begin{equation}
    \frac{1}{|\mathcal{B}_{i}^{(n+1)}|}\sum_{\bm{b} \in \mathcal{B}_{i}^{(n+1)}} \frac{1}{L_b}\lVert \bm{b}- C_i^{(n+1)}(\bm{b})\rVert^2_2 \le \frac{1}{|\mathcal{B}_{i}^{(n+1)}|}\sum_{\bm{b} \in \mathcal{B}_{i}^{(n+1)}} \frac{1}{L_b}\lVert \bm{b}- C_i^{(n)}(\bm{b})\rVert^2_2
\end{equation}

The above equation states that quantizing the given block cluster $\mathcal{B}_i$ after updating the associated codebook from $C_i^{(n)}$ to $C_i^{(n+1)}$ results in lower quantization MSE. Since this is true for all the block clusters, we derive the following result: 
\begin{equation}
\begin{split}
\label{eq:mse_ineq_2}
     J^{(n+1)} &= \frac{1}{N_c} \sum_{i=1}^{N_c} \frac{1}{|\mathcal{B}_{i}^{(n+1)}|}\sum_{\bm{b} \in \mathcal{B}_{i}^{(n+1)}} \frac{1}{L_b}\lVert \bm{b}- C_i^{(n+1)}(\bm{b})\rVert^2_2  \le \tilde{J}^{(n+1)}   
\end{split}
\end{equation}

Following (\ref{eq:mse_ineq_1}) and (\ref{eq:mse_ineq_2}), we find that the quantization MSE is non-increasing for each iteration, that is, $J^{(1)} \ge J^{(2)} \ge J^{(3)} \ge \ldots \ge J^{(M)}$ where $M$ is the maximum number of iterations. 
%Therefore, we can say that if the algorithm converges, then it must be that it has converged to a local minimum. 
\hfill $\blacksquare$


\begin{figure}
    \begin{center}
    \includegraphics[width=0.5\textwidth]{sections//figures/mse_vs_iter.pdf}
    \end{center}
    \caption{\small NMSE vs iterations during LO-BCQ compared to other block quantization proposals}
    \label{fig:nmse_vs_iter}
\end{figure}

Figure \ref{fig:nmse_vs_iter} shows the empirical convergence of LO-BCQ across several block lengths and number of codebooks. Also, the MSE achieved by LO-BCQ is compared to baselines such as MXFP and VSQ. As shown, LO-BCQ converges to a lower MSE than the baselines. Further, we achieve better convergence for larger number of codebooks ($N_c$) and for a smaller block length ($L_b$), both of which increase the bitwidth of BCQ (see Eq \ref{eq:bitwidth_bcq}).


\subsection{Additional Accuracy Results}
%Table \ref{tab:lobcq_config} lists the various LOBCQ configurations and their corresponding bitwidths.
\begin{table}
\setlength{\tabcolsep}{4.75pt}
\begin{center}
\caption{\label{tab:lobcq_config} Various LO-BCQ configurations and their bitwidths.}
\begin{tabular}{|c||c|c|c|c||c|c||c|} 
\hline
 & \multicolumn{4}{|c||}{$L_b=8$} & \multicolumn{2}{|c||}{$L_b=4$} & $L_b=2$ \\
 \hline
 \backslashbox{$L_A$\kern-1em}{\kern-1em$N_c$} & 2 & 4 & 8 & 16 & 2 & 4 & 2 \\
 \hline
 64 & 4.25 & 4.375 & 4.5 & 4.625 & 4.375 & 4.625 & 4.625\\
 \hline
 32 & 4.375 & 4.5 & 4.625& 4.75 & 4.5 & 4.75 & 4.75 \\
 \hline
 16 & 4.625 & 4.75& 4.875 & 5 & 4.75 & 5 & 5 \\
 \hline
\end{tabular}
\end{center}
\end{table}

%\subsection{Perplexity achieved by various LO-BCQ configurations on Wikitext-103 dataset}

\begin{table} \centering
\begin{tabular}{|c||c|c|c|c||c|c||c|} 
\hline
 $L_b \rightarrow$& \multicolumn{4}{c||}{8} & \multicolumn{2}{c||}{4} & 2\\
 \hline
 \backslashbox{$L_A$\kern-1em}{\kern-1em$N_c$} & 2 & 4 & 8 & 16 & 2 & 4 & 2  \\
 %$N_c \rightarrow$ & 2 & 4 & 8 & 16 & 2 & 4 & 2 \\
 \hline
 \hline
 \multicolumn{8}{c}{GPT3-1.3B (FP32 PPL = 9.98)} \\ 
 \hline
 \hline
 64 & 10.40 & 10.23 & 10.17 & 10.15 &  10.28 & 10.18 & 10.19 \\
 \hline
 32 & 10.25 & 10.20 & 10.15 & 10.12 &  10.23 & 10.17 & 10.17 \\
 \hline
 16 & 10.22 & 10.16 & 10.10 & 10.09 &  10.21 & 10.14 & 10.16 \\
 \hline
  \hline
 \multicolumn{8}{c}{GPT3-8B (FP32 PPL = 7.38)} \\ 
 \hline
 \hline
 64 & 7.61 & 7.52 & 7.48 &  7.47 &  7.55 &  7.49 & 7.50 \\
 \hline
 32 & 7.52 & 7.50 & 7.46 &  7.45 &  7.52 &  7.48 & 7.48  \\
 \hline
 16 & 7.51 & 7.48 & 7.44 &  7.44 &  7.51 &  7.49 & 7.47  \\
 \hline
\end{tabular}
\caption{\label{tab:ppl_gpt3_abalation} Wikitext-103 perplexity across GPT3-1.3B and 8B models.}
\end{table}

\begin{table} \centering
\begin{tabular}{|c||c|c|c|c||} 
\hline
 $L_b \rightarrow$& \multicolumn{4}{c||}{8}\\
 \hline
 \backslashbox{$L_A$\kern-1em}{\kern-1em$N_c$} & 2 & 4 & 8 & 16 \\
 %$N_c \rightarrow$ & 2 & 4 & 8 & 16 & 2 & 4 & 2 \\
 \hline
 \hline
 \multicolumn{5}{|c|}{Llama2-7B (FP32 PPL = 5.06)} \\ 
 \hline
 \hline
 64 & 5.31 & 5.26 & 5.19 & 5.18  \\
 \hline
 32 & 5.23 & 5.25 & 5.18 & 5.15  \\
 \hline
 16 & 5.23 & 5.19 & 5.16 & 5.14  \\
 \hline
 \multicolumn{5}{|c|}{Nemotron4-15B (FP32 PPL = 5.87)} \\ 
 \hline
 \hline
 64  & 6.3 & 6.20 & 6.13 & 6.08  \\
 \hline
 32  & 6.24 & 6.12 & 6.07 & 6.03  \\
 \hline
 16  & 6.12 & 6.14 & 6.04 & 6.02  \\
 \hline
 \multicolumn{5}{|c|}{Nemotron4-340B (FP32 PPL = 3.48)} \\ 
 \hline
 \hline
 64 & 3.67 & 3.62 & 3.60 & 3.59 \\
 \hline
 32 & 3.63 & 3.61 & 3.59 & 3.56 \\
 \hline
 16 & 3.61 & 3.58 & 3.57 & 3.55 \\
 \hline
\end{tabular}
\caption{\label{tab:ppl_llama7B_nemo15B} Wikitext-103 perplexity compared to FP32 baseline in Llama2-7B and Nemotron4-15B, 340B models}
\end{table}

%\subsection{Perplexity achieved by various LO-BCQ configurations on MMLU dataset}


\begin{table} \centering
\begin{tabular}{|c||c|c|c|c||c|c|c|c|} 
\hline
 $L_b \rightarrow$& \multicolumn{4}{c||}{8} & \multicolumn{4}{c||}{8}\\
 \hline
 \backslashbox{$L_A$\kern-1em}{\kern-1em$N_c$} & 2 & 4 & 8 & 16 & 2 & 4 & 8 & 16  \\
 %$N_c \rightarrow$ & 2 & 4 & 8 & 16 & 2 & 4 & 2 \\
 \hline
 \hline
 \multicolumn{5}{|c|}{Llama2-7B (FP32 Accuracy = 45.8\%)} & \multicolumn{4}{|c|}{Llama2-70B (FP32 Accuracy = 69.12\%)} \\ 
 \hline
 \hline
 64 & 43.9 & 43.4 & 43.9 & 44.9 & 68.07 & 68.27 & 68.17 & 68.75 \\
 \hline
 32 & 44.5 & 43.8 & 44.9 & 44.5 & 68.37 & 68.51 & 68.35 & 68.27  \\
 \hline
 16 & 43.9 & 42.7 & 44.9 & 45 & 68.12 & 68.77 & 68.31 & 68.59  \\
 \hline
 \hline
 \multicolumn{5}{|c|}{GPT3-22B (FP32 Accuracy = 38.75\%)} & \multicolumn{4}{|c|}{Nemotron4-15B (FP32 Accuracy = 64.3\%)} \\ 
 \hline
 \hline
 64 & 36.71 & 38.85 & 38.13 & 38.92 & 63.17 & 62.36 & 63.72 & 64.09 \\
 \hline
 32 & 37.95 & 38.69 & 39.45 & 38.34 & 64.05 & 62.30 & 63.8 & 64.33  \\
 \hline
 16 & 38.88 & 38.80 & 38.31 & 38.92 & 63.22 & 63.51 & 63.93 & 64.43  \\
 \hline
\end{tabular}
\caption{\label{tab:mmlu_abalation} Accuracy on MMLU dataset across GPT3-22B, Llama2-7B, 70B and Nemotron4-15B models.}
\end{table}


%\subsection{Perplexity achieved by various LO-BCQ configurations on LM evaluation harness}

\begin{table} \centering
\begin{tabular}{|c||c|c|c|c||c|c|c|c|} 
\hline
 $L_b \rightarrow$& \multicolumn{4}{c||}{8} & \multicolumn{4}{c||}{8}\\
 \hline
 \backslashbox{$L_A$\kern-1em}{\kern-1em$N_c$} & 2 & 4 & 8 & 16 & 2 & 4 & 8 & 16  \\
 %$N_c \rightarrow$ & 2 & 4 & 8 & 16 & 2 & 4 & 2 \\
 \hline
 \hline
 \multicolumn{5}{|c|}{Race (FP32 Accuracy = 37.51\%)} & \multicolumn{4}{|c|}{Boolq (FP32 Accuracy = 64.62\%)} \\ 
 \hline
 \hline
 64 & 36.94 & 37.13 & 36.27 & 37.13 & 63.73 & 62.26 & 63.49 & 63.36 \\
 \hline
 32 & 37.03 & 36.36 & 36.08 & 37.03 & 62.54 & 63.51 & 63.49 & 63.55  \\
 \hline
 16 & 37.03 & 37.03 & 36.46 & 37.03 & 61.1 & 63.79 & 63.58 & 63.33  \\
 \hline
 \hline
 \multicolumn{5}{|c|}{Winogrande (FP32 Accuracy = 58.01\%)} & \multicolumn{4}{|c|}{Piqa (FP32 Accuracy = 74.21\%)} \\ 
 \hline
 \hline
 64 & 58.17 & 57.22 & 57.85 & 58.33 & 73.01 & 73.07 & 73.07 & 72.80 \\
 \hline
 32 & 59.12 & 58.09 & 57.85 & 58.41 & 73.01 & 73.94 & 72.74 & 73.18  \\
 \hline
 16 & 57.93 & 58.88 & 57.93 & 58.56 & 73.94 & 72.80 & 73.01 & 73.94  \\
 \hline
\end{tabular}
\caption{\label{tab:mmlu_abalation} Accuracy on LM evaluation harness tasks on GPT3-1.3B model.}
\end{table}

\begin{table} \centering
\begin{tabular}{|c||c|c|c|c||c|c|c|c|} 
\hline
 $L_b \rightarrow$& \multicolumn{4}{c||}{8} & \multicolumn{4}{c||}{8}\\
 \hline
 \backslashbox{$L_A$\kern-1em}{\kern-1em$N_c$} & 2 & 4 & 8 & 16 & 2 & 4 & 8 & 16  \\
 %$N_c \rightarrow$ & 2 & 4 & 8 & 16 & 2 & 4 & 2 \\
 \hline
 \hline
 \multicolumn{5}{|c|}{Race (FP32 Accuracy = 41.34\%)} & \multicolumn{4}{|c|}{Boolq (FP32 Accuracy = 68.32\%)} \\ 
 \hline
 \hline
 64 & 40.48 & 40.10 & 39.43 & 39.90 & 69.20 & 68.41 & 69.45 & 68.56 \\
 \hline
 32 & 39.52 & 39.52 & 40.77 & 39.62 & 68.32 & 67.43 & 68.17 & 69.30  \\
 \hline
 16 & 39.81 & 39.71 & 39.90 & 40.38 & 68.10 & 66.33 & 69.51 & 69.42  \\
 \hline
 \hline
 \multicolumn{5}{|c|}{Winogrande (FP32 Accuracy = 67.88\%)} & \multicolumn{4}{|c|}{Piqa (FP32 Accuracy = 78.78\%)} \\ 
 \hline
 \hline
 64 & 66.85 & 66.61 & 67.72 & 67.88 & 77.31 & 77.42 & 77.75 & 77.64 \\
 \hline
 32 & 67.25 & 67.72 & 67.72 & 67.00 & 77.31 & 77.04 & 77.80 & 77.37  \\
 \hline
 16 & 68.11 & 68.90 & 67.88 & 67.48 & 77.37 & 78.13 & 78.13 & 77.69  \\
 \hline
\end{tabular}
\caption{\label{tab:mmlu_abalation} Accuracy on LM evaluation harness tasks on GPT3-8B model.}
\end{table}

\begin{table} \centering
\begin{tabular}{|c||c|c|c|c||c|c|c|c|} 
\hline
 $L_b \rightarrow$& \multicolumn{4}{c||}{8} & \multicolumn{4}{c||}{8}\\
 \hline
 \backslashbox{$L_A$\kern-1em}{\kern-1em$N_c$} & 2 & 4 & 8 & 16 & 2 & 4 & 8 & 16  \\
 %$N_c \rightarrow$ & 2 & 4 & 8 & 16 & 2 & 4 & 2 \\
 \hline
 \hline
 \multicolumn{5}{|c|}{Race (FP32 Accuracy = 40.67\%)} & \multicolumn{4}{|c|}{Boolq (FP32 Accuracy = 76.54\%)} \\ 
 \hline
 \hline
 64 & 40.48 & 40.10 & 39.43 & 39.90 & 75.41 & 75.11 & 77.09 & 75.66 \\
 \hline
 32 & 39.52 & 39.52 & 40.77 & 39.62 & 76.02 & 76.02 & 75.96 & 75.35  \\
 \hline
 16 & 39.81 & 39.71 & 39.90 & 40.38 & 75.05 & 73.82 & 75.72 & 76.09  \\
 \hline
 \hline
 \multicolumn{5}{|c|}{Winogrande (FP32 Accuracy = 70.64\%)} & \multicolumn{4}{|c|}{Piqa (FP32 Accuracy = 79.16\%)} \\ 
 \hline
 \hline
 64 & 69.14 & 70.17 & 70.17 & 70.56 & 78.24 & 79.00 & 78.62 & 78.73 \\
 \hline
 32 & 70.96 & 69.69 & 71.27 & 69.30 & 78.56 & 79.49 & 79.16 & 78.89  \\
 \hline
 16 & 71.03 & 69.53 & 69.69 & 70.40 & 78.13 & 79.16 & 79.00 & 79.00  \\
 \hline
\end{tabular}
\caption{\label{tab:mmlu_abalation} Accuracy on LM evaluation harness tasks on GPT3-22B model.}
\end{table}

\begin{table} \centering
\begin{tabular}{|c||c|c|c|c||c|c|c|c|} 
\hline
 $L_b \rightarrow$& \multicolumn{4}{c||}{8} & \multicolumn{4}{c||}{8}\\
 \hline
 \backslashbox{$L_A$\kern-1em}{\kern-1em$N_c$} & 2 & 4 & 8 & 16 & 2 & 4 & 8 & 16  \\
 %$N_c \rightarrow$ & 2 & 4 & 8 & 16 & 2 & 4 & 2 \\
 \hline
 \hline
 \multicolumn{5}{|c|}{Race (FP32 Accuracy = 44.4\%)} & \multicolumn{4}{|c|}{Boolq (FP32 Accuracy = 79.29\%)} \\ 
 \hline
 \hline
 64 & 42.49 & 42.51 & 42.58 & 43.45 & 77.58 & 77.37 & 77.43 & 78.1 \\
 \hline
 32 & 43.35 & 42.49 & 43.64 & 43.73 & 77.86 & 75.32 & 77.28 & 77.86  \\
 \hline
 16 & 44.21 & 44.21 & 43.64 & 42.97 & 78.65 & 77 & 76.94 & 77.98  \\
 \hline
 \hline
 \multicolumn{5}{|c|}{Winogrande (FP32 Accuracy = 69.38\%)} & \multicolumn{4}{|c|}{Piqa (FP32 Accuracy = 78.07\%)} \\ 
 \hline
 \hline
 64 & 68.9 & 68.43 & 69.77 & 68.19 & 77.09 & 76.82 & 77.09 & 77.86 \\
 \hline
 32 & 69.38 & 68.51 & 68.82 & 68.90 & 78.07 & 76.71 & 78.07 & 77.86  \\
 \hline
 16 & 69.53 & 67.09 & 69.38 & 68.90 & 77.37 & 77.8 & 77.91 & 77.69  \\
 \hline
\end{tabular}
\caption{\label{tab:mmlu_abalation} Accuracy on LM evaluation harness tasks on Llama2-7B model.}
\end{table}

\begin{table} \centering
\begin{tabular}{|c||c|c|c|c||c|c|c|c|} 
\hline
 $L_b \rightarrow$& \multicolumn{4}{c||}{8} & \multicolumn{4}{c||}{8}\\
 \hline
 \backslashbox{$L_A$\kern-1em}{\kern-1em$N_c$} & 2 & 4 & 8 & 16 & 2 & 4 & 8 & 16  \\
 %$N_c \rightarrow$ & 2 & 4 & 8 & 16 & 2 & 4 & 2 \\
 \hline
 \hline
 \multicolumn{5}{|c|}{Race (FP32 Accuracy = 48.8\%)} & \multicolumn{4}{|c|}{Boolq (FP32 Accuracy = 85.23\%)} \\ 
 \hline
 \hline
 64 & 49.00 & 49.00 & 49.28 & 48.71 & 82.82 & 84.28 & 84.03 & 84.25 \\
 \hline
 32 & 49.57 & 48.52 & 48.33 & 49.28 & 83.85 & 84.46 & 84.31 & 84.93  \\
 \hline
 16 & 49.85 & 49.09 & 49.28 & 48.99 & 85.11 & 84.46 & 84.61 & 83.94  \\
 \hline
 \hline
 \multicolumn{5}{|c|}{Winogrande (FP32 Accuracy = 79.95\%)} & \multicolumn{4}{|c|}{Piqa (FP32 Accuracy = 81.56\%)} \\ 
 \hline
 \hline
 64 & 78.77 & 78.45 & 78.37 & 79.16 & 81.45 & 80.69 & 81.45 & 81.5 \\
 \hline
 32 & 78.45 & 79.01 & 78.69 & 80.66 & 81.56 & 80.58 & 81.18 & 81.34  \\
 \hline
 16 & 79.95 & 79.56 & 79.79 & 79.72 & 81.28 & 81.66 & 81.28 & 80.96  \\
 \hline
\end{tabular}
\caption{\label{tab:mmlu_abalation} Accuracy on LM evaluation harness tasks on Llama2-70B model.}
\end{table}

%\section{MSE Studies}
%\textcolor{red}{TODO}


\subsection{Number Formats and Quantization Method}
\label{subsec:numFormats_quantMethod}
\subsubsection{Integer Format}
An $n$-bit signed integer (INT) is typically represented with a 2s-complement format \citep{yao2022zeroquant,xiao2023smoothquant,dai2021vsq}, where the most significant bit denotes the sign.

\subsubsection{Floating Point Format}
An $n$-bit signed floating point (FP) number $x$ comprises of a 1-bit sign ($x_{\mathrm{sign}}$), $B_m$-bit mantissa ($x_{\mathrm{mant}}$) and $B_e$-bit exponent ($x_{\mathrm{exp}}$) such that $B_m+B_e=n-1$. The associated constant exponent bias ($E_{\mathrm{bias}}$) is computed as $(2^{{B_e}-1}-1)$. We denote this format as $E_{B_e}M_{B_m}$.  

\subsubsection{Quantization Scheme}
\label{subsec:quant_method}
A quantization scheme dictates how a given unquantized tensor is converted to its quantized representation. We consider FP formats for the purpose of illustration. Given an unquantized tensor $\bm{X}$ and an FP format $E_{B_e}M_{B_m}$, we first, we compute the quantization scale factor $s_X$ that maps the maximum absolute value of $\bm{X}$ to the maximum quantization level of the $E_{B_e}M_{B_m}$ format as follows:
\begin{align}
\label{eq:sf}
    s_X = \frac{\mathrm{max}(|\bm{X}|)}{\mathrm{max}(E_{B_e}M_{B_m})}
\end{align}
In the above equation, $|\cdot|$ denotes the absolute value function.

Next, we scale $\bm{X}$ by $s_X$ and quantize it to $\hat{\bm{X}}$ by rounding it to the nearest quantization level of $E_{B_e}M_{B_m}$ as:

\begin{align}
\label{eq:tensor_quant}
    \hat{\bm{X}} = \text{round-to-nearest}\left(\frac{\bm{X}}{s_X}, E_{B_e}M_{B_m}\right)
\end{align}

We perform dynamic max-scaled quantization \citep{wu2020integer}, where the scale factor $s$ for activations is dynamically computed during runtime.

\subsection{Vector Scaled Quantization}
\begin{wrapfigure}{r}{0.35\linewidth}
  \centering
  \includegraphics[width=\linewidth]{sections/figures/vsquant.jpg}
  \caption{\small Vectorwise decomposition for per-vector scaled quantization (VSQ \citep{dai2021vsq}).}
  \label{fig:vsquant}
\end{wrapfigure}
During VSQ \citep{dai2021vsq}, the operand tensors are decomposed into 1D vectors in a hardware friendly manner as shown in Figure \ref{fig:vsquant}. Since the decomposed tensors are used as operands in matrix multiplications during inference, it is beneficial to perform this decomposition along the reduction dimension of the multiplication. The vectorwise quantization is performed similar to tensorwise quantization described in Equations \ref{eq:sf} and \ref{eq:tensor_quant}, where a scale factor $s_v$ is required for each vector $\bm{v}$ that maps the maximum absolute value of that vector to the maximum quantization level. While smaller vector lengths can lead to larger accuracy gains, the associated memory and computational overheads due to the per-vector scale factors increases. To alleviate these overheads, VSQ \citep{dai2021vsq} proposed a second level quantization of the per-vector scale factors to unsigned integers, while MX \citep{rouhani2023shared} quantizes them to integer powers of 2 (denoted as $2^{INT}$).

\subsubsection{MX Format}
The MX format proposed in \citep{rouhani2023microscaling} introduces the concept of sub-block shifting. For every two scalar elements of $b$-bits each, there is a shared exponent bit. The value of this exponent bit is determined through an empirical analysis that targets minimizing quantization MSE. We note that the FP format $E_{1}M_{b}$ is strictly better than MX from an accuracy perspective since it allocates a dedicated exponent bit to each scalar as opposed to sharing it across two scalars. Therefore, we conservatively bound the accuracy of a $b+2$-bit signed MX format with that of a $E_{1}M_{b}$ format in our comparisons. For instance, we use E1M2 format as a proxy for MX4.

\begin{figure}
    \centering
    \includegraphics[width=1\linewidth]{sections//figures/BlockFormats.pdf}
    \caption{\small Comparing LO-BCQ to MX format.}
    \label{fig:block_formats}
\end{figure}

Figure \ref{fig:block_formats} compares our $4$-bit LO-BCQ block format to MX \citep{rouhani2023microscaling}. As shown, both LO-BCQ and MX decompose a given operand tensor into block arrays and each block array into blocks. Similar to MX, we find that per-block quantization ($L_b < L_A$) leads to better accuracy due to increased flexibility. While MX achieves this through per-block $1$-bit micro-scales, we associate a dedicated codebook to each block through a per-block codebook selector. Further, MX quantizes the per-block array scale-factor to E8M0 format without per-tensor scaling. In contrast during LO-BCQ, we find that per-tensor scaling combined with quantization of per-block array scale-factor to E4M3 format results in superior inference accuracy across models. 


% \todo{Add deepseek and other extra metrics to appendix.}

\end{document}
