% \vskip 2em
\begin{figure*}
    \centering
	\begin{tikzpicture}
    \node[text=black, fill=yellow!20, rounded corners, inner sep=2mm, text width=0.8\linewidth, align=center] (question) at (0,0) { 
\textbf{Question}: In rectangle $ABCD$, $AB=100$. Let $E$ be the midpoint of $\overline{AD}$. Given that line $AC$ and line $BE$ are perpendicular, find the greatest integer less than $AD$.
    };
    \node[ below=0.8em of question, ] (cot) {
			\begin{tcolorbox}[width=0.8\linewidth, title={ PPO with Outcome \& Length Reward}, colframe=black]
				\footnotesize

    \textbf{Answer}:

\textbf{\#\# Step 1:}
Draw a diagram of the given scenario, and we have a rectangle $ABCD$ with $AB=100$.


\textbf{\#\# Step 2:}
Let point $E$ be the midpoint of $\overline{AD}$. So, we can define $AE=ED=\frac{AD}{2}$.

...

\textbf{\#\# Step 27:}
Dividing both sides of the equation $10x^2=0$ by 10 gives $x^2=0$.

\textbf{\#\# Step 28:}
We know that $x^2=0$ is only true when $x=0$.

{\color{red}
\textbf{\#\# Step 29:}
However, since $ABCD$ is a rectangle, it has nonzero area, which means that $AD\ne0$.

\textbf{\#\# Step 30:}
Therefore, we must re-examine the steps to see where we went wrong.


\textbf{\#\# Step 31:}
Looking back at step 20, we can see that we derived two different expressions for $AC^2$, which is $2x^2$ and $-4x^2$.
}

...

\textbf{\#\# Step 57:}
The value of $100\sqrt{2}$ is approximately 141.42.

\textbf{\#\# Step 58:}
Since we are asked to find the greatest integer less than $AD$, we can use the value of $100\sqrt{2}$ we calculated in step 57.

\textbf{\#\# Step 59:}
The greatest integer less than $141.42$ is 141.

The final answer is: $\boxed{141}$.

    \end{tcolorbox}
    };
    \end{tikzpicture}
    \caption{The emergence of backtracking, self-correction in Llama-3.1-8B. During PPO training we incentivize the model to use more intermediate steps (length reward) before arriving at the solution. Full CoT trajectory in \autoref{sec:app-rlsp-emergence}.}
    \label{fig:llama_emergence_example}
\end{figure*}



% \vspace{-2cm}
% \vskip 2em
\begin{figure*}
    \centering
	\begin{tikzpicture}
    \node[text=black, fill=yellow!20, rounded corners, inner sep=2mm, text width=0.8\linewidth, align=center] (question) at (0,0) {
    \textbf{Question}: The number \(\sqrt{104\sqrt{6}+468\sqrt{10}+144\sqrt{15}+2006}\) can be written as \(a\sqrt{2}+b\sqrt{3}+c\sqrt{5},\) where \(a, b,\) and \(c\) are positive integers. Find \(abc\).
    };
    \node[ below=0.8em of question, ] (cot) {
			\begin{tcolorbox}[width=0.8\linewidth, title={PPO with Outcome \& Length Reward}, colframe=black]
				\footnotesize

    \textbf{Answer}: To solve the problem, we start by assuming that the expression \(\sqrt{104\sqrt{6} + 468\sqrt{10} + 144\sqrt{15} + 2006}\) can be written in the form \(a\sqrt{2} + b\sqrt{3} + c\sqrt{5}\), where \(a, b,\) and \(c\) are positive integers. We need to find the values of \(a, b,\) and \(c\) such that:
    \[
    (a\sqrt{2} + b\sqrt{3} + c\sqrt{5})^2 = 104\sqrt{6} + 468\sqrt{10} + 144\sqrt{15} + 2006
    \]

...

    \[
    a = \sqrt{171} = 3\sqrt{19}
    \]

    {\color{red} Since $a$ must be an integer, we recheck our calculations and find a mistake.
    }
    
    Let's solve the system again:
    \[
    ab = 52,\quad 
    ac = 234,\quad
    bc = 72
    \]
    ...
    \end{tcolorbox}
    };
    \end{tikzpicture}
    \caption{The emergence of backtracking and self-correction in Qwen-2.5-32B-Instruct; full CoT reasoning trajectory in \autoref{sec:app-rlsp-emergence}.}
    \label{fig:qwen_emergence_example}
\end{figure*}