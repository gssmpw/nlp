	

\begin{figure*}[htbp]
	\centering
	\begin{tikzpicture}

		\node[text=black, fill=yellow!20, rounded corners, inner sep=2mm, text width=0.8\linewidth, align=center] (question) at (0,0) { \textbf{Question:} The positive three-digit integer $N$ has a ones digit of $0$. What is the probability that $N$ is divisible by $4$? Express your answer as a common fraction.};
		% Draw the two main boxes


		\node[below=0.8em of question, anchor=north, xshift=-0.35\linewidth] (cot) {
			\begin{tcolorbox}[width=0.35\linewidth, title={ PPO with Outcome Reward Only}, colframe=black]
				\small
				\textbf{Step 1}:
				To determine the probability that the three-digit integer $N$ is divisible by $4$, we must first establish the total number of possibilities for a three-digit integer with a ones digit of $0$.
				
				\textbf{Step 2}:
				A three-digit number with a ones digit of $0$ is in the form of $\overline{ab0}$, where $a$ and $b$ are digits that can take on values from $1$ to $9$, since the hundreds digit cannot be $0$. \textcolor{red!70}{Therefore, there are $9$ options for $a$ and $9$ options for $b$, giving a total of $9 \cdot 9 = 81$ possible numbers.}
				
				\textbf{Step 3}:
				For a number to be divisible by $4$, the last two digits must be divisible by $4$. Since the last digit is $0$, we only need to consider the possible values of $b$ that make the number $\overline{b0}$ divisible by $4$.
				
				\textbf{Step 4}:
				We can systematically check values of $b$ from $0$ to $9$ to determine which values make $\overline{b0}$ divisible by $4$. By doing so, we find that the values of $b$ that make $\overline{b0}$ divisible by $4$ are $4$ and $8$.
				
				\textbf{Step 5}:
				Thus, there are $2$ possible values for $b$ that make the number $\overline{b0}$ divisible by $4$, and each of these has $9$ options for $a$. Therefore, there are $9 \cdot 2 = 18$ possibilities that are divisible by $4$.
				
				\textbf{Step 6}:
				To find the probability that $N$ is divisible by $4$, we divide the number of possibilities that are divisible by $4$ by the total number of possibilities, which is $\frac{18}{81} = \frac{2}{9}$.
				
				The final answer is: $\boxed{\frac{2}{9}}$
			\end{tcolorbox}
		};
		\node[below=0.8em of question, anchor=north, xshift=0.15\linewidth] (rlsp) {
			\begin{tcolorbox}[width=0.6\linewidth, title={ PPO with Outcome \& Length Reward}, colframe=black]
				\small
				\textbf{Step 1}: Understand the conditions for divisibility by 4
				
				A number is divisible by 4 if the number formed by its last two digits is divisible by 4.
				
				\textbf{Step 2}: Recognize the possible last two digits for N
				
				Since N is a three-digit number and its ones digit is 0, we only need to consider the last two digits which can range from 00 to 90, in increments of 10.
				
				\textbf{Step 3}: List the possible last two digits of N
				
				The possible last two digits are 00, 10, 20, 30, 40, 50, 60, 70, 80, and 90.
				
				\textbf{Step 4}: Determine how many of these are divisible by 4
				
				Among 00, 10, 20, 30, 40, 50, 60, 70, 80, and 90, the numbers that are divisible by 4 are 00, 20, 40, 60, and 80.
				
				\textbf{Step 5}: Count the total number of possibilities
				There are 10 possibilities in total.
				
				\textbf{Step 6}: Calculate the probability that N is divisible by 4
                
				The probability that N is divisible by 4 is the number of successful outcomes (numbers divisible by 4) divided by the total number of outcomes.
				
				\textbf{Step 7}: Calculate the probability
                
				There are 5 successful outcomes (00, 20, 40, 60, 80) and 10 total outcomes. \textbf{So the probability is 5/10 which simplifies to 1/2}.
				
				\textbf{Step 8}: Consider the divisibility rule for 4
				
				\textit{... omitted ...}
				
				
				\textbf{Step 16}: Realize that the probability that N is divisible by 4 is still 1/2
				
				\textbf{This is because 1/2 of the numbers from 00 to 90 are divisible by 4, regardless of the other digits in the number.}
				
				\textbf{Step 17}: Consider the fact that the hundreds digit does not affect the probability
				
				\textit{... omitted ...}
				

				
				\textbf{Step 22}: \textcolor{purple}{Understand that this confirms our previous conclusion. 
                				We have already determined that 5 of the numbers from 00 to 90 are divisible by 4.}
				
				\textbf{Step 23}: Realize that the probability that N is divisible by 4 is 1/2
                
				Since 5 of the 10 numbers from 00 to 90 are divisible by 4, \textbf{the probability is 5/10, which simplifies to 1/2.}
				
				The final answer is: $\boxed{\frac{1}{2}}$
			\end{tcolorbox}
		};
		
	\end{tikzpicture}
	\caption{A side-by-side comparison of regular CoT and CoT after RLSP training. Both experiments use Llama-3.1-8B-Instruct model without any additional SFT. Observe that the model with length reward, which incentivizes more intermediate steps, exhibits emergent behaviors such as verification (step 22) and consideration of alternative possibilities (texts in bold). The full reasoning trajectory in \autoref{sec:app-rlsp-emergence} shows that the model tries 3 different approaches to solve the problem and recognizes that they all lead to the same answer.}
    \label{fig:emergence_example}
\end{figure*}
	