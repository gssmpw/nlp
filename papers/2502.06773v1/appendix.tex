%%%%%%%%%%%%%%%%%%%%%%%%%%%%%%%%%%%%%%%%%%%%%%%%%%%%%%%%%%%%%%%%%%%%%%%%%%%%%%%
%%%%%%%%%%%%%%%%%%%%%%%%%%%%%%%%%%%%%%%%%%%%%%%%%%%%%%%%%%%%%%%%%%%%%%%%%%%%%%%
% EXAMPLES
%%%%%%%%%%%%%%%%%%%%%%%%%%%%%%%%%%%%%%%%%%%%%%%%%%%%%%%%%%%%%%%%%%%%%%%%%%%%%%%
%%%%%%%%%%%%%%%%%%%%%%%%%%%%%%%%%%%%%%%%%%%%%%%%%%%%%%%%%%%%%%%%%%%%%%%%%%%%%%%



\section{Examples of RLSP Emergence Behavior}\label{sec:app-rlsp-emergence}

In this section, we present three examples demonstrating the emergence of self-verification, backtracking, and self-correction behaviors. The models are trained solely using PPO with a length penalty reward on the Llama-3.1-8B-Instruct model, without any supervised fine-tuning. All steps exhibiting these behaviors are highlighted in red.

\begin{itemize}
    \item \textbf{Self-Verification:} The model explicitly checks its own reasoning steps to confirm correctness before finalizing an answer. This often involves re-evaluating computations or validating logical consistency.  
    \item \textbf{Consideration of alternative possibilities:} The model searches over multiple rationales and explores alternative approaches and recognizes inconsistencies or similarities. 
    \item \textbf{Self-Correction:} The model identifies errors in its intermediate steps and revises its solution in the subsequent steps.
    \item \textbf{Backtracking:} Upon identifying errors or dead ends, the model goes back to an earlier step and considers a different reasoning path. 
\end{itemize}

These examples provide strong evidence that RL facilitates the emergence of more systematic and reflective problem-solving behaviours without explicit supervision.

\vskip 2em

\begin{promptbox}[box:fig1-full]{Full Self-Verification Behavior for \autoref{fig:emergence_example}}

\textbf{Question:} The positive three-digit integer $N$ has a ones digit of $0$. What is the probability that $N$ is divisible by $4$? Express your answer as a common fraction.

\textbf{Step 1}: Understand the conditions for divisibility by 4

A number is divisible by 4 if the number formed by its last two digits is divisible by 4.
 
\textbf{Step 2}: Recognize the possible last two digits for N

Since N is a three-digit number and its ones digit is 0, we only need to consider the last two digits which can range from 00 to 90, in increments of 10.
 
\textbf{Step 3}: List the possible last two digits of N

The possible last two digits are 00, 10, 20, 30, 40, 50, 60, 70, 80, and 90.
 
\textbf{Step 4}: Determine how many of these are divisible by 4

Among 00, 10, 20, 30, 40, 50, 60, 70, 80, and 90, the numbers that are divisible by 4 are 00, 20, 40, 60, and 80.
 
\textbf{Step 5}: Count the total number of possibilities

There are 10 possibilities in total.
 
\textbf{Step 6}: Calculate the probability that N is divisible by 4

The probability that N is divisible by 4 is the number of successful outcomes (numbers divisible by 4) divided by the total number of outcomes.
 
\textbf{Step 7}: Calculate the probability

There are 5 successful outcomes (00, 20, 40, 60, 80) and 10 total outcomes. So the probability is 5/10 which simplifies to 1/2.
 
\textbf{Step 8}: Consider the divisibility rule for 4

To be divisible by 4, the last two digits of N must be divisible by 4. This means for every 10 numbers from 00 to 90, exactly one will be divisible by 4.
 
{\color{red}\textbf{Step 9}: Verify the previous conclusion}

Since we know that the numbers 00, 20, 40, 60, and 80 are divisible by 4 and these are evenly spaced, we can confirm that there is one number out of every ten from 00 to 90 that is divisible by 4.
 
\textbf{Step 10}: Calculate the probability

The probability that N is divisible by 4 is therefore 1/2, since half the numbers from 00 to 90 are divisible by 4.
 
\textbf{Step 11}: Consider the fact that 00 is a special case

In a set of numbers from 00 to 90, 00 is counted as 1 of the numbers, and it is divisible by 4.
 
\textbf{Step 12}: Consider the implication of the 00 case for the probability

Since 00 is divisible by 4, it does not change the fact that half the numbers from 00 to 90 are divisible by 4.
 
\textbf{Step 13}: Consider the fact that there are 10 numbers in the set from 00 to 90

This confirms that 5 of the numbers are divisible by 4 and 5 are not.
 
\textbf{Step 14}: Calculate the probability that N is divisible by 4

The probability that N is divisible by 4 is therefore 5/10, which simplifies to 1/2.
 
\textbf{Step 15}: Understand that we don't need to calculate the probability for all 3-digit numbers

Since we are only concerned with three-digit numbers that have a ones digit of 0, the set of numbers we are considering is much smaller than all three-digit numbers.
 
\textbf{Step 16}: Realize that the probability that N is divisible by 4 is still 1/2

This is because 1/2 of the numbers from 00 to 90 are divisible by 4, regardless of the other digits in the number.
 
\textbf{Step 17}: Consider the fact that the hundreds digit does not affect the probability

Since the hundreds digit does not affect whether a number is divisible by 4, it does not change the probability.
 
\textbf{Step 18}: Consider the fact that the tens digit does not affect the probability

Similarly, the tens digit does not affect whether a number is divisible by 4.
 
\textbf{Step 19}: Realize that the probability that N is divisible by 4 is determined by the last two digits

Since the ones digit is always 0, we only need to consider the last two digits.
 
{\color{red}\textbf{Step 20}: Understand that this confirms that 1/2 of the numbers are divisible by 4}

Since there are 5 numbers from 00 to 90 that are divisible by 4, and 10 total numbers, the probability that N is divisible by 4 is 5/10, which simplifies to 1/2.
 
\textbf{Step 21}: Consider the fact that N is a three-digit number with a ones digit of 0

This means that the last two digits can range from 00 to 90.
 
{\color{red} \textbf{Step 22}: Understand that this confirms our previous conclusion}

We have already determined that 5 of the numbers from 00 to 90 are divisible by 4.
 
\textbf{Step 23}: Realize that the probability that N is divisible by 4 is 1/2

Since 5 of the 10 numbers from 00 to 90 are divisible by 4, the probability is 5/10, which simplifies to 1/2.


The final answer is: $\boxed{\frac{1}{2}}$
\end{promptbox}

\vskip 2em

\begin{promptbox}{Example of Self-Verification Behavior}
Problem: The equation of the line that passes through the points $(-3,5)$ and $(0,-4)$ can be expressed in the form $y=mx+b$. What is the value of $m+b$?

assistant 

\textbf{\#\# Step 1:}
To find the equation of the line that passes through the points $(-3,5)$ and $(0,-4)$, we should use the slope-intercept form, which is $y=mx+b$, where $m$ is the slope of the line and $b$ is the y-intercept.
 
\textbf{\#\# Step 2:}
The slope $m$ of a line passing through two points $(x_1,y_1)$ and $(x_2,y_2)$ can be calculated using the formula: $m=\dfrac{y_2-y_1}{x_2-x_1}$. Let's use the given points to find the slope $m$.
 
\textbf{\#\# Step 3:}
Plugging the given points into the slope formula: $m=\dfrac{-4-5}{0-(-3)}=\dfrac{-9}{3}=-3$.
 
\textbf{\#\# Step 4:}
Now that we have the slope $m$, we can find the equation of the line. We know that the line passes through the point $(0,-4)$, so we can find $b$ by plugging this point into the equation $y=mx+b$.
 
\textbf{\#\# Step 5:}
Plugging in the point $(0,-4)$ and the slope $m=-3$ into the equation $y=mx+b$ gives us $-4=-3(0)+b \implies b=-4$.
 
\textbf{\#\# Step 6:}
Now that we have the value of $m$ and $b$, we can find $m+b$. Using the values of $m$ and $b$ we found in steps 3 and 5, we can calculate the value of $m+b$.
 
\textbf{\#\# Step 7:}
Plugging in the values of $m$ and $b$ into the equation $m+b$ gives us $m+b = -3 + (-4)$.
 
\textbf{\#\# Step 8:}
To find the value of $m+b$, we need to perform the addition of $-3$ and $-4$. The result is $m+b = -7$.
 

{\color{red}
\textbf{\#\# Step 9:}
We can also confirm that the points $(-3,5)$ and $(0,-4)$ lie on the line with the equation $y=-3x-4$ by substituting the $x$ and $y$ values of these points into the equation.
}
 
\textbf{\#\# Step 10:}
Using $(-3,5)$ in the equation $y=-3x-4$, we get $5=-3(-3)-4$, which simplifies to $5=9-4$, which is true.
 
\textbf{\#\# Step 11:}
Using $(0,-4)$ in the equation $y=-3x-4$, we get $-4=-3(0)-4$, which simplifies to $-4=-4$, which is also true.
 
\textbf{\#\# Step 12:}
Since the equation $y=-3x-4$ satisfies the points $(-3,5)$ and $(0,-4)$, the equation of the line is indeed $y=-3x-4$.
 
\textbf{\#\# Step 13:}
From steps 3 and 5, we have determined that the slope $m$ of the line is $-3$ and the y-intercept $b$ is $-4$. Therefore, the values of $m$ and $b$ are $-3$ and $-4$.
 
\textbf{\#\# Step 14:}
The problem asks us to find the value of $m+b$ and we have determined that $m=-3$ and $b=-4$.
 
\textbf{\#\# Step 15:}
Therefore, the value of $m+b$ is $-3 + (-4)$.
 
\textbf{\#\# Step 16:}
The value of $-3 + (-4)$ is $-7$.
 
\textbf{\#\# Step 17:}
We have also confirmed that the points $(-3,5)$ and $(0,-4)$ lie on the line with the equation $y=-3x-4$.
 
\textbf{\#\# Step 18:}
Since we have found the correct equation of the line and determined the values of $m$ and $b$, we are confident that the value of $m+b$ is correct.
 

{\color{red} \textbf{\#\# Step 19:}
We need to make sure that our final answer for $m+b$ is accurate.
}
 
\textbf{\#\# Step 20:}
Using the given points $(-3,5)$ and $(0,-4)$ in the equation $y=-3x-4$ confirms that the equation of the line is correct and that the values of $m$ and $b$ are $-3$ and $-4$.
 
\textbf{\#\# Step 21:}
Since we have correctly determined the equation of the line and the values of $m$ and $b$, we can be confident that the value of $m+b$ is accurate.
 
\textbf{\#\# Step 22:}
We have confirmed that the points $(-3,5)$ and $(0,-4)$ lie on the line with the equation $y=-3x-4$.
 

{\color{red}
\textbf{\#\# Step 23:}
The final step is to ensure that we have not made a calculation error.
}
 

{\color{red}\textbf{\#\# Step 24:} 
Upon reviewing our calculations, we have not made any errors.
}
 
\textbf{\#\# Step 25:}
We have correctly determined the equation of the line and the values of $m$ and $b$.
 
\textbf{\#\# Step 26:}
Therefore, the final answer for $m+b$ is $-7$.
 
The final answer is: $\boxed{-7}$.
\end{promptbox}

\vskip 2em



Below is an example of Llama-3.1 that attempts to do self-correction yet failed to solve the problem.


\begin{promptbox}{Example of Self-Correction}
\textbf{Problem:} The largest and smallest of three consecutive terms in an arithmetic sequence differ by 14. Half of the smallest term is added to each term and the sum of the resulting three numbers is 120. What is the value of the original smallest term?

\textbf{ \#\# Step 1:} Let's denote the smallest term in the arithmetic sequence as $a$. Since the sequence is arithmetic, the common difference between terms is constant. Therefore, the next two terms can be represented as $a + d$ and $a + 2d$, where $d$ is the common difference.

\textbf{ \#\# Step 2:} We are told that the largest and smallest of the three terms differ by $14$. This gives us the equation $ (a + 2d) - a = 14 $. Simplifying this equation, we find that $2d = 14$, and thus $d = 7$.

\textbf{ \#\# Step 3:} Now that we know the common difference $d$ is $7$, we can express the next two terms as $a + 7$ and $a + 14$.

\textbf{ \#\# Step 4:} We are also given that half of the smallest term $a$ is added to each term, and the sum of the resulting three numbers is $120$. Therefore, the sum of the modified terms is 
$
\left(a + \frac{a}{2}\right) + \left(a + 7 + \frac{a}{2}\right) + \left(a + 14 + \frac{a}{2}\right) = 120.
$

\textbf{ \#\# Step 5:} Simplify the equation from step 4 by combining like terms. We get 
$
2a + \frac{a}{2} + 2a + \frac{a}{2} + 2a + \frac{a}{2} = 120.
$

\textbf{ \#\# Step 6:} This further simplifies to 
$
6a + a = 120.
$

\textbf{ \#\# Step 7:} To solve for $a$, we divide both sides of the equation by $7$. This gives us 
$
a = \frac{120}{7}.
$

\textbf{ \#\# Step 8:} We must now find a common denominator to see if the value of $a$ is a whole number.

\textbf{ \#\# Step 9:} $\frac{120}{7} = 17.14$ (rounded to two decimal places). However, the original problem suggests a whole number solution for $a$. Therefore, there seems to be an inconsistency in the approach taken so far. We will revisit the steps to ensure accuracy.

\textbf{ \#\# Step 10:} Let's revisit the equation from step 4,
$
\left(a + \frac{a}{2}\right) + \left(a + 7 + \frac{a}{2}\right) + \left(a + 14 + \frac{a}{2}\right) = 120.
$

\textbf{ \#\# Step 11:} We can simplify this as 
$
3a + 21 = 120.
$

\textbf{ \#\# Step 12:} Subtract 21 from both sides of the equation to isolate the term with $a$. This results in 
$
3a = 99.
$

\textbf{ \#\# Step 13:} Divide both sides of the equation by $3$ to solve for $a$. This gives us 
$
a = 33.
$

{\color{red} \textbf{ \#\# Step 14:} However, there seems to have been a misunderstanding in the simplification of the original problem. Let's return to the original statement that the largest and smallest of the three terms differ by $14$.} This difference is actually between the second and first term, not the largest and smallest as initially interpreted. The correct interpretation should be that $ (a + d) - a = 14 $, which was correctly solved as $2d = 14$, and $d = 7$.

\textbf{ \#\# Step 15:} Given the correct common difference $d$ is $7$, the terms in the sequence are $a$, $a + 7$, and $a + 14$.

\textbf{ \#\# Step 16:} The correct equation based on the sum of the terms with the smallest term plus half of itself added to each term should be 
$
\left(a + \frac{a}{2}\right) + \left(a + 7 + \frac{a}{2}\right) + \left(a + 14 + \frac{a}{2}\right) = 120.
$

\textbf{ \#\# Step 17:} Simplify the equation from step 16 to find the correct value of $a$. This simplifies to 
$
3a + 21 = 120.
$

\textbf{ \#\# Step 18:} Subtract 21 from both sides of the equation to isolate the term with $a$. This results in 
$
3a = 99.
$

\textbf{ \#\# Step 19:} Divide both sides of the equation by $3$ to solve for $a$. This gives us 
$
a = \frac{99}{3}.
$

\textbf{ \#\# Step 20:} Now, we perform the division to find the correct value of $a$. This results in 
$
a = 33.
$

The final answer is: $\boxed{33}$.
\end{promptbox}

Below is an example of Llama-3.1 that attempts to propose an alternative approach to solve the problem.


\begin{promptbox}{Example of Using Alternative Approach}
\textbf{Problem:} What is the value of $y$ in the equation $\frac{30}{50}=\sqrt{\frac{y}{50}}$?

\textbf{ \#\# Step 1:} We need to isolate the variable $y$ in the equation $\frac{30}{50}=\sqrt{\frac{y}{50}}$. To begin, we will square both sides of the equation.

\textbf{ \#\# Step 2:} By squaring both sides of the equation, we can eliminate the square root on the right-hand side. The equation becomes $\left(\frac{30}{50}\right)^2=\frac{y}{50}$.

\textbf{ \#\# Step 3:} To solve for $y$, we first need to evaluate the left-hand side of the equation. The left-hand side of the equation can be simplified as $\left(\frac{30}{50}\right)^2=\left(\frac{3}{5}\right)^2$.

\textbf{ \#\# Step 4:} Now, we need to simplify the expression $\left(\frac{3}{5}\right)^2$. The expression can be simplified as $\left(\frac{3}{5}\right)^2=\frac{3^2}{5^2}$.

\textbf{ \#\# Step 5:} After simplifying, we have $\frac{3^2}{5^2}=\frac{9}{25}$.

\textbf{ \#\# Step 6:} Now that we have the simplified expression, we can rewrite the equation as $\frac{9}{25}=\frac{y}{50}$.

\textbf{ \#\# Step 7:} To solve for $y$, we need to multiply both sides of the equation by 50.

\textbf{ \#\# Step 8:} Multiplying both sides of the equation by 50 gives $\frac{9}{25} \times 50=y$.

\textbf{ \#\# Step 9:} To simplify the expression, we can multiply the numerator of the fraction by 50 and divide the result by the denominator. The left-hand side of the equation can be simplified as $\frac{9 \times 50}{25}$.

\textbf{ \#\# Step 10:} Now, we can simplify the expression $\frac{9 \times 50}{25}$. The expression can be simplified as $\frac{9 \times 50}{25}=\frac{450}{25}$.

\textbf{ \#\# Step 11:} After simplifying, we have $\frac{450}{25}$.

\textbf{ \#\# Step 12:} $\frac{450}{25}$ can be simplified further by dividing the numerator by the denominator.

\textbf{ \#\# Step 13:} $\frac{450}{25}$ can be rewritten as $\frac{450\div25}{25\div25}$.

\textbf{ \#\# Step 14:} $\frac{450\div25}{25\div25}=\frac{18}{1}$.

\textbf{ \#\# Step 15:} Now, we have $\frac{18}{1}$.

\textbf{ \#\# Step 16:} $\frac{18}{1}$ can be rewritten as $18$.

\textbf{ \#\# Step 17:} Therefore, $y=18$.

\textbf{ \#\# Step 18:} { \color{red} However, if we look back at the original equation, we can notice that the equation can be simplified before squaring both sides.}

\textbf{ \#\# Step 19:} The left-hand side of the equation can be simplified as $\frac{30}{50}=\frac{3}{5}$.

\textbf{ \#\# Step 20:} Now we can square both sides of the equation.

\textbf{ \#\# Step 21:} By squaring both sides of the equation, we can eliminate the square root on the right-hand side. The equation becomes $\left(\frac{3}{5}\right)^2=\frac{y}{50}$.

...OMITTED...

The final answer is: $\boxed{18}$.
\end{promptbox}


\vskip 2em 

\begin{promptbox}{Example of Self-Correction}
Problem: Find all real $x \in [0, 2 \pi]$ such that
\[\tan 7x - \sin 6x = \cos 4x - \cot 7x.\]
Enter all the real solutions, separated by commas.

\textbf{\#\# Step 1:} The given equation involves trigonometric functions, so the first step is to rewrite the equation in terms of sine and cosine functions. We can rewrite $\tan 7x$ as $\frac{\sin 7x}{\cos 7x}$ and $\cot 7x$ as $\frac{\cos 7x}{\sin 7x}.$

\textbf{\#\# Step 2:} Now, we can substitute these expressions into the given equation.
\[\frac{\sin 7x}{\cos 7x} - \sin 6x = \cos 4x - \frac{\cos 7x}{\sin 7x}.\]

\textbf{\#\# Step 3:} Next, we can multiply through by $\sin 7x \cos 7x$ to clear the denominators. This gives us
\[\sin 7x \cos 7x - \sin 6x \cos 7x = \sin 7x \cos 4x - \cos 7x \sin 4x.\]

\textbf{\#\# Step 4:} We can then use the angle addition formula and the product-to-sum formula to simplify the equation. After using the formulas, the equation becomes
\[\frac{\sin 14x - \sin 7x}{2} = \sin 11x \cos x.\]

\textbf{\#\# Step 5:} By applying the sum-to-product formula to $\sin 14x - \sin 7x$, we can rewrite the equation as
\[\sin 11x \cos x = \frac{\sin 11x \cos x}{2}.\]

\textbf{\#\# Step 6:} Multiplying both sides of the equation by 2 yields
\[\sin 11x \cos x = \sin 11x \cos x.\]

\textbf{\#\# Step 7:} We can now subtract $\sin 11x \cos x$ from both sides, which results in
\[0 = \sin 11x \cos x - \sin 11x \cos x.\]

\textbf{\#\# Step 8:} We can then factor out the common terms to get
\[0 = \sin 11x \cos x - \sin 11x \cos x.\]

\textbf{\#\# Step 9:} This is always true, so we cannot determine any specific values for $x$ from this equation. Therefore, we need to revisit the previous steps and identify any potential errors.

\textbf{\#\# Step 10:} {\color{red} Upon reviewing our steps, we can see that we made an error when we applied the product-to-sum formula. We should have been more careful when using the formula and considered the possibility of a zero product.}

\textbf{\#\# Step 11:} To correct this, we need to revisit the equation
\[\sin 11x \cos x = \frac{\sin 11x \cos x}{2}.\]

\textbf{\#\# Step 12:} We can multiply both sides by 2, giving
\[2 \sin 11x \cos x = \sin 11x \cos x.\]

\textbf{\#\# Step 13:} Next, we can subtract $\sin 11x \cos x$ from both sides to get
\[2 \sin 11x \cos x - \sin 11x \cos x = 0.\]

\textbf{\#\# Step 14:} Then, we can factor out the common term $\sin 11x \cos x$ to get
\[\sin 11x \cos x = 0.\]

\textbf{\#\# Step 15:} This equation gives us the solutions $11x = \frac{\pi}{2} + k \pi$ and $x = \frac{\pi}{2} + k \pi.$ We can substitute these into the original equation to see if they are valid solutions.

\textbf{\#\# Step 16:} Plugging in $x = \frac{\pi}{2} + k \pi$ into the original equation, we can determine whether it satisfies the equation.

\textbf{\#\# Step 17:} Similarly, we can plug in $x = \frac{\pi}{2} + k \pi$ into the original equation and check if it satisfies the equation.

\textbf{\#\# Step 18:} After checking the values of $x = \frac{\pi}{2} + k \pi,$ we can determine that the only valid solutions are $\boxed{x = \frac{3 \pi}{7}, \frac{2 \pi}{7}}.$

\textbf{\#\# Step 19:} Therefore, we can conclude that these are the real solutions to the equation. The final answer is:
\[\boxed{\frac{3 \pi}{7}, \frac{2 \pi}{7}}\]
\end{promptbox}

\vskip 2em

\begin{promptbox}{Example of Backtracking Behavior}
Problem: Find all real $x \in [0, 2 \pi]$ such that
\[\tan 7x - \sin 6x = \cos 4x - \cot 7x.\]
Enter all the real solutions, separated by commas.

\textbf{\#\# Step 1:} The given equation involves trigonometric functions, so the first step is to rewrite the equation in terms of sine and cosine functions. We can rewrite $\tan 7x$ as $\frac{\sin 7x}{\cos 7x}$ and $\cot 7x$ as $\frac{\cos 7x}{\sin 7x}.$

\textbf{\#\# Step 2:} Now, we can substitute these expressions into the given equation.
\[\frac{\sin 7x}{\cos 7x} - \sin 6x = \cos 4x - \frac{\cos 7x}{\sin 7x}.\]

\textbf{\#\# Step 3:} Next, we can multiply through by $\sin 7x \cos 7x$ to clear the denominators. This gives us
\[\sin 7x \cos 7x - \sin 6x \cos 7x = \sin 7x \cos 4x - \cos 7x \sin 4x.\]

\textbf{\#\# Step 4:} We can then use the angle addition formula and the product-to-sum formula to simplify the equation. After using the formulas, the equation becomes
\[\frac{\sin 14x - \sin 7x}{2} = \sin 11x \cos x.\]

\textbf{\#\# Step 5:} By applying the sum-to-product formula to $\sin 14x - \sin 7x$, we can rewrite the equation as
\[\sin 11x \cos x = \frac{\sin 11x \cos x}{2}.\]

\textbf{\#\# Step 6:} Multiplying both sides of the equation by 2 yields
\[\sin 11x \cos x = \sin 11x \cos x.\]

\textbf{\#\# Step 7:} We can now subtract $\sin 11x \cos x$ from both sides, which results in
\[0 = \sin 11x \cos x - \sin 11x \cos x.\]

\textbf{\#\# Step 8:} We can then factor out the common terms to get
\[0 = \sin 11x \cos x - \sin 11x \cos x.\]

\textbf{\#\# Step 9:} {\color{red} This is always true, so we cannot determine any specific values for $x$ from this equation. Therefore, we need to revisit the previous steps and identify any potential errors.}

\textbf{\#\# Step 10:} Upon reviewing our steps, we can see that we made an error when we applied the product-to-sum formula. We should have been more careful when using the formula and considered the possibility of a zero product.

\textbf{\#\# Step 11:} To correct this, we need to revisit the equation
\[\sin 11x \cos x = \frac{\sin 11x \cos x}{2}.\]

\textbf{\#\# Step 12:} We can multiply both sides by 2, giving
\[2 \sin 11x \cos x = \sin 11x \cos x.\]

\textbf{\#\# Step 13:} Next, we can subtract $\sin 11x \cos x$ from both sides to get
\[2 \sin 11x \cos x - \sin 11x \cos x = 0.\]

\textbf{\#\# Step 14:} Then, we can factor out the common term $\sin 11x \cos x$ to get
\[\sin 11x \cos x = 0.\]

\textbf{\#\# Step 15:} This equation gives us the solutions $11x = \frac{\pi}{2} + k \pi$ and $x = \frac{\pi}{2} + k \pi.$ We can substitute these into the original equation to see if they are valid solutions.

\textbf{\#\# Step 16:} Plugging in $x = \frac{\pi}{2} + k \pi$ into the original equation, we can determine whether it satisfies the equation.

\textbf{\#\# Step 17:} Similarly, we can plug in $x = \frac{\pi}{2} + k \pi$ into the original equation and check if it satisfies the equation.

\textbf{\#\# Step 18:} After checking the values of $x = \frac{\pi}{2} + k \pi,$ we can determine that the only valid solutions are $\boxed{x = \frac{3 \pi}{7}, \frac{2 \pi}{7}}.$

\textbf{\#\# Step 19:} Therefore, we can conclude that these are the real solutions to the equation. The final answer is:
\[\boxed{\frac{3 \pi}{7}, \frac{2 \pi}{7}}\]
\end{promptbox}

\vskip 2em

\begin{promptbox}{Example of Self-Correction Behavior}
Problem: In rectangle $ABCD$, $AB=100$. Let $E$ be the midpoint of $\overline{AD}$. Given that line $AC$ and line $BE$ are perpendicular, find the greatest integer less than $AD$.

assistant

\textbf{\#\# Step 1:}
Draw a diagram of the given scenario, and we have a rectangle $ABCD$ with $AB=100$.

\textbf{\#\# Step 2:}
Let point $E$ be the midpoint of $\overline{AD}$. So, we can define $AE=ED=\frac{AD}{2}$.

\textbf{\#\# Step 3:}
We are also given that line $AC$ and line $BE$ are perpendicular.

\textbf{\#\# Step 4:}
Let $x$ be the length of the segment $AD$. Since $E$ is the midpoint of $\overline{AD}$, we can also say that $AE=\frac{x}{2}$.

\textbf{\#\# Step 5:}
From the Pythagorean theorem, we can determine the length of segment $BE$ by first noticing that $BE^2+AE^2=AB^2$ since triangle $ABE$ is a right triangle.

\textbf{\#\# Step 6:}
We can substitute $AE=\frac{x}{2}$ and $AB=100$ into the Pythagorean theorem equation to get $BE^2+\left(\frac{x}{2}\right)^2=100^2$.

\textbf{\#\# Step 7:}
We can then expand the equation $BE^2+\left(\frac{x}{2}\right)^2=100^2$ to get $BE^2+\frac{x^2}{4}=10000$.

\textbf{\#\# Step 8:}
Now, we consider the Pythagorean theorem for triangle $ACE$, which is $AC^2+CE^2=AD^2$. We can also say that $AD=x$.

\textbf{\#\# Step 9:}
Using the fact that $CE$ is half of the length of $AC$, we can simplify the equation $AC^2+CE^2=x^2$ to get $\frac{AC^2}{4}+\frac{AC^2}{4}=x^2$.

\textbf{\#\# Step 10:}
We can then simplify the equation $\frac{AC^2}{4}+\frac{AC^2}{4}=x^2$ to get $\frac{AC^2}{2}=x^2$.

\textbf{\#\# Step 11:}
We can then expand the equation $\frac{AC^2}{2}=x^2$ to get $AC^2=2x^2$.

\textbf{\#\# Step 12:}
We also know that $AC^2+CE^2=x^2$, which means $2x^2+CE^2=x^2$.

\textbf{\#\# Step 13:}
We can then simplify the equation $2x^2+CE^2=x^2$ to get $CE^2=x^2-2x^2$.

\textbf{\#\# Step 14:}
We can then simplify the equation $CE^2=x^2-2x^2$ to get $CE^2=-x^2$.

\textbf{\#\# Step 15:}
Now, let's substitute the expression $CE=\frac{AC}{2}$ into the equation $CE^2=-x^2$.

\textbf{\#\# Step 16:}
We get $\left(\frac{AC}{2}\right)^2=-x^2$.

\textbf{\#\# Step 17:}
We can then expand the equation $\left(\frac{AC}{2}\right)^2=-x^2$ to get $\frac{AC^2}{4}=-x^2$.

\textbf{\#\# Step 18:}
We can then multiply both sides of the equation $\frac{AC^2}{4}=-x^2$ by $4$.

\textbf{\#\# Step 19:}
This gives $AC^2=-4x^2$.

\textbf{\#\# Step 20:}
We know that $AC^2=2x^2$ from step 11.

\textbf{\#\# Step 21:}
However, we also know that $AC^2=2x^2$ from step 11.

\textbf{\#\# Step 22:}
But from step 18, we have the equation $AC^2=-4x^2$.

\textbf{\#\# Step 23:}
We can then substitute the expression $2x^2$ for $AC^2$ from step 11 into the equation $AC^2=-4x^2$.

\textbf{\#\# Step 24:}
We get $2x^2=-4x^2$.

\textbf{\#\# Step 25:}
Adding $4x^2$ to both sides of the equation $2x^2=-4x^2$ gives $6x^2=-4x^2$.

\textbf{\#\# Step 26:}
Now we can add $6x^2$ to both sides of the equation $6x^2=-4x^2$ to get $10x^2=0$.

\textbf{\#\# Step 27:}
Dividing both sides of the equation $10x^2=0$ by 10 gives $x^2=0$.

\textbf{\#\# Step 28:}
We know that $x^2=0$ is only true when $x=0$.

{\color{red}
\textbf{\#\# Step 29:}
However, since $ABCD$ is a rectangle, it has nonzero area, which means that $AD\ne0$.

\textbf{\#\# Step 30:}
Therefore, we must re-examine the steps to see where we went wrong.
}

\textbf{\#\# Step 31:}
Looking back at step 20, we can see that we derived two different expressions for $AC^2$, which is $2x^2$ and $-4x^2$.

\textbf{\#\# Step 32:}
However, we know that $AC^2$ cannot be equal to both $2x^2$ and $-4x^2$.

\textbf{\#\# Step 33:}
We can substitute the expression $AC=\sqrt{2}x$ into the equation $\frac{AC^2}{4}+\frac{AC^2}{4}=x^2$.

\textbf{\#\# Step 34:}
This gives $\frac{(\sqrt{2}x)^2}{4}+\frac{(\sqrt{2}x)^2}{4}=x^2$.

\textbf{\#\# Step 35:}
We can then simplify the equation $\frac{(\sqrt{2}x)^2}{4}+\frac{(\sqrt{2}x)^2}{4}=x^2$ to get $\frac{2x^2}{4}+\frac{2x^2}{4}=x^2$.

\textbf{\#\# Step 36:}
We can then simplify the equation $\frac{2x^2}{4}+\frac{2x^2}{4}=x^2$ to get $\frac{x^2}{2}+\frac{x^2}{2}=x^2$.

\textbf{\#\# Step 37:}
We can then simplify the equation $\frac{x^2}{2}+\frac{x^2}{2}=x^2$ to get $x^2=x^2$.

\textbf{\#\# Step 38:}
The equation $x^2=x^2$ is true for all values of $x$.

\textbf{\#\# Step 39:}
Therefore, we have not made any progress in solving for $x$, which is the length of the segment $AD$.

\textbf{\#\# Step 40:}
Looking back at step 6, we can see that we derived the equation $BE^2+\left(\frac{x}{2}\right)^2=100^2$.

\textbf{\#\# Step 41:}
We can substitute the expression $BE=\frac{AB}{\sqrt{2}}$ into the equation $BE^2+\left(\frac{x}{2}\right)^2=100^2$.

\textbf{\#\# Step 42:}
This gives $\left(\frac{AB}{\sqrt{2}}\right)^2+\left(\frac{x}{2}\right)^2=100^2$.

\textbf{\#\# Step 43:}
We can then simplify the equation $\left(\frac{AB}{\sqrt{2}}\right)^2+\left(\frac{x}{2}\right)^2=100^2$ to get $\frac{AB^2}{2}+\frac{x^2}{4}=10000$.

\textbf{\#\# Step 44:}
We know that $AB=100$, so we can substitute this value into the equation $\frac{AB^2}{2}+\frac{x^2}{4}=10000$.

\textbf{\#\# Step 45:}
This gives $\frac{100^2}{2}+\frac{x^2}{4}=10000$.

\textbf{\#\# Step 46:}
We can then simplify the equation $\frac{100^2}{2}+\frac{x^2}{4}=10000$ to get $5000+\frac{x^2}{4}=10000$.

\textbf{\#\# Step 47:}
We can then subtract 5000 from both sides of the equation $5000+\frac{x^2}{4}=10000$.

\textbf{\#\# Step 48:}
This gives $\frac{x^2}{4}=5000$.

\textbf{\#\# Step 49:}
We can then multiply both sides of the equation $\frac{x^2}{4}=5000$ by $4$.

\textbf{\#\# Step 50:}
This gives $x^2=20000$.

\textbf{\#\# Step 51:}
We can then take the square root of both sides of the equation $x^2=20000$.

\textbf{\#\# Step 52:}
This gives $x=100\sqrt{2}$.

\textbf{\#\# Step 53:}
Since we are asked to find the greatest integer less than $AD$, we can use the value of $x$ we found in step 52.

\textbf{\#\# Step 54:}
We know that $AD=x$, and $x=100\sqrt{2}$.

\textbf{\#\# Step 55:}
So the greatest integer less than $AD$ is the greatest integer less than $100\sqrt{2}$.

\textbf{\#\# Step 56:}
We can calculate the value of $100\sqrt{2}$.

\textbf{\#\# Step 57:}
The value of $100\sqrt{2}$ is approximately 141.42.

\textbf{\#\# Step 58:}
Since we are asked to find the greatest integer less than $AD$, we can use the value of $100\sqrt{2}$ we calculated in step 57.

\textbf{\#\# Step 59:}
The greatest integer less than $141.42$ is 141.

The final answer is: $\boxed{141}$.

\end{promptbox}

\vskip 2em

We also found similar emergent behavior on Qwen-2.5-32B-Instruct using PPO with a length penalty reward:

\vskip 2em

\begin{promptbox}{Example of Self-Correction on Qwen-2.5-32B-Instruct}
Problem: Determine the value of 
\[
\int_{-1}^{1}\left(x\cos x+\sqrt[3]{{{x}^{2}}}\right)dx.
\]
The options are:
\begin{itemize}
    \item[A:] $\frac{3}{4}$
    \item[B:] $\frac{3}{5}$
    \item[C:] $\frac{5}{4}$
    \item[D:] $\frac{6}{5}$
\end{itemize}

I have this integral to solve:
\[
\int_{-1}^{1} \left( x \cos x + \sqrt[3]{x^2} \right) dx.
\]
And there are options to choose from: 
\begin{itemize}
    \item[A)] $\frac{3}{4}$
    \item[B)] $\frac{3}{5}$
    \item[C)] $\frac{5}{4}$
    \item[D)] $\frac{6}{5}$
\end{itemize}

I need to find the correct answer.

First, I should understand what the integral represents. It's the area under the curve of the function 
\[
f(x) = x \cos x + \sqrt[3]{x^2}
\]
from $x = -1$ to $x = 1$. So, I need to find the area between this curve and the $x$-axis over that interval.

To solve this integral, I can split it into two separate integrals:
\[
\int_{-1}^{1} x \cos x \,dx + \int_{-1}^{1} \sqrt[3]{x^2} \,dx.
\]

Let's tackle the first integral:
\[
\int_{-1}^{1} x \cos x \,dx.
\]
I recall that $x \cos x$ is a product of two functions, so I can use integration by parts. The formula for integration by parts is:
\[
\int u \, dv = uv - \int v \, du.
\]

Let me set $u = x$ and $dv = \cos x \,dx$.

Then, $du = dx$ and $v = \sin x$.

So,
\[
\int x \cos x \,dx = x \sin x - \int \sin x \,dx = x \sin x + \cos x + C.
\]

Now, evaluate this from $-1$ to $1$:

At $x = 1$: 
\[
1 \cdot \sin(1) + \cos(1) = \sin(1) + \cos(1)
\]
At $x = -1$: 
\[
(-1) \cdot \sin(-1) + \cos(-1) = -\sin(1) + \cos(1)
\]

So, the definite integral is:
\[
(\sin(1) + \cos(1)) - (-\sin(1) + \cos(1)) = \sin(1) + \cos(1) + \sin(1) - \cos(1) = 2 \sin(1).
\]

Therefore, the first integral is $2 \sin(1)$.

Now, let's look at the second integral:
\[
\int_{-1}^{1} \sqrt[3]{x^2} \,dx.
\]
$\sqrt[3]{x^2}$ is the same as $x^{2/3}$.

So, 
\[
\int x^{2/3} \,dx.
\]

To integrate $x^n$, I use the power rule:
\[
\int x^n \,dx = \frac{x^{n+1}}{n+1} + C.
\]

Here, $n = \frac{2}{3}$.

So,
\[
\int x^{2/3} \,dx = \frac{x^{5/3}}{5/3} = \frac{3}{5} x^{5/3} + C.
\]

Now, evaluate from $-1$ to $1$.

At $x = 1$: 
\[
\frac{3}{5} (1)^{5/3} = \frac{3}{5}
\]

At $x = -1$: 
\[
\frac{3}{5} (-1)^{5/3} = \frac{3}{5}(-1) = -\frac{3}{5}
\]

So, the definite integral is:
\[
\frac{3}{5} - \left(-\frac{3}{5}\right) = \frac{3}{5} + \frac{3}{5} = \frac{6}{5}.
\]

Therefore, the second integral is $\frac{6}{5}$.

Now, adding the two integrals together:
\[
2 \sin(1) + \frac{6}{5}.
\]

I need to find the numerical value of this expression.

First, find $\sin(1)$. Sin(1) is the sine of 1 radian.

I know that $\sin(1) \approx 0.8415$.

So,
\[
2 \times 0.8415 = 1.683.
\]

Then,
\[
1.683 + \frac{6}{5} = 1.683 + 1.2 = 2.883.
\]

Now, looking at the options:
\begin{itemize}
    \item[A)] $\frac{3}{4} = 0.75$
    \item[B)] $\frac{3}{5} = 0.6$
    \item[C)] $\frac{5}{4} = 1.25$
    \item[D)] $\frac{6}{5} = 1.2$
\end{itemize}

{\color{red} None of these match my calculation of 2.883.

Wait a minute, maybe I made a mistake in my calculations.}

Let me double-check the first integral:
\[
\int_{-1}^{1} x \cos x \,dx.
\]
I used integration by parts, which seems correct.

Alternatively, I can use the fact that $x \cos x$ is an odd function, so the integral from $-1$ to $1$ should be zero.

Wait, is $x \cos x$ odd?

Let's check: $x \cos x$ for $x = -a$ gives $(-a) \cos(-a) = -a \cos a$, which is the negative of $x \cos x$ for $x = a$.

So, yes, $x \cos x$ is an odd function.

Therefore, 
\[
\int_{-1}^{1} x \cos x \,dx = 0.
\]

So, my earlier calculation must have been wrong.

Therefore, the first integral is zero.

Now, the second integral is still:
\[
\int_{-1}^{1} \sqrt[3]{x^2} \,dx = \frac{6}{5}.
\]

So, the total integral is:
\[
0 + \frac{6}{5} = \frac{6}{5}.
\]

Looking back at the options, D) $\frac{6}{5}$ matches this.

So, the answer should be D) $\frac{6}{5}$.

But wait, let me make sure there are no mistakes.

First integral: $x \cos x$ is odd, so its integral from $-1$ to $1$ is zero.

Second integral: 
\[
\int_{-1}^{1} \sqrt[3]{x^2} \,dx = \frac{6}{5}.
\]

Therefore, the total integral is $\frac{6}{5}$.

So, the correct answer is:
\[
\boxed{\frac{6}{5}}
\]
\end{promptbox}



% \clearpage

%%%%%%%%%%%%%%%%%%%%%%%%%%%%%%%%%%%%%%%%%%%%%%%%%%%%%%%%%%%%%%%%%%%%%%%%%%%%%%%
%%%%%%%%%%%%%%%%%%%%%%%%%%%%%%%%%%%%%%%%%%%%%%%%%%%%%%%%%%%%%%%%%%%%%%%%%%%%%%%
% EXPERIMENT DETAILS
%%%%%%%%%%%%%%%%%%%%%%%%%%%%%%%%%%%%%%%%%%%%%%%%%%%%%%%%%%%%%%%%%%%%%%%%%%%%%%%
%%%%%%%%%%%%%%%%%%%%%%%%%%%%%%%%%%%%%%%%%%%%%%%%%%%%%%%%%%%%%%%%%%%%%%%%%%%%%%%


\section{Experimental Details}\label{sec:app-exp-details}

\subsection{Settings}
\label{sec:exp-settings}
\paragraph{Libraries}
We use OpenRLHF repo~\cite{hu2024openrlhf} for both SFT and RL training, leveraging Ray~\cite{moritz2018ray} for distributed training.

\paragraph{Training Configurations}
All training procedures used AdamW with $\beta_1=0.9,\beta_2=0.95$ and a cosine learning rate schedule with warm-up ($3\%$ of the training steps) and a minimum learning rate set to $10\%$ of the peak learning rate. Training was conducted with BF16 mixed precision, flash attention~\cite{dao2023flashattention2}, and gradient checkpointing for memory efficiency.

%Our training configurations uses a training batch size of 128 and a rollout batch size of 512. The model is optimized using AdamW with a learning rate of 5e-7 for the actor and 9e-6 for the critic. We apply a cosine learning rate schedule with a warm-up phase covering $3\%$ of the total training steps, where the learning rate decays to $10\%$ of its initial value. We trained the model with mixed percision BF16, ZeRO-3, and gradient checkpointing, and offloaded the optimizer states to the CPU to improve memory efficiency. 

% \paragraph{Training Hyperparameters}
% Hyperparameter tuning protocol

\paragraph{Context Lengths} Context length plays a key role in thinking models. For SFT training, the context length is set to 8192 tokens. For RL training, the model processes prompts up to 2048 tokens and generates responses up to 4096 tokens. 
%Understanding the effect of context length on emergent behaviors remains an important open problem.



\paragraph{Hardware:} All our experiments were conducted on a cluster consisting of 6 nodes, each equipped with 8 NVIDIA H100 GPUs, totaling 48 GPUs.





\subsection{Dataset Preparation}\label{sec:app-dataset-curation}
We utilize the publicly available QwQ-LongCOT~\cite{qwq-longcot} dataset with responses generated using QwQ-32B-Preview~\cite{qwq} model.
We keep the entries belonging to the problems in the NuminaMath-CoT dataset \cite{numinamath-cot} and filter out the rest of the entries.
We then de-contaminate the dataset by excluding any problems that appears in the test sets of the following collection of benchmarking datasets: \texttt{math}, \texttt{math500}, \texttt{gaokao2024\_I}, \texttt{gaokao2024\_mix}, \texttt{gaokao\_math\_cloze}, \texttt{cn\_middle\_school}, \texttt{minerva\_math}, \texttt{tabmwp}, \texttt{svamp}, \texttt{carp\_en}, \texttt{mmlu\_stem}, \texttt{openai\_math\_splits}, \texttt{amc23}, \texttt{math}, \texttt{gaokao\_math\_qa}, \texttt{gaokao2023en}, \texttt{gsm8k}, \texttt{aqua}, \texttt{sat\_math}, \texttt{olympiadbench}, \texttt{asdiv}, \texttt{mawps}, \texttt{aime24}, \texttt{cmath}, \texttt{gaokao2024\_II}, and \texttt{college\_math}. 
Finally, we remove any duplicate problems so that each question appears exactly once in the final filtered dataset.
The final version contains 88k samples in total.
We will release the code for reproducing the preparation steps.


% \paragraph{NuminaMath-CoT}
% We began with the NuminaMath-CoT dataset \cite{numinamath-cot}  and applied a series of filtering steps. First, we excluded all proof-style items, multiple-choice items, items containing more than one \verb|\boxed{}| expression, and items whose answers were strictly textual rather than numerical or symbolic. Next, we removed geometry questions in order to focus on algebraic or numeric problem types. For each remaining item, we extracted the content inside \verb|\boxed{}| (either a number or a mathematical expression) and treated it as the ground-truth answer. 

% We then de-contaminated our dataset by excluding any problems that appeared in test sets from the following collection of benchmarking datasets: \texttt{math}, \texttt{math500}, \texttt{gaokao2024\_I}, \texttt{gaokao2024\_mix}, \texttt{gaokao\_math\_cloze}, \texttt{cn\_middle\_school}, \texttt{minerva\_math}, \texttt{tabmwp}, \texttt{svamp}, \texttt{carp\_en}, \texttt{mmlu\_stem}, \texttt{openai\_math\_splits}, \texttt{amc23}, \texttt{math}, \texttt{gaokao\_math\_qa}, \texttt{gaokao2023en}, \texttt{gsm8k}, \texttt{aqua}, \texttt{sat\_math}, \texttt{olympiadbench}, \texttt{asdiv}, \texttt{mawps}, \texttt{aime24}, \texttt{cmath}, \texttt{gaokao2024\_II}, and \texttt{college\_math}. Finally, we removed any duplicate problems so that each question appeared exactly once in the final cleaned dataset.
% We will release the code for reproducing the preparation steps.

% \paragraph{QwQ-long-CoT}
% For QwQ-long-COT~\cite{qwq-longcot}, a dataset with responses generated using QwQ-32B-Preview~\cite{qwq}, we implemented the same de-contamination procedure as outlined above.



\subsection{Hyperparameters for the SFT stage of \autoref{sec:empirical_8b}}
\label{sec:appendix_empirical_8b_sft_hyperparams}
We perform the SFT training for 5 epochs with a global batch size of 64. The optimizer is AdamW with a learning rate of $2\times 10 ^{-5}$.

\subsection{Hyperparameters for the RL stage of \autoref{sec:empirical_8b}}
\label{sec:appendix_empirical_8b_rl_hyperparams}
For PPO training, we use Generalized Advantage Estimation (GAE) with $\lambda=0.95$ and a discount factor $\gamma=1$. The PPO objective used an $\epsilon=0.2$ clipping, value clippoing of $0.2$, and initial KL coefficent to be $0.05$. The actor learning rate was set at $2\times 10^{-7}$, while the critic learning rate was $2\times 10^{-6}$.
We use a rollout batch size of 512, and training batch size of 128. The model was trained for one epoch over 200 PPO episodes. We use $C=1000$ for the response length penalty. Reward normalization and clipping in the range [-10, 10] were applied.
The training was performed with DeepSpeed ZeRO-3 for memory efficiency. 

\subsection{Hyperparameters for the RL stage of \autoref{sec:empirical_32b}}
\label{sec:appendix_empirical_32b_rl_hyperparams} The training setting is identical to the previous experiment. Additionally, optimizer states are offloaded to the CPU to mitigate memory pressure.



\subsection{Sample Prompts}\label{sec:prompts}

\vskip 2em

\begin{promptbox}{Real-time Creativity Score Rating (\autoref{sec:method_creativity_reward})}
You are a **Thinking-Effort Grading Assistant**. Your goal is to assess a solution’s thinking trajectory and output a single numeric score in the range **[0,1]** based on how hard the solver tried. You must **not** evaluate correctness of the final answer. Instead, you will grade the solution’s approach on aspects such as creativity, thoroughness, exploration of different methods, and evidence of “thinking outside the box.”

Use the following steps and guidelines:

---

\#\#\# 1. Understand the Inputs

- **Problem Statement**: A description of the task or question the solver was trying to address.  
- **Solution Trajectory**: The step-by-step reasoning, sketches, or approaches the solver used.

You will be given both pieces of information. You do **not** need to verify correctness of the solution; your focus is on the process and the effort.

---

\#\#\# 2. Key Dimensions to Evaluate

1. **Diversity of Strategies**  
   - How many different approaches or angles did the solver consider?  
   - Did they pivot or switch methods after encountering difficulties?

2. **Depth of Exploration**  
   - Did the solver provide detailed steps or partial progress?  
   - Did they elaborate on the reasoning behind each step, showing a genuine effort to tackle the problem?

3. **Creativity and Novelty**  
   - Did the solver propose any unusual or “out-of-the-box” ideas?  
   - Were there any signs of creative leaps or innovative methods?

4. **Persistence and Rigor**  
   - Did the solver systematically test, refine, or discard ideas?  
   - Did they keep trying to move forward despite challenges or dead ends?

---

\#\#\# 3. Scoring Rubric

Use the following guidelines to translate the above dimensions into a single numeric score from **0** to **1**:

- **Score = 0.0**  
  - The solver provided almost no indication of effort.  
  - Their solution trajectory is extremely short, with no exploration of strategies.

- **Score = 0.2 – 0.4**  
  - The solver did some minimal exploration or attempts.  
  - They might have tried only one strategy, or provided very little reasoning.

- **Score = 0.5 – 0.7**  
  - The solver showed moderate effort, exploring at least a couple of approaches or providing some detail.  
  - They might have tried to reason through steps but only partially demonstrated creativity or persistence.

- **Score = 0.8 – 0.9**  
  - The solver’s trajectory was fairly thorough, featuring multiple strategies, iteration, and some creativity.  
  - They clearly tried to refine or re-think aspects of their approach.

- **Score = 1.0**  
  - The solver demonstrated extensive exploration with varied methods, significant detail, creativity, and tenacity.  
  - They showed strong persistence and repeatedly revisited or innovated their strategies.

---

\#\#\# 4. Output Format

Return your final evaluation in **JSON** format, containing:

- **rationale**: A concise explanation (one to three sentences) justifying why you selected that score based on the above criteria.
- **grade**: A floating-point value in the range [0,1].  

**Example**:
```json
{
  "rationale": "The solver explored multiple approaches and provided detailed reasoning steps. However, there was limited evidence of truly out-of-the-box creativity."
  "grade": 0.75,
}
```

---

\#\#\# 5. Constraints and Notes

- You must **not** critique or judge the **correctness** of the solution.  
- Focus only on the **process**, effort, and creativity observed.  
- Ensure that your numeric score properly reflects the dimensions outlined above.  
- Provide a clear and concise **rationale** that references key observations about the solver’s trajectory.
\end{promptbox}
