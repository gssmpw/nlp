\section{Experimental Results: Real-World}

In this section, we evaluate a student policy trained using simulated data in real-world environments.
We design experiments to address (1) What is the performance of \ourmethod~when operating over seen and unseen objects in the real-world? (2) Does the performance of \ourmethod~improve when conditioned on a desired grasp pose?


\begin{figure}[t]
    \centering
    \includegraphics[width=0.9\linewidth]{figures/real_objects.pdf}
    \caption{We select several objects with differing sizes and weights that necessitate occluded grasping to evaluate \ourmethod~in the real-world environment.}
    \label{fig:real_objects}
\end{figure}

\begin{table}[t]
    \centering
    \begin{tabular}{c| c | c }
    \toprule
     & \ourmethod & \ourmethod \\
     &  & w/o grasp pose \\
    \hline
    Cuboid-Medium-Heavy (Seen)  & $80\% \ (8/10)$ & $80\% \ (8/10)$ \\
    \rowcolor{gray!20}
    Cuboid-Large-Light  & $90\% \ (9/10)$ & $80\% \ (8/10)$ \\
    Cuboid-Small-Heavy  & $50\% \ (5/10)$ & $60\% (6/10)$ \\
    \rowcolor{gray!20}
    Keyboard  & $80\% \ (8/10)$ & $40\% (4/10)$ \\
    Bag  & $80\% \ (8/10)$ & $80\% \ (8/10)$ \\
    Round-Large-Light  & $30\% \ (3/10)$ & $10\% \ (1/10)$ \\
    \hline
    \rowcolor{gray!20}
    Average & $\mathbf{68.3\% \ (41/60)}$ & $58.3\% \ (35/60)$ \\
    \end{tabular}
    \caption{Performance of \ourmethod~in real-world environments for seen and unseen objects with varying shapes, sizes, and weights.}
    \label{tab:real_world}
\end{table}


\subsection{Experiment Setup}
In this experiment, the student policies are evaluated using both seen and unseen objects with varying shapes, weights, and sizes, as illustrated in Figure~\ref{fig:real_objects}.
To facilitate grasping, we scan the objects to reconstruct their 3D meshes and employ antipodal sampling to generate desired grasp poses. 
This avoids reliance on off-the-shelf grasp pose prediction models~\cite{mousavian20196}, as addressing the inaccuracies of such models is beyond the scope of our evaluation.
Nonetheless, \ourmethod~is compatible with any grasp pose prediction model that takes a segmented object point cloud as input.

When the student policies are conditioned on the desired grasp poses, we estimate the object pose and thus can infer the desired grasp pose during manipulation in real-time, similar to prior work~\cite{wan2023unidexgraspimprovingdexterousgrasping}.
To achieve this, FoundationPose~\cite{wen2024foundationpose} is used to track object pose.
Additioanlly, SAMTrack~\cite{cheng2023segment} is used to generate a segmentation mask to extract the target object point cloud used as input for the policies.


After generating a constraint pose using the student constraint policy, the desired joint positions for the right arm are obtained using the IK solver in CuRobo~\cite{sundaralingam2023curobo}.
Then, MoveIt~\cite{Coleman2014ReducingTB} is used to control the right arm to the desired positions.
After the right arm is positioned in the constraint pose, the left arm begins the student grasping policy rollout.



\subsection{Results}
Table~\ref{tab:real_world} shows the performance of the student policy in the real world.
\ourmethod~effectively tackles occluded grasping tasks for both seen and unseen objects. 
Nonetheless, it encounters challenges with the round box, which is particularly difficult to stabilize in real-world conditions. 
While performance shows a slight decline, \ourmethod~still achieves a reasonable level of success, even without the target grasp pose as input.
\ourmethod, when operating without the target grasp pose, struggles to recover from failed non-prehensile manipulation attempts. 
When an initial push fails to move the object as intended, it is unable to retract and retry the push. 
For instance, pushing a keyboard towards a constraint is particularly challenging due to its thin profile, resulting in a low success rate of only $40\%$, as the left arm often fails to effectively position the keyboard.

In contrast, incorporating the desired grasp pose enables \ourmethod~to recover from such failures by reattempting the task until the left arm successfully reorients and completes the grasp. 
This demonstrates the importance of the desired grasp pose as a critical input for guiding the policy and improving performance.
However, the version of \ourmethod~without the desired grasp pose offers increased flexibility, as it eliminates the need for real-time object pose estimation. 
This trade-off makes it more practical for scenarios where tracking the target grasp pose is infeasible.
