\section{Related work}
When the underlying causal graph is unknown, a natural idea is to first learn the underlying graph via interventions and then apply causal bandit algorithms. But this approach is not efficient because recovering the full causal graph is not necessary for identifying the optimal arm. Lu et al. \cite{lu2021causal} proposed a central node - upper confidence bound (CN-UCB) algorithm to find the atomic intervention with the largest reward, assuming knowledge of observational probabilities instead of the full causal graph. Although observational data are often more accessible, % than experimental data in terms of costs, 
it is not always feasible to obtain enough observational data for accurate estimation of the observational distributions. Additionally, the domain set for all variables except the reward is assumed to be finite and discrete, in which case the exploration of arms can be restricted to the parents of the reward variable.

De Kroon et al. \cite{dekroon2022causal} tackled the problem for discrete and Gaussian models without knowledge of the underlying graph. But their method relies on the existence of a set of non-intervenable variables that d-separate the interventions and the reward variable. And no theoretical guarantee on the cumulative regret was provided. Huang and Zhou \cite{Jireh2022BBB} developed a Bayesian framework that utilizes joint inferences from experimental %\qzcmt{it might be better to call it experimental data (generated under interventions)} 
and observational data without additional assumptions on the underlying graph structure. But the Bayesian framework requires updating parent set posterior probabilities, which is extremely computationally challenging and practically infeasible even when $K$ is moderate.