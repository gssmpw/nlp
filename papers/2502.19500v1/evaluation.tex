
\begin{figure*}[!t]
    \centering
    %\includegraphics[scale=0.2]{icml2024/figs/financial_1.png}
    %\includegraphics[scale=0.07]{images/crossfit_1.png}
    %\includegraphics[scale=0.07]{images/crossfit_2.png}
    %\includegraphics[scale=0.08]{images/crossfit_3.png}
    %\includegraphics[scale=0.07]{images/crossfit_4.png}
    \includegraphics[scale=0.07]{images/crossfit_5.png}
    \includegraphics[scale=0.07]{images/crossfit_6.png}
    \includegraphics[scale=0.07]{images/crossfit_7.png}
    %\includegraphics[scale=0.08]{images/crossfit_8.png}
    %\includegraphics[scale=0.2]{icml2024/figs/financial_4.png}
    %\includegraphics[scale=0.2]{icml2024/figs/financial_5.png}
    \caption{Conversational Coaching for crossfit fitness journey}
    \label{fig:fitness_journey}
\end{figure*}
We performed a qualitative evaluation and found that our proposed framework is powerful in providing interactive planning for user goals across a range of domains, including learning skills, pursuing DIY projects, hobbies, fitness, health, aspirational goals, and many more nuanced personal journeys. Here, we show examples from two important real-world domains: learning \cite{kizilcec2020scaling} and health \cite{davenport2019potential}.  

\noindent \textbf{Conversational Tutoring for a Learning Goal}
Figure \ref{fig:inventors_flow} demonstrates an example of a user-agent interaction in which the user asks for a learning plan: \emph{"How do I explain to my kids about inventors?"} We can see that given the initial query, the meta-controller decided to add three new steps with names \emph{Start with stories}, \emph{Use everyday examples}, and \emph{Visit museums or science centers}. Each of these steps is structured, as executed by sub-policy of \texttt{add-steps}, generating the aforementioned names, a description of why this step is useful, and search keywords. The search keywords are then used by the low-level tool-use policies to fetch and rank content from \texttt{SEARCH}, and \texttt{RECOMMEND-ENGINE}. The content includes animation-based videos and articles regarding children's books on inventions. Additionally, the sub-policy generated follow-up questions per each plan step, as shown at the bottom. The user can interact via either answering one of the provided questions (e.g., Q: \emph{"What's your kids favorite invention?"}, A: \emph{"They're really interested in computers and tablets"}), or by writing their own natural language feedback, i.e., \emph{"I would love to introduce them to female inventors too."}. This feedback becomes part of the context of the agent across all levels of the hierarchical policy, and leads to a plan adaptation where new steps on \emph{Investigate modern inventions}, \emph{Explore the history of the tablet}, \emph{Introduce Charles Babbage} are added to the plan. The plan is further adapted based on the user's free-form feedback on female inventors to include steps on \emph{Ada Lovelace}, \emph{Grace Hopper}, and \emph{Hady Lamarr}. The figure illustrates that the LLM-based policies  fetch relevant content and generate appropriate follow-up questions per step to allow the user to propel their journey forward.  

\noindent \textbf{Conversational Coaching for Personal Health}
Similarly, Figure \ref{fig:fitness_journey} demonstrates an example of a user-agent interaction regarding a personal health journey in the domain of fitness---\emph{"I want to do crossfit"}. The hierarchical agent again  initially decided to \texttt{add-steps}, which then the CoT-prompted LLM-based sub-policy mapped to three steps in the plan: \emph{Learn the basics of crossfit}, \emph{Assess your current fitness level}, and \emph{Set realistic goals}. The tool-enhanced low-level policy then fetched and ranked content for each of the steps. The user can  interact with the sub-policy generated follow-up questions or write their own free-form feedback to further adapt the plan. Here, the user answered the question \emph{"What are your fitness goals?"}, by saying \emph{"I would like to improve my cardiovascular health."} Then, the meta-controller decided to \texttt{alter-step} with argument \emph{Set Realistic goals}. Given the meta-controller's decision, the corresponding sub-policy $\pi_{\texttt{alter-step}}$ executed the macro-action and altered the step, and the low-level policy used the updated search keywords to update the step.% to reflect the user's goals.   