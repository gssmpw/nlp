\documentclass{article}

\usepackage{microtype}
\usepackage[margin=1in]{geometry}
\usepackage{geometry}
\usepackage{graphicx}
\usepackage{subcaption}
\usepackage{booktabs} %
\usepackage{mathpazo}
\usepackage{pifont}
\usepackage{makecell}
\usepackage{tabularx}
\usepackage{pbox}
\usepackage{soul}

\usepackage[dvipsnames]{xcolor}
\usepackage{pgfplots}
\usepackage{pgfplotstable}
\pgfplotsset{compat=1.3}
\usepackage{tikz}
\usetikzlibrary{pgfplots.groupplots}
\usepackage{ragged2e}

\usepackage[toc,page,header]{appendix}
\usepackage{minitoc}
\renewcommand \thepart{}
\renewcommand \partname{}

\usepackage[colorlinks=true,linkcolor=blue,urlcolor=blue,citecolor=blue]{hyperref}
\usepackage{xspace}
\usepackage{bm}
\usepackage{comment}
\usepackage{caption}
\usepackage{makecell}

\captionsetup[table]{position=bottom}
\usepackage[natbib=true,style=alphabetic,minalphanames=3,maxbibnames=99,maxcitenames=2]{biblatex}
\addbibresource{biblio/bib.bib}

\newcommand{\norm}[1]{\left\lVert#1\right\rVert}
\newcommand{\bx}{\textbf{x}}
\newcommand{\bz}{\textbf{z}}
\newcommand{\beps}{\bm{\varepsilon}}
\newcommand{\trak}{\textsc{trak}\xspace}

\newcommand{\tracin}{TracIn\xspace}
\newcommand{\tracincp}{TracInCP\xspace}
\newcommand{\clip}{\textsc{CLIP}\xspace}
\newcommand{\bert}{\textsc{BERT}\xspace}
\newcommand{\bertbase}{\textsc{BERT-base}\xspace}
\newcommand{\cifarten}{\textsc{CIFAR-10}\xspace}
\newcommand{\cifartwo}{\textsc{CIFAR-2}\xspace}
\newcommand{\cifar}{\textsc{CIFAR}\xspace}
\newcommand{\mscoco}{\textsc{MS COCO}\xspace}
\newcommand{\gas}{\textsc{GAS}\xspace}
\newcommand{\xmark}{\ding{55}}%

\newcommand{\spelledout}{Tracing with the Randomly-projected After Kernel}
\newcommand{\thetastar}{\theta^\star}
\newcommand{\thetahat}{\widehat{\theta}}
\newcommand\TODO{\textcolor{red}{[TODO]}}
\newcommand\supp{\textsc{Support}}
\newcommand{\lr}[1]{\left({#1}\right)}
\newcommand\reinfl[2]{\textsc{Greenies}_p\left[#1 \to #2\right]}

\renewcommand{\hat}[1]{\widehat{#1}}
\newcommand{\ind}[1]{\bm{1}\{{#1}\}}
\newcommand{\modeleval}[2]{f({#1};{#2})}
\newcommand{\modelevalbin}[2]{f_{\text{bin}}({#1};{#2})}
\newcommand{\modelevalbinempty}{f_{\text{bin}}}
\newcommand{\modelevalmc}[2]{f_{\text{mc}}({#1};{#2})}
\newcommand{\modelevalmcempty}{f_{\text{bin}}}
\newcommand{\approxmodeleval}[2]{\hat{f}({#1};{#2})}
\newcommand{\estmodeleval}[2]{\widehat{f}_{\mathcal{A}}({#1};{#2})}
\newcommand{\mask}[1]{\bm{1}_{#1}}
\newcommand{\parmap}[1]{g_{\theta}(#1)}
\newcommand{\grad}{g}
\newcommand{\projgrad}{\phi}
\newcommand{\loss}[1]{L(#1; \theta)}
\newcommand{\lossstar}[1]{L(#1; \thetastar)}
\newcommand{\imagemb}[1]{\phi(#1; \theta)}
\newcommand{\textemb}[1]{\psi(#1; \theta)}

\newlength\myindent
\setlength\myindent{2em}
\newcommand\bindent{%
  \begingroup
  \setlength{\itemindent}{\myindent}
  \addtolength{\algorithmicindent}{\myindent}
}
\newcommand\eindent{\endgroup}

\usepackage[normalem]{ulem}
\newcommand{\edit}[1]{\textcolor{ForestGreen}{#1}}
\newcommand{\eout}[1]{\edit{\sout{#1}}}
\newcommand{\meta}[1]{\textcolor{BrickRed}{#1}}


\usepackage{amsmath}
\usepackage{amssymb}
\usepackage{mathtools}
\usepackage{amsthm}
\usepackage{thmtools}
\usepackage{algorithm}
\usepackage{algpseudocode}
\usepackage[T1]{fontenc}
\usepackage{color}
\usepackage[most,skins]{tcolorbox}
\usepackage{varwidth}
\usepackage{adjustbox}
\tcbuselibrary{skins, breakable}

\usepackage{float}
\usepackage{tablefootnote}
\usepackage{changepage}
\usepackage{array}
\usepackage{multirow}
\usepackage{colortbl}

\newcolumntype{R}[2]{%
    >{\adjustbox{angle=#1,lap=\width-(#2)}\bgroup}%
    l%
    <{\egroup}%
}
\newcommand{\rot}[1]{\multicolumn{1}{R{60}{0cm}}{#1}}

\usepackage[capitalize,noabbrev]{cleveref}

\definecolor{mydarkblue}{rgb}{0,0.08,0.85}
\definecolor{mylightblue}{rgb}{0.06,0.56,1.0}
\definecolor{mylightorange}{rgb}{1.0,0.62,0.12}
\definecolor{mylightred}{rgb}{0.99,0.00,0.04}
\hypersetup{colorlinks=true,
linkcolor=mydarkblue,
citecolor=mydarkblue,
filecolor=mydarkblue,
urlcolor=mydarkblue,
bookmarksopen=true,
linktoc=page}

\theoremstyle{plain}
\newtheorem{theorem}{Theorem}[section]
\newtheorem{proposition}[theorem]{Proposition}
\newtheorem{lemma}[theorem]{Lemma}
\newtheorem{corollary}[theorem]{Corollary}
\newtheorem{remark}[theorem]{Remark}
\theoremstyle{definition}
\newtheorem{definition}[theorem]{Definition}
\newtheorem{assumption}[theorem]{Assumption}
\newtheorem{example}[theorem]{Example}
\theoremstyle{remark}

\usepackage{marginnote}
\newcommand{\oftodo}[2]{\marginnote{\color{red} #2}[#1]}

\usepackage{todonotes}
\newcommand{\aleks}[1]{\todo{Aleks: #1}}
\newcommand{\josh}[1]{\todo{Josh: #1}}


\title{Do Large Language Model Benchmarks Test Reliability?}
\author{%
  Joshua Vendrow\thanks{Equal contribution. \textdagger Equal advising.} , Edward Vendrow\footnotemark[1] , Sara Beery\textsuperscript{\textdagger} , Aleksander M\k{a}dry\textsuperscript{\textdagger} \\
  MIT \\
  \texttt{\{jvendrow,evendrow,beery,madry\}@mit.edu} \\
}

\date{}

\makeatletter
\let\c@figure\c@table
\makeatother

\begin{document}
\setcounter{tocdepth}{2}
\doparttoc %
\renewcommand\ptctitle{}
\faketableofcontents %

\maketitle
\begin{abstract}
  \begin{abstract}  
Test time scaling is currently one of the most active research areas that shows promise after training time scaling has reached its limits.
Deep-thinking (DT) models are a class of recurrent models that can perform easy-to-hard generalization by assigning more compute to harder test samples.
However, due to their inability to determine the complexity of a test sample, DT models have to use a large amount of computation for both easy and hard test samples.
Excessive test time computation is wasteful and can cause the ``overthinking'' problem where more test time computation leads to worse results.
In this paper, we introduce a test time training method for determining the optimal amount of computation needed for each sample during test time.
We also propose Conv-LiGRU, a novel recurrent architecture for efficient and robust visual reasoning. 
Extensive experiments demonstrate that Conv-LiGRU is more stable than DT, effectively mitigates the ``overthinking'' phenomenon, and achieves superior accuracy.
\end{abstract}  
\end{abstract}

\section{Introduction}
\label{sec:intro}
\section{Introduction}
Backdoor attacks pose a concealed yet profound security risk to machine learning (ML) models, for which the adversaries can inject a stealth backdoor into the model during training, enabling them to illicitly control the model's output upon encountering predefined inputs. These attacks can even occur without the knowledge of developers or end-users, thereby undermining the trust in ML systems. As ML becomes more deeply embedded in critical sectors like finance, healthcare, and autonomous driving \citep{he2016deep, liu2020computing, tournier2019mrtrix3, adjabi2020past}, the potential damage from backdoor attacks grows, underscoring the emergency for developing robust defense mechanisms against backdoor attacks.

To address the threat of backdoor attacks, researchers have developed a variety of strategies \cite{liu2018fine,wu2021adversarial,wang2019neural,zeng2022adversarial,zhu2023neural,Zhu_2023_ICCV, wei2024shared,wei2024d3}, aimed at purifying backdoors within victim models. These methods are designed to integrate with current deployment workflows seamlessly and have demonstrated significant success in mitigating the effects of backdoor triggers \cite{wubackdoorbench, wu2023defenses, wu2024backdoorbench,dunnett2024countering}.  However, most state-of-the-art (SOTA) backdoor purification methods operate under the assumption that a small clean dataset, often referred to as \textbf{auxiliary dataset}, is available for purification. Such an assumption poses practical challenges, especially in scenarios where data is scarce. To tackle this challenge, efforts have been made to reduce the size of the required auxiliary dataset~\cite{chai2022oneshot,li2023reconstructive, Zhu_2023_ICCV} and even explore dataset-free purification techniques~\cite{zheng2022data,hong2023revisiting,lin2024fusing}. Although these approaches offer some improvements, recent evaluations \cite{dunnett2024countering, wu2024backdoorbench} continue to highlight the importance of sufficient auxiliary data for achieving robust defenses against backdoor attacks.

While significant progress has been made in reducing the size of auxiliary datasets, an equally critical yet underexplored question remains: \emph{how does the nature of the auxiliary dataset affect purification effectiveness?} In  real-world  applications, auxiliary datasets can vary widely, encompassing in-distribution data, synthetic data, or external data from different sources. Understanding how each type of auxiliary dataset influences the purification effectiveness is vital for selecting or constructing the most suitable auxiliary dataset and the corresponding technique. For instance, when multiple datasets are available, understanding how different datasets contribute to purification can guide defenders in selecting or crafting the most appropriate dataset. Conversely, when only limited auxiliary data is accessible, knowing which purification technique works best under those constraints is critical. Therefore, there is an urgent need for a thorough investigation into the impact of auxiliary datasets on purification effectiveness to guide defenders in  enhancing the security of ML systems. 

In this paper, we systematically investigate the critical role of auxiliary datasets in backdoor purification, aiming to bridge the gap between idealized and practical purification scenarios.  Specifically, we first construct a diverse set of auxiliary datasets to emulate real-world conditions, as summarized in Table~\ref{overall}. These datasets include in-distribution data, synthetic data, and external data from other sources. Through an evaluation of SOTA backdoor purification methods across these datasets, we uncover several critical insights: \textbf{1)} In-distribution datasets, particularly those carefully filtered from the original training data of the victim model, effectively preserve the model’s utility for its intended tasks but may fall short in eliminating backdoors. \textbf{2)} Incorporating OOD datasets can help the model forget backdoors but also bring the risk of forgetting critical learned knowledge, significantly degrading its overall performance. Building on these findings, we propose Guided Input Calibration (GIC), a novel technique that enhances backdoor purification by adaptively transforming auxiliary data to better align with the victim model’s learned representations. By leveraging the victim model itself to guide this transformation, GIC optimizes the purification process, striking a balance between preserving model utility and mitigating backdoor threats. Extensive experiments demonstrate that GIC significantly improves the effectiveness of backdoor purification across diverse auxiliary datasets, providing a practical and robust defense solution.

Our main contributions are threefold:
\textbf{1) Impact analysis of auxiliary datasets:} We take the \textbf{first step}  in systematically investigating how different types of auxiliary datasets influence backdoor purification effectiveness. Our findings provide novel insights and serve as a foundation for future research on optimizing dataset selection and construction for enhanced backdoor defense.
%
\textbf{2) Compilation and evaluation of diverse auxiliary datasets:}  We have compiled and rigorously evaluated a diverse set of auxiliary datasets using SOTA purification methods, making our datasets and code publicly available to facilitate and support future research on practical backdoor defense strategies.
%
\textbf{3) Introduction of GIC:} We introduce GIC, the \textbf{first} dedicated solution designed to align auxiliary datasets with the model’s learned representations, significantly enhancing backdoor mitigation across various dataset types. Our approach sets a new benchmark for practical and effective backdoor defense.



\section{Cleaning Up Noisy Benchmarks}
\label{sec:cleaning}
\begin{figure*}[h]
\centering

\begin{tcolorbox}[colback=gray!4, colframe=gray!50, arc=2mm, boxrule=0.5pt,left=9pt,right=9pt]
\small
    
\begin{minipage}{0.48\textwidth}
\textbf{(a)} Mislabeled question, \textit{SVAMP}
\begin{tcolorbox}[colback=white, colframe=gray!50, arc=2mm, boxrule=0.5pt]
        \textbf{Question}: You had 14 bags with equal number of cookies. If you had 28 cookies and 86 candies in total, how many bags of cookies do you have?\\
\textbf{Solution}: 2
\end{tcolorbox}
\hspace{1.4em}\textcolor{red}{There are 14 bags, not 2.}
\end{minipage}
\hfill
\begin{minipage}{0.48\textwidth}
\textbf{(b)} Logical contradiction, \textit{GSM8K}
    \begin{tcolorbox}[colback=white, colframe=gray!50, arc=2mm, boxrule=0.5pt]
Ten stalls have 20 cows each. Mr. Sylas buys 40 cows and divides them equally, putting an equal number of the new cows into each of the twenty stalls. How many cows are in 8 of the stalls?
    \end{tcolorbox}
    \hspace{1.4em}\textcolor{red}{There are both ten and twenty stalls.}
\end{minipage}
\vspace{1em}

\begin{minipage}{0.45\textwidth}
\textbf{(c)} Ambiguity, \textit{VQA v2.0}
\begin{tcolorbox}[colback=white, colframe=gray!50, arc=2mm, boxrule=0.5pt]
\begin{minipage}{0.46\textwidth}
\includegraphics[width=1\textwidth]{figures/vqa_ex.png}
\end{minipage}
\hfill
\begin{minipage}{0.42\textwidth}
\textbf{Question:} Does the baby have socks on?\footnotemark\\
\end{minipage}

\end{tcolorbox}
\hspace{1.4em}\textcolor{red}{There is no way to tell.}
\end{minipage}
\hfill
\begin{minipage}{0.48\textwidth}
\textbf{(d)} Clear flaw / ill-posed, \textit{MMLU HS Math}
    \begin{tcolorbox}[colback=white, colframe=gray!50, arc=2mm, boxrule=0.5pt]
A curve is given parametrically by the equations

Options:\\
A) $\pi/2$\quad
B) $\pi$\quad
C) $2 + \pi$\quad
D) $2\pi$
    \end{tcolorbox}
\hspace{1.4em}\textcolor{red}{The equations for the curve are missing.}
\end{minipage}
\end{tcolorbox}
    \caption{\textbf{Examples of errors in current LLM benchmarks.} \textbf{(a)} For mislabeled questions, we fix the solutions and include the re-labeled examples in our benchmark. We find three common categories of ``bad'' questions: \textbf{(b)} there is a logical contradiction in the problem statement, \textbf{(c)} there is ambiguity leading to many plausible solutions, or \textbf{(d)} there is a clear flaw in the construction of the question, such as missing specifications. We remove such questions when revising benchmarks to make them platinum.}
    \label{fig:benchmark_errors}
\end{figure*}




To make a noisy benchmark ``platinum,'' we need to update it to remove or correct label errors. In this section, we categorize common types of errors and specify our approach for detecting and correcting them. In Section \ref{sec:results}, we will then leverage these platinum benchmarks in order to evaluate the reliability of frontier models.

\footnotetext{We mask identities for privacy.}

\subsection{Experimental setup}

\paragraph{Benchmarks considered} We investigate fifteen benchmarks covering six categories of capabilities: mathematics (SingleOp~\cite{roy2015reasoning}, SingleEq~\cite{koncel2015parsing}, MultiArith~\cite{roy2016solving}, SVAMP~\citep{patel2021nlp}, GSM8K~\citep{cobbe2021training}, MMLU High School Math~\citep{hendrycks2020measuring}), logic (BIG-bench Object Counting, BIG-bench Logical Deduction, BIG-bench Navigate ~\citep{srivastava2022beyond, suzgun2022challenging}), table understanding (TabFact~\citep{chen2019tabfact}), reading comprehension (SQuAD2.0 \cite{rajpurkar2018know}, HotPotQA~\citep{yang2018hotpotqa}, DROP~\citep{dua2019drop}), commonsense reasoning (Winograd WSC~\citep{levesque2012winograd}), and visual understanding (VQA v2.0~\citep{goyal2017making}). For datasets with publicly available test splits with solutions, we use the test split, otherwise we use the validation split. Many of these benchmarks are large (e.g., the VQA v2.0 validation set has over 200,000 questions). In order to ensure the quality of our cleaning process, we select smaller subsets at random from many of these benchmarks.

\paragraph{Models tested}
We test a variety of current frontier models (as of January 2025), including both popular proprietary LLMs \cite{openai2023gpt4,Claude35ModelCardAddendum,reid2024gemini,openai2024o1,openai2025o3mini} and open-weights LLMs \cite{dubey2024llama,qwen2.5,deepseekai2024deepseekv3technicalreport,guo2025deepseek}. The full list of models is included in Table \ref{tab:error_results}.

\paragraph{Prompting}
We use chain-of-thought prompting~\cite{wei2022chain} (i.e., asking the model to think step-by-step) for all models except ``reasoning'' models (the o1 series, DeepSeek-R1, Gemini Thinking), as step-by-step thinking is instilled in them through their training. Furthermore, o1's official prompting guide specifically recommends omitting chain-of-thought prompting\footnote{see \url{https://platform.openai.com/docs/guides/reasoning/advice-on-prompting\#advice-on-prompting}.}. All
questions are asked in a zero-shot setting and using a temperature of 0.5. The exact prompts we use are provided in Appendix \ref{app:template}.

\paragraph{Metrics}
All the benchmarks we use are either multiple choice or have a single correct answer (except for reading comprehension datasets such as SQuAD2.0, for which we manually expand the set of correct responses---see Appendix \ref{app:hotpotqa} for further details). Thus, we can simply compare this answer to the model prediction to evaluate correctness. We  report the number of errors rather than accuracy to better differentiate between models, as for many benchmarks we expect models to have accuracies in the high 90s.

\newcommand*{\belowrulesepcolor}[1]{%
  \noalign{%
    \kern-\belowrulesep 
    \begingroup 
      \color{#1}%
      \hrule height\belowrulesep 
    \endgroup 
  }%
} 
\newcommand*{\aboverulesepcolor}[1]{%
  \noalign{%
    \begingroup 
      \color{#1}%
      \hrule height\aboverulesep 
    \endgroup 
    \kern-\aboverulesep 
  }%
} 

\begin{table*}
    \centering
    \setlength{\tabcolsep}{4pt}
    \begin{tabular}{l|p{0.55cm}p{0.55cm}p{0.55cm}p{0.55cm}p{0.55cm}p{0.55cm}|p{0.55cm}p{0.55cm}p{0.55cm}|l|p{0.55cm}p{0.55cm}p{0.55cm}|l|p{0.55cm}}
 \multicolumn{1}{l}{} & \rot{SingleOp} & \rot{SingleEq} & \rot{MultiArith} & \rot{SVAMP} & \rot{GSM8K} & \rot{MMLU HS Math} & \rot{Logic Ded. 3-Obj} & \rot{Object Counting} & \rot{Navigate} & \rot{TabFact} & \rot{HotpotQA} & \rot{SQuAD2.0} & \rot{DROP} & \rot{Winograd WSC} & \rot{VQA v2.0} \\ \toprule
Type                    & \multicolumn{6}{c|}{Math} & \multicolumn{3}{c|}{Logic}        & Tab & \multicolumn{3}{c|}{RC} &   CR & Vis     \\
\# Original Questions & 159 & 109 & 174 & 300 & 300 & 270 & 200 & 200 & 200 & 200 & 250 & 250 & 250 & 200 & 600 \\ \midrule
\textit{Platinum Labeling} &  &  &  &  &  &  &  &  &  &  &  &  &  &  &  \\
\ \ \ \ \# Bad Questions & 9 & 9 & 3 & 27 & 26 & 2 & 0 & 9 & 0 & 27 & 66 & 86 & 41 & 5 & 352 \\
\ \ \ \ \# Mislabeled & 0 & 0 & 3 & 3 & 1 & 0 & 0 & 0 & 0 & 3 & 6 & 4 & 6 & 0 & \textemdash \\ \midrule
\# Platinum Questions & 150 & 100 & 171 & 273 & 274 & 268 & 200 & 191 & 200 & 173 & 184 & 164 & 209 & 195 & 248 \\
    \bottomrule
    \end{tabular}
    \caption{We revise subsets of fifteen popular benchmarks to remove errors and ambiguities. See Section \ref{sec:identify-bad-questions} for a discussion of what constitutes a ``bad'' question to us. The number of mislabeled examples is missing from VQA v2.0 as our labels follow a different format than the original benchmark, necessitating all labels to be revised (see Appendix \ref{app:vqa} for details on VQA v2.0).}
    \label{tab:cleaning}
\end{table*}


\subsection{Identifying Errors in Benchmarks}\label{sec:identify-bad-questions}
We consider a model to be reliable on some task category if it can always (or nearly always) correctly answer questions within that category, provided that they are unambiguous and clearly defined. Measuring such reliability requires carefully curated benchmarks where every incorrect answer definitively indicates a model failure rather than an issue with the question itself.


\paragraph{What makes a question bad?} Before correcting errors in benchmarks, we first need to categorize the kinds of issues we aim to resolve. Each example in a benchmark consists of a question and a solution. Sometimes, a question can be well-written, but the solution is mislabeled. For instance, Figure \ref{fig:benchmark_errors}(a) shows a question from SVAMP for which the given solution is incorrect. For such examples we can simply re-label the solution, allowing us to keep the example in the benchmark. 


In other cases, though, the question itself is poorly written, so simply re-labeling the solution is inadequate. Figures \ref{fig:benchmark_errors}(b-d) illustrate three common categories of such issues: (1) a logical contradiction in the problem statement, (2) ambiguity that allows for multiple plausible solutions, or (3) a clear flaw in the question's construction. 
We opt to remove all poorly written examples during our revision process, as in many cases there is no simple way to fix them. For example, fixing Figure \ref{fig:benchmark_errors}(d) would require coming up with a set of equations and corresponding question from scratch.

\paragraph{How do we efficiently identify errors in a benchmark?} 

Revising large datasets can be time-consuming, especially when the questions take time to verify (e.g., challenging math problems or retrieval tasks with long contexts). We devise a simple and scalable strategy to find problematic questions by examining the agreement among multiple LLMs.

As described above, we divide potential issues with examples from a benchmark into two categories: (1) mislabeled solutions, and (2) poorly written questions (e.g., ones that are ambiguous or ill-posed). To detect these issues, we give each question to several frontier LLMs. We then manually inspect any example on which at least one LLM makes an error. We expect that when a solution is incorrect, frontier models will often disagree with the given solution, and when the question itself is poorly written, models should disagree among themselves. For every example we inspect, we either mark that example as ``bad'' and remove it if the question is poorly written or relabel the solution if it is incorrect. 
Note that the number of errors that we identify for each benchmark is only a lower bound, as our approach could miss any bad question for which all LLM solutions happen to agree with the benchmark solution. For example, all models could interpret an ambiguous question in the same way as that question's given solution. Further details of the cleaning protocols used for each benchmark are in Appendix \ref{app:exp-details}.


\paragraph{How noisy are saturated LLM benchmarks?}
In Table \ref{tab:cleaning}, we report the number of poorly written and mislabeled questions we identify in each of the fifteen benchmark subsets we investigate. We find that, indeed, many of these benchmarks have a substantial rate of errors, confirming suspicions of the presence of flaws in these benchmarks commonly held by the community. For example, the percentage of mislabeled or poorly written questions that we find in GSM8K, SVAMP, VQA v2.0 and TabFact is greater than the percentage of errors reported by frontier models on these benchmarks.


For reading comprehension datasets (SQuAD2.0, HotpotQA, DROP), we identify issues with up to 30\% of examples. Largely, these issues arise from questions that are sufficiently open-ended such that it is difficult to exhaustively list all possible responses. Additionally, SQuAD2.0 intentionally adds questions that are unanswerable from the given passage to test whether models can abstain from answering (i.e., return N/A). However, the process of making these unanswerable questions often leads to them being highly ambiguous or even nonsensical. For example, one such question asks, ``What isn't the gender income inequality in Bahrain?''; this question can be traced to a worker replacing "is" with "isn't" from an original question within SQuAD~\cite{rajpurkar2016squad}\footnote{For the original question from SQuAD, \href{https://huggingface.co/datasets/rajpurkar/squad/viewer/plain\_text/validation?p=74&row=7464}{see this link}.}. Following our cleaning protocol, we omit such poorly written questions.
In Appendix \ref{app:example-failures}, we show specific examples of bad questions that we identify across the fifteen benchmarks.





 
 
\section{Evaluating Reliability with Platinum Benchmarks}
\label{sec:results}
\definecolor{ReliableGreen}{RGB}{131,251,133}
\definecolor{ReliableYellow}{RGB}{255,251,128}
\definecolor{ReliableRed}{RGB}{255,184,184}
\definecolor{ReliableOrange}{RGB}{255,208,143}
\definecolor{ReliableGray}{RGB}{230,230,230}

\begin{table*}[t]
    \vspace{-4pt}
    
    \centering
    \small


    \setlength{\tabcolsep}{4pt}
    \begin{tabular}{l|p{0.45cm}p{0.45cm}p{0.45cm}p{0.45cm}p{0.45cm}p{0.45cm}|p{0.45cm}p{0.45cm}p{0.45cm}|l|p{0.45cm}p{0.45cm}p{0.45cm}|l|p{0.45cm}|c}
    \multicolumn{1}{l}{} & \rot{SingleOp} & \rot{SingleEq} & \rot{MultiArith} & \rot{SVAMP} & \rot{GSM8K} & \rot{MMLU HS Math} & \rot{Logic Ded. 3-Obj} & \rot{Object Counting} & \rot{Navigate} & \rot{TabFact} & \rot{HotpotQA} & \rot{SQuAD2.0} & \rot{DROP} & \rot{Winograd WSC} & \rot{VQA v2.0} &  \\ \toprule
    
        Type                    & \multicolumn{6}{c|}{Math} & \multicolumn{3}{c|}{Logic}        & Tab & \multicolumn{3}{c|}{RC} &   CR & Vis & \multirow{2}{*}{\textbf{Score}}     \\
    

    \# Platinum Questions & 150 & 100 & 171 & 273 & 274 & 268 & 200 & 191 & 200 & 173 & 184 & 164 & 209 & 195 & 248  \\ \midrule


\ \ o1-2024-12-17 (high) & \cellcolor{ReliableGreen}0 & \cellcolor{ReliableGreen}0 & \cellcolor{ReliableGreen}0 & \cellcolor{ReliableGreen}0 & \cellcolor{ReliableYellow}2 & \cellcolor{ReliableYellow}1 & \cellcolor{ReliableGreen}0 & \cellcolor{ReliableGreen}0 & \cellcolor{ReliableGreen}0 & \cellcolor{ReliableGreen}0 & \cellcolor{ReliableGreen}0 & \cellcolor{ReliableOrange}5 & \cellcolor{ReliableGreen}0 & \cellcolor{ReliableOrange}5 & \cellcolor{ReliableOrange}10 & \cellcolor{ReliableGray}0.75\% \\
\ \ Claude 3.5 Sonnet (Oct) & \cellcolor{ReliableGreen}0 & \cellcolor{ReliableGreen}0 & \cellcolor{ReliableGreen}0 & \cellcolor{ReliableYellow}1 & \cellcolor{ReliableYellow}3 & 21 & \cellcolor{ReliableGreen}0 & \cellcolor{ReliableGreen}0 & \cellcolor{ReliableYellow}1 & \cellcolor{ReliableYellow}1 & \cellcolor{ReliableGreen}0 & \cellcolor{ReliableYellow}3 & \cellcolor{ReliableOrange}6 & \cellcolor{ReliableYellow}3 & 17 & \cellcolor{ReliableGray}1.08\% \\
\ \ o1-2024-12-17 (med) & \cellcolor{ReliableGreen}0 & \cellcolor{ReliableGreen}0 & \cellcolor{ReliableGreen}0 & \cellcolor{ReliableYellow}1 & \cellcolor{ReliableYellow}2 & \cellcolor{ReliableYellow}2 & \cellcolor{ReliableGreen}0 & \cellcolor{ReliableGreen}0 & \cellcolor{ReliableGreen}0 & \cellcolor{ReliableGreen}0 & \cellcolor{ReliableGreen}0 & \cellcolor{ReliableOrange}8 & \cellcolor{ReliableYellow}2 & \cellcolor{ReliableOrange}8 & \cellcolor{ReliableOrange}6 & \cellcolor{ReliableGray}1.27\% \\
\ \ DeepSeek-R1 & \cellcolor{ReliableGreen}0 & \cellcolor{ReliableGreen}0 & \cellcolor{ReliableYellow}1 & \cellcolor{ReliableYellow}1 & \cellcolor{ReliableYellow}1 & \cellcolor{ReliableYellow}2 & \cellcolor{ReliableGreen}0 & \cellcolor{ReliableGreen}0 & \cellcolor{ReliableYellow}1 & \cellcolor{ReliableYellow}3 & \cellcolor{ReliableYellow}1 & \cellcolor{ReliableOrange}6 & \cellcolor{ReliableOrange}6 & \cellcolor{ReliableOrange}7 &  & \cellcolor{ReliableGray}1.64\% \\
\ \ Claude 3.5 Sonnet (June) & \cellcolor{ReliableGreen}0 & \cellcolor{ReliableGreen}0 & \cellcolor{ReliableGreen}0 & \cellcolor{ReliableYellow}2 & \cellcolor{ReliableYellow}5 & 29 & \cellcolor{ReliableGreen}0 & \cellcolor{ReliableGreen}0 & \cellcolor{ReliableGreen}0 & \cellcolor{ReliableYellow}3 & \cellcolor{ReliableGreen}0 & \cellcolor{ReliableYellow}3 & \cellcolor{ReliableYellow}3 & \cellcolor{ReliableOrange}8 & 21 & \cellcolor{ReliableGray}1.83\% \\
\ \ Llama 3.1 405B Inst & \cellcolor{ReliableGreen}0 & \cellcolor{ReliableGreen}0 & \cellcolor{ReliableGreen}0 & \cellcolor{ReliableYellow}3 & \cellcolor{ReliableYellow}2 & 28 & \cellcolor{ReliableGreen}0 & \cellcolor{ReliableYellow}1 & \cellcolor{ReliableOrange}5 & \cellcolor{ReliableYellow}1 & \cellcolor{ReliableYellow}2 & \cellcolor{ReliableOrange}4 & \cellcolor{ReliableYellow}4 & \cellcolor{ReliableOrange}8 &  & \cellcolor{ReliableGray}1.91\% \\
\ \ GPT-4o (Aug) & \cellcolor{ReliableGreen}0 & \cellcolor{ReliableGreen}0 & \cellcolor{ReliableGreen}0 & \cellcolor{ReliableOrange}7 & \cellcolor{ReliableYellow}4 & 22 & \cellcolor{ReliableGreen}0 & \cellcolor{ReliableGreen}0 & \cellcolor{ReliableYellow}4 & \cellcolor{ReliableYellow}2 & \cellcolor{ReliableYellow}3 & 9 & \cellcolor{ReliableYellow}4 & 10 & \cellcolor{ReliableOrange}8 & \cellcolor{ReliableGray}2.40\% \\
\ \ GPT-4o (Nov) & \cellcolor{ReliableGreen}0 & \cellcolor{ReliableGreen}0 & \cellcolor{ReliableGreen}0 & \cellcolor{ReliableOrange}6 & \cellcolor{ReliableOrange}7 & 20 & \cellcolor{ReliableGreen}0 & \cellcolor{ReliableOrange}4 & \cellcolor{ReliableYellow}4 & \cellcolor{ReliableYellow}1 & \cellcolor{ReliableYellow}2 & \cellcolor{ReliableOrange}7 & \cellcolor{ReliableYellow}3 & 12 & \cellcolor{ReliableOrange}6 & \cellcolor{ReliableGray}2.48\% \\
\ \ o1-preview & \cellcolor{ReliableGreen}0 & \cellcolor{ReliableGreen}0 & \cellcolor{ReliableGreen}0 & \cellcolor{ReliableYellow}1 & \cellcolor{ReliableYellow}2 & \cellcolor{ReliableYellow}3 & \cellcolor{ReliableGreen}0 & 15 & \cellcolor{ReliableYellow}3 & \cellcolor{ReliableOrange}4 & \cellcolor{ReliableYellow}3 & 11 & \cellcolor{ReliableOrange}5 & \cellcolor{ReliableOrange}6 &  & \cellcolor{ReliableGray}2.49\% \\
\ \ DeepSeek-V3 & \cellcolor{ReliableYellow}1 & \cellcolor{ReliableGreen}0 & \cellcolor{ReliableGreen}0 & \cellcolor{ReliableYellow}3 & \cellcolor{ReliableYellow}3 & \cellcolor{ReliableOrange}12 & \cellcolor{ReliableGreen}0 & 10 & \cellcolor{ReliableGreen}0 & \cellcolor{ReliableYellow}1 & \cellcolor{ReliableYellow}1 & 9 & \cellcolor{ReliableYellow}3 & 16 &  & \cellcolor{ReliableGray}2.85\% \\
\ \ o1-mini & \cellcolor{ReliableYellow}1 & \cellcolor{ReliableGreen}0 & \cellcolor{ReliableYellow}1 & \cellcolor{ReliableYellow}1 & \cellcolor{ReliableYellow}2 & \cellcolor{ReliableYellow}5 & \cellcolor{ReliableYellow}1 & \cellcolor{ReliableYellow}2 & \cellcolor{ReliableYellow}1 & \cellcolor{ReliableYellow}2 & \cellcolor{ReliableYellow}3 & 9 & \cellcolor{ReliableYellow}3 & 19 &  & \cellcolor{ReliableGray}3.03\% \\
\ \ Gemini Thinking (12/19) & \cellcolor{ReliableGreen}0 & \cellcolor{ReliableGreen}0 & \cellcolor{ReliableYellow}1 & \cellcolor{ReliableGreen}0 & \cellcolor{ReliableYellow}4 & \cellcolor{ReliableYellow}3 & \cellcolor{ReliableYellow}4 & \cellcolor{ReliableOrange}5 & \cellcolor{ReliableYellow}3 & \cellcolor{ReliableYellow}2 & \cellcolor{ReliableYellow}3 & \cellcolor{ReliableOrange}6 & \cellcolor{ReliableOrange}6 & 17 & \cellcolor{ReliableOrange}10 & \cellcolor{ReliableGray}3.03\% \\
\ \ Qwen 2.5 72B Inst & \cellcolor{ReliableGreen}0 & \cellcolor{ReliableGreen}0 & \cellcolor{ReliableGreen}0 & \cellcolor{ReliableYellow}4 & \cellcolor{ReliableOrange}7 & 16 & \cellcolor{ReliableYellow}3 & 10 & \cellcolor{ReliableYellow}3 & \cellcolor{ReliableYellow}3 & \cellcolor{ReliableOrange}4 & \cellcolor{ReliableOrange}8 & \cellcolor{ReliableOrange}5 & 12 &  & \cellcolor{ReliableGray}3.09\% \\
\ \ o3-mini-2025-01-31 (high) & \cellcolor{ReliableGreen}0 & \cellcolor{ReliableGreen}0 & \cellcolor{ReliableYellow}2 & \cellcolor{ReliableYellow}1 & \cellcolor{ReliableYellow}1 & \cellcolor{ReliableYellow}2 & \cellcolor{ReliableYellow}1 & \cellcolor{ReliableYellow}1 & \cellcolor{ReliableGreen}0 & \cellcolor{ReliableGreen}0 & \cellcolor{ReliableYellow}2 & 35 & \cellcolor{ReliableYellow}3 & 14 &  & \cellcolor{ReliableGray}3.18\% \\
\ \ Grok 2 & \cellcolor{ReliableYellow}1 & \cellcolor{ReliableGreen}0 & \cellcolor{ReliableGreen}0 & \cellcolor{ReliableYellow}4 & \cellcolor{ReliableYellow}3 & 15 & \cellcolor{ReliableYellow}1 & \cellcolor{ReliableYellow}2 & \cellcolor{ReliableOrange}5 & \cellcolor{ReliableOrange}5 & \cellcolor{ReliableYellow}1 & \cellcolor{ReliableOrange}8 & \cellcolor{ReliableOrange}5 & 15 & 27 & \cellcolor{ReliableGray}3.20\% \\
\ \ Mistral Large & \cellcolor{ReliableGreen}0 & \cellcolor{ReliableGreen}0 & \cellcolor{ReliableGreen}0 & \cellcolor{ReliableOrange}7 & \cellcolor{ReliableYellow}3 & 34 & \cellcolor{ReliableYellow}4 & \cellcolor{ReliableYellow}1 & 11 & \cellcolor{ReliableYellow}3 & \cellcolor{ReliableOrange}4 & 11 & 11 & 11 &  & \cellcolor{ReliableGray}3.5\% \\
\ \ Gemini 2.0 Flash & \cellcolor{ReliableGreen}0 & \cellcolor{ReliableGreen}0 & \cellcolor{ReliableYellow}1 & \cellcolor{ReliableYellow}4 & \cellcolor{ReliableOrange}8 & \cellcolor{ReliableOrange}6 & \cellcolor{ReliableGreen}0 & \cellcolor{ReliableGreen}0 & \cellcolor{ReliableOrange}7 & \cellcolor{ReliableYellow}2 & \cellcolor{ReliableYellow}3 & \cellcolor{ReliableOrange}7 & \cellcolor{ReliableOrange}6 & 22 & \cellcolor{ReliableOrange}7 & \cellcolor{ReliableGray}3.55\% \\
\ \ Llama 3.3 70B Inst & \cellcolor{ReliableGreen}0 & \cellcolor{ReliableGreen}0 & \cellcolor{ReliableGreen}0 & \cellcolor{ReliableOrange}7 & \cellcolor{ReliableOrange}7 & 44 & \cellcolor{ReliableGreen}0 & \cellcolor{ReliableYellow}1 & \cellcolor{ReliableOrange}10 & \cellcolor{ReliableOrange}4 & \cellcolor{ReliableYellow}2 & \cellcolor{ReliableOrange}8 & \cellcolor{ReliableOrange}7 & 14 &  & \cellcolor{ReliableGray}3.61\% \\
\ \ Llama 3.1 70B Inst & \cellcolor{ReliableYellow}2 & \cellcolor{ReliableGreen}0 & \cellcolor{ReliableGreen}0 & \cellcolor{ReliableOrange}7 & \cellcolor{ReliableOrange}7 & 39 & \cellcolor{ReliableYellow}4 & \cellcolor{ReliableYellow}2 & \cellcolor{ReliableOrange}9 & \cellcolor{ReliableOrange}4 & \cellcolor{ReliableYellow}1 & \cellcolor{ReliableOrange}7 & \cellcolor{ReliableOrange}9 & 17 &  & \cellcolor{ReliableGray}4.02\% \\
\ \ Gemini 1.5 Pro & \cellcolor{ReliableGreen}0 & \cellcolor{ReliableGreen}0 & \cellcolor{ReliableYellow}1 & \cellcolor{ReliableOrange}6 & \cellcolor{ReliableOrange}6 & \cellcolor{ReliableOrange}13 & \cellcolor{ReliableYellow}2 & \cellcolor{ReliableOrange}9 & \cellcolor{ReliableOrange}9 & \cellcolor{ReliableOrange}5 & \cellcolor{ReliableYellow}1 & 14 & \cellcolor{ReliableOrange}7 & 17 & \cellcolor{ReliableOrange}9 & \cellcolor{ReliableGray}4.16\% \\
\ \ GPT-4o mini & \cellcolor{ReliableGreen}0 & \cellcolor{ReliableYellow}1 & \cellcolor{ReliableYellow}1 & \cellcolor{ReliableOrange}6 & \cellcolor{ReliableOrange}6 & 24 & \cellcolor{ReliableYellow}2 & 14 & \cellcolor{ReliableOrange}7 & 14 & \cellcolor{ReliableOrange}4 & 16 & 13 & 27 & 29 & \cellcolor{ReliableGray}6.88\% \\
\ \ Claude 3.5 Haiku & \cellcolor{ReliableGreen}0 & \cellcolor{ReliableGreen}0 & \cellcolor{ReliableYellow}1 & \cellcolor{ReliableOrange}8 & \cellcolor{ReliableOrange}10 & 43 & \cellcolor{ReliableYellow}4 & \cellcolor{ReliableOrange}5 & 11 & 13 & \cellcolor{ReliableYellow}3 & 10 & \cellcolor{ReliableOrange}10 & 31 &  & \cellcolor{ReliableGray}6.88\% \\
\ \ Gemini 1.5 Flash & \cellcolor{ReliableGreen}0 & \cellcolor{ReliableYellow}1 & \cellcolor{ReliableGreen}0 & \cellcolor{ReliableOrange}13 & \cellcolor{ReliableOrange}11 & 23 & \cellcolor{ReliableOrange}5 & 15 & 18 & 17 & \cellcolor{ReliableYellow}3 & 13 & 12 & 21 & \cellcolor{ReliableOrange}11 & \cellcolor{ReliableGray}6.96\% \\
\ \ Mistral Small & \cellcolor{ReliableYellow}1 & \cellcolor{ReliableGreen}0 & \cellcolor{ReliableGreen}0 & \cellcolor{ReliableOrange}11 & 19 & 62 & 27 & 18 & 30 & 12 & 10 & 21 & 25 & 39 &  & \cellcolor{ReliableGray}11.09\% \\
    \bottomrule
    \end{tabular}
    \vspace{-2pt}
    \caption{\textbf{Frontier language models are not reliable on simple tasks.} Here we report the number of errors made by each model on our platinum benchmarks, where every error is manually verified. We observe that even after cleaning benchmarks for errors and ambiguities, almost every model makes mistakes on almost every benchmark. VQA v2.0 is only evaluated for models that support image inputs. The score is computed as the average percentage of errors per benchmark within each category, averaged across the five categories (excluding vision). We color results where models demonstrate \fboxsep2pt\colorbox{green!25}{\bf no errors}, \colorbox{yellow!40}{\bf$\leq$2\% errors}, or \colorbox{orange!40}{\bf$\leq$5\% errors}. \textit{RC: Reading comprehension, Tab: Table understanding, CR: Commonsense reasoning, Vis: Vision.}}
    \label{tab:error_results}

\end{table*}

\begin{table*}
    
    \centering
    \setlength{\tabcolsep}{4pt}
    \begin{tabular}{m{3.3cm}|b{0.6cm}b{0.6cm}b{0.6cm}b{0.6cm}b{0.6cm}b{0.6cm}b{0.6cm}b{0.6cm}b{0.6cm}b{0.6cm}b{0.6cm}b{0.6cm}b{0.6cm}b{0.6cm}}
    \multicolumn{1}{l}{}  & \rot{SingleOp} & \rot{SingleEq} & \rot{MultiArith} & \rot{SVAMP} & \rot{GSM8K} & \rot{MMLU HS Math} & \rot{Logic Ded. 3-Obj} & \rot{Object Counting} & \rot{Navigate} & \rot{TabFact} & \rot{HotpotQA} & \rot{SQuAD2.0} & \rot{DROP} & \rot{Winograd WSC} \\
    \toprule
Avg \# errors, original & 2.5 & 1.1 & 3.5 & 18.8 & 10.1 & 20.5 & 2.4 & 6.3 & 5.9 & 17.3 & 26.4 & 56.2 & 28.4 & 16.6 \\
     Avg \# errors, cleaned &  0.3 & 0.1 & 0.4 & 4.3 & 5.2 & 19.5 & 2.4 & 4.8 & 5.9 & 4.3 & 2.3 & 9.9 & 6.6 & 15.0 \\ \midrule
     
    \textbf{\% errors caused by \newline benchmark errors}  &  90\% & 93\% & 89\% & 77\% & 48\% & 5\% & 0\% & 24\% & 0\% & 75\% & 91\% & 82\% & 77\% & 10\% \\
    
    \bottomrule
    \end{tabular}
    \caption{Here we report the average number of errors across models on each benchmark before and after our revision process. For most benchmarks the number of model errors decreased significantly, often by over 50\%, suggesting that the majority of errors in the original benchmarks can be attributed to label noise rather than genuine model failures. We exclude VQA V2.0, as the original benchmark did not include one single ground-truth label to compare against (see Appendix \ref{app:vqa}).
    For reading comprehension benchmarks, there are often many potential correct answers that are not enumerated within the original label (e.g. answering ``five'' when the solution is ``5''). We use an LLM to resolve such cases in the original benchmark; see Appendix \ref{app:hotpotqa} for details.}
    \label{tab:original_vs_cleaned}
\end{table*}


Now that we have constructed a set of platinum benchmarks, we can use them to measure the reliability of frontier LLMs.

\subsection{Pinpointing the Reliability Frontier}
Within a given category of capabilities (e.g., solving math problems), there exist tasks of a wide spectrum of difficulty (e.g., ranging from a simple addition problem to graduate-level math problems).
So, in order to identify the reliability frontier with greater granularity, we need platinum benchmarks at varying levels of difficulty. Then, we can estimate a model's reliability frontier by identifying the most difficult benchmark it is able to pass (i.e., score 100\% on).


Towards this end, six of the benchmarks that we revised are mathematics benchmarks ranging in difficulty from single operations (SingleOP~\cite{roy2015reasoning}) to high school math problems (MMLU High School Math~\citep{hendrycks2020measuring}). We expect that models might exhibit reliability on sufficiently simple math problems (e.g., current models can generally complete single-digit multiplication 100\% of the time). So, this range should allow is to pinpoint the difficulty at which models begin to lose reliability.






\subsection{Findings}
In Table~\ref{tab:error_results} we report the number of errors made by each model on the revised benchmarks, and in Table~\ref{tab:original_vs_cleaned} we compare the frequency of errors on these benchmarks to the original versions before our revisions. 
We manually inspected all model errors during our revision process, so we can be confident that every model failure we report on our platinum benchmarks is genuine.
Our primary findings are:

\begin{enumerate}
    \item \textbf{Reliability challenges are significant and widespread.} Almost every model makes simple mistakes on almost {\em every} dataset, with the exception of particularly simple math datasets (SingleOP, SingleEq, MultiArith). Considering that these frontier models are now evaluated with PhD-level questions~\cite{rein2023gpqa}, it is alarming that they continue to make such simple mistakes. Several examples of these failures are presented in Appendix~\ref{app:example-failures}.
    \item \textbf{Most ``saturated'' benchmarks are too noisy to evaluate reliability.} Table~\ref{tab:original_vs_cleaned} confirms significant differences in the number of errors made by models on the original and cleaned benchmarks. For most of the original benchmarks, the majority of errors can be attributed to mislabeling or bad questions. For instance, about 75\% the errors that models make on the original SVAMP benchmark are on poorly written or mislabeled questions. Logic datasets show few or no errors, but only because these datasets were programmatically generated~\cite{srivastava2022beyond}.

    \item \textbf{More capable models are also more reliable.} Models that are considered to be more capable (e.g., GPT-4o is more capable than GPT-4o mini, Llama 3.1 405B Instruct is more capable than Llama 3.1 70B Instruct) tend to perform better on our benchmarks. Notably, these more capable models are able to perform perfectly on a few of the benchmarks, whereas our least capable models cannot do so for any benchmark.

    \item \textbf{Models' reliability varies depending on the specific capability.} We find that the o1 series and DeepSeek-R1 exhibit the greatest reliability among mathematics benchmarks, while Claude 3.5 Sonnet (Oct) exhibits the greatest reliability for commonsense reasoning. This finding emphasizes the importance of diverse reliability benchmarking; one might want to choose a different frontier LLM depending on the specific task of interest.
\end{enumerate}


\subsection{Platinum benchmarks allow us to discover new patterns of model failures} 
By investigating model errors on platinum benchmarks and their corresponding chain of thought processes, we can discover \textit{patterns} of failures. We identify two such patterns of questions that lead to consistent collapses in reasoning of frontier LLMs. We initially found an instance of each failure mode by examining models' reasoning processes on failures from our platinum benchmarks. We then verified the consistency of such failures by procedurally constructing similar examples. We outline these failure modes below, and provide further details and a more complete analysis in Appendix \ref{app:patterns}.

\paragraph{Example pattern 1: First event bias} We find that when asked: "What happened second: \{\textit{some event}\} or \{\textit{some other event}\}" given some context, three models (Gemini 1.5 Flash, Gemini 1.5 Pro, and Mistral Small) almost always answer with the first event, and will even explicitly acknowledge they are identifying the first event rather than the second:

\begin{quote}
\small
\textbf{Question:} \textit{\{context paragraph\}} What happened second: Russians blocked Azov or Treaty of Constantinople? \\
\textbf{Solution:}  Treaty of Constantinople\\
\textbf{Gemini 1.5 Pro:} \ldots we can conclude that the Russians blocking Azov happened before the Treaty of Constantinople. Answer: Russians blocked Azov
\end{quote}

\noindent We procedurally generate further examples in this question format and find that each of these three models fails on over 85\% of the examples (see Figure \ref{appfig:event_bias}). Notably, this style of question elicits high failure rates in some models while having no detrimental effect on others.

\paragraph{Example pattern 2: Rounding up primes} We find that when Claude 3.5 Sonnet (June) is asked to solve grade-school math problems involving division, it often incorrectly rounds up the quotient even though the division already resulted in a whole number, leading to an answer greater than the solution by one:

\begin{quote}
\small
\textbf{Question:} The school is planning a field trip. The school has 67 classrooms. There are 66 students in each classroom in the school. If there are 6 seats on each school bus. How many buses are needed to take the trip?\\%(Solution: 737)\\
\textbf{Solution:} 737\\
\textbf{Claude 3.5 Sonnet:} \ldots To find the number of buses needed, we divide the total number of students by the number of seats per bus:
4,422 ÷ 6 = 737. However, since we can't have a fraction of a bus, we need to round up to the next whole number to ensure all students have a seat. Therefore, the school needs 738 buses to take all students on the trip. Answer: 738
\end{quote}

\noindent Interestingly, we find that this behavior is significantly more common when the final answer is closer to being prime (i.e., it has few divisors). For example, when we query Claude 3.5 Sonnet on similarly constructed problems with an answer that is prime, the model makes this exact rounding error 20\% of the time (see Figure \ref{appfig:rounding}).


\section{Related Work}
\label{sec:rel_work}
\paragraph{Errors in machine learning benchmarks} Previous works have studied the identification of errors in machine learning benchmarks, as well as the resulting impact of these label errors on the quality of model evaluations. \citet{tsipras2020from} investigate the original ImageNet labeling process and %
release a refined, multi-label re-labeling of the ImageNet validation set. \citet{northcutt2021pervasive} identify errors in commonly used machine learning benchmarks, and then find that evaluating on benchmarks with significant rates of errors can lead practitioners to incorrectly select less performant models. \citet{bowman2021will} raise concerns similar to ours over issues in benchmarking for NLP tasks, and lay out a set of criteria that good benchmarks should satisfy.
However, they focus on overall design and social impact of benchmarks, whereas we focus specifically on better assessing model reliability. Additionally, while their work identifies similar flaws in NLP benchmarks, we propose an approach that can address some of these flaws (in particular, erroneous examples and ambiguity). %

Recently, \citet{gema2024we} released MMLU-Redux, a re-annotated subset of the MMLU benchmark~\cite{hendrycks2020measuring} created through manual assessment by 14 human experts. Our re-labeling of the MMLU high school mathematics subset actually intersects with MMLU-Redux on 100 examples. Our revised annotations align with MMLU-Redux on all but one example, on which we find that one of their human experts accidentally re-annotated a correct solution to make it incorrect\footnote{See the question marked ``wrong\_groundtruth'' here: \url{https://huggingface.co/datasets/edinburgh-dawg/mmlu-redux/viewer/high_school_mathematics?row=52}}. This slight remaining inconsistency highlights the difficulty of avoiding errors when creating and revising benchmarks.

\paragraph{LLM failures on simple tasks} It is generally known and often discussed that LLMs fail in surprising and unintuitive ways even on simple tasks. For instance, the common example of LLMs failing on the query ``how many r's are there in the word strawberry'' has circulated both social media and news outlets \cite{silberling2024strawberry}. Previous works have investigated specific instantiations of such failures. \citet{yang2024can} find that models frequently fail on simple problems even when they can solve harder versions of these same problems, suggesting inconsistency in their reasoning abilities. 
\citet{nezhurina2024alice} raise similar concerns over breakdowns in LLM reasoning behavior by identifying a specific category of logic tasks on which current frontier LLMs fail consistently.


\paragraph{Adversarial examples}
Adversarial attacks are small, sometimes imperceptible perturbations to model inputs that can drastically change their behavior---these lightly perturbed inputs are referred to as \textit{adversarial examples}. There has been significant work studying adversarial attacks and defenses against them in computer vision (and other) domains \cite{szegedy2014intriguing,carlini2017towards,madry2018towards,papernot16jsma}, and recent work has demonstrated successful adversarial attacks on LLMs, especially in the context of breaking safety alignment~\cite{zou2023universal,xu2023llm}. As a result, one possible approach to identifying LLM failures on simple queries might be to adversarially optimize for queries that result in model failures. However, we aim for our benchmark to assess whether models can be deployed reliably on real-world tasks, and adversarial examples generally do not align with the ``corner cases'' that models might face in the real world (unless the users themselves adversarially optimize their own inputs).



\section{Discussion and Conclusion}
\label{sec:discussion}

\section{Discussion and Conclusion}

% \begin{quote}
% \textit{"We believe it is unethical for social workers not to learn... about technology-mediated social work."} (\citeauthor{singer_ai_2023}, 2023)
% \end{quote}

In this study, we uncovered multiple ways in which GenAI can be used in social service practice. While some concerns did arise, practitioners by and large seemed optimistic about the possibilities of such tools, and that these issues could be overcome. We note that while most participants found the tool useful, it was far from perfect in its outputs. This is not surprising, since it was powered by a generic LLM rather than one fine-tuned for social service case management. However, despite these inadequacies, our participants still found many uses for most of the tool's outputs. Many flaws pointed out by our participants related to highly contextualised, local knowledge. To tune an AI system for this would require large amounts of case files as training data; given the privacy concerns associated with using client data, this seems unlikely to happen in the near future. What our study shows, however, is that GenAI systems need not aim to be perfect to be useful to social service practitioners, and can instead serve as a complement to the critical "human touch" in social service.

We draw both inspiration and comparisons with prior work on AI in other settings. Studies on creative writing tools showed how the "uncertainty" \cite{wan2024felt} and "randomness" \cite{clark2018creative} of AI outputs aid creativity. Given the promise that our tool shows in aiding brainstorming and discussion, future social service studies could consider AI tools explicitly geared towards creativity - for instance, providing side-by-side displays of how a given case would fit into different theoretical frameworks, prompting users to compare, contrast, and adopt the best of each framework; or allowing users to play around with combining different intervention modalities to generate eclectic (i.e. multi-modal) interventions.

At the same time, the concept of supervision creates a different interaction paradigm to other uses of AI in brainstorming. Past work (e.g. \cite{shaer2024ai}) has explored the use of GenAI for ideation during brainstorming sessions, wherein all users present discuss the ideas generated by the system. With supervision in social service practice, however, there is a marked information and role asymmetry: supervisors may not have had the time to fully read up on their supervisee's case beforehand, yet have to provide guidance and help to the latter. We suggest that GenAI can serve a dual purpose of bringing supervisors up to speed quickly by summarising their supervisee's case data, while simultaneously generating a list of discussion and talking points that can improve the quality of supervision. Generalising, this interaction paradigm has promise in many other areas: senior doctors reviewing medical procedures with newer ones \cite{snowdon2017does} could use GenAI to generate questions about critical parts of a procedure to ask the latter, confirming they have been correctly understood or executed; game studio directors could quickly summarise key developmental pipeline concerns to raise at meetings and ensure the team is on track; even in academia, advisors involved in rather too many projects to keep track of could quickly summarise each graduate student's projects and identify potential concerns to address at their next meeting.

In closing, we are optimistic about the potential for GenAI to significantly enhance social service practice and the quality of care to clients. Future studies could focus on 1) longitudinal investigations into the long-term impact of GenAI on practitioner skills, client outcomes, and organisational workflows, and 2) optimising workflows to best integrate GenAI into casework and supervision, understanding where best to harness the speed and creativity of such systems in harmony with the experience and skills of practitioners at all levels.

% GenAI here thus serves as a tool that supervisors can use before rather then using the session, taking just a few minutes of their time to generate a list of discussion points with their supervisees.

% Traditional brainstorming comes up with new things that users discuss. In supervision, supervisors can use AI to more efficiently generate talking points with their supervisees. These are generally not novel ideas, since an experienced worker would be able to come up with these on their own. However, the interesting and novel use of AI here is in its use as a preparation tool, efficiently generating talking and discussion points, saving supervisors' time in preparing for a session, while still serving as a brainstorming tool during the session itself.

% The idea of embracing imperfect AI echoes the findings of \citeauthor{bossen2023batman} (2023) in a clinical decision setting, which examined the successful implementation of an "error-prone but useful AI tool". This study frames human-AI collaboration as "Batman and Robin", where AI is a useful but ultimately less skilled sidekick that plays second fiddle to Batman. This is similar to \citeauthor{yang2019unremarkable}'s (2019) idea of "unremarkable AI", systems designed to be unobtrusive and only visible to the user when they add some value. As compared to \citeauthor{bossen2023batman}, however, we see fewer instances of our AI system producing errors, and more examples of it providing learning and collaborative opportunities and other new use cases. We build on the idea of "complementary performance" \cite{bansal2021does}, which discusses how the unique expertise of AI enhances human decision-making performance beyond what humans can achieve alone. Beyond decision-making, GenAI can now enable "complementary work patterns", where the nature of its outputs enables humans to carry out their work in entirely new ways. Our study suggests that rather being a sidekick - Robin - AI is growing into the role of a "second Batman" or "AI-Batman": an entity with distinct abilities and expertise from humans, and that contributes in its own unique way. There is certainly still a time and place for unremarkable AI, but exploring uses beyond that paradigm uncovers entirely new areas of system design.

% % \cite{gero2022sparks} found AI to be useful for science writers to translate ideas already in their head into words, and to provide new perspectives to spark further inspiration. \textit{But how is ours different from theirs?}

% \subsection{New Avenues of Human-AI Collaboration}
% \label{subsubsec:discussionhaicollaboration}

% Past HCI literature in other areas \cite{nah2023generative} has suggested that GenAI represents a "leap" \cite{singh2023hide} in human-AI collaboration, 
% % Even when an AI system sometimes produces irrelevant outputs, it can still provide users 
% % Such systems have been proposed as ways to 
% helping users discover new viewpoints \cite{singh2023hide}, scour existing literature to suggest new hypotheses 
% \cite{cascella2023evaluating} and answer questions \cite{biswas2023role}, stimulate their cognitive processes \cite{memmert2023towards}, and overcome "writer's block" \cite{singh2023hide, cooper2023examining} (particularly relevant to SSPs and the vast amount of writing required of them). Our study finds promise for AI to help SSPs in all of these areas. By nature of being more verbose and capable of generating large amounts of content, GenAI seems to create a new way in which AI can complement human work and expertise. Our system, as LLMs tend to do, produced a lot of "bullshit" (S6) \cite{frankfurt2005bullshit} - superficially true statements that were often only "tangentially related" and "devoid of meaning" \cite{halloran2023ai}. Yet, many participants cited the page-long analyses and detailed multi-step intervention plans generated by the AI system to be a good starting point for further discussion, both to better conceptualize a particular case and to facilitate general worker growth and development. Almost like throwing mud at a wall to see what sticks, GenAI can quickly produce a long list of ideas or information, before the worker glances through it and quickly identifies the more interesting points to discuss. Playing the proposed role as a "scaffold" for further work \cite{cooper2023examining}, GenAI, literally, generates new opportunities for novel and more effective processes and perspectives that previous systems (e.g., PRMs) could not. This represents an entirely new mode of human-AI collaboration.
% % This represents a new mode of collaboration not possible with the largely quantitative AI models (like PRMs) of the past.

% Our work therefore supports and extends prior research that have postulated the the potential of AI's shifting roles from decision-maker to human-supporter \cite{wang_human-human_2020}. \citeauthor{siemon2022elaborating} (2022) suggests the role of AI as a "creator" or "coordinator", rather than merely providing "process guidance" \cite{memmert2023towards} that does not contribute to brainstorming. Similarly, \citeauthor{memmert2023towards} (2023) propose GenAI as a step forward from providing meta-level process guidance (i.e. facilitating user tasks) to actively contributing content and aiding brainstorming. We suggest that beyond content-support, AI can even create new work processes that were not possible without GenAI. In this sense, AI has come full circle, becoming a "meta-facilitator".

% % --- WIP BELOW ---

% % Our work echoes and extends previous research on HAI collaboration in tasks requiring a human touch. \cite{gero2023social} found AI to be a safe space for creative writers to bounce ideas off of and document their inner thoughts. \cite{dhillon2024shaping} reference the idea of appropriate scaffolding in argumentative writing, where the user is providing with guidance appropriate for their competency level, and also warns of decreased satisfaction and ownership from AI use. 

% Separately, we draw parallels with the field of creative writing, where HAI collaboration has been extensively researched. Writers note the "irreducibly human" aspect of creativity in writing \cite{gero2023social}, similar to the "human touch" core to social service practice (D1); both groups therefore expressed few concerns about AI taking over core aspects of their jobs. Another interesting parallel was how writers often appreciated the "uncertainty" \cite{wan2024felt} and "randomness" \cite{clark2018creative} of AI systems, which served as a source of inspiration. This echoes the idea of "imperfect AI" "expanding [the] perspective[s]" (S4) of our participants when they simply skimmed through what the AI produced. \cite{wan2024felt} cited how the "duality of uncertainty in the creativity process advances the exploration of the imperfection of GenAI models". While social service work is not typically regarded as "creative", practitioners nonetheless go through processes of ideation and iteration while formulating a case. Our study showed hints of how AI can help with various forms of ideation, but, drawing inspiration from creative writing tools, future studies could consider designs more explicitly geared towards creativity - for instance, by attempting to fit a given case into a number of different theories or modalities, and displaying them together for the user to consider. While many of these assessments may be imperfect or even downnright nonsensical, they may contain valuable ideas and new angles on viewing the case that the practitioner can integrate into their own assessment.

% % \cite{foong2024designing}, describing the design of caregiver-facing values elicitation tools, cites the "twin scenarios" that caregivers face - private use, where they might use a tool to discover their patient's values, and collaborative use, where they discuss the resulting values with other parties close to the patient. This closely mirrors how SSPs in our study reference both individual and collaborative uses of our tool. Unlike in \cite{foong2024designing}, however, we do not see a resulting need to design a "staged approach" with distinct interface features for both stages.

% % --- END OF WIP ---

% Having mentioned algorithm aversion previously, we also make a quick point here on the other end of the spectrum - automation bias, or blind trust in an automated system \cite{brown2019toward}. LLMs risk being perceived as an "ultimate epistemic authority" \cite{cooper2023examining} due to their detailed, life-like outputs. While automation bias has been studied in many contexts, including in the social sector or adjacent areas, we suggest that the very nature of GenAI systems fundamentally inhibits automation bias. The tendency of GenAI to produce verbose, lengthy explanations prompts users to read and think through the machine's judgement before accepting it, bringing up opportunities to disagree with the machine's opinion. This guards against blind acceptance of the system's recommendations, particularly in the culture of a social work agency where constant dialogue - including discussing AI-produced work - is the norm.


% % : Perception of AI in Social Service Work ??

% \subsection{Redefining the Boundary}
% \label{subsubsec:discussiontheoretical}

% As \citeauthor{meilvang_working_2023} (2023) describes, the social service profession has sought to distance itself from comprising mostly "administrative work" \cite{abbott2016boundaries}, and workers have long tried to tried to reduce their considerable time \cite{socialraadgiverforening2010notat} spent on such tasks in favour of actual casework with clients \cite{toren1972social}. Our study, however, suggests a blurring of the line between "manual" administrative tasks and "mental" casework that draws on practitioner expertise. Many tasks our participants cited involve elements of both: for instance, documenting a case recording requires selecting only the relevant information to include, and planning an intervention can be an iterative process of drafting a plan and discussing it with colleagues and superiors. This all stems from the fact that GenAI can produce virtually any document required by the user, but this document almost always requires revision under a watchful human eye.

% \citeauthor{meilvang_working_2023} (2023) also describes a more recent shift in the perceived accepted boundary of AI interventions in social service work. From "defending [the] jurisdiction [of social service work] against artificial intelligence" in the early days of PRM and other statistical assessment tools, the community has started to embrace AI as a "decision-support ... element in the assessment process". Our study concurs and frames GenAI as a source of information that can be used to support and qualify the assessments of SSPs \cite{meilvang_working_2023}, but suggests that we can take a step further: AI can be viewed as a \textit{facilitator} rather than just a supporter. GenAI can facilitate a wide range of discussions that promote efficiency, encourage worker learning and growth, and ultimately enhance client outcomes. This entails a much larger scope of AI use, where practitioners use the information provided by AI in a range of new scenarios. 

% Taken together, these suggest a new focus for boundary work and, more broadly, HCI research. GAI can play a role not just in menial documentation or decision-support, but can be deeply ingrained into every facet of the social service workflow to open new opportunities for worker growth, workflow optimisation, and ultimately improved client outcomes. Future research can therefore investigate the deeper, organisational-level effects of these new uses of AI, and their resulting impact on the role of profession discretion in effective social service work.

% % MH: oh i feel this paragraph is quite new to me! Could we elaborate this more, and truncate the first two paragraphs a bit to adjust the word propotion?


% % Our study extensively documents this for the first time in social service practice, and in the process reveals new insights about how AI can play such a role.



% \subsection{Design Implications}

% % Add link from ACE diagram?

% % EJ: it would be interesting to discuss how LLMs could help "hands-on experience" in the discussion section

% Addressing the struggle of integrating AI amidst the tension between machine assessment and expert judgement, we reframe AI as an \textit{facilitator} rather than an algorithm or decision-support tool, alleviating many concerns about trust and explainablity. We now present a high-level framework (Figure \ref{fig:hai-collaboration}) on human-AI collaboration, presenting a new perspective on designing effective AI systems that can be applied to both the social service sector and beyond.

% \begin{figure}
%     \centering
%     \includegraphics[scale=0.15]{images/designframework.png}
%     \caption{Framework for Human-AI Collaboration}
%     \label{fig:hai-collaboration}
%     \Description{An image showing our framework for Human-AI Collaboration. It shows that as stakeholder level increases from junior to senior, the directness of use shifts from co-creation to provision.}
% \end{figure}
% % MH: so this paradigm is proposed by us? I wonder if this could a part of results as well..?

% % \subsubsection{From Creation to Provision}

% In Section \ref{sec:stage2findings}, we uncovered the different ways in which SSPs of varying seniorities use, evaluate, and suggest uses of AI. These are intrinsically tied to the perspectives and levels of expertise that each stakeholder possesses. We therefore position the role of AI along the scale of \textit{creation} to \textit{provision}. 

% With junior workers, we recommend \textbf{designing tools for co-creation}: systems that aid the least experienced workers in creating the required deliverables for their work. Rather than \textit{telling} workers what to do - a difficult task in any case given the complexity of social work solutions - AI systems should instead \textit{co-create} deliverables required of these workers. These encompass the multitude of use cases that junior workers found useful: creating reports, suggesting perspectives from which to formulate a case, and providing a starting template for possible intervention plans. Notably, since AI outputs are not perfect, we emphasise the "co" in "co-creation": AI should only be a part of the workflow that also includes active engagement on the part of the SSPs and proactive discussion with supervisors. 

% For more experienced SSPs, we recommend \textbf{designing tools for provision}. Again, this is not the mere provision of recommendations or courses of action with clients, but rather that of resources which complement the needs of workers with greater responsibilities. This notably includes supplying materials to aid with supervision, a novel use case that to our knowledge has not surfaced in previous literature. In addition, senior workers also benefit greatly from manual tasks such as routine report writing and data processing. Since these workers are more experienced and can better spot inaccuracies in AI output, we suggest that AI can "provide" a more finished product that requires less vetting and corrections, and which can be used more directly as part of required deliverables.

% % MH: can we seperate here? above is about the guidance to paradigm, below is the practical roadmap for implementation
% In terms of concrete design features, given the constant focus on discussing AI outputs between colleagues in our FGDs, we recommend that AI tools, particularly those for junior workers, \textbf{include collaborative features} that facilitate feedback and idea sharing between users. We also suggest that designers work closely with domain experts (i.e. social work practitioners and agencies) to identify areas where the given AI model tends to make more mistakes, and to build in features that \textbf{highlight potential mistakes or inadequacies} in the AI's output to facilitate further discussion and avoid workers adopting suboptimal suggestions. 

% We also point out a fundamental difference between GenAI systems and previous systems: that GenAI can now play an important role in aiding users \textit{regardless of its flaws}. The nature of GenAI means that it promotes discussion and opens up new workflows by nature of its verbose and potentially incomplete outputs. Rather than working towards more accurate or explainable outcomes, which may in any case have minimal improvement on worker outcomes \cite{li2024advanced}, designers can also focus on \textbf{understanding how GenAI outputs can augment existing user flows and create new ones}.

% % for more senior workers...

% % how to differentiate levels of workers?

% % \subsubsection{Provider}

% % The most basic and obvious role of modern AI that we identify leverages the main strength of LLMs. They have the ability to produce high-quality writing from short, point-form, or otherwise messy and disjoint case notes that user often have \textit{[cite participant here]}. 

% Finally, given the limited expertise of many workers at using AI, it is important that systems \textbf{explicitly guide users to the features they need}, rather than simply relying on the ability of GAI to understand complex user instructions. For example, in the case of flexibility in use cases (Section \ref{subsubsec:control}), systems should include user flows that help combine multiple intervention and assessment modalities in order to directly meet the needs of workers.

% \subsection{Limitations and Future Work}

% While we attempt to mimic a contextual inquiry and work environment in our study design, there is no substitute for real data from actual system deployment. The use of an AI system in day-to-day work could reveal a different set of insights. Future studies could in particular study how the longitudinal context of how user attitudes, behaviours, preferences, and work outputs change with extended use of AI. 

% While we tried to include practitioners from different agencies, roles, and seniorities, social service practice may differ culturally or procedurally in other agencies or countries. Future studies could investigate different kinds of social service agencies and in different cultures to see if AI is similarly useful there.

% As the study was conducted in a country with relatively high technology literacy, participants naturally had a higher baseline understanding and acceptance of AI and other computer systems. However, we emphasise that our findings are not contingent on this - rather, we suggest that our proposed lens of viewing AI in the social sector is a means for engaging in relevant stakeholders and ensuring the effective design and implementation of AI in the social sector, regardless of how participants feel about AI to begin with. 



% % \subsection{Notes}

% % 1) safety and risks and 2) privacy - what does the emphasis on this say about a) design recommendations and b) approach to designing/PD of such systems?


% % W9 was presented with "Strengths" and "SFBT" output options. They commented, "solution focus is always building on the person's strengths". W9 therefore requested being able to output strengths and SFBT at the same time. But this would suggest that the SFBT output does not currently emphasise strengths strongly enough. However, W9 did not specifically evaluate that, and only made this comment because they saw the "strengths" option available, and in their head, strengths are key to SFBT.
% % What does this say about system design and UI in relation to user mental models?

\section*{Acknowledgments}
Work supported in part by the NSF grant DMS-2134108 and Open Philanthropy. This work was also supported by the AI and Biodiversity Change (ABC) Global
Climate Center, which is funded by the US National Science Foundation under Award No. 2330423 and Natural Sciences and Engineering Research Council of Canada under Award No. 585136. E.V. is supported by an NSF GRFP fellowship.


\clearpage
\printbibliography

\clearpage
\appendix
\addcontentsline{toc}{section}{Appendix}
\renewcommand\ptctitle{Appendices}
\part{}
\parttoc
\clearpage

\counterwithin{figure}{section}
\counterwithin{table}{section}
\counterwithin{algorithm}{section}

\newpage
\appendix
\onecolumn
\subsection{Lloyd-Max Algorithm}
\label{subsec:Lloyd-Max}
For a given quantization bitwidth $B$ and an operand $\bm{X}$, the Lloyd-Max algorithm finds $2^B$ quantization levels $\{\hat{x}_i\}_{i=1}^{2^B}$ such that quantizing $\bm{X}$ by rounding each scalar in $\bm{X}$ to the nearest quantization level minimizes the quantization MSE. 

The algorithm starts with an initial guess of quantization levels and then iteratively computes quantization thresholds $\{\tau_i\}_{i=1}^{2^B-1}$ and updates quantization levels $\{\hat{x}_i\}_{i=1}^{2^B}$. Specifically, at iteration $n$, thresholds are set to the midpoints of the previous iteration's levels:
\begin{align*}
    \tau_i^{(n)}=\frac{\hat{x}_i^{(n-1)}+\hat{x}_{i+1}^{(n-1)}}2 \text{ for } i=1\ldots 2^B-1
\end{align*}
Subsequently, the quantization levels are re-computed as conditional means of the data regions defined by the new thresholds:
\begin{align*}
    \hat{x}_i^{(n)}=\mathbb{E}\left[ \bm{X} \big| \bm{X}\in [\tau_{i-1}^{(n)},\tau_i^{(n)}] \right] \text{ for } i=1\ldots 2^B
\end{align*}
where to satisfy boundary conditions we have $\tau_0=-\infty$ and $\tau_{2^B}=\infty$. The algorithm iterates the above steps until convergence.

Figure \ref{fig:lm_quant} compares the quantization levels of a $7$-bit floating point (E3M3) quantizer (left) to a $7$-bit Lloyd-Max quantizer (right) when quantizing a layer of weights from the GPT3-126M model at a per-tensor granularity. As shown, the Lloyd-Max quantizer achieves substantially lower quantization MSE. Further, Table \ref{tab:FP7_vs_LM7} shows the superior perplexity achieved by Lloyd-Max quantizers for bitwidths of $7$, $6$ and $5$. The difference between the quantizers is clear at 5 bits, where per-tensor FP quantization incurs a drastic and unacceptable increase in perplexity, while Lloyd-Max quantization incurs a much smaller increase. Nevertheless, we note that even the optimal Lloyd-Max quantizer incurs a notable ($\sim 1.5$) increase in perplexity due to the coarse granularity of quantization. 

\begin{figure}[h]
  \centering
  \includegraphics[width=0.7\linewidth]{sections/figures/LM7_FP7.pdf}
  \caption{\small Quantization levels and the corresponding quantization MSE of Floating Point (left) vs Lloyd-Max (right) Quantizers for a layer of weights in the GPT3-126M model.}
  \label{fig:lm_quant}
\end{figure}

\begin{table}[h]\scriptsize
\begin{center}
\caption{\label{tab:FP7_vs_LM7} \small Comparing perplexity (lower is better) achieved by floating point quantizers and Lloyd-Max quantizers on a GPT3-126M model for the Wikitext-103 dataset.}
\begin{tabular}{c|cc|c}
\hline
 \multirow{2}{*}{\textbf{Bitwidth}} & \multicolumn{2}{|c|}{\textbf{Floating-Point Quantizer}} & \textbf{Lloyd-Max Quantizer} \\
 & Best Format & Wikitext-103 Perplexity & Wikitext-103 Perplexity \\
\hline
7 & E3M3 & 18.32 & 18.27 \\
6 & E3M2 & 19.07 & 18.51 \\
5 & E4M0 & 43.89 & 19.71 \\
\hline
\end{tabular}
\end{center}
\end{table}

\subsection{Proof of Local Optimality of LO-BCQ}
\label{subsec:lobcq_opt_proof}
For a given block $\bm{b}_j$, the quantization MSE during LO-BCQ can be empirically evaluated as $\frac{1}{L_b}\lVert \bm{b}_j- \bm{\hat{b}}_j\rVert^2_2$ where $\bm{\hat{b}}_j$ is computed from equation (\ref{eq:clustered_quantization_definition}) as $C_{f(\bm{b}_j)}(\bm{b}_j)$. Further, for a given block cluster $\mathcal{B}_i$, we compute the quantization MSE as $\frac{1}{|\mathcal{B}_{i}|}\sum_{\bm{b} \in \mathcal{B}_{i}} \frac{1}{L_b}\lVert \bm{b}- C_i^{(n)}(\bm{b})\rVert^2_2$. Therefore, at the end of iteration $n$, we evaluate the overall quantization MSE $J^{(n)}$ for a given operand $\bm{X}$ composed of $N_c$ block clusters as:
\begin{align*}
    \label{eq:mse_iter_n}
    J^{(n)} = \frac{1}{N_c} \sum_{i=1}^{N_c} \frac{1}{|\mathcal{B}_{i}^{(n)}|}\sum_{\bm{v} \in \mathcal{B}_{i}^{(n)}} \frac{1}{L_b}\lVert \bm{b}- B_i^{(n)}(\bm{b})\rVert^2_2
\end{align*}

At the end of iteration $n$, the codebooks are updated from $\mathcal{C}^{(n-1)}$ to $\mathcal{C}^{(n)}$. However, the mapping of a given vector $\bm{b}_j$ to quantizers $\mathcal{C}^{(n)}$ remains as  $f^{(n)}(\bm{b}_j)$. At the next iteration, during the vector clustering step, $f^{(n+1)}(\bm{b}_j)$ finds new mapping of $\bm{b}_j$ to updated codebooks $\mathcal{C}^{(n)}$ such that the quantization MSE over the candidate codebooks is minimized. Therefore, we obtain the following result for $\bm{b}_j$:
\begin{align*}
\frac{1}{L_b}\lVert \bm{b}_j - C_{f^{(n+1)}(\bm{b}_j)}^{(n)}(\bm{b}_j)\rVert^2_2 \le \frac{1}{L_b}\lVert \bm{b}_j - C_{f^{(n)}(\bm{b}_j)}^{(n)}(\bm{b}_j)\rVert^2_2
\end{align*}

That is, quantizing $\bm{b}_j$ at the end of the block clustering step of iteration $n+1$ results in lower quantization MSE compared to quantizing at the end of iteration $n$. Since this is true for all $\bm{b} \in \bm{X}$, we assert the following:
\begin{equation}
\begin{split}
\label{eq:mse_ineq_1}
    \tilde{J}^{(n+1)} &= \frac{1}{N_c} \sum_{i=1}^{N_c} \frac{1}{|\mathcal{B}_{i}^{(n+1)}|}\sum_{\bm{b} \in \mathcal{B}_{i}^{(n+1)}} \frac{1}{L_b}\lVert \bm{b} - C_i^{(n)}(b)\rVert^2_2 \le J^{(n)}
\end{split}
\end{equation}
where $\tilde{J}^{(n+1)}$ is the the quantization MSE after the vector clustering step at iteration $n+1$.

Next, during the codebook update step (\ref{eq:quantizers_update}) at iteration $n+1$, the per-cluster codebooks $\mathcal{C}^{(n)}$ are updated to $\mathcal{C}^{(n+1)}$ by invoking the Lloyd-Max algorithm \citep{Lloyd}. We know that for any given value distribution, the Lloyd-Max algorithm minimizes the quantization MSE. Therefore, for a given vector cluster $\mathcal{B}_i$ we obtain the following result:

\begin{equation}
    \frac{1}{|\mathcal{B}_{i}^{(n+1)}|}\sum_{\bm{b} \in \mathcal{B}_{i}^{(n+1)}} \frac{1}{L_b}\lVert \bm{b}- C_i^{(n+1)}(\bm{b})\rVert^2_2 \le \frac{1}{|\mathcal{B}_{i}^{(n+1)}|}\sum_{\bm{b} \in \mathcal{B}_{i}^{(n+1)}} \frac{1}{L_b}\lVert \bm{b}- C_i^{(n)}(\bm{b})\rVert^2_2
\end{equation}

The above equation states that quantizing the given block cluster $\mathcal{B}_i$ after updating the associated codebook from $C_i^{(n)}$ to $C_i^{(n+1)}$ results in lower quantization MSE. Since this is true for all the block clusters, we derive the following result: 
\begin{equation}
\begin{split}
\label{eq:mse_ineq_2}
     J^{(n+1)} &= \frac{1}{N_c} \sum_{i=1}^{N_c} \frac{1}{|\mathcal{B}_{i}^{(n+1)}|}\sum_{\bm{b} \in \mathcal{B}_{i}^{(n+1)}} \frac{1}{L_b}\lVert \bm{b}- C_i^{(n+1)}(\bm{b})\rVert^2_2  \le \tilde{J}^{(n+1)}   
\end{split}
\end{equation}

Following (\ref{eq:mse_ineq_1}) and (\ref{eq:mse_ineq_2}), we find that the quantization MSE is non-increasing for each iteration, that is, $J^{(1)} \ge J^{(2)} \ge J^{(3)} \ge \ldots \ge J^{(M)}$ where $M$ is the maximum number of iterations. 
%Therefore, we can say that if the algorithm converges, then it must be that it has converged to a local minimum. 
\hfill $\blacksquare$


\begin{figure}
    \begin{center}
    \includegraphics[width=0.5\textwidth]{sections//figures/mse_vs_iter.pdf}
    \end{center}
    \caption{\small NMSE vs iterations during LO-BCQ compared to other block quantization proposals}
    \label{fig:nmse_vs_iter}
\end{figure}

Figure \ref{fig:nmse_vs_iter} shows the empirical convergence of LO-BCQ across several block lengths and number of codebooks. Also, the MSE achieved by LO-BCQ is compared to baselines such as MXFP and VSQ. As shown, LO-BCQ converges to a lower MSE than the baselines. Further, we achieve better convergence for larger number of codebooks ($N_c$) and for a smaller block length ($L_b$), both of which increase the bitwidth of BCQ (see Eq \ref{eq:bitwidth_bcq}).


\subsection{Additional Accuracy Results}
%Table \ref{tab:lobcq_config} lists the various LOBCQ configurations and their corresponding bitwidths.
\begin{table}
\setlength{\tabcolsep}{4.75pt}
\begin{center}
\caption{\label{tab:lobcq_config} Various LO-BCQ configurations and their bitwidths.}
\begin{tabular}{|c||c|c|c|c||c|c||c|} 
\hline
 & \multicolumn{4}{|c||}{$L_b=8$} & \multicolumn{2}{|c||}{$L_b=4$} & $L_b=2$ \\
 \hline
 \backslashbox{$L_A$\kern-1em}{\kern-1em$N_c$} & 2 & 4 & 8 & 16 & 2 & 4 & 2 \\
 \hline
 64 & 4.25 & 4.375 & 4.5 & 4.625 & 4.375 & 4.625 & 4.625\\
 \hline
 32 & 4.375 & 4.5 & 4.625& 4.75 & 4.5 & 4.75 & 4.75 \\
 \hline
 16 & 4.625 & 4.75& 4.875 & 5 & 4.75 & 5 & 5 \\
 \hline
\end{tabular}
\end{center}
\end{table}

%\subsection{Perplexity achieved by various LO-BCQ configurations on Wikitext-103 dataset}

\begin{table} \centering
\begin{tabular}{|c||c|c|c|c||c|c||c|} 
\hline
 $L_b \rightarrow$& \multicolumn{4}{c||}{8} & \multicolumn{2}{c||}{4} & 2\\
 \hline
 \backslashbox{$L_A$\kern-1em}{\kern-1em$N_c$} & 2 & 4 & 8 & 16 & 2 & 4 & 2  \\
 %$N_c \rightarrow$ & 2 & 4 & 8 & 16 & 2 & 4 & 2 \\
 \hline
 \hline
 \multicolumn{8}{c}{GPT3-1.3B (FP32 PPL = 9.98)} \\ 
 \hline
 \hline
 64 & 10.40 & 10.23 & 10.17 & 10.15 &  10.28 & 10.18 & 10.19 \\
 \hline
 32 & 10.25 & 10.20 & 10.15 & 10.12 &  10.23 & 10.17 & 10.17 \\
 \hline
 16 & 10.22 & 10.16 & 10.10 & 10.09 &  10.21 & 10.14 & 10.16 \\
 \hline
  \hline
 \multicolumn{8}{c}{GPT3-8B (FP32 PPL = 7.38)} \\ 
 \hline
 \hline
 64 & 7.61 & 7.52 & 7.48 &  7.47 &  7.55 &  7.49 & 7.50 \\
 \hline
 32 & 7.52 & 7.50 & 7.46 &  7.45 &  7.52 &  7.48 & 7.48  \\
 \hline
 16 & 7.51 & 7.48 & 7.44 &  7.44 &  7.51 &  7.49 & 7.47  \\
 \hline
\end{tabular}
\caption{\label{tab:ppl_gpt3_abalation} Wikitext-103 perplexity across GPT3-1.3B and 8B models.}
\end{table}

\begin{table} \centering
\begin{tabular}{|c||c|c|c|c||} 
\hline
 $L_b \rightarrow$& \multicolumn{4}{c||}{8}\\
 \hline
 \backslashbox{$L_A$\kern-1em}{\kern-1em$N_c$} & 2 & 4 & 8 & 16 \\
 %$N_c \rightarrow$ & 2 & 4 & 8 & 16 & 2 & 4 & 2 \\
 \hline
 \hline
 \multicolumn{5}{|c|}{Llama2-7B (FP32 PPL = 5.06)} \\ 
 \hline
 \hline
 64 & 5.31 & 5.26 & 5.19 & 5.18  \\
 \hline
 32 & 5.23 & 5.25 & 5.18 & 5.15  \\
 \hline
 16 & 5.23 & 5.19 & 5.16 & 5.14  \\
 \hline
 \multicolumn{5}{|c|}{Nemotron4-15B (FP32 PPL = 5.87)} \\ 
 \hline
 \hline
 64  & 6.3 & 6.20 & 6.13 & 6.08  \\
 \hline
 32  & 6.24 & 6.12 & 6.07 & 6.03  \\
 \hline
 16  & 6.12 & 6.14 & 6.04 & 6.02  \\
 \hline
 \multicolumn{5}{|c|}{Nemotron4-340B (FP32 PPL = 3.48)} \\ 
 \hline
 \hline
 64 & 3.67 & 3.62 & 3.60 & 3.59 \\
 \hline
 32 & 3.63 & 3.61 & 3.59 & 3.56 \\
 \hline
 16 & 3.61 & 3.58 & 3.57 & 3.55 \\
 \hline
\end{tabular}
\caption{\label{tab:ppl_llama7B_nemo15B} Wikitext-103 perplexity compared to FP32 baseline in Llama2-7B and Nemotron4-15B, 340B models}
\end{table}

%\subsection{Perplexity achieved by various LO-BCQ configurations on MMLU dataset}


\begin{table} \centering
\begin{tabular}{|c||c|c|c|c||c|c|c|c|} 
\hline
 $L_b \rightarrow$& \multicolumn{4}{c||}{8} & \multicolumn{4}{c||}{8}\\
 \hline
 \backslashbox{$L_A$\kern-1em}{\kern-1em$N_c$} & 2 & 4 & 8 & 16 & 2 & 4 & 8 & 16  \\
 %$N_c \rightarrow$ & 2 & 4 & 8 & 16 & 2 & 4 & 2 \\
 \hline
 \hline
 \multicolumn{5}{|c|}{Llama2-7B (FP32 Accuracy = 45.8\%)} & \multicolumn{4}{|c|}{Llama2-70B (FP32 Accuracy = 69.12\%)} \\ 
 \hline
 \hline
 64 & 43.9 & 43.4 & 43.9 & 44.9 & 68.07 & 68.27 & 68.17 & 68.75 \\
 \hline
 32 & 44.5 & 43.8 & 44.9 & 44.5 & 68.37 & 68.51 & 68.35 & 68.27  \\
 \hline
 16 & 43.9 & 42.7 & 44.9 & 45 & 68.12 & 68.77 & 68.31 & 68.59  \\
 \hline
 \hline
 \multicolumn{5}{|c|}{GPT3-22B (FP32 Accuracy = 38.75\%)} & \multicolumn{4}{|c|}{Nemotron4-15B (FP32 Accuracy = 64.3\%)} \\ 
 \hline
 \hline
 64 & 36.71 & 38.85 & 38.13 & 38.92 & 63.17 & 62.36 & 63.72 & 64.09 \\
 \hline
 32 & 37.95 & 38.69 & 39.45 & 38.34 & 64.05 & 62.30 & 63.8 & 64.33  \\
 \hline
 16 & 38.88 & 38.80 & 38.31 & 38.92 & 63.22 & 63.51 & 63.93 & 64.43  \\
 \hline
\end{tabular}
\caption{\label{tab:mmlu_abalation} Accuracy on MMLU dataset across GPT3-22B, Llama2-7B, 70B and Nemotron4-15B models.}
\end{table}


%\subsection{Perplexity achieved by various LO-BCQ configurations on LM evaluation harness}

\begin{table} \centering
\begin{tabular}{|c||c|c|c|c||c|c|c|c|} 
\hline
 $L_b \rightarrow$& \multicolumn{4}{c||}{8} & \multicolumn{4}{c||}{8}\\
 \hline
 \backslashbox{$L_A$\kern-1em}{\kern-1em$N_c$} & 2 & 4 & 8 & 16 & 2 & 4 & 8 & 16  \\
 %$N_c \rightarrow$ & 2 & 4 & 8 & 16 & 2 & 4 & 2 \\
 \hline
 \hline
 \multicolumn{5}{|c|}{Race (FP32 Accuracy = 37.51\%)} & \multicolumn{4}{|c|}{Boolq (FP32 Accuracy = 64.62\%)} \\ 
 \hline
 \hline
 64 & 36.94 & 37.13 & 36.27 & 37.13 & 63.73 & 62.26 & 63.49 & 63.36 \\
 \hline
 32 & 37.03 & 36.36 & 36.08 & 37.03 & 62.54 & 63.51 & 63.49 & 63.55  \\
 \hline
 16 & 37.03 & 37.03 & 36.46 & 37.03 & 61.1 & 63.79 & 63.58 & 63.33  \\
 \hline
 \hline
 \multicolumn{5}{|c|}{Winogrande (FP32 Accuracy = 58.01\%)} & \multicolumn{4}{|c|}{Piqa (FP32 Accuracy = 74.21\%)} \\ 
 \hline
 \hline
 64 & 58.17 & 57.22 & 57.85 & 58.33 & 73.01 & 73.07 & 73.07 & 72.80 \\
 \hline
 32 & 59.12 & 58.09 & 57.85 & 58.41 & 73.01 & 73.94 & 72.74 & 73.18  \\
 \hline
 16 & 57.93 & 58.88 & 57.93 & 58.56 & 73.94 & 72.80 & 73.01 & 73.94  \\
 \hline
\end{tabular}
\caption{\label{tab:mmlu_abalation} Accuracy on LM evaluation harness tasks on GPT3-1.3B model.}
\end{table}

\begin{table} \centering
\begin{tabular}{|c||c|c|c|c||c|c|c|c|} 
\hline
 $L_b \rightarrow$& \multicolumn{4}{c||}{8} & \multicolumn{4}{c||}{8}\\
 \hline
 \backslashbox{$L_A$\kern-1em}{\kern-1em$N_c$} & 2 & 4 & 8 & 16 & 2 & 4 & 8 & 16  \\
 %$N_c \rightarrow$ & 2 & 4 & 8 & 16 & 2 & 4 & 2 \\
 \hline
 \hline
 \multicolumn{5}{|c|}{Race (FP32 Accuracy = 41.34\%)} & \multicolumn{4}{|c|}{Boolq (FP32 Accuracy = 68.32\%)} \\ 
 \hline
 \hline
 64 & 40.48 & 40.10 & 39.43 & 39.90 & 69.20 & 68.41 & 69.45 & 68.56 \\
 \hline
 32 & 39.52 & 39.52 & 40.77 & 39.62 & 68.32 & 67.43 & 68.17 & 69.30  \\
 \hline
 16 & 39.81 & 39.71 & 39.90 & 40.38 & 68.10 & 66.33 & 69.51 & 69.42  \\
 \hline
 \hline
 \multicolumn{5}{|c|}{Winogrande (FP32 Accuracy = 67.88\%)} & \multicolumn{4}{|c|}{Piqa (FP32 Accuracy = 78.78\%)} \\ 
 \hline
 \hline
 64 & 66.85 & 66.61 & 67.72 & 67.88 & 77.31 & 77.42 & 77.75 & 77.64 \\
 \hline
 32 & 67.25 & 67.72 & 67.72 & 67.00 & 77.31 & 77.04 & 77.80 & 77.37  \\
 \hline
 16 & 68.11 & 68.90 & 67.88 & 67.48 & 77.37 & 78.13 & 78.13 & 77.69  \\
 \hline
\end{tabular}
\caption{\label{tab:mmlu_abalation} Accuracy on LM evaluation harness tasks on GPT3-8B model.}
\end{table}

\begin{table} \centering
\begin{tabular}{|c||c|c|c|c||c|c|c|c|} 
\hline
 $L_b \rightarrow$& \multicolumn{4}{c||}{8} & \multicolumn{4}{c||}{8}\\
 \hline
 \backslashbox{$L_A$\kern-1em}{\kern-1em$N_c$} & 2 & 4 & 8 & 16 & 2 & 4 & 8 & 16  \\
 %$N_c \rightarrow$ & 2 & 4 & 8 & 16 & 2 & 4 & 2 \\
 \hline
 \hline
 \multicolumn{5}{|c|}{Race (FP32 Accuracy = 40.67\%)} & \multicolumn{4}{|c|}{Boolq (FP32 Accuracy = 76.54\%)} \\ 
 \hline
 \hline
 64 & 40.48 & 40.10 & 39.43 & 39.90 & 75.41 & 75.11 & 77.09 & 75.66 \\
 \hline
 32 & 39.52 & 39.52 & 40.77 & 39.62 & 76.02 & 76.02 & 75.96 & 75.35  \\
 \hline
 16 & 39.81 & 39.71 & 39.90 & 40.38 & 75.05 & 73.82 & 75.72 & 76.09  \\
 \hline
 \hline
 \multicolumn{5}{|c|}{Winogrande (FP32 Accuracy = 70.64\%)} & \multicolumn{4}{|c|}{Piqa (FP32 Accuracy = 79.16\%)} \\ 
 \hline
 \hline
 64 & 69.14 & 70.17 & 70.17 & 70.56 & 78.24 & 79.00 & 78.62 & 78.73 \\
 \hline
 32 & 70.96 & 69.69 & 71.27 & 69.30 & 78.56 & 79.49 & 79.16 & 78.89  \\
 \hline
 16 & 71.03 & 69.53 & 69.69 & 70.40 & 78.13 & 79.16 & 79.00 & 79.00  \\
 \hline
\end{tabular}
\caption{\label{tab:mmlu_abalation} Accuracy on LM evaluation harness tasks on GPT3-22B model.}
\end{table}

\begin{table} \centering
\begin{tabular}{|c||c|c|c|c||c|c|c|c|} 
\hline
 $L_b \rightarrow$& \multicolumn{4}{c||}{8} & \multicolumn{4}{c||}{8}\\
 \hline
 \backslashbox{$L_A$\kern-1em}{\kern-1em$N_c$} & 2 & 4 & 8 & 16 & 2 & 4 & 8 & 16  \\
 %$N_c \rightarrow$ & 2 & 4 & 8 & 16 & 2 & 4 & 2 \\
 \hline
 \hline
 \multicolumn{5}{|c|}{Race (FP32 Accuracy = 44.4\%)} & \multicolumn{4}{|c|}{Boolq (FP32 Accuracy = 79.29\%)} \\ 
 \hline
 \hline
 64 & 42.49 & 42.51 & 42.58 & 43.45 & 77.58 & 77.37 & 77.43 & 78.1 \\
 \hline
 32 & 43.35 & 42.49 & 43.64 & 43.73 & 77.86 & 75.32 & 77.28 & 77.86  \\
 \hline
 16 & 44.21 & 44.21 & 43.64 & 42.97 & 78.65 & 77 & 76.94 & 77.98  \\
 \hline
 \hline
 \multicolumn{5}{|c|}{Winogrande (FP32 Accuracy = 69.38\%)} & \multicolumn{4}{|c|}{Piqa (FP32 Accuracy = 78.07\%)} \\ 
 \hline
 \hline
 64 & 68.9 & 68.43 & 69.77 & 68.19 & 77.09 & 76.82 & 77.09 & 77.86 \\
 \hline
 32 & 69.38 & 68.51 & 68.82 & 68.90 & 78.07 & 76.71 & 78.07 & 77.86  \\
 \hline
 16 & 69.53 & 67.09 & 69.38 & 68.90 & 77.37 & 77.8 & 77.91 & 77.69  \\
 \hline
\end{tabular}
\caption{\label{tab:mmlu_abalation} Accuracy on LM evaluation harness tasks on Llama2-7B model.}
\end{table}

\begin{table} \centering
\begin{tabular}{|c||c|c|c|c||c|c|c|c|} 
\hline
 $L_b \rightarrow$& \multicolumn{4}{c||}{8} & \multicolumn{4}{c||}{8}\\
 \hline
 \backslashbox{$L_A$\kern-1em}{\kern-1em$N_c$} & 2 & 4 & 8 & 16 & 2 & 4 & 8 & 16  \\
 %$N_c \rightarrow$ & 2 & 4 & 8 & 16 & 2 & 4 & 2 \\
 \hline
 \hline
 \multicolumn{5}{|c|}{Race (FP32 Accuracy = 48.8\%)} & \multicolumn{4}{|c|}{Boolq (FP32 Accuracy = 85.23\%)} \\ 
 \hline
 \hline
 64 & 49.00 & 49.00 & 49.28 & 48.71 & 82.82 & 84.28 & 84.03 & 84.25 \\
 \hline
 32 & 49.57 & 48.52 & 48.33 & 49.28 & 83.85 & 84.46 & 84.31 & 84.93  \\
 \hline
 16 & 49.85 & 49.09 & 49.28 & 48.99 & 85.11 & 84.46 & 84.61 & 83.94  \\
 \hline
 \hline
 \multicolumn{5}{|c|}{Winogrande (FP32 Accuracy = 79.95\%)} & \multicolumn{4}{|c|}{Piqa (FP32 Accuracy = 81.56\%)} \\ 
 \hline
 \hline
 64 & 78.77 & 78.45 & 78.37 & 79.16 & 81.45 & 80.69 & 81.45 & 81.5 \\
 \hline
 32 & 78.45 & 79.01 & 78.69 & 80.66 & 81.56 & 80.58 & 81.18 & 81.34  \\
 \hline
 16 & 79.95 & 79.56 & 79.79 & 79.72 & 81.28 & 81.66 & 81.28 & 80.96  \\
 \hline
\end{tabular}
\caption{\label{tab:mmlu_abalation} Accuracy on LM evaluation harness tasks on Llama2-70B model.}
\end{table}

%\section{MSE Studies}
%\textcolor{red}{TODO}


\subsection{Number Formats and Quantization Method}
\label{subsec:numFormats_quantMethod}
\subsubsection{Integer Format}
An $n$-bit signed integer (INT) is typically represented with a 2s-complement format \citep{yao2022zeroquant,xiao2023smoothquant,dai2021vsq}, where the most significant bit denotes the sign.

\subsubsection{Floating Point Format}
An $n$-bit signed floating point (FP) number $x$ comprises of a 1-bit sign ($x_{\mathrm{sign}}$), $B_m$-bit mantissa ($x_{\mathrm{mant}}$) and $B_e$-bit exponent ($x_{\mathrm{exp}}$) such that $B_m+B_e=n-1$. The associated constant exponent bias ($E_{\mathrm{bias}}$) is computed as $(2^{{B_e}-1}-1)$. We denote this format as $E_{B_e}M_{B_m}$.  

\subsubsection{Quantization Scheme}
\label{subsec:quant_method}
A quantization scheme dictates how a given unquantized tensor is converted to its quantized representation. We consider FP formats for the purpose of illustration. Given an unquantized tensor $\bm{X}$ and an FP format $E_{B_e}M_{B_m}$, we first, we compute the quantization scale factor $s_X$ that maps the maximum absolute value of $\bm{X}$ to the maximum quantization level of the $E_{B_e}M_{B_m}$ format as follows:
\begin{align}
\label{eq:sf}
    s_X = \frac{\mathrm{max}(|\bm{X}|)}{\mathrm{max}(E_{B_e}M_{B_m})}
\end{align}
In the above equation, $|\cdot|$ denotes the absolute value function.

Next, we scale $\bm{X}$ by $s_X$ and quantize it to $\hat{\bm{X}}$ by rounding it to the nearest quantization level of $E_{B_e}M_{B_m}$ as:

\begin{align}
\label{eq:tensor_quant}
    \hat{\bm{X}} = \text{round-to-nearest}\left(\frac{\bm{X}}{s_X}, E_{B_e}M_{B_m}\right)
\end{align}

We perform dynamic max-scaled quantization \citep{wu2020integer}, where the scale factor $s$ for activations is dynamically computed during runtime.

\subsection{Vector Scaled Quantization}
\begin{wrapfigure}{r}{0.35\linewidth}
  \centering
  \includegraphics[width=\linewidth]{sections/figures/vsquant.jpg}
  \caption{\small Vectorwise decomposition for per-vector scaled quantization (VSQ \citep{dai2021vsq}).}
  \label{fig:vsquant}
\end{wrapfigure}
During VSQ \citep{dai2021vsq}, the operand tensors are decomposed into 1D vectors in a hardware friendly manner as shown in Figure \ref{fig:vsquant}. Since the decomposed tensors are used as operands in matrix multiplications during inference, it is beneficial to perform this decomposition along the reduction dimension of the multiplication. The vectorwise quantization is performed similar to tensorwise quantization described in Equations \ref{eq:sf} and \ref{eq:tensor_quant}, where a scale factor $s_v$ is required for each vector $\bm{v}$ that maps the maximum absolute value of that vector to the maximum quantization level. While smaller vector lengths can lead to larger accuracy gains, the associated memory and computational overheads due to the per-vector scale factors increases. To alleviate these overheads, VSQ \citep{dai2021vsq} proposed a second level quantization of the per-vector scale factors to unsigned integers, while MX \citep{rouhani2023shared} quantizes them to integer powers of 2 (denoted as $2^{INT}$).

\subsubsection{MX Format}
The MX format proposed in \citep{rouhani2023microscaling} introduces the concept of sub-block shifting. For every two scalar elements of $b$-bits each, there is a shared exponent bit. The value of this exponent bit is determined through an empirical analysis that targets minimizing quantization MSE. We note that the FP format $E_{1}M_{b}$ is strictly better than MX from an accuracy perspective since it allocates a dedicated exponent bit to each scalar as opposed to sharing it across two scalars. Therefore, we conservatively bound the accuracy of a $b+2$-bit signed MX format with that of a $E_{1}M_{b}$ format in our comparisons. For instance, we use E1M2 format as a proxy for MX4.

\begin{figure}
    \centering
    \includegraphics[width=1\linewidth]{sections//figures/BlockFormats.pdf}
    \caption{\small Comparing LO-BCQ to MX format.}
    \label{fig:block_formats}
\end{figure}

Figure \ref{fig:block_formats} compares our $4$-bit LO-BCQ block format to MX \citep{rouhani2023microscaling}. As shown, both LO-BCQ and MX decompose a given operand tensor into block arrays and each block array into blocks. Similar to MX, we find that per-block quantization ($L_b < L_A$) leads to better accuracy due to increased flexibility. While MX achieves this through per-block $1$-bit micro-scales, we associate a dedicated codebook to each block through a per-block codebook selector. Further, MX quantizes the per-block array scale-factor to E8M0 format without per-tensor scaling. In contrast during LO-BCQ, we find that per-tensor scaling combined with quantization of per-block array scale-factor to E4M3 format results in superior inference accuracy across models. 

\end{document}
