Using our platinum benchmarks, we identify failures on frontier LLMs on tasks far simpler than their most advanced capabilities. By inspecting the chain-of-thought reasoning processes of models when they fail on these simple tasks, we can better understand the \textit{why} such failures occur. In particular, we identify two patterns of examples that lead to consistent reasoning collapses on frontier LLMs.

\subsection{First event bias} \label{app:patterns-event-bias}

When inspecting model failures on DROP~\cite{dua2019drop}, a reading comprehension dataset, we identified a particular example on which multiple LLMs failed due to the same failure in reasoning:

\begin{tcolorbox}[colback=gray!6, colframe=gray!50, arc=2mm, boxrule=0.5pt]
    \small
    \textbf{Question:} \textit{\{Context paragraph\}} %
    What happened second: Russians blocked Azov or Treaty of Constantinople?\\
    \textbf{Solution:} Treaty of Constantinople

    \medskip

    \vspace{-0.1cm}
    \begin{tcolorbox}[
    colback=white, 
    title={Claude 3.5 Sonnet (June)},
    coltitle=black,
    colbacktitle=gray!25,
    colframe=gray!50,
    arc=2mm,
    boxrule=0.5pt,
    top=3pt,
    bottom=3pt]
    \ldots Given this sequence of events, we can conclude that the Russians blocking Azov happened before the Treaty of Constantinople.
    Answer: Russians blocked Azov
    \end{tcolorbox}
    \begin{tcolorbox}[
    colback=white, 
    title={Claude 3.5 Sonnet (October)},
    coltitle=black,
    colbacktitle=gray!25,
    colframe=gray!50,
    arc=2mm,
    boxrule=0.5pt,
    top=3pt,
    bottom=3pt]
    \ldots Therefore, the Russians blocking Azov (June 27-July 5) happened before the Treaty of Constantinople. Answer: Russians blocked Azov
    \end{tcolorbox}
    \begin{tcolorbox}[colback=white, 
    title={Mistral Small},
    coltitle=black,
    colbacktitle=gray!25,
    colframe=gray!50, arc=2mm, boxrule=0.5pt,
    top=3pt,
    bottom=3pt]
    \ldots Therefore, the Russians blocked Azov first, and the Treaty of Constantinople was signed later. Answer: The Russians blocked Azov second.
    \end{tcolorbox}
    
    \begin{tcolorbox}[colback=white, 
    title={Mistral-Large},
    coltitle=black,
    colbacktitle=gray!25,
    colframe=gray!50,
    arc=2mm,
    boxrule=0.5pt,
    top=3pt,
    bottom=3pt]
    \ldots Therefore, the Russians blocking Azov happened first.
    Answer: Russians blocked Azov
    \end{tcolorbox}
    \begin{tcolorbox}[colback=white, 
    title={Gemini 1.5 Pro},
    coltitle=black,
    colbacktitle=gray!25,
    colframe=gray!50, arc=2mm, boxrule=0.5pt,
    top=3pt,
    bottom=3pt]
    \ldots Since the Treaty of Constantinople followed the events at Azov and the Dnieper campaign, the blockade of Azov happened first.
    Answer: Russians blocked Azov
    \end{tcolorbox}
    \begin{tcolorbox}[colback=white, 
    title={Gemini 1.5 Flash},
    coltitle=black,
    colbacktitle=gray!25,
    colframe=gray!50, arc=2mm, boxrule=0.5pt,
    top=3pt,
    bottom=3pt]
    \ldots Step 2: The passage mentions the Treaty of Constantinople later, after the unsuccessful sieges of Azov and the Dnieper campaign. Answer: Russians blocked Azov
    \end{tcolorbox}
    \begin{tcolorbox}[colback=white, 
    title={Gemini 2.0 Flash},
    coltitle=black,
    colbacktitle=gray!25,
    colframe=gray!50, arc=2mm, boxrule=0.5pt,
    top=3pt,
    bottom=3pt]
    \ldots Therefore, the blocking of Azov happened first. Answer: Russians blocked Azov
    \end{tcolorbox}
    \begin{tcolorbox}[colback=white, 
    title={Qwen 2.5 72B Instruct},
    coltitle=black,
    colbacktitle=gray!25,
    colframe=gray!50, arc=2mm, boxrule=0.5pt,
    top=3pt,
    bottom=3pt]
    \ldots Therefore, the Russians blocking Azov (June 27-July 5) happened before the Treaty of Constantinople. Answer: Russians blocked Azov
    \end{tcolorbox}
    \begin{tcolorbox}[colback=white, 
    title={Deepseek-R1},
    coltitle=black,
    colbacktitle=gray!25,
    colframe=gray!50, arc=2mm, boxrule=0.5pt,
    top=3pt,
    bottom=3pt]
    \ldots So the blocking of Azov happened first, then the Treaty later. Answer: Russians blocked Azov
    \end{tcolorbox}
\end{tcolorbox}
\vspace{0.2cm}

\begin{figure}
    \centering
    \includegraphics[width=1\linewidth]{figures/event_bias.pdf}
    \caption{\textbf{First event bias.} Accuracy of models on synthetically generated questions of the form  ``what happened second: X or Y'' given a context. Most models exhibit frequent failures on these problems despite their relative simplicity, and three models (Gemini 1.5 Flash and Pro, Mistral Small) almost always respond with the first event instead.}
    \label{appfig:event_bias}
\end{figure}

\noindent Even though the question asks what happened second between two events, each model not only answers with the first event, but even explicitly states that it identified the event that happened first.

To further investigate this failure, we construct 50 additional examples in the same format. To construct each example, we use GPT-4o to generate a paragraph-long context describing a historical topic with a sequence of events, and ask it to select two events within this context that have a clear chronological order. To avoid bias stemming from the ordering of the events in the question, we repeat each example twice with the event orders flipped (i.e., ``what happened second: X or Y'' and ``what happened second: Y or X''), leading to 100 total examples.

We find that three models---Gemini 1.5 Flash, Gemini 1.5 Pro, and Mistral Small---fail on over 85\% of the examples, despite random guessing achieving 50\%. Many other models exhibit an error rate of at least 25\%, whereas the median error rate on the overall DROP benchmark, with questions of similar difficulty, was under 3\% (see Figure \ref{appfig:event_bias}).

\subsection{Rounding up primes} \label{app:patterns-rounding}

When inspecting model failures on SVAMP~\citep{patel2021nlp} and GSM8K~\citep{cobbe2021training}, we found two cases in which Claude 3.5 Sonnet (June) rounded up its answer following a division even though the quotient was already a whole number:

\begin{tcolorbox}[colback=gray!6, colframe=gray!50, arc=2mm, boxrule=0.5pt]
    \small
    \textbf{Question:} The school is planning a field trip. The school has 67 classrooms. There are 66 students in each classroom in the school. If there are 6 seats on each school bus. How many buses are needed to take the trip?\\
    \textbf{Solution:} 737
    
    \medskip
    \vspace{-0.1cm}
    \begin{tcolorbox}[
    title={Claude 3.5 Sonnet (June)},
    coltitle=black,
    colbacktitle=gray!25,
    colback=white, 
    colframe=gray!50,
    arc=2mm,
    boxrule=0.5pt]
    \ldots To find the number of buses needed, we divide the total number of students by the number of seats per bus:
4,422 ÷ 6 = 737. However, since we can't have a fraction of a bus, we need to round up to the next whole number to ensure all students have a seat. Therefore, the school needs 738 buses to take all students on the trip. Answer: 738
    \end{tcolorbox}
    \end{tcolorbox}

\begin{tcolorbox}[colback=gray!6, colframe=gray!50, arc=2mm, boxrule=0.5pt]
    \small
    \textbf{Question:} Cecilia just bought a new puppy. According to her veterinarian, she has to feed the puppy 1 cup of dog food every day for the first 180 days. Then she has to feed the puppy 2 cups of dog food every day for the rest of its life. If one bag of dog food contains 110 cups, how many bags of dog food will Cecilia use in the first year?\\
    \textbf{Solution:} 5
    
    \medskip
    \vspace{-0.1cm}
    \begin{tcolorbox}[
    title={Claude 3.5 Sonnet (June)},
    coltitle=black,
    colbacktitle=gray!25,
    colback=white, 
    colframe=gray!50,
    arc=2mm,
    boxrule=0.5pt]
    \ldots Calculate how many bags of dog food this equals:
550 cups ÷ 110 cups per bag = 5 bags. However, since Cecilia can't buy a fraction of a bag, she'll need to round up to the next whole bag. Answer: 6
    \end{tcolorbox}
    \end{tcolorbox}

\vspace{0.2cm}

\begin{figure}
    \centering
    \includegraphics[width=0.75\linewidth]{figures/rounding.pdf}
    \caption{\textbf{Rounding up primes.} Frequency of failures of Claude Sonnet 3.5 (June) on procedurally generated questions involving only a single division step, when the quotient (i.e., the result of the division) has a given number of non-trivial divisor pairs. We filter out failures caused by a failed division, to only include cases where the model intentionally (and incorrectly) rounds up a whole number.}
    \label{appfig:rounding}
\end{figure}

Interestingly, the solutions to these questions both have few divisors (5 is prime, and 737 is the product of 11 and 67, two primes). So, we investigate whether this failure is related to the ``closeness'' of the solution to being prime. Specifically, we create three examples similar to the bus question above that require only a single division, such as the following:
\begin{quote}
\small
\textbf{Question:} A tour group with $\{n*k\}$ people needs to hire buses to travel to their next destination. If each bus can fit $\{k\}$ people, how many buses does the tour group need?\\
\textbf{Solution:} $n$
\end{quote}

We randomly select values of $n$ ranging from 3 to 300, with 0, 1, 2, or 3 non-trivial divisor pairs and measure the average number of failures when varying $k$ from 3 to 9. We isolate only errors caused from incorrectly rounding the whole number quotient by filtering our failures caused by an incorrect division. We find that indeed, the less non-trivial divisor pairs $n$ has (i.e., the closer it is to being prime), the more frequently this failure mode occurs (see Figure \ref{appfig:rounding}).


















