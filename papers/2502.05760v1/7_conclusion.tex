\section{Conclusion}

In this paper, we propose \system, a framework for diversity-aware replay in continual learning specially designed for the challenging setting of malware classification. Our comprehensive evaluation across Domain-IL, Class-IL, and Task-IL scenarios against Windows executable (EMBER) and Android application (AZ) datasets demonstrates that diversity-aware sampling is helpful for effective CL in malware classification. %In Domain-IL scenarios, IFS-R and AWS-R, and in Class-IL scenarios, IFS-U and AWS-U, demonstrate superior adaptability and resource efficiency. 
%These findings affirm the value of our proposed methods in the development of an ever-evolving malware classification system. 
As malware and goodware continue to evolve, these insights steer continual learning towards strategic, resource-efficient methods, ensuring model effectiveness amid the constantly shifting landscape of cybersecurity threats.

%Our extensive evaluation in Domain-IL and Class-IL scenarios in the EMBER and AZ datasets reveals that strategic sampling plays a vital role in continual learning for malware classification. The Ratio variants (IFS-R and AWS-R) perform well in Domain-IL, especially at higher budgets, approaching joint training baselines. In Class-IL, the Uniform variants (IFS-U and AWS-U) consistently outperform other methods, effectively utilizing resources and adapting to new classes to reduce catastrophic forgetting.