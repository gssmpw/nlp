\section{Exploratory Analysis of EMBER}
\label{sec:exploratoryAnalysis}

\begin{figure}[!t]
\tiny
\begin{minipage}[c]{\linewidth}
\centering
\begin{tikzpicture}[scale=1.5]

    \tikzstyle{every node}=[font=\small]
    % Scaling factor for the heights
    \def\scalefactor{0.001}
    
    % Define data for New and Learned Families
    \def\NewFamilies{{913, 425, 257, 208, 183, 184, 118, 133, 287, 23, 66, 102}}
    \def\LearnedFamilies{{0, 551, 641, 596, 726, 761, 658, 784, 873, 370, 508, 652}}

    % Define months
    \def\Months{{"Jan","Feb","Mar","Apr","May","Jun","Jul","Aug","Sep","Oct","Nov","Dec"}}
    
    % Loop through each month
    \foreach \x [count=\xi] in {0,...,11}{
        % Accessing the \x-th element of each list
        \pgfmathsetmacro\NewFamily{\NewFamilies[\x]}
        \pgfmathsetmacro\LearnedFamily{\LearnedFamilies[\x]}
        
        % Calculate bottom and top of the Learned Families bar
        \pgfmathsetmacro\NewHeight{\NewFamily*\scalefactor}
        \pgfmathsetmacro\LearnedHeight{(\NewFamily + \LearnedFamily)*\scalefactor}
        
        % Position for each month's data
        \pgfmathsetmacro\position{\xi*0.35}
        
        % Draw New Families (bottom bar)
        \fill[green!40] (\position,0) rectangle (\position+0.275,\NewHeight);
        % Draw Learned Families (stacked on top of New Families)
        \fill[red!40] (\position,\NewHeight) rectangle (\position+0.275,\LearnedHeight);
        
        % Add value labels for New and Learned Families
        \node at (\position+0.15,\NewHeight/2) {\scriptsize\NewFamily};
        % Display LearnedFamily value only if it's greater than 0
        \ifnum\LearnedFamily = 0
            \node at (\position+0.15,\LearnedHeight*1.05) {\scriptsize\LearnedFamily};
        \fi
         \ifnum\LearnedFamily > 0
            \node at (\position+0.15,\LearnedHeight/1.2) {\scriptsize\LearnedFamily};
        \fi
    }
     % Add month label below the x-axis
    %\node[rotate=45] at (\position+0.15,-0.2) {\tiny\Months[\x]};


    \node[rotate=45] at (0.35,-0.25) {Jan};
    \node[rotate=45] at (0.8,-0.25) {Feb};
    \node[rotate=45] at (1.15,-0.25) {Mar};
    \node[rotate=45] at (1.45,-0.25) {Apr};
    \node[rotate=45] at (1.80,-0.25) {May};
    \node[rotate=45] at (2.15,-0.25) {June};
    \node[rotate=45] at (2.55,-0.25) {July};
    \node[rotate=45] at (2.90,-0.25) {Aug};
    \node[rotate=45] at (3.25,-0.25) {Sep};
    \node[rotate=45] at (3.55,-0.25) {Oct};
    \node[rotate=45] at (3.95,-0.25) {Nov};
    \node[rotate=45] at (4.35,-0.25) {Dec};


    % Draw the axes
    \draw[->] (0.0,-0.05) -- (4.65,-0.05) node[right] {Months};
    \draw[->] (0.05,-0.2) -- (0.05,1.45) node[rotate=90, midway, above] {Frequency};
    

    % Add legends (moved to top right)
    \draw [black!90, fill=green!40] (.3,1.3) rectangle (0.5,1.4) node[right, xshift=0.1cm] {\small New Families};
    \draw [black!90, fill=red!40] (2.0,1.3) rectangle (2.20,1.4) node[right, xshift=0.1cm] {\small Learned Families};
    
\end{tikzpicture}


\vskip -0.1cm
\caption{New and already learned families in each task.}
\label{fig:ember_frequency_families}
\end{minipage}%
\vspace{-0.3cm}
\end{figure}




In this section, we provide an analysis of the EMBER dataset, which sheds light on the distribution across various families and tasks, aiding in selecting representative samples for replay. We identified 2,899 unique malware families within a subset of EMBER, and an additional 11,433 samples lacking clear family labels were assigned the label {\em Other}.

We investigate the prevalence of malware families over time, distinguishing between recurring and newly identified families each month in Figure~\ref{fig:ember_frequency_families}. Unlike many datasets used in CL research, we see significant churn in the representation of families over time. Of the 913 families seen in January, for example, only 551 are seen in February, while 425 new families emerge. This churn indicates a potential issue in training data continuity, which may aggravate catastrophic forgetting and underscores the need for different CL strategies in the malware domain. Table~\ref{ember_task_family_stat} shows the number of goodware and malware samples in each month, as well as the number of families. Generally, each family has its own distinct distribution pattern, and together these patterns make up the total distribution of malware for a particular month. 

Worse, many malware samples do not have family labels at all (see Figure~\ref{fig:ember_noavclass}). Correctly labeling samples is a challenging endeavor and often takes time and expert knowledge~\cite{kantchelian2015better}, so this lack of labels matches real-world conditions. Family labels for malware are based on the {\em av-class} labels provided by the av-test engine~\cite{av-test}. Analysis of the \emph{Other}-labeled samples do not align with the labeled families, which shows that many of them indeed come from unknown families. 


\begin{table}
\scriptsize
\centering
\caption{Number of samples and families per task in EMBER}
\begin{tabular}{l|c|c|c} 
% \hline \hline
\textbf{Task} & \textbf{\#of Goodware} & \textbf{\#of Malware} & \textbf{\#of Families}\\ 
\hline
%\rule{0pt}{2ex}

January     &   29423   &   32491    &  913       \\
February    &   22915   &   31222    &  976      \\
March       &   21373   &   20152    &  898      \\
April       &   25190   &   26892    &  804      \\
May         &   23719   &   22193    &  909      \\
June        &   23285   &   25116    &  945      \\
July        &   24799   &   26622    &  776      \\
August      &   23634   &   21791    &  917      \\
September   &   26707   &   37062    &  1160      \\
October     &   29955   &   56459    &  393      \\
November    &   50000   &   50000    &  574      \\
December    &   50000   &   50000    &  754      \\

\bottomrule

%\rule{0pt}{3ex}
\end{tabular}
\label{ember_task_family_stat}
\vskip -0.4cm   
\end{table}



\begin{figure}[t]
\begin{minipage}[c]{\linewidth}
\centering
\begin{tikzpicture}[scale=1.]
    % Define the style for the bars and nodes
    \tikzstyle{every node}=[font=\small]
    %\tikzset{bar width/.initial=10pt}
    
    % Scaling factor for the heights
    \newcommand{\scalefactor}{0.001}
    
    % Drawing the bars for each month with reduced space
    \draw[fill=blue!40] (0,0) rectangle (0.3,1768*\scalefactor); % Jan
    \draw[fill=blue!40] (0.5,0) rectangle (0.8,1854*\scalefactor); % Feb
    \draw[fill=blue!40] (1,0) rectangle (1.3,1072*\scalefactor); % Mar
    \draw[fill=blue!40] (1.5,0) rectangle (1.8,831*\scalefactor); % Apr
    \draw[fill=blue!40] (2,0) rectangle (2.3,663*\scalefactor); % May
    \draw[fill=blue!40] (2.5,0) rectangle (2.8,1153*\scalefactor); % Jun
    \draw[fill=blue!40] (3,0) rectangle (3.3,742*\scalefactor); % Jul
    \draw[fill=blue!40] (3.5,0) rectangle (3.8,1201*\scalefactor); % Aug
    \draw[fill=blue!40] (4,0) rectangle (4.3,1354*\scalefactor); % Sep
    \draw[fill=blue!40] (4.5,0) rectangle (4.8,116*\scalefactor); % Oct
    \draw[fill=blue!40] (5,0) rectangle (5.3,204*\scalefactor); % Nov
    \draw[fill=blue!40] (5.5,0) rectangle (5.8,475*\scalefactor); % Dec
    
    % Add slanted labels below each bar for months
    \node[rotate=45] at (0.15,-0.25) {Jan};
    \node[rotate=45] at (0.65,-0.25) {Feb};
    \node[rotate=45] at (1.15,-0.25) {Mar};
    \node[rotate=45] at (1.65,-0.25) {Apr};
    \node[rotate=45] at (2.15,-0.25) {May};
    \node[rotate=45] at (2.65,-0.25) {June};
    \node[rotate=45] at (3.15,-0.25) {July};
    \node[rotate=45] at (3.65,-0.25) {Aug};
    \node[rotate=45] at (4.15,-0.25) {Sep};
    \node[rotate=45] at (4.65,-0.25) {Oct};
    \node[rotate=45] at (5.15,-0.25) {Nov};
    \node[rotate=45] at (5.65,-0.25) {Dec};
    
    % Add vertical frequency labels above each bar
    \node[rotate=90] at (0.15,1768*\scalefactor-0.5) {1768};
    \node[rotate=90] at (0.65,1854*\scalefactor-0.5) {1854};
    \node[rotate=90] at (1.15,1072*\scalefactor-0.5) {1072};
    \node[rotate=90] at (1.65,831*\scalefactor-0.5) {831};
    \node[rotate=90] at (2.15,663*\scalefactor-0.3) {663};
    \node[rotate=90] at (2.65,1153*\scalefactor-0.5) {1153};
    \node[rotate=90] at (3.15,742*\scalefactor-0.4) {742};
    \node[rotate=90] at (3.65,1201*\scalefactor-0.5) {1201};
    \node[rotate=90] at (4.15,1354*\scalefactor-0.5) {1354};
    \node[rotate=90] at (4.65,116*\scalefactor+0.3) {116};
    \node[rotate=90] at (5.15,204*\scalefactor+0.3) {204};
    \node[rotate=90] at (5.65,475*\scalefactor+0.3) {475};
    
    % Draw the axes
    \draw[->] (-0.5,0) -- (6.5,0) node[right] {Months};
    
    % Draw the y-axis with label alongside
    \draw[->] (-0.3,-0.2) -- (-0.3,2) node[rotate=90, midway, above] {Frequency};
    
\end{tikzpicture}



\vskip -0.25cm
\caption{EMBER Malware samples without AV-Class labels.}
\label{fig:ember_noavclass}
\end{minipage}%
\vskip -0.4cm
\end{figure}


% Furthermore, we also carried out an analysis to examine how consistently the top malware families appeared across different tasks. This helps us track the trends among the most common malware families in the EMBER dataset. For each task, we identified the 10 most common malware families, determined by the frequency of samples from each family. Our analysis show that the top 10 families changes over time, as the most prevalent families at the start did not stay prevalent in later tasks. For example, the \texttt{emotet} malware family was the most consistent, appearing in 11 out of 12 tasks. The next most consistent families were \texttt{fareit} and \texttt{zusy}, appearing in eight and seven of the tasks, respectively. This shows considerable concept drift in malware data, making it necessary to regularly update classifiers.

Furthermore, our analysis shows that the prominent malware families change with the evolution of tasks. We identified the 10 most common malware families based on the frequency of samples from each family and found that the top 10 families vary across tasks. The most prevalent families at the beginning of the experiment (i.e., January) do not remain prevalent in later tasks (i.e., from February on). For example, the \texttt{emotet} malware family was the most consistent, appearing in 11 out of 12 tasks. The next most consistent families were \texttt{fareit} and \texttt{zusy}, appearing in eight and seven tasks, respectively. This indicates considerable concept drift in malware data, highlighting the need to regularly update classifiers.

In addition, many malware families display complex distributional patterns in feature space, making for additional diversity within classes. Figure~\ref{fig:ember_jan_mal}, for example, shows a t-SNE projection of the EMBER features for all malware samples from January 2018. Each class (represented by color) is not clustered into a single well-defined region. Rather, the larger classes are split up into subsets that spread out in feature space. To accurately represent the malware distribution, it is thus important to select samples not only from each class, but also from multiple areas within each class.

Despite some degree of overlap among different families, it is possible to discern distinct clusters indicative of the diversity across and even within malware families. It is vital to capture samples from each of these smaller clusters, including those belonging to minority families, to accurately represent the malware landscape for a specific task month. 
To further complicate the situation, it is often not possible to provide definitive family labels for a sample due to the inherent subjectivity involved in malware analysis, which results in a large, diverse set of additional unlabeled samples which must be considered~\cite{kantchelian2015better}. These attributes of the malware landscape make it difficult to characterize classes as contiguous regions of the feature space with relatively simple class boundaries. Rather, each class might only be understood as a collection of smaller pockets of the feature space that might be closer to pockets from other classes than the same class. This may explain why prior CL techniques designed for computer vision datasets are less effective when applied to the malware domain~\cite{continual-learning-malware}. 



% These family-specific distributions are, in essence, sub-distributions of a broader global distribution, which underscores the significant diversity in real-world malware. 
%To further complicate the situation, many malware samples do not have family labels and correctly labeling samples often takes time and expert knowledge~\cite{kantchelian2015better}, so the lack of labels matches real-world conditions. %(see Figure~\ref{fig:ember_noavclass}). Family labels for malware are 
%based on the {\em av-class} labels 
%provided by the av-test engine~\cite{av-test}. %The \emph{other}-labeled samples indeed do not seem from our analysis to align with other families, meaning that many of them come from unknown families. 



% Table~\ref{ember_task_family_stat} provides a snapshot of the count of goodware and malware samples for each month, along with the unique malware families encountered in the corresponding month.


%\begin{table}[!t]
\small
% \def\arraystretch{1.35}
\centering
\caption{Top 10 most frequent malware families among 12 task months of EMBER.}
\vspace{-0.25cm}
\begin{tabular}{l|c||l|c}
% \toprule
\textbf{Family} & \textbf{Frequency} & \textbf{Family} & \textbf{Frequency} \\ \hline
Emotet & 11 & Upatre & 5 \\ 
Fareit & 8 & Installmonster & 4 \\
High & 8 &  zbot & 4\\ 
Zusy & 7 & Gandcrab & 3 \\ 
Adposhel & 4 & dealply & 3\\ 
\bottomrule
\end{tabular}
\label{tab:top10_families}
\vspace{-0.3cm}
\end{table}



\begin{figure}[!t]
\centering
\vskip -0.3cm
\includegraphics[scale=0.20,trim={0.5cm 0.8cm 0.5cm 0.5cm},clip]{figures/tsne_db_scan_mal.png}
\vskip -0.3cm
\caption{t-SNE projection of EMBER malware from Jan. 2018.}
\label{fig:ember_jan_mal}
\vskip -0.4cm
\end{figure}


%, which is a challenging issue in its own right~\cite{cade,transcendingtranscend} that continual learning methods should address. 

In light of these analysis, we propose that selecting samples based on families and variations within those families could more effectively capture the diversity within the replay data distribution, potentially mitigating catastrophic forgetting. 





