\begin{abstract}

Millions of new pieces of malicious software (i.e., malware) are introduced each year. This poses significant challenges for antivirus vendors, who use machine learning to detect and analyze malware, and must keep up with changes in the distribution while retaining knowledge of older variants. Continual learning (CL) holds the potential to address this challenge by reducing the storage and computational costs of regularly retraining over all the collected data. Prior work, however, shows that CL techniques, which are designed primarily for computer vision tasks, fare poorly when applied to malware classification. To address these issues, we begin with an exploratory analysis of a typical malware dataset, which reveals that malware families are diverse and difficult to characterize, requiring a wide variety of samples to learn a robust representation. Based on these findings, we propose $\underline{M}$alware $\underline{A}$nalysis with $\underline{D}$iversity-$\underline{A}$ware $\underline{R}$eplay (MADAR), a CL framework that accounts for the unique properties and challenges of the malware data distribution. Through  extensive evaluation on large-scale Windows and Android malware datasets, we show that MADAR significantly outperforms prior work. This highlights the importance of understanding domain characteristics when designing CL techniques and demonstrates a path forward for the malware classification domain. 

\end{abstract}

% \begin{IEEEkeywords}
% Malware Analysis; Windows Malware; Continual Learning; Catastrophic Forgetting
% \end{IEEEkeywords}


