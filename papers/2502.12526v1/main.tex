%%
%% This is file `sample-manuscript.tex',
%% generated with the docstrip utility.
%%
%% The original source files were:
%%
%% samples.dtx  (with options: `manuscript')
%% 
%% IMPORTANT NOTICE:
%% 
%% For the copyright see the source file.
%% 
%% Any modified versions of this file must be renamed
%% with new filenames distinct from sample-manuscript.tex.
%% 
%% For distribution of the original source see the terms
%% for copying and modification in the file samples.dtx.
%% 
%% This generated file may be distributed as long as the
%% original source files, as listed above, are part of the
%% same distribution. (The sources need not necessarily be
%% in the same archive or directory.)
%%
%%
%% Commands for TeXCount
%TC:macro \cite [option:text,text]
%TC:macro \citep [option:text,text]
%TC:macro \citet [option:text,text]
%TC:envir table 0 1
%TC:envir table* 0 1
%TC:envir tabular [ignore] word
%TC:envir displaymath 0 word
%TC:envir math 0 word
%TC:envir comment 0 0
%%
%%
%% The first command in your LaTeX source must be the \documentclass command.
\documentclass[sigconf]{acmart}
%\documentclass[authorversion,sigconf,nonacm]{acmart}
%\documentclass[acmlarge,anonymous,review]{acmart}

%% \BibTeX command to typeset BibTeX logo in the docs
\AtBeginDocument{%
  \providecommand\BibTeX{{%
    \normalfont B\kern-0.5em{\scshape i\kern-0.25em b}\kern-0.8em\TeX}}}

%% Rights management information.  This information is sent to you
%% when you complete the rights form.  These commands have SAMPLE
%% values in them; it is your responsibility as an author to replace
%% the commands and values with those provided to you when you
%% complete the rights form.



\copyrightyear{2025}
\acmYear{2025}
\setcopyright{acmcopyright}
\acmConference[Conference acronym 'XX]{Make sure to enter the correct
  conference title from your rights confirmation email}{June 03--05,
  2018}{Woodstock, NY}
\acmPrice{15.00}
\acmDOI{10.1145/3491102.3501940}
\acmISBN{978-1-4503-9157-3/22/04}



%%
%% Submission ID.
%% Use this when submitting an article to a sponsored event. You'll
%% receive a unique submission ID from the organizers
%% of the event, and this ID should be used as the parameter to this command.
%\acmSubmissionID{2446}

%%
%% The majority of ACM publications use numbered citations and
%% references.  The command \citestyle{authoryear} switches to the
%% "author year" style.
%%
%% If you are preparing content for an event
%% sponsored by ACM SIGGRAPH, you must use the "author year" style of
%% citations and references.
%% Uncommenting
%% the next command will enable that style.
%%\citestyle{acmauthoryear}
\usepackage{subcaption}
\usepackage{tabularx} 
\usepackage{booktabs}
\usepackage{geometry}
\usepackage{colortbl}
\usepackage{xcolor}
\usepackage{soul}
\usepackage{_macros}
\usepackage{caption}
\usepackage{subcaption}
\usepackage{array}
\usepackage{tikz}
\usepackage{hyperref}
\usepackage{hyperxmp}   


\newcommand*\circled[1]{\tikz[baseline=(char.base)]{
            \node[shape=circle,draw,inner sep=1pt] (char) {#1};}}


\newcommand{\name}{\textbf{AnimAlte}}
\sethlcolor{blue!50!red!50!white}


\raggedbottom

%%
%% end of the preamble, start of the body of the document source.
\begin{document}

%%
%% The "title" command has an optional parameter,
%% allowing the author to define a "short title" to be used in page headers.
\title{AnimAlte: Designing AI-Infused Cartoon Videos to Improve Preschoolers' Language Learning with Family Engagement at Home}

\author{Shiya Tsang}
\orcid{0009-0008-4338-3570}
\affiliation{%
  \institution{Hong Kong University of Science and Technology (Guangzhou)}
%   \streetaddress{No.1 DuXue Road}
  \city{Guangzhou}
%   \state{Guangdong}
  \country{China}
%   \postcode{511458}
}
\email{szeng785@connect.hkust-gz.edu.cn}

\author{Ruiyao Miao}
\orcid{0009-0002-1848-4421}
\affiliation{%
  \institution{University of California, Los Angeles}
%   \streetaddress{405 Hilgard Ave, Los Angeles, CA}
  \city{California}
%   \state{LA}
  \country{United States}
%   \postcode{90024}
}
\email{ruiyao0809@g.ucla.edu}

\author{Junren Xiao}
\orcid{0009-0006-5633-2268}
\affiliation{%
  \institution{Hong Kong University of Science and Technology (Guangzhou)}
%   \streetaddress{No.1 DuXue Road}
  \city{Guangzhou}
%   \state{Guangdong}
  \country{China}
%   \postcode{511458}
}
\email{jxiao767@connect.hkust-gz.edu.cn}

\author{Hui Xiong}
\authornote{Corresponding Author.}
\orcid{0000-0001-6016-6465}
\affiliation{%
  \institution{Hong Kong University of Science and Technology (Guangzhou)}
%   \streetaddress{No.1 DuXue Road}
  \city{Guangzhou}
%   \state{Guangdong}
  \country{China}
%   \postcode{511458}
}
\email{xionghui@hkust-gz.edu.cn}



%%
%% The "author" command and its associated commands are used to define
%% the authors and their affiliations.
%% Of note is the shared affiliation of the first two authors, and the
%% "authornote" and "authornotemark" commands
%% used to denote shared contribution to the research.



%%
%% By default, the full list of authors will be used in the page
%% headers. Often, this list is too long, and will overlap
%% other information printed in the page headers. This command allows
%% the author to define a more concise list
%% of authors' names for this purpose.
%\renewcommand{\shortauthors}{}

%%
%% The abstract is a short summary of the work to be presented in the
%% article.
\begin{abstract}

Cartoon videos have proven to be effective in learning vocabulary to preschool children. However, we have little knowledge about integrating AI into cartoon videos to provide systematic, multimodal vocabulary learning support. This late-breaking work present \name{}, an AI-powered cartoon video system that enables real-time Q\&A, vocabulary review, and contextual learning. Preliminary findings contextualized how families interact with \name{} to support vocabulary learning. Parents appreciated the system for its personalized, engaging experiences, fostering collaboration, and encouraging self-reflection on parenting. This study offers valuable design implications for informing future video systems to support vocabulary learning.

%finding1 - 不同的环境中时间
%艺术创作+语言表达不同地方时间疗愈好处的between session activities
%however, 在没有治疗师的情况下如何引导用户通过语言和艺术创作的方式进行自我表达是有挑战的;
% 如何支持治疗师定制家庭作业练习并且追踪利用家庭作业历史数据是有挑战的。
% In HCI,很少有研究探索如何使用技术媒介支持艺术治疗家庭作业的上述挑战。
% 我们present \name{}, human-ai co-creative art-making and conversational interaction 支持了xxx,包含了client application ; therapist application,新手的agent总结
%为期一个月与24个用户在5个治疗师的引导使用了我们的系统引导下进行了实践,
% 研究发现了
%艺术治疗家庭作业是一个通过艺术创作+语言表达在不同地方和时间给予来访者很多艺疗愈好处的between session activities
%==================================================
% 
%门槛高缺乏指导 - 从client角度。
%直接用technology
% customize AI agent 来taiolor homework
% 24用户自己的context进行在五个治疗师指导下
% agent指导homework, art-making同时对话探索image的meanings,作为self-exploration过程
% 自己专业信念和自己的疗愈经验,自己的对话融入到chatbot拥有治疗师的个人印记
%Art therapy homework is between-session activities that facilitate  reflecting on clients' daily feelings and enhance therapeutic collaboration through art-making and verbal expression in their natural environment.
%However, it is challenging to support clients in self-expression through verbalization and art due to a high threshold and limited guidance from therapists, as well as to support therapists in customizing and tracking homework exercises.
%In HCI, few studies have explored how technology can support addressing these challenges of art therapy homework.
%Thus, we present \name{}, a system consisting of a client application that integrates human-AI co-creative art-making and conversational agents for art therapy homework, and a therapist application that allows for the customization of conversational agents to guide homework and review AI-compiled homework data.
%Over a one-month field deployment, 24 clients engaged with \name{} in their own contexts, under the guidance of 5 therapists. Our findings showed that introducing conversational agents while art-making can guide clients in exploring personal feelings and new meanings behind the artwork. Also, \name{} enables therapists to integrate their professional beliefs and practical experiences into the AI agent, allowing it to reflect their unique personal imprint. Homework images and conversation records also helped therapists offer triggers for deeper discussions and intervention resources during in-session activities.
%Building on these findings, we explore the practical implications of incorporating AI agents into therapy homework on future design.
%Art therapy homework is essential for fostering clients' reflection on daily experiences between sessions. However, current practices present challenges: clients often lack guidance for completing tasks that combine art-making and verbal expression, while therapists find it difficult to track and tailor homework. How HCI systems might support art therapy homework remains underexplored. To address this, we present \name{}, comprising a client-facing application leveraging human-AI co-creative art-making and conversational agents to facilitate homework, and a therapist-facing application enabling customization of homework agents and AI-compiled homework history. A 30-day field study with 24 clients and 5 therapists showed how \name{} supported clients’ homework and reflection in their everyday settings. Results also revealed how therapists infused their practice principles and personal touch into the agents to offer tailored homework, and how AI-compiled homework history became a meaningful resource for in-session interactions. Implications for designing human-AI systems to facilitate asynchronous therapist-client collaboration are discussed.


%can help clients express emotions and find new meanings behind their artwork, even without therapists. Additionally, 

%Therapy homework has been widely used in art therapy practices to enhance clients' therapeutic skills in real world and offer continuity between sessions.
% 
%Yet, few studies explored how AI agents could support clients’ art therapy homework and mediate therapist-client collaboration outside the synchronous art therapy sessions.
%
%To address this, we introduce \name{}, a multi-agent system that combines image-based generative AI with conversational AI agents, empowering clients to engage in art therapy homework in natural environments. It can also support therapists customizing and reviewing therapy homework.
%Through one month field deployments involving 24 clients with five art therapists, our results demonstrated that \name{} helped clients externalize their emotions and create new meanings through conversation and art-making across various contexts, such as home or offices.
%AI multi-agents can act as a mediator, enabling therapists to customize them as an extension of their practice to deliver structured guidance aligned with their professional beliefs, while also supporting the use of homework as intervention resources during art therapy sessions.
\end{abstract}

%%
%% The code below is generated by the tool at http://dl.acm.org/ccs.cfm.
%% Please copy and paste the code instead of the example below.
%%
\begin{CCSXML}
<ccs2012>
   <concept>
       <concept_id>10003120.10003121.10011748</concept_id>
       <concept_desc>Human-centered computing~Empirical studies in HCI</concept_desc>
       <concept_significance>500</concept_significance>
       </concept>
 </ccs2012>
\end{CCSXML}

\ccsdesc[500]{Human-centered computing~Empirical studies in HCI}



%%
%% Keywords. The author(s) should pick words that accurately describe
%% the work being presented. Separate the keywords with commas.
\keywords{Vocabulary learning, video-based learning, children, cartoon video, visual language model, large language model}


%% teaser figure 

\begin{teaserfigure}
  \includegraphics[width=\textwidth]{images/shoutu.pdf}
  \vspace{-7mm}
  \caption{\name{} facilitates families to learning vocabulary through real-time question-answering, active review with questions and feedback,real-world associations by linking animated and real-life images, and contextual expansion with sentences or stories.}
  \Description{Enjoying the baseball game from the third-base
  seats. Ichiro Suzuki preparing to bat.}
  \label{fig:teaser}
\end{teaserfigure}


%%
%% This command processes the author and affiliation and title
%% information and builds the first part of the formatted document.

\maketitle

\section{Introduction}

% State of the world (robots for creative activites)
The term ``robot,'' originally signifying `forced labor,' has long been associated with labor and work. Robots have demonstrated their utility in various automated productive and social contexts, where the primary goals are improving productivity, safety, and fostering social interactions with humans~\cite{simoes2022designing, weidemann2021role, honig2018understanding}. However, an increasing number of cases feature using of robots in creative settings. Unlike productive contexts, where the focus is on efficiency and task completion~\cite{arents2022smart}, or social contexts, where communication and trust are prioritized~\cite{nam2020trust, saunderson2019robots}, creative environments prioritize artistic innovation and expression~\cite{hsueh2024counts}. This shift fundamentally alters the dynamics of human-robot interaction, redefining the roles and expectations for both humans and robots.

For instance, robots’ social behaviors are leveraged to support the generation and expression of creative ideas~\cite{hu2021exploring, sandoval2022human, alves2020creativity}, and programmable robotic movements and trajectories are employed to inspire artistic activities such as sketching~\cite{lin2020your}. These studies often engage participants from creative fields who possess limited prior experience with robotics, and are typically conducted in short-term, experimental settings. Consequently, the findings from these studies remain constrained since much can be learned from professional practitioners' experiences to inform system design such as digital fabrication~\cite{hirsch2023nothing}. There is a notable gap in research examining the long-term, active, and practical experience of integrating robotic systems into the creative processes. As a result, the deeper insights into how robots facilitate and shape creative processes, beyond simply augmenting human creativity, remain underexplored. In this study, we aim to better understand the impacts of robots on creative processes and outcomes.

As early as Leonardo da Vinci's 16th century ``Automaton,'' artists have explored the creative affordances of robotic systems~\cite{shanken2002cybernetics, pagliarini2009development, jeon2017robotic}. The artistic creation process typically encompasses various stages, including the exploration of materials and techniques, ongoing experimentation and iteration, and the continual refinement of the artists' insights into their creative subjects~\cite{lewis2023art, sturdee2022state}. Therefore, investigating the artistic process involving robots offers an opportunity to gain deeper insights into robots' creative potential. Robotic art, in particular, provides a compelling case for this exploration.

We define robotic art as artworks that utilize robotic or automated machines to create artistic experiences and tangible artifacts. One example is robotic installation art, in which robots are programmed to follow specific rules that embody the artist’s expression (\autoref{fig:teaser} (a)). Another example is responsive art, in which robots react to their environment, with behaviors that change over time or in response to spectators (\autoref{fig:teaser} (b)). Additionally, there are robotic creators, which possess a degree of agency, allowing them to collaborate with human artists and produce works that extend beyond mere replication of human-created art (\autoref{fig:teaser} (c) and (d)). As such, robotic art becomes a rich case for exploring human-machine interactions in creative contexts. Gaining a deeper understanding of how robots facilitate artistic expression can provide insights for designing computing systems to support creative activities~\cite{gomez2021robot}.

% Therefore, we did...
We draw on semi-structured, in-depth interviews with renowned professional robotic artists to investigate the use of robots in artistic practice. Specifically, our goal is to understand how artistic exploration of robotic systems challenges conventional assumptions about the functions of robots, such as their roles in automating repetitive tasks or serving human needs. We also explore the implications of robots in the artistic process and examine how creativity may emerge within robotic art. To address these interrelated inquiries, our study focuses on the practice of robotic art, posing the research question: \textit{How do robotic artists utilize robots in their artistic practice?} We approach this inquiry through the perspectives and experiences of robotic artists, who creatively design, modify, and repurpose robotic systems for artistic expression and exploration.

% The key findings are...
Our findings highlight the social, material, and temporal dimensions of artists' practices that shape their creativity and artistic outcomes. The creation of robotic art is largely a social process, as artists receive both explicit and implicit feedback through the audience's reactions and reception of their work. Simultaneously, the embodiment and malfunctions inherent to robotic systems drive artistic experimentation. The temporal processes of creation and exhibition, beyond just the final product, further enhance the creative value. Our empirical analysis presents how creativity emerges through the interplay of social, material, and temporal interactions among artists, robots, audiences, and the environment.

% The contributions of this work are...
We make two main contributions to HCI in this study. 
First, we elucidate the interactive mechanisms among key actors---human creators, machines, audiences, and environments---within the practice of robotic art, a topic that remains underexplored in HCI. Our findings reveal the significance of sociality (e.g., interactions between artists and audiences), materiality (e.g., the embodiment and malfunctions of robots), and temporality (e.g., the processes of creation and exhibition) in shaping creative values. We propose that these three facets are central to the creative process and facilitate the emergence of creativity in robotic art.
Second, drawing from the findings, we offer implications for \textit{socially informed}, \textit{material-attentive}, and \textit{process-oriented} creation with computing systems. We suggest leveraging these three aspects to enhance creativity and the creative experience. Specifically, we discuss the value of incorporating implicit audience feedback, designing with technical malfunctions, and focusing on the post-creation process to foster alternative creative experiences with machines~\cite{alter2010designing, juarez2022glitch}.



% 尽管HCI 研究开始关注mental health的homework的支持【】,但是艺术治疗里的homework对于HCI研究仍然是一个尚待理解和探索的新场景
%也尚未有HCI design cases探索如何设计能够较好支持艺术治疗homework的包含AI agents 的系统。
%因此,为了给我们接下来的设计探索收集inputs,我们组织了formative study。我们的主要目的有二:
    % 理解艺术治疗家庭作业的场景
    % 理解设计支持艺术治疗家庭作业的包含AI agent 的系统应该满足哪些需求,符合哪些quality(这里得到的结论应该是下一个阶段设计部分比较重要的交互或者界面或者功能特性,以及比较关键的设计rationales)

\section{Contextual Understanding}
Recent HCI research has pinpointed the significance of understanding therapy homework in mental health~\cite{Oewel_2024}, yet art therapy homework remains a unique and unaddressed domain. 
Therefore, we conducted a contextual study with a group of therapists to gain a concrete understanding of current art therapy homework practice and to identify common needs for technological support.
\begin{figure*}[tb]
  \centering
  \includegraphics[width=\linewidth]{images/1.jpg}
  \vspace{-4mm}
  \caption{Art therapy homework outcomes from the therapists' previous practice: (a): T4; (b)-(c): T5; (d)-(g): T3}
  \Description{This Figure showcases the outcomes of homework practices among art therapists. From left to right: (a) a client completing a homework task on a structured worksheet; (b) a depiction of a volcano represented by yellow patterns, with orange indicating imminent erupting lava; (c) an outline of a small figure containing a floral pattern in black and red; (d) a composition using text alongside red and blue floral designs; (e) a diary entry documented by a client; (f) a handcrafted green mountain created by a client; (g) a client-made black clay figurine placed on a patch of grass.}
  \label{fig:context1}
\end{figure*}

%second, to understand the needs and qualities of human-AI systems in supporting art therapy homework.
% formative study procedure
    % 找了谁
    % 怎么做的
        % 我们组织了一对一的疗愈师访谈,来理解艺术治疗家庭作业的当前practice,包扩(当前的practice,艺术治疗家庭作业是什么样一个形式,怎么布置的,做些什么,有哪些疗愈意义,治疗师想要通过家庭作业达到什么目标)
        % 其次,我们基于艺术治疗理论和相关的前期工作,以及访谈中新获得的理解,以准备了一个初步的demo和mockup来作为formative study的准备材料
            % 相关的理论和研究表明艺术治疗的家庭作业一般需要结合艺术创作与verbal反思两个元素,然而目前并未有将两者结合在一起的系统,为了和疗愈师共创式设计,我们构建了一个简单的家庭作业系统demo,它包含一个让用户通过绘制语义分割来生成图像的画板(类似的画板已被应用于艺术治疗practice,见DeepThink),以及一个可以理解用户在画板上绘制行动并提出问题鼓励用户近一步表达创作过程的AI agent。我们尽量保持系统的simplistic和open-ended以方便疗愈师参与到接下来的协同设计并能最大限度输出他们的经验。
            % 与此同时,我们构想了一个初步的疗愈师界面,目的是辅助疗愈师monitor和review来访者的家庭作业。我们制作了静态的mock up以便疗愈师在此基础上进一步协同创作发展设计。
        %我们用这个初步的demo和mockup组织疗愈师进行了两次的协同设计工作坊,流程:
            % 介绍了协同设计的目标(设计支持疗愈师和来访者的艺术治疗家庭作业的AI工具)
            % 我们展示了demo和mockup
            % 让治疗师进行了交互体验和自由讨论,疗愈师们在本地设备使用了我们的demo,体验了我们的疗愈师端mockup,并且分别进行了在线的讨论,表达了丰富的对于系统设计如何可以更好支持艺术治疗家庭作业的意见,以及交互体验方面的建议,然后我们对
\begin{table*}[tb]
\caption{Demographics of Participant Therapists: Experience refers to the number of years engaged in art therapy; The Number of Case refers to cases related to art therapy; The Number of Online Case refers to cases related to online art therapy}
\label{tab:expert}
\vspace{-3mm}
\small
\resizebox{\textwidth}{!}{
\begin{tabular}{ccccccccc}
\toprule
ID & Age & Gender & Experience & Education Level& Major & Region & The Number of Case&The Number of Online Case\\
\midrule
T1& 39& F& 6 & Master & Art Therapy & United States(Florida) &300+&65\\
T2& 41& F& 10 & Master & Art Therapy & Italy(Puglia) &200+&12\\
T3& 49& F& 8 & Master & Art Therapy & China(Guangdong) &350+&85\\
T4& 37& F& 5 & PhD & Cognitive Psychology\&Art Therapy&China(Hongkong) &100+&52\\
T5& 24& F& 2 & Master & Art Therapy& China(Hangzhou) &100&45\\
\bottomrule
\end{tabular}
}
\Description{Table 1 presents the demographics of the participant therapists. Experience refers to the number of years they have been engaged in art therapy, and the Number of Cases indicates the number of art therapy cases they have handled. The five therapists are as follows: T1 is 39 years old, female, with 6 years of experience. She holds a Master’s degree in Art Therapy and practices in Florida, United States, having managed over 300 cases. T2 is 41 years old, female, with 10 years of experience. She has a Master’s degree in Counseling Psychology and works in Puglia, Italy, with more than 200 cases. T3 is 49 years old, female, with 8 years of experience. She holds a Master’s degree in Art Therapy and is based in Guangdong, China, having overseen over 350 cases. T4 is 37 years old, female, with 5 years of experience. She has a PhD in Cognitive Psychology and Art Therapy and practices in Hong Kong, China, having handled more than 100 cases. T5 is 24 years old, female, with 2 years of experience. She holds a Master’s degree in Art Therapy and is located in Hangzhou, China, with approximately 100 cases managed. The Number of Online Case refers to cases related to online art therapy
}
\end{table*}

\subsection{Procedure and Preparation}
Five art therapists (T1-T5; all self-identified females; aged 24-49) participated in this study. None of the therapists were members of the research team. T3 was a previous collaborator; the other therapists were recruited via T3's professional network, intended for a diverse group of practitioners from various geographical locations.
Their demographics and expertise are detailed in~\autoref{tab:expert}.
We first conducted 60-minute remote one-on-one interviews with each therapist to understand their current homework practice. This was followed by two 60-minute online focus groups with the therapists. The researchers, acting as facilitators, moderated the discussion on the common challenges for homework practice, aiming to identify needs and design opportunities.
In addition, we kept close collaboration with the therapists throughout the development phase and conducted informal follow-ups to gather inputs in formulating design features.
The one-on-one interviews and focus group sessions were screen-recorded and transcribed. We conducted open coding and affinity diagramming to identify emerging insights reported below.

%Initially, we conducted remote, one-on-one semi-structured interviews with each therapist to gather insights into their current practices and the challenges regarding art therapy homework. 
%\textcolor{blue}{Subsequently, we held two 60-minute online focus group discussions with these therapists. The researchers, acting as facilitators, guided the discussions using the challenges and needs identified in prior one-on-one semi-structured interviews to encourage broader conversations. The goal was to identify the common needs and challenges therapists face in their practice, and, secondly, to closely collaborate with them in co-designing the system's core features.}
%Subsequently, we held two online focus groups with these therapists to foster a broader discussion, aiming to identify common needs and challenges in their practices.
%We recruited five art therapists (T1-T5; 5 self-identified females; aged 24-49) whose demographics and expertise are detailed in~\autoref{tab:expert}. 
%We first conducted remote, one-on-one semi-structured interviews with the five therapists in order to understand the current individuals' practices and challenges of art therapy homework. 
%Further, we conducted two remote focus groups with the five therapists in order to promote their discussion about these individuals' practices and challenges and identify common ground.
%First, we showcased the demo and mock-up, enabling the therapists to experience them. Following this, we went through online discussions where they provided feedback on how AI agent system design could enhance support for art therapy homework and offered suggestions for enhancing the interactive experience.


%Further, in order to co-design with the therapists, we developed a demo and mock-up as preparatory materials for context study. 
%First, Existing literature on art therapy homework, along with insights from interviews, suggests that therapy homework could combine art-making with verbal expression, yet no existing system combines these elements. Thus, we developed a demo featuring a drawing tool for AI image generation through semantic segmentation(similar to cases used in art therapy practices, such as DeepThink~\cite{du2024deepthink}) and a conversation agent that understands users' actions and asks questions to encourage description of the creation. 
%Second, we envisioned an therapist interface aimed at assisting therapists in monitoring and reviewing clients' therapy homework. The simplistic and open-ended demo and the static mock-up allowed therapists to contribute their expertise in future co-design workshops. To design an AI agent systems to support therapists and clients with art therapy homework, 


% formative study results  
% understanding current art therapy homework practice
        %介绍visual arts和written的形式
            % 的确有art thearpy homework
            % 疗愈师说了家庭作业是什么样的形式(介绍常有的形式,一般是艺术创作,记日记,拍照,做手工,)
             % 家庭作业很重要,为什么重要,有什么功能,可以怎么样影响来访者:
                % 1
                % 2
        %定制化和数据review
            %介绍定制化需求对于治疗师很重要
                % 治疗师会结合她掌握的疗愈技术和艺术治疗方法来定制家庭作业
                % 治疗师也会根据上一节session灵活调整家庭作业
            % 介绍review的重要性
                %艺术疗愈师需要看到家庭作业,需要用到家庭作业:为什么需要看到,为什么需要用到,怎么用的
                %retrieve,依从性compliance,不知道有没有按时做,
\subsection{Contextual Understanding: Current Practice and Common Challenges}

Our therapists confirmed that art therapy homework plays a crucial role in helping them understand and collaborate with clients between sessions. They shared their current methods for assigning art therapy homework, which often involves multi-modal activities~(see \autoref{fig:context1}) combining visual arts (e.g., drawing, collage-making, photography and clay sculpting) with written or spoken documentation of emotions and experiences (e.g., journaling, social media posts, and audio recordings).
The therapists noted that integrating visual presentations with verbal expression is a common practice, as it helps clients document and articulate their experiences. For example, T4 combined art-making with audio recording to assist clients in expressing their current feelings: \qt{I asked the elderly [clients] to take photos and create collages at home and encouraged them to record audio to share their daily emotions}. The therapists believed that this combination encourages clients to more fully describe their artwork, explore subconscious thoughts behind the creative process, and gain new perspectives on their lives.
Aside from their approach of leveraging art therapy homework in current practice, the therapists also share their challenges regarding art therapy homework. From their shared experiences, three major sets of challenges emerged, which are summarized below:
% As revealed by the therapists, they invited their clients to complete multi-modal forms of art therapy homework, mainly combining visual arts~(e.g., drawing, collage-making, photography) with the written and spoken document of current emotions and experiences~(e.g., journaling, social media posting, and audio recording).
% Our therapists noted that integrating visual presentations with verbal expression is a common practice in art therapy homework, as it helps clients document and articulate their current experiences.
% For example, T4 integrated art-making with audio recording to help clients document their current experiences and feelings. 
% Our therapists believed that combining art-making with verbal expression encourages clients to express and describe their artwork more fully, explore subconscious thoughts behind art-making, and cultivate new perspectives on various aspects of their lives. Further, our therapists emphasized they need to customize homework assignments in art therapy and track their the homework outcomes, which could build an therapeutic collaboration between therapists and clients.

\subsubsection{\textbf{CH1}: Challenges in Homework Threshold without Therapist Guidance} 

Our therapists indicated that art-making-based therapy homework can pose a creative barrier for clients without therapist guidance~(\textbf{CH1-1}). T4 noted that this barrier could lead to stress, self-criticism, and fear of failure: \qt{If a client is self-critical, they may fear creating something `ugly', which can increase pressure and hinder the therapeutic process}. Consistent with prior studies~\cite{Tang2017,Harwood2007}, the therapists also confirmed that clients may lack confidence in completing homework or producing emotional responses without guidance, which can result in lower compliance.
Additionally, therapists expressed concerns that clients might struggle to interpret their artwork in a therapeutic way without support, reducing their motivation for deep reflection~(\textbf{CH1-2}). T1 observed that without proper guiding, it can be difficult for clients to make full use of the exercise: \qt{Last time, I assigned a homework about `your ideal future family', but [...] she just scribbled a bit without expressing any clear thoughts}. The therapists emphasized the importance of guiding clients in verbalizing their emotions alongside art-making. T5 mentioned that while visual art can help explore subconscious thoughts, verbalizing these feelings provides a cathartic outlet and helps clients externalize their emotions.

% Moreover, therapists were concerned that clients might struggle to interpret their artwork in a therapeutic way without guidance, leading to reduced motivation for deep reflection. T1 noted that without clear direction, creating a meaningful drawing that promotes reflection can be difficult for clients.
% Additionally, therapists confirmed the importance of guiding clients to properly verbalize their feelings alongside art-making. As T5 mentioned, visual art can serve as a channel for exploring and expressing subconscious thoughts, while verbalizing these feelings provides a cathartic outlet and helps clients externalize their emotions.

% First, our therapists indicated that art-making-based therapy homework might present a creative threshold for clients.
% For example, T4 explained that the homework is to ensure that the creative process remains therapeutic and accessible, with low threshold, so participants can avoid stress, self-criticism, and fear of failure.
% Second, our therapists was concerned that clients may struggle to interpret their artwork in a therapeutic direction without the guidance of a therapist, which lead to a lack of motivation to engage in deep reflection. 
% Also, T1 suggested that it could be challenging for clients to create a meaningful drawing that effectively promotes reflection without clear guidance.
% Therapists also confirmed that it is crucial to prompt the clients to verbalize their feelings in addition to the art-making. As mentioend by T5, the visual art-making could be a channel for clients to explore and express themsleves at the subconscious level, whereas, verbalizationg could help them externalize the subconscious thoughts and find themself a carthartic outlet.
% Prior studies have shown that clients may struggle with confidence in completing homework and producing emotional arousal without the therapist's guidance~\cite{Tang2017,Harwood2007}, leading to reduced homework compliance.

\subsubsection{\textbf{CH2}: Challenges in Customizing Therapy Homework} 

Our therapists demonstrated their practice of customizing homework assignments in art therapy. For instance, T2 and T5 mentioned tailoring homework tasks and specific instructions based on their practical experience and therapeutic techniques (e.g., cognitive-behavior therapy or mindfulness): \qt{If I suggest therapy homework that integrates mindfulness with art-making, I might ask the client to notice any changes in their breathing [during homework]}~(T4). T1 also adjusted homework tasks based on feedback from previous in-sessions.
However, the therapists noted that adapting structured instructions flexibly was difficult with current verbal or written formats, often leading to clients forgetting or abandoning their guidance or instructions~(\textbf{CH2-1}). Additionally, T3 and T4 observed that offering encouraging words and support during homework could boost motivation, but they found it challenging to provide personalized encouragement outside of in-session times~(\textbf{CH2-2}).

% Our therapists demonstrated their practice of customizing homework assignments in art therapy, e.g., T2 and T5 both mentioned that 
% they tended to tailor diverse homework assignments and specific instructions based on drawing from their own practical experience and therapeutic techniques~(e.g., CBT or mindfulness): \qt{If I suggest therapy homework that integrates mindfulness with art-making, I might suggest that he noticed any changes in his breathing while observing the artwork~(T4)}.
% Also, T1 flexibly adjusted the homework tasks based on feedback from the previous in-session.
% However, our therapists noted that adapting structured instructions flexibly was challenging using existing verbal or written descriptions.
% This often lead to clients forgetting or abandoning their therapy homework.
% Moreover, T3 and T4 noted that offering encouraging words and support during homework could enhance motivation for completion. However, they currently find it challenging to tailor this encouragement and care after in-sessions.

\subsubsection{\textbf{CH3}: Challenges in Tracking Therapy Homework History} 

The therapists confirmed that original homework data---such as the artworks, conversation records about clients' creative states, and details of the creative process---were essential for their assessments. They also encouraged clients to bring homework outcomes to the next session. For example, T1 and T3 prompted clients to share their current feelings and perspectives during one-on-one sessions, while T4 encouraged clients to engage in re-creation based on their homework.
However, therapists commonly expressed difficulty in tracking homework history, as they relied on clients to record and report their own progress~(\textbf{CH3-1}): \qt{The client drew [an artwork] two months ago. When you showed her the artwork, she often didn't remember what had happened at the time~(T3)}. Additionally, T1, T3, and T4 raised concerns that current practices might miss valuable data regarding clients' emotional or mental states at the time the homework was completed~(\textbf{CH3-2}).

% Our therapists confirmed that original homework data, including the artwork, conversation records about clients' current creative states, and the creative process, were all crucial for their assessment.
% Meanwhile, the homework outcomes was encouraged to be brought to the next in-sessions, e.g., T1 and T3 encouraged clients to share their current feelings and perspectives on people and things during the one-on-one sessions. 
% Also, T4 encouraged clients to engage in re-creation activities during the in-sessions, building upon their homework outcomes.
% However, our therapists noted that they found difficult to track the homework history, as they relied on clients to record and report their own progress.
% Also, T1, T3, and T4 raised concerns that current practices might be missing valuable homework data regarding clients' homework assignments, specifically related to the client's status at the time the homework was completed.




   % challenges of current practice
        % 【但是】创作门槛高,便利性。(找话可以支持对应)
        % 【但是】:来访者缺少指导,没有引导,很难知道是否真的发生了反思(找话可以支持对应)
        % 【但是】:难以追踪,难以记录 (找话可以支持对应报告)
%\subsubsection{\textbf{Current challenges}}

%\textbf{D1: Supporting Therapy Homework by Integrating Verbal Expression with Art-making.} Our therapists suggested that therapy homework should be supported through combining art-making with verbal expression.They emphasized the value of integrating art-making and verbal expression in AI-infused art therapy. Likewise, T1 indicated that it not only enabled clients to gain a deeper understanding of their own artwork but also supported their process of self-expression. Further, the therapists envisioned that AI has the potential to further ask in-depth and structured questions based on artwork, thereby eliciting deeper reflections from clients. \textbf{D2: Supporting Customization of Therapy Homework via Agents.} T5 envisioned that conversational agents as \qt{homework assistants} that can guide clients to further explore some deeper self-reflections.Further, T2 suggested that AI agents have the advantage of conveying more caring and supportive messages from therapists to clients. Finally, our therapists noted that they needed to set homework topics and specific instructions in a therapist interface according to their own practice principles.\textbf{D3: Supporting Homework History Gathering and Summarization via AI agents.} The therapists further proposed that AI has the potential to assist in summarizing descriptions of clients' creations and capturing their emotions or experiences. T4 emphasized that AI agents should function as a summary tool rather than providing sophisticated interpretation. For example, T3 suggested that AI could identify and summarize recurring images in clients' artwork. This summarization can facilitate more in-depth discussions during one-on-one sessions.



\section{\name{} SYSTEM}
Informed by the formative study, we designed and developed \name, an AI-infused cartoon video system for promoting preschoolers' language learning. In this section, we discuss our core features, usage scenario, and system implementation.
\subsection{Core Features}
The core features of \name{} were informed by our formative study, existing literature, and insights from experts, culminating in the identification of four key phases: 
The \textbf{Real-time Question-Answering} phase allows families to pause the video and interact with a conversational agent, which answers questions about the current frame to help children learn vocabulary tied to the cartoon's objects and events.
The \textbf{Active Vocabulary Review} phase features the agent, acting as a teachable agent~\cite{matsuda2020effect}, curiously asking children the same questions, presenting relevant screenshots, and providing positive feedback to reinforce newly learned vocabulary.
The \textbf{Real-World Association} phase pairs real-world images with their animated versions to help children connect the object to its real-world counterpart.
The \textbf{Contextual Expansion} phase creates personalized stories and sentences based on family preferences and children's ages~\cite{leong2024putting}. Drawing from previous work~\cite{chen2024retassist}, it also generates related images to enhance word comprehension and recall in context.


\subsection{\name{} Usage Scenario}
\autoref{fig:ui_interface} illustrates the overall user flow of \name{} which is an AI-infused cartoon video system for vocabulary learning. Here we present a typical usage scenario of \name{}: 
Oli is a 3-year-old child who loves watching the cartoon \textit{Peppa Pig}. 
Her mother, Ira, wanted her to learn new vocabulary through the cartoon video. 
One evening, they opened \name{} on a tablet, and Ira encouraged Oli to choose a favorite cartoon character to watch with her (\autoref{fig:ui_interface} (A)). 
Ira assisted Oli in selecting a panda character and the appropriate age group, ensuring the agent could communicate effectively with Oli. 
Then, Ira selected an episode of \textit{Peppa Pig} for them to watch together (\autoref{fig:ui_interface} (B)). 
While watching, Ira paused the video, pointed to the noodles in the cartoon, and asked Oli, ``What is Peppa eating?'', and Oli shook her head.
Ira clicked ``I have a question'' in the interface and asked again. The agent replied, ``Peppa is eating noodles''~(see \autoref{fig:ui_interface} (C)).
Further, Ira encouraged Oli to ask the conversational agent if there was something she didn't understand. 

After watching the video, the conversation agent transformed into a curious teachable agent by showing Oli screenshots from the cartoon and asking her to review the vocabulary she just learned. 
For example, the conversational agent asked, ``What is Peppa eating?''.  Oli could click the ``I want to answer'' button, and the agent would give feedback based on her response (see \autoref{fig:ui_interface} (D)).

In the Contextual Mapping phase, Oli and her mom saw two images: one of noodles from Peppa Pig on the left, and a real-life image on the right. The conversational agent prompted a discussion question, asking if the two images were similar. Oli and her mom then discussed it, with her mom adding, ``Have you seen the noodles we eat at home?'' (see \autoref{fig:ui_interface} (E)).

Finally, they found that the system expanded vocabulary into sentences or stories and generated corresponding images. For example, Ira customized the sentence under the topic ``FOOD'': ``Noodles are made from flour, and when cooked, they become soft and pair perfectly with various delicious soups'', expanding from the word ``noodles'' (see \autoref{fig:ui_interface} (F)).

\begin{figure*}[h]
    \centering
    \includegraphics[width=0.9\linewidth]{images/xitong.pdf}
    \caption{: Overview of \name{} system architecture.}
    \label{fig:enter-label}\vspace{-0.5cm}
\end{figure*}

\subsection{ \name{} System Implementation}
\label{subsec:system-implementation}
\autoref{fig:enter-label} shows our system architecture. 
The front-end application of \name{} is built with \textit{Vue 3}~\cite{vue3}, a JavaScript-based framework that enables cross-platform compatibility across devices. The client communicates with the server via \textit{REST API}. When interacting with \name{}, user audio messages and video frames are uploaded to the backend as input. User audio messages are processed using \textit{iFly's TTS}
\footnote{iFly's TTS, https://global.xfyun.cn/products/text-to-speech} model to generate corresponding voice messages. Also, we have implemented speech-to-text function using the Voice Dictation (Streaming Version) service from \textit{iFlytek}\footnote{iFLYTEK Speech-to-Text, https://global.xfyun.cn/products/speech-to-text}. 

The back-end of \name{} is built with FastAPI~\cite{fastapi}, which facilitates the processing of VLMs and LLMs workflows, AI image generation, and data management. 
The generative pipelines incorporate \textit{CogVLM2}\footnote{CogVLM2, https://github.com/THUDM/CogVLM} to analyze video frames during Real-time Q\&A phase and Active Review phase to provide corresponding feedback. 
In the Real-World Association phase, the conversation data between the agent and families from the first phase was summarized into themes using \textit{Doubao LLM}\footnote{Doubao AI, https://team.doubao.com/zh/}, then transformed into image prompts with realistic styles by another \textit{Doubao LLM}, which were input into Stable Diffusion to generate images.
In the Contextual Expansion phase, we used \textit{Doubao LLM}\footnote{Doubao AI, https://team.doubao.com/zh/} to generate contextual sentences and stories, which are then summarized into prompts for the Stable Diffusion model to generate images (\textit{stabilityai/stable-diffusion-xl-base-1.0}\footnote{Stable Diffusion XL, https://huggingface.co/docs/diffusers/using-diffusers/sdxl}).
We iteratively co-designed AI prompts with language experts.
The system uses MongoDB to store and manage data, including user queries, interaction logs, and AI-generated content (e.g., video frames, dialogue records, and AI-generated sentences or stories).
\begin{table*}[t]
\centering
\tiny
\begin{tabular}{|M{1.2cm}|M{0.7cm}|M{1cm}|M{1cm}|M{1cm}|M{0.8cm}|M{1.2cm}|M{0.7cm}|M{1cm}|M{1cm}|M{1cm}|M{0.8cm}|}
\hline\hline
Model & \#GPU & \#Strategies & Search Time(/s) & Simulation Time(/s) & E2E Time(/s) & Model & \#GPU & \#Strategies & Search Time(/s) & Simulation Time(/s) & E2E Time(/s) \\ \hline
\multirow{4}{*}{Llama-2-7B} & 64 & 23348 & 0.06 & 49.7 & 51.0 & \multirow{4}{*}{Llama-2-13B} & 64 & 23400 & 0.05 & 58.1 & 59.3 \\ \cline{2-6} \cline{8-12} 
 & 256 & 14372 & 0.05 & 43.5 & 44.4 &  & 256 & 13552 & 0.03 & 49.9 & 50.8 \\ \cline{2-6} \cline{8-12} 
 & 1024 & 8856 & 0.04 & 41.8 & 42.2 &  & 1024 & 8920 & 0.02 & 51.0 & 51.7 \\ \cline{2-6} \cline{8-12} 
 & 4096 & 4700 & 0.03 & 33.0 & 33.2 &  & 4096 & 4720 & 0.02 & 44.1 & 44.3 \\ \hline
\multirow{4}{*}{Llama-2-70B} & 64 & 53264 & 0.1 & 68.8 & 75.0 & \multirow{4}{*}{Llama-3-8B} & 64 & 23348 & 0.05 & 48.3 & 49.6 \\ \cline{2-6} \cline{8-12} 
 & 256 & 31440 & 0.06 & 57.7 & 60.9 &  & 256 & 14372 & 0.04 & 42.0 & 42.8 \\ \cline{2-6} \cline{8-12} 
 & 1024 & 20152 & 0.05 & 57.4 & 59.6 &  & 1024 & 8856 & 0.03 & 40.9 & 41.3 \\ \cline{2-6} \cline{8-12} 
 & 4096 & 10948 & 0.04 & 63.2 & 65.0 &  & 4096 & 4700 & 0.03 & 32.7 & 32.9 \\ \hline
\multirow{4}{*}{Llama-3-70B} & 64 & 53264 & 0.1 & 66.8 & 71.8 & \multirow{4}{*}{GLM-67B} & 64 & 20528 & 0.04 & 19.3 & 20.6 \\ \cline{2-6} \cline{8-12} 
 & 256 & 31440 & 0.07 & 56.3 & 59.6 &  & 256 & 12132 & 0.03 & 16.6 & 17.4 \\ \cline{2-6} \cline{8-12} 
 & 1024 & 20152 & 0.05 & 55.5 & 57.6 &  & 1024 & 7948 & 0.02 & 16.9 & 17.3 \\ \cline{2-6} \cline{8-12} 
 & 4096 & 10948 & 0.04 & 62.4 & 63.4 &  & 4096 & 4196 & 0.02 & 21.3 & 21.5 \\ \hline
\multirow{2}{*}{GLM-130B} & 64 & 33540 & 0.06 & 22.4 & 52.4 & \multirow{2}{*}{GLM-130B} & 1024 & 11976 & 0.03 & 16.7 & 18.2 \\ \cline{2-6} \cline{8-12} 
 & 256 & 18776 & 0.04 & 17.2 & 19.4 &  & 4096 & 6040 & 0.02 & 19.2 & 20.1 \\ \hline\hline
\end{tabular}%
\caption{
    The search space and the time cost for \sysname on Heterogeneous GPUs.
  For the pictures of time cost, the light color without hatches represents the time spent searching, while the deep color with hatches represents the time spent simulating.
  We can observe that it only takes \sysname\ about 1 minute to complete the end-to-end simulation. 
}
\label{tab:exp:cost}
\end{table*}

\section{Experiments}\label{sec:exp}


%In this section, we first evaluate \sysname's cost model accuracy under different settings to build the basis for the search in \S\ref{sec:exp:accuracy}.
%We show the search space of \sysname, and the search time cost for the search in \S\ref{sec:exp:cost}.
%Then, t
To prove \sysname's optimal search ability on MegatronLM, we did a comparative analysis between \sysname\ and experts on MegatronLM in \S\ref{sec:exp:expert}.
%After that, we compare \sysname with existing auto-parallel frameworks, including Alpa, Galvatron, etc., in \S\ref{sec:exp:comparison}.
Finally, we evaluate \sysname to search for the finance-optimal plan under different settings in \S\ref{sec:exp:finance}.

%\subsection{Cost Model Accuracy}\label{sec:exp:accuracy}
%



\section{Cost Analysis}\label{sec:exp:cost}

\sssec{Method}.
We did a cost analysis to show the gap between the large search space and the search efficiency of the \sysname.
We selected Llama-2 models (7B, 13B, and 70B) with 64, 256, 1024, and 4096 GPUs.
Then, for all the settings, we implemented \sysname\ on it and recorded the searched strategy number along with the end-to-end time (search time and simulation time)


\sssec{Result}. As shown in Table \ref{tab:exp:cost}, the number of explored strategies grows exponentially with model size. For smaller models like Llama-7B, even with 4096 GPUs, the search space remains relatively small. However, for larger models such as Llama-70B, the search space nearly triples compared to Llama-7B under the same GPU configuration. The end-to-end time reveals that the simulation phase is the main bottleneck, which may take 1 minute to execute on average. While the search time only takes less than 1 second to execute on average. This highlights the need for optimizing the simulation process, particularly in large-scale settings, while \sysname’s search algorithm remains efficient and scalable across different configurations.




\begin{figure*}[thbp]
  \centering
    \subfloat{\includegraphics[width=0.4\textwidth]{figs/fig-expert-legend.pdf}}\\
    \addtocounter{subfigure}{-1}

    \begin{minipage}{\textwidth}
    {\centering{\hspace{2.8cm}A800\hspace{4cm}H100\hspace{4.2cm}H800}}
    \end{minipage}

    \raisebox{0.8cm}{\rotatebox[origin=c]{90}{Llama-2}}
    \subfloat[7B]{\includegraphics[width=0.106\textwidth]{figs/fig-expert-A800-llama2-7b.pdf}}
    \subfloat[13B]{\includegraphics[width=0.106\textwidth]{figs/fig-expert-A800-llama2-13b.pdf}}
    \subfloat[70B]{\includegraphics[width=0.106\textwidth]{figs/fig-expert-A800-llama2-70b.pdf}}
    \subfloat[7B]{\includegraphics[width=0.106\textwidth]{figs/fig-expert-H100-llama2-7b.pdf}}
    \subfloat[13B]{\includegraphics[width=0.106\textwidth]{figs/fig-expert-H100-llama2-13b.pdf}}
    \subfloat[70B]{\includegraphics[width=0.106\textwidth]{figs/fig-expert-H100-llama2-70b.pdf}}
    \subfloat[7B]{\includegraphics[width=0.106\textwidth]{figs/fig-expert-H800-llama2-7b.pdf}}
    \subfloat[13B]{\includegraphics[width=0.106\textwidth]{figs/fig-expert-H800-llama2-13b.pdf}}
    \subfloat[70B]{\includegraphics[width=0.106\textwidth]{figs/fig-expert-H800-llama2-70b.pdf}}
    \\
    \raisebox{0.8cm}{\rotatebox[origin=c]{90}{Llama-3}}
    \subfloat[8B]{\includegraphics[width=0.16\textwidth]{figs/fig-expert-A800-llama3-8b.pdf}}
    \subfloat[70B]{\includegraphics[width=0.16\textwidth]{figs/fig-expert-A800-llama3-70b.pdf}}
    \subfloat[8B]{\includegraphics[width=0.16\textwidth]{figs/fig-expert-H100-llama3-8b.pdf}}
    \subfloat[70B]{\includegraphics[width=0.16\textwidth]{figs/fig-expert-H100-llama3-70b.pdf}}
    \subfloat[8B]{\includegraphics[width=0.16\textwidth]{figs/fig-expert-H800-llama3-8b.pdf}}
    \subfloat[70B]{\includegraphics[width=0.16\textwidth]{figs/fig-expert-H800-llama3-70b.pdf}}
    \\
    \raisebox{0.8cm}{\rotatebox[origin=c]{90}{GLM}}
    \subfloat[67B]{\includegraphics[width=0.16\textwidth]{figs/fig-expert-A800-glm-67b.pdf}}
    \subfloat[130B]{\includegraphics[width=0.16\textwidth]{figs/fig-expert-A800-glm-130b.pdf}}
    \subfloat[67B]{\includegraphics[width=0.16\textwidth]{figs/fig-expert-H100-glm-67b.pdf}}
    \subfloat[130B]{\includegraphics[width=0.16\textwidth]{figs/fig-expert-H100-glm-130b.pdf}}
    \subfloat[67B]{\includegraphics[width=0.16\textwidth]{figs/fig-expert-H800-glm-67b.pdf}}
    \subfloat[130B]{\includegraphics[width=0.16\textwidth]{figs/fig-expert-H800-glm-130b.pdf}}
  \caption{
  We compare \sysname's searched optimal plan's throughput with expert's proposed plan's throughput in single-GPU setting.
  }
  \label{fig:expert:throughput}
  \vspace{-10pt}
\end{figure*}

\subsection{Mode-1: Comparison with Expert Plans}\label{sec:exp:expert}

\sssec{Method}.
To prove the \sysname's ability to search the optimal strategy on MegatronLM, we compared \sysname\ with an expert.
We first selected three models with different parameter sizes (7 model settings in total): Llama-2 (7B, 13B, and 70B), Llama-3 (8B, 70B), and GLM (67B, 130B).
Then, we offer 4 GPU number settings: 32, 128, 256, and 1024.
Next, we asked six experts to craft a parallel strategy for each setting (different models and different GPU settings, overall $7\times 4=28$ settings) based on their expert experience.
Each participant has over six years of industry machine learning service or training experience.
Then, we ran each of the six participants' parallel strategies for each setting on MegatronLM and picked the optimal one (one with the largest throughput) among the six expert-crated strategies as the expert-optimal strategy.
At last, we run \sysname\ to search the optimal parallel strategy automatically and compare the \sysname's parallel strategy's throughput with the expert-optimal parallel strategy's throughput.

\sssec{Results}.
As shown in Fig. \ref{fig:expert:throughput}, \sysname demonstrates its ability to automatically generate parallel strategies that match or exceed expert-tuned plans across various model configurations. This highlights \sysname's capability to generalize and optimize without manual intervention.

\par A key finding is that \sysname consistently matches or outperforms manually designed strategies, showing that its automated search can achieve results on par with domain experts. This adaptability extends across diverse hardware and model types, while specific setups often constrain expert-tuned plans. \sysname dynamically adjusts to different configurations, optimizing parallel strategies based on the specific training environment.

\par Another important observation is \sysname’s flexibility in combining different parallelism techniques—data, tensor, and pipeline. While expert strategies often focus on one type of parallelism, \sysname optimally balances multiple forms, leading to superior performance, especially for large-scale models. This hybrid approach is likely the key to future parallelism strategies, where flexibility and adaptation are critical.
%\subsection{Comparison with Other Schemes}\label{sec:exp:comparison}

\begin{table}[h!]
\centering
\caption{GPT-3 Model Specification}
\label{tab:gpt-3}
\begin{tabular}{ccccc}
\hline
\#params & Hidden size & \#layers & \#heads & \#gpus \\ \hline\hline
350M & 1024 & 24 & 16 & 1 \\ 
1.3B & 2048 & 24 & 32 & 4 \\ 
2.6B & 2560 & 32 & 32 & 8 \\ 
6.7B & 4096 & 32 & 32 & 16 \\ 
15B & 5120 & 48 & 32 & 32 \\ 
39B & 8192 & 48 & 64 & 64 \\ \hline\hline
\end{tabular}
\end{table}


\begin{table}[h!]
\centering
\caption{LLaMA Model Specification}
\label{tab:llama}
\begin{tabular}{ccccc}
\hline
\#params & Hidden size & \#layers & \#heads & \#gpus \\ \hline\hline
7B & 4096 & 32 & 32 & 8 \\
13B & 5120 & 40 & 40 & 16 \\
33B & 6656 & 60 & 52 & 32 \\
70B & 8192 & 80 & 64 & 64 \\ \hline\hline
\end{tabular}
\end{table}

\begin{table}[h!]
\centering
\caption{GShard MoE Model Specification}
\label{tab:moe}
\begin{tabular}{cccccc}
\hline
\#params & Hidden size & \#layers & \#heads & \#experts & \#gpus \\ \hline\hline
380M & 768 & 8 & 16 & 8 & 1 \\
1.3B & 768 & 16 & 16 & 16 & 4 \\
2.4B & 1024 & 16 & 16 & 16 & 8 \\
10B & 1536 & 16 & 16 & 32 & 16 \\
27B & 2048 & 16 & 32 & 48 & 32 \\
70B & 2048 & 32 & 32 & 64 & 64 \\ \hline\hline
\end{tabular}
\end{table}

\sssec{Models and training workflows}.
For our experiments, we target three types of models: GPT-3, LLaMA, and a Mixture of Experts (MoE) model. These models represent a range of architectures, from homogeneous to heterogeneous, providing a comprehensive evaluation of our parallelism strategies. 

\par \textbf{GPT-3} (see Table \ref{tab:gpt-3}) is a homogeneous Transformer-based language model comprising many stacked layers. Its model parallelization plan has been extensively studied and optimized in various research efforts. \textbf{LLaMA} (see Table \ref{tab:llama}) is another advanced Transformer-based model designed for language modeling, with a focus on efficiency and performance in both pre-training and fine-tuning phases. \textbf{MoE} models (see Table \ref{tab:moe}), such as GShard, combine dense and sparse architectures by incorporating a mixture of expert layers. These layers replace the feed-forward layers in every few Transformer layers, making them highly adaptable to different computational environments.

\par To study the scalability and efficiency of training large models, we follow standard machine learning practices by scaling the model size proportionally with the number of GPUs, as reported in Table 4. For GPT-3, we increase the hidden size and the number of layers concurrently with the number of GPUs, following the methodology used in previous studies. For the MoE model, we primarily increase the number of experts, which is crucial for leveraging the model's sparse architecture and optimizing performance across multiple GPUs. For LLaMA, we adjust the model's depth (number of layers) and width (hidden size) to ensure it scales effectively with the available GPU resources.

\par In each experiment, we adopt the recommended global batch size per established ML practices to maintain consistent statistical behavior across different model configurations. We then fine-tune the micro-batch size for each model and system configuration to maximize overall system performance, with gradient accumulation applied across micro-batches.

\sssec{Baselines}. For each model, we compare our system, \sysname, against strong baselines, including Alpa and Galvatron, and manually designed strategies using Megatron-LM.

\par \textbf{Alpa} is chosen as one of the baselines due to its automated parallelization capabilities, particularly for large-scale models. Alpa utilizes a combination of intra-operator and inter-operator parallelism to optimize the training process. We configure Alpa to its best settings by following the guidelines provided in their documentation and research papers. Alpa is known for its comprehensive strategy space, which includes various parallelism paradigms such as data parallelism, tensor parallelism, and pipeline parallelism.

\par \textbf{Galvatron} is another baseline we employ, noted for its efficient transformer training over multiple GPUs using automatic parallelism. Galvatron incorporates multiple popular parallelism dimensions and automatically discovers the most efficient hybrid parallelism strategy through a decision tree decomposition and a dynamic programming search algorithm. We perform a grid search to determine the optimal configurations for Galvatron, ensuring that we fully leverage its capabilities.

\par \textbf{Megatron-LM} serves as the manually designed baseline, specifically for GPT-like models. Megatron-LM v2 is a state-of-the-art system that combines data parallelism, pipeline parallelism, and manually designed operator parallelism (denoted as TMP). This combination is controlled by three integer parameters that specify the degrees of parallelism assigned to each technique. Following the guidance from their research, we conduct a thorough grid search of these parameters and report the best configuration results. While Megatron-LM is highly specialized for GPT-like models, it does not support other models in our evaluation due to its lack of flexibility in handling different architectures.

Our comparison does not include open-source systems like \textbf{FlexFlow} and \textbf{Tofu} due to their limitations. FlexFlow lacks support for essential operators such as layer normalization and mixed-precision operators, and Tofu only supports single-node execution and is not open-sourced. Given these theoretical and practical constraints, we do not expect FlexFlow or Tofu to outperform the state-of-the-art manual baselines in our evaluation.

In summary, our evaluation includes \sysname, Alpa for its automated strategy space, Galvatron for its efficient hybrid parallelism discovery, and manually tuned Megatron-LM for its specialization in GPT-like models. This comprehensive approach thoroughly compares different parallelism strategies and model architectures.

\sssec{Evaluation metrics}. We measure training throughput in our evaluation. We evaluate the system's weak scaling when increasing the model size and the number of GPUs. Following \cite{narayanan2021efficient}, we use the aggregated peta floating-point operations per second (PFLOPS) of the whole cluster as the metric. After proper warmup, we measure it by running a few batches with dummy data. All our results (including those in later sections) have a standard deviation within 0.5\%, so we skip the error bars in our figures.

\sssec{GPT-3 results}.
\textcolor{red}{To be done}

\sssec{Llama results}.
\textcolor{red}{To be done}

\sssec{MoE results}.
\textcolor{red}{To be done}

\subsection{Mode-2: Heterogeneous GPU Search}

\begin{figure}[t]
  \centering
    \subfloat{\includegraphics[width=0.48\textwidth]{figs/fig-heter-legend.pdf}}\\
    \addtocounter{subfigure}{-1}
    
    \subfloat[Llama-2-7B]{\includegraphics[width=0.16\textwidth]{figs/fig-heter-llama2-7b.pdf}}
    \subfloat[Llama-2-13B]{\includegraphics[width=0.16\textwidth]{figs/fig-heter-llama2-13b.pdf}}
    \subfloat[Llama-2-70B]{\includegraphics[width=0.16\textwidth]{figs/fig-heter-llama2-70b.pdf}}
    \\

    \subfloat[Llama-3-8B]{\includegraphics[width=0.24\textwidth]{figs/fig-heter-llama3-8b.pdf}}
    \subfloat[Llama-3-70B]{\includegraphics[width=0.24\textwidth]{figs/fig-heter-llama3-70b.pdf}}
    \\

    \subfloat[GLM-67B]{\includegraphics[width=0.24\textwidth]{figs/fig-heter-glm-67b.pdf}}
    \subfloat[GLM-130B]{\includegraphics[width=0.24\textwidth]{figs/fig-heter-glm-130b.pdf}}
  \caption{
  For the heterogeneous GPU search scene, we compare expert-designed strategies's throughput with \sysname-searched strategies.
  The results prove the that \sysname achieves better throughput in heterogeneous scene.
  }
  \label{fig:exp:heter}
\end{figure}

% Please add the following required packages to your document preamble:
% \usepackage{graphicx}
\begin{table}[t]
\centering
\resizebox{0.5\textwidth}{!}{%
\begin{tabular}{c|cccc}
\hline
Model & H100 & H800 & A800 & Heter. \\ \hline\hline
Llama-2-7B & 10148287 & 9024716 & 3966756 & 5240609 \\
Llama-2-13B & 5721253 & 4937998 & 2187876 & 3040095 \\
Llama-2-70B & 1233850 & 1174362 & 458719 & 654206 \\
Llama-3-8B & 9167338 & 7610698 & 3586433 & 4660743 \\
Llama-3-70B & 1129568 & 1079507 & 425660 & 626050 \\
GLM-67B & 1288107 & 1218933 & 483384 & 699978 \\
GLM-130B & 508377 & 491088 & 202137 & 300193 \\ \hline\hline
\end{tabular}%
}
\caption{
We compare heterogeneous GPU with single-GPU search's optimal strategies' throughput.
The experiment is conducted with 1024 GPUs.
And the heterogeneous GPU setting is activated with A800 and H100.
}
\label{tab:exp:heter}
\end{table}

\sssec{Method}.
To evaluate \sysname's performance in heterogeneous GPU environments, we conducted a comprehensive comparison of \sysname-searched strategies and expert-designed strategies under heterogeneous GPU configurations. 
We use \sysname in the two GPU-heterogeneous environments with Nvidia H100 and A800 activated for search.
Also, we follow the design of \S\ref{sec:exp:expert}, we recruit six experts to craft a heterogeneous parallel strategy for each setting, and we picked the optimal one as the expert-designed strategy.
We offer 4 GPU number settings: 64, 256, 1024, and 4096.

Besides that, we also compared the heterogeneous GPU setting with single GPU setting in the same GPU number setting (1024).
We compare the throughput between the different settings (only A100, H100, H800, and heterogeneous settings)

\sssec{Results}.
As shown in Fig. \ref{fig:exp:heter}, our experiments reveal that \sysname consistently achieves higher throughput than expert-tuned configurations, particularly with larger models. \sysname’s approach dynamically balances data, tensor, and pipeline parallelism across heterogeneous GPUs, a task often challenging for manual tuning. This adaptability highlights the efficiency of automated strategies, especially in cloud-based or distributed environments where GPU types may vary. Overall, \sysname’s heterogeneous GPU search framework offers a scalable, cost-effective solution for optimizing model training in heterogeneous hardware contexts.

Table \ref{tab:exp:heter} shows the heterogeneous GPU setting compared with a single GPU setting.
Though a heterogeneous GPU setting strategy can not beat the performance of a single-GPU setting strategy, \sysname's searched strategy can nearly match with them.
\subsection{Mode-3: Evaluation Performance on Financial Cost}\label{sec:exp:finance}

%\sssec{Models and training workflows}.

\sssec{Search pools for GPU}. To comprehensively evaluate the financial cost performance of \sysname, we incorporate a variety of GPU types commonly used by major cloud service providers. Our search pools include the following GPU models: NVIDIA H100, A800 and H800.

These GPUs represent a range of performance capabilities and costs, providing a realistic and comprehensive basis for evaluating the financial efficiency of our system. By including these diverse GPU options, we can simulate the decision-making process of users who leverage cloud-based GPU resources, allowing us to optimize for both time and financial cost under various configurations.

\begin{figure}[t]
  \centering
    \subfloat[Per Throu. Llama-70B]{\includegraphics[width=0.24\textwidth]{figs/fig-money-per-Llama-2-70B.pdf}}
    \subfloat[Overall Throu. Llama-70B]{\includegraphics[width=0.24\textwidth]{figs/fig-money-all-Llama-2-70B.pdf}}
    \\
    \subfloat[Per Throu. GLM-67B]{\includegraphics[width=0.24\textwidth]{figs/fig-money-per-GLM-67B.pdf}}
    \subfloat[Overall Throu. GLM-67B]{\includegraphics[width=0.24\textwidth]{figs/fig-money-all-GLM-67B.pdf}}
    \\
    \subfloat[Per Throu. GLM-130B]{\includegraphics[width=0.24\textwidth]{figs/fig-money-per-GLM-130B.pdf}}
    \subfloat[Overall Throu. GLM-130B]{\includegraphics[width=0.24\textwidth]{figs/fig-money-all-GLM-130B.pdf}}
  \caption{
  We list the optimal line of \sysname.
  }
  \label{fig:money}
\end{figure}
\section{Results}
% For chunking, It could be condensed -> First, show the final results (line chunking vs ideal chunking) and the naive chunking stats. Then, describe the trends and example of sections not in any naive chunking strategy.
\subsection{(RQ1) Impact of tailored components}
\label{subsec: rq1_result}

\subsubsection{Hierarchy-aware Chunking}
\label{subsubsec: chunking_result}

From the defined metrics that are used to quantify the information loss on each type of chunking in \S\ref{subsubsec: chunk_setup}, we can conclude that a good chunking strategy should minimize \textit{Fail Chunk Ratio}, \textit{Fail Section Ratio} and \textit{Uncovered Section Ratio}.
%
Minimizing these metrics will reduce the information loss of some sections or parts of sections that are missing from the chunks.
%
Additionally, \textit{Sections/Chunk} and \textit{Chunks/Section} should be close to 1 in order for sections \emph{not to be} split into multiple chunks and retain atomicity within each chunk.

We evaluate multiple configurations of chunking strategies, chunk sizes, and overlaps as described in \S\ref{subsubsec: chunk_setup} and present the average metrics for each chunking strategy in Table~\ref{table: chunking_by_type}. 
%
It is observed that the chunking strategy most closely resembling the output of our hierarchy-aware chunking strategy is line-based chunking. 

However, across all strategies, approximately 30\% of sections are not referenced in any chunks, and at least 41.7\% of sections are not fully contained within a single chunk. 
%
Further analysis indicates that around 20\% of all sections cannot be fully covered in a single chunk under any naive chunking strategy due to their extended length. 
%
This necessitates the retrieval model to retrieve multiple chunks to provide sufficient context.

An example of such a section is Section 44 of the \textbf{Emergency Decree on Digital Asset Businesses, B.E. 2561}, which cannot be fully covered in a single chunk across any naive chunking strategy. 
%
This is attributed to its lengthy content and the presence of multiple subsections separated by newline characters, which are commonly used as delimiters in many naive chunking approaches.

% Should this be translated?
\begin{quote}
    \textbf{Section 44 of Emergency Decree on Digital Asset Businesses, B.E. 2561}
    
    It shall be presumed that the following persons, who exhibit behavior involving the buying or selling of digital tokens or engaging in forward contracts related to digital tokens in an unusual manner for themselves, are persons who possess or are aware of inside information as defined under Section 42:
    
    (1) Holders of digital tokens exceeding 5\% of the total tokens sold in each series by the issuer of digital tokens. This includes digital tokens held by their spouses, cohabiting partners in the manner of husband and wife, and their minor children.
    
    (2) Directors, executives, controlling persons, employees, or staff members of the affiliated entities of the digital token issuer who are in positions or roles responsible for, or with access to, inside information.
    
    (3) Ascendants, descendants, adoptive parents, or adopted children of persons specified under Section 43.
    
    (4) Siblings sharing the same father and mother, or the same father or mother as persons specified under Section 43.
    
    (5) Spouses or cohabiting partners in the manner of husband and wife of persons specified under Section 43 or individuals listed under (3) or (4).
    
    The term \enquote{affiliated entities of the digital token issuer} under (2) refers to parent companies, subsidiaries, or associated companies of the digital token issuer, as defined by the criteria set forth by the SEC Board's announcements.
\end{quote}

\begin{table}[!ht]
\centering

\resizebox{\textwidth}{!}{%
\renewcommand{\arraystretch}{1.3} % This increases the cell height by 1.5 times
\small % or \scriptsize
\begin{tabular}{@{}lccccc@{}}
\toprule
\textbf{Chunking Strategy} & \multicolumn{1}{l}{\textbf{Section/Chunk $\rightarrow$1}} & \multicolumn{1}{l}{\textbf{Chunk/Section $\rightarrow$1}} & \multicolumn{1}{l}{\textbf{Fail Chunk Ratio $\downarrow$}} & \multicolumn{1}{l}{\textbf{Fail Section Ratio $\downarrow$}} & \multicolumn{1}{l}{\textbf{Uncovered Section Ratio $\downarrow$}} \\ \midrule
\cellcolor{lightgray}Hierarchy-aware  & \cellcolor{lightgray}{1.000}   & \cellcolor{lightgray}{1.000}    & \cellcolor{lightgray}{0.000}  & \cellcolor{lightgray}{0.000}  & \cellcolor{lightgray}{0.000}                                       \\
Character  & 3.098                             & 1.710                              & 0.819                                & 0.675                                  & 0.397                                       \\

Line       & \textbf{1.689}                    & \textbf{1.234}                    & \textbf{0.658}                       & \textbf{0.417}                         & \textbf{0.294}                              \\

Recursive  & \underline{1.793}                       &\underline{1.27}                        & \underline{0.741}                          & \underline{0.504}                            & \underline{0.381}                                 \\ \bottomrule
\end{tabular}
}
\caption{Information loss comparison between hierarchy-aware chunking compared to other naive chunking strategies. Since hierarchy-aware chunking consistently parses into a single law section, it was treated as an upper bound because no information loss occurred.}
\label{table: chunking_by_type}
\end{table}

For the specific configuration of line chunking that produces chunks most similar to hierarchy-aware chunking, we fix the chunking strategy while varying the chunk overlap and chunk size parameters. 
%
Increasing the chunk size results in more text per chunk, leading to higher \textbf{Sections/Chunk} and \textbf{Chunks/Section} values while reducing the \textbf{Fail Chunk Ratio}, \textbf{Fail Section Ratio}, and \textbf{Uncovered Section Ratio}. 
%
Similarly, increasing the overlap effectively increases the chunk size, producing comparable effects to directly increasing the chunk size.
%
Based on these observations, we select the optimal configuration for naive chunking as line chunking with a chunk size of 553 characters and a chunk overlap of 50 characters. The detailed results for this configuration are displayed in Appendix~\ref{appendix: chunk_hyper}.

Finally, the metrics for the selected naive chunking configuration are compared against hierarchy-aware chunking in Table~\ref{table: chunking_compare_metric}.

\begin{table}[!ht]
\centering

\resizebox{\textwidth}{!}{%
\renewcommand{\arraystretch}{1.3} % This increases the cell height by 1.5 times
\small % or \scriptsize
\begin{tabular}{@{}lccccc@{}}
\toprule
\textbf{Chunking Strategy} & \multicolumn{1}{l}{\textbf{Section/Chunk $\rightarrow$1}} & \multicolumn{1}{l}{\textbf{Chunk/Section $\rightarrow$1}} & \multicolumn{1}{l}{\textbf{Fail Chunk Ratio $\downarrow$}} & \multicolumn{1}{l}{\textbf{Fail Section Ratio $\downarrow$}} & \multicolumn{1}{l}{\textbf{Uncovered Section Ratio $\downarrow$}} \\ \midrule
Hierarchy-aware chunking  & 1.000   & 1.000    & 0.000  & 0.000  & 0.000                                       \\
Line chunking (553 chunk size and 50 chunk overlap)  & 1.956                             & 1.180                              & 0.521                                & 0.323                                  & 0.156                                       \\ \bottomrule
\end{tabular}
}
\caption{Information loss comparison between perfect chunking strategy (hierarchy-aware chunking) and the best naive chunking setup.}
\label{table: chunking_compare_metric}
\end{table}

% Next, we also show results on our benchmark as well
Apart from the evaluation of chunking in isolation in terms of information loss, we also present the evaluation results on our benchmark in Table~\ref{table: chunk_e2e_main}.

\begin{table}[ht!]
\centering
\resizebox{\textwidth}{!}{%
\begin{tabular}{@{}lccccccc@{}}
    \toprule
    \textbf{Settings} & \textbf{Retriever Multi MRR ($\uparrow$)} & \textbf{Retriever Recall ($\uparrow$)} & \textbf{Coverage ($\uparrow$)} & \textbf{Contradiction ($\downarrow$)} & \textbf{E2E Recall ($\uparrow$)} & \textbf{E2E Precision ($\uparrow$)} & \textbf{E2E F1 ($\uparrow$)} \\
    \midrule
    Naïve Chunking            & 0.786 & 0.935 & 86.6 & \textbf{0.050} & 0.882 & 0.613 & 0.722 \\
    Hierarchy-aware Chunking  & \textbf{0.834} & \textbf{0.942} & \textbf{86.7} & 0.054 & \textbf{0.894} & \textbf{0.630} & \textbf{0.739} \\
    \bottomrule
\end{tabular}%
}
\caption{Effect of chunking configuration on E2E performance on NitiBench-CCL}
\label{table: chunk_e2e_main}
\end{table}


% From table~\ref{table: chunk_e2e_main}, the naive chunking strategy performs significantly worse than section-based chunking in terms of retrieval performance on both WCX and Tax Case datasets. This discrepancy likely stems from two factors: First, naive chunking discards chunks that do not fully contain a section. Second, it often splits single sections across multiple chunks, rendering these fragmented sections unusable for evaluation and practical application, as legal responses require complete sections.

% The E2E performance also agrees with the retrieval performance in that the section-based chunking significantly outperforms line chunking. Interestingly,with line chunking, end-to-end (E2E) recall (based on sections cited by the LLM) exceeds retrieval recall. This stems from two factors: 1) LLM sometimes cites unretrieved sections, either from its internal knowledge or through hallucination; and 2) Line chunking’s mapping of chunks to single, fully covered sections can lead to partial section coverage within a chunk, causing LLM to cite portions outside the mapped section. Consequently, line chunking’s E2E recall can surpass its retrieval recall.

From Table~\ref{table: chunk_e2e_main}, the naive chunking strategy performs worse than hierarchy-aware chunking in terms of retrieval performance. 
%
This discrepancy likely arises because naive chunks often contain content from multiple sections, introducing \enquote{noise} that can negatively impact the retrieval model's ranking of relevant documents.  

However, in terms of end-to-end (E2E) performance, the system using hierarchy-aware chunking only slightly outperforms the one using naive chunking. 
%
We suspect that this is because the LLM can effectively filter out the \enquote{noise} in the retrieved sections during answer generation. 
%
As a result, the coverage and contradiction scores are not significantly different between the two systems.
%
Nevertheless, there remains a discrepancy in the E2E citation score.  

In conclusion, \textbf{hierarchy-aware chunking achieves a slight but consistent advantage over the naive chunking strategy.}

\subsubsection{NitiLink}
\label{subsubsec: referencer_result}

The evaluation results of the experiment described in \S\ref{subsubsec: referencer_setup} are presented in Table~\ref{table: augmenter_e2e_main}. 
%
In the table, ``Ref Depth 1'' denotes a RAG system that incorporates a NitiLink component with a maximum depth of 1, while \enquote{No Ref} represents a RAG system without NitiLink. 
%
For the metrics, ``NitiLink'' indicates retrieval metrics calculated on the augmented context, which includes both the initially retrieved sections and the additional sections fetched by NitiLink.

% \begin{table}[!ht]
% \centering
% \caption{Effect of augmenter configuration on E2E performance}
% \renewcommand{\arraystretch}{1.5} % This increases the cell height by 1.5 times
% \label{table: augmenter_e2e_main}
% \begin{tabular}{@{}c|cc|cc@{}}
% \toprule
% Dataset                                  & \multicolumn{2}{c|}{Tax}        & \multicolumn{2}{c}{WCX}         \\ \midrule
% Setting                                  & Ref Depth 1    & No Ref         & Ref Depth 1    & No Ref         \\ \midrule
% Retriever MRR                            & 0.574          & 0.574          & 0.809          & 0.809          \\
% \multicolumn{1}{l|}{Retriever Multi MRR} & 0.333          & 0.333          & 0.809          & 0.809          \\
% Retriever Recall                         & 0.499          & 0.499          & 0.938          & 0.938          \\
% Referencer MRR                           & \textbf{0.582} & 0.574          & 0.800          & \textbf{0.809} \\
% Referencer Multi MRR                     & \textbf{0.345} & 0.333          & 0.800          & \textbf{0.809} \\
% Referencer Recall                        & \textbf{0.602} & 0.499          & \textbf{0.940} & 0.938          \\
% Coverage                                 & 45.0           & \textbf{50.0}  & \textbf{86.3}  & 85.2           \\
% Contradiction                            & 0.520          & \textbf{0.460} & \textbf{0.051} & 0.055          \\
% E2E Recall                               & \textbf{0.354} & 0.333          & \textbf{0.885} & 0.880          \\
% E2E Precision                            & 0.630          & \textbf{0.64}  & 0.579          & \textbf{0.601} \\
% E2E F1                                   & \textbf{0.453} & 0.438          & 0.700          & \textbf{0.714} \\ \bottomrule
% \end{tabular}%
% \end{table}
\begin{table}[!ht]
\centering
\renewcommand{\arraystretch}{1.3}
\newcommand{\gray}{\cellcolor{gray!15}}
\newcommand{\pos}[1]{\textcolor{darkgreen}{(#1\%)}}
\newcommand{\negv}[1]{\textcolor{red}{(#1\%)}}

\begin{tabular}{lcccccc}
\toprule
\multirow{2}{*}{\textbf{Metric}} & \multicolumn{3}{c}{\textbf{NitiBench-CCL}} & \multicolumn{3}{c}{\textbf{NitiBench-Tax}} \\ 
 & \textbf{No Ref} & \gray \textbf{Ref Depth 1} & $\Delta$ & \textbf{No Ref} & \gray \textbf{Ref Depth 1} & $\Delta$ \\ 
\midrule
\multicolumn{7}{c}{\textbf{Retriever Metrics}} \\ 
\midrule
MRR ($\uparrow$)       & \multicolumn{2}{c}{0.809} & -  & \multicolumn{2}{c}{0.574} & -  \\
Multi MRR ($\uparrow$) & \multicolumn{2}{c}{0.809} & -  & \multicolumn{2}{c}{0.333} & -  \\
Recall ($\uparrow$)    & \multicolumn{2}{c}{0.938} & -  & \multicolumn{2}{c}{0.499} & -  \\
\midrule
\multicolumn{7}{c}{\textbf{NitiLink Metrics}} \\ 
\midrule
MRR ($\uparrow$)             & 0.809  & \gray 0.800  & \negv{-1.11}  & 0.574  & \gray \textbf{0.582}  & \pos{+1.39}  \\
Multi MRR ($\uparrow$)       & 0.809  & \gray 0.800  & \negv{-1.11}  & 0.333  & \gray \textbf{0.345}  & \pos{+3.60}  \\
Recall ($\uparrow$)          & 0.938  & \gray \textbf{0.940}  & \pos{+0.21}  & 0.499  & \gray \textbf{0.602}  & \pos{+20.6}  \\
Coverage ($\uparrow$)        & 85.2   & \gray \textbf{86.3}  & \pos{+1.29}  & \textbf{50.0}   & \gray 45.0   & \negv{-10.0}  \\
Contradiction ($\downarrow$) & 0.055  & \gray \textbf{0.051}  & \pos{-7.27}  & \textbf{0.460}  & \gray 0.520  & \negv{+13.0}  \\
E2E Recall ($\uparrow$)      & 0.880  & \gray \textbf{0.885}  & \pos{+0.57}  & 0.333  & \gray \textbf{0.354}  & \pos{+6.31}  \\
E2E Precision ($\uparrow$)   & 0.601  & \gray 0.579  & \negv{-3.66}  & \textbf{0.640}  & \gray 0.630  & \negv{-1.56}  \\
E2E F1 ($\uparrow$)          & \textbf{0.714}  & \gray 0.700  & \negv{-1.96}  & 0.438  & \gray \textbf{0.453}  & \pos{+3.42}  \\
\bottomrule
\end{tabular}
\caption{Effect of NitiLink augmenter configuration on E2E performance. The $\Delta$ column shows the relative percentage change compared to ``No Ref'', with dark green indicating improvement and red indicating degradation.}
\label{table: augmenter_e2e_main}
\end{table}




The results from Table~\ref{table: augmenter_e2e_main} show that there is no clear significant advantage when employing NitiLink in a RAG system. 
%
The results also highlight the differing impacts of incorporating NitiLink across datasets.

\textbf{NitiBench-Tax} For this dataset, we can clearly see that the recall was substantially improved from 0.499 to 0.602.
%
The improvement of recall suggested that NitiLink does provide an additional correct law section to the retrieved documents. 
%
Despite significant improvement over recall, we only see marginal improvements over MRR and Multi MRR.
%
Since we're using a depth-first augmented strategy (see \S\ref{subsubsec: referencer_setup}), this suggested that the document that cited more positives by NitiLink is ranked at the bottom of the retrieved documents.
%
Surprisingly, despite a major improvement in recall, some E2E metrics declined.
%
This might be due to NitiBench-Tax's query complexity, which often demands advanced reasoning capabilities that the LLM, even with the correct documents, struggles to provide. 
%
Another reason that might affect the performance decline even with more relevant documents provided to the LLM is the longer context that the LLM needs to process due to the higher amount of content added by NitiLink.

\textbf{NitiBench-CCL} For NitiBench-CCL showed no significant change in retrieval metrics and most E2E metrics.
%
Incorporating NitiLink yields very little recall gain, while MRR is slightly lower. 
%
This means that NitiLink often pushed the positive lower in the ranking as we're using a depth-first augmentation strategy (see \S\ref{subsubsec: referencer_setup}).
%
We highlight several factors that might contribute to the limited recall gain in this dataset:
\begin{enumerate}
    \item \textbf{Binary recall nature:} NitiBench-CCL queries typically involve a single relevant law, making recall binary and thus harder to improve.
    %
    \item \textbf{Simplicity of NitiBench-CCL queries:} Simple, non-specific NitiBench-CCL queries often rely on many relevant law sections that are similar semantically rather than hierarchically. 
    %
    This is opposed to NitiBench-Tax, where referenced law sections are necessary for legal reasoning.
    %
    This simplicity stems from the fact that the dataset was created by letting the annotator craft a question based on a given law section.
    %
    This explicitly provides bias toward the dataset since the question was created without a referenced law section.
    %
    \item \textbf{Hierarchical limiation:} The hierarchical structure itself presents challenges. 
    %
    Although NitiLink augmented the retrieved law section mentioned in the retrieved document (children reference), it lacks a law section that references retrieved law sections (parent reference).  
    %
    Thus, this version of NitiLink that lacks the ability to fetch parent law sections could result in a suboptimal performance.
    %
    %\item The hierarchical structure itself presents challenges. Many sections in the Revenue Code lack hierarchical connections (41\% have no children, 45\% no parents, and 24\% neither), limiting the referencer's effectiveness. Furthermore, some queries (e.g., those about criminal penalties) require retrieving laws that reference multiple others. The current referencer, retrieving only children of initially retrieved laws, struggles with these, as it cannot retrieve the parent law (which might be the ground truth).
\end{enumerate}

Despite the limited recall gains in NitiBench-CCL, we can see that there's a slight improvement in coverage score as well as recall.
%
This suggests that even small recall improvements can enhance the LLM's ability to answer NitiBench-CCL queries effectively.

\subsection{(RQ2) Impact of Retriever and LLM}
\label{subsec: rq2_result}

\subsubsection{Retriever}
\label{subsubsec: retriever_result}

\textbf{NitiBench-CCL} 
%
Table~\ref{table: retrieval_wangchan} presents the retrieval performance of 8 models as described in \S\ref{subsubsec: retriever_setup} on NitiBench-CCL with hierarchy-aware chunking. 
%
Because each query has only one positive label (as mentioned in \S\ref{subsubsec: wcx_dataset}), the multi-hit-rate and multi-MRR metrics are equivalent to their single-label counterparts. This also applies to recall and hit rate as well. Thus, for this dataset, we only showed Recall (Recall@K) and MRR (MRR@K) since other metrics are considered redundant.

The best-performing model is the human-reranked fine-tuned BGE-M3, achieving an MRR@5 of 0.805. 
%
Close behind are the auto-reranked fine-tuned BGE-M3 (0.800 MRR@5) and the base BGE-M3 (0.579 MRR@5). 
%
BGE-M3's strong performance is likely due to its use of three embedding types for relevance calculation, further enhanced by fine-tuning on in-domain data. 
%
Notably, the auto-reranked version nearly matches human-reranked performance without requiring costly human annotation. 
%

\textbf{Based on these findings, we recommend a cost-effective in-domain adaptation pipeline, notably Auto-Finetuned BGE-M3, that uses a strong LLM to generate synthetic training pairs, retrieves top-k passages with BGE-M3, and then applies a BGE-M3 Reranker. 
%
As shown in the results, this approach closely matches human-reranked performance while significantly saving annotation costs in an in-domain setup.
}

The commercially available Cohere embedding model ranks just below the top-performing BGE-M3 models and is followed by the ColBERT-based and dense embedding models, JINA ColBERT v2 and JINA embeddings v3, respectively. 
%
Among the tested retrievers, NV-Embed v1 shows the lowest performance among non-baseline models (0.713 MRR@5), likely due to its decoder-based architecture and reliance on prefix instruction prompts. 
%
Overall, retrieval performance on NitiBench-CCL is strong, with most models delivering comparable results, except for NV-Embed v1 and BM25. 
%
However, despite this strong performance, a gap between hit-rate and MRR when $k=\{5,10\}$ indicates that \textbf{while relevant documents are frequently retrieved, they are not consistently ranked first, potentially impacting end-to-end performance.}

\begin{table}[!ht]
\centering

% \renewcommand{\arraystretch}{1.5}
\small
\begin{tabular}{@{}clcc@{}}
\toprule
\textbf{Top-K} & \textbf{Model} & \textbf{HR/Recall@k} & \textbf{MRR@k} \\ \midrule
\multirow{8}{*}{k=1} 
  & BM25                   & .481 & .481 \\
  & JINA V2                & .681 & .681 \\
  & JINA V3                & .587 & .587 \\
  & NV-Embed V1            & .492 & .492 \\
  & BGE-M3                 & .700 & .700 \\
  & Human-Finetuned BGE-M3 & \textbf{.735} & \textbf{.735} \\
  & Auto-Finetuned BGE-M3  & \underline{.731} & \underline{.731} \\
  & Cohere                 & .676 & .676 \\ \midrule
\multirow{8}{*}{k=5} 
  & BM25                   & .658 & .548 \\
  & JINA V2                & .852 & .750 \\
  & JINA V3                & .821 & .681 \\
  & NV-Embed V1            & .713 & .579 \\
  & BGE-M3                 & .880 & .773 \\
  & Human-Finetuned BGE-M3 & \textbf{.906} & \textbf{.805} \\
  & Auto-Finetuned BGE-M3  & \underline{.900} & \underline{.800} \\
  & Cohere                 & .870 & .754 \\ \midrule
\multirow{8}{*}{k=10} 
  & BM25                   & .715 & .556 \\
  & JINA V2                & .889 & .755 \\
  & JINA V3                & .875 & .688 \\
  & NV-Embed V1            & .776 & .587 \\
  & BGE-M3                 & .919 & .778 \\
  & Human-Finetuned BGE-M3 & \textbf{.938} & \textbf{.809} \\
  & Auto-Finetuned BGE-M3  & \underline{.934} & \underline{.804} \\
  & Cohere                 & .912 & .760 \\ \bottomrule
\end{tabular}
\caption{Retrieval Evaluation Result on NitiBench-CCL with hierarchy-aware chunking. Since the test split contains a single positive (as mentioned in \S \ref{subsubsec: wcx_dataset}), we collapsed metrics that are duplicated, such as HitRate (HR)/ Recall/ Multi-HitRate and MultiMRR / MRR.}
\label{table: retrieval_wangchan}
\end{table}


\textbf{NitiBench-Tax} Table~\ref{table: retrieval_tax} presents the retrieval performance of various models on NitiBench-Tax using hierarchy-aware chunking. 
%
Unlike NitiBench-CCL, this dataset includes multi-label queries, resulting in different values for single-label and multi-label metrics.

Overall performance is significantly lower on this dataset compared to NitiBench-CCL, likely due to the considerably longer and more nuanced queries in NitiBench-Tax.
%
JINA v3 and BGE-M3 (base, auto-fine-tuned, and human-finetuned) consistently perform among the top, achieving Multi-MRR@10 scores of 0.311, 0.354, 0.345, and 0.333, respectively. 
%
Conversely, JINA v2 and NV-Embed v1 consistently underperform compared to the baseline, potentially because NitiBench-Tax is out-of-distribution relative to their training data, considering the complexity of the query.
%
This is particularly evident with JINA v2, whose Multi-MRR@10 drops dramatically from 0.750 on NitiBench-CCL to 0.091 on NitiBench-Tax.

Similarly, the Human-Finetuned BGE-M3 variants are often outperformed by the base BGE-M3, suggesting different data distributions between NitiBench-CCL and NitiBench-Tax, hindering cross-dataset generalization. 
%
While some models achieve reasonable single-label hit rates, multi-label hit-rate performance is poor across all models. 
%
This, combined with low recall and significantly lower multi-label MRR compared to single-label MRR, indicates that while models can often retrieve some relevant documents, they struggle to retrieve all relevant documents for a given query. 
%
This limitation is critical, as comprehensive legal responses require consideration of all relevant legal sections. 
%
\textbf{Although the proposed pipeline (Human-Finetuned BGE-M3) performs strongly on in-domain data (as seen with NitiBench-CCL), these Tax Case results underscore the critical need for sufficiently diverse in-domain training data, since a narrow domain distribution can lead to inconsistent or contradictory outcomes in real-world settings.}

Despite its lower overall performance, NitiBench-Tax benefits more from increasing the number of top-k retrieved documents compared to NitiBench-CCL. 
%
Its hit rate and recall improve at a faster rate as more documents are retrieved compared to NitiBench-CCL.

\begin{table}[!ht]
\centering
\small
\begin{tabular}{@{}clccccc@{}}
\toprule
\textbf{Top-K} & \textbf{Model}                  & \textbf{HR@k}          & \textbf{Multi HR@k}    & \textbf{Recall@k}      & \textbf{MRR@k}         & \textbf{Multi MRR@k}   \\ \midrule
k=1   & BM25                   & .220          & .080          & .118          & .220          & .118          \\
      & JINA V2                & .140          & .040          & .068          & .140          & .068          \\
      & JINA V3                & .400          & .100          & .203          & .400          & .203          \\
      & NV-Embed V1            & .100          & .020          & .035          & .100          & .035          \\
      & BGE-M3                 & \underline{.500}    & \underline{.140}    & \underline{.269}    & \underline{.500}    & \underline{.269}    \\
      & Human-Finetuned BGE-M3 & .480          & \underline{.140}    & .255          & .480          & .255          \\
      & Auto-Finetuned BGE-M3  & \textbf{.520} & \textbf{.160} & \textbf{.281} & \textbf{.520} & \textbf{.281} \\
      & Cohere                 & .340          & .100          & .179          & .340          & .179          \\ \midrule
k=5   & BM25                   & .480          & .120          & .254          & .318          & .171          \\
      & JINA V2                & .200          & .080          & .114          & .165          & .085          \\
      & JINA V3                & \underline{.720}    & \textbf{.260} & \textbf{.448} & .508          & .297          \\
      & NV-Embed V1            & .200          & .020          & .081          & .126          & .050          \\
      & BGE-M3                 & \underline{.720}    & \underline{.240}    & \underline{.435}    & \underline{.580}    & \textbf{.337} \\
      & Human-Finetuned BGE-M3 & \textbf{.740} & .220          & .411          & .565          & .320          \\
      & Auto-Finetuned BGE-M3  & .700          & .200          & .382          & \textbf{.587} & \underline{.329}    \\
      & Cohere                 & .620          & .200          & .363          & .447          & .256          \\ \midrule
k=10  & BM25                   & .540          & .160          & .320          & .327          & .183          \\
      & JINA V2                & .240          & .100          & .147          & .171          & .091          \\
      & JINA V3                & \textbf{.840} & \underline{.340}    & \underline{.549}    & .524          & .311          \\
      & NV-Embed V1            & .220          & .040          & .097          & .128          & .052          \\
      & BGE-M3                 & \underline{.820}    & \textbf{.360} & \textbf{.555} & \underline{.593}    & \textbf{.354} \\
      & Human-Finetuned BGE-M3 & .800          & .280          & .499          & .574          & .333          \\
      & Auto-Finetuned BGE-M3  & .780          & .260          & .483          & \textbf{.600} & \underline{.345}    \\
      & Cohere                 & .680          & .200          & .414          & .454          & .263          \\ \bottomrule
\end{tabular}
\caption{Retrieval Evaluation Result on NitiBench-Tax with hierarchy-aware chunking. This split contains multiple positives per question.}
\label{table: retrieval_tax}
\end{table}



\subsubsection{LLM}
\label{subsubsec: llm_result}

The evaluation results of the experiments described in \S\ref{subsubsec: llm_setup} are presented in Table~\ref{table: llm_e2e_main_wcx} for NitiBench-CCL and Table~\ref{table: llm_e2e_main_tax} for NitiBench-Tax. 
%
Since experiments in \S\ref{subsubsec: referencer_result} do not provide conclusive results on whether the inclusion of NitiLink is necessary, we also vary the inclusion of NitiLink in this experiment as well.


\begin{table}[!ht]
\centering
\renewcommand{\arraystretch}{1.2}
\resizebox{\textwidth}{!}{%
\begin{tabular}{@{}lcccccccc@{}}
\toprule
\textbf{Setting} & \textbf{NitiLink} & \textbf{Retriever MRR ($\uparrow$)} & \textbf{Retriever Recall ($\uparrow$)} & \textbf{E2E Recall ($\uparrow$)} & \textbf{E2E Precision ($\uparrow$)} & \textbf{E2E F1 ($\uparrow$)} & \textbf{Coverage ($\uparrow$)} & \textbf{Contradiction ($\downarrow$)} \\ \midrule
\multirow{3}{*}{\texttt{gpt-4o-2024-08-06}} 
& No Ref      & \multirow{3}{*}{0.809} & \multirow{3}{*}{0.938} & 0.880  & \textbf{0.601}  & \textbf{0.714}  & 85.2  & 0.055  \\
& \cellcolor{lightgray}Ref Depth 1 &                      &                     & \cellcolor{lightgray}0.885  & \cellcolor{lightgray}\underline{0.579}  & \cellcolor{lightgray}\underline{0.700}  & \cellcolor{lightgray}86.3  & \cellcolor{lightgray}0.051  \\
& $\Delta$    &                      &                     & \textcolor{darkgreen}{+0.6\%}  & \textcolor{red}{-3.7\%}  & \textcolor{red}{-2.0\%}  & \textcolor{darkgreen}{+1.3\%}  & \textcolor{darkgreen}{-7.3\%}  \\ \midrule
\multirow{3}{*}{\texttt{gemini-1.5-pro-002}} 
& No Ref      & \multirow{3}{*}{0.809} & \multirow{3}{*}{0.938} & 0.892  & 0.512  & 0.651  & 86.5  & 0.048  \\
& \cellcolor{lightgray}Ref Depth 1 &                      &                     & \cellcolor{lightgray}\underline{0.895}  & \cellcolor{lightgray}0.491  & \cellcolor{lightgray}0.634  & \cellcolor{lightgray}87.3  & \cellcolor{lightgray}\underline{0.042}  \\
& $\Delta$    &                      &                     & \textcolor{darkgreen}{+0.3\%}  & \textcolor{red}{-4.1\%}  & \textcolor{red}{-2.6\%}  & \textcolor{darkgreen}{+0.9\%}  & \textcolor{darkgreen}{-12.5\%} \\ \midrule
\multirow{3}{*}{\texttt{claude-3-5-sonnet-20240620}} 
& No Ref      & \multirow{3}{*}{0.809} & \multirow{3}{*}{0.938} & \textbf{0.901} & 0.444  & 0.595  & \textbf{89.7} & \textbf{0.040}  \\ 
& \cellcolor{lightgray}Ref Depth 1 &                      &                     & \cellcolor{lightgray}0.894  & \cellcolor{lightgray}0.443  & \cellcolor{lightgray}0.592  & \cellcolor{lightgray}\underline{89.5} & \cellcolor{lightgray}0.044  \\
& $\Delta$    &                      &                     & \textcolor{red}{-0.8\%} & \textcolor{red}{-0.2\%} & \textcolor{red}{-0.5\%}  & \textcolor{red}{-0.2\%}  & \textcolor{red}{+10.0\%}  \\ \midrule
\multirow{3}{*}{\texttt{typhoon-v2-70b-instruct}} 
& No Ref      & \multirow{3}{*}{0.809} & \multirow{3}{*}{0.938} & 0.862  & 0.537  & 0.662  & 81.2  & 0.076  \\
& \cellcolor{lightgray}Ref Depth 1 &                      &                     & \cellcolor{lightgray}0.845  & \cellcolor{lightgray}0.573  & \cellcolor{lightgray}0.683  & \cellcolor{lightgray}79.9  & \cellcolor{lightgray}0.080  \\
& $\Delta$    &                      &                     & \textcolor{red}{-2.0\%} & \textcolor{darkgreen}{+6.7\%} & \textcolor{darkgreen}{+3.2\%}  & \textcolor{red}{-1.6\%}  & \textcolor{red}{+5.3\%}  \\ \midrule
\multirow{3}{*}{\texttt{typhoon-v2-8b-instruct}} 
& No Ref      & \multirow{3}{*}{0.809} & \multirow{3}{*}{0.938} & 0.775  & 0.387  & 0.516  & 70.8  & 0.134  \\
& \cellcolor{lightgray}Ref Depth 1 &                      &                     & \cellcolor{lightgray}0.718  & \cellcolor{lightgray}0.385  & \cellcolor{lightgray}0.501  & \cellcolor{lightgray}68.5  & \cellcolor{lightgray}0.145  \\
& $\Delta$    &                      &                     & \textcolor{red}{-7.4\%} & \textcolor{red}{-0.5\%}  & \textcolor{red}{-2.9\%}  & \textcolor{red}{-3.3\%}  & \textcolor{red}{+8.2\%}  \\ \bottomrule
\end{tabular}
}
\caption{Effect of LLM configuration on E2E performance on NitiBench-CCL. $\Delta$ values are computed relative to No Ref and normalized to percentage change.}
\label{table: llm_e2e_main_wcx}
\end{table}


\begin{table}[!ht]
\centering
\renewcommand{\arraystretch}{1.2}
\resizebox{\textwidth}{!}{%
\begin{tabular}{@{}lccccccccc@{}}
\toprule
{\textbf{Setting}} & {\textbf{NitiLink}} 
& {\textbf{Retriever MRR ($\uparrow$)}} 
& {\textbf{Retriever Multi MRR ($\uparrow$)}} 
& {\textbf{Retriever Recall ($\uparrow$)}} 
& {\textbf{E2E Recall ($\uparrow$)}} 
& {\textbf{E2E Precision ($\uparrow$)}} 
& {\textbf{E2E F1 ($\uparrow$)}} 
& {\textbf{Coverage ($\uparrow$)}} 
& {\textbf{Contradiction ($\downarrow$)}} \\ 
\midrule

%---------------- gpt-4o-2024-08-06 ----------------
\multirow{3}{*}{\texttt{gpt-4o-2024-08-06}} 
& No Ref            
  & \multirow{3}{*}{0.574} 
  & \multirow{3}{*}{0.333}  
  & \multirow{3}{*}{0.499}
  & 0.333 
  & \underline{0.640}
  & 0.438 
  & 50.0  
  & \underline{0.46} \\
& \cellcolor{lightgray}Ref Depth 1  
  & 
  & 
  & 
  & \cellcolor{lightgray}0.354 
  & \cellcolor{lightgray}0.630 
  & \cellcolor{lightgray}0.453 
  & \cellcolor{lightgray}45.0 
  & \cellcolor{lightgray}0.52 \\
& $\Delta$           
  & 
  & 
  & 
  & \textcolor{darkgreen}{+6.3\%}
  & \textcolor{red}{-1.6\%}
  & \textcolor{darkgreen}{+3.4\%}
  & \textcolor{red}{-10.0\%}
  & \textcolor{red}{+13.0\%} \\
\midrule

%---------------- gemini-1.5-pro-002 ----------------
\multirow{3}{*}{\texttt{gemini-1.5-pro-002}} 
& No Ref            
  & \multirow{3}{*}{0.574} & \multirow{3}{*}{0.333} & \multirow{3}{*}{0.499}
  & 0.361 & 0.308 & 0.332 
  & 44.0  & 0.48 \\
& \cellcolor{lightgray}Ref Depth 1  
  & 
  & 
  & 
  & \cellcolor{lightgray}0.354 
  & \cellcolor{lightgray}0.347 
  & \cellcolor{lightgray}0.351 
  & \cellcolor{lightgray}45.0 
  & \cellcolor{lightgray}0.48 \\
& $\Delta$           
  & 
  & 
  & 
  & \textcolor{red}{-1.9\%}
  & \textcolor{darkgreen}{+12.7\%}
  & \textcolor{darkgreen}{+5.7\%}
  & \textcolor{darkgreen}{+2.3\%}
  & \textcolor{black}{+0.0\%} \\
\midrule

%---------------- claude-3-5-sonnet-20240620 ----------------
\multirow{3}{*}{\texttt{claude-3-5-sonnet-20240620}} 
& No Ref            
  & \multirow{3}{*}{0.574} & \multirow{3}{*}{0.333} & \multirow{3}{*}{0.499 }
  & \underline{0.389} & 0.554 & \underline{0.457}
  & \underline{51.0}  & \textbf{0.44} \\
& \cellcolor{lightgray}Ref Depth 1  
  & 
  & 
  & 
  & \cellcolor{lightgray}\textbf{0.417} 
  & \cellcolor{lightgray}0.577 
  & \cellcolor{lightgray}\textbf{0.484} 
  & \cellcolor{lightgray}49.0 
  & \cellcolor{lightgray}0.56 \\
& $\Delta$           
  & 
  & 
  & 
  & \textcolor{darkgreen}{+7.2\%}
  & \textcolor{darkgreen}{+4.2\%}
  & \textcolor{darkgreen}{+5.9\%}
  & \textcolor{red}{-3.9\%}
  & \textcolor{red}{+27.3\%} \\
\midrule

%---------------- typhoon-v2-70b-instruct ----------------
\multirow{3}{*}{\texttt{typhoon-v2-70b-instruct}} 
& No Ref            
  & \multirow{3}{*}{0.574} & \multirow{3}{*}{0.333} & \multirow{3}{*}{0.499}
  & 0.326 & \textbf{0.662} & 0.437 
  & 42.0  & 0.58 \\
& \cellcolor{lightgray}Ref Depth 1  
  & 
  & 
  &  
  & \cellcolor{lightgray}0.333 
  & \cellcolor{lightgray}0.453 
  & \cellcolor{lightgray}0.384 
  & \cellcolor{lightgray}\textbf{54.0 }
  & \cellcolor{lightgray}\underline{0.46} \\
& $\Delta$           
  & 
  & 
  & 
  & \textcolor{darkgreen}{+2.1\%}
  & \textcolor{red}{-31.6\%}
  & \textcolor{red}{-12.1\%}
  & \textcolor{darkgreen}{+28.6\%}
  & \textcolor{darkgreen}{-20.7\%} \\
\midrule

%---------------- typhoon-v2-8b-instruct ----------------
\multirow{3}{*}{\texttt{typhoon-v2-8b-instruct}} 
& No Ref            
  & \multirow{3}{*}{0.574} & \multirow{3}{*}{0.333} & \multirow{3}{*}{0.499 }
  & 0.278 & 0.471 & 0.349 
  & 37.0  & 0.60 \\
& \cellcolor{lightgray}Ref Depth 1  
  & 
  & 
  & 
  & \cellcolor{lightgray}0.319 
  & \cellcolor{lightgray}0.561 
  & \cellcolor{lightgray}0.407 
  & \cellcolor{lightgray}35.0 
  & \cellcolor{lightgray}0.54 \\
& $\Delta$           
  & 
  & 
  & 
  & \textcolor{darkgreen}{+14.7\%}
  & \textcolor{darkgreen}{+19.1\%}
  & \textcolor{darkgreen}{+16.6\%}
  & \textcolor{red}{-5.4\%}
  & \textcolor{darkgreen}{-10.0\%} \\
\bottomrule
\end{tabular}
}
\caption{Effect of LLM configuration on E2E performance on NitiBench-Tax. 
\(\Delta\) values are computed relative to No Ref and normalized to percentage change.}
\label{table: llm_e2e_main_tax}
\end{table}


\textbf{NitiBench-CCL} On NitiBench-CCL, \texttt{claude-3.5-sonnet} excels in Coverage, Contradiction, and E2E Recall. 
%
However, the typhoon family of models struggles to match the performance of the closed-source models in this broader Thai legal QA domain. 
%
As seen with NitiBench-Tax, both \texttt{claude-3.5-sonnet} and \texttt{gemini-1.5-pro-002} exhibit low E2E Precision, leading to lower F1 scores compared to \texttt{gpt-4o}. 
%
This underscores a trade-off between recall and precision, particularly for \texttt{claude-3.5-sonnet} and \texttt{gemini-1.5-pro-002}. 
%
The causes of this precision drop are further analyzed in a later section.

\textbf{NitiBench-Tax} The results on NitiBench-Tax demonstrate that \texttt{claude-3.5-sonnet} also achieves top-2 performance across most end-to-end metrics. 
%
Interestingly, \texttt{typhoon-v2-70b-instruct}, an open-sourced model, delivers comparable results and outperforms others on NitiBench-Tax with the highest coverage score and E2E precision. 
%
However, its smaller variant, \texttt{typhoon-v2-8b-instruct}, ranks the lowest among the tested models.f
%`
Despite its limited parameter size, it manages to avoid falling significantly behind, showcasing a reasonable performance given its constraints.

Notably, both \texttt{claude-3.5-sonnet} and \texttt{gemini-1.5-pro-002} exhibit considerably lower E2E precision compared to \texttt{gpt-4o} and \texttt{typhoon-v2-70b-instruct}. This compromises their suitability for precision-critical applications, even though they excel in other areas. Additionally, the NitiLink module fails to consistently enhance performance, with mixed results indicating no definitive advantage in its current configuration.

\subsubsection{E2E results using best setups}
\label{subsubsec: e2e_best_result}

The results from experiments conducted under the four settings described in \S\ref{subsubsec: e2e_best_setup} are presented in Table~\ref{table: main_exp_full}. 
%
Both the Proposed RAG and Naive RAG settings utilize Human-Finetuned BGE-M3 as the retriever, as it demonstrated the best performance in previous experiments (Section~\ref{subsubsec: retriever_result}). 
%
Similarly, Claude 3.5 Sonnet is chosen as the generator based on its superior results in \S\ref{subsubsec: llm_result}. 
%
NitiLink is excluded in both settings due to its inconclusive impact on E2E performance (Section~\ref{subsubsec: referencer_result} and~\ref{subsubsec: llm_result}). 
%
We opted for this choice instead of the opposite because omitting NitiLink yields similar performance while reducing API costs. 
%
The primary distinction between Naive RAG and Proposed RAG lies in their chunking strategies: Naive RAG employs a naive chunking approach, whereas Proposed RAG utilizes hierarchy-aware chunking.

\begin{table}[!ht]
\centering
\renewcommand{\arraystretch}{1.2} % Increase cell height
\resizebox{\textwidth}{!}{%
\begin{tabular}{@{}lcccccccc@{}}
\toprule
{\textbf{Setting}} & \textbf{{Retriever MRR} ($\uparrow$)} & \textbf{{Retriever Multi-MRR} ($\uparrow$)} & \textbf{{Retriever Recall} ($\uparrow$)} & \textbf{{Coverage} ($\uparrow$)} & \textbf{{Contradiction} ($\downarrow$)} & \textbf{{E2E Recall} ($\uparrow$)} & \textbf{{E2E Precision} ($\uparrow$)} & \textbf{{E2E F1} ($\uparrow$)} \\ 
\midrule
\multicolumn{9}{c}{\textbf{NitiBench-CCL}} \\ \midrule
Parametric     
 & -     
 & -     
 & -     
 & 60.3     
 & 0.199   
 & 0.188   
 & 0.141   
 & 0.161  
\\
Na\"ive RAG    
 & 0.120 
 & 0.048 
 & 0.062 
 & 77.3     
 & 0.097   
 & 0.745   
 & 0.370   
 & 0.495  
\\
Proposed RAG   
 & 0.809 
 & 0.809 
 & 0.938 
 & \textbf{89.7}     
 & \textbf{0.040}   
 & \textbf{0.901}   
 & \textbf{0.444}   
 & \textbf{0.595}  
\\
\rowcolor{gray!15}
Golden Context 
 & 1.0   
 & 1.0   
 & 1.0   
 & 93.4     
 & 0.034   
 & 0.999   
 & 1.000   
 & 1.000  
\\ 
\midrule
\multicolumn{9}{c}{\textbf{NitiBench-Tax}} \\ \midrule
Parametric     
 & -     
 & -     
 & -     
 & 46.0     
 & 0.480   
 & \textbf{0.458}   
 & \textbf{0.629}   
 & \textbf{0.530}  
\\
Na\"ive RAG    
 & 0.120 
 & 0.048 
 & 0.062 
 & 50.0     
 & 0.460   
 & 0.306   
 & 0.463   
 & 0.368  
\\
Proposed RAG   
 & 0.574 
 & 0.333 
 & 0.499 
 & \textbf{51.0}     
 & \textbf{0.440}   
 & {0.389}   
 & {0.554}   
 & {0.457}  
\\
\rowcolor{gray!15}
Golden Context 
 & 1.0   
 & 1.0   
 & 1.0   
 & 52.0     
 & 0.460   
 & 0.694   
 & 1.000   
 & 0.820  
\\ 
\bottomrule
\end{tabular}
}
\caption{E2E Experiment results on NitiBench-CCL and NitiBench-Tax comparing various RAG setups on Human-Finetuned BGE-M3 retriever and \texttt{claude-3-5-sonnet-20240620} as a LLM.}
\label{table: main_exp_full}
\end{table}



\textbf{NitiBench-Tax} On NitiBench-Tax, the four main settings perform similarly, except for the parametric setting's slightly lower coverage and higher contradiction. Two key observations emerge:

First, the parametric setting achieves the second-highest E2E recall and precision despite lacking a retriever. 
%
To investigate this further, we inspected the cited law section generated by LLM. 
%
Surprisingly, we found that out of 105 law sections cited from LLM parametric knowledge, 58 of them aren't even retrieved by the best retriever. 
%
Among those 58 cited documents, 26 were in the correct law section. 
%
In contrast, only 5 of 101 sections cited by the proposed RAG system are \emph{not} retrieved. 
%
This indicates that retriever performance significantly constrains RAG systems, especially with complex queries like those in NitiBench-Tax. 
%
RAG system generators seem discouraged from using internal knowledge, which might sometimes provide better answers. 
%
Furthermore, the substantial disparity between retriever and E2E recall shows that the LLM often underutilizes relevant retrieved sections, particularly those containing primarily terminology (as discussed in \S\ref{subsec: retriever_re_error_analysis_tax}).
%
% We suspect that the high performance of parametric knowledge could be due to Thai law data or the case data from the revenue department website might be contaminated in the pretraining data since it's publicly available.
%
% However, we didn't further investigate this due to time constraint and we leave this to future works.

Second, Table~\ref{table: main_exp_full} (NitiBench-Tax) shows no clear relationship between E2E citation scores and coverage/contradiction. 
%
This suggests the LLM struggles to apply cited sections correctly in its reasoning, leading to incorrect or erroneously reasoned answers. 
%
This resembles the issue in \S\ref{subsubsec: referencer_result}, where improved retriever recall from NitiLink didn't consistently improve E2E metrics. 
%
Here, increased E2E citations with Claude 3.5 Sonnet don't necessarily improve coverage or contradiction.

\textbf{NitiBench-CCL} For NitiBench-CCL (Table~\ref{table: main_exp_full}), the results are as expected: parametric performs worst, followed by Naive RAG, then Proposed RAG, and finally Golden Context (best).

\subsection{(RQ3) Performance of Long-Context LLM (LCLM)}
\label{subsec: rq3_result}

The evaluation results of the Thai legal QA system based on LCLM are presented in Table~\ref{table: main_exp_sampled} in comparison to the same baselines in Table~\ref{table: main_exp_full}. 
%
The evaluation is conducted on a stratified 20\% sample of NitiBench-CCL and the full NitiBench-Tax. 
%
As detailed in \S\ref{subsec: setup_rq3}, the LCLM system processes all 35 Thai financial laws simultaneously as context, with special tokens inserted to serve as section identifiers. 
%
These tokens enable the LCLM to cite relevant sections explicitly when generating responses to queries.

\begin{table}[!ht]
\centering
\renewcommand{\arraystretch}{1.2} % Increase cell height
\resizebox{\textwidth}{!}{%
\begin{tabular}{@{}lcccccccc@{}}
\toprule
\textbf{Setting} & \textbf{Retriever MRR ($\uparrow$)} & \textbf{Retriever Multi-MRR ($\uparrow$)} & \textbf{Retriever Recall ($\uparrow$)} & \textbf{Coverage ($\uparrow$)} & \textbf{Contradiction ($\downarrow$)} & \textbf{E2E Recall ($\uparrow$)} & \textbf{E2E Precision ($\uparrow$)} & \textbf{E2E F1 ($\uparrow$)} \\ 
\midrule
\multicolumn{9}{c}{\textbf{NitiBench-CCL (20\% subsampled)}} \\ \midrule
Parametric     & -     & -     & -     & 60.6  & 0.198 & 0.197 & 0.147 & 0.169 \\
Na\"ive RAG    & 0.549 & 0.549 & 0.649 & 77.7  & 0.092 & 0.740 & 0.379 & 0.501 \\
Proposed RAG   & 0.825 & 0.825 & 0.945 & \textbf{90.1}  & \textbf{0.028} & \textbf{0.920} & 0.453 & 0.607 \\
LCLM (Gemini)  & -     & -     & -     & 83.2  & 0.063 & 0.765 & \textbf{0.514} & \textbf{0.615} \\
\rowcolor{gray!15}
Golden Context & 1.0   & 1.0   & 1.0   & 94.2  & 0.025 & 0.999 & 1.000 & 0.999 \\
\midrule
\multicolumn{9}{c}{\textbf{NitiBench-Tax}} \\ \midrule
Parametric     & -     & -     & -     & 46.0  & 0.480 & \textbf{0.458} & \textbf{0.629} & \textbf{0.530} \\
Na\"ive RAG    & 0.120 & 0.048 & 0.062 & 50.0  & 0.460 & 0.306 & 0.463 & 0.368 \\
Proposed RAG   & 0.574 & 0.333 & 0.499 & \textbf{51.0}  & \textbf{0.440} & 0.389 & 0.554 & 0.457 \\
LCLM (Gemini)  & -     & -     & -     & 36.0  & 0.620 & 0.410 & 0.484 & 0.444 \\
\rowcolor{gray!15}
Golden Context & 1.0   & 1.0   & 1.0   & 52.0  & 0.460 & 0.694 & 1.000 & 0.820 \\ 
\bottomrule
\end{tabular}
}
\caption{Experiment results on sampled NitiBench-CCL and full NitiBench-Tax. In the LCLM setup, we used the Gemini without retriever section, where full legislation books were parsed as a context.}
\label{table: main_exp_sampled}
\end{table}



From Table~\ref{table: main_exp_sampled}, the LCLM-based system performs comparably to the parametric setting on NitiBench-Tax and to the Naive RAG system on NitiBench-CCL. 
%
This performance gap may stem from degradation when processing extremely long contexts (1.2 million tokens). 
%
Regardless of the exact cause, the results suggest that while an LCLM-based Thai legal QA system is feasible, its performance remains significantly behind RAG-based counterparts, highlighting areas for further improvement.

Apart from utilizing LCLM to process the legislations and respond directly to the queries, we also explored using it as a retriever. 
%
As stated in \S\ref{subsec: setup_rq3}, Gemini 1.5 Pro is provided with all 35 legislations and tasked to retrieve 20 relevant laws given a query. 
%
This experiment is also conducted on the same sample of NitiBench-CCL as the previous experiment and the full NitiBench-Tax. 
%
The results are shown in Table~\ref{table: retrieval_wcx_lclm} and~\ref{table: retrieval_tax_lclm}.

\begin{table}[!ht]
\centering
\small
\begin{tabular}{@{}clcc@{}}
\toprule
\textbf{Top-K} & \textbf{Model} & \textbf{HR/Recall@k} & \textbf{MRR@k} \\ \midrule
\multirow{9}{*}{k=1} 
  & BM25                   & .480           & .480 \\
  & JINA V2                & .698           & .698 \\
  & JINA V3                & .601           & .601 \\
  & NV-Embed V1            & .496           & .496 \\
  & BGE-M3                 & .708           & .708 \\
  & Human-Finetuned BGE-M3 & \textbf{.757}  & \textbf{.757} \\
  & Auto-Finetuned BGE-M3  & \underline{.741} & \underline{.741} \\
  & Cohere                 & .707           & .707 \\
  & LCLM-as-a-retriever (Gemini)                   & .590           & .590 \\ \midrule
\multirow{9}{*}{k=5} 
  & BM25                   & .663           & .549 \\
  & JINA V2                & .858           & .761 \\
  & JINA V3                & .828           & .693 \\
  & NV-Embed V1            & .711           & .585 \\
  & BGE-M3                 & \underline{.888} & .779 \\
  & Human-Finetuned BGE-M3 & \textbf{.909}  & \textbf{.819} \\
  & Auto-Finetuned BGE-M3  & \textbf{.909}  & \underline{.807} \\
  & Cohere                 & .867           & .772 \\
  & LCLM-as-a-retriever (Gemini)                   & .776           & .667 \\ \midrule
\multirow{9}{*}{k=10} 
  & BM25                   & .733           & .559 \\
  & JINA V2                & .891           & .766 \\
  & JINA V3                & .878           & .700 \\
  & NV-Embed V1            & .794           & .596 \\
  & BGE-M3                 & .926           & .784 \\
  & Human-Finetuned BGE-M3 & \textbf{.945}  & \textbf{.824} \\
  & Auto-Finetuned BGE-M3  & \underline{.941} & \underline{.812} \\
  & Cohere                 & .913           & .778 \\
  & LCLM-as-a-retriever (Gemini)                   & .807           & .671 \\ \bottomrule
\end{tabular}
\caption{Retrieval Evaluation Result on a 20\% subset of NitiBench-CCL with hierarchy-aware chunking with Long-Context Retriever. Since the test split of NitiBench-CCL is single labeled, duplicated metrics (HR/Recall and MRR) have been collapsed.}
\label{table: retrieval_wcx_lclm}
\end{table}



\begin{table}[!ht]
\centering
\small
\begin{tabular}{@{}clccccc@{}}
\toprule
\textbf{Top-K} & \textbf{Model}                  & \textbf{HR@k}          & \textbf{Multi HR@k}    & \textbf{Recall@k}      & \textbf{MRR@k}         & \textbf{Multi MRR@k}   \\ \midrule
k=1   & BM25                   & .220          & .080          & .118          & .220          & .118          \\
      & JINA V2                & .140          & .040          & .068          & .140          & .068          \\
      & JINA V3                & .400          & .100          & .203          & .400          & .203          \\
      & NV-Embed V1            & .100          & .020          & .035          & .100          & .035          \\
      & BGE-M3                 & \underline{.500}    & \underline{.140}    & \underline{.269}    & \underline{.500}    & \underline{.269}    \\
      & Human-Finetuned BGE-M3 & .480          & \underline{.140}    & .255          & .480          & .255          \\
      & Auto-Finetuned BGE-M3  & \textbf{.520} & \textbf{.160} & \textbf{.281} & \textbf{.520} & \textbf{.281} \\
      & Cohere                 & .340          & .100          & .179          & .340          & .179          \\
      & LCLM                   & .480          & .120          & .227          & .480          & .227          \\ \midrule
k=5   & BM25                   & .480          & .120          & .254          & .318          & .171          \\
      & JINA V2                & .200          & .080          & .114          & .165          & .085          \\
      & JINA V3                & .720          & \underline{.260}    & \underline{.448}    & .508          & .297          \\
      & NV-Embed V1            & .200          & .020          & .081          & .126          & .050          \\
      & BGE-M3                 & .720          & .240          & .435          & \underline{.580}    & \underline{.337}    \\
      & Human-Finetuned BGE-M3 & \underline{.740}    & .220          & .411          & .565          & .320          \\
      & Auto-Finetuned BGE-M3  & .700          & .200          & .382          & \textbf{.587} & \underline{.329}    \\
      & Cohere                 & .620          & .200          & .363          & .447          & .256          \\
      & LCLM-as-a-retriever (Gemini)                   & \textbf{.760} & \textbf{.320} & \textbf{.515} & \textbf{.587} & \textbf{.370} \\ \midrule
k=10  & BM25                   & .540          & .160          & .320          & .327          & .183          \\
      & JINA V2                & .240          & .100          & .147          & .171          & .091          \\
      & JINA V3                & \textbf{.840} & \underline{.340}    & .549          & .524          & .311          \\
      & NV-Embed V1            & .220          & .040          & .097          & .128          & .052          \\
      & BGE-M3                 & \underline{.820}    & \textbf{.360} & \underline{.555}    & \underline{.593}    & \underline{.354}    \\
      & Human-Finetuned BGE-M3 & .800          & .280          & .499          & .574          & .333          \\
      & Auto-Finetuned BGE-M3  & .780          & .260          & .483          & \textbf{.600} & \underline{.345}    \\
      & Cohere                 & .680          & .200          & .414          & .454          & .263          \\
      & LCLM-as-a-retriever (Gemini)                   & .780          & \textbf{.360} & \textbf{.566} & .590          & \textbf{.379} \\ \bottomrule
\end{tabular}
\caption{Retrieval Evaluation Result on NitiBench-Tax with hierarchy-aware chunking. This split contains multiple positives per question.}
\label{table: retrieval_tax_lclm}
\end{table}


The results indicate that the LCLM retriever performs comparably to embedding-based retrievers on NitiBench-Tax, likely due to its superior reasoning capabilities. 
%
However, a noticeable performance gap exists when compared to the best retriever on NitiBench-CCL.
%
Additionally, increasing the number of retrieved documents for the LCLM yields minimal performance improvements relative to other models. 
%
We hypothesize that this limited gain is a result of LLMs' next-token prediction mechanism, which may hinder their ability to effectively retrieve and output relevant laws when those laws are distant from the query context or when the model attempts to generate relevant laws ranked lower in the retrieval order.











\section{Discussion}
\label{sec:discussion}

In this section, we first summarize the conclusion and share some key observations. Then, we reflect on the usability of our method and propose potential applications. In the end, we discuss the limitations and future work.

\subsection{Effectiveness of \name{}}
\label{sec:discuss_effectiveness}
Firstly, based on the results from Section~\ref{sec:experiment}, we can draw the following conclusions:
\begin{itemize}
    \item It is efficient to detect unknown words by combining linguistic characteristics provided by the pre-trained language model (PLM) and gaze trajectory.
    \item The prediction is mainly based on the linguistic features from the textual context captured by PLM.
    \item Gaze locates the region of interest in a timely manner, which is necessary for real-time applications. Gaze also helps improve the model performance, but its contribution is limited compared to PLM.
\end{itemize}

Additionally, it is interesting that while we typically assume that the gaze modality should contribute significantly to the task of unknown word detection, the experimental results show that the contribution of gaze to the model’s improvement is small with the existence of PLM. Based on the previous analysis of line spacing and eye tracker accuracy, a possible reason for this is that under normal reading conditions (single-line spacing, line height 3-5 mm), the eye tracker’s accuracy is insufficient to precisely detect which line the gaze belongs to, thus failing to accurately locate the gaze on the words. Furthermore, changes in user posture during long reading sessions further reduce the accuracy of the eye tracker. In our system, PLM compensates for this issue by providing linguistic information based on the text.

From another perspective, the low contribution of gaze is not necessarily a disadvantage. Our method’s reduced reliance on gaze makes it more tolerant of noise. The model’s good performance on data collected by webcams further supports this conclusion. The reduced dependency on gaze data allows our model to be applied on more affordable and accessible devices, such as webcams.

\subsection{Usability of \name{}}
\label{sec:discuss_usability}
The results from the user evaluation (Section~\ref{sec:user_evaluation}) show that our reading assistance prototype helps users read more fluently and they are more willing to use it compared to traditional click-to-translate methods. In addition to providing real-time translation and explanations during reading, our system can also benefit ESL for long-term learning. For example, based on the unknown word detected by our system, we can generate a vocabulary list for memorizing and offer memory curve tracking. Furthermore, these unknown words can also be used to generate personalized summaries and notes.

The potential issue of generalizability across users, texts and devices can be addressed through fine-tuning and reinforcement learning methods. During the initial phases of usage, the system collects both gaze and text data for fine-tuning and lets users provide feedback on the model's predictions. This allows the model to continuously learn the user's unique gaze patterns and infer their vocabulary proficiency and domain expertise from textual content, thereby improving prediction accuracy.

\subsection{Limitation and Future Works}
\label{sec:discuss_limitation}
The quality of gaze data hinders the improvement model performance. The accuracy of the eye tracker is not enough for word-level detection. Common formatting, such as single-line spacing and 10-point font, results in a line height of approximately 3-5 mm when viewed using the PDF viewer with a sidebar on a 14-inch laptop. This requires an accuracy of about $0.3-0.6^\circ$ at a reading distance of 50-60 cm. However, most eye trackers have a gaze accuracy ranging from $0.2-1.1^\circ$~\cite{gaze_survey_2024}. Combined with additional errors caused by head and upper body movements, this level of accuracy is insufficient for real-world reading scenarios. During data collection and evaluation, some participants reported that even after calibration, the error could span 1-3 lines. This makes it difficult to determine the specific word the user is focusing on based solely on gaze coordinates, explaining why gaze-based baselines performed poorly on our data.

\change{The inaccuracy of the gaze data could also lead to the inaccuracy of data labeling. To mitigate the impact of mouse clicks on gaze behavior, we asked users to label unknown words during their second pass. However, this widely adopted labeling method inherently requires "guessing" which words correspond to a given gaze trajectory. Previous works mapped each gaze coordinate directly to a specific word to establish word-gaze pairs. This method is infeasible for text with normal line spacing, so we establish gaze-word pairs by defining a bounding box based on a segment of gaze to identify the corresponding words instead. While this approach improves robustness, it may also introduce mismatches between gaze and words and thus introduce noise to the dataset. To further improve model performance, more precise labeling methods are needed.}

Additionally, reading time can be longer than several minutes in daily scenarios, so gaze drift can significantly affect data quality. In our experiments, we observed that it is difficult for participants to maintain a fixed posture after calibration, though we required them to do so. The posture shift further increases errors. Therefore, in practical applications, real-time calibration of gaze data based on user posture is crucial to ensure data quality. If the existing eye-tracking technology can combined with user posture detection~\cite{faceori}, it is possible to reduce the impact of user posture on gaze data, thereby improving the quality of gaze data.




\bibliographystyle{ACM-Reference-Format}
\bibliography{reference.bib}

\newpage
\appendix
\onecolumn

\part{}
\section*{\centering \LARGE{Appendix}}
\mtcsettitle{parttoc}{Contents}
\parttoc

\clearpage

\section{Related Work}
\label{sec:relatedwork}
% \paragraph{Tool Usage and Toolchain Management} Research in this area focuses on how intelligent agents design and optimize tool networks to effectively execute complex tasks, particularly by dynamically generating, selecting, and combining tools based on task requirements.This includes methods for automated tool generation and optimization, emphasizing systems that can adaptively choose and adjust tool combinations according to different task needs.

% \paragraph{Multi-Agent Systems and Collaboration} Research in Multi-Agent Systems has explored how multiple intelligent agents can collaboratively solve complex tasks in dynamic environments. One significant contribution is the development of decentralized algorithms that allow agents to autonomously form beneficial collaborations and adapt to changing tasks without the need for a central server (DeLAMA) ~\citep{tang2024decentralizedlifelongadaptivemultiagentcollaborative}. Another key area of study focuses on collaboration among heterogeneous agents, where different agents with varied capabilities work together on complex tasks, such as cleaning large spaces, using hierarchical decision models to allocate sub-tasks effectively~\citep{liu2023heterogeneousembodiedmultiagentcollaboration}. Additionally, collaborative learning approaches like Collaborative Q-learning (CollaQ) enhance agent teamwork by decomposing the Q-function and introducing reward attribution techniques to improve performance in multi-agent environments, such as the StarCraft challenge ~\citep{zhang2020multiagentcollaborationrewardattribution}. Finally, research has also examined how multi-agent collaboration can enhance the performance of large language models (LLMs) in tasks like simulations and software development, highlighting the potential of intelligent agent collaboration to improve task outcomes~\citep{talebirad2023multiagentcollaborationharnessingpower}.

\paragraph{Code Generation and Task Solving with LLMs} Large Language Models (LLMs) have demonstrated remarkable potential in generating code to solve complex tasks. Prior studies highlight their effectiveness in mathematical computation ~\citep{zhou2023solving, wang2023mathcoder, gou2023tora}, tabular reasoning ~\citep{chen2022program, lyu2023faithful, lu2024chameleon}, and visual understanding ~\citep{suris2023vipergpt, choudhury2023zero, gupta2023visual}. Frameworks such as AutoGen ~\citep{wu2023autogen} and CodeActAgent~\citep{wang2024executable} extend this capability to agent-based tasks by interpreting executable code as actions. These models dynamically invoke basic tools based on environmental feedback, significantly expanding their utility. Despite their successes, these approaches often treat program generation processes independently, failing to model shared task features and limiting the reusability of functional modules across tasks.

\paragraph{Reusable Tool Creation and Abstraction} To address the limitations of single-use program generation, recent efforts have focused on creating reusable tools. CREATOR ~\citep{qian2023creator} separates the processes of planning (tool creation) and execution, while LATM ~\citep{cai2023large} and CRAFT ~\citep{yuan2023craft} pre-build tools using training and validation sets for task solving. However, these methods often generate a large number of tools, presenting challenges for their efficient reuse. Furthermore, while abstraction-based approaches like REGAL ~\citep{stengel2024regal} focus on extracting reusable tools from primitive programs, they primarily construct simple tools with limited functional complexity. Similarly, Trove ~\citep{wang2024trove} adopts a training-free approach by dynamically composing high-level tools during testing, but its reliance on self-consistency can lead to hallucinated knowledge, reducing accuracy in complex tasks.

\paragraph{Tool Selection for Complex Task Solving} Currently, research on tool selection and retrieval methods primarily focuses on selecting appropriate tools through retrieval mechanisms and LLM-based approaches. ToolRerank ~\citep{zheng2024toolrerank} uses adaptive truncation and hierarchy-aware reranking to improve retrieval results, while Re-Invoke ~\citep{chen2024reinvoketool} introduces an unsupervised framework with synthetic queries and multi-view ranking, enhancing both single-tool and multi-tool retrieval. COLT ~\citep{Qu_2024COLT} combines semantic matching with graph-based collaborative learning to capture relationships among tools, outperforming larger models in some cases. AvaTaR~\citep{wu2024avataroptimizingllmagents} automates the optimization of LLM prompts for better tool utilization, and DRAFT~\citep{qu2024DAFT} refines tool documentation through iterative feedback and exploration, helping LLMs better understand external tools. Despite progress, existing methods generally overlook cost-effectiveness and scalability in tool selection, and often struggle to efficiently adapt to new tools and task requirements in dynamic environments, leading to performance and efficiency bottlenecks. In contrast, our approach dynamically prioritizes tools by combining their relevance and structural importance, ensuring computational efficiency and scalability, thus enabling more effective solutions for complex tasks.
\section{Experimental Details}
\label{app:apexp}
\subsection{Open-ended Task}
\label{subsec:open}
\paragraph{Benchmark} We employed the benchmark proposed by Voyager~\citep{wang2023voyager}, using Minecraft as the experimental platform. Minecraft provides a sandbox environment where players gather resources and craft tools to achieve various goals. The simulation is built on MineDojo~\citep{fan2022minedojo} and uses Mineflayer~\citep{PrismarineJS2013} JavaScript APIs for motor control. 

\paragraph{Baselines}
We conducted a comprehensive comparison with four baselines. Except for Voyager, these methods were originally designed for NLP tasks without embodiment. Therefore, we had to reinterpret and adapt them for execution within the MineDojo environment, ensuring compatibility with the specific requirements of our experimental setup.
\begin{itemize}
    \item \textbf{ReAct:} ReAct~\citep{yao2022react} uses chain-of-thought prompting [46] by generating both reasoning traces and action
plans with LLMs. We provide it with our environment feedback and the agent states as observations.
    \item \textbf{Reflexion:} Reflexion~\citep{shinn2023reflexion} is built on top of ReAct~\citep{yao2022react}with self-reflection to infer more intuitive future actions.
    \item \textbf{AutoGPT:} AutoGPT~\citep{richardssignificant} is a popular software tool that automates NLP tasks by decomposing a high-level
goal into multiple subgoals and executing them in a ReAct-style loop. We re-implement AutoGPT by using GPT-4O to do task decomposition and provide it with the agent states, environment feedback,
and execution errors as observations for subgoal execution
We provide it with execution errors and our self-verification module.
    \item \textbf{Voyager:} Voyager~\citep{wang2023voyager} is a system that integrates an automated curriculum, a scalable skill library, and an iterative prompting framework based on environmental feedback to explore, store, and accumulate skill library within the Minecraft environment.
\end{itemize}


\paragraph{Metric}
The evaluation metric is based on the number of iterations required to progress through tool upgrades, from wooden to stone, iron, and finally diamond tools. Each execution of code is considered one iteration.

\paragraph{Model}
We leverage GPT-4o for text completion, along with the text-embedding-ada-002 API for text embedding. We set all temperatures to
0 except for the automatic curriculum, which uses temperature = 0.1 to encourage task diversity. 

\paragraph{Setting}
We set the maximum number of iterations to 160. For both \ours\ and Voyager, all agents are controlled by GPT-4o, with the number of tools retrieved per iteration set to 5. To ensure a fairer comparison, we removed the Tool Requirement Stage and bug-free checks in \ours\ , and allowed a maximum of 3 self-checks per iteration.

\paragraph{Item Types and Levels}
In the Minecraft task, there are different types and levels of items. Diamond tools are the highest level, and rare items such as golden apples also exist. High-level tools require some lower-level items to craft. Table \ref{tab:toollist} lists the key items in the Minecraft task.
\begingroup
\begin{table}[H]
\caption{List of item types and levels in the Minecraft task.}
\label{tab:toollist}
\vskip -0.1in
\setlength{\tabcolsep}{10pt} % 调整列间距
\begin{center}
\begin{small}
\begin{sc}
\begin{tabular}{l|c|c}
\toprule
\textnormal{\textbf{Category}} & \textnormal{\textbf{level}} & \textnormal{\textbf{Items}} \\
\midrule         
\midrule
\multirow{4}{*}{\multirow{3}{*}{\normalfont Tools}} 
              & \normalfont Wooden Tools & \normalfont Wooden\_Shovel,Wooden\_Pickaxe,Wooden\_Axe,Wooden\_Hoe,Wooden\_Sword \\
              \cmidrule{2-3}
              & \normalfont Stone Tools &\normalfont stone\_pickaxe, stone\_shovel,Stone\_Axe,Stone\_Hoe,Stone\_Sword   \\
              \cmidrule{2-3}
              & \normalfont Iron Tools &\normalfont iron\_pickaxe, iron\_axe, iron\_sword, iron\_shovel, iron\_hoe    \\
              \cmidrule{2-3}
              & \normalfont Diamond Tools &\normalfont diamond\_pickaxe, diamond\_sword, diamond\_axe, diamond\_shovel    \\
             
\midrule
\multirow{2}{*}{\multirow{1}{*}{\normalfont  Armor}} 
              & \normalfont Iron Armor &\normalfont iron\_chestplate, iron\_helmet, iron\_leggings  \\
              \cmidrule{2-3}
              & \normalfont Diamond Armor &\normalfont diamond\_chestplate, diamond\_helmet, diamond\_leggings, diamond\_boots     \\

\midrule
\multirow{3}{*}{\multirow{2}{*}{\normalfont  Food}} 
              & \normalfont Raw Food &\normalfont chicken, mutton, porkchop, rabbit, raw\_rabbit, spider\_eye, bone  \\
              \cmidrule{2-3}
              & \normalfont Cooked Food &\normalfont cooked\_beef, cooked\_chicken, cooked\_mutton, cooked\_porkchop, cooked rabbit  \\
              \cmidrule{2-3}
              & \normalfont Advanced Food &\normalfont golden apple    \\

\bottomrule
\end{tabular}
\end{sc}
\end{small}
\end{center}
\vskip -0.1in
\end{table}
\endgroup


\subsection{Agent Task}
\label{subsec:agent}
\paragraph{Benchmark}
We conducted experiments on two types of agent tasks, demonstrating {\ours}'s capabilities in both game-related and data science tasks.
\begin{itemize}
     \item \textbf{TextCraft:} We evaluate {\ours} on the TextCraft dataset~\citep{futuyma1988evolution}, which challenges agents to craft Minecraft items in a text-only environment~\citep{cote2019textworld}. Each task instance provides a goal and a sequence of crafting commands, which include distractors. We use depth-2 splits for testing and reserve a subset of depth-1 recipes for development, resulting in a 99/77 train/test split.
    \item \textbf{InfiAgent-DABench:} We also test {\ours} on the InfiAgent-DABench benchmark~\citep{hu2024infiagent}, which evaluates LLM-based agents on end-to-end data analysis tasks. This benchmark consists of 257 questions across 52 CSV files, with each question corresponding to a unique CSV file. Agents are required to generate code to analyze data and produce the specified output format. We randomly selected 20 CSV files and their associated question-answer pairs as training data, resulting in a train/test split of 98/159 instances.
\end{itemize}

\paragraph{Baselines}
We compare \ours\ with three methods described below.
\begin{itemize}
     \item \textbf{ReAct:} In this setting, we employ the executor to interact iteratively with the environment, adopting the think-act-observe prompting style from ReAct~\citep{yao2022react}.
     \item \textbf{Plan-Execution:} In contrast, the Plan-and-Execute approach~\citep{shridhar2023art, yang2023intercode} generates a plan upfront and assigns each sub-task to the executor. To ensure each step is executable without further decomposition, we provide new prompts with more detailed planning instructions.
    \item \textbf{Reflexion:} In the Reflection setting~\citep{shinn2023reflexion}, the agent engages in self-reflection after each step, drawing on environmental feedback and exploration history. 
\end{itemize}

\paragraph{Metric} 
The most practically important aspect of the solutions is correctness. For Textcraft, we verify whether the agent’s inventory contains the goal item. For DABench, we check if the agent’s final answer matches the ground truth.

\paragraph{Model}
During training, we use GPT-4o to construct the tool library with a temperature setting of 0. In the testing phase, we conduct a comprehensive comparison of various open-source and closed-source models. The open-source models include \textit{Qwen2.5-7B-Instruct, Qwen-Coder-7B-Instruct, Qwen2.5-14B-Instruct, Deepseeker-Coder-6.7B-Instruct, and Deepseeker-Coder-33B-Instruct}, while the closed-source models primarily include \textit{gpt-3.5-turbo-1106} and \textit{Claude-3-haiku}. During testing, the temperature is set to 0.3, and each experiment is repeated 3 times, with the average result reported.

\paragraph{Setting} 
For ReAct, Reflexion, and \ours\ , the maximum number of steps is set to 20. For Plan-Execution, the maximum number of steps for each sub-task is set to 8. In \ours\ , the number of tools retrieved during testing is limited to 3.



\subsection{Single-turn Code Task}
\label{subsec:code}
\paragraph{Benchmark}
To further explore {\ours}'s potential, we evaluated it on single-turn code generation tasks spanning mathematical reasoning, date comprehension, and tabular reasoning:
 \begin{itemize}
     \item \textbf{MATH:} We used a subset of the MATH dataset~\citep{hendrycks2021measuring}, focusing on 405 level-4 and level-5 algebra problems (MATH contains 5 levels of difficulty) that require textual understanding and advanced reasoning. We randomly selected 200 examples from the test set of the MATH dataset to construct the tool network, resulting in a train/test split of 200/405.
     \item \textbf{Date:} We use the date understanding task from BigBenchHard~\citep{srivastava2022beyond}, which consists of short word problems requiring date understanding. We follow the data splits provided by REGAL\citep{stengel2024regal}, resulting in a train/test split of 66/180.
     \item \textbf{TabMWP:} We further extend our general experiments on MATH by testing on TabMWP~\citep{grand2023learning}, a tabular reasoning dataset consisting of math word problems about tabular data. Based on the CRAFT~\citep{yuan2023craft} splits, we selected 470 problems from levels 7 and 8 (TabMWP contains 8 levels) from the 1,000 test examples. Additionally, we randomly selected 200 examples from the TabMWP training set, resulting in a train/test split of 200/470.
\end{itemize}

\paragraph{Baselines}
For these tasks, we use Programs of Thoughts (PoT)~\citep{chen2022program} and other existing tool-making methods as baselines for comparison.

\begin{itemize}
    \item \textbf{PoT:} The LLM utilizes a program to reason through the problem step by step~\citep{chen2022program}.
   \item \textbf{LATM:} LATM~\citep{cai2023large} samples 3 examples from the training set and create a tool for the task, which is further verified by 3 samples from the validation set. The created tool is then applied to all test cases.
    \item \textbf{CREATOR:} CREATOR~\citep{qian2023creator} disentangle planning (tool making) from execution, enabling Large Language Models (LLMs) to autonomously create a specific tool for each test case during inference.
     \item \textbf{CRAFT:} CRAFT~\citep{yuan2023craft} constructs task-specific toolsets by generating a tool for each training example. During testing, it utilizes a tool retrieval module and a reasoning process akin to CREATOR, generating a function first and then producing the corresponding invocation code. 
      % \item \textbf{Trove:} Trove~\citep{wang2024trove} introduces a training-free method based on self-consistency, where LMs interact with the toolbox through three modes—IMPORT, SKIP, and CREATE. Each mode is executed K times, and from the 3K responses, the function from the most consistent and optimal response is added to the toolbox.
      \item \textbf{REGAL:} During training, REGAL~\citep{stengel2024regal} refines primitive programs by extracting functions. In the testing phase, it retrieves both tools and refactored programs—comprising original and refactored versions—to generate a program that effectively solves the task. 
\end{itemize}
\paragraph{Metric}
We use correctness as the evaluation metric, measuring whether the execution outcome of the solution program exactly matches the ground-truth answer(s).
\paragraph{Model}
The models for the single-turn code generation task are the same as those used for the Agent Task, as presented in Section \ref{subsec:agent}.
\paragraph{Setting}
To ensure a fair comparison, we make slight adjustments to each method. For all methods, we allow up to 3 times for format checking and correction, as small models may not always follow the required output format. For PoT, we use 6 fixed examples of basic tool usage as few-shot. CREATOR employs the rectifying process, while for CRAFT, we use the same training set as our method and construct the tool library with GPT-4o, retrieving 3 tools during testing. For Regal, we use PoT along with GPT-4o to obtain ground-truth code, select the correct program, and have GPT-4o reconstruct it. To maintain fairness in tool generation quality, we standardize the few-shot examples of basic tools and retrieve 3 tools, along with 3 usage examples from the current tool library, avoiding errors from pruned tools. For our method, we train with GPT-4o, retrieving 3 tools and their corresponding usage examples during testing, while fixing the basic tool few-shot examples to 3, ensuring consistency with PoT’s total few-shot count.
\section{More Results}
\label{app:apresults}
\subsection{Open-ended Task}
\label{subsec:open-results}
\paragraph{More complex tools} 
Our hierarchical graph architecture offers significant advantages in handling complex tasks and large-scale systems. As shown in Figure \ref{fig:toolnet1}, Trial 1 starts with five nodes occupying three layers, and evolves into a five-layer network, with an increasing number of inter-tool calls. As shown in Figure \ref{fig:toolnet2}, Trial 2 starts with four nodes occupying four layers, and evolves into a five-layer network with more inter-tool calls. As shown in Figure \ref{fig:toolnet3}, Trial 3 starts with four nodes occupying three layers, and evolves into a six-layer network structure, with a growing number of inter-tool calls. Our tool graph becomes progressively more complex, flexibly expanding and optimizing its components. These results demonstrate that our method can generate tools that call each other, and combine them into more complex tools. This not only enhances scalability but also facilitates the creation of more sophisticated tools, enabling the solution of increasingly complex problems.


\paragraph{More types of inventory} Our method is able to generate more inventory types than Voyager. As shown in Table \ref{tab:Number}, we can see that {\ours} produces more inventory types in all three trials compared to Voyager.

The inventory collected by {\ours} in each trial is

\begin{itemize}
    \item \textbf{Trial 1:}  \textit{oak\_log, birch\_log, oak\_planks, birch\_planks, crafting\_table, stick, wooden\_pickaxe, dirt, cobblestone, coal, stone\_pickaxe, raw\_copper, furnace, copper\_ingot, andesite, raw\_iron, granite, iron\_ingot, iron\_pickaxe, shield, diorite, raw\_gold, lapis\_lazuli, redstone, diamond, diamond\_pickaxe, bucket, gold\_ingot, iron\_chestplate, arrow, iron\_sword, iron\_helmet, diamond\_sword, diamond\_helmet, lightning\_rod, chest, iron\_axe, iron\_leggings, sandstone, dandelion, spider\_eye, string, iron\_shovel, copper\_block, iron\_door, iron\_hoe, kelp, bow, dried\_kelp, torch, cooked\_beef, gray\_wool, cobbled\_deepslate, tuff, diamond\_leggings, bone, diamond\_chestplate, chicken, white\_banner, cooked\_chicken, egg, feather, oak\_sapling, apple, acacia\_log, golden\_apple, diamond\_axe}

    \item \textbf{Trial 2:}  \textit{oak\_sapling, oak\_log, stick, oak\_planks, crafting\_table, wooden\_pickaxe, dirt, cobblestone, stone\_pickaxe, diorite, raw\_iron, coal, lapis\_lazuli, gravel, furnace, iron\_ingot, raw\_copper, sandstone, granite, iron\_pickaxe, andesite, raw\_gold, gold\_ingot, diamond, diamond\_pickaxe, redstone, cobbled\_deepslate, bucket, iron\_sword, arrow, bow, bone, birch\_log, chest, amethyst\_block, calcite, smooth\_basalt, iron\_chestplate, diamond\_sword, diamond\_helmet, iron\_leggings, diamond\_boots, water\_bucket, string, orange\_tulip, mutton, white\_wool, porkchop, dandelion, cooked\_porkchop, cooked\_mutton}

    \item \textbf{Trial 3:}  \textit{jungle\_log, stick, oak\_sapling, jungle\_planks, crafting\_table, dirt, wooden\_pickaxe, cobblestone, stone\_pickaxe, raw\_iron, raw\_copper, furnace, iron\_ingot, iron\_pickaxe, coal, diorite, lapis\_lazuli, andesite, moss\_block, clay\_ball, redstone, raw\_gold, cobbled\_deepslate, granite, diamond, diamond\_pickaxe, copper\_ingot, gunpowder, bucket, gravel, gold\_ingot, oak\_log, iron\_sword, iron\_chestplate, chest, diamond\_sword, spruce\_sapling, rotten\_flesh, bone, rose\_bush, water\_bucket, string, oak\_planks, grass\_block, diamond\_helmet, iron\_leggings, emerald, snowball, rabbit\_hide, rabbit, spruce\_log, cooked\_rabbit, diamond\_boots}
\end{itemize}


The inventory collected by Voyager in each trial is
\begin{itemize}
    \item \textbf{Trial 1:}  \textit{oak\_log, birch\_log, oak\_sapling, birch\_sapling, oak\_planks, stick, crafting\_table, wooden\_pickaxe, dirt, cobblestone, stone\_pickaxe, raw\_copper, white\_tulip, coal, furnace, copper\_ingot, granite, raw\_iron, iron\_ingot, lightning\_rod, iron\_pickaxe, pink\_tulip, orange\_tulip, sandstone, shears, shield, diorite, cobbled\_deepslate, iron\_block, chest, tuff, lapis\_lazuli, redstone, diamond, raw\_gold, gold\_ingot, diamond\_pickaxe, diamond\_helmet, diamond\_sword, sand, andesite, arrow, bone, iron\_chestplate, beef, leather, oak\_leaves, porkchop, cooked\_beef, leather\_leggings}

    \item \textbf{Trial 2:}  \textit{dirt, oak\_log, oak\_planks, crafting\_table, stick, oak\_sapling, wooden\_pickaxe, cobblestone, coal, stone\_pickaxe, raw\_iron, granite, lapis\_lazuli, raw\_copper, furnace, iron\_ingot, copper\_ingot, iron\_helmet, iron\_pickaxe, diorite, andesite, salmon, ink\_sac, iron\_chestplate, lightning\_rod, cooked\_salmon, stone, stonecutter, rotten\_flesh, gravel, flint, chest, iron\_leggings, copper\_block, cobbled\_deepslate, tuff, diamond, diamond\_pickaxe, raw\_gold, gold\_ingot, redstone, diamond\_sword, egg, diamond\_boots, diamond\_axe}

    \item \textbf{Trial 3:}  \textit{jungle\_log, jungle\_planks, oak\_sapling, oak\_log, crafting\_table, stick, wooden\_pickaxe, dirt, cobblestone, coal, stone\_pickaxe, raw\_copper, furnace, copper\_ingot, magma\_block, lightning\_rod, stone\_axe, jungle\_boat, kelp, sand, sandstone, glass, raw\_iron, granite, lapis\_lazuli, diorite, iron\_ingot, bucket, iron\_pickaxe, chest, andesite, redstone, dried\_kelp, iron\_chestplate, wooden\_sword, shield, iron\_sword}
\end{itemize}

\vskip -0.2in
\begin{table}[H]
\caption{Number of different inventory types produced by each trial}
\label{tab:Number}
% \vskip 0.1in
\setlength{\tabcolsep}{12pt} % 调整列间距
\renewcommand{\arraystretch}{1.0} % 调整行间距
\begin{center}
% \resizebox{\textwidth}{!}{ % 自动调整表格宽度以适应页面
\begin{small}
\begin{sc}
\begin{tabular}{lccc} % 确保列数与标题一致
\toprule
\textnormal{\textbf{Method}} & \textnormal{\textbf{Trial 1}} & \textnormal{\textbf{Trial 2}} & \textnormal{\textbf{Trial 3}}  \\
\midrule
\normalfont Voyager     & 50  & 45  & 37    \\
\normalfont AETG(Ours)  & 67  & 51  & 53    \\
\bottomrule
\end{tabular}
\end{sc}
\end{small}
% }
\end{center}
\vskip -0.1in
\end{table}


\paragraph{Longer exploration path} To better demonstrate the exploration capabilities of the agent, we compared the exploration trajectories and their lengths. As shown in Figure \ref{fig:linermap}, our agent exhibits longer and more persistent exploration capabilities than Voyager. In Table \ref{tab:length}, the trajectory lengths of our agent are consistently much greater than those of Voyager. {\ours}is able to traverse across multiple terrains, with an average distance 2.66 times longer than Voyager. Additionally, {\ours} can explore across different continental plates, while Voyager remains confined to a single plate, highlighting the exceptional exploration capability of {\ours}.

% \vskip -0.2in
\begin{table}[H]
\caption{Exploration trajectory length in each trial, where \textit{Performance Gain} = $\textit{ours}/\textit{voyager}$.}
\label{tab:length}
% \vskip 0.1in
\setlength{\tabcolsep}{12pt} % 调整列间距
% \renewcommand{\arraystretch}{1.0} % 调整行间距
\begin{center}
% \resizebox{\textwidth}{!}{ % 自动调整表格宽度以适应页面
\begin{small}
\begin{sc}
\begin{tabular}{lcccc} % 确保列数与标题一致
\toprule
\textnormal{\textbf{Method}} & \textnormal{\textbf{Trial 1}} & \textnormal{\textbf{Trial 2}} & \textnormal{\textbf{Trial 3}} & \textnormal{\textbf{\textit{Avg}}}\\
\midrule
\normalfont Voyager     & 1925.74  & 4102.99  & 902.13  & 2310.29   \\
\normalfont {\ours}(Ours)  & 5665.75  & 8908.57  & 3895.06 & 6156.46  \\
\midrule
\normalfont \textit{Performance Gain} & 2.94  & 2.17   & 4.32    & 2.66 \\
\bottomrule
\end{tabular}
\end{sc}
\end{small}
% }
\end{center}
\vskip -0.1in
\end{table}


\vskip -0.2in
\begin{figure}[H]
\vskip 0.2in
\begin{center}
\centerline{\includegraphics[width=1\linewidth]{trial-map.png}}
% \vskip -0.2in
\caption{Map coverage: Three bird’s eye views of Minecraft maps. The trajectories are plotted based on the position coordinates where each agent interacts.}
\label{fig:trialmap}
\end{center}
\vskip -0.3in
\end{figure}


\vskip -0.2in
\begin{figure}[H]
\vskip 0.2in
\begin{center}
\centerline{\includegraphics[width=1\linewidth]{liner-map.png}}
% \vskip -0.2in
\caption{Movement trajectory Map: Three bird’s eye views of Minecraft maps. The trajectories are plotted based on the position coordinates where each agent interacts.}
\label{fig:linermap}
\end{center}
\vskip -0.3in
\end{figure}



\paragraph{Efficient Zero-Shot Generalization to Unseen Tasks} Based on the results presented in Table \ref{tab:newtechtree} and Figure \ref{fig:diamon and compass}, we can clearly observe the significant advantages of {\ours} in the open-ended task. Table \ref{tab:newtechtree} shows the number of iterations required for different methods to complete various tasks (Gold Sword, Compass, Diamond Hoe, Lava Bucket), where fewer iterations indicate higher efficiency. Compared to Voyager and {\ours} (w/o toolnet), {\ours} consistently requires significantly fewer iterations across all tasks, demonstrating substantial improvements in efficiency. Notably, in the Gold Sword task, {\ours} (ours) completes the task in just 14.00±1.73 iterations, whereas Voyager requires 46.33±14.57 iterations, showcasing its superior performance.

Figure \ref{fig:diamon and compass} further visualizes the intermediate progress of different methods on the "Craft a Compass" and "Craft a Diamond Hoe" tasks. It is evident that {\ours} learns and masters the necessary skills for crafting items more quickly. As the number of prompting iterations increases, {\ours} reaches the task objectives significantly earlier than the other methods. Additionally, while {\ours}(w/o Tool Graph) performs better than Voyager, it still lags behind {\ours}, indicating that the ToolNet component plays a crucial role in enhancing the model's capability.

Overall, these experimental results demonstrate that {\ours} not only learns new skills and crafting techniques more efficiently but also that its key module, Tool Graph, is essential for overall performance improvement. This further validates the effectiveness of our approach in self-driven exploration and task generalization.


\begingroup
\begin{table}[H]
\caption{The mastery of the tech tree in the Open-ended Task. The number indicates the number of iterations. The fewer the iterations, the more efficient the method. "N/A" indicates that the number of iterations for obtaining the current type of tool is not available.}
\label{tab:newtechtree}
\vskip 0.1in
\setlength{\tabcolsep}{12pt} % 调整列间距
% \renewcommand{\arraystretch}{1.0} % 调整行间距
\begin{center}
% \resizebox{\textwidth}{!}{ % 自动调整表格宽度以适应页面
\begin{small}
\begin{sc}
\begin{tabular}{lccccc} % 确保列数与标题一致
\toprule
\textnormal{\textbf{Method}} & \textnormal{\textbf{Trial}} & \textnormal{\textbf{Gold Sword}} & \textnormal{\textbf{Compass}} & \textnormal{\textbf{Diamond Pickaxe}} & \textnormal{\textbf{Lava Bucket}} \\
\midrule
\multirow{4}{*}{\multirow{2}{*}{\normalfont Voyager}} 
              & \normalfont Trial 1 & 48 & 16 &  24 & N/A         \\
              & \normalfont Trial 2 & 31 & 17 &  25 & 39         \\
              & \normalfont Trial 3 & 60 & 20 & 18  & N/A         \\
              \cmidrule{2-6}
              & \textit{Average} & 46.33$\pm$14.57 & 17.67$\pm$2.08 & 22.33$\pm$3.79 & 39.00$\pm$0.00 \\
\midrule
\multirow{4}{*}{\multirow{2}{*}{\normalfont {\ours}\textit{\small(w/o toolnet)}}} 
               & \normalfont Trial 1 & 26 & 27 & 23  & N/A         \\
              & \normalfont Trial 2 & 18 & 22 & 18  & N/A        \\
              & \normalfont Trial 3 & 56 & 15 & 30  & N/A          \\
              \cmidrule{2-6}
              & \textit{Average} & 33.33$\pm$20.03 & 21.33$\pm$6.03 & 23.67$\pm$6.03 & N/A$\pm$N/A \\
\midrule
\multirow{4}{*}{\multirow{2}{*}{\normalfont {\ours}\textit{\small(ours)}}} 
              & \normalfont Trial 1 & 13 & 28 & 16  & 19       \\
              & \normalfont Trial 2 & 13 & 10 & 14  & 27       \\
              & \normalfont Trial 3 & 16 & 13  & 13  & 18      \\
              \cmidrule{2-6}
              & \textit{Average} & \textbf{14.00$\pm$1.73} & \textbf{17.00$\pm$9.64} & \textbf{14.33$\pm$1.53} & \textbf{21.33$\pm$4.93} \\
             

\bottomrule
\end{tabular}
\end{sc}
\end{small}
% }
\end{center}
\vskip -0.1in
\end{table}
\endgroup



\begin{figure}[H]
\vskip 0.2in
\begin{center}
\centerline{\includegraphics[width=1\linewidth]{compass_and_diamond.png}}
% \vskip -0.2in
\caption{Zero-shot generalization to unseen tasks. Here, we visualize the intermediate progress of each method on the tasks "Craft a Compass" and "Craft a Diamond Hoe."}
\label{fig:diamon and compass}
\end{center}
\vskip -0.3in
\end{figure}



\subsection{Agent Task}
\label{subsec:agent-results}

Figures \ref{fig:toolnet-dabench} and \ref{fig:toolnet-textcraft} present the tool network evolution diagrams of DA-Bench and TextCraft, which visually reflect the call relationships between different tool functions. In these diagrams, each node represents a specific tool function, edges indicate the call dependencies between tools, and the shading of the nodes reflects the frequency of tool calls—darker colors indicate higher call frequency. From Figure \ref{fig:toolnet-dabench}, it can be observed that in DA-Bench, the tool network expands progressively as the task advances, forming multiple core nodes with higher call frequencies. This suggests that certain key tools are frequently called during the task execution, playing a central role. Additionally, the tool call relationships exhibit a hierarchical and well-organized structure, reflecting DA-Bench's efficiency in tool dependency management.

In contrast, Figure \ref{fig:toolnet-textcraft} illustrates the tool network evolution of TextCraft, which also shows a similar expansion trend overall. However, compared to DA-Bench, the tool call frequency in TextCraft is more evenly distributed across multiple nodes, meaning that the system calls a wider variety of tools during task execution, rather than relying on a few core tools. This distribution pattern may suggest that TextCraft adopts a more diverse tool usage strategy in task execution.

A comparative analysis of the two figures reveals that, although both DA-Bench and TextCraft exhibit certain hierarchical and expansive characteristics in their tool call patterns, DA-Bench relies more heavily on a few core tools, whereas TextCraft displays a more dispersed tool call pattern. This contrast not only highlights the differences in tool usage between the two, but also emphasizes the importance and effectiveness of ToolNet.





\subsection{Single-turn Code Task}
\label{subsec:code-results}

As shown in the Figure\ref{fig:toolnet-math} \ref{fig:toolnet-tabmwp}, this illustrates the evolution of the tool graph for the Math and TabMWP tasks. It is evident that the tool graph gradually becomes more complex, creating multiple layers of tools, making the tool graph more intricate. Since the Date task can be solved with fewer tools, there is no evolution of the tool graph. However, the generated tools can still effectively solve the task, while there exists a multi-level calling relationship.


\section{More Ablations}
\label{app:apablation}
\subsection{Open-ended Task}
\label{subsec:open-ablation}

As shown in Figure \ref{fig:ablation}, AETG significantly outperforms methods that lack certain functional modules in discovering new Minecraft items and skills. It can be observed that the performance is worst when "w/o retrieval" is used, indicating that the absence of retrieval has the greatest impact on overall functionality and plays a crucial role, thereby validating the effectiveness of our retrieval method. The performance with "w/o duplication" is slightly better, indicating its importance is weaker than that of "w/o retrieval." The performance of "w/o check" and "w/o pruning" is better, but still far behind AETG, which further demonstrates the importance and effectiveness of each functional component.

\vskip -0.1in
\begin{figure}[H]
% \vskip 0.2in
\begin{center}
\centerline{\includegraphics[width=0.6\linewidth]{toolnumber-ablation.png}}
% \vskip -0.2in
\caption{Ablation study of the iterative prompting mechanism. AETN surpasses all other options, highlighting the essential significance of each functional module in the iterative prompting mechanism.}
\label{fig:ablation}
\end{center}
\vskip -0.3in
\end{figure}


\subsection{Closed-Ended Task}
\label{subsec:closed-ended}
For the Closed-Ended Task, we select Textcraft from the Agent Task and Date from the Single-turn Code Task to evaluate the effectiveness of several components in our method. The results are shown in the Table \ref{tab:closed-toolnumber}.

\begingroup
\begin{table}[H]
\caption{The number of tools in Close-Ended Task.}
\label{tab:closed-toolnumber}
\vskip -0.1in
\setlength{\tabcolsep}{10pt} % 调整列间距
\begin{center}
\begin{small}
\begin{sc}
\begin{tabular}{l|cc}
\toprule
\textnormal{\textbf{Method}} & \textnormal{\textbf{TextCraft}}  & \textnormal{\textbf{Date}} \\
\midrule         

\normalfont W/o Self-Check & 42 & 9 \\
\midrule  
\normalfont W/o Merging & 49 & 11\\
\midrule  
\normalfont W/o pruning & 46 & 9 \\
\midrule  
\normalfont GATE & 44 & 4 \\


\bottomrule
\end{tabular}
\end{sc}
\end{small}
\end{center}
\vskip -0.1in
\end{table}
\endgroup

\section{Tool Making}
\label{app:toolgarph}
\subsection{Basic Tools}
\label{subsec:basic-tools}
As shown in the Table \ref{tab:basictool} , the basic tools generated by each method are displayed.

\begingroup
\begin{table}[H]
\caption{Basic tools in various methods.}
\label{tab:basictool}
\vskip -0.1in
\setlength{\tabcolsep}{10pt} % 调整列间距
\begin{center}
\begin{small}
\begin{sc}
\begin{tabular}{l|p{12cm}}
\toprule
\textnormal{\textbf{Tasks}} & \textnormal{\textbf{Basic Tools}}  \\
\midrule         

\normalfont Other Tasks & \normalfont ToolRequest, NotebookBlock, Terminate, CreateTool, EditTool, Python, Feedback, SendAPI, Feedback, Retrieval \\
\midrule  
\normalfont Minecraft & \normalfont smeltItem, killMob, waitForMobRemoved, givePlacedItemBack, useChest, exploreUntil, craftItem, mineBlock, shoot, placeItem, craftHelper, smeltItem, mineflayer, killMob, useChest, exploreUntil, craftItem, mineBlock, placeItem \\

\bottomrule
\end{tabular}
\end{sc}
\end{small}
\end{center}
\vskip -0.1in
\end{table}
\endgroup


\subsection{Tool construction Lists}
\label{subsec:tool construction}

\paragraph{CREATOR:}
\begin{itemize}[noitemsep, topsep=0pt]
    \item \textbf{MATH:}  \textit{sum of areas, find largest won matches, find K, total distance after bounces, find common ratio sum, count lattice points with distance squared, find c for radius, find circle equation and constants, polynomial degree product, calculate cells, find fiftieth term, find non domain values, inverse function product, find m and n, sum of fractions from roots, find roots of quadratic, main, find coefficients, compute expression, prime factors, find x y, find second largest angle, find y coordinate, find constants, evaluate expression, find b for one solution, find c, find minimum value, find possible s, solve expression, find cone height, solve abc, find minimum expression, \dots, time to hit ground, sum of reciprocals of roots, solve x floor x product, sum of possible x, find constant a, sum of squares of solutions, find cost per extra hour, is triangular number, find smallest b greater than 2011, solve exponential equation, solve club suit equation, find degree of h, f, find vertical asymptotes, domain width, maximize revenue, future value, total savings, find min interest rate, equation, find integers, sum of x coordinates squared, find integer values of a, smallest c for real domain, smallest integer c, find m, required investment, simplify expression, g, distance between midpoints, compute x and power, greatest possible a, find continued fraction value, find a b, solve mnp, compute sum, sum of integers in range,
    }

    \item \textbf{Date:}  \textit{get us thanksgiving date, get date one week from first monday of 2019, calculate anniversary date, calculate yesterday from last day of january, calculate one week ago from first monday, get first monday of 2019, calculate yesterday, calculate yesterday from rescheduled meeting, calculate date a month ago from rescheduled meeting, calculate yesterday from first monday of 2019, get date 10 days before us thanksgiving, calculate one week ago from egg runout, calculate one week ago from end of first quarter, calculate date 24 hours later, calculate date a month ago, calculate date 24 hours after anniversary, calculate one week from today from rescheduled meeting, \dots, get tomorrow from us thanksgiving, calculate yesterday from day before yesterday, calculate yesterday from anniversary, calculate date 10 days ago, calculate one year ago from egg run out date, calculate tomorrow from yesterday, calculate one week from last day of january, calculate one week from anniversary, calculate yesterday from eggs run out, calculate tomorrow from today, calculate tomorrow from day before yesterday, calculate one week ago from today, calculate one week ago, calculate date one month ago from anniversary, calculate one year ago from given date, calculate one week from given date}

    \item \textbf{TabMWP:}  \textit{calculate total cost, smallest points, price difference, cost of river rafts, calculate median, calculate range, calculate total spent, rate of change, cost difference, cost for rides, rate of change vacation days, total participants, calculate mean glasses, find mode of states visited, rate of change straight A students, calculate median basketball hoops, count bins with toys in range, people with at least 3 trips, count teams with fewer than 80 swimmers, calculate median clubs, count exact pushups, children with less than 2 necklaces, people played exactly 3 times, count people with fewer than 80 pullups, range of states visited, find spent amount, \dots, calculate median miles, people with fewer than 3 seashells, calculate median glasses, cost to buy cockatiels, largest broken lights, calculate spent, calculate ice cream cost, range of soccer fields, patrons with at least 2 books, count bushes with 20 roses, total people played golf, range of articles, count shipments with exactly 60 broken plates, total cost for lip balms, rate of change scholarships, count teams with fewer than 50 members, count tests with 34 problems, find mode of soccer fields, rate of change hockey games, find lowest score, count pizzas with exactly 48 pepperoni, count people with at least 30 points, cost of wooden benches, rate of change students, patients with fewer than 2 trips, find mode, total cost for hazelnuts, calculate mean fan letters, readers with at least 4 hats, count classrooms with 41 desks}
\end{itemize}

\paragraph{CRAFT:}
\begin{itemize}[noitemsep, topsep=0pt]
    \item  \textbf{MATH:}  \textit{find pack size, count distinct solutions, calculate points, find tank capacity, solve exponential log equation, total energy equilateral triangle, inverse square law force, find max value, total logs in stack, sum of multiples of 13, calculate exponential growth, gravitational force, find x for piecewise composition, positive difference, specific piecewise func, day exceeds 200 cents, find lattice points, count integer parameters for integer solutions, count zeros in square of power of ten minus one, energy stored, sum of squares of roots, sum odd integers, find d minus e squared, compute complex series sum, total energy configuration, sum of areas, \dots, max item price, solve two variable system, inverse variation power, total distance hopped, is prime, total distance, find constant term of polynomial, total distance moved, find perpendicular slope, calculate inverse proportionality, find value of A, count integer a, find min items for higher score, apply r n times, find min x, day exceeds threshold, calculate area in square yards, solve log equation, total items produced, find variable for distance condition, solve time at speeds, find largest solution, find weight of object, calculate proportional value, calculate material cost, solve for variable, total elements in arithmetic sequence, transformed domain, find day for algae coverage, calculate energy stored, least value of y, solve bowling ball weight, find min froods}

    \item \textbf{Date:} \textit{get today date, calculate one week ago, calculate n days from future date, calculate n days from date in format, calculate date days ago, calculate n months from date, calculate one week from today, calculate date after event, find palindrome day, calculate date a month ago, calculate date after days and months, calculate relative date, calculate n days from reference, calculate one year ago from today, calculate n hours from date, calculate date n days from, get date today, calculate date 10 days ago from deadline, calculate n weeks from date, \dots, calculate n units from date, calculate n years from date, calculate n weeks from first weekday of year, calculate today from tomorrow, find special day, calculate date 10 days ago from future, calculate n days after event, calculate date from days passed, calculate one week from christmas eve, calculate one year ago, calculate date 24 hours later, calculate n weeks from anniversary, calculate tomorrow from uk format date, calculate n days from date, is palindrome, calculate one week from first monday of year, calculate one week ago from anniversary}

    \item \textbf{TabMWP:} \textit{get frequency, calculate volleyballs in lockers, calculate total cost from package prices, calculate total items from group counts, calculate mode, calculate donation difference for person, count bags with 20 to 40 broken cookies, calculate total items from groups and items per group, count commutes of 50 minutes, get received amount, calculate total items for groups, find probability, calculate vacation cost, calculate rate of change, find received amount for transaction, calculate vote difference between two items for group, count customers, find minimum value in stem leaf, calculate metric wrenches, find smallest number, count books with 30 to 50 characters, \dots, count people with 67 pullups, calculate difference in donations for person, calculate total cost from unit price and weight, calculate total items from ratio, calculate total cost from unit weight prices and weight, calculate donation difference between causes, calculate difference, calculate net income, calculate grasshoppers on twigs, count total members in group, calculate expenses on date, find lightest child, calculate difference in amounts, count votes for item from groups, calculate probability from count table, get table cell value, calculate jeans in hampers, count instances with specific value in stem leaf, calculate donation difference for person and causes, calculate total from frequency and additional count, calculate range, calculate total reviews}
\end{itemize}


\paragraph{REGAL:}
\begin{itemize}[noitemsep, topsep=0pt]
    \item \textbf{MATH:}  \textit{solve for largest side, apply function sequence, solve rational equation, calculate expression sum, max sum of products, find b for perpendicular bisector, vertex of quadratic, calculate work days, calculate c for zero coefficient, simplify and rationalize sympy, find a for binomial square, compound interest, calculate inverse variation, expand expression, calculate average speed, calculate rs, sum sequence, solve for p, max consecutive integers, find x intercept, day exceeding threshold, find smallest sum, solve for ac pair, constant function, sum of distances, evaluate expression, sum finite geometric series, factor expression, find common difference, total coins pirates, calculate geometric first term, calculate closest whole number, calculate x minus y squared, solve letter values, find circle center v2, evaluate expression with sqrt, calculate sum of equations, \dots, calculate x3 plus y3, find negative intervals, calculate floor and abs, solve quadratic and find min, calculate y, solve for a, check equations, rationalize and simplify, calculate xyz, calculate distance, solve for x in simplified equation, calculate expression, calculate exponent, sum arithmetic series, complete square form, calculate x2 plus y2
    }

    \item \textbf{Date:}  \textit{subtract weeks from date, add weeks to date, format date, add days to date, subtract months from date, subtract days from date, subtract years from date, calculate date, calculate days between weekdays}

    \item \textbf{TabMWP:}  \textit{count range, find mode, total participants, count bushes with fewer roses, find max frequency, total items, count in range, calculate total items, count below threshold, count teams with minimum size, calculate total, calculate range, calculate fraction, sum frequencies below threshold, sum frequencies, calculate difference, calculate median, total outcomes, count specific height, count numbers in range, difference between groups, access frequency, calculate proportionality constant, count values below threshold, find median, calculate probability, calculate mode, get frequency, convert stem leaf to numbers, find minimum, get total items, count scores above, rate of change, calculate mean}
\end{itemize}



\subsection{The tool graph evolution diagrams of {\ours} for various tasks.}
\label{subsec:tool-graph}
Below are the tool graph evolution diagrams for various tasks. The Date task does not have a tool network evolution diagram, as date reasoning does not heavily rely on tool diversity.


\begin{figure}[H]
\vskip 0.2in
\begin{center}
\centerline{\includegraphics[width=1\linewidth]{toolnet-trial1.png}}
% \vskip -0.2in
\caption{
The tool graph evolution diagram for Minecraft Trial 1. In this diagram, each node represents a tool function, and the edges represent the invocation relationships between tools. The darker the color, the more frequently the tool is invoked. The network consists of a total of 6 layers, with layers 2 to 6 shown here from top to bottom.}
\label{fig:toolnet1}
\end{center}
\vskip -0.3in
\end{figure}

\vskip -0.2in
\begin{figure}[H]
\vskip 0.2in
\begin{center}
\centerline{\includegraphics[width=1\linewidth]{toolnet-trial2.png}}
% \vskip -0.2in
\caption{The tool graph evolution diagram for Minecraft Trial 2. In this diagram, each node represents a tool function, and the edges represent the invocation relationships between tools. The darker the color, the more frequently the tool is invoked. The network consists of a total of 6 layers, with layers 2 to 6 shown here from top to bottom.}
\label{fig:toolnet2}
\end{center}
\vskip -0.3in
\end{figure}

\vskip -0.2in
\begin{figure}[H]
\vskip 0.2in
\begin{center}
\centerline{\includegraphics[width=1\linewidth]{toolnet-trial3.png}}
% \vskip -0.2in
\caption{The tool graph evolution diagram for Minecraft Trial 3. In this diagram, each node represents a tool function, and the edges represent the invocation relationships between tools. The darker the color, the more frequently the tool is invoked. The network consists of a total of 6 layers, with layers 2 to 7 shown here from top to bottom.}
\label{fig:toolnet3}
\end{center}
\vskip -0.3in
\end{figure}


\begin{figure}[H]
\vskip 0.2in
\begin{center}
\centerline{\includegraphics[width=1\linewidth]{toolnet-dabench.png}}
% \vskip -0.2in
\caption{The tool graph evolution diagram of DA-Bench. In this diagram, each node represents a tool function, and the edges represent the invocation relationships between tools. The darker the color, the more frequently the tool is invoked.}
\label{fig:toolnet-dabench}
\end{center}
\vskip -0.3in
\end{figure}

\begin{figure}[H]
\vskip 0.2in
\begin{center}
\centerline{\includegraphics[width=1\linewidth]{toolnet-textcraft.png}}
% \vskip -0.2in
\caption{The tool graph evolution diagram of TextCraft. In this diagram, each node represents a tool function, and the edges represent the invocation relationships between tools. The darker the color, the more frequently the tool is invoked.}
\label{fig:toolnet-textcraft}
\end{center}
\vskip -0.3in
\end{figure}


\begin{figure}[H]
\vskip 0.2in
\begin{center}
\centerline{\includegraphics[width=1\linewidth]{toolnet-math.png}}
% \vskip -0.2in
\caption{The tool graph evolution diagram of MATH. In this diagram, each node represents a tool function, and the edges represent the invocation relationships between tools. The darker the color, the more frequently the tool is invoked.}
\label{fig:toolnet-math}
\end{center}
\vskip -0.3in
\end{figure}

\begin{figure}[H]
\vskip 0.2in
\begin{center}
\centerline{\includegraphics[width=1\linewidth]{toolnet-tabmwp.png}}
% \vskip -0.2in
\caption{The tool graph evolution diagram of TabMWP. In this diagram, each node represents a tool function, and the edges represent the invocation relationships between tools. The darker the color, the more frequently the tool is invoked.}
\label{fig:toolnet-tabmwp}
\end{center}
\vskip -0.3in
\end{figure}
\section{Prompt Template}
\label{app:prompt}
In this section, we provide the prompt templates of different types used throughout our experiment. These prompts were carefully crafted to ensure that the model's output aligns with the specific objectives of each task.

\subsection{Construction Stage}
In open-ended task online training, we made slight modifications to their prompts based on Voyager~\citep{wang2023voyager}. For close-ended tasks, the prompts used during the construction process are as follows:
\begin{tcolorbox}[title=Task Solver's Prompt, breakable, width=\textwidth,top=0mm]
\begin{Verbatim}[breaklines, fontsize=\footnotesize]
# Instruction #
You are the Task Solver in a collaborative team, specializing in reasoning and Python programming. Your role is to analyze tasks, collaborate with the Tool Manager, and solve problems step by step.
Directly solving tasks without tool analysis is not allowed. Request necessary tools before proceeding when needed, based on the task analysis.

# WORKFLOW #
You can decide which step to take based on the environment and current situation, adapting dynamically as the task progresses.
Stage 1. Tool Requests:
    Requesting tool is mandatory. Request generalized and reusable tools to solve the task. Focus on abstract functionality rather than task-specific details to enhance flexibility and adaptability.
Stage 2. Code and Interact: 
    Write notebook blocks incrementally, executing and interacting with the environment step by step. Avoid bundling all steps into a single block; instead, adjust dynamically based on feedback after each interaction.
Stage 3: Validate and Conclude: 
    When confident in the solution, review your work, validate the results, and conclude the task.

# Custom Library #
===api===

# NOTICE #
1. You must fully understand the action space and its parameters before using it.
2. If code execution fails, you should analyze the error and try to resolve it. If you find that the error is caused by the API, please promptly report the error information to the Tool Manager.
3. Regardless of how simple the issue may seem, you should always aim to summarize and refine the tool requirements.


# Tool Request Guidelines #
1. Keep It Simple: Design tools with single and simple functionality to ensure they are easy to implement, understand, and use. Avoid unnecessary complexity.
2. Define Purpose: Clearly outline the tool’s role within broader workflows. Focus on creating reusable tools that solve abstract problems rather than task-specific ones.
3. Specify Input and Output: Define the required input and expected output formats, prioritizing generic structures (e.g., dictionaries or lists) to enhance flexibility and adaptability.
4. Generalize Functionality: Ensure the tool is not tied to a specific task. Abstract its functionality to make it applicable to similar problems in other contexts.


# ACTION SPACE #
You should Only take One action below in one RESPONSE:
## NotebookBlock Action
* Signature: 
NotebookBlock():
```python
executable python script
```
* Description: The NotebookBlock action allows you to create and execute a Jupyter Notebook cell. The action will add a code block to the notebook with the content wrapped inside the paired ``` symbols. If the block already exists, it can be overwritten based on the specified conditions (e.g., execution errors). Once added or replaced, the block will be executed immediately.
* Restrictions: Only one notebook block can be managed or executed per action.
* Example
- Example1: 
NotebookBlock():
```python
# Calculate the area of a circle with a radius of 5
radius = 5
area = 3.1416 * radius ** 2
print(area)
```

## Tool_request Action
* Signature:
{
    "action_name": "tool_request",
    "argument": {
         "request": [
             ...
         ]
    }
}
* Description: The Tool Request Action allows you to send tool requirements to the Tool Manager and request it to create appropriate tools. You need to provide the action in a JSON format, where the argument field contains a request parameter that accepts a list. Each element in the list is a string describing the desired tool.
* Note:
* Examples:
- Example 1:
{
    "action_name": "tool_request",
    "argument": {
        "request": [
            "I need a tool that calculates the average value of a specified column in a dataset. The input should include the column name."
        ]
    }
}
- Example 2:
{
    "action_name": "tool_request",
    "argument": {
        "request": [
            "I need a tool that filters rows in a dataset based on a specified condition. The input should include the column name and the condition to filter by."
        ]
    }
}


## Terminate Action
* Signature: Terminate(result=the result of the task)
* Description: The Terminate action ends the process and provides the task result. The `result` argument contains the outcome or status of task completion.
* Examples:
  - Example1: Terminate(result="A")
  - Example2: Terminate(result="1.23")

# RESPONSE FORMAT #
For each task input, your response should contain:
1. One RESPONSE should only contain One Stage, One Thought and One Action.
2. An current phase of task completion, outlining the steps from planning to review, ensuring progress and adherence to the workflow.  (prefix "Stage: ").
3. An analysis of the task and the current environment, including reasoning to determine the next action based on your role as a SolvingAgent. (prefix "Thought: ").
4. An action from the **ACTION SPACE** (prefix "Action: "). Specify the action and its parameters for this step.

# RESPONSE EXAMPLE #
Observation: ...(the output of last actions, as provided by the environment and the code output, you don't need to generate it)

Stage:...(One Stage from `WORKFLOW`)
Thought: ...
Action: ...(Use an action from the ACTION SPACE no more than once per response.)

# TASK #
===task===
\end{Verbatim}
\end{tcolorbox}

\begin{tcolorbox}[title=Tool Manager's Prompt, breakable, width=\textwidth,top=0mm]
\begin{Verbatim}[breaklines, fontsize=\footnotesize]
# Instruction #
You are a Tool Manager in a collaborative team, specializing in assembling existing APIs to construct hierarchical and reusable abstract tools based on predefined criteria.
You will be provided with a custom library, similar to Python’s built-in modules, containing various functions related to date reasoning. For each task, you will receive:
1. Tool request: The specific goal or functionality the new tool must achieve.
2. Existing tools: A list of available functions from the custom library that you can utilize.
Your task is to analyze the given request and create a reusable tool by effectively leveraging the relevant functions from the existing tools or utilizing basic tools to achieve the desired functionality. 
If an existing tool from the provided library already fully satisfies the requirements, simply return that tool instead of duplicating functionality. Ensure all responses align with reusability and efficiency principles.

# Custom Library #
===api===

# Creation Criteria #
- **Reusability**: The function could be resued for more complex function.
- **Innovation**: Tools should offer innovation, not merely wrap or replicate existing APIs. Simply re-calling an API without significant enhancements does not qualify as innovation.
- **Completeness**: The function should handle potential edge cases to ensure completeness.
- **Leveraging Existing Functions**: The function should effectively utilize existing functions to enhance efficiency and avoid redundancy.
- **Functionality**: Ensure the tool runs successfully and is bug-free, guaranteeing full functionality.

# ACTION SPACE #
You should Only take One action below in one RESPONSE:
## Create tool Action
* Description: The Create Tool action allows you to develop a new tool and temporarily store it in a private repository accessible only to you. Each invocation creates a single tool at a time. You can repeatedly use this action to build smaller components, which can later be assembled into the final tool.
* Signature: 
Create_tool(tool_name=The name of the tool you want to create):
```python
The source code of tool
```
* Example:
Create_tool(tool_name=“calculate_column_statistics”):
```python
def calculate_column_statistics(dataset: pd.DataFrame, column_name: str) -> Dict[str, float]:
    """
    Calculates basic statistics (mean, median, standard deviation) for a specified column in a dataset.
    Parameters:
    - dataset: A pandas DataFrame containing the data.
    - column_name: The name of the column to calculate statistics for.
    Returns:
    - A dictionary containing the mean, median, and standard deviation of the column.
    """
    if column_name not in dataset.columns:
        raise ValueError(f"Column '{column_name}' not found in the dataset.")
    
    column_data = dataset[column_name]
    stats = {
        "mean": column_data.mean(),
        "median": column_data.median(),
        "std_dev": column_data.std()
    }
    return stats
```
## Edit tool Action
* Description: The Edit Tool action allows you to modify an existing tool and temporarily store it in a private repository that only you can access. You must provide the name of the tool to be updated along with the complete, revised code. Please note that only one tool can be edited at a time.
* Signature: 
Edit_tool(tool_name=The name of the tool you want to create):
```python
The edited source code of tool
```
* Examples:
Edit_tool(tool_name="filter_rows_by_condition"):
```python
def filter_rows_by_condition(dataset: pd.DataFrame, column_name: str, condition: str) -> pd.DataFrame:
    """
    Filters rows in a dataset based on a specified condition for a given column.
    Parameters:
    - dataset: A pandas DataFrame containing the data.
    - column_name: The name of the column to apply the condition to.
    - condition: A string representing the condition, e.g., 'value > 10'.
    Returns:
    - A filtered DataFrame containing only the rows that satisfy the condition.
    """
    if column_name not in dataset.columns:
        raise ValueError(f"Column '{column_name}' not found in the dataset.")
    
    try:
        filtered_dataset = dataset.query(f"{column_name} {condition}")
    except Exception as e:
        raise ValueError(f"Invalid condition: {condition}. Error: {e}")
    
    return filtered_dataset
```

# RESPONSE FORMAT #
For each task input, your response should contain:
1. Each response should contain only one "Thought," and one "Action."
2. Determine how to construct your tool to meet tool request and function creation criteria. Check if any functions in the Existing Tool can be invoked to assist in the tool’s development and ensure alignment with the criteria.(prefix "Thought: ").
3. An action dict from the **ACTION SPACE** (prefix "Action: "). Specify the action and its parameters for this step. 

# RESPONSE EXAMPLE  #
1. If you determine that the tool request cannot be solved using existing tools, choose this mode to provide a clear and complete code solution.

Thought: ...
Action: ...

2. If you determine that the tool request is already satisfied by existing tools, choose this mode to directly reference and return the relevant tool without creating additional solutions.
Thought: ...
Tool: {  
    "tool_name": "Name of Existing tools"
}

# NOTICE #
1. You can directly call and use the tool in the custom library in your code or tool without importing it.
2. You can only create or edit one tool per response, so take it one step at a time.

# TASK #
===task===
\end{Verbatim}
\end{tcolorbox}


\begin{tcolorbox}[title=Prompt of Self-Check Step 1, breakable, width=\textwidth,top=0mm]
\begin{Verbatim}[breaklines, fontsize=\footnotesize]
# Instruction #
You are evaluating whether the tools provided by the Tool Manager meet the required standards. 
You follow a defined workflow, take actions from the ACTION SPACE, and apply the evaluation criteria. 

# Evaluation Criteria #
- **Reusability**: The function should be designed for reuse in more complex scenarios. For instance, in the case of the `craft_wooden_sword()` tool, it would be more versatile if it could accept a quantity as an input parameter.
- **Innovation**: Tools should offer innovation, not merely wrap or replicate existing APIs. Simply re-calling an API without significant enhancements does not qualify as innovation. If an existing tool from the provided library already fully satisfies the requirements, simply return that tool instead of duplicating functionality. Ensure all responses align with reusability and efficiency principles.
- **Completeness**: The function should handle potential edge cases to ensure completeness.
- **Leveraging Existing Functions**: Check if any function in "Existing Function" is helpful for completing the task. If such functions exist but are not invoked in the provided code, relevant feedback should be given.

## Tool Abstraction ##
Tool abstraction is essential for enabling tools to adapt to diverse tasks. Key principles include:
- Design generic functions to handle queries of the same type, based on shared reasoning steps, avoiding specific object names or terms.
- Name functions and write docstrings to reflect the core reasoning pattern and data organization, without referencing specific objects.
- Use general variable names and pass all column names as arguments to enhance adaptability.

# ACTION SPACE #
You should Only take One action below in one RESPONSE:
# Feedback Action
* Signature: {
    "action_name": "Feedback",
    "argument": {
        "feedback": ...
        "passed": true/false
    }
}
* Description: The Feedback Action is represented as a JSON string that provides feedback to the Tool Manager or SolvingAgent. The feedback field contains comments or suggestions, while pass indicates whether the tool meets the requirements (true for approval, false for rejection). Feedback should be concise, constructive, and relevant. If pass is true, the feedback can be left empty; otherwise, it must be provided.
* Example:
- Example1:
{
    "action_name": "Feedback",
    "argument": {
        "feedback": "",
        "passed": true
    }
}
- Example2:
{
    "action_name": "Feedback",
    "argument": {
        "feedback": "The tool correctly solves the equation for small numbers, but fails when the coefficients are very large. Consider optimizing the algorithm for handling larger values and improving computational efficiency.",
        "passed": false
    }
}

# RESPONSE FORMAT #
For each task input, your response should contain:
1. One RESPONSE should ONLY contain One Thought and One Action.
2. An comprehensive analysis of the tool code based on the evaluation criteria.(prefix "Thought: ").
3. An action from the **ACTION SPACE** (prefix "Action: "). 

# EXAMPLE RESPONSE #
Observation: ...(output from the last action, provided by the environment and task input, no need for you to generate it)

Thought: 1. Reusability: ...
2. Innovation: ...
3. Completeness: ...
4. Leveraging Existing Functions: ...

Action: ...(Use an action from the ACTION SPACE once per response.)

# Custom Library #
===api===

# TASK #
===task===
\end{Verbatim}
\end{tcolorbox}

\begin{tcolorbox}[title=Prompt of Self-Check Step 2, breakable, width=\textwidth,top=0mm]
\begin{Verbatim}[breaklines, fontsize=\footnotesize]
# Instruction #
You are verifying whether the tools provided by the Tool Manager execute without runtime errors.
You will use a custom library, similar to the built-in library, which provides everything necessary for the tasks. Your task is only to execute the provided tool code and check for runtime errors, not to evaluate the tool’s functionality or correctness.

# Stage and Workflow #
1. **Ensure Bug-Free Tool Operation**:
	- Execute the tool to ensure it runs without any runtime bugs.
	- You don’t need to verify the function’s functionality; simply call it to check for any runtime errors.
	- If the tool is a retrieved API, skip this step and proceed.
2. **Send Feedback**:
	- After executing the code, provide feedback based on the output, indicating whether the operation was successful or not.

# Notice #
1. If any issues with the tool are found, promptly provide clear and critical feedback to the Tool Manager for resolution. 
2. You should not create or edit functions (tools) with the same name as the Existing Functions in the code.
3. You can directly call the APIs from the custom library without needing to import or declare any external libraries.
4. You don’t need to verify the function’s functionality or set up its standard output; simply call it to check for any errors.

# ACTION SPACE #
You should Only take One action below in one RESPONSE:
## Python Action
* Signature: 
Python(file_path=python_file):
```python
executable_python_code
```
* Description: The Python action will create a python file in the field `file_path` with the content wrapped by paired ``` symbols. If the file already exists, it will be overwritten. After creating the file, the python file will be executed. Remember You can only create one python file.
* Examples:
- Example1
Python(file_path="solution.py"):
```python
# Calculate the area of a circle with a radius of 5
radius = 5
area = 3.1416 * radius ** 2
print(f"The area of the circle is {area} square units.")
```
- Example2
Python(file_path="solution.py"):
```python
# Calculate the perimeter of a rectangle with length 8 and width 3
length = 8
width = 3
perimeter = 2 * (length + width)
print(f"The perimeter of the rectangle is {perimeter} units.")
```

# Feedback Action
* Signature: {
    "action_name": "Feedback",
    "argument": {
        "feedback": ...
        "passed": true/false
    }
}
* Description: The Feedback Action is used to provide feedback to the Tool Manager. The feedback field contains detailed comments or suggestions. If the tool encounters an error, you should set passed to false and provide a detailed feedback. If the tool runs without errors, you can set passed to true and leave feedback as an empty string.
* Examples:
- Example 1:
{
    "action_name": "Feedback",
    "argument": {
        "feedback": ""
        "passed": true
    }
}
- Example 2:
{
    "action_name": "Feedback",
    "argument": {
        "feedback": "The tool encountered an error while executing. The variable 'height' is missing in the function call. Please ensure that all required parameters are provided.",
        "passed": false
    }
}

# RESPONSE FORMAT #
For each task input, your response should contain:
1. One RESPONSE should ONLY contain One Thought and One Action.
2. An analysis of the task and current environment, reasoning through the next evaluation step based on your role as CheckingAgent.(prefix "Thought: ").
3. An action from the **ACTION SPACE** (prefix "Action: "). Specify the action and its parameters for this step.

# EXAMPLE RESPONSE #
Observation: ...(output from the last action, provided by the environment and task input, no need for you to generate it)

Thought: ...
Action: ...(Use an action from the ACTION SPACE once per response.)

# Custom Library #
You can use pandas, sklearn, or other Python libraries as part of the custom library.

* Note: You can directly call these tools without importing or redefining them in your code.

Let's think step by step.
# TASK #
===task===
\end{Verbatim}
\end{tcolorbox}

\subsection{Test Stage}
\label{appsub:test_prompt}
During the test stage, the prompts used for different datasets are as follows:
\begin{tcolorbox}[title=Prompt on DABench, breakable, width=\textwidth,top=0mm]
\begin{Verbatim}[breaklines, fontsize=\footnotesize]
# Instruction #
You are a helpful assistant, skilled in data science tasks.
You will be provided with a task description and related files. 
You should complete tasks by writing notebook code to interact with the environment containing the task files.
Additionally, you must strictly adhere to the task constraints. 
Once the task is completed, you need to format the answer as specified in the answer format and invoke the Terminate action to conclude.
You should use actions from the ACTION SPACE, follow the Response Format, and complete the task within 20 steps.

You may also leverage the following helper functions if needed.
===api===


===example===


# Response Format #
Your each response should contain:
1. One RESPONSE should only contain ONLY One Thought and ONLY One Action.
2. Only an analysis of the task and the current environment, including reasoning to determine the next action. (prefix "Thought: ").
3. Only an action from the **ACTION SPACE** (prefix "Action: "). Specify the action and its parameters for this step.

Observation: ...(Provided by the environment, no need for you to generate it.))

Thought: ...
Action: ...

# ACTION SPACE #
## NotebookBlock Action
* Signature: 
NotebookBlock():
```python
executable python script
```
* Description: The NotebookBlock action allows you to create and execute a Jupyter Notebook cell. The action will add a code block to the notebook with the content wrapped inside the paired ``` symbols. If the block already exists, it can be overwritten based on the specified conditions (e.g., execution errors). Once added or replaced, the block will be executed immediately.
* Restrictions: Each response must contain only one notebook block.
* Note: In a single block, you may call multiple tools or single.
* Example:
Action: NotebookBlock():
```python
# Calculate the area of a circle with a radius of 5
radius = 5
area = 3.1416 * radius ** 2
print(area)
```

# Terminate Action
* Signature: Terminate(result="the result of the task")
* Description: The Terminate action marks the completion of a task and presents the final result. It is a formatting guideline, not an executable Python function. The result parameter must contain a clear, specific answer that strictly complies with the task’s output format, with all required values explicitly provided.
Tips:
    - Ensure the result parameter provides a definite and concrete final answer.
    - Do not include unresolved Python expressions, placeholders, or variables (e.g., @value[{x + y}] or @result[{variable_name}] or "@result[{variable_name}]".format(variable_name)).
    - The output must adhere precisely to the task’s formatting specifications, ensuring clarity and consistency.
* Examples:
- Example 1: 
Answer Format: @shapiro_wilk_statistic[test_statistic] @shapiro_wilk_p_value[p_value]
Action: Terminate(result="@shapiro_wilk_statistic[0.56] @shapiro_wilk_p_value[0.0002]")
- Example 2: 
Answer Format: @total_votes_outliers_num[outlier_num]
where "outlier_num" is an integer representing the number of values considered outliers in the 'total_votes' column.
Action: Terminate(result="@total_votes_outliers[10]")
- Example3:
Action: Terminate(result="@normality_test_result[Not Normal] @p_value[0.000]")

## Response Example
Here are four examples of responses.
## Example1
Thought: The dataset has been loaded successfully and it contains the "Close Price" column. Now, we need to calculate the mean of the "Close Price" column using pandas.
Action: NotebookBlock():
```python
# Calculate the mean of the "Close Price" column
mean_close_price = data["Close Price"].mean()
# Round the result to two decimal places
mean_close_price_rounded = round(mean_close_price, 2)
print(mean_close_price_rounded)
```
## Example2
Thought: We need to filter the dataset to only include rows where the “Volume” is greater than 100,000. This will help focus on high-volume trades.
Action: NotebookBlock():
```python
# Filter rows where "Volume" is greater than 100,000
filtered_data = data[data["Volume"] > 100000]
# Display the filtered dataset
print(filtered_data)
```
## Example3
Thought: To analyze the correlation between “Open Price” and “Close Price,” we will calculate the Pearson correlation coefficient using pandas.
Action: NotebookBlock():
```python
# Calculate the correlation between "Open Price" and "Close Price"
correlation = data["Open Price"].corr(data["Close Price"])
# Print the correlation result
print(correlation)
```
## Example4
Thought: To check for missing values in the dataset, we need to check for null values in each column using pandas.
Action: NotebookBlock():
```python
# Check for missing values in each column
missing_values = data.isnull().sum()
# Display the result
print(missing_values)
```

# Begin #
Let's Begin.
## Task 
===task===
\end{Verbatim}
\end{tcolorbox}


\begin{tcolorbox}[title=Prompt on TextCraft, breakable, width=\textwidth,top=0mm]
\begin{Verbatim}[breaklines, fontsize=\footnotesize]
# Instruction #
You are provided with a set of useful crafting recipes to create items in Minecraft.
Crafting commands follow the format: "craft [target object] using [input ingredients]".
You can either "fetch" an object (ingredient) from the inventory or the environment or "craft" the target object using the provided crafting commands.
You are allowed to use only the crafting commands provided; do not invent or use your own crafting commands.
If a crafting command specifies a generic ingredient, such as "planks", you can substitute it with a specific type of that ingredient (e.g., “dark oak planks”).
To complete the crafting tasks, you will write notebook code utilizing tools from the "Custom Library". You should carefully read and understand the tool’s docstrings and code to fully grasp their functionality and usage.
The tools should be invoked by outputting a block of Python code. Additionally, you may incorporate Python constructs such as for-loops, if-statements, and other logic where necessary.
Please always use actions from the ACTION SPACE and follow the Response Format.


# ACTION SPACE #
## NotebookBlock Action
* Signature: 
NotebookBlock():
```python
executable python script
```
* Description: The NotebookBlock action creates and executes a Jupyter Notebook cell. It adds a code block wrapped in ``` symbols, overwriting existing blocks if specified (e.g., after execution errors). The block is executed immediately after being added or replaced.
* Note: In a single block, you may call multiple tools.

## Terminate Action
* Signature: Terminate(result=the result of the task)
* Description: The Terminate action ends the process and provides the task result. The `result` argument contains the outcome or status of task completion. Only the CheckingAgent has the authority to decide whether a task is finished.
* Examples:
  - Example1: Action: Terminate(result="3")
  - Example2: Action: Terminate(result="Successfully craft 2 oak planks")
  - Example3: Action: Terminate(result="Successfully craft 1 milk")


# Response Format #
For each task input, your response should contain:
1. One RESPONSE should only contain ONLY One Thought and ONLY One Action.
2. An analysis of the task and the current environment, including reasoning to determine the next action. (prefix "Thought: ").
3. An action from the **ACTION SPACE** (prefix "Action: "). Specify the action and its parameters for this step.

## Response Example
### Example1
Thought: To craft a crimson fence, I need to use the crafting command: “craft 3 crimson fence using 2 stick, 4 crimson planks.” This means I need two ingredients: 2 sticks and 4 crimson planks. To obtain 4 crimson planks, I can use the command “craft 4 crimson planks using 1 crimson stems,” so I need to fetch 1 crimson stem first. For the sticks, I can use the command “craft 4 stick using 2 planks.” Since the crafting of sticks requires planks, I can directly use the crimson planks I craft earlier for this purpose.
Steps: 
1. Fetch 1 crimson stems.
2. Craft 4 crimson planks using 1 crimson stems.
3. Craft 4 sticks using 2 crimson planks.
4. Craft 3 crimson fences using 2 sticks and 4 crimson planks.
Action: NotebookBlock()
```python
get_object('1 crimson stems')
```
### Example2
Thought: The inventory is currently empty, which means I need to retrieve the necessary spruce logs from the environment. Since we need spruce planks to craft the spruce trapdoor, and each spruce trapdoor requires 6 spruce planks, I should start by fetching an adequate number of spruce logs. Since each spruce log can craft 4 spruce planks, we need at least 2 spruce logs to ensure we can craft 8 spruce planks (more than enough for 2 spruce trapdoors if needed in the future).
Action: NotebookBlock()
```python
# Step 2: Fetch 2 spruce logs from the environment
get_object('2 spruce logs')
```
### Example3
Thought: We have successfully retrieved 2 spruce logs from the environment. 
Action: Terminate(result="Successfully crafted 8 spruce planks")


# Custom Library #
### Tool `check_inventory`
Source Code:
```python
def check_inventory() -> str:
    """
    Retrieves the current inventory state from the environment.
    The function sends an 'inventory' command to the environment
    and processes the observation to return a string representation
    of the inventory, listing items and their quantities.
    Returns:
        str: A string describing the inventory in the format:
             "Inventory: [item_name] (quantity)"
    """
    obs, _ = step('inventory')
    return obs
```
Usage Example:
```python
check_inventory() 
# If the environment has no items, Output: Inventory: You are not carrying anything.
# If the environment contains 2 oak planks, Output: Inventory: [oak planks] (2)
```
### Tool `get_object`
Source Code:
```python
def get_object(target: str) -> None:
    """
    Retrieves an item from the environment.

    The function prints the response message from the environment, 
    indicating whether the retrieval was successful or not.

    Args:
        target (str): The name of the item to be retrieved.

    Returns:
        None
    """
    obs, _ = step("get " + target)
    print(obs)
```
Usage Example:
Craft Command:
craft 2 yellow dye using 1 sunflower
craft 8 yellow carpet using 8 white carpet, 1 yellow dye
```python
get_object("1 sunflower") # Ouput: Got 1 sunflower
get_object("2 sunflower") # Ouput: Got 2 sunflower
# Note: You cannot retrieve yellow dye directly from the environment; it must first be crafted using sunflowers.
get_object("1 yellow dye") # Output: Could not find yellow dye
```
### Tool `craft_object`
Source Code:
```python
def craft_object(target: str, ingredients: List[str]) -> None:
    """
    Crafts a specified item using the given ingredients.

    This function's `target` and `ingredients` parameters correspond to the craft command: 
    "Craft 'target' using [ingredients]".
    
    **Note:** The `ingredients` must exactly match the command format. For example, if the command requires 
    '1 oak logs', providing '1 oak log' instead will not be recognized.

    Prints the environment's response to indicate whether the crafting operation was successful or not.

    Args:
        target (str): The item to craft along with its quantity (e.g., '4 oak planks').
        ingredients (List[str]): A list of required ingredients with their respective quantities 
                                (e.g., ['1 oak logs']).

    Returns:
        None

    """
    obs, _ = step("craft " + target + " using " + ", ".join(ingredients))
    print(obs)
```
Usage Example:
- Example1
Query: Craft 1 black dye using 1 wither rose
Craft Command:
craft 1 black dye using 1 wither rose
craft 1 black dye using 1 ink sac

```python
get_object('1 wither rose')
craft_object('1 black dye', ['1 wither rose'])
# Ouput:
# Got 1 wither rose
# Crafted 1 minecraft:black dye
```
- Example2
Craft Commands: 4 oak planks using 1 oak logs
```python
get_object("1 oak logs") # Got 1 oak logs
check_inventory() # Inventory: ["oak logs"] (1)
craft_object('4 oak planks', ['1 oak logs'])
# Crafted 4 minecraft:oak planks
```

===api===


===example===

# Example #
Here is a demo of how to fetch and craft objects.
Crafting commands:
craft 3 dark oak sign using 6 dark oak planks, 1 stick
craft 4 dark oak planks using 1 dark oak logs
craft 1 stick using 1 planks
craft 4 stick using 2 bamboo
craft 4 oak planks using 1 oak logs
craft 1 dark oak fence using 2 stick, 4 dark oak planks
craft 1 warped stairs using 6 warped planks
craft 3 oak sign using 6 oak planks, 1 stick

craft 2 oak planks

Thought: To craft 2 oak planks, I need to first check if I can fetch them directly from the environment or my inventory. If they are not available, I will craft them using the appropriate crafting recipe. Based on the crafting commands, I know that I can craft 4 oak planks using 1 oak logs. My plan is as follows:
1. Check if 2 oak planks are already in my inventory or can be fetched from the environment.
2. If not, verify if I have the required ingredient (1 oak logs).
3. Gather the missing ingredient, if necessary.
4. Craft 4 oak planks and ensure 2 are available for the task.
Action: NotebookBlock():
```python
check_inventory()
```

Observation: Inventory: [stick] (1) 

Thought: I could not find 2 oak planks or the required ingredient (1 oak logs) in my inventory. I need to fetch 1 oak logs first.
Action: NotebookBlock():
```
get_object("1 oak logs")
```

Observation: Got 1 oak logs

Thought: I now have the required ingredient to craft oak planks. I will use the crafting command: craft 4 oak planks using 1 oak logs.
Action: NotebookBlock():
```
craft_object("4 oak planks", ["1 oak logs"])
check_inventory()
```

Observation: Crafted 4 minecraft:oak_planks
Inventory: [stick] (1) [oak_planks] (4)

Thought: I now have 2 oak planks in my inventory. Task Completed!
Action: Terminate(result=“Successfully craft 2 oak planks”)


# Begin #
Let's Begin. Please follow the Instruction and focus on solve the task.
# Current items in the inventory #
These are the items currently in your inventory. You cannot use items that are not in your inventory to craft new items.
===inventory===

# Task #
===task===
\end{Verbatim}
\end{tcolorbox}

\begin{tcolorbox}[title=Prompt on MATH, breakable, width=\textwidth,top=0mm]
\begin{Verbatim}[breaklines, fontsize=\footnotesize]
Your task is to solve math competition problems by writing Python programs.

You may also leverage the following helper functions, but must avoid fabricating and calling undefined function names.
```python
===api===
```

Examples: 

Examples: 
Query: Point $P$ lies on the line $x= -3$ and is 10 units from the point $(5,2)$. Find the product of all possible $y$-coordinates that satisfy the given conditions.
Program: 
```python
from sympy import symbols, Eq, solve
# Define symbolic variable for y-coordinate of point P
y = symbols('y')
# Step 1: Given conditions
x1 = -3  # Point P lies on the vertical line x = -3
x2, y2 = 5, 2  # Coordinates of the given point (5, 2)
d = 10  # Distance between point P and (5,2)
# Step 2: Apply the distance formula
# Distance formula: sqrt((x2 - x1)^2 + (y - y2)^2) = d
# Squaring both sides to eliminate the square root:
# (x2 - x1)^2 + (y - y2)^2 = d^2
distance_equation = Eq((x2 - x1)**2 + (y - y2)**2, d**2)
# Step 3: Solve for possible values of y
y_solutions = solve(distance_equation, y)
# Step 4: Compute the product of all possible y-values
product = y_solutions[0] * y_solutions[1]
# Step 5: Output the final result
print("Final Answer:", product)
```

Query: If $3p+4q=8$ and $4p+3q=13$, what is $q$ equal to?
Program:
```python
from sympy import symbols, Eq, solve
# Define symbolic variables for the unknowns p and q
p, q = symbols('p q')
# Step 1: Define the given system of equations
eq1 = Eq(3 * p + 4 * q, 8)  # Equation 1: 3p + 4q = 8
eq2 = Eq(4 * p + 3 * q, 13)  # Equation 2: 4p + 3q = 13
# Step 2: Solve the system of equations for p and q
solution = solve((eq1, eq2), (p, q))
# Step 3: Extract and output the value of q
print("Final Answer:", solution[q])
```

Query: Simplify $\frac{3^4+3^2}{3^3-3}$. Express your answer as a common fraction.
Program:
```python
from sympy import symbols, simplify
# Define the variable
x = symbols('x')
# Define the expression
numerator = 3**4 + 3**2
denominator = 3**3 - 3
fraction = numerator / denominator
# Simplify the fraction
simplified_fraction = simplify(fraction)
# Print the result
print("Final Answer:", simplified_fraction)
```

===example===

## Begin !
Please generate ONLY the code wrapped in ```python...``` to solve the query below.

Query: ===task===
Program:
\end{Verbatim}
\end{tcolorbox}



\begin{tcolorbox}[title=Prompt on Date, breakable, width=\textwidth,top=0mm]
\begin{Verbatim}[breaklines, fontsize=\footnotesize]
Your task is to solve simple word problems by creating Python programs.

You may also leverage the following helper functions, but must avoid fabricating and calling undefined function names, such as `calculate_date_by_years`.
```python
===api===
```

Examples:

Query: In the US, Thanksgiving is on the fourth Thursday of November. Today is the US Thanksgiving of 2001. What is the date one week from today in MM/DD/YYYY?
Program:
```python
# import relevant packages
from datetime import date, time, datetime
from dateutil.relativedelta import relativedelta
import calendar
# 1. What is the date of the first Thursday of November? (independent, support: [])
date_1st_thu = date(2001,11,1)
while date_1st_thu.weekday() != calendar.THURSDAY:
    date_1st_thu += relativedelta(days=1)
# 2. How many days are there in a week? (independent, support: ["External knowledge: There are 7 days in a week."])
n_days_of_a_week = 7
# 3. What is the date today? (depends on 1 and 2, support: ["Today is the US Thanksgiving of 2001", "Thanksgiving is on the fourth Thursday of November"])
days_from_1st_to_4th_thu = (4-1) * n_days_of_a_week
date_today = date_1st_thu + relativedelta(days=days_from_1st_to_4th_thu)
# 4. What is the date one week from today? (depends on 3, support: [])
date_1week_from_today = date_today + relativedelta(weeks=1)
# 5. Final Answer: What is the date one week from today in MM/DD/YYYY? (depends on 4, support: [])
answer = date_1week_from_today.strftime("%m/%d/%Y")
# print the answer
print(answer)
```

Query: Yesterday was 12/31/1929. Today could not be 12/32/1929 because December has only 31 days. What is the date tomorrow in MM/DD/YYYY?
Program:
```python
# import relevant packages
from datetime import date, time, datetime
from dateutil.relativedelta import relativedelta
# 1. What is the date yesterday? (independent, support: ["Yesterday was 12/31/1929"])
date_yesterday = date(1929,12,31)
# 2. What is the date today? (depends on 1, support: ["Today could not be 12/32/1929 because December has only 31 days"])
date_today = date_yesterday + relativedelta(days=1)
# 3. What is the date tomorrow? (depends on 2, support: [])
date_tomorrow = date_today + relativedelta(days=1)
# 4. Final Answer: What is the date tomorrow in MM/DD/YYYY? (depends on 3, support: [])
answer = date_tomorrow.strftime("%m/%d/%Y")
# print the answer
print(answer)
```

Query: The day before yesterday was 11/23/1933. What is the date one week from today in MM/DD/YYYY?
Program:
```python
# import relevant packages
from datetime import date, time, datetime
from dateutil.relativedelta import relativedelta
# 1. What is the date the day before yesterday? (independent, support: ["The day before yesterday was 11/23/1933"])
date_day_before_yesterday = date(1933,11,23)
# 2. What is the date today? (depends on 1, support: [])
date_today = date_day_before_yesterday + relativedelta(days=2)
# 3. What is the date one week from today? (depends on 2, support: [])
date_1week_from_today = date_today + relativedelta(weeks=1)
# 4. Final Answer: What is the date one week from today in MM/DD/YYYY? (depends on 3, support: [])
answer = date_1week_from_today.strftime("%m/%d/%Y")
# print the answer
print(answer)
```

===example===

## Begin !
Please generate ONLY the code wrapped in ```python...``` to solve the query below.

Query: ===task===
Program:
\end{Verbatim}
\end{tcolorbox}



\begin{tcolorbox}[title=Prompt on TabMWP, breakable, width=\textwidth,top=0mm]
\begin{Verbatim}[breaklines, fontsize=\footnotesize]
Your task is to solve table-reasoning problems by writing Python programs.
You are given a table. The first row is the name for each column. Each column is seperated by "|" and each row is seperated by "\n".
Pay attention to the format of the table, and what the question asks.

You may also leverage the following helper functions, but must avoid fabricating and calling undefined function names.
```python
===api===
```


Examples: 
### Table
Name: None
Unit: $
Content:
Date | Description | Received | Expenses | Available Funds
 | Balance: end of July | | | $260.85
8/15 | tote bag | | $6.50 | $254.35
8/16 | farmers market | | $23.40 | $230.95
8/22 | paycheck | $58.65 | | $289.60
### Question
This is Akira's complete financial record for August. How much money did Akira receive on August 22?
### Solution code
```python
records = {
    "7/31": {"Description": "Balance: end of July", "Received": "", "Expenses": "", "Available Funds": 260.85},
    "8/15": {"Description": "tote bag", "Received": "", "Expenses": 6.5, "Available Funds": ""},
    "8/16": {"Description": "farmers market", "Received": "", "Expenses": 23.4, "Available Funds": ""},
    "8/22": {"Description": "paycheck", "Received": 58.65, "Expenses": "", "Available Funds": ""}
}
# Access the amount received on August 22
received_aug_22 = records["8/22"]["Received"]
print("Final Answer: ", received_aug_22)
```

### Table
Name: Orange candies per bag
Unit: bags
Content:
Stem | Leaf 
2 | 2, 3, 9
3 | 
4 | 
5 | 0, 6, 7, 9
6 | 0
7 | 1, 3, 9
8 | 5
### Question
A candy dispenser put various numbers of orange candies into bags. How many bags had at least 32 orange candies?
### Solution code
```python
data = {
    2: [2, 3, 9],
    3: [],
    4: [],
    5: [0, 6, 7, 9],
    6: [0],
    7: [1, 3, 9],
    8: [5]
}
# Initialize the count to zero
count = 0
# Iterate over the keys in the dictionary
for key in data:
    # Combine tenth digit and unit digit
    if key * 10 + data[key] >= 32:
        # Increment the count
        count += 1

# Output the result
print("Final Answer: ", count)
```

### Table
Name: Monthly Savings  
Unit: $  
Content:  
Date  | Description       | Received | Expenses | Available Funds |
       | Balance: end of May |   |   | $500.00 |
6/10  | groceries        |   | $45.75 | $454.25 |
6/15  | gas refill       |   | $30.20 | $424.05 |
6/25  | salary           | $1200.00 |   | $1624.05 |
### Question
How much money did Akira receive on June 25?
### Solution code
```python
import pandas as pd
records = {
    "5/31": {"Description": "Balance: end of May", "Received": "", "Expenses": "", "Available Funds": 500.00},
    "6/10": {"Description": "groceries", "Received": "", "Expenses": 45.75, "Available Funds": ""},
    "6/15": {"Description": "gas refill", "Received": "", "Expenses": 30.2, "Available Funds": ""},
    "6/25": {"Description": "salary", "Received": 1200.00, "Expenses": "", "Available Funds": ""}
}
# Access the amount received on June 25
received_june_25 = records["6/25"]["Received"]
print("Final Answer: ", received_june_25)
```

===example===

## Begin!
Please solve the task below and enclose your code within a single code block using ```python```  format.

===task===
### Solution code
\end{Verbatim}
\end{tcolorbox}









\section{Examples}
\label{app:example}
\subsection{Generated Tools}

\textbf{The tools generated for the Open-ended Tasks are as follows:}
\begin{tcolorbox}[title=CraftDiamondHelmet, width=\textwidth,top=0mm,  breakable]
\begin{Verbatim}[breaklines=true, breakanywhere=true, fontsize=\footnotesize]
async function craftDiamondHelmet(bot) {
  const mcData = require('minecraft-data')(bot.version);
  const Vec3 = require('vec3').Vec3;

  // Check inventory for the number of diamonds
  const diamondCount = bot.inventory.count(mcData.itemsByName["diamond"].id);
  const requiredDiamonds = 5;

  // If not enough diamonds, mine diamond ores
  if (diamondCount < requiredDiamonds) {
    const diamondsToMine = requiredDiamonds - diamondCount;
    bot.chat(`Need ${diamondsToMine} more diamond(s). Mining...`);
    await mineBlock(bot, "diamond_ore", diamondsToMine);
  }

  // Ensure a crafting table is placed nearby
  let craftingTable = bot.findBlock({
    matching: mcData.blocksByName.crafting_table.id,
    maxDistance: 32
  });
  if (!craftingTable) {
    bot.chat("Placing crafting table...");
    await placeItem(bot, "crafting_table", bot.entity.position.offset(1, 0, 0));
    craftingTable = bot.findBlock({
      matching: mcData.blocksByName.crafting_table.id,
      maxDistance: 32
    });
  }

  // Craft the diamond helmet
  bot.chat("Crafting diamond helmet...");
  await craftItem(bot, "diamond_helmet", 1);
  bot.chat("Diamond helmet crafted successfully.");
}
\end{Verbatim}
\end{tcolorbox}


\begin{tcolorbox}[title=CraftItemWithMaterials, width=\textwidth,top=0mm,  breakable]
\begin{Verbatim}[breaklines=true, breakanywhere=true, fontsize=\footnotesize]
async function craftItemWithMaterials(bot, itemName, requiredMaterials) {
  const mcData = require('minecraft-data')(bot.version);
  const Vec3 = require('vec3').Vec3;

  // Check inventory for required materials
  for (const material of requiredMaterials) {
    let itemCount = bot.inventory.count(mcData.itemsByName[material.name].id);
    if (itemCount < material.count) {
      const requiredCount = material.count - itemCount;
      bot.chat(`Need ${requiredCount} more ${material.name}(s).`);
      if (material.name === "diamond") {
        let diamondOre = await bot.findBlock({
          matching: mcData.blocksByName["diamond_ore"].id,
          maxDistance: 32
        });
        if (!diamondOre) {
          bot.chat("No diamond ore found nearby. Exploring...");
          diamondOre = await exploreUntil(bot, new Vec3(1, 0, 1), 60, () => {
            return bot.findBlock({
              matching: mcData.blocksByName["diamond_ore"].id,
              maxDistance: 32
            });
          });
        }
        if (diamondOre) {
          await mineBlock(bot, "diamond_ore", requiredCount);
        } else {
          bot.chat("Failed to find diamond ore after exploring.");
          return;
        }
      } else if (material.name === "stick") {
        const woodenPlanksCount = bot.inventory.count(mcData.itemsByName["oak_planks"].id) + bot.inventory.count(mcData.itemsByName["birch_planks"].id);
        if (woodenPlanksCount < 2) {
          const requiredLogs = Math.ceil((2 - woodenPlanksCount) / 4);
          bot.chat(`Need more wooden planks. Gathering ${requiredLogs} logs...`);
          await obtainWoodLogs(bot, requiredLogs);
          await craftItem(bot, "oak_planks", requiredLogs);
        }
        bot.chat("Crafting sticks...");
        await craftItem(bot, "stick", 1);
      }
    }
  }

  // Ensure a crafting table is placed nearby
  let craftingTable = bot.findBlock({
    matching: mcData.blocksByName.crafting_table.id,
    maxDistance: 32
  });
  if (!craftingTable) {
    bot.chat("Placing crafting table...");
    await placeItem(bot, "crafting_table", bot.entity.position.offset(1, 0, 0));
    craftingTable = bot.findBlock({
      matching: mcData.blocksByName.crafting_table.id,
      maxDistance: 32
    });
  }

  // Craft the item
  bot.chat(`Crafting ${itemName}...`);
  await craftItem(bot, itemName, 1);
  bot.chat(`${itemName} crafted successfully.`);
}

async function craftDiamondAxe(bot) {
  const requiredMaterials = [{
    name: "diamond",
    count: 3
  }, {
    name: "stick",
    count: 2
  }];
  await craftItemWithMaterials(bot, "diamond_axe", requiredMaterials);
}
\end{Verbatim}
\end{tcolorbox}


\textbf{The tools generated for the Agent Tasks are as follows:}
Here, we can clearly see the call relationships between functions, thus forming more complex tools.
\begin{tcolorbox}[title=Tools for DA-Bench, width=\textwidth,top=0mm,  breakable]
\begin{Verbatim}[breaklines=true, breakanywhere=true, fontsize=\footnotesize]
def filter_rows_by_non_null(df: pd.DataFrame, column_name: str) -> pd.DataFrame:
    """
    Filters rows in a dataset based on non-null values in a specified column.
    
    Parameters:
    - df (pd.DataFrame): The input DataFrame.
    - column_name (str): The name of the column to filter by non-null values.
    
    Returns:
    - pd.DataFrame: A DataFrame with rows containing non-null values in the specified column.
    
    Raises:
    - ValueError: If the specified column is not found in the DataFrame.
    """
    # Check if the column exists in the DataFrame
    if column_name not in df.columns:
        raise ValueError(f"Column '{column_name}' not found in the DataFrame.")
    
    # Filter rows based on non-null values in the specified column
    filtered_df = df.dropna(subset=[column_name])
    
    return filtered_df

def convert_column_to_numeric(df: pd.DataFrame, column_name: str) -> pd.DataFrame:
    """
    Converts a specified column in a DataFrame to numeric values, handling non-numeric values appropriately.
    
    Parameters:
    - df (pd.DataFrame): The input DataFrame.
    - column_name (str): The name of the column to convert to numeric values.
    
    Returns:
    - pd.DataFrame: The DataFrame with the specified column converted to numeric values.
    
    Raises:
    - ValueError: If the specified column is not found in the DataFrame.
    """
    # Check if the column exists in the DataFrame
    if column_name not in df.columns:
        raise ValueError(f"Column '{column_name}' not found in the DataFrame.")
    
    # Convert the specified column to numeric values, setting non-numeric values to NaN
    df[column_name] = pd.to_numeric(df[column_name], errors='coerce')
    
    # Filter out rows with non-numeric values in the specified column using the existing tool
    df = filter_rows_by_non_null(df, column_name)
    
    return df

def create_sum_feature(df: pd.DataFrame, new_column_name: str, columns_to_sum: list) -> pd.DataFrame:
    """
    Creates a new feature by summing specified columns in a DataFrame.
    
    Parameters:
    - df (pd.DataFrame): The input DataFrame.
    - new_column_name (str): The name of the new column to be created.
    - columns_to_sum (list): A list of column names to sum.
    
    Returns:
    - pd.DataFrame: The DataFrame with the new feature added.
    
    Raises:
    - ValueError: If any of the specified columns are not found in the DataFrame.
    """
    # Check if all specified columns exist in the DataFrame
    for column in columns_to_sum:
        if column not in df.columns:
            raise ValueError(f"Column '{column}' not found in the DataFrame.")
    
    # Convert specified columns to numeric values
    for column in columns_to_sum:
        df = convert_column_to_numeric(df, column)
    
    # Create the new feature by summing the specified columns
    df[new_column_name] = df[columns_to_sum].sum(axis=1)
    
    return df
\end{Verbatim}
\end{tcolorbox}


\begin{tcolorbox}[title=Tools for TextCraft, width=\textwidth,top=0mm, breakable]
\begin{Verbatim}[breaklines=true, breakanywhere=true, fontsize=\footnotesize]
def gather_materials_for_dye(required_materials: dict) -> bool:
    """
    Gathers the required materials for crafting any dye.
    
    Parameters:
    - required_materials (dict): A dictionary where keys are material names and values are the required quantities.
    
    The tool checks the inventory for these materials and gathers them if they are missing.
    
    Returns:
    - bool: True if all materials were successfully gathered, False otherwise.
    """
    # Gather the required materials
    if not gather_materials(required_materials):
        return False
    
    # Check if we have white dye, if not gather bone meal or lily of the valley to craft it
    inventory = check_inventory()
    if "white dye" in required_materials and "white dye" not in inventory:
        if not gather_materials({"bone meal": 1}) and not gather_materials({"lily of the valley": 1}):
            return False
        # Craft white dye using bone meal or lily of the valley
        if "bone meal" in inventory:
            craft_object("1 white dye", ["1 bone meal"])
        elif "lily of the valley" in inventory:
            craft_object("1 white dye", ["1 lily of the valley"])
    
    # Recheck the inventory to ensure all materials are gathered
    missing_items = check_missing_items([f"{qty} {item}" for item, qty in required_materials.items()])
    if missing_items:
        print(f"Missing items: {missing_items}")
        return False
    
    # Successfully gathered all materials
    return True

def craft_orange_dye(quantity: int) -> bool:
    """
    Crafts the specified quantity of orange dye.
    
    Parameters:
    - quantity (int): The number of orange dye to craft.
    
    Returns:
    - bool: True if the orange dye was successfully crafted, False otherwise.
    """
    # Define the required materials for crafting orange dye
    required_materials = {"orange tulip": quantity, "red dye": quantity, "yellow dye": quantity}
    
    # Gather the required materials using the existing gather_materials_for_dye function
    if not gather_materials_for_dye(required_materials):
        return False
    
    # Check the inventory for available materials
    inventory = check_inventory()
    
    # Craft orange dye using orange tulip if available
    if "orange tulip" in inventory:
        craft_object(f"{quantity} orange dye", [f"{quantity} orange tulip"])
        print(f"Crafted {quantity} orange dye using {quantity} orange tulip")
        return True
    
    # Craft orange dye using red dye and yellow dye if available
    if "red dye" in inventory and "yellow dye" in inventory:
        craft_object(f"{quantity} orange dye", [f"{quantity} red dye", f"{quantity} yellow dye"])
        print(f"Crafted {quantity} orange dye using {quantity} red dye and {quantity} yellow dye")
        return True
    
    # If neither method was successful, return False
    print("Failed to craft orange dye.")
    return False
\end{Verbatim}
\end{tcolorbox}


\textbf{The tools generated for the Single-turn Code Task are as follows:}
\begin{tcolorbox}[title=Tools for MATH, width=\textwidth,top=0mm, breakable]
\begin{Verbatim}[breaklines=true, breakanywhere=true, fontsize=\footnotesize]
def find_integer_satisfying_condition(condition):
    """
    Find the smallest positive integer that satisfies the given condition.

    Parameters:
        condition (function): A lambda function representing the condition to be checked.

    Returns:
        int: The smallest positive integer that satisfies the condition.
    """
    x = 1
    while True:
        if condition(x):
            return x
        x += 1

def calculate_min_correct_answers(total_problems, passing_percentage):
    """
    Calculate the minimum number of correct answers required to pass a test based on the total number of problems and the passing percentage.

    Parameters:
        total_problems (int): The total number of problems on the test.
        passing_percentage (float): The passing percentage required to pass the test.

    Returns:
        int: The minimum number of correct answers required to pass the test.
    """
    if total_problems <= 0:
        return "Total number of problems must be greater than zero."
    if not (0 <= passing_percentage <= 100):
        return "Passing percentage must be between 0 and 100."

    required_correct_answers = (passing_percentage / 100) * total_problems

    # Use find_integer_satisfying_condition to find the minimum integer satisfying the condition
    min_correct_answers = find_integer_satisfying_condition(lambda x: x >= required_correct_answers)
    
    return min_correct_answers
\end{Verbatim}
\end{tcolorbox}

\begin{tcolorbox}[title=Tools for Date, width=\textwidth,top=0mm, breakable]
\begin{Verbatim}[breaklines=true, breakanywhere=true, fontsize=\footnotesize]
def calculate_date_by_days(start_date_str: str, days_to_add: int, date_format="%m/%d/%Y") -> str:
    """
    Calculates the date a specified number of days before or after a given date.

    Parameters:
    - start_date_str (str): The starting date as a string in the format MM/DD/YYYY.
    - days_to_add (int): The number of days to add (positive) or subtract (negative) from the start date.
    - date_format (str): The format of the input and output date string. Default is 'MM/DD/YYYY'.

    Returns:
    - str: The resulting date in the format MM/DD/YYYY.
    
    Raises:
    - ValueError: If the input date string does not match the specified format.
    - OverflowError: If the resulting date is out of the valid range for dates.
    """
    from datetime import datetime, timedelta

    try:
        # Parse the input date string into a date object using the provided format
        start_date = datetime.strptime(start_date_str, date_format).date()

        # Calculate the new date by adding the specified number of days
        new_date = start_date + timedelta(days=days_to_add)

        # Format the new date back into the desired string format
        result_date_str = new_date.strftime(date_format)

        return result_date_str
    except ValueError as e:
        raise ValueError("Incorrect date format. Please ensure the date string matches the provided format.") from e
    except OverflowError as e:
        raise OverflowError("The resulting date is out of the valid range for dates.") from e

def calculate_date_by_days_uk_format(start_date_str: str, days_to_add: int) -> str:
    """
    Calculates the date a specified number of days before or after a given date in UK format (DD/MM/YYYY).

    Parameters:
    - start_date_str (str): The starting date as a string in the format DD/MM/YYYY.
    - days_to_add (int): The number of days to add (positive) or subtract (negative) from the start date.

    Returns:
    - str: The resulting date in the format MM/DD/YYYY.
    
    Raises:
    - ValueError: If the input date string does not match the specified format.
    """
    from datetime import datetime

    try:
        # Convert the input date from DD/MM/YYYY to MM/DD/YYYY
        start_date = datetime.strptime(start_date_str, "%d/%m/%Y")
        
        # Use the existing tool to calculate the new date
        result_date_str = calculate_date_by_days(start_date.strftime("%m/%d/%Y"), days_to_add, "%m/%d/%Y")
        
        return result_date_str
    except ValueError as e:
        raise ValueError("Incorrect date format. Please ensure the date string matches the provided format.") from e
\end{Verbatim}
\end{tcolorbox}


\begin{tcolorbox}[title=Tools for TabMWP, width=\textwidth,top=0mm, breakable]
\begin{Verbatim}[breaklines=true, breakanywhere=true, fontsize=\footnotesize]
import pandas as pd

def stem_and_leaf_to_dataframe(stem_leaf_dict: dict) -> pd.DataFrame:
    """
    Converts a stem-and-leaf plot into a DataFrame.

    Parameters:
    - stem_leaf_dict (dict): A dictionary where keys are the stems and values are lists of leaves.

    Returns:
    - pd.DataFrame: A DataFrame with a single column containing the combined values of stems and leaves.
    """
    # Initialize an empty list to store the combined values
    combined_values = []

    # Iterate through the dictionary to combine stems and leaves
    for stem, leaves in stem_leaf_dict.items():
        for leaf in leaves:
            combined_value = int(f"{stem}{leaf}")
            combined_values.append(combined_value)

    # Create a DataFrame from the combined values
    df = pd.DataFrame(combined_values, columns=["Values"])
    
    return df

import pandas as pd

def count_value_occurrences(stem_leaf_dict: dict, value) -> int:
    """
    Counts the occurrences of a specific value in a DataFrame column created from a stem-and-leaf plot.

    Parameters:
    - stem_leaf_dict (dict): A dictionary where keys are the stems and values are lists of leaves.
    - value: The value to count in the DataFrame.

    Returns:
    - int: The count of the specified value in the DataFrame.
    """
    # Convert the stem-and-leaf plot to a DataFrame using the existing tool
    df = stem_and_leaf_to_dataframe(stem_leaf_dict)
    
    # Count the occurrences of the specified value in the DataFrame
    count = df["Values"].value_counts().get(value, 0)
    
    return count
\end{Verbatim}
\end{tcolorbox}

%\balance


\end{document}
\endinput
%%
%% End of file `sample-manuscript.tex'.
