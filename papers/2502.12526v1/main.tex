%%
%% This is file `sample-manuscript.tex',
%% generated with the docstrip utility.
%%
%% The original source files were:
%%
%% samples.dtx  (with options: `manuscript')
%% 
%% IMPORTANT NOTICE:
%% 
%% For the copyright see the source file.
%% 
%% Any modified versions of this file must be renamed
%% with new filenames distinct from sample-manuscript.tex.
%% 
%% For distribution of the original source see the terms
%% for copying and modification in the file samples.dtx.
%% 
%% This generated file may be distributed as long as the
%% original source files, as listed above, are part of the
%% same distribution. (The sources need not necessarily be
%% in the same archive or directory.)
%%
%%
%% Commands for TeXCount
%TC:macro \cite [option:text,text]
%TC:macro \citep [option:text,text]
%TC:macro \citet [option:text,text]
%TC:envir table 0 1
%TC:envir table* 0 1
%TC:envir tabular [ignore] word
%TC:envir displaymath 0 word
%TC:envir math 0 word
%TC:envir comment 0 0
%%
%%
%% The first command in your LaTeX source must be the \documentclass command.
\documentclass[sigconf]{acmart}
%\documentclass[authorversion,sigconf,nonacm]{acmart}
%\documentclass[acmlarge,anonymous,review]{acmart}

%% \BibTeX command to typeset BibTeX logo in the docs
\AtBeginDocument{%
  \providecommand\BibTeX{{%
    \normalfont B\kern-0.5em{\scshape i\kern-0.25em b}\kern-0.8em\TeX}}}

%% Rights management information.  This information is sent to you
%% when you complete the rights form.  These commands have SAMPLE
%% values in them; it is your responsibility as an author to replace
%% the commands and values with those provided to you when you
%% complete the rights form.



\copyrightyear{2025}
\acmYear{2025}
\setcopyright{acmcopyright}
\acmConference[Conference acronym 'XX]{Make sure to enter the correct
  conference title from your rights confirmation email}{June 03--05,
  2018}{Woodstock, NY}
\acmPrice{15.00}
\acmDOI{10.1145/3491102.3501940}
\acmISBN{978-1-4503-9157-3/22/04}



%%
%% Submission ID.
%% Use this when submitting an article to a sponsored event. You'll
%% receive a unique submission ID from the organizers
%% of the event, and this ID should be used as the parameter to this command.
%\acmSubmissionID{2446}

%%
%% The majority of ACM publications use numbered citations and
%% references.  The command \citestyle{authoryear} switches to the
%% "author year" style.
%%
%% If you are preparing content for an event
%% sponsored by ACM SIGGRAPH, you must use the "author year" style of
%% citations and references.
%% Uncommenting
%% the next command will enable that style.
%%\citestyle{acmauthoryear}
\usepackage{subcaption}
\usepackage{tabularx} 
\usepackage{booktabs}
\usepackage{geometry}
\usepackage{colortbl}
\usepackage{xcolor}
\usepackage{soul}
\usepackage{_macros}
\usepackage{caption}
\usepackage{subcaption}
\usepackage{array}
\usepackage{tikz}
\usepackage{hyperref}
\usepackage{hyperxmp}   


\newcommand*\circled[1]{\tikz[baseline=(char.base)]{
            \node[shape=circle,draw,inner sep=1pt] (char) {#1};}}


\newcommand{\name}{\textbf{AnimAlte}}
\sethlcolor{blue!50!red!50!white}


\raggedbottom

%%
%% end of the preamble, start of the body of the document source.
\begin{document}

%%
%% The "title" command has an optional parameter,
%% allowing the author to define a "short title" to be used in page headers.
\title{AnimAlte: Designing AI-Infused Cartoon Videos to Improve Preschoolers' Language Learning with Family Engagement at Home}

\author{Shiya Tsang}
\orcid{0009-0008-4338-3570}
\affiliation{%
  \institution{Hong Kong University of Science and Technology (Guangzhou)}
%   \streetaddress{No.1 DuXue Road}
  \city{Guangzhou}
%   \state{Guangdong}
  \country{China}
%   \postcode{511458}
}
\email{szeng785@connect.hkust-gz.edu.cn}

\author{Ruiyao Miao}
\orcid{0009-0002-1848-4421}
\affiliation{%
  \institution{University of California, Los Angeles}
%   \streetaddress{405 Hilgard Ave, Los Angeles, CA}
  \city{California}
%   \state{LA}
  \country{United States}
%   \postcode{90024}
}
\email{ruiyao0809@g.ucla.edu}

\author{Junren Xiao}
\orcid{0009-0006-5633-2268}
\affiliation{%
  \institution{Hong Kong University of Science and Technology (Guangzhou)}
%   \streetaddress{No.1 DuXue Road}
  \city{Guangzhou}
%   \state{Guangdong}
  \country{China}
%   \postcode{511458}
}
\email{jxiao767@connect.hkust-gz.edu.cn}

\author{Hui Xiong}
\authornote{Corresponding Author.}
\orcid{0000-0001-6016-6465}
\affiliation{%
  \institution{Hong Kong University of Science and Technology (Guangzhou)}
%   \streetaddress{No.1 DuXue Road}
  \city{Guangzhou}
%   \state{Guangdong}
  \country{China}
%   \postcode{511458}
}
\email{xionghui@hkust-gz.edu.cn}



%%
%% The "author" command and its associated commands are used to define
%% the authors and their affiliations.
%% Of note is the shared affiliation of the first two authors, and the
%% "authornote" and "authornotemark" commands
%% used to denote shared contribution to the research.



%%
%% By default, the full list of authors will be used in the page
%% headers. Often, this list is too long, and will overlap
%% other information printed in the page headers. This command allows
%% the author to define a more concise list
%% of authors' names for this purpose.
%\renewcommand{\shortauthors}{}

%%
%% The abstract is a short summary of the work to be presented in the
%% article.
\begin{abstract}

Cartoon videos have proven to be effective in learning vocabulary to preschool children. However, we have little knowledge about integrating AI into cartoon videos to provide systematic, multimodal vocabulary learning support. This late-breaking work present \name{}, an AI-powered cartoon video system that enables real-time Q\&A, vocabulary review, and contextual learning. Preliminary findings contextualized how families interact with \name{} to support vocabulary learning. Parents appreciated the system for its personalized, engaging experiences, fostering collaboration, and encouraging self-reflection on parenting. This study offers valuable design implications for informing future video systems to support vocabulary learning.

%finding1 - 不同的环境中时间
%艺术创作+语言表达不同地方时间疗愈好处的between session activities
%however, 在没有治疗师的情况下如何引导用户通过语言和艺术创作的方式进行自我表达是有挑战的;
% 如何支持治疗师定制家庭作业练习并且追踪利用家庭作业历史数据是有挑战的。
% In HCI,很少有研究探索如何使用技术媒介支持艺术治疗家庭作业的上述挑战。
% 我们present \name{}, human-ai co-creative art-making and conversational interaction 支持了xxx,包含了client application ; therapist application,新手的agent总结
%为期一个月与24个用户在5个治疗师的引导使用了我们的系统引导下进行了实践,
% 研究发现了
%艺术治疗家庭作业是一个通过艺术创作+语言表达在不同地方和时间给予来访者很多艺疗愈好处的between session activities
%==================================================
% 
%门槛高缺乏指导 - 从client角度。
%直接用technology
% customize AI agent 来taiolor homework
% 24用户自己的context进行在五个治疗师指导下
% agent指导homework, art-making同时对话探索image的meanings,作为self-exploration过程
% 自己专业信念和自己的疗愈经验,自己的对话融入到chatbot拥有治疗师的个人印记
%Art therapy homework is between-session activities that facilitate  reflecting on clients' daily feelings and enhance therapeutic collaboration through art-making and verbal expression in their natural environment.
%However, it is challenging to support clients in self-expression through verbalization and art due to a high threshold and limited guidance from therapists, as well as to support therapists in customizing and tracking homework exercises.
%In HCI, few studies have explored how technology can support addressing these challenges of art therapy homework.
%Thus, we present \name{}, a system consisting of a client application that integrates human-AI co-creative art-making and conversational agents for art therapy homework, and a therapist application that allows for the customization of conversational agents to guide homework and review AI-compiled homework data.
%Over a one-month field deployment, 24 clients engaged with \name{} in their own contexts, under the guidance of 5 therapists. Our findings showed that introducing conversational agents while art-making can guide clients in exploring personal feelings and new meanings behind the artwork. Also, \name{} enables therapists to integrate their professional beliefs and practical experiences into the AI agent, allowing it to reflect their unique personal imprint. Homework images and conversation records also helped therapists offer triggers for deeper discussions and intervention resources during in-session activities.
%Building on these findings, we explore the practical implications of incorporating AI agents into therapy homework on future design.
%Art therapy homework is essential for fostering clients' reflection on daily experiences between sessions. However, current practices present challenges: clients often lack guidance for completing tasks that combine art-making and verbal expression, while therapists find it difficult to track and tailor homework. How HCI systems might support art therapy homework remains underexplored. To address this, we present \name{}, comprising a client-facing application leveraging human-AI co-creative art-making and conversational agents to facilitate homework, and a therapist-facing application enabling customization of homework agents and AI-compiled homework history. A 30-day field study with 24 clients and 5 therapists showed how \name{} supported clients’ homework and reflection in their everyday settings. Results also revealed how therapists infused their practice principles and personal touch into the agents to offer tailored homework, and how AI-compiled homework history became a meaningful resource for in-session interactions. Implications for designing human-AI systems to facilitate asynchronous therapist-client collaboration are discussed.


%can help clients express emotions and find new meanings behind their artwork, even without therapists. Additionally, 

%Therapy homework has been widely used in art therapy practices to enhance clients' therapeutic skills in real world and offer continuity between sessions.
% 
%Yet, few studies explored how AI agents could support clients’ art therapy homework and mediate therapist-client collaboration outside the synchronous art therapy sessions.
%
%To address this, we introduce \name{}, a multi-agent system that combines image-based generative AI with conversational AI agents, empowering clients to engage in art therapy homework in natural environments. It can also support therapists customizing and reviewing therapy homework.
%Through one month field deployments involving 24 clients with five art therapists, our results demonstrated that \name{} helped clients externalize their emotions and create new meanings through conversation and art-making across various contexts, such as home or offices.
%AI multi-agents can act as a mediator, enabling therapists to customize them as an extension of their practice to deliver structured guidance aligned with their professional beliefs, while also supporting the use of homework as intervention resources during art therapy sessions.
\end{abstract}

%%
%% The code below is generated by the tool at http://dl.acm.org/ccs.cfm.
%% Please copy and paste the code instead of the example below.
%%
\begin{CCSXML}
<ccs2012>
   <concept>
       <concept_id>10003120.10003121.10011748</concept_id>
       <concept_desc>Human-centered computing~Empirical studies in HCI</concept_desc>
       <concept_significance>500</concept_significance>
       </concept>
 </ccs2012>
\end{CCSXML}

\ccsdesc[500]{Human-centered computing~Empirical studies in HCI}



%%
%% Keywords. The author(s) should pick words that accurately describe
%% the work being presented. Separate the keywords with commas.
\keywords{Vocabulary learning, video-based learning, children, cartoon video, visual language model, large language model}


%% teaser figure 

\begin{teaserfigure}
  \includegraphics[width=\textwidth]{images/shoutu.pdf}
  \vspace{-7mm}
  \caption{\name{} facilitates families to learning vocabulary through real-time question-answering, active review with questions and feedback,real-world associations by linking animated and real-life images, and contextual expansion with sentences or stories.}
  \Description{Enjoying the baseball game from the third-base
  seats. Ichiro Suzuki preparing to bat.}
  \label{fig:teaser}
\end{teaserfigure}


%%
%% This command processes the author and affiliation and title
%% information and builds the first part of the formatted document.

\maketitle

\section{Introduction}

Despite the remarkable capabilities of large language models (LLMs)~\cite{DBLP:conf/emnlp/QinZ0CYY23,DBLP:journals/corr/abs-2307-09288}, they often inevitably exhibit hallucinations due to incorrect or outdated knowledge embedded in their parameters~\cite{DBLP:journals/corr/abs-2309-01219, DBLP:journals/corr/abs-2302-12813, DBLP:journals/csur/JiLFYSXIBMF23}.
Given the significant time and expense required to retrain LLMs, there has been growing interest in \emph{model editing} (a.k.a., \emph{knowledge editing})~\cite{DBLP:conf/iclr/SinitsinPPPB20, DBLP:journals/corr/abs-2012-00363, DBLP:conf/acl/DaiDHSCW22, DBLP:conf/icml/MitchellLBMF22, DBLP:conf/nips/MengBAB22, DBLP:conf/iclr/MengSABB23, DBLP:conf/emnlp/YaoWT0LDC023, DBLP:conf/emnlp/ZhongWMPC23, DBLP:conf/icml/MaL0G24, DBLP:journals/corr/abs-2401-04700}, 
which aims to update the knowledge of LLMs cost-effectively.
Some existing methods of model editing achieve this by modifying model parameters, which can be generally divided into two categories~\cite{DBLP:journals/corr/abs-2308-07269, DBLP:conf/emnlp/YaoWT0LDC023}.
Specifically, one type is based on \emph{Meta-Learning}~\cite{DBLP:conf/emnlp/CaoAT21, DBLP:conf/acl/DaiDHSCW22}, while the other is based on \emph{Locate-then-Edit}~\cite{DBLP:conf/acl/DaiDHSCW22, DBLP:conf/nips/MengBAB22, DBLP:conf/iclr/MengSABB23}. This paper primarily focuses on the latter.

\begin{figure}[t]
  \centering
  \includegraphics[width=0.48\textwidth]{figures/demonstration.pdf}
  \vspace{-4mm}
  \caption{(a) Comparison of regular model editing and EAC. EAC compresses the editing information into the dimensions where the editing anchors are located. Here, we utilize the gradients generated during training and the magnitude of the updated knowledge vector to identify anchors. (b) Comparison of general downstream task performance before editing, after regular editing, and after constrained editing by EAC.}
  \vspace{-3mm}
  \label{demo}
\end{figure}

\emph{Sequential} model editing~\cite{DBLP:conf/emnlp/YaoWT0LDC023} can expedite the continual learning of LLMs where a series of consecutive edits are conducted.
This is very important in real-world scenarios because new knowledge continually appears, requiring the model to retain previous knowledge while conducting new edits. 
Some studies have experimentally revealed that in sequential editing, existing methods lead to a decrease in the general abilities of the model across downstream tasks~\cite{DBLP:journals/corr/abs-2401-04700, DBLP:conf/acl/GuptaRA24, DBLP:conf/acl/Yang0MLYC24, DBLP:conf/acl/HuC00024}. 
Besides, \citet{ma2024perturbation} have performed a theoretical analysis to elucidate the bottleneck of the general abilities during sequential editing.
However, previous work has not introduced an effective method that maintains editing performance while preserving general abilities in sequential editing.
This impacts model scalability and presents major challenges for continuous learning in LLMs.

In this paper, a statistical analysis is first conducted to help understand how the model is affected during sequential editing using two popular editing methods, including ROME~\cite{DBLP:conf/nips/MengBAB22} and MEMIT~\cite{DBLP:conf/iclr/MengSABB23}.
Matrix norms, particularly the L1 norm, have been shown to be effective indicators of matrix properties such as sparsity, stability, and conditioning, as evidenced by several theoretical works~\cite{kahan2013tutorial}. In our analysis of matrix norms, we observe significant deviations in the parameter matrix after sequential editing.
Besides, the semantic differences between the facts before and after editing are also visualized, and we find that the differences become larger as the deviation of the parameter matrix after editing increases.
Therefore, we assume that each edit during sequential editing not only updates the editing fact as expected but also unintentionally introduces non-trivial noise that can cause the edited model to deviate from its original semantics space.
Furthermore, the accumulation of non-trivial noise can amplify the negative impact on the general abilities of LLMs.

Inspired by these findings, a framework termed \textbf{E}diting \textbf{A}nchor \textbf{C}ompression (EAC) is proposed to constrain the deviation of the parameter matrix during sequential editing by reducing the norm of the update matrix at each step. 
As shown in Figure~\ref{demo}, EAC first selects a subset of dimension with a high product of gradient and magnitude values, namely editing anchors, that are considered crucial for encoding the new relation through a weighted gradient saliency map.
Retraining is then performed on the dimensions where these important editing anchors are located, effectively compressing the editing information.
By compressing information only in certain dimensions and leaving other dimensions unmodified, the deviation of the parameter matrix after editing is constrained. 
To further regulate changes in the L1 norm of the edited matrix to constrain the deviation, we incorporate a scored elastic net ~\cite{zou2005regularization} into the retraining process, optimizing the previously selected editing anchors.

To validate the effectiveness of the proposed EAC, experiments of applying EAC to \textbf{two popular editing methods} including ROME and MEMIT are conducted.
In addition, \textbf{three LLMs of varying sizes} including GPT2-XL~\cite{radford2019language}, LLaMA-3 (8B)~\cite{llama3} and LLaMA-2 (13B)~\cite{DBLP:journals/corr/abs-2307-09288} and \textbf{four representative tasks} including 
natural language inference~\cite{DBLP:conf/mlcw/DaganGM05}, 
summarization~\cite{gliwa-etal-2019-samsum},
open-domain question-answering~\cite{DBLP:journals/tacl/KwiatkowskiPRCP19},  
and sentiment analysis~\cite{DBLP:conf/emnlp/SocherPWCMNP13} are selected to extensively demonstrate the impact of model editing on the general abilities of LLMs. 
Experimental results demonstrate that in sequential editing, EAC can effectively preserve over 70\% of the general abilities of the model across downstream tasks and better retain the edited knowledge.

In summary, our contributions to this paper are three-fold:
(1) This paper statistically elucidates how deviations in the parameter matrix after editing are responsible for the decreased general abilities of the model across downstream tasks after sequential editing.
(2) A framework termed EAC is proposed, which ultimately aims to constrain the deviation of the parameter matrix after editing by compressing the editing information into editing anchors. 
(3) It is discovered that on models like GPT2-XL and LLaMA-3 (8B), EAC significantly preserves over 70\% of the general abilities across downstream tasks and retains the edited knowledge better.
\section{Formative Study}

To understand the specific requirements for text-to-SQL dataset annotation, we conducted a formative study by interviewing 5 engineers from Adobe. These interviewees have experienced annotating text-to-SQL datasets in their work.
We describe our interview process in Section~\ref{sec:interview}. Based on these interviews, we identified five major user needs in Section~\ref{sec:user_needs}. 
Finally, we discuss our design rationale in Section~\ref{sec:design}, aiming to address the user needs.

\subsection{Interview}
\label{sec:interview}

We conducted 20-minute semi-structured interviews with each interviewee through a conversational and think-aloud process. 
During these interviews, we first asked about the \textbf{motivation} for text-to-SQL annotation in their use cases, specifically about the schemas they worked on and why obtaining more data was important.
Interviewees reported that when deploying a new service, companies often needed to introduce new entities and restructure the original schema.
However, after updating the schema, they typically found that model performance dropped dramatically. Their regression tests showed an overall accuracy drop of 13.3\% for newly added columns and 9.1\% for new tables. As the schema was further updated, performance continued to decline. Moreover, as the schema changed significantly, they needed a large amount of new data on the updated schema to ensure a robust evaluation.

Second, we asked about their detailed \textbf{workflow} and whether they used any tools to assist with data annotation. Interviewees reported that they did not use any specific tool for annotation, although they sometimes asked ChatGPT to generate initial data. 
Additionally, they often leveraged previous datasets by adapting previous queries to the new schema, such as replacing an outdated column name with a new one.
After annotation, their colleagues performed peer reviews to check and refine the data.





Third, we asked about \textbf{challenges} they had met and the speed of their dataset annotation. Overall, they considered annotation to be very expensive. 
Interviewees mentioned that one engineer could only annotate 50 effective SQL and NL pairs per day in their use case. 
They often lost track and felt overwhelmed during annotation. 
Despite the peer review, they still felt a lack of confidence in the quality of the annotated data. 
They pointed out that randomness existed throughout the entire procedure. 
We summarize more challenges as user needs in Section~\ref{sec:user_needs}.






\subsection{User Needs}
\label{sec:user_needs}

\noindent \textbf{\textit{N1: Effective Schema Comprehension.}} 
Text-to-SQL annotation assumes that users can easily understand the database schema specified in a certain format (e.g., Data Definition Language). However, our interviews indicate that it is cumbersome and error-prone for users to navigate and comprehend complex schemas from such a specification format.



\noindent \textbf{\textit{N2: Creating New Queries.}}
Creating SQL queries requires a deep understanding of both database schema and SQL grammar. When creating a text-to-SQL dataset, users need to continually come up with new, diverse SQL queries. However, it is challenging for them to break free from preconceptions shaped by existing queries they have seen before.

\noindent \textbf{\textit{N3: Detecting Errors in the Annotated Data.}} 
An annotated dataset may include errors, which can deteriorate model performance and evaluation results.
Our interviews suggest that annotators need an effective mechanism for detecting potential errors or ambiguity in the constructed queries.


\noindent \textbf{\textit{N4: Efficiently Correcting the Detected Errors.}} After identifying errors, users need an efficient way to correct these errors to ensure the accuracy and reliability of the dataset. They need to ensure the SQL query is syntactically correct, and the NL is semantically equivalent to the SQL query.
%However, correcting errors in the annotated data is often as labor-intensive as annotating new data.


\noindent \textbf{\textit{N5: Improve Dataset Diversity.}}
Dataset diversity is crucial for improving model performance and ensuring rigorous evaluation.
Human annotation can easily introduce biases due to individual knowledge gaps and a lack of holistic understanding of the dataset composition. 
Thus, interviewees reported the need for an effective way to improve diversity and eliminate biases in the dataset. 




\subsection{Design Rationale}
\label{sec:design}


To support \textbf{N1}, our approach visualizes the database schema as a dynamic, editable graph. This enables users to quickly grasp the overall structure of the database and the relationships between entities. Users can explore detailed information such as data type through further interactions with the graph.


To support \textbf{N2}, our approach alleviates the burden of manually creating new SQL queries. We design an algorithm to randomly sample SQL query templates based on SQL grammar, then fill out this template with entities and values retrieved from the database. We make the SQL generation highly configurable---users can manually adjust keyword probability, or automatically tune the probability by an existing dataset.

To support \textbf{N3}, our approach renders the alignment between the SQL query and the NL question via a step-by-step analysis. Our approach then prompts the LLM to highlight potential misalignments to users.
Subsequently, our approach performs a textual analysis to check the equivalence of the SQL query and NL question and offers users a confidence score about their consistency.

To support \textbf{N4}, we handle two common errors---missing information and including irrelevant information in the NL question. Our approach allows users to fix errors by injecting missing information or removing irrelevant details based on LLM-generated suggestions.

To support \textbf{N5}, our approach first enables users to sample SQL queries based on a probability distribution learned from real-world data rather than creating them manually. 
Furthermore, our approach supports visualizing various dataset compositions through diagrams. 
For example, users can view a bar chart displaying the distribution of column counts in SQL queries. This feature allows users to monitor dataset composition during annotation, maintaining control over the annotation direction and improving data diversity.


\section{\name{} SYSTEM}
This section covers the design and implementation of \name{}. We first present the core design features developed based on our contextual understanding, followed by a typical usage scenario and the technical implementation details.

\begin{figure*}[tb]
  \centering
  \includegraphics[width=\linewidth]{images/UI.png}
  \vspace{-4mm}
  \caption{\name{} consists of the client-facing application and the therapist-facing application (is presented in translated English Version): (1) the client-facing application includes the Art-making Phrase interface and the Discussion Phrase interface; (2) the therapist-facing application includes Homework Agents' Customization interface and AI-compiled Homework Interface}
  \Description{Figure 2 presents two applications: the client-facing application and the therapist-facing application. The left side of the figure shows the client-facing application, while the right side displays the therapist-facing application.
  In the client-facing application, the upper section represents the "Art-Making Phase" interface. This interface includes a conversational agent, an art-making canvas for the user, a co-creative preview canvas, and tools that allow clients to select different brushes. The lower section is the "Discussion Phase" interface, which contains a human-AI co-creative artwork and a conversational agent that enables users to engage in dialogue with it.
  In the therapist-facing application, the upper section shows the "AI-Compiled Homework History" interface, where therapists can review the client's homework history. This interface includes the client's personal information, original creations, the creation process, the final artwork, conversation records, and two questions summarized by the AI assistant based on the dialogue history. The lower section displays the "Homework Agents Customization" interface, which allows therapists to set homework tasks, modify or add principles and sample questions for the conversational agents, and write personal messages for the client.}
  \label{fig:ui}
\end{figure*}


\begin{figure*}[tb]
  \centering
  \includegraphics[width=\linewidth]{images/system_implementation.png}
  \vspace{-4mm}
  \caption{Overview of the \name{} system architecture}
  \Description{
Figure 4 illustrates the system architecture. On the left, the client-side focuses on two main functions: art-making and conversation. The art-making function is powered by ControlNet for color segmentation and the Stable Diffusion model. Additionally, an art-making agent helps clients describe their artwork, summarizing these descriptions into prompts, which are then fed into the Stable Diffusion model. The conversation function is built on large language model agents, and the system supports voice interactions between clients and agents through speech-to-text and text-to-speech APIs.
On the right, the therapist-side also has two main functions: customizing homework agents and tracking homework history. The customizing homework agents feature allows therapists to set personalized homework tasks and define the principles for the conversational agents. These customizations are integrated into the conversational agents. Additionally, summary agents are used to compile and summarize the client's homework data.}
  \label{fig:system}
\end{figure*}

\subsection{\name{} Core Design Features}
To understand how a human-AI system could support both clients' homework (\textbf{RQ1}) and therapist-client collaboration surrounding it (\textbf{RQ2}), we have designed \name{}, which consists of: 
(1) a client-facing application that combines human-AI co-creative art-making with conversational interaction to facilitate homework in daily settings, and (2) a therapist-facing application that offers AI-compiled homework history and customization of client homework agents for tailored guidance (see \autoref{fig:ui}). To address three key challenges we identified in CONTEXTUAL UNDERSTANDING, we collaborated closely with the therapists to develop the following core design features of \name{} (\textbf{DF1}–\textbf{DF3}):


%According to our three common challenges from our context study, we identified three core design features:
\subsubsection{\textbf{DF1}: Combining Human-AI Co-creative Art-making with Conversational Interaction (Client-facing Application)}
To address \textbf{CH1}, the client-facing application leverages a human-AI co-creative canvas (\autoref{fig:ui} (a)) to lower the art-making threshold for clients, and a two-phase conversation workflow to provide clients with structured guidance in both the ``Art-making Phase'' and the ``Discussion Phase''. It features three panels: 
\begin{itemize}
    \item \textit{Human-AI Co-creative Canvas} including various AI brushes~(\autoref{fig:ui}~(a)-\circled{6}) that enable users to draw color-coded segments~(\autoref{fig:ui} (a)-\circled{4}) with each color mapped with a semantic concept, and translate these user-drawn forms into concrete objects, as previewed in~(\autoref{fig:ui}~(a)-\circled{5}). The rationale for enabling AI brushes is that clients can choose brushes representing a variety of semantic concepts, and they can create color-coded forms to directly control the shapes of the corresponding object. Their usage of the semantic brushes can also help therapists understand the conceptual elements of the artworks, which might project client's personality and emotions~\cite{malchiodi2007art}.
    \item \textit{Art-making Phase Conversation} is designed to encourage clients' self-expression by prompting them to verbally describe their artistic concepts as they create color-coded segments using the AI brush. These verbal descriptions are then summarized into text prompts, providing greater control over the generated images, as illustrated in \autoref{fig:ui} (a)-\circled{1} and \circled{2}.
    \item \textit{Discussion Phase Conversation} has been suggested by our therapists to facilitate deliberate self-exploration and reflection right after art-making through multi-round conversations, following our task instructions~(\autoref{fig:ui}~(b)-\circled{1}). The dialogue principles and example questions in the default task instructions came from the art therapy literature~\cite{buchalter2017250,buchalter2004practical,buchalter2009art} and were later customized by the art therapists.
\end{itemize}

\subsubsection{\textbf{DF2}: Supporting the Customization of Client Homework Agents (Therapist-facing Application)} 

To address \textbf{CH2}, customizing homework agents in the therapist-facing application includes three features:
\begin{itemize}
    \item \textit{Dialogue Customization} allows therapists to create, modify, or reorder dialogue principles and their example questions, controlling the dialogue flow of client's conversational agent in Discussion Phase. This enables the therapist to tailor the homework conversation (especially in the Discussion Phase) based on the therapist's professional belief and their understanding about the needs of a specific client~(\autoref{fig:ui}~(d)-\circled{1}).  
    \item \textit{Homework Task Customization} is designed to support therapists in tailoring homework tasks based on their understanding of the client~(\autoref{fig:ui}~(d)-\circled{2}), with these tasks then displayed in the client-facing application's dialog display area~(\autoref{fig:ui}~(a)-\circled{2}); 
    \item \textit{Opening Message Customization} was required by our therapists for tailoring the greeting message to a client, which are displayed in the dialog area of their client-facing application~(\autoref{fig:ui}~(d)-\circled{3}). These messages offer personalized encouragement and emotional support during art therapy homework.
\end{itemize}


%This allows therapists to shape the structured guidance provided by the conversational agent;


\subsubsection{\textbf{DF3}: Enabling AI-compiled Homework History (Therapist-facing Application)}
To address \textbf{CH3}, a AI-compiled homework history interface is designed with three panels:
\begin{itemize}
    \item \textit{Homework Overview} including personal information, a visualization of usage and a short AI-compiled summary of each session~(\autoref{fig:ui}~(c)-\circled{2}, \circled{3} and \circled{4}); 
    \item \textit{Homework Records} show different outcomes of a homework session, including client-drawn color segments (\autoref{fig:ui}~(c)-\circled{5}), the client-AI co-created artwork~(\autoref{fig:ui}~(c)-\circled{7}), the art-making process~(\autoref{fig:ui}~(c)-\circled{6}), and the client-AI conversation history~(\autoref{fig:ui}~(c)-\circled{8}); 
    \item \textit{AI Assistant Summary} eases therapists' review of a client's homework session by summarizing based on the client's verbal inputs and art-making outcomes. It highlights content, experiences, feelings, and reflections expressed during the homework session (\autoref{fig:ui}(c)-\circled{1}). Following therapists' suggestions, it avoids interpretations and instead poses questions from a novice therapist's perspective, pointing out potential relevant aspects to aid the therapist's review.
    
%for summarizing clients' image descriptions, feelings, and experiences, offering therapists insights.
\end{itemize}



\subsection{\name{} Usage Scenario}

Here we present a typical usage scenario of \name{}: Alice had been struggling with her relationship, feeling misunderstood by her partner, so she booked online art therapy sessions with Jessica, her therapist.
So she decided to book online art therapy activities with Jessica, her art therapist.
After their first session, Jessica used \name{}'s therapist-facing application.
Drawing from her experience, she incorporated dialogue principles and example questions into the system (\autoref{fig:ui}~(d)-\circled{1}). Jessica first added a dialogue principle of ``guiding users to describe the overall work'' and provided a few example questions for the conversational agent to reference: e.g., ``I would love to hear how you describe this work.'' Similarly, she added a few other principles and adjusted their order. The conversational agent can then follow these principles and examples in the given structure.
%to guide the conversational agent in asking the right questions~(\textbf{DF2}). 
Further, Jessica also tailored homework tasks related to couple relationships (\autoref{fig:ui}~(d)-\circled{2}) and added a supportive message, ``Your sensitivity and ability to put yourself in others' shoes are truly a gift'' (\autoref{fig:ui}~(d)-\circled{3}), to offer encouragement during Alice's homework~(\textbf{DF2}).

Another day, after a quarrel with her boyfriend, Alice felt overwhelmed and turned to the homework Jessica had assigned using \name{}: drawing two plants, one representing herself and the other her partner. She opened the client-facing app and entered the ``Art-making Phase''~(\textbf{DF1}) on her tablet. She selected the \textit{Tree} brush from the toolbox~(\autoref{fig:ui}~(a)-\circled{6}) and began drawing a tree on the canvas, while the art-making agent prompted her to describe it through voice input~(\autoref{fig:ui}~(a)-\circled{2}). 
Alice described the tree as an apple tree full of blossoms, and the agent summarized text prompts about the current creation based on her input. 
She then added more objects like \textit{Soil}, \textit{Cloud}, and \textit{Grass} and completed her artwork. After selecting the \textit{Watercolor Painting}~(\autoref{fig:ui}~(a)-\circled{3}) style,
she clicked "Generate". The color segments drawn by the client, combined with the text prompts, served as inputs to produce a human-AI collaborative artwork that deeply resonated with her emotions.

Moving into the ``Discussion Phase'', the conversational agent guided Alice in reflecting on her artwork, asking questions like, ``Would you like to describe your tree?'' Alice shared her feelings and realized that she and her boyfriend were like two different plants—independent yet needing to understand each other.

Before their next session, Jessica reviewed Alice's homework history through the therapist-facing interface~(\textbf{DF3}). She examined the original and co-created artworks, along with the conversation records, gaining a deeper understanding of Alice's situation. Noticing that the AI assistant had prompted her to consider Alice's experience of arguing with her partner~(\autoref{fig:ui}~(c)-\circled{1}), Jessica decided to address these insights during their upcoming session. After revisiting Alice's homework sessions, Jessica decided to moved the principle ``naming the artwork'' from the second to the end of the dialogue. She believes this would help Alice better articulate her thoughts.






% She opened the client-facing application and entered the “Art-making Phase”~(\textbf{DF1}) in her browser on her tablet. First, she selected the \textit{Tree} brush from the toolbox~(\autoref{fig:ui}~(a)-(6)). 
% She then drew a tree on her canvas using the \textit{Tree} brush, while the art-making agent simultaneously prompted her to describe the tree she had created through voice interaction~(\autoref{fig:ui}~(a)-(2)).
% Alice described that \textit{Tree} is an apple tree full of apple blossoms through audio.
% At the same time, the agent summarized a prompt based on her description of \textit{Tree}.
% Then, she utilized \textit{Tree}, \textit{Soil}, \textit{Cloud} and \textit{Grass} brushes to create an artwork.
% After completing the artwork and describing the various artistic elements in words, she clicked ``\textit{Use This Description}'' button. All of her verbal descriptions of the artwork were displayed in \autoref{fig:ui}~(a)-(1) , where they can be modified and updated. Then, she selected the artistic style of \textit{Watercolor Painting}~(autoref{fig
% }~(a)-(3)), and finally clicked the ``\textit{Generate}'' button.
% A human-AI co-creative artwork was shown on the preview canvas in \autoref{fig:system}~(a)-(5). 
% Alice loved this artwork because it captured and externalize her emotions using two \textit{Tree} brushes. 

% She then moved into the "Discussion Phase" to facilitate deep self-exploration and reflection on the artwork through an AI conversational agent~(\textbf{DF1}). 
% First, the AI conversational agent follows Alice's customized conversation principles to guide clients in exploring the thoughts and meanings behind their artworks, e.g., describing the overall artwork~(\qt{would you like to describe your tree?}).
% Alice shared many of her feelings and experiences with the AI agent and came to the realization that she and her boyfriend were like two different plants—both independent individuals who needed to understand and accept each other.

% Before the next online session with Jessica, the therapist Jessica reviewed Alice's homework history through the therapist-facing interface~(\textbf{DF3}). 
% She reviewed the client's original and co-creative artworks, along with the conversation records, to provide the therapist with a deeper understanding of the client's needs. 
% Additionally, she noticed that the AI assistant, much like an apprentice, inquired whether the therapist had considered the client’s experience of arguing with her boyfriend~(\autoref{fig:ui}~(c)-(1)). She decided to discuss the insights she had discovered during their upcoming session.


%\subsection{\name{} Usage Scenario}The \name{} system is designed with two distinct user interfaces, tailored to the roles of the visitor and the therapist, respectively. Both interfaces are web-based and optimized for mobile devices, ensuring functionality across various devices including tablets, laptops, and desktops.\subsubsection{\textbf{Visitor Interface: }}The visitor interface consists of two main components:\textit{1. Drawing Module:} This includes tools such as a canvas and brushes. Users create artwork which is then processed by the system to generate results.\textit{2. Dialogue Module: } Includes a chatbot dialogue box that facilitates communication between the user and the system. The use of this interface is divided into two sequential stages.  \textit{In the first stage,} the user paints with the drawing module and verbally describes their artwork to the chatbot. The system automatically summarises the user's description alongside the visual content. Based on the user's satisfaction with the generated artwork, they can either proceed to the second stage or request changes, which can be self-directed or managed by an AI re-generation process. \textit{In the second stage,} the chatbot, customised by the therapist, interacts dynamically with the user based on the thematic content of the user's drawing. \subsubsection{\textbf{Therapist Interface: }}The therapist interface consists of several functional modules:
%\textit{1. Drawing Records Viewing Module:} Therapists can access comprehensive records of user sessions. This module displays the user's drawing activities, including the drawing process, final results, and accompanying dialogue transcripts.
%\textit{2. Task and Messaging Module:} Therapists can assign tasks or send messages to visitors. These messages are delivered to the visitor through the dialogue module during subsequent interactions with the system.
%Both the visitor and therapist interfaces are designed to be user-friendly, ensuring ease of use regardless of the user's technical expertise. This dual-interface approach not only enhances the therapeutic process but also facilitates a more personalized interaction between the user and the system.





% % % % % % % % % % % % % % % % % % % % % % % % % % % % % % % % % 

\subsection{\name{} System Implementation}
As \autoref{fig:system} shows, the \name{} system consists of the front-end of both client-facing application and therapist-facing application, as well as the back-end including LLM Agents module, AIGC module, and the memory module.
% \subsubsection{Overview of System Architecture}

% In this section, we present in detail how we implemented the \name{} system. As shown in \autoref{fig:system}, the system consists of the client-facing application 


% two distinct user interfaces (UIs) - one for users engaged in therapeutic art making, and another for therapists monitoring and intervening in sessions. 
% The system also has a back-end server that uses a large language model (LLM) at its core service designed to support real-time interactions, support the dual-interface model, and handle various data generated during use.


\subsubsection{Front-end}

The front-end of both client and therapist applications are web-based UIs built with Quasar Framework, which is based on Vue.js, and extends its capabilities by providing a rich UI component library, tools and cross-platform development support. %需要加这句对quasar的介绍吗?
For the conversational interaction of client-facing application, client messages are processed using OpenAI's TTS model\footnote{OpenAI TTS Models, https://platform.openai.com/docs/models/tts} to generate a corresponding voice message that autoplays by default. We also implemented speech to text function using the Voice Dictation (Streaming Version)\footnote{iFLYTEK Speech-to-Text, https://global.xfyun.cn/products/speech-to-text} service from iFlytek. For the therapist-facing application, all data is stored and transmitted in JSON format, interacting with the back-end server via HTTP requests, allowing therapists to select a client from a list of names, and view the session logs, assign customized homework, and modify the agent's guidelines. 

% As mentioned earlier, the client-facing application is designed with two phases. The \textit{"Art-making Phase"} and the \textit{"Discussion Phase"}. Both phases feature a Dialogue Module, which was custom-built to suit our specific needs. In this module, each AI response returns a set of interactive functions. For example, dialogue text is processed using OpenAI's TTS model\footnote{OpenAI TTS Models, https://platform.openai.com/docs/models/tts} to generate a corresponding voice message that autoplays by default. Users also have the option to play or pause the message by clicking the 'PLAY THE VOICE' button. Additionally, when the AI generates a summary of the drawing log to assist with the prompt, it provides a 'USE THIS DESCRIPTION' button, allowing users to directly apply or modify the AI-generated prompt. We also implemented speech to text function using the Voice Dictation (Streaming Version)\footnote{iFLYTEK Speech-to-Text, https://global.xfyun.cn/products/speech-to-text} service from iFlytek.

%The \textit{"Art-making Phase"} also includes a drawing module with a 512x512 canvas and a variety of color-segmentation brushes. This module supports freehand drawing while recording the entire drawing process. Upon completion, the system analyzes each pixel to match the colors used to their respective brushes, optimizing performance to avoid latency and ensure a smooth experience for therapeutic purposes. When the "Generate" button is clicked, the system sends the user's drawing, prompt, drawing log, and dialogue history from the \textit{"Art-making Phase"} to the back-end via HTTP, where they are stored along with the generated image.

% \subsubsection{Frontend: Therapist Interface}



\subsubsection{Back-end}

The backend is implemented using Flask, responsible for generating images, processing LLM workflows, and managing data. 
%表明我们的模型没有训练
The image generation model pipeline uses \textit{ControlNetModel} and \textit{StableDiffusionControlNetPipeline} from huggingface's diffusers\cite{von-platen-etal-2022-diffusers}, and the base model is \textit{runwayml/stable-diffusion-v1-5}, the control net segmentation model is \textit{lllyasviel/control\_v11p\_sd15\_seg}\footnote{Controlnet - v1.1 - seg Version, https://huggingface.co/lllyasviel/control\_v11p\_sd15\_seg}. The system uses asynchronous, queue-based processing with multi-threading to efficiently handle multiple concurrent image generation requests. 

The backend for the LLM workflows consists of five prompted LLM Agents, all utilizing GLM-4\footnote{GLM-4 GitHub, https://github.com/THUDM/GLM-4}. 
The first LLM, refer to "Art-making Agent" of \autoref{fig:system}-\circled{2}, is utilized during the Art-making Phase of the client app, where it summarizes users' drawings and descriptions into prompts for the Stable Diffusion model. The second LLM, refer to "Conversation Agent" of \autoref{fig:system}-\circled{2}, is employed during the Discussion Phase, using a template that incorporates principles and customized questions from therapists. The third and fourth LLMs, refer to "Conversation Agent" of \autoref{fig:system}-\circled{2}, process data from the Art-making and Discussion Phases, respectively, and condense the information into short summaries for the therapist. The fifth LLM, also refer to "Conversation Agent" of \autoref{fig:system}-\circled{2}, integrates dialogue from both phases and generates three therapist-focused questions based on the conversation history. All the prompt are included in the supplementary materials of the paper.

These services communicate via RESTful APIs, the entire backend service has 15 APIs, ensuring the functionality and a smooth front-end and back-end interaction.
The back-end handles data storage and retrieval, managing user-generated content including AI-generated and original artwork, creative process data, homework dialogue data, color segmentation area data, and customized homework data. Images are stored in PNG format, while logs and settings are saved in JSON format.



% api的个数,保存的数据格式。













% [
%     {
%         ‘description‘:’[user-drawn] I drew the TV. [canvas content] The canvas now has the building, the TV.’ ,
%         ‘time": ’11:11:05’
%     },
%     {
%         ‘description‘:’[user-drawn] I drew the table. [canvas content] The canvas now has buildings, a TV, and a table.’ ,
%         ‘time": ’11:11:40’
%     },
%     {
%         ‘description‘:’[user-drawn] I drew the seats. [canvas content] The canvas now has buildings, TVs, tables, and seats.’ ,
%         ‘time": ’11:12:07’
%     },
% ]

\begin{table*}[tb]
\centering
\caption{Demographics of Participant Clients: Previous Art Therapy Sessions indicates the number of times the client has previously participated in art therapy; Familiarity with Traditional Drawing reflects the client's level of experience with traditional drawing techniques (0-not familiar; 1-very familiar); Familiarity with Digital Drawing reflects the client's level of experience with digital drawing techniques (0-not familiar; 1-very familiar); Participation Purposes reflects the reasons clients choose to engage in the activity.}
\vspace{-3mm}
\label{tab:clients}
\small
\resizebox{1\linewidth}{!}{
\begin{tabular}{cccccccccc}
\toprule
\textbf{ID} & \textbf{Gender} & \textbf{Age} & \textbf{Education} & \textbf{Region} & \parbox[t]{2.5cm}{\centering\textbf{Previous Art Therapy Sessions}} & \parbox[t]{3cm}{\centering\textbf{Familiarity with Traditional Drawing}} & \parbox[t]{2cm}{\centering\textbf{Familiarity with Digital Drawing}} & \parbox[t]{2cm}{\centering\textbf{Therapist Assignment}} & \parbox[t]{2.5cm}{\centering\textbf{Participation Purposes}} \\
\midrule
C1  & Female & 37  & Bachelor's & China/Shanghai & 0                            & 1                                   & 0.25  &T3 & Personal Growth                   \\
C2  & Female & 35  & Bachelor's & China/Shenzhen & 3                            & 0.5                                   & 0.5   &T3 & Career Development and Family                 \\
C3  & Female & 28  & Master's   & China/Hebei    & 2                            & 0.75                                  & 0.75   &T3  & Family and Emotional Management                \\
C4  & Female & 36  & Bachelor's & China/Beijing  & 10                           & 0.75                                   & 0   &T3  &Career Development                \\
C5  & Male   & 28  & Master's   & Germany       & 0                            & 1                                   & 0.75   &T3   &  Emotional Management and Personal Growth                       \\
C6  & Other  & 26  & Associate's & China/Heilongjiang & 1                            & 0.5                                   & 0.25  &T5  & Emotional Exploration and Intimate Relationships                           \\
C7  & Female & 23  & Master's   & China/Shanghai & 0                            & 1                                   & 1     &T5     &  Intimate Relationships                    \\
C8  & Female & 20  & Bachelor's & China/Shenzhen & 0                            & 0.5                                   & 0.5    &T5   &  Emotional Management and Intimate Relationships                       \\
C9  & Female & 25  & Bachelor's & China/Guangxi  & 4                            & 0                                   & 0.5    &T5    &  Self-Expression and Emotional Exploration                      \\
C10 & Male   & 23  & Master's   & China/Shenzhen & 0                            & 0.75                                   & 0.5   &T5   &             Self-Expression and Social Skills             \\
C11 & Female & 26  & Master's   & China/Hangzhou & 0                            & 0.5                                   & 0.25    &T4  &        Emotional Management, Social Skills and Intimate Relationships                 \\
C12 & Female & 26  & Master's   & China/Shanghai & 2                            & 0.75                                   & 0.5    &T4   &                   Stress Relieving and Intimate Relationships  \\
C13 & Female & 30  & Master's   & China/Dalian   & 0                            & 0.5                                   & 0.25   &T4    &             Family and Emotional Management            \\
C14 & Female & 19  & Bachelor's & China/Chongqing & 0                            & 0.25                                   & 0.25   &T4  &                Personal Growth and Self-Exploration           \\
C15 & Male   & 27  & Bachelor's & China/Beijing  & 0                            & 0.25                                  & 0.25   &T4    &                 Stress Relieving and Personal Growth        \\
C16 & Female & 22  & Bachelor's & China/Shandong & 0                            & 0.5                                   & 0.25   &T1     &              Emotional Management and Social Skills       \\
C17 & Male   & 38  & Master's   & China/Sichuan  & 0                            & 0.75                                   & 0.75   &T1     &                    Personal Growth      \\
C18 & Female & 40  & Master's   & China/Beijing  & 20                           & 1                                   & 0.75    &T1      &               Stress Relieving and Emotional Management          \\
C19 & Female & 28  & Bachelor's & China/Guangzhou & 0                            & 0.5                                   & 0   &T1       &                 Future Career Planning and Personal Growth      \\
C20 & Male   & 25  & Master's   & China/Guangzhou & 0                            & 1                                   & 1   &T1        &                    Academic Pressure Relieving   \\
C21 & Male   & 24  & Master's   & China/Hubei    & 0                            & 0                                   & 0   &T2        &                Childhood Family and Dreams Exploration  \\
C22 & Female & 24  & Master's   & China/Shenzhen & 0                            & 0.25                                   & 0.25    &T2  &                Emotional Management and Personal Growth     \\
C23 & Male   & 25  & Master's   & China/Zhejiang & 10                           & 0.5                                   & 0.5    &T2   &                  Emotional Development and Self-Expression        \\
C24 & Male & 55  & Bachelor's & Dubai& 0 & 0.5& 0.5&T2 &                           Emotional Management \\
\bottomrule

\end{tabular}}
\Description{The table 2 describes 24 participants in art therapy sessions. The participants are from diverse locations, including China (Shanghai, Shenzhen, Hebei, Beijing, Heilongjiang, Guangxi, Hangzhou, Chongqing, Shandong, Sichuan, Hubei, and Zhejiang), Germany, and Dubai. The ages range from 19 to 55 years old, with varying levels of education from associate degrees to master's degrees and bachelor's degrees. Their familiarity with traditional drawing techniques ranges from no familiarity to very familiar, while their familiarity with digital drawing techniques also varies across the spectrum. The participants have attended between 0 and 20 previous art therapy sessions and are assigned to different therapists identified by codes T1 to T5.Participation Purposes reflects the reasons clients choose to engage in the activity}
\end{table*}

\section{Field study}
Using \name{} as both a novel system to study and a research tool to study with, we aim to explore how a human-AI system support clients' art therapy homework in their daily settings (\textbf{RQ1}) and how such a system could mediate therapist-client collaboration surrounding art therapy homework (\textbf{RQ2}). To this end, we conducted a field deployment involving 24 recruited clients and five therapists over the course of one month.



%参与者与实验的setup
    %参与者招募
        % 我们招募的途径:To recruit our clients, we distributed digital recruitment flyers through social media platforms.
        % 海报上描述了什么:The recruitment flyer described the art therapy activities as "promoting self-exploration using a digital software".
        % 我首先要求参与者填写pre-问卷,这个问卷主要包括descriptions of the art therapy activities, demographic information, the number of art therapy sessions they attended, familiarity with digital drawing, and specific needs for the art therapy activities.
        % Participants were included in this study with the aim of reducing stress and anxiety, fostering personal growth, improving emotional regulation, and strengthening social skills.
        % 此外,we tried to selection of participants based on their regions, occupations, the types of devices they used, and the number of times they participated in art therapy.
        % finally, 有27名参与者开始使用这个系统,其中有3名参与者drop out因为缺乏时间
\subsection{Participants and Study Procedure}
\subsubsection{Participants}

The five therapists who participated in the field evaluation were the same ones from our contextual study (see \autoref{tab:expert}). Each therapist was compensated at their regular hourly rate.
For client recruitment, we distributed digital flyers through social media platforms, describing the art therapy activities as an "online art therapy experience promoting self-exploration using a digital software." This aligns with the common goal of art therapy sessions, which are widely used to promote self-exploration for all clients, beyond treating mental illness~\cite{kahn1999art, riley2003family}.

Participants first completed a pre-questionnaire, which provided an overview of the activities and collected demographics, and prior experiences with art therapy experience and with digital drawing---to ensure that we include both novices and experienced user---and their personal goals for participation. 
The therapists guided the recruitment and screening of the the clients, and included individuals seeking for reducing stress, fostering personal growth, enhancing emotional regulation, and strengthening social skills. The therapists excluded individuals with serious mental health conditions to minimize ethical risks.
%Based on the therapists' advice, clients with goals such as reducing stress and anxiety, fostering personal growth, enhancing emotional regulation, and strengthening social skills were included, avoiding ethical concerns related to clinically diagnosed mental health conditions. 
%We also considered participants' regions, device types, drawing familiarity, and prior art therapy experience to create a balanced selection.

In total, 27 clients began using \name{}, but 3 withdrew early due to scheduling conflicts. The final group of 24 clients (C1-C24; 8 self-identified males, 15 self-identified females, 1 identifying as other; aged 19-55) completed the study (client demographics are detailed in the~\autoref{tab:clients}). Clients who completed the full process were compensated with \$37, others were compensated with a prorated fee.
Our study protocol was approved by the institutional research ethics board, and all participant names in this paper have been changed to pseudonyms. Participants reviewed and signed informed consent forms before taking part, acknowledging their understanding of the study.

% The five therapists participated in the field evaluation were the ones who also participated in our contextual study (see \autoref{tab:expert}).
% Five art therapists were compensated with their regular hourly rate.
% For the clients recruitment, we distributed digital recruitment flyers through social media platforms. 
% The recruitment flyer described the art therapy activities as ``online art therapy experience promoting self-exploration using a digital software''.
% This is due to that this is a common goal for art therapy sessions, since Art therapy activities are not only effective in treating mental illness but also widely promote self-exploration for every clients, as commonly integrated into practice~\cite{kahn1999art,riley2003family}.
% First, participants completed a pre-questionnaire that provided an overview of the art therapy activities and gathered details such as their demographics, the number of art therapy sessions they've attended, familiarity with digital drawing, and any specific needs they hoped to address.
% Following that, based on the advices from the therapists, clients were included with the goal of reducing stress and anxiety, fostering personal growth, enhancing emotional regulation, and strengthening social skills.
% The therapists suggest so since they agree that these therapeutic goals would be beneficial for eavery day therapy clients and would could It might avoid the potential ethical and safety risks associated with clinically diagnosed mental health issue.
% Further, we selected participants based on a balance of their regions, the types of devices they used, the familiarity with drawing and their prior experience with art therapy. 

% In total, 27 clients began using \name{}, but 3 withdrew from the study at the early stage due to scheduling conflicts.
% Finally, 24 clients (C1-C24; 8 self-identified males, 15 self-identified females, 1 identifying as other; aged 19-55) completed our field study. 
% APPENDIX shows the specific client demographics.
% We compensated clients based on their level of involvement, with those who completed the full one-month study receiving 200 RMB as a bonus, and clients who dropped out receiving a prorated fee according to the duration of their participation.

% Our protocol was approved by the institutional research ethics board, and all names in this paper have been changed to pseudonyms.
% Also, before participating in the activity, participants carefully reviewed and signed the informed consent form, acknowledging their understanding.

%在与治疗师协商讨论下,这些用户被分到5位治疗师(see Table),其中T2有4位来访者,其余治疗师有5位来访者。
%这个研究. .
%在活动开始前,我们邀请每位参与者开展了一场介绍session. 主要是目的是介绍活动目的与流程,并且演示如何使用\name{},并且为每位来访者可以接触到系统的URL的链接;
%介绍活动结束后,来访者被鼓励有规律地去自行探索使用\name{};
%每隔一周,我们会安排治疗师与来访者进行线上一对一的session。我们会鼓励治疗师在线上一对一session之前提前review来访者的使用数据,并通过即时通讯软件与我们交流review之后的洞见与想法。
%在线上一对一session时,在不干扰治疗师艺术治疗实践的基础上,我们鼓励治疗师在线上一对一session时利用这些数据。在艺术创作阶段,来访者可以通过分享屏幕的方式使用系统的第一个阶段进行创作并与治疗师进行讨论交流,在session快结束前治疗师会给来访者推荐家庭作业。
%在session结束后,治疗师会在治疗师系统上安排家庭作业并给予来访者的个人赠言。此外,来访者在结束线上session后可以按照治疗师的推荐完成家庭作业或者自行探索使用系统。
\subsubsection{Procedures}

Clients were distributed in coordination with the five therapists, as shown in \autoref{tab:expert}. T2 was assigned four clients, while the other therapists each had five clients. The field study consisted of two main activities: (1) three online in-session activities, where clients had one-on-one conversations and collaborated with the therapist, and (2) unstructured between-session activities, where clients practiced therapy homework using \name{} following the therapist’s recommendations.
Before the study, we held online introductory sessions to familiarize the clients with \name{}, and provided both demonstrations and hands-on exploration on their preferred devices. Similarly, we offered online training for therapists on customizing and reviewing homework, while allowing them to explore both the therapist-facing and client-facing applications. After the session, clients were encouraged to regularly explore \name{}.
Two weeks into the study, we scheduled weekly one-on-one online sessions between therapists and clients, each lasting approximately 60 minutes. Therapists were encouraged to review the clients' homework history using \autoref{fig:ui}(c) before each session. During the online session, therapists used this data to inform their practices without interrupting the flow of therapy. We encouraged clients in advance, to create artworks during the Art-making Phase~(\autoref{fig:qual_results}(a)), sharing screens and discussing their creations with the therapist, but did not interfere with the therapeutic process.

%Clients also used \autoref{fig:qual_results}(a) to create artwork, sharing their screens and discussing their creations with the therapist.

At the end of each session, therapists recommended homework tasks based on insights gained during the conversation. After the session, therapists might customize homework agents, including customizing conversational principles, assigning homework tasks, and providing personal messages through \autoref{fig:ui}~(d). Clients could then either complete the assigned homework or engage in self-exploration using \name{} between sessions.

% Clients were distributed In coordination with the five therapists, as shown by \autoref{tab:clients}: T2 was assigned with four clients, while each of the other therapists was assigned with five clients.
% The procedure for the field study consisted of two activities: (1) three online in-session activities where they have one-on-one conversation and collaboration with the therapist and (2) unstructured between-session activities where they perform therapy homework practices either upon recommendations of usage from the therapist or volunteerily use it in their daily lives.
% Before the study, we conducted an introductory session for each client to explain the activities, demonstrate how to use \name{}, and provide access to \name{} via a URL on their preferred devices.
% After the introductory session, the clients were encouraged to explore the use of \name{} on a regular basis.

% After two weeks of self-exploration, we started scheduling weekly one one-on-one online sessions between the therapists and the clients.
% Therapists were encouraged to review clients' homework history using \autoref{fig:ui}~(c) before the online session.
% During the online one-on-one session, we encouraged therapists to use these history data without interfering with their art therapy practices. 
% Also, they would utilize \autoref{fig:ui}~(a) to create their artwork by sharing their screens and discussing their artworks with therapists. 
% Before the end of the session, the therapist would recommend the homework tasks for the client based on the insights gained from the one-on-one session.
% After the online session ends, therapists would customize homework agents, including modifying or updating the conversational agent principles, assigning homework tasks and providing therapist's messages to the client through \autoref{fig:system}~(d). 
% Correspondingly, clients could either complete the homework or engage in self-exploration using \name{} between sessions.

% 对于异步session场景数据收集下,所有来访者使用系统的图像以及对话记录等日志数据以及治疗师在治疗师系统中使用定制功能的日志数据在保存在数据库中。
% 此外,我们鼓励来访者和治疗师通过即时通讯软件发送给我们images以及comments关于使用系统的实践以及感受。
% 对于线上session的场景数据收集,首先,online sessions were audio- and video-recorded.
% 此外,at the end of each online session, we conducted a 5-minute interview with therapists, mainly to collect their practices and experiences about the session.
% Upon concluding all the sessions,我们与治疗师以及来访者开展了约为30分钟的semi-structured interview to 探索ai agents如何支持艺术治疗场景的家庭作业(RQ1)以及AI agents如何mediate 治疗师与来访者合作(RQ2). We used 治疗师与来访者在 the trial period使用系统的log 数据以及他们的反馈作为stimuli 去问特定的使用实践的问题。
% With participants' consent, we recorded the interviews and transcribed them for thematic analysis.
% First, two researchers conducted collaborative inductive coding. They initially annotated the transcript to identify relevant quotes, key concepts, and recurring patterns in the data. These findings were further developed through regular discussions, leading to a detailed coding scheme aligned with the research questions. Quotes were then coded and clustered into a hierarchy of emerging themes, continually reviewed, and refined in recurrent meetings, where exemplar quotes were also selected for presenting each theme and sub-theme. 
% Also, we collected the log data from 治疗师和来访者 作为证据以及examples for the thematic analysis results.

\subsection{Data Gathering Methods} 

For between-sessions, we stored all homework-related data in a database, including artwork, dialogue, usage logs, as well as information on homework customization such as conversational principles, tasks, and personal messages.
We encouraged participants to use personal messaging (WeChat) to share pictures and comments about on-the-spot experience and feelings after homework with \name{} to compensate for semi-structured interviews.
During online sessions, we recorded audio and video. 
The researchers did not observe the therapy session in live, but reviewed post hoc, as the therapists believed a third party's presence could affect a client's emotional expression and the therapist-client dynamic.
After each session, we conducted a brief 5-minute interview with the therapists to gather their insights and feelings.

Upon the completion of the final one-on-one sessions, we conducted 30-minute semi-structured interviews with both therapists and clients. These interviews aimed to explore how \name{} supported art therapy homework in clients' daily lives (\textbf{RQ1}) and how therapists and clients collaborated surrounding art therapy homework (\textbf{RQ2}). We used feedback and homework outcomes from the trial period to ask targeted questions about their practices.
With participants' consent, we recorded and transcribed the brief 5-minute interviews and the 30-minute interviews for thematic analysis~\cite{braun2006using}. This analysis also included the personal messages shared by the participants about their on-the-spot experiences.
%we recorded and transcribed the interviews for thematic analysis. 
Two researchers then engaged in inductive coding, annotating transcripts to identify relevant quotes, key concepts, and patterns. They developed a detailed coding scheme through regular discussions, grouping quotes into a hierarchical structure of themes and sub-themes. Exemplar quotes were selected to represent each theme. We also used homework history (e.g., images or conversation data) and customization data (e.g., homework dialogue principle data) as evidences or examples to back up the findings in our thematic analysis.



% In between sessions, all homework history data~(e.g., artwork, creative process data and dialogue data) and history data on homework customization~(e.g., principles of conversational agents, homework tasks and personal messages) were stored in the database.
% In addition, we encouraged clients and therapists to send us images and comments about their experiences and feelings when using \name{} via an instant messaging app.
% For online in-sessions, the sessions were first audio- and video-recorded.
% At the end of each in-session, we conducted a brief 5-minute interview with the therapists to gather insights into their practices and feelings during the session.
% Upon concluding all the sessions, we conducted approximately 30-minute semi-structured interviews with both the therapists and the clients to explore how \name{} support art therapy homework in clients' daily settings~(\textbf{RQ1}), and how therapists tailored the homework and tracked the homework history surrounding art therapy homework~(\textbf{RQ2}). 
% Further, we employed the homework outcomes and feedback from both therapists and clients during the trial period as stimuli to ask specific questions about their practices. 

% With participants' consent, we recorded the interviews and transcribed them for thematic analysis~\cite{braun2006using}.
% Initially, two researchers engaged in collaborative inductive coding. They began by annotating the transcript to highlight relevant quotes, key concepts, and recurring patterns in the data. Through regular discussions, they expanded these insights into a detailed coding scheme that aligned with their research questions. The quotes were then systematically coded and grouped into a hierarchical structure of emerging themes, which were continuously reviewed and refined during recurring meetings. During these discussions, exemplar quotes were also chosen to represent each theme and sub-theme.
% We also gathered homework history and customization data, including artworks and conversation records from both therapists and clients, as evidence and examples to support the results of the thematic analysis.

\begin{figure*}[tb]
  \centering
  \includegraphics[width=\linewidth]{images/findings_1.png}
  \vspace{-7mm}
  \caption{Overview of The Homework Engagement of Clients with \name{}: (a) Homework Activity Date Distribution; (b) Accumulated Homework Activity Hourly Distribution of the Day; (c) Usage of AI Brushes in Artworks; 
  }
  \Description{Figure 5 contains three sub-figures. Figure 5a shows the Homework Activity Date Distribution for 24 clients over a four-week period, using seven different shades of purple to represent varying levels of participation in the homework sessions. Figure 5b illustrates the frequency of AI brush usage during clients' homework art-making, with the top 20 most frequently used brushes highlighted in larger font. Figure 5c depicts the distribution of homework sessions across different times of the day, revealing that clients tend to engage in homework sessions more frequently in the afternoon and evening.}
  \label{fig:quan_results}
\end{figure*}




\section{Results}
%Below we report quantitative results from the study sessions and survey. 
Among 54 task instances, participants successfully completed the programming task in 50 instances, passing all test cases. 
In 4 instances, the task was halted as participants did not pass all test cases within 30 minutes.
The mean task completion time was 16 minutes 46 seconds, with no significant differences across system conditions, task orders, or tasks.
To understand the effects of proactivity on human-AI programming collaboration (RQ2), we first report participants' user experience comparison between prompt-based AI tools (e.g. ChatGPT), their perceived effort of use, and the sense of disruption.
We then describe participants' evaluation of the \sys{} probe's key design features, including the timing of proactive interventions, the AI agent presence, and context management.
Analyzing the 1004 human-AI interaction episodes, we illustrate how users interacted with the AI agent under different programming processes, as well as discuss participants' preference to utilize proactive AI in different task contexts and workflows (RQ3).
We also discuss the human-AI interplay between users and different versions of the system, covering their reliance and trust towards AI, and their own sense of control, ownership, and level of code understanding while using the tools.
% From a software engineering perspective, we discuss participants' preference to use proactive AI in different programming task contexts and workflow processes (RQ3).
% Finally, we report on task-based metrics to quantitatively evaluate the three versions of the system, as well as an analysis of the 1004 human-AI interaction episodes to illustrate how users interacted and made use of the AI agent under different programming processes. 

% Potential results we can report:
% Talk about the effect of proactivity on disruption and how our design helped
% Awareness of our AI agent design and effect on collaboration experience
% Which stages of programming were most suitable for proactivity?
% Time and effort with proactivity
% CONTEXT: At which stage, which subtask, what type of AI actions are preferred by the users
% PERCEPTION: Is acceptance or usability connected to user's biases or perceptions of AI tools
% Can lead to design principles of proactive AI in different domains



\subsection{\sys{} Reduces Expression Effort and Alleviates Disruptions}
% Increased productivity, compare to base line
Overall, participants found the increased AI proactivity in the CodeGhost and \sys{} conditions led to higher efficiency (P1, P2, P13, P15, P18). 
% P2 commented \textit{``I couldn't accomplish this [task] in a short time, so that's the reason I use that [AI support].'' }
% Similarly, P15 remarked: 
% \begin{quote}
%    \textit{ ``Now that I have experienced this AI assistant, I think that the arguments about AIs are out there taking programming jobs...has some merit to it... Just for the convenience of programming, I would love to have one of these in my home. (P15)''}
% \end{quote}
Participants commented that prompt-based tools, like Github Copilot or the PromptOnly in the study, required more effort to interact with (P7, P8, P10, P12, P14).
This was due to the proactive systems' ability to provide suggestions preemptively (P7), making the interaction feel more natural (P8).
After experiencing the CodeGhost and \sys{} conditions, P10 felt that \textit{``in the third one [PromptOnly], there was not enough [proactivity]. Like I had to keep on prompting and asking.''}

% This result was supported quantitatively based on time to convey and interpret...
The proactive agent interventions also resulted in less effort for the user to interpret each AI action in both CodeGhost and \sys{} compared to in PromptOnly (Figure \ref{fig:time_convey_interpret}). 
Among 857 recorded episodes where both the user and the AI agent had at least one turn of interaction (i.e. AI responded to the user's query or the user engaged with AI proactive intervention), we observed a significant difference in the amount of time to interpret the AI agent's actions (e.g., chat messages, editor code changes, presence cues) per interaction across three conditions (\textit{F}(2,856)= 41.1, \textit{p} < 0.001) using one-way ANOVA.
Using pairwise T-test with Bonferroni Correction, we found the interpretation time significantly higher in PromptOnly ($\mu$ = 34.5 seconds, $\sigma$ = 30.1) than in CodeGhost ($\mu$ = 19.8 seconds, $\sigma$ = 17.2; \textit{p} < 0.001) and \sys{} ($\mu$ = 18.7 seconds, $\sigma$ = 14.9; \textit{p} < 0.001). 
There was no significant difference in the time to interpret between the CodeGhost and \sys{} conditions (\textit{p} = 0.398; Figure \ref{fig:time_convey_interpret}). 
This indicates that when the system was proactive, participants spent less time interpreting AI's response and incorporating them into their own code, potentially due to the context awareness of the assistance to present just-in-time help.
We did not find a significant difference in the time to express user intent to the AI agent per interaction (e.g. respond to AI intervention via chat message, in-line comment, or breakout chat) (\textit{F}(2,652) = 2.36, \textit{p} = 0.095), despite qualitative feedback that the PromptOnly without proactivity was the most effortful to communicate with.


\begin{figure}[t]

\centering
\includegraphics[width=\columnwidth]{figures/Time_to_convey_and_interpret.pdf}
%https://docs.google.com/drawings/d/1CWPtvVLIvQhpS99MhPdVq3V5_PZS9gdYvgYkQgENFII/edit
\caption{\textbf{The time to express user intentions to the AI and the time to interpret the AI response per interaction.} \textnormal{Users' expression time was not significantly different across conditions \textit{F}(2,652) = 2.36, \textit{p} = 0.095). Users' interpretation time varied (\textit{F}(2,856)= 41.1, \textit{p} < 0.001), and was significantly lower for CodeGhost and \sys{} conditions than in PromptOnly.}}
\label{fig:time_convey_interpret}
% Link to google drawing: https://docs.google.com/drawings/d/1CWPtvVLIvQhpS99MhPdVq3V5_PZS9gdYvgYkQgENFII/edit?usp=sharing
\end{figure}

While proactivity allowed participants to feel more productive and efficient, they also experienced an increased sense of disruption.
% A Chi-squared test (\textit{$\chi^2$} = 13.8, \textit{df} = 2, \textit{p} < 0.01) shows significant difference in the number of disruptions 
This was especially prominent in the CodeGhost condition, when the AI agent did not exhibit its presence and provide context management (P1, P9, P10, P14).
% Overall, introducing proactive AI support led to an increased sense of disruption.
Disruptions occurred in different patterns across the three conditions.
In PromptOnly, the scarce disruptions arose from users accidentally triggering AI responses via the in-line comments (similar to Github Copilot's autocompletion) while documenting code or making manual changes during system feedback, leading to interruptions. 
% P10 specifically disliked using comments as instructions for AI: ``I feel like when I think of comments, I think of just writing helpful little notes for myself. Like I don't see them necessarily as instructions. So I feel like it would have been a little distracting right now.''
In CodeGhost, disruptions were due to users' lack of awareness of the AI's state, leading to unanticipated AI actions while they attempted to manually code or move to another task, making interventions feel abrupt. 
For example, P14 found the lack of visual feedback on which part of the code the AI modified made the collaboration chaotic.
Similarly, P12 felt that the automatic response disrupted their flow of thinking, leading to confusion.
In \sys{}, similar disruptions occurred less frequently with the addition of AI presence and threaded interaction.
% However, the additional agent visual signals and organizations of breakout chat messages could be overwhelming and lead to disruptions.
% However, miscommunications about turn-taking between the user and AI sometimes arose, resulting in both parties acting simultaneously and causing interruptions.
% Add quantitative support

Analyzing the Likert-scale survey data (Fig. \ref{fig:survey}) using the Friedman test, participants perceived different levels of disruptions among three conditions (\textit{$\chi^2$} = 22.1, \textit{df} = 2, \textit{p} < 0.001, Fig.\ref{fig:survey} Q1), with the highest in CodeGhost ($\mu$ = 4.61, $\sigma$ = 1.58), then \sys{} ($\mu$ = 3.78, $\sigma$ = 1.86) and PromptOnly ($\mu$ = 1.56, $\sigma$ = 1.15).
Using Wilcoxon signed-rank test with Bonferroni Correction, we found higher perceived disruption in CodeGhost than PromptOnly (\textit{Z} = 3.44, \textit{p} < 0.01), and in \sys{} than PromptOnly (\textit{Z} = 3.10, \textit{p} < 0.01).
We did not find a statistically significant difference in perceived disruption between CodeGhost and \sys{} (\textit{Z} = -1.51, \textit{p} = 0.131).
The perceived disruptions in \sys{} might be due to the additional visual cues exhibited by the AI agent and the breakout chat, which we further discuss in Section \ref{Results:presence_context}.

% The results presented a multifaceted outcome of using proactive AI assistance in programming.
% On the one hand, the AI reduced users' effort to specify and initiate help-seeking, enhancing productivity and efficiency by leveraging LLM's generative capabilities.
% Meanwhile, AI's increased involvement in user workflows created disruptions, but this was alleviated to an extent by our design of \sys{}, although participants perceived level of disruption varied widely (Fig.\ref{fig:survey} Q1).
% We further discuss designs to adapt the salience of the AI presence in the user interface 
% % and improvements to the timing of service 
% in the Discussion.



\begin{figure*}[h]

\centering
\includegraphics[width=0.85\textwidth]{figures/Codellaborator_Timing_Heuristics.pdf}
%https://docs.google.com/drawings/d/1CWPtvVLIvQhpS99MhPdVq3V5_PZS9gdYvgYkQgENFII/edit
\caption{\textbf{Summary of Heuristics for Proactive Assistance Timing.} Overall, we recorded 398 instances of AI proactivity defined by our timing heuristics (Table \ref{table:proactive-features}), with 212 (53.3\%) instances leading to effective user engagement, 48 (12.1\%) instances of disruptions, and 138 (34.7\%) instances of ignored AI proactivity.}
\label{fig:timing_heuristics}
% Link to google drawing: https://docs.google.com/drawings/d/1UW-dSkpbG0QZd4AoKhKW00GZha-VE8aNPmTWE58lVqA/edit
\end{figure*}


\subsection{Measuring Programming Sub-Task Boundary Is Effective to Time Proactive AI Assistance}
% Talk about the frequency, effectiveness of each design
% Discuss any qualitative feedback on each timing of design
To evaluate the design heuristics for the timing of proactive AI assistance (DG1, Table \ref{table:proactive-features}), we analyzed how each system feature and heuristic was utilized. 
Derived from the interaction data, we summarize the frequency, duration, and outcome of each heuristic design (Fig.\ref{fig:timing_heuristics}). 
Overall, we recorded 398 proactivity instances, with 212 (53.3\%) interactions leading to the user's effective engagement (e.g. adapting AI code changes, respond to the agent's message), 48 disruptions (12.1\%), and 138 interactions (34.7\%) where the user did not engage with the proactive agent (i.e. ignored or did not notice).
The most frequently triggered heuristics were code block completion (107 times), program execution (102 times), and user-written in-line comment (78 times). 
Additionally, the most effective heuristics that led to user engagement are multi-line change (73.1\%), user-written comment (69.2\%) and program execution (66.7\%).
Reflecting on the proactivity features, Design Rationale 2 --- intervening at programmer's task boundary --- was the most effective design principle overall.
The only exception is the heuristic of intervening at code block completion, which resulted in \revise{excessive AI responses. Many were affirmatory messages} to acknowledge the completed code and ask if the user needs further help. This led users to ignore around \revise{50\% of the proactive agent signals (Fig.\ref{fig:timing_heuristics})} to avoid disruption to their workflow.

% Add a couple lines on the user's comments on those
\revise{On the other hand}, the implementation of Design Rationale 3 --- intervening based on the user's implicit signals of adding a code comment or selecting a range of code --- resulted in many false positives that led to workflow disruptions.
Code comments and cursor selections conveyed different utilities for different users, which led to misinterpretation of user intent.
For example, P10 did not perceive comments as instructions for AI: ``\textit{I feel like when I think of comments, I think of just writing helpful little notes for myself. Like I don't see them necessarily as instructions. So I feel like it would have been a little distracting right now.}''
Code selection, similarly, was used by some participants as a habitual behavior to focus their own attention on a part of code. Therefore, the agent's proactivity could be perceived as unexpected and unnecessary.

Design Rationale 1 --- intervening at moments of low mental workload --- \revise{was not effectively operationalized}. 
Participants reported that when they were inactive for an extended period, they were likely thinking through the code design or solving an issue, which represents high mental workload. 
While it is likely that idleness is a signal to assist, participants preferred to initiate the help-seeking after they could not resolve the issue themselves, rather than having the AI agent intervene at a potentially mentally occupied moment.
The design rationale requires more involved modeling of the programmer's mental state to render it effective.
We outline the design implications from these finding in the discussion.



\begin{figure*}[t]

\centering
\includegraphics[width=\textwidth]{figures/Codellaborator_User_interaction_journey.pdf}

\caption{\textbf{Human-AI interaction timelines for P1.} \textnormal{For each task, we visualized each interaction initiated by the user and the AI, along with the time spent expressing the user's intent and interpreting the AI agent's response. We also visualized the annotated programming stages over the task. 
The Misc stage colored in black represents when the user was not actively engaged in the task (e.g. performing think-out-loud).
In PromptOnly, we observed the traditional command-response interaction paradigm where the user initiated most interactions. However, P1 unexpectedly triggered the AI agent when documenting code with comments, causing disruptions. In the CodeGhost condition, AI initiated most interactions, but this caused 6 disruptions, mainly during the Organize stage when P1 was making low-level edits and did not expect AI intervention. 
In the \sys{} condition, AI remained proactive but caused fewer disruptions, as P1 engaged in more back-and-forth interactions with higher awareness of the AI's actions and processes. See supplementary material for all timelines.}}
\label{fig:timeline}
% link to google drawing: https://docs.google.com/drawings/d/1Gz6nBWUN-ja1zI0tPYowdSLWafotLV5jUtYCOJjdFmw/edit?usp=sharing
\end{figure*}


\subsection{Users Adapted to AI Proactivity and Established New Collaboration Patterns}
% Describe how the users develop trust to the system, and some users become reliant or over-reliant on the AI
% Throughout the user study session, participants demonstrated calibration of their mental model of the AI agent's capabilities in the editor.
% Users naturally developed more trust and reliance as they used the AI to aid with their tasks, especially after they were exposed to proactive assistance.
Throughout the user study, participants calibrated their mental model of the AI agent's capabilities in the editor, developing a level of trust and reliance after experiencing proactive assistance.
% P15 used an analogy to a leader in a software engineering team, and that as the team establishes a \textit{``good track record of performing well, you're just naturally going to trust it.''}
Half of the participants ($N$ = 9) exhibited a level of reliance on the AI's generative power to tackle the coding task at hand and resorted to an observer and code reviewer role.
% P7 described their mentality shift as such: 
% \textit{``As programmers you're never really going to do extra work if you don't have to...You might as well take a little bit of a backseat on it and kind of only start working on it yourself once it's like complex logic that you need to understand yourself.''}
With this role change, participants shifted their mental process to focus more on high-level task design and away from syntax-level code-writing.
P3 reflected on their shift: \textit{``I kind of shifted more from `I want to try and solve the problem' to what are the keywords to use to get this [AI agent] to solve the problem for me... I could also feel myself paying less attention to what exactly was being written...So I think my shift focus from less like problem-solving and more so like prompts.''}
% P15 had a similar view of the proactive system: 
% \begin{quote}
%    \textit{ ``It [the AI] was the driver and I was the tool. Basically, I was the post-mortem tool, right. I was checking whether his code is correct, right. But it was the driver writing it. So in that case, I was not disrupted at all.. because the paradigm of the workflow has shifted. I am not in a position to be disrupted anymore, right. It was doing the heavy lifting. I was just doing the code review. (P15)''}
% \end{quote}
P10 expressed optimism toward developers' transition from code-writing to more high-level engineering and designing tasks:
\begin{quote}
    \textit{``I think with the increase of... low code, or even no code sort of systems, I feel like the coding part is becoming less and less important. And so I really do see this as a good thing that can really empower software engineers to do more. Like this sort of more wrote software engineering, more wrote code writing is just... it's not needed anymore.''}
\end{quote}

Under this trend of allowing the AI to drive the programming tasks, four participants (P3, P6, P7, P15) commented that they were still able to maintain overall control of the programming collaboration and steer the AI toward their goal.
P7 described their control as they adjusted to the level of AI assistance and navigated division of labor: \textit{``It's great to the point where you have the autonomy and agency to tell it if you want it to implement it for you, or if you want suggestions or something like you can tell [the AI] with the way it's written. It's always kind of like asking you, do you want me to do this for you? And I think that's like, perfect.''}
These findings highlight the potential for users to adopt proactive AI support in their programming workflows, fostering productive and balanced collaborations, provided the systems show clear signals of its capabilities for users to align their understanding to.






\subsection{Users Desired Varying Proactivity at Different Programming Processes}
\label{Results:programming_process}
% Some users expressed ambivalence to using proactive AI, good for efficiency, but bad for code understanding
Through analyzing the 1004 human-AI interaction episodes, we found that participants engaged with the AI the most (38.2\%) during the implement stage, 
followed by debug (26.4\%), 
analyze (i.e., examine existing code or querying technical questions like how to use an API; 11.5\%), design (i.e., planning the implementation; 10.9\%), organize (i.e., formatting, re-arranging code; 6.67\%), refactor (5.48\%), and miscellaneous interactions (e.g., user thanks the AI agent for its help; 0.697\%). 
We visualize P1's user interaction timeline as an example to illustrate different interaction types and frequencies under different programming stages (Fig.\ref{fig:timeline}).
% To perform this analysis, we referenced CUPS, an existing process taxonomy on AI programming usage \cite{Mozannar2022ReadingBT}, and adapted it to our research questions and applicable stages observed in our tasks.
% We acknowledge that the listed programming processes do not comprehensively represent different tasks and software engineering contexts.
% Rather, we cross-reference our analysis with qualitative feedback to identify user experiences in different stages of programming at a high level.
% With this broad categorization, we probed participants during the post-interview and identified processes of programming where proactive AI assistance was desired, and where it was disruptive and unhelpful.
To conduct this analysis, we adapted CUPS, an existing process taxonomy on AI programming usage \cite{Mozannar2022ReadingBT}, to align with our research questions and the stages observed in our tasks. 
By cross-referencing the interaction analysis with qualitative feedback, \revise{we identified programming stages where proactive AI assistance was most desired or disruptive.}
% This broad categorization allowed us to identify where proactive AI assistance was most desired and where it could be disruptive or unhelpful, providing valuable insights into optimizing AI support in programming.

In general, participants preferred to engage with the AI during well-defined boundaries between high-level processes, like providing scaffolds to the initial design or executing the code, and repetitive processes, such as refactoring.
They additionally desired AI intervention when they were stuck, for example during debugging.
In contrast, for more low-level tasks that require high mental focus, like implementing \revise{planned} functionality, participants were more often disrupted by proactive AI support and would prefer to take control and initiate interactions themselves.

This was corroborated by our interaction analysis results.
When examining the number of disruptions, we found that most disruptions occurred during the implementation process (32.7\%, 18 disruptions).
% and the implement (25.7\%, 9 disruptions). A disproportional amount of organize stage disruptions took place when participants were not intentionally moving forward with their processes or trying to advance the task. 
% Instead, participants were conducting low-level code organization (e.g., moving blocks around or adding empty lines between code blocks) to improve readability or satisfy personal preferences. 
% The AI agent's intervention during this stage was considered unnecessary. P3, who often documented and re-arranged their code, lamented about proactivity in P condition intruding their organization process: ``\textit{I would like type a comment, and then like, it would appear...giving me like three different text messages}.''
In contrast, \revise{very few} disruptions occurred during the debugging (7.27\%, 4 disruptions) and refactor phases (1.82\%, 1 disruption), which comprised 26.3\% and 5.48\% of all the interactions, respectively.
Most participants expressed the need to seek help from the AI agent in these stages and anticipated AI intervention as there were clear indications of turn-taking (i.e., program execution) and information to act on (i.e., program output, code to be refactored). 
After experiencing proactive assistance, P9 felt that ``\textit{[PromptOnly] wasn't responsive enough in the sense that when I ran the tests, I was kind of looking for immediate feedback regarding what's wrong with my tests and how I can fix it.}''
This corresponds with our proactivity design guideline to initiate intervention during subtask boundaries (Table \ref{table:proactive-features}, Design Rationale 2). 
In a sense, participants desired meaningful actions to be taken before AI intervened.
As P13 described, \textit{``If I'm like paste [code], something big, I run the program, the proactivity in that way, it's good. But if it's proactive because I'm idle or proactive because of a tiny action or like a fidget, then I don't really like that [AI] initiation.''}
% In different context, they would want different levels of proactivity, the action should match the level of severity
% Despite general agreement on preferred and less preferred programming processes to engage with proactive AI, participants did not reach a consensus and often expressed conflicting views on specific processes.
% For example, while P9, P16, and P17 desired proactive feedback after executing their programs and receiving errors, P14 and P18 were against using proactive AI for debugging as it might recurse into more errors, making the program harder to debug.
% Thus, in addition to adhering to general trends, future systems should also aim to be adaptive to the user's preferences, exhibiting different levels of proactive AI assistance according to best fit the user's personal needs and use cases.
% We propose a detailed design suggestion in Section \ref{Discussion:design_implication}.
\revise{Participants generally expressed preferences for programming processes where they wanted proactive support, but their opinions varied regarding which specific processes required more or less proactive assistance.}
For instance, while P9, P16, and P17 welcomed proactive feedback after program errors, P14 and P18 opposed it, fearing it could lead to more errors and complicate debugging. 
Therefore, future systems should adapt to individual user preferences, offering varying levels of proactive AI assistance based on personal needs and use cases. A detailed design suggestion is provided in Section \ref{Discussion:design_implication}.



\subsection{\sys{} and CodeGhost Feel \revise{More} Like Programming with a Partner than a Tool}
% Another effect of presence and context is a different sense of collaboration
% Another effect we observed from participants using the proactive conditions was an elevated sense of collaboration rather than using the programming assistant as a tool.
% While in all three conditions, the AI agent was initialized with the same prompt that enforces pair programming practices (Appendix \ref{appendix:prompt}), some participants expressed that working with the \sys{} and CodeGhost conditions felt more like collaborating with a more human-like agent with presence ($N$ = 6) than the PromptOnly.
We observed that participants in the proactive conditions perceived an elevated sense of collaboration with the AI, rather than viewing it as just a tool. 
Despite all three conditions using the same pair programming prompt (Appendix \ref{appendix:prompt}), six participants noted that \sys{} and CodeGhost felt more like collaborating with a human-like agent compared to PromptOnly.
% P1 commented that \textit{``it's like a person that's on your side [and says] `that's over here. You add that here' and kind of felt that way.''}
P6 reflected that \textit{``just the fact that it was talking with me and checking in with a code editor. I maybe treated it more like an actual human.''}
A part of this is due to the \revise{local scope of interaction with the agent in} the code editor (DG3), as P14 reflected \textit{``by changing the code that I'm working on instead of like on the side window...it feels more like physically interacting with my task.''}
Even the disruptions arising from the proactive AI actions facilitated a human-like interaction experience.
P9 recalled an interaction where they encountered a conflict in turn-taking with the AI: \textit{``[AI agent] was like, `Do you want to read the import statement? Or should I?' I was like, `No, I'll write it' and it [AI agent] said `Great I'll do it' and it just did it. Okay, yeah. True to the human experience.''}

This different sense of collaboration was \revise{reflected in} the survey results ((\textit{$\chi^2$} = 22.1, \textit{df} = 2, \textit{p} < 0.001, Fig.\ref{fig:survey} Q8).
Participants rated the AI assistant in the PromptOnly to be much like a tool ($\mu$ = 5.67, $\sigma$ = 1.58), while both the \sys{} ($\mu$ = 3.61, $\sigma$ = 1.65) and the CodeGhost conditions ($\mu$ = 4.17, $\sigma$ = 1.72) felt more like a programming partner (both \textit{p} < 0.001 compared to PromptOnly).
This more humanistic collaboration experience introduced by proactive AI systems naturally brings questions to its implications for programmers' workflow. We further share our analysis across programming processes in Section \ref{Results:programming_process} and the corresponding design suggestions in the Discussion.



\begin{figure*}[]

\centering
\includegraphics[width=0.8\textwidth]{figures/Codellaborator_survey.pdf}
% new link to google drawing:
% https://docs.google.com/drawings/d/1oiOvDttY3Xibk0P3hm0y1BJWbRh3cKMkh6or_Gu4XB4/edit?usp=sharing
% link to google drawing: https://docs.google.com/drawings/d/1rELkSVN1rroPm8ccWyVykW72SDOhkBKn3MjoaZZq4ec/edit?usp=sharing
\caption{\textbf{Likert-scale Response displayed in box and whisker plots comparing three conditions}. \textnormal{Anchors are 1 - Strongly disagree and 7 - Strongly agree. The green dotted lines represent the mean values for each question. Using the Friedman test, we identified significant differences in rating in Q1 for disruption, Q5 for awareness, and Q8 for partner versus tool use experience.}}
\label{fig:survey}
\end{figure*}



\subsection{Presence and \revise{Local Scope of Interaction} Increase User Awareness on AI Action and Process}
\label{Results:presence_context}
% To specifically evaluate our design of the \sys{} technology probe, we collected qualitative feedback on the AI presence and context management features, and their effects on the user experience compared to other conditions.
% Eight participants expressed that the AI agent's presence in the editor increased their awareness of the AI's actions, intentions, and processes.
We gathered qualitative feedback on the \sys{} technology probe's AI presence and \revise{breakout} features to assess their impact on user experience. 
Eight participants noted that the AI's presence in the editor enhanced their awareness of its actions, intentions, and processes (DG2).
Visualizing the AI's edit traces in the editor using a caret and cursor helped guide the users (P1, P4, P7, P12, P18) and allowed them to understand the system's focus and thinking (P12, P13, P18).
As P13 commented \textit{``I... like the cursor implementation of like, be able to see what it's highlighting, be able to move that cursor all the way just to see like, what part of the file it's focusing on.''}
The presence features also helped users identify the provenance of code and clarified the human-AI turn-taking.
As P10 remarked, \textit{``it was really clear when the AI was taking the turn with writing out the text and like the cursor versus when I was writing it.''}
On the other hand, the different scopes of interaction further increased users' awareness by reducing their cognitive load and enhancing the granularity of control (DG3).
For example, compared to a standard chat interface where \textit{``everything is just one very long line of like, long stream of chat''}, P6 preferred the threaded breakout conversations that decomposed and organized past exchanges.
P4 also found that the breakout \textit{``could be sort of like a plus towards steerability because you can really highlight what you want it to do.''}

% In survey, more awareness...
% Analyzing the survey response, we found that participants generally found the system to be highly aware of the user's actions, with no significant difference across conditions (\textit{$\chi^2$} = 5.83, \textit{df} = 2, \textit{p} = 0.054, Fig.\ref{fig:survey} Q6).
% Conversely, participants rated their own awareness of the AI differently (\textit{$\chi^2$} = 12.7, \textit{df} = 2, \textit{p} < 0.001, Fig.\ref{fig:survey} Q5), with the highest in PromptOnly ($\mu$ = 6.56, $\sigma$ = 0.511), then \sys{} ($\mu$ = 5.44, $\sigma$ = 1.79), then CodeGhost ($\mu$ = 4.17, $\sigma$ = 1.86).
% Specifically, the users felt like they were less aware of the CodeGhost condition prototype compared to the non-proactive PromptOnly (\textit{Z} = 2.5, \textit{p} < 0.001).
% We did not identify any other significant pairwise comparisons after Bonferroni Correction.
% This suggests that proactivity alone in CodeGhost induced more workflow interruptions, which in turn lowered users' perceived awareness of the AI system's action and process. 
% But similar to alleviating workflow disruptions, the \sys{} condition with visual presence and context management also improved user awareness.
Analyzing the survey responses, we found that participants generally rated the system as highly aware of their actions, with no significant difference across conditions (\textit{$\chi^2$} = 5.83, \textit{df} = 2, \textit{p} = 0.054, Fig.\ref{fig:survey} Q6). 
However, participants' \revise{own} awareness of the AI \revise{agent's actions} varied significantly (\textit{$\chi^2$} = 12.7, \textit{df} = 2, \textit{p} < 0.001, Fig.\ref{fig:survey} Q5), with the highest ratings in PromptOnly ($\mu$ = 6.56, $\sigma$ = 0.511), followed by \sys{} ($\mu$ = 5.44, $\sigma$ = 1.79), and the lowest in CodeGhost ($\mu$ = 4.17, $\sigma$ = 1.86). 
Specifically, participants felt less aware of the AI in CodeGhost compared to the non-proactive PromptOnly condition (\textit{Z} = 2.5, \textit{p} < 0.001). 
No other significant pairwise differences were found after applying Bonferroni correction. 
\revise{This can be attributed to CodeGhost's increased proactivity, which, in the absence of sufficient presence signals and a manageable local interaction scope, resulted in more frequent workflow interruptions. These interruptions, in turn, diminished users' ability to remain aware of the AI's actions and interpret its signals effectively, ultimately reducing their sense of control and understanding during the interaction.}

% These findings suggest that CodeGhost's proactivity led to more workflow interruptions and reduced users' perceived awareness of the AI's actions, presenting drawbacks to proactive assistance.

% However, not everyone was fond of presence and context, some find it distracting, and some don't want to go back to old conversations
% While the \sys{} condition with visual presence and context management improved user awareness,  not every participant found the agent design helpful. 
% Four participants thought that the AI presence could be distracting (P8, P10, P17), and two participants did not prefer the integration of AI in the editor, as they took up screen real estate (P6, P9).
% Similarly, P16 pointed out that their personal workflow would not involve using breakout chats for managing past interactions: \textit{``I also don't think I would have that many discussions with the AI once the coding is done and I have this working, then I'm probably not gonna look back at the discussions I've taken.''}
% The mixed findings on the system design indicate that different users, given different programming styles, workflows, preferences, and task contexts, desire different types of systems. We detail our design implications and suggestions in Section \ref{Discussion:design_implication}.
While the \sys{} condition showed improvements on user awareness, not all participants found the agent design helpful. 
Four participants felt the AI presence was distracting (P8, P10, P17), and two thought it occupied too much screen space (P6, P9). 
P16 noted that their workflow wouldn't involve using breakout chats to manage \revise{interaction context}: \textit{“Once the coding is done and I have this working, then I'm probably not gonna look back at the discussions I've taken.”} 
These mixed responses suggest that users with different programming styles, workflows, and preferences require varied system designs. Design implications are discussed in Section \ref{Discussion:design_implication}.




\subsection{Over-Reliance on Proactive Assistance Led to Loss of Control, Ownership, and Code Understanding}
% Discuss how some users are against overly trusting and relying on AI
% These users express concerns on their loss of control
Despite optimism in adopting proactive AI support in many participants, some participants (P6, P10, P11, P13, P16, P18) voiced concerns about over-relying on AI help, \revise{citing} a loss of control.
P10 felt like they were \textit{``fighting against the AI''} in terms of planning for the coding task, as the agent proactively makes coding changes during the implementation phase.
% P11 described that the AI agent \textit{``didn't let me implement it the way I wanted to implement it, it just kind of implemented it the way it felt fit.''}
They further expanded on the potential limitations of LLM code generation, particularly with regards to devising innovative solutions: \textit{``If it was too proactive with that, it would almost force you into a box of whatever data it's already been trained on, right?... It would probably give you whatever is the most common choice, as opposed to what's best for your specific project (P11).''}

% The effects also affect ownership of the code
The capability to understand rich task context and quickly generate solutions also lowered users' sense of ownership of the completed code.
P7 concluded that \textit{``the more proactivity there was, the less ownership I felt...it feels like the AI is kind of ahead of you in terms of its understanding.''}
% Code understanding, maintainability
This lack of code understanding was referenced by multiple participants (P11, P16, P18), raising issues on the maintainability of the code (P11, P14, P15) and security risks (P18).
As P11 suggested: \textit{``It's... not facilitating code understanding or your knowledge transfer. And yes, it's not very easily understood by others, if they just take a look at it.''}
Additionally, some participants believed that programmers should still invest time and effort to cultivate a deep understanding of the codebase, even if AI took the initiative to write the code.
P4 commented \textit{``I think the more that you leave it up to the AI, the more that you sort of have to take it upon yourself to understand what it's doing, assuming that you're being you know, responsible as [a] programmer.''}
Upon noticing that the AI was overtaking the control, P14 adjusted the way they utilized the proactive assistance and found a more balanced paradigm: \textit{``It's more like a conversation, like I gave him [AI] something so it did something, and then step by step I give another instruction and then you [AI] did something. I was being more involved, which allows me to like step by step understand what the AI is doing also to oversee, I was able to check it.''}

However, not all participants shared this concern. P10 expressed a different opinion as they felt like they are not \textit{``emotionally attached''} to their code, and that in the industry setting, code has been written and modified by many stakeholders anyways, so that \textit{``me typing it versus me asking the AI to type it, it's just not that much of a difference.''}
% This prompts an adaptive and balanced design that emphasizes user's control and reliance...
% This part of our findings uncovered different trade-offs between convenience and productivity from the utilization of more proactive and autonomous AI tools, and the potential loss of control during the programming process and less ownership of the end result.
\revise{This finding highlights the trade-offs between convenient productivity and potential risks in user control and code quality.}
While the system can be used to increase efficiency and free programmers from low-level tasks like learning syntax, documentation, and debugging minor issues, it remains a challenge to design balanced human-AI interaction, where the users' influences are not diminished and developers can work with AI, not driven by AI, to tackle new engineering problems.
We condense our findings into design implications in Section \ref{Discussion:design_implication}.


% \subsection{Old Disruption Subsection}

% % \subsection{Social Transparency Cues Reduced the Number of Disruptions Caused by Proactivity}
% Analyzing the Likert-scale survey data using the Friedman test, participants perceived different levels of disruptions among three conditions (\textit{$\chi^2$} = 17.4, \textit{df} = 2, \textit{p} < 0.001, Fig.\ref{fig:survey} Q1). 
% Using Wilcoxon signed-rank test with Bonferroni Correction, we found higher perceived disruption in the P condition ($\mu$ = 4.75, $\sigma$ = 1.42) than in the B condition ($\mu$ = 1.50, $\sigma$ = 0.90, \textit{Z} = 3.1, \textit{p} < 0.01), as well as a higher level of perceived disruption in condition S than B ($\mu$ = 3.75, $\sigma$ = 1.91, \textit{Z} = 2.9, \textit{p} < 0.01).
% We did not find a significant difference in the perceived level of disruption between condition P and condition S ($\mu$ = 3.75, $\sigma$ = 1.9, \textit{Z} = -1.7, \textit{p} = 0.085).
% The higher level of perceived disruptiveness compared to the baseline might be due to the added AI agent's visual signals burdening users' cognitive load. P6 commented that the presence of the AI cursor and thought bubble ``\textit{take up some, I guess real estate of like the editor itself.}'' Similarly, P7 found that the AI agent is S condition is ``\textit{very proactive and very like present with what you're doing. I think that can be potentially a tad bit overwhelming at times because especially with like the highlighting if you can see like what it's doing and it's doing everything.}''
% This indicates that while social transparency cues allow for more visibility of the AI agent's process, it can be overwhelming for the user to perceive all the visual feedback.

% Comparing all three system conditions using a one-way ANOVA test, we observe a significant difference in the mean number of disruptions participants experience per task (\textit{F(2,33)}= 5.86, \textit{p} < 0.01).
% We then conduct post-hoc analysis using Tukey's Honestly Significant Difference (HSD) for multi-group pairwise comparisons with control in overall familywise error rate.
% The P condition ($\mu$ = 2.08, $\sigma$ = 2.11) resulted in significantly more disruptions than the B condition ($\mu$=0.33, $\sigma$ = 0.89; \textit{p} < 0.01). 
% % Disruptions during the B condition were mainly due to the user unintentionally triggering an AI response when documenting code (i.e., activating a comment action), or the user attempting to manually make code changes while waiting for system feedback but then being disrupted by system feedback. 
% % Disruptions during the P condition stemmed from the lack of awareness the user had about the AI agent's state. Participants perceived no AI actions and desired to take control (i.e., manually coding or examining the file), potentially advancing to the next sub-task with a different context. However, the AI agent reacted to user actions without visibility about the interaction process and existing context, causing the interventions to be jarring and hard to interpret.
% In contrast, participants experienced significantly fewer disruptions during the S condition ($\mu$ = 0.58, $\sigma$ = 0.51) than the P condition (\textit{p} < 0.05). There was not a significant difference between the S and B conditions (\textit{p} = 0.89). 
% This result suggests a mediating effect of social transparency cues for alleviating the disruptions brought upon by a proactive AI agent.

% Disruptions occurred in different patterns across the three conditions.
% In condition B, disruptions were mainly due to the user unintentionally triggering an AI response when documenting code (i.e., activating a comment action), or the user attempting to manually make code changes while waiting for system feedback but then being disrupted by system feedback. 
% Disruptions during the P condition stemmed from the lack of awareness the user had about the AI agent's state. Participants perceived no AI actions and desired to take control (i.e., manually coding or examining the file), potentially advancing to the next sub-task with a different context. However, the AI agent reacted to user actions without visibility about the interaction process and existing context, causing the interventions to be jarring.
% Disruptions during the S condition are similar to those of the P condition, where users expected to take control but were interrupted, but they occurred less frequently. Another type of disruption with the S condition occurs when the user perceives and communicates with the AI about the turn-taking, but the coordination is ambiguous, so both the human and the AI agent act, causing disruptions.
% % Disruptions during the B condition occurred more often when neither the participant nor the AI agent was active, but both attempted to initiate interaction with poor coordination. Due to a period of participant inactivity, the AI agent could not accurately identify the participant's needs and current thought process. Thus, the timing of system intervention appeared more unforeseen.



% % \subsection{Proactivity and Social Transparency's Effects on Disruptions}
% % Comparing all three system conditions using a one-way ANOVA test, we observe a significant difference in the mean number of disruptions participants experience per task (\textit{F}(2,33)= 5.86, \textit{p} < 0.01).
% % We then conduct pairwise comparisons using T-test and Bonferroni Correction, judging statistical significance at \textit{p} < 0.0167 [0.05 / 3]. With this metric, we found that the P condition ($\mu$ = 2.08, $\sigma$ = 2.11) resulted in significantly more disruptions than the B condition ($\mu$=0.33, $\sigma$ = 0.89; \textit{p} < 0.0167). 
% % Participants experienced fewer disruptions during the S condition ($\mu$ = 0.58, $\sigma$ = 0.51), but we did not identify statistical significance when comparing to the P condition (\textit{p} = 0.0256). 
% % There was also not a significant difference between the S and B conditions (\textit{p} = 0.408). 

% % Disruptions during the B condition were mainly due to the user unintentionally triggering an AI response when documenting code (i.e., activating a comment action), or the user attempting to manually make code changes while waiting for system feedback but then being disrupted by system feedback. 
% % Disruptions during the P condition stemmed from the lack of awareness the user had about the AI agent's state. Participants perceived no AI actions and desired to take control (i.e., manually coding or examining the file), potentially advancing to the next sub-task with a different context. However, the AI agent reacted to user actions without visibility about the interaction process and existing context, causing the interventions to be jarring and hard to interpret.
% % Disruptions during the S condition are similar to those of the P condition, where users expected to take control but were interrupted, although less frequent. Another type of disruption with the S condition occurs when the user perceives and communicates with the AI about the turn-taking, but the coordination is ambiguous, so both the human and the AI agent act, causing disruptions.
% % nor the AI agent was active, but both attempted to initiate interaction with poor coordination. Due to a period of participant inactivity, the AI agent could not accurately identify the participant's needs and current thought process. Thus, the timing of system intervention appeared more unforeseen.

% % Analyzing the Likert-scale survey using the Friedman test, participants perceived different levels of disruptions among three conditions (\textit{$\chi^2$} = 17.4, \textit{df} = 2, \textit{p} < 0.001, Fig.\ref{fig:survey} Q1). 
% % Using Wilcoxon signed-rank test with Bonferroni Correction, we found higher perceived disruption in the P condition ($\mu$ = 4.75, $\sigma$ = 1.42) than in the B condition ($\mu$ = 1.50, $\sigma$ = 0.90, \textit{Z} = 3.1, \textit{p} < 0.01), and higher perceived disruption in condition S than B ($\mu$ = 3.75, $\sigma$ = 1.91, \textit{Z} = 2.9, \textit{p} < 0.01).
% % However, we did not find a significant difference in the perceived level of disruption between condition P and condition S (\textit{Z} = -1.7, \textit{p} = 0.085).

% % Despite a lower mean number of disruptions and lower mean rating for perceived disruption in condition S than P, we did not identify an effect of social transparency cues reducing disruptions significantly. 
% % This might be due to the visual signals indicating AI agent status in S condition adds to users' cognitive load. P6 commented that the presence of AI cursor and chat bubble ``\textit{take up some, I guess real estate of like the editor itself.}'' Similarly, P7 found that the AI agent is S condition is ``\textit{very proactive and very like present with what you're doing. I think that can be potentially a tad bit overwhelming at times because especially with like the highlighting if you can see like what it's doing and it's doing everything.}''
% % This indicates that while social transparency cues allow for more visibility of the AI agent's process, it can be overbearing for the user to perceive all the visual feedback.



% \subsection{Time and Effort to Convey to, and Comprehend, the AI}
% % \yc{is there a statement to make in the title? should have one takeaway msg per result as a paragraph-header}
% To examine the effects of proactivity on the facilitation of productivity and better human-AI collaboration, we analyzed the overall completion time for each condition. We also draw from interaction-level data to inspect the effort users needed to convey their intentions to the AI agent and to comprehend the system response.

% All tasks were completed within the 30-minute time allotment except for three. During these three instances, the participants did not come close to fulfilling the specifications of the tasks and were halted without passing all test cases. We observe one failed instance of each coding task, with two occurring during the P condition and one during the S condition. 
% The interaction episodes in the failed instances are still included in the data analysis, as they were natural interactions from first-time users.

% A one-way ANOVA did not reveal a significant difference in overall completion time for each condition (\textit{F}(2,33)= 0.425, \textit{p} = 0.657). This demonstrates that despite feeling more disrupted in proactive AI systems, the interventions do not slow down participants from completing the tasks.
% % , consistent with prior studies \cite{vaithilingam2022expectation}.
% % \yc{this is consistent with prior studies}.
% % or each task (\textit{F(2,33)}= 1.77, \textit{p} = 0.19). 
% % Thus, there is no evidence that the difference in task difficulties was a confounding factor. 
% % There was also no significant effect of ordering (\textit{F(X,X)} = 1.16, \textit{p} = 0.32). However, the range of task completion time is drastic, with the fastest completion being only 169 seconds (task order 1, condition P, budget tracker), which was lower than the time allotted and deviated from the overall mean completion time of 991 seconds. This could have been due to the inconsistent code generation quality from the LLM despite our effort to configure it with the least randomness.
% Despite no significant differences in task completion time, the proactive agent interventions in the P and S conditions resulted in less effort for the user to interpret each AI action than during the B condition (Figure \ref{fig:time_convey_interpret}). 
% We observe a significant difference in the amount of time to interpret the AI agent's actions (e.g., chat messages, editor code changes, presence cues) per interaction across three conditions (\textit{F}(2,637)= 46.7, \textit{p} < 0.001) from ANOVA.
% Using Tukey's HSD, we found the time to interpret per interaction was significantly higher in condition B ($\mu$ = 36.4 seconds, $\sigma$ = 30.8) than in condition P ($\mu$ = 17.7 seconds, $\sigma$ = 16.1; \textit{p} < 0.001) and S ($\mu$ = 17.0 seconds, $\sigma$ = 15.9; \textit{p} < 0.001). There was no significant difference in the time to interpret between the P and S conditions (\textit{p} = 0.90; Figure \ref{fig:time_convey_interpret}). 


% There was also not a significant effect of condition on the time to convey user intention to the AI agent (\textit{F}(2,432) = 0.744, \textit{p} = 0.476). The degree of freedom is lower as only user-initiated interactions required users to convey their intentions. However, five participants expressed that out of the three conditions, they felt that they spent the most effort communicating with the AI agent in condition B since they had to manually draft a message with a specific context (P1, P3, P6, P7, P12).

% Overall, these findings indicate that proactivity in AI programming systems might not result in a significant increase in productivity or efficiency, however, AI-initiated contextualized assistance could potentially contribute to more explainable and interpretable AI, with the decreased effort to understand AI responses and no extra cost to convey user intentions, at the scope of each interaction.

% \subsection{Proactivity Affects Awareness and Collaboration Experience}
% Further analyzing the survey results using Friedman test, we found that across three conditions,
% participants rated different levels on their sense of awareness of the AI agent (\textit{$\chi^2$} = 8.06, \textit{df} = 2, \textit{p} < 0.05, Fig.\ref{fig:survey} Q5). 
% Using Wilcoxon signed-rank tests with Bonferroni Correction, we found that compared to the baseline condition ($\mu$ = 6.33, $\sigma$ = 0.49), participants felt the AI agent was less aware of their actions in P condition ($\mu$ = 4.33, $\sigma$ = 2.06, \textit{Z} = -2.4, \textit{p} < 0.0167 [0.05 / 3]).
% We did not identify any significant difference between conditions B and S ($\mu$ = 5.58, $\sigma$ = 1.62, \textit{Z} = -1.2, \textit{p} = 0.202), nor between conditions P and S (\textit{Z} = 1.7, \textit{p} = 0.092).


% On the other hand, we found a significant difference in users' rating of the AI agent's awareness of the user's actions (\textit{$\chi^2$} = 8.21, \textit{df} = 2, \textit{p} < 0.05, Fig.\ref{fig:survey} Q6).
% % participants perceived different levels of disruptions among three conditions (\textit{$\chi^2$} = 17.61, \textit{df} = 2, \textit{p} < 0.001, Fig.\ref{fig:survey} Q1). 
% % Using Wilcoxon signed-rank tests, we found a similar pattern in higher perceived disruption in P condition ($\mu$ = 4.8, $\sigma$ = 1.4) than in B condition ($\mu$ = 1.5, $\sigma$ = 0.90, \textit{Z} = 3.1, \textit{p} < 0.01).
% Performing pairwise condition comparisons, we found that compared to the baseline condition ($\mu$ = 5.33, $\sigma$ = 0.98), participants felt the AI agent was more aware of their actions in P condition ($\mu$ = 6.42, $\sigma$ = 0.79, \textit{Z} = 2.4, \textit{p} < 0.0167). 
% We did not identify a significant difference between condition B and S ($\mu$ = 6.17, $\sigma$ = 0.83, \textit{Z} = 2.0, \textit{p} = 0.0442) nor between P and S conditions (\textit{Z} = -0.89, \textit{p} = 0.37).


% These results demonstrate that the proactivity in conditions P without social transparency cues simultaneously made users feel like the AI agent was more aware of the user's actions, while they were less aware of the AI agent's actions themselves. 
% % Whereas in condition S, participants did not necessarily feel less aware of the AI. 
% % This potentially indicates that the social transparency cues alleviated the abruptness of the AI agent's proactive actions. While we did not identify a significant pairwise difference between conditions P and S, 

% Interestingly, proactivity also affects users' perspectives on whether the collaboration felt like a partnership or merely utilizing a tool. 
% When asked whether they felt like the AI agent was more like a tool than a programming partner (Fig.\ref{fig:survey} Q8), we discovered significant differences across conditions (\textit{$\chi^2$} = 8.04, \textit{df} = 2, \textit{p} < 0.05). 
% % Similar to the pattern in system awareness, the conditions P ($\mu$ = 3.83, $\sigma$ = 1.59, \textit{Z} = -2.21, \textit{p} < 0.05) and S ($\mu$ = 4.0, $\sigma$ = 1.86, \textit{Z} = -2.35, \textit{p} < 0.05) were both perceived as closer to a programming partner than a tool when compared to the baseline ($\mu$ = 5.25, $\sigma$ = 1.71, \textit{Z} = 2.4, \textit{p} < 0.05). we did not find a significant pairwise difference between conditions P and S (\textit{Z} = 0.42, \textit{p} = 0.67).
% While we did not identify any significant pairwise comparisons between conditions, we received qualitative feedback that conditions P ($\mu$ = 3.83, $\sigma$ = 1.59) and S ($\mu$ = 4.0, $\sigma$ = 1.86) felt less like using a tool and more like a partnership experience than in baseline condition ($\mu$ = 5.25, $\sigma$ = 1.71).
% From the interviews, participants expressed that proactivity and social transparency cues led to a human-like programming collaboration. P1 in condition S commented that ``\textit{it's like a person that's on your side that [says] `here, you add that [code] here'}.'' P8 also expressed that in condition S, ``\textit{it felt a bit more human in the way that it was kind of interrupting you}.'' Similarly, P6 recalled that in conditions P and S, ``\textit{the fact that it was talking with me and checking in with a code editor. I maybe treated it more like an actual human}.'' 

% Based on these findings, AI proactivity could lead to users feeling less aware of the system's actions but more observed by the system with higher AI-to-user awareness. 
% When introduced with social transparency cues in condition S, participants did not feel less aware of the AI compared to the baseline. 
% However, more data needs to be collected to further explore if social transparency mediates the awareness gap when users interact with proactive AI tools.
% Proactivity and social transparency also potentially serve the purpose of facilitating a social presence, creating a partnership-like experience. 
% Future research can explore the implications of proactive AI design and how to best utilize this human-like collaboration experience.

% % The Likert-scale responses revealed that participants felt that the AI agent was more aware of the users' actions during the P ($\mu$ = 6.42, $\sigma$ = XXX; \textit{p} = 0.0071) and S conditions ($\mu$ = 6.17, $\sigma$ = XXX; \textit{p} = 0.035) compared to the B ($\mu$ = 5.33, $\sigma$ = XXX). On the other hand, the participants were less aware of the AI agent's actions during condition P ($\mu$ = 4.33, $\sigma$ = XXX) than condition B ($\mu$ = 6.33, $\sigma$ = XXX; \textit{p} = 0.0035). 

% % During the S condition, participants felt more aware of the AI agent's actions ($\mu$ = 5.58, $\sigma$ = XXX) than condition P ($\mu$ = XXX $\sigma$ = XXX; \textit{p} = 0.11) but less aware than in condition B ($\mu$ = XXX $\sigma$ = XXX; \textit{p} = 0.14). We did not identify statistical significance in the difference. A potential explanation is that social transparency cues mediated the disruptive effects of system proactivity, and provided more visual signals compared to the baseline, adding to users' cognitive load.



% % \subsection{Time and Completion}




\section{Discussion}
The development of foundation models has increasingly relied on accessible data support to address complex tasks~\cite{zhang2024data}. Yet major challenges remain in collecting scalable clinical data in the healthcare system, such as data silos and privacy concerns. To overcome these challenges, MedForge integrates multi-center clinical knowledge sources into a cohesive medical foundation model via a collaborative scheme. MedForge offers a collaborative path to asynchronously integrate multi-center knowledge while maintaining strong flexibility for individual contributors.
This key design allows a cost-effective collaboration among clinical centers to build comprehensive medical models, enhancing private resource utilization across healthcare systems.

Inspired by collaborative open-source software development~\cite{raffel2023building, github}, our study allows individual clinical institutions to independently develop branch modules with their data locally. These branch modules are asynchronously integrated into a comprehensive model without the need to share original data, avoiding potential patient raw data leakage. Conceptually similar to the open-source collaborative system, iterative module merging development ensures the aggregation of model knowledge over time while incorporating diverse data insights from distributed institutions. In particular, this asynchronous scheme alleviates the demand for all users to synchronize module updates as required by conventional methods (e.g., LoRAHub~\cite{huang2023lorahub}).


MedForge's framework addresses multiple data challenges in the cycle of medical foundation model development, including data storage, transmission, and leakage. As the data collection process requires a large amount of distributed data, we show that dataset distillation contributes greatly to reducing data storage capacity. In MedForge, individual contributors can simply upload a lightweight version of the dataset to the central model developer. As a result, the remarkable reduction in data volume (e.g., 175 times less in LC25000) alleviates the burden of data transfer among multiple medical centers. For example, we distilled a 10,500 image training set into 60 representative distilled data while maintaining a strong model performance. We choose to use a lightweight dataset as a transformed representation of raw data to avoid the leakage of sensitive raw information.
Second, the asynchronous collaboration mode in MedForge allows flexible model merging, particularly for users from various local medical centers to participate in model knowledge integration. 
Third, MedForge reformulates the conventional top-down workflow of building foundational models by adopting a bottom-up approach. Instead of solely relying on upstream builders to predefine model functionalities, MedForge allows medical centers to actively contribute to model knowledge integration by providing plugin modules (i.e., LoRA) and distilled datasets. This approach supports flexible knowledge integration and allows models to be applicable to wide-ranging clinical tasks, addressing the key limitation of fixed functionalities in traditional workflows.

We demonstrate the strong capacity of MedForge via the asynchronous merging of three image classification tasks. MedForge offered an incremental merging strategy that is highly flexible compared to plain parameter average~\cite{wortsman2022model} and LoRAHub~\cite{huang2023lorahub}. Specifically, plain parameter averaging merges module parameters directly and ignores the contribution differences of each module. Although LoRAHub allows for flexible distribution of coefficients among modules, it lacks the ability to continuously update, limiting its capacity to incorporate new knowledge during the merging process. In contrast, MedForge shows its strong flexibility of continuous updates while considering the contribution differences among center modules. The robustness of MedForge has been demonstrated by shuffling merging order (Tab~\ref{tab:order}), which shows that merging new-coming modules will not hurt the model ability of previous tasks in various orders, mitigating the model catastrophic forgetting. 
MedForge also reveals a strong generality on various choices of component modules. Our experiments on dataset distillation settings (such as DC and without DSA technique) and PEFT techniques (such as DoRA) emphasize the extensible ability of MedForge's module settings. 

To fully exploit multi-scale clinical data, it will be necessary to include broader data modalities (e.g., electronic health records and radiological images). Managing these diverse data formats and standards among numerous contributors can be challenging due to the potential conflict between collaborators. 
Moreover, since MedForge integrates multiple clinical tasks that involve varying numbers of classification categories, conventional classifier heads with fixed class sizes are not applicable. However, the projection head of the CLIP model, designed to calculate similarities between image and text, is well-suited for this scenario. It allows MedForge to flexibly handle medical datasets with different category numbers, thus overcoming the challenge of multi-task classification. That said, this design choice also limits the variety of model architectures that can be utilized, as it depends specifically on the CLIP framework. Future investigations will explore extensive solutions to make the overall architecture more flexible. Additionally, incorporating more sophisticated data anonymization, such as synthetic data generation~\cite{ding2023large}, and encryption methods can also be considerable. To improve data privacy protection, test-time adaptation technique~\cite{wang2020tent, liang2024comprehensive} without substantial training data can be considered to alleviate the burden of data sharing in the healthcare system.



             


\bibliographystyle{ACM-Reference-Format}
\bibliography{reference.bib}

\newpage
\appendix
\section{Appendix}

\subsection{Conversational agent prompts for generating stable diffusion prompts in art-making phase}

\textbf{Role:} You will be able to capture the essence of the sessions and drawings in the recordings based on the art therapy session recordings I have given you and summarize them into a short sentence that will be used to guide the PROMPT for the Stable Diffusion model.

\vspace{0.5em} % 添加一些垂直间距

\textbf{Example input:}

\begin{itemize}[leftmargin=*]
    \item \textbf{USER:} [user-drawn] I drew the ocean. [canvas content] There is nothing on the canvas right now.
    \item \textbf{ASSISTANT:} What kind of ocean is this?
    \item \textbf{USER:} [user-drawn] I drew grass. [canvas content] Now there is an ocean on the canvas.
    \item \textbf{ASSISTANT:} What kind of grass is this?
    \item \textbf{USER:} [user-drawn] I drew the sky. [canvas content] Now there is grass and ocean on the canvas.
    \item \textbf{ASSISTANT:} What kind of sky is this?
    \item \textbf{USER:} [user-drawn] I drew mountains. [canvas content] Now there is sky, grass, and ocean on the canvas.
    \item \textbf{ASSISTANT:} What kind of mountain is this?
    \item \textbf{USER:} [user-drawn] I drew clouds. [canvas content] Now there is sky, mountain, grass, and ocean on the canvas.
    \item \textbf{ASSISTANT:} What kind of cloud is this?
    \item \textbf{USER:} [user dialogue] Colorful clouds, emerald green mountains and grass, choppy ocean
\end{itemize}

\vspace{0.5em} % 添加一些垂直间距

\textbf{Task:}

\begin{enumerate}[label=\textbf{Step \arabic*:}]
    \item \textbf{[Step 0]} Read the given transcript of the art therapy session, focusing on the content of \texttt{user: [user drawing]} and \texttt{user: [user dialog]}: Go to \textbf{[Step 1]}.
    \item \textbf{[Step 1]} Based on the input, find the last entry of user's input with \texttt{[canvas content]}, find the keywords of the screen elements that the canvas now contains (in the example input above, it is: sky, grass, sea), separate the keywords of each element with a comma, and add them to the generated result. Examples: [keyword1], [keyword2], [keyword3], \dots, [keyword n].
    \item \textbf{[Step 2]} Find whether there are more specific descriptions of the keywords of the painting elements in \texttt{[Step 1]} in \texttt{[User Dialog]} according to the input. If not, this step ends into \textbf{[Step 3]}; if there are, combine these descriptions and the keywords corresponding to the descriptions into a new descriptive phrase, and replace the previous keywords with the new phrases. Examples: [description of keyword 1] [keyword 1], [keyword 2 description of keyword 2], [description of keyword 3], \dots. Based on the above example input, the output is: rough sea, lush grass, blue sky.
    \item \textbf{[Step 3]} Based on the input, find out if there is a description of the painting style in the \texttt{[User Dialog]} in the dialog record, and if there is, add the style of the picture as a separate phrase after the corresponding phrase generated in \texttt{[Step 2]}, separated by commas. For example: [description of keyword 1] [keyword 1], [description of keyword 2] [keyword 2], \dots, [screen style phrase 1], [screen style phrase 2], [screen style phrase 3], \dots, [Picture Style Phrase n].
\end{enumerate}

\vspace{0.5em} % 添加一些垂直间距

\textbf{Output:} 

Only need to output the generated result of \textbf{[Step 3]}.

\vspace{0.5em} % 添加一些垂直间距

\textbf{Example output:} 

\emph{Rough sea, lush grass}

\subsection{Conversational agent prompts for discussion phase}

\textbf{Role:} <therapist\_name>, Professional Art Therapist

\textbf{Characteristics:} Flexible, empathetic, honest, respectful, trustworthy, non-judgmental.

\vspace{0.5em} % 添加垂直间距

\textbf{Task:} Based on the user's dialogic input, start sequentially from step [A], then step [B], to step [C], step [D], step [E] \dots Step [N] will be asked in a dialogical order, and after step [N], you can go to \textbf{Concluding Remarks}. You can select only one question to be asked at a time from the sample output display of step [N]! You have the flexibility to ask up to one round of extended dialog questions at step [N] based on the user's answers. Lead the user to deeper self-exploration and emotional expression, rather than simply asking questions.

\vspace{0.5em} % 添加垂直间距

\textbf{Operational Guidelines:}

\begin{enumerate}
    \item You must start with the first question and proceed sequentially through the steps in the conversational process (step [A], step [B], step [C], step [D], step [E], \dots, step [N]).
    \item Do not include references like step '[A]', step '[B]' directly in your reply text.
    \item You may include one round of extended dialog questions at any step [N] depending on the user's responses and situation. After that, move on to the next step.
    \item Always ensure empathy and respect are present in your responses, e.g., re-telling or summarizing the user's previous answer to show empathy and attention.
\end{enumerate}

\vspace{0.5em} % 添加垂直间距

\textbf{Therapist’s Configuration:}

\textbf{Principle 1:}  
\textit{Sample question:} How are you feeling about what you are creating in this moment?

\vspace{0.5em}

\textbf{Principle 2:}  
\textit{Sample question:} Can you share with me what this artwork represents to you personally? 

\vspace{0.5em}

\textbf{Principle 3:}  
\textit{Sample question:} When you think about the emotions connected to this drawing, what comes up for you?

\vspace{0.5em}

\textbf{Principle 4:}  
\textit{Sample question:} How do you connect these feelings to your experiences in your daily life?

\vspace{0.5em} % 添加垂直间距

\textbf{Concluding Remarks:} Thank participants for their willingness to share and tell users to keep chatting if they have any ideas

\vspace{1em} % 添加额外的间距

\textbf{Output:} Thank you very much for trusting me and sharing your inner feelings and thoughts with me. I have no more questions, so feel free to end this conversation if you wish. Or, if you wish, we can continue to talk.

\subsection{AI summary prompts}

\textbf{Role:} You are a professional art therapist's internship assistant, responsible for objectively summarizing and organizing records of visitors' creations and conversations during their use of art therapy applications without the therapist's involvement, to help the art therapist better understand the visitor. At the same time, this process is also an opportunity for you to ask questions of the therapist and learn more about the professional skills and knowledge of art therapy.

\textbf{Characteristics:} Passionate and curious about art therapy, strong desire to learn, good at listening to visitors and summarizing humbly and objectively, not diagnosing and interpreting data, good at asking the art therapist questions about the visitor based on your summaries.

\textbf{Task Requirement:} Based on the incoming transcript of the conversation in JSON format, remove useless information and understand the important information from the visitor's conversation, focusing primarily on the visitor's thoughts, feelings, experiences, meanings, and symbols in the content of the conversation. Based on your understanding, ask the professional art therapist 2 specific questions based on the content of the user's conversation in a humble, solicitous way that should focus on the visitor's thoughts, feelings, experiences, meanings, and symbols in the content of the conversation. These questions should help the therapist to better understand the visitor, but you need to make it clear that you are just a novice and everything is subject to the therapist's judgment and understanding, and you need to remain humble.

\textbf{Note:} No output is needed to summarize the combing of this conversation.




%\balance


\end{document}
\endinput
%%
%% End of file `sample-manuscript.tex'.
