\section{FINDINGS}
\subsection{RQ1: How Families Interacted with \name{} to Learn Vocabulary in Home Settings}
In this pilot study, we contextualized the process of families interact with \name{} to learn vocabulary: In \textbf{Real-Time Q\&A} phase, several parents utilized real-time Q\&A interactions to expand their knowledge of vocabulary-related topics, e.g, C1 mentioned that they enhanced children's vocabulary by asking for additional words related to those they had already learned: ``\textit{We saw Peppa eating spaghetti. I asked the AI, `Why is the pasta yellow?' and it explained that it's made from wheat, helping me learn why it's yellow}''. In \textbf{Active Vocabulary Review} phase, interactions between parents and children often followed a turn-taking pattern. When C7 hesitated to review certain vocabulary, the parent acted as a ``role model'', answering the conversational agent's questions and demonstrating correct responses. This encouraged C8 to motivate the child to answer the next question, gradually fostering more active participation in the review process. In \textbf{Real-World Association} phase, C4 utilized real-life object images from \name{} to encourage children to discuss whether they have these items at home, connecting cartoon vocabulary to their living environment. Similarly, C6 used real-life images to spark multisensory discussions, asking questions like, ``What does this onion smell like? Do you remember the sound when Dad was cutting it that day?''. In \textbf{Contextual Expansion} phase, C10 customized sentences based on the vocabulary children learn, focusing on topics they frequently encounter in daily life, e.g., food-related topics helped children understand the various uses of different foods. Meanwhile, the AI-generated stories for C8 and C7 adopted a more interactive approach, encouraging them to take on roles within the narrative and retell the story.

\subsection{RQ2: How They Perceived Cartoon-based Vocabulary Learning with \name{}}
In our study, parents valued the system for its ability to provide personalized and engaging learning experiences for preschool children. 
C10 appreciated that AI can offer tailored responses based on the child's vocabulary level: ``\textit{He only knew basic words like `apple' and `banana', and I found that the AI responded with age-appropriate, simple sentences. The voice was also cute, something I couldn't replicate}".
C8 also noted that AI-generated images might enhance children's engagement during sentence expansion: ``\textit{For 3-year-olds with limited interest in text, highly relevant visuals combined with expanded sentences can better capture their attention, and I noticed she really enjoyed these images}.

Some parents appreciated the system for encouraging self-reflection on their parenting. For example, C4 noted how the AI patiently answered C3's random questions, saying, ``\textit{I asked how many pigs were in the animated scene... and then if I could provide a formula. The AI answered even absurd queries, and C3 enjoyed chatting with it. I didn't respond to these questions and began to wonder if I might be a bit too impatient}''.

Finally, our parents reported that \name{} can foster their collaboration during learning vocabulary. C6 stated that the real-time Q\&A might as a great medium to spark more questions, encouraging deeper communication beyond vocabulary: ``\textit{I inquired about what George was holding, and the AI responded with a dinosaur. I then asked my child if he knew how dinosaurs became extinct. After he shook his head, I encouraged him to ask the AI}''. Also, C2 perceived the teachable agent as a bridge during vocabulary review, encouraging the child to seek help: ``\textit{The AI prompted the child to ask about words they'd learned. If the child didn't understand, they'd come to me for an explanation, which I encouraged them to share with the AI}''.