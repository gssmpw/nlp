\vspace*{-0.2cm}\section{Pilot Study and Evaluation}
We conducted a remote pilot study to explore two key questions: how families engage with AI-infused cartoon videos to learn vocabulary in home settings (\textbf{RQ1}), and how they perceive cartoon-based vocabulary learning with \name{} (\textbf{RQ2}).

\subsection{Participants}
The research team recruited family groups by distributing digital flyers on a social media platform, describing the pilot study as an activity to enhance preschoolers' language learning through a digital cartoon video system. Five families participated, comprising 10 individuals: 5 children (ages 3–5), 4 mothers, and 1 father. Participant demographics are detailed in APPENDIX \autoref{tab:familystudy}, with all families sharing the same native language to ensure cultural and linguistic consistency. The study was approved by the institutional ethics board, and pseudonyms were used to protect anonymity. Each family received 100 RMB as compensation.

%This age group was chosen for several reasons. Firstly, it represents a critical period for early language development, where children begin to acquire foundational vocabulary and communication skills. Secondly, this stage heavily relies on parent-mediated learning, which is central to our research focus on  language acquisition and parent-child interaction.

%我们通过重新联系上一轮formative study中的实验者以及滚雪球的方式,招募了3对参与者。
%每对参与者由一名2-5岁的儿童和一名家长组成
%所有参与者均来自中国南方,以中文为母语。
%每对参与者都获得了100RMB作为补偿
\subsection{Study Procedure and Data Analysis}
Each hour-long user study session was conducted remotely via Tencent Meeting. Families accessed \name{} on their personal computers or tablets through a web browser, sharing their screens with researchers and enabling cameras when possible. Before the study, We held a 20-minute online session to introduce the activity, using slides to demonstrate \name{}'s interface and features. Families were encouraged to explore the system freely, with parents asking 2-3 questions based on their understanding of the cartoon videos and motivating their children to ask questions as well. After the introductory session, each family pair used \name{} to watch an episode of \textit{Peppa Pig}. This cartoon was chosen for its suitable length, everyday language, rich visual cues, and relevance to children's lives, aligning with language development goals. Parents utilized \name{}'s interactive features during the four phases (Section 3.2) to support collaborative vocabulary learning. Researchers acted only as technical support, without intervening in the session. After using \name{}, we conducted 20-minute semi-structured interviews with each parent to explore our research questions. All sessions were recorded, totaling 347 minutes of audio and 216 minutes of screen recordings, which were transcribed for thematic analysis~\cite{braun2006using}.
%Two researchers conducted inductive coding on the 30-minute interviews, extracting quotes, identifying patterns, and developing a thematic framework. The team also reviewed and annotated session videos, selecting key segments to support the findings.



%We adopted a thematic analysis approach and triangulated the transcriptions with observational notes to enrich the analytical process. The first author independently coded 20\% of the data to generate an initial codebook. Subsequently, all authors engaged in a collaborative discussion to review individual coding outcomes, address discrepancies, and finalize the codebook (achieving an inter-coder reliability exceeding 90\%). The finalized codebook was then employed by the first author to systematically code the remaining dataset.


%\subsubsection*{Session 2: Semi-structured Interview (20 minutes)}
%The final stage of the study consisted of a 20-minute semi-structured interview with each parent. These interviews sought to gather in-depth insights into the parents’ experiences using AnimAlte, focusing on usability, effectiveness in supporting children’s learning, and the dynamics of parent-child interaction. Parents were encouraged to share their perspectives on their children’s language learning practices, their overall experiences with AnimAlte, and any challenges they encountered. Additionally, they were invited to offer suggestions for enhancing AnimAlte’s functionality. These interviews provided valuable qualitative data to assess AnimAlte’s impact on parent-child learning interactions and its practical utility in everyday contexts.




%Then, we held online introductory sessions with families to explain activities and demonstrate \name{}We began each session with a 10-minute introductory phase. Due to geographical limitations, all sessions were conducted remotely using . Participants were encouraged to keep their cameras on throughout the session to facilitate observation and interaction. Before starting the session, participants signed an informed consent form to confirm their voluntary participation. The researcher provided a brief overview of the study’s objectives and played a 4-minute instructional video pre-recorded by the research team. This video demonstrated the core functionalities and operational steps of AnimAlte. After viewing the video, parents accessed the AnimAlte platform using Chrome browsers on their own devices, such as laptops, iPads, or other tools.

%\subsubsection*{Session 1: Tasks with AnimAlte (30 minutes)}
%In the subsequent 30-minute session, each parent-child dyad engaged with AnimAlte to explore its full set of features, including animated video playback, interactive visual question-and-answer activities, vocabulary review, and vocabulary extension. Using AnimAlte’s collaborative viewing mode, children and parents watched videos together, leveraging AnimAlte to facilitate interactive discussions about the scenes. Parents or children could pause the video at any point to ask scene-specific questions, which AnimAlte answered in real time using its Visual Language Model (VLM)-driven contextual and visual comprehension capabilities. Questions posed during the session were logged and revisited in a follow-up review phase, where children or parents were prompted to answer them, reinforcing learning outcomes.


%Upon completing the video viewing and review phases, AnimAlte guided children through two additional modules: \textit{“Association”} and \textit{“Extension”}. These modules linked vocabulary from the animated content to real-world scenarios, enabling children to enhance their understanding through sentence construction and image generation tasks. This stage aimed to ensure that children not only remembered the vocabulary but also grasped its contextual meanings and could apply it flexibly. Figure~\ref{fig:example} illustrates a typical session setup, where parents sat beside their children, supporting them throughout the AnimAlte session. Parents actively participated in their children’s learning while engaging in natural conversations, fostering a dynamic and collaborative learning environment.





%每个用户研究将持续1个小时左右的时间,由于地区的问题,我们采用了腾讯会议的形式进行远程参加。如果条件允许,我们鼓励参与者在全程开启摄像头。所有用户研究均有视频全程记录。
%在每次实验开始前,并签署了同意书。
%在实验人员简要介绍研究内容后,家长观看又研究员录制的长度为4分钟的软件使用介绍,学习如何使用AnimAlte。
%在完成学习后,家长将使用自己的笔记本电脑,ipad或其他工具上的chrome浏览器,打开由研究人员发送的AnimAlte网址。
%每组参与者虽然完整的使用两次AnimAlte的全部功能,包括卡通片观看,画面问答,词汇复习和词汇拓展。
%在家长与AI的共同陪伴观看模式下,儿童和家长开始观看视频,并借助AnimAlte促进针对卡通片画面的互动问答。家长和儿童可以随时点击暂停按钮,并提出画面相关的问题,AnimAlte将及时理解画面,并给出回答。在此期间,家长和儿童均可以提出问题,而所有的问题将在视频播放完成后,以反问的复习形式被重新提出,由儿童进行回答。
%在复习完成后,AnimAlte将通过关联和拓展两部分,帮助儿童通过动画联系生活实际场景,并通过拓展造句和文生图的方式,帮助儿童更好的理解词语的含义和运用方式。
%图X展示了在我们的研究中,家长和孩子如何与AnimAlte互动的情景:家长坐在孩子旁边,陪伴孩子一起使用AnimAlte进行动画片的观看和词汇的学习,在整个过程中,他们可以随时进行相互聊天。
%在尝试使用AnimAlte后,我们对每位参与家长进行了20分钟的半结构化访谈,了解他们与AnimAlte互动的体验。 