\section{FORMATIVE STUDY}
%Recent HCI research highlighted the growing importance of AI-enhanced video learning in educational contexts~\cite{tanprasert2023scripted,kim2023older,tang2021conceptguide,adcock2010talkminer}. %However, we have little knowledge about how AI integrates into cartoon videos to provide a systematic and flexible language guidance for preschool children and also promote parental involvement at home. 
To inform the design of \name{}, we first conducted semi-structured interviews with parents to understand current practices and challenges in using cartoon videos to support preschoolers' language development. 
Next, we interviewed experts to discuss challenges identified from the parent interviews and explore the potential role of AI in enhancing language learning for preschoolers.

\subsection{Procedure and Analysis}
First, we recruited six families (mothers and their children aged 2–5) through a previously partnered kindergarten in [the city of Anonymity], with demographics detailed in APPENDIX \autoref{tab:family1}.
All families had experience using cartoon videos for language learning. We conducted 30-minute interviews with each parent to explore their practices and challenges in using cartoons for vocabulary development. Additionally, we held one-hour interviews with three child language specialists and a speech-language pathologist, all experienced in early childhood education. We introduced AI technologies such as visual language models (VLMs), large language models (LLMs), and image-based generative AI, emphasizing their natural language capabilities, and invited insights on their potential for video-based learning.

All sessions were recorded, transcribed, and analyzed using open coding and affinity diagramming~\cite{braun2006using}. The study was approved by the research ethics board, with pseudonyms used for participants. Parents received 100 RMB (~13 USD), and experts were compensated at their standard hourly rate.
%The \textit{Peppa Pig} cartoon videos were selected as a research probe in this study for several reasons: (1) target audience: it is specifically designed for preschool-aged children; (2) content simplicity: the episodes have clear storylines and simple language, making them suitable for young children; (3) visual and auditory appeal: bright colors, simple animation, and engaging voices effectively capture and hold young children's attention.

%contacted s located in southern China and 
%Each family consisted of at least one parent and one child aged 2 to 5 years. It is worth noting that all participating parents were female, primarily mothers or sisters. 

%Through a literature review, we found that there is relatively little research on applying video learning to early language acquisition. Inspired by theories in education and linguistics, we divided the study into two parts: demonstration sessions and expert interviews. The study aims to identify pain points for families in early language education, gather feedback on the acceptance of video language learning, and obtain expert suggestions on the interaction process, which provide important guidance for our system development.

%\subsection{Technology Probe}
%In designing and developing the Probe system, we incorporated the latest technologies and design concepts to explore preschool children's comprehension and acceptance in video learning.

%According to Mauricio Pilleux's theory of auditory input-driven language acquisition, timely feedback and question-and-answer sessions are key to enhancing children's language learning. Based on this theory, we designed a video question-and-answer demo that combines audio-visual question answering (AVQA) with text-to-speech (TTS) technology. While watching the video, users can ask questions related to the content, and the system understands and responds in real-time. For example, when a child asks, "What toy is George holding?" the system responds instantly, "George is holding a green dinosaur," enhancing interaction and the learning experience.

%The core concept of the system is "contextualized learning." It analyzes video content using a multimodal model and provides precise feedback based on contextual information. Parents can collaborate with the system to observe their child's language development in real-time, identify shortcomings, and offer support.

%

%\subsubsection{Parents and children.}After obtaining ethical approval from the institution for the research, we directly contacted kindergartens located in southern China and recruited 10 families to participate in the study (a total of 6 boys and 4 girls). Each family consisted of at least one parent and one child aged 2 to 5 years. It is worth noting that all participating parents were female, primarily mothers or sisters. To understand how parents guide preschool children’s language learning in their daily lives and to collect their feedback and suggestions on the technology probe experience, we conducted approximately one-hour semi-structured interviews. The interviews focused on parents' practices and challenges in language learning, particularly their acceptance of video learning and views on interactive feedback. Families received 50 RMB as a thank-you after participating, while participants who withdrew early were not compensated.
%\subsubsection{Experts.}In order to obtain authoritative feedback and further optimize the system design, we invited four experts from fields such as child education and educational psychology to participate in interviews. Besides one doctoral candidate with many years of experience in children's education programs, the other three experts are currently professionals engaged in preschool children's language education at relevant educational institutions. The backgrounds and experiences of these experts will provide us with valuable insights, helping us to gain a deeper understanding of the system's application and potential in children's language learning, as well as guiding future system optimization.

%\subsection{Formative Investigation}
%At the beginning of the experiment, we first explained the purpose of the study to the parents, clearly stating that the aim was to support preschool children's daily language learning through AI technology. 

%Next, we discussed with the parents their practices and challenges in language learning, particularly how they promote their children's language development in daily life, as well as their acceptance and views on video learning methods. 

%Following this, the parents and children experienced the technology probe system together, interacting through video Q&A. Researchers provided necessary support but tried to avoid excessive intervention in order to collect feedback from parents and children about the system. During the interaction, we focused on recording the quality of children's responses to the questions, including whether there was a verbal response, the number of words in the response, and the relevance of the answers. At the same time, the research team also documented the children's behavioral performances, such as visual attention (e.g., whether they looked at the screen, at family members, or elsewhere), facial expressions (e.g., smiling, frowning), and non-verbal expressions (e.g., nodding or shaking their heads). 

%At the end of the experiment, we conducted interviews with the parents again to understand their feelings about the video learning experience and to solicit their suggestions for improving the system.

%\definecolor{customcolor}{HTML}{FEF9F2} % 定义颜色
%\begin{table}[ht]
%\label{tab:2}
%\centering
%\begin{tabularx}{\columnwidth}{XXXX}
%\toprule
%ID & Profession  & Experience & Case \\\rowcolor{customcolor}
%\midrule
%01 & SLP         & 6 yrs      & 100+ \\
%02 & SLP         & 8 yrs      & 100+ \\\rowcolor{customcolor}
%03 & Researcher  & 1 yr       & -    \\
%\bottomrule
%\end{tabularx}
%\caption{Expert}
%\end{table}

\subsection{Findings from the Interviews}
\subsubsection{CH1: Challenges in Interactive Feedback in Cartoon-Based Vocabulary Learning} 
Most parents mentioned that their children lack interactive feedback while learning vocabulary through cartoon videos. P6 mentioned that the lack of interaction with children may limit children's motivation to express curiosity: ``\textit{My child is naturally curious. He asked about the meaning of taxi colors while watching cartoons. If I'm distracted by my phone, he loses interest in continuing to ask}''. Our experts highlighted that the lack of interactive feedback in vocabulary learning through cartoons might result in poor outcomes, as ``\textit{language is a two-way tool, while cartoons offer one-way interaction, leaving children passive and unable to correct mistakes or deepen their understanding(E1)}''.

Experts emphasized that parental involvement, through interactive feedback, can support preschool children's vocabulary learning and strengthen the parent-child relationship during cartoons. 
However, some parents mentioned struggling to understand and respond to their children's questions, leading to communication barriers. P10 noted that when children asked questions about the plot of the animation,  it might be difficult for her to answer if she had not watched the cartoon, thus reducing the interaction with the children. Also, P12 mentioned that her child's questions about Vienna were beyond her knowledge.
Further, many parents struggled to adjust their communication to their child's developmental stage, e.g., P4 may need to speak more slowly and use simple, cute words to help their child understand, as the child might not fully grasp it otherwise. 
 

\subsubsection{CH2: Challenges in Vocabulary Review through Cartoon Videos}
Our parents indicated that using cartoon videos to help preschool children learn new vocabulary might lack support for repetitive review. 
For instance, P8 suggested that cartoons may not offer enough memory support for recalling newly learned vocabulary: ``\textit{The last time we watched Frozen, my child learned the word `brave', but after the cartoon ended, P7 didn't seem to mention the word again}''.
To address this, P4 replayed specific cartoon segments to help her children review vocabulary. P8 wrote vocabulary or sentences on the walls to read with her child, while E1 took screenshots of cartoon scenes for effective review.
Some parents complained that when their child asked too many questions or discussed too much vocabulary during an episode, the review process became time-consuming and tedious, burdening them with organizing and revisiting the material.

\subsubsection{CH3: Challenges in Bridging the Connection Between the Virtual and Real Worlds}
Most parents noted that characters and objects in cartoon videos often seem disconnected from the real world, e.h., P8 worried that anthropomorphized characters in cartoons might confuse her cognition, so she used real photos of objects, like cards with apples, to help the child learn these words.
E1 explained that disconnection from the real world could hinder learners' ability to relate new words to real-life contexts, making it harder to internalize and apply them meaningfully.
E1, E3, and E4 stressed the importance of comparing animated items with real objects to aid children's vocabulary understanding. E1 helped children recognize cartoon items by buying their real-life counterparts, while E4 connected the cartoon sun to real sunsets and noons, helping children identify the sun in various images and link the word to their everyday experiences.

\subsubsection{CH4: Challenges in Contextualizing Vocabulary Beyond Cartoon Videos}
Several parents mentioned that learning vocabulary through watching cartoon videos sometimes lacks application and extension in various contexts.
For example, P4 observed that after watching cartoons, their child could name objects and animals but struggled to talk about them in different settings. P6 noted that learning vocabulary and events through cartoons might lack sentence expansion or story extension, limiting the child's broader thinking: ``\textit{My child learned the name `apple', but I felt that wasn't enough. I wanted to expand by sharing the story of Newton and the apple}''.
E1 emphasized the importance of expanding sentences and stories, noting that learners can not only recognize an onion's appearance but also understand its taste and uses through stories and sentences, deepening their vocabulary and gradually forming a semantic network.






%\subsubsection{Importance of AI-structured Systematic Language Learning Guidance for Children's Vocabulary Acquisition} 

%In our interviews, about 6/10 parents reported that the existing technology probe system lacks a systematic learning process, particularly in areas such as review, spelling, vocabulary explanation, and association. Parents expect the system to provide a scientifically sound and logically clear comprehensive language learning plan, rather than just serving a "supportive" function. One parent mentioned, "The Q&A feature is good, but simply identifying the words that the child doesn't know is not enough. Without repetition and review, what the child learned in the morning can easily be forgotten by the evening." Another parent showcased her family’s "Peppa Pig" vocabulary book and shared how she uses it at home: "The content of the book is completely consistent with the animated series. After watching an episode, I reinforce the child's memory of the vocabulary through the questions in the book."

%In the field of language education, vocabulary review and reinforcement are widely acknowledged as essential components of language learning. Research by Mauricio Pilleux indicates that using dialogues with rhetorical questions as review materials is an effective tool for promoting children's recall of forgotten content. Through repeated exposure and practice with these dialogues, children can smoothly transition to higher-order learning content. Therefore, systematic review not only helps children consolidate newly learned vocabulary but also deepens their memory, enhancing the continuity and coherence of language learning.

%Based on parental feedback, to better meet parents' needs in language education, the system should help children reinforce vocabulary memory through repeated questioning and answering. A systematic learning process can assist children in deepening their grasp of new vocabulary, thereby effectively supporting the language learning process and improving learning outcomes.

            
%\subsubsection{Multimodal Interaction: Accelerating Children's Vocabulary Association and Expansion}In our interviews, 7/10 parents mentioned the issue of a monotonous daily learning mode, especially for younger children, who are exposed to very limited forms of media primarily focused on picture books, everyday conversations, videos, and music. Parents reported that although the existing educational materials are rich in content, their presentation is overly simplistic and lacks multimodal connections, resulting in a gradual decline in children's interest and focus. For example, one parent mentioned, "I usually read picture books to my child. In the beginning, he was quite interested, but after a few times, he no longer wanted to look at them." Additionally, two parents spoke about their children's experiences listening to children's songs. They noted that although their children learned to sing after repeated listenings, without the combination of images or context, the children did not truly understand the meaning of the songs.

%Research has shown that multisensory interaction significantly promotes the effectiveness of language learning in children. In language learning, creative activities that integrate text, images, and sounds through multimodal approaches not only enhance children's understanding of the content but also better capture their attention and improve language skills. Compared to single-mode learning methods, this multimodal approach tends to more effectively stimulate children's interests while helping them achieve a deeper internalization of knowledge through interaction.

%Therefore, our research concludes that systematic design should focus on a multimodal approach, combining dynamic content, interactive functions, and parental involvement to create a comprehensive platform that can stimulate children's interests, promote semantic understanding, and enhance language learning efficiency. This design not only maintains children's motivation to learn but also aids them in better understanding and mastering language content through a richer learning experience.

           
%\subsubsection{Our System Enhancing Parent-Child Interaction and Stimulating Learning Potential}
%In our interviews, about 4 out of 10 parents mentioned the issue of parental involvement in interactions. Some parents indicated that the system should encourage them to participate more actively in their children's learning processes. One parent noted, "If the system could add some prompts or pose some questions to me, it would help me engage more consciously." Another parent stated, "When my child watches videos, I often find myself scrolling through my phone beside them, but with the addition of this system, I feel that watching videos with my child has become really interesting." These feedbacks highlight the significant role of active parental participation in promoting children's language learning.

%In educational psychology, Vygotsky (1978) emphasized the importance of dialogue between children and the people around them, arguing that these interactions form the basis for language and cognitive development. He pointed out that thinking is essentially internalized language, and language develops through social interactions. Additionally, Vygotsky suggested that children perform better under adult guidance than when working independently, as adult guidance enhances their ability to understand and complete tasks.

%Therefore, in order to maximize the system's support for children's language learning, we believe that system design should emphasize the active role of parents in interactions. By providing system prompts and guiding parental involvement, we can enable parents to become co-facilitators of their children's language learning, thus creating a more effective learning support environment within the family.

%\subsection{Experts group interview}

\begin{figure*}[ht]
    \centering  \includegraphics[width=1\linewidth]{images/jiemian.pdf}
    \caption{User interface flow of \name{} (A), episode navigation (B), cartoon watching and question answering (C), review (D), real-world association (E), and contextual expansion (F).}
    \label{fig:ui_interface}\vspace{-0.5cm}
\end{figure*}