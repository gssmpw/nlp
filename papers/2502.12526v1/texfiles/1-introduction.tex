\section{Introduction}
Vocabulary plays a vital role in early childhood literacy development~\cite{song2015tracing,hoffman2014assessing}.
Research consistently shows that a well-developed vocabulary in preschool children might improve social interaction~\cite{erdemir2022vocabulary}, future academic success~\cite{cox2014children}, and reading comprehension~\cite{hadley2016examining}. 
An effective strategy for supporting preschool children's vocabulary learning is using cartoon videos, thanks to their unique style and expression~\cite{Vitasmoro2019/11,meng2020influence}.
Cartoon videos with repetitive input and contextual learning can engage children while improving vocabulary, pronunciation, and interest in language learning~\cite{perween2020impact,okunolaexploring,arifani2020cartoon}.
However, most cartoon videos rely on passive input, offering limited opportunities for interaction with either the content or caregivers, which might reduce their effectiveness for language learning~\cite{10.1145/3078072.3079717,meng2020influence}.
Learners also tend to forget information from video-based learning since it isn't systematically processed, retained, or reviewed~\cite{goodianti2007learners}. 
Additionally, studies suggest young children may struggle to distinguish between real and fantastical events, hindering their ability to apply the language learned from cartoon in different context of real life~\cite{Beege2024,Prosic-Santovac2017,Vitasmoro2019/11,li2015can}.

In the field of Human-Computer Interaction (HCI), AI-infused video learning has emerged as an innovative approach within the context of education and learning~\cite{10.1145/3544548.3581153,10.1145/3544549.3585804,10.1145/3442381.3449808,10.1145/3613904.3642587}. 
For example, Xu et al. showed that incorporating a conversational agent into video programs, which engages children with questions and feedback, enhances active learning during video watching~\cite{10.1145/3491102.3502050}.
However, few studies have explored how AI conversational agents can understand video frames to provide offer personalized, interactive feedback that enhances preschoolers' language learning.
In the non-video learning context, research shows that large language models (LLMs) improve vocabulary learning by generating personalized, contextually relevant content~\cite{leong2024putting}. 
Additionally, AI-generated images may aid in learning target words through reading and retelling a story~\cite{chen2024retassist}. 
These studies highlight the potential of AI to enhance preschool language learning, suggesting an opportunity to integrate AI into cartoon videos for a more systematic and multimodal approach.

To this end, we proposed an AI-infused cartoon video system, \name{}, designed to support preschool children's vocabulary learning~(\autoref{fig:teaser}). 
This system features four phases for supporting vocabulary learning with family engagement through real-time question-answering during cartoon videos, active review with questions and feedback, real-world associations by linking animated and real-life images, and contextual expansion with sentences or stories tailored to topics and age groups. 
In our late-breaking work, we conducted a user study involving 5 pairs of parents and children to explore how families interact with \name{} for vocabulary learning in home settings (\textbf{RQ1}) and how they perceive cartoon-based vocabulary learning using \name{} (\textbf{RQ2}).
The preliminary findings helped contextualize the process of how families interact with \name{} to learn vocabulary. The data also illustrated how parents appreciated \name{} for providing their personalized and engaging learning experiences, helping them reflect on their parenting and fostering their collaboration.