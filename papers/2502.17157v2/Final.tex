\documentclass[10pt,twocolumn,letterpaper]{article}

\usepackage{iccv}
\usepackage{times}
\usepackage{epsfig}
\usepackage{graphicx}
\usepackage{amsmath}
\usepackage{amssymb}

\usepackage{booktabs}
\usepackage{multirow}
\usepackage{adjustbox}

\usepackage{caption}
\captionsetup{labelfont=bf,tableposition=top,font=small}

\usepackage{float}

\usepackage{xcolor}
\newcommand{\red}[1]{\textcolor{red}{#1}}

\usepackage[colorlinks=true,citecolor=blue,linkcolor=blue]{hyperref}

\usepackage{bold-extra}
\usepackage{algorithm}
\usepackage{algorithmic}



\iccvfinalcopy % *** Uncomment this line for the final submission

\def\iccvPaperID{****} % *** Enter the ICCV Paper ID here
\def\httilde{\mbox{\tt\raisebox{-.5ex}{\symbol{126}}}}

% Pages are numbered in submission mode, and unnumbered in camera-ready
% \ificcvfinal\pagestyle{empty}\fi

\begin{document}

%%%%%%%%% TITLE
\newcommand{\ours}{{\sc Di\-Ce\-p\-ti\-on}}
\newcommand{\oursbf}{{\bfseries\scshape Di\-Ce\-p\-ti\-on}}



\title{\oursbf: A Generalist Diffusion Model for 
Visual
Perceptual Tasks
}


\author{%
  Canyu Zhao$^{\rm 1,*}$ ~~ Mingyu Liu$^{\rm 1,2,}$\thanks{Equal Contribution} ~~ Huanyi Zheng$^{\rm 1}$ ~~ Muzhi Zhu$^{\rm 1}$ ~~  
  \\ Zhiyue Zhao$^{\rm 1}$ ~~
  Hao Chen$^{\rm 1}$ ~~ Tong He$^{\rm 2}$ ~~ Chunhua Shen$^{\rm 1}$\\[0.25cm]
  \textsuperscript{\rm 1} Zhejiang University~~~~~
  \textsuperscript{\rm 2} Shanghai AI Laboratory\\
}


%CS: 
\input fig1.tex



\maketitle
% Remove page # from the first page of camera-ready.

% \ificcvfinal\thispagestyle{empty}\fi

%%%%%%%%% ABSTRACT
\begin{abstract}

Our primary goal here is to create a good, generalist perception model that can tackle multiple tasks, within limits on computational resources and training data. To achieve this, we resort to text-to-image diffusion models pre-trained on billions of images and introduce our visual generalist model: \oursbf. 
Our exhaustive evaluation metrics demonstrate that \ours\ effectively tackles multiple perception tasks, achieving performance on par with state-of-the-art models.
\textbf{We achieve results on par with SAM-vit-h using only 0.06\% of their data (\textit{e.g.}, 600K vs.\ 1B 
   pixel-level annotated 
   images)}.
Inspired by
Wang \textit{et al.}\ 
\cite{wang2023images}, 
   \ours\ formulates the outputs of various perception tasks using color encoding;
   and we show
   that the strategy of assigning random colors to different instances is highly effective in both entity segmentation and semantic segmentation. 
   Unifying various perception tasks as 
   conditional 
   image generation
   enables us to fully leverage  
   pre-trained text-to-image models. Thus,
   \ours\ can be efficiently
   trained at a  cost of orders of magnitude lower,
   compared to conventional models that were trained from scratch.
   When adapting our model to other tasks, it only requires fine-tuning on as few as 50 images and $\sim$1\% of its parameters.
   \ours\ provides valuable insights and a more promising solution for visual generalist models. 
   \href{https://aim-uofa.github.io/Diception/}{Project webpage}, and 
   \href{https://huggingface.co/spaces/Canyu/Diception-Demo}{huggingface demo} are available. 
   
\end{abstract}

%%%%%%%%% BODY TEXT
\section{Introduction}
\label{sec:intro}

\begin{figure*}[tb]
    \centering
    \includegraphics[width=0.848\linewidth]{figs/circuitnn.pdf} 
    \caption{Illustration of differentiable CircuitNN. CircuitNN is designed based on differentiable NAND gates. After DAS is guided by PI and PO pairs of the truth table, CircuitNN can get the precise circuit architecture logic equivalent to the truth table.}
    \label{fig:circuitnn}
\end{figure*}

% 1. Describe the importance of logic synthesis
% 2. Existing Problems
% (a) Neural Architecture Search: Unstable, Predefined Setting, etc.
% (b) Circuit Generation: Probabilistic Model, Logic Equivalence

With the rapid advancement of technology, the scale of integrated circuits (ICs) has expanded exponentially. 
This expansion has introduced significant challenges in chip manufacturing, particularly concerning power and area metrics.
A primary objective in IC design is achieving the same circuit function with fewer transistors, thereby reducing power usage and area occupancy.

Logic synthesis~\cite{hachtel2005logicsynth}, a critical step in electronic design automation (EDA), transforms behavioral-level circuit designs into optimized gate-level circuits, ultimately yielding the final IC layout. 
The primary goal of logic synthesis is to identify the physical implementation with the fewest gates for a given circuit function. 
This task constitutes a challenging NP-hard combinatorial optimization problem. 
Current logic synthesis tools~\cite{brayton2010abc, wolf2013yosys} rely on human-designed heuristics, often leading to sub-optimal outcomes.

Differentiable architecture search (DAS) techniques~\cite{liu2018darts, chu2020darts} offer novel perspectives on addressing challenges in this problem.
Circuit functions can be represented through truth tables, which map binary inputs to their corresponding outputs. 
Truth tables provide a precise representation of input-output relationships, ensuring the design of functionally equivalent circuits.
Inspired by this, researchers~\cite{deepmind2024ai4sys, wang2024tnet} have begun exploring the application of DAS to synthesize circuits directly from truth tables.
Specifically, \citet{deepmind2024ai4sys} proposed CircuitNN, a framework that learns differentiable connection structures with logic gates, enabling the automatic generation of logic circuits from truth tables.
This approach significantly reduces the complexity of traditional circuit generation. 
Building on this, \citet{wang2024tnet} introduced T-Net, a triangle-shaped variant of CircuitNN, incorporating regularization techniques to enhance the efficiency of DAS.

Despite these advancements, several challenges remain. 
The computational complexity of DAS grows quadratically with the number of gates, posing scalability issues.
Although triangle-shaped architecture~\cite{wang2024tnet} partially mitigates this problem, redundancy persists. 
%Additionally, DAS is susceptible to converging to local optima, limiting the ability to search architectures that satisfy the given truth tables~\cite{liu2018darts}. 
%Furthermore, hyperparameters (network depth and layer width) require extensive searches, introducing complexity and prolonging the synthesis process. 
Additionally, DAS is susceptible to converging to local optima~\cite{liu2018darts} and hyperparameters (network depth and layer width) require extensive searches. 
The challenges arise from the vast search space in DAS. 
% Even with predefined settings for CircuitNN, finding a configuration that meets the truth table requires extensive trial and error during the DAS process. 
Intuitively, limiting the search space through predefined parameters (network depth, gates per layer, and connection probabilities) can significantly reduce the complexity.

Recent advances~\cite{openai2023gpt4, abramson2024alphafold3, esser2024sd3, li2024mar} in conditional generative models have demonstrated remarkable performance across language, vision, and graph generation tasks. 
Motivated by these developments, we propose a novel approach to circuit generation that generates preliminary circuit structures to guide DAS in generating refined circuits matching specified truth tables. 
Firstly, we introduce CircuitVQ, a tokenizer with a discrete codebook for circuit tokenization. 
Built upon our Circuit AutoEncoder framework~\cite{hou2022graphmae,li2023maskgae,wu2025mgvga}, CircuitVQ is trained through a circuit reconstruction task. 
Specifically, the CircuitVQ encoder encodes input circuits into discrete tokens using a learnable codebook, while the decoder reconstructs the circuit adjacency matrix based on these tokens.
Subsequently, the CircuitVQ encoder serves as a circuit tokenizer for CircuitAR pretraining, which employs a masked autoregressive modeling paradigm~\cite{chang2022maskgit, li2023mage}. 
In this process, the discrete codes function as supervision signals. 
After training, CircuitAR can generate discrete tokens progressively, which can be decoded into initial circuit structures by the decoder of the CircuitVQ. 
These prior insights can guide DAS in producing refined circuits that match the target truth tables precisely.

Our key contributions can be summarized as follows:
\begin{itemize}
\item We introduce CircuitVQ, a circuit tokenizer that facilitates graph autoregressive modeling for circuit generation, based on our Circuit AutoEncoder framework;
\item Develop CircuitAR, a model trained using masked autoregressive modeling, which generates initial circuit structures conditioned on given truth tables;
\item Propose a refinement framework that integrates differentiable architecture search to produce functionally equivalent circuits guided by target truth tables;
\item Comprehensive experiments demonstrating the scalability and capability emergence of our CircuitAR and the superior performance of the proposed circuit generation approach.
\end{itemize}

% Motivation
% (a) Diffusion (Vision, Graph), Autoregressive (Language, Vision)
% (b) Circuit Generation for Predefined Setting
% (c) Neural Architecture Search for Strict Logic Equivalence

% Contribution
% (a) Circuit Tokenizer (new transformer arch, training strategy)
% (b) CircuitAR (train and gen strategies, post-ar strategy)
% (c) Extensive Evaluation including BitD (Bit Distance) for Scalability



\subsection{Plasticity in Neural Networks}
In recent years, various methods have been proposed to address plasticity loss.
Several works have focused on maintaining active units \cite{abbas2023loss, elsayed2024addressing} or re-initializing dead units \cite{sokar2023dormant, dohare2024loss}.
Other studies have explored limiting deviations from the initial statistics of model parameters \cite{kumar2023maintaining, lewandowski2023curvature, elsayed2024weight}.
Additionally, some methods rely on architectural modifications \cite{nikishin2024deep, lee2024slow, lewandowski2024plastic}.  
Plasticity loss also occurs in the reinforcement learning due to its inherent non-stationary. \citet{nikishin2022primacy} proposed resetting the model, while \citet{asadi2024resetting} suggested resetting the optimizer state. 

As noted by \citet{berariu2021study}, loss of plasticity can be divided into two distinct aspects: a decreased ability of networks to minimize training loss on new data (trainability) and a decreased ability to generalize to unseen data (generalizability).
While most previous works focused on trainability, \citet{lee2024slow} addressed generalizability loss.
They demonstrated that plasticity loss also occurs under a stationary distribution, as in a warm-start learning scenario where the model is pretrained on a subset of the training data and then fine-tuned on the full dataset.

Most existing studies have focused on only one of the following challenges: trainability, generalizability, or reinforcement learning.
However, in this study, we validate our AID method across all three aspects, demonstrating its effectiveness in each scenario.



\subsection{Activation Function}
Our AID method is a stochastic approach similar to Dropout while also functioning as an activation function.
Therefore, we aim to discuss previously proposed probabilistic activation functions.
Although the field of probabilistic activation functions has not seen extensive research, two noteworthy studies exist.
The first is the Randomized ReLU (RReLU) function, introduced in the Kaggle NDSB Competition \cite{xu2015empirical}.
The original ReLU function maps all negative values to zero, whereas RReLU maps negative values linearly based on a random slope.
During testing, negative values are mapped using the mean of the slope distribution.
Their experimental results suggest that RReLU effectively prevents overfitting.
Another example of a probabilistic activation function is DropReLU \cite{liang2021drop}.
DropReLU randomly determines whether a node's activation is processed through a ReLU function or a linear function.
The authors claim that DropReLU improves the generalization performance of neural networks.
The fundamental distinction between these probabilistic activation functions and our method lies in the generality of our approach.
Unlike simple probabilistic activation functions, our method encompasses techniques such as Dropout and ReLU, providing a more comprehensive framework.

Another related approach involves activation functions designed to address plasticity loss.
\citep{abbas2023loss} proposed the Concatenated Rectified Linear Units (CReLU), which concatenates the outputs of the standard ReLU applied to the input and its negation.
This structure prevents the occurrence of dead units, thereby improving plasticity.
Additionally, trainable activation functions have also been shown to effectively mitigate plasticity loss in reinforcement learning \citep{delfosseadaptive}.
Specifically, they introduced a trainable rational activation function that prevents value overfitting and overestimation in reinforcement learning.



\begin{figure*}[ht!]
    \centering
    \includegraphics[width=0.3\textwidth]{figures/sources/mainnet_pls_acc.pdf}
    \includegraphics[width=0.3\textwidth]{figures/sources/subnet_pls_acc.pdf}
    \includegraphics[width=0.3\textwidth]{figures/sources/warm_start_dropout.pdf}
    \caption{\textbf{Left.} Random label MNIST experiment using an 8-layer MLP. Higher dropout probabilities result in significant trainability loss. 
    \textbf{Middle.} Accuracy of the subnetworks trained on random target. Each subnetworks are sampled from original network after each epoch. Subnetworks of the Dropout also experience trainability loss. \textbf{Right.} Warm-start scenario of Resnet-18 model with CIFAR100 dataset. Dropout improves generalization performance; however, the reduction in accuracy compared to the cold-start scenario is nearly identical to that of the vanilla model.}
    \label{exp_dropout}
\end{figure*}



% \begin{figure}
%     \centering
%     \includegraphics[width=0.5\linewidth]{Move_teaser.pdf}
%     \caption{Comparison of different dynamic compute approaches. length of arrow indicates residual transformation per token while width indicates velocity of transformation.}
%     \label{fig:enter-label}
% \end{figure}

\section{Method}
\label{sec:method}
Residual connections play a crucial role in shaping token representations, yet their dynamics remain underexplored in the context of efficient decoding. In this work, we delve deeper into transformer residual dynamics and investigate how modulating residual transformation velocity can improve inference efficiency in token-level processing, optimizing both dense and sparse MoE transformers.


\subsection{Residual Dynamics and Motivation for Multi-rate Residuals} \label{sec:motivation}

To analyze how hidden representations evolve across different layers of a transformer architecture, it's crucial to consider the effect of residual connections. Each transformer decoder layer typically has residual connections across attention and MLP submodules. As the residual stream $h_i$ traverses from interval $E_j$ to $E_{j+1}$, it undergoes a residual transformation given by:  
% \begin{equation}
% \label{eq:slow_residual_transformation}
% H_{E_{j+1}} = H_{E_j} \prod_{i=E_j}^{E_{j+1}} \left( I + \mathcal{A}_i \right) \left( I + \mathcal{M}_i \right) \quad \text{where} \quad \mathcal{A}_i = f(c_i, h_{i}), \mathcal{M}_i = g(h_i)
% \end{equation}

\begin{equation} \label{eq:slow_residual_transformation}
h_{E_{j+1}} = h_{E_j} + \sum_{i=E_j}^{E_{j+1}-1} \left( \mathcal{A}_i(h_i) + \mathcal{M}_i(h_i + \mathcal{A}_i(h_i)) \right) \quad \text{where} \quad \mathcal{A}_i = f(c_i, h_{i}), \mathcal{M}_i = g(h_i). 
\end{equation}

Here, \( \mathcal{A}_i \) denotes the non-linear transformation introduced by the multi-head attention mechanism at layer \( i \), while \( \mathcal{M}_i \) corresponds to the non-linear transformation of the MLP block at the same layer. These transformations depend on the input residual stream \( h_i \) and, in the case of \( \mathcal{A}_i \), the previous contextual representation \( c_i \).\footnote{Normalization layers are typically applied in practice but are omitted here for simplicity of the argument.}


% For easy tokens, the magnitude and direction of this delta transformation become progressively smaller with each successive layer as shown in \cref{fig:delta_transformation}. Consequently, it is feasible to predict these tokens after only a few residual connections, whereas harder tokens necessitate more extensive processing through additional layers.

\begin{figure}[ht]
    \centering
    \begin{subfigure}{0.48\textwidth}
        \centering
        \includegraphics[width=\textwidth]{sections/figures/residual_change.pdf}
        \caption{}
        \label{fig:residual_change}
    \end{subfigure}%
    \hfill
    \begin{subfigure}{0.48\textwidth}
        \centering
        \includegraphics[width=\textwidth]{sections/figures/alignment_wrt_dedicated_model.pdf}
        \caption{}
    \label{fig:alignment_wrt_dedicated_model}
    \end{subfigure}
    \caption{(a) As residual streams propagate through the model, the directional shifts in the residuals become progressively smaller. (b) A dedicated model with $k$ layers achieves a faster rate of change in residual streams and higher alignment than base model leveraging early exit mechanisms at layer $k$.}
    \label{fig}
\end{figure}


To examine whether residual transformations can be accelerated across layers, we conducted experiments using a diverse set of prompts on a pre-trained Phi3 model~\cite{phi3_report}. As illustrated in \cref{fig:residual_change}, we measured the directional shift in residual states as \( 1 - \mathcal{C}(h_{i-1}, h_i) \), where \(\mathcal{C}\) denotes normalized cosine similarity. This shift is notably higher in the initial layers, gradually decreasing in subsequent layers. This behavior allows traditional early exit approaches to effectively accelerate decoding by enabling earlier exits for simpler tokens. However, these approaches typically rely on a distance-based approximation, where the full residual transformation of the model is approximated by the residual transformations of the initial layers. To gain deeper insights into the distance versus velocity aspects of residual transformation, we conducted a comparative study. Specifically, we trained an early exit head at layer $k$ of the Phi3 model, which consists of 32 layers, restricting the distance traveled by each token. To accelerate the residual transformation relative to number of layers, we trained a smaller model consisting of only $k$ layers, while keeping all other hyperparameters consistent. We then compared the next-token prediction accuracy of the early exit head of the base model with that of the smaller model. To ensure an equal number of trainable parameters, we inserted low-rank adapters into the smaller model and trained only these adapters, whereas, in the distance-based approach, we trained solely the early exit head. In addition, to accelerate the residual transformation in smaller model, we distilled the residual streams from the larger model by incorporating a distillation loss ~\cite{sanh2019distilbert} between the residual state at layer \(i\) of the smaller model and the residual state at layer \(4 \times i\) of the larger model. As shown in ~\cref{fig:alignment_wrt_dedicated_model} the smaller model demonstrates a significantly faster rate of change in residual streams, leading to higher next token prediction accuracy after $k$ layers compared to the base model that employs traditional early exit mechanisms after $k$ layers \cite{schuster2022confident, chen2023eellm, varshney-etal-2024-investigating}. This experimental setup, which modifies only the rate of change in residual streams while keeping other factors constant, suggests that dense transformers, trained with a fixed number of layers, may inherently possess a slow residual transformation bias.

This observation raises an intriguing question: if the rate of change in residual streams could be accelerated relative to the number of layers, is it possible to facilitate earlier alignment for a greater proportion of tokens? Earlier alignment would be beneficial to not only facilitate dynamic computation but also for generating speculative tokens efficiently with high acceptance rates in speculative decoding setups ~\cite{leviathan2023fast, chen2023accelerating}. 

%thereby enhancing the efficiency of early exiting? 
 % This bias likely constrains the effectiveness of early exiting, particularly for easier tokens. By addressing this limitation through accelerated residual transformations, we hypothesize that it is possible to substantially improve the efficiency and accuracy of early exit strategies in transformer models.

\subsection{Multi-Rate Residual Transformation} \label{m2r2_method}

To address the slow residual transformation bias described in ~\cref{sec:motivation}, we introduce \textit{accelerated residual streams} that operate at rate $R$ relative to original slow residual stream. We pair slow residual stream, $h$ with an accelerated residual stream, $p$, which has an intrinsic bias towards earlier alignment. Relative to ~\cref{eq:slow_residual_transformation}, accelerated residual transformation from interval $E_j$ to $E_{j+1}$ can be represented as: 

% \begin{equation}
% \label{eq:fast_residual_transformation}
% P_{E_{j+1}} = P_{E_j} \prod_{i=E_j}^{E_{j+1}} \left( I + \hat{\mathcal{A}_i} \right) \left( I + \hat{\mathcal{M}_i} \right) \quad \text{where} \quad \hat{\mathcal{A}_i} = \hat{f}(c_i, P_{i}), \hat{\mathcal{M}_i} = \hat{g}(P_{i})
% \end{equation}


\begin{equation} \label{eq:fast_residual_transformation}
p_{E_{j+1}} = p_{E_j} + \sum_{i=E_j}^{E_{j+1}-1} \left( \hat{\mathcal{A}_i}(p_i) + \hat{\mathcal{M}_i}(p_i + \hat{\mathcal{A}_i}(p_i)) \right) \quad \text{where} \quad \hat{\mathcal{A}_i} = \hat{f}(c_i, p_{i}), \hat{\mathcal{M}_i} = \hat{g}(h_i), 
\end{equation}



where $\hat{\mathcal{A}_i}$ and $\hat{\mathcal{M}_i}$ denote non-linear transformation added by layer $i$ to previous accelerated residual $p_{i}$. Similar to $\mathcal{A}_i$, non-linear transformation $\hat{\mathcal{A}_i}$ attends to same context $c_i$ but uses a different transformation $\hat{f}$ for accelerating $p_{E_j}$ relative to $h_{E_j}$. 

We integrate accelerated residual transformation directly into the base network using parallel accelerator adapters such that rank of accelerator adapters $R_p << d$ where $d$ denotes base model hidden dimension. This setup allows the slow residual stream $h_{E_j}$ to pass through the base model layers while the accelerated residual stream $p_{E_j}$ utilizes these parallel adapters as shown in ~\cref{fig:m2r2_main}. Both slow and accelerated residuals are processed in same forward pass via attention masking and incur negligible additional inference latency in memory bound decoding setups, while in compute bound decoding setups where FLOPs optimization is essential, accelerated residual stream utilizes a fraction of attention heads that of slow residual (see ~\cref{sec:flops_optimization}). Additionally, to maximize the utility of accelerated residual transformations without introducing dedicated KV caches, we propose a shared caching mechanism between the slow and accelerated streams which minimally impact alignment benefits of our approach while offering substantial memory savings (see ~\cref{fig:koala_alignment}). Specifically, the attention operation on the slow residuals \( \text{MHA}(h_t, h_{\leq t}, h_{\leq t}) \) is redefined for accelerated residuals as 
\[
\hat{\mathcal{A}} = MHA(p_t, h_{<t} \oplus p_t, h_{<t} \oplus p_t),
\]
where the accelerated residual at time-step $t$, \( p_t \) attends to the slow residual’s KV cache, facilitating the reuse of contextual information across both residual streams without incurring additional caching costs. Here, \(MHA(q, k, v) \) represents multi-head attention between query \( q \), key \( k \), and value \( v \).

\begin{figure}
    \centering
    \includegraphics[width=0.8\linewidth]{sections//figures/m2r2_main2.pdf}
    \caption{Multi-rate Residuals Framework: Slow residual stream of base model is accompanied by a faster stream that operates at a $2-(J+1)\times$ rate relative to the slow stream, undergoing transformations via accelerator adapters as detailed in \cref{m2r2_method}, where J denotes number of early exit intervals. Colors within the slow and fast residual streams indicate similarity, with matching colors representing the most closely aligned residual states. At the beginning of the forward pass and at each exit point, the accelerated residual state is initialized from the corresponding slow residual state to avoid gradient conflict during training (see ~\cref{sec:grad_conflict}). Early exiting decisions are informed by the Accelerated Residual Latent Attention (ARLA) mechanism, described in \cref{method_arla}, which evaluates residual dynamics across consecutive exit gates.}
    \label{fig:m2r2_main}
\end{figure}

% Furthermore. to maximize the benefits of fast residual transformations without using dedicated KV caches, we propose sharing the fast network’s cache with the slow network. Formally speaking, We modify attention operation on slow residuals $MHA(H_t, H_{<=t}, H_{<=t})$ as $MHA(P_{t}, H_{<t} \oplus P_t, H_{<t}  \oplus P_t)$ such that accelerated residuals attend to previous slow context KV cache, where $MHA(q,k,v)$ denotes multi head attention between query, $q$, key $k$ and value $v$.


\subsection{Enhanced Early Residual Alignment}
Early residual alignment is instrumental in optimizing early exiting, speculative decoding, and Mixture-of-Experts (MoE) inference mechanisms. In this section, we provide a detailed analysis of how accelerated residuals enhance these inference setups.

% By aligning the residual states of intermediate layers with the final output representations, the model can maintain high prediction accuracy even when computations are truncated at earlier layers. This enables more reliable early exiting, reducing the overall computational cost while preserving performance. Additionally, in speculative decoding, early residual alignment allows the model to make confident predictions using faster, partial computations, thereby accelerating inference without sacrificing output quality.


\subsubsection{Early Exiting} \label{method_early_exiting}

A prevalent strategy for enabling early exiting at an intermediate layer $E_{j}$ involves approximating the residual transformation between $E_{j}$ and the final layer $N-1$ using a linear, context independent mapping, $\mathcal{T}$, such that $H_{N-1} \approx \mathcal{T}(H_{E_{j}})$. This approximation has been extensively employed in conventional approaches ~\cite{schuster2022confident, chen2023eellm, varshney-etal-2024-investigating}, providing a computationally efficient means to project the output of deeper layers from intermediate states. Specifically, residual state of layer $N-1$ with this approximation can be expressed as:


% \begin{equation}
% \label{eq: vanila_ea_assumption}
% \Phi(H_{E_{j}}) \sim H_{E_{j}} \prod_{i=E_{j}}^{N}\left( I + \mathcal{A}_i \right) \left( I + \mathcal{M}_i \right) \quad \text{where} \quad \Phi \perp C
% \end{equation}

\begin{equation} \label{eq:early_exiting}
h_{E_j} + \sum_{i=E_j}^{N-1} \left( \mathcal{A}_i(h_i) + \mathcal{M}_i(h_i + \mathcal{A}_i(h_i)) \right) \sim \mathcal{T}(h_{E_{j}})  \quad \text{where} \quad \mathcal{T} \perp c. 
\end{equation}


Here, $\mathcal{A}_i$ and $\mathcal{M}_i$ represent the residual contributions of the multi-head attention and MLP layers, respectively, while $\mathcal{T}$ remains independent of $c$, the preceding context.

This approach is inherently limited by two major factors: first, the assumption of linearity between $h_{E_{j}}$ and $h_{N-1}$ may not hold uniformly for all tokens, particularly when $E_j \ll N$. Second, the linear transformation $\mathcal{T}$ disregards the influence of the context $c$ and fails to account for the latent representations of previous contextual states. In contrast, M2R2 accelerated residual states mitigate both of these challenges by approximating the slow residual transformation of all layers via a faster residual transformation of fewer layers as:
% \begin{equation}
% H_{E_j} \prod_{i=E_j}^{N}\left( I + \mathcal{A}_i \right) \left( I + \mathcal{M}_i \right) \sim P_{E_j} \prod_{i=E_j}^{E_j+1}\left( I + \hat{\mathcal{A}_i} \right) \left( I + \hat{\mathcal{M}_i} \right)
% \end{equation}


\begin{equation} \label{eq:m2r2_approximating_ea}
h_{E_j} + \sum_{i=E_j}^{N-1} \left( \mathcal{A}_i(h_i) + \mathcal{M}_i(h_i + \mathcal{A}_i(h_i)) \right) \sim p_{E_j} + \sum_{i=E_j}^{E_{j+1}-1} \left( \hat{\mathcal{A}_i}(p_i) + \hat{\mathcal{M}_i}(p_i + \hat{\mathcal{A}_i}(p_i)) \right), 
\end{equation}

% \begin{equation} \label{eq:fast_residual_transformation}
% p_{E_{j+1}} = p_{E_j} + \sum_{i=E_j}^{E_{j+1}-1} \left( \hat{\mathcal{A}_i}(p_i) + \hat{\mathcal{M}_i}(p_i + \hat{\mathcal{A}_i}(p_i)) \right) \quad \text{where} \quad \hat{\mathcal{A}_i} = \hat{f}(c_i, p_{i}), \hat{\mathcal{M}_i} = \hat{g}(h_i) 
% \end{equation}






where $p_{E_j}$ is initialized from the slow residual state $h_{E_j}$ at each early exit interval $E_j$ using an identity transformation (see ~\cref{fig:m2r2_main}). As shown in ~\cref{fig:m2r2_residual_sim}, accelerated residuals offer a smoother, more consistent shift in residual direction across layers, in contrast to the abrupt changes typically seen at early exit points in standard early exit methods. Moreover, the normalized cosine similarity between accelerated states at early exit intervals and final residual states is substantially higher compared to traditional early exit techniques, highlighting improved alignment with final layer representations. Traditional adaptive compute methods are constrained by two principal factors: the number of tokens eligible for early exit at intermediate layers and the precision of early exit decision. If residual streams fail to saturate early, the majority of tokens remain ineligible for exit, thereby diminishing potential speedups. Additionally, imprecise delineations between tokens suitable for early exit can lead to underthinking (premature exits that adversely affect accuracy) or overthinking (unnecessary processing that compromises efficiency) ~\cite{zhou2020self, dai2020dynamic}. Enhanced early alignment using ~\cref{eq:m2r2_approximating_ea} helps to address  first issue. To address the second issue we introduce Accelerated Residual Latent Attention, which dynamically assesses the saturation of the residual stream, allowing for a more precise differentiation between tokens that can exit early and those requiring further processing.

% This results in uniform change in residual direction    
% % We keep $\mathcal{A} = \hat{\mathcal{A}}$, while $\hat{\mathcal{M}}$ is accelerated by a factor of $2 - (N_{E}+1)X$ relative to the slower residual transformation $\mathcal{M}$, where $N_E$ represents number of early exiting intervals.
% Figure~\cref{fig:rate_change_comparison} illustrates the comparative rate of change between these transformation streams.



% fig:rate_change_comparison
% - grid plot x axis -> layer id (0, 8) , y axis -> layer id -> dark color cell for max similarity , lighter for lower 
% 
-------------------------------------------------------
Let's consider residual stream $h_i$ traverses through interval $E_j$ to $E_{j+1}$ and undergoes residual transformation given by 
\begin{equation}
h_{E_{j+1}} = h_{E_j} \prod_{i=E_j}^{E_{j+1}} \left( 1 + \delta_i \right)    
\end{equation}

where $\delta_i$ denotes non-linear transformation added by layer $i$. Each non-linear transformation of layer $i$ is a function of previous contextual representation, $c_i$ and input residual stream $h_i-1$ as
$\delta_i = f(c_i, h_{i-1})$ 

One way to exit early at exit $E_j+1$ is to assume that residual transformation from $E_j+1$ to final layer $N-1$ can be approximated by a linear function $\phi$ as $h_{N-1} \sim \Phi(h_{E_j+1})$ and most conventional approaches such as \todo{cite EA papers} use this approach. In other words, 

\begin{equation}
\Phi(h_{E_j+1} \sim h_{E_j+1} \prod_{i=E_j+1}^{N} \left( 1 + \delta_i \right)   
\end{equation}

This approach suffers from two primary issues, linearity assumption from $h_E_j+1$ to $H_N-1$ if often incorrect, particularly when $E_j << N$. More importantly, linear transformation $\Phi$ doesn't consider effect of context $C_i$. M2R2  effectively addresses these issues as accelerated residual stream at interval $E_j+1$ can be represented as 

\begin{equation}
r_{E_{j+1}} = r_{E_j} \prod_{i=E_j}^{E_{j+1}} \left( 1 + \gamma_i \right)    
\end{equation}

where $\gamma_i$ denotes non-linear transformation added by layer $i$ to previous accelerated residual $r_i-1$. Similar to $\delta_i$, non-linear transformation $\gamma_i$ considers context $C_i$ as 
$\gamma_i = g(c_i, r_{i-1})$. So in summary, slow residual transformation is approximated by accelerated residual as: 

\begin{equation}
h_{E_j} \prod_{i=E_j}^{N} \left( 1 + \delta_i \right) \sim h_{E_j} \prod_{i=E_j}^{E_j+1} \left( 1 + \gamma_i \right)
\end{equation}

It's worth noting that accelerated residual $r_i$ and slow residual $h_i$ are processed concurrently at layer $i$ by constructing proper attention mask such as attention of slow residual is represented as 

$MHA(H_it, H_{i<=t}, H_{i<=t}$ while attention of fast residual is computed as 

$MHA(r_it, H_{i<=t}, H_{i<=t}$ where $MHA(q,k,v$ denotes multi head attention between query, $q$, key $k$ and value $v$.


------------------------------------------------------------------

Vertical latent attention on accelerated residual is computed as 
$MHA(S_mt, S(Ej<=i<=m)t, S(Ej<=i<=m)t)$ where $Smt$ denotes query/key/value projection in latent domain at layer $m$ at time $t$. 
------------------------------------------------------------------

Gradient conflict Avoidance: 

Let's consider $w_j$ is a trainable parameter that belongs to a layer between $E_j$ and $E_j+1$. Consider early exit loss at gate $E_j+1$, $L_j+1$, gradient propagation of $w_j$ at another trainable parameter $w_j-n$ can be gives as 

$\sum_{k=E_j-n}^{E_j} \beta_k \frac{\partial L_{E_k}}{\partial w_k}$

where $\beta_j$ denotes backward transformation coefficient for weight $w_j$ to reach gate $E_j$. 
 
On the other hand, gradient propagation in proposed approach can be represented as 

\[
\frac{\partial L_{E_j}}{\partial w_j} = 
\begin{cases} 
\beta_j \frac{\partial L_{E_j}}{\partial w_j} & \text{if } E_j \leq w_j \leq E_{j+1} \\
0 & \text{otherwise}
\end{cases}
\]







% \begin{figure}[ht]
%     \centering
%     \includegraphics[width=0.8\textwidth, height=5cm]{rate_change_comparison.png}
%     \caption{Rate of change comparison between fast and slow residual streams.}
%     \label{fig:rate_change_comparison}
% \end{figure}

%vary k and and plot EA accuracy for larger and smaller models. 

% \begin{figure}[ht]
%     \centering
%     \includegraphics[width=0.5\textwidth,height=5cm]{sections/figures/alignment_comparison_dialogsum.pdf}
%     \caption{Alignment of exited tokens for different early exit layers using traditional early exiting heads, dedicated faster networks, and faster residuals.}
%     \label{fig:small_model_early_exiting}
% \end{figure}


\textbf{Accelerated Residual Latent Attention} \label{method_arla}

In the context of residual streams, we observe that the decision to exit at a given layer can be more effectively informed by analyzing the dynamics of residual stream transformations, instead of solely relying on a classification head applied at the early exit interval $E_j$. To capture the subtle dynamics of residual acceleration, we propose a \textit{Accelerated Residual Latent Attention} (ARLA) mechanism. This approach involves making the exit decision at gate $E_j$ by attending to the residuals spanning from gate $E_{j-1}$ to $E_j$, rather than considering only the residual at gate $E_j$. To minimize the computational overhead associated with exit decision-making, the attention mechanism operates within the latent domain as depicted in ~\cref{fig:arla_arch}. Formally, for each interval $[E_j, E_{j+1}]$, the accelerated residuals are projected into Query ($Q^s_{E_j}, \ldots, Q^s_{E_{j+1}}$), Key ($K^s_{E_j}, \ldots, K^s_{E_{j+1}}$), and Value ($V^s_{E_j}, \ldots, V^s_{E_{j+1}}$) vectors, with latent dimension $d^s$ for $Q^s$, $K^s$, and $V^s$ being significantly smaller than hidden dimension of $p$.\footnote{We use $d^s = 64$ for experiments described in ~\cref{sec:experiments}.} Notably, when the router is allowed to make exit decisions at gate $E_j$ based on residual change dynamics, we observe that the attention is not confined to the residual state at $E_j$ but is distributed across residual states from $E_{j-1}$ to $E_j$, %as illustrated in Figure~\ref{fig:vertical_latent_attention_dynamics}. 
This broader focus on residual dynamics significantly reduces decision ambiguity in early exits, as demonstrated in Figure~\ref{fig:roc_arla}, which contrasts routers based on the last hidden state, and the proposed ARLA router.

%show R -> S transformation. 
%show parameter and flop overhead as compared to adapter on last hidden state.

% \begin{figure}[ht]
%     \centering
%     \includegraphics[width=0.5\textwidth,height=5cm]{sections/figures/roc_arla.pdf}
%     \caption{ROC curves of early exit decision strategies: confidence-based methods (CALM/LITE), routers based on the accelerated hidden state, and latent attention routers.}
%     \label{fig:decision_making_comparison}
% \end{figure}

% \begin{figure}[ht]
%     \centering
%     \includegraphics[width=0.5\textwidth,height=5cm]{vertical_latent_attention.png}
%     \caption{Vertical latent attention mechanism for optimizing early exit decisions by considering residuals from gate \(M\) through \(M-1\).}
%     \label{fig:vertical_latent_attention}
% \end{figure}

\begin{figure}[ht]
    \centering
    \begin{subfigure}{0.52\textwidth}
        \centering
        \includegraphics[width=\textwidth, height = 4cm]{sections/figures/arla_arch.pdf}
        \caption{Accelerated Residual Latent Attention (ARLA): Accelerated residuals between early exit gates are projected into latent domain and attention over residual states within the interval is computed to capture residual dynamics and exit decision is made based on residual saturation.}
        \label{fig:arla_arch}
    \end{subfigure}%
    \hfill
    \begin{subfigure}{0.45\textwidth}
        \centering
        \includegraphics[width=\textwidth, height = 4.5cm]{sections/figures/vla_roc.pdf}
        \caption{ROC classification curves of early exit decision strategies using a linear router used on last residual state ~\cite{schuster2022confident, varshney-etal-2024-investigating, chen2023eellm}  and using ARLA approach that considers residual dynamics. }
        \label{fig:roc_arla}
    \end{subfigure}
    \caption{Effectiveness of ARLA in capturing residual dynamics for early exiting decisions.}


\end{figure}



% \begin{figure}[ht]
%     \centering
%     \includegraphics[width=1\textwidth,height=5cm]{sections/figures/arla.pdf}
%     \caption{fig that plots 32 rows 2 cols heatmap showing attention at each gate}
%     \label{fig:vertical_latent_attention_dynamics}
% \end{figure}

\subsubsection{Self Speculative Decoding} \label{method_self_speculative_decoding}

An alternative means to exploit the early alignment properties of our approach is through the use of accelerated residual states for speculative token sampling to accelerate autoregressive decoding. Speculative decoding aims to speed up memory-bound transformer inference by employing a lightweight draft model to predict candidate tokens, while verifying speculated tokens in parallel and advancing token generation by more than one token per full model invocation \cite{leviathan2023fast, chen2023accelerating, xia2023speculative, miao2023specinfer}. Despite its effectiveness in accelerating large language models (LLMs), speculative decoding introduces substantial complexity in both deployment and training. A separate draft model must be specifically trained and aligned with the target model for each application, which increases the training load and operational complexity ~\cite{chen2023accelerating}. Additionally, this approach is resource-inefficient, as it requires both the draft and target models to be simultaneously maintained in memory during inference \cite{leviathan2023fast, chen2023accelerating}. 

One strategy to address this inefficiency is to leverage the initial layers of the target model itself to generate speculative candidates, as depicted in ~\cite{Tang2024}. While this method reduces the autoregressive overhead associated with speculation, it suffers from suboptimal acceptance rates. This occurs because the linear transformation employed for translating hidden states from layer $k$ to the final layer $N$ is typically a poor approximation, as discussed in ~\cref{sec:motivation} and ~\cref{method_early_exiting}. Our approach resolves this limitation by utilizing accelerated residuals, which demonstrate higher fidelity to their slower counterparts. By utilizing accelerated residuals operating at a rate of $N/k$, where $k$ denotes the number of layers used for candidate speculation, we are able to efficiently generate speculative tokens for decoding.\footnote{We typically set $k = 4$ to balance the trade-off between autoregressive drafting overhead and acceptance rate, as discussed in~\cref{sec:experiments}.}
 This technique not only obviates the need for multiple models during inference but also improves the overall efficiency and effectiveness of speculative decoding.

\begin{figure}
    \centering    \includegraphics[width=1\linewidth]{sections/figures/m2r2_aot_loading.pdf}
    \caption{Ahead-of-Time Expert Loading: M2R2 accelerated residual stream predicts experts required for future layers, reducing reliance on on-demand lazy loading. Speculative pre-loading is efficiently overlapped with computation of multi-head attention (MHA) and MLP transformations. Only incorrectly speculated experts are loaded lazily, resulting in faster inference steps and improved computational efficiency. Here, H indicates LBM Host while D indicates HBM Device.}
    \label{fig:moe_expert_aot_loading}
\end{figure}


\subsubsection{Ahead of Time Expert Loading:} \label{method_aot_expert_loading}

Recent advancements in sparse Mixture-of-Experts (MoE) architectures ~\cite{shazeer2017outrageously, fedus2022switch, artetxe2019massively, lepikhin2020gshard, zoph2022designing} have introduced a paradigm shift in token generation by dynamically activating only a subset of experts per input, achieving superior efficiency in comparison to dense models, particularly under memory-bound constraints of autoregressive decoding \cite{fedus2022switch, zoph2022designing}. This sparse activation approach enables MoE-based language models to generate tokens more swiftly, leveraging the efficiency of selective expert usage and avoiding the overhead of full dense layer invocation. In dense transformer models, pre-loading layers is a common strategy to enhance throughput, as computations of current layer can be overlapped with pre-loading of next layer parameters ~\cite{narayanan2021efficient, shoeybi2020megatron}. However, MoE models face a unique challenge: expert selection occurs dynamically based on previous layer’s output, making it infeasible to preload next layer’s experts in parallel. This limitation results in inherent latency, as expert loading becomes a sequential, on-demand process ~\cite{lepikhin2020gshard, fedus2022switch}.

To address this inefficiency, our method introduces a mechanism with \textit{accelerated residuals}, which not only captures key characteristics of base slower residual states but also exhibit high cosine similarity with their final counterparts (as illustrated in \cref{fig:m2r2_residual_sim}). By employing accelerated residual streams, we can effectively predict the necessary experts for future layers well in advance of their actual invocation. Specifically, using a $2\times$ accelerated residual, the experts needed for layers $2i+2$ and $2i+3$ can be identified while still computing in layer $i$, thus overcoming the bottleneck of sequential, on-demand expert selection and mitigating latency in the decoding pipeline, as shown in \cref{fig:moe_expert_aot_loading}. Note that, we use fixed set of accelerator adapters for transforming accelerated residuals (as discussed in ~\cref{m2r2_method}) while slow residual is transformed via expert routing mechanism. 

Furthermore, our approach integrates a Least Recently Used (LRU) caching strategy, which enhances memory efficiency by replacing the least recently used experts with speculated experts that are anticipated to be needed in upcoming layers. This hybrid approach of preemptive expert loading with LRU caching yields substantial improvements over traditional on-demand loading or standalone caching strategies. By minimizing cache misses and efficiently managing memory, this approach addresses both compute and memory bottlenecks, leading to faster, more resource-efficient token generation in MoE architectures. A comprehensive evaluation of this strategy, in relation to state-of-the-art methods, is provided in \cref{experiments_aot}, and the compute and memory traces on an A100 GPU are detailed in \cref{fig:moe_aot_cuda_trace}.



% Recent advancements in sparse Mixture-of-Experts (MoE) architectures have introduced the concept of utilizing distinct computational paths for different tokens \cite{shazeer2017outrageously}. This approach, wherein only a subset of experts are activated per input, enables MoE-based language models to generate tokens more swiftly compared to their dense counterparts due to memory-bound nature of auto-regressive decoding. In dense models, pre-loading layers in advance is a common strategy to enhance computational efficiency. However, this technique is not applicable to MoE models, where expert selection occurs dynamically based on the outputs of previous layers, preventing parallel pre-fetching of experts.

% Our proposed method addresses this inefficiency. Accelerated residuals, which are highly similar to their slower counterparts (see \cref{fig:similarity}), can reliably predict the necessary experts ahead of time. For instance, by utilizing $2X$ accelerated residual stream, we can predict the experts needed for the layer $2i+1$ and $2i+3$ while carrying out computation in layer $i$. This enables us to commence expert loading significantly earlier, as illustrated in \cref{expert_loading}, effectively mitigating the delays observed with the naive on-demand expert loading. Additionally, our method benefits from incorporating a Least Recently Used (LRU) strategy, where speculated experts replace those that are least recently utilized, resulting in improved performance compared to using either strategy alone. For a comprehensive evaluation, refer to \cref{moe_trace}, which provides a CUDA compute and memory trace of our approach executed on <>.



% A naive solution involves using the residual state of the previous layer along with the gating function of the next layer to predict which experts need to be loaded, and initiating the expert loading process in parallel with the attention computation of the next layer. Yet, as shown in \cref{fig:MOE_attn_vs_loading_time}, the attention computation for medium to long contexts is considerably faster than the expert loading time, making this approach inefficient.




\subsection{Training} \label{method_training}
% This approach is feasible due to the absence of gradient conflicts, as discussed in \cref{sec:grad_conflict}.

To accelerate residual streams, we employ parallel accelerator adapters as described in \cref{m2r2_method}.  For the early exiting use-case outlined in \cref{method_early_exiting}, we define the training objective for these adapters using the following loss function, which combines cross-entropy loss at each exit $E_j$ with distillation loss at each layer $i$. Loss weights coefficients $\alpha_0$ and $\alpha_1$ are employed to balance contribution of corresponding losses.

\begin{align} \label{eq:mr_loss}
L_{\text{m2r2}} = \underbrace{-\alpha_0 \sum_{j=1}^{J} \sum_{t=1}^{T} \log p_{\theta} \left( \hat{y}_t^{E_j} \mid y_{<t}, x \right)}_{\text{cross-entropy loss}} 
+ \underbrace{\alpha_1\sum_{i=1}^{E_{J-1}} \sum_{t=1}^{T} \| \mathbf{p}_{t}^{i} - \mathbf{h}_{t}^{((i - E_{j(i)}) \cdot R_i) + E_{j(i)})} \|^2}_{\text{distillation loss}}.
\end{align}

where $\hat{y}_t^{E_j}$ denotes the predictions from the accelerated residual stream at layer $E_j$ and time step $t$, $y_t$ represents the corresponding ground truth tokens, and $x$ indicates previous context tokens. The distillation loss at each layer $i$ is computed by comparing accelerated residuals at layer $i$ with slow residuals at layer $(i - E_{j(i)}) \cdot R_i + E_{j(i)}$, where $R_i$ denotes the rate of accelerated residuals at layer $i$ while $E_{j(i)}$ represents the most recent gate layer index such that $E_{j(i)} <= i$. \( J \) represents the total number of early exit gates, N denotes number of hidden layers and $E_j$ denotes layer index corresponding to gate index $j$ and \( T \) denotes the sequence length. 

In dynamic compute settings, after training of accelerator adapters, we optimize the query, key, and value parameters governing the ARLA routers (see ~\cref{method_arla}) across all exits in parallel on binary cross entropy loss between predicted decision and ground truth exiting decision. The ground truth labels for the router are determined based on whether the application of the final logit head on $\hat{y}_t^{E_j}$ yields the correct next-token prediction. 


% The objective for this optimization is defined by the following loss function:


%TODO are equations required ? 
% \begin{equation} \label{eq:arla_loss_combined}\small
%     L_{\text{arla}} = -\frac{1}{N} \sum_{t=1}^{T} \left( \sum_{j=1}^{E_n} \left[ O_t^{E_j} \log(\hat{O}_t^{E_j}) + (1 - O_t^{E_j}) \log(1 - \hat{O}_t^{E_j}) \right] \right), \quad \text{where} \quad 
%     O_t^{E_j} = \begin{cases} 
%     1, & \text{if } L(\hat{y}_t^{E_j}) = y_t^{E_j} \\
%     0, & \text{otherwise}
%     \end{cases}
% \end{equation}

% where $\hat{O}_t^{E_j}$ represents the binary predicted logits produced by the vertical latent attention router, as described in \cref{sec:arla}, at gate $E_j$ and time step $t$, and $O_t^{E_j}$ denotes the corresponding ground truth labels. The ground truth labels for the router are determined based on whether the application of the logit head on $\hat{y}_t^{E_j}$ yields the correct next-token prediction. The parameters controlling vertical latent attention are trained concurrently to ensure consistency and efficient use of computational resources.

For self-speculative decoding, as described in \cref{method_self_speculative_decoding}, the training objective remains the same as \cref{eq:mr_loss}, but with the number of intervals set to $J = 1$ and the rate of residual transformation set to $R_n = N/k$, where the first $k$ layers generate speculative candidate tokens. In the context of Ahead-of-Time Expert Loading for Mixture-of-Experts (MoE) models (see \cref{method_aot_expert_loading}), setting the rate of residual transformation to $R_n = 2$ typically offers a good trade-off between the accuracy of expert speculation and AoT pre-loading of experts. 

% Thus, we set $J = 1$ and $E_1 = 16$.


~\subsection{FLOPs Optimization} \label{sec:flops_optimization}

Naively implemented, M2R2 incurs higher FLOP overhead compared to traditional speculative decoding and early exiting approaches such as ~\cite{medusa, schuster2022confident, Tang2024}. However, modern accelerators demonstrate compute bandwidth that exceeds memory access bandwidth by an order of magnitude or more~\cite{databricksLLMInference2023, jouppi2021ten}, meaning increased FLOPs do not necessarily translate to increased decoding latency. Nevertheless, to ensure fair comparison and efficiency in compute bound scenarios, we introduce targeted optimizations.

~\textbf{Attention FLOPs Optimization} For medium-to-long context lengths, attention computation dominates FLOPs in the self-attention layer, surpassing the contribution from MLP layers. Specifically, matrix multiplications involving queries, cached keys, and cached values scale with $l_{kv} * l_{q}$ where $l_{kv}$ denotes previous context length and $l_q$ denotes current query length. Since M2R2 pairs accelerated residuals with slow residuals, a naive implementation results in twice the FLOPs consumption compared to a standard attention layer. To address this, we limit the attention of accelerated residual stream to selectively attend to the top-k most relevant tokens, identified by the slow residual stream based on top attention coefficients\footnote{We set to k = 64 and attend to top 64 tokens as identified by the slow residual stream.}. This is possible since slow and accelerated residual streams are processed in same forward pass and accelerated streams have access to attention coefficients of slow stream. Note that, the faster residual stream still retains the flexibility to assign distinct attention coefficients to these tokens. Furthermore, we design the faster residual stream to employ only 8 attention heads, compared to the 32 heads used in the slow residual stream of the Phi-3 model, reducing query, key, value, and output projection FLOPs by a factor of 1/4. ~\cref{fig:m2r2_num_heads_ablation} indicates effect of using a slicker stream on alignment. As depicted, using $\hat{n}_h = 8$ offers a good trade-off between alignment and FLOPs overhead. 

~\textbf{MLP FLOPs Optimization} The accelerator adapters operating on the accelerated residual stream are intentionally designed with lower rank than their counterparts in the base model. This reduces FLOP overhead by a factor proportional to $hiddenSize / rank$. Additionally, since the faster residual stream uses only 8 attention heads (compared to 32 in the slow residual stream of Phi-3), the subsequent MLP layers process a smaller set of activations, further reducing FLOPs by another factor of 1/4.

These optimizations significantly reduce the FLOP overhead per speculative draft generation, as illustrated in ~\cref{fig:flops_optmization}. Notably, while traditional early-exiting speculative approaches such as DEED require propagating the full slow residual state through the initial layers, incurring substantial computational costs, M2R2 achieves efficient token generation via slimmer, low-rank faster residual streams. In contrast, Medusa introduces considerable FLOP overhead due to per-head computations scaling with $d^2+dv$\footnote{Here $d$ denotes hidden state dimension while $v$ denotes vocab size.}, whereas M2R2 employs low-rank layers for both MLP and language modeling heads, maintaining computational efficiency. All experiments involving the M2R2 approach, as detailed in ~\cref{sec:experiments}, are conducted using these FLOPs optimizations.









% \[
% O_t^{E_j} = 
% \begin{cases} 
% 1, & \text{if } L(\hat{y}_t^{E_j}) = y_t^{E_j} \\
% 0, & \text{otherwise}
% \end{cases}
% \]




%add distillation
% We train accelerator adapters described in \cref{m2r2_method} to accelerate residual streams on next token prediction all in parallel since there are no gradient conflict issues as described in \cref{sec:grad_conflict}.

% \begin{align} \label{eq:mr_loss}
% L_{mr} =  & -\sum_{j = 1}^{E_n} (\sum_{t=1}^{T}\log p_{\theta} (\hat{y}_t^{E_j} | \hat{y}_{<t}, x)) \nonumber
% \end{align}

% where $\hat{y_t^{E_j}}$ denotes predicted logits obtained from accelerated residual stream at gate $E_j$ and time-step $t$ while $y_t^{E_j}$ denotes corresponding truth tokens. 

% Upon training of adapters responsible for accelerating residual streams, we train query, key, value parameters responsible for vertical latent attention of all gates in parallel as

% \begin{equation} \label{eq:arla_loss}
%     L_{arla} = -\frac{1}{N} (\sum_{t=1}^{T}(1\sum_{j=1}^{E_n} \left[ O_t^{E_j} \log(\hat{O}_t^{E_j}) + (1 - o_t^{E_j}) \log(1 - \hat{o_t}_{E_j}) \right]))
% \end{equation}

% where $\hat{O_t^{E_j}}$ denotes binary predicted logits obtained from vertical latent attention router described in \cref{sec:arla} at gate $E_j$ and timestep $t$ while $O_t^{E_j}$ denotes corresponding truth label. Truth labels for router are obtained by computing whether logit head application on $\hat{y}_t^j$ results in true next token prediction. Formally speaking, 

% $O_t^{E_j} = 1 if L(\hat{y_t^{E_j}}) == y_t^{E_j} , 0 otherwise$. 

% Parameters responsible for vertical latent attention are also trained in parallel as well. 

%todo: training slow and fast residuals together and distillation can be two training mdoes. 
%Distillation can be an ablation. 




% Although transformer decoding is memory bound on most mainstream accelerators, there could be scenarios where flop savings are crucial. For instance, on on-device settings power consumption is directly correlated with flops per decoding step and reducing flops does help with overall energy consumption. Vanilla early exiting methods help with flop reduction but suffer from mismatch between training and inference due to early exited tokens. If token at decoding step $t$, $T_t$ exited at layer $E_i$, while token $T_{t+k}$ exits at layer $E_j$ such that $E_i < E_j$, hidden state $H_{t+k}l$ does not have corresponding hidden state $H_tl$ to attend to where $E_i < l <= E_j$. One solution that's often used in literature is to rely on last hidden state available, $H_t{E_j}$, however it tends to be sub-optimal and does affect generation quality \cite{ref}.  To alleviate this mismatch while reducing flops, we train router such that attention mask between token $T_{t+k}$ and token $T_{<t+k}$ is given by: 

% \begin{equation}
%     a_{T_{{t+k}{T_{<t+k}}} = 1 if  E_{T_{<t+k}} >= E{T_{t+k}}
%     else 0
% \end{equation}

% This attention mask enables router to account for exited tokens and get trained accordingly. Since attention mechanism during decoding remains exactly same as that during training, impact on generation quality tends to be minimal as noted in \cref{fig:gen_auality_with_and_without_recompute_attention_show_flops}.  Although MoD does not suffer from training and inference mismatch, we observe that it suffers from discountinuity between pre-training and super-vised fine-tuning resulting in sub-optimal perplexity. On the other hand, our method doesn't not require pre-training , doesn't suffer from discountinuity, and achieves much better perplexity in super-vised fine-tuning and instruction tuning setups as shown in \cref{fig:Mod_vs_m2r2_loss_curves}.






% Our techniques are directly applicable in such scenarios.    




%expert loading with cuda streams in experiments
\section{Experiments: Planning outperforms Heuristics}
\label{sec:experiment}

We begin our empirical demonstrations by showcasing the effectiveness of our planning framework on both synthetic and real datasets. We focus on the simplest planning algorithm, 1-step lookaheads (Algorithm~\ref{alg:complete}), and show that even basic planning can hold great promise. 
We illustrate our framework using two uncertainty quantification modules---GPs and 
\ensembles/ \ensembleplus. 

Throughout this section, we focus on evaluating the mean squared error of 
a regression model $\model$,  and develop adaptive policies that minimize uncertainty on $g(f)$ defined in~\eqref{eqn:l2-g-f}.
When GPs provide a valid model of uncertainty, 
our experiments show that our planning framework significantly outperforms other baselines. 
We further demonstrate that our conceptual framework extends to deep learning-based uncertainty quantification methods such as  \ensembleplus while highlighting computational challenges that need to be resolved in order to scale our ideas. 
For simplicity, we assume a naive predictor, i.e., $\psi(\cdot) \equiv 0$. However, we emphasize that this problem is just as complex as if we were using a sophisticated model $\psi(.)$. The performance gap between the algorithms 
primarily depends
on the level  of uncertainty in our prior beliefs.

To evaluate the performance of our algorithm, we benchmark it against several baselines. 
%Active learning baselines use an acquisition function $\ac$ to select points that have the highest   function value: $X\opt_t \in \argmax_{X \in \xpoolj{t}} \ac({X})$ at every step $t$. These methods may also need an UQ module, which we simply use the same UQ module as in our algorithm, and it  outputs $V(X)$ that measures the the uncertainty of each point $X \in \xpoolj{t}$.
Our first set of baselines are from active learning~\citep{AggarwalKoGuHaPh14}:
\\ % \noindent\textbf{Active Learning Heuristics:} 
\textbf{(1)} 
\textsf{Uncertainty Sampling (Static):}  In this approach, we query the samples for which the model is least certain about. Specifically, we estimate the variance of the latent output $f(X)$ for each $X \in \xpool$ using the UQ module and select the top-$K$ points with the highest uncertainty. \\
\textbf{(2)} \textsf{Uncertainty Sampling (Sequential):} This is a greedy heuristic that sequentially selects the points with the highest uncertainty within a batch, while updating the posterior beliefs using pseudo labels from the current posterior state. Unlike \textsf{Uncertainty Sampling (Static)}, this method takes into account the information gained from each point within batch, and hence tries to diversify the selected points within a batch. 

 
We also compare our approach to the  \textbf{(3)} \textsf{Random Sampling}, which selects each batch uniformly at random from the pool. Additionally, we compare solving the planning problem using  \textsf{REINFORCE}-based policy gradients with   $\mathsf{Smoothed\text{-}Autodiff}$ policy gradients.\footnote{Our code repository is available at
  \url{https://github.com/namkoong-lab/adaptive-labeling}.}
%Detailed experimental setups are provided in Section \ref{sec:details-experiments}.

%We repeat all experiments with 10 random seeds.




\begin{figure}[t]
\centering
\begin{minipage}[b]{0.49\textwidth}
\centering
\includegraphics[width=\textwidth, height=5cm]{figures/original_scale/Var_of_l_2_loss.pdf}
\caption{(Synthetic data) Variance of mean squared loss evaluated through the posterior belief $\mu_t$ at each horizon $t$. This is the objective that policy gradient methods like \textsf{REINFORCE} and $\ouralgo$ optimizes. 1-step lookaheads are surprisingly effective even in long horizons.}
\label{fig:var-l2-sim}
\end{minipage}
\hfill
\begin{minipage}[b]{0.49\textwidth}
\centering \includegraphics[width=\textwidth, height=5cm]{figures/original_scale/Error_of_estimated_model_l_2_loss.pdf}
\caption{(Synthetic data) Error between MSE calculated based on collected data $\mc{D}^{0:T}$ vs. population oracle MSE over $\mc{D}_{\rm eval} \sim P_X$. Reducing uncertainty over posteriors directly leads to better OOD evaluations. 1-step lookaheads significantly outperform active learning heuristics in small horizons.}
\label{fig:mean-l2-sim}
\end{minipage}
%\caption{Simulated data for GPs}
%\label{fig:both_plots}
\end{figure}

\subsection{Planning with Gaussian processes}
\label{sec:experiment-plan-GP}
We now briefly describe the data generation process for the GP experiments,  deferring a more detailed discussion of the dataset generation to Section~\ref{sec:details-experiments}. 
We use both the synthetic data and the real data to test our methodology.
For the \emph{simulated data},  we construct a setting where the general population is distributed across \emph{51 non-overlapping clusters} while the initial labeled data $\dtrain$ just comes from one cluster. In contrast, both $\dpool \defeq (\xpool,\ypool),\deval \defeq (\xeval,\yeval)$ are generated   from all the clusters. 
We begin with a low-dimensional scenario, generating a one-dimensional regression setting using a GP. %Gaussian Process (GP).
Although the data-generating process is not known to the algorithms,  we assume that the GP hyperparameters are known to all the algorithms
to ensure fair comparisons. This can be viewed as a setting where our prior is well-specified, allowing us to isolate the effects
of different policy optimization approaches
 without any concerns about the misspecified priors. We select $10$ batches, each of size $K=5$ across $T = 10$ time horizons.

To examine the robustness of our method against the distributional assumptions made  in the simulated case, we then move to a real dataset where the correct prior is not known. We simulate selection bias from the eICU dataset~\citep{PollardJoRaCeMaBa18}, which contains real-world patient data with in-hospital mortality outcomes. 
We conduct a $k$-means clustering to generate 51 clusters and then select data from those clusters. We view this to be a credible replication of practice, as severe distribution shifts are common due to selection bias in clinical labels.  To convert the binary mortality labels into a regression setting, we train a  random forest classifier and fit a GP on predicted scores, which serves as the UQ module for all the algorithms. As before, the task is to select 10 batches, each consisting of 5 samples, across 10 time horizons.

 In Figures~\ref{fig:var-l2-sim} and~\ref{fig:mean-l2-sim}, we present results for the simulated data. 
Figure~\ref{fig:var-l2-sim} shows the variance of $\ell_2$ loss, and Figure~\ref{fig:mean-l2-sim} presents the error in the estimated $\ell_2$ loss using $\mu_t$ (relative to true $\ell_2$ loss, that is unknown to the algorithm). 
As we can see from these plots, our method one-step lookahead  gives substantial improvements  over active learning baselines and random sampling. In addition,
compared to the one-step lookahead planning approach using \textsf{REINFORCE}-based policy gradients, 
we observe that $\mathsf{Smoothed\text{-}Autodiff}$-based policy gradients provide significantly more robust performance over all horizons.

In Figures~\ref{fig:var-l2-real}~and~\ref{fig:mean-l2-real}, we observe similar findings on the eICU data. We see that planning policies (\textsf{REINFORCE} and $\mathsf{Smoothed\text{-}Autodiff}$) consistently outperform other heuristics by a large margin.  Active learning baselines perform poorly in these small-horizon batched problems and can sometimes be even worse than the random search baselines.  Overall, our results show the importance of careful planning in adaptive labeling for reliable model evaluation. 

We offer some intuition as to why one-step lookahead planning may outperform other heuristic algorithms. 
 First,  \textsf{Uncertainty sampling (Static)} while myopically selects the
 top-$K$ inputs with the highest uncertainty, it fails to consider 
the overlap in information content among the ``best” instances; see \citep{AggarwalKoGuHaPh14} for more details. 
In other words,  it might acquire points from the same region with high uncertainty while failing to induce diversity among the batch.
Although \textsf{Uncertainty Sampling (Sequential)} somewhat addresses the issue of information overlap, a significant drawback of 
this algorithm
is the disconnect between the objective we aim to optimize and the algorithm. For example, it might sample from a region with high uncertainty but very low density. 

\begin{figure}[t]
\centering
\begin{minipage}[b]{0.48\textwidth}
\centering
\includegraphics[width=\textwidth, height=5cm]{figures/original_scale/Var_of_l_2_loss_real.pdf}
\caption{(Real-world eICU data) Variance of mean squared loss evaluated through the posterior belief $\mu_t$ at each horizon $t$. Even 1-step lookaheads are extremely effective planners, and auto-differentiation-based pathwise policy gradients provide a reliable optimization algorithm based on low-variance gradient estimates.}
\label{fig:var-l2-real}
\end{minipage}
\hfill
\begin{minipage}[b]{0.48\textwidth}
\centering \includegraphics[width=\textwidth, height=5cm]{figures/original_scale/Error_of_estimated_model_l_2_loss_real.pdf}
\caption{(Real-world eICU data) Error between MSE calculated based on collected data $\mc{D}^{0:T}$ vs. population oracle MSE over $\mc{D}_{\rm eval} \sim P_X$. Reducing uncertainty over posteriors directly leads to better OOD evaluations. Our method significantly outperforms active learning-based heuristics, and random sampling.}
\label{fig:mean-l2-real}
\end{minipage}
%\caption{Real data for GPs}
\end{figure}
 
%\vspace{-1.5cm}
% \begin{wrapfigure}{r}{.32\columnwidth}
%   \vspace{-.5cm} 
%   \centering
% \includegraphics[scale=.29]{figures/Var of l2l_2 loss.pdf}
%   \vspace{-0.2cm}
%   \caption{Results of GP}
% \label{fig:var-l2-gp}
%   \vspace{-0.1cm}
% \end{wrapfigure}


% Attempts have been made  in the past to address these  drawbacks heuristically  (see \citep{AggarwalKoGuHaPh14}). We give a unified computational framework while approaching the problem in a more principled manner and solving it more optimally.




\subsection{Planning with  neural network-based uncertainty quantification methods ($\ensembleplus$)}


We now provide a proof-of-concept that shows the generalizability of our conceptual framework  to the deep learning-based UQ modules, specifically focusing on $\ensembleplus$ due to their previously observed superior performance~\citep{OsbandWenAsDwIbLuRo23}. Recall that implementing our framework with deep learning-based UQ modules  requires us to retrain the model across multiple possible random actions $\bm{a}(\theta)$ sampled from the current policy $\pi_\theta$.
This requires significant computational resources, in sharp contrast to the GPs where the posteriors are in closed form and can be readily updated and differentiated. 

Due to the computational constraints, we test $\ensembleplus$ on a toy setting to demonstrate the generalizability of our framework. We consider a setting where the general population consists of four clusters, while the initial labeled data only comes from one cluster. Again we generate data using GPs.  The task is to select a batch of 2 points in one horizon. We detail the $\ensembleplus$ architecture in Section \ref{sec:details-experiments}, and we assume prior uncertainty to be large (depends on the scaling of the prior generating functions). 
The results are summarized in the Table~\ref{tab:UQ_ensemble}.

% \begin{table}[H]
% \vspace{-10pt}
% \caption{Performance under \ensembleplus as UQ module}
%     \centering
%     \begin{tabular}{|m{3cm}|m{2.5cm}|m{2cm}|} 
%     \hline
%       Algorithm   & Variance of $\loss_2$ loss estimate & Error of $\loss_2$ loss estimate  \\ \hline Random Sampling 
%          & $1710.9 \pm 1352.1$ & $8.67\pm6.62$ 
%       \\ \hline \ouralgo & $1.30 \pm 0.68$ & $0.91\pm0.25$ \\ \hline
%     \end{tabular}
%     \label{tab:UQ_ensemble}
%     %\vspace{-10pt}
% \end{table}




\begin{table}[h]
\vspace{-10pt}
\caption{Performance under \ensembleplus as the UQ module}
\centering
\begin{tabular}{|l|l|l|}
\hline
Algorithm   & Variance of $\loss_2$ loss estimate & Error of $\loss_2$ loss estimate  \\
\hline
\textsf{Random sampling} & 7129.8 $\pm$ 1027.0 & 136.2 $\pm$ 8.28 \\ \hline
\textsf{Uncertainty sampling (Static)} & 10852 $\pm$ 0.0 & 162.156 $\pm$ 0.0 \\ \hline
\textsf{Uncertainty sampling (Sequential)} & 8585.5 $\pm$ 898.9 & 144 $\pm$ 6.93 \\ \hline
\textsf{REINFORCE} & 1697.1 $\pm$ 0.0 & 45.27 $\pm$ 0.0 \\ \hline
\ouralgo & 1697.1 $\pm$ 0.0 & 45.27 $\pm$ 0.0 \\ \hline
\end{tabular}
%\caption{Comparison of different algorithms based on variance   and   error in $\ell_2$ loss estimation with Ensemble $+$ as the UQ module. Our results demonstrate that {\ouralgo} and REINFORCE outperformthe other active learning based heuristics, confirming the benefits of our MDP formulation for the adaptive labeling problem, as also demonstrated in Section 4.\\
%\footnotesize{Experimental details: We use Gaussian Processes as our data generating process, GP parameters are the same as in Section D.3.  The task is to select a batch of 2 points along one horizon.The marginal distribution $p_X$ has 4 \textit{non-overlapping} clusters. Initial data comes from one cluster, while pool and evaluation points comes from all the clusters. We have $20$ initial labeled data points, $10$ pool points, and $252$ evaluation points.  Training procedures are similar to the one in Section D.3.} }
\label{tab:UQ_ensemble}
\end{table}



% We faced  issues in scaling up these experiments which will be our focus in the future. 





% \begin{itemize}
%     \item Posteriors should be consistent. Two dimensions: even with less training,  
%     \item the inference should be  fast enough
% \end{itemize}


% Potential research directions for uncertainty quantification

% In this section we consider a simple setting We consider a simpler setting and 


% For synthetic dataset generation, we use ...... For real datasets, we use ...... We compare our methodolgy to several baselines ()    This Section is structured as follows:
% \begin{itemize}
%     \item \textbf{GPs, square loss objective} (Section \ref{}): 
%     %the broad aim of the experiments  in this section is to isolate the performance of our methodology without any concerns for the inefficiencies induced due to a mis-specified prior or imperfect posterior inference. To accomplish this we generate synthetic datasets using GPs (detailed later). We use the well specified prior (GPs - with same hyperparameter setting) as our UQ module.   
%      As GPs provide differentaible posterior inference - any errors induced due to imperfect posterior updates are also isolated. We note that under this setting
%      \item In Section\ref{} we demonstrate why our methodology performs better than other baselines - by devising various synthetic experiments ()
%     \item  \textbf{UQ Benchmarking }(Section \ref{}): Before diving into the experiments using $\ensembleplus$ and ENNs,  we showcase our benchmarking experiments in Section \ref{}. We use real datasets We observe that ENNs perform better
%      \item \textbf{Ensemble $+$}, objective: recall, accuracy
%     \item \textbf{ENN}, objective: recall, accuracy
% \end{itemize}




% In Section {}, we test 
% \subsection{Experimental details}

% \begin{itemize}
%     \item UQ methodologies - GPs, ENNs
%     \item Objectives - Recall,  ATE
%     \item Datasets - ATE-synthetic datasets, Recall-synthetic, real datasets
%     \item Baselines - 
%     \begin{itemize}
%         \item Random sampling
%         \item Active learning - Uncertainty based sampling - In regression setting almost all of the 
%         \item Myopic greedy - Greedy Batch based sampling
%         \item Policy Gradient
%     \end{itemize}
    
% \end{itemize}

% \subsection{Experiments}
%     \begin{itemize}
%     \item GPs with square loss
%     \item Benchmarking ENN
%         \item ENNs with ATE
%         \item ENNs with Recall
%     \end{itemize}

% \subsection{Benefits over other algorithms - intuition and experiments}

%Active learning - Myopic greedy / Don't rely on the objective rather some entropy version.


%%% Local Variables:
%%% mode: latex
%%% TeX-master: "main"
%%% End:

\section*{Conclusion}
This paper aims to enhance our understanding of the computational complexity of computing various Shapley value variants. We found that for various ML models --- including decision trees, regression tree ensembles, weighted automata, and linear regression --- both local and global interventional and baseline SHAP can be computed in polynomial time under HMM modeled distributions. This extends popular algorithms, such as TreeSHAP, beyond their empirical distributional scope. We also establish strict complexity gaps between the various SHAP variants (baseline, interventional, and conditional) and prove the intractability of computing SHAP for tree ensembles and neural networks in simplified scenarios. Overall, we present SHAP as a versatile framework whose complexity depends on four key factors: \begin{inparaenum}[(i)] \item model type, \item SHAP variant, \item distribution modeling approach, \item and local vs. global explanations\end{inparaenum}. We believe this perspective provides deeper insight into the computational complexity of SHAP, paving the way for future work.




%We believe that our framework provides a more intricate understanding of SHAP computation complexity across different models, distributions, and variants, paving the way for further research.

Our work opens promising directions for future research. First, expanding our computational analysis to other SHAP-related metrics, such as asymmetric SHAP~\citep{frye20} and SAGE~\citep{covert2020understanding}, would be valuable. Additionally, we aim to explore more expressive distribution classes and relaxed assumptions beyond those in Section \ref{sec:tractable} while maintaining tractable SHAP computation. Finally, when exact computation is intractable (Section \ref{sec:intractable}), investigating the approximability of SHAP metrics through approximation and parameterized complexity theory~\citep{downey2012parameterized} is an important direction.

%Our work opens several promising avenues for future research on the computational properties of explainable AI methods, with a particular focus on SHAP. First, it would be interesting to broaden the computational analysis conducted in this work to include other popular SHAP-related metrics in the literature, such as asymmetric SHAP \cite{frye20} and SAGE \cite{covert2020understanding}. Also, in the future, we aim to explore more expressive distribution classes and relaxed distributional assumptions—extending beyond those examined in Section \ref{sec:tractable} —that still yield tractable SHAP computation. Finally, when exact computation proves intractable (Section \ref{sec:intractable}), it is worthwhile to theoretically investigate the question of the approximability of computing the SHAP metrics across various configurations, through the lens of approximation and parametrized complexity theory \cite{arora2009computational}.

%This paper aims to deepen our understanding of the computational complexity involved in obtaining different Shapley value variants. We found that for a variety of ML models, including decision trees, tree ensembles for regression, weighted automata, and linear regression models — computing both local and global interventional and baseline SHAP can be done in polynomial time when distributions are modeled by HMMs. This extends the distributional scope of popular algorithms like TreeSHAP, which is limited to empirical distributions. Additionally, we demonstrate a strict complexity gap between SHAP variants, showing that interventional and baseline SHAP can be strictly easier to compute than conditional SHAP. Despite these positive results, we uncovered intractability for various SHAP variants in neural networks and tree ensembles. Finally, we provided generalized complexity relations across SHAP variants. We believe that our framework offers a deeper understanding of the complexity involved in computing SHAP across various variants, models, distributions, as well as in both local and global computations, laying the groundwork for future research.

\clearpage
{\small
\bibliographystyle{ieee_fullname}
\bibliography{Main}
}




\clearpage
\appendix


\renewcommand\thesection{\Alph{section}}
\renewcommand\thefigure{S\arabic{figure}}
\renewcommand\thetable{S\arabic{table}}
\renewcommand\theequation{S\arabic{equation}}
\setcounter{figure}{0}
\setcounter{table}{0}
\setcounter{equation}{0}



\section*{Appendix}




\section{Dataset}
\begin{figure*}[htbp]
  \centering
  \includegraphics[width=1\linewidth]{iccv2023AuthorKit/Figures/demo.pdf}
  \caption{
   Additional visualizations. Our one single model tackles multiple perception tasks.
  }
  \phantomsection
  \label{fig:demo}
\end{figure*}

\label{appendix:dataset}
We summarize the datasets used in our work in Table~\ref{tab:data}. The depth and normal data samples are obtained by randomly selecting 500K images from OpenImages~\cite{kuznetsova2020open} and labeling them using Depth Pro~\cite{bochkovskii2024depth} and StableNormal~\cite{ye2024stablenormal}, respectively. The 400K point segmentation data samples are obtained by randomly selecting images from the SA-1B dataset~\cite{kirillov2023segment}. For the synthesis of point segmentation data, we extract the foreground from P3M-10K~\cite{li2021privacy}, AIM500~\cite{li2021deep} and AM2K~\cite{li2022bridging}, randomly applying transformations such as rotation, resizing, and flipping. These transformed foregrounds are then pasted onto different background images, resulting in 200K synthetic images with fine-grained hair segmentation.

For the validation set, we evaluate depth using the same evaluation protocol as Genpercept~\cite{xu2024diffusion}, conducting tests on the NYUv2~\cite{nyu}, KITTI~\cite{kitti}, ScanNet~\cite{scannet}, DIODE~\cite{diode}, ETH3D~\cite{eth3d}. Similarly, for normal estimation, we followed the evaluation protocol of StableNormal~\cite{ye2024stablenormal} and performed evaluations on the NYUv2~\cite{nyu}, ScanNet~\cite{scannet}, DIODE~\cite{diode}. For point segmentation, we conducted extensive comparisons across 23 datasets. The remaining tasks, including Entity Segmentation, Semantic Segmentation, and Human Keypoints, were evaluated on the MS COCO 2017 dataset~\cite{lin2015microsoftcococommonobjects}. We believe the comprehensive experiments provide solid evidence of the remarkable performance of our method.
\begin{figure*}
    \centering
    \includegraphics[width=1\linewidth]{bar2.pdf}
    \caption{(a) shows the bar chart of the raw data, (b) presents the results of applying Moving Average Smoothing to reduce anomalies in prediction percentages, and (c) highlights the reduction of visual clutter and emphasizes sequential behavior patterns after merging behaviors of the same category.}
    \label{fig:bar}
    \Description{(a) shows the bar chart of the raw data, (b) presents the results of applying Moving Average Smoothing to reduce anomalies in prediction percentages, and (c) highlights the reduction of visual clutter and emphasizes sequential behavior patterns after merging behaviors of the same category.}
\end{figure*}

\section{Data Collection and Processing}
\label{sec:data}
\RR{In this section, we provided an overview of the data collection context and introduced the collaborative programming performance framework along with its metric quantification methods.}

\subsection{Data Collection}
We collaborated with Professor E1, an expert in programming education, and teaching assistants (TA1 and TA2), experienced in Python, to collect data from E1's Spring 2023 Python course with 66 non-computer science freshmen in 22 groups. Using non-intrusive methods, we recorded group discussions, screen activities (without audio), and code submissions. Session lengths ranged from 10 to 60 minutes based on question completion. 
Due to data quality issues, we selected data from 19 groups (57 students) for analysis.


\subsection{Data Preprocessing}
In collaborative programming analysis, students' spoken content was key to understanding discussion and evaluating collaboration. We used the Faster-Whisper model~\cite{fasterwhisper} for speech recognition and the Pyannote-audio model~\cite{pyannoteaudio} for speaker diarization. 
For groups lacking clear problem-solving strategies, we used Tesseract OCR~\cite{tesseract} to analyze screen recordings and extract key frames through screenshots.

\subsection{Scope of Collaborative Programming Performance Framework}
Evaluating student and group performance in collaborative programming required considering multiple dimensions~\cite{hawlitschek2023empirical}.  
Building on literature and expert input (E1), we proposed the following comprehensive analytical framework to assess performance. 



\subsubsection{Student Performance Assessment}
\label{shema}
Previous research demonstrated that students' skills, backgrounds, and personalities in the classroom vary significantly, affecting their engagement and learning outcomes~\cite{wu2019analysing}. 
Therefore, we focus on each student's \textit{background} (prior academic performance and major), \textit{role transitions}, \textit{behavioral engagement}, and \textit{cognitive engagement}.






\textbf{Problem-solving Categorization:}
Based on previous frameworks~\cite{wu2019analysing}, team theory~\cite{zhao2023analysing}, and collaborative coding processes~\cite{sun2021three}, we developed a coding scheme (Fig.~\ref{fig:scheme}) to capture group problem-solving in collaborative programming. 
The scheme used four color-coded categories to represent discussion types. 
The first three categories followed a hierarchical structure, indicating discussion depth, while the green category focuses on situation awareness and specific behaviors.

Building on the scheme, we used tailored prompts with the ChatGPT-4o model~\cite{gpt4o} to classify behavioral patterns in transcribed dialogue \RR{(More details are in appendix B)}. 
\RR{The model provided a prediction percentage of uncertainty for each classification, improving result interpretability. }
To minimize anomalies, we applied a ``moving window'' technique with Moving Average Smoothing~\cite{chang2022muse}, stabilizing prediction percentages (Fig.\ref{fig:bar}-b). To reduce visual clutter in long time-series data, we aggregated consecutive instances of the same category, averaging prediction percentages (Fig.\ref{fig:bar}-c). These results were displayed in the timeline panel's progress bar, enabling detailed analysis by zooming into specific behavior categories in Sec.~\ref{barchart}. 




\textbf{Roles Extraction:}
We analyzed each speaker's dynamic roles (Driver, Navigator, and Monitor) during programming~\cite{lewis2011pair}. Using ChatGPT-4o and prompts based on the Thought Chain Model~\cite{wei2022chain}, we guided the model through step-by-step reasoning to generate role classifications. Prompts were iterated for clarity, and the model's responses were structured hierarchically and returned in JSON format. Each query was repeated ten times, with the majority result adopted for classification.

\RR{\textbf{Behavioral Engagement:} reflected the level of effort and participation students invested in learning~\cite{fredricks2022measurement}. 
In our study, we focused on the duration and frequency of student speech.} 
We extracted conversation data, excluding irrelevant chat, and divided each conversation into two parts: the first half and the full conversation. We then measured speaking duration, frequency, and degree centrality using co-occurrence networks~\cite{ng1999toward}. For each question, we created and normalized two networks, followed by Non-negative Matrix Factorization (NMF)~\cite{lee2000algorithms} to identify key behavioral patterns for dynamic group comparison.


\RR{\textbf{Cognitive Engagement:} referred to the cognitive investment students made in their learning. We highlighted the role changes and behavior frequencies of students during the collaborative process. }
To capture dynamic changes in student cognitive engagement, we split the dialogue for each question into two segments: the first half and the full dialogue. We extracted the frequency of each speaker's 14 behavioral categories and their roles at each timestamp. After normalizing these features for consistency, we applied NMF to reduce dimensionality and assess each speaker's cognitive engagement.

\begin{figure*}
  \includegraphics[width=\textwidth]{CPVis.pdf}
  \caption{\RR{A screenshot of Group 10 view.} \textit{CPVis} applies multimodal learning analysis to provide instructors with evidence for evaluating group and student performance. It consists of three views:
Filter View (A) Provides an overview and allows group selection. The selected groups appear in the lasso selection area (A2), and the similarity panel (A3) displays the most similar and different groups based on the search (A1a).
Content View (B) Displays group performance, with the B1 panel showing completed codes, the B3a panel illustrating the behavior sequence, and the B3b panel showing student engagement over time.
Detail View (C) Presents the group's collaborative programming video (C1) and raw conversation data (C2).}
  \Description{A screenshot of Group 10 view. \textit{CPVis} applies multimodal learning analysis to provide instructors with evidence for evaluating group and student performance. It consists of three views:
Filter View (A) Provides an overview and allows group selection. The selected groups appear in the lasso selection area (A2), and the similarity panel (A3) displays the most similar and different groups based on the search (A1a).
Content View (B) Displays group performance, with the B1 panel showing completed codes, the B3a panel illustrating the behavior sequence, and the B3b panel showing student engagement over time.
Detail View (C) Presents the group's collaborative programming video (C1) and raw conversation data (C2).}
  \label{fig:teaser}
  \end{figure*}

\subsubsection{Group Performance Assessment}
We evaluated group performance based on three dimensions: code quality, collaborative problem-solving, and teacher scaffolding. 
Through in-depth discussions with domain experts, we assessed how each dimension was valued and measured in the context of our study.




\label{code}
\textbf{Code quality}, reflecting students' mastery of course concepts, was a key metric for evaluating group performance. To assess student submissions, we used ChatGPT-4o~\cite{gpt4o} to evaluate dimensions such as problem-solving, code integrity, accuracy, and algorithmic innovation, scoring each on a 1–5 scale. After refining evaluation prompts, we ran the assessment ten times per submission, averaging the results to ensure consistency and reliability.





\textbf{Collaborative Problem-Solving (CPS):} 
Earlier studies categorized CPS into team effectiveness and task effectiveness~\cite{rosen2020towards}. Team effectiveness was measured by student engagement, while task effectiveness was assessed through code quality. %Our analysis captured problem-solving behaviors by frequency and sequence.
To evaluate CPS, we examined task effectiveness, represented by the average question score (\(\bar{s}\)), and team effectiveness, assessed through the standard deviation of engagement (\(\sigma_e\)) and the average engagement score (\(\bar{e}\)) as shown in Equation \ref{eq:1}. We then used the coefficient of variation (\(CV_e\)) \RR{to account for both engagement variability and engagement}. Finally, the overall collaboration quality was calculated using Equation \ref{eq:2}, combining question performance and engagement balance. 
\begin{equation}
\sigma_e = \sqrt{\frac{1}{n} \sum_{i=1}^{n} (e_i - \bar{e})^2}, \quad CV_e = \frac{\sigma_e}{\bar{e}}
\label{eq:1}
\end{equation}

\begin{equation}
Quality = \bar{s} \cdot (1 - CV_e)
\label{eq:2}
\end{equation}
As shown in Table \ref{table:comparison}, Group 19, despite achieving a respectable average score, exhibited imbalanced engagement, leading to a lower collaboration quality score. In contrast, Group 20 demonstrated more balanced and higher engagement, resulting in a better overall collaboration quality.
\begin{table}[htbp]
\centering
\begin{tabular}{cccccc}
\toprule
\textbf{Group} & \(\bar{s}\) & \textbf{Engagement Levels} & \(\sigma_e\) & \(\text{CV}_e\) & \textbf{CQ} \\
\midrule
Group 19 & \(4.11\) & (10.515, 9.725, 4.575) & \(2.80\) & \(0.24\) & \(2.80\) \\
Group 20 & \(4.14\) & (10.06, 9.32, 8.62) & \(0.73\) & \(0.08\) & \(3.88\) \\
\bottomrule
\end{tabular}
\caption{Comparison of Group 19 and Group 20 on Collaboration Quality (CQ).}
\label{table:comparison}
\end{table}

\textbf{Teacher Scaffolding,} categorized into cognitive (low, medium, high-control) and metacognitive forms~\cite{ouyang2022applying}, reflected the level of support provided to a group and its impact on programming performance. We evaluated four scaffolding dimensions, leveraging GPT-4o for annotation. By using targeted prompts and examples, we improved classification accuracy, while teacher scaffolding was categorized according to the type of support based on a semantic analysis of interactions.





\section{Post-processing}

\label{appendix:post_processing}

\begin{figure*}[h!]
  \centering
  \includegraphics[width=1\linewidth]{iccv2023AuthorKit/Figures/fur.pdf}
  \caption{
   Segmentation results on furry objects.
  }
  \phantomsection
  \label{fig:fur}
\end{figure*}




\begin{table}
    \caption{Evaluation of human keypoints estimation on MS COCO.}
    \centering
    \resizebox{\linewidth}{!}{
    \begin{tabular}{c|ccc|ccc}
       \hline
        & HRNet\cite{hrnet}  & HRFormer\cite{hrformer} & ViTPose\cite{vitpose} & Painter\cite{wang2023images} & 
       Ours  \\
       \hline
        AP$\uparrow$ & 76.3 & 77.2 & 78.3 & 72.5 & 
        57.8  \\
       \hline
        
    \end{tabular}
    }
    \label{tab:pose}
    \vspace{-2mm}
\end{table}



\section{Speech Tokenization Methods: Semantic Tokens}

\label{sec:semantic}
Semantic tokens refer to discrete speech representations from discriminative or self-supervised learning (SSL) models.
While we use the term \textit{semantic tokens} to maintain consistency with prior works, some researchers recently argue that SSL features are more accurately described as \textit{phonetic} than \textit{semantic}~\cite{choi24b_interspeech} in nature.
Hence to clarify, in this review, semantic tokens should be more accurately defined as the complementary set of acoustic tokens, such that they are not primarily aimed at reconstruction purposes.
In practice, the vast majority of these tokens are designed for discriminative tasks and are believed to have a strong correlation with phonetic and semantic information~\cite{wells22_interspeech,mohamed2022self,sicherman2023analysing,yeh2024estimating}.

\subsection{Semantic Tokens from General-Purpose SSL}
\label{sec:semantic-general}
\subsubsection{Motivation}
% A large branch of semantic tokens come from speech SSL features. 
Speech SSL models have consistently outperformed many traditional methods in various speech tasks~\cite{superb,mohamed2022self}.
Their potential has been extensively mined in discriminative tasks such as automatic speech recognition (ASR)~\cite{wav2vec,vq-wav2vec,hsu2021hubert,zhang2020pushing}, automatic speaker verification (ASV)~\cite{chen2022wavlm,jung2024espnet,miara24_interspeech}, speech emotion recognition (SER)~\cite{morais2022speech,chen2022wavlm,MADANIAN2023200266,ma-etal-2024-emotion2vec} and speech translation (ST)~\cite{wu20g_interspeech,nguyen20_interspeech,babu22_interspeech}.
% \textcolor{red}{TODO: add citations on these tasks with SSL inputs.}
Discretized SSL tokens are initially favored for reducing computation costs and improving robustness against irrelevant information for ASR~\cite{chang23b_interspeech}.
As language models have gained increasing attention, these SSL tokens have been further explored in generative tasks such as TTS~\cite{VQTTS,kharitonov2023speak,vectokspeech} and SLM~\cite{lakhotia2021generative,borsos2023audiolm,hassid2024textually}.
This is because they can be considered high-level abstractions of speech semantics that are largely independent of acoustic details.
% \textcolor{red}{TODO: not finished. Perhaps should have a logic plan first.}

\begin{figure}
    \centering
    % \includegraphics[width=0.85\linewidth]{figs/semantic1.png}
    % \includegraphics[width=0.7\linewidth]{figs/semantic2.png}
    \includegraphics[width=0.99\linewidth]{figs/semantic.png}
    \caption{Representatives in different kinds of semantic tokens. 
    % Upper: semantic tokens from \textbf{general-purpose SSL models}; Middle: \textbf{perturbation-invariant SSL models}; Bottom: semantic tokens from \textbf{supervised models}. 
    ``Q.'' denotes quantizer, which can be optional (dotted line).}
    \vspace{-0.1in}
    \label{fig:semantic-types}
\end{figure}
\subsubsection{Approaches}

SSL models initiate the learning process by defining a pretext task which enables the model to learn meaningful representations directly from the data itself. 
Typical speech SSL models employ CNNs and Transformer encoders to extract deep contextual embeddings.
When it comes to semantic tokens, there are mainly two ways to extract those discrete tokens from an SSL model (see upper part of Fig.\ref{fig:semantic-types}):
\begin{itemize}[leftmargin=5mm]
    \item External quantization, like clustering or training a VQ-VAE. This refers to extracting continuous embeddings from a certain layer or multiple layers in a pretrained SSL model, and performing quantization manually.
    For example, a common semantic token is the HuBERT+kmeans units, where k-means clustering is performed on a HuBERT Transformer layer with a portion of training data~\cite{lakhotia2021generative,kharitonov-etal-2022-text}.
    It is also feasible to perform clustering on multiple layers~\cite{shi24h_interspeech,mousavi2024should}, or train a VQ-VAE on the SSL hidden embeddings~\cite{huang2023repcodec,wang2024maskgct}.
    \item When an SSL model contains an inner quantizer that is trained together with other network modules, its outputs can also be regarded as semantic tokens.
    Many SSL models involve quantizers to produce targets for their training objectives~\cite{vq-wav2vec,baevski2020wav2vec,chiu2022self,zhu2025muq}.
    This approach provides an efficient and effective way of extracting semantic tokens.
\end{itemize}
Note that for SSL models with an inner quantizer, it is still practical to perform external quantization on its continuous embeddings, like wav2vec 2.0~\cite{baevski2020wav2vec}.
However, these two methods -- internal and external quantization -- may result in different patterns of information exhibition, which we will investigate in Section \ref{sec:analysis}.

For general-purpose SSL models, there are different designs on the pretext task~\cite{mohamed2022self}.
Table \ref{tab:semantic-metadata} provides a high-level summary of well-known semantic tokens.

\paragraph{Contrastive} This type of speech SSL models aims to learn representations by distinguishing a target sample (positives) from distractors (negatives) given an anchor~\cite{mohamed2022self}.
They minimize the latent space similarity of negative pairs and maximize that of the positive pairs.
For semantic tokens, vq-wav2vec~\cite{vq-wav2vec} and wav2vec 2.0~\cite{baevski2020wav2vec} are two representative contrastive SSL models.
They involve a quantizer to produce localized features that is contrastively compared to contextualized continuous features.
Vq-wav2vec~\cite{vq-wav2vec} uses pure CNN blocks while wav2vec 2.0~\cite{baevski2020wav2vec} adopts a Transformer for stronger capacity.
Both use GVQ quantizers with two groups to expand the VQ space.
Wav2vec 2.0 has also been extended to massively multilingual versions~\cite{conneau21_interspeech,babu22_interspeech,pratap2024scaling}.

\paragraph{Predictive}
This type of speech SSL models incorporates an external target for prediction, either from signal processing features or another teacher network.
A popular line of work is HuBERT~\cite{hsu2021hubert}.
It takes raw waveforms as inputs, applies random masks on the hidden representations before Transformer contextual blocks, and then predicts k-means quantized targets from MFCC or another HuBERT teacher.
% It can take more self-iterations by using a trained HuBERT teacher model and applying k-means clustering as targets.
WavLM~\cite{chen2022wavlm} augments HuBERT by additional speaker and noise perturbations to achieve superior performance in more paralinguistic-related tasks.
There are no inner quantizers in both models, so external quantization like k-means clustering is necessary to obtain semantic tokens.
BEST-RQ~\cite{chiu2022self} changes the prediction target to the output of a random projection quantizer.
% Similar to acoustic tokens, training a VQ-VAE to compress continuous semantic features in a vector quantized space is also explored. \textcolor{red}{RepCodec~\cite{huang2023repcodec}, token in MaskGCT, etc.}
% Data2vec~\cite{baevski2022data2vec,baevski2023efficient} proposes a general teacher-student masked prediction framework the masked and original view of data are fed to the student and teacher respectively, and the student network predicts the teacher outputs. 
The next-token prediction criterion from language models (LMs) have also been adopted into speech SSL~\cite{turetzky2024last,han2024nest}, either with or without a pretrained text LM.
This method emphasizes the autoregressive prediction property of learned tokens that may be better suited for the LM use case.

\subsubsection{Challenges}
% 0. data
Firstly, SSL models typically require large amount of data to train, as indicated in Table \ref{tab:semantic-metadata}.
% 1. clustering problems
For SSL models without a built-in quantizer during pretraining, k-means clustering is a prevalent approach to obtain discrete units.
% However, since most SSL models work in a high-dimensional space (e.g. with 768 or 1024 dimensions), the space and time complexity of such k-means procedures are large.
However, given that most SSL models operate in high-dimensional spaces (e.g., with 768 or 1024 dimensions), the space and time complexity of k-means clustering are substantial. 
% The clustering result is sometimes unreliable because of the curse of dimensionality in the Euclidean space.
The clustering results can sometimes be unreliable due to the curse of dimensionality in Euclidean space.
% 2. Acoustics and reconstruction
Moreover, it is often reported, and will also be shown by experiments in Section \ref{sec:analysis}, that discretized SSL units lose much acoustic details after quantization~\cite{polyak21,sicherman2023analysing,mousavi2024dasb}.
Different clustering settings, such as the chosen layer and vocabulary size, can lead to different outcomes within a single model.
% 3. causality and stream-ability
Finally, since most SSL models utilize Transformer blocks, their causality and streaming ability are compromised.

\subsection{Semantic Tokens from Perturbation-Invariant SSL}
\label{sec:semantic-invariant}
\subsubsection{Motivation}
As SSL tokens feature semantic or phonetic information, a major concern is to improve the resistance against perturbations in the input signal.
This kind of invariance includes noise and speaker aspects that don't affect the contents of speech.
Noise invariance refers to the invariance against signal augmentations such as additive noise, reverberations, etc.
Speaker invariance aims to remove speaker information, similar to speaker-disentangled acoustic tokens.
% SSL semantic tokens with perturbation invariance are often obtained by training with explicit perturbations.
% Perturbations are often explicitly introduced in training of these perturbation-invariant SSL models.
In the training process, perturbations are often explicitly introduced in these perturbation-invariant SSL models.
The original and perturbed view of an utterance are both fed to the same network (or teacher and student networks), and an external loss to reduce the impact of perturbation is applied.
The middle part of Fig.\ref{fig:semantic-types} depicts a typical perturbation-invariant SSL model.

\subsubsection{Approaches}

% \textcolor{red}{Another way to organize this section is to first introduce noise and speaker perturbation methods, and then treat them like the same, and introduce contrastive, distribution-similarity, CTC respectively.}

\paragraph{Perturbations}
The perturbations can either be designed to augment the acoustics or alter the speaker timbre, depending on the objective of invariance.
These perturbations usually preserve temporal alignments, meaning that the perturbed utterance and the original one are strictly synchronized.
For noise-invariant SSL tokens, basic signal variations like time stretching, pitch shifting, additive noise, random replacing, reverberation, and SpecAugment~\cite{park2020specaugment} are commonly applied~\cite{gat2023augmentation,ccc-wav2vec2.0,messica2024nast,huang2022spiral}.
% ~\cite{park2020specaugment} is also used in \cite{huang2022spiral}.
Typical speaker timbre perturbations include formant and pitch scaling as well as random equalization~\cite{qian2022contentvec,chang23_interspeech,chang2024dc}.
In contrast, random time stretching is applied as speaker perturbation in \cite{hwang2024removing}, which alters the tempo in each random segment.

\paragraph{Contrastive-based Methods}
Contrastive loss is a common method to obtain perturbation-invariant representations.
In this context, the contrastive loss is a modified version of that used in wav2vec 2.0~\cite{baevski2020wav2vec}.
Given two embedding sequences derived from the original and perturbed utterances, assuming the perturbation preserves frame-wise alignment, the positive sample of an anchor is taken from the same position in the other utterance.
This is because the content remains unchanged by the perturbation, thus the same position of two representation sequences should encode the same information.
In noise-invariant models~\cite{huang2022spiral,ccc-wav2vec2.0}, negative samples are selected from the other utterance relative to the anchor.
However, in speaker-invariant models~\cite{qian2022contentvec,hwang2024removing}, negative samples are selected from the same utterance as the anchor.
Specifically, in \cite{hwang2024removing}, soft attention pooling is applied to create equal-length representation sequences from two utterances with different durations.
This approach forces SSL models to ignore acoustic differences and focus solely on the unperturbed content.

\paragraph{Distribution-based Methods}
Another method to achieve invariance is to minimize some distance metrics between the representations extracted from the original and perturbed utterances.
In existing perturbation-invariant SSL models, this is typically accomplished using a cross-entropy loss between the underlying distributions in the VQ module of the SSL model.
NAST~\cite{messica2024nast} trains a Gumbel-based VQ-VAE on HuBERT features and enforces similarity between the Gumbel distributions Eq.\eqref{eq:gumbel-softmax} derived from the original and augmented utterances.
Spin~\cite{chang23_interspeech} and DC-Spin~\cite{chang2024dc} explore a speaker-invariant clustering algorithm for HuBERT features.
Similar to NAST~\cite{messica2024nast}, Spin employs a cross-entropy loss to ensure that the distributions over codebook entries are similar between the original and perturbed utterances.
% Spin uses a distribution smoothing technique before pushing the distributions to be similar, thereby preventing collapse into a trivial solution~\cite{chang23_interspeech}.
This distribution-based approach forces the same content to be quantized to the same index regardless of acoustic conditions.
% DC-Spin~\cite{chang2024dc} uses Spin units to train a HuBERT model and extends the Spin algorithm to incorporate two VQ codebooks, both optimized with the same objective
% The auxiliary codebook is designed to be larger than the primary one, allowing for more fine-grained acoustic details
% Additionally, DC-Spin explores fine-tuning with mel reconstruction and supervised ASR, which are anticipated to further enhance speaker invariance.

\paragraph{CTC-based Methods}
Noise invariance can also be achieved like an ASR task with perturbed speech inputs.
As semantic tokens from SSL models are highly content-related, these tokens extracted from the original clean utterance can serve as some pseudo-label for a perturbed view.
% Normally, 
In \cite{gat2023augmentation}, a connectionist temporal classification (CTC)~\cite{ctc} loss is calculated between quantized tokens from the augmented signal and a pretrained HuBERT+kmeans pseudo-labels from the clean signal.
This pushes the quantized tokens to have the same phonetic structure with the pseudo-labels.

\subsubsection{Challenges}
While noise and speaker-invariance have emerged as promising approaches in semantic tokens, they currently rely on content-preserving perturbations that are typically hand-crafted.
Most existing methods have only been evaluated on small-scale data and models.
It also remains unclear how these methods will generally benefit generative tasks such as speech generation and spoken language modeling.

% Contrastive approaches are explored in \cite{huang2022spiral,ccc-wav2vec2.0}.
% There, the original and augmented utterances are fed to the same network (or the teacher and student respectively) to obtain two sequences of representations.
% The contrastive loss from wav2vec 2.0 is borrowed, but the positive and negative samples are taken from the other utterance instead of the same utterance.


% \paragraph{Speaker invariance}
% Common speaker perturbation in this line of work include 
% ContentVec~\cite{qian2022contentvec} and \cite{hwang2024removing} adopt a contrastive objective similar to noise invariance SSL.
% ContentVec chooses to base on the HuBERT architecture and take negative samples from the same utterance than the perturbed utterance.
% ContentVec also introduces a voice conversion module to provide teacher labels from another speaker, for further eliminating the speaker information.
% Hwang et al.~\cite{hwang2024removing}, instead, chooses the CPC framework~\cite{oord2018representation,wav2vec} and introduces variable-length soft-pooling.



\begin{table}[]
\centering
\caption{A high-level summary of famous semantic speech tokens. Notations follow Table.\ref{tab:acoustic-metadata}.
Symbol `/' denotes different versions. 
``Inner Quantizer'' refers to whether the model has a quantizer, or external quantization (e.g. clustering) must be performed.
$F$ denotes frame rate.
In case there are inner quantizers, $Q,V$ denote number of quantizers and vocabulary size for each quantizer, respectively.
% $Q$ denotes number of quantizers (if there are), and $F$ denotes frame rate.
``\textit{NR}.'' means not reported.
% \textcolor{red}{Shall we change this table? Should more info be included?}
}
\label{tab:semantic-metadata}
\resizebox{\columnwidth}{!}{
% {
\begin{tabular}{@{}lcccccc@{}}
\toprule
\textbf{\makecell{Semantic \\Speech Tokens}} & \textbf{\makecell{Criterion \\ / Objective}} & \textbf{\makecell{Training\\Data (h)}} & $F$ \textbf{(Hz)} & \textbf{{Inner Quantizer}} \\ \midrule
\multicolumn{5}{l}{\textbf{\textit{General-purpose self-supervised learning (SSL) models}}} \\
vq-wav2vec~\cite{vq-wav2vec} & Contrastive & 0.96k & 100 & GVQ, $Q=2,V=320$ \\
wav2vec 2.0~\cite{baevski2020wav2vec} & Contrastive & 60k & 50 & GVQ, $Q=2,V=320$ \\
XLSR-53~\cite{conneau21_interspeech} & Contrastive & 50k & 50 & GVQ, $Q=2,V=320$ \\
HuBERT~\cite{hsu2021hubert} & Predictive & 60k & 50 & No \\
WavLM~\cite{chen2022wavlm} & Predictive & 94k & 50 & No \\
BEST-RQ~\cite{chiu2022self} & Predictive & 60k & 25 & {No} \\ 
w2v-BERT~\cite{chung2021w2v} & {Predictive+Contrastive} & 60k & 50 & VQ, $Q=1,V=1024$ \\
w2v-BERT 2.0~\cite{barrault2023seamless} & {Predictive+Contrastive} & 4500k & 50 & GVQ, $Q=2,V=320$ \\
% data2vec 2.0~\cite{baevski2023efficient} & Predictive & 60k& 50Hz  & No \\
DinoSR~\cite{liu2024dinosr} & Predictive & 0.96k & 50 & VQ, $Q=8,V=256$ \\
NEST-RQ~\cite{han2024nest} & {Predictive} & 300k &  25 & {No} \\
LAST~\cite{turetzky2024last} & {Predictive} & \textit{NR.} & 50 & VQ, $Q=1,V=500$ \\
\midrule
\multicolumn{5}{l}{\textbf{\textit{SSL models with perturbation-invariance}}} \\
{Gat et al.~\cite{gat2023augmentation}} & Noise Invariance & 0.10k & 50 & VQ, $G=1,V=50$-$500$  \\
ContentVec~\cite{qian2022contentvec} & Speaker Invariance & 0.96k & 50 & No \\
SPIRAL~\cite{huang2022spiral} & Noise Invariance & 60k & 12.5Hz & No\\
CCC-wav2vec 2.0~\cite{ccc-wav2vec2.0} & Noise Invariance & 0.36k & 50 & GVQ, $G=2,V=320$ \\
Spin~\cite{chang23_interspeech} & Speaker Invariance & 0.10k & 50 & VQ, $Q=1,V=128$-$2048$\\
NAST~\cite{messica2024nast} & Noise Invariance & 0.96k & 50 & VQ, $Q=1,V=50$-$200$\\
DC-Spin~\cite{chang2024dc} & Speaker Invariance & 0.96k & 50 & VQ, $Q=2,V=(50$-$500)$+$4096$ \\
% \textcolor{red}{Hwang et al.~\cite{hwang2024removing}} & Speaker Invariance & 0.96k & \\
\midrule
\multicolumn{5}{l}{\textbf{\textit{Supervised models}}} \\
% Whisper~\cite{whisper} & Supervised ASR & 680k & 50Hz & No \\
$\mathcal S^3$ Tokenizer~\cite{du2024cosyvoice}  & Supervised ASR & 172k & 25 / 50  & VQ, $Q=1,V=4096$ \\
Zeng et al.~\cite{zeng2024scaling} & Supervised ASR & 90k & 12.5 & VQ, $Q=1,Q=16384$ \\
Du et al. \scriptsize{(CosyVoice 2)}~\cite{cosyvoice2} & Supervised ASR & 200k & 12.5 & FSQ, $Q=8,V=3$ \\
\bottomrule
\end{tabular}
}
\vspace{-0.15in}
\end{table}

\IEEEpubidadjcol

\subsection{Semantic Tokens from Supervised  Models}
\label{sec:semantic-supervised}
As representing semantic or phonetic information is the major purpose of semantic tokens, a more direct way to achieve this is through supervised learning.
A famous example shown at the bottom of Fig.\ref{fig:semantic-types} is the $\mathcal S^3$ Tokenizer from CosyVoice~\cite{du2024cosyvoice}.
It places a single-codebook VQ layer between two Transformer encoder modules and optimizes the network through an ASR loss similar to Whisper~\cite{whisper}.
The same method is adopted in \cite{zeng2024scaling,zeng2024glm} where the frame rate is further reduced to 12.5Hz.
CosyVoice 2~\cite{cosyvoice2} improves $\mathcal S^3$ Tokenizer by replacing plain VQ with FSQ for better codebook utilization.
Note that in this kind of supervised semantic tokens, it is the output of the VQ layer that serves as tokens.
This allows for more preservation of paralinguistic information than directly transcribing speech into text.
% Whisper~\cite{whisper}, on the other hand, needs an extra quantization step to produce discrete semantic tokens since it operates on a continuous embedding space.
These supervised tokenizers are trained on massive paired speech-text data, and have demonstrated rich speech content understanding capabilities~\cite{du2024cosyvoice,fang2024llamaomni}.
% citing llama-omni because it uses continuous whisper as speech encoder.

However, training these models is highly costly due to the heavy data demands.
Training with only the ASR task may still result in the loss of some prosody information.
Although \cite{cosyvoice2} has demonstrated that its supervised tokenizer trained on Chinese and English can also work in Japanese and Korean, it remains unclear how well these supervised tokenizers generalize to more unseen languages.


\subsection{Post-processing for Keypoints}
For keypoints, since all keypoints were labeled in red during training, our first step in post-processing is to extract all red regions from the RGB output. Next, we identify all connected components within the extracted red regions. For each connected component, we further extract sub-regions that approximate a circular shape. This step is crucial because, in some cases, multiple predicted keypoints may overlap, requiring us to separate them as much as possible. For example, when a person clasps his hands together, the keypoints for both hands may overlap.

Once the circular regions are identified, we compute their center points as the predicted keypoint coordinates. Since our model does not explicitly predict the type of each keypoint (\textit{e.g.}, hand, foot), we assign keypoint types by measuring the distance between the extracted keypoints and the ground truth (GT) keypoints. Each predicted keypoint is assigned the type of its nearest GT keypoint. To ensure robustness, we apply a distance threshold, considering only those predicted keypoints that are sufficiently close to a GT keypoint. Finally, all extracted keypoints that are successfully matched to a GT keypoint form our final predicted keypoint coordinates after post-processing. The algorithm is shown in Algorithm~\ref{alg:pose}.

\subsection{Post-processing for RGB Masks}
For entity segmentation and semantic segmentation RGB masks, we employ clustering algorithms to extract the object masks. Specifically, we first compute the histogram peaks for each of the three RGB channels and estimate the number of clusters by averaging the peak counts across the three channels. We then use KMeans clustering to group the colors and identify the clustered regions in the RGB mask.
For each identified cluster, we extract regions with RGB values close to the cluster's centroid. This step is followed by morphological operations to refine the extracted masks, such as filling holes and removing small, fragmented regions. We further filter the masks by computing their area, excluding any regions that are too small to be meaningful.

Additionally, we also consider the number of connected components within the extracted masks, discarding overly fragmented results that have too many connected components. Finally, we refine the extracted masks by calculating the Intersection over Union (IoU) between them, removing any duplicate or overlapping masks. The algorithm is shown in Algorithm~\ref{alg:segmentation}.

\subsection{Performance Degradation of RGB Masks}
\label{appendix:mask_degradation}
\begin{figure*}[htbp]
  \centering
  \includegraphics[width=.85\linewidth]{iccv2023AuthorKit/Figures/semantic_degradation.pdf}
  \caption{
   When post-processing RGB masks, small regions and excessive numbers of objects lead to significant metric degradation.
  }
  \phantomsection
  \label{fig:degradation}
\end{figure*}


\begin{table}
    \centering
    \caption{When post-processing RGB masks, small regions and excessive numbers of objects significantly lead to performance degradation.}
    \resizebox{.4\linewidth}{!}{
    \begin{tabular}{c|cccc}
    \hline
        Category & AP $\uparrow$ \\
        \hline
        Bear & 76.3 \\
        Dog & 68.9 \\
        Cat & 71.7 \\
        Person & 18.6 \\
        Bird & 10.4 \\
        Book & 10.8 \\
    \hline
    \end{tabular}
    }
    
    \label{tab:mask_degradation}
\end{table}

We observe that while the quality of our semantic segmentation visualizations is high, the average precision (AP) for certain categories remains unsatisfactory.
For example, for the Person category, we conducted exhaustive experiments and achieved good visualization results (highlighted by the green rectangle in Figure~\ref{fig:degradation}), but AP is low (as in Table~\ref{tab:mask_degradation}).
Although, clearly, there is room for us to improve the semantic segmentation results, we do not intend to fit the data bias of those existing datasets, as pointed out by other authors, \textit{e.g.}, \cite{ravi2024sam}.  

We trace the root cause of this issue to degradation during post-processing, particularly due to small objects and an excessive number of objects. Specifically, during mask processing, we filter out small noise regions, but this also removes some positive samples, such as the crowd and the bird highlighted in red in rows 3 to 5 in Figure~\ref{fig:degradation}. However, if we do not filter these noise regions, they further degrade the results.
In our setting, filtering noise regions results in better metrics compared to not filtering them. Additionally, when an image contains an excessive number of objects of the same category (as in row 6 of Figure~\ref{fig:degradation}), post-processing may erroneously group similarly colored but distinct objects into a single class, leading to lower metrics. 
Furthermore, as in Table~\ref{tab:mask_degradation}, we examine categories with fewer small objects and instances of those categories, such as bear, dog, and cat, and observe higher AP scores. However, for categories with opposite characteristics, their AP scores tend to be lower.

Although we can optimize post-processing for individual 
datasets 
by adjusting hyperparameters for each image to achieve the best results, this approach becomes impractical for large-scale \textit{in-the-wild} evaluation, as it requires significant manual effort. Consequently, the dependency on post-processing remains a limitation of our approach.
Table \ref{tab:binary_vs_continuous} compares the binary and continuous variants of CoT-corruption-based metrics. Table \ref{tab:diagnosticity_memit} reports the diagnosticity scores when the knowledge editing method is switched from ICE to MEMIT, while Table \ref{tab:diagnosticity_model_generated} presents the scores when using model-generated explanations instead of synthetic ones. Table \ref{tab:scaling_diagnosticity} examines how diagnosticity scores vary with model size for selected metrics. Additionally, Figure \ref{fig:ppl_comparison_memit_vs_ice} compares MEMIT and ICE in terms of edit success across three tasks, whereas Figure \ref{fig:ppl_comparison_size} illustrates the edit success of models of different sizes across four tasks.

\begin{table*}[t!]
        \centering
        \resizebox{\linewidth}{!}{
        \begin{tabular}{lcccccccc}
        \toprule
        \multirow{2}{*}{\textbf{Metric}} & \multicolumn{2}{c}{\textbf{FactCheck}} & \multicolumn{2}{c}{\textbf{Analogy}} & \multicolumn{2}{c}{\textbf{Object Counting}} & \multicolumn{2}{c}{\textbf{Multi-hop}} \\
        \cmidrule(lr){2-3} \cmidrule(lr){4-5} \cmidrule(lr){6-7} \cmidrule(lr){8-9}
        & \textbf{Bin.} & \textbf{Cont.} ($\Delta$)
        & \textbf{Bin.} & \textbf{Cont.} ($\Delta$)
        & \textbf{Bin.} & \textbf{Cont.} ($\Delta$)
        & \textbf{Bin.} & \textbf{Cont.} ($\Delta$) \\
        \midrule
    Early Answering & 0.030 & 0.752  \textcolor{forestgreen}{(+0.722)} & 0.014 & 0.501  \textcolor{forestgreen}{(+0.487)} & 0.082 & 0.468  \textcolor{forestgreen}{(+0.386)} & 0.060 & 0.490  \textcolor{forestgreen}{(+0.430)}  \\
            Filler Tokens & 0.008 & 0.824  \textcolor{forestgreen}{(+0.816)}& 0.004 & 0.554  \textcolor{forestgreen}{(+0.550)}& 0.047 & 0.432  \textcolor{forestgreen}{(+0.385)}& 0.050 & 0.515  \textcolor{forestgreen}{(+0.465)}  \\
            Adding Mistakes & 0.074 & 0.493  \textcolor{forestgreen}{(+0.419)} & 0.082 & 0.580  \textcolor{forestgreen}{(+0.498)} & 0.150 & 0.579  \textcolor{forestgreen}{(+0.429)} & 0.085 & 0.440  \textcolor{forestgreen}{(+0.355)}  \\
            Paraphrasing & 0.278 & 0.561  \textcolor{forestgreen}{(+0.283)} & 0.025 & 0.561  \textcolor{forestgreen}{(+0.536)} & 0.186 & 0.609  \textcolor{forestgreen}{(+0.423)} & 0.050 & 0.510  \textcolor{forestgreen}{(+0.460)}  \\
                \bottomrule
        \end{tabular}
        }
        \caption{Comparison of diagnosticity scores between continuous and binary variants of CoT corruption-based metrics using \texttt{qwen-2.5-7b}.}
        \label{tab:binary_vs_continuous}
    \end{table*}


\begin{figure}[htb]
    \centering
    \includegraphics[width=\linewidth]{figures/ppl_comparison_memit_vs_ice.pdf}
    \caption{Comparison of the edit reliability of two editing methods across three tasks using \texttt{qwen2.5-7b}. A higher frequency indicates greater success in applied edits.}
    \label{fig:ppl_comparison_memit_vs_ice}
\end{figure}

\begin{figure}[htb]
    \centering
    \includegraphics[width=\linewidth]{figures/ppl_comparison_size.pdf}
    \caption{Comparison of the edit reliability across four tasks using models of varying sizes: \texttt{qwen2.5-7b-instruct}, \texttt{qwen2.5-32b-instruct-awq}, \texttt{qwen2.5-72b-instruct-awq}. A higher frequency indicates greater success in applied edits.}
    \label{fig:ppl_comparison_size}
\end{figure}

\begin{table*}[t!]
        \centering
        \resizebox{0.8\linewidth}{!}{
        \begin{tabular}{lcccc}
        \toprule
        \textbf{Metric} & \textbf{FactCheck} & \textbf{Analogy} & \textbf{Object Counting} \\
        \midrule

        \multicolumn{4}{l}{\textbf{Post-hoc}} \\
        \quad CC-SHAP & \textbf{\underline{0.541}} & \textbf{0.130} & \textbf{\underline{0.580}}   \\
                \quad Simulatability & 0.019 & 0.022 & 0.000   \\
                \quad Counterfact. Edits & 0.000 & 0.000 & 0.000   \\
                \midrule

        \multicolumn{4}{l}{\textbf{CoT}} \\
        \quad Early Answering & 0.484 & 0.356 & 0.478   \\
                \quad Filler Tokens & 0.459 & \textbf{\underline{0.715}} & 0.487   \\
                \quad Adding Mistakes & 0.468 & \underline{0.553} & 0.471   \\
                \quad Paraphrasing & 0.478 & \underline{0.683} & 0.487   \\
                \quad CC-SHAP & \textbf{0.493} & 0.246 & \textbf{\underline{0.580}}   \\
                \bottomrule
        \end{tabular}
        }
        \caption{The diagnosticity scores of each metric across three tasks using \texttt{qwen2.5-7b} as model and \textbf{MEMIT as knowledge editing method}. Bold numbers indicate the highest scores on each task across the two categories of faithfulness
metrics: post-hoc and CoT. Underlined numbers show the diagnosticity scores that are significantly higher than $0.5$ (Binomial, $p < 0.05$).}
        \label{tab:diagnosticity_memit}
    \end{table*}



\begin{table*}[t!]
        \centering
        \resizebox{0.8\linewidth}{!}{
        \begin{tabular}{lcccc}
        \toprule
        \textbf{Metric} & \textbf{FactCheck} & \textbf{Analogy} & \textbf{Object Counting} & \textbf{Multi-Hop} \\
        \midrule

        \multicolumn{4}{l}{\textbf{Post-hoc}} \\
        \quad CC-SHAP & \textbf{\underline{0.547}} & \textbf{0.233} & \textbf{0.414} & \textbf{\underline{0.575}}   \\
                \quad Simulatability & 0.083 & 0.010 & 0.018 & 0.000   \\
                \quad Counterfact. Edits & 0.001 & 0.000 & 0.000 & 0.000   \\
                \midrule

        \multicolumn{4}{l}{\textbf{CoT}} \\
        \quad Early Answering & 0.504 & \textbf{\underline{0.582}} & 0.504 & 0.515   \\
                \quad Filler Tokens & 0.475 & \underline{0.544} & 0.496 & 0.490   \\
                \quad Adding Mistakes & 0.478 & 0.506 & 0.473 & 0.470   \\
                \quad Paraphrasing & \textbf{\underline{0.565}} & 0.498 & \textbf{\underline{0.581}} & \textbf{0.520}   \\
                \quad CC-SHAP & 0.510 & 0.353 & 0.447 & 0.480   \\
                \bottomrule
        \end{tabular}
        }
        \caption{Diagnosticity scores of each metric across three tasks using \texttt{qwen2.5-7b} as the model and ICE as the knowledge editing method, \textbf{with model-generated explanations}. Bold numbers indicate the highest scores on each task across the two categories of faithfulness
metrics: post-hoc and CoT. Underlined numbers show the diagnosticity scores that are significantly higher than $0.5$ (Binomial, $p < 0.05$).}
        \label{tab:diagnosticity_model_generated}
    \end{table*}


\begin{table*}[t!]
        \centering
        \resizebox{\linewidth}{!}{
        \begin{tabular}{lcccc}
        \toprule
        \textbf{Metric} & \textbf{FactCheck} & \textbf{Analogy} & \textbf{Object Counting} & \textbf{Multi-Hop} \\
        \midrule
    \multicolumn{5}{c}{\textbf{7B}} \\ 
 \midrule \\ 
    Simulatability & 0.018 & 0.004 & 0.011 & 0.000   \\
                    Filler Tokens & 0.299 & 0.229 & 0.330 & 0.520   \\
                    Adding Mistakes & 0.243 & 0.504 & 0.481 & 0.435   \\
                    Paraphrasing & 0.509 & \underline{0.711} & \underline{0.741} & 0.525   \\
                
 \midrule 
\multicolumn{5}{c}{\textbf{32B}} \\ 
 \midrule \\ 
    Simulatability & 0.017 & 0.009 & 0.011 & 0.000   \\
                    Filler Tokens & 0.327  \textcolor{forestgreen}{(+0.03)}& 0.336  \textcolor{forestgreen}{(+0.11)}& 0.195  \textcolor{red}{(--0.14)}& 0.475  \textcolor{red}{(--0.05)}  \\
                    Adding Mistakes & 0.465  \textcolor{forestgreen}{(+0.22)}& 0.310  \textcolor{red}{(--0.19)}& 0.418  \textcolor{red}{(--0.06)}& 0.445  \textcolor{forestgreen}{(+0.01)}  \\
                    Paraphrasing & 0.522  \textcolor{forestgreen}{(+0.01)}& \underline{0.582}  \textcolor{red}{(--0.13)}& \underline{0.847}  \textcolor{forestgreen}{(+0.11)}& 0.515  \textcolor{red}{(--0.01)}  \\
                
 \midrule 
\multicolumn{5}{c}{\textbf{72B}} \\ 
 \midrule \\ 
    Simulatability & 0.020 & 0.009 & 0.010 & 0.000   \\
                    Filler Tokens & 0.442  \textcolor{forestgreen}{(+0.14)}& 0.320  \textcolor{forestgreen}{(+0.09)}& 0.084  \textcolor{red}{(--0.25)}& 0.475  \textcolor{red}{(--0.05)}  \\
                    Adding Mistakes & 0.310  \textcolor{forestgreen}{(+0.07)}& \underline{0.557}  \textcolor{forestgreen}{(+0.05)}& 0.473  \textcolor{red}{(--0.01)}& 0.505  \textcolor{forestgreen}{(+0.07)}  \\
                    Paraphrasing & \underline{0.754}  \textcolor{forestgreen}{(+0.24)}& 0.396  \textcolor{red}{(--0.31)}& \underline{0.936}  \textcolor{forestgreen}{(+0.20)}& \underline{0.615}  \textcolor{forestgreen}{(+0.09)}  \\
                
 \midrule 
    \bottomrule
        \end{tabular}
        }
        \caption{The change in diagnosticity scores across with respect to model size across four tasks.}
        \label{tab:scaling_diagnosticity}
    \end{table*}
One limitation of this study is that it only evaluated LLaVA as the target Vision Language Model (VLM), which may limit the generalizability of the findings to other models. Additionally, the alignment of visual attention heatmaps for non-existing objects was not assessed, indicating that further analysis is needed in this area. 

Moreover, the experiments were conducted solely using the MSCOCO dataset, and future work should expand the evaluation to include additional datasets to ensure the robustness and broader applicability of the results. Furthermore, since datasets that contain both questions and corresponding answers alongside matching segmentation data, which can be used to evaluate object hallucination, are scarce, it may be necessary to develop such datasets.


\end{document}
