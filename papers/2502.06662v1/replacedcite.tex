\section{Background and Related Work}
\label{sec:background}

\paragraph{Dependency Management in npm}

\begin{figure}[b]
    \scriptsize
    \centering
\begin{lstlisting}[style=json,numbers=none]
{ "name": "postcss-loader", "version": "7.3.3",
  "dependencies": {
    "webpack": ">=5.0.0"     (*\color{red}$\leftarrow$~\textsf{\textbf{floating - major}}*)
    "cosmiconfig": "^8.3.5", (*\color{red}$\leftarrow$~\textsf{\textbf{floating - minor: $\ge$8.3.5, $<$9.0.0}}*)
    "jiti": "~1.21.0",       (*\color{red}$\leftarrow$~\textsf{\textbf{floating - patch: $\ge$1.21.0, $<$1.22.0}}*)
    "semver": "7.5.4"        (*\color{red}$\leftarrow$~\textsf{\textbf{pinned}}*)
    "postcss": "^7.0.0 || ^8.0.1", (*\color{red}$\leftarrow$~\textsf{\textbf{other}}*)
    "lodash": "git+ssh://git@github.com:lodash/lodash.git#v4.17", (*\color{red}$\leftarrow$~\textsf{\textbf{other}}*)
  }, "devDependencies": {...}, ... }
\end{lstlisting}
\caption{
An example \Code{package.json} file. 
We modified version constraints in the original file to illustrate different types of version constraints defined in Section~\ref{sec:method-rq1}. Other details irrelevant to this paper are omitted.}
\label{fig:example-pkgjson}
\end{figure}

npm is the most widely used dependency manager for  JavaScript and TypeScript____.
It consists of a package registry that hosts open-source components (called \emph{packages} in npm) and a command-line tool that helps projects reuse packages from the package registry.  
As of September 2024, the npm registry is the largest package registry, hosting more than 3 million packages and 25 million releases____.
The npm command-line tool reads direct dependencies from a configuration file named \Code{package.json}.
%resolves all direct and indirect dependencies and their specific versions as a \emph{dependency graph}, and installs all the packages in the dependency graph to the local machine.
In the \Code{package.json} file (see Figure~\ref{fig:example-pkgjson} for an example), developers specify package names as the \emph{dependencies} of the project with a \emph{version constraint} for each dependency.
In addition, npm distinguishes different kinds of dependencies---usually, developers focus more on \emph{production dependencies}____, i.e., dependencies that will be included in the production build (for applications) or passed downstream (for packages), and are less concerned about security vulnerabilities if the dependencies are not used in production____.

Given a \Code{package.json} file, npm will resolve a \textit{dependency graph} including not only dependencies specified in the \Code{package.json} file, but also all transitive dependencies required by dependencies.
The npm ecosystem is known as particularly fragmented with many tiny packages and high levels of reuse____; a project commonly has orders of magnitude more transitive dependencies than direct ones____.
Dependency resolution can be complicated when multiple dependencies depend on the same third dependency but with conflicting version constraints; this is known as the \textit{diamond problem} or \textit{dependency hell} (see Figure~\ref{fig:rq2-example}). 
While it is highly problematic in other ecosystems____, npm can install multiple distinct versions for the same package, but it will generally try to produce a dependency graph with fewer duplicate versions (i.e., deduplication____). 
Optimal deduplication is NP-Complete____, so npm uses a variety of heuristics to search for a ``good'' dependency graph satisfying all the version constraints while having fewer duplicate versions____. 
%Attempts to resolve optimal dependency graphs using an SMT solver are ongoing____. 

Importantly, \emph{dependency graphs are volatile}. 
Since npm always installs the latest version inside a floating versioning constraint, the same \Code{package.json} may result in different graphs as new package updates are released____.
With floating, projects can keep dependencies up-to-date automatically, whereas developers often do not update pinned dependencies timely, even for those with known security vulnerabilities____.

A key challenge to floating dependencies is \textit{breaking changes}: When an update changes expected behavior, the application can break without any action by its developer. Developers then need to react and either adapt their own code or downgrade the dependency____.
To control breaking changes, npm advocates for \emph{semantic versioning}____: A release should increment its major version number (e.g., from \Code{1.2.1} to \Code{2.0.0}) if it contains breaking changes, minor version number (e.g., from \Code{1.2.1} to \Code{1.3.0}) if it contains new features, and patch version number (e.g., from \Code{1.2.1} to \Code{1.2.2}) if it contains only bug fixes. If packages follow those conventions, automatic minor and patch updates should be safe. 
To support this practice, npm provides specific patterns to declare version ranges accepting minor (e.g., \Code{\^{}8.3.5}) and patch updates (e.g., \Code{\~{}1.21.0}); it also adopts floating versions for minor updates by default when a developer adds a dependency with  \Code{npm install}.
Although previous research has shown that 
%However, one important limitation of semantic versioning is that package developers need to decide whether their release contains breaking changes, which is a challenging open problem____.
%also covered by Hyrum's Law: \textit{``with a sufficient number of users of an API, it does not matter what you promise in the contract: all observable behaviors of your system will be depended on by somebody''}____.
packages may fail to obey semantic versioning____ and developers may turn to pinning instead if they have lost trust in its reliability____,
 semantic-versioning-based floating is still widely accepted and used by npm developers____.
In fact, without discussion of supply chain attacks, the survey by____ shows that npm developers generally consider pinned dependencies as a ``smell,'' indicating a compromise that will accumulate technical debt and bring negative consequences in the long term.

Finally, it is worth noting that the above discussion applies not only to floating direct dependencies but also to floating transitive dependencies even if the developer pins all their direct dependencies.
Developers usually have no control over how their dependencies float or pin transitive dependencies.
As a workaround, npm (as many other ecosystems) provides functionality to \textit{freeze} and store a project's entire dependency graph with a lock file (\Code{package-lock.json}____). % storing the entire resolved dependency graph. 
A lock file ensures the reproducibility of npm builds, but
it has two limitations: (1)~it is restricted to applications (the root of the dependency graph) whereas components used as dependencies cannot pass their lock decisions to downstream applications; it is hence not very useful for packages; (2)~it is not designed (hence very difficult) to be manually maintained, so floating version constraints in \Code{package.json} will still cause automatic updates every time a lock file is generated.

\paragraph{Software Supply Chain Security}

The term \textit{software supply chain} is an umbrella term referring to all the software components, processes, and tools that a software project relies on____.
For a project using npm, all packages in its dependency graph are part of its supply chain.
Most software projects nowadays are built upon a collaboratively developed open-source software supply chain____; the 3~million packages in the npm package registry are a prominent example.

Software supply chains are fragile.
Such fragility comes from the collaborative and distributed nature of open-source development: Projects are often built with a large number of open-source components, each of them maintained by different groups of developers____.
Meanwhile, popular components are directly or transitively used by a large number of downstream applications.
Consequently, any single weakness in the software supply chain could have a huge impact on the entire ecosystem____, creating a large attack vector for malicious actors and making the software industry increasingly concerned with software supply chain security____.

In recent years, a significant amount of practitioner attention has been drawn to the rising threat of \emph{software supply chain attacks}____. 
Although the attackers may attempt to trick malware installation in other ways (e.g., typosquatting____), the most concerning incidents of software supply chain attacks are \emph{malicious package updates}, in which malicious actors \textit{intentionally} compromise an open-source component to inject malware into its (often many) downstream dependents.
While not every security vulnerability is exploitable in practice (leading developers to treat them as false positives)____, malicious package updates can immediately harm the infected downstream dependents____. 
They may also be well obfuscated and often very difficult to detect____.
Their high catastrophic potential and lack of controllability amplify the signal of real incidents and elevate public risk perception (e.g., in a way similar to nuclear hazards____).
As a result, malicious package updates are considered one of the primary threats in today's threat landscape____.
 
As our previous discussion of the pinning versus floating debate illustrates, many dependency management practices have security implications and involve trade-offs between conflicting priorities____.
Previous studies have investigated various practices, such as dependency adoption, updates, abandonment, migration, and the use of dependency management tools~\cite[e.g.,][]{DBLP:journals/ese/KulaGOII18, DBLP:conf/sigsoft/VargasATBG20, DBLP:conf/sigsoft/HeHGZ21, DBLP:conf/sigsoft/MillerKV23, DBLP:journals/tse/HeHZZ23}.
%and reproducible builds____. 
It is worth noting, however, that most prior research in the software engineering community is centered around the mitigation of security vulnerabilities.
%less attention has been paid to \textit{practices that may defend against the rising threat of software supply chain attacks.}
%Researchers have also conducted qualitative studies to inquire practitioner opinions on certain practices____.
%In support of floating, researchers have revealed how floating may help, and how pinning may block, the propagation of security fixes in packages____. 
%On the other hand, security practitioners argue that pinning is necessary for applications to defend against supply chain attacks, and security vulnerabilities should be handled by manual updates with careful reviews.
On the other hand, the security community has recently consolidated best practices to defend against supply chain attacks, such as the OpenSSF Scorecard project____ and the survey by ____ introduced in Section~\ref{sec:introduction}.
However, these (supposed) security best practices may interact with existing \emph{software engineering} trade-offs, among which the pinning versus floating problem is particularly complicated (Table~\ref{tab:trade-offs});  the strongly held convictions from both sides lead to a considerable amount of controversy among practitioners____.

\paragraph{Novelty of Our Research.} 
There have been many prior studies on the use of version constraints____ and the trade-offs  around semantic versioning with regards to maintenance effort, breaking changes, and security vulnerabilities____.
However, the new threat of \textit{malicious package updates} complicates the previously discussed trade-offs, as \emph{malicious actors intentionally break semantic versioning}, often disguising their attacks as patch releases for maximum propagation____.
The changed threat landscape and the complexity of this pinning versus floating problem eventually lead to controversies and divergence in practitioner opinions____.
Prior studies are either collecting these opinions____ or aggregating descriptive statistics from software repositories____; we are not aware of any prior study that attempted to quantify the impact of version constraints using statistical modeling methods.
To shed light on resolving the controversy, we believe it is vital to conduct data-driven research and quantitatively measure the actual impact of version constraints while taking all the previously discussed trade-offs into consideration (Table~\ref{tab:trade-offs}).
This forms the main motivation of our study.