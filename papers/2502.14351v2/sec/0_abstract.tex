\begin{abstract}
Positron Emission Tomography (PET) imaging plays a crucial role in modern medical diagnostics by revealing the metabolic processes within a patient's body, which is essential for quantification of therapy response and monitoring treatment progress. However, the segmentation of PET images presents unique challenges due to their lower contrast and less distinct boundaries compared to other structural medical modalities. Recent developments in segmentation foundation models have shown superior versatility across diverse natural image segmentation tasks. Despite the efforts of medical adaptations, these works primarily focus on structural medical images with detailed physiological structural information and exhibit poor generalization ability when adapted to molecular PET imaging.
In this paper, we collect and construct PETS-5k, the largest PET segmentation dataset to date, comprising 5,731 three-dimensional whole-body PET images and encompassing over 1.3M 2D images. Based on the established dataset, we develop SegAnyPET, a modality-specific 3D foundation model for universal promptable segmentation from PET images. To issue the challenge of discrepant annotation quality of PET images, we adopt a cross prompting confident learning (CPCL) strategy with an uncertainty-guided self-rectification process to robustly learn segmentation from high-quality labeled data and low-quality noisy labeled data. Experimental results demonstrate that SegAnyPET can correctly segment seen and unseen targets using only one or a few prompt points, outperforming state-of-the-art foundation models and task-specific fully supervised models with higher accuracy and strong generalization ability for universal segmentation. As the first foundation model for PET images, we believe that SegAnyPET will advance the applications to various downstream tasks for molecular imaging.\footnote{The model and code will be publicly available at \href{https://github.com/YichiZhang98/SegAnyPET}{the project page}.}
\end{abstract}