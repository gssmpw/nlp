% This must be in the first 5 lines to tell arXiv to use pdfLaTeX, which is strongly recommended.
\pdfoutput=1
% In particular, the hyperref package requires pdfLaTeX in order to break URLs across lines.

\documentclass[11pt]{article}

% Change "review" to "final" to generate the final (sometimes called camera-ready) version.
% Change to "preprint" to generate a non-anonymous version with page numbers.
\usepackage[preprint]{acl}

% Standard package includes
\usepackage{times}
\usepackage{latexsym}

% For proper rendering and hyphenation of words containing Latin characters (including in bib files)
\usepackage[T1]{fontenc}
% For Vietnamese characters
% \usepackage[T5]{fontenc}
% See https://www.latex-project.org/help/documentation/encguide.pdf for other character sets

% This assumes your files are encoded as UTF8
\usepackage[utf8]{inputenc}

% This is not strictly necessary, and may be commented out,
% but it will improve the layout of the manuscript,
% and will typically save some space.
\usepackage{microtype}

% This is also not strictly necessary, and may be commented out.
% However, it will improve the aesthetics of text in
% the typewriter font.
\usepackage{inconsolata}

%Including images in your LaTeX document requires adding
%additional package(s)
\usepackage{graphicx}

\usepackage[utf8]{inputenc} % allow utf-8 input
\usepackage[T1]{fontenc}    % use 8-bit T1 fonts
\usepackage{url}            % simple URL typesetting
%\usepackage[colorlinks=true]{hyperref} % hyperlinks
\usepackage{booktabs}       % professional-quality tables
\usepackage{amsfonts}       % blackboard math symbols
\usepackage{nicefrac}       % compact symbols for 1/2, etc.
\usepackage{microtype}      % microtypography
\usepackage{xcolor}         % colors
\usepackage{multirow}
\usepackage{graphicx}


\newcommand{\fref}[1]{Figure~\ref{#1}}
\newcommand{\tref}[1]{Table~\ref{#1}}
\newcommand{\sref}[1]{\S\ref{#1}}
\newcommand{\jw}[1]{\textcolor{orange}{\bf\small [#1 --Junjie]}}

\usepackage{microtype}
\usepackage{graphicx}
\usepackage{amsmath}
%\usepackage{hyperref}
\usepackage{bbm}
\usepackage{tcolorbox}
\newcommand\BibTeX{B\textsc{ib}\TeX}

\newtheorem{definition}{Definition}
\usepackage{enumitem}
\usepackage{CJKutf8}

\usepackage[]{todonotes}
\newcommand{\fixme}[2][]{{\todo[color=yellow,size=\scriptsize,fancyline,caption={},#1]{#2}}}
\newcommand{\note}[4][]{{\todo[author=#2,color=#3,size=\scriptsize,fancyline,caption={},#1]{#4}}}
\newcommand{\mo}[2][]{{\note[#1]{MO}{blue!20}{#2}}}
\newcommand{\Mo}[2][]{\mo[inline,#1]{#2}\noindent}

% If the title and author information does not fit in the area allocated, uncomment the following
%
%\setlength\titlebox{<dim>}
%
% and set <dim> to something 5cm or larger.

%\title{Evaluate LLM Inductive Reasoning}
\title{On the Fluid Intelligence Evaluation for LLMs}
\title{Understanding LLMs' Fluid Intelligence Deficiency: \\ An Analysis of the ARC Task}

% Author information can be set in various styles:
% For several authors from the same institution:
% \author{Author 1 \and ... \and Author n \\
%         Address line \\ ... \\ Address line}
% if the names do not fit well on one line use
%         Author 1 \\ {\bf Author 2} \\ ... \\ {\bf Author n} \\
% For authors from different institutions:
% \author{Author 1 \\ Address line \\  ... \\ Address line
%         \And  ... \And
%         Author n \\ Address line \\ ... \\ Address line}
% To start a separate ``row'' of authors use \AND, as in
% \author{Author 1 \\ Address line \\  ... \\ Address line
%         \AND
%         Author 2 \\ Address line \\ ... \\ Address line \And
%         Author 3 \\ Address line \\ ... \\ Address line}

\newcommand{\authorsep}{\quad}
\newcommand{\footnotemarksep}{\enspace\space\!\!}

\author{
Junjie Wu$^1$\authorsep
Mo Yu$^2$\thanks{Co-corresponding authors.}\authorsep
Lemao Liu$^2$\authorsep
Dit-Yan Yeung$^1$\footnotemark[1]\authorsep
Jie Zhou$^2$\authorsep
\\
\textsuperscript{1}Hong Kong University of Science and Technology\\
\textsuperscript{2}WeChat AI, Tencent\\
\texttt{junjie.wu@connect.ust.hk} \quad \texttt{moyumyu@global.tencent.com} \\ \texttt{\{redmondliu, withtomzhou\}@tencent.com} \quad \texttt{dyyeung@ust.hk}
}

%\author{
%  \textbf{First Author\textsuperscript{1}},
%  \textbf{Second Author\textsuperscript{1,2}},
%  \textbf{Third T. Author\textsuperscript{1}},
%  \textbf{Fourth Author\textsuperscript{1}},
%\\
%  \textbf{Fifth Author\textsuperscript{1,2}},
%  \textbf{Sixth Author\textsuperscript{1}},
%  \textbf{Seventh Author\textsuperscript{1}},
%  \textbf{Eighth Author \textsuperscript{1,2,3,4}},
%\\
%  \textbf{Ninth Author\textsuperscript{1}},
%  \textbf{Tenth Author\textsuperscript{1}},
%  \textbf{Eleventh E. Author\textsuperscript{1,2,3,4,5}},
%  \textbf{Twelfth Author\textsuperscript{1}},
%\\
%  \textbf{Thirteenth Author\textsuperscript{3}},
%  \textbf{Fourteenth F. Author\textsuperscript{2,4}},
%  \textbf{Fifteenth Author\textsuperscript{1}},
%  \textbf{Sixteenth Author\textsuperscript{1}},
%\\
%  \textbf{Seventeenth S. Author\textsuperscript{4,5}},
%  \textbf{Eighteenth Author\textsuperscript{3,4}},
%  \textbf{Nineteenth N. Author\textsuperscript{2,5}},
%  \textbf{Twentieth Author\textsuperscript{1}}
%\\
%\\
%  \textsuperscript{1}Affiliation 1,
%  \textsuperscript{2}Affiliation 2,
%  \textsuperscript{3}Affiliation 3,
%  \textsuperscript{4}Affiliation 4,
%  \textsuperscript{5}Affiliation 5
%\\
%  \small{
%    \textbf{Correspondence:} \href{mailto:email@domain}{email@domain}
%  }
%}

\begin{document}
\maketitle
\begin{abstract}
While LLMs have exhibited strong performance on various NLP tasks, it is noteworthy that most of these tasks rely on utilizing the vast amount of knowledge encoded in LLMs' parameters, rather than solving new problems without prior knowledge. In cognitive research, the latter ability is referred to as fluid intelligence, which is considered to be critical for assessing human intelligence. Recent research on fluid intelligence assessments has highlighted significant deficiencies in LLMs' abilities. In this paper, we analyze the challenges LLMs face in demonstrating fluid intelligence through controlled experiments, using the most representative ARC task as an example.
Our study revealed three major limitations in existing LLMs: limited ability for skill composition, unfamiliarity with abstract input formats, and the intrinsic deficiency of left-to-right decoding.
Our data and code can be found in \url{https://wujunjie1998.github.io/araoc-benchmark.github.io/}.

% The ARC task is widely recognized as an assessment of machines' inductive reasoning capabilities, on which all existing approaches, including powerful large language models (LLMs), have failed.
% We aim to analyze the challenge that ARC presents to current LLMs from multiple perspectives through controlled experiments. 
% First, according to task decomposition, we decompose ARC tasks into several atomic operations, which leads to a simplified version of ARC for evaluation. Then we evaluate the composition ability for LLMs from the task composition perspective. %We define inductive reasoning examples that reflect atomic operations over input grids.
% Moreover, we investigate the challenges from the abstract input-format perspective as well as modeling perspective. By thorough empirical analyses, we obtain the following key findings: 1)
% on the simplified ARC tasks, LLMs surprisingly fail in most cases. 2) Their composition ability on atomic operations is limited. 3) The abstract input-format is a crucial challenge for LLMs on ARC tasks. 4) The dominant left-to-right paradigm of LLMs have an intrinsic deficiency in achieving advanced inductive reasoning. Our data and code will be publicly released, and the data is also attached in the submission.
% % A comprehensive analysis suggests that the dominant left-to-right paradigm of LLMs may have an intrinsic deficiency in achieving advanced inductive reasoning. First, inductive reasoning is inherently a long-sequence understanding task, as comprehending each example in the input requires mining its connections to others in a back-and-forth manner. Second, a solution to inductive reasoning typically follows a hierarchical structure with planning, resulting in a mismatch with autoregressive decoding, which often emphasizes irrelevant parts (salience) of the inputs.

\end{abstract}

\newcommand{\lemao}[1]{\textcolor{red}{\textbf{#1 --Lemao}}}

\section{Introduction}
\label{sec:introduction}
The business processes of organizations are experiencing ever-increasing complexity due to the large amount of data, high number of users, and high-tech devices involved \cite{martin2021pmopportunitieschallenges, beerepoot2023biggestbpmproblems}. This complexity may cause business processes to deviate from normal control flow due to unforeseen and disruptive anomalies \cite{adams2023proceddsriftdetection}. These control-flow anomalies manifest as unknown, skipped, and wrongly-ordered activities in the traces of event logs monitored from the execution of business processes \cite{ko2023adsystematicreview}. For the sake of clarity, let us consider an illustrative example of such anomalies. Figure \ref{FP_ANOMALIES} shows a so-called event log footprint, which captures the control flow relations of four activities of a hypothetical event log. In particular, this footprint captures the control-flow relations between activities \texttt{a}, \texttt{b}, \texttt{c} and \texttt{d}. These are the causal ($\rightarrow$) relation, concurrent ($\parallel$) relation, and other ($\#$) relations such as exclusivity or non-local dependency \cite{aalst2022pmhandbook}. In addition, on the right are six traces, of which five exhibit skipped, wrongly-ordered and unknown control-flow anomalies. For example, $\langle$\texttt{a b d}$\rangle$ has a skipped activity, which is \texttt{c}. Because of this skipped activity, the control-flow relation \texttt{b}$\,\#\,$\texttt{d} is violated, since \texttt{d} directly follows \texttt{b} in the anomalous trace.
\begin{figure}[!t]
\centering
\includegraphics[width=0.9\columnwidth]{images/FP_ANOMALIES.png}
\caption{An example event log footprint with six traces, of which five exhibit control-flow anomalies.}
\label{FP_ANOMALIES}
\end{figure}

\subsection{Control-flow anomaly detection}
Control-flow anomaly detection techniques aim to characterize the normal control flow from event logs and verify whether these deviations occur in new event logs \cite{ko2023adsystematicreview}. To develop control-flow anomaly detection techniques, \revision{process mining} has seen widespread adoption owing to process discovery and \revision{conformance checking}. On the one hand, process discovery is a set of algorithms that encode control-flow relations as a set of model elements and constraints according to a given modeling formalism \cite{aalst2022pmhandbook}; hereafter, we refer to the Petri net, a widespread modeling formalism. On the other hand, \revision{conformance checking} is an explainable set of algorithms that allows linking any deviations with the reference Petri net and providing the fitness measure, namely a measure of how much the Petri net fits the new event log \cite{aalst2022pmhandbook}. Many control-flow anomaly detection techniques based on \revision{conformance checking} (hereafter, \revision{conformance checking}-based techniques) use the fitness measure to determine whether an event log is anomalous \cite{bezerra2009pmad, bezerra2013adlogspais, myers2018icsadpm, pecchia2020applicationfailuresanalysispm}. 

The scientific literature also includes many \revision{conformance checking}-independent techniques for control-flow anomaly detection that combine specific types of trace encodings with machine/deep learning \cite{ko2023adsystematicreview, tavares2023pmtraceencoding}. Whereas these techniques are very effective, their explainability is challenging due to both the type of trace encoding employed and the machine/deep learning model used \cite{rawal2022trustworthyaiadvances,li2023explainablead}. Hence, in the following, we focus on the shortcomings of \revision{conformance checking}-based techniques to investigate whether it is possible to support the development of competitive control-flow anomaly detection techniques while maintaining the explainable nature of \revision{conformance checking}.
\begin{figure}[!t]
\centering
\includegraphics[width=\columnwidth]{images/HIGH_LEVEL_VIEW.png}
\caption{A high-level view of the proposed framework for combining \revision{process mining}-based feature extraction with dimensionality reduction for control-flow anomaly detection.}
\label{HIGH_LEVEL_VIEW}
\end{figure}

\subsection{Shortcomings of \revision{conformance checking}-based techniques}
Unfortunately, the detection effectiveness of \revision{conformance checking}-based techniques is affected by noisy data and low-quality Petri nets, which may be due to human errors in the modeling process or representational bias of process discovery algorithms \cite{bezerra2013adlogspais, pecchia2020applicationfailuresanalysispm, aalst2016pm}. Specifically, on the one hand, noisy data may introduce infrequent and deceptive control-flow relations that may result in inconsistent fitness measures, whereas, on the other hand, checking event logs against a low-quality Petri net could lead to an unreliable distribution of fitness measures. Nonetheless, such Petri nets can still be used as references to obtain insightful information for \revision{process mining}-based feature extraction, supporting the development of competitive and explainable \revision{conformance checking}-based techniques for control-flow anomaly detection despite the problems above. For example, a few works outline that token-based \revision{conformance checking} can be used for \revision{process mining}-based feature extraction to build tabular data and develop effective \revision{conformance checking}-based techniques for control-flow anomaly detection \cite{singh2022lapmsh, debenedictis2023dtadiiot}. However, to the best of our knowledge, the scientific literature lacks a structured proposal for \revision{process mining}-based feature extraction using the state-of-the-art \revision{conformance checking} variant, namely alignment-based \revision{conformance checking}.

\subsection{Contributions}
We propose a novel \revision{process mining}-based feature extraction approach with alignment-based \revision{conformance checking}. This variant aligns the deviating control flow with a reference Petri net; the resulting alignment can be inspected to extract additional statistics such as the number of times a given activity caused mismatches \cite{aalst2022pmhandbook}. We integrate this approach into a flexible and explainable framework for developing techniques for control-flow anomaly detection. The framework combines \revision{process mining}-based feature extraction and dimensionality reduction to handle high-dimensional feature sets, achieve detection effectiveness, and support explainability. Notably, in addition to our proposed \revision{process mining}-based feature extraction approach, the framework allows employing other approaches, enabling a fair comparison of multiple \revision{conformance checking}-based and \revision{conformance checking}-independent techniques for control-flow anomaly detection. Figure \ref{HIGH_LEVEL_VIEW} shows a high-level view of the framework. Business processes are monitored, and event logs obtained from the database of information systems. Subsequently, \revision{process mining}-based feature extraction is applied to these event logs and tabular data input to dimensionality reduction to identify control-flow anomalies. We apply several \revision{conformance checking}-based and \revision{conformance checking}-independent framework techniques to publicly available datasets, simulated data of a case study from railways, and real-world data of a case study from healthcare. We show that the framework techniques implementing our approach outperform the baseline \revision{conformance checking}-based techniques while maintaining the explainable nature of \revision{conformance checking}.

In summary, the contributions of this paper are as follows.
\begin{itemize}
    \item{
        A novel \revision{process mining}-based feature extraction approach to support the development of competitive and explainable \revision{conformance checking}-based techniques for control-flow anomaly detection.
    }
    \item{
        A flexible and explainable framework for developing techniques for control-flow anomaly detection using \revision{process mining}-based feature extraction and dimensionality reduction.
    }
    \item{
        Application to synthetic and real-world datasets of several \revision{conformance checking}-based and \revision{conformance checking}-independent framework techniques, evaluating their detection effectiveness and explainability.
    }
\end{itemize}

The rest of the paper is organized as follows.
\begin{itemize}
    \item Section \ref{sec:related_work} reviews the existing techniques for control-flow anomaly detection, categorizing them into \revision{conformance checking}-based and \revision{conformance checking}-independent techniques.
    \item Section \ref{sec:abccfe} provides the preliminaries of \revision{process mining} to establish the notation used throughout the paper, and delves into the details of the proposed \revision{process mining}-based feature extraction approach with alignment-based \revision{conformance checking}.
    \item Section \ref{sec:framework} describes the framework for developing \revision{conformance checking}-based and \revision{conformance checking}-independent techniques for control-flow anomaly detection that combine \revision{process mining}-based feature extraction and dimensionality reduction.
    \item Section \ref{sec:evaluation} presents the experiments conducted with multiple framework and baseline techniques using data from publicly available datasets and case studies.
    \item Section \ref{sec:conclusions} draws the conclusions and presents future work.
\end{itemize}
%\section{Introducing atomic operations}
\iffalse
\begin{figure}
    \centering
    \includegraphics[width=0.45\textwidth]{figures/figure2.pdf}
    \caption{An example of the input format conversion. We follow~\citet{wang2023hypothesis} to convert 2D input grids into matrices (represented by NumPy arrays), where each pixel is transformed into a number denoting a specific color (e.g., ``8'' denotes blue and ``2'' denotes red).\lemao{Remove this Figure.}} 
    \label{fig:example matrix input}
\end{figure}
\fi



\section{Evaluating Fluid Intelligence on ARC}
\label{evaluate llm on arc}

\subsection{ARC Benchmark}
\label{sec:arc setting}
%\paragraph{ARC Benchmark.}
We start by evaluating the fluid intelligences of existing LLMs using the ARC benchmark, which comprises 400 training and 400 evaluation tasks. As shown in~\tref{tab:inductive reasoning examples}, each ARC task includes several 2D input-output grid pairs that define a unique transformation rule, with each grid ranging from $1 \times 1$ to $30 \times 30$ pixels, and each pixel being one of ten colors (see~\fref{fig:original prompt} for the names of the ten colors). An LLM must induct the transformation rule from the given input-output grid pairs and use it to predict the output grid for a testing input grid. Due to the high cost of closed-source LLMs, we follow~\citet{wang2023hypothesis} and use a subset of 100 training tasks in ARC for evaluation~\footnote{\scriptsize{Additionally, we evaluated GPT-4o on all 400 training tasks, where it achieved an Acc score of 18.50. This result aligns with the score reported in~\tref{tab:arc performance}, further supporting the representativeness of the subset.
}}.  



\begin{table}[tb]
\small
\centering
\setlength{\tabcolsep}{4mm}
\begin{tabular}{lcc}
\toprule
\textbf{LLM} & \textbf{Acc}$\uparrow$ 
%& \textbf{$\text{P-Acc}_{\text{A}}$}$\uparrow$ & \textbf{$\text{P-Acc}_{\text{M}}$}$\uparrow$ 
& \textbf{Not M\%}$\downarrow$ \\
\midrule
Mistral & 2.00 
%& 32.59 & 62.67 
& 48.00 \\
Llama-3 & 5.00 
%& 49.56 & 73.98 
& 33.00 \\
\midrule
$\text{Mistral-FT}_{\text{ARC}}$ & 3.00 
%&44.74 &67.79 
&34.00 \\
$\text{Llama-3-FT}_{\text{ARC}}$ & 9.00
%& 54.20& 76.34
& 29.00\\
\midrule
$\text{Mistral-8*7B}$ &3.00
%&44.74 &67.79 
& 27.00\\
$\text{Llama-3-70B}$ &9.00
%& 54.20& 76.34
& 24.00\\
\midrule
GPT-3.5 & 6.00 
%& 46.38 & 71.35
& 35.00 \\
%GPT-4 & \textbf{17.00} 
%& \textbf{69.52} & \textbf{82.76} 
%& \textbf{16.00} \\
GPT-4o & \textbf{19.00} &\textbf{11.00} \\
\midrule
GPT-o1* & 18.00 & 10.00 \\
\bottomrule
\end{tabular}
\caption{Evaluation results on the 100 ARC tasks, where Acc %$\text{P-Acc}_{\text{A}}$, and $\text{P-Acc}_{\text{M}}$ 
is represented as percentages. $\text{FT}_{\text{ARC}}$ denotes fine-tuning on ARC tasks. The best results in each column are \textbf{boldfaced}. *GPT-o1 is evaluated on a partial subset, where GPT-4o obtains \emph{16.00} and \emph{10.00} for both scores.
}
\vspace{-0.2in}
\label{tab:arc performance}
\end{table}

\begin{table*}[tb]
  \renewcommand\arraystretch{1.1}
  \centering
  \setlength{\tabcolsep}{2mm}
  \small
  \begin{tabular}{p{2cm}p{7cm}p{5cm}}
    \toprule[1pt]
   
     \textbf{Name} & \textbf{Description/Transformation Rule} & \textbf{Example} \\
     \midrule[0.5pt]
    \textbf{Move} & Move a subgrid in the input grid for several steps towards a single direction in one of \{Up, Down, Left, Right, Up-left, Up-right, Down-left, Down-right\} to form the output grid. Note that the moved subgrid could not surpass the boundary of the input grid. & \begin{center}\vspace{-1mm}\includegraphics[width=5cm]{figures/move.pdf}\vspace{-3mm}\end{center} \\
    \midrule[0.5pt]
   \textbf{Change Color} & Change the color of a subgrid in the input grid to another color other than black to form the output grid. & \vspace{-3.5mm} \begin{center}\includegraphics[width=5cm]{figures/change_color.pdf}\end{center}\vspace{-5mm} \\
   \midrule[0.5pt]
   \textbf{Copy} & Copy a subgrid in the input grid and move it with Move to form the output grid, while making sure that the copied subgrid could neither surpass the boundary of the input grid, nor overlap with the original subgrid. & \vspace{-3.5mm} \begin{center}\includegraphics[width=5cm]{figures/copy.pdf}\end{center} \vspace{-5.5mm} \\
   \midrule[0.5pt]
   \textbf{Mirror} & Mirror the input grid towards a single direction in one of \{Up, Down, Left, Right\} to form the output grid. & \vspace{-4mm} \begin{center}\includegraphics[width=5cm]{figures/mirror.pdf}\end{center} \vspace{-5.5mm}\\
   \midrule[0.5pt]
   \textbf{Fill Internal} & The input grid has a closed subgrid whose internal is black. Fill the internal black part of this subgrid with another color to form the output grid. &\vspace{-3mm} \begin{center}\includegraphics[width=5cm]{figures/fill_internal.pdf}\end{center} \vspace{-6.5mm} \\
   \midrule[0.5pt]
   \textbf{Scale} & Some pixels in the input grid are colored with a specific color. Let the number of rows and columns of the input grid be \(a\) and \(b\), respectively. First, the input grid will be copied \(a \times b\) times. These copies will then be arranged in an output grid with \(a \times a\) rows and \(b \times b\) columns, placed from top to bottom and left to right. Finally, if the position \((i, j)\) in the input grid is black, the \(i \times j\)-th copy in the output grid will be converted to black. & \vspace{-4.5mm} \begin{center}\includegraphics[width=5cm]{figures/scale.pdf}\end{center} \vspace{-6mm} \\
    \bottomrule[1pt]
  \end{tabular}
  \caption{Descriptions and examples of the six atomic operations we use in this paper.}
  \vspace{-0.1in}
  \label{tab:atom operations}
\end{table*}

\begin{table*}[tb]
\renewcommand\arraystretch{1.1}
\centering
\setlength{\tabcolsep}{0.8mm}
\small
\begin{tabular}{lcc|cc|cc|cc|cc|cc}
\toprule[1pt]
\multirow{2}*{LLM} & \multicolumn{2}{c}{\textbf{Move}} & \multicolumn{2}{c}{\textbf{Change Color}} & \multicolumn{2}{c}{\textbf{Copy}} & \multicolumn{2}{c}{\textbf{Mirror}} & \multicolumn{2}{c}{\textbf{Fill Internal}} & \multicolumn{2}{c}{\textbf{Scale}} \\
 & Acc$\uparrow$ & Not M\%$\downarrow$ & Acc$\uparrow$ & Not M\%$\downarrow$ & Acc$\uparrow$ & Not M\%$\downarrow$ & Acc$\uparrow$ & Not M\%$\downarrow$ & Acc$\uparrow$ & Not M\%$\downarrow$ & Acc$\uparrow$ & Not M\%$\downarrow$ \\
\midrule[0.5pt]
Mistral & 2.00 & 36.00 & 15.00 & 30.00 & 2.00 & 43.00 & 1.00 & 97.00 & 9.00 & 31.00 & 0.00 & 98.00 \\
Llama-3 & 1.00 & 19.00 & 39.00 & 17.00 & 4.00 & 13.00 & 2.00 & 96.00 & 63.00 & 6.00 & 1.00 & 89.00 \\
\midrule
Mistral-8*7B &2.00&10.00&57.00&5.00&2.00&7.00&5.00&95.00&50.00&3.00&\textbf{3.00}&81.00 \\
Llama-3-70B &8.00&15.00&92.00&1.00&4.00&11.00&7.00&75.00&64.00&3.00&\textbf{3.00}&80.00 \\
\midrule
GPT-3.5 & 4.00 & 27.00 & 48.00 & 13.00 & 4.00 & 29.00 & 6.00 & 89.00 & 58.00 & 12.00 & 1.00 & 80.00 \\
%GPT-4 & \textbf{14.00} & \textbf{3.00} & \textbf{97.00} & \textbf{0.00} & \textbf{13.00} & \textbf{6.00} & \textbf{14.00} & \textbf{52.00} & \textbf{100.00} & \textbf{0.00} & \textbf{3.00} & \textbf{70.00} \\
GPT-4o &\textbf{13.00}&\textbf{0.00}&\textbf{98.00}&\textbf{0.00}&\textbf{15.00}&\textbf{0.00}&\textbf{12.00}&\textbf{48.00}&\textbf{96.00}&\textbf{0.00}&2.00&\textbf{72.00} \\
%\midrule
%$\text{Mistral-FT}_{\text{Atom}}$ & 12.00 & 11.00 & 100.00 & 0.00 & 20.00 & 6.00 & 26.00 & 52.00 & 97.00 & 2.00 & 89.00 & 0.00 \\
%$\text{Llama-3-FT}_{\text{Atom}}$ & 13.00 & 9.00 & 98.00 & 1.00 & 14.00 & 8.00 & 27.00 & 54.00 & 97.00 & 1.00 & 78.00 & 99.08 \\
\bottomrule[1pt]
\end{tabular}
\caption{Results on ARAOC. %$\text{FT}_{\text{Atom}}$ denotes fine-tuning on atomic operation data. 
Acc is shown in percentage. The best results under each column are \textbf{boldfaced}.}
\vspace{-0.2in}
\label{tab:araoc results}
\end{table*}

\subsection{Comparing Text- and Visual-Based LLMs}
Since ARC tasks are presented in a 2D visual grid format, we can employ both visual-based LLMs (\textbf{Visual}) and text-based LLMs through
%Since LLMs cannot process visual inputs directly, 
converting input-output grids into matrices represented by NumPy arrays following existing works~\cite{xullms,wang2023hypothesis} (\textbf{Textual}). Therefore, we first investigate the performances of these two types of LLMs on ARC by prompting GPT-4o with 5 different input-output formats (check Appendix~\ref{appendix:prompts} for the prompts). As shown in~\tref{tab:different format}, prompting GPT-4o solely with textual input-output format yielding the best performance on the 100 ARC tasks. On the other hand, it seems extremely challenging for visual-based LLMs to finish ARC tasks, where we provide detailed analysis in Appendix~\ref{appendix:visual analysis}. \textbf{Based on the results, we apply the textual only input-output format and refer \textit{``LLMs''} to text-based LLMs in the rest of the paper.}
%, as illustrated in~\fref{fig:example matrix input}.

\subsection{Comparing Different LLMs on ARC}
\label{sec:evaluate on original arc}
\paragraph{Evaluated LLMs.}
\label{evaluated llms}
We evaluate both closed-source and open-source LLMs. For closed-source models, we use GPT-4o and GPT-3.5. For open-source LLMs, we select Mistral (\texttt{Mistral-7B-Instruct-v0.2})~\cite{jiang2023mistral} and Llama-3 (\texttt{Llama-3-8B-Instruct})~\cite{llama3}. Additionally, we include the recently released GPT-o1 (\texttt{o1-preview}) model, known for its strong reasoning abilities, for comparison. Due to the slow inference speed and limited quota of GPT-o1, we evaluate it on a subset of 50 tasks and report the performance of both GPT-4o and GPT-o1 on this subset. Check Appendix~\ref{appendix:inference config} for details on the inference configurations.

%For all the models, we maintain their official prompt templates unchanged and the inference configurations are listed in Appendix~\ref{appendix:inference config}.

\paragraph{Evaluation Metrics.}
The primary metric we use to evaluate the performance of LLMs is the accuracy of their predictions (Acc). Additionally, since we observe that the shape of the LLMs' predicted output grids does not always align with the ground truth, we report the percentage of mismatched predictions for each LLM (Not M\%), where lower scores indicate better performance.



\iffalse
\begin{table*}[tb]
\renewcommand\arraystretch{1.1}
\centering
\setlength{\tabcolsep}{0.2mm}
\small
\begin{tabular}{l|cccc|cccc|cccc}
\toprule[1pt]
\multirow{2}*{LLM} & \multicolumn{4}{c}{\textbf{Move}} & \multicolumn{4}{c}{\textbf{Change Color}} & \multicolumn{4}{c}{\textbf{Copy}} \\
 & \textbf{Acc}$\uparrow$ & \textbf{$\text{P-Acc}_{\text{A}}$}$\uparrow$ & \textbf{$\text{P-Acc}_{\text{M}}$}$\uparrow$ & \textbf{Not M\%}$\downarrow$ & \textbf{Acc}$\uparrow$ & \textbf{$\text{P-Acc}_{\text{A}}$}$\uparrow$ & \textbf{$\text{P-Acc}_{\text{M}}$}$\uparrow$ & \textbf{Not M\%}$\downarrow$ & \textbf{Acc}$\uparrow$ & \textbf{$\text{P-Acc}_{\text{A}}$}$\uparrow$ & \textbf{$\text{P-Acc}_{\text{M}}$}$\uparrow$ & \textbf{Not M\%}$\downarrow$ \\
\midrule[0.5pt]
Mistral & 2.00 & 50.23 & 78.48 & 36.00 & 15.00 & 59.33 & 84.76 & 30.00 & 2.00 & 48.09 & 84.37 & 43.00 \\
Llama-3 & 1.00 & 65.91 & 81.36 & 19.00 & 39.00 & 73.85 & 88.98 & 17.00 & 4.00 & 78.06 & \underline{\textbf{89.72}} & 13.00 \\
\midrule
GPT-3.5 & 4.00 & 60.88 & 83.39 & 27.00 & 48.00 & 80.64 & 92.68 & 13.00 & 4.00 & 61.75 & 86.97 & 29.00 \\
GPT4o & \underline{\textbf{14.00}} & \underline{\textbf{85.93}} & \underline{\textbf{88.59}} & \underline{\textbf{3.00}} & \underline{97.00} & \underline{99.67} & \underline{99.67} & \underline{\textbf{0.00}} & \underline{13.00} & \underline{\textbf{84.00}} & 89.36 & \underline{\textbf{6.00}}\\
\midrule
$\text{Mistral-FT}_{\text{Atom}}$&12.00&78.31&87.99&11.00&\textbf{100.00}&\textbf{100.00}&\textbf{100.00}&\textbf{0.00}&\textbf{20.00}&83.96&89.32&\textbf{6.00} \\
$\text{Llama-3-FT}_{\text{Atom}}$ &13.00&79.69&87.57&9.00&98.00&98.99&99.99&1.00&14.00&82.54&89.71&8.00 \\
\midrule[1pt]
\multirow{2}*{LLM} & \multicolumn{4}{c}{\textbf{Mirror}} & \multicolumn{4}{c}{\textbf{Fill Internal}} & \multicolumn{4}{c}{\textbf{Scale}} \\
 & \textbf{Acc}$\uparrow$ & \textbf{$\text{P-Acc}_{\text{A}}$}$\uparrow$ & \textbf{$\text{P-Acc}_{\text{M}}$}$\uparrow$ & \textbf{Not M\%}$\downarrow$ & \textbf{Acc}$\uparrow$ & \textbf{$\text{P-Acc}_{\text{A}}$}$\uparrow$ & \textbf{$\text{P-Acc}_{\text{M}}$}$\uparrow$ & \textbf{Not M\%}$\downarrow$ & \textbf{Acc}$\uparrow$ & \textbf{$\text{P-Acc}_{\text{A}}$}$\uparrow$ & \textbf{$\text{P-Acc}_{\text{M}}$}$\uparrow$ & \textbf{Not M\%}$\downarrow$ \\
\midrule[0.5pt]
Mistral & 1.00 & 2.29 & 76.39 & 97.00 & 9.00 & 58.96 & 85.45 & 31.00 & 0.00 & 1.32 & 66.05 & 98.00 \\
Llama-3 & 2.00 & 2.58 & 64.58 & 96.00 & 63.00 & 89.12 & 94.81 & 6.00 & 1.00 & 8.38 & 76.15 & 89.00 \\
\midrule
GPT-3.5 & 6.00 & 10.48 & \underline{95.30} & 89.00 & 58.00 & 83.44 & 94.82 & 12.00 & 1.00 & 14.85 & 74.26 & 80.00 \\
GPT4o & \underline{14.00} & \underline{42.16} & 87.84 & \underline{\textbf{52.00}} & \underline{\textbf{100.00}} & \underline{\textbf{100.00}} & \underline{\textbf{100.00}} & \underline{\textbf{0.00}} & \underline{3.00} & \underline{24.28} &\underline{80.93}& \underline{70.00} \\
\midrule
$\text{Mistral-FT}_{\text{Atom}}$&26.00&\textbf{44.64}&92.99&\textbf{52.00}&97.00&97.91&99.91&2.00&\textbf{89.00}&\textbf{99.51}&\textbf{99.51}&\textbf{0.00} \\
$\text{Llama-3-FT}_{\text{Atom}}$ &\textbf{27.00}&43.97&\textbf{95.59}&54.00&97.00&98.96&99.96&1.00&78.00&95.11&4.00&99.08 \\
\bottomrule[1pt]
\end{tabular}
\caption{Evaluation results on ARAOC on atomic operation-level. $\text{FT}_{\text{Atom}}$ denotes fine-tuning on atomic operation data. Acc
%$\text{P-Acc}_{\text{A}}$ and $\text{P-Acc}_{\text{M}}$ are 
is shown in percentage. 
The best results under each column are \textbf{boldfaced}, and the best results among not fine-tuned LLMs are \underline{underlined}.
}
\label{tab:araoc results}
\end{table*}
\fi



\paragraph{Results.}
The evaluation results are presented in~\tref{tab:arc performance}. We observe that, although GPT-4o significantly outperforms other LLMs, its performance remains far from ideal. Moreover, GPT-o1 shows almost no improvement over GPT-4o on the evaluated subset. Hence, due to its low speed and limited quota, we do not include GPT-o1 in the following experiments. 

For the other LLMs, handling ARC tasks seems extremely challenging, with more than one-third of their predictions failing to match the shape of the corresponding ground truth. To examine the impact of model size on ARC performance, we further experiment with Mistral-8*7B (\texttt{Mixtral-8x7B-Instruct-v0.1}) and Llama-3-70B (\texttt{Llama-3-70B-Instruct}). As shown in~\tref{tab:arc performance}, larger LLMs consistently outperform smaller ones across all tasks, indicating that models with more parameters exhibit stronger fluid intelligence on ARC tasks. However, their overall performance remains poor. We hypothesize that this poor performance is due to the LLMs' unfamiliarity with the style of these tasks. Consequently, we further fine-tuned Mistral and Llama-3 on a separate ARC evaluation set that do not overlap with the 100 ARC tasks used in~\tref{tab:arc performance} using LoRA~\cite{hu2021lora}, and evaluated them on the 100 ARC tasks (check fine-tuning details in Appendix~\ref{appendix:lora}). However, as shown in~\tref{tab:arc performance}, even though fine-tuning on ARC tasks improves the LLMs' performance, the results remain suboptimal, with Acc scores below 10\%.

 

In summary, these experiments demonstrate the significant challenge LLMs face in successfully completing ARC tasks, motivating us to further investigate the underlying reasons for this difficulty.


%\lemao{Please insert finetuning experiments into this subsection to further highlight the challenge of ARC tasks for LLMs.}




\section{Breaking ARC into Atomic Operations}
\label{sec: atom operation}
%The poor performance of LLMs on ARC tasks motivates us to explore the reasons behind this issue. 
As mentioned in~\sref{intro}, the transformation rule of each ARC task can be decomposed into several atomic operations (e.g., the rule in~\tref{tab:inductive reasoning examples} can be broken into moving the subgrid and changing its color), which motivates us to analyze the challenges of LLMs from a task decomposition perspective. To this end, we first decompose the ARC tasks into simplified tasks and form the ARAOC benchmark that consists of various atomic operations, then use ARAOC to evaluate the fluid intelligence of LLMs.


\subsection{ARAOC Benchmark}

%\lemao{Remove the finetuning results from Table 3 and combine them into Table 6 to show performance gap on ARAOC and ARC, which demonstrates the composition challenge in sec 4.}

\label{sec:araoc benchmark}
To evaluate LLMs' fluid intelligence with atomic operations, we first manually go through all the tasks in ARC's training and evaluation sets, then conclude six atomic operations that can compose the transformation rules for most of the ARC tasks. Check~\tref{tab:atom operations} for atomic operations' descriptions. 

For each atomic operation, we use it as the transformation rule to build 100 tasks with 3 input-output training pairs and 1 testing pair, which follows the standard ARC setting (check Appendix~\ref{appendix:araoc} for the crafting details). This finally leads to a benchmark named \textbf{A}bstraction and \textbf{R}easoning on \textbf{A}tom \textbf{O}peration  \textbf{C}orpus (\textbf{ARAOC}) with 600 distinct tasks. %Specifically, for each task in ARAOC, we have three input-output grid pairs as the few-shot examples, and a single input grid that needs LLMs to infer its corresponding output grid. 
We evaluate all LLMs in~\sref{evaluated llms} on ARAOC and additionally include Mistral-8*7B and Llama-3-70B to study the impact of model size.

%list the results in~\tref{tab:araoc results}.



\iffalse
\begin{table}[tb]
\small
\centering
\setlength{\tabcolsep}{1mm}
\begin{tabular}{llcccc}
\toprule
& \textbf{COMB}& \textbf{Mistral}& \textbf{Llama-3} & \textbf{GPT-3.5}& \textbf{GPT-4o} \\
\midrule[0.5pt]
\multirow{6}{*}{\textbf{Move}}&Up 1 &0.00&12.00&0.00&26.00\\
&Up 2 &2.00&6.00&2.00&12.00\\
&Up 3 &4.00&4.00&0.00&8.00\\
\cmidrule{2-6}
&Up-right 1 &0.00&2.00&0.00&10.00\\
&Up-right 2 &0.00&0.00&0.00&0.00\\
&Up-right 3 &2.00&2.00&0.00&4.00\\
\midrule[0.5pt]
\multirow{6}{*}{\textbf{Copy}}&Up 1 &4.00&16.00&6.00&40.00\\
&Up 2 &8.00&10.00&6.00&26.00\\
&Up 3 &10.00&12.00&8.00&16.00\\
\cmidrule{2-6}
&Up-right 1 &2.00&8.00&0.00&16.00\\
&Up-right 2 &4.00&4.00&4.00&4.00\\
&Up-right 3 &2.00&4.00&0.00&2.00\\
\bottomrule
\end{tabular}
\caption{Further analysis results regarding Move and Copy. \textbf{COMB} is the abbreviation of combination. We only list Acc scores (in percentage) here for simplicity, and other metric scores are listed in Table Y.}
\label{tab:controllable}
\end{table}
\fi



\paragraph{Results.}
\label{sec:araoc results}
As shown in~\tref{tab:araoc results}, GPT-4o largely outperforms other LLMs across almost all tasks in the ARAOC benchmark, achieving nearly 100\% Acc scores on the Change Color and Fill Internal tasks, demonstrating its high fluid intelligence. Additionally, %GPT-3.5 and Llama-3 produce comparable results, 
Llama-3/Llama-3-70B outperforms Mistral/Mistral-8*7B, suggesting that pre-training %on higher-quality data 
with a greater number of parameters can enhance the fluid intelligence of LLMs. Also, similar to~\tref{tab:arc performance}, larger LLMs continue to outperform smaller ones across tasks, further illustrating the above point. However, all LLMs still encounter substantial difficulties with tasks related to Move, Copy, Mirror, and Scale, failing to predict the correct shapes of output grids for the latter two atomic operations on more than \textasciitilde50 tasks.



\subsection{Further Analysis}
%Moreover, the results in~\tref{tab:araoc results} %and~\tref{tab:composition} 
%show that LLMs' Acc scores on Move and Copy tasks in ARAOC are still less than 20\%, even they have been trained on similar data. 
\paragraph{Analysis I: Internal Factors.}
As concluded from~\sref{sec:araoc results}, all the LLMs exhibit poor performances on Move and Copy tasks in ARAOC. To analyze whether this is caused by the internal complexity of Move and Copy, we investigate factors that may affect the complexity of Move and Copy, and their influences on LLMs' performances. Given that Copy can actually be viewed as first copying the original subgrid, then moving the copied subgrid several steps in specific directions, we intuitively consider two factors in this study: 1) the number of steps the subgrid/copied subgrid moves; 2) the direction in which the subgrid/copied subgrid moves.

\paragraph{Setup.}
Specifically, we choose {Up, Up-right} and {1 step, 2 steps, 3 steps} as our candidate moving directions and steps, respectively. We then generate 50 input grids for each atomic operation, ensuring that these grids can be transformed into valid output grids based on any combination of the two candidate sets (e.g., Up for 1 step). For each input grid, we create 6 tasks corresponding to all 6 combinations of the candidate sets, and evaluate the closed-source (GPT-4o) and open-source (Llama-3) LLMs, which performed better in~\tref{tab:araoc results}, as representatives on these tasks.


\paragraph{Results.}
Results are shown in~\tref{tab:controllable}. We observe that 
%Mistral and GPT-3.5 can hardly finish the given tasks, and even scoring 0 under ``Up'', ``Up-right 1'' and ``Up-right 2'', rendering their results not indicative. 
%For GPT-4 and Llama-3, we notice that 
for both Move and Copy, a larger number of steps would lead to lower Acc scores. This could be because as the number of steps increases, LLMs need to focus on a longer context to induce the atomic operation, which leads to more challenges. Additionally, LLMs appear to be more adept with subgrids that move in a straight direction, as their performance on "Up"-related tasks is significantly higher than on "Up-right"-related tasks. Even when considering "Up-right 1" as a two-step move (one step "Up" and one step "Right"), LLMs' Acc scores on "Up-right 1" are still lower than those on "Up 2", further supporting our previous assertion. %Overall, we conclude that the inductive reasoning ability of LLMs can be influenced by various intrinsic factors related to different operations, which control the complexity of such atomic operations. %Future work should pay more attention to these factors when evaluating LLMs' inductive reasoning capabilities with atomic operations.

\begin{table}[tb]
\small
\centering
\setlength{\tabcolsep}{3mm}
\begin{tabular}{llcc}
\toprule
& \textbf{COMB}& \textbf{Llama-3} & \textbf{GPT-4o} \\
\midrule[0.5pt]
\multirow{6}{*}{\textbf{Move}}&Up 1 &12.00&24.00\\
&Up 2 &6.00&26.00\\
&Up 3 &4.00&17.00\\
\cmidrule{2-4}
&Up-right 1 &2.00&9.00\\
&Up-right 2 &0.00&2.00\\
&Up-right 3 &2.00&1.00\\
\midrule[0.5pt]
\multirow{6}{*}{\textbf{Copy}}&Up 1 &16.00&46.00\\
&Up 2 &10.00&38.00\\
&Up 3 &12.00&27.00\\
\cmidrule{2-4}
&Up-right 1 &8.00&11.00\\
&Up-right 2 &4.00&6.00\\
&Up-right 3 &4.00&10.00\\
\bottomrule
\end{tabular}
\caption{Analysis I's Acc scores. \textbf{COMB} refers to combination. See~\tref{tab:controllable_plus} for the Not M\% scores.}
\vspace{-0.2in}
\label{tab:controllable}
\end{table}


\paragraph{Analysis II: Effect on Input Size.}
%Another reason LLMs struggle with ARAOC may relate to the size of input grids, where larger input grids should bring more difficult tasks and vice versa. To test this hypothesis, 
We %evaluate LLMs on Move and Copy tasks for the effect on input size via 
evaluate LLMs on 100 Move and Copy tasks with smaller sizes (crafting details are included in Appendix~\ref{appendix:small size}). 
%and we randomly initialized the 100 tasks for each atomic operation from a range that is half of the original range listed in Appendix~\ref{appendix:araoc}. This results in 100 new tasks for both atomic operations, with an average size of $4.96 \times 4.89$ and $4.81 \times 4.80$, respectively. For comparison, the original average sizes are $10.07 \times 10.16$ and $9.72 \times 9.62$, respectively. 
The evaluation results on these tasks are listed in Table \ref{tab:large matrix}, where LLMs perform significantly better on Move and Copy tasks with smaller input sizes. This indicates that the size of matrix-format input does affect LLMs' understanding of ARAOC tasks and thus influences their performance on ARAOC. %See Appendix X for prompts used in this section.
%\mo{Conclusion: ARC is a natural long sequence understanding task (Combined with Conclusion 2 and 4.3)}

\textbf{Overall, this section shows that the performances of LLMs is largely affected by the superficial properties of the input grids, 
and LLMs fail to grasp the underlying concept of the operations. This result further suggests that LLMs rely more on pattern recognition and memorization, akin to crystallized intelligence, rather than reasoning through abstract, novel relationships (fluid intelligence).}
In the following, we provide further insights into LLMs' deficiencies through the lens of three challenges on ARC and ARAOC: task composition (\sref{sec:factor}), LLMs' understanding of task inputs (\sref{sec:matrix}), and LLMs' modeling strategies (\sref{sec:model}).

% Overall, we conclude that the inductive reasoning capabilities of LLMs are still far from ideal, as they even fail to abstract simple transformation rules, such as atomic operations, from the given examples.
% %, let alone ARC tasks that require composing different atomic operations to form transformation rules. 

% Therefore, in the following sections, we analyze three challenges that cause LLMs to fail on ARC and ARAOC tasks: task composition (\sref{sec:factor}), LLMs' understanding of task inputs (\sref{sec:matrix}), and LLMs' modeling of ARAOC tasks (\sref{sec:model}).



%\paragraph{Input Size.}

%As concluded from Section \ref{sec:araoc results}, all the LLMs exhibit poor performances on ARAOC tasks involving Move and Copy. To analyze whether this is caused by the internal complexity of Move and Copy, we investigate factors that may affect the complexity of these two atomic operations, and their influences on LLMs' performances. Given that Copy can actually be viewed as first copying the original subgrid, then moving the copied subgrid several steps in specific directions, we intuitively consider two factors in this study: 

%1. The number of steps the subgrid/copied subgrid moves.
%2. The direction in which the subgrid/copied subgrid moves.

%Specifically, we select \{Up, Up-right\} and \{1 step, 2 ste%ps, 3 steps\} as our candidate moving directions and steps, respectively. Subsequently, we generate 50 input grids for each atomic operation, ensuring that these input grids can be transformed into valid output grids following any combination of the two candidate sets (e.g., Up for 1 step). For each input grid, we then craft 6 tasks with all the 6 combinations of the two candidate sets, and evaluate all the LLMs on these tasks.




%which further explains LLMs' poor performances on ARC.

%Although these atomic operations are decomposed from ARC tasks and should make ARAOC tasks less challenging, there exists a possibility that tasks in ARAOC are still difficult to answer, which leads to LLMs' poor performances. Therefore, we invite three kindergarten kids to finish ARAOC tasks for comparison. Specifically, we select 100 tasks from Move and Copy (50 for each) that GPT-4 wrongly answers and ask the kids to finish these tasks. To our surprise, these kids obtain an average Acc of xxx and xxx on Move and Copy, indicating that tasks in ARAOC are actually not challenging for kids. Considering that GPT-4 only obtains 14.00 and 13.00 Acc scores on Move and Copy in ARAOC, \textbf{a thorough analysis on why LLMs cannot handle inductive reasoning tasks well is necessary}.



\iffalse
\begin{table}[tb]
\small
\centering
\setlength{\tabcolsep}{0.7mm}
\begin{tabular}{lcccc}
\toprule
\textbf{LLM} & \textbf{Acc}$\uparrow$ & \textbf{$\text{P-Acc}_{\text{A}}$}$\uparrow$ & \textbf{$\text{P-Acc}_{\text{M}}$}$\uparrow$ & \textbf{Not M\%}$\downarrow$ \\
\midrule
%Mistral & 2.00 & 32.59 & 62.67 & 48.00 \\
%Llama-3 & 5.00 & 49.56 & 73.98 & 33.00 \\
%\midrule
$\text{Mistral-FT}_{\text{Atom}}$ &1.00 &42.46 &68.49 &38.00 \\
%$\text{Mistral-FT}_{\text{ARC+Atom}}$ & 6.00&51.31 &71.27 &28.00 \\
%\midrule
$\text{Llama-3-FT}_{\text{Atom}}$ & 2.00&42.77 &71.28 &40.00 \\
%$\text{Llama-3-FT}_{\text{ARC+Atom}}$ &6.00 &51.81 &74.02 &30.00 \\
\bottomrule
\end{tabular}
\caption{Performances of LLMs fine-tuned on atomic operation data on ARAOC and the 100 ARC tasks. %\lemao{Reorg this table to show the performance gap between ARROC and ARC. Merge the finetuning results together.}
}
\label{tab:fine-tune arc performance}
\end{table}
\fi



\section{Challenge on Task Composition}
\label{sec:factor}
In this section, we assess the deficiency of LLM's fluid intelligence from a task composition perspective. %To be specific, we evaluate the task composition ability of LLMs from two aspects. 
First, we consider a simple composition experiment that controllably evaluates the composition for Move and Copy in ARAOC (\sref{sec:simple composition}). Moreover, we design a complex composition experiment utilizing ARC tasks to evaluate LLMs' abilities to compose all atomic operations (\sref{sec:complex composition}).



\subsection{Evaluation on Simple Composition}
\label{sec:simple composition}
%One possible reason why LLMs fine-tuned on atomic operation data still fail on ARC is that their abilities to compose multiple atomic operations are weak. 
We start from evaluating LLMs' compositional ability on a simple composition task. 
To be specific, we compose Move and Copy to create 100 new tasks for evaluation. Since Mistral and Llama-3 are facing severe challenges on inducting these two atomic operations, we fine-tune them on three types of data: 1) 3,000 Move tasks; 2) 3,000 Copy tasks ; 3) 1,500 Move tasks and 1,500 Copy tasks, while making sure that these tasks do not overlap with those in ARAOC. We evaluate these fine-tuned LLMs and the GPT models on the newly crafted Move and Copy tasks and list the results in~\tref{tab:composition}.

%To further investigate the above assumption, 
%we fine-tune Mistral and Llama-3 on 3000 Move and Copy tasks that do not overlap with ARAOC (ensuring the number of fine-tuning examples remains the same as~\tref{tab:fine-tune arc performance}), respectively. Then, we test these fine-tuned LLMs on both the Move and Copy tasks in ARAOC, as well as on 100 additional tasks composed of Move and Copy operations. 

As can be seen, fine-tuning on single atomic operation's data can boost LLMs' performances on corresponding tasks, while fine-tuning on both atomic operations can achieve enhancement on both tasks. However, all the fine-tuned LLMs as well as GPT models face severe challenges when dealing with the composition tasks, which is not a complex composition, indicating that the composition abilities of LLMs are limited.



%Nevertheless, fine-tuning on both atomic operations' data could lead to better performances on the compositional task, raising up a straightforward question: \textit{could fine-tune with all the atomic operations help LLMs handle complex compositions in general ARC tasks?}

%This experimental result demonstrates our previous assumption that LLMs lack the capability to compose atomic operations, thus leads to their poor performances on ARC.
%.\mo{Conclusion1: Failure of transfer and composition of SFT paradigm}


%\lemao{This section is too short, you should consider how to add some new experiments or some contents? For example, you can use gpt4 for experiments without finetuning on both table 4 \& 5. In addition, you can add composition for other operations such as Change Color and Fill without finetuning?}



\subsection{Evaluation on Complex Composition}
\label{sec:complex composition}
%\lemao{Reorganize this subsection. Note that ARC can be considered as the compositions of different atomic operations.}

%Given that LLMs still perform poorly on ARAOC, 
%Similar to Section \ref{sec:evaluate on original arc}, we fine-tune Mistral and Llama-3 on tasks built upon atomic operations to see if this could enhance their performances on ARAOC. 
Furthermore, we examine the LLMs' abilities to compose atomic operations in complex ways. As mentioned in~\sref{sec: atom operation}, the ARC tasks can be decomposed into atomic operations listed in Table \ref{tab:atom operations}. Therefore, we regard ARC tasks as complex compositions of atomic operations for evaluation. Here we evaluate Llama-3 and GPT-4o since they are the better open-sourced and close-sourced LLMs in~\tref{tab:composition}.
In addition, % to the LLMs in~\sref{evaluated llms}, 
we fine-tune Llama-3 on tasks built upon atomic operations (check fine-tune details in Appendix~\ref{appendix:lora}) to see if this leads to improvement on ARC (\textbf{FT-atomic}).
%, using the same strategy and configurations as described in Section \ref{sec:evaluate on original arc}. Specifically, we generated an additional 500 tasks for each atomic operation beyond the 100 tasks in ARAOC, resulting in a total of 3000 tasks for fine-tuning. %We fine-tuned Mistral and Llama-3 on these data using the same strategy and configurations as described in Section \ref{sec:evaluate on original arc} (FT-atomic). 
In addition, we apply three more strategies to fine-tune Llama-3 for comparison: 1) using both the aforementioned operation data and 400 ARC tasks that do not overlap with the 100 evaluation tasks (\textbf{FT-atomic-arc}); 2) using only the 400 ARC tasks (\textbf{FT-arc}).
  %We evaluated these LLMs on ARC, and also listed their results on ARAOC for extra comparison.

Results are show in~\tref{tab:fine-tune arc performance}. We observe that fine-tuning on atomic operation data largely improves the performance of Llama-3 on ARAOC~\footnote{\scriptsize{We perform an additional experiment in Appendix~\ref{appendix:further fine-tuning} to further support this point.
}}. In particular, both fine-tuned LLMs achieve high accuracy on Color, Fill Internal, and Scale tasks, which Llama-3 struggles with. However, Llama-3-FT-atomic performs even worse than Llama-3 on ARC tasks. This could be due to the loss of compositional ability after solely fine-tuning on atomic operations, an issue that Llama-3-FT-atomic-arc does not encounter. On the other hand, fine-tuning on ARC tasks enhances LLMs' performance on ARC, but the improvement on ARAOC tasks is relatively limited compared to fine-tuning on ARAOC tasks. This is likely because the transformation rules in ARC are highly complex, and LLMs struggle to decompose these rules into atomic operations. Nonetheless, all LLMs still perform poorly on ARC tasks, which is unsurprising given their difficulties with even the simple compositions presented in~\tref{tab:composition}.
 
 %Overall, the experiment outcome again suggests that even if LLMs have a good understanding of individual atomic operations, they have limited ability to compose these atomic operations, causing them to fail on complex inductive reasoning tasks.

\textbf{Overall, while fine-tuning on atomic operations may assist LLMs in understanding these operations, it does not enable them to infer such operations from in-context examples. This limitation explains LLMs' poor performance on compositional tasks and further highlights their lack of intrinsic mechanisms for abstract reasoning, a core characteristic of fluid intelligence.
}



%the reason LLMs struggle with ARC tasks may be due to their weak ability to compose different atomic operations into the complex transformation rules required for ARC tasks.
 %\mo{Composition is one of the important aspects. The problems left are: (1) the LLMs learned the six operations instead of induction (this is actually the problem of 1DARC design); (2) the six operations do not cover every operation in ARC.}




%\usepackage{fancyhdr}
%\pagestyle{fancy}

\usepackage{amsfonts}
\usepackage{times}
%\usepackage[pdftex]{graphicx}
%\DeclareGraphicsExtensions{.jpg}
%\usepackage[dvips]{graphicx}
%\DeclareGraphicsExtensions{.eps}
\usepackage{latexsym}
\usepackage{amssymb}
\usepackage{amsmath}
%\usepackage{cite}
\usepackage{verbatim}
\newtheorem{theorem}{Theorem}

\newtheorem{acknowledgement}[theorem]{Acknowledgement}
%\newtheorem{algorithm}[theorem]{Algorithm}
\newtheorem{assumption}[theorem]{Assumption}
\newtheorem{axiom}[theorem]{Axiom}
\newtheorem{case}[theorem]{Case}
\newtheorem{claim}[theorem]{Claim}
\newtheorem{conclusion}[theorem]{Conclusion}
\newtheorem{condition}[theorem]{Condition}
\newtheorem{conjecture}[theorem]{Conjecture}
\newtheorem{corollary}[theorem]{Corollary}
\newtheorem{rem}{Remark}
\newtheorem{definition}[theorem]{Definition}
\newtheorem{example}[theorem]{Example}
\newtheorem{exercise}[theorem]{Exercise}
\newtheorem{fact}[theorem]{Fact}
\newtheorem{lemma}[theorem]{Lemma}
%\newtheorem{notation}[theorem]{Notation}
%\newtheorem{problem}[theorem]{Problem}
%\newtheorem{proposition}[theorem]{Proposition}
%\newtheorem{solution}[theorem]{Solution}
%\newtheorem{summary}[theorem]{Summary}
%\newtheorem{remark}[theorem]{Remark}
\newenvironment{proof}{ \textbf{Proof:} }{ \hfill $\Box$}
%\newtheorem{definition}{Definition}
%\newtheorem{prop}{Proposition}
%\newtheorem{lemma}{Lemma}
%\newtheorem{corr}{Corrolary}
%\newtheorem{theorem}{Theorem}
%\newtheorem{conjecture}{Conjecture}

\newcommand{\R}{\mathbb{R}}
\newcommand{\Z}{\mathbb{Z}}
\newcommand{\ra}{\rightarrow} 
\newcommand{\ua}{\uparrow}
\newcommand{\prob}[1]{P\left(#1\right)} 
\newcommand{\imp}{\Rightarrow}
\newcommand{\re}{\mathbb{R}} 
\newcommand{\Exp}[1]{\mathbb{E}\left[#1\right]} %Expectation
\newcommand{\eqdist}{\stackrel{d}{=}}
\newcommand{\std}{\leq_{\mathrm{s.t.}}}
\newcommand{\indicator}[1]{\boldsymbol{1}_{\{#1\}}}
\newcommand{\ceil}[1]{\left\lceil #1 \right\rceil}
\newcommand{\floor}[1]{\left\lfloor #1 \right\rfloor}
%\DeclareMathOperator*{\argmax}{arg\,max}
%\DeclareMathOperator*{\argmin}{arg\,min}
\newcommand{\figref}[1]{{Fig.}~\ref{#1}}
\newcommand{\tabref}[1]{{Table}~\ref{#1}}
\newcommand{\bookemph}[1]{ {\em #1}}
\newcommand{\Ns}{N_s}
\newcommand{\Ut}{U_t}
\newcommand{\fig}[1]{Fig.\ \ref{#1}}
\def\onehalf{\frac{1}{2}}
\def\etal{et.\/ al.\/}
\newcommand{\bydef}{\triangleq}
\newcommand{\tr}{{\it{tr}}}
\def\SNR{{\textsf{SNR}}}
\def\Pe{{P_e}}
\def\SINR{{\mathsf{SINR}}}
\def\SIR{{\mathsf{SIR}}}
\def\MI{{\mathsf{MI}}}
\def\greedy{{\mathsf{GREEDY}}}

% blackboard lowercase
\def\bydef{:=}
\def\bba{{\mathbb{a}}}
\def\bbb{{\mathbb{b}}}
\def\bbc{{\mathbb{c}}}
\def\bbd{{\mathbb{d}}}
\def\bbee{{\mathbb{e}}}
\def\bbff{{\mathbb{f}}}
\def\bbg{{\mathbb{g}}}
\def\bbh{{\mathbb{h}}}
\def\bbi{{\mathbb{i}}}
\def\bbj{{\mathbb{j}}}
\def\bbk{{\mathbb{k}}}
\def\bbl{{\mathbb{l}}}
\def\bbm{{\mathbb{m}}}
\def\bbn{{\mathbb{n}}}
\def\bbo{{\mathbb{o}}}
\def\bbp{{\mathbb{p}}}
\def\bbq{{\mathbb{q}}}
\def\bbr{{\mathbb{r}}}
\def\bbs{{\mathbb{s}}}
\def\bbt{{\mathbb{t}}}
\def\bbu{{\mathbb{u}}}
\def\bbv{{\mathbb{v}}}
\def\bbw{{\mathbb{w}}}
\def\bbx{{\mathbb{x}}}
\def\bby{{\mathbb{y}}}
\def\bbz{{\mathbb{z}}}
\def\bb0{{\mathbb{0}}}

% Bold lowercase
\def\bydef{:=}
\def\ba{{\mathbf{a}}}
\def\bb{{\mathbf{b}}}
\def\bc{{\mathbf{c}}}
\def\bd{{\mathbf{d}}}
\def\bee{{\mathbf{e}}}
\def\bff{{\mathbf{f}}}
\def\bg{{\mathbf{g}}}
\def\bh{{\mathbf{h}}}
\def\bi{{\mathbf{i}}}
\def\bj{{\mathbf{j}}}
\def\bk{{\mathbf{k}}}
\def\bl{{\mathbf{l}}}
\def\bm{{\mathbf{m}}}
\def\bn{{\mathbf{n}}}
\def\bo{{\mathbf{o}}}
\def\bp{{\mathbf{p}}}
\def\bq{{\mathbf{q}}}
\def\br{{\mathbf{r}}}
\def\bs{{\mathbf{s}}}
\def\bt{{\mathbf{t}}}
\def\bu{{\mathbf{u}}}
\def\bv{{\mathbf{v}}}
\def\bw{{\mathbf{w}}}
\def\bx{{\mathbf{x}}}
\def\by{{\mathbf{y}}}
\def\bz{{\mathbf{z}}}
\def\b0{{\mathbf{0}}}
\def\opt{\mathsf{OPT}}
\def\off{\mathsf{OFF}}
% Bold capital letters
\def\bA{{\mathbf{A}}}
\def\bB{{\mathbf{B}}}
\def\bC{{\mathbf{C}}}
\def\bD{{\mathbf{D}}}
\def\bE{{\mathbf{E}}}
\def\bF{{\mathbf{F}}}
\def\bG{{\mathbf{G}}}
\def\bH{{\mathbf{H}}}
\def\bI{{\mathbf{I}}}
\def\bJ{{\mathbf{J}}}
\def\bK{{\mathbf{K}}}
\def\bL{{\mathbf{L}}}
\def\bM{{\mathbf{M}}}
\def\bN{{\mathbf{N}}}
\def\bO{{\mathbf{O}}}
\def\bP{{\mathbf{P}}}
\def\bQ{{\mathbf{Q}}}
\def\bR{{\mathbf{R}}}
\def\bS{{\mathbf{S}}}
\def\bT{{\mathbf{T}}}
\def\bU{{\mathbf{U}}}
\def\bV{{\mathbf{V}}}
\def\bW{{\mathbf{W}}}
\def\bX{{\mathbf{X}}}
\def\bY{{\mathbf{Y}}}
\def\bZ{{\mathbf{Z}}}
\def\b1{{\mathbf{1}}}


% Blackboard capital letters
\def\bbA{{\mathbb{A}}}
\def\bbB{{\mathbb{B}}}
\def\bbC{{\mathbb{C}}}
\def\bbD{{\mathbb{D}}}
\def\bbE{{\mathbb{E}}}
\def\bbF{{\mathbb{F}}}
\def\bbG{{\mathbb{G}}}
\def\bbH{{\mathbb{H}}}
\def\bbI{{\mathbb{I}}}
\def\bbJ{{\mathbb{J}}}
\def\bbK{{\mathbb{K}}}
\def\bbL{{\mathbb{L}}}
\def\bbM{{\mathbb{M}}}
\def\bbN{{\mathbb{N}}}
\def\bbO{{\mathbb{O}}}
\def\bbP{{\mathbb{P}}}
\def\bbQ{{\mathbb{Q}}}
\def\bbR{{\mathbb{R}}}
\def\bbS{{\mathbb{S}}}
\def\bbT{{\mathbb{T}}}
\def\bbU{{\mathbb{U}}}
\def\bbV{{\mathbb{V}}}
\def\bbW{{\mathbb{W}}}
\def\bbX{{\mathbb{X}}}
\def\bbY{{\mathbb{Y}}}
\def\bbZ{{\mathbb{Z}}}

% Caligraphic capital letters
\def\cA{\mathcal{A}}
\def\cB{\mathcal{B}}
\def\cC{\mathcal{C}}
\def\cD{\mathcal{D}}
\def\cE{\mathcal{E}}
\def\cF{\mathcal{F}}
\def\cG{\mathcal{G}}
\def\cH{\mathcal{H}}
\def\cI{\mathcal{I}}
\def\cJ{\mathcal{J}}
\def\cK{\mathcal{K}}
\def\cL{\mathcal{L}}
\def\cM{\mathcal{M}}
\def\cN{\mathcal{N}}
\def\cO{\mathcal{O}}
\def\cP{\mathcal{P}}
\def\cQ{\mathcal{Q}}
\def\cR{\mathcal{R}}
\def\cS{\mathcal{S}}
\def\cT{\mathcal{T}}
\def\cU{\mathcal{U}}
\def\cV{\mathcal{V}}
\def\cW{\mathcal{W}}
\def\cX{\mathcal{X}}
\def\cY{\mathcal{Y}}
\def\cZ{\mathcal{Z}}

% Sans serif capital letters
\def\sfA{\mathsf{A}}
\def\sfB{\mathsf{B}}
\def\sfC{\mathsf{C}}
\def\sfD{\mathsf{D}}
\def\sfE{\mathsf{E}}
\def\sfF{\mathsf{F}}
\def\sfG{\mathsf{G}}
\def\sfH{\mathsf{H}}
\def\sfI{\mathsf{I}}
\def\sfJ{\mathsf{J}}
\def\sfK{\mathsf{K}}
\def\sfL{\mathsf{L}}
\def\sfM{\mathsf{M}}
\def\sfN{\mathsf{N}}
\def\sfO{\mathsf{O}}
\def\sfP{\mathsf{P}}
\def\sfQ{\mathsf{Q}}
\def\sfR{\mathsf{R}}
\def\sfS{\mathsf{S}}
\def\sfT{\mathsf{T}}
\def\sfU{\mathsf{U}}
\def\sfV{\mathsf{V}}
\def\sfW{\mathsf{W}}
\def\sfX{\mathsf{X}}
\def\sfY{\mathsf{Y}}
\def\sfZ{\mathsf{Z}}


% sans serif lowercase
\def\bydef{:=}
\def\sfa{{\mathsf{a}}}
\def\sfb{{\mathsf{b}}}
\def\sfc{{\mathsf{c}}}
\def\sfd{{\mathsf{d}}}
\def\sfee{{\mathsf{e}}}
\def\sfff{{\mathsf{f}}}
\def\sfg{{\mathsf{g}}}
\def\sfh{{\mathsf{h}}}
\def\sfi{{\mathsf{i}}}
\def\sfj{{\mathsf{j}}}
\def\sfk{{\mathsf{k}}}
\def\sfl{{\mathsf{l}}}
\def\sfm{{\mathsf{m}}}
\def\sfn{{\mathsf{n}}}
\def\sfo{{\mathsf{o}}}
\def\sfp{{\mathsf{p}}}
\def\sfq{{\mathsf{q}}}
\def\sfr{{\mathsf{r}}}
\def\sfs{{\mathsf{s}}}
\def\sft{{\mathsf{t}}}
\def\sfu{{\mathsf{u}}}
\def\sfv{{\mathsf{v}}}
\def\sfw{{\mathsf{w}}}
\def\sfx{{\mathsf{x}}}
\def\sfy{{\mathsf{y}}}
\def\sfz{{\mathsf{z}}}
\def\sf0{{\mathsf{0}}}

\def\Nt{{N_t}}
\def\Nr{{N_r}}
\def\Ne{{N_e}}
\def\Ns{{N_s}}
\def\Es{{E_s}}
\def\No{{N_o}}
\def\sinc{\mathrm{sinc}}
\def\dmin{d^2_{\mathrm{min}}}
\def\vec{\mathrm{vec}~}
\def\kron{\otimes}
\def\Pe{{P_e}}
\newcommand{\expeq}{\stackrel{.}{=}}
\newcommand{\expg}{\stackrel{.}{\ge}}
\newcommand{\expl}{\stackrel{.}{\le}}
\def\SIR{{\mathsf{SIR}}}

% Added by Takao
\def\nn{\nonumber}




\section{Model}
\label{sec:model}
Let $[N] = \{1, 2, \dots, N \}$ be a set of $N$ agents.
We examine an environment in which a system interacts with the agents over $T$ rounds.
Every round $t\leq T$ comprises $N$ \emph{sessions}, each session represents an encounter of the system with exactly one agent, and each agent interacts exactly once with the system every round.
I.e., in each round $t$ the agents arrive sequentially. 


\paragraph{Arrival order} The \emph{arrival order} of round $t$, denoted as $\ordv_t=(\ord_t(1),\dots, \ord_t(N))$, is an element from set of all permutations of $[N]$. Each entry $q$ in $\ordv_t$ is the index of the agent that arrives in the $q^{\text{th}}$ session of round $t$.
For example, if $\ord_t(1) = 2$, then agent $2$ arrives in the first session of round $t$.
Correspondingly, $\ord_t^{-1}(i)=q$ implies that agent $i$ arrives in the $q^{\text{th}}$ session of round $t$. 

As we demonstrate later, the arrival order has an immediate impact on agent rewards. We call the mechanism by which the arrival order is set \emph{arrival function} and denote it by $\ordname$. Throughout the paper, we consider several arrival functions such as the \emph{uniform arrival} function, denoted by $\uniord$, and the \emph{nudged arrival} $\sugord$; we introduce those formally in Sections~\ref{sec:uniform} and~\ref{sec:nudge}, respectively.

%We elaborate more on this concept in Section~\ref{sec: arrival}.


\paragraph{Arms} We consider a set of $K \geq 2$ arms, $A = \{a_1, \ldots, a_K\}$. The reward of arm $a_i$ in round $t$ is a random variable $X_i^t \sim \mathcal{D}^t_i$, where the rewards $(X_i^t)_{i,t}$ are mutually independent and bounded within the interval $[0,1]$. The reward distribution $\mathcal{D}^t_i$ of arm $a_i$, $i\in [K]$ at round $t\in T$ is assumed to be non-stationary but independent across arms and rounds. We denote the realized reward of arm $a_i$ in round $t$ by $x_i^t$. We assume \emph{reward consistency}, meaning that rewards may vary between rounds but remain constant within the sessions of a single round. Specifically, if an arm $a_i$ is selected multiple times during round~$t$, each selection yields the same reward $x_i^t$, where the superscript $t$ indicates its dependence on the round rather than the session. This consistency enables the system to leverage information obtained from earlier sessions to make more informed decisions in later sessions within the same round. We provide further details on this principle in Subsection~\ref{subsec:information}.


\paragraph{Algorithms} An algorithm is a mapping from histories to actions. We typically expect algorithms to maximize some aggregated agent metric like social welfare. Let $\mathcal H^{t,q}$ denote the information observed during all sessions of rounds $1$ to $t-1$ and sessions $1$ to $q-1$ in round $t$.  The history $\mathcal H^{t,q}$ is an element from $(A \times [0,1])^{(t-1) \cdot N +q-1}$, consisting of pairs of the form (pulled arm, realized reward). Notice that we restrict our attention to \emph{anonymous} algorithms, i.e., algorithms that do not distinguish between agents based on their identities. Instead, they only respond to the history of arms pulled and rewards observed, without conditioning on which specific agent performed each action.
%In the most general case, algorithms make decisions at session $q$ of round $t$  based on the entire history $\mathcal H^{t,q}$ and the index of the arriving agent $\ord_t(q)$. %Furthermore, we sometimes assume that algorithms have Bayesian information, i.e., algorithms are aware of the distributions $(\mathcal D_i)^K_{i=1}$. 
Furthermore, we sometimes assume that algorithms have Bayesian information, meaning they are aware of the reward distributions $(\mathcal{D}^t_i)_{i,t}$. If such an assumption is required to derive a result, we make it explicit. %Otherwise, we do not assume any additional knowledge about the algorithm’s information. %This distinction allows us to analyze both general algorithms without prior distributional knowledge and specialized algorithms that leverage Bayesian information.


\paragraph{Rewards} Let $\rt{i}$ denote the reward received by agent $i \in [N]$ at round $t$, and let $\Rt{i}$ denote her cumulative reward at the end of round $t$, i.e., $\Rt{i} = \sum_{\tau=1}^{t}{r^{\tau}_{i}}$. We further denote the \emph{social welfare} as the sum of the rewards all agents receive after $T$ rounds. Formally, $\sw=\sum^{N}_{i=1}{R^T_i}$. We emphasize that social welfare is independent of the arrival order. 


\paragraph{Envy}
We denote by $\adift{i}{j}$ the reward discrepancy of agents $i$ and $j$ in round $t$; namely, $\adift{i}{j}= \rt{i} - \rt{j}$. %We call this term \omer{name??} reward discrepancy in round $t$. 
The (cumulative) \emph{envy} between two agents at round $t$ is the difference in their cumulative rewards. Formally, $\env_{i,j}^t= \Rt{i} - \Rt{j}$ is the envy after $t$ rounds between agent $i$ and $j$. We can also formulate envy as the sum of reward discrepancies, $\env_{i,j}^t= \sum^{t}_{\tau=1}{\adif{i}{j}^\tau}$. Notice that envy is a signed quantity and can be either positive or negative. Specifically, if $\env_{i,j}^t < 0$, we say that agent $i$ envies agent $j$, and if $\env_{i,j}^t > 0$, agent $j$ envies agent $i$. The main goal of this paper is to investigate the behavior of the \emph{maximal envy}, defined as
\[
\env^t = \max_{i,j \in [N]} \env^t_{i,j}.
\]
For clarity, the term \emph{envy} will refer to the maximal envy.\footnote{ We address alternative definitions of envy in Section~\ref{sec:discussion}.} % Envy can also be defined in alternative ways, such as by averaging pairwise envy across all agents. We address average envy in Section~\ref{sec:avg_envy}.}
Note that $\env_{i,j}^t$ are random variables that depend on the decision-making algorithm, realized rewards, and the arrival order, and therefore, so is $\env^t$. If a result we obtain regarding envy depends on the arrival order $\ordname$, we write $\env^t(\ordname)$. Similarly, to ease notation, if $\ordname$ can be understood from the context, it is omitted.



\paragraph{Further Notation} We use the subscript $(q)$ to address elements of the $q^{\text{th}}$ session, for $q\in [N]$.
That is, we use the notation $\rt{(q)}$ to denote the reward granted to the agent that arrives in the $q^{\text{th}}$ session of round $t$ and $\Rt{(q)}$ to denote her cumulative reward. %Additionally, we introduce the notation $\at{(q)}$ to denote the arm pulled in that session.
Correspondingly, $\sdift{q}{w} = \rt{(q)} - \rt{(w)}$ is the reward discrepancy of the agents arriving in the $q^{\text{th}}$ and $w^{\text{th}}$ sessions of round $t$, respectively. 
To distinguish agents, arms, sessions and rounds, we use the letters $i,j$ to mark agents and arms, $q,w$ for sessions, and $t,\tau$ for rounds.


\subsection{Example}
\label{sec: example}
To illustrate the proposed setting and notation, we present the following example, which serves as a running example throughout the paper.

\begin{table}[t]
\centering
\begin{tabular}{|c|c|c|c|}
\hline
$t$ (round) & $\ordv_t$ (arrival order) & $x_1^t$ & $x_2^t$ \\ \hline
1           & 2, 1                     & 0.6     & 0.92    \\ \hline
2           & 1, 2                     & 0.48    & 0.1     \\ \hline
3           & 2, 1                     & 0.15    & 0.8     \\ \hline
\end{tabular}
\caption{
    Data for Example~\ref{example 1}.
}
\label{tbl: example}
\end{table}

\begin{algorithm}[t]
\caption{Algorithm for Example~\ref{example 1}}
\label{alguni}
\DontPrintSemicolon 
\For{round $t = 1$ to $T$}{
    pull $a_{1}$ in the first session\label{alguniexample: first}\\
    \lIf{$x^t_1 \geq \frac{1}{2}$}{pull $a_{1}$ again in second session \label{alguniexample: pulling a again}}
    \lElse{pull $a_{2}$ in second session \label{alguniexample: sopt else}}
}
\end{algorithm}


\begin{example}\label{example 1}
We consider $K=2$ uniform arms, $X_1,X_2 \sim \uni{0,1}$, and $N=2$ for some $T\geq 3$. We shall assume arm decision are made by Algorithm~\ref{alguni}: In the first session, the algorithm pulls $a_1$; if it yields a reward greater than $\nicefrac{1}{2}$, the algorithm pulls it again in the second session (the ``if'' clause). Otherwise, it pulls $a_2$.



We further assume that the arrival orders and rewards are as specified in Table~\ref{tbl: example}. Specifically, agent 2 arrives in the first session of round $t=1$, and pulling arm $a_2$ in this round would yield a reward of $x^1_2 = 0.92$. Importantly, \emph{this information is not available to the decision-making algorithm in advance} and is only revealed when or if the corresponding arms are pulled.




In the first round, $\boldsymbol{\eta}^1 = \left(2,1\right)$; thus, agent 2 arrives in the first session.
The algorithm pulls arm $a_1$, which means, $a^1_{(1)} = a_1$, and the agent receives $r_{2}^1=r_{(1)}^1=x_1^1=0.6$.
Later that round, in the second session, agent 1 arrives, and the algorithm pulls the same arm again since $x^1_1 = 0.6 \geq \nicefrac{1}{2}$ due to the ``if'' clause.
I.e., $a^1_{(2)} = a_1$ and $r_{1}^1 = r_{(2)}^1 = x_1^1 = 0.6$.
Even though the realized reward of arm $a_2$ in that round is higher ($0.92$), the algorithm is not aware of that value.
At the end of the first round, $R^1_1 = R^1_{(2)} = R^1_2 = R^1_{(1)} = 0.6$. The reward discrepancy is thus $\adif{1}{2}^1 = \adif{2}{1}^1= \sdif{2}{1}^1 = 0.6 - 0.6 =0$. 



In the second round, agent 1 arrives first, followed by agent 2.
Firstly, the algorithm pulls arm $a_1$ and agent 1 receives a reward of $r_{1}^2 = r_{(1)}^2 = x_1^2 = 0.48$.
Because the reward is lower than $\nicefrac{1}{2}$, in the second session the algorithm pulls the other arm ($a^2_{(2)} = a_2$), granting agent 2 a reward of $r_{2}^2 = r_{(2)}^2 = x_2^2 = 0.1$.
At the end of the second round, $R^2_1 = R^2_{(1)} = 0.6 + 0.48 = 1.08$ and $R^2_2 = R^2_{(2)} = 0.6 + 0.1 = 0.7$. Furthermore, $\sdif{2}{1}^2 = \adif{2}{1}^2 = r^2_{2} - r^2_{1} = 0.1 - 0.48 = -0.38$.

In the third and final round, agent 2 arrives first again, and receives a reward  of $0.15$ from $a_1$. When agent 1 arrives in the second session, the algorithm pulls arm $a_2$, and she receives a reward of $0.8$. As for the reward discrepancy, $\sdif{2}{1}^3 = \adif{2}{1}^3 = r^3_{2} - r^3_{1} = 0.15 - 0.8 = -0.75$. 

Finally, agent 1 has a cumulative reward of $R^3_1 = R^3_{(2)} = 0.6 + 0.48 + 0.8 = 1.88$, whereas agent~2 has a cumulative reward of $R^3_2 = R^3_{(1)} = 0.6 + 0.1 + 0.15 = 0.85$. In terms of envy, $\env^1_{1,2}= \adif{1}{2}^1 =0$, $\env^2_{1,2}=\adif{1}{2}^1+\adif{1}{2}^2= 0.38$, and $\env^3_{1,2} = -\env^3_{2,1} = R^3_1-R^3_2 = 1.88-0.85 = 1.03$, and consequently the envy in round 3 is $\env^3 = 1.03$.
\end{example}


\subsection{Information Exploitation}
\label{subsec:information}

In this subsection, we explain how algorithms can exploit intra-round information.
Since rewards are consistent in the sessions of each round, acquiring information in each session can be used to increase the reward of the following sessions.
In other words, the earlier sessions can be used for exploration, and we generally expect agents arriving in later sessions to receive higher rewards.
Taken to the extreme, an agent that arrives after all arms have been pulled could potentially obtain the highest reward of that round, depending on how the algorithm operates.

To further demonstrate the advantage of late arrival, we reconsider Example~\ref{example 1} and Algorithm~\ref{alguni}. 
The expected reward for the agent in the first session of round $t$ is $\E{\rt{(1)}}=\mu_1=\frac{1}{2}$, yet the expected reward of the agent in the second session is
\begin{align*}
\E{\rt{(2)}}=\E{\rt{(2)}\mid X^t_1 \geq \frac{1}{2} }\prb{X^t_1 \geq \frac{1}{2}} + \E{\rt{(2)}\mid X^t_1 < \frac{1}{2} }\prb{X^t_1 < \frac{1}{2}};
\end{align*}
thus, $\E{\rt{(2)}} =\E{X^t_1\mid X^t_1 \geq \frac{1}{2} }\cdot \frac{1}{2} + \mu_2\cdot\frac{1}{2} = \frac{5}{8}$.
Consequently, the expected welfare per round is $\E{\rt{(1)}+\rt{(2)}}=1+\frac{1}{8}$, and the benefit of arriving in the second session of any round $t$ is $\E{\rt{(2)} - \rt{(1)}} = \frac{1}{8}$. This gap creates envy over time, which we aim to measure and understand.
%This discrepancy generates envy over time, and our paper aims to better understand it.
\subsection{Socially Optimal Algorithms}
\label{sec: sw}
Since our model is novel, particularly in its focus on the reward consistency element, studying social welfare maximizing algorithms represents an important extension of our work. While the primary focus of this paper is to analyze envy under minimal assumptions about algorithmic operations, we also make progress in the direction of social welfare optimization. See more details in Section~\ref{sec:discussion}.%Due to space limitations, we defer the discussion on socially optimal algorithms to  \ifnum\Includeappendix=0{the appendix}\else{Section~\ref{appendix:sociallyopt}}\fi.




% Since our model is novel and specifically the reward consistency element, it might be interesting to study social welfare optimization. While the main focus of our paper is to study envy under minimal assumptions on how the algorithm operates, we take steps toward this direction as well. Due to space limitations, we defer the discussion on socially optimal algorithms to  \ifnum\Includeappendix=0{the appendix}\else{Section~\ref{appendix:sociallyopt}}\fi.  We devise a socially optimal algorithm for the two-agent case, offer efficient and optimal algorithms for special cases of $N>2$ agents, and an inefficient and approximately optimal algorithm for any instance with $N>2$. Moreover, we address the welfare-envy tradeoff in Section~\ref{sec:extensions}.


% Social welfare, unlike envy, is entirely independent of the arrival order. While the main focus of our paper is to study envy under minimal assumptions on how the algorithm operates, socially optimal algorithms might also be of interest. Due to space limitations, we defer the discussion on socially optimal algorithms to  \ifnum\Includeappendix=0{the appendix}\else{Section~\ref{appendix:sociallyopt}}\fi. We devise a socially optimal algorithm for the two-agent case, offer efficient and optimal algorithms for special cases of $N>2$ agents, and an inefficient and approximately optimal algorithm for any instance with $N>2$. %Furthermore, we treat the welfare-envy tradeoff of the special case of Example~\ref{example 1}.



\section{RELATED WORK}
\label{sec:relatedwork}
In this section, we describe the previous works related to our proposal, which are divided into two parts. In Section~\ref{sec:relatedwork_exoplanet}, we present a review of approaches based on machine learning techniques for the detection of planetary transit signals. Section~\ref{sec:relatedwork_attention} provides an account of the approaches based on attention mechanisms applied in Astronomy.\par

\subsection{Exoplanet detection}
\label{sec:relatedwork_exoplanet}
Machine learning methods have achieved great performance for the automatic selection of exoplanet transit signals. One of the earliest applications of machine learning is a model named Autovetter \citep{MCcauliff}, which is a random forest (RF) model based on characteristics derived from Kepler pipeline statistics to classify exoplanet and false positive signals. Then, other studies emerged that also used supervised learning. \cite{mislis2016sidra} also used a RF, but unlike the work by \citet{MCcauliff}, they used simulated light curves and a box least square \citep[BLS;][]{kovacs2002box}-based periodogram to search for transiting exoplanets. \citet{thompson2015machine} proposed a k-nearest neighbors model for Kepler data to determine if a given signal has similarity to known transits. Unsupervised learning techniques were also applied, such as self-organizing maps (SOM), proposed \citet{armstrong2016transit}; which implements an architecture to segment similar light curves. In the same way, \citet{armstrong2018automatic} developed a combination of supervised and unsupervised learning, including RF and SOM models. In general, these approaches require a previous phase of feature engineering for each light curve. \par

%DL is a modern data-driven technology that automatically extracts characteristics, and that has been successful in classification problems from a variety of application domains. The architecture relies on several layers of NNs of simple interconnected units and uses layers to build increasingly complex and useful features by means of linear and non-linear transformation. This family of models is capable of generating increasingly high-level representations \citep{lecun2015deep}.

The application of DL for exoplanetary signal detection has evolved rapidly in recent years and has become very popular in planetary science.  \citet{pearson2018} and \citet{zucker2018shallow} developed CNN-based algorithms that learn from synthetic data to search for exoplanets. Perhaps one of the most successful applications of the DL models in transit detection was that of \citet{Shallue_2018}; who, in collaboration with Google, proposed a CNN named AstroNet that recognizes exoplanet signals in real data from Kepler. AstroNet uses the training set of labelled TCEs from the Autovetter planet candidate catalog of Q1–Q17 data release 24 (DR24) of the Kepler mission \citep{catanzarite2015autovetter}. AstroNet analyses the data in two views: a ``global view'', and ``local view'' \citep{Shallue_2018}. \par


% The global view shows the characteristics of the light curve over an orbital period, and a local view shows the moment at occurring the transit in detail

%different = space-based

Based on AstroNet, researchers have modified the original AstroNet model to rank candidates from different surveys, specifically for Kepler and TESS missions. \citet{ansdell2018scientific} developed a CNN trained on Kepler data, and included for the first time the information on the centroids, showing that the model improves performance considerably. Then, \citet{osborn2020rapid} and \citet{yu2019identifying} also included the centroids information, but in addition, \citet{osborn2020rapid} included information of the stellar and transit parameters. Finally, \citet{rao2021nigraha} proposed a pipeline that includes a new ``half-phase'' view of the transit signal. This half-phase view represents a transit view with a different time and phase. The purpose of this view is to recover any possible secondary eclipse (the object hiding behind the disk of the primary star).


%last pipeline applies a procedure after the prediction of the model to obtain new candidates, this process is carried out through a series of steps that include the evaluation with Discovery and Validation of Exoplanets (DAVE) \citet{kostov2019discovery} that was adapted for the TESS telescope.\par
%



\subsection{Attention mechanisms in astronomy}
\label{sec:relatedwork_attention}
Despite the remarkable success of attention mechanisms in sequential data, few papers have exploited their advantages in astronomy. In particular, there are no models based on attention mechanisms for detecting planets. Below we present a summary of the main applications of this modeling approach to astronomy, based on two points of view; performance and interpretability of the model.\par
%Attention mechanisms have not yet been explored in all sub-areas of astronomy. However, recent works show a successful application of the mechanism.
%performance

The application of attention mechanisms has shown improvements in the performance of some regression and classification tasks compared to previous approaches. One of the first implementations of the attention mechanism was to find gravitational lenses proposed by \citet{thuruthipilly2021finding}. They designed 21 self-attention-based encoder models, where each model was trained separately with 18,000 simulated images, demonstrating that the model based on the Transformer has a better performance and uses fewer trainable parameters compared to CNN. A novel application was proposed by \citet{lin2021galaxy} for the morphological classification of galaxies, who used an architecture derived from the Transformer, named Vision Transformer (VIT) \citep{dosovitskiy2020image}. \citet{lin2021galaxy} demonstrated competitive results compared to CNNs. Another application with successful results was proposed by \citet{zerveas2021transformer}; which first proposed a transformer-based framework for learning unsupervised representations of multivariate time series. Their methodology takes advantage of unlabeled data to train an encoder and extract dense vector representations of time series. Subsequently, they evaluate the model for regression and classification tasks, demonstrating better performance than other state-of-the-art supervised methods, even with data sets with limited samples.

%interpretation
Regarding the interpretability of the model, a recent contribution that analyses the attention maps was presented by \citet{bowles20212}, which explored the use of group-equivariant self-attention for radio astronomy classification. Compared to other approaches, this model analysed the attention maps of the predictions and showed that the mechanism extracts the brightest spots and jets of the radio source more clearly. This indicates that attention maps for prediction interpretation could help experts see patterns that the human eye often misses. \par

In the field of variable stars, \citet{allam2021paying} employed the mechanism for classifying multivariate time series in variable stars. And additionally, \citet{allam2021paying} showed that the activation weights are accommodated according to the variation in brightness of the star, achieving a more interpretable model. And finally, related to the TESS telescope, \citet{morvan2022don} proposed a model that removes the noise from the light curves through the distribution of attention weights. \citet{morvan2022don} showed that the use of the attention mechanism is excellent for removing noise and outliers in time series datasets compared with other approaches. In addition, the use of attention maps allowed them to show the representations learned from the model. \par

Recent attention mechanism approaches in astronomy demonstrate comparable results with earlier approaches, such as CNNs. At the same time, they offer interpretability of their results, which allows a post-prediction analysis. \par


%\input{latex/sections/enhance performance.tex}


\section{Conclusion}
This paper presents an in-depth study of LLMs' fluid intelligence deficiencies using the ARC tasks, with a series of controlled experiments from multiple perspectives. %and uncovering several key findings. 
Through task decomposition, we introduce the atomic ARAOC benchmark, revealing that LLMs struggle with atomic operations despite their simplicity for humans. We further demonstrate that LLMs' task composition abilities are limited, as improvements on the decomposed ARAOC tasks via fine-tuning do not lead to better performance on ARC tasks. Additionally, our study shows that LLMs' difficulty in encoding abstract input formats is a major obstacle in addressing ARC tasks. Lastly, it shows an intrinsic limitation in the left-to-right paradigm of LLMs, which hinders their ability to achieve advanced fluid intelligence.



\section*{Limitations}
Due to the experiment budget, on all the ARC related experiments, we only evaluate LLMs on 100 tasks rather than the whole corpus following~\citet{wang2023hypothesis}, which may lead to potential bias in the evaluation results. Also, although most of the ARC tasks can be composed by the six atomic operations proposed by our work, there may still exist very few tasks that cannot be composed by our atomic operations, which may also introducing few bias to~\tref{tab:fine-tune arc performance}. We will try to provide more comprehensive results in future works once we get more experimental budgets, and propose more atomic operations that could be used to cover more ARC tasks.

%\section*{Ethical Considerations}
%Since this paper includes many responses generated by LLMs, it is possible that these LLM generated contents include toxic and harmful parts, requiring users to perform comprehensive data processing if they want to use our methods.

\section*{Acknowledgment}
This work has been made possible by a Research Impact Fund project (RIF R6003-21) and a General Research Fund project (GRF 16203224) funded by the Research Grants Council (RGC) of the Hong Kong Government.

\bibliography{custom}

\appendix
\clearpage

\subsection{Lloyd-Max Algorithm}
\label{subsec:Lloyd-Max}
For a given quantization bitwidth $B$ and an operand $\bm{X}$, the Lloyd-Max algorithm finds $2^B$ quantization levels $\{\hat{x}_i\}_{i=1}^{2^B}$ such that quantizing $\bm{X}$ by rounding each scalar in $\bm{X}$ to the nearest quantization level minimizes the quantization MSE. 

The algorithm starts with an initial guess of quantization levels and then iteratively computes quantization thresholds $\{\tau_i\}_{i=1}^{2^B-1}$ and updates quantization levels $\{\hat{x}_i\}_{i=1}^{2^B}$. Specifically, at iteration $n$, thresholds are set to the midpoints of the previous iteration's levels:
\begin{align*}
    \tau_i^{(n)}=\frac{\hat{x}_i^{(n-1)}+\hat{x}_{i+1}^{(n-1)}}2 \text{ for } i=1\ldots 2^B-1
\end{align*}
Subsequently, the quantization levels are re-computed as conditional means of the data regions defined by the new thresholds:
\begin{align*}
    \hat{x}_i^{(n)}=\mathbb{E}\left[ \bm{X} \big| \bm{X}\in [\tau_{i-1}^{(n)},\tau_i^{(n)}] \right] \text{ for } i=1\ldots 2^B
\end{align*}
where to satisfy boundary conditions we have $\tau_0=-\infty$ and $\tau_{2^B}=\infty$. The algorithm iterates the above steps until convergence.

Figure \ref{fig:lm_quant} compares the quantization levels of a $7$-bit floating point (E3M3) quantizer (left) to a $7$-bit Lloyd-Max quantizer (right) when quantizing a layer of weights from the GPT3-126M model at a per-tensor granularity. As shown, the Lloyd-Max quantizer achieves substantially lower quantization MSE. Further, Table \ref{tab:FP7_vs_LM7} shows the superior perplexity achieved by Lloyd-Max quantizers for bitwidths of $7$, $6$ and $5$. The difference between the quantizers is clear at 5 bits, where per-tensor FP quantization incurs a drastic and unacceptable increase in perplexity, while Lloyd-Max quantization incurs a much smaller increase. Nevertheless, we note that even the optimal Lloyd-Max quantizer incurs a notable ($\sim 1.5$) increase in perplexity due to the coarse granularity of quantization. 

\begin{figure}[h]
  \centering
  \includegraphics[width=0.7\linewidth]{sections/figures/LM7_FP7.pdf}
  \caption{\small Quantization levels and the corresponding quantization MSE of Floating Point (left) vs Lloyd-Max (right) Quantizers for a layer of weights in the GPT3-126M model.}
  \label{fig:lm_quant}
\end{figure}

\begin{table}[h]\scriptsize
\begin{center}
\caption{\label{tab:FP7_vs_LM7} \small Comparing perplexity (lower is better) achieved by floating point quantizers and Lloyd-Max quantizers on a GPT3-126M model for the Wikitext-103 dataset.}
\begin{tabular}{c|cc|c}
\hline
 \multirow{2}{*}{\textbf{Bitwidth}} & \multicolumn{2}{|c|}{\textbf{Floating-Point Quantizer}} & \textbf{Lloyd-Max Quantizer} \\
 & Best Format & Wikitext-103 Perplexity & Wikitext-103 Perplexity \\
\hline
7 & E3M3 & 18.32 & 18.27 \\
6 & E3M2 & 19.07 & 18.51 \\
5 & E4M0 & 43.89 & 19.71 \\
\hline
\end{tabular}
\end{center}
\end{table}

\subsection{Proof of Local Optimality of LO-BCQ}
\label{subsec:lobcq_opt_proof}
For a given block $\bm{b}_j$, the quantization MSE during LO-BCQ can be empirically evaluated as $\frac{1}{L_b}\lVert \bm{b}_j- \bm{\hat{b}}_j\rVert^2_2$ where $\bm{\hat{b}}_j$ is computed from equation (\ref{eq:clustered_quantization_definition}) as $C_{f(\bm{b}_j)}(\bm{b}_j)$. Further, for a given block cluster $\mathcal{B}_i$, we compute the quantization MSE as $\frac{1}{|\mathcal{B}_{i}|}\sum_{\bm{b} \in \mathcal{B}_{i}} \frac{1}{L_b}\lVert \bm{b}- C_i^{(n)}(\bm{b})\rVert^2_2$. Therefore, at the end of iteration $n$, we evaluate the overall quantization MSE $J^{(n)}$ for a given operand $\bm{X}$ composed of $N_c$ block clusters as:
\begin{align*}
    \label{eq:mse_iter_n}
    J^{(n)} = \frac{1}{N_c} \sum_{i=1}^{N_c} \frac{1}{|\mathcal{B}_{i}^{(n)}|}\sum_{\bm{v} \in \mathcal{B}_{i}^{(n)}} \frac{1}{L_b}\lVert \bm{b}- B_i^{(n)}(\bm{b})\rVert^2_2
\end{align*}

At the end of iteration $n$, the codebooks are updated from $\mathcal{C}^{(n-1)}$ to $\mathcal{C}^{(n)}$. However, the mapping of a given vector $\bm{b}_j$ to quantizers $\mathcal{C}^{(n)}$ remains as  $f^{(n)}(\bm{b}_j)$. At the next iteration, during the vector clustering step, $f^{(n+1)}(\bm{b}_j)$ finds new mapping of $\bm{b}_j$ to updated codebooks $\mathcal{C}^{(n)}$ such that the quantization MSE over the candidate codebooks is minimized. Therefore, we obtain the following result for $\bm{b}_j$:
\begin{align*}
\frac{1}{L_b}\lVert \bm{b}_j - C_{f^{(n+1)}(\bm{b}_j)}^{(n)}(\bm{b}_j)\rVert^2_2 \le \frac{1}{L_b}\lVert \bm{b}_j - C_{f^{(n)}(\bm{b}_j)}^{(n)}(\bm{b}_j)\rVert^2_2
\end{align*}

That is, quantizing $\bm{b}_j$ at the end of the block clustering step of iteration $n+1$ results in lower quantization MSE compared to quantizing at the end of iteration $n$. Since this is true for all $\bm{b} \in \bm{X}$, we assert the following:
\begin{equation}
\begin{split}
\label{eq:mse_ineq_1}
    \tilde{J}^{(n+1)} &= \frac{1}{N_c} \sum_{i=1}^{N_c} \frac{1}{|\mathcal{B}_{i}^{(n+1)}|}\sum_{\bm{b} \in \mathcal{B}_{i}^{(n+1)}} \frac{1}{L_b}\lVert \bm{b} - C_i^{(n)}(b)\rVert^2_2 \le J^{(n)}
\end{split}
\end{equation}
where $\tilde{J}^{(n+1)}$ is the the quantization MSE after the vector clustering step at iteration $n+1$.

Next, during the codebook update step (\ref{eq:quantizers_update}) at iteration $n+1$, the per-cluster codebooks $\mathcal{C}^{(n)}$ are updated to $\mathcal{C}^{(n+1)}$ by invoking the Lloyd-Max algorithm \citep{Lloyd}. We know that for any given value distribution, the Lloyd-Max algorithm minimizes the quantization MSE. Therefore, for a given vector cluster $\mathcal{B}_i$ we obtain the following result:

\begin{equation}
    \frac{1}{|\mathcal{B}_{i}^{(n+1)}|}\sum_{\bm{b} \in \mathcal{B}_{i}^{(n+1)}} \frac{1}{L_b}\lVert \bm{b}- C_i^{(n+1)}(\bm{b})\rVert^2_2 \le \frac{1}{|\mathcal{B}_{i}^{(n+1)}|}\sum_{\bm{b} \in \mathcal{B}_{i}^{(n+1)}} \frac{1}{L_b}\lVert \bm{b}- C_i^{(n)}(\bm{b})\rVert^2_2
\end{equation}

The above equation states that quantizing the given block cluster $\mathcal{B}_i$ after updating the associated codebook from $C_i^{(n)}$ to $C_i^{(n+1)}$ results in lower quantization MSE. Since this is true for all the block clusters, we derive the following result: 
\begin{equation}
\begin{split}
\label{eq:mse_ineq_2}
     J^{(n+1)} &= \frac{1}{N_c} \sum_{i=1}^{N_c} \frac{1}{|\mathcal{B}_{i}^{(n+1)}|}\sum_{\bm{b} \in \mathcal{B}_{i}^{(n+1)}} \frac{1}{L_b}\lVert \bm{b}- C_i^{(n+1)}(\bm{b})\rVert^2_2  \le \tilde{J}^{(n+1)}   
\end{split}
\end{equation}

Following (\ref{eq:mse_ineq_1}) and (\ref{eq:mse_ineq_2}), we find that the quantization MSE is non-increasing for each iteration, that is, $J^{(1)} \ge J^{(2)} \ge J^{(3)} \ge \ldots \ge J^{(M)}$ where $M$ is the maximum number of iterations. 
%Therefore, we can say that if the algorithm converges, then it must be that it has converged to a local minimum. 
\hfill $\blacksquare$


\begin{figure}
    \begin{center}
    \includegraphics[width=0.5\textwidth]{sections//figures/mse_vs_iter.pdf}
    \end{center}
    \caption{\small NMSE vs iterations during LO-BCQ compared to other block quantization proposals}
    \label{fig:nmse_vs_iter}
\end{figure}

Figure \ref{fig:nmse_vs_iter} shows the empirical convergence of LO-BCQ across several block lengths and number of codebooks. Also, the MSE achieved by LO-BCQ is compared to baselines such as MXFP and VSQ. As shown, LO-BCQ converges to a lower MSE than the baselines. Further, we achieve better convergence for larger number of codebooks ($N_c$) and for a smaller block length ($L_b$), both of which increase the bitwidth of BCQ (see Eq \ref{eq:bitwidth_bcq}).


\subsection{Additional Accuracy Results}
%Table \ref{tab:lobcq_config} lists the various LOBCQ configurations and their corresponding bitwidths.
\begin{table}
\setlength{\tabcolsep}{4.75pt}
\begin{center}
\caption{\label{tab:lobcq_config} Various LO-BCQ configurations and their bitwidths.}
\begin{tabular}{|c||c|c|c|c||c|c||c|} 
\hline
 & \multicolumn{4}{|c||}{$L_b=8$} & \multicolumn{2}{|c||}{$L_b=4$} & $L_b=2$ \\
 \hline
 \backslashbox{$L_A$\kern-1em}{\kern-1em$N_c$} & 2 & 4 & 8 & 16 & 2 & 4 & 2 \\
 \hline
 64 & 4.25 & 4.375 & 4.5 & 4.625 & 4.375 & 4.625 & 4.625\\
 \hline
 32 & 4.375 & 4.5 & 4.625& 4.75 & 4.5 & 4.75 & 4.75 \\
 \hline
 16 & 4.625 & 4.75& 4.875 & 5 & 4.75 & 5 & 5 \\
 \hline
\end{tabular}
\end{center}
\end{table}

%\subsection{Perplexity achieved by various LO-BCQ configurations on Wikitext-103 dataset}

\begin{table} \centering
\begin{tabular}{|c||c|c|c|c||c|c||c|} 
\hline
 $L_b \rightarrow$& \multicolumn{4}{c||}{8} & \multicolumn{2}{c||}{4} & 2\\
 \hline
 \backslashbox{$L_A$\kern-1em}{\kern-1em$N_c$} & 2 & 4 & 8 & 16 & 2 & 4 & 2  \\
 %$N_c \rightarrow$ & 2 & 4 & 8 & 16 & 2 & 4 & 2 \\
 \hline
 \hline
 \multicolumn{8}{c}{GPT3-1.3B (FP32 PPL = 9.98)} \\ 
 \hline
 \hline
 64 & 10.40 & 10.23 & 10.17 & 10.15 &  10.28 & 10.18 & 10.19 \\
 \hline
 32 & 10.25 & 10.20 & 10.15 & 10.12 &  10.23 & 10.17 & 10.17 \\
 \hline
 16 & 10.22 & 10.16 & 10.10 & 10.09 &  10.21 & 10.14 & 10.16 \\
 \hline
  \hline
 \multicolumn{8}{c}{GPT3-8B (FP32 PPL = 7.38)} \\ 
 \hline
 \hline
 64 & 7.61 & 7.52 & 7.48 &  7.47 &  7.55 &  7.49 & 7.50 \\
 \hline
 32 & 7.52 & 7.50 & 7.46 &  7.45 &  7.52 &  7.48 & 7.48  \\
 \hline
 16 & 7.51 & 7.48 & 7.44 &  7.44 &  7.51 &  7.49 & 7.47  \\
 \hline
\end{tabular}
\caption{\label{tab:ppl_gpt3_abalation} Wikitext-103 perplexity across GPT3-1.3B and 8B models.}
\end{table}

\begin{table} \centering
\begin{tabular}{|c||c|c|c|c||} 
\hline
 $L_b \rightarrow$& \multicolumn{4}{c||}{8}\\
 \hline
 \backslashbox{$L_A$\kern-1em}{\kern-1em$N_c$} & 2 & 4 & 8 & 16 \\
 %$N_c \rightarrow$ & 2 & 4 & 8 & 16 & 2 & 4 & 2 \\
 \hline
 \hline
 \multicolumn{5}{|c|}{Llama2-7B (FP32 PPL = 5.06)} \\ 
 \hline
 \hline
 64 & 5.31 & 5.26 & 5.19 & 5.18  \\
 \hline
 32 & 5.23 & 5.25 & 5.18 & 5.15  \\
 \hline
 16 & 5.23 & 5.19 & 5.16 & 5.14  \\
 \hline
 \multicolumn{5}{|c|}{Nemotron4-15B (FP32 PPL = 5.87)} \\ 
 \hline
 \hline
 64  & 6.3 & 6.20 & 6.13 & 6.08  \\
 \hline
 32  & 6.24 & 6.12 & 6.07 & 6.03  \\
 \hline
 16  & 6.12 & 6.14 & 6.04 & 6.02  \\
 \hline
 \multicolumn{5}{|c|}{Nemotron4-340B (FP32 PPL = 3.48)} \\ 
 \hline
 \hline
 64 & 3.67 & 3.62 & 3.60 & 3.59 \\
 \hline
 32 & 3.63 & 3.61 & 3.59 & 3.56 \\
 \hline
 16 & 3.61 & 3.58 & 3.57 & 3.55 \\
 \hline
\end{tabular}
\caption{\label{tab:ppl_llama7B_nemo15B} Wikitext-103 perplexity compared to FP32 baseline in Llama2-7B and Nemotron4-15B, 340B models}
\end{table}

%\subsection{Perplexity achieved by various LO-BCQ configurations on MMLU dataset}


\begin{table} \centering
\begin{tabular}{|c||c|c|c|c||c|c|c|c|} 
\hline
 $L_b \rightarrow$& \multicolumn{4}{c||}{8} & \multicolumn{4}{c||}{8}\\
 \hline
 \backslashbox{$L_A$\kern-1em}{\kern-1em$N_c$} & 2 & 4 & 8 & 16 & 2 & 4 & 8 & 16  \\
 %$N_c \rightarrow$ & 2 & 4 & 8 & 16 & 2 & 4 & 2 \\
 \hline
 \hline
 \multicolumn{5}{|c|}{Llama2-7B (FP32 Accuracy = 45.8\%)} & \multicolumn{4}{|c|}{Llama2-70B (FP32 Accuracy = 69.12\%)} \\ 
 \hline
 \hline
 64 & 43.9 & 43.4 & 43.9 & 44.9 & 68.07 & 68.27 & 68.17 & 68.75 \\
 \hline
 32 & 44.5 & 43.8 & 44.9 & 44.5 & 68.37 & 68.51 & 68.35 & 68.27  \\
 \hline
 16 & 43.9 & 42.7 & 44.9 & 45 & 68.12 & 68.77 & 68.31 & 68.59  \\
 \hline
 \hline
 \multicolumn{5}{|c|}{GPT3-22B (FP32 Accuracy = 38.75\%)} & \multicolumn{4}{|c|}{Nemotron4-15B (FP32 Accuracy = 64.3\%)} \\ 
 \hline
 \hline
 64 & 36.71 & 38.85 & 38.13 & 38.92 & 63.17 & 62.36 & 63.72 & 64.09 \\
 \hline
 32 & 37.95 & 38.69 & 39.45 & 38.34 & 64.05 & 62.30 & 63.8 & 64.33  \\
 \hline
 16 & 38.88 & 38.80 & 38.31 & 38.92 & 63.22 & 63.51 & 63.93 & 64.43  \\
 \hline
\end{tabular}
\caption{\label{tab:mmlu_abalation} Accuracy on MMLU dataset across GPT3-22B, Llama2-7B, 70B and Nemotron4-15B models.}
\end{table}


%\subsection{Perplexity achieved by various LO-BCQ configurations on LM evaluation harness}

\begin{table} \centering
\begin{tabular}{|c||c|c|c|c||c|c|c|c|} 
\hline
 $L_b \rightarrow$& \multicolumn{4}{c||}{8} & \multicolumn{4}{c||}{8}\\
 \hline
 \backslashbox{$L_A$\kern-1em}{\kern-1em$N_c$} & 2 & 4 & 8 & 16 & 2 & 4 & 8 & 16  \\
 %$N_c \rightarrow$ & 2 & 4 & 8 & 16 & 2 & 4 & 2 \\
 \hline
 \hline
 \multicolumn{5}{|c|}{Race (FP32 Accuracy = 37.51\%)} & \multicolumn{4}{|c|}{Boolq (FP32 Accuracy = 64.62\%)} \\ 
 \hline
 \hline
 64 & 36.94 & 37.13 & 36.27 & 37.13 & 63.73 & 62.26 & 63.49 & 63.36 \\
 \hline
 32 & 37.03 & 36.36 & 36.08 & 37.03 & 62.54 & 63.51 & 63.49 & 63.55  \\
 \hline
 16 & 37.03 & 37.03 & 36.46 & 37.03 & 61.1 & 63.79 & 63.58 & 63.33  \\
 \hline
 \hline
 \multicolumn{5}{|c|}{Winogrande (FP32 Accuracy = 58.01\%)} & \multicolumn{4}{|c|}{Piqa (FP32 Accuracy = 74.21\%)} \\ 
 \hline
 \hline
 64 & 58.17 & 57.22 & 57.85 & 58.33 & 73.01 & 73.07 & 73.07 & 72.80 \\
 \hline
 32 & 59.12 & 58.09 & 57.85 & 58.41 & 73.01 & 73.94 & 72.74 & 73.18  \\
 \hline
 16 & 57.93 & 58.88 & 57.93 & 58.56 & 73.94 & 72.80 & 73.01 & 73.94  \\
 \hline
\end{tabular}
\caption{\label{tab:mmlu_abalation} Accuracy on LM evaluation harness tasks on GPT3-1.3B model.}
\end{table}

\begin{table} \centering
\begin{tabular}{|c||c|c|c|c||c|c|c|c|} 
\hline
 $L_b \rightarrow$& \multicolumn{4}{c||}{8} & \multicolumn{4}{c||}{8}\\
 \hline
 \backslashbox{$L_A$\kern-1em}{\kern-1em$N_c$} & 2 & 4 & 8 & 16 & 2 & 4 & 8 & 16  \\
 %$N_c \rightarrow$ & 2 & 4 & 8 & 16 & 2 & 4 & 2 \\
 \hline
 \hline
 \multicolumn{5}{|c|}{Race (FP32 Accuracy = 41.34\%)} & \multicolumn{4}{|c|}{Boolq (FP32 Accuracy = 68.32\%)} \\ 
 \hline
 \hline
 64 & 40.48 & 40.10 & 39.43 & 39.90 & 69.20 & 68.41 & 69.45 & 68.56 \\
 \hline
 32 & 39.52 & 39.52 & 40.77 & 39.62 & 68.32 & 67.43 & 68.17 & 69.30  \\
 \hline
 16 & 39.81 & 39.71 & 39.90 & 40.38 & 68.10 & 66.33 & 69.51 & 69.42  \\
 \hline
 \hline
 \multicolumn{5}{|c|}{Winogrande (FP32 Accuracy = 67.88\%)} & \multicolumn{4}{|c|}{Piqa (FP32 Accuracy = 78.78\%)} \\ 
 \hline
 \hline
 64 & 66.85 & 66.61 & 67.72 & 67.88 & 77.31 & 77.42 & 77.75 & 77.64 \\
 \hline
 32 & 67.25 & 67.72 & 67.72 & 67.00 & 77.31 & 77.04 & 77.80 & 77.37  \\
 \hline
 16 & 68.11 & 68.90 & 67.88 & 67.48 & 77.37 & 78.13 & 78.13 & 77.69  \\
 \hline
\end{tabular}
\caption{\label{tab:mmlu_abalation} Accuracy on LM evaluation harness tasks on GPT3-8B model.}
\end{table}

\begin{table} \centering
\begin{tabular}{|c||c|c|c|c||c|c|c|c|} 
\hline
 $L_b \rightarrow$& \multicolumn{4}{c||}{8} & \multicolumn{4}{c||}{8}\\
 \hline
 \backslashbox{$L_A$\kern-1em}{\kern-1em$N_c$} & 2 & 4 & 8 & 16 & 2 & 4 & 8 & 16  \\
 %$N_c \rightarrow$ & 2 & 4 & 8 & 16 & 2 & 4 & 2 \\
 \hline
 \hline
 \multicolumn{5}{|c|}{Race (FP32 Accuracy = 40.67\%)} & \multicolumn{4}{|c|}{Boolq (FP32 Accuracy = 76.54\%)} \\ 
 \hline
 \hline
 64 & 40.48 & 40.10 & 39.43 & 39.90 & 75.41 & 75.11 & 77.09 & 75.66 \\
 \hline
 32 & 39.52 & 39.52 & 40.77 & 39.62 & 76.02 & 76.02 & 75.96 & 75.35  \\
 \hline
 16 & 39.81 & 39.71 & 39.90 & 40.38 & 75.05 & 73.82 & 75.72 & 76.09  \\
 \hline
 \hline
 \multicolumn{5}{|c|}{Winogrande (FP32 Accuracy = 70.64\%)} & \multicolumn{4}{|c|}{Piqa (FP32 Accuracy = 79.16\%)} \\ 
 \hline
 \hline
 64 & 69.14 & 70.17 & 70.17 & 70.56 & 78.24 & 79.00 & 78.62 & 78.73 \\
 \hline
 32 & 70.96 & 69.69 & 71.27 & 69.30 & 78.56 & 79.49 & 79.16 & 78.89  \\
 \hline
 16 & 71.03 & 69.53 & 69.69 & 70.40 & 78.13 & 79.16 & 79.00 & 79.00  \\
 \hline
\end{tabular}
\caption{\label{tab:mmlu_abalation} Accuracy on LM evaluation harness tasks on GPT3-22B model.}
\end{table}

\begin{table} \centering
\begin{tabular}{|c||c|c|c|c||c|c|c|c|} 
\hline
 $L_b \rightarrow$& \multicolumn{4}{c||}{8} & \multicolumn{4}{c||}{8}\\
 \hline
 \backslashbox{$L_A$\kern-1em}{\kern-1em$N_c$} & 2 & 4 & 8 & 16 & 2 & 4 & 8 & 16  \\
 %$N_c \rightarrow$ & 2 & 4 & 8 & 16 & 2 & 4 & 2 \\
 \hline
 \hline
 \multicolumn{5}{|c|}{Race (FP32 Accuracy = 44.4\%)} & \multicolumn{4}{|c|}{Boolq (FP32 Accuracy = 79.29\%)} \\ 
 \hline
 \hline
 64 & 42.49 & 42.51 & 42.58 & 43.45 & 77.58 & 77.37 & 77.43 & 78.1 \\
 \hline
 32 & 43.35 & 42.49 & 43.64 & 43.73 & 77.86 & 75.32 & 77.28 & 77.86  \\
 \hline
 16 & 44.21 & 44.21 & 43.64 & 42.97 & 78.65 & 77 & 76.94 & 77.98  \\
 \hline
 \hline
 \multicolumn{5}{|c|}{Winogrande (FP32 Accuracy = 69.38\%)} & \multicolumn{4}{|c|}{Piqa (FP32 Accuracy = 78.07\%)} \\ 
 \hline
 \hline
 64 & 68.9 & 68.43 & 69.77 & 68.19 & 77.09 & 76.82 & 77.09 & 77.86 \\
 \hline
 32 & 69.38 & 68.51 & 68.82 & 68.90 & 78.07 & 76.71 & 78.07 & 77.86  \\
 \hline
 16 & 69.53 & 67.09 & 69.38 & 68.90 & 77.37 & 77.8 & 77.91 & 77.69  \\
 \hline
\end{tabular}
\caption{\label{tab:mmlu_abalation} Accuracy on LM evaluation harness tasks on Llama2-7B model.}
\end{table}

\begin{table} \centering
\begin{tabular}{|c||c|c|c|c||c|c|c|c|} 
\hline
 $L_b \rightarrow$& \multicolumn{4}{c||}{8} & \multicolumn{4}{c||}{8}\\
 \hline
 \backslashbox{$L_A$\kern-1em}{\kern-1em$N_c$} & 2 & 4 & 8 & 16 & 2 & 4 & 8 & 16  \\
 %$N_c \rightarrow$ & 2 & 4 & 8 & 16 & 2 & 4 & 2 \\
 \hline
 \hline
 \multicolumn{5}{|c|}{Race (FP32 Accuracy = 48.8\%)} & \multicolumn{4}{|c|}{Boolq (FP32 Accuracy = 85.23\%)} \\ 
 \hline
 \hline
 64 & 49.00 & 49.00 & 49.28 & 48.71 & 82.82 & 84.28 & 84.03 & 84.25 \\
 \hline
 32 & 49.57 & 48.52 & 48.33 & 49.28 & 83.85 & 84.46 & 84.31 & 84.93  \\
 \hline
 16 & 49.85 & 49.09 & 49.28 & 48.99 & 85.11 & 84.46 & 84.61 & 83.94  \\
 \hline
 \hline
 \multicolumn{5}{|c|}{Winogrande (FP32 Accuracy = 79.95\%)} & \multicolumn{4}{|c|}{Piqa (FP32 Accuracy = 81.56\%)} \\ 
 \hline
 \hline
 64 & 78.77 & 78.45 & 78.37 & 79.16 & 81.45 & 80.69 & 81.45 & 81.5 \\
 \hline
 32 & 78.45 & 79.01 & 78.69 & 80.66 & 81.56 & 80.58 & 81.18 & 81.34  \\
 \hline
 16 & 79.95 & 79.56 & 79.79 & 79.72 & 81.28 & 81.66 & 81.28 & 80.96  \\
 \hline
\end{tabular}
\caption{\label{tab:mmlu_abalation} Accuracy on LM evaluation harness tasks on Llama2-70B model.}
\end{table}

%\section{MSE Studies}
%\textcolor{red}{TODO}


\subsection{Number Formats and Quantization Method}
\label{subsec:numFormats_quantMethod}
\subsubsection{Integer Format}
An $n$-bit signed integer (INT) is typically represented with a 2s-complement format \citep{yao2022zeroquant,xiao2023smoothquant,dai2021vsq}, where the most significant bit denotes the sign.

\subsubsection{Floating Point Format}
An $n$-bit signed floating point (FP) number $x$ comprises of a 1-bit sign ($x_{\mathrm{sign}}$), $B_m$-bit mantissa ($x_{\mathrm{mant}}$) and $B_e$-bit exponent ($x_{\mathrm{exp}}$) such that $B_m+B_e=n-1$. The associated constant exponent bias ($E_{\mathrm{bias}}$) is computed as $(2^{{B_e}-1}-1)$. We denote this format as $E_{B_e}M_{B_m}$.  

\subsubsection{Quantization Scheme}
\label{subsec:quant_method}
A quantization scheme dictates how a given unquantized tensor is converted to its quantized representation. We consider FP formats for the purpose of illustration. Given an unquantized tensor $\bm{X}$ and an FP format $E_{B_e}M_{B_m}$, we first, we compute the quantization scale factor $s_X$ that maps the maximum absolute value of $\bm{X}$ to the maximum quantization level of the $E_{B_e}M_{B_m}$ format as follows:
\begin{align}
\label{eq:sf}
    s_X = \frac{\mathrm{max}(|\bm{X}|)}{\mathrm{max}(E_{B_e}M_{B_m})}
\end{align}
In the above equation, $|\cdot|$ denotes the absolute value function.

Next, we scale $\bm{X}$ by $s_X$ and quantize it to $\hat{\bm{X}}$ by rounding it to the nearest quantization level of $E_{B_e}M_{B_m}$ as:

\begin{align}
\label{eq:tensor_quant}
    \hat{\bm{X}} = \text{round-to-nearest}\left(\frac{\bm{X}}{s_X}, E_{B_e}M_{B_m}\right)
\end{align}

We perform dynamic max-scaled quantization \citep{wu2020integer}, where the scale factor $s$ for activations is dynamically computed during runtime.

\subsection{Vector Scaled Quantization}
\begin{wrapfigure}{r}{0.35\linewidth}
  \centering
  \includegraphics[width=\linewidth]{sections/figures/vsquant.jpg}
  \caption{\small Vectorwise decomposition for per-vector scaled quantization (VSQ \citep{dai2021vsq}).}
  \label{fig:vsquant}
\end{wrapfigure}
During VSQ \citep{dai2021vsq}, the operand tensors are decomposed into 1D vectors in a hardware friendly manner as shown in Figure \ref{fig:vsquant}. Since the decomposed tensors are used as operands in matrix multiplications during inference, it is beneficial to perform this decomposition along the reduction dimension of the multiplication. The vectorwise quantization is performed similar to tensorwise quantization described in Equations \ref{eq:sf} and \ref{eq:tensor_quant}, where a scale factor $s_v$ is required for each vector $\bm{v}$ that maps the maximum absolute value of that vector to the maximum quantization level. While smaller vector lengths can lead to larger accuracy gains, the associated memory and computational overheads due to the per-vector scale factors increases. To alleviate these overheads, VSQ \citep{dai2021vsq} proposed a second level quantization of the per-vector scale factors to unsigned integers, while MX \citep{rouhani2023shared} quantizes them to integer powers of 2 (denoted as $2^{INT}$).

\subsubsection{MX Format}
The MX format proposed in \citep{rouhani2023microscaling} introduces the concept of sub-block shifting. For every two scalar elements of $b$-bits each, there is a shared exponent bit. The value of this exponent bit is determined through an empirical analysis that targets minimizing quantization MSE. We note that the FP format $E_{1}M_{b}$ is strictly better than MX from an accuracy perspective since it allocates a dedicated exponent bit to each scalar as opposed to sharing it across two scalars. Therefore, we conservatively bound the accuracy of a $b+2$-bit signed MX format with that of a $E_{1}M_{b}$ format in our comparisons. For instance, we use E1M2 format as a proxy for MX4.

\begin{figure}
    \centering
    \includegraphics[width=1\linewidth]{sections//figures/BlockFormats.pdf}
    \caption{\small Comparing LO-BCQ to MX format.}
    \label{fig:block_formats}
\end{figure}

Figure \ref{fig:block_formats} compares our $4$-bit LO-BCQ block format to MX \citep{rouhani2023microscaling}. As shown, both LO-BCQ and MX decompose a given operand tensor into block arrays and each block array into blocks. Similar to MX, we find that per-block quantization ($L_b < L_A$) leads to better accuracy due to increased flexibility. While MX achieves this through per-block $1$-bit micro-scales, we associate a dedicated codebook to each block through a per-block codebook selector. Further, MX quantizes the per-block array scale-factor to E8M0 format without per-tensor scaling. In contrast during LO-BCQ, we find that per-tensor scaling combined with quantization of per-block array scale-factor to E4M3 format results in superior inference accuracy across models. 


\end{document}
