Recently, large language models (LLMs) have revolutionaized the landscape of Natural Language Processing (NLP), demonstrating its effectiveness in various domains.
Especially, LLMs become popular in the field of drug discovery, driven by the wealth of literature in the domain.

However, most efforts in the field of drug discovery rely on the costly process of fine-tuning LLMs to domain-specific data. 
That is, rather than using readily available general-purpose LLMs, researchers have found that fine-tuned LLMs yield promising results in domain-specific, knowledge-intensive tasks due to the inherent complexity and specialized nature of the field \cite{lee2020biobert,edwards2024molcap}.
Although it has shown success in molecular property prediction \cite{zhao2023gimlet} and molecule captioning tasks \cite{edwards2022translation}, this expensive fine-tuning approach hinders the field of drug discovery from fully benefiting from the fast-changing landscape of advancements in both AI and the domain itself.
% Despite its success, this expensive fine-tuning approach hinders the field of drug discovery from fully benefiting from the fast-changing landscape of advancements in both AI and the domain itself.

Specifically, from the perspective of the AI field, new models and techniques in LLMs are being published every few months or even weeks \cite{minaee2024large,zhao2023survey}. 
While these advancements are driving impressive progress in AI, they make it challenging to fine-tune models for domain-specific applications. 
Furthermore, when new knowledge is discovered in biology, incorporating this newly discovered information into already fine-tuned domain-specific LLMs is not trivial. 
This process may encounter issues such as catastrophic forgetting, where the model loses previously learned information when encountering the new training data \cite{luo2023empirical}.
Therefore, it is crucial to develop AI methods that can generally applied to various drug discovery tasks without expensive fine-tuning.


Instead of an expensive fine-tuning strategy, retrieval-augmented generation (RAG) methods can offer a promising solution by retrieving relevant information from external knowledge bases rather than embedding the knowledge directly into the model parameters \cite{gao2023retrieval}.
While RAG systems have proven effective in general question-answering tasks, the knowledge needed for drug discovery is significantly more intricate and multidisciplinary. 
That is, drug discovery requires retrieving knowledge from multiple sources, yet LLMs face substantial difficulties in integrating scientific and factual information from multiple sources \cite{harris2023large}.
Hence, it is essential to create an AI system capable of handling intricate scientific data more efficiently to advance drug discovery.

% Recent retrieval-augmented generation (RAG) approaches overcome the limitation of embedding knowledge directly into model parameters by retrieving relevant information from external knowledge bases.
% However, the knowledge essential for drug discovery is highly intricate and multidisciplinary, while LLMs are widely recognized for their limitations in integrating comprehensive scientific and factual information \cite{harris2023large}.

% To overcome these issues, collaboration frameworks among multiple LLM agents got a surge of interest in various scientific domains \cite{kim2024mdagents}.
% In this framework, multiple agents collaborate as a team to tackle complex tasks with improved accuracy, embodying the cooperative nature of scientific research. 
% While it has shown promising results in various domains such as medical \cite{kim2024mdagents} and materials science \cite{zhang2024honeycomb}, the effective integration of multiple LLM agents for drug discovery remains largely unexplored.


To achieve this, we introduce~\proposed, a multi-agent collaborative LLM framework designed to utilize complex and multidisciplinary data sources for diverse drug discovery tasks, eliminating the need for costly fine-tuning.
To do so, rather than relying on a single LLM, we propose employing multiple LLM agents that collaborate to address complex drug discovery tasks, with each agent focusing on a specific data source.
More specifically, our framework consists of three specialized teams: the Planning Team, the Molecule Relationship Analysis Team, and the Molecule Understanding Team. 
For a given target molecule, the Planning Team determines which external databases to utilize. 
Following this decision, the Molecular Relationship Analysis Team examines the target molecule by comparing it with other drugs retrieved from knowledge graphs. 
This analysis focuses on the structural similarity and biological relationships between the target molecule and the related drugs. 
Meanwhile, the Molecule Understanding Team focuses on the target molecule itself, analyzing its structure and leveraging captions obtained from the external database.
% On the other hand, the Drug Relation Analyst Team analyzes the target molecule by comparing it to other drugs, which are obtained through the knowledge graphs (KGs).
% Through the process, the Drug Relation Analyst Team analyzes the structural similarity and biological relationship between the target molecule and the drugs.
This framework facilitates the efficient integration of diverse information from multiple sources by emulating the collaborative dynamics of a drug discovery team.
In this paper, we make the following contributions:
\begin{itemize}
    \item We present a multi-agent collaborative framework for drug discovery that supports a range of tasks by utilizing external knowledge with general-purpose LLMs. This approach also offers the advantage of flexible model outputs, facilitating genuine interaction between scientists and agent systems.
    \item In the framework, we propose a novel strategy to analyze the target molecule by comparing it to the relevant drugs from the knowledge graphs, which have shown to be effective in delivering relevant information to the general purpose LLMs.
    \item Extensive experiments across various drug discovery tasks demonstrate the effectiveness of a multi-agent collaborative framework for drug discovery.
\end{itemize}
To the best of our knowledge, this is the first work that proposes a general multi-agent collaboration framework for various drug discovery tasks.