

\subsection{Problem Setup}

% \noindent \textbf{Problem Setup.}
Given a query molecule $g_q$ and a textual prompt describing a task of interest $\mathcal{I}$, we consider the general problem of generating a relevant response  $\mathcal{A}_{g_q}$. For instance, given $g_q=\text{`C1=CC(=C(C=C1CCN)O)O
'}$ and $\mathcal{I}=\text{`\emph{Predict liver toxicity}'}$, our model should be able to generate an answer stating that $\mathcal{A}_{g_q}=\text{`\emph{this molecule does not have liver toxicity concerns}'}$. Such a general QA setup can be adapted to tasks such as multi-class classification, captioning, and set-based predictions. 
 
We assume access to two types of external databases: (1)~molecular annotation databases $\mathcal{C}$, which contain textual annotation about molecules (for example, detailing their functions and properties) and (2)~knowledge graphs (KGs) connecting molecules to other biomedical entities.
% In contrast to molecular annotation databases, KG databases do not directly provide a direct annotation of a molecule, but rather provide a relative description, given the position of a given molecule in the graph. 
In particular, a KG $\mathcal{G}$ is composed of a set of entities $\mathcal{E}$ and a set of relations $\mathcal{R}$ connecting them. 
KG can include various types of entities, such as drugs, proteins, and diseases. 
In this paper, we only assume that molecule (or drug) entities are present in KG, while any other types of entities can exist.
% We do not assume the type of entities in the KG, but our work assumes that molecule (or drug) entities are present.

Additionally, we assume access to pre-trained molecular captioning models that can be used as external tools to complement the external databases. In general, any predictive model on molecules can be considered a captioning model \cite{edwards2022translation,pei2023biot5}, given that its output can be simply represented as text. 

%a molecule caption database $\mathcal{C}$ and a biology knowledge graph $\mathcal{G}$. Together, these account for the majority of molecular databases typically available. 

%The molecule caption database $\mathcal{C}$ consists of a collection of tuples $(g,c)$, where each molecule $g$ is associated with a textual description $c$.

%The knowledge graph $\mathcal{G}$ is composed of a set of entities $\mathcal{E}$ and a set of relation $\mathcal{R}$ connecting these entities. While entities can include various concepts such as drugs, proteins, or diseases, our work assumes that molecule (or drug) entities are present.

%Given any molecule $g$ that is not necessarily in databases $\mathcal{C}$ and $\mathcal{G}$, our goal is to predict specific properties of interest of that molecule.

\subsection{\proposed}

Here, we introduce \proposed, a multi-agent framework for general molecular question-answering that supports multiple drug discovery tasks. Each agent is implemented by an off-the-shelf LLM prompted to elicit a particular behavior.
Our framework is composed of three teams, each composed of several agents: 
the \textbf{Planning Team}, which identifies the most appropriate data sources and overall strategy given the task and the query molecule (Section~\ref{sec: Planning Team}); 
the \textbf{Knowledge Graph (KG) Team}, which retrieves relevant contextual information about the molecule from available KG databases (Section~\ref{sec:KnowledgeGraphTeam});
and the \textbf{Molecular Understanding (MU) Team}, which retrieves and integrates information from molecular annotation databases and external tools for molecule description (Section~\ref{sec: MolecualrUnderstandingTeam}); 
%and the \textbf{Prediction Team} that integrates the retrieved information and produces the final answer (Section~\ref{sec:PredictionTeam}).
Finally, the \textbf{Prediction Agent}  integrates the findings from the MU and KG teams to generate the final answer.
In the following sub-sections, we describe each team in detail. The overall framework is depicted in Figure~\ref{fig:fig1}.

\subsubsection{Planning Team}
\label{sec: Planning Team}

The Planning Team assesses prior knowledge for a given query molecule. The team separately assesses the relevance of the molecular annotations and the knowledge graph using a MolAnn Planner agent and a KG Planner agent. 

%For the molecular annotations, the agent retrieves its molecular annotation $c$ from the molecular databases and decides wether it complementary information from captioning tools are needed. 

%For the knowledge graph, the KG Planner agent first identifies an anchor molecule $g_a$ in the knowledge graph. For a query molecule $g$, the anchor molecule is chosen as the molecule that is the exhibits the highest cosine similarity, computed on an embedding obtained from a pre-trained graph neural network. The 

%and identifies the most related \emph{anchor drug} within the biological knowledge graph $\mathcal{G}$.
%The retrieved caption and anchor drug are then provided to the Caption Evaluator and the KG Evaluator, respectively, to decide which data source to use and which data should be supplemented. This workflow is summarized in Figure \ref{fig:fig1}(a).

%In particular, we assume prior knowledge can exist in two forms: as textual descriptions of the molecules (for example, detailing their functions, properties, and chemical features) in unstructured datasets $\mathcal{C}$, and/or as part of a knowledge graph $\mathcal{G}$ connecting molecules to other biochemical entities (proteins, diseases, etc.).
% The role of the Planning Team is to assess prior knowledge available for the query molecule. In particular, we assume prior knowledge can exist in two forms: as a description of the molecular structure (for example, detailing its functions, properties, and chemical features) in a caption database $\mathcal{C}$, and/or as part of a knowledge graph $\mathcal{G}$ connecting molecular structures to other biochemical entities (proteins, diseases, etc.). 
% The Planning Team consists of a caption evaluator and a knowledge graph (KG) evaluator, aiming to plan which data source to use and which data should be supplemented, as depicted in Figure \ref{fig:fig1} (a).

%For a given query molecule, the team first retrieves its caption $c_t \in \mathcal{C}$ and identifies the most related \emph{anchor drug} within the biological knowledge graph $\mathcal{G}$.
%The retrieved caption and anchor drug are then provided to the Caption Evaluator and the KG Evaluator, respectively, to decide which data source to use and which data should be supplemented. This workflow is summarized in Figure \ref{fig:fig1}(a).
% whether to generate a new caption or utilize the biological knowledge graph $\mathcal{G}$, respectively.

\noindent \textbf{Molecule Annotation (MolAnn) Planner.}
This agent first retrieves annotations for the query molecule, $c_q$, from the annotation database $\mathcal{C}$. While these annotations can provide valuable biochemical knowledge~\cite{yu2024llasmol}, they are often sparse and may lack sufficient details due to the vastness of the chemical space~\cite{lee2024vision}.
Given the vastness of the chemical space, it is also not uncommon for molecules to be completely absent from databases.

To this end, the MolAnn Planner determines whether the retrieved annotations provide enough information for subsequent analyses.
Specifically, given a query molecule $g_q$ and retrieved annotations $c_q$, the agent is invoked as follows:
\begin{equation} 
\small
    o_{\text{MAP}} = \text{MolAnn Planner}(g_{q}, c_{q}).
\end{equation}
$o_{\text{MAP}}$ is a Boolean indicating whether annotations should be complemented with additional information from tools.

%then passed to the Molecular Understanding Team, which may invoke an additional caption generator if the evaluator indicates that more information is required. 
% While the retrieved caption is scientific and trustworthy since it is generated by human experts, due to the large space of the molecule, this caption is very sparse and sometimes uninformative.
% Therefore, if the retrieved caption is insufficient or absent, it calls the caption generator to create a caption $\Tilde{c}_{t}$ for the query molecule, and concatenate it with the retrieved caption, i.e., $c_{t} = c_{t} || \Tilde{c}_{t}$.
% Otherwise, if the retrieved caption is enough, it keeps the original caption $c_{t}$ without calling the caption generator.

\noindent \textbf{Knowledge Graph (KG) Planner.}
In parallel to analyzing the available description for the query molecule, we analyze the relevance of the contextual information present in the KG. 
Unlike previous works in general QA tasks, which primarily rely on identifying exact entity matches within the KG~\cite{baek2023knowledge}, the vast chemical search space and the limited coverage of existing knowledge bases hinder such approaches.

To address this challenge, we propose leveraging the knowledge of drugs that are structurally similar to the query drug, building upon the well-established biochemical principle that structurally similar molecules often exhibit related biological activity \cite{martin2002structurally}. Specifically, we define the \emph{anchor drug} $g_{a}$ as the entity drug that maximizes the cosine similarity between its embedding and that of the query molecule, among the set of all molecules in the KG ($g_{\mathcal{G}}$):
\begin{equation}
    \small
    g_{a} = \underset{g \in g_{\mathcal{G}}}{\text{argmax}} \, \frac{\textit{emb}(g_q) \cdot \textit{emb}(g)} {\|\textit{emb}(g_q)\| \|\textit{emb}(g)\|},
\label{eq: GNN}
\end{equation}
where $\textit{emb}$ is a graph neural network (GNN) pre-trained with 3D geometry~\cite{liu2021pre} that outputs structure-aware molecular embeddings.


Then, the KG Planner agent decides whether to use the KG based on the structural similarity between the query molecule and the retrieved anchor drug.
To do so, we also provide the  Tanimoto similarity\footnote{We provide details on the Tanimoto similarity in Appendix \ref{app: Preliminaries}.}
% instead of cosine similarity as input
to the KG Planner, as this domain-specific metric can be leveraged by  the LLM's reasoning about chemical structural similarity as follows:
\begin{equation} 
\small
    o_{\text{KGP}} = \text{KG Planner}(g_q, g_{a}, s_{q,a}),
\end{equation}
where $s_{q,a}$ is the Tanimoto similarity between the query and anchor molecules.
$o_{\text{KGP}}$ is a Boolean indicating whether the KG should be used for the prediction. %, and \text{FALSE} otherwise. %the Planning Team calls the Drug Relation Analyst (DRA) Team to further analyze the knowledge graph.
%Otherwise, it calls the Molecule Understanding Team to analyze the query molecule without using the knowledge graph.


\subsubsection{Knowledge Graph Team}
\label{sec:KnowledgeGraphTeam}
% The role of this team is to provide relevant contextual information about the query molecule by leveraging the knowledge graph. 
% It consists of the \textbf{Structure Relation Agent (SRA)} and the \textbf{Biological Relation Agent (BRA)}, which aim to write reports on the query molecule by analyzing its relationship with related drugs in the knowledge graph, accounting for structural and graph-based similarity, respectively.
% After the Planning Team decides to use the knowledge graph, the MRA Team begins their work by identifying reference drugs through the use of the knowledge graph structure as shown in Figure \ref{fig:fig1}(b).

This team aims to provide relevant contextual information about the query molecule by leveraging the KG, and it is only called if $o_{\text{KGP}} = \text{TRUE}$. 
It consists of the Drug Relation (DrugRel) Agent and the Biological Relation (BioRel) Agent, both of which generate reports on the query molecule based on different aspects of the KG. 
Specifically, the DrugRel Agent focuses on related drug entities within the KG, primarily leveraging its internal knowledge, 
whereas the BioRel Agent focuses on summarizing and assessing biological relationships between drugs present in the KG.
% relationships and non-drug entities within the graph to uncover additional context. 
%After the Planning Team decides to use the knowledge graph, the MRA Team begins their work by identifying reference drugs through the structure and relations in the knowledge graph, as shown in Figure \ref{fig:fig1}(b).}

\noindent \textbf{Related Drugs Retrieval.}
The typical approach to leveraging a KG for QA tasks involves identifying multiple entities in the query and extracting the subgraph that encompasses those entities~\cite{baek2023knowledge,wen2023mindmap}.
% However, in molecular understanding for applications related to drug discovery tasks, the question often involves only a single entity, which rarely appears as-is in the KG, making it challenging to identify information in the KG relevant to the task. 
However, in molecular understanding for applications related to drug discovery tasks, the question often involves only a single entity, i.e., the query molecule $g_q$, making it challenging to identify information in the KG relevant to the task. 

Here, we introduce a novel approach for extracting relevant information for the query molecule $g_q$ by utilizing the retrieved anchor drug $g_a$, which exhibits high structural similarity to the query molecule.
In particular, while the drug entities in the KG $\mathcal{G}$ are mainly connected to other types of biological entities (e.g., proteins, diseases), we can infer relationships among drugs by considering the biological entities they share. % define the relationship between the drugs through the biological entity that the two drugs share concurrently.
For example, we can determine the relatedness of the drugs Trastuzumab and Lapatinib by observing their connectivity to the protein HER2 in the KG, as both drugs specifically target and inhibit HER2 to treat HER2-positive breast cancer~\cite{de2014lapatinib}.
Therefore, to identify relevant related drugs, we first compute the 2-hop paths connecting the anchor drug $g_{a}$ to other drugs $g_{\mathcal{G}}^{i}$ in the KG $\mathcal{G}$, i.e., $(g_{a}, r_{a \rightarrow e}, e, r_{i \rightarrow e}, g_{\mathcal{G}}^{i})$, where $r \in \mathcal{R}$, $e \in \mathcal{E}$, and $i$ denotes the index of the other drug.
Then, we select the top-$k$ \emph{related drugs}, denoted as $g_{r^{1}}, \ldots, g_{r^{k}}$, corresponding to the molecules that have the greatest number of 2-hop paths to the anchor drug.
Note that while the anchor drug $g_{a}$ is selected based on its structural similarity to the query molecule $g_{q}$, these reference drugs are biologically related to $g_{a}$, reflecting the relationships captured within the KG.

\noindent \textbf{Drug Relation (DrugRel) Agent.}
The DrugRel Agent generates a report on the query molecule, contextualizing it in relation to relevant drugs present in the knowledge base for the specific task instruction.
Given a query molecule $g_q$, its anchor drug $g_{a}$, and the set of related drugs $g_{r^{1}}, \ldots, g_{r^{k}}$,
%, that are structurally and biologically related to the query drug, 
the DrugRel Agent is defined as follows: %with information on these relationships as follows:
\begin{equation} 
\small
    o_{\text{DRA}} = \text{DrugRel Agent}~(g_q, g_{a}, g_{r^{1}}, \ldots, g_{r^{k}}, \mathcal{T}, \mathcal{I}),
\end{equation}
where $\mathcal{T} = \{s_{q,a}, s_{q, r^{1}}, \ldots,  s_{q, r^{k}}\}$ is the set of Tanimoto similarities between the query molecule and the retrieved drugs, and $\mathcal{I}$ is the task instruction.
This allows the agent to leverage its internal knowledge about related drugs while effectively assessing the relatedness of the information to the target molecule based on the Tanimoto similarity.

\noindent \textbf{Biological Relation (BioRel) Agent.}
On the other hand, drugs related to the query molecule in the KG may exhibit limited structural similarity, underscoring the importance of utilizing additional relevant information, such as shared toxicity profiles or interactions with the same target, rather than relying solely on structural resemblance.
Therefore, the BioRel Agent summarizes how the anchor drug and the related drugs are biologically related, integrating additional biochemical information present in the KG.
% Although it might be straightforward to provide all the possible paths $\mathcal{P}$ between the anchor drug $g_{a}$ and the reference drugs $g_{r^{1}}, \ldots, g_{r^{k}}$ into the agent's prompt, we found that some of the connections between drugs are very complex, consisting in up to 300 paths between them.
Specifically, given an anchor drug $g_{a}$, a set of reference drugs $g_{r^{1}}, \ldots, g_{r^{k}}$, the collection of all 2-hop paths $\mathcal{P}$ linking the anchor drug to the reference drugs, and the instruction $\mathcal{I}$, the agent generates the report as follows:
% Therefore, it is crucial to extract and summarize the key paths relevant to the task instruction $\mathcal{I}$ from the set of all paths $\mathcal{P}$ as follows:
% we ask the agent to first summarize all possible paths $\mathcal{P}$ between the drugs as follows:
\begin{equation} 
\small
    % p_{\text{BRA}} &= \text{Template}_{\text{BRA}}(\mathcal{P}, \mathcal{I}, s_{t,a}), \\
    o_{\text{BRA}} = \text{BioRel Agent}(\mathcal{P}, \mathcal{I}, g_q, g_{a}, s_{q,a}).
    \vspace{-5pt}
\end{equation}
This enables us to obtain a task-relevant summary of the subgraph connected to the anchor drug. % within the knowledge graph, where the nodes are the anchor drug and the reference drugs, formatted in human-readable language.

Importantly, while both the DrugRel Agent and BioRel Agent aim to reason about the query molecule in relation to other relevant drugs in the KG for the specific task, they leverage distinct knowledge sources and perform different roles. 
% Specifically, the biological relation agent primarily relies on external knowledge, utilizing the connectivity of drugs within the knowledge graph. Its main role is to interpret and summarize such knowledge, linking it to the specific task.
% In contrast, the structural relation agent primarily draws on its internal knowledge triggered by the names of the related drugs in the knowledge base, also accounting for their similarity.
% We observe that these agents effectively complement one another, demonstrating a synergistic effect when combined together.
Specifically, the BioRel Agent primarily leverages the connectivity between drugs and other biological entities in the KG, focusing on interpreting and summarizing this network of relationships and aligning it with the specific task at hand. 
In contrast, the DrugRel Agent primarily draws on its internal knowledge, triggered by the names of the related drug entities in the KG, and incorporates structural similarity between them. 
In Section \ref{sec: Experiments}, we demonstrate how these agents complement each other effectively, producing a synergistic effect when combined together.


\subsubsection{Molecular Understanding Team}
\label{sec: MolecualrUnderstandingTeam}

While the KG Team compiles the report by aggregating contextual knowledge, 
the Molecule Understanding (MU) Team focuses primarily on the query molecule itself.
The MU Team is composed of a single  Molecule Understanding (MU) Agent, which aims to write a report on the query molecule by leveraging its structural information, annotations from tools, and reports from other agents.

\noindent \textbf{Molecule Understanding (MU) Agent.}
The MU Agent retrieves annotations for the query molecule, denoted as $c_q$. If the Planning Team decides to use external annotation tools (\emph{i.e.,} $o_{\text{MAP}} = \text{TRUE}$), additional captions $ \Tilde{c}_q$ are generated with the external captioning tools as follows:
\begin{equation} 
\small
    \Tilde{c}_{q} = \text{Captioning Tools}(g_q),
    \vspace{-5pt}
\end{equation}
and concatenated to the annotations retrieved from the database: $c_{q} = c_{q} || \Tilde{c}_{q}$. 
External captioning tools allow the system to easily harness recent advances in LLM-driven molecular understanding~\cite{pei2023biot5,yu2024llasmol}, and can potentially include any specialized tool.

%In the current implementation, \proposed leverages MolT5 \cite{edwards2022translation} as Molecule Captioner.

%The generated caption $\Tilde{c}_{t}$ is concatenated to the retrieved caption, i.e., $c_{t} = c_{t} || \Tilde{c}_{t}$.


%generates a description of the query molecule based on the anno  of the Caption Evaluator in the Planning Team $o_{\text{CE}}$ (Section~\ref{sec: Planning Team}).
%Specifically, if the Caption Evaluator considers the retrieved caption is insufficient or absent, MC generates anew caption as follows:
%\begin{equation} 
%\small
%    \Tilde{c}_{t} = \text{Molecule Captioner}(g_{t}).
%\end{equation}
%The generated caption $\Tilde{c}_{t}$ is concatenated to the retrieved caption, i.e., $c_{t} = c_{t} || \Tilde{c}_{t}$.
% Note that both the generated and retrieved captions are in text format, allowing them to be easily combined into a single caption through concatenation.
%Otherwise, if the Caption Evaluator considers the retrieved caption is enough, it keeps the original caption $c_{t}$ without calling the Molecule Captioner.
%Importantly, including the Molecule Captioner allows the system to easily harness recent advances in LLM-driven molecular understanding~\cite{pei2023biot5,yu2024llasmol,edwards2024molcap}, since any of these recently proposed methods for molecular captioning can be seamlessly integrated as a plug-in replacement for this agent. In the current implementation, \proposed leverages MolT5 \cite{edwards2022translation} as Molecule Captioner.


%\noindent \textbf{query molecule Agent.}
The agent then analyzes the structure of the molecule, contextualizing it with reports generated by the other agents. 
Using the SMILES representation, the caption of the query molecule, and the reports from the KG Team, it compiles a comprehensive molecular annotation report as follows:
\begin{equation} 
    % p_{\text{TMA}} &= \text{Template}_{\text{TMA}}(g_{t}, c_{t}, o_{\text{SRA}}, o_{\text{BRA}}, \mathcal{I}), \\
\small
    o_{\text{MUA}} = \text{MU Agent}(g_{q}, c_{q}, o_{\text{DRA}}, o_{\text{BRA}}, \mathcal{I}).
    \vspace{-5pt}
\end{equation}
\subsubsection{Prediction Agent}
\label{sec:PredictionAgent}
Finally, the Prediction Agent performs the user-defined task by considering the reports from the various agents, including the MU and KG teams, %structure relation agent, biological relation agent, and molecule agent,
as follows:
\begin{equation}
\small
    % p_{\text{TA}} &= \text{Template}_{\text{TA}}(g_{t}, o_{\text{TMA}}, o_{\text{SRA}}, o_{\text{BRA}}, \mathcal{I}), \\
    \mathcal{A}_{g_q} = \text{Task Agent}(g_q, o_{\text{MUA}}, o_{\text{DRA}}, o_{\text{BRA}}, \mathcal{I}).
    \vspace{-5pt}
\end{equation}
By integrating this evidence, the Prediction Agent can perform a comprehensive analysis of the query molecule. Importantly, the output of the Prediction Agent can be flexibly adjusted based on the specific task requirements. For instance, it can be a descriptive caption, a simple yes/no response for binary classification, or a more complex answer listing the top-k targets associated with the query molecule. Such behavior leverages the zero-shot capabilities of LLMs~\cite{kojima2022large} and does not require additional fine-tuning.
Therefore, a key advantage of \proposed is its flexibility, which enhances scientist-AI interactions. 
% \ehsan{is $\mathcal{I}$ here different that previous equations? if yes, you might use another notation.}