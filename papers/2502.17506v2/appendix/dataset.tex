In this section, we provide further details on the datasets we used in Section \ref{sec: Experiments}.
We provide a summary of data statistics in Table \ref{app tab: data stats}.

\begin{figure}[t]
    \centering
    \includegraphics[width=\linewidth]{figures/data_stats/dataset_distribution.pdf}
    \vspace{-5mm}
    \caption{\textbf{Statistics of 3D mask-text datasets.} We show the total number of scenes, tokens for generated captions. Our \dataname significantly surpasses previous datasets in scale, combining multiple datasets to create the largest 3D mask-text dataset to date.}
    \label{fig:data_stats}
\end{figure}


\subsection{Drug Toxicity Prediction Task}
For the drug toxicity prediction task, we use four datasets: \textbf{hERG}, \textbf{DILI}, \textbf{Skin}, and \textbf{Carcinogens}.
\begin{itemize}[leftmargin=.1in]
\item The Human ether-a-go-go related gene (\textbf{hERG}) \cite{wang2016admet} plays a critical role in regulating the heart's rhythm. Thus, accurately predicting hERG liability is essential in drug discovery. In this task, we assess the model's ability to predict whether a drug blocks hERG.
\item Drug-induced liver injury (\textbf{DILI}) \cite{xu2015deep} is a severe liver condition caused by medications. In this task, we evaluate the model's capability to predict whether a drug is likely to cause liver injury.
\item Repeated exposure to a chemical agent can trigger an immune response in inherently susceptible individuals, resulting in \textbf{Skin} \cite{alves2015predicting} sensitization. In this task, we evaluate the model's capability to predict whether the drug induces a skin reaction.
\item A \textbf{Carcinogen} \cite{lagunin2009computer} refers to any substance, radionuclide, or type of radiation that contributes to carcinogenesis, the development of cancer. In this task, we assess the model's ability to predict whether a drug has carcinogenic properties.
\end{itemize}

\subsection{Property-Specific Molecular Captioning Task}
For the property-specific molecular captioning task, we use four datasets in MoleculeNet \cite{wu2018moleculenet}: \textbf{BBBP}, \textbf{Sider}, \textbf{Clintox}, \textbf{BACE}
\begin{itemize}[leftmargin=.1in]
\item The blood-brain barrier penetration \textbf{(BBBP)} dataset consists of compounds categorized by their ability to penetrate the barrier, addressing a significant challenge in developing drugs targeting the central nervous system.
\item The side effect resource \textbf{(Sider)} dataset organizes the side effects of approved drugs into 27 distinct organ system categories.
\item The \textbf{Clintox} dataset includes two classification tasks: 1) predicting toxicity observed during clinical trials, and 2) determining FDA approval status.
\item The \textbf{BACE} dataset provides qualitative binding results for a set of inhibitors aimed at human $\beta$-secretase 1.
\end{itemize}

\subsection{Drug-Target Prediction Task Task}
We rely on annotated molecular targets present in the Drug Repurposing Hub~\cite{corsello2017drug}, DrugBank~\cite{wishart2018drugbank} and STITCH v5.0~\cite{szklarczyk2016stitch}, as combined and preprocessed in \citealp{zheng2023chempert}. 
As we explained in Section \ref{sec: Experiments}, 
we separately report the performance on the test set for molecules based on their information availability in the external databases (``Overlap''/``No Overlap'').
% we separately report the performance on the test set for molecules present/not present in the external databases (``Overlap''/``No Overlap'').
More specifically, for ``No Overlap" cases, we exclude the molecules in the following criteria:
\begin{itemize}[leftmargin=.1in]
\item We exclude the molecules if they exist in the knowledge graph.
\item However, we noticed that many molecules have uninformative annotations, as also discussed in Section \ref{app: Implementation Details}. Consequently, we decided to exclude molecules from the test set only if they have sufficient annotations relevant to the task, as determined by GPT-4o mini.
\end{itemize}
After this process, 5771 molecules remained in the test set for ``No Overlap" scenario.

% While we eliminate the molecules from the test set if they exist in the knowledge graph,
% we noticed that many molecules have uninformative annotations, as also discussed in Section \ref{app: Implementation Details}. 
% Consequently, we decided to exclude molecules from the test set only if they have sufficient annotations relevant to the task, as determined by GPT-4o mini. 