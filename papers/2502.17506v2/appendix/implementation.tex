In this section, we provide further details on the implementation of \proposed.

\subsection{Software Configuration}
Our model is implemented using Python 3.11, PyTorch 2.5.1, Torch-Geometric 2.6.1, RDKit 2023.9.6, and LangGraph 0.2.59.
% The source code for \proposed~is available at~\url{https://anonymous.4open.science/r/CLADD-0281}.

% Our code is publicly available at \textcolor{magenta}{\url{https://github.com/Namkyeong/DrugAgent}}.

\subsection{External Databases}
In all experiments, we employ the PubChem database \cite{kim2021pubchem} as the annotation database $\mathcal{C}$ and PrimeKG \cite{chandak2023building} as the biological knowledge graph $\mathcal{G}$.

The \textbf{PubChem} database is one of the most extensive public molecular databases available.
Pubchem database consists of multiple data sources, including DrugBank, CTD, PharmGKB, and more (\url{https://pubchem.ncbi.nlm.nih.gov/sources/)}.
The PubChem database used in this study includes 299K unique molecules and 336K textual descriptions associated with them (that is, a single molecule can have multiple captions sourced from different datasets associated with it). 
On average, each molecule has 1.115 descriptions, ranging from a minimum of one to a maximum of 17, as shown in Figure \ref{app fig: data analysis pubchem} (a).
In this study, if a molecule had multiple captions, they were concatenated to form a single caption.
On the other hand, as shown in Figure \ref{app fig: data analysis pubchem} (b), most captions consist of fewer than 20 words, underscoring the limited informativeness of human-generated captions. Even after concatenating multiple captions for each molecule, the majority still contain fewer than 50 words.
% Therefore, we propose incorporating an additional Molecule Captioner to supplement the human-generated descriptions.


\textbf{PrimeKG} is a widely used knowledge graph for biochemical research.
The knowledge graph contains 4,037,851 triplets and encompasses 10 entity types, including \{\texttt{anatomy, biological processes, cellular components, diseases, drugs, effects/phenotypes, exposures, genes/proteins, molecular functions, and pathways}\}. 
Additionally, it includes 18 relationship types: \{\texttt{associated with, carrier, contraindication, enzyme, expression absent, expression present, indication, interacts with, linked to, off-label use, parent-child, phenotype absent, phenotype present, ppi, side effect, synergistic interaction, target, and transporter}\}.
The number of triplets associated with each entity and relation type is shown in Figure \ref{app fig: data analysis primekg} (a) and (b), respectively.

\begin{figure*}[t]
    \centering
    \begin{minipage}{0.5\linewidth}
        \centering
        \includegraphics[width=0.95\linewidth]{imgs/data_analysis_pubchem.pdf}
        \caption{Data analysis on PubChem database.}
        \label{app fig: data analysis pubchem}
    \end{minipage}
    \begin{minipage}{0.49\linewidth}
        \centering
        \includegraphics[width=0.95\linewidth]{imgs/data_analysis_primekg.pdf}
        \caption{Data analysis on PrimeKG knowledge graph.}
        \label{app fig: data analysis primekg}
    \end{minipage}
\end{figure*}

\subsection{KG Planner}
In section \ref{sec: Planning Team}, we propose to utilize 3D geometrically pre-trained GNNs to retrieve molecules highly structurally similar to the query molecule.
We use GIN architecture \cite{xu2018powerful}, which is pre-trained with GraphMVP \cite{liu2021pre} approach.
The checkpoint of the model is available at \footnote{\url{https://huggingface.co/chao1224/MoleculeSTM/tree/main/pretrained_GraphMVP}}.

% We use GIN architecture \cite{xu2018powerful} as a molecular encoder in Equation \ref{eq: GNN}, which has been widely used as the backbone model in recent graph self-supervised learning works \cite{hu2019strategies,liu2021pre}.
% Additionally, we initialize the encoder parameters using the GraphMVP \cite{liu2021pre} checkpoints provided by the original authors\footnote{\url{https://huggingface.co/chao1224/MoleculeSTM/tree/main/pretrained_GraphMVP}}.
% The molecule encoder contains a total 1,885,206 number of parameters.



