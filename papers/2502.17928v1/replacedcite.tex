\section{Related Works}
\label{r_w}
\subsection{Source Localization}
As the inverse problem of information propagation on networks, source localization refers to inferring the initial propagation sources given the current diffused observation, such as the states of the specified sensors or a snapshot of the whole network status____. It can be applied to tasks like rumor source identification and finding the origin of rolling blackouts in intelligent power grids____. Early approaches are rule-based and rely on metrics or heuristics derived from the network’s
topology for source identification____. For example, ____ develop a rumor-centrality-based maximum likelihood estimator under the Susceptible-Infected (SI)____ propagation pattern. This kind of method fails to effectively encode topology information. Later, deep learning-based methods devised for capturing the topological effect exhibited in
empirical data have been proposed____. However, most of them fail to model the uncertainty of the location of sources, as the forward propagation process is stochastic. To overcome this, deep generative models have been adopted____. SLVAE____ utilizes the Variational Auto-Encoders~(VAEs) backbone and optimizes the posterior for better prediction. However, it is difficult to converge when the propagation pattern is complicated due to the nature of VAEs. DDMSL____ models the Susceptible-Infected-Recovered~(SIR) ____infection process into the discrete Diffusion Model~(DM)____, and design a reversible residual block based on Graph Convolutional Networks~(GCNs)____. However, it requires additional intermediate propagation data and cannot be generalized to other propagation patterns. Our method demonstrates superior functionality and adaptability for real-world applications, requiring fewer input data while addressing existing limitations, thus offering greater practical value. {We provide a comparison of typical source localization methods in the Appendix~\ref{app:comp}.} 
% can be put back to main content

\subsection{Typical Propagation Models}
Information propagation estimation models information spread in networks and explains propagation sources, with applications in event prediction____, adverse event detection____, and disease spread prediction____. Two main model categories exist: infection models and influence models. Infection models like Susceptible-Infected (SI) and Susceptible-Infected-Susceptible (SIS) manage transitions between susceptible and infected states in networks____. In these models, infected nodes attempt to infect adjacent nodes with probability $\beta$ at each iteration, while in SIS, infected nodes may revert to susceptible with probability $\lambda$. The Susceptible-Infected-Recovered (SIR) model extends this by adding a recovered state.

Independent Cascade (IC) and Linear Threshold (LT)____ are influence models that examine influence spread in social and infrastructure networks. In the IC model, nodes are either active or inactive, starting with initial active nodes. Newly activated nodes get one chance to activate inactive neighbors, with activation probability based on edge weight. The LT model activates inactive nodes when accumulated neighbor influence exceeds a threshold.
% Information propagation estimation involves approximating and reproducing the spread of information in a network and providing explanations based on propagation sources. This task has applications in event prediction____, adverse event detection____, and disease spread prediction____. Models for this purpose fall into two main categories: infection models and influence models. Infection models, such as the Susceptible-Infected (SI) and Susceptible-Infected-Susceptible (SIS), manage transitions between susceptible and infected statuses in networks, offering different switching paths for these changes____. Specifically, every infected node attempts to infect adjacent nodes with probability $\beta$ at each iteration. However, in the SIS model, infected nodes might revert to being susceptible with a certain probability $\lambda$. A more complex case is the Susceptible-Infected-Recovered (SIR) model, which additionally considers the recovered state.

% Independent Cascade (IC) and Linear Threshold (LT)____ are two typical influence models examining how influence spreads in social networks or infrastructure networks. The IC model involves nodes that can either be active or inactive. The process begins with a set of initial active nodes. At each step, any newly activated node can activate its inactive neighbors with a single chance. The probability of activation is dependent on the weight of the edge between nodes. As for the LT model, each inactive node becomes active only if it receives enough influence (over a threshold) from its neighbors.