\documentclass{article} % For LaTeX2e

\usepackage{geometry}
\geometry{a4paper, scale=0.75}
\usepackage{natbib}

\usepackage[utf8]{inputenc} % allow utf-8 input
\usepackage[T1]{fontenc}    % use 8-bit T1 fonts

\usepackage{titletoc}
\usepackage[toc, page, header]{appendix} %%% MAKE SURE TO PUT THIS BEFORE hyperref PACKAGE

\usepackage[colorlinks=true, linkcolor=blue, citecolor=blue,urlcolor=black]{hyperref}
\usepackage{url}            % simple URL typesetting
\usepackage{booktabs}       % professional-quality tables
\usepackage{amsfonts}       % blackboard math symbols
\usepackage{nicefrac}       % compact symbols for 1/2, etc.
\usepackage{microtype}      % microtypography
\usepackage{xcolor}         % colors
\usepackage{footnote}
% \usepackage{microtype}
\usepackage{graphicx}
% \usepackage{subfigure}
\usepackage{subcaption}
\usepackage{booktabs} % for professional tables

% tables
\usepackage{multirow}
\usepackage{colortbl}
\usepackage{arydshln}
\setlength\dashlinedash{0.4pt}
\setlength\dashlinegap{2pt}
\setlength\arrayrulewidth{0.4pt}

\usepackage{amsmath}
\usepackage{amsthm}
\usepackage{amssymb}
\usepackage{mathtools}

\usepackage{comment}
\usepackage{bm}
\usepackage{bbm}
\usepackage{enumitem}
\setitemize{leftmargin=*}
\setenumerate{leftmargin=*}

% Attempt to make hyperref and algorithmic work together better:
\newcommand{\theHalgorithm}{\arabic{algorithm}}
\newcommand{\alglinelabel}{%
  \addtocounter{ALC@line}{-1}% Reduce line counter by 1
  \refstepcounter{ALC@line}% Increment line counter with reference capability
  \label% Regular \label
}
\usepackage{algorithm}
\usepackage[noend]{algorithmic}
\renewcommand{\algorithmiccomment}[1]{ \hfill $\triangleright$ {\color{blue} #1}}
\renewcommand{\algorithmicrequire}{\textbf{Input:}}
\renewcommand{\algorithmicensure}{\textbf{Output:}}

\allowdisplaybreaks
\renewcommand\topfraction{0.85}
\renewcommand\bottomfraction{0.85}
\renewcommand\textfraction{0.1}
\renewcommand\floatpagefraction{0.85}

\renewcommand{\tilde}{\widetilde}
\renewcommand{\hat}{\widehat}
\renewcommand{\bar}{\overline}
\renewcommand{\O}{\operatorname{\mathcal O}}
\newcommand{\Otil}{\operatorname{\tilde{\mathcal O}}}

\usepackage{cleveref}
\newtheorem{theorem}{Theorem}
\newtheorem{lemma}[theorem]{Lemma}
\newtheorem{proposition}[theorem]{Proposition}
\newtheorem{corollary}[theorem]{Corollary}
\newtheorem{definition}[theorem]{Definition}
\newtheorem{assumption}[theorem]{Assumption}
\newtheorem{remark}[theorem]{Remark}

\Crefname{ALC@line}{Line}{Lines}
\Crefname{assumption}{Assumption}{Assumptions}
\Crefformat{equation}{Eq. #2(#1)#3}
\Crefrangeformat{equation}{Eqs. #3(#1)#4 to #5(#2)#6}
\Crefmultiformat{equation}{Eqs. #2(#1)#3}{ and #2(#1)#3}{, #2(#1)#3}{ and #2(#1)#3}
\Crefrangemultiformat{equation}{Eqs. #3(#1)#4 to #5(#2)#6}{ and #3(#1)#4 to #5(#2)#6}{, #3(#1)#4 to #5(#2)#6}{ and #3(#1)#4 to #5(#2)#6}
\newcommand{\creflastconjunction}{, and\nobreakspace}

\newenvironment{tightcenter}{%
\setlength\topsep{2pt}
\setlength\parskip{0pt}
\begin{center}
}{%
\end{center}
}

% \newcommand{\longbo}[1]{{\color{red}  [\text{Longbo:} #1]}}
% \newcommand{\daiyan}[1]{{\color{violet} [\text{Yan:} #1]}}
% \newcommand{\yu}[1]{{\color{cyan}  [\text{Yu:} #1]}}
% \newcommand{\jiatai}[1]{{\color{magenta}  [\text{Jiatai:} #1]}}

%%%%% NEW MATH DEFINITIONS %%%%%

\usepackage{amsmath,amsfonts,bm}
\usepackage{derivative}
% Mark sections of captions for referring to divisions of figures
\newcommand{\figleft}{{\em (Left)}}
\newcommand{\figcenter}{{\em (Center)}}
\newcommand{\figright}{{\em (Right)}}
\newcommand{\figtop}{{\em (Top)}}
\newcommand{\figbottom}{{\em (Bottom)}}
\newcommand{\captiona}{{\em (a)}}
\newcommand{\captionb}{{\em (b)}}
\newcommand{\captionc}{{\em (c)}}
\newcommand{\captiond}{{\em (d)}}

% Highlight a newly defined term
\newcommand{\newterm}[1]{{\bf #1}}

% Derivative d 
\newcommand{\deriv}{{\mathrm{d}}}

% Figure reference, lower-case.
\def\figref#1{figure~\ref{#1}}
% Figure reference, capital. For start of sentence
\def\Figref#1{Figure~\ref{#1}}
\def\twofigref#1#2{figures \ref{#1} and \ref{#2}}
\def\quadfigref#1#2#3#4{figures \ref{#1}, \ref{#2}, \ref{#3} and \ref{#4}}
% Section reference, lower-case.
\def\secref#1{section~\ref{#1}}
% Section reference, capital.
\def\Secref#1{Section~\ref{#1}}
% Reference to two sections.
\def\twosecrefs#1#2{sections \ref{#1} and \ref{#2}}
% Reference to three sections.
\def\secrefs#1#2#3{sections \ref{#1}, \ref{#2} and \ref{#3}}
% Reference to an equation, lower-case.
\def\eqref#1{equation~\ref{#1}}
% Reference to an equation, upper case
\def\Eqref#1{Equation~\ref{#1}}
% A raw reference to an equation---avoid using if possible
\def\plaineqref#1{\ref{#1}}
% Reference to a chapter, lower-case.
\def\chapref#1{chapter~\ref{#1}}
% Reference to an equation, upper case.
\def\Chapref#1{Chapter~\ref{#1}}
% Reference to a range of chapters
\def\rangechapref#1#2{chapters\ref{#1}--\ref{#2}}
% Reference to an algorithm, lower-case.
\def\algref#1{algorithm~\ref{#1}}
% Reference to an algorithm, upper case.
\def\Algref#1{Algorithm~\ref{#1}}
\def\twoalgref#1#2{algorithms \ref{#1} and \ref{#2}}
\def\Twoalgref#1#2{Algorithms \ref{#1} and \ref{#2}}
% Reference to a part, lower case
\def\partref#1{part~\ref{#1}}
% Reference to a part, upper case
\def\Partref#1{Part~\ref{#1}}
\def\twopartref#1#2{parts \ref{#1} and \ref{#2}}

\def\ceil#1{\lceil #1 \rceil}
\def\floor#1{\lfloor #1 \rfloor}
\def\1{\bm{1}}
\newcommand{\train}{\mathcal{D}}
\newcommand{\valid}{\mathcal{D_{\mathrm{valid}}}}
\newcommand{\test}{\mathcal{D_{\mathrm{test}}}}

\def\eps{{\epsilon}}


% Random variables
\def\reta{{\textnormal{$\eta$}}}
\def\ra{{\textnormal{a}}}
\def\rb{{\textnormal{b}}}
\def\rc{{\textnormal{c}}}
\def\rd{{\textnormal{d}}}
\def\re{{\textnormal{e}}}
\def\rf{{\textnormal{f}}}
\def\rg{{\textnormal{g}}}
\def\rh{{\textnormal{h}}}
\def\ri{{\textnormal{i}}}
\def\rj{{\textnormal{j}}}
\def\rk{{\textnormal{k}}}
\def\rl{{\textnormal{l}}}
% rm is already a command, just don't name any random variables m
\def\rn{{\textnormal{n}}}
\def\ro{{\textnormal{o}}}
\def\rp{{\textnormal{p}}}
\def\rq{{\textnormal{q}}}
\def\rr{{\textnormal{r}}}
\def\rs{{\textnormal{s}}}
\def\rt{{\textnormal{t}}}
\def\ru{{\textnormal{u}}}
\def\rv{{\textnormal{v}}}
\def\rw{{\textnormal{w}}}
\def\rx{{\textnormal{x}}}
\def\ry{{\textnormal{y}}}
\def\rz{{\textnormal{z}}}

% Random vectors
\def\rvepsilon{{\mathbf{\epsilon}}}
\def\rvphi{{\mathbf{\phi}}}
\def\rvtheta{{\mathbf{\theta}}}
\def\rva{{\mathbf{a}}}
\def\rvb{{\mathbf{b}}}
\def\rvc{{\mathbf{c}}}
\def\rvd{{\mathbf{d}}}
\def\rve{{\mathbf{e}}}
\def\rvf{{\mathbf{f}}}
\def\rvg{{\mathbf{g}}}
\def\rvh{{\mathbf{h}}}
\def\rvu{{\mathbf{i}}}
\def\rvj{{\mathbf{j}}}
\def\rvk{{\mathbf{k}}}
\def\rvl{{\mathbf{l}}}
\def\rvm{{\mathbf{m}}}
\def\rvn{{\mathbf{n}}}
\def\rvo{{\mathbf{o}}}
\def\rvp{{\mathbf{p}}}
\def\rvq{{\mathbf{q}}}
\def\rvr{{\mathbf{r}}}
\def\rvs{{\mathbf{s}}}
\def\rvt{{\mathbf{t}}}
\def\rvu{{\mathbf{u}}}
\def\rvv{{\mathbf{v}}}
\def\rvw{{\mathbf{w}}}
\def\rvx{{\mathbf{x}}}
\def\rvy{{\mathbf{y}}}
\def\rvz{{\mathbf{z}}}

% Elements of random vectors
\def\erva{{\textnormal{a}}}
\def\ervb{{\textnormal{b}}}
\def\ervc{{\textnormal{c}}}
\def\ervd{{\textnormal{d}}}
\def\erve{{\textnormal{e}}}
\def\ervf{{\textnormal{f}}}
\def\ervg{{\textnormal{g}}}
\def\ervh{{\textnormal{h}}}
\def\ervi{{\textnormal{i}}}
\def\ervj{{\textnormal{j}}}
\def\ervk{{\textnormal{k}}}
\def\ervl{{\textnormal{l}}}
\def\ervm{{\textnormal{m}}}
\def\ervn{{\textnormal{n}}}
\def\ervo{{\textnormal{o}}}
\def\ervp{{\textnormal{p}}}
\def\ervq{{\textnormal{q}}}
\def\ervr{{\textnormal{r}}}
\def\ervs{{\textnormal{s}}}
\def\ervt{{\textnormal{t}}}
\def\ervu{{\textnormal{u}}}
\def\ervv{{\textnormal{v}}}
\def\ervw{{\textnormal{w}}}
\def\ervx{{\textnormal{x}}}
\def\ervy{{\textnormal{y}}}
\def\ervz{{\textnormal{z}}}

% Random matrices
\def\rmA{{\mathbf{A}}}
\def\rmB{{\mathbf{B}}}
\def\rmC{{\mathbf{C}}}
\def\rmD{{\mathbf{D}}}
\def\rmE{{\mathbf{E}}}
\def\rmF{{\mathbf{F}}}
\def\rmG{{\mathbf{G}}}
\def\rmH{{\mathbf{H}}}
\def\rmI{{\mathbf{I}}}
\def\rmJ{{\mathbf{J}}}
\def\rmK{{\mathbf{K}}}
\def\rmL{{\mathbf{L}}}
\def\rmM{{\mathbf{M}}}
\def\rmN{{\mathbf{N}}}
\def\rmO{{\mathbf{O}}}
\def\rmP{{\mathbf{P}}}
\def\rmQ{{\mathbf{Q}}}
\def\rmR{{\mathbf{R}}}
\def\rmS{{\mathbf{S}}}
\def\rmT{{\mathbf{T}}}
\def\rmU{{\mathbf{U}}}
\def\rmV{{\mathbf{V}}}
\def\rmW{{\mathbf{W}}}
\def\rmX{{\mathbf{X}}}
\def\rmY{{\mathbf{Y}}}
\def\rmZ{{\mathbf{Z}}}

% Elements of random matrices
\def\ermA{{\textnormal{A}}}
\def\ermB{{\textnormal{B}}}
\def\ermC{{\textnormal{C}}}
\def\ermD{{\textnormal{D}}}
\def\ermE{{\textnormal{E}}}
\def\ermF{{\textnormal{F}}}
\def\ermG{{\textnormal{G}}}
\def\ermH{{\textnormal{H}}}
\def\ermI{{\textnormal{I}}}
\def\ermJ{{\textnormal{J}}}
\def\ermK{{\textnormal{K}}}
\def\ermL{{\textnormal{L}}}
\def\ermM{{\textnormal{M}}}
\def\ermN{{\textnormal{N}}}
\def\ermO{{\textnormal{O}}}
\def\ermP{{\textnormal{P}}}
\def\ermQ{{\textnormal{Q}}}
\def\ermR{{\textnormal{R}}}
\def\ermS{{\textnormal{S}}}
\def\ermT{{\textnormal{T}}}
\def\ermU{{\textnormal{U}}}
\def\ermV{{\textnormal{V}}}
\def\ermW{{\textnormal{W}}}
\def\ermX{{\textnormal{X}}}
\def\ermY{{\textnormal{Y}}}
\def\ermZ{{\textnormal{Z}}}

% Vectors
\def\vzero{{\bm{0}}}
\def\vone{{\bm{1}}}
\def\vmu{{\bm{\mu}}}
\def\vtheta{{\bm{\theta}}}
\def\vphi{{\bm{\phi}}}
\def\va{{\bm{a}}}
\def\vb{{\bm{b}}}
\def\vc{{\bm{c}}}
\def\vd{{\bm{d}}}
\def\ve{{\bm{e}}}
\def\vf{{\bm{f}}}
\def\vg{{\bm{g}}}
\def\vh{{\bm{h}}}
\def\vi{{\bm{i}}}
\def\vj{{\bm{j}}}
\def\vk{{\bm{k}}}
\def\vl{{\bm{l}}}
\def\vm{{\bm{m}}}
\def\vn{{\bm{n}}}
\def\vo{{\bm{o}}}
\def\vp{{\bm{p}}}
\def\vq{{\bm{q}}}
\def\vr{{\bm{r}}}
\def\vs{{\bm{s}}}
\def\vt{{\bm{t}}}
\def\vu{{\bm{u}}}
\def\vv{{\bm{v}}}
\def\vw{{\bm{w}}}
\def\vx{{\bm{x}}}
\def\vy{{\bm{y}}}
\def\vz{{\bm{z}}}

% Elements of vectors
\def\evalpha{{\alpha}}
\def\evbeta{{\beta}}
\def\evepsilon{{\epsilon}}
\def\evlambda{{\lambda}}
\def\evomega{{\omega}}
\def\evmu{{\mu}}
\def\evpsi{{\psi}}
\def\evsigma{{\sigma}}
\def\evtheta{{\theta}}
\def\eva{{a}}
\def\evb{{b}}
\def\evc{{c}}
\def\evd{{d}}
\def\eve{{e}}
\def\evf{{f}}
\def\evg{{g}}
\def\evh{{h}}
\def\evi{{i}}
\def\evj{{j}}
\def\evk{{k}}
\def\evl{{l}}
\def\evm{{m}}
\def\evn{{n}}
\def\evo{{o}}
\def\evp{{p}}
\def\evq{{q}}
\def\evr{{r}}
\def\evs{{s}}
\def\evt{{t}}
\def\evu{{u}}
\def\evv{{v}}
\def\evw{{w}}
\def\evx{{x}}
\def\evy{{y}}
\def\evz{{z}}

% Matrix
\def\mA{{\bm{A}}}
\def\mB{{\bm{B}}}
\def\mC{{\bm{C}}}
\def\mD{{\bm{D}}}
\def\mE{{\bm{E}}}
\def\mF{{\bm{F}}}
\def\mG{{\bm{G}}}
\def\mH{{\bm{H}}}
\def\mI{{\bm{I}}}
\def\mJ{{\bm{J}}}
\def\mK{{\bm{K}}}
\def\mL{{\bm{L}}}
\def\mM{{\bm{M}}}
\def\mN{{\bm{N}}}
\def\mO{{\bm{O}}}
\def\mP{{\bm{P}}}
\def\mQ{{\bm{Q}}}
\def\mR{{\bm{R}}}
\def\mS{{\bm{S}}}
\def\mT{{\bm{T}}}
\def\mU{{\bm{U}}}
\def\mV{{\bm{V}}}
\def\mW{{\bm{W}}}
\def\mX{{\bm{X}}}
\def\mY{{\bm{Y}}}
\def\mZ{{\bm{Z}}}
\def\mBeta{{\bm{\beta}}}
\def\mPhi{{\bm{\Phi}}}
\def\mLambda{{\bm{\Lambda}}}
\def\mSigma{{\bm{\Sigma}}}

% Tensor
\DeclareMathAlphabet{\mathsfit}{\encodingdefault}{\sfdefault}{m}{sl}
\SetMathAlphabet{\mathsfit}{bold}{\encodingdefault}{\sfdefault}{bx}{n}
\newcommand{\tens}[1]{\bm{\mathsfit{#1}}}
\def\tA{{\tens{A}}}
\def\tB{{\tens{B}}}
\def\tC{{\tens{C}}}
\def\tD{{\tens{D}}}
\def\tE{{\tens{E}}}
\def\tF{{\tens{F}}}
\def\tG{{\tens{G}}}
\def\tH{{\tens{H}}}
\def\tI{{\tens{I}}}
\def\tJ{{\tens{J}}}
\def\tK{{\tens{K}}}
\def\tL{{\tens{L}}}
\def\tM{{\tens{M}}}
\def\tN{{\tens{N}}}
\def\tO{{\tens{O}}}
\def\tP{{\tens{P}}}
\def\tQ{{\tens{Q}}}
\def\tR{{\tens{R}}}
\def\tS{{\tens{S}}}
\def\tT{{\tens{T}}}
\def\tU{{\tens{U}}}
\def\tV{{\tens{V}}}
\def\tW{{\tens{W}}}
\def\tX{{\tens{X}}}
\def\tY{{\tens{Y}}}
\def\tZ{{\tens{Z}}}


% Graph
\def\gA{{\mathcal{A}}}
\def\gB{{\mathcal{B}}}
\def\gC{{\mathcal{C}}}
\def\gD{{\mathcal{D}}}
\def\gE{{\mathcal{E}}}
\def\gF{{\mathcal{F}}}
\def\gG{{\mathcal{G}}}
\def\gH{{\mathcal{H}}}
\def\gI{{\mathcal{I}}}
\def\gJ{{\mathcal{J}}}
\def\gK{{\mathcal{K}}}
\def\gL{{\mathcal{L}}}
\def\gM{{\mathcal{M}}}
\def\gN{{\mathcal{N}}}
\def\gO{{\mathcal{O}}}
\def\gP{{\mathcal{P}}}
\def\gQ{{\mathcal{Q}}}
\def\gR{{\mathcal{R}}}
\def\gS{{\mathcal{S}}}
\def\gT{{\mathcal{T}}}
\def\gU{{\mathcal{U}}}
\def\gV{{\mathcal{V}}}
\def\gW{{\mathcal{W}}}
\def\gX{{\mathcal{X}}}
\def\gY{{\mathcal{Y}}}
\def\gZ{{\mathcal{Z}}}

% Sets
\def\sA{{\mathbb{A}}}
\def\sB{{\mathbb{B}}}
\def\sC{{\mathbb{C}}}
\def\sD{{\mathbb{D}}}
% Don't use a set called E, because this would be the same as our symbol
% for expectation.
\def\sF{{\mathbb{F}}}
\def\sG{{\mathbb{G}}}
\def\sH{{\mathbb{H}}}
\def\sI{{\mathbb{I}}}
\def\sJ{{\mathbb{J}}}
\def\sK{{\mathbb{K}}}
\def\sL{{\mathbb{L}}}
\def\sM{{\mathbb{M}}}
\def\sN{{\mathbb{N}}}
\def\sO{{\mathbb{O}}}
\def\sP{{\mathbb{P}}}
\def\sQ{{\mathbb{Q}}}
\def\sR{{\mathbb{R}}}
\def\sS{{\mathbb{S}}}
\def\sT{{\mathbb{T}}}
\def\sU{{\mathbb{U}}}
\def\sV{{\mathbb{V}}}
\def\sW{{\mathbb{W}}}
\def\sX{{\mathbb{X}}}
\def\sY{{\mathbb{Y}}}
\def\sZ{{\mathbb{Z}}}

% Entries of a matrix
\def\emLambda{{\Lambda}}
\def\emA{{A}}
\def\emB{{B}}
\def\emC{{C}}
\def\emD{{D}}
\def\emE{{E}}
\def\emF{{F}}
\def\emG{{G}}
\def\emH{{H}}
\def\emI{{I}}
\def\emJ{{J}}
\def\emK{{K}}
\def\emL{{L}}
\def\emM{{M}}
\def\emN{{N}}
\def\emO{{O}}
\def\emP{{P}}
\def\emQ{{Q}}
\def\emR{{R}}
\def\emS{{S}}
\def\emT{{T}}
\def\emU{{U}}
\def\emV{{V}}
\def\emW{{W}}
\def\emX{{X}}
\def\emY{{Y}}
\def\emZ{{Z}}
\def\emSigma{{\Sigma}}

% entries of a tensor
% Same font as tensor, without \bm wrapper
\newcommand{\etens}[1]{\mathsfit{#1}}
\def\etLambda{{\etens{\Lambda}}}
\def\etA{{\etens{A}}}
\def\etB{{\etens{B}}}
\def\etC{{\etens{C}}}
\def\etD{{\etens{D}}}
\def\etE{{\etens{E}}}
\def\etF{{\etens{F}}}
\def\etG{{\etens{G}}}
\def\etH{{\etens{H}}}
\def\etI{{\etens{I}}}
\def\etJ{{\etens{J}}}
\def\etK{{\etens{K}}}
\def\etL{{\etens{L}}}
\def\etM{{\etens{M}}}
\def\etN{{\etens{N}}}
\def\etO{{\etens{O}}}
\def\etP{{\etens{P}}}
\def\etQ{{\etens{Q}}}
\def\etR{{\etens{R}}}
\def\etS{{\etens{S}}}
\def\etT{{\etens{T}}}
\def\etU{{\etens{U}}}
\def\etV{{\etens{V}}}
\def\etW{{\etens{W}}}
\def\etX{{\etens{X}}}
\def\etY{{\etens{Y}}}
\def\etZ{{\etens{Z}}}

% The true underlying data generating distribution
\newcommand{\pdata}{p_{\rm{data}}}
\newcommand{\ptarget}{p_{\rm{target}}}
\newcommand{\pprior}{p_{\rm{prior}}}
\newcommand{\pbase}{p_{\rm{base}}}
\newcommand{\pref}{p_{\rm{ref}}}

% The empirical distribution defined by the training set
\newcommand{\ptrain}{\hat{p}_{\rm{data}}}
\newcommand{\Ptrain}{\hat{P}_{\rm{data}}}
% The model distribution
\newcommand{\pmodel}{p_{\rm{model}}}
\newcommand{\Pmodel}{P_{\rm{model}}}
\newcommand{\ptildemodel}{\tilde{p}_{\rm{model}}}
% Stochastic autoencoder distributions
\newcommand{\pencode}{p_{\rm{encoder}}}
\newcommand{\pdecode}{p_{\rm{decoder}}}
\newcommand{\precons}{p_{\rm{reconstruct}}}

\newcommand{\laplace}{\mathrm{Laplace}} % Laplace distribution

\newcommand{\E}{\mathbb{E}}
\newcommand{\Ls}{\mathcal{L}}
\newcommand{\R}{\mathbb{R}}
\newcommand{\emp}{\tilde{p}}
\newcommand{\lr}{\alpha}
\newcommand{\reg}{\lambda}
\newcommand{\rect}{\mathrm{rectifier}}
\newcommand{\softmax}{\mathrm{softmax}}
\newcommand{\sigmoid}{\sigma}
\newcommand{\softplus}{\zeta}
\newcommand{\KL}{D_{\mathrm{KL}}}
\newcommand{\Var}{\mathrm{Var}}
\newcommand{\standarderror}{\mathrm{SE}}
\newcommand{\Cov}{\mathrm{Cov}}
% Wolfram Mathworld says $L^2$ is for function spaces and $\ell^2$ is for vectors
% But then they seem to use $L^2$ for vectors throughout the site, and so does
% wikipedia.
\newcommand{\normlzero}{L^0}
\newcommand{\normlone}{L^1}
\newcommand{\normltwo}{L^2}
\newcommand{\normlp}{L^p}
\newcommand{\normmax}{L^\infty}

\newcommand{\parents}{Pa} % See usage in notation.tex. Chosen to match Daphne's book.

\DeclareMathOperator*{\argmax}{arg\,max}
\DeclareMathOperator*{\argmin}{arg\,min}

\DeclareMathOperator{\sign}{sign}
\DeclareMathOperator{\Tr}{Tr}
\let\ab\allowbreak

\newcommand{\EX}{\mathbb{E}}
\newcommand{\pll}{\kern 0.56em/\kern -0.8em /\kern 0.56em} 
\newcommand{\defeq}{\overset{\underset{\triangle}{}}{=}}
\newcommand{\Hesse}{\text{Hess}}
\newcommand{\sgn}{\text{sgn}}
\newcommand{\T}{\text{T}}
\newcommand{\holder}{H{\"o}lder's inequality}
\newcommand{\Span}{\text{span}}
\newcommand{\weakto}{\rightharpoonup}
\newcommand{\lloc}{L_{\text{loc}}}
\newcommand{\supp}{\text{supp}}
\newcommand{\dd}{\textnormal{d}}

\newcommand{\longbo}[1]{\textcolor{red}{[\text{Longbo:} #1]}}
\newcommand{\yu}[1]{{\color{cyan}[\text{Yu:} #1]}}
\newcommand{\rui}[1]{{\color{purple}[\text{Rui:} #1]}}
\newcommand{\yuhaoDone}{{\color{green}  [\text{Done}]}}

\usepackage{pifont}
\newcommand{\cmark}{{\color{green!70!black}\ding{51}}}%
\newcommand{\xmark}{{\color{red!70!black}\ding{55}}}%
\newcommand{\adv}{{\color{green!70!black}\textbf{Adv.}}}%
\newcommand{\stoc}{{\color{red!70!black}\textbf{Only Stoc.}}}%
\newcommand{\advred}{{\color{red!70!black}\textbf{Only Adv.}}}%
\newcommand{\stocgreen}{{\color{green!70!black}\textbf{Stoc.}}}%

\newcommand{\Clip}{{\text{clip}}}
\newcommand{\Skip}{{\text{skip}}}

\title{
Finite-Time Analysis of Discrete-Time Stochastic Interpolants
}

% \usepackage{authblk}
\author{
Yuhao Liu \thanks{IIIS, Tsinghua University. Email: \texttt{liuyuhao21@mails.tsinghua.edu.cn}.}
\and
Yu Chen \thanks{IIIS, Tsinghua University. Email: \texttt{chenyu23@mails.tsinghua.edu.cn}.}
\and
Rui Hu \thanks{IIIS, Tsinghua University. Email: \texttt{hu-r24@mails.tsinghua.edu.cn}.}
\and
Longbo Huang \thanks{IIIS, Tsinghua University. Email: \texttt{longbohuang@tsinghua.edu.cn}. Corresponding author.}
}
\date{}

\newcommand{\fix}{\marginpar{FIX}}
\newcommand{\new}{\marginpar{NEW}}


\begin{document}
\maketitle


\begin{abstract}
The stochastic interpolant framework offers a powerful approach for constructing generative models based on ordinary differential equations (ODEs) or stochastic differential equations (SDEs) to transform arbitrary data distributions. However, prior analyses of this framework have primarily focused on the continuous-time setting, assuming a perfect solution of the underlying equations. In this work, we present the first discrete-time analysis of the stochastic interpolant framework, where we introduce an innovative discrete-time sampler and derive a finite-time upper bound on its distribution estimation error. Our result provides a novel quantification of how different factors, including 
the distance between source and target distributions and estimation accuracy, affect the convergence rate and also offers a new principled way to design efficient schedules for convergence acceleration. Finally, numerical experiments are conducted on the discrete-time sampler to corroborate our theoretical findings. 
\end{abstract}

\section{Introduction}
\label{sec:introduction}
The business processes of organizations are experiencing ever-increasing complexity due to the large amount of data, high number of users, and high-tech devices involved \cite{martin2021pmopportunitieschallenges, beerepoot2023biggestbpmproblems}. This complexity may cause business processes to deviate from normal control flow due to unforeseen and disruptive anomalies \cite{adams2023proceddsriftdetection}. These control-flow anomalies manifest as unknown, skipped, and wrongly-ordered activities in the traces of event logs monitored from the execution of business processes \cite{ko2023adsystematicreview}. For the sake of clarity, let us consider an illustrative example of such anomalies. Figure \ref{FP_ANOMALIES} shows a so-called event log footprint, which captures the control flow relations of four activities of a hypothetical event log. In particular, this footprint captures the control-flow relations between activities \texttt{a}, \texttt{b}, \texttt{c} and \texttt{d}. These are the causal ($\rightarrow$) relation, concurrent ($\parallel$) relation, and other ($\#$) relations such as exclusivity or non-local dependency \cite{aalst2022pmhandbook}. In addition, on the right are six traces, of which five exhibit skipped, wrongly-ordered and unknown control-flow anomalies. For example, $\langle$\texttt{a b d}$\rangle$ has a skipped activity, which is \texttt{c}. Because of this skipped activity, the control-flow relation \texttt{b}$\,\#\,$\texttt{d} is violated, since \texttt{d} directly follows \texttt{b} in the anomalous trace.
\begin{figure}[!t]
\centering
\includegraphics[width=0.9\columnwidth]{images/FP_ANOMALIES.png}
\caption{An example event log footprint with six traces, of which five exhibit control-flow anomalies.}
\label{FP_ANOMALIES}
\end{figure}

\subsection{Control-flow anomaly detection}
Control-flow anomaly detection techniques aim to characterize the normal control flow from event logs and verify whether these deviations occur in new event logs \cite{ko2023adsystematicreview}. To develop control-flow anomaly detection techniques, \revision{process mining} has seen widespread adoption owing to process discovery and \revision{conformance checking}. On the one hand, process discovery is a set of algorithms that encode control-flow relations as a set of model elements and constraints according to a given modeling formalism \cite{aalst2022pmhandbook}; hereafter, we refer to the Petri net, a widespread modeling formalism. On the other hand, \revision{conformance checking} is an explainable set of algorithms that allows linking any deviations with the reference Petri net and providing the fitness measure, namely a measure of how much the Petri net fits the new event log \cite{aalst2022pmhandbook}. Many control-flow anomaly detection techniques based on \revision{conformance checking} (hereafter, \revision{conformance checking}-based techniques) use the fitness measure to determine whether an event log is anomalous \cite{bezerra2009pmad, bezerra2013adlogspais, myers2018icsadpm, pecchia2020applicationfailuresanalysispm}. 

The scientific literature also includes many \revision{conformance checking}-independent techniques for control-flow anomaly detection that combine specific types of trace encodings with machine/deep learning \cite{ko2023adsystematicreview, tavares2023pmtraceencoding}. Whereas these techniques are very effective, their explainability is challenging due to both the type of trace encoding employed and the machine/deep learning model used \cite{rawal2022trustworthyaiadvances,li2023explainablead}. Hence, in the following, we focus on the shortcomings of \revision{conformance checking}-based techniques to investigate whether it is possible to support the development of competitive control-flow anomaly detection techniques while maintaining the explainable nature of \revision{conformance checking}.
\begin{figure}[!t]
\centering
\includegraphics[width=\columnwidth]{images/HIGH_LEVEL_VIEW.png}
\caption{A high-level view of the proposed framework for combining \revision{process mining}-based feature extraction with dimensionality reduction for control-flow anomaly detection.}
\label{HIGH_LEVEL_VIEW}
\end{figure}

\subsection{Shortcomings of \revision{conformance checking}-based techniques}
Unfortunately, the detection effectiveness of \revision{conformance checking}-based techniques is affected by noisy data and low-quality Petri nets, which may be due to human errors in the modeling process or representational bias of process discovery algorithms \cite{bezerra2013adlogspais, pecchia2020applicationfailuresanalysispm, aalst2016pm}. Specifically, on the one hand, noisy data may introduce infrequent and deceptive control-flow relations that may result in inconsistent fitness measures, whereas, on the other hand, checking event logs against a low-quality Petri net could lead to an unreliable distribution of fitness measures. Nonetheless, such Petri nets can still be used as references to obtain insightful information for \revision{process mining}-based feature extraction, supporting the development of competitive and explainable \revision{conformance checking}-based techniques for control-flow anomaly detection despite the problems above. For example, a few works outline that token-based \revision{conformance checking} can be used for \revision{process mining}-based feature extraction to build tabular data and develop effective \revision{conformance checking}-based techniques for control-flow anomaly detection \cite{singh2022lapmsh, debenedictis2023dtadiiot}. However, to the best of our knowledge, the scientific literature lacks a structured proposal for \revision{process mining}-based feature extraction using the state-of-the-art \revision{conformance checking} variant, namely alignment-based \revision{conformance checking}.

\subsection{Contributions}
We propose a novel \revision{process mining}-based feature extraction approach with alignment-based \revision{conformance checking}. This variant aligns the deviating control flow with a reference Petri net; the resulting alignment can be inspected to extract additional statistics such as the number of times a given activity caused mismatches \cite{aalst2022pmhandbook}. We integrate this approach into a flexible and explainable framework for developing techniques for control-flow anomaly detection. The framework combines \revision{process mining}-based feature extraction and dimensionality reduction to handle high-dimensional feature sets, achieve detection effectiveness, and support explainability. Notably, in addition to our proposed \revision{process mining}-based feature extraction approach, the framework allows employing other approaches, enabling a fair comparison of multiple \revision{conformance checking}-based and \revision{conformance checking}-independent techniques for control-flow anomaly detection. Figure \ref{HIGH_LEVEL_VIEW} shows a high-level view of the framework. Business processes are monitored, and event logs obtained from the database of information systems. Subsequently, \revision{process mining}-based feature extraction is applied to these event logs and tabular data input to dimensionality reduction to identify control-flow anomalies. We apply several \revision{conformance checking}-based and \revision{conformance checking}-independent framework techniques to publicly available datasets, simulated data of a case study from railways, and real-world data of a case study from healthcare. We show that the framework techniques implementing our approach outperform the baseline \revision{conformance checking}-based techniques while maintaining the explainable nature of \revision{conformance checking}.

In summary, the contributions of this paper are as follows.
\begin{itemize}
    \item{
        A novel \revision{process mining}-based feature extraction approach to support the development of competitive and explainable \revision{conformance checking}-based techniques for control-flow anomaly detection.
    }
    \item{
        A flexible and explainable framework for developing techniques for control-flow anomaly detection using \revision{process mining}-based feature extraction and dimensionality reduction.
    }
    \item{
        Application to synthetic and real-world datasets of several \revision{conformance checking}-based and \revision{conformance checking}-independent framework techniques, evaluating their detection effectiveness and explainability.
    }
\end{itemize}

The rest of the paper is organized as follows.
\begin{itemize}
    \item Section \ref{sec:related_work} reviews the existing techniques for control-flow anomaly detection, categorizing them into \revision{conformance checking}-based and \revision{conformance checking}-independent techniques.
    \item Section \ref{sec:abccfe} provides the preliminaries of \revision{process mining} to establish the notation used throughout the paper, and delves into the details of the proposed \revision{process mining}-based feature extraction approach with alignment-based \revision{conformance checking}.
    \item Section \ref{sec:framework} describes the framework for developing \revision{conformance checking}-based and \revision{conformance checking}-independent techniques for control-flow anomaly detection that combine \revision{process mining}-based feature extraction and dimensionality reduction.
    \item Section \ref{sec:evaluation} presents the experiments conducted with multiple framework and baseline techniques using data from publicly available datasets and case studies.
    \item Section \ref{sec:conclusions} draws the conclusions and presents future work.
\end{itemize}
% !TEX root =  ../main.tex
\section{Background on causality and abstraction}\label{sec:preliminaries}

This section provides the notation and key concepts related to causal modeling and abstraction theory.

\spara{Notation.} The set of integers from $1$ to $n$ is $[n]$.
The vectors of zeros and ones of size $n$ are $\zeros_n$ and $\ones_n$.
The identity matrix of size $n \times n$ is $\identity_n$. The Frobenius norm is $\frob{\mathbf{A}}$.
The set of positive definite matrices over $\reall^{n\times n}$ is $\pd^n$. The Hadamard product is $\odot$.
Function composition is $\circ$.
The domain of a function is $\dom{\cdot}$ and its kernel $\ker$.
Let $\mathcal{M}(\mathcal{X}^n)$ be the set of Borel measures over $\mathcal{X}^n \subseteq \reall^n$. Given a measure $\mu^n \in \mathcal{M}(\mathcal{X}^n)$ and a measurable map $\varphi^{\V}$, $\mathcal{X}^n \ni \mathbf{x} \overset{\varphi^{\V}}{\longmapsto} \V^\top \mathbf{x} \in \mathcal{X}^m$, we denote by $\varphi^{\V}_{\#}(\mu^n) \coloneqq \mu^n(\varphi^{\V^{-1}}(\mathbf{x}))$ the pushforward measure $\mu^m \in \mathcal{M}(\mathcal{X}^m)$. 


We now present the standard definition of SCM.

\begin{definition}[SCM, \citealp{pearl2009causality}]\label{def:SCM}
A (Markovian) structural causal model (SCM) $\scm^n$ is a tuple $\langle \myendogenous, \myexogenous, \myfunctional, \zeta^\myexogenous \rangle$, where \emph{(i)} $\myendogenous = \{X_1, \ldots, X_n\}$ is a set of $n$ endogenous random variables; \emph{(ii)} $\myexogenous =\{Z_1,\ldots,Z_n\}$ is a set of $n$ exogenous variables; \emph{(iii)} $\myfunctional$ is a set of $n$ functional assignments such that $X_i=f_i(\parents_i, Z_i)$, $\forall \; i \in [n]$, with $ \parents_i \subseteq \myendogenous \setminus \{ X_i\}$; \emph{(iv)} $\zeta^\myexogenous$ is a product probability measure over independent exogenous variables $\zeta^\myexogenous=\prod_{i \in [n]} \zeta^i$, where $\zeta^i=P(Z_i)$. 
\end{definition}
A Markovian SCM induces a directed acyclic graph (DAG) $\mathcal{G}_{\scm^n}$ where the nodes represent the variables $\myendogenous$ and the edges are determined by the structural functions $\myfunctional$; $ \parents_i$ constitutes then the parent set for $X_i$. Furthermore, we can recursively rewrite the set of structural function $\myfunctional$ as a set of mixing functions $\mymixing$ dependent only on the exogenous variables (cf. \cref{app:CA}). A key feature for studying causality is the possibility of defining interventions on the model:
\begin{definition}[Hard intervention, \citealp{pearl2009causality}]\label{def:intervention}
Given SCM $\scm^n = \langle \myendogenous, \myexogenous, \myfunctional, \zeta^\myexogenous \rangle$, a (hard) intervention $\iota = \operatorname{do}(\myendogenous^{\iota} = \mathbf{x}^{\iota})$, $\myendogenous^{\iota}\subseteq \myendogenous$,
is an operator that generates a new post-intervention SCM $\scm^n_\iota = \langle \myendogenous, \myexogenous, \myfunctional_\iota, \zeta^\myexogenous \rangle$ by replacing each function $f_i$ for $X_i\in\myendogenous^{\iota}$ with the constant $x_i^\iota\in \mathbf{x}^\iota$. 
Graphically, an intervention mutilates $\mathcal{G}_{\mathsf{M}^n}$ by removing all the incoming edges of the variables in $\myendogenous^{\iota}$.
\end{definition}

Given multiple SCMs describing the same system at different levels of granularity, CA provides the definition of an $\alpha$-abstraction map to relate these SCMs:
\begin{definition}[$\abst$-abstraction, \citealp{rischel2020category}]\label{def:abstraction}
Given low-level $\mathsf{M}^\ell$ and high-level $\mathsf{M}^h$ SCMs, an $\abst$-abstraction is a triple $\abst = \langle \Rset, \amap, \alphamap{} \rangle$, where \emph{(i)} $\Rset \subseteq \datalow$ is a subset of relevant variables in $\mathsf{M}^\ell$; \emph{(ii)} $\amap: \Rset \rightarrow \datahigh$ is a surjective function between the relevant variables of $\mathsf{M}^\ell$ and the endogenous variables of $\mathsf{M}^h$; \emph{(iii)} $\alphamap{}: \dom{\Rset} \rightarrow \dom{\datahigh}$ is a modular function $\alphamap{} = \bigotimes_{i\in[n]} \alphamap{X^h_i}$ made up by surjective functions $\alphamap{X^h_i}: \dom{\amap^{-1}(X^h_i)} \rightarrow \dom{X^h_i}$ from the outcome of low-level variables $\amap^{-1}(X^h_i) \in \datalow$ onto outcomes of the high-level variables $X^h_i \in \datahigh$.
\end{definition}
Notice that an $\abst$-abstraction simultaneously maps variables via the function $\amap$ and values through the function $\alphamap{}$. The definition itself does not place any constraint on these functions, although a common requirement in the literature is for the abstraction to satisfy \emph{interventional consistency} \cite{rubenstein2017causal,rischel2020category,beckers2019abstracting}. An important class of such well-behaved abstractions is \emph{constructive linear abstraction}, for which the following properties hold. By constructivity, \emph{(i)} $\abst$ is interventionally consistent; \emph{(ii)} all low-level variables are relevant $\Rset=\datalow$; \emph{(iii)} in addition to the map $\alphamap{}$ between endogenous variables, there exists a map ${\alphamap{}}_U$ between exogenous variables satisfying interventional consistency \cite{beckers2019abstracting,schooltink2024aligning}. By linearity, $\alphamap{} = \V^\top \in \reall^{h \times \ell}$ \cite{massidda2024learningcausalabstractionslinear}. \cref{app:CA} provides formal definitions for interventional consistency, linear and constructive abstraction.

\begin{table*}[t]
\centering
\fontsize{11pt}{11pt}\selectfont
\begin{tabular}{lllllllllllll}
\toprule
\multicolumn{1}{c}{\textbf{task}} & \multicolumn{2}{c}{\textbf{Mir}} & \multicolumn{2}{c}{\textbf{Lai}} & \multicolumn{2}{c}{\textbf{Ziegen.}} & \multicolumn{2}{c}{\textbf{Cao}} & \multicolumn{2}{c}{\textbf{Alva-Man.}} & \multicolumn{1}{c}{\textbf{avg.}} & \textbf{\begin{tabular}[c]{@{}l@{}}avg.\\ rank\end{tabular}} \\
\multicolumn{1}{c}{\textbf{metrics}} & \multicolumn{1}{c}{\textbf{cor.}} & \multicolumn{1}{c}{\textbf{p-v.}} & \multicolumn{1}{c}{\textbf{cor.}} & \multicolumn{1}{c}{\textbf{p-v.}} & \multicolumn{1}{c}{\textbf{cor.}} & \multicolumn{1}{c}{\textbf{p-v.}} & \multicolumn{1}{c}{\textbf{cor.}} & \multicolumn{1}{c}{\textbf{p-v.}} & \multicolumn{1}{c}{\textbf{cor.}} & \multicolumn{1}{c}{\textbf{p-v.}} &  &  \\ \midrule
\textbf{S-Bleu} & 0.50 & 0.0 & 0.47 & 0.0 & 0.59 & 0.0 & 0.58 & 0.0 & 0.68 & 0.0 & 0.57 & 5.8 \\
\textbf{R-Bleu} & -- & -- & 0.27 & 0.0 & 0.30 & 0.0 & -- & -- & -- & -- & - &  \\
\textbf{S-Meteor} & 0.49 & 0.0 & 0.48 & 0.0 & 0.61 & 0.0 & 0.57 & 0.0 & 0.64 & 0.0 & 0.56 & 6.1 \\
\textbf{R-Meteor} & -- & -- & 0.34 & 0.0 & 0.26 & 0.0 & -- & -- & -- & -- & - &  \\
\textbf{S-Bertscore} & \textbf{0.53} & 0.0 & {\ul 0.80} & 0.0 & \textbf{0.70} & 0.0 & {\ul 0.66} & 0.0 & {\ul0.78} & 0.0 & \textbf{0.69} & \textbf{1.7} \\
\textbf{R-Bertscore} & -- & -- & 0.51 & 0.0 & 0.38 & 0.0 & -- & -- & -- & -- & - &  \\
\textbf{S-Bleurt} & {\ul 0.52} & 0.0 & {\ul 0.80} & 0.0 & 0.60 & 0.0 & \textbf{0.70} & 0.0 & \textbf{0.80} & 0.0 & {\ul 0.68} & {\ul 2.3} \\
\textbf{R-Bleurt} & -- & -- & 0.59 & 0.0 & -0.05 & 0.13 & -- & -- & -- & -- & - &  \\
\textbf{S-Cosine} & 0.51 & 0.0 & 0.69 & 0.0 & {\ul 0.62} & 0.0 & 0.61 & 0.0 & 0.65 & 0.0 & 0.62 & 4.4 \\
\textbf{R-Cosine} & -- & -- & 0.40 & 0.0 & 0.29 & 0.0 & -- & -- & -- & -- & - & \\ \midrule
\textbf{QuestEval} & 0.23 & 0.0 & 0.25 & 0.0 & 0.49 & 0.0 & 0.47 & 0.0 & 0.62 & 0.0 & 0.41 & 9.0 \\
\textbf{LLaMa3} & 0.36 & 0.0 & \textbf{0.84} & 0.0 & {\ul{0.62}} & 0.0 & 0.61 & 0.0 &  0.76 & 0.0 & 0.64 & 3.6 \\
\textbf{our (3b)} & 0.49 & 0.0 & 0.73 & 0.0 & 0.54 & 0.0 & 0.53 & 0.0 & 0.7 & 0.0 & 0.60 & 5.8 \\
\textbf{our (8b)} & 0.48 & 0.0 & 0.73 & 0.0 & 0.52 & 0.0 & 0.53 & 0.0 & 0.7 & 0.0 & 0.59 & 6.3 \\  \bottomrule
\end{tabular}
\caption{Pearson correlation on human evaluation on system output. `R-': reference-based. `S-': source-based.}
\label{tab:sys}
\end{table*}



\begin{table}%[]
\centering
\fontsize{11pt}{11pt}\selectfont
\begin{tabular}{llllll}
\toprule
\multicolumn{1}{c}{\textbf{task}} & \multicolumn{1}{c}{\textbf{Lai}} & \multicolumn{1}{c}{\textbf{Zei.}} & \multicolumn{1}{c}{\textbf{Scia.}} & \textbf{} & \textbf{} \\ 
\multicolumn{1}{c}{\textbf{metrics}} & \multicolumn{1}{c}{\textbf{cor.}} & \multicolumn{1}{c}{\textbf{cor.}} & \multicolumn{1}{c}{\textbf{cor.}} & \textbf{avg.} & \textbf{\begin{tabular}[c]{@{}l@{}}avg.\\ rank\end{tabular}} \\ \midrule
\textbf{S-Bleu} & 0.40 & 0.40 & 0.19* & 0.33 & 7.67 \\
\textbf{S-Meteor} & 0.41 & 0.42 & 0.16* & 0.33 & 7.33 \\
\textbf{S-BertS.} & {\ul0.58} & 0.47 & 0.31 & 0.45 & 3.67 \\
\textbf{S-Bleurt} & 0.45 & {\ul 0.54} & {\ul 0.37} & 0.45 & {\ul 3.33} \\
\textbf{S-Cosine} & 0.56 & 0.52 & 0.3 & {\ul 0.46} & {\ul 3.33} \\ \midrule
\textbf{QuestE.} & 0.27 & 0.35 & 0.06* & 0.23 & 9.00 \\
\textbf{LlaMA3} & \textbf{0.6} & \textbf{0.67} & \textbf{0.51} & \textbf{0.59} & \textbf{1.0} \\
\textbf{Our (3b)} & 0.51 & 0.49 & 0.23* & 0.39 & 4.83 \\
\textbf{Our (8b)} & 0.52 & 0.49 & 0.22* & 0.43 & 4.83 \\ \bottomrule
\end{tabular}
\caption{Pearson correlation on human ratings on reference output. *not significant; we cannot reject the null hypothesis of zero correlation}
\label{tab:ref}
\end{table}


\begin{table*}%[]
\centering
\fontsize{11pt}{11pt}\selectfont
\begin{tabular}{lllllllll}
\toprule
\textbf{task} & \multicolumn{1}{c}{\textbf{ALL}} & \multicolumn{1}{c}{\textbf{sentiment}} & \multicolumn{1}{c}{\textbf{detoxify}} & \multicolumn{1}{c}{\textbf{catchy}} & \multicolumn{1}{c}{\textbf{polite}} & \multicolumn{1}{c}{\textbf{persuasive}} & \multicolumn{1}{c}{\textbf{formal}} & \textbf{\begin{tabular}[c]{@{}l@{}}avg. \\ rank\end{tabular}} \\
\textbf{metrics} & \multicolumn{1}{c}{\textbf{cor.}} & \multicolumn{1}{c}{\textbf{cor.}} & \multicolumn{1}{c}{\textbf{cor.}} & \multicolumn{1}{c}{\textbf{cor.}} & \multicolumn{1}{c}{\textbf{cor.}} & \multicolumn{1}{c}{\textbf{cor.}} & \multicolumn{1}{c}{\textbf{cor.}} &  \\ \midrule
\textbf{S-Bleu} & -0.17 & -0.82 & -0.45 & -0.12* & -0.1* & -0.05 & -0.21 & 8.42 \\
\textbf{R-Bleu} & - & -0.5 & -0.45 &  &  &  &  &  \\
\textbf{S-Meteor} & -0.07* & -0.55 & -0.4 & -0.01* & 0.1* & -0.16 & -0.04* & 7.67 \\
\textbf{R-Meteor} & - & -0.17* & -0.39 & - & - & - & - & - \\
\textbf{S-BertScore} & 0.11 & -0.38 & -0.07* & -0.17* & 0.28 & 0.12 & 0.25 & 6.0 \\
\textbf{R-BertScore} & - & -0.02* & -0.21* & - & - & - & - & - \\
\textbf{S-Bleurt} & 0.29 & 0.05* & 0.45 & 0.06* & 0.29 & 0.23 & 0.46 & 4.2 \\
\textbf{R-Bleurt} & - &  0.21 & 0.38 & - & - & - & - & - \\
\textbf{S-Cosine} & 0.01* & -0.5 & -0.13* & -0.19* & 0.05* & -0.05* & 0.15* & 7.42 \\
\textbf{R-Cosine} & - & -0.11* & -0.16* & - & - & - & - & - \\ \midrule
\textbf{QuestEval} & 0.21 & {\ul{0.29}} & 0.23 & 0.37 & 0.19* & 0.35 & 0.14* & 4.67 \\
\textbf{LlaMA3} & \textbf{0.82} & \textbf{0.80} & \textbf{0.72} & \textbf{0.84} & \textbf{0.84} & \textbf{0.90} & \textbf{0.88} & \textbf{1.00} \\
\textbf{Our (3b)} & 0.47 & -0.11* & 0.37 & 0.61 & 0.53 & 0.54 & 0.66 & 3.5 \\
\textbf{Our (8b)} & {\ul{0.57}} & 0.09* & {\ul 0.49} & {\ul 0.72} & {\ul 0.64} & {\ul 0.62} & {\ul 0.67} & {\ul 2.17} \\ \bottomrule
\end{tabular}
\caption{Pearson correlation on human ratings on our constructed test set. 'R-': reference-based. 'S-': source-based. *not significant; we cannot reject the null hypothesis of zero correlation}
\label{tab:con}
\end{table*}

\section{Results}
We benchmark the different metrics on the different datasets using correlation to human judgement. For content preservation, we show results split on data with system output, reference output and our constructed test set: we show that the data source for evaluation leads to different conclusions on the metrics. In addition, we examine whether the metrics can rank style transfer systems similar to humans. On style strength, we likewise show correlations between human judgment and zero-shot evaluation approaches. When applicable, we summarize results by reporting the average correlation. And the average ranking of the metric per dataset (by ranking which metric obtains the highest correlation to human judgement per dataset). 

\subsection{Content preservation}
\paragraph{How do data sources affect the conclusion on best metric?}
The conclusions about the metrics' performance change radically depending on whether we use system output data, reference output, or our constructed test set. Ideally, a good metric correlates highly with humans on any data source. Ideally, for meta-evaluation, a metric should correlate consistently across all data sources, but the following shows that the correlations indicate different things, and the conclusion on the best metric should be drawn carefully.

Looking at the metrics correlations with humans on the data source with system output (Table~\ref{tab:sys}), we see a relatively high correlation for many of the metrics on many tasks. The overall best metrics are S-BertScore and S-BLEURT (avg+avg rank). We see no notable difference in our method of using the 3B or 8B model as the backbone.

Examining the average correlations based on data with reference output (Table~\ref{tab:ref}), now the zero-shoot prompting with LlaMA3 70B is the best-performing approach ($0.59$ avg). Tied for second place are source-based cosine embedding ($0.46$ avg), BLEURT ($0.45$ avg) and BertScore ($0.45$ avg). Our method follows on a 5. place: here, the 8b version (($0.43$ avg)) shows a bit stronger results than 3b ($0.39$ avg). The fact that the conclusions change, whether looking at reference or system output, confirms the observations made by \citet{scialom-etal-2021-questeval} on simplicity transfer.   

Now consider the results on our test set (Table~\ref{tab:con}): Several metrics show low or no correlation; we even see a significantly negative correlation for some metrics on ALL (BLEU) and for specific subparts of our test set for BLEU, Meteor, BertScore, Cosine. On the other end, LlaMA3 70B is again performing best, showing strong results ($0.82$ in ALL). The runner-up is now our 8B method, with a gap to the 3B version ($0.57$ vs $0.47$ in ALL). Note our method still shows zero correlation for the sentiment task. After, ranks BLEURT ($0.29$), QuestEval ($0.21$), BertScore ($0.11$), Cosine ($0.01$).  

On our test set, we find that some metrics that correlate relatively well on the other datasets, now exhibit low correlation. Hence, with our test set, we can now support the logical reasoning with data evidence: Evaluation of content preservation for style transfer needs to take the style shift into account. This conclusion could not be drawn using the existing data sources: We hypothesise that for the data with system-based output, successful output happens to be very similar to the source sentence and vice versa, and reference-based output might not contain server mistakes as they are gold references. Thus, none of the existing data sources tests the limits of the metrics.  


\paragraph{How do reference-based metrics compare to source-based ones?} Reference-based metrics show a lower correlation than the source-based counterpart for all metrics on both datasets with ratings on references (Table~\ref{tab:sys}). As discussed previously, reference-based metrics for style transfer have the drawback that many different good solutions on a rewrite might exist and not only one similar to a reference.


\paragraph{How well can the metrics rank the performance of style transfer methods?}
We compare the metrics' ability to judge the best style transfer methods w.r.t. the human annotations: Several of the data sources contain samples from different style transfer systems. In order to use metrics to assess the quality of the style transfer system, metrics should correctly find the best-performing system. Hence, we evaluate whether the metrics for content preservation provide the same system ranking as human evaluators. We take the mean of the score for every output on each system and the mean of the human annotations; we compare the systems using the Kendall's Tau correlation. 

We find only the evaluation using the dataset Mir, Lai, and Ziegen to result in significant correlations, probably because of sparsity in a number of system tests (App.~\ref{app:dataset}). Our method (8b) is the only metric providing a perfect ranking of the style transfer system on the Lai data, and Llama3 70B the only one on the Ziegen data. Results in App.~\ref{app:results}. 


\subsection{Style strength results}
%Evaluating style strengths is a challenging task. 
Llama3 70B shows better overall results than our method. However, our method scores higher than Llama3 70B on 2 out of 6 datasets, but it also exhibits zero correlation on one task (Table~\ref{tab:styleresults}).%More work i s needed on evaluating style strengths. 
 
\begin{table}%[]
\fontsize{11pt}{11pt}\selectfont
\begin{tabular}{lccc}
\toprule
\multicolumn{1}{c}{\textbf{}} & \textbf{LlaMA3} & \textbf{Our (3b)} & \textbf{Our (8b)} \\ \midrule
\textbf{Mir} & 0.46 & 0.54 & \textbf{0.57} \\
\textbf{Lai} & \textbf{0.57} & 0.18 & 0.19 \\
\textbf{Ziegen.} & 0.25 & 0.27 & \textbf{0.32} \\
\textbf{Alva-M.} & \textbf{0.59} & 0.03* & 0.02* \\
\textbf{Scialom} & \textbf{0.62} & 0.45 & 0.44 \\
\textbf{\begin{tabular}[c]{@{}l@{}}Our Test\end{tabular}} & \textbf{0.63} & 0.46 & 0.48 \\ \bottomrule
\end{tabular}
\caption{Style strength: Pearson correlation to human ratings. *not significant; we cannot reject the null hypothesis of zero corelation}
\label{tab:styleresults}
\end{table}

\subsection{Ablation}
We conduct several runs of the methods using LLMs with variations in instructions/prompts (App.~\ref{app:method}). We observe that the lower the correlation on a task, the higher the variation between the different runs. For our method, we only observe low variance between the runs.
None of the variations leads to different conclusions of the meta-evaluation. Results in App.~\ref{app:results}.
% TODO: section name
\section{Schedule Design for Faster Convergence}
\label{sec:instance}

% TODO: highlight the importance, how it affects convergence rate (done)
% prove the the time schedule design for the second \gamma, add another theorem (done?)

%longbo{first give a high level description about what we are doing here. something like the previous theorem provides a general framework, so we want to design optimized time schedule for faster convergence blabla}\yuhaoDone

In \Cref{thm:main}, we provide an upper bound on the KL divergence from the target distribution to the estimated distribution for a general class of SDE-based generative models. Since the bound depends on the choice of latent scale $\gamma(t)$ and schedule $\{t_k\}_{k=0}^N$, we are able to carefully design a time schedule for a given latent scale, thereby achieving a provably bounded error within a minimum number of steps.

% TODO: add more references
Specifically, we consider the common choice of latent scale in stochastic interpolants, $\gamma(t)=\sqrt{at(1-t)}$, which is first introduced in \citet{interpolation}. 
% In fact, the process $x_t=(1-t)x_0+tx_1+\sqrt{2t(1-t)}z$ is a variance-preserving process. 
This choice is equivalent to changing the definition $$x_t=I(t,x_0,x_1)+\gamma(t)z$$
to $$x_t=I(t,x_0,x_1)+\sqrt{a}\dd B_t,$$
where $B_t$ is a standard Brownian bridge process independent of $(x_0,x_1)$. %(We can just write $B_t=W_t-tW_1$ to obtain a Brownian bridge process.) 
For this $\gamma(t)$, we present the following time schedule to optimize the sample complexity.
% Now, for the given choice of $\gamma(t)$, we give the following time schedule to optimize the sample complexity, which is the number of steps required to achieve a specific error bound.

% TODO: intuition behind the schedule design, explain why
\paragraph{Exponentially Decaying Time Schedule} 
% Consider the bound given by \Cref{thm:main}, we can see that for those steps with smaller $\bar{\gamma}_k$, the error terms in the bound are larger. When $\gamma(t)=\sqrt{at(1-t)}$, for $t_k$ close to $0$ or $1$, the error term in the bound is larger. Therefore, to reduce the error bound for a fixed number of steps $N$, we want to take shorter steps when $t$ is close to $0$ or $1$, and longer steps when $t$ is around $0.5$. 

As suggested by \Cref{thm:main}, smaller steps need to be taken in order to balance the error terms. Moreover, to exactly cancel the $\gamma$-terms, we need $h_k=O(\bar{\gamma}_k^2)$ where $\bar{\gamma}$ is defined in \Cref{thm:main}. Hence, we propose an exponentially decaying time schedule inspired by the approach of \citet{dlinear}. %\longbo{is it the same form? "adapted to our setting" make our result look very weak. if it is not the same, you can just say "inspired".}
Specifically, we first select a midpoint $t_M=\frac{1}{2}$. Let $h\in(0,1)$ be a parameter that controls the step size. We then define the time steps as follows:
$$t_{k+1}-t_k=\begin{cases}\frac{1}{2}h(1-h)^{M-k-1},&k<M\\\frac{1}{2}h(1-h)^{k-M},&k\ge M.\end{cases}$$
This leads to $$t_k=\begin{cases}
    \frac{1}{2}(1-h)^{M-k},&k<M\\
    1-\frac{1}{2}(1-h)^{k-M},&k\ge M.
\end{cases}$$
The parameter $h$ determines the overall scale of the step sizes. A smaller $h$ results in a finer discretization of the time interval. 

Let $h_k=t_{k+1}-t_k$ denote the step size at the $k$-th step. We observe that $$h_k=O(h\min\{t_k,1-t_{k+1}\})=O(h\bar{\gamma}_k^2),$$
which satisfies the condition of canceling the $\gamma$-terms.
% where $\bar{\gamma}_k$ is defined in \Cref{thm:main}. 
% This design ensures that smaller step sizes are taken when $\bar{\gamma}_k$ is smaller, which is crucial for mitigating the error contributions from regions with smaller latent scales, as suggested by \Cref{thm:main}. 
Moreover, the total number of steps is given by
$$\begin{aligned}
    N&=O\left(\frac{\log(1/t_0)+\log(1/(1-t_N))}{\log(1/(1-h))}\right)\\
    &=O\left(h^{-1}\log\left(\frac{1}{t_0(1-t_N)}\right)\right).
\end{aligned}$$
%\longbo{can you state the schedule design using a different order: 1. from theorem 4.3, we know that in order to cancel the gamma terms, we need $h_k=O(sth)$. thus, we use the following step size inspired by xxx. 3. we then see that by choosing this step fulfill our purpose}

Now we can provide the following bound:

\begin{proposition}
    Consider the same settings as in \Cref{thm:main}. Suppose $h_k=t_{t+1}-t_k=O(h\bar{\gamma}^2)$, $\epsilon=\Theta(1)$ and $h=O(\frac{1}{d})$. %\longbo{explain what is $\lesssim$ } 
    Then, we have
    $$\begin{aligned}
        \textnormal{KL}(\rho(t_N)\Vert\hat{\rho}(t_N))&\lesssim\varepsilon_{b_F}^2+\textnormal{KL}(\rho(t_0)\Vert\hat{\rho}(t_0))\\
        &+hd\sqrt{\mathbb{E}\Vert x_0-x_1\Vert^4}+Nh^2d^2.
    \end{aligned}$$
    \label{cor:schedule}
\end{proposition}

\Cref{cor:schedule} provides the KL error bound when the step sizes is chosen so that the $\gamma$-terms are canceled.

\begin{corollary}
    Using $\gamma=\sqrt{at(1-t)}$ and the time schedule defined above, suppose that $\textnormal{KL}(\rho(t_0)\Vert\hat{\rho}(t_0))\le\varepsilon^2$ and $\varepsilon^2_{b_F}\le\varepsilon^2$. Furthermore, assume that $\epsilon=\Theta(1)$ and $h=O(\frac{1}{d})$. Then, under the same settings as in \Cref{thm:main}, the total number of steps required to achieve $\textnormal{KL}(\rho(t_N)\Vert\hat{\rho}(t_N))=O(\varepsilon^2)$ is:
    $$\begin{aligned}
        N=O\left\{\frac{1}{\varepsilon^2}\left[\sqrt{\mathbb{E}\Vert x_0-x_1\Vert^4}d\log\left(\frac{1}{t_0(1-t_N)}\right)\right.\right.\\
        \left.\left.+d^2\log^2\left(\frac{1}{t_0(1-t_N)}\right)\right]\right\}.
    \end{aligned}$$
    \label{cor:instant}
\end{corollary}


Corollary \ref{cor:instant} provides the computational complexity of sampling data using the forward SDE. For a fixed error bound $\varepsilon$, the complexity scales proportionally to $\varepsilon^{-2}$. We can further decompose the complexity into distance-related complexity and Gaussian diffusion complexity. 
% \yu{Do not present a single expression here, try to make a name for them, for example, the distance-related error, gaussian diffusion error or something like that.} (done)
Here $O\left(\frac{1}{\varepsilon^2}\sqrt{\mathbb{E}\Vert x_0-x_1\Vert^4}d\log\left(\frac{1}{t_0(1-t_N)}\right)\right)$ is the distance-related complexity representing the number of steps required to achieve a sufficiently small discretization error associated with the velocity function $v(t, x)$. $O\left(\frac{1}{\varepsilon^2}d^2\log^2\left(\frac{1}{t_0(1-t_N)}\right)\right)$ is the Gaussian diffusion complexity  representing the number of steps required to achieve a sufficiently small discretization error associated with the score function $s(t, x)$.

We briefly explain how to obtain this complexity. First, given a desired number of steps $N$, we select $$h=\Theta\left(N^{-1}\log\left(\frac{1}{t_0(1-t_N)}\right)\right)$$
to achieve the specified number of steps. Since $h_k=O(\bar{\gamma}_k^2h)$, we have: 
%\longbo{dont use phrases like this: "by direct calculation,"}
$$\begin{aligned}
    \sum_{k=0}^{N-1}h_k^3\left[M_2+\bar{\gamma}_k^{-6}d^3+\bar{\gamma}_k^{-2}d\sqrt{\mathbb{E}\Vert x_0-x_1\Vert^{8}}\right]\\\le Nh^3d^3+h^2\left(M_2+d\sqrt{\mathbb{E}\Vert x_0-x_1\Vert^8}\right),
\end{aligned}$$
and 
$$\begin{aligned}
    \sum_{k=0}^{N-1}\left(h^2d^2+h_khd\sqrt{\mathbb{E}\Vert x_0-x_1\Vert^4}\right)\\
    \le Nh^2d^2+hd\sqrt{\mathbb{E}\Vert x_0-x_1\Vert^4}.
\end{aligned}$$
By substituting the chosen value of $h$ for the given $N$ into \Cref{thm:main}, we can derive the stated complexity bound.

\paragraph{Comparison to a Uniform Schedule.} 
%Here we want to show the advantages of our schedule. We first compare our schedule to a natural uniform schedule, which satisfies $h_k=\frac{t_N-t_0}{N}\approx\frac{1}{N}$. 
To highlight the benefits of our proposed exponentially decaying time schedule, we compare it with a natural uniform schedule that satisfies $h_k = \frac{t_N - t_0}{N} \approx \frac{1}{N}$. 
We further assume the ideal case where  $\varepsilon_{b_F}^2=0$ and $\rho(t_0)=\hat{\rho}(t_0)$ in our analysis. 

According to \Cref{thm:main}, the error bound for the uniform schedule is given by
$$\begin{aligned}
    &\quad\sum_{k=0}^{N-1}h_k^3(M_2+\bar{\gamma}_k^{-6}d^3+\bar{\gamma}_k^{-2}d\sqrt{\mathbb{E}\Vert x_0-x_1\Vert^8})\\
    &+\sum_{k=0}^{N-1}h_k^2(\bar{\gamma}_k^{-4}d^2+\bar{\gamma}_k^{-2}d\sqrt{\mathbb{E}\Vert x_0-x_1\Vert^4}).
\end{aligned}$$
Since $\bar{\gamma}_k^2=\Theta(\min\{t_k,1-t_{k+1}\})$, and noting that $$\int_{\delta}^{0.5}t^{-p}\dd t=
\begin{cases}
    \Theta(\log(1/\delta)),&p=1\\
    \Theta(\delta^{-(p-1)}),&p>1
\end{cases}$$
for a uniform schedule, the overall error bound becomes:
$$\begin{aligned}
    &\qquad\textnormal{KL}(\rho(t_N)\Vert\hat{\rho}(t_N))\\&=O\left(\frac{1}{N}\left[\sqrt{\mathbb{E}\Vert x_0-x_1\Vert^4}d\log\left(\frac{1}{t_0(1-t_N)}\right)\right.\right.\\
    &\qquad\qquad\qquad\left.\left.+\frac{1}{t_0(1-t_N)}d^2\right]\right).
\end{aligned}$$
Consequently, the complexity of using a uniform schedule is given by 
\begin{eqnarray*}
N&=&O\bigg(\varepsilon^{-2}\bigg[\log\left(\frac{1}{t_0(1-t_N)}\right)d\sqrt{\mathbb{E}\Vert x_0-x_1\Vert^4}\\&& \qquad\qquad\qquad+\frac{1}{t_0(1-t_N)}d^2\bigg]\bigg),
\end{eqnarray*}
which exhibits a higher computational complexity compared to the proposed exponentially decaying schedule.

\paragraph{Comparison to Diffusion Models Results.} By setting $I(t,x_0,x_1)=(1-t)x_0+tx_1$, $\gamma(t)=\sqrt{2t(1-t)}$, $x_0\sim\rho_0=N(0,I_d)$, and assuming that $x_0$ and $x_1$ are independent, the stochastic interpolant reduces to $x_t=\sqrt{1-t^2}\bar{z}+tx_1$ for some $\bar{z}\sim N(0,I_d)$, which fits the diffusion model setting \cite{song2021scorebased}. %\longbo{give a diffusion ref}. 
Assuming that the fourth moment of $\rho_1$ is bounded by a constant (see \Cref{appendix:reduce-to-gaussian} for details), the complexity of our approach simplifies to $$N=O\left(\varepsilon^{-2}d^2\log^2\left(\frac{1}{1-t_N}\right)\right).$$
% Furthermore, $\text{KL}(\rho(t_0)\Vert\rho_0)\lesssim dt_0^2$ (see, e.g., Proposition 4 in \citealt{dlinear}). By selecting $t_0\lesssim\sqrt{\varepsilon^2/d}$ and setting $\hat{\rho}(t_0)=\rho_0=N(0,I_d)$, the assumption $\text{KL}(\rho(t_0)\Vert\rho(t_0))\le\varepsilon^2$ is satisfied.\longbo{I am confused by this two sentences. what are you trying to say?}

For diffusion models with an early stopping time $\delta$, \citet{chen2023improved} established a complexity bound of $\tilde{O}\left(\varepsilon^{-2}d^2\log^2\left(\frac{1}{\delta}\right)\right)$.
By setting $\delta=1-t_N$ in our analysis, we recover the same complexity bound as that obtained for diffusion models. % \longbo{give a ref}. {\color{orange}[the first ref is added now; I have already referenced Chen et al. so idk if there are any other refs to give.]}
% 
While \citet{dlinear} further improves the complexity bound for diffusion models to $\tilde{O}\left(\varepsilon^{-2}d\log^2\left(\frac{1}{\delta}\right)\right)$ by leveraging techniques from stochastic localization, these techniques heavily rely on the Gaussian structure of diffusion models and cannot be directly applied to the more general stochastic interpolant framework.

% TODO: our framework also applies in analyzing other \gamma
% make it a formal corollary
\paragraph{Other Choices of $\gamma(t)$.} In addition to the commonly used $\gamma(t)=\sqrt{at(1-t)}$, our framework can readily be extended to analyze other choices of $\gamma(t)$. In Appendix \ref{appendix:another}, we present an analysis for $\gamma^2(t)=(1-s)^2s$, which is equivalent to the definition in \citet{chen2024forcasting}. We show that the proposed time schedule in Appendix \ref{appendix:another} also outperforms the uniform schedule in terms of computational complexiting the effectiveness of our schedule design, demonstrating the effectiveness of our schedule design. %\longbo{explain why this is an important feature?} 

% [generated by Gemini] This ability to analyze and optimize for different choices of $\gamma(t)$$ is a significant advantage of our framework, as it provides greater flexibility in designing and optimizing the generative process.
%applies for the analyses of other $\gamma$. In 
%For example, Appendix \ref{appendix:another} studies the choice $\gamma^2(t)=(1-s)^2s$, which is equivalent to the definition in \cite{chen2024forcasting}.

% \longbo{experiments? }
\section{Experiments}
\label{sec:experiments}
The experiments are designed to address two key research questions.
First, \textbf{RQ1} evaluates whether the average $L_2$-norm of the counterfactual perturbation vectors ($\overline{||\perturb||}$) decreases as the model overfits the data, thereby providing further empirical validation for our hypothesis.
Second, \textbf{RQ2} evaluates the ability of the proposed counterfactual regularized loss, as defined in (\ref{eq:regularized_loss2}), to mitigate overfitting when compared to existing regularization techniques.

% The experiments are designed to address three key research questions. First, \textbf{RQ1} investigates whether the mean perturbation vector norm decreases as the model overfits the data, aiming to further validate our intuition. Second, \textbf{RQ2} explores whether the mean perturbation vector norm can be effectively leveraged as a regularization term during training, offering insights into its potential role in mitigating overfitting. Finally, \textbf{RQ3} examines whether our counterfactual regularizer enables the model to achieve superior performance compared to existing regularization methods, thus highlighting its practical advantage.

\subsection{Experimental Setup}
\textbf{\textit{Datasets, Models, and Tasks.}}
The experiments are conducted on three datasets: \textit{Water Potability}~\cite{kadiwal2020waterpotability}, \textit{Phomene}~\cite{phomene}, and \textit{CIFAR-10}~\cite{krizhevsky2009learning}. For \textit{Water Potability} and \textit{Phomene}, we randomly select $80\%$ of the samples for the training set, and the remaining $20\%$ for the test set, \textit{CIFAR-10} comes already split. Furthermore, we consider the following models: Logistic Regression, Multi-Layer Perceptron (MLP) with 100 and 30 neurons on each hidden layer, and PreactResNet-18~\cite{he2016cvecvv} as a Convolutional Neural Network (CNN) architecture.
We focus on binary classification tasks and leave the extension to multiclass scenarios for future work. However, for datasets that are inherently multiclass, we transform the problem into a binary classification task by selecting two classes, aligning with our assumption.

\smallskip
\noindent\textbf{\textit{Evaluation Measures.}} To characterize the degree of overfitting, we use the test loss, as it serves as a reliable indicator of the model's generalization capability to unseen data. Additionally, we evaluate the predictive performance of each model using the test accuracy.

\smallskip
\noindent\textbf{\textit{Baselines.}} We compare CF-Reg with the following regularization techniques: L1 (``Lasso''), L2 (``Ridge''), and Dropout.

\smallskip
\noindent\textbf{\textit{Configurations.}}
For each model, we adopt specific configurations as follows.
\begin{itemize}
\item \textit{Logistic Regression:} To induce overfitting in the model, we artificially increase the dimensionality of the data beyond the number of training samples by applying a polynomial feature expansion. This approach ensures that the model has enough capacity to overfit the training data, allowing us to analyze the impact of our counterfactual regularizer. The degree of the polynomial is chosen as the smallest degree that makes the number of features greater than the number of data.
\item \textit{Neural Networks (MLP and CNN):} To take advantage of the closed-form solution for computing the optimal perturbation vector as defined in (\ref{eq:opt-delta}), we use a local linear approximation of the neural network models. Hence, given an instance $\inst_i$, we consider the (optimal) counterfactual not with respect to $\model$ but with respect to:
\begin{equation}
\label{eq:taylor}
    \model^{lin}(\inst) = \model(\inst_i) + \nabla_{\inst}\model(\inst_i)(\inst - \inst_i),
\end{equation}
where $\model^{lin}$ represents the first-order Taylor approximation of $\model$ at $\inst_i$.
Note that this step is unnecessary for Logistic Regression, as it is inherently a linear model.
\end{itemize}

\smallskip
\noindent \textbf{\textit{Implementation Details.}} We run all experiments on a machine equipped with an AMD Ryzen 9 7900 12-Core Processor and an NVIDIA GeForce RTX 4090 GPU. Our implementation is based on the PyTorch Lightning framework. We use stochastic gradient descent as the optimizer with a learning rate of $\eta = 0.001$ and no weight decay. We use a batch size of $128$. The training and test steps are conducted for $6000$ epochs on the \textit{Water Potability} and \textit{Phoneme} datasets, while for the \textit{CIFAR-10} dataset, they are performed for $200$ epochs.
Finally, the contribution $w_i^{\varepsilon}$ of each training point $\inst_i$ is uniformly set as $w_i^{\varepsilon} = 1~\forall i\in \{1,\ldots,m\}$.

The source code implementation for our experiments is available at the following GitHub repository: \url{https://anonymous.4open.science/r/COCE-80B4/README.md} 

\subsection{RQ1: Counterfactual Perturbation vs. Overfitting}
To address \textbf{RQ1}, we analyze the relationship between the test loss and the average $L_2$-norm of the counterfactual perturbation vectors ($\overline{||\perturb||}$) over training epochs.

In particular, Figure~\ref{fig:delta_loss_epochs} depicts the evolution of $\overline{||\perturb||}$ alongside the test loss for an MLP trained \textit{without} regularization on the \textit{Water Potability} dataset. 
\begin{figure}[ht]
    \centering
    \includegraphics[width=0.85\linewidth]{img/delta_loss_epochs.png}
    \caption{The average counterfactual perturbation vector $\overline{||\perturb||}$ (left $y$-axis) and the cross-entropy test loss (right $y$-axis) over training epochs ($x$-axis) for an MLP trained on the \textit{Water Potability} dataset \textit{without} regularization.}
    \label{fig:delta_loss_epochs}
\end{figure}

The plot shows a clear trend as the model starts to overfit the data (evidenced by an increase in test loss). 
Notably, $\overline{||\perturb||}$ begins to decrease, which aligns with the hypothesis that the average distance to the optimal counterfactual example gets smaller as the model's decision boundary becomes increasingly adherent to the training data.

It is worth noting that this trend is heavily influenced by the choice of the counterfactual generator model. In particular, the relationship between $\overline{||\perturb||}$ and the degree of overfitting may become even more pronounced when leveraging more accurate counterfactual generators. However, these models often come at the cost of higher computational complexity, and their exploration is left to future work.

Nonetheless, we expect that $\overline{||\perturb||}$ will eventually stabilize at a plateau, as the average $L_2$-norm of the optimal counterfactual perturbations cannot vanish to zero.

% Additionally, the choice of employing the score-based counterfactual explanation framework to generate counterfactuals was driven to promote computational efficiency.

% Future enhancements to the framework may involve adopting models capable of generating more precise counterfactuals. While such approaches may yield to performance improvements, they are likely to come at the cost of increased computational complexity.


\subsection{RQ2: Counterfactual Regularization Performance}
To answer \textbf{RQ2}, we evaluate the effectiveness of the proposed counterfactual regularization (CF-Reg) by comparing its performance against existing baselines: unregularized training loss (No-Reg), L1 regularization (L1-Reg), L2 regularization (L2-Reg), and Dropout.
Specifically, for each model and dataset combination, Table~\ref{tab:regularization_comparison} presents the mean value and standard deviation of test accuracy achieved by each method across 5 random initialization. 

The table illustrates that our regularization technique consistently delivers better results than existing methods across all evaluated scenarios, except for one case -- i.e., Logistic Regression on the \textit{Phomene} dataset. 
However, this setting exhibits an unusual pattern, as the highest model accuracy is achieved without any regularization. Even in this case, CF-Reg still surpasses other regularization baselines.

From the results above, we derive the following key insights. First, CF-Reg proves to be effective across various model types, ranging from simple linear models (Logistic Regression) to deep architectures like MLPs and CNNs, and across diverse datasets, including both tabular and image data. 
Second, CF-Reg's strong performance on the \textit{Water} dataset with Logistic Regression suggests that its benefits may be more pronounced when applied to simpler models. However, the unexpected outcome on the \textit{Phoneme} dataset calls for further investigation into this phenomenon.


\begin{table*}[h!]
    \centering
    \caption{Mean value and standard deviation of test accuracy across 5 random initializations for different model, dataset, and regularization method. The best results are highlighted in \textbf{bold}.}
    \label{tab:regularization_comparison}
    \begin{tabular}{|c|c|c|c|c|c|c|}
        \hline
        \textbf{Model} & \textbf{Dataset} & \textbf{No-Reg} & \textbf{L1-Reg} & \textbf{L2-Reg} & \textbf{Dropout} & \textbf{CF-Reg (ours)} \\ \hline
        Logistic Regression   & \textit{Water}   & $0.6595 \pm 0.0038$   & $0.6729 \pm 0.0056$   & $0.6756 \pm 0.0046$  & N/A    & $\mathbf{0.6918 \pm 0.0036}$                     \\ \hline
        MLP   & \textit{Water}   & $0.6756 \pm 0.0042$   & $0.6790 \pm 0.0058$   & $0.6790 \pm 0.0023$  & $0.6750 \pm 0.0036$    & $\mathbf{0.6802 \pm 0.0046}$                    \\ \hline
%        MLP   & \textit{Adult}   & $0.8404 \pm 0.0010$   & $\mathbf{0.8495 \pm 0.0007}$   & $0.8489 \pm 0.0014$  & $\mathbf{0.8495 \pm 0.0016}$     & $0.8449 \pm 0.0019$                    \\ \hline
        Logistic Regression   & \textit{Phomene}   & $\mathbf{0.8148 \pm 0.0020}$   & $0.8041 \pm 0.0028$   & $0.7835 \pm 0.0176$  & N/A    & $0.8098 \pm 0.0055$                     \\ \hline
        MLP   & \textit{Phomene}   & $0.8677 \pm 0.0033$   & $0.8374 \pm 0.0080$   & $0.8673 \pm 0.0045$  & $0.8672 \pm 0.0042$     & $\mathbf{0.8718 \pm 0.0040}$                    \\ \hline
        CNN   & \textit{CIFAR-10} & $0.6670 \pm 0.0233$   & $0.6229 \pm 0.0850$   & $0.7348 \pm 0.0365$   & N/A    & $\mathbf{0.7427 \pm 0.0571}$                     \\ \hline
    \end{tabular}
\end{table*}

\begin{table*}[htb!]
    \centering
    \caption{Hyperparameter configurations utilized for the generation of Table \ref{tab:regularization_comparison}. For our regularization the hyperparameters are reported as $\mathbf{\alpha/\beta}$.}
    \label{tab:performance_parameters}
    \begin{tabular}{|c|c|c|c|c|c|c|}
        \hline
        \textbf{Model} & \textbf{Dataset} & \textbf{No-Reg} & \textbf{L1-Reg} & \textbf{L2-Reg} & \textbf{Dropout} & \textbf{CF-Reg (ours)} \\ \hline
        Logistic Regression   & \textit{Water}   & N/A   & $0.0093$   & $0.6927$  & N/A    & $0.3791/1.0355$                     \\ \hline
        MLP   & \textit{Water}   & N/A   & $0.0007$   & $0.0022$  & $0.0002$    & $0.2567/1.9775$                    \\ \hline
        Logistic Regression   &
        \textit{Phomene}   & N/A   & $0.0097$   & $0.7979$  & N/A    & $0.0571/1.8516$                     \\ \hline
        MLP   & \textit{Phomene}   & N/A   & $0.0007$   & $4.24\cdot10^{-5}$  & $0.0015$    & $0.0516/2.2700$                    \\ \hline
       % MLP   & \textit{Adult}   & N/A   & $0.0018$   & $0.0018$  & $0.0601$     & $0.0764/2.2068$                    \\ \hline
        CNN   & \textit{CIFAR-10} & N/A   & $0.0050$   & $0.0864$ & N/A    & $0.3018/
        2.1502$                     \\ \hline
    \end{tabular}
\end{table*}

\begin{table*}[htb!]
    \centering
    \caption{Mean value and standard deviation of training time across 5 different runs. The reported time (in seconds) corresponds to the generation of each entry in Table \ref{tab:regularization_comparison}. Times are }
    \label{tab:times}
    \begin{tabular}{|c|c|c|c|c|c|c|}
        \hline
        \textbf{Model} & \textbf{Dataset} & \textbf{No-Reg} & \textbf{L1-Reg} & \textbf{L2-Reg} & \textbf{Dropout} & \textbf{CF-Reg (ours)} \\ \hline
        Logistic Regression   & \textit{Water}   & $222.98 \pm 1.07$   & $239.94 \pm 2.59$   & $241.60 \pm 1.88$  & N/A    & $251.50 \pm 1.93$                     \\ \hline
        MLP   & \textit{Water}   & $225.71 \pm 3.85$   & $250.13 \pm 4.44$   & $255.78 \pm 2.38$  & $237.83 \pm 3.45$    & $266.48 \pm 3.46$                    \\ \hline
        Logistic Regression   & \textit{Phomene}   & $266.39 \pm 0.82$ & $367.52 \pm 6.85$   & $361.69 \pm 4.04$  & N/A   & $310.48 \pm 0.76$                    \\ \hline
        MLP   &
        \textit{Phomene} & $335.62 \pm 1.77$   & $390.86 \pm 2.11$   & $393.96 \pm 1.95$ & $363.51 \pm 5.07$    & $403.14 \pm 1.92$                     \\ \hline
       % MLP   & \textit{Adult}   & N/A   & $0.0018$   & $0.0018$  & $0.0601$     & $0.0764/2.2068$                    \\ \hline
        CNN   & \textit{CIFAR-10} & $370.09 \pm 0.18$   & $395.71 \pm 0.55$   & $401.38 \pm 0.16$ & N/A    & $1287.8 \pm 0.26$                     \\ \hline
    \end{tabular}
\end{table*}

\subsection{Feasibility of our Method}
A crucial requirement for any regularization technique is that it should impose minimal impact on the overall training process.
In this respect, CF-Reg introduces an overhead that depends on the time required to find the optimal counterfactual example for each training instance. 
As such, the more sophisticated the counterfactual generator model probed during training the higher would be the time required. However, a more advanced counterfactual generator might provide a more effective regularization. We discuss this trade-off in more details in Section~\ref{sec:discussion}.

Table~\ref{tab:times} presents the average training time ($\pm$ standard deviation) for each model and dataset combination listed in Table~\ref{tab:regularization_comparison}.
We can observe that the higher accuracy achieved by CF-Reg using the score-based counterfactual generator comes with only minimal overhead. However, when applied to deep neural networks with many hidden layers, such as \textit{PreactResNet-18}, the forward derivative computation required for the linearization of the network introduces a more noticeable computational cost, explaining the longer training times in the table.

\subsection{Hyperparameter Sensitivity Analysis}
The proposed counterfactual regularization technique relies on two key hyperparameters: $\alpha$ and $\beta$. The former is intrinsic to the loss formulation defined in (\ref{eq:cf-train}), while the latter is closely tied to the choice of the score-based counterfactual explanation method used.

Figure~\ref{fig:test_alpha_beta} illustrates how the test accuracy of an MLP trained on the \textit{Water Potability} dataset changes for different combinations of $\alpha$ and $\beta$.

\begin{figure}[ht]
    \centering
    \includegraphics[width=0.85\linewidth]{img/test_acc_alpha_beta.png}
    \caption{The test accuracy of an MLP trained on the \textit{Water Potability} dataset, evaluated while varying the weight of our counterfactual regularizer ($\alpha$) for different values of $\beta$.}
    \label{fig:test_alpha_beta}
\end{figure}

We observe that, for a fixed $\beta$, increasing the weight of our counterfactual regularizer ($\alpha$) can slightly improve test accuracy until a sudden drop is noticed for $\alpha > 0.1$.
This behavior was expected, as the impact of our penalty, like any regularization term, can be disruptive if not properly controlled.

Moreover, this finding further demonstrates that our regularization method, CF-Reg, is inherently data-driven. Therefore, it requires specific fine-tuning based on the combination of the model and dataset at hand.
\section{Conclusion}
In this work, we propose a simple yet effective approach, called SMILE, for graph few-shot learning with fewer tasks. Specifically, we introduce a novel dual-level mixup strategy, including within-task and across-task mixup, for enriching the diversity of nodes within each task and the diversity of tasks. Also, we incorporate the degree-based prior information to learn expressive node embeddings. Theoretically, we prove that SMILE effectively enhances the model's generalization performance. Empirically, we conduct extensive experiments on multiple benchmarks and the results suggest that SMILE significantly outperforms other baselines, including both in-domain and cross-domain few-shot settings.

\bibliographystyle{apalike}
\bibliography{ref}

\newpage
\appendix
% \renewcommand{\appendixpagename}{\centering \LARGE Supplementary Materials}
\appendixpage

% \begin{itemize}
%     \item \textbf{Appendix \ref{appendix:preliminaries}: Supplementary details for Section \ref{sec:preliminaries}} - This section provides additional details from \cite{interpolation} that were not included in Section \ref{sec:preliminaries}. It includes a formal statement of key equations and their associated conditions. Furthermore, it outlines the optimization objectives for the score and velocity estimators, which are used for model training in Section \ref{sec:experiments}.  %\longbo{explain why we need them}
    
%     \item \textbf{Appendix \ref{appendix:lemmas}: Useful lemmas in bounding derivatives} - This section presents lemmas that are essential for bounding the derivatives of velocity functions and score functions, along with their corresponding proofs. The proofs primarily rely on properties and inequalities related to (conditional) expectations. These lemmas play a crucial role in deriving the overall KL error bounds.
    
%     \item \textbf{Appendix \ref{appendix:overall}: Proofs of results in Section \ref{sec:results} and \ref{sec:instance}} - This section provides the complete proofs for the results presented in Section \ref{sec:results} and Section \ref{sec:instance}. This includes the proof of \Cref{thm:main} (\Cref{appendix:proofofmain}), the proof of \Cref{cor:schedule} and \Cref{cor:instant} (Appendices \ref{appendix:proofofcor},\ref{appendix:proofofschedule}), and additional details regarding the discussions in Section \ref{sec:instance} (Appendix \ref{appendix:another}).

%     \item \textbf{Appendix \ref{appendix:experiments}: More details of numerical experiments} - This section provides omitted details for Section \ref{sec:experiments}, including the parameterization of estimators, choice of $(t_0,t_N)$ and optimizers, and how the TV distance is estimated. We also include additional experiments for $\gamma(t)=\sqrt{(1-t)^2t}$.
% \end{itemize}


\startcontents[section]
\printcontents[section]{l}{1}{\setcounter{tocdepth}{2}}
\newpage


\section*{Notations}
%\yu{Do not use bullets here. Just write a paragraph or make a table. (done)}

We use $\Vert\cdot\Vert$ to denote $\ell_2$ norm for both vectors and matrices. For a matrix $A$, we use $\Vert A\Vert_F=\sqrt{\sum_{ij}A_{ij}^2}$ to denote the Frobenious norm of $A$. We use $\frac{\dd}{\dd u}$, $\frac{\partial}{\partial u}$, or just $\partial_u$ to denote the (partial) derivative with respect to $u$. We use $\nabla$ to denote the gradient or Jacobian, depending on whether the function is scalar-valued or vector-valued. If not specified, for the function in form of $f(t,x)$ where $t$ is a scalar and $x$ is a vector, $\nabla f(t,x)$ means the gradient vector or Jacobian matrix with respect to $x$ rather than $t$. We use $\Delta f(t,x)=\sum_{i=1}^d\frac{\partial^2}{\partial x_i^2}f$ as the Laplace operator. We use $\mathbb{E}[X]$ to denote the expectation of a random variable $X$, and $\text{Cov}(X,Y)$ to denote the covariance of two random variables $X,Y$. $\mathbb{E}[X|c]$ and $\text{Cov}(X,Y|c)$ denote the corresponding conditional expectation and conditional covariance given condition $c$. We use the notation $f(x)\lesssim g(x)$ or $f(x)=O(g(x))$ to denote that there exists a constant $C>0$ such that $f(x)\le Cg(x)$.

% \begin{itemize}
%     \item ($\ell_2$-)norm: $\Vert\cdot\Vert$, Frobenius norm: $\Vert\cdot\Vert_F$.
%     \item We use $\frac{\dd}{\dd u}$, $\frac{\partial}{\partial u}$, or just $\partial_u$ to denote the (partial) derivative. 
%     \item We use $\nabla$ to denote the gradient or Jacobian, depending on whether the function is scalar-valued or vector-valued. If not specified, for the function in form of $f(t,x)$ where $t$ is a scalar and $x$ is a vector, $\nabla f(t,x)$ means the gradient vector or Jacobian matrix with respect to $x$ rather than $t$.
%     \item $\Delta f=\sum_{i=1}^d\frac{\partial^2}{\partial x_i^2}f$ is the Laplace operator.
%     \item We use $\mathbb{E}$ to denote the expectation and $\text{Cov}$ to denote the covariance of two random variables. $\mathbb{E}[\cdot|c]$ and $\text{Cov}[\cdot,\cdot|c]$ denote the corresponding conditional version.
%     \item We use the notation $f(x)\lesssim g(x)$ or $f(x)=O(g(x))$ to hide constant factors.
% \end{itemize}

\section{Supplementary Details for Section \ref{sec:preliminaries}}
\label{appendix:preliminaries}

This part summarizes some of the results from \cite{interpolation} that are not introduced in Section \ref{sec:preliminaries}.

\begin{proposition}
    (\cite{interpolation}, Theorem 2.6, Corollaries 2.10 and 2.18, and their proofs)
    Suppose that the joint measure $\nu$ and the function $I$ satisfies 
    \begin{equation}
        \underset{(x_0,x_1)\sim\nu}{\mathbb{E}}\Vert\partial_tI(t,x_0,x_1)\Vert^4\le M_1<\infty,\quad\underset{(x_0,x_1)\sim\nu}{\mathbb{E}}\Vert\partial_t^2I(t,x_0,x_1)\Vert^2\le M_2<\infty,\quad \forall t\in[0,1].
        \label{eq:assumption1}
    \end{equation}
    Then, $\rho\in C^1((0,1),C^p(\mathbb{R}^d))$, $s\in C^1((0,1),(C^p(\mathbb{R}^d))^d)$ and $b\in C^0((0,1),(C^p(\mathbb{R}^d))^d)$, and both the solution of the probability flow ODE
    $$\frac{\dd}{\dd t}X_t=b(t,X_t), \qquad X_0\sim\rho_0$$
    and the solution of the forward SDE
    $$\dd X_t^F=b_F(t,X_t^F)\dd t+\sqrt{2\epsilon(t)}\dd W_t, \qquad X_0^F\sim\rho_0$$
    have the same marginal densities as $(x_t)_{t\in[0,1]}$. Here $\epsilon\in C[0,1]$ with $\epsilon(t)\ge0$ for all $t\in[0,1]$ and $b_F$ is defined as \begin{equation}
        b_F(t,x)=b(t,x)+\epsilon(t)s(t,x).
        \label{eq:defbf}
    \end{equation}
    
    Moreover, suppose that the densities $\rho_0,\rho_1$ are strictly positive elements of $C^2(\mathbb{R}^d)$, and are such that
    $$\int_{\mathbb{R}^d}\Vert\nabla\log\rho_0(x)\Vert^2\rho_0(x)dx<\infty,\qquad\int_{\mathbb{R}^d}\Vert\nabla\log\rho_0(x)\Vert^2\rho_1(x)dx<\infty.$$
    Then $\rho\in C^1([0,1],C^p(\mathbb{R}^d))$, $s\in C^1([0,1],(C^p(\mathbb{R}^d))^d)$ and $b\in C^0([0,1],(C^p(\mathbb{R}^d))^d)$. The notation is adapted from \cite{interpolation} where $f\in C^1([0,1],C^p(\mathbb{R}^d))$ means that the function $f$ is $C^1$ in $t\in[0,1]$ and $C^p$ in $x\in\mathbb{R}^d$.
    \label{prop:generative-modeling}
\end{proposition}

% Also, by the proof of Theorem 2.6 in \cite{interpolation} and study how the assumptions are used, we can derive the following result when we do not have the boundary assumptions.

% \begin{proposition}
%     (\cite{interpolation}, Appendix B.1)
%     With only the condition (\ref{eq:assumption1}) in Proposition \ref{prop:generative-modeling}, we still have the same result for any subintervals of $(0,1)$ as Proposition \ref{prop:generative-modeling}.
%     \label{prop:generative-modeling2}
% \end{proposition}

The above proposition provides a generative modeling in the form of 
$$\frac{\dd}{\dd t}X_t=b(t,X_t)$$
and
$$\dd X_t^F=b_F(t,X_t^F)\dd t+\sqrt{2\epsilon(t)}\dd W_t.$$

In practice, we need to train an estimator to estimate velocity functions. By the following proposition, we can use the optimization objectives to train the estimators. 

\begin{proposition}
    (\cite{interpolation}, Theorems 2.7 and 2.8)
    $b$ is the unique minimizer of
    $$\mathcal{L}_b[\hat{b}]=\int_0^1\mathbb{E}\left[\frac{1}{2}\Vert\hat{b}(t,x_t)\Vert^2-(\partial_tI(t,x_0,x_1)+\dot{\gamma}(t)z)\cdot\hat{b}(t,x_t)\right]\dd t,$$
    and $s$ is the unique minimizer of
    $$\mathcal{L}_s[\hat{s}]=\int_0^1\mathbb{E}\left[\frac{1}{2}\Vert\hat{s}(t,x_t)\Vert^2+\gamma^{-1}(t)z\cdot\hat{s}(t,x_t)\right]\dd t.$$
    Here the notation ``$\cdot$" represents the inner product of two vectors.
    \label{prop:objectives}
\end{proposition}


\section{Bounding the Velocities and Scores}
\label{appendix:lemmas}

\subsection{Useful Lemmas}

To begin with, we first provide moment bounds on the Gaussian variable $z\sim N(0,I_d)$.

\begin{lemma}
    For any $p\ge1$,
    $$\mathbb{E}\Vert z\Vert^{2p}\le C(p)d^p,$$
    where $C(p)$ is a constant that only depends on $p$.
    \label{lem:moment-z}
\end{lemma}

\begin{proof}
    First, $\Vert z\Vert^2=\sum_{i=1}^n z_i^2$, where we represent $z=(z_1,z_2,\cdots,z_d)^T$. For any $n$ positive numbers $a_1,a_2,\dots,a_n$, using Jensen's inequality,
    $$\left(\sum_{i=1}^na_i\right)^p=n^p\left(\frac{1}{n}\sum_{i=1}^na_i\right)^p\le n^p\cdot\frac{1}{n}\sum_{i=1}^na_i^p.$$
    Then,
    $$\begin{aligned}
        \mathbb{E}\Vert z\Vert^{2p}&=\mathbb{E}\left[\left(\sum_{i=1}^dz_i^2\right)^p\right]\\
        &\le d^p\cdot\frac{1}{d}\sum_{i=1}^d\mathbb{E}[|z_i|^{2p}]&(\text{Jensen's inequality})\\
        &\le d^p\mathbb{E}\left[|z_1|^{2p}\right]&(\{z_i\}_{i=1}^d\text{ are i.i.d.})\\
        &=C(p)d^p.
    \end{aligned}$$
    Here the constant $$C(p)=\int_{-\infty}^\infty\frac{1}{\sqrt{2\pi}}e^{-\frac{x^2}{2}}|x|^{2p}\dd x<\infty$$
    only depends on $p$.
\end{proof}

Also, the following is another simple fact that is useful for our analysis.

\begin{lemma}
    For two vectors $u\in\mathbb{R}^n$, $v\in\mathbb{R}^m$, the matrix $uv^T\in\mathbb{R}^{n\times m}$ satisfies
    $$\Vert uv^T\Vert_F=\Vert u\Vert\cdot\Vert v\Vert,$$
    where $\Vert\cdot\Vert_F$ denotes the Frobenious norm and $\Vert\cdot\Vert$ denotes the 2-norm.
    \label{f-norm}
\end{lemma}

\begin{proof} By the definition of the Frobenious norm,
    $$\begin{aligned}
        \Vert uv^T\Vert_F^2&=\sum_{i=1}^n\sum_{j=1}^m(uv^T)_{ij}^2\\
        &=\sum_{i=1}^n\sum_{j=1}^mu_i^2v_j^2\\
        &=\sum_{i=1}^nu_i^2\cdot\sum_{j=1}^mv_j^2\\
        &=\Vert u\Vert^2\cdot\Vert v\Vert^2.
    \end{aligned}$$
\end{proof}

Recall that we have defined $v(t,x)=\mathbb{E}[\partial_tI(t,x_0,x_1)|x_t=x]$. We then give bounds for the score functions and the velocity functions.

\begin{lemma}
    For $p\ge 1$, there exists a constant $C(p)$ that depends only on $p$, s.t. for $t\in(0,1)$, 
    $$\begin{aligned}
        \mathbb{E}\Vert s(t,x_t)\Vert^p&\le C(p)\gamma^{-p}d^{p/2},\\
        \mathbb{E}\Vert v(t,x_t)\Vert^p&\le C(p)\mathbb{E}\Vert x_1-x_0\Vert^p,\\
        \mathbb{E}\Vert b(t,x_t)\Vert^p&\le C(p)\left[\mathbb{E}\Vert x_1-x_0\Vert^p+\dot{\gamma}d^{p/2}\right],\\
        \mathbb{E}\Vert b_F(t,x_t)\Vert^p&\le C(p)\left[\mathbb{E}\Vert x_1-x_0\Vert^p+(\dot{\gamma}^p-\gamma^{-p}\epsilon^p)d^{p/2}\right].
    \end{aligned}$$
    \label{lem:vsb-bound}
\end{lemma}

\begin{proof}
    When $p\ge 1$, use the conditional expectation form of $s$ and $v$ and apply Jensen's inequality, we then obtain
    $$\begin{aligned}
        \mathbb{E}\Vert s(t,x_t)\Vert^p&=\mathbb{E}\Vert\gamma^{-1}\mathbb{E}[z|x_t=x]\Vert^p\le\gamma^{-p}\mathbb{E}\Vert z\Vert^p\le C(p)\gamma^{-p}d^{p/2},\\
        \mathbb{E}\Vert v(t,x_t)\Vert^p&=\mathbb{E}\Vert\mathbb{E}[\partial_tI|x_t=x]\Vert^p\le\mathbb{E}\Vert\partial_tI\Vert^p\le C(p)\mathbb{E}\Vert x_1-x_0\Vert^p,
    \end{aligned}$$
    Moreover, since $b(t,x)=v(t,x)+\dot{\gamma}\gamma s(t,x)$ and $b_F(t,x)=b(t,x)+\epsilon s(t,x)$,
    $$\begin{aligned}
        \mathbb{E}\Vert b(t,x_t)\Vert^p&\le C(p)\left[\mathbb{E}\Vert x_1-x_0\Vert^p+\dot{\gamma}^pd^{p/2}\right],\\
        \mathbb{E}\Vert b_F(t,x_t)\Vert^p&\le C(p)\left[\mathbb{E}\Vert x_1-x_0\Vert^p+(\dot{\gamma}^p-\gamma^{-p}\epsilon^p)d^{p/2}\right].
    \end{aligned}$$
\end{proof}

\subsection{Bounds on Time and Space Derivatives}

\textbf{Note:} In the following sections, we will use the fact that $\frac{\dd}{\dd t}\gamma^2(t)=O(1)$ and $\frac{\dd^2}{\dd t^2}\gamma^2(t)=O(1)$.

Before we move on to the lemmas, we first discuss the conditional expectation itself. By the definition $x_t=I(t,x_0,x_1)+\gamma(t)z$, we can just know that the density of $x_t$ can be expressed as
$$\rho(t,x)=\int_{\mathbb{R}^d\times\mathbb{R}^d}\frac{1}{(2\pi\gamma(t)^2)^{d/2}}\exp\left(-\frac{\Vert x-I(t,x_0,x_1)\Vert^2}{2\gamma(t)^2}\right)\dd\nu(x_0,x_1).$$
Also, under the condition $x_t=x$, the conditional measure of $(x_0,x_1)$ is then 
$$\frac{1}{\rho(t,x)}\cdot\frac{1}{(2\pi\gamma(t)^2)^{d/2}}\exp\left(-\frac{\Vert x-I(t,x_0,x_1)\Vert^2}{2\gamma(t)^2}\right)\dd\nu(x_0,x_1).$$
Therefore, for any function $f_t(x_t,x_0,x_1)$, its conditional expectation can be written as
$$\begin{aligned}
    \mathbb{E}[f_t(x_t,x_0,x_1)|x_t=x]&=\int_{\mathbb{R}^d\times\mathbb{R}^d}\frac{f_t(x,x_0,x_1)}{\rho(t,x)}\cdot\frac{1}{(2\pi\gamma(t)^2)^{d/2}}\exp\left(-\frac{\Vert x-I(t,x_0,x_1)\Vert^2}{2\gamma(t)^2}\right)\dd\nu(x_0,x_1)\\
    &=\frac{\underset{(x_0,x_1)\sim\nu}{\mathbb{E}}\left[\exp\left(-\frac{\Vert x-I(t,x_0,x_1)\Vert^2}{2\gamma(t)^2}\right)\cdot f_t(x,x_0,x_1)\right]}{\underset{(x_0,x_1)\sim\nu}{\mathbb{E}}\left[\exp\left(-\frac{\Vert x-I(t,x_0,x_1)\Vert^2}{2\gamma(t)^2}\right)\right]}.
\end{aligned}$$

We first consider the time derivative of $v$ in the sense of expectation.

\begin{lemma}
    We have $$\mathbb{E}\Vert\partial_tv(t,x_t)\Vert^2\lesssim\mathbb{E}\Vert\partial_t^2I\Vert^2+\gamma^{-2}d\mathbb{E}\Vert x_0-x_1\Vert^4+\gamma^{-2}\dot{\gamma}^4d^3$$
for $t\in(0,1)$.
    \label{lem:v-time}
\end{lemma}

\begin{proof}
    For $t\in(0,1)$, we can first explicitly write $$v(t,x)=\frac{\underset{(x_0,x_1)\sim\nu}{\mathbb{E}}\left[\exp\left(-\frac{\Vert x-I(t)\Vert^2}{2\gamma(t)^2}\right)\cdot\partial_tI(t)\right]}{\underset{(x_0,x_1)\sim\nu}{\mathbb{E}}\left[\exp\left(-\frac{\Vert x-I(t)\Vert^2}{2\gamma(t)^2}\right)\right]}.$$
    Here we write $I(t)=I(t,x_0,x_1)$ for simplicity, and below we will omit $t$ when it is clear in the context. We now want to compute $\partial_tv(t,x)$. First notice that
    $$\frac{\dd}{\dd t}\left[\exp\left(-\frac{\Vert x-I\Vert^2}{2\gamma^2}\right)\cdot\partial_tI\right]=\exp\left(-\frac{\Vert x-I\Vert^2}{2\gamma^2}\right)\cdot\left[\partial_t^2I+\partial_tI\cdot\left(\frac{\Vert x-I\Vert^2}{\gamma(t)^3}\dot{\gamma}+\frac{x-I}{\gamma^2}\cdot\partial_tI\right)\right].$$
    Note that $\sup_{x\in\mathbb{R}}\exp(-x^2/2)x=e^{-1/2}=C_1<\infty$, $\sup_{x\in\mathbb{R}}\exp(-x^2/2)x^2=2e^{-1}=C_2<\infty$,
    we know that $$\left\Vert\frac{\dd}{\dd t}\left[\exp\left(-\frac{\Vert x-I\Vert^2}{2\gamma^2}\right)\cdot\partial_tI\right]\right\Vert\le\Vert\partial_t^2I\Vert+C_2\gamma^{-1}\dot{\gamma}\Vert\partial_tI\Vert+C_1\gamma^{-1}\Vert\partial_tI\Vert^2,$$
Therefore, using dominated convergence theorem, we know that
    $$\frac{\dd}{\dd t}\underset{(x_0,x_1)\sim\nu}{\mathbb{E}}\left[\exp\left(-\frac{\Vert x-I\Vert^2}{2\gamma^2}\right)\cdot\partial_tI\right]=\underset{(x_0,x_1)\sim\nu}{\mathbb{E}}\left[\frac{\dd}{\dd t}\left(\exp\left(-\frac{\Vert x-I\Vert^2}{2\gamma^2}\right)\cdot\partial_tI\right)\right].$$
Similarly we can do this for the denominator, so that we can compute the overall derivative. Let $f_t(x_0,x_1)=-\frac{\Vert x-I(t)\Vert^2}{2\gamma^2}$, for simplicity we may just write $f_t$. Then,
    $$\begin{aligned}
        \partial_tv(t,x)&=\frac{\underset{(x_0,x_1)\sim\nu}{\mathbb{E}}\left[\exp\left(f_t\right)\cdot\partial_t^2I\right]}{\underset{(x_0,x_1)\sim\nu}{\mathbb{E}}\left[\exp\left(f_t\right)\right]}\\
        &\qquad+\frac{\underset{(x_0,x_1)\sim\nu}{\mathbb{E}}\left[\exp\left(f_t\right)\cdot\partial_tI\cdot\partial_tf_t\right]}{\underset{(x_0,x_1)\sim\nu}{\mathbb{E}}\left[\exp\left(f_t\right)\right]}\\
&\qquad-\frac{\underset{(x_0,x_1)\sim\nu}{\mathbb{E}}\left[\exp\left(f_t\right)\cdot\partial_tI\right]\cdot\underset{(x_0,x_1)\sim\nu}{\mathbb{E}}\left[\exp\left(f_t\right)\cdot\partial_tf_t\right]}{\left[\underset{(x_0,x_1)\sim\nu}{\mathbb{E}}\left[\exp\left(f_t\right)\right]\right]^2}\\
        &=\mathbb{E}[\partial_t^2I|x_t=x]\\
        &\qquad+\text{Cov}(\partial_tI,\partial_tf_t|x_t=x),
    \end{aligned}$$
    where the last equality uses the previous explanations of conditional expectations. Hence,
    $$\begin{aligned}
        \Vert\partial_tv(t,x)\Vert&\le\mathbb{E}[\Vert\partial_t^2I\Vert|x_t=x]+\sqrt{\mathbb{E}[|\partial_tf_t|^2|x_t=x]}\sqrt{\mathbb{E}[\Vert\partial_tI\Vert^2|x_t=x]}.
    \end{aligned}$$
    Therefore, we have 
    $$\begin{aligned}
        \mathbb{E}\Vert\partial_tv(t,x_t)\Vert^2&\le2\mathbb{E}[\mathbb{E}[\Vert\partial_t^2I\Vert^2|x_t]]+2\mathbb{E}[\mathbb{E}[|\partial_tf_t|^2|x_t]\cdot\mathbb{E}[\Vert\partial_tI\Vert^2|x_t]]&((a+b)^2\le2a^2+2b^2)\\
        &\le2\mathbb{E}\Vert\partial_t^2I\Vert^2+2\sqrt{\mathbb{E}[\mathbb{E}[|\partial_tf_t|^2|x_t]^2]}\cdot\sqrt{\mathbb{E}[\mathbb{E}[\Vert\partial_tI\Vert^2|x_t]^2]}&(\text{Cauchy-Schwarz inequality})\\
        &\le2\mathbb{E}\Vert\partial_t^2I\Vert^2+2\sqrt{\mathbb{E}[\mathbb{E}[|\partial_tf_t|^4|x_t]]}\cdot\sqrt{\mathbb{E}[\mathbb{E}[\Vert\partial_tI\Vert^4|x_t]]}&(\text{Jensen's inequality})\\
        &\le2\mathbb{E}\Vert\partial_t^2I\Vert^2+2\sqrt{\mathbb{E}|\partial_tf_t|^4}\sqrt{\mathbb{E}\Vert\partial_tI\Vert^4}.
    \end{aligned}$$

    Using the requirement $\partial_tI\le C\Vert x_0-x_1\Vert$ in the definition of stochastic interpolants, $\Vert\partial_tI\Vert^4\lesssim\Vert x_0-x_1\Vert^4$. For $\partial_tf_t$, we can directly obtain
    $$\partial_tf=\frac{\Vert x-I\Vert^2}{\gamma^3}\dot{\gamma}+\gamma^{-2}(x-I)\cdot\partial_tI=\gamma^{-1}\dot{\gamma}\Vert z\Vert^2+\gamma^{-1}z\cdot\partial_tI.$$
Recall that we have defined $x_t=I(t,x_0,x_1)+\gamma(t)z$ where $z$ is an independent gaussian variable $z\sim\mathcal{N}(0,I_d)$. By \Cref{lem:moment-z}, $$\mathbb{E}\Vert z\Vert^8\lesssim d^4,\qquad\mathbb{E}\Vert z\Vert^4\lesssim d^2,$$
we have $$\mathbb{E}|\partial_tf_t|^4\lesssim(\gamma^{-1}\dot{\gamma})^4d^4+\gamma^{-4}d^2\mathbb{E}\Vert x_0-x_1\Vert^4.$$
Therefore, we can finally deduce that
    $$\mathbb{E}\Vert\partial_tv(t,x_t)\Vert^2\lesssim\mathbb{E}\Vert\partial_t^2I\Vert^2+\gamma^{-2}d\mathbb{E}\Vert x_0-x_1\Vert^4+\gamma^{-2}\dot{\gamma}^4d^3.$$
\end{proof}

In addition, we want to consider the space derivative of the velocity for a fixed $t\in(0,1)$. That is, we want to give a bound for $\nabla v(t,x)$. Here we use the notation $\nabla v(t,x)$ to denote the Jacobian matrix $\left(\frac{d}{dx^i}v(t,x)_j\right)_{ij}$, where $x^i$ represents the value of vector $x$ at the $i$-th dimension.

\begin{lemma}
    We have $$\mathbb{E}\Vert\nabla v(t,x)\Vert_F^p\le C(p)\gamma^{-p}d^{p/2}\sqrt{\mathbb{E}\Vert x_0-x_1\Vert^{2p}}$$
    for $p\ge1$, $t\in(0,1)$, where $C(p)$ is a constant that only depends on $p$ and $\Vert\cdot\Vert_F$ denotes the Frobenius norm.
    \label{lem:v-space}
\end{lemma}

\begin{proof}
    Similar to the proof of Lemma \ref{lem:v-time}, $$\nabla\left(\exp\left(-\frac{\Vert x-I\Vert^2}{2\gamma^2}\right)\cdot\partial_tI\right)=\exp\left(-\frac{\Vert x-I\Vert^2}{2\gamma^2}\right)\left(\partial_tI\otimes\nabla\left(-\frac{\Vert x-I\Vert^2}{2\gamma^2}\right)\right),$$
    where $\otimes$ denotes the tensor product, which denotes $\partial_tI\otimes\nabla\left(-\frac{\Vert x-I\Vert^2}{2\gamma^2}\right)=\partial_tI\cdot\nabla\left(-\frac{\Vert x-I\Vert^2}{2\gamma^2}\right)^T$ here in the matrix form. Again, by dominated convergence theorem we can move the gradient operator into the expectation. Using the same notations (i.e., $f_t$ and so on), we can deduce that
    $$\begin{aligned}
    \nabla v(t,x)&=\frac{\underset{(x_0,x_1)\sim\nu}{\mathbb{E}}[\exp(f_t)\cdot(\partial_tI\otimes\nabla f_t)]}{\underset{(x_0,x_1)\sim\nu}{\mathbb{E}}[\exp(f_t)]}\\
    &\qquad-\frac{\underset{(x_0,x_1)\sim\nu}{\mathbb{E}}[\exp(f_t)\cdot\partial_tI]\otimes\underset{(x_0,x_1)\sim\nu}{\mathbb{E}}[\exp(f_t)\cdot\nabla f_t]}{\left[\underset{(x_0,x_1)\sim\nu}{\mathbb{E}}[\exp(f_t)]\right]^2}\\
    &=\text{Cov}(\partial_tI,\nabla f_t|x_t=x).
    \end{aligned}$$
    Again, the last equality uses the definition of covariance. Thus, by Cauchy-Schwarz inequality,
    $$\begin{aligned}
        \Vert\nabla v(t,x)\Vert_F&\le\sqrt{\mathbb{E}[\Vert\partial_tI\Vert^2|x_t=x]}\sqrt{\mathbb{E}[\Vert\nabla f_t\Vert^2|x_t=x]}.
    \end{aligned}$$
    Therefore, we can use Cauchy-Schwarz inequality again and apply Jensen's inequality to deduce that for any $p\ge1$,
    $$\begin{aligned}
        \mathbb{E}\Vert\nabla v(t,x_t)\Vert_F^p&\le\sqrt{\left[\mathbb{E}[\mathbb{E}\Vert\partial_tI\Vert^2|x_t]\right]^{p}}\cdot\sqrt{\left[\mathbb{E}[\mathbb{E}\Vert\nabla f_t\Vert^2|x_t]\right]^{p}}\\
        &\le\sqrt{\mathbb{E}\Vert\partial_tI\Vert^{2p}}\cdot\sqrt{\mathbb{E}\Vert\nabla f_t\Vert^{2p}}.
    \end{aligned}$$
    It is clear that $\mathbb{E}\Vert\partial_tI\Vert^{2p}\lesssim\mathbb{E}\Vert x_0-x_1\Vert^{2p}$. Note $$\nabla f_t=-\frac{x-I}{\gamma^2}=-\gamma^{-1}z,$$
    we then deduce that $$\mathbb{E}\Vert\nabla f_t\Vert^{2p}\le C(p)\gamma^{-2p}d^p$$
    for some constant that only depends on $p$. The lemma is then obtained.
\end{proof}

Despite the function $v(t,x)$, we are also interested in the score function $s(t,x)$. The following lemmas provide some similar bounds for $s(t,x)$.

\begin{lemma}
    $$\mathbb{E}\Vert\partial_t\left(\gamma s(t,x_t)\right)\Vert^2\lesssim\gamma^{-2}\dot{\gamma}^2d^3+\gamma^{-2}d^2\sqrt{\mathbb{E}\Vert x_0-x_1\Vert^4}$$
    and 
    $$\mathbb{E}\Vert\partial_ts(t,x_t)\Vert^2\lesssim\gamma^{-4}\dot{\gamma}^2d^3+\gamma^{-4}d^2\sqrt{\mathbb{E}\Vert x_0-x_1\Vert^4}.$$
    for any $t\in(0,1)$, 
    \label{lem:s-time}
\end{lemma}

\begin{proof}
    First using the analysis for the conditional expectations, we obtain that
$$s(t,x)=\nabla\log\rho(t,x)=-\frac{\underset{(x_0,x_1)\sim\nu}{\mathbb{E}}\left[\exp\left(-\frac{\Vert x-I\Vert^2}{2\gamma^2}\right)\cdot\frac{x-I}{\gamma^2}\right]}{\underset{(x_0,x_1)\sim\nu}{\mathbb{E}}\left[\exp\left(-\frac{\Vert x-I\Vert^2}{2\gamma^2}\right)\right]}.$$
    In order to compute $\partial_t(\gamma s(t,x))$, we apply a similar analysis as the proof of \Cref{lem:v-time} with exactly the same notations to deduce that
    $$\begin{aligned}
        \partial_ts(t,x)&=\frac{\underset{(x_0,x_1)\sim\nu}{\mathbb{E}}\left[\exp(f_t)\cdot\partial_t(\gamma\nabla f_t)\right]}{\underset{(x_0,x_1)\sim\nu}{\mathbb{E}}\left[\exp(f_t)\right]}\\
        &\qquad+\frac{\underset{(x_0,x_1)\sim\nu}{\mathbb{E}}\left[\exp(f_t)\cdot\partial_tf_t\cdot\gamma\nabla f_t\right]}{\underset{(x_0,x_1)\sim\nu}{\mathbb{E}}\left[\exp(f_t)\right]}\\
        &\qquad-\frac{\underset{(x_0,x_1)\sim\nu}{\mathbb{E}}\left[\exp(f_t)\cdot\gamma\nabla f_t\right]\cdot\underset{(x_0,x_1)\sim\nu}{\mathbb{E}}\left[\exp(f_t)\cdot\partial_tf_t\right]}{\left[\underset{(x_0,x_1)\sim\nu}{\mathbb{E}}\left[\exp(f_t)\right]\right]^2}\\
        &=\mathbb{E}[\partial_t(\gamma\nabla f_t)|x_t=x]+\text{Cov}(\gamma\nabla f_t,\partial_tf_t|x_t=x)
    \end{aligned}$$
    The above term has exactly the same form as which in the proof of Lemma \ref{lem:v-time}, so by a similar analysis we can obtain that
    $$\mathbb{E}\Vert\partial_t(\gamma s(t,x_t))\Vert^2\le2\mathbb{E}\Vert\partial_t(\gamma\nabla f_t)\Vert^2+2\sqrt{\mathbb{E}|\partial_tf_t|^4}\cdot\sqrt{\mathbb{E}\Vert\gamma\nabla f_t\Vert^4}.$$

    We have already deduced that $$\mathbb{E}\Vert\nabla f_t\Vert^4\lesssim\gamma^{-4}d^2,$$
and $$\mathbb{E}|\partial_tf_t|^4\lesssim(\gamma^{-1}\dot{\gamma})^4d^4+\gamma^{-4}d^2\mathbb{E}\Vert x_0-x_1\Vert^4.$$
Also, $$\partial_t(\gamma\nabla f_t)=\partial_t\left(-\frac{x-I}{\gamma}\right)=\gamma^{-1}\partial_tI+\gamma^{-2}\dot{\gamma}(x-I)=\gamma^{-1}\partial_tI+\gamma^{-1}\dot{\gamma}z$$
Hence, $$\mathbb{E}\Vert\partial_ts(t,x_t)\Vert^2\lesssim\gamma^{-2}\dot{\gamma}^2d^3+\gamma^{-2}d^2\sqrt{\mathbb{E}\Vert x_0-x_1\Vert^4},$$
which completes the first part. The proof of the second part is exactly the same by replacing $\gamma\nabla f_t$ with $\nabla f_t$.
\end{proof}

\begin{lemma}
    For any $p\ge1$, there exists a constant $C(p)<\infty$ that only depends on $p$ such that
    $$\mathbb{E}\Vert\nabla s(t,x)\Vert_F^{p}\le C(p)\gamma^{-2p}d^p.$$
    \label{lem:s-space}
\end{lemma}
\begin{proof}
    With exactly the same ideas of the previous lemmas, we can obtain
    $$\begin{aligned}
        \nabla s(t,x)&=\mathbb{E}[\nabla^2f_t|x_t=x]+\text{Cov}(\nabla f_t,\nabla f_t|x_t=x)\\
        &=-\gamma^{-2}I+\gamma^{-2}\text{Cov}(z,z|x_t=x)
    \end{aligned}$$
    Then, for $p\ge1$, we have
    $$\begin{aligned}
        \mathbb{E}\Vert\nabla s(t,x_t)\Vert_F^{p}&\le2^{p-1}\Vert\gamma^{-2}I\Vert_F^p+2^{p-1}\gamma^{-2p}\mathbb{E}\Vert\mathbb{E}[\Vert z\Vert^2|x_t=x]\Vert^p\\
        &\le 2^{p-1}\gamma^{-2p}d^{p/2}+2^{p-1}\gamma^{-2p}\mathbb{E}\Vert z\Vert^{2p}&(\text{Jensen's inequality})\\
        &\le C(p)\gamma^{-2p}d^p.
    \end{aligned}$$
    Here for the first inequality we have used the fact $(a+b)^p\le 2^{p-1}a^p+2^{p-1}b^p$ for $a,b\ge0$.
\end{proof}

We also need some bounds for $\Delta s$ and $\Delta v$, where $\Delta$ represents the Laplace operator.

\begin{lemma}
    $$\mathbb{E}\Vert\Delta v(t,x_t)\Vert^2\lesssim\gamma^{-2}d\mathbb{E}\Vert x_0-x_1\Vert^4+\gamma^{-4}d^2$$
    for all $t\in(0,1)$.
    \label{lem:v-laplace}
\end{lemma}

\begin{proof}
    We still use the notations in the proof of Lemma \ref{lem:v-time}. First, in the proof of Lemma \ref{lem:v-space}, we have already shown that
    $$\begin{aligned}
        \partial_{x^i}v(t,x)&=
        \frac{\underset{(x_0,x_1)\sim\nu}{\mathbb{E}}[\exp(f_t)\cdot(\partial_tI\cdot\partial_{x^i}f_t)]}{\underset{(x_0,x_1)\sim\nu}{\mathbb{E}}[\exp(f_t)]}\\
        &\qquad-\frac{\underset{(x_0,x_1)\sim\nu}{\mathbb{E}}[\exp(f_t)\cdot\partial_tI]\cdot\underset{(x_0,x_1)\sim\nu}{\mathbb{E}}[\exp(f_t)\cdot\partial_{x^i}f_t]}{\left[\underset{(x_0,x_1)\sim\nu}{\mathbb{E}}[\exp(f_t)]\right]^2}\\
        &=\frac{\underset{(x_0,x_1)\sim\nu}{\mathbb{E}}\left[\underset{(\bar{x}_0,\bar{x}_1)\sim\nu}{\mathbb{E}}[\exp(f_t)\exp(\bar{f}_t)(\partial_tI-\partial_t\bar{I})\cdot(\partial_{x^i}f_t-\partial_{x^i}\bar{f}_t)]\right]}{2\underset{(x_0,x_1)\sim\nu}{\mathbb{E}}\left[\underset{(\bar{x}_0,\bar{x}_1)\sim\nu}{\mathbb{E}}[\exp(f_t)\exp(\bar{f}_t)]\right]}.
    \end{aligned}$$
    The last equality is an alternative form of the covariance, and we use notations $\bar{I}=I(t,\bar{x}_0,\bar{x}_1)$ and $\bar{f}_t=f_t(\bar{x}_0,\bar{x}_1)$ for intermediate variables $(\bar{x}_0,\bar{x}_1)$.
    Hence, $$\begin{aligned}
        \partial^2_{x^i}v(t,x)&=\text{Cov}(\partial_tI,\partial_{x^i}^2f_t|x_t=x)\\
        &\qquad+\frac{1}{2}\text{Cov}[(\partial_tI-\partial_t\bar{I})(\partial_{x^i}f_t-\partial_{x^i}\bar{f}_t),\partial_{x^i}f_t+\partial_{x^i}\bar{f}_t|x_t=\bar{x}_t=x].
    \end{aligned}$$
    For the first term, note that $\partial^2_{x^i}f_t=-\gamma^{-2}$ is fixed. So,
    $$\Delta v(t,x)=\frac{1}{2}\text{Cov}[(\partial_tI-\partial_t\bar{I})(\nabla f_t-\nabla \bar{f}_t),\nabla f_t+\nabla\bar{f}_t|x_t=\bar{x}_t=x].$$
    Here the covariance refers to the expectation of dot product instead of the expectation of tensor product. Then, use the fact $\mathbb{E}\Vert X-\mathbb{E}X\Vert^2\le\mathbb{E}\Vert X\Vert^2$, we know that
    $$\begin{aligned}
        \Vert\Delta v(t,x)\Vert
        &\le\sqrt{\mathbb{E}[\left\Vert(\partial_tI-\partial_t\bar{I})(\nabla f_t-\nabla\bar{f}_t)^T\right\Vert^2|x_t=\bar{x}_t=x]}\\
        &\qquad\cdot\sqrt{\mathbb{E}[\Vert\nabla f_t+\nabla\bar{f}_t\Vert^2|x_t=\bar{x}_t=x]}&(\text{Cauchy-Schwarz inequality})\\
        &\lesssim\left[\mathbb{E}[\left\Vert\partial_tI-\partial_t\bar{I}\right\Vert^4|x_t=\bar{x}_t=x]\right]^{1/4}\\
        &\qquad\cdot\left[\mathbb{E}[\left\Vert\nabla f_t-\nabla\bar{f}_t\right\Vert^4|x_t=\bar{x}_t=x]\right]^{1/4}&(\text{Cauchy-Schwarz inequality})\\
        &\qquad\cdot\sqrt{\mathbb{E}[\Vert\nabla f_t\Vert^2|x_t=x]}&(\text{by symmetry})\\
        &\lesssim\left[\mathbb{E}[\left\Vert\partial_tI\right\Vert^4|x_t=x]\right]^{1/4}\cdot\sqrt{\mathbb{E}[\Vert\nabla f_t\Vert^4|x_t=x]}.&(\text{by symmetry})
    \end{aligned}$$
    Therefore, $$\begin{aligned}
        \mathbb{E}\Vert\Delta v(t,x_t)\Vert^2
        &\lesssim\mathbb{E}\left[\sqrt{\mathbb{E}[\left\Vert\partial_tI\right\Vert^4|x_t=x]}\cdot\mathbb{E}[\Vert\nabla f_t\Vert^4|x_t=x]\right]\\
        &\lesssim\sqrt{\mathbb{E}\left[\mathbb{E}[\left\Vert\partial_tI\right\Vert^4|x_t=x]\right]}\cdot\sqrt{\mathbb{E}\left[\mathbb{E}[\Vert\nabla f_t\Vert^4|x_t=x]^2\right]}&(\text{Cauchy-Schwarz inequality})\\
        &\lesssim\gamma^{-4}\sqrt{\mathbb{E}\Vert\partial_tI\Vert^4}\cdot\sqrt{\mathbb{E}\Vert z\Vert^{8}}&(\text{Jensen's inequality})\\
        &\lesssim\gamma^{-4}\sqrt{\mathbb{E}\Vert x_0-x_1\Vert^4}\cdot d^2\\
        &\lesssim\gamma^{-2}d\mathbb{E}\Vert x_0-x_1\Vert^4+\gamma^{-4}d^2.
    \end{aligned}$$
\end{proof}

\begin{lemma}
    $$\mathbb{E}\Vert\Delta s(t,x)\Vert^2\lesssim\gamma^{-6}d^3$$
    for $t\in(0,1)$.
    \label{lem:s-laplace}
\end{lemma}

\begin{proof}
    $$\begin{aligned}
        \nabla s(t,x)&=-\gamma^{-2}I+\text{Cov}(\nabla f_t,\nabla f_t|x_t=x).
    \end{aligned}$$
    Hence, with similar calculations and notations as in the proof of Lemma \ref{lem:v-laplace}, we can deduce that
    $$\begin{aligned}
        \Delta s(t,x)&=2\text{Cov}(\nabla f_t,\Delta f_t|x_t=x)\\
        &\qquad+\frac{1}{2}\text{Cov}[(\nabla f_t-\nabla\bar{f}_t)(\nabla f_t-\nabla\bar{f}_t)^T,\nabla f_t-\nabla\bar{f}_t|x_t=\bar{x}_t=x].\\
        &=\frac{1}{2}\text{Cov}[(\nabla f_t-\nabla\bar{f}_t)(\nabla f_t-\nabla\bar{f}_t)^T,\nabla f_t-\nabla\bar{f}_t|x_t=\bar{x}_t=x].
    \end{aligned}$$
    Then, with H\"older's inequality, we have
    $$\begin{aligned}
        \Vert\Delta s(t,x)\Vert&\lesssim\left[\mathbb{E}[\Vert\nabla f_t-\nabla\bar{f}_t\Vert^3|x_t=\bar{x}_t=x]\right]^{1/3}\\
        &\qquad\cdot\left[\mathbb{E}[\Vert(\nabla f_t-\nabla\bar{f}_t)(\nabla f_t-\nabla\bar{f}_t)^T\Vert^{3/2}]\right]^{2/3}\\
        &\lesssim\left[\mathbb{E}[\Vert\nabla f_t\Vert^3|x_t=x]\right]^{1/3}&(\text{by symmetry})\\
        &\qquad\cdot\left[\mathbb{E}[\Vert\nabla f_t-\nabla\bar{f}_t\Vert^3|x_t=\bar{x}_t=x]\right]^{2/3}\\
        &\lesssim\mathbb{E}[\Vert\nabla f_t\Vert^3|x_t=x].&(\text{by symmetry})
    \end{aligned}$$
    Hence, by Jensen's inequality, $$\mathbb{E}\Vert\Delta s(t,x)\Vert^2\lesssim\mathbb{E}[\mathbb{E}[\Vert\nabla f_t\Vert^3|x_t=x]^2]\lesssim\gamma^{-6}\mathbb{E}\Vert z\Vert^6\lesssim\gamma^{-6}d^3.$$
\end{proof}



\section{Omitted Proofs in Sections \ref{sec:results} and \ref{sec:instance}}
\label{appendix:overall}

\subsection{Bounds along the forward Path}

Recall the forward ODE $$\dd X_t=b(t,X_t)\dd t$$
and the forward SDE $$\dd X_t^F=b_F(t,X_t^F)\dd t+\sqrt{2\epsilon}dW_t.$$
Their solutions are denoted by $X_t$ and $X_t^F$, respectively. Using the chain rule or It\^o's formula, for a function $f(t,x)$ that is twice continuously differentiable, we have 
$$\dd f(t,X_t)=[\partial_tf(t,X_t)+\nabla f(t,X_t)\cdot b(t,X_t)]\dd t,$$
and $$\dd f(t,X_t^F)=[\partial_tf(t,X_t^F)+\nabla f(t,X_t^F)\cdot b_F(t,X_t^F)+\epsilon\Delta f]\dd t+\sqrt{2\epsilon}\nabla f(t,X_t^F)\cdot \dd W_t.$$

With the above formula, we can now provide the following bound on the discretization error.

\begin{lemma}
    % Suppose that we take $\epsilon=\Theta(1)$. Then 
    For $0<t_0\le t_1<1$, suppose $\epsilon=O(1)$, then,
    $$\begin{aligned}
        \mathbb{E}\Vert v(t_0,X_{t_0}^F)-v(t_1,X_{t_1}^F)\Vert^2&\lesssim(t_1-t_0)^2\left[M_2+\gamma_{\min}^{-6}d^3+\gamma_{\min}^{-2}d\sqrt{\mathbb{E}\Vert x_0-x_1\Vert^{8}}\right]\\
        &\qquad+\epsilon(t_1-t_0)\gamma_{\min}^{-2}d\sqrt{\mathbb{E}\Vert x_0-x_1\Vert^{4}},
    \end{aligned}$$
    and $$\mathbb{E}\Vert s(t_0,X_{t_0}^F)-s(t_1,X_{t_1}^F)\Vert^2\lesssim(t_1-t_0)^2\left[\gamma_{\min}^{-4}d^2\sqrt{\mathbb{E}\Vert x_0-x_1\Vert^4}+\gamma_{\min}^{-6}d^3\right]+\epsilon(t_1-t_0)\gamma_{\min}^{-4}d^2.$$
    Here we denote $\gamma_{\min}=\min_{u\in[t_0,t_1]}\gamma$.
    \label{lem:discretize}
\end{lemma}

\begin{proof}
    According to the formula 
    $$\dd v(t,X_t^F)=[\partial_tv(t,X_t^F)+\nabla v(t,X_t^F)\cdot b_F(t,X_t^F)+\epsilon\Delta v(t,X_t^F)]\dd t+\sqrt{2\epsilon}\nabla v(t,X_t^F)\cdot \dd W_t,$$
    we know that 
    $$\begin{aligned}
        \Vert v(t_1,X_{t_1}^F)-v(t_0,X_{t_0}^F)\Vert^2
        &\le4\left\Vert\int_{t_0}^{t_1}\partial_uv(u,X_u^F)\dd u\right\Vert^2\\
        &\qquad+4\left\Vert\int_{t_0}^{t_1}\nabla v(u,X_u^F)\cdot b(u,X_u^F)\dd u\right\Vert^2\\
        &\qquad+4\left\Vert\int_{t_0}^{t_1}\epsilon\Delta v(u,X_u^F)\dd u\right\Vert^2\\
        &\qquad+4\left\Vert\int_{t_0}^{t_1}\sqrt{2\epsilon}\nabla v(u,X_u^F)\cdot \dd W_u\right\Vert^2.
    \end{aligned}$$
    For the first three terms, by Jensen's inequality we know that for any function $Y$, we have $$\left\Vert\int_{t_0}^{t_1}Y(u)du\right\Vert^2\le(t_1-t_0)\int_{t_0}^{t_1}\Vert Y(u)\Vert^2du.$$
    For the last term, use It\^o's isometry (\citealt{le2016brownian}, Equation 5.8), we can get $$\mathbb{E}\left[\left\Vert\int_{t_0}^{t_1}\sqrt{2\epsilon}\nabla v(u,X_u^F)\cdot dW_u\right\Vert^2\right]=\int_{t_0}^{t_1}\mathbb{E}\Vert\sqrt{2\epsilon}\nabla v(u,X_u^F)\Vert_F^2du.$$
    Therefore, we can use Fubini's theorem to change the order of expectation and integral, and combine the results of Lemma \ref{lem:v-time}, \ref{lem:v-space}, \ref{lem:v-laplace} and \ref{lem:vsb-bound} and Assumption \ref{a:regularity} to get
    $$\begin{aligned}
        \Vert v(t_1,X_{t_1}^F)-v(t_0,X_{t_0}^F)\Vert^2&\lesssim(t_1-t_0)\int_{t_0}^{t_1}\mathbb{E}\Vert\partial_uv(u,X_u)\Vert^2\dd u\\
        &\qquad+(t_1-t_0)\int_{t_0}^{t_1}\left[\gamma_{\min}^2d^{-1}\mathbb{E}\Vert\nabla v(u,X_u)\Vert^4+\gamma_{\min}^{-2}d\mathbb{E}\Vert b_F(u,X_u)\Vert^4\right]\dd u\\
        &\qquad+\epsilon^2(t_1-t_0)\int_{t_0}^{t_1}\mathbb{E}\Vert\Delta v(u,X_u)\Vert^2\dd u\\
        &\qquad+2\epsilon\int_{t_0}^{t_1}\mathbb{E}\Vert\nabla v(u,X_u^F)\Vert_F^2\dd u\\
        &\lesssim(t_1-t_0)^2\left[M_2+\gamma_{\min}^{-6}d^3+\gamma_{\min}^{-2}d\sqrt{\mathbb{E}\Vert x_0-x_1\Vert^{8}}\right]\\
        &\qquad+\epsilon(t_1-t_0)\gamma^{-2}d^{1}\sqrt{\mathbb{E}\Vert x_0-x_1\Vert^{4}}
    \end{aligned}$$
    Note that we have already used the condition $\gamma^2\in C^2[0,1]$ and $\gamma\dot{\gamma}=O(1)$.

    Similarly, we can use Lemma \ref{lem:s-time}, \ref{lem:s-space}, \ref{lem:s-laplace} and \ref{lem:vsb-bound}, and the formula
    $$\dd s(t,X_t^F)=[\partial_ts(t,X_t^F)+\nabla s(t,X_t^F)\cdot b_F(t,X_t^F)+\epsilon\Delta s(t,X_t^F)]\dd t+\sqrt{2\epsilon}\nabla s(t,X_t^F)\cdot \dd W_t$$
    to bound
    $$\begin{aligned}
        \mathbb{E}\Vert s(t_1,X_{t_1}^F)-s(t_0,X_{t_0}^F)\Vert^2\lesssim(t_1-t_0)^2\left[\gamma_{\min}^{-4}d^2\sqrt{\mathbb{E}\Vert x_0-x_1\Vert^4}+\gamma_{\min}^{-6}d^3\right]+\epsilon(t_1-t_0)\gamma_{\min}^{-4}d^2,
    \end{aligned}$$
    where $\gamma_{\min}=\min_{u\in[s,t]}\gamma$.
\end{proof}

\subsection{Proof of Theorem \ref{thm:main}}
\label{appendix:proofofmain}

We first give the following proposition, which is a result from \cite{chen2023ddpm}.

\begin{proposition}
    (Section 5.2 of \cite{chen2023ddpm}) Let $P$, $Q$ be the path measures of solutions of SDE (\ref{eq:forward-sde}) and (\ref{eq:estimated-sde}), where they both start from the same distribution $\rho(t_0)$ at time $t=t_0$ and end at time $t=t_N$. Then, if
    $$\mathbb{E}[\Vert b_F(t,X_t^F)-\hat{b}_F(t_k,X_{t_k}^F)\Vert^2]\le C$$
    for any $t\in[t_0,t_N]$ and some constant $C$, we have 
    $$\text{KL}(P\Vert Q)=\frac{1}{4\epsilon}\sum_{k=0}^{N-1}\int_{t_k}^{t_{k+1}}\mathbb{E}[\Vert b_F(t,X_t^F)-\hat{b}_F(t_k,X_{t_k}^F)\Vert^2]\dd t.$$
    Here the expectations are taken over the ground-truth forward process $(X_t^F)_{t\in[t_0,t_N]}\sim P$.
    \label{prop:girsanov}
\end{proposition}

Now, using the above proposition, we are ready to prove \Cref{thm:main}.

\begin{proof}
    Let $P$, $Q$ be the path measures of the solutions to the SDE (\ref{eq:forward-sde}) and (\ref{eq:estimated-sde}), where the solutions start from the same distribution $\rho(t_0)$ at time $t=t_0$, as in Proposition \ref{prop:girsanov}. We first want to check the condition of Proposition \ref{prop:girsanov}. Note that 
    $$\begin{aligned}
        \mathbb{E}\Vert\hat{b}_F(t_k,X_{t_k}^F)-b_F(t,X_t^F)\Vert^2&\overset{(a)}{\le}2\mathbb{E}\Vert\hat{b}_F(t_k,X_{t_k}^F)-b_F(t_k,X_{t_k}^F)\Vert^2+2\mathbb{E}\Vert b_F(t_k,X_{t_k}^F)-b_F(t,X_t^F)\Vert^2\\
        &\overset{(b)}{\le}2\mathbb{E}\Vert\hat{b}_F(t_k,X_{t_k}^F)-b_F(t_k,X_{t_k}^F)\Vert^2+4\mathbb{E}\Vert v(t_k,X_{t_k}^F)-v(t,X_t^F)\Vert^2\\
        &\qquad+4\mathbb{E}\Vert(-\gamma(t_k)\dot{\gamma}(t_k)+\epsilon)s(t_k,X_{t_k}^F)-(-\gamma(t_k)\dot{\gamma}(t_k)+\epsilon)s(t,X_t^F)\Vert^2\\
        &\overset{(c)}{\le}2\mathbb{E}\Vert\hat{b}_F(t_k,X_{t_k}^F)-b_F(t_k,X_{t_k}^F)\Vert^2+4\mathbb{E}\Vert v(t_k,X_{t_k}^F)-v(t,X_t^F)\Vert^2\\
        &\qquad+8(-\gamma(t_k)\dot{\gamma}(t_k)+\epsilon)^2\mathbb{E}\Vert s(t_k,X_{t_k}^F)-s(t,X_t^F)\Vert^2\\
        &\qquad+8(\gamma(t)\dot{\gamma}(t)-\gamma(t_k)\dot{\gamma}(t_k))^2\mathbb{E}\Vert s(t,X_t^F)\Vert^2.
    \end{aligned}$$
    Here (a), (b) and (c) use the triangle inequality and the fact $(a+b)^2\le a^2+b^2$. By Lemmas \ref{lem:vsb-bound} and \ref{lem:discretize}, this term is uniformly bounded in the closed interval $[t_0,t_N]$. In fact, we can apply these lemmas to obtain that
    $$\begin{aligned}
        \mathbb{E}\Vert\hat{b}_F(t_k,X_{t_k}^F)-b_F(t,X_t^F)\Vert^2&\overset{\text{(a)}}{\lesssim}\mathbb{E}\Vert\hat{b}_F(t_k,X_{t_k}^F)-b_F(t_k,X_{t_k}^F)\Vert^2\\
        &\qquad+(t-t_k)^2\left[M_2+\bar{\gamma}_k^{-6}d^3+\bar{\gamma}_k^{-2}d\sqrt{\mathbb{E}\Vert x_0-x_1\Vert^{8}}\right]\\
        &\qquad+\epsilon(t-t_k)\bar{\gamma}_k^{-2}d\sqrt{\mathbb{E}\Vert x_0-x_1\Vert^{4}}\\
        &\qquad+(t-t_k)^2\left[\bar{\gamma}_k^{-4}d^2\sqrt{\mathbb{E}\Vert x_0-x_1\Vert^4}+\bar{\gamma}_k^{-6}d^3\right]+\epsilon(t-t_k)\bar{\gamma}_k^{-4}d^2\\
        &\qquad+(t-t_k)^2\bar{\gamma}_k^{-2}d\\
        &\overset{\text{(b)}}{\lesssim}\mathbb{E}\Vert\hat{b}_F(t_k,X_{t_k}^F)-b_F(t_k,X_{t_k}^F)\Vert^2\\
        &\qquad+(t-t_k)^2\left[M_2+\bar{\gamma}_k^{-6}d^3+\bar{\gamma}_k^{-2}d\sqrt{\mathbb{E}\Vert x_0-x_1\Vert^{8}}\right]\\
        &\qquad+\epsilon(t-t_k)\bar{\gamma}_k^{-2}d\left[\sqrt{\mathbb{E}\Vert x_0-x_1\Vert^4}+\bar{\gamma}_k^{-2}d\right].
    \end{aligned}$$
    Here step (a) directly expands the discretization error using Lemmas \ref{lem:vsb-bound} and \ref{lem:discretize}; step (b) simplifies the terms by applying Young's inequality and that $1+\epsilon^2=O(1)$. Then, by Proposition \ref{prop:girsanov},
    $$\begin{aligned}
        \text{KL}(P\Vert Q)&=\frac{1}{4\epsilon}\sum_{k=0}^{N-1}\int_{t_k}^{t_{k+1}}\mathbb{E}[\Vert b_F(t,X_t^F)-\hat{b}_F(t_k,X_{t_k}^F)\Vert^2]\dd t\\
        &\overset{\text{(a)}}{\lesssim}\varepsilon_{b_F}^2+\epsilon^{-1}\sum_{k=0}^{N-1}(t_{k+1}-t_k)^3\left[M_2+\bar{\gamma}_k^{-6}d^3+\bar{\gamma}_k^{-2}d\sqrt{\mathbb{E}\Vert x_0-x_1\Vert^{8}}\right]\\
        &\qquad+\sum_{k=0}^{N-1}(t_{k+1}-t_k)^2\bar{\gamma}_k^{-2}d\left[\sqrt{\mathbb{E}\Vert x_0-x_1\Vert^4}+\bar{\gamma}_k^{-2}d\right].
    \end{aligned}$$
    Here step (a) just integrates over the upper bound of the disretization error. Now, consider $\text{KL}(\rho(t_N)\Vert\hat{\rho}(t_N))$. Let $\hat{Q}$ be the path measure of solutions of (\ref{eq:estimated-sde}) starting from $\hat{\rho}(t_0)$ instead of $\rho(t_0)$. Then,
    $$\begin{aligned}
        \text{KL}(\rho(t_N)\Vert\hat{\rho}(t_N))\le\text{KL}(P\Vert\hat{Q})&=\mathbb{E}_P\left[\log\frac{\dd P}{\dd\hat{Q}}(X)\right]\\
        &=\mathbb{E}_P\left[\log\left(\frac{\dd P}{\dd Q}(X)\cdot\frac{\dd Q}{\dd\hat{Q}}(X)\right)\right]\\
        &=\mathbb{E}_P\left[\log\frac{\dd P}{\dd Q}(X)\right]+\mathbb{E}_P\left[\log\frac{\dd\rho(t_0)}{\dd\hat{\rho}(t_0)}(X_{t_0})\right]\\
        &=\text{KL}(P\Vert Q)+\text{KL}(\rho(t_0)\Vert\hat{\rho}(t_0)).
    \end{aligned}$$
    The proof is then completed.
\end{proof}

\subsection{Proof of Proposition \ref{cor:schedule}}
\label{appendix:proofofcor}

\begin{proof}
    Using the results of Theorem \ref{thm:main}, 
    $$\begin{aligned}
        \text{KL}(\rho(t_N)\Vert\hat{\rho}(t_N))&\overset{\text{(a)}}{\lesssim}\varepsilon_{b_F}^2+\text{KL}(\rho(t_0)\Vert\hat{\rho}(t_0))+\epsilon^{-1}\sum_{k=0}^{N-1}(t_{k+1}-t_k)^3\left[M_2+\bar{\gamma}_k^{-6}d^3+\bar{\gamma}_k^{-2}d\sqrt{\mathbb{E}\Vert x_0-x_1\Vert^{8}}\right]\\
        &\qquad+\sum_{k=0}^{N-1}(t_{k+1}-t_k)^2\bar{\gamma}_k^{-2}d\left[\sqrt{\mathbb{E}\Vert x_0-x_1\Vert^4}+\bar{\gamma}_k^{-2}d\right]\\
        &\overset{\text{(b)}}{\lesssim}\varepsilon_{b_F}^2+\text{KL}(\rho(t_0)\Vert\hat{\rho}(t_0))+\epsilon^{-1}\sum_{k=0}^{N-1}\left[M_2h_k^3+h^3d^3+hh_k^2d\sqrt{\mathbb{E}\Vert x_0-x_1\Vert^8}\right]\\
        &\qquad+\sum_{k=0}^{N-1}\left[hh_kd\sqrt{\mathbb{E}\Vert x_0-x_1\Vert^4}+h^2d^2\right]\\
        &\overset{\text{(c)}}{\lesssim}\varepsilon_{b_F}^2+\text{KL}(\rho(t_0)\Vert\hat{\rho}(t_0))+\epsilon^{-1}h^2\left(M_2+d\sqrt{\mathbb{E}\Vert x_0-x_1\Vert^8}\right)+\epsilon^{-1}Nh^3d^3\\
        &\qquad+hd\sqrt{\mathbb{E}\Vert x_0-x_1\Vert^4}+Nh^2d^2\\
        &\overset{\text{(d)}}{\lesssim}\varepsilon_{b_F}^2+\text{KL}(\rho(t_0)\Vert\hat{\rho}(t_0))+hd\sqrt{\mathbb{E}\Vert x_0-x_1\Vert^4}+Nh^2d^2.
    \end{aligned}$$
    Here step (a) is the result of Theorem \ref{thm:main}; step (b) uses the fact $h_k=t_{k+1}-t_k=O(h\bar{\gamma}_k^2)$; step (c) uses the fact $\sum_{k=0}^{N-1}h_k=t_N-t_0\le 1$; step (d) omits the higher-order terms.
\end{proof}

\subsection{Proof of Corollary \ref{cor:instant}}
\label{appendix:proofofschedule}

\begin{proof}
    When the number of steps is $N$, we have 
    $$h=\Theta\left(N^{-1}\log\left(\frac{1}{t_0(1-t_N)}\right)\right).$$ 
    Then, by Corollary \ref{cor:schedule} and the assumptions,
    $$\text{KL}(\rho(t_N)\Vert\hat{\rho}(t_N))\lesssim\varepsilon^2+N^{-1}\left[d\sqrt{\mathbb{E}\Vert x_0-x_1\Vert^4}\log\left(\frac{1}{t_0(1-t_N)}\right)+d^2\log^2\left(\frac{1}{t_0(1-t_N)}\right)\right].$$
    This gives the complexity to to make $\text{KL}(\rho(t_N)\Vert\hat{\rho}(t_N))\lesssim\varepsilon^2$.
\end{proof}
%     $$\begin{aligned}
%         \text{KL}(\rho(t_N)\Vert\hat{\rho}(t_N))&\lesssim\varepsilon_{b_F}^2+\text{KL}(\rho(t_0)\Vert\hat{\rho}(t_0))+\sum_{k=0}^{N-1}(t_{k+1}-t_k)^3\left[M_2+\bar{\gamma}_k^{-6}d^3+\bar{\gamma}_k^{-2}d\sqrt{\mathbb{E}\Vert x_0-x_1\Vert^{8}}\right]\\
%         &\qquad+\sum_{k=0}^{N-1}(t_{k+1}-t_k)^2\bar{\gamma}_k^{-2}d\left[\sqrt{\mathbb{E}\Vert x_0-x_1\Vert^4}+\bar{\gamma}_k^{-2}d\right]\\
%         &\lesssim\varepsilon_{b_F}^2+\text{KL}(\rho(t_0)\Vert\hat{\rho}(t_0))+\sum_{k=0}^{N-1}\left[h_k^3M_2+h^3d^3+hh_k^2d\sqrt{\mathbb{E}\Vert x_0-x_1\Vert^8}\right]\\
%         &\qquad+\sum_{k=0}^{N-1}\left[hh_kd\sqrt{\mathbb{E}\Vert x_0-x_1\Vert^4}+h^2d^2\right]\\
%         &\lesssim\varepsilon_{b_F}^2+\text{KL}(\rho(t_0)\Vert\hat{\rho}(t_0))+N(h^3d^3+h^2d^2)\\
%         &\qquad+M_2h^3\sum_{k=0}^{N-1}\left(\min\{t_k,1-t_{k+1}\}\right)^3\\
%         &\qquad+\sqrt{\mathbb{E}\Vert x_0-x_1\Vert^8}h^3d\sum_{k=0}^{N-1}\left(\min\{t_k,1-t_{k+1}\}\right)^2\\
%         &\qquad+\sqrt{\mathbb{E}\Vert x_0-x_1\Vert^4}h^2d\sum_{k=0}^{N-1}\min\{t_k,1-t_{k+1}\}.
%     \end{aligned}$$
%     It is easy to see that $$\sum_{k=0}^{N-1}\min\{t_k,1-t_{k+1}\}^p\lesssim\sum_{k=0}^\infty(1-h)^{kp}\le\frac{1}{ph},$$
%     so $$\begin{aligned}
%         \text{KL}(\rho(t_N)\Vert\hat{\rho}(t_N))&\lesssim\varepsilon_{b_F}^2+\text{KL}(\rho(t_0)\Vert\hat{\rho}(t_0))+(h^2d^3+hd^2)(\log(1/t_0)+\log(1/\delta))\\
%         &\qquad+h^2\left(M_2+\sqrt{\mathbb{E}\Vert x_0-x_1\Vert^8}d\right)\\
%         &\qquad+h\sqrt{\mathbb{E}\Vert x_0-x_1\Vert^4}d.
%     \end{aligned}$$
% When $h\lesssim 1/d$, view $\sqrt{\mathbb{E}\Vert x_0-x_1\Vert^8}$ and $M_2$ as constants (Assumption \ref{a:regularity}), then we have
% $$\begin{aligned}
%     \text{KL}(\rho(t_N)\Vert\hat{\rho}(t_N))&\lesssim\varepsilon_{b_F}^2+\text{KL}(\rho(t_0)\Vert\hat{\rho}(t_0))+h\left[d^2\log\left(\frac{1}{t_0(1-t_0)}\right)+d\sqrt{\mathbb{E}\Vert x_0-x_1\Vert^4}\right].
% \end{aligned}$$

% $$\begin{aligned}
%     \text{KL}(\rho(t_N)\Vert\hat{\rho}(t_N))&\lesssim\varepsilon_{b_F}^2+\text{KL}(\rho(t_0)\Vert\hat{\rho}(t_0))\\
%     &+N^{-1}\left[d^2\log^2\left(\frac{1}{t_0(1-t_0)}\right)+d\sqrt{\mathbb{E}\Vert x_0-x_1\Vert^4}\log\left(\frac{1}{t_0(1-t_0)}\right)\right].
% \end{aligned}$$
% This gives the result.


\subsection{Reducing to Diffusion Models}
\label{appendix:reduce-to-gaussian}

By modifying the definition of stochastic interpolant to $$x_t=I(t,x_1)+\gamma(t)z$$
and change the condition on $I$ to $\Vert\partial_tI(t,x_1)\Vert\le C\Vert x_1\Vert$, we can repeat the previous analysis while replacing $\sqrt{\mathbb{E}\Vert x_0-x_1\Vert^p}$ by $\sqrt{\mathbb{E}\Vert x_1\Vert^p}$. For the case of diffusion models, we can choose $I(t,x_1)=tx_1$ and $\gamma(t)=\sqrt{1-t^2}$ to obtain a process with the same marginal distributions. Moreover, under this definition of interpolants, we can choose $t_0=0$ and $h_k=t_{k+1}-t_k\propto(1-t_k)$ as the time schedule to recover the sample complexity of diffusion models.

\subsection{Omitted Proofs for $\gamma^2(t)=(1-t)^2t$}
\label{appendix:another}

In this section, we will design a schedule for $\gamma^2(t)=(1-t)^2t$, and provide the corresponding complexity deduced using \Cref{thm:main}. Moreover, we also derived the complexity of using a uniform schedule for comparison.

\begin{corollary}
    For $\gamma^2(t)=(1-t)^2t$, there exists a schedule so that under the same assumptions as \Cref{cor:instant}, the complexity is given by
    $$N=O\left(\frac{1}{\varepsilon^2}\left[\sqrt{\mathbb{E}\Vert x_0-x_1\Vert^4}d\left(\frac{1}{\sqrt{1-t_N}}+\log\left(\frac{1}{t_0}\right)\right)+d^2\left(\frac{1}{(1-t_N)^2}+\log^2\left(\frac{1}{t_0}\right)\right)\right]\right).$$
    In addition, the complexity for using a uniform schedule is $$N=O\left(\frac{1}{\varepsilon^2}\left[\sqrt{\mathbb{E}\Vert x_0-x_1\Vert^4}d\left(\frac{1}{1-t_N}+\log\left(\frac{1}{t_0}\right)\right)+d^2\left(\frac{1}{(1-t_N)^3}+\frac{1}{t_0}\right)\right]\right).$$
\end{corollary}

\begin{proof}
    Here we also take $h_M=0.5$ for some $M>0$. Then, we define
    $$h_k=\begin{cases}
        h_A\cdot t_{k+1},&k<M\\
        h_B\cdot (1-t_k)^{1.5},&k\ge M.
    \end{cases}$$
    for some $h_A\in[0,0.5),h_B\in[0,1)$.
    For the part $k<M$ and $t_k\in[0,0.5)$, $\gamma^2(t_k)=\Theta(t_k)$, so it is the same as what we have discussed for the case, and we need $$M=N_1=O\left(\frac{1}{\varepsilon^2}\left[\sqrt{\mathbb{E}\Vert x_0-x_1\Vert^4}d\log\left(\frac{1}{t_0}\right)+d^2\log^2\left(\frac{1}{t_0}\right)\right]\right)$$
    steps to make the discretization error $$\begin{aligned}
        &\varepsilon_{b_F}^2+\sum_{k=0}^{M-1}(t_{k+1}-t_k)^3\left[M_2+\bar{\gamma}_k^{-6}d^3+\bar{\gamma}_k^{-2}d\sqrt{\mathbb{E}\Vert x_0-x_1\Vert^{8}}\right]\\
        &\qquad+\sum_{k=0}^{M-1}(t_{k+1}-t_k)^2\bar{\gamma}_k^{-2}d\left[\sqrt{\mathbb{E}\Vert x_0-x_1\Vert^4}+\bar{\gamma}_k^{-2}d\right]\lesssim\varepsilon^2.
    \end{aligned}$$
    For the part $k\ge M$, 
    $$\sum_{k=M}^{N-1}(t_{k+1}-t_k)^3\left[M_2+\bar{\gamma}_k^{-6}d^3+\bar{\gamma}_k^{-2}d\sqrt{\mathbb{E}\Vert x_0-x_1\Vert^{8}}\right]=O(h_B^2),$$
    and by that $h_k=\Theta(h_B\bar{\gamma}_k^{1.5})=\Theta(h_B(1-t_k)^{1.5})$ (use in step (a) below),
    $$\begin{aligned}
        &\qquad\sum_{k=M}^{N-1}h_k^2\bar{\gamma}_k^{-2}d\left[\sqrt{\mathbb{E}\Vert x_0-x_1\Vert^4}+\bar{\gamma}_k^{-2}d\right]\\
        &\overset{\text{(a)}}{\lesssim} h_B\sum_{k=M}^{N-1}h_k\left[d\sqrt{\mathbb{E}\Vert x_0-x_1\Vert^4}\bar{\gamma}_k^{-0.5}+\bar{\gamma}_k^{-2.5}d^2\right]\\
        &\overset{\text{(b)}}{\lesssim} h_B\left[d\sqrt{\mathbb{E}\Vert x_0-x_1\Vert^4}\int_{0.5}^{t_N}(1-s)^{-0.5}\dd s+d^2\int_{0.5}^{t_N}(1-s)^{-2.5}\dd s\right]\\
        &\lesssim h_B\left[d\sqrt{\mathbb{E}\Vert x_0-x_1\Vert^4}+d^2\frac{1}{(1-t_N)^{1.5}}\right].
    \end{aligned}$$
    Here the inequality (b) is by that $\bar{\gamma}_k=\Theta(1-t)$ for $t\in[t_k,t_{k+1}]$. Now, we want to compute the number of steps $N_2=N-M$ for the part $k\ge M$. Note that if $t_k=1-2^{-p}$, it takes $O(2^{p/2}h_B^{-1})$ more steps to reach $1-2^{-p-1}$. Hence $N_2=O\left(h_B^{-1}(1-t_N)^{-0.5}\right)$, so we need to take $h_B=\Theta\left(N^{-1}(1-t_N)^{-0.5}\right)$.
    Therefore,
    $$\begin{aligned}
        &\qquad\sum_{k=M}^{N-1}h_k^2\bar{\gamma}_k^{-2}d\left[\sqrt{\mathbb{E}\Vert x_0-x_1\Vert^4}+\bar{\gamma}_k^{-2}d\right]\\
        &\lesssim N^{-1}\left[\frac{d\sqrt{\mathbb{E}\Vert x_0-x_1\Vert^4}}{\sqrt{1-t_N}}+\frac{d^2}{(1-t_N)^{2}}\right].
    \end{aligned}$$
    Thus, for the part $k>M$, we need $$N-M=N_2=O\left(\frac{1}{\varepsilon^2}\left(\frac{d\sqrt{\mathbb{E}\Vert x_0-x_1\Vert^4}}{\sqrt{1-t_N}}+\frac{d^2}{(1-t_N)^{2}}\right)\right)$$
    steps to make the discretization error bounded by $O(\varepsilon^2)$. Hence, the overall complexity is given by $N=N_1+N_2$, which is our result.

    If we use a uniform schedule, by \Cref{thm:main} and that $\gamma^2(t)=\Theta(\min\{t,(1-t)^2\})$, we can bound
    $$\begin{aligned}
        \text{KL}(\rho(t_N)\Vert\hat{\rho}(t_N))&\overset{\text{(a)}}{\lesssim}\varepsilon_{b_F}^2+\text{KL}(\rho(t_0)\Vert\hat{\rho}(t_0))\\
        &\qquad+\frac{1}{N}\sqrt{\mathbb{E}\Vert x_0-x_1\Vert^4}d\left(\int_{t_0}^{0.5}s^{-1}\dd s+\int_{0.5}^{t_N}(1-s)^{-2}\dd s\right)\\
        &\qquad+\frac{1}{N}d^2\left(\int_{t_0}^{0.5}s^{-2}\dd s+\int_{0.5}^{t_N}(1-s)^{-4}\dd s\right)\\
        &\lesssim\varepsilon_{b_F}^2+\text{KL}(\rho(t_0)\Vert\hat{\rho}(t_0))\\
        &\qquad+\frac{1}{N}\sqrt{\mathbb{E}\Vert x_0-x_1\Vert^4}d\left(\frac{1}{1-t_N}+\log\left(\frac{1}{t_0}\right)\right)\\
        &\qquad+\frac{1}{N}d^2\left(\frac{1}{(1-t_N)^3}+\frac{1}{t_0}\right),
    \end{aligned}$$
    which further gives the complexity bound for uniform schedule. Here the inequality (a) is by applying \Cref{thm:main} and replacing $\bar{\gamma}_k$ with the term of the same order. This bound is worse than using the schedule satisfying that $h_k\lesssim h\bar{\gamma}_k$.
\end{proof}
% \section{Reproducibility Checklist}
% Please refer to the technical appendix titled 'Reproducibility Checklist', which is attached in the supplementary material of this submission.

\section{Experimental Setup \label{sec:hyperParams}} 
All training experiments were performed
on public datasets using a single A100 40GB GPU for a
maximum of two days. All experiments were conducted using PyTorch, and results are averaged over three seeds. All hyperparameters are detailed in Tab.~\ref{tab:NLPhyperpams} and Tab.~\ref{tab:Vsionhyperpams}.

\begin{table}[h]
\centering
\small
\begin{tabular}{l c}
\toprule
\textbf{Parameter} & \textbf{Value} \\
\midrule
Model-width & 192 \\
Number of layers & 24 \\
Number of patches & 196 \\
%Scan Mode {\color{red} ???} & one directional \\
Batch-size & 512 \\
Optimizer & AdamW \\
Momentum & \( \beta_1, \beta_2 = 0.9, 0.999 \) \\
Base learning rate & $5e-4$ \\
Weight decay & 0.1 \\
Dropout & 0 \\
Training epochs & 300 \\
Learning rate schedule & cosine decay \\
Warmup epochs & 5 \\
Warmup schedule & linear \\ 
Degree of Taylor approx. (Eq.~\ref{eq:simplifiedModel}) & 3 \\
\bottomrule
\end{tabular}
\caption{Hyperparameters for image-classification via Vision Mamba variants} 
\label{tab:Vsionhyperpams}
\end{table}

\begin{table}[h]
\centering
\small
\begin{tabular}{l c}
\toprule
\textbf{Parameter} & \textbf{Value} \\
\midrule
Model-width & 386 \\
Number of layers & 12 \\
Context-length (training) & 1024 \\
Batch-size & 32 \\
Optimizer & AdamW \\
Momentum & \( \beta_1, \beta_2 = 0.9, 0.999 \) \\
Base learning rate & $1.5e-3$ \\
Weight decay & 0.01 \\
Dropout & 0 \\
Training epochs & 20 \\
Learning rate schedule & cosine decay  \\
Warmup epochs & 1  \\
Warmup schedule & linear  \\ 
Degree of Taylor approx. (Eq.~\ref{eq:simplifiedModel}) &  3\\
\bottomrule
\end{tabular}
\caption{Hyperparameters for language modeling via Mamba-based LMs} 
\label{tab:NLPhyperpams}
\end{table}

% \section{Additional Background Material}

% {\noindent\textbf{Rademacher Complexities}}
% We explore the generalization capabilities of overparameterized NNs by analyzing their Rademacher complexity. This measure provides an upper bound on the worst-case generalization gap, which represents the difference between training and testing errors within a specific hypothesis class. It is defined as the expected performance of the class averaged over all possible data labelings, with labels independently and uniformly drawn from the set $\{\pm 1\}$. For further details, refer to \citep{mohri2018foundations, Shalev-Shwartz2014, bartlett2002rademacher}.


% \begin{definition}[Rademacher Complexity] Let $\mathcal{F}$ be a set of real-valued functions $f_w:\mathcal{X} \to \mathbb{R}^\mathcal{C}$ defined over a set $\mathcal{X}$. Given a fixed sample $X = \{ x_j\}_{j=1}^m \in \mathcal{X}^m$, the empirical Rademacher complexity of $\mathcal{F}$ is defined as follows: 
% \begin{equation*}
% \mathcal{R}_{X}(\mathcal{F}) ~:=~ \frac{1}{m} \mathbb{E}_{\xi: \xi_{ic} \sim U[\{\pm 1\}]} \left[ \sup_{f_w \in \mathcal{F}} \sum^{m}_{j=1}\sum^{\mathcal{C}}_{c=1} \xi_{ic} f_w(x_j)_c  \right].
% \end{equation*}
% \end{definition}

% {\color{red}
% It should be moved:
% We focus on training models for classification tasks. The problem is formally defined by a distribution $P$ over pairs $(x,y)\in \cX\times \cY$, where $\cX \subset \R^{\mathcal{D}}$ represents the input space, and $\cY \subset \R^\mathcal{C}$ is the label space containing one-hot encoded labels for the integers $1,\ldots,\mathcal{C}$. 

% We define a hypothesis class $\cF \subset \{f_w:\cX\to \R^\mathcal{C}\}$ (such as a neural network architecture), where each function \( f_w \in \cF \) is parameterized by a vector of trainable weights $w$. Given any input \( x \in \cX \), \( f_w \) provides a predicted label, and the performance is evaluated based on the \emph{expected error}, \(\err_P(f_w) := \mathbb{E}_{(x, y) \sim P}\left[\mathbb{I}\left[\max_{j \neq y}(f_w(x)_j) \geq f_w(x)_{y}\right]\right]\). Here, the indicator function \(\mathbb{I}\) returns 1 for True and 0 for False.

% Since the full distribution \( P \) is not directly accessible, our objective is to train a model \( f_w \) using a training dataset \( S = \{(x_i, y_i)\}_{i = 1}^m \) consisting of independent and identically distributed (i.i.d.) samples from \( P \). We aim to achieve accurate predictions while applying regularization to control the complexity of \( f_w \). 

% To denote the entire $n$th row and $n$th column of a matrix $M$, we use the notation:
% \begin{align*}
% M_{n*} & \text{ denotes the entire } n \text{th row of matrix } M. \\
% M_{*n} & \text{ denotes the entire } n \text{th column of matrix } M. \\
% M_{nk} & \text{ denotes the element in the } n \text{th row and } k \text{th column.}
% \end{align*}
% {\bf Selective State Space Models.\enspace} 
% Time-variant SSMs, namely, the matrices $A,B,C$ of each channel are modified over $L$ time steps. We are focusing on selective SSMs of the following form. 
% A neural network $f_w : \mathbb{R}^{D \times L} \rightarrow \mathbb{R}^{\mathcal{C}}$ takes a sequence $x=(x_{*1},...,x_{*L}) \in \mathbb{R}^{D \times L}$ as input where $L$ is the length of the sequence and $D$ is the dimension of the tokens $x_i$. We denote
% \begin{align*}
%     B_i &= B x_{*i}, B \in \mathbb{R}^{N \times D} \\
%     C_i &= C x_{*i}, C \in \mathbb{R}^{N \times D} \\
%     \Delta_{*i} &= S_{\Delta} x_{*i}, S_{\Delta} \in \mathbb{R}^{D \times D} \rightarrow \Delta \in \mathbb{R}^{D \times L} \\
%     \bar{A}_{j,i} &= (z^2 + \alpha z)(\Delta_{j,i} * \underbrace{A_{j*}}_{\textbf{ $j$th row of A } } ), A \in \mathbb{R}^{D \times N}, \text{$(z^2+\alpha z)$ is an activation function, } \alpha \geq 0 \\ 
%     W &\in \mathbb{R}^{\mathcal{C} \times D} 
% \end{align*}
% }

\end{document}
