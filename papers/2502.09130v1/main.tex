\documentclass{article} % For LaTeX2e

\usepackage{geometry}
\geometry{a4paper, scale=0.75}
\usepackage{natbib}

\usepackage[utf8]{inputenc} % allow utf-8 input
\usepackage[T1]{fontenc}    % use 8-bit T1 fonts

\usepackage{titletoc}
\usepackage[toc, page, header]{appendix} %%% MAKE SURE TO PUT THIS BEFORE hyperref PACKAGE

\usepackage[colorlinks=true, linkcolor=blue, citecolor=blue,urlcolor=black]{hyperref}
\usepackage{url}            % simple URL typesetting
\usepackage{booktabs}       % professional-quality tables
\usepackage{amsfonts}       % blackboard math symbols
\usepackage{nicefrac}       % compact symbols for 1/2, etc.
\usepackage{microtype}      % microtypography
\usepackage{xcolor}         % colors
\usepackage{footnote}
% \usepackage{microtype}
\usepackage{graphicx}
% \usepackage{subfigure}
\usepackage{subcaption}
\usepackage{booktabs} % for professional tables

% tables
\usepackage{multirow}
\usepackage{colortbl}
\usepackage{arydshln}
\setlength\dashlinedash{0.4pt}
\setlength\dashlinegap{2pt}
\setlength\arrayrulewidth{0.4pt}

\usepackage{amsmath}
\usepackage{amsthm}
\usepackage{amssymb}
\usepackage{mathtools}

\usepackage{comment}
\usepackage{bm}
\usepackage{bbm}
\usepackage{enumitem}
\setitemize{leftmargin=*}
\setenumerate{leftmargin=*}

% Attempt to make hyperref and algorithmic work together better:
\newcommand{\theHalgorithm}{\arabic{algorithm}}
\newcommand{\alglinelabel}{%
  \addtocounter{ALC@line}{-1}% Reduce line counter by 1
  \refstepcounter{ALC@line}% Increment line counter with reference capability
  \label% Regular \label
}
\usepackage{algorithm}
\usepackage[noend]{algorithmic}
\renewcommand{\algorithmiccomment}[1]{ \hfill $\triangleright$ {\color{blue} #1}}
\renewcommand{\algorithmicrequire}{\textbf{Input:}}
\renewcommand{\algorithmicensure}{\textbf{Output:}}

\allowdisplaybreaks
\renewcommand\topfraction{0.85}
\renewcommand\bottomfraction{0.85}
\renewcommand\textfraction{0.1}
\renewcommand\floatpagefraction{0.85}

\renewcommand{\tilde}{\widetilde}
\renewcommand{\hat}{\widehat}
\renewcommand{\bar}{\overline}
\renewcommand{\O}{\operatorname{\mathcal O}}
\newcommand{\Otil}{\operatorname{\tilde{\mathcal O}}}

\usepackage{cleveref}
\newtheorem{theorem}{Theorem}
\newtheorem{lemma}[theorem]{Lemma}
\newtheorem{proposition}[theorem]{Proposition}
\newtheorem{corollary}[theorem]{Corollary}
\newtheorem{definition}[theorem]{Definition}
\newtheorem{assumption}[theorem]{Assumption}
\newtheorem{remark}[theorem]{Remark}

\Crefname{ALC@line}{Line}{Lines}
\Crefname{assumption}{Assumption}{Assumptions}
\Crefformat{equation}{Eq. #2(#1)#3}
\Crefrangeformat{equation}{Eqs. #3(#1)#4 to #5(#2)#6}
\Crefmultiformat{equation}{Eqs. #2(#1)#3}{ and #2(#1)#3}{, #2(#1)#3}{ and #2(#1)#3}
\Crefrangemultiformat{equation}{Eqs. #3(#1)#4 to #5(#2)#6}{ and #3(#1)#4 to #5(#2)#6}{, #3(#1)#4 to #5(#2)#6}{ and #3(#1)#4 to #5(#2)#6}
\newcommand{\creflastconjunction}{, and\nobreakspace}

\newenvironment{tightcenter}{%
\setlength\topsep{2pt}
\setlength\parskip{0pt}
\begin{center}
}{%
\end{center}
}

% \newcommand{\longbo}[1]{{\color{red}  [\text{Longbo:} #1]}}
% \newcommand{\daiyan}[1]{{\color{violet} [\text{Yan:} #1]}}
% \newcommand{\yu}[1]{{\color{cyan}  [\text{Yu:} #1]}}
% \newcommand{\jiatai}[1]{{\color{magenta}  [\text{Jiatai:} #1]}}

%%%%% NEW MATH DEFINITIONS %%%%%

% \usepackage{amsmath,amsfonts,bm}
\usepackage{amsmath,amsfonts}

\usepackage{pifont}


\newcommand{\R}{\mathbb{R}}


\def\va{{\mathbf{a}}}
\def\vg{{\mathbf{g}}}

% Sets
\def\sR{\mathbb{R}}
\def\sC{\mathbb{C}}
\def\sZ{\mathbb{Z}}
\def\sN{\mathbb{N}}
\def\sQ{\mathbb{Q}}

\def\sS{\mathcal{S}}



% Vectors
\def\vzero{{\mathbf{0}}}
\def\vone{{\mathbf{1}}}
\def\vmu{{\mathbf{\mu}}}
\def\vtheta{{\mathbf{\theta}}}
\def\va{{\mathbf{a}}}
\def\vb{{\mathbf{b}}}
\def\vc{{\mathbf{c}}}
\def\vd{{\mathbf{d}}}
\def\ve{{\mathbf{e}}}
\def\vf{{\mathbf{f}}}
\def\vg{{\mathbf{g}}}
\def\vh{{\mathbf{h}}}
\def\vi{{\mathbf{i}}}
\def\vj{{\mathbf{j}}}
\def\vk{{\mathbf{k}}}
\def\vl{{\mathbf{l}}}
\def\vm{{\mathbf{m}}}
\def\vn{{\mathbf{n}}}
\def\vo{{\mathbf{o}}}
\def\vp{{\mathbf{p}}}
\def\vq{{\mathbf{q}}}
\def\vr{{\mathbf{r}}}
\def\vs{{\mathbf{s}}}
\def\vt{{\mathbf{t}}}
\def\vu{{\mathbf{u}}}
\def\vv{{\mathbf{v}}}
\def\vw{{\mathbf{w}}}
\def\vx{{\mathbf{x}}}
\def\vy{{\mathbf{y}}}
\def\vz{{\mathbf{z}}}
\def\vzeta{{\mathbf{\zeta}}}

% Matrix
\def\mA{{\mathbf{A}}}
\def\mB{{\mathbf{B}}}
\def\mC{{\mathbf{C}}}
\def\mD{{\mathbf{D}}}
\def\mE{{\mathbf{E}}}
\def\mF{{\mathbf{F}}}
\def\mG{{\mathbf{G}}}
\def\mH{{\mathbf{H}}}
\def\mI{{\mathbf{I}}}
\def\mJ{{\mathbf{J}}}
\def\mK{{\mathbf{K}}}
\def\mL{{\mathbf{L}}}
\def\mM{{\mathbf{M}}}
\def\mN{{\mathbf{N}}}
\def\mO{{\mathbf{O}}}
\def\mP{{\mathbf{P}}}
\def\mQ{{\mathbf{Q}}}
\def\mR{{\mathbf{R}}}
\def\mS{{\mathbf{S}}}
\def\mT{{\mathbf{T}}}
\def\mU{{\mathbf{U}}}
\def\mV{{\mathbf{V}}}
\def\mW{{\mathbf{W}}}
\def\mX{{\mathbf{X}}}
\def\mY{{\mathbf{Y}}}
\def\mZ{{\mathbf{Z}}}
\def\mBeta{{\mathbf{\beta}}}
\def\mPhi{{\mathbf{\Phi}}}
\def\mLambda{{\mathbf{\Lambda}}}
\def\mSigma{{\mathbf{\Sigma}}}


% Expectation
% \def\eE{\mathop{\mathbb{E}}\limits}
\def\eE{\mathbb{E}}

% Probability
\def\pP{\mathbb{P}}

% Tilde
\def\tf{\tilde{f}}
\def\tS{\tilde{S}}
\def\wtF{\widetilde{\mathcal{F}}}
\def\whR{\widehat{R}}
\def\tvx{\tilde{\mathbf{x}}}
\def\ty{\tilde{y}}


\def\defeq{\overset{\textup{def}}{=}}
% \def\defeq{\overset{.}{=}}
\def\defone{\overset{\text{\ding{172}}}{=}}
\def\deftwo{\overset{\text{\ding{173}}}{=}}
\def\leqone{\overset{\text{\ding{172}}}{\leq}}
\def\leqtwo{\overset{\text{\ding{173}}}{\leq}}
\def\leqthree{\overset{\text{\ding{174}}}{\leq}}
\def\leqfour{\overset{\text{\ding{175}}}{\leq}}
\def\eqone{\overset{\text{\ding{172}}}{=}}
\def\eqtwo{\overset{\text{\ding{173}}}{=}}
\def\eqthree{\overset{\text{\ding{174}}}{=}}
\def\eqfour{\overset{\text{\ding{175}}}{=}}
\def\geqfive{\overset{\text{\ding{176}}}{\geq}}
\newcommand{\EX}{\mathbb{E}}
\newcommand{\pll}{\kern 0.56em/\kern -0.8em /\kern 0.56em} 
\newcommand{\defeq}{\overset{\underset{\triangle}{}}{=}}
\newcommand{\Hesse}{\text{Hess}}
\newcommand{\sgn}{\text{sgn}}
\newcommand{\T}{\text{T}}
\newcommand{\holder}{H{\"o}lder's inequality}
\newcommand{\Span}{\text{span}}
\newcommand{\weakto}{\rightharpoonup}
\newcommand{\lloc}{L_{\text{loc}}}
\newcommand{\supp}{\text{supp}}
\newcommand{\dd}{\textnormal{d}}

\newcommand{\longbo}[1]{\textcolor{red}{[\text{Longbo:} #1]}}
\newcommand{\yu}[1]{{\color{cyan}[\text{Yu:} #1]}}
\newcommand{\rui}[1]{{\color{purple}[\text{Rui:} #1]}}
\newcommand{\yuhaoDone}{{\color{green}  [\text{Done}]}}

\usepackage{pifont}
\newcommand{\cmark}{{\color{green!70!black}\ding{51}}}%
\newcommand{\xmark}{{\color{red!70!black}\ding{55}}}%
\newcommand{\adv}{{\color{green!70!black}\textbf{Adv.}}}%
\newcommand{\stoc}{{\color{red!70!black}\textbf{Only Stoc.}}}%
\newcommand{\advred}{{\color{red!70!black}\textbf{Only Adv.}}}%
\newcommand{\stocgreen}{{\color{green!70!black}\textbf{Stoc.}}}%

\newcommand{\Clip}{{\text{clip}}}
\newcommand{\Skip}{{\text{skip}}}

\title{
Finite-Time Analysis of Discrete-Time Stochastic Interpolants
}

% \usepackage{authblk}
\author{
Yuhao Liu \thanks{IIIS, Tsinghua University. Email: \texttt{liuyuhao21@mails.tsinghua.edu.cn}.}
\and
Yu Chen \thanks{IIIS, Tsinghua University. Email: \texttt{chenyu23@mails.tsinghua.edu.cn}.}
\and
Rui Hu \thanks{IIIS, Tsinghua University. Email: \texttt{hu-r24@mails.tsinghua.edu.cn}.}
\and
Longbo Huang \thanks{IIIS, Tsinghua University. Email: \texttt{longbohuang@tsinghua.edu.cn}. Corresponding author.}
}
\date{}

\newcommand{\fix}{\marginpar{FIX}}
\newcommand{\new}{\marginpar{NEW}}


\begin{document}
\maketitle


\begin{abstract}
The stochastic interpolant framework offers a powerful approach for constructing generative models based on ordinary differential equations (ODEs) or stochastic differential equations (SDEs) to transform arbitrary data distributions. However, prior analyses of this framework have primarily focused on the continuous-time setting, assuming a perfect solution of the underlying equations. In this work, we present the first discrete-time analysis of the stochastic interpolant framework, where we introduce an innovative discrete-time sampler and derive a finite-time upper bound on its distribution estimation error. Our result provides a novel quantification of how different factors, including 
the distance between source and target distributions and estimation accuracy, affect the convergence rate and also offers a new principled way to design efficient schedules for convergence acceleration. Finally, numerical experiments are conducted on the discrete-time sampler to corroborate our theoretical findings. 
\end{abstract}

\documentclass[../main.tex]{subfiles}
\graphicspath{{../images/}}
\makeatletter
\def\input@path{{../images/}}
\makeatother
\begin{document}
\section{Introduction}
\begin{figure}
\centering
\begin{tikzpicture}
\node[inner sep=0pt] (ws) at (0, 0) {
\includegraphics[height=.4\textwidth, trim={10cm 0 10cm 0},clip]{world_space.png}};
\node[inner sep=0pt] (cs) at (6,0) {\includegraphics[height=.4\textwidth, trim={10cm 1cm 10cm 4cm},clip]{conf_space.png}};
\end{tikzpicture}
\vspace{-5pt}
\label{fig:pbrm_intro}
\caption{\textbf{Left}: Shows world space obstacles as grey spheres. Robots start and goal configuration is colored red and green, respectively. Configurations along the computed path are colored transparent blue. \textbf{Right:} Mapped world space scenario to configuration space. Obstacle region is the grey mesh. Red spheres are collision-free regions computed by the neural SCDF. The optimized shortest path in the convex corridor is the blue curve.}
\vspace{-25pt}
\end{figure}
Motion planning is the problem of finding a collision-free trajectory that connects a given start and goal configuration. The planning takes place in the configuration space of the robot. For single body robots, like mobile robots or drones, the configuration space and the world space are usually the same. This simplifies the planning, since explicit obstacle representations are available which enables geometrical tools like separating hyperplanes, smallest distance to obstacles etc., to be used when designing motion planning algorithms. For multi-body robots like manipulators, the situation is completely different. The world space obstacles are usually mapped to non-convex regions, and to make the problem even harder, the mapping is usually not known. Forming explicit representations of the obstacle region in the configuration space is usually too expensive or intractable. Despite all of this, sampling based planners are used with great success, which mainly is due to their use of implicit representations of the obstacle region. The basic idea is to construct a graph in the configuration space that covers and connects the collision-free region. From this graph, a path can be extracted that connects a given start and goal configuration. The approach is computationally expensive, since the graph is constructed with the smallest geometrical building block available, points, which represents a collision-check. Furthermore, the extracted paths from the graph are non-smooth and jagged due to the stochastic nature of the approach. This adds an additional post-processing step to the process, where the paths are shortcutted and smoothened, before the path can be used for tracking. Clearly a lot of time is invested to form this graph and produce smooth paths. Thus, if the obstacles start to move, then all of this work is done in no use, since all points that make up this graph need to be re-verified, which is simply too time consuming to be done in real time.
\\\\
In this work, we want to address the existing drawbacks of the sampling based planners. Our main contribution is an improved motion planner where each vertex in the graph covers a collision-free region in the form of a sphere instead of a point and where the edges are formed with neighboring intersecting spheres. This representation has the advantage of instead of returning piecewise linear paths, returning a sequence of overlapping spheres, i.e. a convex corridor, that connects a given start and goal configuration, illustrated in Figure \ref{fig:pbrm_intro}. This convex corridor allows us to use convex optimization to produce smooth trajectories, instead of computationally expensive post-processing methods. The representation further allows us to estimate the coverage of the collision-free space, which gives us awareness and feedback in the offline roadmap construction phase. Finally, our representation is simple to adapt to moving obstacles, simply requery for the new radii and recheck for intersections. 
\\\\
The spherical collision-free regions are formed using a signed distance function (SDF), which is a function that returns the smallest distance from an arbitrary point to the boundary of an obstacle. As the name implies, the distance is signed, thus if the point is inside the obstacle it is negative otherwise positive. If the distance is positive, a sphere with radius equal to the distance is guaranteed to cover a collision-free region. Using an SDF in motion planning is not new, but what is novel about our approach is that we express the distance in the configuration space instead of the world space and by doing so allows us to form these convex collision-free regions. We refer to the resulting SDF as a signed configuration distance function (SCDF). Computing an SCDF analytically is non-trivial, our approach is therefore to parameterize the SCDF with a deep neural network and learn the mapping by supervised learning. Our resulting neural SCDF can compute distances for different parameter values of obstacle shapes and we also show how multiple distances can be combined, thus making our approach flexible.
\section{Related work}
Motion planning algorithms can roughly be divided into three families, grid-based, sampling based and optimization based methods. Grid-based methods (GBM) discretize the planning space from which a graph is then compiled. A standard search method is A$^\star$ \citep{a_star}, which is classified as an \textit{informed} search method, since it employs a heuristic function to speed up the search. A$^\star$ guarantees to return an optimal path at the level of discretization used. GBMs usually discretize the planning space by a regular lattice and this limits the GBMs to problems with low dimensionality due to the curse of dimensionality. Thus, GBMs are usually limited to single-body robots where the degrees of freedom (DOF) are low. To overcome the inherent scaling problem with the GBMs, stochastic methods are usually used for multi-body robots. These methods are termed as sampling-based methods (SBM) and core members within this family are the rapidly-exploring random trees (RRT) \citep{rrt} and the probabilistic roadmap (PRM) \citep{prm}. RRT grows a tree from the start configuration and explores the collision-free region in a rapid way until it is able to connect to the goal region. RRT is usually improved by bi-directional planning \citep{rrt_connect}, i.e. an additional tree is grown from the goal configuration and the trees are tested for connection after any tree has been expanded. RRT is a single-query method, thus it searches for a path from scratch each time it is queried. Contrary to this, PRM is a multi-query method, which solves for multiple queries without starting from scratch. PRM does this by creating a roadmap (graph) that covers the collision-free space as an offline step. The graph is then used to solve for multiple queries. PRMs are used in cases where the environment does not change since the extra offline step is too computationally costly and needs to be re-done if the environment is changed. In our work, we address this inherent issue by using a different roadmap representation. Our vertices in the graph cover a collision-free region in the form of spheres and we form the edges by checking for intersecting spheres. If something in the environment changes, we recompute the spheres radii and recheck the intersections, without relying on collision detection. We use a trained neural network to compute the sphere radius, therefore querying for the radius can be done fast, hence our representation enables the PRM for dynamic environments.
\\\\
In the recent decades, optimization based methods (OBM) \citep{chomp, schulman, itomp, stomp} have been introduced as an alternative to SBM for multi-body robots. Like the SBM, the OBMs scale well to higher dimensional problems and produce smoother motion. It is common to use a SDF in the optimization since it is a smooth function, thus enabling gradient-based methods. However, the standard way of expressing the SDF is in world space. The distance therefore needs to be mapped to the configuration space by the forward kinematics. This mapping makes the optimization problem a non-linear program (NLP), which is computationally expensive to solve. Recently, a different approach has been proposed. In \cite{mp_gcs} motion planning is formulated as a convex optimization problem by using the graph of convex sets framework \citep{gcs}. The underlying idea is to decompose the collision-free space into intersecting convex sets from which a convex optimization problem is formulated. In cases where an explicit representation of the obstacles in the configuration space exists, like for single-body robots, creating collision-free convex regions can be done fast \citep{iris}. For multi-body robots, this is non-trivial. Existing work does this successfully \citep{iris_nlp, iris_c} by an optimization based approach, but the methods are still too time consuming to be used in the presence of moving obstacles. Our approach is instead to use deep learning to learn an SDF expressed in the configuration space. With this, we can query for shortest distances to the collision boundary, which allows us to expand spherical regions which are collision-free. Our approach is fast and therefore enables our suggested roadmap planner to be used in dynamic environments.
\\\\
Recent research has focused on learning collision detection \citep{fk_kernel_distance, diffco, graphdistnet} by predicting the signed distance between the robot links and the surrounding obstacles in the world space. The learned SDF is used in trajectory optimization but since the distance is expressed in the world space, the problem becomes an NLP and therefore takes a long time to solve. We take a novel approach and suggest to instead express the signed distance in the configuration space. This allows us to improve the PRM at the same time as it enables convex optimization for trajectory optimization, which runs faster and is more reliable than NLP solvers. In \cite{cspf} a learned signed distance function in the configuration space is proposed similar to our approach. However, their approach is restricted to point cloud representations, while we propose to represent the obstacles as parameterized geometric shapes, e.g. spheres. Furthermore, we also show how to use our learned SCDF to improve an existing roadmap planner.
\section{Problem formulation}
A robot is located in the world space, $\W \subset \R^3 $. The unique location of the robot is given by its configuration $\q \in \C$, where $\C$ is the configuration space. The set of points covered by the robots bodies at a certain configuration is expressed as $\B(\q) \subset \W$. The robot is surrounded by $\NrObst$ obstacles $\O = \bigcup_{i=1}^{\NrObst} \O_i$, where  $\O_i \subset \W$. The representation of the obstacle in the configuration space is the set $\C\O_i = \{\q \in \C \: |\: \B(\q) \cap \O_i \neq \emptyset \}$. The obstacle space is formed as $\Co = \bigcup_{i=1}^{\NrObst} \C \O_i$. The complement is referred to as the free space, $\Cf = \C \setminus \Co$. The path planning problem is a tuple, ($\Cf$, $\qStart$, $\qGoal$), where we want to connect a query pair, consisting of a start, $\qStart$, and goal configuration, $\qGoal$, with a geometric path, $\q(s): [0, 1] \mapsto \Cf$, such that $\q(0)=\qStart$ and $\q(1)=\qGoal$, or report correctly when such a path does not exist.
\end{document}


\section{Preliminaries}\label{sec:preliminaries}



%We denote by $(\Ac(x_\Ac),\Bc(x_\Bc))(z)$ a random execution of $\pi$ with private inputs $(x_\Ac,y_\Ac)$, and common input $z$.

%\Jnote{Move to DP}
% At the end of such an execution, the protocol outputs a public transcript denoted by the random variable $\trans_\pi(x_\Ac,x_\Ac,z)$ we denotes the common as $\out(\trans_\pi(x_\Ac,x_\Ac,z)$, and each party $\Pc \in \set{\Ac,\Bc}$ obtains his view denoted $\view^\Pc_\pi(x_\Ac,x_\Bc,z)$, which may also contain a ``local output'' \Jnote{Local} $\out^\Pc(x_\Ac,x_\Bc,z)$ (if the protocol specifies such an output). \Jnote{Common output, and parties output}


\subsection{Distributions and Random Variables}\label{sec:prelim:dist}
The support of a distribution $P$ over a finite set $\cS$ is defined by $\Supp(P) \eqdef \set{x\in \cS: P(x)>0}$. For a distribution or a random variable $D$, let $d\from D$ denote that $d$ was sampled according to $D$. Similarly,  for a set $\cS$, let $x \from \cS$ denote that $x$ is drawn uniformly from $\cS$, and denote by $\cU_{\cS}$ the uniform distribution over $\cS$. For a finite set $\cX$ and a distribution $C_X$ over $\cX$, we use the capital letter $X$ to denote the random variable that takes values in $\cX$ and is sampled according to $C_X$. The {\sf statistical distance} (\aka {\sf~variation distance}) of two distributions $P$ and $Q$ over a discrete domain $\cX$ is defined by $\sdist{P}{Q} \eqdef \max_{\cS\subseteq \cX} \size{P(\cS)-Q(\cS)} = \frac{1}{2} \sum_{x \in \cS}\size{P(x)-Q(x)}$. 
For a vector $x = (x_1,\ldots,x_n)$ and index $i\in [n]$, we let $x_{-i} = (x_1,\ldots,x_{i-1},x_{i+1},\ldots,x_n)$ and $x^{(i)} = (x_1,\ldots,x_{i-1}, -x_i, x_{i+1},\ldots,x_n)$, for a set $\cS \subseteq [n]$ we let $x_{\cS} = (x_i)_{i \in \cS}$ and $x_{-\cS} = (x_i)_{i \in [n]\setminus \cS}$, and for a vector $r \in \zo^n$ we let $x_r = (x_i)_{\set{i \colon r_i = 1}}$ and $x_{-r} = (x_i)_{\set{i \colon r_i = 0}}$.

%For $n \in \N$ we let $U_n$ be the uniform distribution over $\oo^n$, and let $S_n$ be the distribution induces by the sum of $n$ i.i.d.\ random variables, each is distributed according to $U_1$. Let $\cN(0,1)$ be the standard normal distribution.
%For a distribution $\cD$ and a function $f$, we define by $f(\cD)$ the distribution that is induced by the output of $f(x)$ for $x \from \cD$. 





% \begin{theorem}[\cite{McGregorMPRTV10}]\label{thm:sv-extracotr}
% 	\Enote{Remove if not needed}
% 	There is a constant $c$ to make the following holds. Let $X$ be an $\alpha$-SV source on $\{0,1\}^n$, let $Y$ be a source on $\{0,1\}^n$ with min-entropy at least $\beta n$ (independent from $X$), and let $Z=\ip{X,Y}\mbox{mod m}$ for some $m\in\mathbb{N}$. Then for every $\delta\in[0,1]$, the random variable $(Y,Z)$ is $\delta$-close to $(Y,U)$ where $U$ is uniform on $\mathbb{Z}_m$ and independent of $Y$, provided that
% 	$$
% 	n\geq c\cdot\frac{m^2}{\alpha\beta}\cdot\log(\frac{m}{\beta})\cdot\log(\frac{m}{\delta}).
% 	$$
% \end{theorem}



\Enote{I removed the definition of DP since it already appears in the intro}
\remove{
\subsection{Differential Privacy}\label{sec:prelim:DP}
We use the following standard definition of (information theoretic) differential privacy, due to \citet{DMNS06}. For notational convenience, we focus on databases over $\oo$.
\begin{definition}[Differentially private mechanisms]\label{def:mech}
	A randomized function $f\colon\oo^n\mapsto \zs$ is an {\sf $n$-size, $(\eps,\delta)$-differentially private mechanism} (denoted $(\eps,\delta)$-\DP) if for every neighboring $w,w'\in \oo^n$ and every function $g\colon \zs\mapsto \zo$, it holds that 
	$$
	\pr{g(f(w))=1}\leq e^{\eps}\cdot \pr{g(f(w'))=1} +\delta.
	$$ 	
	If $\delta=0$, we omit it from the notation.
\end{definition}
}


\subsubsection{Computational Differential Privacy}
There are several ways for defining computational differential privacy (see \cref{sec:related-works}). We use the most relaxed version due to \cite{BNO08}. For notational convenience, we focus on databases over $\oo$.
\begin{definition}[Computational differentially private mechanisms]\label{def:ComMech}
	A randomized function ensemble $f=\set{f_\pk\colon\oo^{n(\pk)}\mapsto \zs}$ is an {\sf $n$-size, $(\eps,\delta)$-computationally differentially private} (denoted $(\eps,\delta)$-$\CDP$) if for every poly-size circuit family $\set{\Ac_\pk}_{\pk\in \N}$, the following holds for every large enough $\pk$ and every neighboring $w,w'\in\oo^{n(\pk)}$:
	$$
	\pr{\Ac_\pk(f_\pk(w))=1}\leq e^{\eps(\pk)}\cdot \pr{\Ac_\pk(f_\pk(w'))=1} +\delta(\pk).
	$$ 
	If $\delta(\pk) = \negl(\pk)$, we omit it from the notation. 
\end{definition}



\subsubsection{Two-Party Differential Privacy}\label{sec:DP}
In this section we formally define distributed differential privacy mechanism (\ie protocols). %For the ease of notation, we consider protocol with no common input.

\begin{definition}\label{def:DP}%\Nnote{fix security parameter}
	A two-party protocol $\Pi=(\Ac,\Bc)$ is {\sf $(\eps,\delta)$-differentially private}, denoted $(\eps,\delta)$-$\DP$, if the following holds for every algorithm $\Dc$: let $\V^\Pc(x,y)(\pk)$ be the view of party $\Pc$ in a random execution of $\Pi(x,y)(1^\pk)$. Then for every $\pk,n \in \N$, $x\in \oo^n$ and neighboring $y,y'\in\oo^n$:
	\begin{align*}
	\pr{\Dc(V^\Ac(x,y)(\pk))=1}\le e^{\eps(\pk)}\cdot \pr{\Dc(V^\Ac (x,y')(\pk))=1}+\delta(\pk),
	\end{align*} 
	and for every $y\in \oo^n$ and neighboring $x,x'\in\oo^{n}$:
	\begin{align*}
	\pr{\Dc(V^\Bc(x,y)(\pk))=1}\le e^{\eps(\pk)}\cdot \pr{\Dc(V^\Bc (x',y)(\pk))=1}+\delta(\pk).
	\end{align*} 	
	Protocol $\Pi$ is {\sf $(\eps,\delta)$-computational differentially private}, denoted $(\eps,\delta)$-$\CDP$, if the above inequalities only hold for a non-uniform \ppt $\Dc$ and large enough $\pk$. We omit $\delta = \negl(\pk)$ from the notation. \footnote{Note that define we give for two-party differentially private protocols is a semi-honest definition, in which we ask for the security to hold when the parties interact in an honest execution of the protocol. Since we are proving a lower bound, starting from this weaker guarantee (as opposed to security against malicious players), yields a stronger result.}
\end{definition}
%We omit $\delta$ from the notation if $\delta$ is a negligible function of $n$.

%\Enote{simulation-based}
\begin{remark}[The definition for computational differential privacy we use]\label{rem:comDPChannel} 
	An alternative, stronger definition of computational differential privacy, known as simulation-based computational differential privacy, requires that the distribution of each party’s view be computationally indistinguishable from a distribution that ensures privacy in an information-theoretic sense. \cref{def:DP} is a weaker notion in comparison. Consequently, establishing a lower bound for a protocol that satisfies this weaker guarantee (as we do in this work) yields a stronger result.%Actually, our lower bound only requires the privacy to hold against \emph{uniform} external observer.
	%\Nnote{Maybe add: When only interesting in \Dp against external observer, the two definitions can be achieve using key-agreement and (single-party) \Dp mechanism. }
\end{remark}




\subsection{Useful Claims}
\remove{
In this section, we state generic lemmas and propositions that we will use later in our proofs.

The following lemma which we prove in \cref{sec:missing-proofs:distance-I}, measures the distance between two uniform stings conditioned one a random index $i$ either being fixed to $0$ or to $1$.

\def\distanceILemma{
    Let $R \la \zo^n$. For any (randomized) function $f:\{0,1\}^n\rightarrow \{0,1\}$ and $\alpha > 0$, it holds that
    \begin{align}\label{eq:f-alpha}
        \ppr{i \la [n]}{\size{\:\ex{f(R) \mid R_i = 0}-\ex{f(R) \mid R_i = 1}\:}\geq \alpha} \leq \frac{2}{n \alpha^2},
    \end{align}
    where the expectations are taken over $R$ and the randomness of $f$.
}

\begin{lemma}\label{lem:distance-I}
    \distanceILemma
\end{lemma}
}

The following two propositions state that given the output of a differentially private function, it is not possible to predict well even a random index (even if all other indexes are leaked). The first proposition handles the information-theoretic case and the second handles the computation case. Both propositions are proven in \cref{sec:missing-proofs:hard-to-guess}. 

\def\propHardToGuessInf{
    Let $f\colon \oo^n \rightarrow \cY$ be an $(\eps,\delta)$-\DP function, let $g \colon [n] \times \oo^{n-1} \times \cY \rightarrow \set{-1,1,\bot}$ be a (randomized) function, and let $X = (X_1,\ldots,X_n) \la \oo^n$. Then the following holds for every $i \in [n]$ where $X_i^* = g(i,X_{-i},f(X_1,\ldots,X_n))$:
    \begin{align*}
        \pr{X_i^* = X_i} \leq e^{\eps}\cdot \pr{X_i^* = -X_i} + \delta.
    \end{align*}
}

\begin{proposition}\label{prop:hard-to-guess-inf}
    \propHardToGuessInf
\end{proposition}


\def\propHardToGuessComp{
    Let $f = \set{f_{\pk} \colon \oo^{n(\pk)} \rightarrow \zo^{m(\pk)}}_{\pk \in \bbN}$ be an $(\eps,\delta)$-\CDP function ensemble, and let $\set{g_{\pk}}_{\pk \in \bbN}$ be a poly-size circuit family. Then, for large enough $\pk$ and $X = (X_1,\ldots,X_{n(\pk)}) \la \oo^{n(\pk)}$, the following holds for every $i \in [n(\pk)]$ where $X_i^* = g_{\pk}(i,X_{-i},f_{\pk}(X_1,\ldots,X_n))$:
    \begin{align*}
        \pr{X_i^* = X_i} \leq e^{\eps(\pk)}\cdot \pr{X_i^* = -X_i} + \delta(\pk).
    \end{align*}
}

\begin{proposition}\label{prop:hard-to-guess-comp}
    \propHardToGuessComp
\end{proposition}





\remove{
\Enote{Chao's old statement:}
\begin{lemma}\label{lem:distance-I-old}
        Let $R \la \zo^n$. 
	For any function $f:\{0,1\}^n\rightarrow \{0,1\}$ and $\alpha<0.01$, it holds that
	$$
	\Pr_{i\la[n]}\left[\: \size{\:\mathbb{E}[f(R) \mid R_i = 0]-\mathbb{E}[f(R) \mid R_i = 1]\:}\geq \alpha\right]\leq \frac{2+2\log(\frac{1}{\alpha})}{n\alpha^2}.
	$$
\end{lemma}
\begin{proof}
	Define $S_1=\{r \in \zo^n \colon f(r)=1\}$. Then for any $i\in[n]$, we have
	$$
	\begin{array}{rl}
		\size{\mathbb{E}[f(R) \mid R_i = 0]-\mathbb{E}[f(R) \mid R_i = 1]}
		&=\size{\Pr[R\in S_1|R_i=0]-\Pr[R\in S_1|R_i=1]}\\
		&=\size{\frac{\Pr[R_i=0|R\in S_1]\cdot\Pr[R\in S_1]}{\Pr[R_i=0]}-\frac{\Pr[R_i=1|R\in S_1]\cdot\Pr[R\in S_1]}{\Pr[R_i=1]}}\\
		&=\frac{2\size{S_1}}{2^n}\size{\Pr[R_i=0|R\in S_1]-\Pr[R_i=1|R\in S_1]}
	\end{array}
	$$
	When $|S_1|\leq \alpha\cdot 2^{n-1}$, we have $\size{\mathbb{E}[f(R) \mid R_i = 0]-\mathbb{E}[f(R) \mid R_i = 1]}\leq\frac{2\size{S_1}}{2^n}\leq \alpha$ for any $i\in[n]$. Hence, in the following, we assume $|S_1|> \alpha\cdot 2^{n-1}$.

	%Define $I_{bad}=\{i|\size{\Pr[R_i=0|R\in S_1]-\Pr[R_i=1|R\in S_1]}>2\alpha\}$ and $k=\size{I_{bad}}$, then for any $i\notin I_{bad}$, we have 
    %$$
    %\begin{array}{rl}
    %    2\alpha&\geq \size{\Pr[R_i=0|R\in S_1]-\Pr[R_i=1|R\in S_1]}\\
    %    &=\size{\frac{\Pr[R\in S_1|R_i=0]\cdot\Pr[R_i=0]}{\Pr[R\in S_1]}-\frac{\Pr[R\in S_1|R_i=1]\cdot\Pr[R_i=1]}{\Pr[R\in S_1]}}\\
    %    &=\size{\Pr[R\in S_1|R_i=0]-\Pr[R\in S_1|R_i=1]}\cdot\frac{1}{2\Pr[R\in S_1]}\\
    %    &\geq \size{\mathbb{E}[f(R) \mid R_i = 0]-\mathbb{E}[f(R) \mid R_i = 1]}\cdot \frac{1}{2},
    %\end{array}
    %$$ 
    %where the last inequality is because $\Pr[R\in S_1]\leq 1$. So that $\size{\mathbb{E}}[f(R) \mid R_i = 0]-\mathbb{E}[f(R) \mid R_i = 1]\leq %4\alpha$.
    Define $I_{bad}=\{i \colon \size{\Pr[R_i=0|R\in S_1]-\Pr[R_i=1|R\in S_1]} \geq 2\alpha\}$ and $k=\size{I_{bad}}$, and denote $I_{bad}=\{i_1,\dots,i_k\}$. Define $(X_{i_1}, \ldots X_{i_k}) = (R_{i_1},\dots,R_{i_k})\mid_{R \in S_1}$. 
    Consider the min-entropy
	$$
	\begin{array}{rl}
		H_{min}(X_{i_1},\dots,X_{i_k})&\leq H(X_{i_1},\dots,X_{i_k})\\
		&\leq \sum_{j=1}^k H(X_{i_j})\\
		&\leq k\cdot \left(-(\frac{1}{2}+2\alpha)\cdot\log(\frac{1}{2}+2\alpha)-(\frac{1}{2}-2\alpha)\cdot\log(\frac{1}{2}-2\alpha)\right)\\
            &=k\cdot \left(-(\frac{1}{2}+2\alpha)\cdot(\log(1+4\alpha)-1)-(\frac{1}{2}-2\alpha)\cdot(\log(1-4\alpha)-1)\right)\\
            &=k\cdot \left(1-(\frac{1}{2}+2\alpha)\cdot\log(1+4\alpha)-(\frac{1}{2}-2\alpha)\cdot\log(1-4\alpha)\right),
		
	\end{array}
	$$
	where $H_{min}(Y)$ is the minimum entropy of $Y$ and $H(Y)$ is the Shannon entropy of $Y$.\Enote{add to preliminaries.}
        The third inequality holds since by the definition of $I_{bad}$, for every $j \in [k]$ it holds that $\size{\pr{X_{i_j} = 1}-\pr{X_{i_j} = 0}} > 2\alpha$, and therefore $H(X_{i_j}) \leq H(1/2 + 2\alpha)$\Enote{define}.
	
	Therefore, there exists $b_1,\dots,b_k\in\{0,1\}$, such that 
	
	\begin{align}\label{eq:min-entropy-result}
		\Pr\left[(R_{i_1},\ldots,R_{i_k}) = (b_1,\ldots,b_k) \mid R\in S_1\right]
		&= \pr{(X_{i_1},\ldots,X_{i_k}) = (b_1,\ldots,b_k)}\\
		&= 2^{-H_{min}(X_{i_1},\dots,X_{i_k})}\nonumber\\
		&\geq 2^{k\cdot \left(-1+(\frac{1}{2}+2\alpha)\cdot\log(1+4\alpha)+(\frac{1}{2}-2\alpha)\cdot\log(1-4\alpha)\right)}.\nonumber
	\end{align}
	
	Let $S_{bad}=\{r \in \zo^n  \colon \set{(r_{i_1},\ldots,r_{i_k}) = (b_1,\ldots,b_k)} \land \set{r\in S_1}\}$.
	It holds that
	\begin{align*}
		|S_{bad}|
		&= \size{S_1} \cdot \Pr\left[(R_{i_1},\ldots,R_{i_k}) = (b_1,\ldots,b_k) \mid R\in S_1\right]\\
		&\geq \alpha\cdot 2^{n-1}\cdot2^{k\cdot \left(-1+(\frac{1}{2}+2\alpha)\cdot\log(1+4\alpha)+(\frac{1}{2}-2\alpha)\cdot\log(1-4\alpha)\right)},
	\end{align*} 
	where the inequality holds by \cref{eq:min-entropy-result} and since $\size{S_1} \geq \alpha\cdot 2^{n-1}$.
	Notice that any string in $S_{bad}$ depends on at most $n-k$ bits. It implies that $|S_{bad}|\leq 2^{n-k}$. Therefore, we have
	$$
	\begin{array}{rl}
		&2^{n-k}\geq \alpha\cdot 2^{n-1}\cdot2^{k\cdot \left(-1+(\frac{1}{2}+2\alpha)\cdot\log(1+4\alpha)+(\frac{1}{2}-2\alpha)\cdot\log(1-4\alpha)\right)} \\
		\Rightarrow& n-k \geq \log \alpha+n-1+k\cdot \left(-1+(\frac{1}{2}+2\alpha)\cdot\log(1+4\alpha)+(\frac{1}{2}-2\alpha)\cdot\log(1-4\alpha)\right)\\
		\Rightarrow& 1-\log \alpha \geq k\cdot((\frac{1}{2}+2\alpha)\cdot\log(1+4\alpha)+(\frac{1}{2}-2\alpha)\cdot\log(1-4\alpha))\\
		\Rightarrow& 1-\log \alpha \geq k\cdot(4\alpha\cdot\log(1+4\alpha)+(\frac{1}{2}-2\alpha)\cdot\log(1-16\alpha^2))\\
        \Rightarrow& 1-\log\alpha \geq k\cdot(15.9\alpha^2-8\alpha^2+32\alpha^3)=k\cdot(7.9\alpha^2+32\alpha^3)>0.5k\alpha^2\\
		\Rightarrow& k\leq \frac{2-2\log \alpha}{\alpha^2} = \frac{2+2\log (1/\alpha)}{\alpha^2},
	\end{array}
	$$
	Where the third transition holds since 
	\begin{align*}
		\lefteqn{(\frac{1}{2}+2\alpha)\cdot\log(1+4\alpha)+(\frac{1}{2}-2\alpha)\cdot\log(1-4\alpha)}\\
		&= 4\alpha\cdot\log(1+4\alpha) + (\frac{1}{2}-2\alpha)\paren{\log(1+4\alpha)+\log(1-4\alpha)}\\
		&= 4\alpha\cdot\log(1+4\alpha)+(\frac{1}{2}-2\alpha)\cdot\log(1-16\alpha^2),
	\end{align*}
	and the forth transition holds since $4\alpha\cdot\log(1+4\alpha)+(\frac{1}{2}-2\alpha)\cdot\log(1-16\alpha^2) > 15.9\alpha^2-8\alpha^2+32\alpha^3$ for $\alpha < 0.01$.
	Thus, we conclude that 
	$$
	\Pr_{i\la[n]}\left[\size{\mathbb{E}[f(R) \mid R_i=0]-\mathbb{E}[f(R) \mid R_i = 1]}\geq \alpha\right]\leq \frac{k}{n}\leq \frac{2+2\log (1/\alpha)}{n\alpha^2}.
	$$
\end{proof}
}


\subsection{Channels and Two-Party Protocols}\label{sec:protocol}

\paragraph{Channels.}A channel is simply a distribution of a pair of tuples defined as follows. 
\begin{definition}[Channels]\label{def:channel} A {\sf channel} $C_{(X,U)(Y,V)}$ of size $\isize$ over alphabet $\Sigma$ is a probability distribution over $(\Sigma^\isize \times\zo^\ast) \times(\Sigma^\isize \times\zo^\ast)$. The ensemble $C_{(X,U)(Y,V)}= \set{C_{(X_\pk,U_\pk)(Y_\pk,V_\pk)}}_{\pk\in \N}$ is an $\isize$-size channel ensemble, if for every $\pk\in \N$, $C_{(X_\pk,U_\pk)(Y_\pk,V_\pk)}$ is an $\isize(\pk)$-size channel. %We denote a channel of size one by a \emph{single-bit} channel. 
We refer to $X$ and $Y$ as the {\sf local outputs}, and to $U$ and $V$ as the {\sf views}.	
\end{definition}

We view a  channel as the experiment in which there are two parties $\Ac$ and $\Bc$.  Party $\Ac$ receives ``output'' $X$ and ``view'' $U$, and party $\Bc$ receives ``output'' $Y$ and ``view'' $V$. Unless stated otherwise, the channels we consider are over the alphabet $\Sigma = \oo$. We naturally identify channels with the distribution that characterizes their output.








\subsubsection{Two-Party Protocols}

A two-party protocol $\Pi=(\Ac,\Bc)$ is \ppt if the running time of both parties is polynomial in their input length. We let $\Pi(x,y)(z)$ or $(\Ac(x),\Bc(y))(z)$ denote a random execution of $\Pi$ on a common input $z$, and private inputs $x,y$.%We assume \wlg that a protocol has a common output (part of its transcript).\Jnote{This is not really the case we consider in this paper..}

\begin{definition}[Oracle-aided protocols]\label{def:ChannelAidedProtocol}
	In a two-party protocol $\Pi$ with oracle access to a {\sf protocol} $\Psi$, denoted $\Pi^\Psi$, the parties make use of the \textit{next-message function} of $\Psi$.\footnote{The function that on a partial view of one of the parties, returns its next message.} In a two-party protocol $\Pi$ with oracle access to a {\sf channel} $C_{Z W}$, denoted $\Pi^C$, the parties can jointly invoke $C$ for several times. In each call, an independent pair $(z,w)$ is sampled according to $C_{Z W}$, one party gets $z$, the other gets $w$.
\end{definition}


\begin{definition}[The channel of a protocol]\label{def:ChannlOfProtocol}
	For a no-input two-party protocol $\Pi= (\Ac,\Bc)$, we associate the channel $C_\Pi$, defined by $\C_\Pi= C_{(X, U),(Y, V)}$, where $X$ and $Y$ are the local outputs of $\Ac$ and $\Bc$ (respectively) and
	$U$ and $V$ are the local views of $\Ac$ and $\Bc$ (respectively).
    
	For a two-party protocol $\Pi$ that gets a security parameter $1^\pk$ as its (only, common) input, we associate the channel ensemble $ \set{C_{\Pi(1^\pk)}}_{\pk\in \N}$. 
\end{definition}

\begin{definition}[$(\alpha,\gamma)$-Accurate channel]\label{def:accurate-func}
	A channel $C = C_{(X, U),(Y, V)}$ is {\sf $(\alpha,\gamma)$-accurate for the function $f$}, if $\ppr{C}{\size{\out(V)-f(X,Y)}\leq \alpha}\ge \gamma$, where $\out(V)$ is the designated output.
    A channel ensemble $C_{(X, U),(Y, V)}= \set{C_{(X_\pk, U_\pk),(Y_\pk, V_\pk)}}_{\pk\in \N}$ is  $(\alpha,\gamma)$-accurate for  $f$ if $C_{(X_\pk, U_\pk),(Y_\pk, V_\pk)}$ is $(\alpha(\pk),\gamma(\pk))$-accurate for $f$, for every $\pk \in \N$.
\end{definition}

\subsubsection{Differentially Private Channels}\label{sec:DPChannel}
Differentially private channels are naturally defined as follows:
\begin{definition}[Differentially private channels]\label{def:DPChannel}
	An $n$-size channel $C = C_{(X, U),(Y, V)}$ with $X, Y$ over $\oo^n$ 
	is {\sf$(\eps,\delta)$-differentially private} (denoted $(\eps,\delta)$-$\DP$) if for every $x \in \Supp(X)$ there exists an $n$-size $(\eps,\delta)$-$\DP$ mechanisms $\Mc_x$ such that $(X,Y,U) \equiv (X,Y,\Mc_X(Y))$, and for every $y \in \Supp(Y)$ there exists an $n$-size $(\eps,\delta)$-$\DP$ mechanisms $\Mc_y'$ such that $(X,Y,V) \equiv (X,Y,\Mc_Y'(X))$. In addition, we say that the channel is \emph{uniform} if $X$ and $Y$ are independent random variables uniformly distributed in $\oo^n$. 
\end{definition}

\begin{definition}[Computational differentially private channels]\label{def:CDPChannel}
	An $n$-size channel ensemble $C = \set{C_{(X_\pk, U_\pk),(Y_\pk, V_\pk)}}_{\pk\in\N}$ with $X_\pk, Y_\pk$ over $\oo^n$ 
	is {\sf$(\eps,\delta)$-computationally differentially private} (denoted $(\eps,\delta)$-$\CDP$) if for every ensemble $\set{x_\pk \in \Supp(X_\pk)}_{\pk\in\N}$ there exists an $n$-size $(\eps,\delta)$-\CDP mechanisms ensemble $\set{\Mc_{x_\pk}}_{\pk\in\N}$ such that $(X_\pk,Y_\pk,U_\pk) \equiv (X_\pk,Y_\pk,\Mc_{X_\pk}(Y_\pk))$, for every $\pk\in\N$, and for every ensemble $\set{y_\pk \in \Supp(Y_\pk)}_{\pk\in\N}$ there exists an $n$-size $(\eps,\delta)$-$\CDP$ mechanisms ensemble $\set{\Mc'_{y_\pk}}_{\pk\in\N}$ such that $(X_\pk,Y_\pk,V_\pk) \equiv (X_\pk,Y_\pk,\Mc_{Y_\pk}'(X_\pk))$ for every $\pk\in \N$. In addition, we say that the channel is \emph{uniform} if $X_\pk$ and $Y_\pk$ are independent random variables uniformly distributed in $\{\pm 1\}^n$ for all $\pk\in\N$.
\end{definition}




% \begin{lemma}~\label{lem:dp-sv-source}
% 	Let $P$ be an $\varepsilon$-DP randomized protocol. Let $X$ and $Y$ be independent random variables uniformly distributed in $\{\pm 1\}^n$ and let random variable $\Pi(X,Y)$ denote the transcript of running $P(X,y)$. Then for every $\pi\in Supp(\Pi)$, the random variables corresponding to the inputs conditioned on transcript $\pi$, $X_\pi$ and $Y_\pi$, are independent $e^{-\varepsilon}$-strong SV source.
% \end{lemma}





\subsubsection{Weak Erasure Channel (\WEC)}

\begin{definition}[\WEC]\label{def:WEC}
	A channel $((O_A,V_A), (O_B,V_B))$ with $O_A \in \set{0,1}$ and $O_B \in \set{0,1,\bot}$ is a {\sf weak erasure channel}, denoted $(\alpha,p,q)$-$\WEC$, if:
	\begin{itemize}
		%\item $O_A\in \set{-1,1}$ and $O_B\in \set{-1,1,\bot}$.
		\item Random erasure: $\pr{O_B = \perp} = 1/2$.
		
		\item Agreement: $\pr{O_A\ne O_B\mid O_B\ne \bot}\le \alpha$.
		
		\item Secrecy:
		
		\begin{enumerate}
			\item For every algorithm $\Dc$ it holds that\label{WEC:item:A}
			\begin{align*}
				%\size{\pr{\Ac(O_A,V_A) = 1 \mid O_B \neq \perp} - \pr{\Ac(O_A,V_A) = 1 \mid O_B = \perp}} \le p
				\size{\pr{\Dc(V_A) = 1 \mid O_B \neq \perp} - \pr{\Dc(V_A) = 1 \mid O_B = \perp}} \le p
			\end{align*}
			(Alice doesn't know if $O_B = \perp$.)
			
			\item For every algorithm $\Dc$ it holds that\label{WEC:item:B}
			\begin{align*}
				\pr{\Dc(V_B) = O_A \mid O_B=\bot} \leq \frac{1+q}{2}.
			\end{align*}
			(i.e., if $O_B=\bot$, Bob don't know what is the value of $O_A$).
			
			%\item $SD((O_A U|O_B=\bot),(O_A U|O_B\ne \bot))\le p$ (The sender don't know if $O_B=\bot$).
			
			%\item $SD(V O_A|O_B=\bot,V(-O_A)|O_B=\bot)\le q$ (If $O_B=\bot$, Bob don't know what the value of $O_A$).
		\end{enumerate}
	\end{itemize}
   We say that a channel ensemble $C=\set{C_\pk}_{\pk\in N}$ is a {\sf computational weak erasure channel}, denoted $(\alpha,p,q)$-\CompWEC, if for every \ppt algorithm $\Dc$ and every sufficiently large $\pk\in\N$, $C_\pk$ satisfies the properties stated in the items above, where the secrecy property holds with respect to a \ppt algorithm $\Dc$. A protocol $\Lambda$ is said to be $(\alpha,p,q)$-$\CompWEC$, if the ensemble induces by the protocol (that is, $C=\set{C_{\Lambda(\pk)}}_{\pk\in\N}$) is $(\alpha,p,q)$-$\CompWEC$.  
\end{definition}



\subsubsection{Approximate Weak Erasure Channel (\AWEC)}\label{sec:AWEC}

\begin{definition}[\AWEC]\label{def:AWEC}
	A channel $C = ((O_A,V_A), (O_B,V_B))$ over $([-n,n] \times \zo^*) \times (([-n,n] \cup \bot)  \times \zo^*)$ is an {\sf approximate weak erasure channel}, denoted $(\ell,\alpha,p,q)$-\AWEC if:
	\begin{itemize}
		
		\item Random erasure: $\pr{O_B = \perp} = 1/2$.
		
		\item Accuracy: $\pr{\size{O_A - O_B} > \ell \mid O_B \ne \bot}\le \alpha$.
		
		\item Secrecy:
		
		\begin{enumerate}
			\item For every algorithm $\Dc$ it holds that\label{AWEC:item:A}
			\begin{align*}
				%\size{\pr{\Ac(O_A,V_A) = 1 \mid O_B \neq \perp} - \pr{\Ac(O_A,V_A) = 1 \mid O_B = \perp}} \le p
				\size{\pr{\Dc(V_A) = 1 \mid O_B \neq \perp} - \pr{\Dc(V_A) = 1 \mid O_B = \perp}} \le p
			\end{align*}
			(Alice doesn't know if $O_B=\bot$).
			
			\item For every algorithm $\Dc$ it holds that\label{AWEC:item:B}
			\begin{align*}
				\pr{\size{\Dc(V_B) - O_A} \leq 1000 \ell \mid O_B=\bot} \leq q.
			\end{align*}
			(i.e., if $O_B=\bot$, Bob can't estimate the value of $O_A$ with error $\leq 1000 \ell$).
		\end{enumerate}
	\end{itemize}
     We say that a channel ensemble $C=\set{C_\pk}_{\pk\in N}$ is a {\sf computational approximate weak erasure channel}, denoted $(\ell,\alpha,p,q)$-\CompAWEC, if for every \ppt algorithm $\Dc$ and every sufficiently large $\pk\in\N$, $C_\pk$ satisfies the properties stated in the items above. A protocol $\Gamma$ is said to be $(\ell,\alpha,p,q)$-$\CompAWEC$, if the ensemble induced by the protocol (that is, $C=\set{C_{\Gamma(\pk)}}_{\pk\in\N}$) is $(\ell,\alpha,p,q)$-$\CompAWEC$.  
\end{definition}

We will make use of the following lemma, which shows that for some choices of the parameters, \AWEC implies \WEC. The lemma is proven in \cref{sec:AWEC-to-WEC}.

\begin{lemma}\label{lemma:AWEC-to-WEC}
	For every $\ell> 0$, there exists a \ppt protocol $\Lambda = (\Pc_1,\Pc_2)$ such that given an oracle access to an $(\ell,\alpha,p,q)$-\AWEC $C$, the channel $\tilde{C}$ induced by $\Lambda^C$ is $(\alpha'=\alpha+0.001,\: p' = p ,\:  q' = 1/2 + 2(q+0.01))$-\WEC.
	Furthermore, the proof is constructive in a black-box manner:
	\begin{enumerate}
		\item There exists an oracle-aided \ppt algorithm $\Ec_1$ such that for every channel $C = ((\OA,\VA), (\OB,\VB))$ and algorithm $\Dc$ violating the \WEC secrecy property~\ref{WEC:item:A} of $\tilde{C}$, algorithm $\Ec_1^{\Dc}$ violates the \AWEC secrecy property~\ref{AWEC:item:A} of $C$.
		
		\item There exists an oracle-aided \ppt algorithm $\Ec_2$ such that for every channel $C = ((\OA,\VA), (\OB,\VB))$ and algorithm $\Dc$ violating the \WEC secrecy property~\ref{WEC:item:B} of $\tilde{C}$, algorithm $\Ec_2^{\Dc}$ violates the \AWEC secrecy property~\ref{AWEC:item:B} of $C$.
	\end{enumerate}
\end{lemma}

Since \cref{lemma:AWEC-to-WEC} is constructive, the following is an immediate corollary.
\begin{corollary}\label{cor:CompAWEC to CompWEC}
There exists an oracle aided \ppt protocol $\Lambda$, such that given a protocol $\Gamma$ that induces $(\ell,\alpha,p,q)$-\CompAWEC, it holds that $\Lambda^\Gamma$ is $(\alpha'=\alpha+0.001,\: p' = p ,\:  q' = 1/2 + 2(q+0.01))$-\CompWEC.  
\end{corollary}
\begin{proof}[Proof of \ref{cor:CompAWEC to CompWEC}]
Let $\Lambda$ be the \ppt algorithm guaranteed  by Lemma \ref{lemma:AWEC-to-WEC}. Given an $(\ell,\alpha,p,q)$-\CompAWEC protocol $\Gamma$, we define $\Lambda(\pk)={\Lambda^{\Gamma(\pk)}(\pk)}$. Assume towards a contradiction that $\Lambda$ is not a $(\alpha',p',q')$-\CompWEC. It follows that there exists a \ppt $\Dc$ that for infinity many $\pk\in\N$ contradicts one of the \WEC secrecy properties of channel ensemble $\set{C_{\Lambda(\pk)}}_{\pk\in\N}$. Fix $\pk\in\N$ for which this holds. By Lemma \ref{lemma:AWEC-to-WEC}, there exists a \ppt $\Ec^\Dc$ that for every such $\pk$  contradicts one of the secrecy properties of the channel $C_{\Gamma(\pk)}$. This implies that for infinity many $\pk\in\N$, $\Ec^\Dc$  contradict the secrecy of the channel ensemble $\set{C_{\Gamma(\pk)}}_{\pk\in\N}$, which is a contradiction since this would means that $\Gamma$ is not a $(\ell,\alpha,p,q)$-\CompAWEC.       
\end{proof}



\subsection{Oblivious Transfer (\OT)}

\paragraph{Secure Computation.}
We use the standard notion of securely computing a functionality, \cf  \cite{Goldreich04}.
\begin{definition}[Secure computation]\label{def:SFE}
	A two-party protocol {\sf securely computes a functionality $f$}, if it does so according to the real/ideal paradigm.   We add the term perfectly/statistically/computationally/non-uniform computationally, if the simulator's output is  perfect/statistical/computationally indistinguishable/  non-uniformly indistinguishable from  the real distribution.  The protocol have the above notions of security {\sf against semi-honest  adversaries}, if its security only  guaranteed to holds against an adversary that follows the prescribed protocol.   Finally, for the case of perfectly secure computation, we naturally apply the above notion also to the non-asymptotic case: the protocol with no security parameter perfectly  compute a functionality $f$.
	
	A two-party protocol {\sf securely computes a functionality ensemble $f$ with oracle to a channel $C$}, if it does so according to the above definition when the parties have access to a trusted party computing $C$. All the above adjectives naturally extend to this setting.
\end{definition}

\paragraph{Oblivious Transfer.}
The (one-out-of-two) oblivious transfer functionality is defined as follows.
\begin{definition}[oblivious transfer functionality $f_{\OT}$]\label{def:OTfunc}
	The oblivious transfer functionality over $\zo \times (\zs)^2$ is defined by  $f_{\OT} (i,(\sigma_0,\sigma_1)) = (\perp,\sigma_i)$.
\end{definition}
A protocol is $\ast$ secure OT,   for \\$\ast\in \set{\text{semi-honest statistically/computationally/computationally non-uniform}}$, if it  compute the $f_{\OT}$  functionality with $\ast$ security.





% \begin{definition}[Computational oblivious transfer, semi-honest model]
% A protocol $\Pi=(\Ac,\Bc)$ is a semi-honest 1-out-of-2 computational oblivious transfer (comp-OT) protocol if the following holds. Given a common input $1^{\pk}$, the parties $\Ac$ and $\Bc$ run the protocol $\Pi(1^\pk)$ (in an honest manner) and    
% $\Ac$ outputs $X=(m_1,m_2)\in \zo\times\zo$ and has a view $U$ and $\Bc$ outputs $Y=(i,\hat{m})\in\zo\times\zo$ and has a view $V$, and the following properties are satisfied:
% \begin{enumerate}
%     \item \textbf{Correctness:} 
%     $\pr{\hat{m}\neq m_i}<\negl(\pk).$ 
    
%     \item \textbf{A's Privacy:} For every \ppt $\Dc$ and every sufficiently large $\pk$:
%     $\pr{\Dc(V)=m_{i-1}}<(1+\negl(\pk))/2$
    
%     \item \textbf{B's Privacy:} For every \ppt $\Dc$ and every sufficiently large $\pk$:
%     $\pr{\Dc(U)=i}<(1+\negl(\pk))/2$  
% \end{enumerate}
% \end{definition}

We make use of the following useful results by Wullschleger on oblivious transfer amplification from weak channels.
\begin{theorem}[\cite{Wullschleger09}, from \WEC to statistically secure \OT]\label{thm:WEC TO OT IT}
    There exists an oracle aided protocol $\Pi$ such that the following holds: Given a $(\alpha,p,q)$-\WEC $C$, if $44(\alpha+p)\le 1-q$ then $\Pi^{C}(1^\pk)$ is a semi-honest statistically secure \OT.
\end{theorem}

The following computational version of \cref{thm:WEC TO OT IT} is implicit in \cite{Wullschleger09} and is based on the computational proof explicitly stated in \cite{Wul07} (see Section 6 in \cite{Wullschleger09} for discussion).   

\begin{theorem}[\cite{Wullschleger09,   Wul07}, from \CompWEC to computinally secure \OT]\label{thm:WEC TO OT Comp}
    There exists an oracle aided protocol $\Pi$ such that the following holds: Given a $(\alpha,p,q)$-\CompWEC protocol $\Lambda$, if $44(\alpha+p)\le 1-q$ then $\Pi^{\Lambda}$ is a semi-honest computational secure \OT.
\end{theorem}



% \begin{definition}[Computational 1-out-of-2 Oblivious Transfer, semi-honest model]
% A protocol $\Pi=(\Ac,\Bc)$ is a semi-honest 1-out-of-2 $(\eps,\alpha,\beta)$-oblivious transfer (OT) protocol if the following holds. 

% The parties $\Ac$ and $\Bc$ run the protocol (in an honest manner) and    
% $\Ac$ outputs $X=(m_1,m_2)\in \zo\times\zo$ and has a view $U$ and $\Bc$ outputs $Y=(i,\hat{m})\in\zo\times\zo$ and has a view $V$, and following properties are satisfied:
% \begin{enumerate}
%     \item \textbf{Correctness:} 
%     $\pr{\hat{m}\neq m_i}<\eps.$ 
    
%     \item \textbf{A's Privacy:} For every adversary $\Dc$:
%     $\pr{\Dc(V)=m_{i-1}}<(1+\alpha)/2$
    
%     \item \textbf{B's Privacy:} For every adversary $\Dc$: $\pr{\Dc(U)=i}<(1+\beta)/2$  
% \end{enumerate}
% \end{definition}
\begin{table}[ht!]
\centering
\caption{\textbf{Super Resolution Performance Results.} Our proposed WGAN EEG Spatial Upsampling method significantly outperforms a baseline of Bicubic Interpolation commonly used in EEG upsampling pipelines.}
\label{tab:results}
\resizebox{0.8\linewidth}{!}{%
\begin{tabular}{@{}cccccc@{}}
\toprule
\multirow{2}{*}{\textbf{Dataset}} & \multirow{2}{*}{\textbf{Scale}} & \multicolumn{2}{c}{\textbf{Bicubic}} & \multicolumn{2}{c}{\textbf{WGAN}} \\ \cmidrule(l){3-6} 
                      &   & \textbf{MSE} & \textbf{MAE} & \textbf{MSE}    & \textbf{MAE}   \\
\toprule
\multirow{2}{*}{Val}  & 2 & 3.71E7       & 3.89E3       & \textbf{2.01E3} & \textbf{24.38} \\
                      & 4 & 7.23E7       & 6.42E3       & \textbf{8.53E3} & \textbf{63.83} \\
\midrule
\multirow{2}{*}{Test} & 2 & 3.75E7       & 3.91E3       & \textbf{2.06E3} & \textbf{24.66} \\
                      & 4 & 7.30E7       & 6.45E3       & \textbf{8.68E3} & \textbf{64.39} \\
\bottomrule
\end{tabular}%
}
\end{table}
% TODO: section name
\section{Schedule Design for Faster Convergence}
\label{sec:instance}

% TODO: highlight the importance, how it affects convergence rate (done)
% prove the the time schedule design for the second \gamma, add another theorem (done?)

%longbo{first give a high level description about what we are doing here. something like the previous theorem provides a general framework, so we want to design optimized time schedule for faster convergence blabla}\yuhaoDone

In \Cref{thm:main}, we provide an upper bound on the KL divergence from the target distribution to the estimated distribution for a general class of SDE-based generative models. Since the bound depends on the choice of latent scale $\gamma(t)$ and schedule $\{t_k\}_{k=0}^N$, we are able to carefully design a time schedule for a given latent scale, thereby achieving a provably bounded error within a minimum number of steps.

% TODO: add more references
Specifically, we consider the common choice of latent scale in stochastic interpolants, $\gamma(t)=\sqrt{at(1-t)}$, which is first introduced in \citet{interpolation}. 
% In fact, the process $x_t=(1-t)x_0+tx_1+\sqrt{2t(1-t)}z$ is a variance-preserving process. 
This choice is equivalent to changing the definition $$x_t=I(t,x_0,x_1)+\gamma(t)z$$
to $$x_t=I(t,x_0,x_1)+\sqrt{a}\dd B_t,$$
where $B_t$ is a standard Brownian bridge process independent of $(x_0,x_1)$. %(We can just write $B_t=W_t-tW_1$ to obtain a Brownian bridge process.) 
For this $\gamma(t)$, we present the following time schedule to optimize the sample complexity.
% Now, for the given choice of $\gamma(t)$, we give the following time schedule to optimize the sample complexity, which is the number of steps required to achieve a specific error bound.

% TODO: intuition behind the schedule design, explain why
\paragraph{Exponentially Decaying Time Schedule} 
% Consider the bound given by \Cref{thm:main}, we can see that for those steps with smaller $\bar{\gamma}_k$, the error terms in the bound are larger. When $\gamma(t)=\sqrt{at(1-t)}$, for $t_k$ close to $0$ or $1$, the error term in the bound is larger. Therefore, to reduce the error bound for a fixed number of steps $N$, we want to take shorter steps when $t$ is close to $0$ or $1$, and longer steps when $t$ is around $0.5$. 

As suggested by \Cref{thm:main}, smaller steps need to be taken in order to balance the error terms. Moreover, to exactly cancel the $\gamma$-terms, we need $h_k=O(\bar{\gamma}_k^2)$ where $\bar{\gamma}$ is defined in \Cref{thm:main}. Hence, we propose an exponentially decaying time schedule inspired by the approach of \citet{dlinear}. %\longbo{is it the same form? "adapted to our setting" make our result look very weak. if it is not the same, you can just say "inspired".}
Specifically, we first select a midpoint $t_M=\frac{1}{2}$. Let $h\in(0,1)$ be a parameter that controls the step size. We then define the time steps as follows:
$$t_{k+1}-t_k=\begin{cases}\frac{1}{2}h(1-h)^{M-k-1},&k<M\\\frac{1}{2}h(1-h)^{k-M},&k\ge M.\end{cases}$$
This leads to $$t_k=\begin{cases}
    \frac{1}{2}(1-h)^{M-k},&k<M\\
    1-\frac{1}{2}(1-h)^{k-M},&k\ge M.
\end{cases}$$
The parameter $h$ determines the overall scale of the step sizes. A smaller $h$ results in a finer discretization of the time interval. 

Let $h_k=t_{k+1}-t_k$ denote the step size at the $k$-th step. We observe that $$h_k=O(h\min\{t_k,1-t_{k+1}\})=O(h\bar{\gamma}_k^2),$$
which satisfies the condition of canceling the $\gamma$-terms.
% where $\bar{\gamma}_k$ is defined in \Cref{thm:main}. 
% This design ensures that smaller step sizes are taken when $\bar{\gamma}_k$ is smaller, which is crucial for mitigating the error contributions from regions with smaller latent scales, as suggested by \Cref{thm:main}. 
Moreover, the total number of steps is given by
$$\begin{aligned}
    N&=O\left(\frac{\log(1/t_0)+\log(1/(1-t_N))}{\log(1/(1-h))}\right)\\
    &=O\left(h^{-1}\log\left(\frac{1}{t_0(1-t_N)}\right)\right).
\end{aligned}$$
%\longbo{can you state the schedule design using a different order: 1. from theorem 4.3, we know that in order to cancel the gamma terms, we need $h_k=O(sth)$. thus, we use the following step size inspired by xxx. 3. we then see that by choosing this step fulfill our purpose}

Now we can provide the following bound:

\begin{proposition}
    Consider the same settings as in \Cref{thm:main}. Suppose $h_k=t_{t+1}-t_k=O(h\bar{\gamma}^2)$, $\epsilon=\Theta(1)$ and $h=O(\frac{1}{d})$. %\longbo{explain what is $\lesssim$ } 
    Then, we have
    $$\begin{aligned}
        \textnormal{KL}(\rho(t_N)\Vert\hat{\rho}(t_N))&\lesssim\varepsilon_{b_F}^2+\textnormal{KL}(\rho(t_0)\Vert\hat{\rho}(t_0))\\
        &+hd\sqrt{\mathbb{E}\Vert x_0-x_1\Vert^4}+Nh^2d^2.
    \end{aligned}$$
    \label{cor:schedule}
\end{proposition}

\Cref{cor:schedule} provides the KL error bound when the step sizes is chosen so that the $\gamma$-terms are canceled.

\begin{corollary}
    Using $\gamma=\sqrt{at(1-t)}$ and the time schedule defined above, suppose that $\textnormal{KL}(\rho(t_0)\Vert\hat{\rho}(t_0))\le\varepsilon^2$ and $\varepsilon^2_{b_F}\le\varepsilon^2$. Furthermore, assume that $\epsilon=\Theta(1)$ and $h=O(\frac{1}{d})$. Then, under the same settings as in \Cref{thm:main}, the total number of steps required to achieve $\textnormal{KL}(\rho(t_N)\Vert\hat{\rho}(t_N))=O(\varepsilon^2)$ is:
    $$\begin{aligned}
        N=O\left\{\frac{1}{\varepsilon^2}\left[\sqrt{\mathbb{E}\Vert x_0-x_1\Vert^4}d\log\left(\frac{1}{t_0(1-t_N)}\right)\right.\right.\\
        \left.\left.+d^2\log^2\left(\frac{1}{t_0(1-t_N)}\right)\right]\right\}.
    \end{aligned}$$
    \label{cor:instant}
\end{corollary}


Corollary \ref{cor:instant} provides the computational complexity of sampling data using the forward SDE. For a fixed error bound $\varepsilon$, the complexity scales proportionally to $\varepsilon^{-2}$. We can further decompose the complexity into distance-related complexity and Gaussian diffusion complexity. 
% \yu{Do not present a single expression here, try to make a name for them, for example, the distance-related error, gaussian diffusion error or something like that.} (done)
Here $O\left(\frac{1}{\varepsilon^2}\sqrt{\mathbb{E}\Vert x_0-x_1\Vert^4}d\log\left(\frac{1}{t_0(1-t_N)}\right)\right)$ is the distance-related complexity representing the number of steps required to achieve a sufficiently small discretization error associated with the velocity function $v(t, x)$. $O\left(\frac{1}{\varepsilon^2}d^2\log^2\left(\frac{1}{t_0(1-t_N)}\right)\right)$ is the Gaussian diffusion complexity  representing the number of steps required to achieve a sufficiently small discretization error associated with the score function $s(t, x)$.

We briefly explain how to obtain this complexity. First, given a desired number of steps $N$, we select $$h=\Theta\left(N^{-1}\log\left(\frac{1}{t_0(1-t_N)}\right)\right)$$
to achieve the specified number of steps. Since $h_k=O(\bar{\gamma}_k^2h)$, we have: 
%\longbo{dont use phrases like this: "by direct calculation,"}
$$\begin{aligned}
    \sum_{k=0}^{N-1}h_k^3\left[M_2+\bar{\gamma}_k^{-6}d^3+\bar{\gamma}_k^{-2}d\sqrt{\mathbb{E}\Vert x_0-x_1\Vert^{8}}\right]\\\le Nh^3d^3+h^2\left(M_2+d\sqrt{\mathbb{E}\Vert x_0-x_1\Vert^8}\right),
\end{aligned}$$
and 
$$\begin{aligned}
    \sum_{k=0}^{N-1}\left(h^2d^2+h_khd\sqrt{\mathbb{E}\Vert x_0-x_1\Vert^4}\right)\\
    \le Nh^2d^2+hd\sqrt{\mathbb{E}\Vert x_0-x_1\Vert^4}.
\end{aligned}$$
By substituting the chosen value of $h$ for the given $N$ into \Cref{thm:main}, we can derive the stated complexity bound.

\paragraph{Comparison to a Uniform Schedule.} 
%Here we want to show the advantages of our schedule. We first compare our schedule to a natural uniform schedule, which satisfies $h_k=\frac{t_N-t_0}{N}\approx\frac{1}{N}$. 
To highlight the benefits of our proposed exponentially decaying time schedule, we compare it with a natural uniform schedule that satisfies $h_k = \frac{t_N - t_0}{N} \approx \frac{1}{N}$. 
We further assume the ideal case where  $\varepsilon_{b_F}^2=0$ and $\rho(t_0)=\hat{\rho}(t_0)$ in our analysis. 

According to \Cref{thm:main}, the error bound for the uniform schedule is given by
$$\begin{aligned}
    &\quad\sum_{k=0}^{N-1}h_k^3(M_2+\bar{\gamma}_k^{-6}d^3+\bar{\gamma}_k^{-2}d\sqrt{\mathbb{E}\Vert x_0-x_1\Vert^8})\\
    &+\sum_{k=0}^{N-1}h_k^2(\bar{\gamma}_k^{-4}d^2+\bar{\gamma}_k^{-2}d\sqrt{\mathbb{E}\Vert x_0-x_1\Vert^4}).
\end{aligned}$$
Since $\bar{\gamma}_k^2=\Theta(\min\{t_k,1-t_{k+1}\})$, and noting that $$\int_{\delta}^{0.5}t^{-p}\dd t=
\begin{cases}
    \Theta(\log(1/\delta)),&p=1\\
    \Theta(\delta^{-(p-1)}),&p>1
\end{cases}$$
for a uniform schedule, the overall error bound becomes:
$$\begin{aligned}
    &\qquad\textnormal{KL}(\rho(t_N)\Vert\hat{\rho}(t_N))\\&=O\left(\frac{1}{N}\left[\sqrt{\mathbb{E}\Vert x_0-x_1\Vert^4}d\log\left(\frac{1}{t_0(1-t_N)}\right)\right.\right.\\
    &\qquad\qquad\qquad\left.\left.+\frac{1}{t_0(1-t_N)}d^2\right]\right).
\end{aligned}$$
Consequently, the complexity of using a uniform schedule is given by 
\begin{eqnarray*}
N&=&O\bigg(\varepsilon^{-2}\bigg[\log\left(\frac{1}{t_0(1-t_N)}\right)d\sqrt{\mathbb{E}\Vert x_0-x_1\Vert^4}\\&& \qquad\qquad\qquad+\frac{1}{t_0(1-t_N)}d^2\bigg]\bigg),
\end{eqnarray*}
which exhibits a higher computational complexity compared to the proposed exponentially decaying schedule.

\paragraph{Comparison to Diffusion Models Results.} By setting $I(t,x_0,x_1)=(1-t)x_0+tx_1$, $\gamma(t)=\sqrt{2t(1-t)}$, $x_0\sim\rho_0=N(0,I_d)$, and assuming that $x_0$ and $x_1$ are independent, the stochastic interpolant reduces to $x_t=\sqrt{1-t^2}\bar{z}+tx_1$ for some $\bar{z}\sim N(0,I_d)$, which fits the diffusion model setting \cite{song2021scorebased}. %\longbo{give a diffusion ref}. 
Assuming that the fourth moment of $\rho_1$ is bounded by a constant (see \Cref{appendix:reduce-to-gaussian} for details), the complexity of our approach simplifies to $$N=O\left(\varepsilon^{-2}d^2\log^2\left(\frac{1}{1-t_N}\right)\right).$$
% Furthermore, $\text{KL}(\rho(t_0)\Vert\rho_0)\lesssim dt_0^2$ (see, e.g., Proposition 4 in \citealt{dlinear}). By selecting $t_0\lesssim\sqrt{\varepsilon^2/d}$ and setting $\hat{\rho}(t_0)=\rho_0=N(0,I_d)$, the assumption $\text{KL}(\rho(t_0)\Vert\rho(t_0))\le\varepsilon^2$ is satisfied.\longbo{I am confused by this two sentences. what are you trying to say?}

For diffusion models with an early stopping time $\delta$, \citet{chen2023improved} established a complexity bound of $\tilde{O}\left(\varepsilon^{-2}d^2\log^2\left(\frac{1}{\delta}\right)\right)$.
By setting $\delta=1-t_N$ in our analysis, we recover the same complexity bound as that obtained for diffusion models. % \longbo{give a ref}. {\color{orange}[the first ref is added now; I have already referenced Chen et al. so idk if there are any other refs to give.]}
% 
While \citet{dlinear} further improves the complexity bound for diffusion models to $\tilde{O}\left(\varepsilon^{-2}d\log^2\left(\frac{1}{\delta}\right)\right)$ by leveraging techniques from stochastic localization, these techniques heavily rely on the Gaussian structure of diffusion models and cannot be directly applied to the more general stochastic interpolant framework.

% TODO: our framework also applies in analyzing other \gamma
% make it a formal corollary
\paragraph{Other Choices of $\gamma(t)$.} In addition to the commonly used $\gamma(t)=\sqrt{at(1-t)}$, our framework can readily be extended to analyze other choices of $\gamma(t)$. In Appendix \ref{appendix:another}, we present an analysis for $\gamma^2(t)=(1-s)^2s$, which is equivalent to the definition in \citet{chen2024forcasting}. We show that the proposed time schedule in Appendix \ref{appendix:another} also outperforms the uniform schedule in terms of computational complexiting the effectiveness of our schedule design, demonstrating the effectiveness of our schedule design. %\longbo{explain why this is an important feature?} 

% [generated by Gemini] This ability to analyze and optimize for different choices of $\gamma(t)$$ is a significant advantage of our framework, as it provides greater flexibility in designing and optimizing the generative process.
%applies for the analyses of other $\gamma$. In 
%For example, Appendix \ref{appendix:another} studies the choice $\gamma^2(t)=(1-s)^2s$, which is equivalent to the definition in \cite{chen2024forcasting}.

% \longbo{experiments? }
\section{Experiments: Planning outperforms Heuristics}
\label{sec:experiment}

We begin our empirical demonstrations by showcasing the effectiveness of our planning framework on both synthetic and real datasets. We focus on the simplest planning algorithm, 1-step lookaheads (Algorithm~\ref{alg:complete}), and show that even basic planning can hold great promise. 
We illustrate our framework using two uncertainty quantification modules---GPs and 
\ensembles/ \ensembleplus. 

Throughout this section, we focus on evaluating the mean squared error of 
a regression model $\model$,  and develop adaptive policies that minimize uncertainty on $g(f)$ defined in~\eqref{eqn:l2-g-f}.
When GPs provide a valid model of uncertainty, 
our experiments show that our planning framework significantly outperforms other baselines. 
We further demonstrate that our conceptual framework extends to deep learning-based uncertainty quantification methods such as  \ensembleplus while highlighting computational challenges that need to be resolved in order to scale our ideas. 
For simplicity, we assume a naive predictor, i.e., $\psi(\cdot) \equiv 0$. However, we emphasize that this problem is just as complex as if we were using a sophisticated model $\psi(.)$. The performance gap between the algorithms 
primarily depends
on the level  of uncertainty in our prior beliefs.

To evaluate the performance of our algorithm, we benchmark it against several baselines. 
%Active learning baselines use an acquisition function $\ac$ to select points that have the highest   function value: $X\opt_t \in \argmax_{X \in \xpoolj{t}} \ac({X})$ at every step $t$. These methods may also need an UQ module, which we simply use the same UQ module as in our algorithm, and it  outputs $V(X)$ that measures the the uncertainty of each point $X \in \xpoolj{t}$.
Our first set of baselines are from active learning~\citep{AggarwalKoGuHaPh14}:
\\ % \noindent\textbf{Active Learning Heuristics:} 
\textbf{(1)} 
\textsf{Uncertainty Sampling (Static):}  In this approach, we query the samples for which the model is least certain about. Specifically, we estimate the variance of the latent output $f(X)$ for each $X \in \xpool$ using the UQ module and select the top-$K$ points with the highest uncertainty. \\
\textbf{(2)} \textsf{Uncertainty Sampling (Sequential):} This is a greedy heuristic that sequentially selects the points with the highest uncertainty within a batch, while updating the posterior beliefs using pseudo labels from the current posterior state. Unlike \textsf{Uncertainty Sampling (Static)}, this method takes into account the information gained from each point within batch, and hence tries to diversify the selected points within a batch. 

 
We also compare our approach to the  \textbf{(3)} \textsf{Random Sampling}, which selects each batch uniformly at random from the pool. Additionally, we compare solving the planning problem using  \textsf{REINFORCE}-based policy gradients with   $\mathsf{Smoothed\text{-}Autodiff}$ policy gradients.\footnote{Our code repository is available at
  \url{https://github.com/namkoong-lab/adaptive-labeling}.}
%Detailed experimental setups are provided in Section \ref{sec:details-experiments}.

%We repeat all experiments with 10 random seeds.




\begin{figure}[t]
\centering
\begin{minipage}[b]{0.49\textwidth}
\centering
\includegraphics[width=\textwidth, height=5cm]{figures/original_scale/Var_of_l_2_loss.pdf}
\caption{(Synthetic data) Variance of mean squared loss evaluated through the posterior belief $\mu_t$ at each horizon $t$. This is the objective that policy gradient methods like \textsf{REINFORCE} and $\ouralgo$ optimizes. 1-step lookaheads are surprisingly effective even in long horizons.}
\label{fig:var-l2-sim}
\end{minipage}
\hfill
\begin{minipage}[b]{0.49\textwidth}
\centering \includegraphics[width=\textwidth, height=5cm]{figures/original_scale/Error_of_estimated_model_l_2_loss.pdf}
\caption{(Synthetic data) Error between MSE calculated based on collected data $\mc{D}^{0:T}$ vs. population oracle MSE over $\mc{D}_{\rm eval} \sim P_X$. Reducing uncertainty over posteriors directly leads to better OOD evaluations. 1-step lookaheads significantly outperform active learning heuristics in small horizons.}
\label{fig:mean-l2-sim}
\end{minipage}
%\caption{Simulated data for GPs}
%\label{fig:both_plots}
\end{figure}

\subsection{Planning with Gaussian processes}
\label{sec:experiment-plan-GP}
We now briefly describe the data generation process for the GP experiments,  deferring a more detailed discussion of the dataset generation to Section~\ref{sec:details-experiments}. 
We use both the synthetic data and the real data to test our methodology.
For the \emph{simulated data},  we construct a setting where the general population is distributed across \emph{51 non-overlapping clusters} while the initial labeled data $\dtrain$ just comes from one cluster. In contrast, both $\dpool \defeq (\xpool,\ypool),\deval \defeq (\xeval,\yeval)$ are generated   from all the clusters. 
We begin with a low-dimensional scenario, generating a one-dimensional regression setting using a GP. %Gaussian Process (GP).
Although the data-generating process is not known to the algorithms,  we assume that the GP hyperparameters are known to all the algorithms
to ensure fair comparisons. This can be viewed as a setting where our prior is well-specified, allowing us to isolate the effects
of different policy optimization approaches
 without any concerns about the misspecified priors. We select $10$ batches, each of size $K=5$ across $T = 10$ time horizons.

To examine the robustness of our method against the distributional assumptions made  in the simulated case, we then move to a real dataset where the correct prior is not known. We simulate selection bias from the eICU dataset~\citep{PollardJoRaCeMaBa18}, which contains real-world patient data with in-hospital mortality outcomes. 
We conduct a $k$-means clustering to generate 51 clusters and then select data from those clusters. We view this to be a credible replication of practice, as severe distribution shifts are common due to selection bias in clinical labels.  To convert the binary mortality labels into a regression setting, we train a  random forest classifier and fit a GP on predicted scores, which serves as the UQ module for all the algorithms. As before, the task is to select 10 batches, each consisting of 5 samples, across 10 time horizons.

 In Figures~\ref{fig:var-l2-sim} and~\ref{fig:mean-l2-sim}, we present results for the simulated data. 
Figure~\ref{fig:var-l2-sim} shows the variance of $\ell_2$ loss, and Figure~\ref{fig:mean-l2-sim} presents the error in the estimated $\ell_2$ loss using $\mu_t$ (relative to true $\ell_2$ loss, that is unknown to the algorithm). 
As we can see from these plots, our method one-step lookahead  gives substantial improvements  over active learning baselines and random sampling. In addition,
compared to the one-step lookahead planning approach using \textsf{REINFORCE}-based policy gradients, 
we observe that $\mathsf{Smoothed\text{-}Autodiff}$-based policy gradients provide significantly more robust performance over all horizons.

In Figures~\ref{fig:var-l2-real}~and~\ref{fig:mean-l2-real}, we observe similar findings on the eICU data. We see that planning policies (\textsf{REINFORCE} and $\mathsf{Smoothed\text{-}Autodiff}$) consistently outperform other heuristics by a large margin.  Active learning baselines perform poorly in these small-horizon batched problems and can sometimes be even worse than the random search baselines.  Overall, our results show the importance of careful planning in adaptive labeling for reliable model evaluation. 

We offer some intuition as to why one-step lookahead planning may outperform other heuristic algorithms. 
 First,  \textsf{Uncertainty sampling (Static)} while myopically selects the
 top-$K$ inputs with the highest uncertainty, it fails to consider 
the overlap in information content among the ``best” instances; see \citep{AggarwalKoGuHaPh14} for more details. 
In other words,  it might acquire points from the same region with high uncertainty while failing to induce diversity among the batch.
Although \textsf{Uncertainty Sampling (Sequential)} somewhat addresses the issue of information overlap, a significant drawback of 
this algorithm
is the disconnect between the objective we aim to optimize and the algorithm. For example, it might sample from a region with high uncertainty but very low density. 

\begin{figure}[t]
\centering
\begin{minipage}[b]{0.48\textwidth}
\centering
\includegraphics[width=\textwidth, height=5cm]{figures/original_scale/Var_of_l_2_loss_real.pdf}
\caption{(Real-world eICU data) Variance of mean squared loss evaluated through the posterior belief $\mu_t$ at each horizon $t$. Even 1-step lookaheads are extremely effective planners, and auto-differentiation-based pathwise policy gradients provide a reliable optimization algorithm based on low-variance gradient estimates.}
\label{fig:var-l2-real}
\end{minipage}
\hfill
\begin{minipage}[b]{0.48\textwidth}
\centering \includegraphics[width=\textwidth, height=5cm]{figures/original_scale/Error_of_estimated_model_l_2_loss_real.pdf}
\caption{(Real-world eICU data) Error between MSE calculated based on collected data $\mc{D}^{0:T}$ vs. population oracle MSE over $\mc{D}_{\rm eval} \sim P_X$. Reducing uncertainty over posteriors directly leads to better OOD evaluations. Our method significantly outperforms active learning-based heuristics, and random sampling.}
\label{fig:mean-l2-real}
\end{minipage}
%\caption{Real data for GPs}
\end{figure}
 
%\vspace{-1.5cm}
% \begin{wrapfigure}{r}{.32\columnwidth}
%   \vspace{-.5cm} 
%   \centering
% \includegraphics[scale=.29]{figures/Var of l2l_2 loss.pdf}
%   \vspace{-0.2cm}
%   \caption{Results of GP}
% \label{fig:var-l2-gp}
%   \vspace{-0.1cm}
% \end{wrapfigure}


% Attempts have been made  in the past to address these  drawbacks heuristically  (see \citep{AggarwalKoGuHaPh14}). We give a unified computational framework while approaching the problem in a more principled manner and solving it more optimally.




\subsection{Planning with  neural network-based uncertainty quantification methods ($\ensembleplus$)}


We now provide a proof-of-concept that shows the generalizability of our conceptual framework  to the deep learning-based UQ modules, specifically focusing on $\ensembleplus$ due to their previously observed superior performance~\citep{OsbandWenAsDwIbLuRo23}. Recall that implementing our framework with deep learning-based UQ modules  requires us to retrain the model across multiple possible random actions $\bm{a}(\theta)$ sampled from the current policy $\pi_\theta$.
This requires significant computational resources, in sharp contrast to the GPs where the posteriors are in closed form and can be readily updated and differentiated. 

Due to the computational constraints, we test $\ensembleplus$ on a toy setting to demonstrate the generalizability of our framework. We consider a setting where the general population consists of four clusters, while the initial labeled data only comes from one cluster. Again we generate data using GPs.  The task is to select a batch of 2 points in one horizon. We detail the $\ensembleplus$ architecture in Section \ref{sec:details-experiments}, and we assume prior uncertainty to be large (depends on the scaling of the prior generating functions). 
The results are summarized in the Table~\ref{tab:UQ_ensemble}.

% \begin{table}[H]
% \vspace{-10pt}
% \caption{Performance under \ensembleplus as UQ module}
%     \centering
%     \begin{tabular}{|m{3cm}|m{2.5cm}|m{2cm}|} 
%     \hline
%       Algorithm   & Variance of $\loss_2$ loss estimate & Error of $\loss_2$ loss estimate  \\ \hline Random Sampling 
%          & $1710.9 \pm 1352.1$ & $8.67\pm6.62$ 
%       \\ \hline \ouralgo & $1.30 \pm 0.68$ & $0.91\pm0.25$ \\ \hline
%     \end{tabular}
%     \label{tab:UQ_ensemble}
%     %\vspace{-10pt}
% \end{table}




\begin{table}[h]
\vspace{-10pt}
\caption{Performance under \ensembleplus as the UQ module}
\centering
\begin{tabular}{|l|l|l|}
\hline
Algorithm   & Variance of $\loss_2$ loss estimate & Error of $\loss_2$ loss estimate  \\
\hline
\textsf{Random sampling} & 7129.8 $\pm$ 1027.0 & 136.2 $\pm$ 8.28 \\ \hline
\textsf{Uncertainty sampling (Static)} & 10852 $\pm$ 0.0 & 162.156 $\pm$ 0.0 \\ \hline
\textsf{Uncertainty sampling (Sequential)} & 8585.5 $\pm$ 898.9 & 144 $\pm$ 6.93 \\ \hline
\textsf{REINFORCE} & 1697.1 $\pm$ 0.0 & 45.27 $\pm$ 0.0 \\ \hline
\ouralgo & 1697.1 $\pm$ 0.0 & 45.27 $\pm$ 0.0 \\ \hline
\end{tabular}
%\caption{Comparison of different algorithms based on variance   and   error in $\ell_2$ loss estimation with Ensemble $+$ as the UQ module. Our results demonstrate that {\ouralgo} and REINFORCE outperformthe other active learning based heuristics, confirming the benefits of our MDP formulation for the adaptive labeling problem, as also demonstrated in Section 4.\\
%\footnotesize{Experimental details: We use Gaussian Processes as our data generating process, GP parameters are the same as in Section D.3.  The task is to select a batch of 2 points along one horizon.The marginal distribution $p_X$ has 4 \textit{non-overlapping} clusters. Initial data comes from one cluster, while pool and evaluation points comes from all the clusters. We have $20$ initial labeled data points, $10$ pool points, and $252$ evaluation points.  Training procedures are similar to the one in Section D.3.} }
\label{tab:UQ_ensemble}
\end{table}



% We faced  issues in scaling up these experiments which will be our focus in the future. 





% \begin{itemize}
%     \item Posteriors should be consistent. Two dimensions: even with less training,  
%     \item the inference should be  fast enough
% \end{itemize}


% Potential research directions for uncertainty quantification

% In this section we consider a simple setting We consider a simpler setting and 


% For synthetic dataset generation, we use ...... For real datasets, we use ...... We compare our methodolgy to several baselines ()    This Section is structured as follows:
% \begin{itemize}
%     \item \textbf{GPs, square loss objective} (Section \ref{}): 
%     %the broad aim of the experiments  in this section is to isolate the performance of our methodology without any concerns for the inefficiencies induced due to a mis-specified prior or imperfect posterior inference. To accomplish this we generate synthetic datasets using GPs (detailed later). We use the well specified prior (GPs - with same hyperparameter setting) as our UQ module.   
%      As GPs provide differentaible posterior inference - any errors induced due to imperfect posterior updates are also isolated. We note that under this setting
%      \item In Section\ref{} we demonstrate why our methodology performs better than other baselines - by devising various synthetic experiments ()
%     \item  \textbf{UQ Benchmarking }(Section \ref{}): Before diving into the experiments using $\ensembleplus$ and ENNs,  we showcase our benchmarking experiments in Section \ref{}. We use real datasets We observe that ENNs perform better
%      \item \textbf{Ensemble $+$}, objective: recall, accuracy
%     \item \textbf{ENN}, objective: recall, accuracy
% \end{itemize}




% In Section {}, we test 
% \subsection{Experimental details}

% \begin{itemize}
%     \item UQ methodologies - GPs, ENNs
%     \item Objectives - Recall,  ATE
%     \item Datasets - ATE-synthetic datasets, Recall-synthetic, real datasets
%     \item Baselines - 
%     \begin{itemize}
%         \item Random sampling
%         \item Active learning - Uncertainty based sampling - In regression setting almost all of the 
%         \item Myopic greedy - Greedy Batch based sampling
%         \item Policy Gradient
%     \end{itemize}
    
% \end{itemize}

% \subsection{Experiments}
%     \begin{itemize}
%     \item GPs with square loss
%     \item Benchmarking ENN
%         \item ENNs with ATE
%         \item ENNs with Recall
%     \end{itemize}

% \subsection{Benefits over other algorithms - intuition and experiments}

%Active learning - Myopic greedy / Don't rely on the objective rather some entropy version.


%%% Local Variables:
%%% mode: latex
%%% TeX-master: "main"
%%% End:

\section*{Conclusion}
This paper aims to enhance our understanding of the computational complexity of computing various Shapley value variants. We found that for various ML models --- including decision trees, regression tree ensembles, weighted automata, and linear regression --- both local and global interventional and baseline SHAP can be computed in polynomial time under HMM modeled distributions. This extends popular algorithms, such as TreeSHAP, beyond their empirical distributional scope. We also establish strict complexity gaps between the various SHAP variants (baseline, interventional, and conditional) and prove the intractability of computing SHAP for tree ensembles and neural networks in simplified scenarios. Overall, we present SHAP as a versatile framework whose complexity depends on four key factors: \begin{inparaenum}[(i)] \item model type, \item SHAP variant, \item distribution modeling approach, \item and local vs. global explanations\end{inparaenum}. We believe this perspective provides deeper insight into the computational complexity of SHAP, paving the way for future work.




%We believe that our framework provides a more intricate understanding of SHAP computation complexity across different models, distributions, and variants, paving the way for further research.

Our work opens promising directions for future research. First, expanding our computational analysis to other SHAP-related metrics, such as asymmetric SHAP~\citep{frye20} and SAGE~\citep{covert2020understanding}, would be valuable. Additionally, we aim to explore more expressive distribution classes and relaxed assumptions beyond those in Section \ref{sec:tractable} while maintaining tractable SHAP computation. Finally, when exact computation is intractable (Section \ref{sec:intractable}), investigating the approximability of SHAP metrics through approximation and parameterized complexity theory~\citep{downey2012parameterized} is an important direction.

%Our work opens several promising avenues for future research on the computational properties of explainable AI methods, with a particular focus on SHAP. First, it would be interesting to broaden the computational analysis conducted in this work to include other popular SHAP-related metrics in the literature, such as asymmetric SHAP \cite{frye20} and SAGE \cite{covert2020understanding}. Also, in the future, we aim to explore more expressive distribution classes and relaxed distributional assumptions—extending beyond those examined in Section \ref{sec:tractable} —that still yield tractable SHAP computation. Finally, when exact computation proves intractable (Section \ref{sec:intractable}), it is worthwhile to theoretically investigate the question of the approximability of computing the SHAP metrics across various configurations, through the lens of approximation and parametrized complexity theory \cite{arora2009computational}.

%This paper aims to deepen our understanding of the computational complexity involved in obtaining different Shapley value variants. We found that for a variety of ML models, including decision trees, tree ensembles for regression, weighted automata, and linear regression models — computing both local and global interventional and baseline SHAP can be done in polynomial time when distributions are modeled by HMMs. This extends the distributional scope of popular algorithms like TreeSHAP, which is limited to empirical distributions. Additionally, we demonstrate a strict complexity gap between SHAP variants, showing that interventional and baseline SHAP can be strictly easier to compute than conditional SHAP. Despite these positive results, we uncovered intractability for various SHAP variants in neural networks and tree ensembles. Finally, we provided generalized complexity relations across SHAP variants. We believe that our framework offers a deeper understanding of the complexity involved in computing SHAP across various variants, models, distributions, as well as in both local and global computations, laying the groundwork for future research.

\bibliographystyle{apalike}
\bibliography{ref}

\newpage
\appendix
% \renewcommand{\appendixpagename}{\centering \LARGE Supplementary Materials}
\appendixpage

% \begin{itemize}
%     \item \textbf{Appendix \ref{appendix:preliminaries}: Supplementary details for Section \ref{sec:preliminaries}} - This section provides additional details from \cite{interpolation} that were not included in Section \ref{sec:preliminaries}. It includes a formal statement of key equations and their associated conditions. Furthermore, it outlines the optimization objectives for the score and velocity estimators, which are used for model training in Section \ref{sec:experiments}.  %\longbo{explain why we need them}
    
%     \item \textbf{Appendix \ref{appendix:lemmas}: Useful lemmas in bounding derivatives} - This section presents lemmas that are essential for bounding the derivatives of velocity functions and score functions, along with their corresponding proofs. The proofs primarily rely on properties and inequalities related to (conditional) expectations. These lemmas play a crucial role in deriving the overall KL error bounds.
    
%     \item \textbf{Appendix \ref{appendix:overall}: Proofs of results in Section \ref{sec:results} and \ref{sec:instance}} - This section provides the complete proofs for the results presented in Section \ref{sec:results} and Section \ref{sec:instance}. This includes the proof of \Cref{thm:main} (\Cref{appendix:proofofmain}), the proof of \Cref{cor:schedule} and \Cref{cor:instant} (Appendices \ref{appendix:proofofcor},\ref{appendix:proofofschedule}), and additional details regarding the discussions in Section \ref{sec:instance} (Appendix \ref{appendix:another}).

%     \item \textbf{Appendix \ref{appendix:experiments}: More details of numerical experiments} - This section provides omitted details for Section \ref{sec:experiments}, including the parameterization of estimators, choice of $(t_0,t_N)$ and optimizers, and how the TV distance is estimated. We also include additional experiments for $\gamma(t)=\sqrt{(1-t)^2t}$.
% \end{itemize}


\startcontents[section]
\printcontents[section]{l}{1}{\setcounter{tocdepth}{2}}
\newpage


\section*{Notations}
%\yu{Do not use bullets here. Just write a paragraph or make a table. (done)}

We use $\Vert\cdot\Vert$ to denote $\ell_2$ norm for both vectors and matrices. For a matrix $A$, we use $\Vert A\Vert_F=\sqrt{\sum_{ij}A_{ij}^2}$ to denote the Frobenious norm of $A$. We use $\frac{\dd}{\dd u}$, $\frac{\partial}{\partial u}$, or just $\partial_u$ to denote the (partial) derivative with respect to $u$. We use $\nabla$ to denote the gradient or Jacobian, depending on whether the function is scalar-valued or vector-valued. If not specified, for the function in form of $f(t,x)$ where $t$ is a scalar and $x$ is a vector, $\nabla f(t,x)$ means the gradient vector or Jacobian matrix with respect to $x$ rather than $t$. We use $\Delta f(t,x)=\sum_{i=1}^d\frac{\partial^2}{\partial x_i^2}f$ as the Laplace operator. We use $\mathbb{E}[X]$ to denote the expectation of a random variable $X$, and $\text{Cov}(X,Y)$ to denote the covariance of two random variables $X,Y$. $\mathbb{E}[X|c]$ and $\text{Cov}(X,Y|c)$ denote the corresponding conditional expectation and conditional covariance given condition $c$. We use the notation $f(x)\lesssim g(x)$ or $f(x)=O(g(x))$ to denote that there exists a constant $C>0$ such that $f(x)\le Cg(x)$.

% \begin{itemize}
%     \item ($\ell_2$-)norm: $\Vert\cdot\Vert$, Frobenius norm: $\Vert\cdot\Vert_F$.
%     \item We use $\frac{\dd}{\dd u}$, $\frac{\partial}{\partial u}$, or just $\partial_u$ to denote the (partial) derivative. 
%     \item We use $\nabla$ to denote the gradient or Jacobian, depending on whether the function is scalar-valued or vector-valued. If not specified, for the function in form of $f(t,x)$ where $t$ is a scalar and $x$ is a vector, $\nabla f(t,x)$ means the gradient vector or Jacobian matrix with respect to $x$ rather than $t$.
%     \item $\Delta f=\sum_{i=1}^d\frac{\partial^2}{\partial x_i^2}f$ is the Laplace operator.
%     \item We use $\mathbb{E}$ to denote the expectation and $\text{Cov}$ to denote the covariance of two random variables. $\mathbb{E}[\cdot|c]$ and $\text{Cov}[\cdot,\cdot|c]$ denote the corresponding conditional version.
%     \item We use the notation $f(x)\lesssim g(x)$ or $f(x)=O(g(x))$ to hide constant factors.
% \end{itemize}

\section{Supplementary Details for Section \ref{sec:preliminaries}}
\label{appendix:preliminaries}

This part summarizes some of the results from \cite{interpolation} that are not introduced in Section \ref{sec:preliminaries}.

\begin{proposition}
    (\cite{interpolation}, Theorem 2.6, Corollaries 2.10 and 2.18, and their proofs)
    Suppose that the joint measure $\nu$ and the function $I$ satisfies 
    \begin{equation}
        \underset{(x_0,x_1)\sim\nu}{\mathbb{E}}\Vert\partial_tI(t,x_0,x_1)\Vert^4\le M_1<\infty,\quad\underset{(x_0,x_1)\sim\nu}{\mathbb{E}}\Vert\partial_t^2I(t,x_0,x_1)\Vert^2\le M_2<\infty,\quad \forall t\in[0,1].
        \label{eq:assumption1}
    \end{equation}
    Then, $\rho\in C^1((0,1),C^p(\mathbb{R}^d))$, $s\in C^1((0,1),(C^p(\mathbb{R}^d))^d)$ and $b\in C^0((0,1),(C^p(\mathbb{R}^d))^d)$, and both the solution of the probability flow ODE
    $$\frac{\dd}{\dd t}X_t=b(t,X_t), \qquad X_0\sim\rho_0$$
    and the solution of the forward SDE
    $$\dd X_t^F=b_F(t,X_t^F)\dd t+\sqrt{2\epsilon(t)}\dd W_t, \qquad X_0^F\sim\rho_0$$
    have the same marginal densities as $(x_t)_{t\in[0,1]}$. Here $\epsilon\in C[0,1]$ with $\epsilon(t)\ge0$ for all $t\in[0,1]$ and $b_F$ is defined as \begin{equation}
        b_F(t,x)=b(t,x)+\epsilon(t)s(t,x).
        \label{eq:defbf}
    \end{equation}
    
    Moreover, suppose that the densities $\rho_0,\rho_1$ are strictly positive elements of $C^2(\mathbb{R}^d)$, and are such that
    $$\int_{\mathbb{R}^d}\Vert\nabla\log\rho_0(x)\Vert^2\rho_0(x)dx<\infty,\qquad\int_{\mathbb{R}^d}\Vert\nabla\log\rho_0(x)\Vert^2\rho_1(x)dx<\infty.$$
    Then $\rho\in C^1([0,1],C^p(\mathbb{R}^d))$, $s\in C^1([0,1],(C^p(\mathbb{R}^d))^d)$ and $b\in C^0([0,1],(C^p(\mathbb{R}^d))^d)$. The notation is adapted from \cite{interpolation} where $f\in C^1([0,1],C^p(\mathbb{R}^d))$ means that the function $f$ is $C^1$ in $t\in[0,1]$ and $C^p$ in $x\in\mathbb{R}^d$.
    \label{prop:generative-modeling}
\end{proposition}

% Also, by the proof of Theorem 2.6 in \cite{interpolation} and study how the assumptions are used, we can derive the following result when we do not have the boundary assumptions.

% \begin{proposition}
%     (\cite{interpolation}, Appendix B.1)
%     With only the condition (\ref{eq:assumption1}) in Proposition \ref{prop:generative-modeling}, we still have the same result for any subintervals of $(0,1)$ as Proposition \ref{prop:generative-modeling}.
%     \label{prop:generative-modeling2}
% \end{proposition}

The above proposition provides a generative modeling in the form of 
$$\frac{\dd}{\dd t}X_t=b(t,X_t)$$
and
$$\dd X_t^F=b_F(t,X_t^F)\dd t+\sqrt{2\epsilon(t)}\dd W_t.$$

In practice, we need to train an estimator to estimate velocity functions. By the following proposition, we can use the optimization objectives to train the estimators. 

\begin{proposition}
    (\cite{interpolation}, Theorems 2.7 and 2.8)
    $b$ is the unique minimizer of
    $$\mathcal{L}_b[\hat{b}]=\int_0^1\mathbb{E}\left[\frac{1}{2}\Vert\hat{b}(t,x_t)\Vert^2-(\partial_tI(t,x_0,x_1)+\dot{\gamma}(t)z)\cdot\hat{b}(t,x_t)\right]\dd t,$$
    and $s$ is the unique minimizer of
    $$\mathcal{L}_s[\hat{s}]=\int_0^1\mathbb{E}\left[\frac{1}{2}\Vert\hat{s}(t,x_t)\Vert^2+\gamma^{-1}(t)z\cdot\hat{s}(t,x_t)\right]\dd t.$$
    Here the notation ``$\cdot$" represents the inner product of two vectors.
    \label{prop:objectives}
\end{proposition}


\section{Bounding the Velocities and Scores}
\label{appendix:lemmas}

\subsection{Useful Lemmas}

To begin with, we first provide moment bounds on the Gaussian variable $z\sim N(0,I_d)$.

\begin{lemma}
    For any $p\ge1$,
    $$\mathbb{E}\Vert z\Vert^{2p}\le C(p)d^p,$$
    where $C(p)$ is a constant that only depends on $p$.
    \label{lem:moment-z}
\end{lemma}

\begin{proof}
    First, $\Vert z\Vert^2=\sum_{i=1}^n z_i^2$, where we represent $z=(z_1,z_2,\cdots,z_d)^T$. For any $n$ positive numbers $a_1,a_2,\dots,a_n$, using Jensen's inequality,
    $$\left(\sum_{i=1}^na_i\right)^p=n^p\left(\frac{1}{n}\sum_{i=1}^na_i\right)^p\le n^p\cdot\frac{1}{n}\sum_{i=1}^na_i^p.$$
    Then,
    $$\begin{aligned}
        \mathbb{E}\Vert z\Vert^{2p}&=\mathbb{E}\left[\left(\sum_{i=1}^dz_i^2\right)^p\right]\\
        &\le d^p\cdot\frac{1}{d}\sum_{i=1}^d\mathbb{E}[|z_i|^{2p}]&(\text{Jensen's inequality})\\
        &\le d^p\mathbb{E}\left[|z_1|^{2p}\right]&(\{z_i\}_{i=1}^d\text{ are i.i.d.})\\
        &=C(p)d^p.
    \end{aligned}$$
    Here the constant $$C(p)=\int_{-\infty}^\infty\frac{1}{\sqrt{2\pi}}e^{-\frac{x^2}{2}}|x|^{2p}\dd x<\infty$$
    only depends on $p$.
\end{proof}

Also, the following is another simple fact that is useful for our analysis.

\begin{lemma}
    For two vectors $u\in\mathbb{R}^n$, $v\in\mathbb{R}^m$, the matrix $uv^T\in\mathbb{R}^{n\times m}$ satisfies
    $$\Vert uv^T\Vert_F=\Vert u\Vert\cdot\Vert v\Vert,$$
    where $\Vert\cdot\Vert_F$ denotes the Frobenious norm and $\Vert\cdot\Vert$ denotes the 2-norm.
    \label{f-norm}
\end{lemma}

\begin{proof} By the definition of the Frobenious norm,
    $$\begin{aligned}
        \Vert uv^T\Vert_F^2&=\sum_{i=1}^n\sum_{j=1}^m(uv^T)_{ij}^2\\
        &=\sum_{i=1}^n\sum_{j=1}^mu_i^2v_j^2\\
        &=\sum_{i=1}^nu_i^2\cdot\sum_{j=1}^mv_j^2\\
        &=\Vert u\Vert^2\cdot\Vert v\Vert^2.
    \end{aligned}$$
\end{proof}

Recall that we have defined $v(t,x)=\mathbb{E}[\partial_tI(t,x_0,x_1)|x_t=x]$. We then give bounds for the score functions and the velocity functions.

\begin{lemma}
    For $p\ge 1$, there exists a constant $C(p)$ that depends only on $p$, s.t. for $t\in(0,1)$, 
    $$\begin{aligned}
        \mathbb{E}\Vert s(t,x_t)\Vert^p&\le C(p)\gamma^{-p}d^{p/2},\\
        \mathbb{E}\Vert v(t,x_t)\Vert^p&\le C(p)\mathbb{E}\Vert x_1-x_0\Vert^p,\\
        \mathbb{E}\Vert b(t,x_t)\Vert^p&\le C(p)\left[\mathbb{E}\Vert x_1-x_0\Vert^p+\dot{\gamma}d^{p/2}\right],\\
        \mathbb{E}\Vert b_F(t,x_t)\Vert^p&\le C(p)\left[\mathbb{E}\Vert x_1-x_0\Vert^p+(\dot{\gamma}^p-\gamma^{-p}\epsilon^p)d^{p/2}\right].
    \end{aligned}$$
    \label{lem:vsb-bound}
\end{lemma}

\begin{proof}
    When $p\ge 1$, use the conditional expectation form of $s$ and $v$ and apply Jensen's inequality, we then obtain
    $$\begin{aligned}
        \mathbb{E}\Vert s(t,x_t)\Vert^p&=\mathbb{E}\Vert\gamma^{-1}\mathbb{E}[z|x_t=x]\Vert^p\le\gamma^{-p}\mathbb{E}\Vert z\Vert^p\le C(p)\gamma^{-p}d^{p/2},\\
        \mathbb{E}\Vert v(t,x_t)\Vert^p&=\mathbb{E}\Vert\mathbb{E}[\partial_tI|x_t=x]\Vert^p\le\mathbb{E}\Vert\partial_tI\Vert^p\le C(p)\mathbb{E}\Vert x_1-x_0\Vert^p,
    \end{aligned}$$
    Moreover, since $b(t,x)=v(t,x)+\dot{\gamma}\gamma s(t,x)$ and $b_F(t,x)=b(t,x)+\epsilon s(t,x)$,
    $$\begin{aligned}
        \mathbb{E}\Vert b(t,x_t)\Vert^p&\le C(p)\left[\mathbb{E}\Vert x_1-x_0\Vert^p+\dot{\gamma}^pd^{p/2}\right],\\
        \mathbb{E}\Vert b_F(t,x_t)\Vert^p&\le C(p)\left[\mathbb{E}\Vert x_1-x_0\Vert^p+(\dot{\gamma}^p-\gamma^{-p}\epsilon^p)d^{p/2}\right].
    \end{aligned}$$
\end{proof}

\subsection{Bounds on Time and Space Derivatives}

\textbf{Note:} In the following sections, we will use the fact that $\frac{\dd}{\dd t}\gamma^2(t)=O(1)$ and $\frac{\dd^2}{\dd t^2}\gamma^2(t)=O(1)$.

Before we move on to the lemmas, we first discuss the conditional expectation itself. By the definition $x_t=I(t,x_0,x_1)+\gamma(t)z$, we can just know that the density of $x_t$ can be expressed as
$$\rho(t,x)=\int_{\mathbb{R}^d\times\mathbb{R}^d}\frac{1}{(2\pi\gamma(t)^2)^{d/2}}\exp\left(-\frac{\Vert x-I(t,x_0,x_1)\Vert^2}{2\gamma(t)^2}\right)\dd\nu(x_0,x_1).$$
Also, under the condition $x_t=x$, the conditional measure of $(x_0,x_1)$ is then 
$$\frac{1}{\rho(t,x)}\cdot\frac{1}{(2\pi\gamma(t)^2)^{d/2}}\exp\left(-\frac{\Vert x-I(t,x_0,x_1)\Vert^2}{2\gamma(t)^2}\right)\dd\nu(x_0,x_1).$$
Therefore, for any function $f_t(x_t,x_0,x_1)$, its conditional expectation can be written as
$$\begin{aligned}
    \mathbb{E}[f_t(x_t,x_0,x_1)|x_t=x]&=\int_{\mathbb{R}^d\times\mathbb{R}^d}\frac{f_t(x,x_0,x_1)}{\rho(t,x)}\cdot\frac{1}{(2\pi\gamma(t)^2)^{d/2}}\exp\left(-\frac{\Vert x-I(t,x_0,x_1)\Vert^2}{2\gamma(t)^2}\right)\dd\nu(x_0,x_1)\\
    &=\frac{\underset{(x_0,x_1)\sim\nu}{\mathbb{E}}\left[\exp\left(-\frac{\Vert x-I(t,x_0,x_1)\Vert^2}{2\gamma(t)^2}\right)\cdot f_t(x,x_0,x_1)\right]}{\underset{(x_0,x_1)\sim\nu}{\mathbb{E}}\left[\exp\left(-\frac{\Vert x-I(t,x_0,x_1)\Vert^2}{2\gamma(t)^2}\right)\right]}.
\end{aligned}$$

We first consider the time derivative of $v$ in the sense of expectation.

\begin{lemma}
    We have $$\mathbb{E}\Vert\partial_tv(t,x_t)\Vert^2\lesssim\mathbb{E}\Vert\partial_t^2I\Vert^2+\gamma^{-2}d\mathbb{E}\Vert x_0-x_1\Vert^4+\gamma^{-2}\dot{\gamma}^4d^3$$
for $t\in(0,1)$.
    \label{lem:v-time}
\end{lemma}

\begin{proof}
    For $t\in(0,1)$, we can first explicitly write $$v(t,x)=\frac{\underset{(x_0,x_1)\sim\nu}{\mathbb{E}}\left[\exp\left(-\frac{\Vert x-I(t)\Vert^2}{2\gamma(t)^2}\right)\cdot\partial_tI(t)\right]}{\underset{(x_0,x_1)\sim\nu}{\mathbb{E}}\left[\exp\left(-\frac{\Vert x-I(t)\Vert^2}{2\gamma(t)^2}\right)\right]}.$$
    Here we write $I(t)=I(t,x_0,x_1)$ for simplicity, and below we will omit $t$ when it is clear in the context. We now want to compute $\partial_tv(t,x)$. First notice that
    $$\frac{\dd}{\dd t}\left[\exp\left(-\frac{\Vert x-I\Vert^2}{2\gamma^2}\right)\cdot\partial_tI\right]=\exp\left(-\frac{\Vert x-I\Vert^2}{2\gamma^2}\right)\cdot\left[\partial_t^2I+\partial_tI\cdot\left(\frac{\Vert x-I\Vert^2}{\gamma(t)^3}\dot{\gamma}+\frac{x-I}{\gamma^2}\cdot\partial_tI\right)\right].$$
    Note that $\sup_{x\in\mathbb{R}}\exp(-x^2/2)x=e^{-1/2}=C_1<\infty$, $\sup_{x\in\mathbb{R}}\exp(-x^2/2)x^2=2e^{-1}=C_2<\infty$,
    we know that $$\left\Vert\frac{\dd}{\dd t}\left[\exp\left(-\frac{\Vert x-I\Vert^2}{2\gamma^2}\right)\cdot\partial_tI\right]\right\Vert\le\Vert\partial_t^2I\Vert+C_2\gamma^{-1}\dot{\gamma}\Vert\partial_tI\Vert+C_1\gamma^{-1}\Vert\partial_tI\Vert^2,$$
Therefore, using dominated convergence theorem, we know that
    $$\frac{\dd}{\dd t}\underset{(x_0,x_1)\sim\nu}{\mathbb{E}}\left[\exp\left(-\frac{\Vert x-I\Vert^2}{2\gamma^2}\right)\cdot\partial_tI\right]=\underset{(x_0,x_1)\sim\nu}{\mathbb{E}}\left[\frac{\dd}{\dd t}\left(\exp\left(-\frac{\Vert x-I\Vert^2}{2\gamma^2}\right)\cdot\partial_tI\right)\right].$$
Similarly we can do this for the denominator, so that we can compute the overall derivative. Let $f_t(x_0,x_1)=-\frac{\Vert x-I(t)\Vert^2}{2\gamma^2}$, for simplicity we may just write $f_t$. Then,
    $$\begin{aligned}
        \partial_tv(t,x)&=\frac{\underset{(x_0,x_1)\sim\nu}{\mathbb{E}}\left[\exp\left(f_t\right)\cdot\partial_t^2I\right]}{\underset{(x_0,x_1)\sim\nu}{\mathbb{E}}\left[\exp\left(f_t\right)\right]}\\
        &\qquad+\frac{\underset{(x_0,x_1)\sim\nu}{\mathbb{E}}\left[\exp\left(f_t\right)\cdot\partial_tI\cdot\partial_tf_t\right]}{\underset{(x_0,x_1)\sim\nu}{\mathbb{E}}\left[\exp\left(f_t\right)\right]}\\
&\qquad-\frac{\underset{(x_0,x_1)\sim\nu}{\mathbb{E}}\left[\exp\left(f_t\right)\cdot\partial_tI\right]\cdot\underset{(x_0,x_1)\sim\nu}{\mathbb{E}}\left[\exp\left(f_t\right)\cdot\partial_tf_t\right]}{\left[\underset{(x_0,x_1)\sim\nu}{\mathbb{E}}\left[\exp\left(f_t\right)\right]\right]^2}\\
        &=\mathbb{E}[\partial_t^2I|x_t=x]\\
        &\qquad+\text{Cov}(\partial_tI,\partial_tf_t|x_t=x),
    \end{aligned}$$
    where the last equality uses the previous explanations of conditional expectations. Hence,
    $$\begin{aligned}
        \Vert\partial_tv(t,x)\Vert&\le\mathbb{E}[\Vert\partial_t^2I\Vert|x_t=x]+\sqrt{\mathbb{E}[|\partial_tf_t|^2|x_t=x]}\sqrt{\mathbb{E}[\Vert\partial_tI\Vert^2|x_t=x]}.
    \end{aligned}$$
    Therefore, we have 
    $$\begin{aligned}
        \mathbb{E}\Vert\partial_tv(t,x_t)\Vert^2&\le2\mathbb{E}[\mathbb{E}[\Vert\partial_t^2I\Vert^2|x_t]]+2\mathbb{E}[\mathbb{E}[|\partial_tf_t|^2|x_t]\cdot\mathbb{E}[\Vert\partial_tI\Vert^2|x_t]]&((a+b)^2\le2a^2+2b^2)\\
        &\le2\mathbb{E}\Vert\partial_t^2I\Vert^2+2\sqrt{\mathbb{E}[\mathbb{E}[|\partial_tf_t|^2|x_t]^2]}\cdot\sqrt{\mathbb{E}[\mathbb{E}[\Vert\partial_tI\Vert^2|x_t]^2]}&(\text{Cauchy-Schwarz inequality})\\
        &\le2\mathbb{E}\Vert\partial_t^2I\Vert^2+2\sqrt{\mathbb{E}[\mathbb{E}[|\partial_tf_t|^4|x_t]]}\cdot\sqrt{\mathbb{E}[\mathbb{E}[\Vert\partial_tI\Vert^4|x_t]]}&(\text{Jensen's inequality})\\
        &\le2\mathbb{E}\Vert\partial_t^2I\Vert^2+2\sqrt{\mathbb{E}|\partial_tf_t|^4}\sqrt{\mathbb{E}\Vert\partial_tI\Vert^4}.
    \end{aligned}$$

    Using the requirement $\partial_tI\le C\Vert x_0-x_1\Vert$ in the definition of stochastic interpolants, $\Vert\partial_tI\Vert^4\lesssim\Vert x_0-x_1\Vert^4$. For $\partial_tf_t$, we can directly obtain
    $$\partial_tf=\frac{\Vert x-I\Vert^2}{\gamma^3}\dot{\gamma}+\gamma^{-2}(x-I)\cdot\partial_tI=\gamma^{-1}\dot{\gamma}\Vert z\Vert^2+\gamma^{-1}z\cdot\partial_tI.$$
Recall that we have defined $x_t=I(t,x_0,x_1)+\gamma(t)z$ where $z$ is an independent gaussian variable $z\sim\mathcal{N}(0,I_d)$. By \Cref{lem:moment-z}, $$\mathbb{E}\Vert z\Vert^8\lesssim d^4,\qquad\mathbb{E}\Vert z\Vert^4\lesssim d^2,$$
we have $$\mathbb{E}|\partial_tf_t|^4\lesssim(\gamma^{-1}\dot{\gamma})^4d^4+\gamma^{-4}d^2\mathbb{E}\Vert x_0-x_1\Vert^4.$$
Therefore, we can finally deduce that
    $$\mathbb{E}\Vert\partial_tv(t,x_t)\Vert^2\lesssim\mathbb{E}\Vert\partial_t^2I\Vert^2+\gamma^{-2}d\mathbb{E}\Vert x_0-x_1\Vert^4+\gamma^{-2}\dot{\gamma}^4d^3.$$
\end{proof}

In addition, we want to consider the space derivative of the velocity for a fixed $t\in(0,1)$. That is, we want to give a bound for $\nabla v(t,x)$. Here we use the notation $\nabla v(t,x)$ to denote the Jacobian matrix $\left(\frac{d}{dx^i}v(t,x)_j\right)_{ij}$, where $x^i$ represents the value of vector $x$ at the $i$-th dimension.

\begin{lemma}
    We have $$\mathbb{E}\Vert\nabla v(t,x)\Vert_F^p\le C(p)\gamma^{-p}d^{p/2}\sqrt{\mathbb{E}\Vert x_0-x_1\Vert^{2p}}$$
    for $p\ge1$, $t\in(0,1)$, where $C(p)$ is a constant that only depends on $p$ and $\Vert\cdot\Vert_F$ denotes the Frobenius norm.
    \label{lem:v-space}
\end{lemma}

\begin{proof}
    Similar to the proof of Lemma \ref{lem:v-time}, $$\nabla\left(\exp\left(-\frac{\Vert x-I\Vert^2}{2\gamma^2}\right)\cdot\partial_tI\right)=\exp\left(-\frac{\Vert x-I\Vert^2}{2\gamma^2}\right)\left(\partial_tI\otimes\nabla\left(-\frac{\Vert x-I\Vert^2}{2\gamma^2}\right)\right),$$
    where $\otimes$ denotes the tensor product, which denotes $\partial_tI\otimes\nabla\left(-\frac{\Vert x-I\Vert^2}{2\gamma^2}\right)=\partial_tI\cdot\nabla\left(-\frac{\Vert x-I\Vert^2}{2\gamma^2}\right)^T$ here in the matrix form. Again, by dominated convergence theorem we can move the gradient operator into the expectation. Using the same notations (i.e., $f_t$ and so on), we can deduce that
    $$\begin{aligned}
    \nabla v(t,x)&=\frac{\underset{(x_0,x_1)\sim\nu}{\mathbb{E}}[\exp(f_t)\cdot(\partial_tI\otimes\nabla f_t)]}{\underset{(x_0,x_1)\sim\nu}{\mathbb{E}}[\exp(f_t)]}\\
    &\qquad-\frac{\underset{(x_0,x_1)\sim\nu}{\mathbb{E}}[\exp(f_t)\cdot\partial_tI]\otimes\underset{(x_0,x_1)\sim\nu}{\mathbb{E}}[\exp(f_t)\cdot\nabla f_t]}{\left[\underset{(x_0,x_1)\sim\nu}{\mathbb{E}}[\exp(f_t)]\right]^2}\\
    &=\text{Cov}(\partial_tI,\nabla f_t|x_t=x).
    \end{aligned}$$
    Again, the last equality uses the definition of covariance. Thus, by Cauchy-Schwarz inequality,
    $$\begin{aligned}
        \Vert\nabla v(t,x)\Vert_F&\le\sqrt{\mathbb{E}[\Vert\partial_tI\Vert^2|x_t=x]}\sqrt{\mathbb{E}[\Vert\nabla f_t\Vert^2|x_t=x]}.
    \end{aligned}$$
    Therefore, we can use Cauchy-Schwarz inequality again and apply Jensen's inequality to deduce that for any $p\ge1$,
    $$\begin{aligned}
        \mathbb{E}\Vert\nabla v(t,x_t)\Vert_F^p&\le\sqrt{\left[\mathbb{E}[\mathbb{E}\Vert\partial_tI\Vert^2|x_t]\right]^{p}}\cdot\sqrt{\left[\mathbb{E}[\mathbb{E}\Vert\nabla f_t\Vert^2|x_t]\right]^{p}}\\
        &\le\sqrt{\mathbb{E}\Vert\partial_tI\Vert^{2p}}\cdot\sqrt{\mathbb{E}\Vert\nabla f_t\Vert^{2p}}.
    \end{aligned}$$
    It is clear that $\mathbb{E}\Vert\partial_tI\Vert^{2p}\lesssim\mathbb{E}\Vert x_0-x_1\Vert^{2p}$. Note $$\nabla f_t=-\frac{x-I}{\gamma^2}=-\gamma^{-1}z,$$
    we then deduce that $$\mathbb{E}\Vert\nabla f_t\Vert^{2p}\le C(p)\gamma^{-2p}d^p$$
    for some constant that only depends on $p$. The lemma is then obtained.
\end{proof}

Despite the function $v(t,x)$, we are also interested in the score function $s(t,x)$. The following lemmas provide some similar bounds for $s(t,x)$.

\begin{lemma}
    $$\mathbb{E}\Vert\partial_t\left(\gamma s(t,x_t)\right)\Vert^2\lesssim\gamma^{-2}\dot{\gamma}^2d^3+\gamma^{-2}d^2\sqrt{\mathbb{E}\Vert x_0-x_1\Vert^4}$$
    and 
    $$\mathbb{E}\Vert\partial_ts(t,x_t)\Vert^2\lesssim\gamma^{-4}\dot{\gamma}^2d^3+\gamma^{-4}d^2\sqrt{\mathbb{E}\Vert x_0-x_1\Vert^4}.$$
    for any $t\in(0,1)$, 
    \label{lem:s-time}
\end{lemma}

\begin{proof}
    First using the analysis for the conditional expectations, we obtain that
$$s(t,x)=\nabla\log\rho(t,x)=-\frac{\underset{(x_0,x_1)\sim\nu}{\mathbb{E}}\left[\exp\left(-\frac{\Vert x-I\Vert^2}{2\gamma^2}\right)\cdot\frac{x-I}{\gamma^2}\right]}{\underset{(x_0,x_1)\sim\nu}{\mathbb{E}}\left[\exp\left(-\frac{\Vert x-I\Vert^2}{2\gamma^2}\right)\right]}.$$
    In order to compute $\partial_t(\gamma s(t,x))$, we apply a similar analysis as the proof of \Cref{lem:v-time} with exactly the same notations to deduce that
    $$\begin{aligned}
        \partial_ts(t,x)&=\frac{\underset{(x_0,x_1)\sim\nu}{\mathbb{E}}\left[\exp(f_t)\cdot\partial_t(\gamma\nabla f_t)\right]}{\underset{(x_0,x_1)\sim\nu}{\mathbb{E}}\left[\exp(f_t)\right]}\\
        &\qquad+\frac{\underset{(x_0,x_1)\sim\nu}{\mathbb{E}}\left[\exp(f_t)\cdot\partial_tf_t\cdot\gamma\nabla f_t\right]}{\underset{(x_0,x_1)\sim\nu}{\mathbb{E}}\left[\exp(f_t)\right]}\\
        &\qquad-\frac{\underset{(x_0,x_1)\sim\nu}{\mathbb{E}}\left[\exp(f_t)\cdot\gamma\nabla f_t\right]\cdot\underset{(x_0,x_1)\sim\nu}{\mathbb{E}}\left[\exp(f_t)\cdot\partial_tf_t\right]}{\left[\underset{(x_0,x_1)\sim\nu}{\mathbb{E}}\left[\exp(f_t)\right]\right]^2}\\
        &=\mathbb{E}[\partial_t(\gamma\nabla f_t)|x_t=x]+\text{Cov}(\gamma\nabla f_t,\partial_tf_t|x_t=x)
    \end{aligned}$$
    The above term has exactly the same form as which in the proof of Lemma \ref{lem:v-time}, so by a similar analysis we can obtain that
    $$\mathbb{E}\Vert\partial_t(\gamma s(t,x_t))\Vert^2\le2\mathbb{E}\Vert\partial_t(\gamma\nabla f_t)\Vert^2+2\sqrt{\mathbb{E}|\partial_tf_t|^4}\cdot\sqrt{\mathbb{E}\Vert\gamma\nabla f_t\Vert^4}.$$

    We have already deduced that $$\mathbb{E}\Vert\nabla f_t\Vert^4\lesssim\gamma^{-4}d^2,$$
and $$\mathbb{E}|\partial_tf_t|^4\lesssim(\gamma^{-1}\dot{\gamma})^4d^4+\gamma^{-4}d^2\mathbb{E}\Vert x_0-x_1\Vert^4.$$
Also, $$\partial_t(\gamma\nabla f_t)=\partial_t\left(-\frac{x-I}{\gamma}\right)=\gamma^{-1}\partial_tI+\gamma^{-2}\dot{\gamma}(x-I)=\gamma^{-1}\partial_tI+\gamma^{-1}\dot{\gamma}z$$
Hence, $$\mathbb{E}\Vert\partial_ts(t,x_t)\Vert^2\lesssim\gamma^{-2}\dot{\gamma}^2d^3+\gamma^{-2}d^2\sqrt{\mathbb{E}\Vert x_0-x_1\Vert^4},$$
which completes the first part. The proof of the second part is exactly the same by replacing $\gamma\nabla f_t$ with $\nabla f_t$.
\end{proof}

\begin{lemma}
    For any $p\ge1$, there exists a constant $C(p)<\infty$ that only depends on $p$ such that
    $$\mathbb{E}\Vert\nabla s(t,x)\Vert_F^{p}\le C(p)\gamma^{-2p}d^p.$$
    \label{lem:s-space}
\end{lemma}
\begin{proof}
    With exactly the same ideas of the previous lemmas, we can obtain
    $$\begin{aligned}
        \nabla s(t,x)&=\mathbb{E}[\nabla^2f_t|x_t=x]+\text{Cov}(\nabla f_t,\nabla f_t|x_t=x)\\
        &=-\gamma^{-2}I+\gamma^{-2}\text{Cov}(z,z|x_t=x)
    \end{aligned}$$
    Then, for $p\ge1$, we have
    $$\begin{aligned}
        \mathbb{E}\Vert\nabla s(t,x_t)\Vert_F^{p}&\le2^{p-1}\Vert\gamma^{-2}I\Vert_F^p+2^{p-1}\gamma^{-2p}\mathbb{E}\Vert\mathbb{E}[\Vert z\Vert^2|x_t=x]\Vert^p\\
        &\le 2^{p-1}\gamma^{-2p}d^{p/2}+2^{p-1}\gamma^{-2p}\mathbb{E}\Vert z\Vert^{2p}&(\text{Jensen's inequality})\\
        &\le C(p)\gamma^{-2p}d^p.
    \end{aligned}$$
    Here for the first inequality we have used the fact $(a+b)^p\le 2^{p-1}a^p+2^{p-1}b^p$ for $a,b\ge0$.
\end{proof}

We also need some bounds for $\Delta s$ and $\Delta v$, where $\Delta$ represents the Laplace operator.

\begin{lemma}
    $$\mathbb{E}\Vert\Delta v(t,x_t)\Vert^2\lesssim\gamma^{-2}d\mathbb{E}\Vert x_0-x_1\Vert^4+\gamma^{-4}d^2$$
    for all $t\in(0,1)$.
    \label{lem:v-laplace}
\end{lemma}

\begin{proof}
    We still use the notations in the proof of Lemma \ref{lem:v-time}. First, in the proof of Lemma \ref{lem:v-space}, we have already shown that
    $$\begin{aligned}
        \partial_{x^i}v(t,x)&=
        \frac{\underset{(x_0,x_1)\sim\nu}{\mathbb{E}}[\exp(f_t)\cdot(\partial_tI\cdot\partial_{x^i}f_t)]}{\underset{(x_0,x_1)\sim\nu}{\mathbb{E}}[\exp(f_t)]}\\
        &\qquad-\frac{\underset{(x_0,x_1)\sim\nu}{\mathbb{E}}[\exp(f_t)\cdot\partial_tI]\cdot\underset{(x_0,x_1)\sim\nu}{\mathbb{E}}[\exp(f_t)\cdot\partial_{x^i}f_t]}{\left[\underset{(x_0,x_1)\sim\nu}{\mathbb{E}}[\exp(f_t)]\right]^2}\\
        &=\frac{\underset{(x_0,x_1)\sim\nu}{\mathbb{E}}\left[\underset{(\bar{x}_0,\bar{x}_1)\sim\nu}{\mathbb{E}}[\exp(f_t)\exp(\bar{f}_t)(\partial_tI-\partial_t\bar{I})\cdot(\partial_{x^i}f_t-\partial_{x^i}\bar{f}_t)]\right]}{2\underset{(x_0,x_1)\sim\nu}{\mathbb{E}}\left[\underset{(\bar{x}_0,\bar{x}_1)\sim\nu}{\mathbb{E}}[\exp(f_t)\exp(\bar{f}_t)]\right]}.
    \end{aligned}$$
    The last equality is an alternative form of the covariance, and we use notations $\bar{I}=I(t,\bar{x}_0,\bar{x}_1)$ and $\bar{f}_t=f_t(\bar{x}_0,\bar{x}_1)$ for intermediate variables $(\bar{x}_0,\bar{x}_1)$.
    Hence, $$\begin{aligned}
        \partial^2_{x^i}v(t,x)&=\text{Cov}(\partial_tI,\partial_{x^i}^2f_t|x_t=x)\\
        &\qquad+\frac{1}{2}\text{Cov}[(\partial_tI-\partial_t\bar{I})(\partial_{x^i}f_t-\partial_{x^i}\bar{f}_t),\partial_{x^i}f_t+\partial_{x^i}\bar{f}_t|x_t=\bar{x}_t=x].
    \end{aligned}$$
    For the first term, note that $\partial^2_{x^i}f_t=-\gamma^{-2}$ is fixed. So,
    $$\Delta v(t,x)=\frac{1}{2}\text{Cov}[(\partial_tI-\partial_t\bar{I})(\nabla f_t-\nabla \bar{f}_t),\nabla f_t+\nabla\bar{f}_t|x_t=\bar{x}_t=x].$$
    Here the covariance refers to the expectation of dot product instead of the expectation of tensor product. Then, use the fact $\mathbb{E}\Vert X-\mathbb{E}X\Vert^2\le\mathbb{E}\Vert X\Vert^2$, we know that
    $$\begin{aligned}
        \Vert\Delta v(t,x)\Vert
        &\le\sqrt{\mathbb{E}[\left\Vert(\partial_tI-\partial_t\bar{I})(\nabla f_t-\nabla\bar{f}_t)^T\right\Vert^2|x_t=\bar{x}_t=x]}\\
        &\qquad\cdot\sqrt{\mathbb{E}[\Vert\nabla f_t+\nabla\bar{f}_t\Vert^2|x_t=\bar{x}_t=x]}&(\text{Cauchy-Schwarz inequality})\\
        &\lesssim\left[\mathbb{E}[\left\Vert\partial_tI-\partial_t\bar{I}\right\Vert^4|x_t=\bar{x}_t=x]\right]^{1/4}\\
        &\qquad\cdot\left[\mathbb{E}[\left\Vert\nabla f_t-\nabla\bar{f}_t\right\Vert^4|x_t=\bar{x}_t=x]\right]^{1/4}&(\text{Cauchy-Schwarz inequality})\\
        &\qquad\cdot\sqrt{\mathbb{E}[\Vert\nabla f_t\Vert^2|x_t=x]}&(\text{by symmetry})\\
        &\lesssim\left[\mathbb{E}[\left\Vert\partial_tI\right\Vert^4|x_t=x]\right]^{1/4}\cdot\sqrt{\mathbb{E}[\Vert\nabla f_t\Vert^4|x_t=x]}.&(\text{by symmetry})
    \end{aligned}$$
    Therefore, $$\begin{aligned}
        \mathbb{E}\Vert\Delta v(t,x_t)\Vert^2
        &\lesssim\mathbb{E}\left[\sqrt{\mathbb{E}[\left\Vert\partial_tI\right\Vert^4|x_t=x]}\cdot\mathbb{E}[\Vert\nabla f_t\Vert^4|x_t=x]\right]\\
        &\lesssim\sqrt{\mathbb{E}\left[\mathbb{E}[\left\Vert\partial_tI\right\Vert^4|x_t=x]\right]}\cdot\sqrt{\mathbb{E}\left[\mathbb{E}[\Vert\nabla f_t\Vert^4|x_t=x]^2\right]}&(\text{Cauchy-Schwarz inequality})\\
        &\lesssim\gamma^{-4}\sqrt{\mathbb{E}\Vert\partial_tI\Vert^4}\cdot\sqrt{\mathbb{E}\Vert z\Vert^{8}}&(\text{Jensen's inequality})\\
        &\lesssim\gamma^{-4}\sqrt{\mathbb{E}\Vert x_0-x_1\Vert^4}\cdot d^2\\
        &\lesssim\gamma^{-2}d\mathbb{E}\Vert x_0-x_1\Vert^4+\gamma^{-4}d^2.
    \end{aligned}$$
\end{proof}

\begin{lemma}
    $$\mathbb{E}\Vert\Delta s(t,x)\Vert^2\lesssim\gamma^{-6}d^3$$
    for $t\in(0,1)$.
    \label{lem:s-laplace}
\end{lemma}

\begin{proof}
    $$\begin{aligned}
        \nabla s(t,x)&=-\gamma^{-2}I+\text{Cov}(\nabla f_t,\nabla f_t|x_t=x).
    \end{aligned}$$
    Hence, with similar calculations and notations as in the proof of Lemma \ref{lem:v-laplace}, we can deduce that
    $$\begin{aligned}
        \Delta s(t,x)&=2\text{Cov}(\nabla f_t,\Delta f_t|x_t=x)\\
        &\qquad+\frac{1}{2}\text{Cov}[(\nabla f_t-\nabla\bar{f}_t)(\nabla f_t-\nabla\bar{f}_t)^T,\nabla f_t-\nabla\bar{f}_t|x_t=\bar{x}_t=x].\\
        &=\frac{1}{2}\text{Cov}[(\nabla f_t-\nabla\bar{f}_t)(\nabla f_t-\nabla\bar{f}_t)^T,\nabla f_t-\nabla\bar{f}_t|x_t=\bar{x}_t=x].
    \end{aligned}$$
    Then, with H\"older's inequality, we have
    $$\begin{aligned}
        \Vert\Delta s(t,x)\Vert&\lesssim\left[\mathbb{E}[\Vert\nabla f_t-\nabla\bar{f}_t\Vert^3|x_t=\bar{x}_t=x]\right]^{1/3}\\
        &\qquad\cdot\left[\mathbb{E}[\Vert(\nabla f_t-\nabla\bar{f}_t)(\nabla f_t-\nabla\bar{f}_t)^T\Vert^{3/2}]\right]^{2/3}\\
        &\lesssim\left[\mathbb{E}[\Vert\nabla f_t\Vert^3|x_t=x]\right]^{1/3}&(\text{by symmetry})\\
        &\qquad\cdot\left[\mathbb{E}[\Vert\nabla f_t-\nabla\bar{f}_t\Vert^3|x_t=\bar{x}_t=x]\right]^{2/3}\\
        &\lesssim\mathbb{E}[\Vert\nabla f_t\Vert^3|x_t=x].&(\text{by symmetry})
    \end{aligned}$$
    Hence, by Jensen's inequality, $$\mathbb{E}\Vert\Delta s(t,x)\Vert^2\lesssim\mathbb{E}[\mathbb{E}[\Vert\nabla f_t\Vert^3|x_t=x]^2]\lesssim\gamma^{-6}\mathbb{E}\Vert z\Vert^6\lesssim\gamma^{-6}d^3.$$
\end{proof}



\section{Omitted Proofs in Sections \ref{sec:results} and \ref{sec:instance}}
\label{appendix:overall}

\subsection{Bounds along the forward Path}

Recall the forward ODE $$\dd X_t=b(t,X_t)\dd t$$
and the forward SDE $$\dd X_t^F=b_F(t,X_t^F)\dd t+\sqrt{2\epsilon}dW_t.$$
Their solutions are denoted by $X_t$ and $X_t^F$, respectively. Using the chain rule or It\^o's formula, for a function $f(t,x)$ that is twice continuously differentiable, we have 
$$\dd f(t,X_t)=[\partial_tf(t,X_t)+\nabla f(t,X_t)\cdot b(t,X_t)]\dd t,$$
and $$\dd f(t,X_t^F)=[\partial_tf(t,X_t^F)+\nabla f(t,X_t^F)\cdot b_F(t,X_t^F)+\epsilon\Delta f]\dd t+\sqrt{2\epsilon}\nabla f(t,X_t^F)\cdot \dd W_t.$$

With the above formula, we can now provide the following bound on the discretization error.

\begin{lemma}
    % Suppose that we take $\epsilon=\Theta(1)$. Then 
    For $0<t_0\le t_1<1$, suppose $\epsilon=O(1)$, then,
    $$\begin{aligned}
        \mathbb{E}\Vert v(t_0,X_{t_0}^F)-v(t_1,X_{t_1}^F)\Vert^2&\lesssim(t_1-t_0)^2\left[M_2+\gamma_{\min}^{-6}d^3+\gamma_{\min}^{-2}d\sqrt{\mathbb{E}\Vert x_0-x_1\Vert^{8}}\right]\\
        &\qquad+\epsilon(t_1-t_0)\gamma_{\min}^{-2}d\sqrt{\mathbb{E}\Vert x_0-x_1\Vert^{4}},
    \end{aligned}$$
    and $$\mathbb{E}\Vert s(t_0,X_{t_0}^F)-s(t_1,X_{t_1}^F)\Vert^2\lesssim(t_1-t_0)^2\left[\gamma_{\min}^{-4}d^2\sqrt{\mathbb{E}\Vert x_0-x_1\Vert^4}+\gamma_{\min}^{-6}d^3\right]+\epsilon(t_1-t_0)\gamma_{\min}^{-4}d^2.$$
    Here we denote $\gamma_{\min}=\min_{u\in[t_0,t_1]}\gamma$.
    \label{lem:discretize}
\end{lemma}

\begin{proof}
    According to the formula 
    $$\dd v(t,X_t^F)=[\partial_tv(t,X_t^F)+\nabla v(t,X_t^F)\cdot b_F(t,X_t^F)+\epsilon\Delta v(t,X_t^F)]\dd t+\sqrt{2\epsilon}\nabla v(t,X_t^F)\cdot \dd W_t,$$
    we know that 
    $$\begin{aligned}
        \Vert v(t_1,X_{t_1}^F)-v(t_0,X_{t_0}^F)\Vert^2
        &\le4\left\Vert\int_{t_0}^{t_1}\partial_uv(u,X_u^F)\dd u\right\Vert^2\\
        &\qquad+4\left\Vert\int_{t_0}^{t_1}\nabla v(u,X_u^F)\cdot b(u,X_u^F)\dd u\right\Vert^2\\
        &\qquad+4\left\Vert\int_{t_0}^{t_1}\epsilon\Delta v(u,X_u^F)\dd u\right\Vert^2\\
        &\qquad+4\left\Vert\int_{t_0}^{t_1}\sqrt{2\epsilon}\nabla v(u,X_u^F)\cdot \dd W_u\right\Vert^2.
    \end{aligned}$$
    For the first three terms, by Jensen's inequality we know that for any function $Y$, we have $$\left\Vert\int_{t_0}^{t_1}Y(u)du\right\Vert^2\le(t_1-t_0)\int_{t_0}^{t_1}\Vert Y(u)\Vert^2du.$$
    For the last term, use It\^o's isometry (\citealt{le2016brownian}, Equation 5.8), we can get $$\mathbb{E}\left[\left\Vert\int_{t_0}^{t_1}\sqrt{2\epsilon}\nabla v(u,X_u^F)\cdot dW_u\right\Vert^2\right]=\int_{t_0}^{t_1}\mathbb{E}\Vert\sqrt{2\epsilon}\nabla v(u,X_u^F)\Vert_F^2du.$$
    Therefore, we can use Fubini's theorem to change the order of expectation and integral, and combine the results of Lemma \ref{lem:v-time}, \ref{lem:v-space}, \ref{lem:v-laplace} and \ref{lem:vsb-bound} and Assumption \ref{a:regularity} to get
    $$\begin{aligned}
        \Vert v(t_1,X_{t_1}^F)-v(t_0,X_{t_0}^F)\Vert^2&\lesssim(t_1-t_0)\int_{t_0}^{t_1}\mathbb{E}\Vert\partial_uv(u,X_u)\Vert^2\dd u\\
        &\qquad+(t_1-t_0)\int_{t_0}^{t_1}\left[\gamma_{\min}^2d^{-1}\mathbb{E}\Vert\nabla v(u,X_u)\Vert^4+\gamma_{\min}^{-2}d\mathbb{E}\Vert b_F(u,X_u)\Vert^4\right]\dd u\\
        &\qquad+\epsilon^2(t_1-t_0)\int_{t_0}^{t_1}\mathbb{E}\Vert\Delta v(u,X_u)\Vert^2\dd u\\
        &\qquad+2\epsilon\int_{t_0}^{t_1}\mathbb{E}\Vert\nabla v(u,X_u^F)\Vert_F^2\dd u\\
        &\lesssim(t_1-t_0)^2\left[M_2+\gamma_{\min}^{-6}d^3+\gamma_{\min}^{-2}d\sqrt{\mathbb{E}\Vert x_0-x_1\Vert^{8}}\right]\\
        &\qquad+\epsilon(t_1-t_0)\gamma^{-2}d^{1}\sqrt{\mathbb{E}\Vert x_0-x_1\Vert^{4}}
    \end{aligned}$$
    Note that we have already used the condition $\gamma^2\in C^2[0,1]$ and $\gamma\dot{\gamma}=O(1)$.

    Similarly, we can use Lemma \ref{lem:s-time}, \ref{lem:s-space}, \ref{lem:s-laplace} and \ref{lem:vsb-bound}, and the formula
    $$\dd s(t,X_t^F)=[\partial_ts(t,X_t^F)+\nabla s(t,X_t^F)\cdot b_F(t,X_t^F)+\epsilon\Delta s(t,X_t^F)]\dd t+\sqrt{2\epsilon}\nabla s(t,X_t^F)\cdot \dd W_t$$
    to bound
    $$\begin{aligned}
        \mathbb{E}\Vert s(t_1,X_{t_1}^F)-s(t_0,X_{t_0}^F)\Vert^2\lesssim(t_1-t_0)^2\left[\gamma_{\min}^{-4}d^2\sqrt{\mathbb{E}\Vert x_0-x_1\Vert^4}+\gamma_{\min}^{-6}d^3\right]+\epsilon(t_1-t_0)\gamma_{\min}^{-4}d^2,
    \end{aligned}$$
    where $\gamma_{\min}=\min_{u\in[s,t]}\gamma$.
\end{proof}

\subsection{Proof of Theorem \ref{thm:main}}
\label{appendix:proofofmain}

We first give the following proposition, which is a result from \cite{chen2023ddpm}.

\begin{proposition}
    (Section 5.2 of \cite{chen2023ddpm}) Let $P$, $Q$ be the path measures of solutions of SDE (\ref{eq:forward-sde}) and (\ref{eq:estimated-sde}), where they both start from the same distribution $\rho(t_0)$ at time $t=t_0$ and end at time $t=t_N$. Then, if
    $$\mathbb{E}[\Vert b_F(t,X_t^F)-\hat{b}_F(t_k,X_{t_k}^F)\Vert^2]\le C$$
    for any $t\in[t_0,t_N]$ and some constant $C$, we have 
    $$\text{KL}(P\Vert Q)=\frac{1}{4\epsilon}\sum_{k=0}^{N-1}\int_{t_k}^{t_{k+1}}\mathbb{E}[\Vert b_F(t,X_t^F)-\hat{b}_F(t_k,X_{t_k}^F)\Vert^2]\dd t.$$
    Here the expectations are taken over the ground-truth forward process $(X_t^F)_{t\in[t_0,t_N]}\sim P$.
    \label{prop:girsanov}
\end{proposition}

Now, using the above proposition, we are ready to prove \Cref{thm:main}.

\begin{proof}
    Let $P$, $Q$ be the path measures of the solutions to the SDE (\ref{eq:forward-sde}) and (\ref{eq:estimated-sde}), where the solutions start from the same distribution $\rho(t_0)$ at time $t=t_0$, as in Proposition \ref{prop:girsanov}. We first want to check the condition of Proposition \ref{prop:girsanov}. Note that 
    $$\begin{aligned}
        \mathbb{E}\Vert\hat{b}_F(t_k,X_{t_k}^F)-b_F(t,X_t^F)\Vert^2&\overset{(a)}{\le}2\mathbb{E}\Vert\hat{b}_F(t_k,X_{t_k}^F)-b_F(t_k,X_{t_k}^F)\Vert^2+2\mathbb{E}\Vert b_F(t_k,X_{t_k}^F)-b_F(t,X_t^F)\Vert^2\\
        &\overset{(b)}{\le}2\mathbb{E}\Vert\hat{b}_F(t_k,X_{t_k}^F)-b_F(t_k,X_{t_k}^F)\Vert^2+4\mathbb{E}\Vert v(t_k,X_{t_k}^F)-v(t,X_t^F)\Vert^2\\
        &\qquad+4\mathbb{E}\Vert(-\gamma(t_k)\dot{\gamma}(t_k)+\epsilon)s(t_k,X_{t_k}^F)-(-\gamma(t_k)\dot{\gamma}(t_k)+\epsilon)s(t,X_t^F)\Vert^2\\
        &\overset{(c)}{\le}2\mathbb{E}\Vert\hat{b}_F(t_k,X_{t_k}^F)-b_F(t_k,X_{t_k}^F)\Vert^2+4\mathbb{E}\Vert v(t_k,X_{t_k}^F)-v(t,X_t^F)\Vert^2\\
        &\qquad+8(-\gamma(t_k)\dot{\gamma}(t_k)+\epsilon)^2\mathbb{E}\Vert s(t_k,X_{t_k}^F)-s(t,X_t^F)\Vert^2\\
        &\qquad+8(\gamma(t)\dot{\gamma}(t)-\gamma(t_k)\dot{\gamma}(t_k))^2\mathbb{E}\Vert s(t,X_t^F)\Vert^2.
    \end{aligned}$$
    Here (a), (b) and (c) use the triangle inequality and the fact $(a+b)^2\le a^2+b^2$. By Lemmas \ref{lem:vsb-bound} and \ref{lem:discretize}, this term is uniformly bounded in the closed interval $[t_0,t_N]$. In fact, we can apply these lemmas to obtain that
    $$\begin{aligned}
        \mathbb{E}\Vert\hat{b}_F(t_k,X_{t_k}^F)-b_F(t,X_t^F)\Vert^2&\overset{\text{(a)}}{\lesssim}\mathbb{E}\Vert\hat{b}_F(t_k,X_{t_k}^F)-b_F(t_k,X_{t_k}^F)\Vert^2\\
        &\qquad+(t-t_k)^2\left[M_2+\bar{\gamma}_k^{-6}d^3+\bar{\gamma}_k^{-2}d\sqrt{\mathbb{E}\Vert x_0-x_1\Vert^{8}}\right]\\
        &\qquad+\epsilon(t-t_k)\bar{\gamma}_k^{-2}d\sqrt{\mathbb{E}\Vert x_0-x_1\Vert^{4}}\\
        &\qquad+(t-t_k)^2\left[\bar{\gamma}_k^{-4}d^2\sqrt{\mathbb{E}\Vert x_0-x_1\Vert^4}+\bar{\gamma}_k^{-6}d^3\right]+\epsilon(t-t_k)\bar{\gamma}_k^{-4}d^2\\
        &\qquad+(t-t_k)^2\bar{\gamma}_k^{-2}d\\
        &\overset{\text{(b)}}{\lesssim}\mathbb{E}\Vert\hat{b}_F(t_k,X_{t_k}^F)-b_F(t_k,X_{t_k}^F)\Vert^2\\
        &\qquad+(t-t_k)^2\left[M_2+\bar{\gamma}_k^{-6}d^3+\bar{\gamma}_k^{-2}d\sqrt{\mathbb{E}\Vert x_0-x_1\Vert^{8}}\right]\\
        &\qquad+\epsilon(t-t_k)\bar{\gamma}_k^{-2}d\left[\sqrt{\mathbb{E}\Vert x_0-x_1\Vert^4}+\bar{\gamma}_k^{-2}d\right].
    \end{aligned}$$
    Here step (a) directly expands the discretization error using Lemmas \ref{lem:vsb-bound} and \ref{lem:discretize}; step (b) simplifies the terms by applying Young's inequality and that $1+\epsilon^2=O(1)$. Then, by Proposition \ref{prop:girsanov},
    $$\begin{aligned}
        \text{KL}(P\Vert Q)&=\frac{1}{4\epsilon}\sum_{k=0}^{N-1}\int_{t_k}^{t_{k+1}}\mathbb{E}[\Vert b_F(t,X_t^F)-\hat{b}_F(t_k,X_{t_k}^F)\Vert^2]\dd t\\
        &\overset{\text{(a)}}{\lesssim}\varepsilon_{b_F}^2+\epsilon^{-1}\sum_{k=0}^{N-1}(t_{k+1}-t_k)^3\left[M_2+\bar{\gamma}_k^{-6}d^3+\bar{\gamma}_k^{-2}d\sqrt{\mathbb{E}\Vert x_0-x_1\Vert^{8}}\right]\\
        &\qquad+\sum_{k=0}^{N-1}(t_{k+1}-t_k)^2\bar{\gamma}_k^{-2}d\left[\sqrt{\mathbb{E}\Vert x_0-x_1\Vert^4}+\bar{\gamma}_k^{-2}d\right].
    \end{aligned}$$
    Here step (a) just integrates over the upper bound of the disretization error. Now, consider $\text{KL}(\rho(t_N)\Vert\hat{\rho}(t_N))$. Let $\hat{Q}$ be the path measure of solutions of (\ref{eq:estimated-sde}) starting from $\hat{\rho}(t_0)$ instead of $\rho(t_0)$. Then,
    $$\begin{aligned}
        \text{KL}(\rho(t_N)\Vert\hat{\rho}(t_N))\le\text{KL}(P\Vert\hat{Q})&=\mathbb{E}_P\left[\log\frac{\dd P}{\dd\hat{Q}}(X)\right]\\
        &=\mathbb{E}_P\left[\log\left(\frac{\dd P}{\dd Q}(X)\cdot\frac{\dd Q}{\dd\hat{Q}}(X)\right)\right]\\
        &=\mathbb{E}_P\left[\log\frac{\dd P}{\dd Q}(X)\right]+\mathbb{E}_P\left[\log\frac{\dd\rho(t_0)}{\dd\hat{\rho}(t_0)}(X_{t_0})\right]\\
        &=\text{KL}(P\Vert Q)+\text{KL}(\rho(t_0)\Vert\hat{\rho}(t_0)).
    \end{aligned}$$
    The proof is then completed.
\end{proof}

\subsection{Proof of Proposition \ref{cor:schedule}}
\label{appendix:proofofcor}

\begin{proof}
    Using the results of Theorem \ref{thm:main}, 
    $$\begin{aligned}
        \text{KL}(\rho(t_N)\Vert\hat{\rho}(t_N))&\overset{\text{(a)}}{\lesssim}\varepsilon_{b_F}^2+\text{KL}(\rho(t_0)\Vert\hat{\rho}(t_0))+\epsilon^{-1}\sum_{k=0}^{N-1}(t_{k+1}-t_k)^3\left[M_2+\bar{\gamma}_k^{-6}d^3+\bar{\gamma}_k^{-2}d\sqrt{\mathbb{E}\Vert x_0-x_1\Vert^{8}}\right]\\
        &\qquad+\sum_{k=0}^{N-1}(t_{k+1}-t_k)^2\bar{\gamma}_k^{-2}d\left[\sqrt{\mathbb{E}\Vert x_0-x_1\Vert^4}+\bar{\gamma}_k^{-2}d\right]\\
        &\overset{\text{(b)}}{\lesssim}\varepsilon_{b_F}^2+\text{KL}(\rho(t_0)\Vert\hat{\rho}(t_0))+\epsilon^{-1}\sum_{k=0}^{N-1}\left[M_2h_k^3+h^3d^3+hh_k^2d\sqrt{\mathbb{E}\Vert x_0-x_1\Vert^8}\right]\\
        &\qquad+\sum_{k=0}^{N-1}\left[hh_kd\sqrt{\mathbb{E}\Vert x_0-x_1\Vert^4}+h^2d^2\right]\\
        &\overset{\text{(c)}}{\lesssim}\varepsilon_{b_F}^2+\text{KL}(\rho(t_0)\Vert\hat{\rho}(t_0))+\epsilon^{-1}h^2\left(M_2+d\sqrt{\mathbb{E}\Vert x_0-x_1\Vert^8}\right)+\epsilon^{-1}Nh^3d^3\\
        &\qquad+hd\sqrt{\mathbb{E}\Vert x_0-x_1\Vert^4}+Nh^2d^2\\
        &\overset{\text{(d)}}{\lesssim}\varepsilon_{b_F}^2+\text{KL}(\rho(t_0)\Vert\hat{\rho}(t_0))+hd\sqrt{\mathbb{E}\Vert x_0-x_1\Vert^4}+Nh^2d^2.
    \end{aligned}$$
    Here step (a) is the result of Theorem \ref{thm:main}; step (b) uses the fact $h_k=t_{k+1}-t_k=O(h\bar{\gamma}_k^2)$; step (c) uses the fact $\sum_{k=0}^{N-1}h_k=t_N-t_0\le 1$; step (d) omits the higher-order terms.
\end{proof}

\subsection{Proof of Corollary \ref{cor:instant}}
\label{appendix:proofofschedule}

\begin{proof}
    When the number of steps is $N$, we have 
    $$h=\Theta\left(N^{-1}\log\left(\frac{1}{t_0(1-t_N)}\right)\right).$$ 
    Then, by Corollary \ref{cor:schedule} and the assumptions,
    $$\text{KL}(\rho(t_N)\Vert\hat{\rho}(t_N))\lesssim\varepsilon^2+N^{-1}\left[d\sqrt{\mathbb{E}\Vert x_0-x_1\Vert^4}\log\left(\frac{1}{t_0(1-t_N)}\right)+d^2\log^2\left(\frac{1}{t_0(1-t_N)}\right)\right].$$
    This gives the complexity to to make $\text{KL}(\rho(t_N)\Vert\hat{\rho}(t_N))\lesssim\varepsilon^2$.
\end{proof}
%     $$\begin{aligned}
%         \text{KL}(\rho(t_N)\Vert\hat{\rho}(t_N))&\lesssim\varepsilon_{b_F}^2+\text{KL}(\rho(t_0)\Vert\hat{\rho}(t_0))+\sum_{k=0}^{N-1}(t_{k+1}-t_k)^3\left[M_2+\bar{\gamma}_k^{-6}d^3+\bar{\gamma}_k^{-2}d\sqrt{\mathbb{E}\Vert x_0-x_1\Vert^{8}}\right]\\
%         &\qquad+\sum_{k=0}^{N-1}(t_{k+1}-t_k)^2\bar{\gamma}_k^{-2}d\left[\sqrt{\mathbb{E}\Vert x_0-x_1\Vert^4}+\bar{\gamma}_k^{-2}d\right]\\
%         &\lesssim\varepsilon_{b_F}^2+\text{KL}(\rho(t_0)\Vert\hat{\rho}(t_0))+\sum_{k=0}^{N-1}\left[h_k^3M_2+h^3d^3+hh_k^2d\sqrt{\mathbb{E}\Vert x_0-x_1\Vert^8}\right]\\
%         &\qquad+\sum_{k=0}^{N-1}\left[hh_kd\sqrt{\mathbb{E}\Vert x_0-x_1\Vert^4}+h^2d^2\right]\\
%         &\lesssim\varepsilon_{b_F}^2+\text{KL}(\rho(t_0)\Vert\hat{\rho}(t_0))+N(h^3d^3+h^2d^2)\\
%         &\qquad+M_2h^3\sum_{k=0}^{N-1}\left(\min\{t_k,1-t_{k+1}\}\right)^3\\
%         &\qquad+\sqrt{\mathbb{E}\Vert x_0-x_1\Vert^8}h^3d\sum_{k=0}^{N-1}\left(\min\{t_k,1-t_{k+1}\}\right)^2\\
%         &\qquad+\sqrt{\mathbb{E}\Vert x_0-x_1\Vert^4}h^2d\sum_{k=0}^{N-1}\min\{t_k,1-t_{k+1}\}.
%     \end{aligned}$$
%     It is easy to see that $$\sum_{k=0}^{N-1}\min\{t_k,1-t_{k+1}\}^p\lesssim\sum_{k=0}^\infty(1-h)^{kp}\le\frac{1}{ph},$$
%     so $$\begin{aligned}
%         \text{KL}(\rho(t_N)\Vert\hat{\rho}(t_N))&\lesssim\varepsilon_{b_F}^2+\text{KL}(\rho(t_0)\Vert\hat{\rho}(t_0))+(h^2d^3+hd^2)(\log(1/t_0)+\log(1/\delta))\\
%         &\qquad+h^2\left(M_2+\sqrt{\mathbb{E}\Vert x_0-x_1\Vert^8}d\right)\\
%         &\qquad+h\sqrt{\mathbb{E}\Vert x_0-x_1\Vert^4}d.
%     \end{aligned}$$
% When $h\lesssim 1/d$, view $\sqrt{\mathbb{E}\Vert x_0-x_1\Vert^8}$ and $M_2$ as constants (Assumption \ref{a:regularity}), then we have
% $$\begin{aligned}
%     \text{KL}(\rho(t_N)\Vert\hat{\rho}(t_N))&\lesssim\varepsilon_{b_F}^2+\text{KL}(\rho(t_0)\Vert\hat{\rho}(t_0))+h\left[d^2\log\left(\frac{1}{t_0(1-t_0)}\right)+d\sqrt{\mathbb{E}\Vert x_0-x_1\Vert^4}\right].
% \end{aligned}$$

% $$\begin{aligned}
%     \text{KL}(\rho(t_N)\Vert\hat{\rho}(t_N))&\lesssim\varepsilon_{b_F}^2+\text{KL}(\rho(t_0)\Vert\hat{\rho}(t_0))\\
%     &+N^{-1}\left[d^2\log^2\left(\frac{1}{t_0(1-t_0)}\right)+d\sqrt{\mathbb{E}\Vert x_0-x_1\Vert^4}\log\left(\frac{1}{t_0(1-t_0)}\right)\right].
% \end{aligned}$$
% This gives the result.


\subsection{Reducing to Diffusion Models}
\label{appendix:reduce-to-gaussian}

By modifying the definition of stochastic interpolant to $$x_t=I(t,x_1)+\gamma(t)z$$
and change the condition on $I$ to $\Vert\partial_tI(t,x_1)\Vert\le C\Vert x_1\Vert$, we can repeat the previous analysis while replacing $\sqrt{\mathbb{E}\Vert x_0-x_1\Vert^p}$ by $\sqrt{\mathbb{E}\Vert x_1\Vert^p}$. For the case of diffusion models, we can choose $I(t,x_1)=tx_1$ and $\gamma(t)=\sqrt{1-t^2}$ to obtain a process with the same marginal distributions. Moreover, under this definition of interpolants, we can choose $t_0=0$ and $h_k=t_{k+1}-t_k\propto(1-t_k)$ as the time schedule to recover the sample complexity of diffusion models.

\subsection{Omitted Proofs for $\gamma^2(t)=(1-t)^2t$}
\label{appendix:another}

In this section, we will design a schedule for $\gamma^2(t)=(1-t)^2t$, and provide the corresponding complexity deduced using \Cref{thm:main}. Moreover, we also derived the complexity of using a uniform schedule for comparison.

\begin{corollary}
    For $\gamma^2(t)=(1-t)^2t$, there exists a schedule so that under the same assumptions as \Cref{cor:instant}, the complexity is given by
    $$N=O\left(\frac{1}{\varepsilon^2}\left[\sqrt{\mathbb{E}\Vert x_0-x_1\Vert^4}d\left(\frac{1}{\sqrt{1-t_N}}+\log\left(\frac{1}{t_0}\right)\right)+d^2\left(\frac{1}{(1-t_N)^2}+\log^2\left(\frac{1}{t_0}\right)\right)\right]\right).$$
    In addition, the complexity for using a uniform schedule is $$N=O\left(\frac{1}{\varepsilon^2}\left[\sqrt{\mathbb{E}\Vert x_0-x_1\Vert^4}d\left(\frac{1}{1-t_N}+\log\left(\frac{1}{t_0}\right)\right)+d^2\left(\frac{1}{(1-t_N)^3}+\frac{1}{t_0}\right)\right]\right).$$
\end{corollary}

\begin{proof}
    Here we also take $h_M=0.5$ for some $M>0$. Then, we define
    $$h_k=\begin{cases}
        h_A\cdot t_{k+1},&k<M\\
        h_B\cdot (1-t_k)^{1.5},&k\ge M.
    \end{cases}$$
    for some $h_A\in[0,0.5),h_B\in[0,1)$.
    For the part $k<M$ and $t_k\in[0,0.5)$, $\gamma^2(t_k)=\Theta(t_k)$, so it is the same as what we have discussed for the case, and we need $$M=N_1=O\left(\frac{1}{\varepsilon^2}\left[\sqrt{\mathbb{E}\Vert x_0-x_1\Vert^4}d\log\left(\frac{1}{t_0}\right)+d^2\log^2\left(\frac{1}{t_0}\right)\right]\right)$$
    steps to make the discretization error $$\begin{aligned}
        &\varepsilon_{b_F}^2+\sum_{k=0}^{M-1}(t_{k+1}-t_k)^3\left[M_2+\bar{\gamma}_k^{-6}d^3+\bar{\gamma}_k^{-2}d\sqrt{\mathbb{E}\Vert x_0-x_1\Vert^{8}}\right]\\
        &\qquad+\sum_{k=0}^{M-1}(t_{k+1}-t_k)^2\bar{\gamma}_k^{-2}d\left[\sqrt{\mathbb{E}\Vert x_0-x_1\Vert^4}+\bar{\gamma}_k^{-2}d\right]\lesssim\varepsilon^2.
    \end{aligned}$$
    For the part $k\ge M$, 
    $$\sum_{k=M}^{N-1}(t_{k+1}-t_k)^3\left[M_2+\bar{\gamma}_k^{-6}d^3+\bar{\gamma}_k^{-2}d\sqrt{\mathbb{E}\Vert x_0-x_1\Vert^{8}}\right]=O(h_B^2),$$
    and by that $h_k=\Theta(h_B\bar{\gamma}_k^{1.5})=\Theta(h_B(1-t_k)^{1.5})$ (use in step (a) below),
    $$\begin{aligned}
        &\qquad\sum_{k=M}^{N-1}h_k^2\bar{\gamma}_k^{-2}d\left[\sqrt{\mathbb{E}\Vert x_0-x_1\Vert^4}+\bar{\gamma}_k^{-2}d\right]\\
        &\overset{\text{(a)}}{\lesssim} h_B\sum_{k=M}^{N-1}h_k\left[d\sqrt{\mathbb{E}\Vert x_0-x_1\Vert^4}\bar{\gamma}_k^{-0.5}+\bar{\gamma}_k^{-2.5}d^2\right]\\
        &\overset{\text{(b)}}{\lesssim} h_B\left[d\sqrt{\mathbb{E}\Vert x_0-x_1\Vert^4}\int_{0.5}^{t_N}(1-s)^{-0.5}\dd s+d^2\int_{0.5}^{t_N}(1-s)^{-2.5}\dd s\right]\\
        &\lesssim h_B\left[d\sqrt{\mathbb{E}\Vert x_0-x_1\Vert^4}+d^2\frac{1}{(1-t_N)^{1.5}}\right].
    \end{aligned}$$
    Here the inequality (b) is by that $\bar{\gamma}_k=\Theta(1-t)$ for $t\in[t_k,t_{k+1}]$. Now, we want to compute the number of steps $N_2=N-M$ for the part $k\ge M$. Note that if $t_k=1-2^{-p}$, it takes $O(2^{p/2}h_B^{-1})$ more steps to reach $1-2^{-p-1}$. Hence $N_2=O\left(h_B^{-1}(1-t_N)^{-0.5}\right)$, so we need to take $h_B=\Theta\left(N^{-1}(1-t_N)^{-0.5}\right)$.
    Therefore,
    $$\begin{aligned}
        &\qquad\sum_{k=M}^{N-1}h_k^2\bar{\gamma}_k^{-2}d\left[\sqrt{\mathbb{E}\Vert x_0-x_1\Vert^4}+\bar{\gamma}_k^{-2}d\right]\\
        &\lesssim N^{-1}\left[\frac{d\sqrt{\mathbb{E}\Vert x_0-x_1\Vert^4}}{\sqrt{1-t_N}}+\frac{d^2}{(1-t_N)^{2}}\right].
    \end{aligned}$$
    Thus, for the part $k>M$, we need $$N-M=N_2=O\left(\frac{1}{\varepsilon^2}\left(\frac{d\sqrt{\mathbb{E}\Vert x_0-x_1\Vert^4}}{\sqrt{1-t_N}}+\frac{d^2}{(1-t_N)^{2}}\right)\right)$$
    steps to make the discretization error bounded by $O(\varepsilon^2)$. Hence, the overall complexity is given by $N=N_1+N_2$, which is our result.

    If we use a uniform schedule, by \Cref{thm:main} and that $\gamma^2(t)=\Theta(\min\{t,(1-t)^2\})$, we can bound
    $$\begin{aligned}
        \text{KL}(\rho(t_N)\Vert\hat{\rho}(t_N))&\overset{\text{(a)}}{\lesssim}\varepsilon_{b_F}^2+\text{KL}(\rho(t_0)\Vert\hat{\rho}(t_0))\\
        &\qquad+\frac{1}{N}\sqrt{\mathbb{E}\Vert x_0-x_1\Vert^4}d\left(\int_{t_0}^{0.5}s^{-1}\dd s+\int_{0.5}^{t_N}(1-s)^{-2}\dd s\right)\\
        &\qquad+\frac{1}{N}d^2\left(\int_{t_0}^{0.5}s^{-2}\dd s+\int_{0.5}^{t_N}(1-s)^{-4}\dd s\right)\\
        &\lesssim\varepsilon_{b_F}^2+\text{KL}(\rho(t_0)\Vert\hat{\rho}(t_0))\\
        &\qquad+\frac{1}{N}\sqrt{\mathbb{E}\Vert x_0-x_1\Vert^4}d\left(\frac{1}{1-t_N}+\log\left(\frac{1}{t_0}\right)\right)\\
        &\qquad+\frac{1}{N}d^2\left(\frac{1}{(1-t_N)^3}+\frac{1}{t_0}\right),
    \end{aligned}$$
    which further gives the complexity bound for uniform schedule. Here the inequality (a) is by applying \Cref{thm:main} and replacing $\bar{\gamma}_k$ with the term of the same order. This bound is worse than using the schedule satisfying that $h_k\lesssim h\bar{\gamma}_k$.
\end{proof}
\section{Experiments}\label{sec:experiments_extra}
The code used in this work will be made publicly available later.
%{The code to replicate our results can be found in this public Github repositroy:}
%\AL{make a new public github repo and include link later}
%\AN{We should adhere to dual blind policy during review process. Please refer to https://icml.cc/Conferences/2025/AuthorInstructions}
%\AN{This mean it is safe to submit the code as a supplement file (zipped). After the acceptance, we can make the code open on github.}
%\AL{Yes, I will consult you on this tomorrow, thank you.}

\subsection{Pseudocode and training settings for mean-field experiments} \label{subsec:pseudocode}

For experiments concerning MFNNs, the output of a neuron in a two-layer MFNN is modelled by: $h(x_i, z_i) = R\tanh(x_i^3) \tanh(x_i^{1\top} z_i + x_i^2)$, where $x_i = (x_i^1, x_i^2, x_i^3) \in \bR^{d + 1 + 1}$ is its parameter, $z_i$ is the given input and $R$ is a scaling constant. The $\tanh$ activation function is placed on the second layer as boundedness of the model is crucial for our analysis. Noisy gradient descent is then used to train neural networks for $T$ epochs each. We omit the pseudocode for training MFNNs with MFLD since it is identical to the backpropagation with noisy gradient descent algorithm.

\paragraph{Algorithm \ref{alg:circle_data}} Generate the double circle data: $\mathcal{D} = \left( z_i, y_i\right)^n_{i=1}$, $z_i \in \bR^2, y_i \in \bR$ before splitting it into $\mathcal{D}_\text{train}$ and $\mathcal{D}_{\text{test}}$. We set $n=200$, $r_\text{inner}=1$, $r_\text{outer}=2$ and use an 80-20 train-test split for the data.

\paragraph{Algorithm \ref{alg:multi_index_data}} Generate the $k$ multi-index data: $\mathcal{D} = \left( z_i, y_i\right)^n_{i=1}$, $z_i \in \bR^d, y_i \in \bR$. A key step is normalizing and projecting $z_i$ to the inside of a $d$-dimensional hypersphere. We set $n = 500$, $d = 100$, $r=5$, $k=100$ and $\bar{R} = 100$. 

\paragraph{Algorithm \ref{alg:classification}} Describes how we obtain and test the performance of merged MFNNs against (an approximation to) the mean-field limit by computing the sup-norm between both outputs. The relevant results are stored into a dictionary for plotting the heatmaps. The training procedure is identical for both the classification and regression problem. Let $M_\text{max} = 20$ and $N_\text{list} = \{50, 100, \dots ,500 \}$ denote the maximum number of networks to merge and list of neuron settings to train in parallel respectively. We set the hyperparameters for training as follows:
\begin{itemize}
    \item Classification: $R=10$, $N_\infty=10000$, $\eta = 0.1$, $\lambda' = 0.1$, $\lambda = 0.01$, $T=200$ and loss function: logistic loss
    \item Regression: $R=10$, $N_\infty=10000$, $\eta = 0.01$, $\lambda' = 0.1$, $\lambda = 0.01$, $T=100$ and loss function: mean squared error
\end{itemize}


\begin{algorithm} 
\caption{Generate data points along cocentric 2D circles}\label{alg:circle_data}
\begin{algorithmic}[1]
\REQUIRE $n$, $r_{\text{inner}}$, $r_{\text{outer}}$
\ENSURE Dataset $\mathcal{D} = \{(z_i, y_i)\}_{i=1}^n$
\STATE Initialize $\mathcal{D} \gets \emptyset$
\FOR{$i = 1$ to $n$}
    \STATE Sample $\theta \sim \text{Uniform}(0, 2\pi)$
    \STATE Sample $\xi_1, \xi_2 \sim \text{Normal}(0, 0.1)$
    \IF{$i <  n/2$}
        \STATE $r \gets r_{\text{inner}}$
        \STATE $y_i \gets -1$
    \ELSE
        \STATE $r \gets r_{\text{outer}}$
        \STATE $y_i \gets +1$
    \ENDIF
    \STATE Compute Cartesian coordinates: $z_i = (r \cos(\theta) + \xi_1, r \sin(\theta) + \xi_2)$
    \STATE Add $(z_i, y_i)$ to $\mathcal{D}$
\ENDFOR
\STATE Randomly shuffle $\mathcal{D}$
\STATE Split $\mathcal{D}$ into $\mathcal{D}_{\text{train}}$ (80\%) and $\mathcal{D}_{\text{test}}$ (20\%)
\STATE \textbf{return} $\mathcal{D}_{\text{train}}, \mathcal{D}_{\text{test}}$
\end{algorithmic}
\end{algorithm}

\clearpage
\begin{figure}[H]
\vspace{-1.5em}
\begin{algorithm}[H] 
\caption{Generate $k$ multi-index data}
\begin{algorithmic}[1] \label{alg:multi_index_data}
\REQUIRE $n$, $d$, $r$, $k$, $\bar{R}$
\ENSURE Dataset $\mathcal{D} = \{(z_i, y_i)\}_{i=1}^n$
\STATE Initialize $\mathcal{D} \gets \emptyset$
\FOR{$i = 1$ to $n$}
    \STATE Sample $\zeta \sim \text{Normal}(0,1)$
    \STATE $\zeta \gets \zeta^{(1/d)} \times r$ \hfill \COMMENT{Get scaling constant}
    \STATE Sample $z \sim \text{Normal}\left(0, \text{I}_d \right)$
    
    \STATE $z_i \gets z/ |z|$ \hfill \COMMENT{Normalize} 
    \STATE $z_i \gets z_i \times \zeta$ \hfill \COMMENT{Project}
    \STATE $y_i \gets 0$
    \FOR{$j = 1$ to $k$}
        \STATE $y_i \gets y_i + \tanh \left(z_i^j \right)$
    \ENDFOR
\STATE $y_i \gets y_i \times (\bar{R}/k)$
\STATE Add $(z_i, y_i)$ to $\mathcal{D}$
\ENDFOR
\STATE Split $\mathcal{D}$ into $\mathcal{D}_{\text{train}}$ (80\%) and $\mathcal{D}_{\text{test}}$ (20\%)
\STATE \textbf{return} $\mathcal{D}_{\text{train}}, \mathcal{D}_{\text{test}}$
\end{algorithmic}
\end{algorithm}
\end{figure}

\begin{algorithm} 
\caption{Training and merging MFNNs}\label{alg:classification}
\begin{algorithmic}[H]
\REQUIRE $\mathcal{D}_{\text{train}}, \mathcal{D}_{\text{test}} = (z_\text{test}, y_\text{test})$, $N_\infty$, $N_\text{list}$, $M_\text{max}$
\ENSURE Dictionary \textit{sup\_norm\_dic} maps $N$ to the average sup\_norm
\STATE $h_\infty \gets$ Train a MFNN with $N_\infty$ neurons on $\mathcal{D}_{\text{train}}$
\STATE $\hat{y}_\infty \gets$Use $h_\infty$ to predict on $\mathcal{D}_{\text{test}}$
\STATE Initialize \textit{sup\_norm\_dic} $\gets \{\}$
\FOR{$N \in N_\text{list}$}
    \STATE  $\{h_N^{1}, h_N^{2}, \dots h^{M_\text{max}}_N \} \gets$Train $M_\text{max}$ MFNNs with $N$ neurons on $\mathcal{D}_{\text{train}}$
    \STATE Initialize \textit{sup\_norm\_lst} $\gets []$

    \FOR{$M \in \{1, 2, \dots M_\text{max} \}$}
        \STATE \textit{sup\_norm\_total} $\gets 0$
        \FOR{50 iterations}
            \STATE Randomly sample $M$ networks from $\left \{ h_N^1, h_N^2, \dots, h^{M_\text{max}}_N \right \}$
            \STATE $h_{MN} \gets$Merge the $M$ networks to form a new neural network 
            \STATE $\hat{y} \gets$Use $h_{MN}$ to predict on $\mathcal{D}_{\text{test}}$
            \STATE \textit{sup\_norm} $\gets \text{max}\left(|\hat{y} - \hat{y}_\infty |\right)$
            \STATE $\text{\textit{sup\_norm\_total}} \gets \text{\textit{sup\_norm\_total}} + \text{\textit{sup\_norm}}$
        \ENDFOR
        \STATE Append \textit{sup\_norm\_total}/ 50 to \textit{sup\_norm\_lst}
    \ENDFOR
    \STATE \textit{sup\_norm\_dic}[\textit{N}] $\gets$ \textit{sup\_norm\_lst}
\ENDFOR
\STATE \textbf{return} \textit{sup\_norm\_dic}
\end{algorithmic}
\end{algorithm}


\clearpage

\subsection{Additional MFNN experiments} \label{subsec:additional_experiments}
Beyond examining the effect of both $M$ and $N$ on sup norm, we also compare the convergence rate of MFNNs using different $\lambda \in \{10^{-1}, 10^{-2}, 10^{-3}, 10^{-4} \}$ on the multi-index regression problem. Since the training dataset is small and we intend to investigate high $\lambda$, we have to consider the low epoch setting to prevent deterioration of generalization capabilities. We train 20 networks in parallel and average the MSE (in log-scale) at each epoch, repeating this for $N\in \{300, 400, \dots, 800\}$. Figure \ref{fig:experiments_extra} shows that higher $\lambda$ improves the convergence speed of particles and makes training more stable. Finally, we merge networks with the same hyperparameters for comparison across different $\lambda$. A similar trend is observed in Table \ref{table:experiments_extra}, highlighting the efficacy of PoC-based ensembling when training for fewer epochs with a high $\lambda$.

\begin{figure}[H]
\vskip 0.2in
\begin{center}
\centerline{\includegraphics[width=\columnwidth]{extra_experiment.png}}
\caption{Averaged test ln(MSE) of singular MFNNs, across different $N$ and $\lambda$ for 5 epochs}  
\label{fig:experiments_extra}
\end{center}
\end{figure}

\begin{table*}[th]
    \centering
    \caption{MSE comparison between merging $M=20$ networks across different $N$ and $\lambda$ after 5 epochs.}
    \label{table:experiments_extra}
    \begin{footnotesize}
    \begin{tabular}{ccccccc} 
    \toprule
    & \multicolumn{6}{c}{$\boldsymbol{N}$} \\
    $\boldsymbol{\lambda}$ & 300 & 400 & 500 & 600 & 700 & 800\\
    \midrule
    $10^{-1}$ & \underline{0.9132253} & \underline{0.9040508}& \underline{0.9075238}& \underline{0.9044338}& \underline{0.9030165}&  \underline{0.9022377}\\
    $10^{-2}$ & 1.2325489& 1.2229528& 1.2166352& 1.1978958 & 1.1921849& 1.1654898\\
    $10^{-3}$ & 1.5718020& 1.5668763& 1.5607907& 1.5581368& 1.5282313& 1.5234329\\
    $10^{-4}$ & 1.6987042& 1.6887244& 1.6631799& 1.6135653& 1.5860944& 1.5821924\\
    \bottomrule
    \end{tabular}
    \end{footnotesize}
\end{table*}


\clearpage


\subsection{LoRA for finetuning language models}\label{subsec:experiments_extra_lora}
To examine the effect of $\lambda$, we perform LoRA and PoC-based merging by varying $\lambda \in \{0,10^{-5},10^{-4}\}$ with one-epoch training. We optimize eight LoRA parameters of rank $N=32$ in parallel using noisy AdamW with the speficied $\lambda$. Table \ref{table:LoRA_comparison_1epoch} summarizes the results. For LoRA, the table lists the best result among the eight LoRA parameters based on the average accuracy across all datasets and also provides the average accuracies of the eight parameters for each dataset. We observed that for Llama2-7B with $\lambda=0$ and  $\lambda=10^{-5}$, the chances of the optimization converging are very low. Consequently, both the average accuracy of eight LoRAs and the accuracy of PoC-based merging are also low. This is because the regularization strength $\lambda$ controls the optimization speed as seen in Theorem \ref{theorem:mfld_convergence}. On the other hand, by using a high constant $\lambda=10^{-4}$ the average performance was improved, and PoC-based merging achieved quite high accuracy even with only one-epoch of training. This result suggests using high $\lambda$ to reduce the training costs, provided it does not negatively affect generalization error. For Llama3-8B, one-epoch training is sufficient to converge, and while LoRA performed well and PoC-based merging further improved the accuracies.

\begin{table*}[th]
    \centering
    \caption{Accuracy comparison of LoRA and PoC-based merging for finetuning Llama models (1 epoch).}
    \label{table:LoRA_comparison_1epoch}
    \begin{footnotesize}
    \begin{tabular}{cccccccccccc}
        \toprule
        \textbf{Model} & \textbf{Method} & \textbf{$\lambda$} & \textbf{SIQA} & \textbf{PIQA} & \textbf{WinoGrande} & \textbf{OBQA} & \textbf{ARC-c} & \textbf{ARC-e} & \textbf{BoolQ} & \textbf{HellaSwag} & \textbf{Ave.} \\
        \midrule
        \multirow{11}{*}{\begin{tabular}{c}Llama2\\7B\end{tabular}}
            & LoRA (best) & $0$  & $80.55$ & $82.86$ & $83.19$ & $81.60$ & $71.08$ & $84.51$ & $71.90$ & $90.21$ & $80.74$ \\
            & LoRA (ave.) & $0$ & $64.73$  & $76.31$ & $77.76$ & $68.70$ & $57.02$ & $69.02$ & $69.04$ & $70.63$ & $69.15$ \\
            & \textbf{PoC merge} & $0$ & $32.29$ & $62.57$ & $83.58$ & $22.20$ & $28.41$ & $29.42$ & $61.53$ & $28.50$ & $43.56$ \\
            \cmidrule(lr){2-12}
            & LoRA (best) & $10^{-5}$ & $80.14$ & $82.37$ & $83.43$ & $80.40$ & $68.86$ & $83.42$ & $71.68$ & $89.94$ & $80.03$ \\
            & LoRA (ave.) & $10^{-5}$  & $74.37$ & $74.12$ & $80.55$ & $67.50$ & $58.34$ & $71.98$ & $69.43$ & $66.25$ & $70.32$ \\
            & \textbf{PoC merge} & $10^{-5}$ & $74.56$ & $83.84$ & $85.16$ & $60.00$ & $63.14$ & $78.37$ & $68.72$ & $92.77$ & $75.82$ \\
            \cmidrule(lr){2-12}
            & LoRA (best) & $10^{-4}$ & $78.20$ & $80.90$ & $81.22$ & $78.40$ & $65.19$ & $79.00$ & $69.97$ & $86.50$ & $77.42$ \\
            & LoRA (ave.) & $10^{-4}$  & $74.42$ & $77.70$ & $76.08$ & $75.93$ & $60.93$ & $76.25$ & $65.68$ & $66.71$ & $71.71$ \\
            & \textbf{PoC merge} & $10^{-4}$ & $80.76$ & $82.15$ & $84.85$ & $84.80$ & $71.25$ & $85.35$ & $72.26$ & $91.65$ & $81.63$ \\
        \midrule
        \multirow{11}{*}{\begin{tabular}{c}Llama3\\8B\end{tabular}}
            & LoRA (best) & $0$ & $80.45$ & $88.47$ & $86.82$ & $87.60$ & $82.25$ & $90.87$ & $73.85$ & $95.78$ & $85.76$ \\
            & LoRA (ave.) & $0$  & $80.51$ & $88.87$ & $86.85$ & $87.00$ & $80.78$ & $90.98$ & $73.71$ & $95.84$ & $85.57$ \\
            & \textbf{PoC merge} & $0$ & $81.73$ & $88.96$ & $87.77$ & $88.00$ & $81.40$ & $91.71$ & $74.46$ & $96.45$ & $86.31$ \\
            \cmidrule(lr){2-12}
            & LoRA (best) & $10^{-5}$ & $80.50$ & $88.68$ & $86.98$ & $86.80$ & $81.48$ & $91.12$ & $75.14$ & $95.97$ & $85.83$ \\
            & LoRA (ave.) & $10^{-5}$  & $80.83$ & $88.64$ & $86.85$ & $87.05$ & $80.39$ & $90.76$ & $71.54$ & $95.87$ & $85.24$ \\
            & \textbf{PoC merge} & $10^{-5}$ & $81.53$ & $89.45$ & $87.92$ & $87.80$ & $82.25$ & $91.79$ & $75.54$ & $96.44$ & $86.59$ \\
            \cmidrule(lr){2-12}
            & LoRA (best) & $10^{-4}$ & $80.30$ & $88.57$ & $86.42$ & $87.20$ & $78.07$ & $89.81$ & $73.61$ & $95.14$ & $84.89$ \\
            & LoRA (ave.) & $10^{-4}$ & $80.00$ & $88.20$ & $85.69$ & $86.23$ & $78.86$ & $89.48$ & $73.08$ & $95.05$ & $84.57$ \\
            & \textbf{PoC merge} & $10^{-4}$ & $80.71$ & $89.72$ & $88.08$ & $89.00$ & $82.17$ & $91.79$ & $74.56$ & $96.36$ & $86.55$ \\
        \bottomrule
    \end{tabular}
    \end{footnotesize}
\end{table*}


\end{document}
