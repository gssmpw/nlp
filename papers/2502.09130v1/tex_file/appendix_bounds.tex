
\section{Bounding the Velocities and Scores}
\label{appendix:lemmas}

\subsection{Useful Lemmas}

To begin with, we first provide moment bounds on the Gaussian variable $z\sim N(0,I_d)$.

\begin{lemma}
    For any $p\ge1$,
    $$\mathbb{E}\Vert z\Vert^{2p}\le C(p)d^p,$$
    where $C(p)$ is a constant that only depends on $p$.
    \label{lem:moment-z}
\end{lemma}

\begin{proof}
    First, $\Vert z\Vert^2=\sum_{i=1}^n z_i^2$, where we represent $z=(z_1,z_2,\cdots,z_d)^T$. For any $n$ positive numbers $a_1,a_2,\dots,a_n$, using Jensen's inequality,
    $$\left(\sum_{i=1}^na_i\right)^p=n^p\left(\frac{1}{n}\sum_{i=1}^na_i\right)^p\le n^p\cdot\frac{1}{n}\sum_{i=1}^na_i^p.$$
    Then,
    $$\begin{aligned}
        \mathbb{E}\Vert z\Vert^{2p}&=\mathbb{E}\left[\left(\sum_{i=1}^dz_i^2\right)^p\right]\\
        &\le d^p\cdot\frac{1}{d}\sum_{i=1}^d\mathbb{E}[|z_i|^{2p}]&(\text{Jensen's inequality})\\
        &\le d^p\mathbb{E}\left[|z_1|^{2p}\right]&(\{z_i\}_{i=1}^d\text{ are i.i.d.})\\
        &=C(p)d^p.
    \end{aligned}$$
    Here the constant $$C(p)=\int_{-\infty}^\infty\frac{1}{\sqrt{2\pi}}e^{-\frac{x^2}{2}}|x|^{2p}\dd x<\infty$$
    only depends on $p$.
\end{proof}

Also, the following is another simple fact that is useful for our analysis.

\begin{lemma}
    For two vectors $u\in\mathbb{R}^n$, $v\in\mathbb{R}^m$, the matrix $uv^T\in\mathbb{R}^{n\times m}$ satisfies
    $$\Vert uv^T\Vert_F=\Vert u\Vert\cdot\Vert v\Vert,$$
    where $\Vert\cdot\Vert_F$ denotes the Frobenious norm and $\Vert\cdot\Vert$ denotes the 2-norm.
    \label{f-norm}
\end{lemma}

\begin{proof} By the definition of the Frobenious norm,
    $$\begin{aligned}
        \Vert uv^T\Vert_F^2&=\sum_{i=1}^n\sum_{j=1}^m(uv^T)_{ij}^2\\
        &=\sum_{i=1}^n\sum_{j=1}^mu_i^2v_j^2\\
        &=\sum_{i=1}^nu_i^2\cdot\sum_{j=1}^mv_j^2\\
        &=\Vert u\Vert^2\cdot\Vert v\Vert^2.
    \end{aligned}$$
\end{proof}

Recall that we have defined $v(t,x)=\mathbb{E}[\partial_tI(t,x_0,x_1)|x_t=x]$. We then give bounds for the score functions and the velocity functions.

\begin{lemma}
    For $p\ge 1$, there exists a constant $C(p)$ that depends only on $p$, s.t. for $t\in(0,1)$, 
    $$\begin{aligned}
        \mathbb{E}\Vert s(t,x_t)\Vert^p&\le C(p)\gamma^{-p}d^{p/2},\\
        \mathbb{E}\Vert v(t,x_t)\Vert^p&\le C(p)\mathbb{E}\Vert x_1-x_0\Vert^p,\\
        \mathbb{E}\Vert b(t,x_t)\Vert^p&\le C(p)\left[\mathbb{E}\Vert x_1-x_0\Vert^p+\dot{\gamma}d^{p/2}\right],\\
        \mathbb{E}\Vert b_F(t,x_t)\Vert^p&\le C(p)\left[\mathbb{E}\Vert x_1-x_0\Vert^p+(\dot{\gamma}^p-\gamma^{-p}\epsilon^p)d^{p/2}\right].
    \end{aligned}$$
    \label{lem:vsb-bound}
\end{lemma}

\begin{proof}
    When $p\ge 1$, use the conditional expectation form of $s$ and $v$ and apply Jensen's inequality, we then obtain
    $$\begin{aligned}
        \mathbb{E}\Vert s(t,x_t)\Vert^p&=\mathbb{E}\Vert\gamma^{-1}\mathbb{E}[z|x_t=x]\Vert^p\le\gamma^{-p}\mathbb{E}\Vert z\Vert^p\le C(p)\gamma^{-p}d^{p/2},\\
        \mathbb{E}\Vert v(t,x_t)\Vert^p&=\mathbb{E}\Vert\mathbb{E}[\partial_tI|x_t=x]\Vert^p\le\mathbb{E}\Vert\partial_tI\Vert^p\le C(p)\mathbb{E}\Vert x_1-x_0\Vert^p,
    \end{aligned}$$
    Moreover, since $b(t,x)=v(t,x)+\dot{\gamma}\gamma s(t,x)$ and $b_F(t,x)=b(t,x)+\epsilon s(t,x)$,
    $$\begin{aligned}
        \mathbb{E}\Vert b(t,x_t)\Vert^p&\le C(p)\left[\mathbb{E}\Vert x_1-x_0\Vert^p+\dot{\gamma}^pd^{p/2}\right],\\
        \mathbb{E}\Vert b_F(t,x_t)\Vert^p&\le C(p)\left[\mathbb{E}\Vert x_1-x_0\Vert^p+(\dot{\gamma}^p-\gamma^{-p}\epsilon^p)d^{p/2}\right].
    \end{aligned}$$
\end{proof}

\subsection{Bounds on Time and Space Derivatives}

\textbf{Note:} In the following sections, we will use the fact that $\frac{\dd}{\dd t}\gamma^2(t)=O(1)$ and $\frac{\dd^2}{\dd t^2}\gamma^2(t)=O(1)$.

Before we move on to the lemmas, we first discuss the conditional expectation itself. By the definition $x_t=I(t,x_0,x_1)+\gamma(t)z$, we can just know that the density of $x_t$ can be expressed as
$$\rho(t,x)=\int_{\mathbb{R}^d\times\mathbb{R}^d}\frac{1}{(2\pi\gamma(t)^2)^{d/2}}\exp\left(-\frac{\Vert x-I(t,x_0,x_1)\Vert^2}{2\gamma(t)^2}\right)\dd\nu(x_0,x_1).$$
Also, under the condition $x_t=x$, the conditional measure of $(x_0,x_1)$ is then 
$$\frac{1}{\rho(t,x)}\cdot\frac{1}{(2\pi\gamma(t)^2)^{d/2}}\exp\left(-\frac{\Vert x-I(t,x_0,x_1)\Vert^2}{2\gamma(t)^2}\right)\dd\nu(x_0,x_1).$$
Therefore, for any function $f_t(x_t,x_0,x_1)$, its conditional expectation can be written as
$$\begin{aligned}
    \mathbb{E}[f_t(x_t,x_0,x_1)|x_t=x]&=\int_{\mathbb{R}^d\times\mathbb{R}^d}\frac{f_t(x,x_0,x_1)}{\rho(t,x)}\cdot\frac{1}{(2\pi\gamma(t)^2)^{d/2}}\exp\left(-\frac{\Vert x-I(t,x_0,x_1)\Vert^2}{2\gamma(t)^2}\right)\dd\nu(x_0,x_1)\\
    &=\frac{\underset{(x_0,x_1)\sim\nu}{\mathbb{E}}\left[\exp\left(-\frac{\Vert x-I(t,x_0,x_1)\Vert^2}{2\gamma(t)^2}\right)\cdot f_t(x,x_0,x_1)\right]}{\underset{(x_0,x_1)\sim\nu}{\mathbb{E}}\left[\exp\left(-\frac{\Vert x-I(t,x_0,x_1)\Vert^2}{2\gamma(t)^2}\right)\right]}.
\end{aligned}$$

We first consider the time derivative of $v$ in the sense of expectation.

\begin{lemma}
    We have $$\mathbb{E}\Vert\partial_tv(t,x_t)\Vert^2\lesssim\mathbb{E}\Vert\partial_t^2I\Vert^2+\gamma^{-2}d\mathbb{E}\Vert x_0-x_1\Vert^4+\gamma^{-2}\dot{\gamma}^4d^3$$
for $t\in(0,1)$.
    \label{lem:v-time}
\end{lemma}

\begin{proof}
    For $t\in(0,1)$, we can first explicitly write $$v(t,x)=\frac{\underset{(x_0,x_1)\sim\nu}{\mathbb{E}}\left[\exp\left(-\frac{\Vert x-I(t)\Vert^2}{2\gamma(t)^2}\right)\cdot\partial_tI(t)\right]}{\underset{(x_0,x_1)\sim\nu}{\mathbb{E}}\left[\exp\left(-\frac{\Vert x-I(t)\Vert^2}{2\gamma(t)^2}\right)\right]}.$$
    Here we write $I(t)=I(t,x_0,x_1)$ for simplicity, and below we will omit $t$ when it is clear in the context. We now want to compute $\partial_tv(t,x)$. First notice that
    $$\frac{\dd}{\dd t}\left[\exp\left(-\frac{\Vert x-I\Vert^2}{2\gamma^2}\right)\cdot\partial_tI\right]=\exp\left(-\frac{\Vert x-I\Vert^2}{2\gamma^2}\right)\cdot\left[\partial_t^2I+\partial_tI\cdot\left(\frac{\Vert x-I\Vert^2}{\gamma(t)^3}\dot{\gamma}+\frac{x-I}{\gamma^2}\cdot\partial_tI\right)\right].$$
    Note that $\sup_{x\in\mathbb{R}}\exp(-x^2/2)x=e^{-1/2}=C_1<\infty$, $\sup_{x\in\mathbb{R}}\exp(-x^2/2)x^2=2e^{-1}=C_2<\infty$,
    we know that $$\left\Vert\frac{\dd}{\dd t}\left[\exp\left(-\frac{\Vert x-I\Vert^2}{2\gamma^2}\right)\cdot\partial_tI\right]\right\Vert\le\Vert\partial_t^2I\Vert+C_2\gamma^{-1}\dot{\gamma}\Vert\partial_tI\Vert+C_1\gamma^{-1}\Vert\partial_tI\Vert^2,$$
Therefore, using dominated convergence theorem, we know that
    $$\frac{\dd}{\dd t}\underset{(x_0,x_1)\sim\nu}{\mathbb{E}}\left[\exp\left(-\frac{\Vert x-I\Vert^2}{2\gamma^2}\right)\cdot\partial_tI\right]=\underset{(x_0,x_1)\sim\nu}{\mathbb{E}}\left[\frac{\dd}{\dd t}\left(\exp\left(-\frac{\Vert x-I\Vert^2}{2\gamma^2}\right)\cdot\partial_tI\right)\right].$$
Similarly we can do this for the denominator, so that we can compute the overall derivative. Let $f_t(x_0,x_1)=-\frac{\Vert x-I(t)\Vert^2}{2\gamma^2}$, for simplicity we may just write $f_t$. Then,
    $$\begin{aligned}
        \partial_tv(t,x)&=\frac{\underset{(x_0,x_1)\sim\nu}{\mathbb{E}}\left[\exp\left(f_t\right)\cdot\partial_t^2I\right]}{\underset{(x_0,x_1)\sim\nu}{\mathbb{E}}\left[\exp\left(f_t\right)\right]}\\
        &\qquad+\frac{\underset{(x_0,x_1)\sim\nu}{\mathbb{E}}\left[\exp\left(f_t\right)\cdot\partial_tI\cdot\partial_tf_t\right]}{\underset{(x_0,x_1)\sim\nu}{\mathbb{E}}\left[\exp\left(f_t\right)\right]}\\
&\qquad-\frac{\underset{(x_0,x_1)\sim\nu}{\mathbb{E}}\left[\exp\left(f_t\right)\cdot\partial_tI\right]\cdot\underset{(x_0,x_1)\sim\nu}{\mathbb{E}}\left[\exp\left(f_t\right)\cdot\partial_tf_t\right]}{\left[\underset{(x_0,x_1)\sim\nu}{\mathbb{E}}\left[\exp\left(f_t\right)\right]\right]^2}\\
        &=\mathbb{E}[\partial_t^2I|x_t=x]\\
        &\qquad+\text{Cov}(\partial_tI,\partial_tf_t|x_t=x),
    \end{aligned}$$
    where the last equality uses the previous explanations of conditional expectations. Hence,
    $$\begin{aligned}
        \Vert\partial_tv(t,x)\Vert&\le\mathbb{E}[\Vert\partial_t^2I\Vert|x_t=x]+\sqrt{\mathbb{E}[|\partial_tf_t|^2|x_t=x]}\sqrt{\mathbb{E}[\Vert\partial_tI\Vert^2|x_t=x]}.
    \end{aligned}$$
    Therefore, we have 
    $$\begin{aligned}
        \mathbb{E}\Vert\partial_tv(t,x_t)\Vert^2&\le2\mathbb{E}[\mathbb{E}[\Vert\partial_t^2I\Vert^2|x_t]]+2\mathbb{E}[\mathbb{E}[|\partial_tf_t|^2|x_t]\cdot\mathbb{E}[\Vert\partial_tI\Vert^2|x_t]]&((a+b)^2\le2a^2+2b^2)\\
        &\le2\mathbb{E}\Vert\partial_t^2I\Vert^2+2\sqrt{\mathbb{E}[\mathbb{E}[|\partial_tf_t|^2|x_t]^2]}\cdot\sqrt{\mathbb{E}[\mathbb{E}[\Vert\partial_tI\Vert^2|x_t]^2]}&(\text{Cauchy-Schwarz inequality})\\
        &\le2\mathbb{E}\Vert\partial_t^2I\Vert^2+2\sqrt{\mathbb{E}[\mathbb{E}[|\partial_tf_t|^4|x_t]]}\cdot\sqrt{\mathbb{E}[\mathbb{E}[\Vert\partial_tI\Vert^4|x_t]]}&(\text{Jensen's inequality})\\
        &\le2\mathbb{E}\Vert\partial_t^2I\Vert^2+2\sqrt{\mathbb{E}|\partial_tf_t|^4}\sqrt{\mathbb{E}\Vert\partial_tI\Vert^4}.
    \end{aligned}$$

    Using the requirement $\partial_tI\le C\Vert x_0-x_1\Vert$ in the definition of stochastic interpolants, $\Vert\partial_tI\Vert^4\lesssim\Vert x_0-x_1\Vert^4$. For $\partial_tf_t$, we can directly obtain
    $$\partial_tf=\frac{\Vert x-I\Vert^2}{\gamma^3}\dot{\gamma}+\gamma^{-2}(x-I)\cdot\partial_tI=\gamma^{-1}\dot{\gamma}\Vert z\Vert^2+\gamma^{-1}z\cdot\partial_tI.$$
Recall that we have defined $x_t=I(t,x_0,x_1)+\gamma(t)z$ where $z$ is an independent gaussian variable $z\sim\mathcal{N}(0,I_d)$. By \Cref{lem:moment-z}, $$\mathbb{E}\Vert z\Vert^8\lesssim d^4,\qquad\mathbb{E}\Vert z\Vert^4\lesssim d^2,$$
we have $$\mathbb{E}|\partial_tf_t|^4\lesssim(\gamma^{-1}\dot{\gamma})^4d^4+\gamma^{-4}d^2\mathbb{E}\Vert x_0-x_1\Vert^4.$$
Therefore, we can finally deduce that
    $$\mathbb{E}\Vert\partial_tv(t,x_t)\Vert^2\lesssim\mathbb{E}\Vert\partial_t^2I\Vert^2+\gamma^{-2}d\mathbb{E}\Vert x_0-x_1\Vert^4+\gamma^{-2}\dot{\gamma}^4d^3.$$
\end{proof}

In addition, we want to consider the space derivative of the velocity for a fixed $t\in(0,1)$. That is, we want to give a bound for $\nabla v(t,x)$. Here we use the notation $\nabla v(t,x)$ to denote the Jacobian matrix $\left(\frac{d}{dx^i}v(t,x)_j\right)_{ij}$, where $x^i$ represents the value of vector $x$ at the $i$-th dimension.

\begin{lemma}
    We have $$\mathbb{E}\Vert\nabla v(t,x)\Vert_F^p\le C(p)\gamma^{-p}d^{p/2}\sqrt{\mathbb{E}\Vert x_0-x_1\Vert^{2p}}$$
    for $p\ge1$, $t\in(0,1)$, where $C(p)$ is a constant that only depends on $p$ and $\Vert\cdot\Vert_F$ denotes the Frobenius norm.
    \label{lem:v-space}
\end{lemma}

\begin{proof}
    Similar to the proof of Lemma \ref{lem:v-time}, $$\nabla\left(\exp\left(-\frac{\Vert x-I\Vert^2}{2\gamma^2}\right)\cdot\partial_tI\right)=\exp\left(-\frac{\Vert x-I\Vert^2}{2\gamma^2}\right)\left(\partial_tI\otimes\nabla\left(-\frac{\Vert x-I\Vert^2}{2\gamma^2}\right)\right),$$
    where $\otimes$ denotes the tensor product, which denotes $\partial_tI\otimes\nabla\left(-\frac{\Vert x-I\Vert^2}{2\gamma^2}\right)=\partial_tI\cdot\nabla\left(-\frac{\Vert x-I\Vert^2}{2\gamma^2}\right)^T$ here in the matrix form. Again, by dominated convergence theorem we can move the gradient operator into the expectation. Using the same notations (i.e., $f_t$ and so on), we can deduce that
    $$\begin{aligned}
    \nabla v(t,x)&=\frac{\underset{(x_0,x_1)\sim\nu}{\mathbb{E}}[\exp(f_t)\cdot(\partial_tI\otimes\nabla f_t)]}{\underset{(x_0,x_1)\sim\nu}{\mathbb{E}}[\exp(f_t)]}\\
    &\qquad-\frac{\underset{(x_0,x_1)\sim\nu}{\mathbb{E}}[\exp(f_t)\cdot\partial_tI]\otimes\underset{(x_0,x_1)\sim\nu}{\mathbb{E}}[\exp(f_t)\cdot\nabla f_t]}{\left[\underset{(x_0,x_1)\sim\nu}{\mathbb{E}}[\exp(f_t)]\right]^2}\\
    &=\text{Cov}(\partial_tI,\nabla f_t|x_t=x).
    \end{aligned}$$
    Again, the last equality uses the definition of covariance. Thus, by Cauchy-Schwarz inequality,
    $$\begin{aligned}
        \Vert\nabla v(t,x)\Vert_F&\le\sqrt{\mathbb{E}[\Vert\partial_tI\Vert^2|x_t=x]}\sqrt{\mathbb{E}[\Vert\nabla f_t\Vert^2|x_t=x]}.
    \end{aligned}$$
    Therefore, we can use Cauchy-Schwarz inequality again and apply Jensen's inequality to deduce that for any $p\ge1$,
    $$\begin{aligned}
        \mathbb{E}\Vert\nabla v(t,x_t)\Vert_F^p&\le\sqrt{\left[\mathbb{E}[\mathbb{E}\Vert\partial_tI\Vert^2|x_t]\right]^{p}}\cdot\sqrt{\left[\mathbb{E}[\mathbb{E}\Vert\nabla f_t\Vert^2|x_t]\right]^{p}}\\
        &\le\sqrt{\mathbb{E}\Vert\partial_tI\Vert^{2p}}\cdot\sqrt{\mathbb{E}\Vert\nabla f_t\Vert^{2p}}.
    \end{aligned}$$
    It is clear that $\mathbb{E}\Vert\partial_tI\Vert^{2p}\lesssim\mathbb{E}\Vert x_0-x_1\Vert^{2p}$. Note $$\nabla f_t=-\frac{x-I}{\gamma^2}=-\gamma^{-1}z,$$
    we then deduce that $$\mathbb{E}\Vert\nabla f_t\Vert^{2p}\le C(p)\gamma^{-2p}d^p$$
    for some constant that only depends on $p$. The lemma is then obtained.
\end{proof}

Despite the function $v(t,x)$, we are also interested in the score function $s(t,x)$. The following lemmas provide some similar bounds for $s(t,x)$.

\begin{lemma}
    $$\mathbb{E}\Vert\partial_t\left(\gamma s(t,x_t)\right)\Vert^2\lesssim\gamma^{-2}\dot{\gamma}^2d^3+\gamma^{-2}d^2\sqrt{\mathbb{E}\Vert x_0-x_1\Vert^4}$$
    and 
    $$\mathbb{E}\Vert\partial_ts(t,x_t)\Vert^2\lesssim\gamma^{-4}\dot{\gamma}^2d^3+\gamma^{-4}d^2\sqrt{\mathbb{E}\Vert x_0-x_1\Vert^4}.$$
    for any $t\in(0,1)$, 
    \label{lem:s-time}
\end{lemma}

\begin{proof}
    First using the analysis for the conditional expectations, we obtain that
$$s(t,x)=\nabla\log\rho(t,x)=-\frac{\underset{(x_0,x_1)\sim\nu}{\mathbb{E}}\left[\exp\left(-\frac{\Vert x-I\Vert^2}{2\gamma^2}\right)\cdot\frac{x-I}{\gamma^2}\right]}{\underset{(x_0,x_1)\sim\nu}{\mathbb{E}}\left[\exp\left(-\frac{\Vert x-I\Vert^2}{2\gamma^2}\right)\right]}.$$
    In order to compute $\partial_t(\gamma s(t,x))$, we apply a similar analysis as the proof of \Cref{lem:v-time} with exactly the same notations to deduce that
    $$\begin{aligned}
        \partial_ts(t,x)&=\frac{\underset{(x_0,x_1)\sim\nu}{\mathbb{E}}\left[\exp(f_t)\cdot\partial_t(\gamma\nabla f_t)\right]}{\underset{(x_0,x_1)\sim\nu}{\mathbb{E}}\left[\exp(f_t)\right]}\\
        &\qquad+\frac{\underset{(x_0,x_1)\sim\nu}{\mathbb{E}}\left[\exp(f_t)\cdot\partial_tf_t\cdot\gamma\nabla f_t\right]}{\underset{(x_0,x_1)\sim\nu}{\mathbb{E}}\left[\exp(f_t)\right]}\\
        &\qquad-\frac{\underset{(x_0,x_1)\sim\nu}{\mathbb{E}}\left[\exp(f_t)\cdot\gamma\nabla f_t\right]\cdot\underset{(x_0,x_1)\sim\nu}{\mathbb{E}}\left[\exp(f_t)\cdot\partial_tf_t\right]}{\left[\underset{(x_0,x_1)\sim\nu}{\mathbb{E}}\left[\exp(f_t)\right]\right]^2}\\
        &=\mathbb{E}[\partial_t(\gamma\nabla f_t)|x_t=x]+\text{Cov}(\gamma\nabla f_t,\partial_tf_t|x_t=x)
    \end{aligned}$$
    The above term has exactly the same form as which in the proof of Lemma \ref{lem:v-time}, so by a similar analysis we can obtain that
    $$\mathbb{E}\Vert\partial_t(\gamma s(t,x_t))\Vert^2\le2\mathbb{E}\Vert\partial_t(\gamma\nabla f_t)\Vert^2+2\sqrt{\mathbb{E}|\partial_tf_t|^4}\cdot\sqrt{\mathbb{E}\Vert\gamma\nabla f_t\Vert^4}.$$

    We have already deduced that $$\mathbb{E}\Vert\nabla f_t\Vert^4\lesssim\gamma^{-4}d^2,$$
and $$\mathbb{E}|\partial_tf_t|^4\lesssim(\gamma^{-1}\dot{\gamma})^4d^4+\gamma^{-4}d^2\mathbb{E}\Vert x_0-x_1\Vert^4.$$
Also, $$\partial_t(\gamma\nabla f_t)=\partial_t\left(-\frac{x-I}{\gamma}\right)=\gamma^{-1}\partial_tI+\gamma^{-2}\dot{\gamma}(x-I)=\gamma^{-1}\partial_tI+\gamma^{-1}\dot{\gamma}z$$
Hence, $$\mathbb{E}\Vert\partial_ts(t,x_t)\Vert^2\lesssim\gamma^{-2}\dot{\gamma}^2d^3+\gamma^{-2}d^2\sqrt{\mathbb{E}\Vert x_0-x_1\Vert^4},$$
which completes the first part. The proof of the second part is exactly the same by replacing $\gamma\nabla f_t$ with $\nabla f_t$.
\end{proof}

\begin{lemma}
    For any $p\ge1$, there exists a constant $C(p)<\infty$ that only depends on $p$ such that
    $$\mathbb{E}\Vert\nabla s(t,x)\Vert_F^{p}\le C(p)\gamma^{-2p}d^p.$$
    \label{lem:s-space}
\end{lemma}
\begin{proof}
    With exactly the same ideas of the previous lemmas, we can obtain
    $$\begin{aligned}
        \nabla s(t,x)&=\mathbb{E}[\nabla^2f_t|x_t=x]+\text{Cov}(\nabla f_t,\nabla f_t|x_t=x)\\
        &=-\gamma^{-2}I+\gamma^{-2}\text{Cov}(z,z|x_t=x)
    \end{aligned}$$
    Then, for $p\ge1$, we have
    $$\begin{aligned}
        \mathbb{E}\Vert\nabla s(t,x_t)\Vert_F^{p}&\le2^{p-1}\Vert\gamma^{-2}I\Vert_F^p+2^{p-1}\gamma^{-2p}\mathbb{E}\Vert\mathbb{E}[\Vert z\Vert^2|x_t=x]\Vert^p\\
        &\le 2^{p-1}\gamma^{-2p}d^{p/2}+2^{p-1}\gamma^{-2p}\mathbb{E}\Vert z\Vert^{2p}&(\text{Jensen's inequality})\\
        &\le C(p)\gamma^{-2p}d^p.
    \end{aligned}$$
    Here for the first inequality we have used the fact $(a+b)^p\le 2^{p-1}a^p+2^{p-1}b^p$ for $a,b\ge0$.
\end{proof}

We also need some bounds for $\Delta s$ and $\Delta v$, where $\Delta$ represents the Laplace operator.

\begin{lemma}
    $$\mathbb{E}\Vert\Delta v(t,x_t)\Vert^2\lesssim\gamma^{-2}d\mathbb{E}\Vert x_0-x_1\Vert^4+\gamma^{-4}d^2$$
    for all $t\in(0,1)$.
    \label{lem:v-laplace}
\end{lemma}

\begin{proof}
    We still use the notations in the proof of Lemma \ref{lem:v-time}. First, in the proof of Lemma \ref{lem:v-space}, we have already shown that
    $$\begin{aligned}
        \partial_{x^i}v(t,x)&=
        \frac{\underset{(x_0,x_1)\sim\nu}{\mathbb{E}}[\exp(f_t)\cdot(\partial_tI\cdot\partial_{x^i}f_t)]}{\underset{(x_0,x_1)\sim\nu}{\mathbb{E}}[\exp(f_t)]}\\
        &\qquad-\frac{\underset{(x_0,x_1)\sim\nu}{\mathbb{E}}[\exp(f_t)\cdot\partial_tI]\cdot\underset{(x_0,x_1)\sim\nu}{\mathbb{E}}[\exp(f_t)\cdot\partial_{x^i}f_t]}{\left[\underset{(x_0,x_1)\sim\nu}{\mathbb{E}}[\exp(f_t)]\right]^2}\\
        &=\frac{\underset{(x_0,x_1)\sim\nu}{\mathbb{E}}\left[\underset{(\bar{x}_0,\bar{x}_1)\sim\nu}{\mathbb{E}}[\exp(f_t)\exp(\bar{f}_t)(\partial_tI-\partial_t\bar{I})\cdot(\partial_{x^i}f_t-\partial_{x^i}\bar{f}_t)]\right]}{2\underset{(x_0,x_1)\sim\nu}{\mathbb{E}}\left[\underset{(\bar{x}_0,\bar{x}_1)\sim\nu}{\mathbb{E}}[\exp(f_t)\exp(\bar{f}_t)]\right]}.
    \end{aligned}$$
    The last equality is an alternative form of the covariance, and we use notations $\bar{I}=I(t,\bar{x}_0,\bar{x}_1)$ and $\bar{f}_t=f_t(\bar{x}_0,\bar{x}_1)$ for intermediate variables $(\bar{x}_0,\bar{x}_1)$.
    Hence, $$\begin{aligned}
        \partial^2_{x^i}v(t,x)&=\text{Cov}(\partial_tI,\partial_{x^i}^2f_t|x_t=x)\\
        &\qquad+\frac{1}{2}\text{Cov}[(\partial_tI-\partial_t\bar{I})(\partial_{x^i}f_t-\partial_{x^i}\bar{f}_t),\partial_{x^i}f_t+\partial_{x^i}\bar{f}_t|x_t=\bar{x}_t=x].
    \end{aligned}$$
    For the first term, note that $\partial^2_{x^i}f_t=-\gamma^{-2}$ is fixed. So,
    $$\Delta v(t,x)=\frac{1}{2}\text{Cov}[(\partial_tI-\partial_t\bar{I})(\nabla f_t-\nabla \bar{f}_t),\nabla f_t+\nabla\bar{f}_t|x_t=\bar{x}_t=x].$$
    Here the covariance refers to the expectation of dot product instead of the expectation of tensor product. Then, use the fact $\mathbb{E}\Vert X-\mathbb{E}X\Vert^2\le\mathbb{E}\Vert X\Vert^2$, we know that
    $$\begin{aligned}
        \Vert\Delta v(t,x)\Vert
        &\le\sqrt{\mathbb{E}[\left\Vert(\partial_tI-\partial_t\bar{I})(\nabla f_t-\nabla\bar{f}_t)^T\right\Vert^2|x_t=\bar{x}_t=x]}\\
        &\qquad\cdot\sqrt{\mathbb{E}[\Vert\nabla f_t+\nabla\bar{f}_t\Vert^2|x_t=\bar{x}_t=x]}&(\text{Cauchy-Schwarz inequality})\\
        &\lesssim\left[\mathbb{E}[\left\Vert\partial_tI-\partial_t\bar{I}\right\Vert^4|x_t=\bar{x}_t=x]\right]^{1/4}\\
        &\qquad\cdot\left[\mathbb{E}[\left\Vert\nabla f_t-\nabla\bar{f}_t\right\Vert^4|x_t=\bar{x}_t=x]\right]^{1/4}&(\text{Cauchy-Schwarz inequality})\\
        &\qquad\cdot\sqrt{\mathbb{E}[\Vert\nabla f_t\Vert^2|x_t=x]}&(\text{by symmetry})\\
        &\lesssim\left[\mathbb{E}[\left\Vert\partial_tI\right\Vert^4|x_t=x]\right]^{1/4}\cdot\sqrt{\mathbb{E}[\Vert\nabla f_t\Vert^4|x_t=x]}.&(\text{by symmetry})
    \end{aligned}$$
    Therefore, $$\begin{aligned}
        \mathbb{E}\Vert\Delta v(t,x_t)\Vert^2
        &\lesssim\mathbb{E}\left[\sqrt{\mathbb{E}[\left\Vert\partial_tI\right\Vert^4|x_t=x]}\cdot\mathbb{E}[\Vert\nabla f_t\Vert^4|x_t=x]\right]\\
        &\lesssim\sqrt{\mathbb{E}\left[\mathbb{E}[\left\Vert\partial_tI\right\Vert^4|x_t=x]\right]}\cdot\sqrt{\mathbb{E}\left[\mathbb{E}[\Vert\nabla f_t\Vert^4|x_t=x]^2\right]}&(\text{Cauchy-Schwarz inequality})\\
        &\lesssim\gamma^{-4}\sqrt{\mathbb{E}\Vert\partial_tI\Vert^4}\cdot\sqrt{\mathbb{E}\Vert z\Vert^{8}}&(\text{Jensen's inequality})\\
        &\lesssim\gamma^{-4}\sqrt{\mathbb{E}\Vert x_0-x_1\Vert^4}\cdot d^2\\
        &\lesssim\gamma^{-2}d\mathbb{E}\Vert x_0-x_1\Vert^4+\gamma^{-4}d^2.
    \end{aligned}$$
\end{proof}

\begin{lemma}
    $$\mathbb{E}\Vert\Delta s(t,x)\Vert^2\lesssim\gamma^{-6}d^3$$
    for $t\in(0,1)$.
    \label{lem:s-laplace}
\end{lemma}

\begin{proof}
    $$\begin{aligned}
        \nabla s(t,x)&=-\gamma^{-2}I+\text{Cov}(\nabla f_t,\nabla f_t|x_t=x).
    \end{aligned}$$
    Hence, with similar calculations and notations as in the proof of Lemma \ref{lem:v-laplace}, we can deduce that
    $$\begin{aligned}
        \Delta s(t,x)&=2\text{Cov}(\nabla f_t,\Delta f_t|x_t=x)\\
        &\qquad+\frac{1}{2}\text{Cov}[(\nabla f_t-\nabla\bar{f}_t)(\nabla f_t-\nabla\bar{f}_t)^T,\nabla f_t-\nabla\bar{f}_t|x_t=\bar{x}_t=x].\\
        &=\frac{1}{2}\text{Cov}[(\nabla f_t-\nabla\bar{f}_t)(\nabla f_t-\nabla\bar{f}_t)^T,\nabla f_t-\nabla\bar{f}_t|x_t=\bar{x}_t=x].
    \end{aligned}$$
    Then, with H\"older's inequality, we have
    $$\begin{aligned}
        \Vert\Delta s(t,x)\Vert&\lesssim\left[\mathbb{E}[\Vert\nabla f_t-\nabla\bar{f}_t\Vert^3|x_t=\bar{x}_t=x]\right]^{1/3}\\
        &\qquad\cdot\left[\mathbb{E}[\Vert(\nabla f_t-\nabla\bar{f}_t)(\nabla f_t-\nabla\bar{f}_t)^T\Vert^{3/2}]\right]^{2/3}\\
        &\lesssim\left[\mathbb{E}[\Vert\nabla f_t\Vert^3|x_t=x]\right]^{1/3}&(\text{by symmetry})\\
        &\qquad\cdot\left[\mathbb{E}[\Vert\nabla f_t-\nabla\bar{f}_t\Vert^3|x_t=\bar{x}_t=x]\right]^{2/3}\\
        &\lesssim\mathbb{E}[\Vert\nabla f_t\Vert^3|x_t=x].&(\text{by symmetry})
    \end{aligned}$$
    Hence, by Jensen's inequality, $$\mathbb{E}\Vert\Delta s(t,x)\Vert^2\lesssim\mathbb{E}[\mathbb{E}[\Vert\nabla f_t\Vert^3|x_t=x]^2]\lesssim\gamma^{-6}\mathbb{E}\Vert z\Vert^6\lesssim\gamma^{-6}d^3.$$
\end{proof}

