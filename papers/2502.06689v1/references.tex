\begin{thebibliography}{00}
\bibitem{SGTbook} Chung, Fan RK. Spectral graph theory. Vol. 92. American Mathematical Soc., 1997.
\bibitem{ChungYau1997} Chung, F.R. and Yau, S.T., 1997. Eigenvalue inequalities for graphs and convex subgraphs. Communications in Analysis and Geometry, 5(4), pp.575-623.
\bibitem{sgt1} Chung, F.R.K., Graham, R.L. and Yau, S.T., 1996. On sampling with Markov chains. Random Structures \& Algorithms, 9(1‐2), pp.55-77.
\bibitem{sgt2} Chung, F.R. and Yau, S.T., 1994. A Harnack inequality for homogeneous graphs and subgraphs. Communications in Analysis and Geometry, 2(4), pp.627-640.
\bibitem{sgt3} Chung, F.R. and Yau, S.T., 1997. Eigenvalue inequalities for graphs and convex subgraphs. Communications in Analysis and Geometry, 5(4), pp.575-623.
\bibitem{sgt4} Tan, J., 1999. Eigenvalue comparison theorems of Neumann laplacian for graphs. Interdisciplinary information sciences, 5(2), pp.157-159.
\bibitem{landmark0} Czaja, W. and Ehler, M., 2012. Schroedinger eigenmaps for the analysis of biomedical data. IEEE Transactions on Pattern Analysis and Machine Intelligence, 35(5), pp.1274-1280.
\bibitem{coifman2005geometric} Coifman, R.R. and Lafon, S., 2006. Diffusion maps. Applied and computational harmonic analysis, 21(1), pp.5-30.
\bibitem{landmark1} De Silva, V. and Tenenbaum, J.B., 2004. Sparse multidimensional scaling using landmark points (Vol. 120). technical report, Stanford University.
\bibitem{landmark2} Silva, V. and Tenenbaum, J., 2002. Global versus local methods in nonlinear dimensionality reduction. Advances in neural information processing systems, 15.
\bibitem{landmark3} Webster, M.A.R.K. and Sheets, H.D., 2010. A practical introduction to landmark-based geometric morphometrics. The paleontological society papers, 16, pp.163-188.
\bibitem{landmark4} Vladymyrov, M. and Carreira-Perpinán, M.Á., 2013. Locally linear landmarks for large-scale manifold learning. In Machine Learning and Knowledge Discovery in Databases: European Conference, ECML PKDD 2013, Prague, Czech Republic, September 23-27, 2013, Proceedings, Part III 13 (pp. 256-271). Springer Berlin Heidelberg.
\bibitem{landmark5} Thongprayoon, C. and Masuda, N., 2024. Online landmark replacement for out-of-sample dimensionality reduction methods. Proceedings of the Royal Society A, 480(2300), p.20230966.
\bibitem{landmark6} Long, A.W. and Ferguson, A.L., 2019. Landmark diffusion maps (L-dMaps): Accelerated manifold learning out-of-sample extension. Applied and Computational Harmonic Analysis, 47(1), pp.190-211.
% \bibitem{landmark7} Chi, J. and Crawford, M.M., 2014. Active landmark sampling for manifold learning based spectral unmixing. IEEE Geoscience and Remote Sensing Letters, 11(11), pp.1881-1885.
\bibitem{landmark8} Pai, G., Talmon, R., Bronstein, A. and Kimmel, R., 2019, January. Dimal: Deep isometric manifold learning using sparse geodesic sampling. In 2019 IEEE Winter Conference on Applications of Computer Vision (WACV) (pp. 819-828). IEEE.
\bibitem{kernelreweighting1} Evans, L., Cameron, M.K. and Tiwary, P., 2023. Computing committors in collective variables via Mahalanobis diffusion maps. Applied and Computational Harmonic Analysis, 64, pp.62-101.
\bibitem{kernelreweighting2} Trstanova, Z., Leimkuhler, B. and Lelièvre, T., 2020. Local and global perspectives on diffusion maps in the analysis of molecular systems. Proceedings of the Royal Society A, 476(2233), p.20190036.
\bibitem{kernelreweighting3} Sule, S., Evans, L. and Cameron, M., 2023. Sharp error estimates for target measure diffusion maps with applications to the committor problem. arXiv preprint arXiv:2312.14418.
\bibitem{kernelreweighting4} Hoyos P, Kileel J. Diffusion maps for group-invariant manifolds. arXiv preprint arXiv:2303.16169. 2023 Mar 28.

\bibitem{landmarkdmap1} Yeh, S.Y., Wu, H.T., Talmon, R. and Tsui, M.P., 2024. Landmark Alternating Diffusion. arXiv preprint arXiv:2404.19649.
\bibitem{landmarkdmap2} Shen, C. and Wu, H.T., 2022. Scalability and robustness of spectral embedding: landmark diffusion is all you need. Information and Inference: A Journal of the IMA, 11(4), pp.1527-1595.
\bibitem{landmarkdmap3} Shen, C., Lin, Y.T. and Wu, H.T., 2022. Robust and scalable manifold learning via landmark diffusion for long-term medical signal processing. Journal of Machine Learning Research, 23(86), pp.1-30.
\bibitem{landmarkdmap4} Cahill, N.D., Czaja, W. and Messinger, D.W., 2014, June. Schroedinger eigenmaps with nondiagonal potentials for spatial-spectral clustering of hyperspectral imagery. In Algorithms and technologies for multispectral, hyperspectral, and ultraspectral imagery XX (Vol. 9088, pp. 27-39). SPIE.
\bibitem{landmarkdmap5} Czaja, W., Dong, D., Jabin, P.E. and Ndjakou Njeunje, F.O., 2021. Transport model for feature extraction. SIAM Journal on Mathematics of Data Science, 3(1), pp.321-341.
% \bibitem{landmarkdmap6} Landa, B., Kluger, Y. and Ma, R., 2024. Entropic Optimal Transport Eigenmaps for Nonlinear Alignment and Joint Embedding of High-Dimensional Datasets. arXiv preprint arXiv:2407.01718.
\bibitem{datafold} Lehmberg et al., (2020). datafold: data-driven models for point clouds and time series on manifolds. Journal of Open Source Software, 5(51), 2283, https://doi.org/10.21105/joss.02283
\bibitem{cvsimportant} Bonati, L., Trizio, E., Rizzi, A. and Parrinello, M., 2023. A unified framework for machine learning collective variables for enhanced sampling simulations: mlcolvar. The Journal of Chemical Physics, 159(1).
\bibitem{cvsdynamical1} Legoll, F. and Lelievre, T., 2010. Effective dynamics using conditional expectations. Nonlinearity, 23(9), p.2131.
\bibitem{cvsdynamical2} Legoll, F., Lelievre, T. and Olla, S., 2017. Pathwise estimates for an effective dynamics. Stochastic Processes and their Applications, 127(9), pp.2841-2863.
\bibitem{cvsdynamical3} Legoll, F. and Lelievre, T., 2011, August. Some remarks on free energy and coarse-graining. In Numerical Analysis of Multiscale Computations: Proceedings of a Winter Workshop at the Banff International Research Station 2009 (pp. 279-329). Berlin, Heidelberg: Springer Berlin Heidelberg.
% \bibitem{cvsdynamical4} Zhang, W., Hartmann, C. and Schütte, C., 2016. Effective dynamics along given reaction coordinates, and reaction rate theory. Faraday discussions, 195, pp.365-394.
\bibitem{cvsrr1} Palacio-Rodriguez, K. and Pietrucci, F., 2022. Free energy landscapes, diffusion coefficients, and kinetic rates from transition paths. Journal of chemical theory and computation, 18(8), pp.4639-4648.
\bibitem{cvsrr2} Mouaffac, L., Palacio-Rodriguez, K. and Pietrucci, F., 2023. Optimal reaction coordinates and kinetic rates from the projected dynamics of transition paths. Journal of Chemical Theory and Computation, 19(17), pp.5701-5711.
\bibitem{cvssampling} Yang, Y.I., Shao, Q., Zhang, J., Yang, L. and Gao, Y.Q., 2019. Enhanced sampling in molecular dynamics. The Journal of chemical physics, 151(7).

\bibitem{transitionstate} Gao, Y.Q. and Yang, L., 2006. On the enhanced sampling over energy barriers in molecular dynamics simulations. The Journal of chemical physics, 125(11).
\bibitem{slowfast} Legoll, F., Lelièvre, T., Myerscough, K. and Samaey, G., 2020. Parareal computation of stochastic differential equations with time-scale separation: a numerical convergence study. Computing and Visualization in Science, 23, pp.1-18.
\bibitem{anisotropic} Singer, A., Erban, R., Kevrekidis, I.G. and Coifman, R.R., 2009. Detecting intrinsic slow variables in stochastic dynamical systems by anisotropic diffusion maps. Proceedings of the National Academy of Sciences, 106(38), pp.16090-16095.
\bibitem{equivariance} Batzner, S., Musaelian, A., Sun, L., Geiger, M., Mailoa, J.P., Kornbluth, M., Molinari, N., Smidt, T.E. and Kozinsky, B., 2022. E (3)-equivariant graph neural networks for data-efficient and accurate interatomic potentials. Nature communications, 13(1), p.2453.
\bibitem{cv1} Ribeiro, J.M.L., Bravo, P., Wang, Y. and Tiwary, P., 2018. Reweighted autoencoded variational Bayes for enhanced sampling (RAVE). The Journal of chemical physics, 149(7).
\bibitem{cv2} Vani, B.P., Aranganathan, A., Wang, D. and Tiwary, P., 2023. Alphafold2-rave: From sequence to boltzmann ranking. Journal of chemical theory and computation, 19(14), pp.4351-4354.
\bibitem{cv3} Wang, D., Wang, Y., Evans, L. and Tiwary, P., 2024. From latent dynamics to meaningful representations. Journal of Chemical Theory and Computation, 20(9), pp.3503-3513.
\bibitem{cv4} Belkacemi, Z., Gkeka, P., Lelièvre, T. and Stoltz, G., 2021. Chasing collective variables using autoencoders and biased trajectories. Journal of chemical theory and computation, 18(1), pp.59-78.
\bibitem{cv5} Lelièvre, T., Pigeon, T., Stoltz, G. and Zhang, W., 2024. Analyzing multimodal probability measures with autoencoders. The Journal of Physical Chemistry B, 128(11), pp.2607-2631.

\bibitem{cv5} Rogal, J., Schneider, E. and Tuckerman, M.E., 2019. Neural-network-based path collective variables for enhanced sampling of phase transformations. Physical Review Letters, 123(24), p.245701.

\bibitem{fp1} Nadler, B., Lafon, S., Coifman, R.R. and Kevrekidis, I.G., 2006. Diffusion maps, spectral clustering and reaction coordinates of dynamical systems. Applied and Computational Harmonic Analysis, 21(1), pp.113-127.
\bibitem{fp2} Coifman, R.R., Kevrekidis, I.G., Lafon, S., Maggioni, M. and Nadler, B., 2008. Diffusion maps, reduction coordinates, and low dimensional representation of stochastic systems. Multiscale Modeling \& Simulation, 7(2), pp.842-864.

\bibitem{nystrom} Czaja, W., Doster, T. and Halevy, A. "An overview of numerical acceleration techniques for nonlinear dimension reduction." Recent Applications of Harmonic Analysis to Function Spaces, Differential Equations, and Data Science: Novel Methods in Harmonic Analysis, Volume 2 (2017): 797-829.

\end{thebibliography}