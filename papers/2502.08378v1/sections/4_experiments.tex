





% \begin{figure}
%     \centering
%     \includegraphics[width=0.75\linewidth]{figures/umap.pdf}
%     \caption{Trajectory visualization.}
%     \label{fig:trajectory}
% \end{figure}

\begin{figure*}[t]
    \centering
    \includegraphics[width=0.96\textwidth]{figures/robustness.pdf}
    \vspace{-0.02in}
    \caption{\textbf{Robustness analysis in simulation}. Evaluation of control policies under four environmental disturbances demonstrates the robustness of our controllers. The poor performance of \ours-History1 indicates the importance of historical information for robustness, while \ours-Bound0.25's high energy consumption reveals limitations in motion quality under disturbance, demonstrating the effect of curriculum setup of action bound.}
    \label{fig:robustness}
    \vspace{-0.08in}
\end{figure*}

\begin{figure}[t]
    \centering
    \includegraphics[width=1\linewidth]{figures/smoothness_vs_speed.pdf}
    \caption{\textbf{Trade-off analysis in simulation}. Trade-offs between motion speed, smoothness, and energy across terrains. Results show the inverse speed-smoothness relationship, indicating the importance of constrained motion speed achieved by our method for real-world deployment.}
    \label{fig:tradeoff}
    \vspace{-0.1in}
\end{figure}

\section{Simulation Experiments}
\subsection{Experimenrt Setup} 
\subsubsection{Evaluation Metrics.} While the design of evaluation metrics for humanoid standing-up control remains an open question~\cite{subburaman2023survey}, we aim to make a step forward by proposing the following metrics: 
\begin{itemize}[leftmargin=4mm]
\vspace{-0.02in}
\item \textbf{Success rate} $E_{succ}$: The episode is considered successful if the robot's base height, $h_{\mathrm{base}}$, exceeds a target height $h_\mathrm{targ}$ and is maintained for the remainder of the episode, indicating stable standing. 
\item \textbf{Feet movement} $E_{\mathrm{feet}}$: The distance traveled by the robot's feet after reaching the target height $h_\mathrm{targ}$, indicating stability in the standing pose. 
\item \textbf{Motion smoothness} $E_{\mathrm{smth}}$: We aggregate the movement of all joint angles of consecutive control steps to measure the smoothness of the motion. It indicates that the robot should keep a smooth motion during the whole episode. 
\item \textbf{Energy} $E_{\mathrm{engy}}$: The energy consumed before reaching $h_\mathrm{targ}$, indicating the avoidance of violent standing-up motion. 
\vspace{-0.02in} 
\end{itemize}

\subsubsection{Baselines} To evaluate the effectiveness of the key design choices in \ours, we compare it against the following ablated versions:
\begin{itemize}[leftmargin=4mm]
\vspace{-0.02in}
    \item \textbf{Single critic}: A baseline using a single critic RL to assess the impact of multiple critics on motor skill learning. 
    \item \textbf{Exploration strategy}: Baselines with random noise and curiosity-based rewards (e.g., RND~\cite{burda2019exploration}) to evaluate the effectiveness of the force curriculum. 
    \item \textbf{Motion constraints}: Ablation of action bounds $\beta$ and smoothness regularization L2C2 to test their influence on motion smoothness.
    \item \textbf{Historical states}: Ablation of the number of historical states to assess their effect on standing-up motion.
\vspace{-0.02in}
\end{itemize}
 
\subsection{Main Results} 
\ours demonstrates good efficacy in learning standing-up control across all terrains, as shown in \cref{table:main_results}. The effect of key design choices is summarized as follows:

\paragraphbegin{Multiple critics are crucial for learning motor skills} Using the same reward functions, the performance of the single critic version of \ours deteriorates significantly across all terrains, achieving zero success rates. This highlights the importance of multiple critics in learning and integrating motor skills while also reducing the hyperparameter tuning burden.

\paragraphbegin{Force curriculum enhances exploration efficiency.} Without the proposed force curriculum, the robot fails to stand up on all terrains except the platform, as the other terrains require exploration from a fully fallen state to stable kneeling. While curiosity-based exploration partially alleviates this challenge, performance remains unsatisfactory. In contrast, the force curriculum greatly improves exploration efficiency.

\paragraphbegin{Action bound prevents abrupt motions.}  While the robot can learn to stand up without action bounds (\ours-w/o-Bound), its movements are excessively violent, as indicated by three performance metrics. With action bounds, \ours demonstrates smoother motions and higher success rates. Although \ours-Bound0.25 performs well, its motions are less natural due to restricted exploration during training.

\paragraphbegin{Smoothness regularization prevents motion oscillation.} Adding smoothness constraints significantly reduces motion oscillation and increases energy efficiency, validating the effectiveness of smooth regularization. Further discussion is presented in \cref{sec:real_robot_exp}.


\begin{figure*}[t]
    \centering
    \includegraphics[width=1\textwidth]{figures/real_main_snapshot.png}
    \caption{\textbf{Snapshot of real robot motion}. We directly transfer our policies from simulation to four real-world scenes that correspond to four simulation terrains. We conclude that (1) our policies can produce smooth and successful standing-up motion in all tested scenes and (2) smooth regularization of L2C2 is important to avoid oscillation and improve stability.}
    \label{fig:real_snapshot}
    \vspace{-0.05in}
\end{figure*}

\begin{figure*}[t]
    \centering
    \vspace{-0.1in}
    \includegraphics[width=1\textwidth]{figures/real_outdoor_snapshot.png}
    \caption{\textbf{Snapshot of outdoor experiments}. We test our controllers in diverse outdoor environments, demonstrating smooth motion on unseen terrains such as grassland, wooden platforms, and stone roads, as well as successful performance on stone platforms and tree-leaning postures.}
    \label{fig:outdoor_snapshot}
\end{figure*}
% \endgroup
\begingroup
\setlength{\tabcolsep}{4pt}
\begin{table*}[t]
    \centering
    \vspace{-0.1in} 
    \caption{\textbf{Main results for real robot experiments.} We report the success rate and motion smoothness to quantitatively compare our methods with the baseline. The results demonstrate the superiority of our method and the importance of adding smooth regularization into our method.} %.}
    % \vspace{0.05cm}
    
    % \resizebox{0.95\linewidth}{!}{%
 \begin{tabular}{lc c cc c cc  c cc c cc c cc} 
 \toprule
  \multirow{2}{*}{Method} & & \multicolumn{2}{c}{Ground} & & \multicolumn{2}{c}{Platform} & & \multicolumn{2}{c}{Wall} & & \multicolumn{2}{c}{Slope} & & \multicolumn{2}{c}{Overall} \\ %[0.2ex]
   \cmidrule{3-4}\cmidrule{6-7}\cmidrule{9-10} \cmidrule{12-13}  \cmidrule{15-16} 
    & 
    & $E_{\mathrm{succ}}\uparrow$ &  $E_{\mathrm{smth}}\downarrow$ 
    & 
    & $E_{\mathrm{succ}}$ $\uparrow$ & $E_{\mathrm{smth}}$ $\downarrow$ & 
    & $E_{\mathrm{succ}}$ $\uparrow$ & $E_{\mathrm{smth}}$ $\downarrow$ & 
    & $E_{\mathrm{succ}}$ $\uparrow$ &  $E_{\mathrm{smth}}$ $\downarrow$ &
    & $E_{\mathrm{succ}}$ $\uparrow$ &  $E_{\mathrm{smth}}$ $\downarrow$ &
    \\ %[0.3ex] 
 \midrule 
 \ours-w/o-L2C2 & 
 & \scalebox{1.5}{\sfrac{5}{5}} & 2.09 & 
 & \scalebox{1.5}{\sfrac{2}{5}} & 7.85 & 
 & \scalebox{1.5}{\sfrac{4}{5}} & 13.36 & 
 & \scalebox{1.5}{\sfrac{0}{5}} & 2.89 & 
 & \scalebox{1.5}{\sfrac{11}{20}} & 6.54\\ 
 \ourrow \ours (ours) & 
 & \scalebox{1.5}{\sfrac{5}{5}} & 1.83 & 
 & \scalebox{1.5}{\sfrac{5}{5}} & 5.06 & 
 & \scalebox{1.5}{\sfrac{5}{5}} & 7.22 & 
 & \scalebox{1.5}{\sfrac{5}{5}} & 1.94 & 
 & \scalebox{1.5}{\sfrac{20}{20}} & 4.01 & 
 \\
\bottomrule
\end{tabular}%}
\label{table:main_real_results}
% \vspace{-0.4cm}
\vspace{-0.05in}
\end{table*}
\endgroup

\paragraphbegin{Medium history length yields great performance.} \ours with short history length underperforms in contact-rich scenarios, such as the Wall terrain. In contrast, a longer history length improves performance, though it slightly reduces motion smoothness and increases energy consumption compared to the default setting.

\subsection{More Analyses}  
\paragraphbegin{Trajectory analysis (\cref{fig:trajectory}).} Following~\cite{haarnoja2024learning}, we apply Uniform Manifold Approximation and Projection (UMAP;~\cite{mcinnes2018umap}) to project joint-space motion trajectories into 2D, providing a visualization of the humanoid robot’s motion across diverse terrains. The resulting UMAP figure demonstrates distinct motion patterns: smooth, controlled movement on flat ground, while more complex, yet consistent, trajectories emerge on challenging terrains such as Wall. Additionally, in the 3D trajectory plots, the coordinated motion of the robot's hands and feet reveals significant posture adaptability, as the robot adjusts its stance dynamically for balance and stability. These observations highlight the harmonious whole-body coordination achieved by our controllers and validate the effectiveness of our proposed framework.

\paragraphbegin{Robustness analysis (\cref{fig:robustness})}. We comprehensively evaluate the robustness of our learned control policies by simulating various environmental disturbances. Specifically, we test four types of external perturbations: CoM position offset in the sagittal direction, consistent sagittal force,  initial joint angle offset, and random torque dropout ratio. Our results demonstrate that the policies exhibit remarkable robustness across all disturbances, achieving high success rates and efficient motion energy utilization. Notably, the poor performance of \ours-History1 underscores the critical role of historical information, which implicitly encodes contact dynamics, in maintaining robustness. Furthermore, while \ours-Bound0.25 achieves a high success rate, its elevated energy consumption highlights its limited ability to maintain motion smoothness under disturbance. These findings validate the robustness of our policies while indicating the importance of historical context and curriculum of action bound for robust standing-up.


\begin{figure*}[t]
    \centering
    % \vspace{-0.1in}
    \includegraphics[width=1\textwidth]{figures/domain_randomization.png}
    \vspace{-0.2in}
    \caption{\textbf{Sim-to-real analysis}. (a) We analyze the effect of each domain randomization term, showing that our randomization terms effectively mitigate the sim-to-real gap, with the CoM position being particularly influential. (b) To further investigate the sim-to-real gap, we compare the phases of knee and hip joints that are crucial for standing-up control. The results reveal significant discrepancies in joint velocities, suggesting a sim-to-real gap in joint torques. }
    \vspace{-0.13in}
    \label{fig:sim2real}
\end{figure*}
\begin{figure*}[t]
    \centering
    % \vspace{-0.05in}
    \includegraphics[width=1\textwidth]{figures/robustness_analysis.png}
    \caption{\textbf{Emergent properties in real robot experiments}. (a) our controllers show great robustness to the external force (3kg ball), blocking objects on the ground, and payload mass up to 12kg (2x mass of trunk. (b) Our controllers also exhibit a surprising ability to recover from very large external forces without fully falling down. (c) Our policies also exhibit the ability of dynamic balancing over a 15$^\circ$ slippery slope without falling down.}
    \label{fig:real_robustness}
    \vspace{-0.17in}
\end{figure*}

\paragraphbegin{Trade-off analysis (\cref{fig:tradeoff}).} We examine trade-offs between motion speed, smoothness, and energy consumption across terrains. On the left, motion speed and smoothness exhibit an inverse relationship: longer fall-to-standing times enhance smoothness but reduce speed, a trend consistent across all terrains. On the right, energy consumption increases with fall-to-standing time, with terrain-specific variations. For example, the Slope terrain requires higher energy for balancing. Interestingly, the Wall terrain shows a distinct trend: energy consumption rises sharply at longer fall-to-standing times despite low motion speed, suggesting greater energy intensity. This is likely due to the need for increased force or modified body mechanics to push against a vertical surface, making the motion in Wall less energy-efficient than other terrains. Overall, the results reveal a clear inverse relationship between motion speed and smoothness, indicating the importance of constrained motion speed for real-world deployment and validating the necessity of our approach to achieve such motions.



\begingroup
\setlength{\tabcolsep}{4pt}
\begin{table}[t]
    \centering
    % \vspace{0.1cm}
    \caption{\textbf{Robustness to payload and random torque dropout.} } %.}
    % \vspace{0.05cm}
    
    \resizebox{0.96\linewidth}{!}{%
 \begin{tabular}{lc c ccccc  c ccc c  } 
 \toprule
  \multirow{2}{*}{Metric} & & \multicolumn{5}{c}{Payload Mass} &  &\multicolumn{4}{c}{Torque Dropout Ratio} \\ %[0.2ex]
   \cmidrule{3-7} \cmidrule{9-12} 
    & 
    & 4kg &  6kg 
    & 8kg & 10kg 
    & 12kg  & 
    & 0.05 & 0.1 & 0.15 & 0.2
    % & $E_{\mathrm{succ}}$ $\uparrow$ &  $E_{\mathrm{smth}}$ $\downarrow$ &
    \\ %[0.3ex] 
 \midrule 
  $E_{\mathrm{smth}}\downarrow$ & 
  & 1.75 & 1.92 & 1.86 & 1.82 & 1.85 & & 2.00 & 2.16 & 2.61 & / &
 \\[0.4ex]
 $E_{\mathrm{succ}}\uparrow$ & & \scalebox{1.5}{\sfrac{3}{3}} & \scalebox{1.5}{\sfrac{3}{3}} & \scalebox{1.5}{\sfrac{3}{3}} & \scalebox{1.5}{\sfrac{3}{3}} & \scalebox{1.5}{\sfrac{2}{3}} & & \scalebox{1.5}{\sfrac{3}{3}} & \scalebox{1.5}{\sfrac{3}{3}} & \scalebox{1.5}{\sfrac{3}{3}} & \scalebox{1.5}{\sfrac{0}{3}} &
 \\
\bottomrule
\end{tabular}}
\label{table:payload_torque}
\vspace{-0.07in}
\end{table}
\endgroup





% \begin{table*}
%     \centering
%     % \vspace{-0.17in}
%     \caption{\label{table:main_results}\textbf{Key design choices of RPM.} }  

%     % \vspace{-0.1in}
%     \begin{tabular}{lc} 
%         \toprule[0.4mm]
%         Ablated Components & Success Rate (\%)  \\
%         \midrule
%         \ourrow \textbf{Training Objective}& \\
%         [0.3mm]\cdashline{1-2}\noalign{\vskip 0.6mm}
%         wo. objective $\mathcal{L}_{\mathrm{dyn}}$ & 66.2\ci{0.8}   \\
%         w/o. objective $\mathcal{L}_{\mathrm{act}}$ &71.3\ci{1.2} \\
%         w/o. objective $\mathcal{L}_{\mathrm{tcl}}$  & 72.0\ci{1.2}  \\        
%             \midrule
%         \ourrow \textbf{Dynamic Chunk} &\\
%         [0.3mm]\cdashline{1-2}\noalign{\vskip 0.6mm}
%         Length $l$:  3$\rightarrow$1  & 72.1\ci{2.9}\\
%         Length $l$: 3$\rightarrow$5  & 76.8\ci{2.4} \\
%         Length $l$: 3$\rightarrow$7  & 76.8\ci{2.2} \\
%         \midrule
%         \ourrow \textbf{Encoder Backbone} &\\
%         [0.3mm]\cdashline{1-2}\noalign{\vskip 0.6mm}
%         ResNet-: 50$\rightarrow$18  & 77.3\ci{1.8} \\
%         ResNet-: 50$\rightarrow$34  & 77.9\ci{1.7}  \\
%         \midrule
%         \ourrow \ours~(original) &\textbf{83.2}\ci{1.3}   \\
%         \bottomrule
%      \end{tabular}
%    % \vspace{-3in}
% \end{table*}




\section{Real Robot Experiments}\label{sec:real_robot_exp}

\subsection{Main Results}
We evaluate our method in both laboratory and outdoor environments corresponding to simulation terrains, using \ours-w/o-L2C2 as the baseline to examine the effect of smoothness regularization during deployment.

\paragraphbegin{Smooth regularization improves motions (\cref{fig:real_snapshot}).} Motion oscillations are observed in all scenes without smoothness regularization, often leading to standing-up failures. In contrast, our method produces smooth and stable motions, especially on 10.5$^\circ$ slope. Quantitative results in \cref{table:main_real_results} strengthen this conclusion, with our approach achieving a 100\% success rate and high motion smoothness across all scenes.\footnote{We select the successful episode to compute smoothness to reflect the effect of L2C2 regularization better. Due to the unavailability of the height, we compute the smoothness $E_{\mathrm{smth}}$ within two seconds after starting up. }

\begin{figure}[t]
    \centering
    % \vspace{-0.1in}
    \includegraphics[width=1\linewidth]{figures/standing_stability.png}
    \caption{\textbf{Standing stability.} Our control policies demonstrate great stability against external disturbances after successful standing up.}
    \vspace{-0.1in}
    \label{fig:standing_stability}
\end{figure}

\paragraphbegin{Generalization to outdoor environments (\cref{fig:outdoor_snapshot}).} We evaluate our learned controllers in a variety of outdoor environments, testing their ability to generalize to terrains not encountered during training. On flat ground, the controllers produce stable, smooth motions across grassland, wooden platforms, and stone roads. Notably, these terrains were not included in the training simulations. Additionally, our controllers successfully handle more complex scenarios, including stone platforms and tree-leaning postures, demonstrating their adaptability to diverse real-world conditions.


\subsection{Sim-to-real Analysis}
In this analysis, we investigate the effect of various domain randomization terms on the sim-to-real gap, as shown in \cref{fig:sim2real}. Our results demonstrate that the introduction of these randomization terms significantly reduces the sim-to-real gap, particularly with respect to the Center of Mass (CoM) position.

\paragraphbegin{Phase plot.} To further investigate the sources of this gap, we examine the phase plots of the knee and hip roll joints. These joints are considered most important for standing-up motions. We observe a notable discrepancy between simulated and real-world joint velocities, suggesting a gap in joint torques. This highlights the need for more accurate actuator modeling to bridge the sim-to-real gap in humanoid standing-up tasks, which is also suggested by previous work on quadrupedal robots~\cite{hwangbo2019learning}. Despite this, our controllers remain effective in handling these discrepancies, exhibiting joint paths consistent with the simulated ones.


\subsection{Emergent Properties} 
% Training with diverse postures and suitable domain randomization, our control policies exhibit strong robustness in the following aspects, as demonstrated in \cref{fig:real_robustness}.


\paragraphbegin{Robustness to external disturbance (\cref{fig:real_robustness}a).} The robustness of our control policies was tested through experiments involving external disturbances, such as a 3 kg ball impact and obstructive objects. The controllers maintained stability even under significant disturbances, like objects disrupting the robot's center of gravity. Additionally, the controllers managed payloads up to 12kg, twice the mass of the humanoid robot's trunk. We also quantitatively verify the great robustness of payload and torque dropout ratio in \cref{table:payload_torque}.  

\paragraphbegin{Fall recovery (\cref{fig:real_robustness}b).} Our controllers also exhibited strong resilience in recovering from large external forces without fully falling down. This capability is vital for humanoid robots navigating unpredictable real-world scenarios with sudden impacts or balance shifts. Testing showed that, even under abrupt perturbations, the robots regained their upright posture, demonstrating the effectiveness of our control strategies in maintaining dynamic stability.

\paragraphbegin{Dynamic balance (\cref{fig:real_robustness}c).} We further tested our controllers on a 15$^\circ$ slippery slope, simulating challenging real-world conditions such as unstable surfaces. The controllers not only maintained stability on the incline but also adjusted posture and center of mass in real time to counteract the slippery conditions. These results highlight the adaptability and stability of our controllers, ensuring humanoid robots can operate safely on diverse and unpredictable terrains.

\paragraphbegin{Standing stability (\cref{fig:standing_stability}).} Our controllers demonstrate strong standing stability, effectively resisting external disturbances after successful standing up. This stability is beneficial for integrating our controllers into existing control systems.
