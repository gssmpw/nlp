\section{Related Work}
In this section, we provide a brief overview of recent work related to 4D radar(-inertial) odometry. Radar can acquire information about the azimuth, elevation angle, radial Doppler velocity, and distance of the object being measured.
 % Of these, Doppler velocity is a significant advantage of radar point clouds over laser point clouds. 
 4D radar odometry algorithms are primarily based on Doppler velocity, point cloud matching, or a combination of them.

Doppler velocity measurements can provide precise radial velocity, and many algorithms utilize this information for state estimation. Doer et al.\cite{DoerMFI2020} use a 3-Point RANSAC Least Squares approach for ego velocity estimation. It requires a single radar scan only making use of the direction and Doppler velocity of each detected object. After this, they proposed EKF-RIO\cite{DoerENC2020}\cite{DoerJGN2022}, based on the EKF framework, fuses IMU with the velocities calculated by radar to determine odometry. Kubelka et al.\cite{kubelka2023we,yoon2023need,wu2022picking,ng2021continuous} also solely utilized Doppler and IMU data for fusion, yielding results that surpassed those achieved with point cloud matching. Hexsel at al. proposed DICP algorithm\cite{hexsel2022dicp}, it derived an observation equation based on Doppler velocity. This equation is solely related to the measured speed of the point and is independent of the spatial structure of the surrounding environment. This helps to address the issue of constraint degradation that ICP faces in unstructured scenarios.
% Additionally, the work \cite{nissov2024degradation} \cite{kramer2020radar}\cite{huang2024multi}\cite{hong2021radar}also utilized radar Doppler velocity to address the degradation problem of other sensors.
Additionally, the work \cite{nissov2024degradation,kramer2020radar,huang2024multi,hong2021radar} also utilized radar Doppler velocity to address the degradation problem of other sensors.

The RIO system based on Doppler velocity will inevitably produce cumulative errors in position and heading. Many works use point cloud matching to reduce cumulative errors. Zhang et al. proposed 4DRadarSLAM\cite{zhang20234dradarslam}, which employs ADPGICP for scan-to-scan matching to perform trajectory calculations. In the backend, ScanContext\cite{kim2018scan} is used for loop closure detection. Once a loop closure is detected, pose optimization is carried out.
4D iRIOM\cite{zhuang20234d} fusion IMU and radar information based on the filtering framework. It uses the speed information solved by a single frame and IMU for loose coupling fusion, and then updates the point cloud observation in a distribution-to-multidistribution manner. Michalczyk et al.\cite{michalczyk2022tightly} \cite{michalczyk2023multi} use ESKF to tightly couple fuse IMU, Doppler velocity observation and radar point-to-point matching observation. Huang et al.\cite{huang2024less} use IMU as a prior to eliminate dynamic points of radar, use RCS bounded data association for point-to-point matching, and use a factor gragh\cite{loeliger2004introduction} tightly-coupled  fuse  Doppler velocity residuals and scan matching residuals. Xu et al.\cite{xu2024modeling} derived the uncertainties of Doppler and point cloud observations from the sensor's measurement model. They then used these uncertainties for data association and fusion, thereby enhancing the accuracy of RIO.
Besides, there are also some learning-based approaches \cite{lu2020milliego,10160681,yin2021rall} that employ radar for pose estimation.

Doppler velocity constraints can effectively constrain local changes in state, but cannot avoid cumulative drift. Scan matching can reduce cumulative drift, but the noise and sparsity of radar point clouds make matching extremely difficult. In principle, a combination of them can achieve better results.
% In addition,owing to the impact of physical factors related to radar, the point cloud characteristics that radar displays vary significantly in different scenarios, such as ground scenes versus aerial drone scenes. 
% Moreover, in drone scenarios, it's challenging for radar measurements to generate repeated feature point observations between consecutive frames, making point-to-point  matching almost impossible. To address the issue of sparse and noisy point cloud matching in drone scenarios, we project multiple nearby frames to the same moment based on the local pose constrained by Doppler, remove outliers , and then perform point-to-distribution matching with the local map. Our experiments show that the method proposed in this paper can provide accurate and robust pose estimation in drone scenarios.
However, the radar point cloud characteristics  varies significantly across different scenarios. Radar measurements  in the UAV flight scenario are sparser and noisier compared to those on the ground.
% Additionally, the intense motion in drone scenarios makes it challenging to generate repeated feature point observations between adjacent frames of radar measurements. Consequently, it's almost impossible to use the point-to-point method for point cloud matching constraints in such scenarios.
Very few radar odometry algorithms are specifically designed for UAV navigation.
To address the issue of sparse and noisy point cloud matching constraints in UAV flight scenarios, we particularly design the RIO system to integrate Doppler observations and scan matching observations in order to achieve better  accuracy.
% Experiments have proven that the method proposed in this paper can achieve accurate and robust pose estimation in UAV flight scenarios.