%ol is an important topic, but  often need sufficient data to train well; when the data is scarce
%when simulation is complex 
\begin{abstract}
Neural operators have shown great potential in surrogate modeling. However, training a well-performing neural operator typically requires a substantial amount of data, which can pose a major challenge in complex applications. In such scenarios, detailed physical knowledge can be unavailable or difficult to obtain, and collecting extensive data is often prohibitively expensive. To mitigate this challenge, we propose the Pseudo Physics-Informed Neural Operator (PPI-NO) framework. PPI-NO  constructs a surrogate physics system for the target system using partial differential equations (PDEs) derived from simple, rudimentary physics principles, such as basic differential operators. 
This surrogate system is coupled with a neural operator model, using an alternating update and learning process to iteratively enhance the model's predictive power.
While the physics derived via PPI-NO may not mirror the ground-truth underlying physical laws --- hence the term ``pseudo physics'' ---  this approach significantly improves the accuracy of standard operator learning models in data-scarce scenarios, which is evidenced by extensive evaluations across five benchmark tasks and a fatigue modeling application. % demonstrate that PPI-NO consistently outperforms standard neural operator models, achieving substantial improvements in prediction accuracy. 
  %competing methods by a significant margin. The success of PPI-NO may introduce a new paradigm in physics-informed machine learning, one that requires minimal physics knowledge and opens the door to broader applications in data-driven physics learning and  simulations.


%Recent advancements in operator learning are transforming the landscape of computational physics and engineering, especially alongside  the rapidly evolving field of physics-informed machine learning. The convergence of these areas offers exciting opportunities for innovative research and applications. However, merging these two realms  often demands deep expertise and explicit knowledge of physical systems, which may be challenging or even impractical in relatively complex applications.
%To address this limitation, we propose a novel framework: Pseudo Physics-Informed Neural Operator (PPI-NO). In this framework, we construct a surrogate physics system for the target system using partial differential equations (PDEs) derived from simple, rudimentary physics knowledge, such as basic differential operators. We then couple the surrogate system with the neural operator model, utilizing  an alternating update and learning process to iteratively enhance the model's predictive power. While the physics derived via PPI-NO may not mirror the ground-truth underlying physical laws --- hence the term ``pseudo physics'' ---  this approach significantly enhances the accuracy of current operator learning models, particularly in data scarce scenarios. Through extensive evaluations across five benchmark operator learning tasks and an application in fatigue modeling, PPI-NO consistently outperforms competing methods by a significant margin. The success of PPI-NO may introduce a new paradigm in physics-informed machine learning, one that requires minimal physics knowledge and opens the door to broader applications in data-driven physics learning and  simulations.
\end{abstract}