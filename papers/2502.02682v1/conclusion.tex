\section{Conclusion}
We have presented a  Pseudo Physics-Informed Neural Operator (PPI-NO) learning framework. PPI-NO is based on our observation that a PDE system is often characterized by a \textit{local} combination of the solution and its derivatives. This property makes it feasible to learn an effective representation of the PDE system, even with limited data.
 %the derivation of many training points from the function sampling locations, facilitating learning of the PDE systems through a neural network. %The learned black-box PDE is subsequently utilized to guide the learning of neural operators. 
While the physics delineated by PPI-NO might not precisely reflect true physical phenomena, our findings reveal that this method significantly enhances the efficiency of operator learning with limited data quantity. %In the future, we plan to develop more powerful neural networks that can learn a wider range of PDEs, and apply our method in more applications.

%In the future, we have two plans. Firstly, we plan to develop effective co-learning approaches that mutually refine the ``pseudo'' physics laws and enhance operator learning.  Our current experience shows that jointly or alternatively updating the neural operators and the $\phi$ network can deteriorate the performance on both sides.
%Secondly, we plan to learn a more general representation of the PDE system, which couples the state derivatives and the input function together:
% \[
% \phi\left(\mathbf{x}, u(\mathbf{x}), S_1(u)(\mathbf{x}), \ldots, S_Q(u), f(\mathbf{x})\right) = 0.
% \]
% This approach would allow us to extend the usage of the learned physics laws, such as in facilitating the learning of the mapping between the PDE parameter functions (\eg the conductivity in Darcy flow) and the solutions.


\section{Impact Statements}
This paper presents work whose goal is to advance machine learning for operator learning. There are many potential societal consequences of our work, none of which we feel must be specifically highlighted here.