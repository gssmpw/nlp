%

\documentclass{article}

%
\usepackage{microtype}
\usepackage{graphicx}
%
\usepackage{booktabs} %

%
%
%
%
\usepackage{hyperref}

\usepackage[utf8]{inputenc}
\usepackage[ruled,linesnumbered]{algorithm2e}
\usepackage[noend]{algpseudocode}
\usepackage{amsmath,amsthm,amsfonts,amssymb,mathtools}
\usepackage{color}
\usepackage{mathrsfs}
\usepackage{enumitem}
\usepackage{multirow}
\usepackage{makecell}
\usepackage{caption}
\usepackage{thmtools}
\usepackage{thm-restate}
\usepackage{hhline}
\usepackage{cite}
\usepackage[table]{xcolor}
\usepackage{rotating} 
\usepackage{natbib}
\usepackage{bm}
\usepackage{diagbox}
\usepackage{cancel}
\usepackage{subcaption}

%
\newcommand{\theHalgorithm}{\arabic{algorithm}}

%
\usepackage[accepted]{icml2025}

%
%

%
\usepackage{amsmath}
\usepackage{amssymb}
\usepackage{mathtools}
\usepackage{amsthm}

%
\usepackage[capitalize,noabbrev]{cleveref}

%
%
%
\theoremstyle{plain}
\newtheorem{theorem}{Theorem}[section]
\newtheorem{proposition}[theorem]{Proposition}
\newtheorem{lemma}[theorem]{Lemma}
\newtheorem{corollary}[theorem]{Corollary}
\theoremstyle{definition}
\newtheorem{definition}[theorem]{Definition}
\newtheorem{assumption}[theorem]{Assumption}
\theoremstyle{remark}
\newtheorem{remark}[theorem]{Remark}

%
%
%
\usepackage[textsize=tiny]{todonotes}


%
%
\icmltitlerunning{Can RLHF be More Efficient with Imperfect Reward Models?  A Policy Coverage Perspective}

\section{Summary of Mathematical Notations}
\label{sec:notations}

We summarize the main mathematical notations used in the main paper in Table \ref{table:notations}.

\begin{table*}[h]
	\centering
	\caption{Summary of main mathematical notations.}
	%\resizebox{1\columnwidth}{!}{
		\begin{tabular}{c|l}
			\toprule
			Notation  & Description  \\
			\hline
            $ \mathcal{G} $ & a document graph \\
			$ \mathcal{D} $ & a corpus of documents, $ \mathcal{D}=\{d_i\}_{i=1}^{N} $ \\
			$ N $ & number of documents in the corpus, $ N=|\mathcal{D}| $ \\
			$ d_i $ & document $ i $ containing a sequence of words, $ d_i=\{w_{i,v}\}_{v=1}^{|d_i|}\subset\mathcal{V} $ \\
			$ \mathcal{V} $ & vocabulary \\
			$ |d_i| $ & number of words in document $ i $ \\
			$ \mathcal{E} $ & a set of graph edges connecting documents, $ \mathcal{E}=\{e_{ij}\} $ \\
			$ \mathcal{N}(i) $ & the neighbor set of document $ i $ \\
            $ \Bbb H^{n,K} $ & Hyperboloid model with dimension $ n $ and curvature $ -1/K $ \\
			$ \mathcal{T}_{\textbf{x}}\Bbb H^{n,K} $ & tangent (Euclidean) space around hyperbolic vector $ x\in\Bbb H^{n,K} $ \\
			$ \exp_{\textbf{x}}^K(\textbf{v}) $ & exponential map, projecting tangent vector $ \textbf{v} $ to hyperbolic space \\
			$ \log_{\textbf{x}}^K(\textbf{y}) $ & logarithmic map, projecting hyperbolic vector $ \textbf{y} $ to $ \textbf{x} $'s tangent space \\
			$ d_{\mathcal{L}}^K(\textbf{x},\textbf{y}) $ & hyperbolic distance between hyperbolic vectors $ \textbf{x} $ and $ \textbf{y} $ \\
			$ \text{PT}_{\textbf{x}\rightarrow\textbf{y}}^K(\textbf{v}) $ & parallel transport, transporting $ \textbf{v} $ from $ \textbf{x} $'s tangent space to $ \textbf{y} $'s \\
			$ H $ & length of a path on topic tree \\
            $ \sigma(t,i) $ & similarity between topic $ t $ and document $ i $ \\
            $ \bm{\pi}_i $ & path distribution of document $ i $ over topic tree \\
            $ \textbf{z}_{t,p} $ & hyperbolic ancestral hidden state of topic $ t $ \\
            $ \textbf{z}_{t,s} $ & hyperbolic fraternal hidden state of topic $ t $ \\
			$ \textbf{z}_t $ & hyperbolic hidden state of topic $ t $ \\
            $ \sigma(h,i) $ & similarity between topic $ t $ and document $ i $ \\
            $ \textbf{z}_h $ & hyperbolic hidden state of level $ h $ \\
			$ \bm{\delta}_i $ & level distribution of document $ i $ over topic tree \\
            $ \bm{\theta}_i $ & topic distribution of document $ i $ over topic tree \\
            $ \textbf{e}_i $ & hierarchical tree embedding of document $ i $ \\
            $ T $ & number of topics on topic tree \\
            $ \textbf{g}_i $ & hierarchical graph embedding of document $ i $ \\
			$ \textbf{U} $ & a matrix of word embeddings, $ \textbf{U}\in\Bbb R^{|\mathcal{V}|\times(n+1)} $ \\
			$ \bm{\beta} $ & topic-word distribution $ \bm{\beta}\in\Bbb R^{T\times |\mathcal{V}|} $ \\
			\bottomrule
		\end{tabular}
	%}
	%\vspace{-0.2cm}
	\label{table:notations}
\end{table*}

\begin{document}

\twocolumn[
\icmltitle{Can RLHF be More Efficient with Imperfect Reward Models? \\ A Policy Coverage Perspective}

%
%
%
%

%
%
%
%

%
%
%
%

\begin{icmlauthorlist}
\icmlauthor{Jiawei Huang}{ethz}
\icmlauthor{Bingcong Li}{ethz}
\icmlauthor{Christoph Dann}{google}
\icmlauthor{Niao He}{ethz}
%
%
%
%
%
%
%
%
\end{icmlauthorlist}

\icmlaffiliation{ethz}{Department of Computer Science, ETH Zurich}
\icmlaffiliation{google}{Google Research}
%

\icmlcorrespondingauthor{Jiawei Huang}{jiawei.huang@inf.ethz.ch}
%

%
%
%
\icmlkeywords{Machine Learning, ICML}

\vskip 0.3in
]

%

%
%
%
%
%

\printAffiliationsAndNotice{}  %
%




\allowdisplaybreaks

\begin{abstract}
    Sample efficiency is critical for online Reinforcement Learning from Human Feedback (RLHF). While existing works investigate sample-efficient online exploration strategies, the potential of utilizing misspecified yet relevant reward models to accelerate learning remains underexplored. This paper studies how to transfer knowledge from those imperfect reward models in online RLHF. We start by identifying a novel property of the KL-regularized RLHF objective: \emph{a policy's coverability of the optimal policy is captured by its sub-optimality}. Building on this insight, we propose novel transfer learning principles and a theoretical algorithm with provable benefits compared to standard online learning. Our approach achieves low regret in the early stage by quickly adapting to the best available source reward models without prior knowledge of their quality, and over time, it attains an $\tilde{O}(\sqrt{T})$ regret bound \emph{independent} of structural complexity measures. Empirically, inspired by our theoretical findings, we develop a win-rate-based transfer policy selection method with improved computational efficiency. Moreover, our empirical transfer learning technique is modular and can be integrated with various policy optimization methods, such as DPO, IPO and XPO, to further enhance their performance. We validate the effectiveness of our method through experiments on summarization tasks.

    %
    %
    %
    %
    %
    %
    %
\end{abstract}

\section{Introduction}

Large language models (LLMs) have achieved remarkable success in automated math problem solving, particularly through code-generation capabilities integrated with proof assistants~\citep{lean,isabelle,POT,autoformalization,MATH}. Although LLMs excel at generating solution steps and correct answers in algebra and calculus~\citep{math_solving}, their unimodal nature limits performance in plane geometry, where solution depends on both diagram and text~\citep{math_solving}. 

Specialized vision-language models (VLMs) have accordingly been developed for plane geometry problem solving (PGPS)~\citep{geoqa,unigeo,intergps,pgps,GOLD,LANS,geox}. Yet, it remains unclear whether these models genuinely leverage diagrams or rely almost exclusively on textual features. This ambiguity arises because existing PGPS datasets typically embed sufficient geometric details within problem statements, potentially making the vision encoder unnecessary~\citep{GOLD}. \cref{fig:pgps_examples} illustrates example questions from GeoQA and PGPS9K, where solutions can be derived without referencing the diagrams.

\begin{figure}
    \centering
    \begin{subfigure}[t]{.49\linewidth}
        \centering
        \includegraphics[width=\linewidth]{latex/figures/images/geoqa_example.pdf}
        \caption{GeoQA}
        \label{fig:geoqa_example}
    \end{subfigure}
    \begin{subfigure}[t]{.48\linewidth}
        \centering
        \includegraphics[width=\linewidth]{latex/figures/images/pgps_example.pdf}
        \caption{PGPS9K}
        \label{fig:pgps9k_example}
    \end{subfigure}
    \caption{
    Examples of diagram-caption pairs and their solution steps written in formal languages from GeoQA and PGPS9k datasets. In the problem description, the visual geometric premises and numerical variables are highlighted in green and red, respectively. A significant difference in the style of the diagram and formal language can be observable. %, along with the differences in formal languages supported by the corresponding datasets.
    \label{fig:pgps_examples}
    }
\end{figure}



We propose a new benchmark created via a synthetic data engine, which systematically evaluates the ability of VLM vision encoders to recognize geometric premises. Our empirical findings reveal that previously suggested self-supervised learning (SSL) approaches, e.g., vector quantized variataional auto-encoder (VQ-VAE)~\citep{unimath} and masked auto-encoder (MAE)~\citep{scagps,geox}, and widely adopted encoders, e.g., OpenCLIP~\citep{clip} and DinoV2~\citep{dinov2}, struggle to detect geometric features such as perpendicularity and degrees. 

To this end, we propose \geoclip{}, a model pre-trained on a large corpus of synthetic diagram–caption pairs. By varying diagram styles (e.g., color, font size, resolution, line width), \geoclip{} learns robust geometric representations and outperforms prior SSL-based methods on our benchmark. Building on \geoclip{}, we introduce a few-shot domain adaptation technique that efficiently transfers the recognition ability to real-world diagrams. We further combine this domain-adapted GeoCLIP with an LLM, forming a domain-agnostic VLM for solving PGPS tasks in MathVerse~\citep{mathverse}. 
%To accommodate diverse diagram styles and solution formats, we unify the solution program languages across multiple PGPS datasets, ensuring comprehensive evaluation. 

In our experiments on MathVerse~\citep{mathverse}, which encompasses diverse plane geometry tasks and diagram styles, our VLM with a domain-adapted \geoclip{} consistently outperforms both task-specific PGPS models and generalist VLMs. 
% In particular, it achieves higher accuracy on tasks requiring geometric-feature recognition, even when critical numerical measurements are moved from text to diagrams. 
Ablation studies confirm the effectiveness of our domain adaptation strategy, showing improvements in optical character recognition (OCR)-based tasks and robust diagram embeddings across different styles. 
% By unifying the solution program languages of existing datasets and incorporating OCR capability, we enable a single VLM, named \geovlm{}, to handle a broad class of plane geometry problems.

% Contributions
We summarize the contributions as follows:
We propose a novel benchmark for systematically assessing how well vision encoders recognize geometric premises in plane geometry diagrams~(\cref{sec:visual_feature}); We introduce \geoclip{}, a vision encoder capable of accurately detecting visual geometric premises~(\cref{sec:geoclip}), and a few-shot domain adaptation technique that efficiently transfers this capability across different diagram styles (\cref{sec:domain_adaptation});
We show that our VLM, incorporating domain-adapted GeoCLIP, surpasses existing specialized PGPS VLMs and generalist VLMs on the MathVerse benchmark~(\cref{sec:experiments}) and effectively interprets diverse diagram styles~(\cref{sec:abl}).

\iffalse
\begin{itemize}
    \item We propose a novel benchmark for systematically assessing how well vision encoders recognize geometric premises, e.g., perpendicularity and angle measures, in plane geometry diagrams.
	\item We introduce \geoclip{}, a vision encoder capable of accurately detecting visual geometric premises, and a few-shot domain adaptation technique that efficiently transfers this capability across different diagram styles.
	\item We show that our final VLM, incorporating GeoCLIP-DA, effectively interprets diverse diagram styles and achieves state-of-the-art performance on the MathVerse benchmark, surpassing existing specialized PGPS models and generalist VLM models.
\end{itemize}
\fi

\iffalse

Large language models (LLMs) have made significant strides in automated math word problem solving. In particular, their code-generation capabilities combined with proof assistants~\citep{lean,isabelle} help minimize computational errors~\citep{POT}, improve solution precision~\citep{autoformalization}, and offer rigorous feedback and evaluation~\citep{MATH}. Although LLMs excel in generating solution steps and correct answers for algebra and calculus~\citep{math_solving}, their uni-modal nature limits performance in domains like plane geometry, where both diagrams and text are vital.

Plane geometry problem solving (PGPS) tasks typically include diagrams and textual descriptions, requiring solvers to interpret premises from both sources. To facilitate automated solutions for these problems, several studies have introduced formal languages tailored for plane geometry to represent solution steps as a program with training datasets composed of diagrams, textual descriptions, and solution programs~\citep{geoqa,unigeo,intergps,pgps}. Building on these datasets, a number of PGPS specialized vision-language models (VLMs) have been developed so far~\citep{GOLD, LANS, geox}.

Most existing VLMs, however, fail to use diagrams when solving geometry problems. Well-known PGPS datasets such as GeoQA~\citep{geoqa}, UniGeo~\citep{unigeo}, and PGPS9K~\citep{pgps}, can be solved without accessing diagrams, as their problem descriptions often contain all geometric information. \cref{fig:pgps_examples} shows an example from GeoQA and PGPS9K datasets, where one can deduce the solution steps without knowing the diagrams. 
As a result, models trained on these datasets rely almost exclusively on textual information, leaving the vision encoder under-utilized~\citep{GOLD}. 
Consequently, the VLMs trained on these datasets cannot solve the plane geometry problem when necessary geometric properties or relations are excluded from the problem statement.

Some studies seek to enhance the recognition of geometric premises from a diagram by directly predicting the premises from the diagram~\citep{GOLD, intergps} or as an auxiliary task for vision encoders~\citep{geoqa,geoqa-plus}. However, these approaches remain highly domain-specific because the labels for training are difficult to obtain, thus limiting generalization across different domains. While self-supervised learning (SSL) methods that depend exclusively on geometric diagrams, e.g., vector quantized variational auto-encoder (VQ-VAE)~\citep{unimath} and masked auto-encoder (MAE)~\citep{scagps,geox}, have also been explored, the effectiveness of the SSL approaches on recognizing geometric features has not been thoroughly investigated.

We introduce a benchmark constructed with a synthetic data engine to evaluate the effectiveness of SSL approaches in recognizing geometric premises from diagrams. Our empirical results with the proposed benchmark show that the vision encoders trained with SSL methods fail to capture visual \geofeat{}s such as perpendicularity between two lines and angle measure.
Furthermore, we find that the pre-trained vision encoders often used in general-purpose VLMs, e.g., OpenCLIP~\citep{clip} and DinoV2~\citep{dinov2}, fail to recognize geometric premises from diagrams.

To improve the vision encoder for PGPS, we propose \geoclip{}, a model trained with a massive amount of diagram-caption pairs.
Since the amount of diagram-caption pairs in existing benchmarks is often limited, we develop a plane diagram generator that can randomly sample plane geometry problems with the help of existing proof assistant~\citep{alphageometry}.
To make \geoclip{} robust against different styles, we vary the visual properties of diagrams, such as color, font size, resolution, and line width.
We show that \geoclip{} performs better than the other SSL approaches and commonly used vision encoders on the newly proposed benchmark.

Another major challenge in PGPS is developing a domain-agnostic VLM capable of handling multiple PGPS benchmarks. As shown in \cref{fig:pgps_examples}, the main difficulties arise from variations in diagram styles. 
To address the issue, we propose a few-shot domain adaptation technique for \geoclip{} which transfers its visual \geofeat{} perception from the synthetic diagrams to the real-world diagrams efficiently. 

We study the efficacy of the domain adapted \geoclip{} on PGPS when equipped with the language model. To be specific, we compare the VLM with the previous PGPS models on MathVerse~\citep{mathverse}, which is designed to evaluate both the PGPS and visual \geofeat{} perception performance on various domains.
While previous PGPS models are inapplicable to certain types of MathVerse problems, we modify the prediction target and unify the solution program languages of the existing PGPS training data to make our VLM applicable to all types of MathVerse problems.
Results on MathVerse demonstrate that our VLM more effectively integrates diagrammatic information and remains robust under conditions of various diagram styles.

\begin{itemize}
    \item We propose a benchmark to measure the visual \geofeat{} recognition performance of different vision encoders.
    % \item \sh{We introduce geometric CLIP (\geoclip{} and train the VLM equipped with \geoclip{} to predict both solution steps and the numerical measurements of the problem.}
    \item We introduce \geoclip{}, a vision encoder which can accurately recognize visual \geofeat{}s and a few-shot domain adaptation technique which can transfer such ability to different domains efficiently. 
    % \item \sh{We develop our final PGPS model, \geovlm{}, by adapting \geoclip{} to different domains and training with unified languages of solution program data.}
    % We develop a domain-agnostic VLM, namely \geovlm{}, by applying a simple yet effective domain adaptation method to \geoclip{} and training on the refined training data.
    \item We demonstrate our VLM equipped with GeoCLIP-DA effectively interprets diverse diagram styles, achieving superior performance on MathVerse compared to the existing PGPS models.
\end{itemize}

\fi 


\section{Preliminaries}
\label{Preliminaries}
\subsection{Multi-Agent Reinforcement Learning}
A MARL problem can be formulated as a decentralized partially observed Markov decision process (Dec-POMDP)~\cite{oliehoek2016concise}, which is described as a tuple $\langle n,\boldsymbol{S},\boldsymbol{A},P,R,\boldsymbol{O},\boldsymbol{\Omega},\gamma\rangle $, where $n$ represents the number of agents, $\boldsymbol{S}$ is the global state space. $\boldsymbol{A}$ is the action space. $\boldsymbol{O}=\{O_{i}\}_{i=1,\cdots,n}$ is the observation space. At timestep $t$, each agent $i$ receives an observation $o_{i}^t\in O_{i}$ according to the observation function $\boldsymbol{\Omega}(s^t,i):\boldsymbol{S}\to O_i$ and then selects an action $a_i^t\in\boldsymbol{A}$. The joint action $\boldsymbol{a}^t=(a_1^t,\ldots,a_n^t)$ is then applied to the environment, resulting in a transition to the next state $s^{t+1}$ and a global reward signal $r^{t}$ according to the transition function $P(s^{t+1}\mid s^{t},\boldsymbol{a}^t)$ and the reward function $R(s^t,\boldsymbol{a}^t)$. $\gamma\in[0,1]$ is the discount factor. The objective is to learn a joint policy $\pi$ that maximizes the expected cumulative reward $\mathbb{E}\left[\sum_{t=0}^{\infty}\gamma^{t}r^{t}\right|\pi]$.

\subsection{Centralized Training With Decentralized Execution}
Centralized Training with Decentralized Execution (CTDE) is a commonly employed architecture in MARL~\cite{lowe2017multi,rashid2020monotonic}. In CTDE, each agent utilizes an actor network to make decisions based on local observations. Additionally, the training process incorporates global information to train a centralized value function. The centralized value function provides a centralized gradient to update the actor network based on its outputs.

\subsection{Generalizable Model Structure in MARL}
To handle varying state/observation/action spaces, previous works like UPDeT~\cite{hu2021updet} and ASN~\cite{wang2019action} propose a generalizable model that treats all agents as entities. In such models, observation $o_i$ can be conducted as entity-observations: $[o_{i,1},o_{i,2},...,o_{i,m}]$, where $m$ denotes the number of all entities in the environment. Based on the criterion of whether entites can be observed, entity-observations can be splited into two subsets: observed entity-observations $o_{\mathrm{obs},i}$ and unobserved entity-observations $o_{\mathrm{mask},i}$. We denote the number of observed entities and masked entities as $n_{\mathrm{obs}}$ and $n_{\mathrm{mask}}$, respectively, and it holds that $m=n_{\mathrm{obs}}+n_{\mathrm{mask}}$.
Additionally, action space $\boldsymbol{A}$ can be decomposed into two subsets:$\boldsymbol{A}^{\mathrm{self}}$ containing actions that affect the environment or itself  and $\boldsymbol{A}^{\mathrm{out}}$ representing actions that directly interact with other entities.

%
%
\section{The Blessing of Regularization: A Policy Coverage Perspective}\label{sec:transfer_coverage_perspective}
Unlike classical pure-reward maximization RL, the RLHF objective in \eqref{eq:rlhf_obj} incorporates regularization with respect to $\pi_\textref$.
We start by identifying distinctive properties associated with such regularization in Sec.~\ref{sec:new_structure}, and discuss their implications on transfer learning in RLHF in Sec.~\ref{sec:new_insights}.
%
%
%
%
%

\subsection{Structural Property Induced by Regularization}\label{sec:new_structure}
%
%
%
%
%
%
%
\begin{restatable}{lemma}{LemCovValGap}\label{lem:coverage_and_value_gap}
    Under Assump.~\ref{assump:policy} and assume $r^w\in[0,R]$ for all $w\in[W]$, then, for any policy $\pi\in\conv(\Pi)\cup\{\pi^*_{r^w}\}_{w=1}^W$,
    %
    %
    %
    \begin{align}
        \cov^{\pi^*_{r^*}|{\pi}} \leq 1 + \kappa(e^{\frac{2{\Rmax}}{\beta}}) \cdot \frac{J_\beta(\pi^*_{r^*}) - J_\beta({\pi})}{\beta},\label{eq:cov_value_gap}
    \end{align}
    where $\kappa(x) := \frac{(x-1)^2}{x-1- \log x} = O(x)$.
\end{restatable}
The key insight of Lem.~\ref{lem:coverage_and_value_gap} is that: for any prospective candidates of the optimal policy (i.e., $\conv(\Pi)$\footnote{Here we consider the convex hull in order to incorporate all possible uniform mixture policies induced by $\Pi$.}) or any transfer candidates (i.e., $\{\pi^*_{r^w}\}_{w=1}^W$), \emph{its coverability of $\pi^*_{r^*}$ is controlled by its policy value gap}.
Intuitively, $\cov^{\tpi|\pi}$ becomes extremely large or even unbounded if there is a significant distribution shift between $\pi$ and $\tpi$.
However, in the presence of regularization ($\beta > 0$), we should only consider policies with bounded policy ratio relative to $\pi_\textref$ (see Assump.~\ref{assump:policy}-(II)), and exclude those (near-)deterministic ones from our policy candidate class $\Pi$, because none of them can be (near-)optimal.
%
In other words, regularization leverages prior knowledge from $\pi_\textref$ and enables a free policy filtration step before learning begins, ensuring that the remaining policies exhibit a favorable structure (Lem.~\ref{lem:coverage_and_value_gap}).

%
%
%

To understand why such property is uniquely arising from regularization, consider a bandit instance with a single optimal arm and multiple suboptimal arms yields rewards $\Rmax$ and $\Rmax - 2\epsilon$, respectively.
In pure reward maximization RL ($\beta = 0$), the optimal policy $\pi^*_{r^*}$ is deterministic.
A policy class $\Pi$ satisfying Assump.~\ref{assump:policy} may include several suboptimal deterministic policies.
The coverage coefficient between any of them and $\pi^*_{r^*}$ is infinity, while their suboptimal gaps are $2\epsilon$ and can be arbitrarily small.

%
%
%
%
%


\subsection{New Insights for Transfer Learning in RLHF}\label{sec:new_insights}
In the online RLHF, the primary goal of exploration is to discover high-reward regions, i.e., the states and actions covered by the optimal policy.
Therefore, in our reward transfer setup, we propose to \textbf{\emph{transfer from the policy with the best coverage of $\pi^*_{r^*}$}}.
Inspired by Lem.~\ref{lem:coverage_and_value_gap}, we identify two novel principles for transfer learning for RLHF objective, which we will further explore in later sections.

\textbf{Principle 1: Select Transfer Policies with High Policy Value}~By Lem.~\ref{lem:coverage_and_value_gap}, exploiting a policy with high value for data collection could ``help'' exploration, because such a policy inherently provides good coverage for $\pi^*_{r^*}$.
In other words, regularization reconciles the trade-off between exploration and exploitation.
This insight allows us to use policy value as a criterion and transfer from the policy achieving the highest value among all candidates.
This strategy is also practical given that policy values can be estimated well.

We emphasize that this principle is unique in the regularized setting.
As exemplified by the bandit instance before, near-optimality does not imply good coverage for $\pi^*_{r^*}$ in the absence of regularization.
To avoid negative transfer in pure reward maximization setting, previous algorithms typically rely on additional assumptions about task similarity and employ sophisticated strategies to balance exploiting good source tasks with exploration~\citep{golowich2022can,huang2023robust}, which can be challenging to generalize beyond the tabular setting.
In contrast, regularization enables us to filter transfer policies directly with their policy value, facilitating the applicablity beyond the tabular setup.


\textbf{Principle 2: Transfer from the Policy Distilled from Online Data---the ``Self-Transfer Learning''}~
We first introduce a key result by combining Lem.~\ref{lem:coverage_and_value_gap} and offline RLHF result in Lem.~\ref{lem:offline_RLHF}.
\begin{restatable}{theorem}{ThmOnlineOffline}\label{thm:general_val_gap}
    Under Assump.~\ref{assump:policy}, w.p. $1-\delta$, given an online dataset $\cD$ generated\footnote{See Cond.~\ref{cond:seq_data} for the definition of data generation process.} by a policy series $\{\pi^t\}_{t=1}^T \in \conv(\Pi)$, running $\RPO$ with $\conv(\Pi)$ and $\cR^\Pi$ on $\cD$ yields a distilled policy $\pi_\SELF$, such that,
    \begin{align*}
        & J_\beta(\pi^*_{r^*}) - J_\beta(\pi_\SELF) \leq \numberthis\label{eq:offline_policy_covergence} \\
        &\tilde{O}\Big(e^{2{\Rmax}} \Big(1 + \kappa(e^{\frac{2{\Rmax}}{\beta}})  \sum_{t=1}^T\frac{J_\beta(\pi^*_{r^*}) - J_\beta(\pi^t)}{\beta T}\Big)\sqrt{\frac{1}{T}}\Big).
        %
    \end{align*}
\end{restatable}
To understand the significance of Thm.~\ref{thm:general_val_gap}, consider the case when $\{\pi^t\}_{t\in[T]}$ are produced by a no-regret online learning algorithm, such as $\XPO$ in Lem.~\ref{lem:online_RLHF}.
As a result, the term $\sum_{t=1}^T\frac{J_\beta(\pi^*_{r^*}) - J_\beta(\pi^t)}{\beta T}$ in Eq.~\eqref{eq:offline_policy_covergence} diminishes to 0 as $T$ increases. This implies that the policy $\pi_\SELF$ distilled from online data by offline learning techniques converges to $\pi^*_{r^*}$ at a rate of $O(T^{-\frac{1}{2}})$, which \textbf{\emph{does not depend on}} $|\cS|,|\cA|$ or other complexity measures such as $\cov_{\infty}(\Pi)$ in Lem.~\ref{lem:online_RLHF}.
This result not only strictly improves the sample complexity bounds\footnote{Beyond sample complexity, a regret bound improved to $\tilde{O}(\sqrt{T})$ for online RLHF can be established. We defer it to Coro.~\ref{coro:sqrtT_reg} after presenting our main results.} for existing online RLHF algorithms \citep{xiong2024iterative, xie2024exploratory,cen2024value,zhang2024self}, but also reveals a fundamental difference from the pure reward maximization setting, where lower bounds depending on those structural complexity factors have been established \citep{auer2002nonstochastic,dani2008stochastic}.
We defer detailed discussions to Appx.~\ref{appx:proof_offline_policy_gap}.

More importantly, the faster rate of the convergence of $\pi_\SELF$ to $\pi^*_{r^*}$ also indicates the potential of using $\pi_\SELF$ as a candidate for policy transfer.
%
We term this regime as ``\emph{self-transfer learning}'', and refer $\pi_\SELF$ as the ``\emph{self-transfer policy}''.
Notably, $\pi_\SELF$ continuously improves and converges to $\pi^*_{r^*}$ as the dataset grows, while the source policies $\{\pi^*_{r^w}\}_{w=1}^W$ retain fixed non-zero value gaps due to the imperfections in reward models $\{r^w\}_{w=1}^W$.
This reveals another benefit of self-transfer learning: it helps to avoid being restricted by suboptimal source reward models.

%
%


%
%
%
%
%

%
%
%
%
%
%

%
%
%
%
%

%
%


%
%
%
%

%

%

%


%
%
%

%
%

%
%



%
%
%
%

%

\iffalse
Moreover, Thm.~\ref{thm:general_val_gap} also suggests one may regard the offline policy as another candidates used for transfer, which we call the ``\emph{\textbf{self-transfer learning}}''.
More concretely, suppose at step $t$, the accumulative regret by the previous policies $\pi^1,...,\pi^t$ is low, e.g. $\sum_{i=1}^t J_\beta(\pi^*_{r^*}) - J_\beta(\pi^i) = O(\sqrt{t})$. 
By running an offline algorithm (RPO) on the collected dataset, Thm.~\ref{thm:general_val_gap} suggests we will get a good policy $\pi^t_\SELF$ with value gap $\tilde{O}((1 + e^{\frac{{\Rmax}}{\beta}}\sqrt{\frac{1}{\beta t}})\cdot \sqrt{\frac{1}{t}}) \approx \tilde{O}(\frac{1}{\sqrt{t}})$.
When $t$ is relatively large, $\pi^t_\SELF$ would be a better policy to transfer than $\pi^*_{r^w}$ since it has diminishing policy value gap.
Moreover, the convergence rate of $\pi^t_\SELF$ does not have the coefficient depending on $\Complexity(\Pi)$, which suggests its possible superiority over the policies for exploration-based online learning.


\blue{
    Intuition about the structure: the regularization smooths the optimal policy, and therefore, near-optimal policy implies a good coverage.
}

Next, we interpret what Thm.~\ref{thm:general_val_gap} reveals for the main focus of this paper: the reward transfer setting.
As we can see, the coefficient term $1 + e^{\frac{{\Rmax}}{\beta}} \sqrt{\sum_{t=1}^T\frac{J_\beta(\pi^*_{r^*}) - J_\beta(\pi^t)}{2\beta T}}$ is directly related to the accumulative regret of the policy sequence $\pi^1,...,\pi^T$.
This suggests the goal of the reward transfer should be reducing the accumulative regret.
If we have multiple source reward models $\{r^w\}_{w=1}^W$, it also suggests we should consider the one whose corresponding optimal policy achieves the highest regularized policy value $w^* \gets \arg\max_{w\in[W]} J_\beta(\pi^*_{r^w})$.
\fi

%
%


\iffalse
\discussion{
    \paragraph{Comparison with Pure Online Learning from Preferences}
    Suppose we directly run XPO for $T$ steps, which results in a policy
    \begin{align}
        J(\pi^*_{r^*}) - J(\hpi^*_{r^*}) \leq \EE_{s\sim\rho}[\blue{\frac{1}{T} \cdot \XPOReg(T)}] = \tilde{O}(\EE_{s\sim\rho}[\blue{\sqrt{\frac{\cov^\Pi}{T}}}]). \label{eq:no_transfer}
    \end{align}
    Comparing with Eq.~\eqref{eq:transfer} and Eq.~\eqref{eq:no_transfer}, the benefits of the transfer is reflected by, for those transferable states, the gap is reduced from $\sqrt{\cov^\Pi}$ to $O(\exp(\frac{\xi}{\beta}))$ (suppose $W$ and $\alpha$ are constants).
    In the worst case, $\cov^\Pi = \Theta(e^{\frac{{\Rmax}}{\beta}})$, which can be potentially much larger than $O(\exp(\frac{\xi}{\beta}))$.

    Intuitively, for those $\xi$-transferable states, because some of $\pi^*_{r^w}$ already provide good coverage for $\pi^*_{r^*}$ than those exploration policies $\pi_{\mix}$. By separating some sample budgets for $\pi^*_{r^w}$ can results in better performance on those states.

    On the other hand, for those non-transferable states, $\pi^*_{r^w}$ are useless. Therefore, in the worst case, if $\cS^\xi_\transfer = \emptyset$, Eq.~\eqref{eq:transfer} will pay $\frac{1}{\alpha}$ factor because of the waste of samples on $\pi^*_{r^w}$.


    %
    %

    %
    %
    %

    \paragraph{Without regularization, bounded reward gap does not imply policy coverage between optimal policies}
    Consider a Bernoulli bandit setting with $A$ arms, where the true mean reward $r^*$ is unknown.
    Suppose we have a reward model $r$ predicting the mean reward $r(a_1) = 0.5$ and $r(a_i) = 0.5 - \xi$ for all arm $i > 1$.
    In the pure reward maximization without regularization (i.e. $\beta = 0$), Eq.~\eqref{eq:universal_gap} does not rule out the probability of being the optimal action for any $a\in\cA$, and therefore, it does not imply bounded coverage between $\pi^*_{r^*}$ and $\pi^*_{\hat r}$.

    %

}
\fi

\section{Provably Efficient Transfer Learning}\label{sec:main_theory}
In this section, we develop provably efficient transfer learning algorithms based on the principles in Sec.~\ref{sec:new_insights}.

\textbf{Outline of Main Algorithm}~
Our main algorithm $\TPO$, short for \textbf{T}ransfer \textbf{P}olicy \textbf{O}ptimization, is provided in Alg.~\ref{alg:main_algorithm}, which leverages Alg.~\ref{alg:transfer_policy_computing} ($\TPS$, short for \textbf{T}ransfer \textbf{P}olicy \textbf{S}election) as a subroutine to select source policies to transfer from.
$\TPO$ can be regarded as a mixture of standard online learning and transfer learning, balanced through a hyper-parameter $\alpha \in (0, 1)$.
Motivated by the implication of Thm.~\ref{thm:general_val_gap}, $\TPO$ returns the offline policy computed with all the data collected.
For convenience, we divide the total number of iterations $T$ into $K=T/N$ blocks, each containing $N$ sub-iterations.
In each block, we first run $\alpha N$ iterations of an \textbf{{O}}n\textbf{{L}}ine learning algorithm $\AlgOnline$, followed by $(1-\alpha)N$ iterations of transfer learning with policy selected by Alg.~\ref{alg:transfer_policy_computing}.
Here, $\AlgOnline$ can be any online algorithm with per-step no-regret guarantees, for example, $\XPO$ in Lem.~\ref{lem:online_RLHF}.
To save space, we defer to Appx.~\ref{appx:online_oracle} the formal behavior assumption on $\AlgOnline$ (Def.~\ref{def:online_oracle}) and concrete examples with verifications.

Now we are ready to take a closer look at the transfer policy selection steps in Alg.~\ref{alg:transfer_policy_computing} in Sec.~\ref{sec:alg_explanation}, after which, the provable merits of proposed approach are showcased in Sec.~\ref{sec:alg_main_results}.
\begin{algorithm*}[t]
    \textbf{Input}: Block size $N$; Number of blocks $K = T / N$; $\{r^w\}_{w=1}^W$; $\Pi$; $\alpha \in (0,1)$; $\delta\in(0,1)$ \\
    For all $(k,n)$, 
    $\cD^{k,n}$ denotes all the data collected up to $(k,n)$, and $\cD^{k,n}_\Online$ only includes those collected by $\AlgOnline$. See detailed definitions in Appx.~\ref{appx:main_alg_details}.\\
    \For{$k=1,2...,K$}{
        %
        \For{$n=1,...,N$}{
            \lIf{$n \leq \alpha N$}{
                $\pi^{k,n} \gets \AlgOnline(\alpha T,\Pi,\delta;\cD^{k,n}_\Online)$\label{line:online_learning}
            }
            \lElse{
                ~$\pi^{k,n} \gets \TPS(T,\Pi,\delta,\{r^w\}_{w=1}^W;\cD^{k,n})$\label{line:transfer_learning}
            }
            %
            Collect data $(s^{k,n}, a^{k,n},\ta^{k,n}, y^{k,n}) \sim \rho\times\pi^{k,n}\times\pi_\textref\times\mP_{r^*}(\cdot|\cdot,\cdot,\cdot)$. 
        }
        %
        %
        %
        %
    }
    \Return $\hat{\pi}^*_{r^*}$ computed by $\RPO$ with $\cD^{K,N+1}$.
    \caption{\textbf{T}ransfer \textbf{P}olicy \textbf{O}ptimization (\TPO)}\label{alg:main_algorithm}
\end{algorithm*}

%
%


\subsection{Details for Alg.~\ref{alg:transfer_policy_computing}: The Transfer Policy Selection}\label{sec:alg_explanation}
%
%
%
%
%

%
%

%
%
%

%

%
%
%
%
%
%
%
%
%
%
%
%
%
%
%
%
%

%

%
The design of Alg.~\ref{alg:transfer_policy_computing} follows the two principles in Sec.~\ref{sec:new_insights}, which are: 
%
%
%
%
(1) transfer the policy with the highest (estimated) policy value, because higher policy value implies better coverage for $\pi^*_{r^*}$;
(2) include the self-transfer policy as a candidate, because it progressively converges to $\pi^*_{r^*}$ at a faster rate than the best-known ones for online policies.

Before diving into details, we first clarify some notation.
Given a dataset $\cD:=\{(s^i,a^i,\ta^i,y^i,\pi^i)\}_{i=1}^{|\cD|}$, $L_{\cD}(r)$ denotes the average negative log-likelihood (NLL) loss regarding the reward model $r$:
\begin{align*}
        L_{\cD}(r) := & \frac{1}{|\cD|}\sum_{i \leq |\cD|} -y^i \log \sigma\Big(r(s^i,a^i) - r(s^i,\ta^i)\Big) \\
        & - (1 - y^i) \log \sigma\Big(r(s^i,\ta^i) - r(s^i,a^i)\Big). \numberthis\label{eq:def_likelihood}
    %
    %
    %
\end{align*}
We will use $\EE_{\rho, \pi}[r] := \EE_{s\sim\rho, a\sim\pi}[r(s,a)]$ as a short note.
In Line~\ref{line:counter_N} of Alg.~\ref{alg:transfer_policy_computing}, $N(w;\cD) := \sum_{i\leq|\cD|} \mI[\pi^i = \pi^*_{r^w}]$ denotes the number samples collected with $\pi^*_{r^w}$ in the dataset, following the convention that $1/N(\cdot,\cdot) = +\infty$ if $N(\cdot,\cdot)=0$.
In Line~\ref{line:RPO}, we leverage $\RPO$ \citep{liu2024provably} to compute the self-transfer policy $\pi_\SELF$ and a reward function $\hr_\SELF$.
To save space, we defer the details of $\RPO$ to Appx.~\ref{appx:adaption_offline}.
%
%
%
%

\begin{algorithm*}[t]
    \textbf{Input}: Source tasks $\{r^w\}_{w=1}^W$; Policy class $\Pi$; $T$, $\delta$; Dataset $\cD:=\{(s^i,a^i,\ta^i,y^i,\pi^i)\}_{i\leq|\cD|}$. \\
    // \blue{Optimistic estimation for $J_\beta(\pi^*_{r^w}) - J_\beta(\pi_\textref)$.} \\
    $\hr_\MLE \gets \argmin_{r\in\cR^\Pi} L_{\cD}(r)$. \label{line:MLE}\\
    $\forall w\in [W],~\hat{V}(\pi^*_{r^w};\cD) \gets \EE_{\rho,\pi^*_{r^w}}[\hr_\MLE] - \EE_{\rho,\pi_\textref}[\hr_\MLE] - \beta \KL(\pi^*_{r^w}\|\pi_\textref) + 16 e^{2{\Rmax}} \sqrt{\frac{1}{N(w;\cD)} \log\frac{|\Pi|WT}{\delta}}.$ \label{line:counter_N}\\
    %
    // \blue{Pessimistic estimation for $J_\beta(\pi_\SELF) - J_\beta(\pi_\textref)$.} \\
    $\pi_\SELF, \hr_\SELF \gets \RPO(\conv(\Pi),\cR^{\Pi},\cD,\eta)$ with $\eta = c\cdot (1+e^{{\Rmax}})^{-2} \sqrt{\frac{1}{|\cD|}\log\frac{|\Pi|T}{\delta}}.$
    %
    \label{line:RPO}\\
    $\hV(\pi_\SELF;\cD) \gets \EE_{\rho,\pi_\SELF}[\hr_\SELF] - \EE_{\rho,\pi_\textref}[\hr_\SELF] - \beta \KL(\pi_\SELF\|\pi_\textref) + \frac{1}{\eta} L_{\cD}(\hr_\SELF) - \frac{1}{\eta} L_{\cD}(\hr_\MLE) - 2c e^{2{\Rmax}} \sqrt{\frac{1}{|\cD|}\log\frac{|\Pi|T}{\delta}}.$ \label{line:LB_pi_value}\\
    %
    \Return $\argmax_{\pi \in \{\pi^*_{r^w}\}_{w\in[W]} \cup \{\pi_\SELF\}} \hat{V}(\pi;\cD) $. // \blue{Selecting Transfer Policy by Estimated Value}
    \caption{\textbf{T}ransfer \textbf{P}olicy \textbf{S}election (\TPS)}\label{alg:transfer_policy_computing}
    %
\end{algorithm*}

Next, we explain our value estimation strategy.
Note that in RLHF setting, we cannot access $r^*$ directly but only the preference comparison samples following the BT model.
Thus, we instead estimate the value gain relative to $J_\beta(\pi_\textref)$.

\textbf{Optimistic Estimation for} $J_\beta(\pi^*_{r^w}) - J_\beta(\pi_\textref)$~
For policies induced by imperfect source reward models, we adopt UCB-style optimistic policy evaluation to efficiently balance exploration and exploitation.
Intuitively, by utilizing the MLE reward estimator $\hr_\MLE$, the estimation error $\hr_\MLE - r^*$ under the distribution of $\pi^*_{r^w}$ is related to the number of samples from $\pi^*_{r^w}$ occurring in the dataset. Therefore, we can quantify the value estimation error as follows.
%
\begin{restatable}[Value Est Error for $\{\pi^*_{r^w}\}_{w\in[W]}$]{lemma}{LemOptismValErr}\label{lem:formal_optism_val_est_error}
    Under Assump.~\ref{assump:policy} and Def.~\ref{def:online_oracle}, w.p. $1-\delta$, in each call of Alg.~\ref{alg:transfer_policy_computing}:
    \begin{align*}
        \forall w\in[W],\quad & J_\beta(\pi^*_{r^w}) - J_\beta(\pi_\textref) \leq \hat{V}(\pi^*_{r^w};\cD) \\
        \leq & J_\beta(\pi^*_{r^w}) - J_\beta(\pi_\textref) + \tilde{O}(\frac{e^{2{\Rmax}}}{\sqrt{N(w;\cD)}}).
    \end{align*}
\end{restatable}

%
%
%
\textbf{Pessimistic Estimation for} $J_\beta(\pi_\SELF)-J_\beta(\pi_\textref)$~
The main challenge in estimating the value of $\pi_\SELF$ is that, $\pi_\SELF$ is not fixed but changing and improving.
The previous optimistic strategy is not applicable here, since the coverage of $\pi_\SELF$ by the dataset is unclear, making it difficult to quantify the uncertainty in estimation via count-based bonus term.
Fortunately, given that $\pi_\SELF$ is improving over time, it is more important when it surpasses all the other source policies. Therefore, it suffices to construct a tight lower bound for $J_\beta(\pi_\SELF)-J_\beta(\pi_\textref)$; see line~\ref{line:LB_pi_value}. 
By leveraging $\hat{r}_\MLE$ and the optimality of $(\pi_\SELF, \hr_\SELF)$ for the $\RPO$ loss, we can show:
%
%
%
%
%
%
%
%
%

%
%
%

\begin{restatable}[Value Est Error for $\pi_\SELF$]{lemma}{LemSelfTransErr}\label{lem:formal_val_est_error}
    Under Assump.~\ref{assump:policy} and Def.~\ref{def:online_oracle}, w.p. $1-\delta$, in each call of Alg.~\ref{alg:transfer_policy_computing}:
    %
    %
    %
    %
    %
    \begin{align*}
        &J_\beta(\pi^*_{r^*}) - J_\beta(\pi_\textref) - \tilde{O}\Big(\frac{{\Rmax} e^{2{\Rmax}}}{\sqrt{|\cD|}} \cdot (\cov^{\pi^*_{r^*}|\pi_\mix^{\cD}} \wedge \frac{\sqrt{\Complexity(\Pi)}}{\alpha})\Big)\\
        & \leq \hat{V}(\pi_\SELF; \cD) \leq J_\beta(\pi_\SELF) - J_\beta(\pi_\textref) \numberthis\label{eq:offline_est_err}.
    \end{align*}
\end{restatable}
Here $\pi_\mix^{\cD}:=\frac{1}{|\cD|}\sum_{i\leq|\cD|}\pi^i$ denotes the mixture policy.
In the LHS, the coefficient takes minimum over two factors $\cov^{\pi^*_{r^*}|\pi_\mix^{\cD}}$ and $\sqrt{\cC(\Pi)}/\alpha$, resulting from two different ways to estimate the value gap of $\pi_\SELF$.
According to offline RLHF theory (see Lem.~\ref{lem:offline_learning}), $\pi_\SELF$ is competitive with any $\pi \in \conv(\Pi)$ well-covered by the dataset distribution, or equivalently, $J_\beta(\pi^*_{r^*}) - J_\beta(\pi_\SELF) = J_\beta(\pi^*_{r^*}) - J_\beta(\pi) + \tilde{O}(|\cD|^{-\frac{1}{2}}\cov^{\pi|\pi_\mix^{\cD}})$.
By choosing $\pi = \pi^*_{r^*}$, we obtain the first bound with the factor $\cov^{\pi^*_{r^*}|\pi_\mix^D}$.
Next, considering $\pi = \frac{1}{\alpha kN}\sum_{i\leq k, j\leq \alpha N}\pi^{i,j}$, the uniform mixture of policies generated by $\AlgOnline$ so far, leads to the second bound involving $\sqrt{\cC(\Pi)}/\alpha$.
This also explains why we still involve normal online learning in $\TPO$---to provide another safeguard for the quality of the transfer policy.
%

%
%

%

%
%
%
%
%

%
%
%
%



\subsection{Main Results and Interpretation}\label{sec:alg_main_results}


We establish the per-step regret bound for $\TPO$ below.
%
\begin{restatable}[Total Regret]{theorem}{ThmMainReg}\label{thm:regret_guarantees}
    Suppose $\AlgOnline$ is a no-regret instance satisfying Def.~\ref{def:online_oracle}, whose regret grows as $\tilde{O}(\Rmax e^{2\Rmax} \sqrt{\cC(\Pi)\tilde{T}})$ for any intermediate step $\tilde{T}$ and some policy class complexity measure $\cC(\Pi)$.
    Then, w.p. $1-2\delta$, for any $T/K \leq t \leq T$, running $\TPO$ yields a regret bound:
    %
    %
    %
    %
    %
    %
    %
    %
    %
    %
    \begin{align*}
        \sum_{\tau \leq t} J_\beta(\pi^*_{r^*}) -&  J_\beta(\pi^{k(\tau),n(\tau)}) = \Regret_\Online^{(t)} + \Regret_\Transfer^{(t)}\\
        \Regret_\Online^{(t)} :=& \tilde{O}({\Rmax} e^{2{\Rmax}} \sqrt{\alpha\Complexity(\Pi)t}),\numberthis\label{eq:transfer_regret_bound_1}\\
        \Regret_\Transfer^{(t)} :=& \tilde{O}\Big(\sum_{\substack{\tau\leq t: \alpha N < n(\tau) \leq N}} \Delta_{\min} \wedge \iota^{k(\tau),n(\tau)}\numberthis\label{eq:transfer_regret_bound_2}\\
         &+ e^{2{\Rmax}} \sqrt{(1-\alpha)Wt} \wedge  \sum_{w:\Delta(w) > 0} \frac{e^{4{\Rmax}}}{\Delta(w)} \Big).
         %
    \end{align*}
    Here we denote $k(\tau) := \lceil \frac{\tau}{N} \rceil$ and $n(\tau) := \tau~\text{mod}~N$ to be the block index and inner iteration index for step $\tau$;
    $\iota^{k(\tau),n(\tau)} := \tilde{O}({\Rmax} e^{2{\Rmax}} \Big(\cov^{\pi^*_{r^*}|\pi_\mix^{\tau}} \wedge \frac{\sqrt{\Complexity(\Pi)}}{\alpha}\Big) \sqrt{\frac{1}{\tau}})$, where $\pi_\mix^{\tau} := \frac{1}{\tau}\sum_{i\leq \tau} \pi^{k(i),n(i)}$ is the mixture policy up to $\tau$; $\Delta(w)$ and $\Delta_{\min}$ denote value gaps as defined in Sec.~\ref{sec:transfer_setting}.
    %
    %
    %
    %
    %
    %
\end{restatable}
We decompose the total regret into two parts depending on their origins. $\Regret_\Online^{(t)}$ comes from the regret by running the online algorithm $\AlgOnline$. It is weighted by $\alpha$ since we only allocate $\alpha$-proportion of the samples for $\AlgOnline$.
$\Regret_\Transfer^{(t)}$ represents the regret from transfer policies.
The first term in Eq.~\eqref{eq:transfer_regret_bound_2} reflects the benefits of utilizing transfer policy over online learning.
Here $\Delta_{\min}$ is contributed by source reward models $\{r^w\}_{w\in[W]}$, and the term $\iota^{k(\tau),n(\tau)}$ is due to the ``self-transfer policy'' $\pi_\SELF$, as we derived in Lem.~\ref{lem:formal_val_est_error}.
The second term in Eq.~\eqref{eq:transfer_regret_bound_2} results from the imperfection of source reward models: without prior knowledge on their quality, additional cost has to be paid during exploration.

Next, we elaborate the benefits of transfer learning by taking a closer look at $\Regret_\Transfer^{(t)}$ in Eq.~\eqref{eq:transfer_regret_bound_2}.
Note that the lower $\Regret_\Transfer^{(t)}$ is, the faster $\hat{\pi}^*_{r^*}$ in $\TPO$ converges to $\pi^*_{r^*}$.
When $\Delta_{\min} = 0$, i.e. $r^*$ is realizable in $\{r^w\}_{w\in[W]}$, we have $\Regret_\Transfer^{(t)} = \tilde{O}(\sqrt{Wt} \wedge \sum_{w:\Delta(w)>0} \frac{1}{\Delta(w)})$ and the benefit of transfer learning is clear.
Thus, in the following, we only focus on the case $\Delta_{\min} > 0$. We separately consider two scenarios, according to the relationship between $t$ and $\Delta_{\min}$. For clarity, we will omit the constant terms ${\Rmax}$ and $e^{{\Rmax}}$.
%
%
%
%
%
%
%

%
\textbf{Stage 1: $t<\frac{W^2}{\Delta^2_{\min}}$}~
%
%
This corresponds to the early learning stage, when $t$ is relatively small.
In this case, Thm.~\ref{thm:regret_guarantees} implies the following regret bound:
%
%
%
%
\begin{align*}
    \Regret_\Transfer^{(t)} = \tilde{O}(\sqrt{1-\alpha} (\sqrt{Wt} + \Delta_{\min}t)) = \tilde{O}(W\sqrt{t}),\numberthis\label{eq:case_1}
\end{align*}
%
which can be further improved to $\tilde{O}(\sqrt{Wt})$ if $t < \frac{W}{\Delta_{\min}^2}$.
This suggests at the earlier stage, the benefits of transfer is contributed mostly by the source reward models $\{r^w\}_{w\in[W]}$.
In general, we can expect the number of source tasks $W$ much lower than the policy class complexity measure $\Complexity(\Pi)$.
Therefore, Eq.~\eqref{eq:case_1} implies a significant improvement over the typical online learning regret bound without transfer.
%
%
%

%
\textbf{Stage 2: $\Tt\geq \frac{W^2}{\Delta^2_{\min}}$}~
In this case, the second term in Eq.~\eqref{eq:transfer_regret_bound_2} is controlled by $O(\sum_{w\in[W]}\frac{1}{\Delta(w)}) = O(\frac{W}{\Delta_{\min}}) = O(\sqrt{\Tt})$, and we have the following regret bound:
%
%
%
%
%
%
%
%
%
%
%
%
\begin{align}
    \textstyle \Regret_\Transfer^{(t)} = \tilde{O}\Big(\sqrt{\frac{\Complexity(\Pi)\Tt}{\alpha^2} \wedge \sum_{\tau\leq\Tt} (\cov^{\pi^*_{r^*}|\pi_\mix^{\tau}})^2}\Big).\numberthis\label{eq:case_2}
\end{align}
%
%
At the first glance, the RHS is controlled by $\tilde{O}(\sqrt{\Complexity(\Pi)\Tt}/\alpha)$, which implies transfer learning at most suffer a factor of $1/\alpha$ larger regrets than no transfer.
However, in fact, the term $\sqrt{\sum_{\tau\leq\Tt} (\cov^{\pi^*_{r^*}|\pi_\mix^{\tau}})^2}$ yields a much tighter bound, which only grows as $\tilde{O}(\sqrt{\Tt})$ after finite time, and is independent of $\Complexity(\Pi)$.
%
To see this, by Lem.~\ref{lem:coverage_and_value_gap} and the concavity of $J_\beta(\cdot)$, we have $\cov^{\pi^*_{r^*}|\pi_\mix^{t}} = 1 + \tilde{O}({\kappa(e^{\frac{2{\Rmax}}{\beta}}) }\cdot \frac{\Regret_\Online^{(t)} + \Regret_\Transfer^{(t)}}{\beta t})$.
Note that Eq.~\eqref{eq:case_1} and~\eqref{eq:case_2} already indicate a regret upper bound $\Regret_\Transfer^{(t)}=\tilde{O}(\Coeff\sqrt{t})$, where $\Coeff$ is a short note of a coefficient depending on $\alpha,~W,~\{\Delta(w)\}_{w\in[W]}$ and $\Complexity(\Pi)$, but not $t$.
%
%
%
%
%
This implies $\cov^{\pi^*_{r^*}|\pi_\mix^{t}}$ converges to 1 at the rate of $O(1/\sqrt{t})$, and $\Regret_\Transfer^{(t)} = \tilde{O}(\sqrt{t})$ after finite time.
%
%
%

Although the above provable benefits in Stage 2 result primarily from ``self-transfer learning'', high-quality source reward models also play an important role here.
According to Eq.~\eqref{eq:case_1}, small $\Delta_{\min}$ can lead to small $\Coeff$ and therefore, accelerate the convergence of $\cov^{\pi^*_{r^*}|\pi_\mix^{t}}$ towards $1$.
%
%
%
%

%
%
%



%
\begin{algorithm*}[t]
    \textbf{Input}: $K$, $N$ and $\{r^w\}_{w\in[W]}$; \\
    %
    %
    %

    For all $(k,n) \in [K]\times[N]$, and all $w\in[W]$,
    $\cD^{k,n} := \cup_{j=1}^{n-1}\{(s^{k,j}, a^{k,j},\ta^{k,j}, y^{k,j})\}$, $N^{k,n}(\cdot) := \sum_{j<n}\mathbb{I}[\cdot = \pi^{k,j}]$, and $\hat{\mP}_{r^*}^{k,n}(\cdot \succ \pi^k_\base) := \frac{1}{N^{k,n}(\cdot)} \sum_{j<n}\mathbb{I}[\cdot = \pi^{k,j}]y^{k,j}$. \\
    Initialize $\pi^1_\base \gets \pi_\textref$; \\
    \For{$k=1,2...,K$}{
        \For{$n=1,... N$}{
            %
            %
            %
            %
                %
            %
            $\forall w\in[W],~\hat{\text{WR}}^{\pi^*_{r^w}} \gets \hat{\mP}_{r^*}^{k,n}(\pi^*_{r^w} \succ \pi^k_\base) + c \sqrt{\frac{1}{N^{k,n}(\pi^*_{r^w})}\log\frac{1}{\delta}}$; \\
            $\hat{\text{WR}}^{\pi^k_\base} \gets \mP_{r^*}(\pi^k_\base\succ\pi^k_\base) = 0.5.$ \blue{// $\hat{\text{WR}}^{\pi^k_\base}$ can be treated as a hyperparameter taking value other than 0.5.} \\
            $\pi^{k,n} \gets \argmax_{\pi \in \{\pi^*_{r^w}\}_{w=1}^W \cup \{\pi_\base^k\}} \hat{\text{WR}}^\pi$. \label{line:UCB} \\ 
            %
            %
            %
            %
            %
            %
            Collect online data $(s^{k,n}, a^{k,n},\ta^{k,n}, y^{k,n}) \sim \rho\times\pi^{k,n}\times\pi^k_\Online\times\mP_{r^*}(\cdot|\cdot,\cdot,\cdot)$. \\
            %
            %
            %
            %
        }
        $\pi^{k+1}_{\base} \gets \text{Alg}_{\text{PO}}(\pi^{k}_{\base},\cD^{k,N+1})$; \\
    }
    \Return $\pi^{K+1}_{\base}$.
    \caption{Empirical $\TPO$}\label{alg:empirical}
\end{algorithm*}

\textbf{Choice of $\alpha$ and the total regret of $\TPO$}~
Our algorithm treats $\alpha$ as a hyperparameter. 
Based on the discussion above, with proper choice of $\alpha$ we have the following corollary.
\begin{corollary}[Total Regret of $\TPO$]\label{coro:total_regret}
    By choosing XPO \citep{xie2024exploratory} as $\AlgOnline$ and setting $\alpha = e^{-\frac{R}{\beta}} \leq \cov_{\infty}^{-1}(\Pi)$, Thm.~\ref{thm:regret_guarantees} implies that $\TPO$ achieves $\tilde{O}(W\sqrt{T})$ regret if $T < \frac{W^2}{\Delta^2_{\min}}$, or $\tilde{O}(\sqrt{T})$ regret if $T > \frac{W^2}{\Delta^2_{\min}}$ and large enough.
\end{corollary}
Depending on the concrete scenarios, if we have stronger prior beliefs that one of $\{r^w\}_{w\in[W]}$ is similar to $r^*$, we should prefer larger $\alpha$.
Besides, one may gradually decay $\alpha$ to 0 when the iteration number is large enough, which can also result in a total regret growing with $\tilde{O}(\sqrt{T})$ over time.

\textbf{Improved regret bound in standard online RLHF without source rewards}~Although our focus is transfer learning, our results can be extended to the standard online RLHF setting where no source tasks are present (i.e., $W=0$).
As implied by Corollary~\ref{coro:total_regret}, only utilizing self-transfer learning can result in an $\tilde{O}(\sqrt{T})$ online regret, thereby strictly improving existing results \citep{xiong2024iterative, xie2024exploratory,cen2024value,zhang2024self}.
%
%
%
%
%
%
%
%
%

\iffalse
%
%
Our discussion for Case 1 enlightens the benefits when the source reward models have high quality, i.e. $\Delta_{\min}$ is small.
Note that the sub-optimality of $\pi^t_{\mix}$ depends on the accumulative regret up to step $t$. A lower $\Delta_{\min}$ implies a lower $\Coeff$ in the above analysis, which implies $\cov^{\pi^*_{r^*}|\pi_\mix^{t}}$ can have a faster convergence to 1.
%
%


%
%
\fi



%
%
%



%
%
%
%
%
%
%
%
%
%
%
%
%
%
%


\section{From Theory to an Empirical Algorithm}\label{sec:empirical_alg}
In terms of computational overheads, $\TPO$ requires solving multiple minimax optimization problems, which restricts its applicability to fine-tune LLMs in practice.
To address this, adhering to the design principles of $\TPO$, we introduce a more computationally efficient alternative in Alg.~\ref{alg:empirical}.

\textbf{Key Insight: Estimating Win Rates instead of Policy Values}~
As discussed in Sec.~\ref{sec:main_theory}, several optimization steps are designed to estimate policy values used for transfer policy selection, because they help to identify the policies' coverability for optimal policy (i.e. $\cov^{\pi^*_{r^*}|\cdot}$).
The key insight in our empirical algorithm design is to \emph{find a more accessible indicator to infer $\cov^{\pi^*_{r^*}|\cdot}$}.
This leads us to the policy win rates, i.e., the probability that human prefer the generation by one policy over another. Formally, given two policies $\pi, \tpi$, the win rate of $\tpi$ over $\pi$ is defined by:
$
    \mP_{r^*}(\tpi \succ \pi) := \EE_{s\sim\rho,a\sim\tpi,a'\sim\pi}[\mP_{r^*}(y=1|s,a,a')].
$

Win rates between two policies can be unbiasedly estimated by querying human preferences with their generated responses.
Moreover, win rates can be used to construct a lower bound for $\cov^{\pi^*_{r^*}|\cdot}$, as stated in Lem.~\ref{lem:BT_LB_coverage} below.
\begin{lemma}\label{lem:BT_LB_coverage}
    Under BT-model\footnote{
        Lem.~\ref{lem:BT_LB_coverage} can be generalized beyond BT-model.
        Besides, it is possible to construct a lower bound involving $\mP_{r^*}(\bpi\succ\pi)$ instead.
        See Lem.~\ref{lem:LB_coverage_formal} and Remark~\ref{remark:LB_coverage} in Appx.~\ref{appx:win_rate_and_coverage} for more details.
        }, for any $\pi$:
    %
    %
    %
    %
    %
    %
    %
    %
    %
    %
    $\displaystyle\cov^{\pi^*_{r^*}|\pi}\geq \!$$
    \displaystyle\max_{\gamma > 0, \bpi} $$({\sqrt{(\gamma\! +\! 2\mP_{r^*}(\pi\!\succ \!\bpi))  \log \frac{1+\gamma}{\gamma}} + \sqrt{\frac{J_\beta(\pi^*_{r^*}) - J_\beta(\bar{\pi})}{2\beta}}})^{-1}.$
    %
    %
    %
\end{lemma}
Note that we may not identify the policy with the best coverage for $\pi^*_{r^*}$ through the lower bound above.
However, it still provides useful guidance for practice: we can filter out policies yielding high lower bound.
In Lem.~\ref{lem:BT_LB_coverage}, for any fixed $\gamma$ and comparator $\bpi$, the lower bound for $\cov^{\pi^*_{r^*}|\pi}$ increases as $\mP_{r^*}(\pi \succ \bpi)$ decay to 0, suggesting prioritizing transferring from policies with high win rates.

The key question now is how to choose the comparator $\bpi$. According to Lem.~\ref{lem:BT_LB_coverage}, ideally, the comparator should be close to $\pi^*_{r^*}$, so that $J_\beta(\pi^*_{r^*}) - J_\beta(\bar{\pi})$ becomes negligible, allowing the win rate term to dominate the lower bound.
Since we do not know $\pi^*_{r^*}$ in advance, empirically, we can choose the learning policy as the comparator, which is optimized and progressively converges to $\pi^*_{r^*}$.
%
%
%

\textbf{From Insights to Practice}~
Next, we walk through empirical $\TPO$ in Alg.~\ref{alg:empirical} and explain how we integrate these insights into the algorithm design.
Alg.~\ref{alg:empirical} utilizes an iterative online learning framework, which repeatedly collects online data and optimizes the policy.
We start by initializing the online learning policy $\pi^1_\base$ with the reference policy $\pi_\textref$.
For computational efficiency, in each iteration $k$, we avoid separately computing online exploration policies and self-transfer learning policies as done in $\TPO$. Instead, we only compute one policy $\pi^k_\base$ (updated from $\pi^{k-1}_\base$) by $\text{Alg}_{\text{PO}}$. Here $\text{Alg}_{\text{PO}}$ is a placeholder for an arbitrary \textbf{P}olicy \textbf{O}ptimization algorithm, and we do not restrict the concrete choice.
Such a design increases the modularity of our empirical TPO, making it possible to combine with various policy optimization methods and enhance their performance.
For example, $\text{Alg}_{\text{PO}}$ may be instantiated by DPO \citep{rafailov2024direct}, resulting in a transfer learning framework built on iterative-DPO  \citep{xiong2024iterative, yuan2024self}.
Besides, one may consider other advanced (online) methods, such as XPO \citep{xie2024exploratory}, IPO \citep{azar2024general}, etc.

As the core ingredients of our empirical TPO, during data collection, the algorithm selects the policy $\pi^{k,n} \in \{\pi^*_{r^*}\}_{w=1}^W \cup \{\pi^k_\base\}$ with the highest win rate when competing against $\pi^k_\base$.
Intuitively, we encourage transfer learning if $\{\pi^*_{r^*}\}_{w=1}^W$ includes high-quality candidates; otherwise, the algorithm conducts standard iterative policy optimization with $\text{Alg}_{\text{PO}}$ by default.
This strategy also aligns with the heuristic principle: \textbf{\emph{learn from an expert until surpassing it}}.
Lastly, since the win rates are unknown in advance, the selection process is formulated as a multi-armed bandit problem. We employ a UCB subroutine (line~\ref{line:UCB}) to balance the exploration and exploitation during the win rates estimation.
%
%



%
%
%
%
%
%
%
%

%
%
%
%
%




%
%

%

%
%
%
%

%
%

%
%
%
%
%
%


%

%
%

%
%
%

%
%
%
%
%
%
%
%
%

%
%
%
%
%
%
%
%
%
%
%
    
%
%
%
%
%
%

%
%
%

%
%
%
%

%
%
%
%
%


%
%

%

%
%
%

%

%
%
%

\iffalse
\subsubsection{Theoretical Investigation}
\paragraph{No-regret behaviors by EXP3}
We use vector $l^k := \{\Pr(\pi^*_{r^w} \succ \pi^k|r^*)\}_{w=1}^W \cup \{0.5\}$ to denote the probability of win rates.
\begin{align*}
    \max_{p\in\Delta([0,1]^{W+1})} \EE[\sum_{b=1}^B \langle p - p^{k,n}, l^k \rangle] = o(B).
\end{align*}
Suppose there exists $w\in[W]$, such that $\Pr(\pi^*_{r^w} \succ \pi^k|r^*) \leq 0.5 - \xi$, consider the vector $p$, such that $p(w') = \frac{1}{B} \sum_{b=1}^B p^{k,n}(w')$ for $w'\not\in\{w,W+1\}$ and $p(w) = 0$ and $p(W+1) = \frac{1}{B} \sum_{b=1}^B p^{k,n}(w) + p^{k,n}(W+1)$, we have:
\begin{align*}
    \xi \cdot \sum_{b=1}^B p^{k,n}(w)  \leq o(B).
\end{align*}
%
%
%
%
\paragraph{Turning back to ``self-transfer learning'' Eventually}
As the training iteration increases, the learned policy will gradually converge to the true policy. In theory, the value gap reflects the KL divergence, which can be used to control the coverage coefficient, by the Lem.~\ref{lem:upper_bound_coverage}:
As a result, for any fixed soruce tasks, as long as they do not cover the true optimal policy (non-zero reward gap), the online policy (which continuously getting improved) would be a better choice to cover the optimal policy.

On the one hand, it suggests that we should give up all the source tasks eventually (which also makes sense).
On the other hand, as a by-product, such results reflects a way to improve the basic gaurantees by XPO, because of the ``self-transfer'' learning.

\red{
TODO: how to understand the benefits of the transfer learning in the early stages? 
Would our theorems still useful in some sense? using preference as an indicator of the coverage coefficient.
}

In some sense, as the policy improves, the learning policy will beat the others, so we will eventually turn to this mode. However, there is no guarantee, if there are some policy which achieves high KL divergence, even if the reward value is high.


\paragraph{}
From Lem.~\ref{lem:upper_bound_coverage}, we know that, a meaningful source task selection technique is to find the policy which results in the lowest policy gap.

We can just treat $J_\beta(\pi)$ as reward function of the bandit, and trying to estimate it.
Note that, we can get access to the entire policy, so the KL value can be easily computed.
If we can get access to the reward of the responses, the expected return part can be unbiasedly estimated, then, it is a standard bandit setting with stochastic feedback.

However, in the worst case, we can only get access to preference-based feedback, and we need to estimate the reward from it.

%
%
%
%
%
%
%
%

%
%


\red{In practice, computing the $\hat{r}$ can be chanllenging (?)}, so we use win rate instead. Or maybe we can just use $\log \pi$ as the estimated reward?
\fi

\iffalse
\newpage
\subsection{A Refined Reward Models Selection Algorithm}
In the second algorithm, we do a refined state-wisely reward model selection, which requires additional estimation of the performance for each state.
We introduce another scalar function class $\cF$, such that $\forall, f\in\cF$, $f:\cS\times\cA \rightarrow [0,1]^W \cup \{\frac{1}{2}\} \in \cR^{W+1}$. $\cF$ will be used to approximate the success rate for $\pi^*_{r^w}$ in each state.

\begin{algorithm}[h]
    \textbf{Input}: Iteraction number $K$; Batch size $B$; Imperfect reward models $\{r^w\}_{w\in[W]}$; Hyper-parameter $\alpha$ and $\eta$; Scalar function class $\cF$\\
    $\cD \gets \emptyset$; Initialize $p^0(\cdot|\cdot) = (\frac{1}{W+1},\frac{1}{W+1}...,\frac{1}{W+1})\in\mR^{W+1}$.\\
    \For{$k=1,...,K$}{
        $\pi^k \gets \texttt{Online-Alg}(\cD)$; \\
        \For{$b=1,..., B$}{
            Sample $s^{k,n}\sim\rho$. \\
            \textbf{If} $b \leq \alpha B$ \textbf{then} $w^{k,n} \gets 0$  
                \textbf{else} $w^{k,n} \sim p^k(\cdot|s^{k,n})$ \\
            Sample responses $(a^{k,n},\tilde{a}^{k,n}) \sim \pi^*_{r_{w^{k,n}}}(\cdot|s^{k,n}) \times \pi^k(\cdot|s^{k,n})$. \\
            Query the human preference $y^{k,n} \sim \Pr(a^{k,n} \succ \tilde{a}^{k,n}|s^{k,n})$. \\
                $\cD \gets \cD \cup \{(s^{k,n},a^{k,n},\tilde{a}^{k,n},y^{k,n})\}$ 
        }
        \blue{// train $f^k(\cdot|s^{k,n})$ to predict $\Pr(\pi^*_{r^w}\succ \pi^k|s^{k,n})$.} \\
        \blue{// Here $\texttt{CE}(y,p) := y\log p + (1-y) \log(1-p)$} \\
        %
        $\forall w\in[W],~ f^k_w \gets \argmin_{f\in\cF} \sum_{b= \alpha B + 1}^B \mathbb{I}[w^{k,n} = w] \texttt{CE}(y^{k,n}, f(\cdot|s^{k,n}))$ \\
        Update $p^k(w|\cdot) \propto p^{k-1}(w|\cdot) \cdot \exp(\eta \cdot f^{k}_w(\cdot))$ \\
    }
    \Return $\hat{\pi}^*_{r^*} \gets \texttt{Offline-Alg}(\cD)$.
    \caption{A Refined Reward Models Selection Algorithm}
\end{algorithm}

\fi

\section{Experimental Analysis}
\label{sec:exp}
We now describe in detail our experimental analysis. The experimental section is organized as follows:
%\begin{enumerate}[noitemsep,topsep=0pt,parsep=0pt,partopsep=0pt,leftmargin=0.5cm]
%\item 

\noindent In {\bf 
Section~\ref{exp:setup}}, we introduce the datasets and methods to evaluate the previously defined accuracy measures.

%\item
\noindent In {\bf 
Section~\ref{exp:qual}}, we illustrate the limitations of existing measures with some selected qualitative examples.

%\item 
\noindent In {\bf 
Section~\ref{exp:quant}}, we continue by measuring quantitatively the benefits of our proposed measures in terms of {\it robustness} to lag, noise, and normal/abnormal ratio.

%\item 
\noindent In {\bf 
Section~\ref{exp:separability}}, we evaluate the {\it separability} degree of accurate and inaccurate methods, using the existing and our proposed approaches.

%\item
\noindent In {\bf 
Section~\ref{sec:entropy}}, we conduct a {\it consistency} evaluation, in which we analyze the variation of ranks that an AD method can have with an accuracy measures used.

%\item 
\noindent In {\bf 
Section~\ref{sec:exectime}}, we conduct an {\it execution time} evaluation, in which we analyze the impact of different parameters related to the accuracy measures and the time series characteristics. 
We focus especially on the comparison of the different VUS implementations.
%\end{enumerate}

\begin{table}[tb]
\caption{Summary characteristics (averaged per dataset) of the public datasets of TSB-UAD (S.: Size, Ano.: Anomalies, Ab.: Abnormal, Den.: Density)}
\label{table:charac}
%\vspace{-0.2cm}
\footnotesize
\begin{center}
\scalebox{0.82}{
\begin{tabular}{ |r|r|r|r|r|r|} 
 \hline
\textbf{\begin{tabular}[c]{@{}c@{}}Dataset \end{tabular}} & 
\textbf{\begin{tabular}[c]{@{}c@{}}S. \end{tabular}} & 
\textbf{\begin{tabular}[c]{c@{}} Len.\end{tabular}} & 
\textbf{\begin{tabular}[c]{c@{}} \# \\ Ano. \end{tabular}} &
\textbf{\begin{tabular}[c]{c@{}c@{}} \# \\ Ab. \\ Points\end{tabular}} &
\textbf{\begin{tabular}[c]{c@{}c@{}} Ab. \\ Den. \\ (\%)\end{tabular}} \\ \hline
Dodgers \cite{10.1145/1150402.1150428} & 1 & 50400   & 133.0     & 5612.0  &11.14 \\ \hline
SED \cite{doi:10.1177/1475921710395811}& 1 & 100000   & 75.0     & 3750.0  & 3.7\\ \hline
ECG \cite{goldberger_physiobank_2000}   & 52 & 230351  & 195.6     & 15634.0  &6.8 \\ \hline
IOPS \cite{IOPS}   & 58 & 102119  & 46.5     & 2312.3   &2.1 \\ \hline
KDD21 \cite{kdd} & 250 &77415   & 1      & 196.5   &0.56 \\ \hline
MGAB \cite{markus_thill_2020_3762385}   & 10 & 100000  & 10.0     & 200.0   &0.20 \\ \hline
NAB \cite{ahmad_unsupervised_2017}   & 58 & 6301   & 2.0      & 575.5   &8.8 \\ \hline
NASA-M. \cite{10.1145/3449726.3459411}   & 27 & 2730   & 1.33      & 286.3   &11.97 \\ \hline
NASA-S. \cite{10.1145/3449726.3459411}   & 54 & 8066   & 1.26      & 1032.4   &12.39 \\ \hline
SensorS. \cite{YAO20101059}   & 23 & 27038   & 11.2     & 6110.4   &22.5 \\ \hline
YAHOO \cite{yahoo}  & 367 & 1561   & 5.9      & 10.7   &0.70 \\ \hline 
\end{tabular}}
\end{center}
\end{table}











\subsection{Experimental Setup and Settings}
\label{exp:setup}
%\vspace{-0.1cm}

\begin{figure*}[tb]
  \centering
  \includegraphics[width=1\linewidth]{figures/quality.pdf}
  %\vspace{-0.7cm}
  \caption{Comparison of evaluation measures (proposed measures illustrated in subplots (b,c,d,e); all others summarized in subplots (f)) on two examples ((A)AE and OCSM applied on MBA(805) and (B) LOF and OCSVM applied on MBA(806)), illustrating the limitations of existing measures for scores with noise or containing a lag. }
  \label{fig:quality}
  %\vspace{-0.1cm}
\end{figure*}

We implemented the experimental scripts in Python 3.8 with the following main dependencies: sklearn 0.23.0, tensorflow 2.3.0, pandas 1.2.5, and networkx 2.6.3. In addition, we used implementations from our TSB-UAD benchmark suite.\footnote{\scriptsize \url{https://www.timeseries.org/TSB-UAD}} For reproducibility purposes, we make our datasets and code available.\footnote{\scriptsize \url{https://www.timeseries.org/VUS}}
\newline \textbf{Datasets: } For our evaluation purposes, we use the public datasets identified in our TSB-UAD benchmark. The latter corresponds to $10$ datasets proposed in the past decades in the literature containing $900$ time series with labeled anomalies. Specifically, each point in every time series is labeled as normal or abnormal. Table~\ref{table:charac} summarizes relevant characteristics of the datasets, including their size, length, and statistics about the anomalies. In more detail:

\begin{itemize}
    \item {\bf SED}~\cite{doi:10.1177/1475921710395811}, from the NASA Rotary Dynamics Laboratory, records disk revolutions measured over several runs (3K rpm speed).
	\item {\bf ECG}~\cite{goldberger_physiobank_2000} is a standard electrocardiogram dataset and the anomalies represent ventricular premature contractions. MBA(14046) is split to $47$ series.
	\item {\bf IOPS}~\cite{IOPS} is a dataset with performance indicators that reflect the scale, quality of web services, and health status of a machine.
	\item {\bf KDD21}~\cite{kdd} is a composite dataset released in a SIGKDD 2021 competition with 250 time series.
	\item {\bf MGAB}~\cite{markus_thill_2020_3762385} is composed of Mackey-Glass time series with non-trivial anomalies. Mackey-Glass data series exhibit chaotic behavior that is difficult for the human eye to distinguish.
	\item {\bf NAB}~\cite{ahmad_unsupervised_2017} is composed of labeled real-world and artificial time series including AWS server metrics, online advertisement clicking rates, real time traffic data, and a collection of Twitter mentions of large publicly-traded companies.
	\item {\bf NASA-SMAP} and {\bf NASA-MSL}~\cite{10.1145/3449726.3459411} are two real spacecraft telemetry data with anomalies from Soil Moisture Active Passive (SMAP) satellite and Curiosity Rover on Mars (MSL).
	\item {\bf SensorScope}~\cite{YAO20101059} is a collection of environmental data, such as temperature, humidity, and solar radiation, collected from a sensor measurement system.
	\item {\bf Yahoo}~\cite{yahoo} is a dataset consisting of real and synthetic time series based on the real production traffic to some of the Yahoo production systems.
\end{itemize}


\textbf{Anomaly Detection Methods: }  For the experimental evaluation, we consider the following baselines. 

\begin{itemize}
\item {\bf Isolation Forest (IForest)}~\cite{liu_isolation_2008} constructs binary trees based on random space splitting. The nodes (subsequences in our specific case) with shorter path lengths to the root (averaged over every random tree) are more likely to be anomalies. 
\item {\bf The Local Outlier Factor (LOF)}~\cite{breunig_lof_2000} computes the ratio of the neighbor density to the local density. 
\item {\bf Matrix Profile (MP)}~\cite{yeh_time_2018} detects as anomaly the subsequence with the most significant 1-NN distance. 
\item {\bf NormA}~\cite{boniol_unsupervised_2021} identifies the normal patterns based on clustering and calculates each point's distance to normal patterns weighted using statistical criteria. 
\item {\bf Principal Component Analysis (PCA)}~\cite{aggarwal_outlier_2017} projects data to a lower-dimensional hyperplane. Outliers are points with a large distance from this plane. 
\item {\bf Autoencoder (AE)} \cite{10.1145/2689746.2689747} projects data to a lower-dimensional space and reconstructs it. Outliers are expected to have larger reconstruction errors. 
\item {\bf LSTM-AD}~\cite{malhotra_long_2015} use an LSTM network that predicts future values from the current subsequence. The prediction error is used to identify anomalies.
\item {\bf Polynomial Approximation (POLY)} \cite{li_unifying_2007} fits a polynomial model that tries to predict the values of the data series from the previous subsequences. Outliers are detected with the prediction error. 
\item {\bf CNN} \cite{8581424} built, using a convolutional deep neural network, a correlation between current and previous subsequences, and outliers are detected by the deviation between the prediction and the actual value. 
\item {\bf One-class Support Vector Machines (OCSVM)} \cite{scholkopf_support_1999} is a support vector method that fits a training dataset and finds the normal data's boundary.
\end{itemize}

\subsection{Qualitative Analysis}
\label{exp:qual}



We first use two examples to demonstrate qualitatively the limitations of existing accuracy evaluation measures in the presence of lag and noise, and to motivate the need for a new approach. 
These two examples are depicted in Figure~\ref{fig:quality}. 
The first example, in Figure~\ref{fig:quality}(A), corresponds to OCSVM and AE on the MBA(805) dataset (named MBA\_ECG805\_data.out in the ECG dataset). 

We observe in Figure~\ref{fig:quality}(A)(a.1) and (a.2) that both scores identify most of the anomalies (highlighted in red). However, the OCSVM score points to more false positives (at the end of the time series) and only captures small sections of the anomalies. On the contrary, the AE score points to fewer false positives and captures all abnormal subsequences. Thus we can conclude that, visually, AE should obtain a better accuracy score than OCSVM. Nevertheless, we also observe that the AE score is lagged with the labels and contains more noise. The latter has a significant impact on the accuracy of evaluation measures. First, Figure~\ref{fig:quality}(A)(c) is showing that AUC-PR is better for OCSM (0.73) than for AE (0.57). This is contradictory with what is visually observed from Figure~\ref{fig:quality}(A)(a.1) and (a.2). However, when using our proposed measure R-AUC-PR, OCSVM obtains a lower score (0.83) than AE (0.89). This confirms that, in this example, a buffer region before the labels helps to capture the true value of an anomaly score. Overall, Figure~\ref{fig:quality}(A)(f) is showing in green and red the evolution of accuracy score for the 13 accuracy measures for AE and OCSVM, respectively. The latter shows that, in addition to Precision@k and Precision, our proposed approach captures the quality order between the two methods well.

We now present a second example, on a different time series, illustrated in Figure~\ref{fig:quality}(B). 
In this case, we demonstrate the anomaly score of OCSVM and LOF (depicted in Figure~\ref{fig:quality}(B)(a.1) and (a.2)) applied on the MBA(806) dataset (named MBA\_ECG806\_data.out in the ECG dataset). 
We observe that both methods produce the same level of noise. However, LOF points to fewer false positives and captures more sections of the abnormal subsequences than OCSVM. 
Nevertheless, the LOF score is slightly lagged with the labels such that the maximum values in the LOF score are slightly outside of the labeled sections. 
Thus, as illustrated in Figure~\ref{fig:quality}(B)(f), even though we can visually consider that LOF is performing better than OCSM, all usual measures (Precision, Recall, F, precision@k, and AUC-PR) are judging OCSM better than AE. On the contrary, measures that consider lag (Rprecision, Rrecall, RF) rank the methods correctly. 
However, due to threshold issues, these measures are very close for the two methods. Overall, only AUC-ROC and our proposed measures give a higher score for LOF than for OCSVM.

\subsection{Quantitative Analysis}
\label{exp:case}

\begin{figure}[t]
  \centering
  \includegraphics[width=1\linewidth]{figures/eval_case_study.pdf}
  %\vspace*{-0.7cm}
  \caption{\commentRed{
  Comparison of evaluation measures for synthetic data examples across various scenarios. S8 represents the oracle case, where predictions perfectly align with labeled anomalies. Problematic cases are highlighted in the red region.}}
  %\vspace*{-0.5cm}
  \label{fig:eval_case_study}
\end{figure}
\commentRed{
We present the evaluation results for different synthetic data scenarios, as shown in Figure~\ref{fig:eval_case_study}. These scenarios range from S1, where predictions occur before the ground truth anomaly, to S12, where predictions fall within the ground truth region. The red-shaded regions highlight problematic cases caused by a lack of adaptability to lags. For instance, in scenarios S1 and S2, a slight shift in the prediction leads to measures (e.g., AUC-PR, F score) that fail to account for lags, resulting in a zero score for S1 and a significant discrepancy between the results of S1 and S2. Thus, we observe that our proposed VUS effectively addresses these issues and provides robust evaluations results.}

%\subsection{Quantitative Analysis}
%\subsection{Sensitivity and Separability Analysis}
\subsection{Robustness Analysis}
\label{exp:quant}


\begin{figure}[tb]
  \centering
  \includegraphics[width=1\linewidth]{figures/lag_sensitivity_analysis.pdf}
  %\vspace*{-0.7cm}
  \caption{For each method, we compute the accuracy measures 10 times with random lag $\ell \in [-0.25*\ell,0.25*\ell]$ injected in the anomaly score. We center the accuracy average to 0.}
  %\vspace*{-0.5cm}
  \label{fig:lagsensitivity}
\end{figure}

We have illustrated with specific examples several of the limitations of current measures. 
We now evaluate quantitatively the robustness of the proposed measures when compared to the currently used measures. 
We first evaluate the robustness to noise, lag, and normal versus abnormal points ratio. We then measure their ability to separate accurate and inaccurate methods.
%\newline \textbf{Sensitivity Analysis: } 
We first analyze the robustness of different approaches quantitatively to different factors: (i) lag, (ii) noise, and (iii) normal/abnormal ratio. As already mentioned, these factors are realistic. For instance, lag can be either introduced by the anomaly detection methods (such as methods that produce a score per subsequences are only high at the beginning of abnormal subsequences) or by human labeling approximation. Furthermore, even though lag and noises are injected, an optimal evaluation metric should not vary significantly. Therefore, we aim to measure the variance of the evaluation measures when we vary the lag, noise, and normal/abnormal ratio. We proceed as follows:

\begin{enumerate}[noitemsep,topsep=0pt,parsep=0pt,partopsep=0pt,leftmargin=0.5cm]
\item For each anomaly detection method, we first compute the anomaly score on a given time series.
\item We then inject either lag $l$, noise $n$ or change the normal/abnormal ratio $r$. For 10 different values of $l \in [-0.25*\ell,0.25*\ell]$, $n \in [-0.05*(max(S_T)-min(S_T)),0.05*(max(S_T)-min(S_T))]$ and $r \in [0.01,0.2]$, we compute the 13 different measures.
\item For each evaluation measure, we compute the standard deviation of the ten different values. Figure~\ref{fig:lagsensitivity}(b) depicts the different lag values for six AD methods applied on a data series in the ECG dataset.
\item We compute the average standard deviation for the 13 different AD quality measures. For example, figure~\ref{fig:lagsensitivity}(a) depicts the average standard deviation for ten different lag values over the AD methods applied on the MBA(805) time series.
\item We compute the average standard deviation for the every time series in each dataset (as illustrated in Figure~\ref{fig:sensitivity_per_data}(b to j) for nine datasets of the benchmark.
\item We compute the average standard deviation for the every dataset (as illustrated in Figure~\ref{fig:sensitivity_per_data}(a.1) for lag, Figure~\ref{fig:sensitivity_per_data}(a.2) for noise and Figure~\ref{fig:sensitivity_per_data}(a.3) for normal/abnormal ratio).
\item We finally compute the Wilcoxon test~\cite{10.2307/3001968} and display the critical diagram over the average standard deviation for every time series (as illustrated in Figure~\ref{fig:sensitivity}(a.1) for lag, Figure~\ref{fig:sensitivity}(a.2) for noise and Figure~\ref{fig:sensitivity}(a.3) for normal/abnormal ratio).
\end{enumerate}

%height=8.5cm,

\begin{figure}[tb]
  \centering
  \includegraphics[width=\linewidth]{figures/sensitivity_per_data_long.pdf}
%  %\vspace*{-0.3cm}
  \caption{Robustness Analysis for nine datasets: we report, over the entire benchmark, the average standard deviation of the accuracy values of the measures, under varying (a.1) lag, (a.2) noise, and (a.3) normal/abnormal ratio. }
  \label{fig:sensitivity_per_data}
\end{figure}

\begin{figure*}[tb]
  \centering
  \includegraphics[width=\linewidth]{figures/sensitivity_analysis.pdf}
  %\vspace*{-0.7cm}
  \caption{Critical difference diagram computed using the signed-rank Wilkoxon test (with $\alpha=0.1$) for the robustness to (a.1) lag, (a.2) noise and (a.3) normal/abnormal ratio.}
  \label{fig:sensitivity}
\end{figure*}

The methods with the smallest standard deviation can be considered more robust to lag, noise, or normal/abnormal ratio from the above framework. 
First, as stated in the introduction, we observe that non-threshold-based measures (such as AUC-ROC and AUC-PR) are indeed robust to noise (see Figure~\ref{fig:sensitivity_per_data}(a.2)), but not to lag. Figure~\ref{fig:sensitivity}(a.1) demonstrates that our proposed measures VUS-ROC, VUS-PR, R-AUC-ROC, and R-AUC-PR are significantly more robust to lag. Similarly, Figure~\ref{fig:sensitivity}(a.2) confirms that our proposed measures are significantly more robust to noise. However, we observe that, among our proposed measures, only VUS-ROC and R-AUC-ROC are robust to the normal/abnormal ratio and not VUS-PR and R-AUC-PR. This is explained by the fact that Precision-based measures vary significantly when this ratio changes. This is confirmed by Figure~\ref{fig:sensitivity_per_data}(a.3), in which we observe that Precision and Rprecision have a high standard deviation. Overall, we observe that VUS-ROC is significantly more robust to lag, noise, and normal/abnormal ratio than other measures.




\subsection{Separability Analysis}
\label{exp:separability}

%\newline \textbf{Separability Analysis: } 
We now evaluate the separability capacities of the different evaluation metrics. 
\commentRed{The main objective is to measure the ability of accuracy measures to separate accurate methods from inaccurate ones. More precisely, an appropriate measure should return accuracy scores that are significantly higher for accurate anomaly scores than for inaccurate ones.}
We thus manually select accurate and inaccurate anomaly detection methods and verify if the accuracy evaluation scores are indeed higher for the accurate than for the inaccurate methods. Figure~\ref{fig:separability} depicts the latter separability analysis applied to the MBA(805) and the SED series. 
The accurate and inaccurate anomaly scores are plotted in green and red, respectively. 
We then consider 12 different pairs of accurate/inaccurate methods among the eight previously mentioned anomaly scores. 
We slightly modify each score 50 different times in which we inject lag and noises and compute the accuracy measures. 
Figure~\ref{fig:separability}(a.4) and Figure~\ref{fig:separability}(b.4) are divided into four different subplots corresponding to 4 pairs (selected among the twelve different pairs due to lack of space). 
Each subplot corresponds to two box plots per accuracy measure. 
The green and red box plots correspond to the 50 accuracy measures on the accurate and inaccurate methods. 
If the red and green box plots are well separated, we can conclude that the corresponding accuracy measures are separating the accurate and inaccurate methods well. 
We observe that some accuracy measures (such as VUS-ROC) are more separable than others (such as RF). We thus measure the separability of the two box-plots by computing the Z-test. 

\begin{figure*}[tb]
  \centering
  \includegraphics[width=1\linewidth]{figures/pairwise_comp_example_long.pdf}
  %\vspace*{-0.5cm}
  \caption{Separability analysis applied on 4 pairs of accurate (green) and inaccurate (red) methods on (a) the MBA(805) data series, and (b) the SED data series.}
  %\vspace*{-0.3cm}
  \label{fig:separability}
\end{figure*}

We now aggregate all the results and compute the average Z-test for all pairs of accurate/inaccurate datasets (examples are shown in Figures~\ref{fig:separability}(a.2) and (b.2) for accurate anomaly scores, and in Figures~\ref{fig:separability}(a.3) and (b.3) for inaccurate anomaly scores, for the MBA(805) and SED series, respectively). 
Next, we perform the same operation over three different data series: MBA (805), MBA(820), and SED. 
Then, we depict the average Z-test for these three datasets in Figure~\ref{fig:separability_agg}(a). 
Finally, we show the average Z-test for all datasets in Figure~\ref{fig:separability_agg}(b). 


We observe that our proposed VUS-based and Range-based measures are significantly more separable than other current accuracy measures (up to two times for AUC-ROC, the best measures of all current ones). Furthermore, when analyzed in detail in Figure~\ref{fig:separability} and Figure~\ref{fig:separability_agg}, we confirm that VUS-based and Range-based are more separable over all three datasets. 

\begin{figure}[tb]
  \centering
  \includegraphics[width=\linewidth]{figures/agregated_sep_analysis.pdf}
  %\vspace*{-0.5cm}
  \caption{Overall separability analysis (averaged z-test between the accuracy values distributions of accurate and inaccurate methods) applied on 36 pairs on 3 datasets.}
  \label{fig:separability_agg}
\end{figure}


\noindent \textbf{Global Analysis: } Overall, we observe that VUS-ROC is the most robust (cf. Figure~\ref{fig:sensitivity}) and separable (cf. Figure~\ref{fig:separability_agg}) measure. 
On the contrary, Precision and Rprecision are non-robust and non-separable. 
Among all previous accuracy measures, only AUC-ROC is robust and separable. 
Popular measures, such as, F, RF, AUC-ROC, and AUC-PR are robust but non-separable.

In order to visualize the global statistical analysis, we merge the robustness and the separability analysis into a single plot. Figure~\ref{fig:global} depicts one scatter point per accuracy measure. 
The x-axis represents the averaged standard deviation of lag and noise (averaged values from Figure~\ref{fig:sensitivity_per_data}(a.1) and (a.2)). The y-axis corresponds to the averaged Z-test (averaged value from Figure~\ref{fig:separability_agg}). 
Finally, the size of the points corresponds to the sensitivity to the normal/abnormal ratio (values from Figure~\ref{fig:sensitivity_per_data}(a.3)). 
Figure~\ref{fig:global} demonstrates that our proposed measures (located at the top left section of the plot) are both the most robust and the most separable. 
Among all previous accuracy measures, only AUC-ROC is on the top left section of the plot. 
Popular measures, such as, F, RF, AUC-ROC, AUC-PR are on the bottom left section of the plot. 
The latter underlines the fact that these measures are robust but non-separable.
Overall, Figure~\ref{fig:global} confirms the effectiveness and superiority of our proposed measures, especially of VUS-ROC and VUS-PR.


\begin{figure}[tb]
  \centering
  \includegraphics[width=\linewidth]{figures/final_result.pdf}
  \caption{Evaluation of all measures based on: (y-axis) their separability (avg. z-test), (x-axis) avg. standard deviation of the accuracy values when varying lag and noise, (circle size) avg. standard deviation of the accuracy values when varying the normal/abnormal ratio.}
  \label{fig:global}
\end{figure}




\subsection{Consistency Analysis}
\label{sec:entropy}

In this section, we analyze the accuracy of the anomaly detection methods provided by the 13 accuracy measures. The objective is to observe the changes in the global ranking of anomaly detection methods. For that purpose, we formulate the following assumptions. First, we assume that the data series in each benchmark dataset are similar (i.e., from the same domain and sharing some common characteristics). As a matter of fact, we can assume that an anomaly detection method should perform similarly on these data series of a given dataset. This is confirmed when observing that the best anomaly detection methods are not the same based on which dataset was analyzed. Thus the ranking of the anomaly detection methods should be different for different datasets, but similar for every data series in each dataset. 
Therefore, for a given method $A$ and a given dataset $D$ containing data series of the same type and domain, we assume that a good accuracy measure results in a consistent rank for the method $A$ across the dataset $D$. 
The consistency of a method's ranks over a dataset can be measured by computing the entropy of these ranks. 
For instance, a measure that returns a random score (and thus, a random rank for a method $A$) will result in a high entropy. 
On the contrary, a measure that always returns (approximately) the same ranks for a given method $A$ will result in a low entropy. 
Thus, for a given method $A$ and a given dataset $D$ containing data series of the same type and domain, we assume that a good accuracy measure results in a low entropy for the different ranks for method $A$ on dataset $D$.

\begin{figure*}[tb]
  \centering
  \includegraphics[width=\linewidth]{figures/entropy_long.pdf}
  %\vspace*{-0.5cm}
  \caption{Accuracy evaluation of the anomaly detection methods. (a) Overall average entropy per category of measures. Analysis of the (b) averaged rank and (c) averaged rank entropy for each method and each accuracy measure over the entire benchmark. Example of (b.1) average rank and (c.1) entropy on the YAHOO dataset, KDD21 dataset (b.2, c.2). }
  \label{fig:entropy}
\end{figure*}

We now compute the accuracy measures for the nine different methods (we compute the anomaly scores ten different times, and we use the average accuracy). 
Figures~\ref{fig:entropy}(b.1) and (b.2) report the average ranking of the anomaly detection methods obtained on the YAHOO and KDD21 datasets, respectively. 
The x-axis corresponds to the different accuracy measures. We first observe that the rankings are more separated using Range-AUC and VUS measures for these two datasets. Figure~\ref{fig:entropy}(b) depicts the average ranking over the entire benchmark. The latter confirms the previous observation that VUS measures provide more separated rankings than threshold-based and AUC-based measures. We also observe an interesting ranking evolution for the YAHOO dataset illustrated in Figure~\ref{fig:entropy}(b.1). We notice that both LOF and MatrixProfile (brown and pink curve) have a low rank (between 4 and 5) using threshold and AUC-based measures. However, we observe that their ranks increase significantly for range-based and VUS-based measures (between 2.5 and 3). As we noticed by looking at specific examples (see Figure~\ref{exp:qual}), LOF and MatrixProfile can suffer from a lag issue even though the anomalies are well-identified. Therefore, the range-based and VUS-based measures better evaluate these two methods' detection capability.


Overall, the ranking curves show that the ranks appear more chaotic for threshold-based than AUC-, Range-AUC-, and VUS-based measures. 
In order to quantify this observation, we compute the Shannon Entropy of the ranks of each anomaly detection method. 
In practice, we extract the ranks of methods across one dataset and compute Shannon's Entropy of the different ranks. 
Figures~\ref{fig:entropy}(c.1) and (c.2) depict the entropy of each of the nine methods for the YAHOO and KDD21 datasets, respectively. 
Figure~\ref{fig:entropy}(c) illustrates the averaged entropy for all datasets in the benchmark for each measure and method, while Figure~\ref{fig:entropy}(a) shows the averaged entropy for each category of measures.
We observe that both for the general case (Figure~\ref{fig:entropy}(a) and Figure~\ref{fig:entropy}(c)) and some specific cases (Figures~\ref{fig:entropy}(c.1) and (c.2)), the entropy is reducing when using AUC-, Range-AUC-, and VUS-based measures. 
We report the lowest entropy for VUS-based measures. 
Moreover, we notice a significant drop between threshold-based and AUC-based. 
This confirms that the ranks provided by AUC- and VUS-based measures are consistent for data series belonging to one specific dataset. 


Therefore, based on the assumption formulated at the beginning of the section, we can thus conclude that AUC, range-AUC, and VUS-based measures are providing more consistent rankings. Finally, as illustrated in Figure~\ref{fig:entropy}, we also observe that VUS-based measures result in the most ordered and similar rankings for data series from the same type and domain.










\subsection{Execution Time Analysis}
\label{sec:exectime}

In this section, we evaluate the execution time required to compute different evaluation measures. 
In Section~\ref{sec:synthetic_eval_time}, we first measure the influence of different time series characteristics and VUS parameters on the execution time. In Section~\ref{sec:TSB_eval_time}, we  measure the execution time of VUS (VUS-ROC and VUS-PR simultaneously), R-AUC (R-AUC-ROC and R-AUC-PR simultaneously), and AUC-based measures (AUC-ROC and AUC-PR simultaneously) on the TSB-UAD benchmark. \commentRed{As demonstrated in the previous section, threshold-based measures are not robust, have a low separability power, and are inconsistent. 
Such measures are not suitable for evaluating anomaly detection methods. Thus, in this section, we do not consider threshold-based measures.}


\subsubsection{Evaluation on Synthetic Time Series}\hfill\\
\label{sec:synthetic_eval_time}

We first analyze the impact that time series characteristics and parameters have on the computation time of VUS-based measures. 
to that effect, we generate synthetic time series and labels, where we vary the following parameters: (i) the number of anomalies {\bf$\alpha$} in the time series, (ii) the average \textbf{$\mu(\ell_a)$} and standard deviation $\sigma(\ell_a)$ of the anomalies lengths in the time series (all the anomalies can have different lengths), (iii) the length of the time series \textbf{$|T|$}, (iv) the maximum buffer length \textbf{$L$}, and (v) the number of thresholds \textbf{$N$}.


We also measure the influence on the execution time of the R-AUC- and AUC- related parameter, that is, the number of thresholds ($N$).
The default values and the range of variation of these parameters are listed in Table~\ref{tab:parameter_range_time}. 
For VUS-based measures, we evaluate the execution time of the initial VUS implementation, as well as the two optimized versions, VUS$_{opt}$ and VUS$_{opt}^{mem}$.

\begin{table}[tb]
    \centering
    \caption{Value ranges for the parameters: number of anomalies ($\alpha$), average and standard deviation anomaly length ($\mu(\ell_a)$,$\sigma(\ell_a)$), time series length ($|T|$), maximum buffer length ($L$), and number of thresholds ($N$).}
    \begin{tabular}{|c|c|c|c|c|c|c|} 
 \hline
 Param. & $\alpha$ & $\mu(\ell_a)$ & $\sigma(\ell_{a})$ & $|T|$ & $L$ & $N$ \\ [0.5ex] 
 \hline\hline
 \textbf{Default} & 10 & 10 & 0 & $10^5$ & 5 & 250\\ 
 \hline
 Min. & 0 & 0 & 0 & $10^3$ & 0 & 2 \\
 \hline
 Max. & $2*10^3$ & $10^3$ & $10$ & $10^5$ & $10^3$ & $10^3$ \\ [1ex] 
 \hline
\end{tabular}
    \label{tab:parameter_range_time}
\end{table}


Figure~\ref{fig:sythetic_exp_time} depicts the execution time (averaged over ten runs) for each parameter listed in Table~\ref{tab:parameter_range_time}. 
Overall, we observe that the execution time of AUC-based and R-AUC-based measures is significantly smaller than VUS-based measures.
In the following paragraph, we analyze the influence of each parameter and compare the experimental execution time evaluation to the theoretical complexity reported in Table~\ref{tab:complexity_summary}.

\vspace{0.2cm}
\noindent {\bf [Influence of $\alpha$]}:
In Figure~\ref{fig:sythetic_exp_time}(a), we observe that the VUS, VUS$_{opt}$, and VUS$_{opt}^{mem}$ execution times are linearly increasing with $\alpha$. 
The increase in execution time for VUS, VUS$_{opt}$, and VUS$_{opt}^{mem}$ is more pronounced when we vary $\alpha$, in contrast to $l_a$ (which nevertheless, has a similar effect on the overall complexity). 
We also observe that the VUS$_{opt}^{mem}$ execution time grows slower than $VUS_{opt}$ when $\alpha$ increases. 
This is explained by the use of 2-dimensional arrays for the storage of predictions, which use contiguous memory locations that allow for faster access, decreasing the dependency on $\alpha$.

\vspace{0.2cm}
\noindent {\bf [Influence of $\mu(\ell_a)$]}:
As shown in Figure~\ref{fig:sythetic_exp_time}(b), the execution time variation of VUS, VUS$_{opt}$, and VUS$_{opt}^{mem}$ caused by $\ell_a$ is rather insignificant. 
We also observe that the VUS$_{opt}$ and VUS$_{opt}^{mem}$ execution times are significantly lower when compared to VUS. 
This is explained by the smaller dependency of the complexity of these algorithms on the time series length $|T|$. 
Overall, the execution time for both VUS$_{opt}$ and VUS$_{opt}^{mem}$ is significantly lower than VUS, and follows a similar trend. 

\vspace{0.2cm}
\noindent {\bf [Influence of $\sigma(\ell_a)$]}: 
As depicted in Figure~\ref{fig:sythetic_exp_time}(d) and inferred from the theoretical complexities in Table~\ref{tab:complexity_summary}, none of the measures are affected by the standard deviation of the anomaly lengths.

\vspace{0.2cm}
\noindent {\bf [Influence of $|T|$]}:
For short time series (small values of $|T|$), we note that O($T_1$) becomes comparable to O($T_2$). 
Thus, the theoretical complexities approximate to $O(NL(T_1+T_2))$, $O(N*(T_1+T_2))+O(NLT_2)$ and $O(N(T_1+T_2))$ for VUS, VUS$_{opt}$, and VUS$_{opt}^{mem}$, respectively. 
Indeed, we observe in Figure~\ref{fig:sythetic_exp_time}(c) that the execution times of VUS, VUS$_{opt}$, and VUS$_{opt}^{mem}$ are similar for small values of $|T|$. However, for larger values of $|T|$, $O(T_1)$ is much higher compared to $O(T_2)$, thus resulting in an effective complexity of $O(NLT_1)$ for VUS, and $O(NT_1)$ for VUS$_{opt}$, and VUS$_{opt}^{mem}$. 
This translates to a significant improvement in execution time complexity for VUS$_{opt}$ and VUS$_{opt}^{mem}$ compared to VUS, which is confirmed by the results in Figure~\ref{fig:sythetic_exp_time}(c).

\vspace{0.2cm}
\noindent {\bf [Influence of $N$]}: 
Given the theoretical complexity depicted in Table~\ref{tab:complexity_summary}, it is evident that the number of thresholds affects all measures in a linear fashion.
Figure~\ref{fig:sythetic_exp_time}(e) demonstrates this point: the results of varying $N$ show a linear dependency for VUS, VUS$_{opt}$, and VUS$_{opt}^{mem}$ (i.e., a logarithmic trend with a log scale on the y axis). \commentRed{Moreover, we observe that the AUC and range-AUC execution time is almost constant regardless of the number of thresholds used. The latter is explained by the very efficient implementation of AUC measures. Therefore, the linear dependency on the number of thresholds is not visible in Figure~\ref{fig:sythetic_exp_time}(e).}

\vspace{0.2cm}
\noindent {\bf [Influence of $L$]}: Figure~\ref{fig:sythetic_exp_time}(f) depicts the influence of the maximum buffer length $L$ on the execution time of all measures. 
We observe that, as $L$ grows, the execution time of VUS$_{opt}$ and VUS$_{opt}^{mem}$ increases slower than VUS. 
We also observe that VUS$_{opt}^{mem}$ is more scalable with $L$ when compared to VUS$_{opt}$. 
This is consistent with the theoretical complexity (cf. Table~\ref{tab:complexity_summary}), which indicates that the dependence on $L$ decreases from $O(NL(T_1+T_2+\ell_a \alpha))$ for VUS to $O(NL(T_2+\ell_a \alpha)$ and $O(NL(\ell_a \alpha))$ for $VUS_{opt}$, and $VUS_{opt}^{mem}$.





\begin{figure*}[tb]
  \centering
  \includegraphics[width=\linewidth]{figures/synthetic_res.pdf}
  %\vspace*{-0.5cm}
  \caption{Execution time of VUS, R-AUC, AUC-based measures when we vary the parameters listed in Table~\ref{tab:parameter_range_time}. The solid lines correspond to the average execution time over 10 runs. The colored envelopes are to the standard deviation.}
  \label{fig:sythetic_exp_time}
\end{figure*}


\vspace{0.2cm}
In order to obtain a more accurate picture of the influence of each of the above parameters, we fit the execution time (as affected by the parameter values) using linear regression; we can then use the regression slope coefficient of each parameter to evaluate the influence of that parameter. 
In practice, we fit each parameter individually, and report the regression slope coefficient, as well as the coefficient of determination $R^2$.
Table~\ref{tab:parameter_linear_coeff} reports the coefficients mentioned above for each parameter associated with VUS, VUS$_{opt}$, and VUS$_{opt}^{mem}$.



\begin{table}[tb]
    \centering
    \caption{Linear regression slope coefficients ($C.$) for VUS execution times, for each parameter independently. }
    \begin{tabular}{|c|c|c|c|c|c|c|} 
 \hline
 Measure & Param. & $\alpha$ & $l_a$ & $|T|$ & $L$ & $N$\\ [0.5ex] 
 \hline\hline
 \multirow{2}{*}{$VUS$} & $C.$ & 21.9 & 0.02 & 2.13 & 212 & 6.24\\\cline{2-7}
 & {$R^2$} & 0.99 & 0.15 & 0.99 & 0.99 & 0.99 \\   
 \hline
  \multirow{2}{*}{$VUS_{opt}$} & $C.$ & 24.2  & 0.06 & 0.19 & 27.8 & 1.23\\\cline{2-7}
  & $R^2$& 0.99 & 0.86 & 0.99 & 0.99 & 0.99\\ 
 \hline
 \multirow{2}{*}{$VUS_{opt}^{mem}$} & $C.$ & 21.5 & 0.05 & 0.21 & 15.7 & 1.16\\\cline{2-7}
  & $R^2$ & 0.99 & 0.89 & 0.99 & 0.99 & 0.99\\[1ex] 
 \hline
\end{tabular}
    \label{tab:parameter_linear_coeff}
\end{table}

Table~\ref{tab:parameter_linear_coeff} shows that the linear regression between $\alpha$ and the execution time has a $R^2=0.99$. Thus, the dependence of execution time on $\alpha$ is linear. We also observe that VUS$_{opt}$ execution time is more dependent on $\alpha$ than VUS and VUS$_{opt}^{mem}$ execution time.
Moreover, the dependence of the execution time on the time series length ($|T|$) is higher for VUS than for VUS$_{opt}$ and VUS$_{opt}^{mem}$. 
More importantly, VUS$_{opt}$ and VUS$_{opt}^{mem}$ are significantly less dependent than VUS on the number of thresholds and the maximal buffer length. 







\subsubsection{Evaluation on TSB-UAD Time Series}\hfill\\
\label{sec:TSB_eval_time}

In this section, we verify the conclusions outlined in the previous section with real-world time series from the TSB-UAD benchmark. 
In this setting, the parameters $\alpha$, $\ell_a$, and $|T|$ are calculated from the series in the benchmark and cannot be changed. Moreover, $L$ and $N$ are parameters for the computation of VUS, regardless of the time series (synthetic or real). Thus, we do not consider these two parameters in this section.

\begin{figure*}[tb]
  \centering
  \includegraphics[width=\linewidth]{figures/TSB2.pdf}
  \caption{Execution time of VUS, R-AUC, AUC-based measures on the TSB-UAD benchmark, versus $\alpha$, $\ell_a$, and $|T|$.}
  \label{fig:TSB}
\end{figure*}

Figure~\ref{fig:TSB} depicts the execution time of AUC, R-AUC, and VUS-based measures versus $\alpha$, $\mu(\ell_a)$, and $|T|$.
We first confirm with Figure~\ref{fig:TSB}(a) the linear relationship between $\alpha$ and the execution time for VUS, VUS$_{opt}$ and VUS$_{opt}^{mem}$.
On further inspection, it is possible to see two separate lines for almost all the measures. 
These lines can be attributed to the time series length $|T|$. 
The convergence of VUS and $VUS_{opt}$ when $\alpha$ grows shows the stronger dependence that $VUS_{opt}$ execution time has on $\alpha$, as already observed with the synthetic data (cf. Section~\ref{sec:synthetic_eval_time}). 

In Figure~\ref{fig:TSB}(b), we observe that the variation of the execution time with $\ell_a$ is limited when compared to the two other parameters. We conclude that the variation of $\ell_a$ is not a key factor in determining the execution time of the measures.
Furthermore, as depicted in Figure~\ref{fig:TSB}(c), $VUS_{opt}$ and $VUS_{opt}^{mem}$ are more scalable than VUS when $|T|$ increases. 
We also confirm the linear dependence of execution time on the time series length for all the accuracy measures, which is consistent with the experiments on the synthetic data. 
The two abrupt jumps visible in Figure~\ref{fig:TSB}(c) are explained by significant increases of $\alpha$ in time series of the same length. 

\begin{table}[tb]
\centering
\caption{Linear regression slope coefficients ($C.$) for VUS execution time, for all time series parameters all-together.}
\begin{tabular}{|c|ccc|c|} 
 \hline
Measure & $\alpha$ & $|T|$ & $l_a$ & $R^2$ \\ [0.5ex] 
 \hline\hline
 \multirow{1}{*}{${VUS}$} & 7.87 & 13.5 & -0.08 & 0.99  \\ 
 %\cline{2-5} & $R^2$ & \multicolumn{3}{c|}{ 0.99}\\
 \hline
 \multirow{1}{*}{$VUS_{opt}$} & 10.2 & 1.70 & 0.09 & 0.96 \\
 %\cline{2-5} & $R^2$ & \multicolumn{3}{c|}{0.96}\\
\hline
 \multirow{1}{*}{$VUS_{opt}^{mem}$} & 9.27 & 1.60 & 0.11 & 0.96 \\
 %\cline{2-5} & $R^2$ & \multicolumn{3}{c|}{0.96} \\
 \hline
\end{tabular}
\label{tab:parameter_linear_coeff_TSB}
\end{table}



We now perform a linear regression between the execution time of VUS, VUS$_{opt}$ and VUS$_{opt}^{mem}$, and $\alpha$, $\ell_a$ and $|T|$.
We report in Table~\ref{tab:parameter_linear_coeff_TSB} the slope coefficient for each parameter, as well as the $R^2$.  
The latter shows that the VUS$_{opt}$ and VUS$_{opt}^{mem}$ execution times are impacted by $\alpha$ at a larger degree than $\alpha$ affects VUS. 
On the other hand, the VUS$_{opt}$ and VUS$_{opt}^{mem}$ execution times are impacted to a significantly smaller degree by the time series length when compared to VUS. 
We also confirm that the anomaly length does not impact the execution time of VUS, VUS$_{opt}$, or VUS$_{opt}^{mem}$.
Finally, our experiments show that our optimized implementations VUS$_{opt}$ and VUS$_{opt}^{mem}$ significantly speedup the execution of the VUS measures (i.e., they can be computed within the same order of magnitude as R-AUC), rendering them practical in the real world.











\subsection{Summary of Results}


Figure~\ref{fig:overalltable} depicts the ranking of the accuracy measures for the different tests performed in this paper. The robustness test is divided into three sub-categories (i.e., lag, noise, and Normal vs. abnormal ratio). We also show the overall average ranking of all accuracy measures (most right column of Figure~\ref{fig:overalltable}).
Overall, we see that VUS-ROC is always the best, and VUS-PR and Range-AUC-based measures are, on average, second, third, and fourth. We thus conclude that VUS-ROC is the overall winner of our experimental analysis.

\commentRed{In addition, our experimental evaluation shows that the optimized version of VUS accelerates the computation by a factor of two. Nevertheless, VUS execution time is still significantly slower than AUC-based approaches. However, it is important to mention that the efficiency of accuracy measures is an orthogonal problem with anomaly detection. In real-time applications, we do not have ground truth labels, and we do not use any of those measures to evaluate accuracy. Measuring accuracy is an offline step to help the community assess methods and improve wrong practices. Thus, execution time should not be the main criterion for selecting an evaluation measure.}


\section{Conclusion}
This paper studies reward transfer in the context of online RLHF.
We contribute $\TPO$, a provable and efficient transfer learning algorithm that leverages the structure induced by the KL regularizer.
Based on that, we further develop a UCB-based empirical alternative and evaluate its effectiveness through LLM experiments.
Several promising directions remain for future exploration.
Firstly, an interesting avenue is to develop transfer learning strategies beyond RLHF setting, for example, the Nash Learning from Human Feedback setting.
Secondly, while we focus on policy-level transfer, a finer-grained prompt-wise knowledge transfer may be possible, which allows transfer from different policies in different states.
Thirdly, due to resource limitations, we leave the examination of our methods in fine-tuning much larger-scale language models to the future work.


%
%
%
%
%
%
%
%
%
%
%
%

\newpage
\section*{Acknowledgement}
The work is supported by ETH research grant and Swiss National Science Foundation (SNSF) Project Funding No. 200021-207343 and SNSF Starting Grant.
%
%

\section*{Impact Statement}
This paper presents work whose goal is to advance the field of 
Machine Learning. There are many potential societal consequences 
of our work, none which we feel must be specifically highlighted here.

\section*{Reproducibility Statement}
The code of all the experiments and the running instructions can be found in \url{https://github.com/jiaweihhuang/RLHF_RewardTransfer}.

\bibliography{references}
\bibliographystyle{icml2025}

\newpage
\appendix
\onecolumn

\section*{Outline of the Appendix}
\begin{itemize}
    \item Appx.~\ref{appx:freq_notations}: Frequently Used Notation.
    \item Appx.~\ref{appx:missing_details}: Missing Details in the Main Text.
    \item Appx.~\ref{appx:adaption_offline}: Offline Learning Results in Previous Literature.
    \item Appx.~\ref{appx:online_oracle}: Verification for Online Learning Oracle Example in Sec.~\ref{sec:main_theory}.
    \item Appx.~\ref{appx:coverage_related}: Proofs for Results in Section~\ref{sec:transfer_coverage_perspective}.
    \item Appx.~\ref{appx:proof_task_selection}: Proofs for the Main Algorithm and Results in Sec.~\ref{sec:main_theory}.
    \item Appx.~\ref{appx:win_rate_and_coverage}: Connection between Win Rate and Policy Coverage Coefficient.
    \item Appx.~\ref{appx:basic_lemma}: Useful Lemmas.
    \item Appx.~\ref{appx:experiment}: Missing Experiment Details.
\end{itemize}
\newpage

\section{Frequently Used Notation}\label{appx:freq_notations}

\begin{table}[h]
    \centering
    \def\arraystretch{1.2}
    %
    \begin{tabular}{ll}
        \hline
        \textbf{Notation} & \textbf{Description} \\
        \hline
        $\cS,\cA$ & State space and action space \\
        $\rho$ & Prompt distribution (initial state distribution) \\
        $r$ & Reward model \\
        $r^*$ & Ground-truth reward model (reflecting human preferences) \\
        $\mP_r(y|s,a,a')$ & Preference under $r$ \\
        $\mP_r(\pi\succ\tpi)$ & Win rate of $\pi$ over $\tpi$ under $r$ \\
        $\{r^w\}_{w\in[W]}$ & Imperfect source reward model \\
        $\pi$ & LLM policy \\
        $\pi^t_\mix$ & \makecell[tl]{Uniform mixture policy $\frac{1}{t}\sum_{i\leq t}\pi^i$ of a policy sequence $\pi^1,...,\pi^t$. \\ Sometimes, given a dataset $\cD=\{(x^i,\pi^i)\}_{i\leq |\cD|}$, with a bit abuse of notation, \\ we use $\pi^\cD_\mix$ to refer the mixture policy $\frac{1}{|\cD|}\sum_{i\leq|\cD|} \pi^i$.} \\
        $\cov^{\tpi|\pi}$ & Coverage coefficient \\
        %
        $\Pi$ & The policy class \\
        $\cR^\Pi$ & The reward function class converted from $\Pi$, see Appx.~\ref{appx:extend_prelim}\\
        $\conv(\Pi)$ & Convex hull of $\Pi$ \\
        $\beta$ & Regularization coefficient in RLHF objective \\
        $J_\beta(\cdot)$ & Regularized policy value (Eq.~\eqref{eq:rlhf_obj})\\
        $\Delta(w)$ & Value gap for $\pi^*_{r^w}$, i.e. $J_\beta(\pi^*_{r^*}) - J_\beta(\pi^*_{r^w})$\\
        $\Delta_{\min}$ & Minimal value gap $\min_{w\in[W]} \Delta(w)$ \\
        $a \wedge b$ & $\min\{a,b\}$ \\
        $[n]$ & $\{1,2,...,n\}$ \\
        $O(\cdot),\Omega(\cdot),\Theta(\cdot),\tilde{O}(\cdot),\tilde{\Omega}(\cdot),\tilde{\Theta}(\cdot)$ & Standard Big-O notations, $\tilde{(\cdot)}$ omits the log terms.\\
        \hline
        
    \end{tabular}
\end{table}

For completeness, we provide the definition of convex hull here.
\begin{definition}[Convex Hull]\label{def:convex_hull}
    Given a policy class $\Pi$ with finite cardinality (i.e. $|\Pi| < +\infty$), we denote $\conv(\Pi)$ as its convex hull, such that, $\forall n \in [\mN^*],~\forall \lambda^1,...,\lambda^n \geq 0$ with $\sum_{i=1}^n \lambda^i = 1$, and any $\pi^1,...,\pi^n \in \Pi$, we have:
    \begin{align*}
        \sum_{i=1}^n \lambda^i \pi^i \in \conv(\Pi).
    \end{align*}
\end{definition}


\begin{remark}
    Note that in the contextual bandit setting, the state action density induced by a policy and the policy distribution collapse with each other.
    Therefore, given a policy sequence $\pi^1,...,\pi^t$, the uniform mixture policy $\pi_\mix^t(\cdot|\cdot) = \frac{1}{t}\sum_{i\leq t} \pi^t(\cdot|\cdot)$ is directly a valid policy as a mapping from $\cS$ to $\Delta(\cA)$, which induces the state-action density $\pi_\mix^t(\cdot|\cdot)$.

    Besides, we will use $\Online$ and $\Offline$ as abbreviations of ``online learning'' and ``offline learning'', respectively.
    %
\end{remark}


\section{Missing Details in the Main Text}\label{appx:missing_details}

\subsection{Extended Preliminary}\label{appx:extend_prelim}
\paragraph{More Elaborations on the Necessity of Regularization} The RLHF objective Eq.~\eqref{eq:rlhf_obj} typically involves a regularization term $\beta\neq 0$. This regularization is critical in practice for several reasons.
Firstly, it prevents overfitting to human preferences, which can possibly be noisy and biased \citep{gao2023scaling, ouyang2022training}.
%
Moreover, pure reward maximization prefers near-deterministic policies, potentially causing mode collapse. In contrast, regularization encourages the fine-tuned model to retain diversity from the reference policy \citep{jaques2017sequence, jaques2019way}.
Thirdly, reference policies are pretrained on a significantly larger corpus than the post-training data, enabling them to encode more general-purpose knowledge. Regularization helps mitigate catastrophic forgetting, ensuring the model retains this broad knowledge base.

\paragraph{Formal Definition for $\cR^\Pi$}
Given a policy class $\Pi$ satisfying Assump.~\ref{assump:policy}, we use $\cR^{\Pi}$ to denote the reward function class converted from $\Pi$, such that (1) $\forall r\in\cR^\Pi$, $r(\cdot,\cdot)\in[0, R]$; (2) $\exists r\in\cR^{\Pi}, \pi_r^* = \pi^*_{r^*}$.
A possible construction satisfying this is given by
\begin{align}
    \cR^{\Pi} := \{r_{|\pi}|r_{|\pi}(s,a):=\text{Clip}_{[0,{\Rmax}]}[\beta\log\frac{\pi(a|s)}{\pi_\textref(a|s)} - \min_{a'\in\cA}\beta \log \frac{\pi(a'|s)}{\pi_\textref(a'|s)}],~\pi\in\Pi\}.\label{eq:reward_class_conversion}
\end{align}
The rationale behind such a construction lies in that $r_{|\pi^*_{r^*}}$ provably differs from $r^*$ by at most of an action-independent constant under Assump.~\ref{assump:policy}.
We prove this in the following.

For any $s\in\cS$, we denote $a_s := \argmin_{a'\in\cA} \log \frac{\pi^*_{r^*}(a'|s)}{\pi_\textref(a'|s)}$.
According to Eq.~\eqref{eq:closed_form} and the fact that $r^* \in [0, {\Rmax}]$, for any $s\in\cS$ and $a,a'\in\cA$, we have:
\begin{align*}
    0 \leq \beta\log\frac{\pi^*_{r^*}(a|s)}{\pi_\textref(a|s)} - \min_{a'\in\cA}\beta \log \frac{\pi^*_{r^*}(a'|s)}{\pi_\textref(a'|s)} = r^*(s,a) - r^*(s,a_s) \leq {\Rmax},
\end{align*}
where the first inequality is because $a_s$ takes the minimal over $\cA$.
Therefore, $r_{|\pi^*_{r^*}}(s,a) \in [0, {\Rmax}]$ and $r_{|\pi^*_{r^*}}(s,a) - r^*(s,a) = r^*(s,a_s)$, which is action-independent.
In another word, $r_{|\pi^*_{r^*}}$ induces the same optimal policy $\pi^*_{r^*}$.

Under the objective in Eq.~\eqref{eq:rlhf_obj}, the realizability assumption in \citep{liu2024provably} can be relaxed and the reward model class $\cR^{\Pi}$ can be used in their $\RPO$ objective. Because any per-state action-independent shift on the reward space does not change the induced policy in Eq.~\eqref{eq:closed_form}.
As a result, throughout this paper, we will not distinguish between $r^*$ and $r_{|\pi^*_{r^*}}$.

\paragraph{Remarks on Assumption~\ref{assump:policy}-(II)}
\begin{lemma}\label{lem:bounded_ratio}
    If $r^*(s,a)\in[0, R]$ for all $(s,a)\in\cS\times\cA$, we have:
    \begin{align*}
        \max_{s\in\cS,a\in\cA} |\log\frac{\pi^*_{r^*}(a|s)}{\pi_{\textref}(a|s)}| \leq \frac{R_{\max}}{\beta}.
    \end{align*}
\end{lemma}
\begin{proof}
    By definition, for any $s$, 
    \begin{align*}
        \forall a\in\cA,\quad \pi^*_{r^*}(a|s) = \pi_{\textref}(a|s) e^{\frac{r^*(s,a)}{\beta}} / Z(s),
    \end{align*}
    where $Z(s) := \sum_{a\in\cA} \pi_{\textref}(a|s) e^{\frac{r^*(s,a)}{\beta}}$.
    Because $r^*(s,a) \in [0, \Rmax]$, obviously, $1 \leq Z(s) \leq e^{\frac{\Rmax}{\beta}}$. Therefore,
    \begin{align*}
        \forall a\in\cA,\quad |\log\frac{\pi^*_{r^*}(a|s)}{\pi_{\textref}(a|s)}| = |\frac{r^*(s,a)}{\beta} - \log Z(s)| \leq \max\{\frac{\Rmax}{\beta} - \log Z(s), \log Z(s)\} \leq \frac{\Rmax}{\beta}.
    \end{align*}
\end{proof}

\subsection{Other Related Works}\label{appx:related_workds}

\paragraph{Other Related RLHF Literature}
%
Various approaches have been developed for reward-model-free online exploration. For example, DPO \citep{rafailov2024direct} implicitly optimizes the same objective as RLHF without explicit reward modeling. DPO is further extended to different settings; see e.g., online DPO \citep{guo2024direct}, iterative DPO \citep{xu2023some, pang2024iterative, dong2405rlhf}, etc.

%

Another direction is to go beyond Bradley-Terry reward model assumption. A particularly promising set of techniques formulates RLHF as a two-player zero-sum game \citep{yue2012k}, aiming to select policies preferred by the rater to others \citep{rosset2024direct, ye2024theoretical, munos2023nash, swamy2024minimaximalist}.
Investigating knowledge transfer within this framework is an exciting direction for future work.

\paragraph{RL Theory in Pure-Reward Maximization Setting}~
In the classical pure-reward maximization RL setting, sample efficiency is a central topic, with extensive research dedicated to strategic exploration and fundamental complexity measures for online learning \citep{russo2013eluder, jiang2017contextual, jin2021bellman, foster2021statistical, du2021bilinear}.

Besides the literature already mentioned in Sec.~\ref{sec:background_policy_coverage}, there is a rich literature \citep{uehara2020minimax,jiang2020minimax,jin2021pessimism,xie2021bellman} investigating the role of policy coverage (or density ratio) in offline learning.

\paragraph{Regularized RL}
Sample complexity in regularized RL has also been studied in previous works \citep{ziebart2008maximum,ziebart2010modeling,geist2019theory,tiapkin2023fast}.
Nonetheless, most of them focus on tabular settings and do not consider the transfer learning.



%
%
%
%
%


%
%
%


%
%

\section{Offline Learning Results in Previous Literature}\label{appx:adaption_offline}
In this section, we recall and adapt some results from \citep{liu2024provably}, which are useful for proofs in other places.
\paragraph{$\RPO$ Optimization Objective}
For completeness, we provide the optimization objective of $\RPO$.
Given a policy class $\tPi$ and a reward function class $\cR$, the $\RPO$ objective solves a mini-max optimization problem defined as follows:
\begin{align}
    \RPO(\tPi,\cR,\cD,\eta) = \arg\max_{\pi\in\tPi}\min_{r\in\cR} L_{\cD}(r) + \eta \EE_{s\sim\rho,a\sim\pi,\ta\sim\pi_\textref}[r(s,a)-r(s,\ta)] - \beta \KL(\pi\|\pi_\textref), \label{eq:RPO_objective} 
\end{align}
where we choose $\pi_\textref$ as the base policy in \citep{liu2024provably}.
In Alg.~\ref{alg:transfer_policy_computing}, we set $\tPi = \conv(\Pi)$ and $\cR = \cR^\Pi$.
\begin{condition}[Sequential Data Generation]\label{cond:seq_data}
    We say a dataset $\cD := \{(s^i,a^i,\ta^i,y^i,\pi^i)\}_{i\leq |\cD|}$ is generated sequentially, if it is generated following:
    \begin{align*}
        \forall i\leq |\cD|,\quad & \pi^i \sim \text{Alg}(\cdot|\{(s^j,a^j,\ta^j,y^j,\pi^j)\}_{j<i}),\\
        & s^i\sim\rho,~a^i\sim\pi^i(\cdot|s^i),~\ta^i\sim\pi_\textref(\cdot|s^i),~y^i\sim \mP_{r^*}(\cdot|s^i,a^i,\ta^i),
    \end{align*}
    where $\text{Alg}$ denotes an algorithm computing the next policy only with the interaction history.
\end{condition}


%
\begin{restatable}{lemma}{LemOfflineLearning}\label{lem:offline_learning}[Adapted from Thm.~5.3 in \citep{liu2024provably}]
    Under Assump.~\ref{assump:policy}, given any $\delta \in (0,1)$, by running $\RPO$ (Eq.~\eqref{eq:RPO_objective}) with $\conv(\Pi), \cR^{\Pi}, \delta$ and a dataset $\cD := \{(s^i,a^i,\ta^i,y^i,\pi^i)\}_{i\leq |\cD|}$ satisfying Cond.~\ref{cond:seq_data}, by choosing $\eta = (1+e^{{\Rmax}})^2 \sqrt{24|\cD|\log\frac{|\Pi|}{\delta}}$, we have:
    \begin{align*}
        \forall \pi \in \conv(\Pi),\quad J_\beta(\pi) - J_\beta(\pi_\SELF) \leq C_\Offline e^{2{\Rmax}}\cdot \cov^{\pi|\pi_\mix^\cD}\sqrt{\frac{1}{|\cD|}\log\frac{|\Pi|}{\delta}},
    \end{align*}
    where we use $\pi_\mix^\cD := \frac{1}{|\cD|}\sum_{i\leq |\cD|} \pi^{i}$ as a short note of the uniform mixture policy.
\end{restatable}
\begin{proof}
    The main difference comparing with \citep{liu2024provably} is that we consider sequentially generated dataset while they study dataset generated by a fixed dataset distribution.
    In the following, we show how to extend their results to our setting.

    Firstly, we check the assumptions. Note that we consider feed $\RPO$ \citep{liu2024provably} by the reward function class $\cR^{\Pi}$ converted from a policy class $\Pi$ satisfying Assump.~\ref{assump:policy}, through Eq.~\eqref{eq:reward_class_conversion}. 
    Therefore, the optimal reward is also realizabile in $\cR^{\Pi}$, and the basic assumptions required by $\RPO$ \citep{liu2024provably} are satisfied.

    %
    %
    %
    %
    %
    %
    %
    %
    %
    %
    %
    %
    %

    Next, we adapt the proofs in \citep{liu2024provably}. Note that we can directly start with their Eq.~(D.4), because their bounds in Eq.~(D.2) and Eq.~(D.3) only involve optimality of the choice of $\pi_\SELF$ and realizability.
    We move the KL-regularization terms to the LHS and merge to $J_\beta(\pi)$ and $J_\beta(\pi_\SELF)$, and we choose $\pi_\textref$ as the base policy in $\RPO$. The adapted results to our notations would be:
    \begin{align*}
        \forall \pi \in \conv(\Pi),~ J_\beta(\pi) & - J_\beta(\pi_\SELF) \\
        \leq & \max_{r\in\cR^{\Pi}} \EE_{s\sim\rho,a\sim\pi(\cdot|s),\ta\sim\pi_\textref(\cdot|s)}[(r^*(s,a) - r^*(s,\ta)) - (r(s,a) - r(s,\ta))] \\
        & + \eta^{-1} (\cL_{\cD}(r^*) - \cL_{\cD}(r)).
    \end{align*}
    %
    Recall $\cL_\cD$ is the (unnormalized) negative log-likelihood (NLL) loss, defined in Eq.~\eqref{eq:def_likelihood}.
    Since the dataset $\cD$ is generated sequentially (Cond.~\ref{cond:seq_data}), we can apply the concentration results in Lem.~\ref{lem:MLE_Estimation}, which is a variant of Lemma D.1 in \citep{liu2024provably} for sequentially generated data:
    \begin{align*}
        \text{w.p.}~1-\delta,\quad \forall \pi\in \conv(\Pi),~ J_\beta(\pi) & - J_\beta(\pi_\SELF) \\
        \leq & \EE_{s\sim\rho,a\sim\pi(\cdot|s),\ta\sim\pi_\textref(\cdot|s)}[(r^*(s,a) - r^*(s,\ta)) - (r_{\gets\pi}(s,a) - r_{\gets\pi}(s,\ta))] \\
        & + \eta^{-1} (\cL_{\cD}(r^*) - \cL_{\cD}(r_{\gets\pi})) \\
        \leq &  \EE_{s\sim\rho,a\sim\pi(\cdot|s),\ta\sim\pi_\textref(\cdot|s)}[(r^*(s,a) - r^*(s,\ta)) - (r_{\gets\pi}(s,a) - r_{\gets\pi}(s,\ta))] \\
        & - \frac{1}{\eta|\cD|}\sum_{i \leq |\cD|} \EE_{s\sim\rho,a\sim\pi^i(\cdot|s),\ta\sim\pi_\textref(\cdot|s)}[\mH^2(\mP_{r_{\gets\pi}}(\cdot|s,a,\ta)\| \mP_{r^*}(\cdot|s,a,\ta))] + \frac{2}{\eta|\cD|}\log\frac{|\Pi|}{\delta}\\
        = &  \EE_{s\sim\rho,a\sim\pi(\cdot|s),\ta\sim\pi_\textref(\cdot|s)}[(r^*(s,a) - r^*(s,\ta)) - (r_{\gets\pi}(s,a) - r_{\gets\pi}(s,\ta))] \\
        & - \frac{1}{\eta} \EE_{s\sim\rho,a\sim\pi^\cD_\mix(\cdot|s),\ta\sim\pi_\textref(\cdot|s)}[\mH^2(\mP_{r_{\gets\pi}}(\cdot|s,a,\ta)\| \mP_{r^*}(\cdot|s,a,\ta))] + \frac{2}{\eta|\cD|}\log\frac{|\Pi|}{\delta},
    \end{align*}
    where we denote $r_{\gets\pi} := \argmax_{r\in\cR^{\Pi}} \EE_{s\sim\rho,a\sim\pi(\cdot|s),\ta\sim\pi_\textref(\cdot|s)}[(r^*(s,a) - r^*(s,\ta)) - (r(s,a) - r(s,\ta))]$.

    The rest of the proofs in \citep{liu2024provably} can be adapted here, and by choosing $\eta = (1 + e^{{\Rmax}})^{-2} \sqrt{\frac{24}{|\cD|}\log\frac{|\Pi|}{\delta}}$, we can inherit the following guarantee:
    \begin{align*}
        \text{w.p.}~1-\delta,\quad \forall \pi\in \conv(\Pi),~ J_\beta(\pi) - J_\beta(\pi_\SELF) \leq \frac{\sqrt{6}}{4}\cdot (1+e^{{\Rmax}})^2(C_{\pi_\mix^\cD}(\cR^{\Pi};\pi;\pi_\textref)^2 + 1) \sqrt{\frac{1}{|\cD|} \log\frac{|\Pi|}{\delta}}.
    \end{align*}
    Here $C_{\pi_\mix^\cD}(\cR^{\Pi};\pi;\pi_\textref)$ is the coverage coefficient (adapted from Assump. 5.2 \citep{liu2024provably}) with $\pi_\mix^\cD$, which can be upper bounded by:
    \begin{align*}
        C_{\pi_\mix^\cD}(\cR^{\Pi};\pi;\pi_\textref) \leq & \max_{r\in\cR^{\Pi}}\frac{\EE_{s\sim\rho,a\sim\pi(\cdot|s),\ta\sim\pi_\textref(\cdot|s)}[|(r^*(s,a) - r^*(s,\ta)) - (r(s,a) - r(s,\ta))|]}{\sqrt{\EE_{s\sim\rho,a\sim\pi_\mix^\cD(\cdot|s),\ta\sim\pi_\textref(\cdot|s)}[|(r^*(s,a) - r^*(s,\ta)) - (r(s,a) - r(s,\ta))|^2]}} \\
        \leq & \sqrt{\EE_{s\sim\rho,a\sim\pi(\cdot|s)}[\frac{\pi(a|s)}{\pi_\mix^\cD(a|s)}]} \tag{AM-GM inequality; Holds for any $r$ and therefore including the one achieves the maximum}\\
        = & \sqrt{\cov^{\pi|\pi_\mix^\cD}}.
    \end{align*}
    %
    Therefore, we finish the proof. We simplify the upper bound by using $\cov^{\pi|\pi_\mix^\cD} \geq 1$ and $e^{{\Rmax}} \geq 1$.
\end{proof}





\section{Details for Online Learning Oracle Example in Sec.~\ref{sec:main_theory}}\label{appx:online_oracle}

\begin{definition}[$L_\infty$ Coverability; \citep{xie2022role,xie2024exploratory}]\label{def:l_inf_coverage}
    The $L_\infty$ coverability is defined by:
    \begin{align*}
        \cov_\infty(\Pi) := \inf_{\mu\in\Delta(\cS)\times\Delta(\cA)} \sup_{\pi\in\Pi} \sup_{s\in\cS,a\in\cA} \frac{\pi(a|s)}{\mu(a|s)}
    \end{align*}
\end{definition}


\begin{definition}[No-Regret Online Algorithm]\label{def:online_oracle}
    Given any $\delta \in (0,1)$, iteration number $\tT$, and a policy class $\Pi$ satisfying Assump.~\ref{assump:policy}, the online learning algorithm $\AlgOnline$ iteratively computes policy to collect samples and conducts no-regret learning. W.p. $1-\delta$, it produces a sequence of online policies $\pi^1,...,\pi^{\tilde T}$, such that, $\forall t\in[\tilde T]$,
    \begin{align*}
        \sum_{i\leq t} J_\beta(\pi^*_{r^*}) - J_\beta(\pi^i)  
        \leq C_\Online {\Rmax} e^{2{\Rmax}}  \sqrt{\Complexity(\Pi) t \log^{c_0}\frac{|\Pi|\tT}{\delta}},
    \end{align*}
    where $C_\Online > 0$ and $c_0 \geq 1$ are absolute constants, and $\Complexity(\Pi)$ denotes some complexity measure for $\Pi$.
    %
\end{definition}


\begin{restatable}{proposition}{ExampleOnline}[Example for Online Oracle in Def.~\ref{def:online_oracle}]\label{example:online_oracle}
    The $\XPO$ algorithm in \citep{xie2024exploratory} can fulfill the requirements in Def.~\ref{def:online_oracle}.
\end{restatable}
\begin{proof}
    We start by generalizing Eq.(35) in the proof of Theorem 3.1 in \citep{xie2024exploratory} to all $t \in [\tT]$. 
    Note that we consider bandit setting so ${\Rmax}$ in \citep{xie2024exploratory} collapse with ${\Rmax}$.
    Suppose at the end of iteration $t$, $\XPO$ generated a sequence of policies $\pi^1,...,\pi^t$, we have:
    \begin{align*}
        &\frac{1}{t}\sum_{i=1}^t J_\beta(\pi^*_{r^*}) - J_\beta(\pi^i) \\
        %
        %
        %
        \leq & \frac{6{\Rmax}}{t} + \frac{\text{SEC}_{\text{RLHF}}(\Pi,t,\beta,\pi_\textref)}{2\eta t} + \frac{\eta}{2}{\Rmax}^2 + \frac{1}{t} \EE_{i=2}^t \EE_{s\sim\rho,a\sim\pi_\textref(\cdot|s)}[\beta \log\pi^t(a|s) - \beta \log \pi^*(a|s)] \\
        & + \frac{\eta}{2t} \sum_{i=2}^t (i-1) \EE_{s\sim\rho,a\sim\pi_\mix^{i-1}(\cdot|s),\ta\sim\textref(\cdot|s)}\Big[\Big(\beta\log\frac{\pi^i(a|s)}{\pi_\textref(a'|s)} - r^*(s,a) - \beta\log\frac{\pi^i(\ta|s)}{\pi_\textref(\ta|s)} + r^*(s,\ta) \Big)^2\Big],
    \end{align*}
    where we choose $\tpi^{(t)}$ in \citep{xie2024exploratory} to be $\pi_\textref$ and denote $\pi_\mix^{i-1} = \frac{1}{i-1}\sum_{j=1}^{i-1} \pi^j$.
    %
    %
    %
    %

    %
    %
    %
    %
    %
    %
    %
    %
    %
    %
    %
    %
    %
    %
    %
    Note that Lemma C.5 in \citep{xie2024exploratory} holds w.p. $1-\delta$ for all $t\in[\tT]$. Following their proofs till Eq.(43) in \citep{xie2024exploratory}, we can show that for any $t\in[\tT]$
    \begin{align*}
        \frac{1}{t}\sum_{i=1}^t J_\beta(\pi^*_{r^*}) - J_\beta(\pi^i) \leq O({\Rmax} e^{2{\Rmax}} \sqrt{\frac{\text{SEC}_{\text{RLHF}}(\Pi,t,\beta,\pi_\textref)}{t}\log\frac{|\Pi|T}{\delta}}),
    \end{align*}
    Based on the arguments in \citep{xie2024exploratory}, $\text{SEC}_{\text{RLHF}}(\Pi,t,\beta,\pi_\textref)$ can be controlled by $c_0 \cdot \cov_\infty^\Pi\log^{c_1}(|\Pi|t)$ for some absolute constant $c_0,c_1 > 0$. Here $\cov_\infty^\Pi$ is the $L_\infty$ coverability coefficient (Def.~\ref{def:l_inf_coverage}) and plays the role.
    Therefore, we finish the verification.
\end{proof}



\iffalse

\begin{proof}
    We start with $\cI^{(t)}$ defined in Eq.(36) in \citep{xie2024exploratory}.
    \begin{align*}
        \cI^{(t)} := \frac{(\EE_{s\sim\rho, a\sim\pi^t, a'\sim\pi_\textref}[f(s,a,a',r^*)])^2}{{\Rmax}^2 \vee (t-1) \EE_{s\sim\rho, (a, a')\sim\mu^t}[f(s,a,a',r^*)^2]},
    \end{align*}
    where $f(s,a,a',r^*) := \beta\log\frac{\pi^t(a|s)}{\pi_\textref(a'|s)} - r(s,a) - \beta\log\frac{\pi^t(a'|s)}{\pi_\textref(a'|s)} + r(s,a')$ as a short note.
    Adapt to our setting, we have $\max_{s,a,a'}|f(s,a,a',r^*)| \leq 4{\Rmax}$.

    Note that,
    \begin{align*}
        \cI^{(t)} \leq & \frac{2(\EE_{s\sim\rho, a\sim\pi^t, a'\sim\pi_\textref}[f(s,a,a',r)])^2}{{\Rmax}^2 + (t-1) \EE_{s\sim\rho, (a, a')\sim\mu^t}[f(s,a,a',r)^2]} \tag{$\frac{1}{a\vee b} \leq \frac{2}{a+b}$}\\
        \leq & \frac{2(\EE_{s\sim\rho, a\sim\pi^t, a'\sim\pi_\textref}[\frac{1}{\sqrt{\frac{1}{4} + (t-1)\mu^t(a,a'|s)}} \cdot \sqrt{\frac{1}{4} + (t-1)\mu^t(a,a'|s)} f(s,a,a',r)])^2}{{\Rmax}^2 + (t-1) \EE_{s\sim\rho, (a, a')\sim\mu^t}[f(s,a,a',r)^2]} \\
        \leq & \frac{2\EE_{s\sim\rho, a\sim\pi^t, a'\sim\pi_\textref}[\frac{\pi^t(a|s)\pi_\textref(a'|s)}{\frac{1}{4} + (t-1)\mu^t(a,a'|s)}] \cdot ({\Rmax}^2 + (t-1) \EE_{s\sim\rho, a, a'\sim\mu^t(a,a'|s)}[f^2(s,a,a',r)])}{{\Rmax}^2 + (t-1) \EE_{s\sim\rho, (a, a')\sim\mu^t}[f(s,a,a',r)^2]} \\
        = & 2\EE_{s\sim\rho, a\sim\pi^t, a'\sim\pi_\textref}[\frac{\pi^t(a|s)\pi_\textref(a'|s)}{\frac{1}{4} + (t-1)\mu^t(a,a'|s)}] \\
        \leq & 2\EE_{s\sim\rho, a\sim\pi^t, a'\sim\pi_\textref}[\frac{\pi^t(a|s)\pi_\textref(a'|s)}{\frac{1}{4}\pi_\textref(a'|s) + (t-1)\mu^t(a,a'|s)}] \\
        = & 2\EE_{s\sim\rho, a\sim\pi^t}[\frac{\pi^t(a|s)}{\frac{1}{4} + \sum_{i<t}\pi^i(a|s)}].
    \end{align*}
    Given a $a\in\cA$, we denote 
    $$
        \text{t}(s,a) = \min\{t|\sum_{i<t}\pi^i(a|s) \geq \cov \cdot \bpi(a|s)\}
    $$
    Then we have:
    \begin{align*}
        \frac{1}{2}\sum_{t=1}^T \cI^{(t)} \leq & \sum_{t=1}^T\EE_{s\sim\rho, a\sim\pi^t}[\frac{\pi^t(a|s)}{\frac{1}{4} + \sum_{i<t}\pi^i(a|s)}] \\
        =&\sum_{t=1}^T\EE_{s\sim\rho, a\sim\pi^t}[\frac{\pi^t(a|s) \mI[t < \text{t}(s,a)]}{\frac{1}{4} + \sum_{i<t}\pi^i(a|s)}] + \sum_{t=1}^T\EE_{s\sim\rho, a\sim\pi^t}[\frac{\pi^t(a|s)\mI[t \geq \text{t}(s,a)]}{\frac{1}{4} + \sum_{i<t}\pi^i(a|s)}].
    \end{align*}
    For the first term,
    \begin{align*}
        & \sum_{t=1}^T\EE_{s\sim\rho, a\sim\pi^t}[\frac{\pi^t(a|s) \mI[t < \text{t}(s,a)]}{\frac{1}{4} + \sum_{i<t}\pi^i(a|s)}] \\
        \leq & \sum_{t=1}^T 4 \EE_{s\sim\rho, a\sim\pi^t}[\mI[t < \text{t}(s,a)]] \\
        =& 4 \sum_{s,a} \sum_{t=1}^{\text{t}(s,a)-2} \rho(s)\pi^i(a|s) + 4 \sum_{s,a} \rho(s) \pi^{\text{t}(s,a)-1}(a|s) \\
        \leq & 4 \cov \sum_{s,a}\rho(s) \bpi(a|s) + 4 \sum_{s}\rho(s) \sum_a \frac{\pi^{\text{t}(s,a)-1}(a|s)}{\sqrt{\bpi(a|s)}}\sqrt{\bpi(a|s)} \\
        \leq & 4\cov + 4\cov^{}
    \end{align*}


    We denote $\nu^t(\cdot,\cdot|\cdot) := \frac{1}{4} + (t-1)\mu^t(\cdot,\cdot|\cdot)$ as a short note. As a result, $\nu^t(\cdot,\cdot|s) - \nu^{t-1}(\cdot,\cdot|s) = \pi^t(a|s)\pi_\textref(a'|s)$. 
    In the following, we denote $\bpi \gets \argmin_{\bpi} \max_{\pi \in \Pi} \cov^{\bpi|\pi}$.
    Therefore,
    \begin{align*}
        \frac{\cI^{(t)}}{2} \leq & \EE_{s\sim\rho}[\sum_{a,a'} \frac{(\pi^t(a|s)\pi_\textref(a'|s))^2}{\nu^t(a,a'|s)}] \\
        \leq & \sum_{s,a,a'} \sqrt{\rho(s)} \frac{(\pi^t(a|s))^2\pi_\textref(a'|s)}{\bpi(a|s)} \cdot \sqrt{\rho(s)} \frac{\pi_\textref(a'|s)\bpi(a|s)}{\nu^t(a,a'|s)}
    \end{align*}

    \begin{align*}
        \frac{\cI^{(t)}}{2} \leq & \EE_{s\sim\rho}[\sum_{a,a'} \frac{(\nu^t(a,a'|s) - \nu^{t-1}(a,a'|s))^2}{\nu^t(a,a'|s)}] \\
        %
        \leq & \sum_{s,a,a'} \sqrt{\rho(s)}\frac{\pi^t(a|s)\sqrt{\pi_\textref(a'|s)}}{\sqrt{\bpi(a|s)}} \cdot \sqrt{\rho(s)}\sqrt{\bpi(a|s) \pi_\textref(a'|s)} \log \frac{\nu^t(a,a'|s)}{\nu^{t-1}(a,a'|s)} \tag{$1 - x \leq \log\frac{1}{x}$} \\
        \leq & \sqrt{\EE_{s\sim\rho,a\sim\pi^t(\cdot|s)}[\frac{\pi^t(a|s)}{\bpi(a|s)}] \cdot \EE_{s\sim\rho,a\sim\bpi(\cdot|s),a'\sim\pi_\textref(\cdot|s)}[\log^2 \frac{\nu^t(a,a'|s)}{\nu^{t-1}(a,a'|s)}]} \\
        \leq & \sqrt{\min_{\bpi} \max_{\pi \in \Pi} \cov^{\bpi|\pi}} \cdot \sqrt{\EE_{s\sim\rho,a\sim\bpi(\cdot|s),a'\sim\pi_\textref(\cdot|s)}[\log^2 \frac{\nu^t(a,a'|s)}{\nu^{t-1}(a,a'|s)}]}.
    \end{align*}
    Therefore,
    \begin{align*}
        \sum_{t=1}^T \cI^{(t)} \leq & 2 \sqrt{\min_{\bpi} \max_{\pi \in \Pi} \cov^{\bpi|\pi}} \cdot \sum_{t=1}^T \sqrt{\EE_{s\sim\rho,a\sim\bpi(\cdot|s),a'\sim\pi_\textref(\cdot|s)}[\log^2 \frac{\nu^t(a,a'|s)}{\nu^{t-1}(a,a'|s)}]} \\
        \leq & \sqrt{\min_{\bpi} \max_{\pi \in \Pi} \cov^{\bpi|\pi}T} \cdot \sqrt{\EE_{s\sim\rho,a\sim\bpi(\cdot|s),a'\sim\pi_\textref(\cdot|s)}[\sum_{t=1}^T \log^2 \frac{\nu^t(a,a'|s)}{\nu^{t-1}(a,a'|s)}]} \\
        \leq & \sqrt{\min_{\bpi} \max_{\pi \in \Pi} \cov^{\bpi|\pi}T} \cdot \sqrt{\EE_{s\sim\rho,a\sim\bpi(\cdot|s),a'\sim\pi_\textref(\cdot|s)}[(\sum_{t=1}^T \log \frac{\nu^t(a,a'|s)}{\nu^{t-1}(a,a'|s)})^2]} \tag{For $x_1,...,x_T \geq 1$, $\sum_{t=1}^T \log^2 x_t \leq (\sum_{t=1}^T \log x_t)^2$ } \\
        \leq & \sqrt{\min_{\bpi} \max_{\pi \in \Pi} \cov^{\bpi|\pi}T} \cdot \sqrt{\EE_{s\sim\rho,a\sim\bpi(\cdot|s),a'\sim\pi_\textref(\cdot|s)}[\log^2 \frac{\nu^T(a,a'|s)}{\nu^{0}(a,a'|s)}]} \\
        = & \tilde{O}(\sqrt{\min_{\bpi} \max_{\pi \in \Pi} \cov^{\bpi|\pi}T}).
    \end{align*}
    where in the last step, we use $\nu^0(\cdot,\cdot|\cdot) = \frac{1}{4}$ and $\nu^T(\cdot,\cdot|\cdot) \leq \frac{1}{4} + T$.
    The rest of the proof follows \citep{xie2024exploratory}, as long as we replace the SEC related terms with our bound above.
    We finish the proof.
\end{proof}


\fi

\section{Proofs for Results in Section~\ref{sec:transfer_coverage_perspective}}\label{appx:coverage_related}
\subsection{Proof for Lemma~\ref{lem:coverage_and_value_gap}}
We first introduce some useful results from \citep{sason2016f}. Given two probability distribution $P, Q \in \Delta(\cA)$, we use $D_{+\infty}(P\|Q)$ to denote the Renyi divergence of order $\alpha = +\infty$. 
We follow the definition of $\chi^2$-divergence in \citep{sason2016f} as follows:
\begin{align*}
    \chi^2(P\|Q) = \EE_{s\sim P}[\frac{P(x)}{Q(x)}] - 1.
\end{align*}
%
\begin{lemma}[Theorem 7 in \citep{sason2016f}]\label{lem:KL_reverse_KL}
    Given $P,Q\in\Delta(\cA)$, such that $P\neq Q$ and $P(a),Q(a) > 0$ for all $a\in\cA$, we have:
    \begin{align*}
        \KL(P\|Q) \leq \kappa_1(e^{D_{+\infty}(P\|Q)}) \cdot \KL(Q\|P),
    \end{align*}
    where $\kappa_1:(0,1)\cup(1,+\infty) \rightarrow (0,+\infty),~\kappa_1(t) = \frac{t\log t + (1-t)}{(t-1) - \log t}$.
\end{lemma}
\begin{lemma}[Eq. 182; Theorem 9 in \citep{sason2016f} for $\alpha= 2$]\label{lem:chi_KL}
    Under the same condition as Lem.~\ref{lem:KL_reverse_KL}, 
    \begin{align*}
        \chi^2(P\|Q) \leq \frac{\KL(P\|Q)}{\kappa_2(e^{D_{+\infty}(P\|Q)})},
    \end{align*}
    where $\kappa_2(t) := \frac{t\log t + (1-t)}{(t-1)^2}$.
\end{lemma}


%
%
%
%
%
%
%
%
%
%
%
%
%
%
%
%
%
%
%

%
%
%
%
%
%


%
\begin{lemma}\label{lem:KL_as_value_gap}
    For any policy $\pi$,
    \begin{align*}
        J_\beta(\pi^*_{r^*}) - J_\beta(\pi) = \beta \EE_{s\sim\rho}[\KL(\pi(\cdot|s)\|\pi^*_{r^*}(\cdot|s))].
    \end{align*}
\end{lemma}
\begin{proof}
    A shorter proof can be done by directly assigning $\nu = \pi$ in Lemma 3.1 of \citep{xie2024exploratory}, and here we provide another one without detouring through it.
    \begin{align*}
        &J_\beta(\pi^*_{r^*}) - J_\beta(\pi) \\
        =& \EE_{s\sim\rho,a\sim\pi^*_{r^*}}[r^*(s,a)] - \EE_{s\sim\rho,a\sim\pi}[r^*(s,a)] - \beta \EE_{s\sim\rho,a\sim\pi^*_{r^*}}[\log\frac{\pi^*_{r^*}(a|s)}{\pi_\textref(a|s)}] + \beta \EE_{s\sim\rho,a\sim\pi}[\log\frac{\pi(a|s)}{\pi_\textref(a|s)}] \\
        =& \cancel{\beta \EE_{s\sim\rho,a\sim\pi^*_{r^*}}[\log\frac{\pi^*_{r^*}(a|s)}{\pi_\textref(a|s)}]} - \beta \EE_{s\sim\rho,a\sim\pi}[\log\frac{\pi^*_{r^*}(a|s)}{\pi_\textref(a|s)}] - \cancel{\beta \EE_{s\sim\rho,a\sim\pi^*_{r^*}}[\log\frac{\pi^*_{r^*}(a|s)}{\pi_\textref(a|s)}]} + \beta \EE_{s\sim\rho,a\sim\pi}[\log\frac{\pi(a|s)}{\pi_\textref(a|s)}] \\
        =& \beta \EE_{s\sim\rho,a\sim\pi}[\log\frac{\pi(a|s)}{\pi^*_{r^*}(a|s)}] \\
        =& \beta \EE_{s\sim\rho}[\KL(\pi(\cdot|s)\|\pi^*_{r^*}(\cdot|s))].
    \end{align*}
    where the second equality holds because for any $s,a$
    \begin{align*}
        r^*(s,a) = \beta \log\frac{\pi^*_{r^*}(a|s)}{\pi_\textref(a|s)} + Z(s)
    \end{align*}
    for some $Z(s)$ independent w.r.t. $a$.
\end{proof}


\LemCovValGap*
We prove a stronger result in Lem.~\ref{lem:cov_value_gap_stronger} below, where we consider the policy class including all the policy having bounded ratio with $\pi_\textref$.
\begin{align*}
    \Pi_{\leq\frac{\Rmax}{\beta}} := \{\pi:\cS\rightarrow\Delta(\cA)| \max_{s,a} |\log\frac{\pi(a|s)}{\pi_\textref(a|s)}| \leq \frac{\Rmax}{\beta}\}.
\end{align*}
Lem.~\ref{lem:coverage_and_value_gap} then holds directly as a corollary by combining with Lem.~\ref{lem:convex_hull_property}, Lem.~\ref{lem:bounded_ratio} and the fact that $r^w \in [0, R]$ for all $w\in[W]$.
\begin{lemma}\label{lem:cov_value_gap_stronger}
    For any policy $\pi \in \Pi_{\leq\frac{\Rmax}{\beta}}$,
    \begin{align}
        \cov^{\pi^*_{r^*}|{\pi}} \leq 1 + \kappa(e^{\frac{2{\Rmax}}{\beta}}) \cdot \frac{J_\beta(\pi^*_{r^*}) - J_\beta({\pi})}{\beta},
    \end{align}
    where $\kappa(x) := \frac{(x-1)^2}{x-1- \log x} = O(x)$.
\end{lemma}
\begin{proof}
    Given any $\pi \in \Pi_{\leq\frac{\Rmax}{\beta}}$, we consider a fixed $s > 0$, and apply Lem.~\ref{lem:KL_reverse_KL} and Lem.~\ref{lem:chi_KL} with $P = \pi^*_{r^*}(\cdot|s)$ and $Q = \pi(\cdot|s)$. Since those two lemmas holds when $P \neq Q$, we first check the case when $\pi^*_{r^*}(\cdot|s) \neq \pi(\cdot|s)$:
    \begin{align*}
        \EE_{a\sim\pi^*_{r^*}(a|s)}[\frac{\pi^*_{r^*}(a|s)}{\pi(a|s)}] - 1 =& \chi^2(\pi^*_{r^*}(\cdot|s)\|\pi(\cdot|s)) \leq \frac{1}{\kappa_2(\zeta)} \KL(\pi^*_{r^*}(\cdot|s)\|\pi(\cdot|s)) \\
        \leq & \frac{1}{\kappa_2(\zeta)} \cdot \kappa_1(\zeta) \cdot \KL(\pi(\cdot|s)\|\pi^*_{r^*}(\cdot|s)) \\
        =& \frac{(\zeta - 1)^2}{\zeta - 1 - \log \zeta} \cdot \KL(\pi(\cdot|s)\|\pi^*_{r^*}(\cdot|s)).
    \end{align*}
    where we use $\zeta := e^{D_{+\infty}(\pi^*_{r^*}(\cdot|s)\|\pi(\cdot|s))} > 1$ as a short note.

    We define $\kappa(x) = \frac{(x - 1)^2}{x - 1 - \log x}$. Note that,
    \begin{align*}
        \kappa'(x) =& \frac{2(x - 1)}{x - 1 - \log x} - \frac{(x - 1)^2(1 - x^{-1})}{(x - 1 - \log x)^2} \\
        =&\frac{x - 1}{x - 1 - \log x} \frac{2(x - 1) - 2\log x - (x - 1)(1 - x^{-1})}{x - 1 - \log x} \\
        =&\frac{x - 1}{x - 1 - \log x} \frac{x - x^{-1} - 2\log x }{x - 1 - \log x}.
    \end{align*}
    Now, we consider $g(x) := x - x^{-1} - 2\log x$ for $x \in (1, +\infty)$. Note that, $g(1) = 0$ and
    \begin{align*}
        g'(x) = 1 + \frac{1}{x^2} - \frac{2}{x} \geq 0.
    \end{align*}
    Therefore, $\kappa'(x) \geq 0$, which implies $\kappa(x)$ is increasing for all $x > 1$.

    %
    %
    %
    %
    %
    %
    %
    %
    %

    %
    %
    %
    %
    %
    %
    %
    %
    %

    Under Assump.~\ref{assump:policy},
    \begin{align*}
        D_{+\infty}(\pi^*_{r^*}(\cdot|s)\|\pi(\cdot|s)) = \log \exp(\max_a \frac{\pi^*_{r^*}(a|s)}{\pi(a|s)}) \leq \frac{2{\Rmax}}{\beta},
    \end{align*}
    which implies $\zeta \leq e^{\frac{{2\Rmax}}{\beta}}$.
    Therefore,
    \begin{align*}
        \EE_{a\sim\pi^*_{r^*}(a|s)}[\frac{\pi^*_{r^*}(a|s)}{\pi(a|s)}] - 1 \leq & \kappa(e^{\frac{2{\Rmax}}{\beta}})  \cdot \KL(\pi(\cdot|s)\|\pi^*_{r^*}(\cdot|s)).
    \end{align*}
    Note that the above inequality also holds when $\pi(\cdot|s) = \pi^*_{r^*}(\cdot|s)$. Therefore, combining with Lem.~\ref{lem:KL_as_value_gap}, we have:
    \begin{align*}
        \cov^{\pi^*_{r^*}|\pi} =& \EE_{s\sim\rho,a\sim\pi^*_{r^*}(a|s)}[\frac{\pi^*_{r^*}(a|s)}{\pi(a|s)}] \leq 1 + \kappa(e^{\frac{2{\Rmax}}{\beta}}) \cdot \EE_{s\sim\rho}[\KL(\pi(\cdot|s)\|\pi^*_{r^*}(\cdot|s))] \\
        = & 1 + \kappa(e^{\frac{2{\Rmax}}{\beta}}) \cdot \frac{J_\beta(\pi^*_{r^*}) - J_\beta(\pi)}{\beta}. 
    \end{align*}

    %
    %
    %
    %
    %
    %
    %
    %
    %
    %
    %
\end{proof}




%
%
%
%
%
%
%
%
%
%
%
%
%

\subsection{Another Bound for Policy Coverage Coefficient}
In the following, we provide another bound for the coverage coefficient between the optimal policies induced by different reward models.
Although we do not use this lemma in the proofs for other results in this paper, it indicates a different upper bound, and possibly, it is tighter than the one in Lem.~\ref{lem:coverage_and_value_gap} in some cases.
\begin{restatable}{lemma}{LemUBCov}\label{lem:UB_Cov}
    Under Assump.~\ref{assump:policy}, given any bounded reward model $r$, and the associated optimal policy $\pi^*_r$ (defined by Eq.~\eqref{eq:rlhf_obj}), the coverage coefficient between $\pi^*_r$ and $\pi^*_{r^*}$ can be controlled by:
    \begin{align*}
        \cov^{\pi^*_{r^*}|\pi^*_r} \leq \min_{b\in\mR}\EE_{s\sim\rho}[\EE^2_{a\sim\pi^*_{r^*}}[\exp(\frac{|r^*(s,a) - r(s,a)-b|}{\beta})]].
    \end{align*}
\end{restatable}
\begin{proof}
    By definition, the state-wise coverage coefficient
    \begin{align*}
        \cov^{\pi^*_{r^*}|\pi^*_{r}}(s) :=& \EE_{a\sim \pi^*_{r^*}(\cdot|s)}[\frac{\pi^*_{r^*}(a|s)}{\pi^*_{r}(a|s)}] \\
        =&\EE_{a\sim \pi^*_{r^*}(\cdot|s)}[\exp(\frac{r^*(s,a) - r(s,a)}{\beta})] \cdot \frac{Z_{r}(s)}{Z_{r^*}(s)}
    \end{align*}
    Here we denote $Z_r(s) = \sum_{a} \pi_\textref(a|s) \exp(\frac{1}{\beta} r(s,a))$ and similar for $Z_{r^*}(s)$. Therefore,
    \begin{align*}
        \frac{Z_{r}(s)}{Z_{r^*}(s)} = \sum_{a} \frac{\pi_\textref(a|s)\exp(\frac{1}{\beta} r(s,a))}{Z_{r^*}(s)} = \sum_{a} \pi^*_{r^*}(s) \cdot \exp(\frac{1}{\beta}({r}(s,a) - r^*(s,a))) = \EE_{a\sim \pi^*_{r^*}}[\exp(\frac{r(s,a) - r^*(s,a)}{\beta})].
    \end{align*}
    We remark that one important fact we leverage in the second equality is that $\pi^*_{r^*}(a|s) > 0$ for all $a\in\cA$.
    Considering introducing an arbitrary $b \in \cR$, we should have:
    \begin{align*}
        \cov^{\pi^*_{r^*}|\pi^*_{r}} =& \EE_{a\sim \pi^*_{r^*}(\cdot|s)}[\exp(\frac{r^*(s,a) - {r}(s,a) + b}{\beta})] \cdot \EE_{a\sim \pi^*_{r^*}(\cdot|s)}[\exp(\frac{{r}(s,a) - \tilde  r(s,a) - b}{\beta})]\\
        \leq & \EE_{a\sim \pi^*_{r^*}(\cdot|s)}^2[\exp(\frac{|r^*(s,a) - r(s,a) + b|}{\beta})]
    \end{align*}
    Given that $b$ is arbirtary, we can pick the best one:
    \begin{align*}
        \cov^{\pi^*_{r^*}|\pi^*_{r}} \leq \min_{b\in\mR}\EE^2_{a\sim\pi^*_{r^*}}[\exp(\frac{|r^*(s,a) - {r}(s,a)-b|}{\beta})].
    \end{align*}
\end{proof}


\subsection{Proof for Theorem~\ref{thm:general_val_gap}}\label{appx:proof_offline_policy_gap}

\ThmOnlineOffline*
We refer to Lem.~\ref{lem:offline_learning} for the detailed hyperparameter setups.
\begin{proof}
    By Lem.~\ref{lem:offline_learning}, w.p. $1-\delta$,
    \begin{align*}
        \quad J_\beta(\pi^*_{r^*}) - J_\beta(\pi_\SELF) \leq C_\Offline e^{2{\Rmax}}\cdot \cov^{\pi^*_{r^*}|\pi_\mix^T}\sqrt{\frac{1}{T}\log\frac{|\Pi|}{\delta}},
    \end{align*}
    where $\pi_\mix^T:=\frac{1}{T}\sum_{t\in[T]}\pi^t$ is the uniform mixture policy, and the coverage coefficient can be upper bounded by:
    \begin{align*}
        \cov^{\pi^*_{r^*}|\pi_\mix^T}\leq &  1 + \kappa(e^{\frac{2{\Rmax}}{\beta}}) \cdot \frac{J_\beta(\pi^*_{r^*}) - J_\beta({\pi^T_\mix})}{\beta} \tag{Lem.~\ref{lem:coverage_and_value_gap}}\\
        \leq & 1 + \kappa(e^{\frac{2{\Rmax}}{\beta}}) \cdot \sum_{t=1}^T \frac{J_\beta(\pi^*_{r^*}) - J_\beta({\pi^t_\mix})}{\beta T}
    \end{align*}
    %
    %
    %
    %
    Here in the last step, we use the fact that KL divergence is convex, and therefore, $\KL(\pi^T_\mix\|\pi_\textref) \leq \frac{1}{T}\sum_{t=1}^T \KL(\pi^t \| \pi_\textref)$, which implies $J_\beta(\pi^T_\mix) \geq \frac{1}{T} \sum_{t=1}^T J_\beta(\pi^t)$.
    %
    %
    %
    %
    %
    %
    %
    %
    %
    %
    %
    %

    %
    %
    %
\end{proof}


\paragraph{Implication for Online RLHF}
If we consider the policy sequence generated by a no-regret online learning algorithm, we have the following corollary.
\begin{corollary}\label{coro:offline_gap}
    Under Assump.~\ref{assump:policy}, suppose $\pi^1,...,\pi^T$ is generated by a no-regret online learning algorithm with $\sum_{t=1}^T J_\beta(\pi^*_{r^*}) - J_\beta(\pi^t) = \tilde{O}(\Complexity(\Pi)\sqrt{T})$ for some structural complexity measure $\Complexity(\Pi)$, as long as $T = \tilde{\Omega}(\beta^{-2}\Complexity(\Pi)^2\kappa^2(e^{\frac{2{\Rmax}}{\beta}}))$, running $\RPO$ yields an offline policy s.t. $J_\beta(\pi^*_{r^*}) - J_\beta(\pi_\SELF) = \tilde{O}(e^{2{\Rmax}} T^{-\frac{1}{2}})$.
\end{corollary}
The proof is straightforward by noting that $\sum_{t=1}^T\frac{J_\beta(\pi^*_{r^*}) - J_\beta(\pi^t)}{\beta T} = O(\cC(\Pi)\sqrt{T}/T)$, which decays to 0 as $T$ increases.
Coro.~\ref{coro:offline_gap} is remarkable as it implies an $\tilde{O}(\epsilon^{-2})$ sample complexity bound to learn an $\epsilon$-optimal policy for online RLHF (for $\epsilon$ smaller than a threshold), which \textbf{\emph{does not depend on}} the number of states and actions or other complexity measures.
In contrast, in previous online RLHF literature \citep{xiong2024iterative, xie2024exploratory,cen2024value,zhang2024self}, for the uniform mixture policy $\pi^T_\mix := \frac{1}{T}\sum_{t=1}^T \pi^t$, the regret-to-PAC conversion implies a value gap $J_\beta(\pi^*_{r^*}) - J_\beta(\pi^T_\mix) = \tilde{O}(\sqrt{\frac{\Complexity(\Pi)}{T}})$, which has an additional factor $\Complexity(\Pi)$ regarding the complexity of the function class.
This suggests a strict improvement.
%


Moreover, this marks a fundamental difference from the pure reward maximization setting, where lower bounds depending on those factors has been established \citep{auer2002nonstochastic,dani2008stochastic}.
%


\paragraph{Other Previous Works Reporting Faster Convergence Rate}
Several recent works also report faster convergence rate than the information-theoretic lower bounds for online pure reward maximization RL, by exploiting the structure induced by KL regularization.
\citep{shi2024crucial} investigates the tabular softmax parametrization setting and establishes quadratic convergence results.
In contrast, our result is more general, applying to arbitrary policy class.

The work of \citep{zhao2024sharp} is more related to ours. They consider general reward function classes and derive an $O(\epsilon^{-1} \text{Poly}(D))$ sample complexity bound, where $D$ is a coefficient related to the coverage of the distribution $\rho\times\pi_\textref$.
While their dependence on $\epsilon$ is better than ours, their definition of $D$ is not always satisfactory. For example, in the worst case one would have $D = \Omega(\frac{1}{\min_{s\in\cS}\rho(s)})$. This indicates that their bound scales with the number of states, once noticing that $\frac{1}{\min_{s\in\cS}\rho(s)}$ is no smaller than $|\cS|$.
In contrast, the largest coverage-related coefficient in our result is $O(\kappa^2(e^{\frac{2R}{\beta}})) = O(e^{\frac{4R}{\beta}})$, which remains small and is free of $|\cS|$.
Therefore, our Coro.~\ref{coro:offline_gap} can outperform the bound in \citep{zhao2024sharp} in many scenarios.

More importantly, the primary focus of our work is on reward transfer, which is orthogonal to these studies.
%
%

\section{Proofs for the Main Algorithm and Results in Sec.~\ref{sec:main_theory}}\label{appx:proof_task_selection}
\subsection{Additional Algorithm Details}\label{appx:main_alg_details}
\paragraph{Missing Details for $\TPO$ (Alg.~\ref{alg:main_algorithm})}
For any given $(k,n) \in [K]\times[N]$, we use $\cD^{k,n} := \cup_{i< k \text{ or } i = k, j<n}\{s^{i,j},a^{i,j},\ta^{i,j},y^{i,j},\pi^{i,j}\}$ to denote all the collected data up to step $(k,n)$; $\cD^{k,n}_\Online := \cup_{i<k,j\leq\alpha N\text{ or }i=k,j\leq n\wedge \alpha N}\{s^{i,j},a^{i,j},\ta^{i,j},y^{i,j},\pi^{i,j}\}$ denotes the data collected by $\AlgOnline$ up to step $(k,n)$.


\subsection{Some Useful Lemmas}
\begin{lemma}[MLE Reward Estimation Error]\label{lem:reward_est_error}
    In each call of Alg.~\ref{alg:transfer_policy_computing} with a policy class $\Pi$ satisfying Assump.~\ref{assump:policy} and a dataset $\cD$ generated by a sequence of policies $\pi^1,...,\pi^{|\cD|}$, then, for any policy $\pi$, given any $\delta\in(0,1)$, with probability at least $1-\delta$, for all $w\in[W]$, we have:
    \begin{align*}
        \Big|\Big(\EE_{\rho,\pi}[r^*] - \EE_{\rho,\pi_\textref}[r^*]\Big) - \Big(\EE_{\rho,\pi}[\hr_\MLE] - \EE_{\rho,\pi_\textref}[\hr_\MLE]\Big)\Big| \leq 16e^{2{\Rmax}} \sqrt{\frac{\cov^{\pi|\pi^\cD_\mix}}{|\cD|}\cdot \log\frac{|\Pi|}{\delta}},
    \end{align*}
    where we use $\pi_\mix^\cD := \frac{1}{|\cD|} \sum_{i \leq |\cD|} \pi^i$ as a short note.
\end{lemma}
\begin{proof}
    For any policy $\pi\in\Pi$, by applying Lem.~\ref{lem:r_err_to_Hellinger} with $\pi_\mix^\cD$ and $r \gets \hr_\MLE$, we have:
    \begin{align*}
        &\Big|\Big(\EE_{\rho,\pi}[r^*] - \EE_{\rho,\pi_\textref}[r^*]\Big) - \Big(\EE_{\rho,\pi}[\hr_\MLE] - \EE_{\rho,\pi_\textref}[\hr_\MLE]\Big)\Big| \\
        \leq & \EE_{s\sim\rho,a\sim\pi(\cdot|s),\ta\sim\pi_\textref(\cdot|s)}[|\Big(r^*(s,a) - r^*(s,\ta)\Big) - \Big(\hr_\MLE(s,a) - \hr_\MLE(s,\ta)\Big)|] \\
        \leq& 8\sqrt{2}e^{2{\Rmax}} \sqrt{\cov^{\pi|\pi^\cD_\mix} \cdot \frac{1}{|\cD|} \cdot \sum_{i\leq|\cD|} \EE_{s\sim\rho,a\sim\pi^i(\cdot|s),\ta\sim\pi_\textref(\cdot|s)}[\mH^2(\mP_{\hr_\MLE}(\cdot|s,a,\ta)\|\mP_{r^*}(\cdot|s,a,\ta))]}. 
    \end{align*}
    By applying Lem.~\ref{lem:MLE_Estimation}, and the fact that $\hr_\MLE, r^* \in \cR^{\Pi}$, for any $\delta\in(0,1)$, w.p. $1-\delta$, we have:
    \begin{align*}
        & \frac{1}{|\cD|} \sum_{i\leq |\cD|} \EE_{s\sim\rho,a\sim\pi^i(\cdot|s),\ta\sim\pi_\textref(\cdot|s)}[\mH^2(\mP_{\hr_\MLE}(\cdot|s,a,\ta)\|\mP_{r^*}(\cdot|s,a,\ta))] \\
        \leq & L_{\cD}(\hr_\MLE) - L_{\cD}(r^*) + \frac{2}{|\cD|}\log\frac{|\Pi|}{\delta} \\
        \leq & \frac{2}{|\cD|} \log\frac{|\Pi|}{\delta} \tag{Assump.~\ref{assump:policy} and $\hr_\MLE$ minimizes the negative log-likelihood}.
    \end{align*}
    %
    %
    %
    %
    %
    %
    %
    Therefore, we finish the proof.
\end{proof}
%
%
%
%
%
%
%
%

\LemOptismValErr*
\begin{proof}
    Note that $\frac{\cov^{\pi^*_{r^w}|\pi^\cD_\mix}}{|\cD|} \leq \frac{1}{N(w;\cD)}$, where we recall that $N(w;\cD) := \sum_{i\leq|\cD|} \mI[\pi^i = \pi^*_{r^w}]$ denotes the number of occurrences of $\pi^*_{r^w}$ in the dataset. By Lem.~\ref{lem:reward_est_error}, w.p. $1-\delta'$, for all $w\in[W]$, and any $(k,n)\in[K]\times[N]$ occurs in the call of Alg.~\ref{alg:main_algorithm} such that $n>\alpha N$:
    \begin{align*}
        \Big|\Big(\EE_{\rho,\pi^*_{r^w}}[r^*] - \EE_{\rho,\pi_\textref}[r^*]\Big) - \Big(\EE_{\rho,\pi^*_{r^w}}[\hr_\MLE] - \EE_{\rho,\pi_\textref}[\hr_\MLE]\Big)\Big| \leq & 16e^{2{\Rmax}} \sqrt{\frac{1}{N(w;\cD^{k,n})} \log\frac{|\Pi|W}{\delta'}}.
    \end{align*}
    Recall
    \begin{align*}
        \hV(\pi^*_{r^w};\cD) :=& \EE_{\rho,\pi^*_{r^w}}[\hr_\MLE] - \EE_{\rho,\pi_\textref}[\hr_\MLE] - \beta \KL(\pi^*_{r^w}\|\pi_\textref) + 16e^{2{\Rmax}} \sqrt{\frac{1}{N(w;\cD^{k,n})} \log\frac{|\Pi|WT}{\delta}}.
    \end{align*}
    By taking the union bound for all $T$ iterations (choosing $\delta' = \delta/T$), we finish the proof.
\end{proof}


%
%
%
%
%
%
%
%
%
%
%
%
%
%
%
%
%
%
%
%
%
%
%
%
%
%
%


\begin{lemma}[Estimation Error for Self-Transfer Policy]\label{lem:est_error_self_transfer}
    For any $k > 1$ and $\alpha N < n \leq N$, in each call of Alg.~\ref{alg:transfer_policy_computing} in the iteration $(k,n)$ of Alg.~\ref{alg:main_algorithm} with a dataset $\cD := \{(s^i,a^i,\ta^i,y^i,\pi^i)\}_{i\leq |\cD|}$ satisfying Cond.~\ref{cond:seq_data}, then, given any $\delta\in(0,1)$, w.p. $1-\delta$:
    \begin{align*}
        \hat{V}(\pi_\SELF; \cD) \leq & J_\beta(\pi_\SELF) - J_\beta(\pi_\textref) \\
        \hat{V}(\pi_\SELF; \cD) \geq & J_\beta(\pi^*_{r^*}) - J_\beta(\pi_\textref) - c'\cdot {\Rmax} e^{2{\Rmax}}\cdot \Big(\cov^{\pi^*_{r^*}|\pi_\mix^\cD} \wedge \frac{\sqrt{\Complexity(\Pi)}}{\alpha}\Big) \cdot \sqrt{\frac{1}{|\cD|}\log^{c_0}\frac{|\Pi|T}{\delta}},
    \end{align*}
    where we use $\pi_\mix^\cD := \frac{1}{|\cD|}\sum_{i\leq |\cD|} \pi^{|\cD|}$ as a short note, and $c'$ is some absolute constant.
\end{lemma}
\begin{proof}
    Recall that
    \begin{align*}
        \hat{V}(\pi_\SELF; \cD) :=& \EE_{\rho,\pi_\SELF}[\hr_\SELF] - \EE_{\rho,\pi_\textref}[\hr_\SELF] - \beta \KL(\pi_\SELF\|\pi_\textref) \\
         & + \frac{1}{\eta} L_{\cD}(\hr_\SELF) - \frac{1}{\eta} L_{\cD}(\hr_\MLE) - \bonus,
    \end{align*}
    Here we use $\bonus := 2c\cdot e^{2{\Rmax}} \sqrt{\frac{1}{|\cD|}\log\frac{|\Pi|T}{\delta}}$ as a short note of the bonus term.
    By definition,
    %
    \begin{align*}
        &\hat{V}(\pi_\SELF; \cD) \\
        \leq & \EE_{\rho,\pi_\SELF}[r^*] - \EE_{\rho,\pi_\textref}[r^*] - \beta \KL(\pi_\SELF\|\pi_\textref) + \frac{1}{\eta} L_{\cD}(r^*) - \frac{1}{\eta} L_{\cD}(\hr_\MLE) - \bonus \tag{Pessimistic estimation of $\hr_\SELF$ in Eq.~\eqref{eq:RPO_objective}}\\
        \leq & J_\beta(\pi_\SELF) - J_\beta(\pi_\textref) + \frac{2}{\eta|\cD|}\log\frac{|\Pi|}{\delta} - \bonus \tag{Lem.~\ref{lem:MLE_Estimation}} \\
        \leq & J_\beta(\pi_\SELF) - J_\beta(\pi_\textref) + 2c\cdot e^{2{\Rmax}}\sqrt{\frac{1}{|\cD|}\log\frac{|\Pi|T}{\delta}} - \bonus.\numberthis\label{eq:Vhat_Offline_upper_bound}
    \end{align*}
    The last step is because of our choice of $\eta = (1+e^{{\Rmax}})^{-2} \sqrt{\frac{24}{|\cD|}\log\frac{|\Pi|T}{\delta}}$.

    For the lower bound, note that for any policy $\pi \in \conv(\Pi)$, we have:
    \begin{align*}
        & J_\beta(\pi) - J_\beta(\pi_\textref) - \hat{V}(\pi_\SELF; \cD)\\
        = & \Big(\EE_{\rho,\pi}[r^*] - \EE_{\rho,\pi_\textref}[r^*] - \beta \KL(\pi\|\pi_\textref) \Big) \\
        & - \Big(\EE_{\rho,\pi_\SELF}[\hr_\SELF] - \EE_{\rho,\pi_\textref}[\hr_\SELF]- \beta \KL(\pi_\SELF\|\pi_\textref) + \frac{1}{\eta} L_{\cD}(\hr_\SELF)\Big) + \frac{1}{\eta} L_{\cD}(\hr_\MLE) + \bonus \\
        \leq & \Big(\EE_{\rho,\pi}[r^*] - \EE_{\rho,\pi_\textref}[r^*] - \beta \KL(\pi\|\pi_\textref) \Big) - \min_{r\in\cR^{\Pi}}\Big(\EE_{\rho,\pi}[r] - \EE_{\rho,\pi_\textref}[r]- \beta \KL(\pi\|\pi_\textref) + \frac{1}{\eta} L_{\cD}(r)\Big) \tag{Optimality of $\pi_\SELF$ in $\RPO$;}\\
        & + \frac{1}{\eta} L_{\cD}(r^*)  + \bonus \tag{$\hr_\MLE$ minimizes $L_{\cD}$ }\\
        \leq & \EE_{s\sim\rho,a\sim\pi,\ta\sim\pi_\textref}[|r^*(s,a) - r^*(s,\ta) - r_{\pi;\cD}(s,a) + r_{\pi;\cD}(s,\ta)|] + \frac{1}{\eta} L_{\cD}(r^*) - \frac{1}{\eta} L_{\cD}(r_{\pi;\cD}) + \bonus \tag{We use $r_{\pi;\cD}$ to denote the reward achieves the above minimum} \\
        \leq & \frac{2}{\eta|\cD|} \log\frac{|\Pi|}{\delta} + 8\sqrt{2}e^{2{\Rmax}} \sqrt{\frac{\cov^{\pi|\pi_\mix^\cD}}{|\cD|} \cdot \sum_{i\leq|\cD|} \EE_{s\sim\rho,a\sim\pi^i(\cdot|s),\ta\sim\pi_\textref(\cdot|s)}[\mH^2(\mP_{r_{\pi;\cD}}(\cdot|s,a,\ta)\|\mP_{r^*}(\cdot|s,a,\ta))]} \\
        & - \frac{1}{\eta} \sum_{i \leq |\cD|} \EE_{s\sim\rho,a\sim\pi^i,\ta\sim\pi_\textref}[\mH^2(\mP_{r_{\pi;\cD}}(\cdot|s,a,\ta) \| \mP_{r^*}(\cdot|s,a,\ta))] \tag{Lem.~\ref{lem:MLE_Estimation} and Lem.~\ref{lem:r_err_to_Hellinger}}  + \bonus \\
        \leq & \frac{2}{\eta|\cD|} \log\frac{|\Pi|}{\delta} + 64 \eta e^{4{\Rmax}} \frac{\cov^{\pi|\pi_\mix^\cD}}{|\cD|} \tag{$ax - b x^2 \leq \frac{a^2}{4b}$}  + \bonus\\
        \leq & 4 c_2 \cdot e^{2{\Rmax}}\cdot \cov^{\pi|\pi_\mix^\cD} \cdot \sqrt{\frac{1}{|\cD|}\log\frac{|\Pi|T}{\delta}}  + \bonus. \numberthis\label{eq:V_pi_off_LB}
    \end{align*}
    where the last step is because of our choice of $\eta = c\cdot (1+e^{{\Rmax}})^{-2} \sqrt{\frac{24}{|\cD|}\log\frac{|\Pi|T}{\delta}}$.

    Next, we evaluate some choice of $\pi$. We first consider $\pi = \pi^*_{r^*} \in \conv(\Pi)$, the above result implies,
    \begin{align*}
        J_\beta(\pi^*_{r^*}) - J_\beta(\pi_\textref) - \hat{V}(\pi_\SELF; \cD) \leq 4 c_2 \cdot e^{2{\Rmax}}\cdot \cov^{\pi^*_{r^*}|\pi_\mix^\cD} \cdot \sqrt{\frac{1}{|\cD|}\log\frac{|\Pi|T}{\delta}}  + \bonus. \numberthis\label{eq:Vhat_Offline_lower_bound_1}
    \end{align*}
    Secondly, we consider the mixture policy $\pi = \pi_{\mix}^{k-1} := \frac{1}{\alpha (k-1)N}\sum_{i=1}^{k-1} \pi^{i,j} \in \conv(\Pi)$. Because of the convexity of KL divergence, $J(\pi)$ is concave in $\pi$, by Jensen's inequality, we have:
    \begin{align*}
        J_\beta(\pi^*_{r^*}) -  J_\beta(\pi^{k-1}_{\mix}) =&J_\beta(\pi^*_{r^*}) - \EE_{s\sim\rho,a\sim\pi^{k-1}_\mix(\cdot|s)}[r^*(s,a)] + \beta \KL(\pi^{k-1}_\mix\|\pi_\textref) \\
        \leq & J_\beta(\pi^*_{r^*}) -  \frac{1}{\alpha (k-1)N}\sum_{i=1}^{k-1} \sum_{1\leq j\leq \alpha N} \Big(\EE_{s\sim\rho,a\sim\pi_\Online^i(\cdot|s)}[r^*(s,a)] - \beta \KL(\pi^{i,j}\|\pi_\textref)\Big)\\
        \leq & C_\Online {\Rmax} e^{2{\Rmax}} \sqrt{\frac{\Complexity(\Pi)}{\alpha (k-1)N} \log^{c_0} \frac{|\Pi|T}{\delta}} \leq C_\Online {\Rmax} e^{2{\Rmax}} \sqrt{\frac{2\Complexity(\Pi)}{\alpha kN} \log^{c_0} \frac{|\Pi|T}{\delta}} \tag{Cond.~\ref{def:online_oracle}}.
    \end{align*}
    Note that $|\cD|\pi^\cD_\mix \geq \alpha(k-1)N\pi^{k-1}_\mix$, which implies $\cov^{\pi^{k-1}_\mix|\pi_\mix^\cD} \leq \frac{|\cD|}{\alpha(k-1)N} \leq \frac{kN}{\alpha(k-1)N}\leq \frac{2}{\alpha}$.
    Therefore, by Eq.~\eqref{eq:V_pi_off_LB},
    \begin{align*}
        J_\beta(\pi^{k-1}_\mix) - J_\beta(\pi_\textref) - \hat{V}(\pi_\SELF; \cD) \leq 4 c_2 \cdot e^{2{\Rmax}}\cdot \frac{2}{\alpha} \cdot \sqrt{\frac{1}{|\cD|}\log\frac{|\Pi|T}{\delta}}.
    \end{align*}
    Combining the above two inequalities together, we have:
    \begin{align*}
        &J_\beta(\pi^*_{r^*}) - J_\beta(\pi_\textref) - \hat{V}(\pi_\SELF; \cD) \\
        \leq & J_\beta(\pi^*_{r^*}) - J_\beta(\pi_{\mix}^{k-1}) + J_\beta(\pi_{\mix}^{k-1}) - J_\beta(\pi_\textref) - \hat{V}(\pi_\SELF; \cD)\\
        \leq &  C_\Online {\Rmax} e^{2{\Rmax}} \sqrt{\frac{2\Complexity(\Pi)}{\alpha kN} \log^{c_0} \frac{|\Pi|T}{\delta}} + 4 c_2 \cdot e^{2{\Rmax}}\cdot \frac{2}{\alpha} \cdot \sqrt{\frac{1}{|\cD|}\log\frac{|\Pi|T}{\delta}}  + \bonus \\
        \leq & c_3 {\Rmax} \cdot e^{2{\Rmax}} \sqrt{\frac{\Complexity(\Pi)}{\alpha^2|\cD|} \log^{c_0}\frac{|\Pi|T}{\delta}} + \bonus.\numberthis\label{eq:Vhat_Offline_lower_bound_2}
    \end{align*}
    Therefore, under our choice of $\bonus = 2c\cdot\sqrt{\frac{1}{|\cD|}\log\frac{|\Pi|T}{\delta}}$, Eq.~\eqref{eq:Vhat_Offline_upper_bound}, Eq.~\eqref{eq:Vhat_Offline_lower_bound_1} and Eq.~\eqref{eq:Vhat_Offline_lower_bound_2} imply,
    \begin{align*}
        \hat{V}(\pi_\SELF; \cD) \leq & J_\beta(\pi_\SELF) - J_\beta(\pi_\textref) \\
        \hat{V}(\pi_\SELF; \cD) \geq & J_\beta(\pi^*_{r^*}) - J_\beta(\pi_\textref) - c' {\Rmax} e^{2{\Rmax}}\cdot \Big(\cov^{\pi^*_{r^*}|\pi_\mix^\cD} \wedge \frac{\sqrt{\Complexity(\Pi)}}{\alpha}\Big) \cdot \sqrt{\frac{1}{|\cD|}\log^{c_0}\frac{|\Pi|T}{\delta}}.
    \end{align*}
\end{proof}


%
%
%
%
%
%
%
%
\LemSelfTransErr*
\begin{proof}
    By applying Lem.~\ref{lem:est_error_self_transfer} with appropriate constants, and taking the union bound over all iterations, we can finish the proof.
\end{proof}


\subsection{Proof for Thm.~\ref{thm:regret_guarantees}}

\ThmMainReg*
Throught the proof, we follow the convention that $1/0 = +\infty$.
\begin{proof}
    Since we divide the total budget $T$ to $K$ batches with batch size $N$, we will use two indices $\tK\in[K]$ and $\tN\in[N]$ to represent the current iteration number, i.e. the $\tN$-th iteration in the $\tK$-th batch.
    We will divide the indices of previous iterations to two parts, depending on whether we conduct normal online learning (the first $\alpha N$ samples in each batch) or do transfer learning (the rest $(1-\alpha) N$ samples in each batch):
    \begin{align*}
        &\cI^{\Online}_{\tK,\tN}:=\{(k,n)|k< \tK, n\leq \alpha N,\text{~or~}k=\tK, n\leq \tN \wedge \alpha N\},\\
        &\cI^{\Transfer}_{\tK,\tN}:=\{(k,n)|k< \tK, \alpha N < n\leq N,\text{~or~}k=\tK, \alpha N < n\leq \tN \}, \\
        &\cI_{\tK,\tN} := \cI^{\Online}_{\tK,\tN} \cup \cI^{\Transfer}_{\tK,\tN} = \{(k,n)|k< \tK, n\leq N,\text{~or~}k=\tK, n\leq \tN \}.
    \end{align*}
    For the policies generated by online algorithm, under the condition in Def.~\ref{def:online_oracle}, w.p. $1-\delta$, for any $\tK\in[K], \tN\in[N]$ we have:
    \begin{align}
        \sum_{(k,n)\in\cI^{\Online}_{\tK,\tN}} J_\beta(\pi^*_{r^*}) - J_\beta(\pi^{k,n}) \leq C_\Online {\Rmax} e^{2{\Rmax}} \sqrt{\Complexity(\Pi) |\cI^{\Online}_{\tK,\tN}| \log^{c_0}\frac{|\Pi|T}{\delta}}.\label{eq:online_regret}
    \end{align}
    Next, we focus on the performance of transfer policies. We first introduce a few notation for convenience.

    \paragraph{Additional Notations}
    We use $\pi^{k,n}_\SELF$ to denote the offline policy computed by Alg.~\ref{alg:transfer_policy_computing} called by Alg.~\ref{alg:main_algorithm} at iteration $(k,n)$ for some $\alpha N < n \leq N$.
    We denote $\cE^{k,n}_\SELF := \{\pi^{k,n}_\SELF = \pi^{k,n}\}$ to be the event that Alg.~\ref{alg:transfer_policy_computing} returns $\pi^{k,n}_\SELF$ as the policy, and use $\cE^{k,n}_w := \{\pi^*_{r^w} = \pi^{k,n}\}$ to denote the event that Alg.~\ref{alg:transfer_policy_computing} pick and return $\pi^*_{r^w}$.
    Besides, we use $\neg\cE^{k,n}_\SELF := \bigcup_{w\in[W]} \cE^{k,n}_w$ as a short note for the event that Alg.~\ref{alg:transfer_policy_computing} does not return the offline policy $\pi^{k,n}_\SELF$.
    Recall the definition $\Delta(w) := J_\beta(\pi^*_{r^*}) - J_\beta(\pi^*_{r^w})$, and $\Delta_{\min} = \min_{w\in[W]} \Delta(w)$. 
    We will use $w^*$ to denote the index of the task achieves $\Delta_{\min}$ (or any of the tasks if multiple maximizers exist).
    Given the dataset $\cD^{k,n}$ we use $\pi^{k,n}_\mix := \frac{1}{|\cD^{k,n}|} \sum_{i,j\in \cD^{k,n}} \pi^{i,j}$ to be the uniform mixture policy from $\cD^{k,n}$.

    Then, we decompose the accumulative value gap depending on whether $\cE^{k,n}_\SELF$ is true or not. We use $\mathbb{I}[\cE]$ as the indicator function, which takes value 1 if $\cE$ happens and otherwise 0.
    For any $\tK\in[K], \tN\in[N]$, we have:
    \begin{align}
        &\sum_{(k,n)\in \cI^{\Transfer}_{\tK,\tN}} J_\beta(\pi^*_{r^*}) - J_\beta(\pi^{k,n}) \nonumber\\
        =& \sum_{(k,n)\in \cI^{\Transfer}_{\tK,\tN}} \mathbb{I}[\cE^{k,n}_\SELF] (J_\beta(\pi^*_{r^*}) - J_\beta(\pi^{k,n})) + \sum_{(k,n)\in \cI^{\Transfer}_{\tK,\tN}} \mathbb{I}[\neg\cE^{k,n}_\SELF](J_\beta(\pi^*_{r^*}) - J_\beta(\pi^{k,n})). \label{eq:value_gap_decomposition}
    \end{align}
    \paragraph{Part-(1) Upper Bound the First Part in Eq.~\eqref{eq:value_gap_decomposition}}
    We first bound the accumulative error when $\mathbb{I}[\cE^{k,n}_\SELF] = 1$.
    On the good events in Lem.~\ref{lem:formal_optism_val_est_error} and Lem.~\ref{lem:formal_val_est_error} (which holds w.p. $1-\delta$), $\mathbb{I}[\cE^{k,n}_\SELF] = 1$ implies
    \begin{align*}
        \hV(\pi_\SELF; \cD^{k,n}) \geq \max_{w\in[W]} \hV(\pi^*_{r^w};\cD^{k,n}) \geq \max_{w\in[W]} J_\beta(\pi^*_{r^w}) - J_\beta(\pi_\textref) = J_\beta(\pi^*_{r^*}) - J_\beta(\pi_\textref) - \Delta_{\min},
    \end{align*}
    and as implied by Lem.~\ref{lem:formal_val_est_error}
    \begin{align*}
        J_\beta(\pi^*_{r^*}) -  J_\beta(\pi_\SELF) \leq & \Delta_{\min},\\
        J_\beta(\pi^*_{r^*}) -  J_\beta(\pi_\SELF) \leq & c_2 {\Rmax} e^{2{\Rmax}}\cdot \Big(\cov^{\pi^*_{r^*}|\pi_\mix^{k,n}} \wedge \frac{\sqrt{\Complexity(\Pi)}}{\alpha}\Big) \cdot \sqrt{\frac{1}{|\cD^{k,n}|}\log^{c_0}\frac{|\Pi|T}{\delta}}.
    \end{align*}
    Combining all the results above, we conclude that
    \begin{align*}
        J_\beta(\pi^*_{r^*}) - J_\beta(\pi_\SELF) \leq & \Delta_{\min} \wedge c_2 {\Rmax} e^{2{\Rmax}}\cdot \Big(\cov^{\pi^*_{r^*}|\pi_\mix^{k,n}} \wedge \frac{\sqrt{\Complexity(\Pi)}}{\alpha}\Big) \cdot \sqrt{\frac{1}{|\cD^{k,n}|}\log^{c_0}\frac{|\Pi|T}{\delta}} \\
        =& \Delta_{\min} \wedge \iota^{k,n}.
    \end{align*}
    Here for simplicity, we use 
    $$
    \iota^{k,n} := c_2 {\Rmax} e^{2{\Rmax}}\cdot \Big(\cov^{\pi^*_{r^*}|\pi_\mix^{k,n}} \wedge \frac{\sqrt{\Complexity(\Pi)}}{\alpha}\Big) \cdot \sqrt{\frac{1}{|\cD^{k,n}|}\log^{c_0}\frac{|\Pi|T}{\delta}}
    $$
    as a short note, indexed by $k,n$.
    Therefore, 
    \begin{align}
        \sum_{(k,n)\in \cI^{\Transfer}_{\tK,\tN}} \mathbb{I}[\cE^{k,n}_\SELF] (J_\beta(\pi^*_{r^*}) - J_\beta(\pi^{k,n})) = \sum_{(k,n)\in \cI^{\Transfer}_{\tK,\tN}} \mathbb{I}[\cE^{k,n}_\SELF] (\Delta_{\min} \wedge \iota^{k,n}).\label{eq:offline_accum_gap}
    \end{align}

    \paragraph{Part-(2) Upper Bound the Second Part in Eq.~\eqref{eq:value_gap_decomposition}}
    Next, we bound the accumulative error when $\mathbb{I}[\neg\cE^{k,n}_\SELF] = 1$.
    Note that,
    \begin{align*}
        & \sum_{(k,n)\in \cI^{\Transfer}_{\tK,\tN}} \mathbb{I}[\neg\cE^{k,n}_\SELF](J_\beta(\pi^*_{r^*}) - J_\beta(\pi^{k,n})) = \sum_{(k,n)\in \cI^{\Transfer}_{\tK,\tN}}\sum_{\substack{w\in[W] \\ \Delta(w) > 0}} \mathbb{I}[\cE^{n,k}_w] \Delta(w)
    \end{align*}
    Here we only focus on those source tasks with $\Delta(w) > 0$, since transferring from $\pi^*_{r^w}$ with $\Delta(w) = 0$ does not incur regret.
    We separate source tasks into two sets $\cW_{\leq 2\Delta_{\min}} := \{w\in[W]|\Delta(w) \leq 2\Delta_{\min}\}$ and $\cW_{> 2\Delta_{\min}} := \{w\in[W]|\Delta(w) > 2\Delta_{\min}\}$.
    For $w\in \cW_{> 2\Delta_{\min}}$, on the same good events in Lem.~\ref{lem:formal_optism_val_est_error} and Lem.~\ref{lem:formal_val_est_error}, $\mathbb{I}[\cE^{n,k}_w] = 1$ implies
    \begin{align*}
        \Delta_{\min} =& J_\beta(\pi^*_{r^*}) - J_\beta(\pi_\textref) - J_\beta(\pi^*_{r^{w^*}}) + J_\beta(\pi_\textref)\\
        \geq & J_\beta(\pi^*_{r^*}) - J_\beta(\pi_\textref) - \hV^{k,n}(\pi^*_{r^{w^*}};\cD{}^{k,n-1}) \\
        \geq & J_\beta(\pi^*_{r^*}) - J_\beta(\pi_\textref) - \hV^{k,n}(\pi^*_{r^w};\cD{}^{k,n-1}) \\
        \geq & J_\beta(\pi^*_{r^*}) - J_\beta(\pi_\textref) - J_\beta(\pi^*_{r^{w}}) + J_\beta(\pi_\textref) - 32\cdot e^{2{\Rmax}}\sqrt{\frac{1}{N(w;\cD^{k,n})}\log\frac{|\Pi|WT}{\delta}} \\
        = & \Delta(w) - 32\cdot e^{2{\Rmax}}\sqrt{\frac{1}{N(w;\cD^{k,n})}\log\frac{|\Pi|WT}{\delta}}.
    \end{align*}
    In the following, we use $c_1 = 32$ as a short note, then the above implies
    \begin{align*}
        N(w;\cD^{k,n}) \leq \frac{c_1^2 e^{4{\Rmax}}}{(\Delta(w) - \Delta_{\min})^2} \log\frac{|\Pi|WT}{\delta} \leq \frac{4c_1^2 e^{4{\Rmax}}}{\Delta(w)^2} \log\frac{|\Pi|WT}{\delta}
    \end{align*}
    and therefore,
    \begin{align*}
        \forall w\in\cW_{>2\Delta_{\min}},\quad \sum_{(k,n)\in \cI^{\Transfer}_{\tK,\tN}} \mathbb{I}[\cE^{n,k}_w] \Delta(w) \leq \frac{16c_1^2 e^{4{\Rmax}}}{\Delta(w)} \log\frac{|\Pi|WT}{\delta},
        %
    \end{align*}
    %
    For $w\in\cW_{\leq 2\Delta_{\min}}$, we introduce a new event $\cE^{n,k}_{2\iota < \Delta_{\min}} := \{2\iota^{k,n} \leq \Delta_{\min}\}$. Note that, when $\mI[\cE^{n,k}_{2\iota < \Delta_{\min}}]=0$, i.e. $2\iota \geq \Delta_{\min}$, we automatically have:
    \begin{align}
        \forall w\in\cW_{\leq 2\Delta_{\min}},\quad \mI[\cE^{n,k}_w] \Delta(w) \leq 2\mI[\cE^{n,k}_w]\Delta_{\min} \leq 4\mI[\cE^{n,k}_w] \cdot (\Delta_{\min} \wedge \iota^{n,k}).\label{eq:iota_cases}
    \end{align}
    On the other hand, on the good events of Lem.~\ref{lem:formal_optism_val_est_error} and Lem.~\ref{lem:formal_val_est_error}, when $\mI[\cE^{n,k}_w\cap\cE^{n,k}_{2\iota < \Delta_{\min}}]=1$, we must have:
    \begin{align*}
        J_\beta(\pi^*_{r^*}) - J_\beta(\pi_\textref) - \iota^{k,n} \leq & \hV^{k,n}(\pi^{k,n}_\SELF;\cD^{k,n-1}{}) \tag{Lem.~\ref{lem:formal_optism_val_est_error} and Lem.~\ref{lem:formal_val_est_error}}\\
        \leq & \hV^{k,n}(\pi^*_{r^{w}};\cD^{k,n-1}{}) \tag{$w$ is chosen}\\
        \leq & J_\beta(\pi^*_{r^w}) - J_\beta(\pi_\textref) + 32\cdot e^{2{\Rmax}}\sqrt{\frac{1}{N(w;\cD^{k,n})}\log\frac{|\Pi|WT}{\delta}},
    \end{align*}
    which implies,
    \begin{align*}
        N(w;\cD^{k,n}) \leq \frac{c_1^2 e^{4{\Rmax}}}{(\Delta_{\min} - \iota^{k,n})^2}\log\frac{|\Pi|WT}{\delta} \leq \frac{4c_1^2 e^{4{\Rmax}}}{\Delta_{\min}^2}\log\frac{|\Pi|WT}{\delta}.
    \end{align*}
    Therefore,
    \begin{align*}
        \forall w\in \cW_{\leq 2\Delta_{\min}}, \quad \sum_{(k,n)\in \cI^{\Transfer}_{\tK,\tN}} \mI[\cE^{n,k}_w \cap \cE^{n,k}_{2\iota < \Delta_{\min}}] \Delta(w) \leq \frac{8c_1^2 e^{4{\Rmax}}}{\Delta(w)}\log\frac{|\Pi|WT}{\delta}.
    \end{align*}
    %
    Combining with Eq.~\eqref{eq:iota_cases}, we have:
    \begin{align*}
        \forall w\in \cW_{\leq 2\Delta_{\min}}, \quad \sum_{(k,n)\in \cI^{\Transfer}_{\tK,\tN}} \mI[\cE^{n,k}_w] \Delta(w) \leq 4\sum_{(k,n)\in \cI^{\Transfer}_{\tK,\tN}} \mI[\cE^{n,k}_w] \cdot (\Delta_{\min} \wedge \iota^{n,k}) + \frac{8c_1^2 e^{4{\Rmax}}}{\Delta(w)}\log\frac{|\Pi|WT}{\delta}.
    \end{align*}
    By merging the analysis for $w\in\cW_{\leq 2\Delta_{\min}}$ and $w\in\cW_{>2\Delta_{\min}}$, we have:
    \begin{align*}
        \forall w\in[W],\quad \sum_{(k,n)\in \cI^{\Transfer}_{\tK,\tN}} \mI[\cE^{n,k}_w] \Delta(w) \leq & \frac{16c_1^2 e^{4{\Rmax}}}{\Delta(w)}\log\frac{|\Pi|WT}{\delta} + 4\sum_{(k,n)\in \cI^{\Transfer}_{\tK,\tN}} \mI[\cE^{n,k}_w]\Delta_{\min} \wedge \iota^{n,k},\numberthis\label{eq:reg_1}
    \end{align*}
    %
    %
    %
    %
    Note that for those $\Delta(w) \leq 4c_1 e^{2_{\max}} \cdot \sqrt{\frac{1}{\sum_{(k,n)\in \cI^{\Transfer}_{\tK,\tN}} \mI[\cE^{n,k}_w ]}\log\frac{|\Pi|WT}{\delta}}$, we automatically have 
    %
    %
    %
    \begin{align*}
        \sum_{(k,n)\in \cI^{\Transfer}_{\tK,\tN}} \mI[\cE^{n,k}_w ] \Delta(w) \leq & 4 c_1 e^{2_{\max}} \cdot \sqrt{\frac{1}{\sum_{(k,n)\in \cI^{\Transfer}_{\tK,\tN}} \mI[\cE^{n,k}_w ]}\log\frac{|\Pi|WT}{\delta}} \sum_{(k,n)\in \cI^{\Transfer}_{\tK,\tN}} \mI[\cE^{n,k}_w ]\\
        =&4 c_1 e^{2_{\max}} \cdot \sqrt{\sum_{(k,n)\in \cI^{\Transfer}_{\tK,\tN}} \mI[\cE^{n,k}_w]\log\frac{|\Pi|WT}{\delta}}.
    \end{align*}
    On the other hand, when $\Delta(w) > 4c_1 e^{2_{\max}} \cdot \sqrt{\frac{1}{\sum_{(k,n)\in \cI^{\Transfer}_{\tK,\tN}} \mI[\cE^{n,k}_w ]}\log\frac{|\Pi|WT}{\delta}}$, the bound in Eq.~\eqref{eq:reg_1} is tighter, since
    \begin{align*}
        \frac{16c_1^2 e^{4{\Rmax}}}{\Delta(w)}\log\frac{|\Pi|WT}{\delta} \leq 4c_1 e^{2_{\max}} \sqrt{\sum_{(k,n)\in \cI^{\Transfer}_{\tK,\tN}} \mI[\cE^{n,k}_w]\log\frac{|\Pi|WT}{\delta}}.
    \end{align*}
    Combining the above discussions,
    \begin{align*}
        &\sum_{(k,n)\in \cI^{\Transfer}_{\tK,\tN}} \mathbb{I}[\neg\cE^{k,n}_\SELF](J_\beta(\pi^*_{r^*}) - J_\beta(\pi^{k,n})) = \sum_{(k,n)\in \cI^{\Transfer}_{\tK,\tN}}\sum_{\substack{w\in[W] \\ \Delta(w) > 0}} \mI[\cE^{n,k}_w] \Delta(w) \\
        \leq & \sum_{\substack{w\in[W] \\ \Delta(w) > 0}} \min\{\frac{16c_1^2 e^{4{\Rmax}}}{\Delta(w)}\log\frac{|\Pi|WT}{\delta}, 4c_1 e^{2_{\max}} \sqrt{\sum_{(k,n)\in \cI^{\Transfer}_{\tK,\tN}} \mI[\cE^{n,k}_w]\log\frac{|\Pi|WT}{\delta}}\} \\
        & + 4\sum_{(k,n)\in \cI^{\Transfer}_{\tK,\tN}} \mI[\cE^{n,k}_w]\Delta_{\min} \wedge \iota^{n,k} \\
        \leq & \min\{\sum_{\substack{w\in[W] \\ \Delta(w) > 0}} \frac{16c_1^2 e^{4{\Rmax}}}{\Delta(w)}\log\frac{|\Pi|WT}{\delta}, \sum_{\substack{w\in[W] \\ \Delta(w) > 0}} 4c_1 e^{2_{\max}} \sqrt{\sum_{(k,n)\in \cI^{\Transfer}_{\tK,\tN}} \mI[\cE^{n,k}_w]\log\frac{|\Pi|WT}{\delta}}\} \tag{$\min\{a,b\} + \min\{x,y\} \leq \min\{a+x, b+y\}$}\\
        & + 4\sum_{(k,n)\in \cI^{\Transfer}_{\tK,\tN}} \mI[\cE^{n,k}_w]\Delta_{\min} \wedge \iota^{n,k} \\
        \leq & \min\{\sum_{\substack{w\in[W] \\ \Delta(w) > 0}} \frac{16c_1^2 e^{4{\Rmax}}}{\Delta(w)}\log\frac{|\Pi|WT}{\delta}, 4c_1 e^{2_{\max}} \sqrt{W |\cI^{\Transfer}_{\tK,\tN}|\log\frac{|\Pi|WT}{\delta}}\} \tag{Cauchy-Schwarz inequality and $\sum_{\substack{w\in[W] \\ \Delta(w) > 0}} \mI[\cE^{n,k}_w] \leq |\cI^{\Transfer}_{\tK,\tN}|$}\\
        & + 4\sum_{(k,n)\in \cI^{\Transfer}_{\tK,\tN}} \mI[\cE^{n,k}_w]\Delta_{\min} \wedge \iota^{n,k}.\numberthis\label{eq:reg_2}
    \end{align*}
    %
    %
    %
    %
    %
    \paragraph{Merge Everything Together}
    Combining Eq.~\eqref{eq:offline_accum_gap} and Eq.~\eqref{eq:reg_2}, we have:
    \begin{align*}
        &\sum_{(k,n)\in \cI^{\Transfer}_{\tK,\tN}} J_\beta(\pi^*_{r^*}) - J_\beta(\pi^{k,n}) \\
        =& \sum_{(k,n)\in \cI^{\Transfer}_{\tK,\tN}} \mathbb{I}[\cE^{k,n}_\SELF](J_\beta(\pi^*_{r^*}) - J_\beta(\pi^{k,n})) + \sum_{(k,n)\in \cI^{\Transfer}_{\tK,\tN}} \mathbb{I}[\neg\cE^{k,n}_\SELF](J_\beta(\pi^*_{r^*}) - J_\beta(\pi^{k,n}))\\
        \leq & \min\{\sum_{\substack{w\in[W] \\ \Delta(w) > 0}} \frac{16c_1^2 e^{4{\Rmax}}}{\Delta(w)}\log\frac{|\Pi|WT}{\delta}, 4c_1 e^{2_{\max}} \sqrt{W |\cI^{\Transfer}_{\tK,\tN}|\log\frac{|\Pi|WT}{\delta}}\} \\
        & + 4\sum_{(k,n)\in \cI^{\Transfer}_{\tK,\tN}} \Delta_{\min} \wedge \iota^{n,k}.
    \end{align*}
    Combining the value gap for the online parts, we have:
    \begin{align*}
        & \sum_{(k,n)\in\cI_{\tK,\tN}} J_\beta(\pi^*_{r^*}) - J_\beta(\pi^{k,n}) \\
        \leq & C_\Online {\Rmax} e^{2{\Rmax}} \sqrt{\Complexity(\Pi)|\cI^{\Online}_{\tK,\tN}| \log^{c_0}\frac{|\Pi|\alpha T}{\delta}} + 4\sum_{(k,n)\in \cI^{\Transfer}_{\tK,\tN}} \Delta_{\min} \wedge \iota^{n,k} \\
        & + \min\{\sum_{\substack{w\in[W] \\ \Delta(w) > 0}} \frac{16c_1^2 e^{4{\Rmax}}}{\Delta(w)}\log\frac{|\Pi|WT}{\delta}, 4c_1 e^{2_{\max}} \sqrt{W |\cI^{\Transfer}_{\tK,\tN}|\log\frac{|\Pi|WT}{\delta}}\}.
        %
    \end{align*}
    Note that for $\tK > 1$, denote $t_{\tK,\tN} := (\tK - 1)N + \tN$, we have:
    \begin{align*}
        |\cI^{\Online}_{\tK,\tN}| \leq \alpha \tK N \leq 2\alpha t_{\tK,\tN},\quad |\cI^{\Transfer}_{\tK,\tN}| \leq (1-\alpha) \tK N \leq 2(1-\alpha) t_{\tK,\tN}.
    \end{align*}
    Therefore,
    \begin{align*}
        \sum_{(k,n)\in\cI_{\tK,\tN}} J_\beta(\pi^*_{r^*}) - & J_\beta(\pi^{k,n}) = \tilde{O}\Big(\sum_{(k,n)\in \cI^{\Transfer}_{\tK,\tN}} \Delta_{\min} \wedge \iota^{n,k} +  {\Rmax} e^{2{\Rmax}} \sqrt{\alpha\Complexity(\Pi)t_{\tK,\tN}} + e^{2_{\max}} \sqrt{(1-\alpha)Wt_{\tK,\tN}} \wedge \sum_{\substack{w\in[W] \\ \Delta(w) > 0}} \frac{e^{4{\Rmax}}}{\Delta(w)}\Big),
    \end{align*}
    where in the last step, we omit the constant and logarithmic terms.
    By replacing $t_{\tK,\tN} \gets t$, $\sum_{(k,n) \in \cI_{\tK,\tN}} \gets \sum_{\tau \leq t} $, $k \gets k(\tau) := \lceil \frac{\tau}{N} \rceil$ and $n \gets n(\tau) := \tau \% N$, we finish the proof.
    

    %
    %
    %
    %
    %
    %
    %
    %
    %
    %
    %
    %
    %
    %
    %
    %
    %
    %
    %
    %
    %
    %
    %
    %
    %
    %
    %
    %
    %
    %


    \iffalse
    Next, we try to simplify
    \begin{align*}
        \sum_{\substack{1\leq k\leq K,\\ \alpha N < n \leq N}} \Delta_{\min} \wedge \iota^{n,k} \leq \min \{\Delta_{\min} T, \sum_{\substack{1\leq k\leq K,\\ \alpha N < n \leq N}} \iota^{n,k}\}.
    \end{align*}
    By Cauchy–Schwarz inequality,
    \begin{align*}
        \sum_{\substack{1\leq k\leq K,\\ \alpha N < n \leq N}} \iota^{n,k} \leq & c_2 {\Rmax} e^{2{\Rmax}} \cdot \sqrt{\Big(\sum_{\substack{1\leq k\leq K,\\ \alpha N < n \leq N}} \Big(\cov^{\pi^*_{r^*}|\pi_\mix^{k,n}} \wedge \frac{\sqrt{\Complexity(\Pi)}}{\alpha}\Big)^2\Big) \cdot \Big(\sum_{\substack{1\leq k\leq K,\\ \alpha N < n \leq N}} \frac{1}{|\cD^{k,n}|}\Big)} \log^{c_0}\frac{|\Pi|T}{\delta}  \\
        = & \tilde{O}\Big({\Rmax} e^{2{\Rmax}} \cdot \sqrt{\frac{\sum_{\substack{1\leq k\leq K,\\ \alpha N < n \leq N}} \Big(\cov^{\pi^*_{r^*}|\pi_\mix^{k,n}} \Big)^2}{T}} \wedge \sqrt{\frac{(1-\alpha)\Complexity(\Pi)}{\alpha}} \sqrt{T})
    \end{align*}
    By merging them together, we have:
    \begin{align*}
        & \sum_{k=1}^K \sum_{1 < n\leq N} J_\beta(\pi^*_{r^*}) - J_\beta(\pi^{k,n}) \\
        \leq & \tilde{O}\Big(\min\{\Delta_{\min}(1-\alpha) T, {\Rmax} e^{2{\Rmax}} \cdot \sqrt{\sum_{\substack{1\leq k\leq K,\\ \alpha N < n \leq N}} \Big(\cov^{\pi^*_{r^*}|\pi_\mix^{k,n}} \Big)^2}, \sqrt{\frac{(1-\alpha)\Complexity(\Pi)T}{\alpha}}\}\Big) \\
        & + \tilde{O}\Big({\Rmax} e^{2{\Rmax}}\sqrt{\alpha \Complexity(\Pi)T} + \sqrt{(1-\alpha)}e^{2{\Rmax}}\min\{\frac{e^{2{\Rmax}}W}{\Delta_{\min}}, \sqrt{WT}\}\Big)
    \end{align*}
    \fi

    %
    %
    %
    %
    %
    %
    %
    %
    %

    %

    %
    %
    %
    %
    %
    %
    %
    %

\end{proof}




\section{Connection between Win Rate and Policy Coverage Coefficient}\label{appx:win_rate_and_coverage}

\begin{lemma}\label{lem:diff_and_preference}
    Given two probability vector $u, v \in \Delta(\cA)$ and a reward function $r:\cA\rightarrow\mR$, consider a preference model based on $r$, satisfying,
    \begin{align*}
        \mP_r(y=1|a,a') \geq \frac{1}{2},
    \end{align*}
    for any $a,a'\in\cA$ satisfying $r(a) \geq r(a')$. Then,
    \begin{align*}
        \sum_a \sqrt{u(a) v(a)} \leq \min_{\gamma > 0} \sqrt{(\gamma + 2\mP_r(v\succ u))\log\frac{1 + \gamma}{\gamma}},
    \end{align*}
    where $\mP_r(u \succ v) := \EE_{a\sim u,a'\sim v}[\mP_r(y=1|a, a')]$.
\end{lemma}
\begin{proof}
    We first of all sort the action space $\cA$ to $\cA_{\sorted} := \{a_1,a_2,...,a_{|\cA|}\}$ according to reward function $r$, such that, for any $1\leq i<j \leq |\cA|$, $r(a_i) \leq r(a_j)$.
    Besides, we use $F^u(\cdot)$ to denote the cumulative distribution function regarding $u$:
    $$
        \forall 1\leq i \leq |\cA|,\quad F^u(a_i) := \sum_{j=1}^{i} u(a_i),
    $$
    and $F^{v}$ is defined similarly. Then we have:
    \begin{align*}
        \sum_{a\in\cA} \sqrt{u(a)v(a)} =& \sum_{i=1}^{|\cA|} \sqrt{u(a_i)v(a_i)} = \sum_{i=1}^{|\cA|} \sqrt{\frac{u(a_i)}{\gamma + F^u(a_i)}} \cdot \sqrt{(\gamma + F^u(a_i))v(a)} \tag{Introducting a parameter $\gamma > 0$}\\
        \leq & \sqrt{\sum_{i=1}^{|\cA|} \frac{u(a_i)}{\gamma + F^u(a_i)}} \cdot \sqrt{\sum_{i=1}^{|\cA|}  (\gamma + F^u(a_i))v(a_i)} \tag{Cauchy–Schwarz inequality}\\
        \leq & \sqrt{\sum_{i=1}^{|\cA|} \frac{u(a_i)}{\gamma + F^u(a_i)}} \cdot \sqrt{\gamma + 2\mP_r(v\succ u)},
    \end{align*}
    where in the last step, we use the fact that $\gamma\sum_{i=1}^{|\cA|} v(a_i) = \gamma$ and 
    \begin{align*}
        \mP_r(v\succ u) =& \EE_{a\sim v,a' \sim u}[\mP_r(y=1|a,a')] \geq \sum_{i=1}^{|\cA|} v(a_i) \sum_{j=1}^i u(a_j) \mP_r(y=1|a_i,a_j) \geq \frac{1}{2}\sum_{i=1}^{|\cA|} v(a_i) F^u(a_i).
    \end{align*}
    For the first part, we can upper bound by the following:
    \begin{align*}
        \sum_{i=1}^{|\cA|} \frac{u(a_i)}{\gamma + F^u(a_i)} = & \sum_{i=1}^{|\cA|} \frac{u(a_i)}{\gamma + \sum_{j=1}^{i} u(a_j)} = \sum_{i=1}^A 1 - \frac{\gamma + \sum_{j=1}^{i-1} u(a_j)}{\gamma + \sum_{j=1}^{i} u(a_j)} \\
        \leq & \sum_{i=1}^A \log \frac{\gamma + \sum_{j=1}^{i} u(a_j)}{\gamma + \sum_{j=1}^{i-1} u(a_j)} \tag{$1 - x \leq \log \frac{1}{x}$} \\
        \leq & \log \frac{1+\gamma}{\gamma}
    \end{align*}
    Therefore, $\sum_{a\in\cA}\sqrt{u(a)v(a)} \leq \sqrt{(\gamma + 2\mP_r(v\succ u))\log\frac{1+\gamma}{\gamma}}$.
    Since $\gamma$ is arbitrary, we can take the minimum over $\gamma > 0$, and finish the proof.
    %
    %
    %
    %
    %
\end{proof}


%
%
%
%
%
%
%
%
%
%
%
%
%
%
%
%
%
%
%
%


\begin{lemma}\label{lem:TV_to_Preference}
    %
    %
    %
    For any policy $\pi, \tpi$,
    \begin{align*}
        &1 - \TV(\pi(\cdot|s)\|\tpi(\cdot|s)) \leq \min_{\gamma > 0} \sqrt{(\gamma + 2\mP_{r^*}(\pi(\cdot|s) \succ \tpi(\cdot|s))) \log \frac{1 + \gamma}{\gamma}},\\
        &1 - \EE_{s\sim\rho}[\TV(\pi(\cdot|s)\|\tpi(\cdot|s))] \leq \min_{\gamma > 0} \sqrt{(\gamma + 2 \mP_{r^*}(\pi\succ \tpi)) \log \frac{1 + \gamma}{\gamma}}.
    \end{align*}
\end{lemma}
\begin{proof}
    \begin{align*}
        1 - \TV(\pi(\cdot|s)\|\tpi(\cdot|s)) \leq & 1 - \mH^2(\pi(\cdot|s)\|\tpi(\cdot|s)) = \sum_{a\in\cA} \sqrt{\pi(a|s) \tpi(a|s)} \\
        \leq & \min_{\gamma > 0}\sqrt{\Big(\gamma + 2 \mP_{r^*}(\pi(\cdot|s) \succ \tpi(\cdot|s)) \Big)\log\frac{1+\gamma}{\gamma}}.
        %
        %
        %
        %
        %
    \end{align*}
    where in the last step we apply Lem.~\ref{lem:diff_and_preference} with $\pi(\cdot|s)$ as $v(\cdot)$, $\tpi(\cdot|s)$ as $u(\cdot)$, $r^*(s,\cdot)$ as the reward function.
    %
    %
    Then, we finish the proof for the first inequality.

    For the second inequality, by taking the expectation over $s\sim\rho$ and the concavity of $\sqrt{\cdot}$ function, we have:
    \begin{align*}
        1 - \EE_{s\sim\rho}[\TV(\pi(\cdot|s)\|\tpi(\cdot|s))] \leq \min_{\gamma > 0}\sqrt{\Big(\gamma + 2\mP_{r^*}(\pi\succ\tpi)\Big) \log\frac{1 + \gamma}{\gamma}}.
    \end{align*}
    By choosing $\gamma = \mP_{r^*}(\pi\succ\tpi)$ (note that this choice ensures $\gamma > 0$ since $\tpi(\cdot|\cdot) > 0$), we finish the proof.
\end{proof}

\begin{restatable}{lemma}{LemLBCoverage}[The Complete Version of Lem.~\ref{lem:BT_LB_coverage}]\label{lem:LB_coverage_formal}
    Given any policy $\pi$, under the assumption that $\mP_{r^*}(y=1|s,a,a') \geq \frac{1}{2}$ for any $a,a'\in\cA$ satisfying $r^*(s,a) \geq r^*(s,a')$, we have:
    \begin{align}
        \cov^{\pi^*_{r^*}|\pi}\geq & \max_{\gamma > 0, \bpi} \Big(\sqrt{\gamma + 2\mP_{r^*}(\bpi\succ\pi)\log\frac{1+\gamma}{\gamma}} +\sqrt{\frac{J_\beta(\pi^*_{r^*}) - J_\beta(\bpi)}{2\beta}}\Big)^{-1} \label{eq:LB_2}\\
        \cov^{\pi^*_{r^*}|\pi} \geq & \max_{\gamma > 0, \bpi} \Big(\sqrt{\gamma + 2\mP_{r^*}(\pi\succ\bpi)\log\frac{1+\gamma}{\gamma}}+\sqrt{\frac{J_\beta(\pi^*_{r^*}) - J_\beta(\bpi)}{2\beta}}\Big)^{-1}. \label{eq:LB_1}
    \end{align}
    where $\bar{\pi}$ is an arbitrary intermediate policy, $\mP_{r^*}(\pi\succ \tpi) := \EE_{s\sim\rho,a\sim\pi,a'\sim\tpi}[\mP_{r^*}(y=1|s,a,a')]$ and $\mP_{r^*}(\tpi\succ \pi) = 1 - \mP_{r^*}(\pi\succ \tpi) = \EE_{s\sim\rho,a\sim\tpi,a'\sim\pi}[\mP_{r^*}(y=1|s,a,a')]$.
\end{restatable}
\begin{proof}
    We have:
    \begin{align*}
        \EE_{a\sim \pi^*_{r^*}(\cdot|s)}[\frac{\pi^*_{r^*}(a|s)}{\pi(a|s)}] - 1=& \chi^2(\pi^*_{r^*}(\cdot|s)\|\pi(\cdot|s)) \\
        \geq & \exp(\KL(\pi^*_{r^*}(\cdot|s) \| \pi(\cdot|s))) - 1 \tag{Theorem 5 in \citep{gibbs2002choosing}}\\
        \geq & \frac{1}{2} \cdot \frac{1}{1 - \TV(\pi^*_{r^*}(\cdot|s)\|\pi(\cdot|s))} - 1 \tag{Bretagnolle–Huber inequality}.
         %
    \end{align*}
    Now, we introduce an arbitrary intermediate policy $\tpi$, 
    \begin{align*}
        \TV(\pi^*_{r^*}(\cdot|s)\|\pi(\cdot|s))
        \geq & \TV(\bpi(\cdot|s)\|\pi(\cdot|s)) - \TV(\bpi(\cdot|s)\|\pi^*_{r^*}(\cdot|s)) \tag{Reverse triangle inequality}\\
        %
        \geq & \TV(\bpi(\cdot|s)\|\pi(\cdot|s)) - \sqrt{\frac{1}{2} \KL(\bpi(\cdot|s)\|\pi^*_{r^*}(\cdot|s))} \tag{Pinsker's inequality}.
    \end{align*}
    %
    %
    %
    %
    Applying Lem.~\ref{lem:TV_to_Preference} with $(\pi, \tpi) \gets (\pi, \pi)$, we have:
    \begin{align*}
        %
        \EE_{a\sim \pi^*_{r^*}(\cdot|s)}[\frac{\pi^*_{r^*}(a|s)}{\pi(a|s)}] \geq & \frac{1}{1 - \TV(\bpi(\cdot|s)\|\pi(\cdot|s)) + \sqrt{\frac{1}{2} \KL(\bpi(\cdot|s)\|\pi^*_{r^*}(\cdot|s))}} \\
        \geq & \frac{1}{\sqrt{(\gamma + 2\mP_{r^*}(\bpi(\cdot|s) \succ \pi(\cdot|s))) \log \frac{1+\gamma}{\gamma}} + \sqrt{\frac{1}{2} \KL(\bpi(\cdot|s)\|\pi^*_{r^*}(\cdot|s))}}.
    \end{align*}
    By taking the expectation over $s\sim\rho$, and leveraging the convexity of $1/x$ and the concavity of $\sqrt{\cdot}$ functions, we have:
    \begin{align*}
        \EE_{s\sim\rho,a\sim \pi^*_{r^*}(\cdot|s)}[\frac{\pi^*_{r^*}(a|s)}{\pi(a|s)}] \geq& \frac{1}{\EE_{s\sim\rho}[\sqrt{(\gamma + 2\mP_{r^*}(\bpi(\cdot|s) \succ \pi(\cdot|s))) \log \frac{1+\gamma}{\gamma}}] + \EE_{s\sim\rho}[\sqrt{\frac{1}{2} \KL(\bpi(\cdot|s)\|\pi^*_{r^*}(\cdot|s))}]} \\
        \geq & \frac{1}{\sqrt{(\gamma + 2\mP_{r^*}(\bpi\succ\pi))\log\frac{1+\gamma}{\gamma}}+\sqrt{\frac{J_\beta(\pi^*_{r^*}) - J_\beta(\pi)}{2\beta}}}.
        %
        %
    \end{align*}
    %
    Note that the above results hold for any $\gamma > 0$ and any $\bpi$, we finish the proof by taking the maximum over them.
    %

    The second inequality in Lem.~\ref{lem:LB_coverage_formal} can be proved similarly by applying Lem.~\ref{lem:TV_to_Preference} with $(\pi, \tpi) \gets (\pi, \bpi)$. All the discussion are the same.
\end{proof}

\begin{remark}\label{remark:LB_coverage}
    We provide some remarks about Lem.~\ref{lem:LB_coverage_formal}.
    \begin{itemize}
        \item Lem.~\ref{lem:BT_LB_coverage} is a direct corollary of Eq.~\eqref{eq:LB_1}.

        \item Notably, Eq.~\eqref{eq:LB_2} has a different implication compared with Eq.~\eqref{eq:LB_1} that we follow in the algorithm design in Sec.~\ref{sec:empirical_alg}.
        More concretely, Eq.~\eqref{eq:LB_2} suggests we should also disregard those source policies that strongly dominate $\bpi$ when $\bpi$ is close to $\pi^*_{r^*}$.
        This makes sense because those source policies may achieve high rewards or win rates by incurring a high KL divergence with $\pi_\textref$, and therefore, they may not provide good coverage for $\pi^*_{r^*}$.
    
        \item However, in Alg.~\ref{alg:empirical}, we intentionally do not filter out but instead prioritize source policies with exceptionally high win rates. 
        Because they likely provide good coverage for high-reward regions and can be advantageous for practical LLM training.
        Nonetheless, for completeness, we bring this theory-practice gap into attention.
    \end{itemize}
\end{remark}

\section{Useful Lemmas}\label{appx:basic_lemma}

\begin{lemma}[Convex Hull Fulfills Assump.~\ref{assump:policy}]\label{lem:convex_hull_property}
    Given $\Pi$ satisfying Assump.~\ref{assump:policy}, $\conv(\Pi)$ also satisfies Assump.~\ref{assump:policy}.
\end{lemma}
\begin{proof}
    The realizability condition is obviously. We verify Assump.~\ref{assump:policy}.
    Note that for any $\pi \in \conv(\Pi)$, there exists $\lambda^1,...,\lambda^n \geq 0$ and $\pi^1,...,\pi^n \in \Pi$, s.t. $\sum_{i=1}^n \lambda^i = 1$ and $\pi = \sum_{i=1}^n \lambda^i \pi^i$, which implies,
    \begin{align*}
        \forall s,a\quad & \frac{\pi(a|s)}{\pi_\textref(a|s)} = \sum_{i=1}^n \lambda^i \frac{\pi^i(a|s)}{\pi_\textref(a|s)} \geq \exp(-\frac{\Rmax}{\beta}),
        & \frac{\pi(a|s)}{\pi_\textref(a|s)} = \sum_{i=1}^n \lambda^i \frac{\pi^i(a|s)}{\pi_\textref(a|s)} \leq \exp(\frac{\Rmax}{\beta}).
    \end{align*}
    Therefore, $\conv(\Pi)$ fulfills Assump.~\ref{assump:policy}-(II), which finishes the proof.
\end{proof}

\begin{lemma}[MLE Guarantees; Adapated from Lemma C.6 in \citep{xie2024exploratory}]\label{lem:MLE_Estimation}
    Consider a policy class $\Pi$ satisfying Assump.~\ref{assump:policy}, and recall the reward class $\cR^\Pi$ converted by Eq.~\eqref{eq:reward_class_conversion}.
    Given a dataset $\cD := \{(s^i,a^i,\ta^i,y^i,\pi^i)\}_{i\leq |\cD|}$ satisfying Cond.~\ref{cond:seq_data} and any $\delta\in(0,1)$, w.p. $1-\delta$,
    %
    %
    %
    %
    %
    \begin{align*}
        \forall r\in\cR^\Pi,~\frac{1}{|\cD|}\sum_{i \leq |\cD|} \EE_{s\sim\rho,a\sim\pi^i(\cdot|s),\ta\sim\pi_\textref(\cdot|s)}[\mH^2(\mP_{r}(\cdot|s,a,\ta)\| \mP_{r^*}(\cdot|s,a,\ta))] \leq L_{\cD}(r) - L_{\cD}(r^*) + \frac{2}{|\cD|}\log\frac{|\Pi|}{\delta}.
    \end{align*}
    %
    %
    %
\end{lemma}
\begin{proof}
    The proof is almost identical to Lemma C.6 in \citep{xie2024exploratory}, except we replace $\mP_\pi$ in their paper by $\mP_r$. So we omit it here.
    Besides, note that our NLL loss is normalized, while \citep{xie2024exploratory} consider unnormalized version. This results in the additional $\frac{1}{|\cD|}$ factor here.
\end{proof}
%
%
%
%
%
%
%
%
%
%
%
%
%
%

%
%
%
%
%
%
%
%
%
%
%
%
%
%
%
%

\begin{lemma}[From reward error to Hellinger  Distance]\label{lem:r_err_to_Hellinger}
    Given any policy $\pi$, any reward function $r$ with bounded value range $[-\Rmax,{\Rmax}]$, and another arbitrary $\tpi$ with positive support on $\cS\times\cA$, we have:
    \begin{align*}
        &\EE_{s\sim\rho,a\sim\pi(\cdot|s),\ta\sim\pi_\textref(\cdot|s)}[|\Big(r^*(s,a) - r^*(s,\ta)\Big) - \Big(r(s,a) - r(s,\ta)\Big)|] \\
        \leq & 8\sqrt{2}e^{2{\Rmax}} \sqrt{\cov^{\pi|\tpi} \cdot \EE_{s\sim\rho,a\sim\tpi(\cdot|s),\ta\sim\pi_\textref(\cdot|s)}[\mH^2(\mP_{r}(\cdot|s,a,\ta)\|\mP_{r^*}(\cdot|s,a,\ta))]}.
    \end{align*}
\end{lemma}
\begin{proof}
    For any reward function $r$, we have:
    \begin{align*}
        &\Big|\Big(\EE_{\rho,\pi}[r^*] - \EE_{\rho,\pi_\textref}[r^*]\Big) - \Big(\EE_{\rho,\pi}[r] - \EE_{\rho,\pi_\textref}[r]\Big)\Big| \\
        \leq & \EE_{s\sim\rho,a\sim\pi(\cdot|s),\ta\sim\pi_\textref(\cdot|s)}[|\Big(r^*(s,a) - r^*(s,\ta)\Big) - \Big(r(s,a) - r(s,\ta)\Big)|] \\
        \leq & 4e^{2{\Rmax}} \EE_{s\sim\rho,a\sim\pi(\cdot|s),\ta\sim\pi_\textref(\cdot|s)}[|\sigma\Big(r^*(s,a) - r^*(s,\ta)\Big) - \sigma\Big(r(s,a) - r(s,\ta)\Big)|] \tag{Lem.~\ref{lem:sigmoid} with $C = 2\Rmax$} \\
        = & 4e^{2{\Rmax}} \sum_{s,a,\ta} \sqrt{\rho(s) \pi_\textref(\ta|s)} \frac{\pi(a|s)}{\sqrt{\tpi(a|s)}} \\
        &\quad\quad\quad\quad\qquad\qquad \cdot \sqrt{\rho(s) \pi_\textref(\ta|s)}\sqrt{\tpi(a|s)}\Big|\sigma\Big(r^*(s,a) - r^*(s,\ta)\Big) - \sigma\Big(r(s,a) - r(s,\ta)\Big)\Big| \\
        \leq & 4e^{2{\Rmax}} \sqrt{\sum_{s,a,\ta} \rho(s) \pi_\textref(\ta|s) \frac{\pi^2(a|s)}{\tpi(a|s)}} \\
        &\quad\quad\quad\quad\qquad\qquad \sqrt{\EE_{s\sim\rho,a\sim\tpi(\cdot|s),\ta\sim\pi_\textref(\cdot|s)}[\Big|\sigma\Big(r^*(s,a) - r^*(s,\ta)\Big) - \sigma\Big(r(s,a) - r(s,\ta)\Big)\Big|^2]} \tag{Cauchy–Schwarz inequality} \\
        =& 4e^{2{\Rmax}} \sqrt{\cov^{\pi|\tpi}} \cdot \sqrt{\EE_{s\sim\rho,a\sim\tpi(\cdot|s),\ta\sim\pi_\textref(\cdot|s)}[\Big|\sigma\Big(r^*(s,a) - r^*(s,\ta)\Big) - \sigma\Big(r(s,a) - r(s,\ta)\Big)\Big|^2]} \numberthis\label{eq:eq_ref_2}.
    \end{align*}
    Note that, 
    \begin{align*}
        &\EE_{s\sim\rho,a\sim\tpi(\cdot|s),\ta\sim\pi_\textref(\cdot|s)}[\Big|\sigma\Big(r^*(s,a) - r^*(s,\ta)\Big) - \sigma\Big(r(s,a) - r(s,\ta)\Big)\Big|^2]\\
        \leq & 8 \EE_{s\sim\rho,a\sim\tpi(\cdot|s),\ta\sim\pi_\textref(\cdot|s)}[|\sqrt{\sigma(r^*(s,a) - r^*(s,\ta))} - \sqrt{\sigma(r(s,a) - r(s,\ta))}|^2] \tag{$(x-y)^2 \leq 4(x+y)(\sqrt{x} - \sqrt{y})^2$}\\
        \leq & 8 \EE_{s\sim\rho,a\sim\tpi(\cdot|s),\ta\sim\pi_\textref(\cdot|s)}[\mH^2(\mP_{r}(\cdot|s,a,\ta)\|\mP_{r^*}(\cdot|s,a,\ta))]\numberthis\label{eq:eq_ref_1}.
    \end{align*}
    By plugging into Eq.~\eqref{eq:eq_ref_2}, we finish the proof.
\end{proof}

\begin{lemma}[Sigmoid Function]\label{lem:sigmoid}
    Given $x,y \in [-C, C]$ for some $C > 0$,
    \begin{align*}
        |x - y| \leq 4\exp(C)|\sigma(x) - \sigma(y)|.
    \end{align*}
\end{lemma}
\begin{proof}
    Without loss of generality, we assume $x \geq y$. Because $\sigma(\cdot)$ is a monotonically increasing function and it is continuous, we know there exists $z \in [y, x]$ s.t. 
    \begin{align*}
        \frac{\sigma(x) - \sigma(y)}{x - y} = \sigma'(z) = \sigma(z)(1 - \sigma(z)) = \sigma(z) \sigma(-z).
    \end{align*}
    Because $x,y \in [-C,C]$, we have $\sigma(-C) \leq \sigma(z) \leq \sigma(C)$.
    Note that the axis of symmetry of function $f(a) = a(1-a)$ is $1/2$ and $\frac{1}{2} - \sigma(C) = \sigma(-C) - \frac{1}{2}$. Therefore,
    \begin{align*}
        |x - y| =& \frac{1}{\sigma'(z)}|\sigma(x) - \sigma(y)| \leq (1 + \exp(C))(1 + \exp(-C)) \cdot |\sigma(x) - \sigma(y)| \\
        \leq & 2(1+\exp(C))|\sigma(x) - \sigma(y)| \leq 4\exp(C)|\sigma(x) - \sigma(y)|
    \end{align*}
\end{proof}

%
%
%
%
%
%
%
%
%
%
%
%
%
%
%
%
%

%
%
%
%
%


%
%
%
%
%
%
%
%
%
%
%
%
%



\section{Experiment Details and Additional Results}\label{appx:experiment}

\subsection{Details in Experiment Setup}
\paragraph{Setup of $r^*$}
Due to the high cost of collecting real human feedback, we use preferences generated by Llama3-8B \citep{dubey2024llama} to simulate the ground-truth human annotations.
More concretely, we adopt \texttt{sfairXC/FsfairX-LLaMA3-RM-v0.1} \citep{dong2405rlhf} as the true reward model $r^*$, which is distilled from \texttt{meta-llama/Meta-Llama-3-8B-Instruct}.
This reward model can be queried with prompt-response pairs and returns reward scores for each of them.


\paragraph{Best-of-N as an Approximation of $\pi^*_{r^w}$}
We recall that we consider 4 source tasks for transfer learning, including, (a) ROUGE-Lsum score \citep{lin2004rouge}, (b) BERTScore \citep{zhang2019bertscore}, (c) T5-base (250M) $\pi_{\text{base}}$, (d) T5-large (770M) $\pi_{\text{large}}$.

To reduce computational complexity, instead of explicitly training the optimal policies $\{\pi^*_{r^w}\}_{w\in[W]}$ associated with each source reward model, we use Best-of-N (BoN) approach, a.k.a. rejection sampling, to approximate the responses generated by $\pi^*_{r^w}$.
Specifically, we generate $\texttt{N}$\footnote{To distinguish with $N$ used to denote the block size in Alg.~\ref{alg:empirical}, we use $\texttt{N}$ to denote the size of BoN.} responses by the online learning policy $\pi^k_\base$ in Alg.~\ref{alg:empirical}, rank them according to the reward model, and select the top-ranked ones.

Furthermore, even for source LLM policies (3) and (4), we find that transferring from BoN-selected responses generated by $\pi^k_\base$ actually outperforms directly using the responses generated by T5-base/large.
We hypothesize that it is because the responses by T5-base/large usually have quite low probability of being generated by the online learning policy, leading to a distribution shift that complicates learning.
In contrast, BoN-selected responses maintain non-trivial probability of being sampled, without significant distributional mismatch.

Next, we elaborate the BoN process with more details.
In our experiment, we choose $\texttt{N}=32$.
For source reward models (1) and (2), we generate $\texttt{N}$ responses, and we compute the ROUGE-Lsum/BERTScore between the generated response and the human-provided summary in XSum dataset as the reward value.
For source policies (3) and (4), motivated by the closed-form solution in Eq.~\eqref{eq:closed_form}, given any prompt $s$ and response $a$, we infer the log-probability of T5-base/large predicting $a$ given $s$ as the reward score, i.e. $\log \pi_{\text{base}}(a|s)$ or $\log \pi_{\text{large}}(a|s)$.
In another word, we interpret T5-base/large as the optimal policies fine-tuned from uniform distribution to align with some reward models, which we treated as source rewards for transfer learning.


\paragraph{Training Details}
Our training setup is based on and adapted from \citep{xiong2024iterative,xie2024exploratory}.
We run for 3 iterations ($K=3$), and in each iteration, we sample a training dataset of size 10k (i.e., the block size $N=$10k). For each prompt, we collect 8 responses as follows.
Firstly, we generate $\texttt{N} + 4$ responses by the online learning policy $\pi^k_\base$.
The initial $\texttt{N}$ responses are used for Best-of-N (BoN) selection, where we choose a source reward model via the UCB strategy in Alg.~\ref{alg:empirical}, and then pick the top 4 from $\texttt{N}$ responses with the highest source rewards.
These 4 responses are merged with the remaining 4 responses and we get a total of 8 responses.
After that, we query $r^*$ to label the reward for those responses, and record the ones with the highest and lowest rewards to serve as positive and negative samples for $\DPO$ training.

In contrast to the procedure presented in Alg.~\ref{alg:empirical}, we utilize 8 responses for each prompt.
Therefore, the win rates are computed in a relatively different way.
Specially, we set $y^{k,n} = 1$ if the response achieving the highest reward comes from the 4 responses selected by the BoN step, and $y^{k,n} = 0$ otherwise.
The updates of the win rates estimation and the computation of UCB bonus terms align with Alg.~\ref{alg:empirical}, except that we set $\hat{\text{WR}}^{\pi^k_\base} = 0.55$ instead of 0.5.
This adjustment establishes a higher threshold for enabling transfer learning, requiring source tasks to outperform the baseline policy $\pi^k_\base$ by a larger margin before being considered.
We believe it enhances overall performance.


Regarding other hyperparameters during the training, the learning rate is 5e-5 with a cosine annealing schedule.
Training is conducted on 4 H-100 GPUs with total batch size 64.
We set the constant parts in the UCB bonus $c \sqrt{\log\frac{1}{\delta}} = 1.0$ in practice, considering the value range of win rates is [0, 1].


\paragraph{Evaluation Details}
During the evaluation phase, we randomly sample 10k prompts from XSum test dataset without repetition. 
For each prompt, we generate one response for each of the policies being compared, and query their reward values from $r^*$ (i.e., the \texttt{sfairXC}\texttt{/FsfairX-LLaMA3-RM-v0.1} reward model).
The win rate is then estimated as the frequency of that one policy generates a response with higher rewards than the other across the 10k prompts.

\subsection{Additional Experiment Results}\label{appx:additional_results}

\paragraph{Results under Other Choices for $\text{Alg}_{\text{PO}}$ in Alg.~\ref{alg:empirical}}
In the following, we report the results with two alternative instaniations of $\text{Alg}_{\text{PO}}$: by optimizing the XPO loss \citep{xie2024exploratory} or the IPO loss \citep{azar2024general}.
All the training setups are the same as the experiments where $\text{Alg}_{\text{PO}}$ is DPO, except that we choose a smaller learning rate 1e-5 for $\text{Alg}_{\text{PO}}$ is IPO.

\begin{remark}
    Different from DPO and IPO, XPO is an online algorithm itself and in their original design, the pairs of online data are generated by an online exploration policy and another fixed base policy, respectively.
    However, empirically, \citet{xie2024exploratory} follow the iterative-DPO and utilize the same online learning policy to generate pairs of online data.
    This exactly aligns with the no transfer baseline we compete with---instantiating $\text{Alg}_{\text{PO}}$ with XPO in Alg.~\ref{alg:empirical} and setting $W=0$, which we refer as iterative-XPO in this paper.
\end{remark}

\begin{table}[h]
    \begin{subtable}{0.52\textwidth}
    \begin{tabular}{cccc}
        \hline
                & \makecell{Without \\ Transfer} & \makecell{Purely Exploit \\ ROUGE-Lsum} & \makecell{Purely Exploit \\ T5-Large} \\
                \hline
         Iter 1 &  $52.3\pm1.0$ & $50.4\pm1.6$ & $49.9\pm0.4$\\
        %
         Iter 2 &  $55.2\pm1.4$ & $52.3\pm0.3$ & $50.1\pm0.3$\\
        %
         Iter 3 &  $55.3\pm1.1$ & $51.8\pm0.5$ & $50.3\pm0.5$\\\hline
    \end{tabular}
    \caption{IPO as $\text{Alg}_{\text{PO}}$ in Alg.~\ref{alg:empirical}}
    \label{tab:IPO}
    \end{subtable}
    \begin{subtable}{0.52\textwidth}
    \begin{tabular}{cccc}
        \hline
                & \makecell{Without \\ Transfer} & \makecell{Purely Exploit \\ ROUGE-Lsum} & \makecell{Purely Exploit \\ T5-Large} \\
                \hline
         Iter 1 &  $52.3\pm 1.1$ & $53.4\pm0.8$ & $50.2\pm0.3$\\
         Iter 2 &  $51.6\pm1.3$ & $54.7\pm1.6$ & $49.1\pm1.3$\\
         Iter 3 &  $52.2\pm1.6$ & $53.8\pm2.9$ & $49.2\pm1.1$\\\hline
    \end{tabular}
    \caption{XPO as $\text{Alg}_{\text{PO}}$ in Alg.~\ref{alg:empirical}}
    \label{tab:XPO}
    \end{subtable}
    \caption{Similar to Table~\ref{tab:experiment}, we report the win rates (\%) of the policies trained by empirical $\TPO$ (Alg.~\ref{alg:empirical}) competed with 3 baselines, presented across 3 columns. {Baseline (I)}: without transfer, i.e., iterative-IPO or iterative-XPO. {Baseline (II)}:  purely utilizing ROUGE-LSum (the lowest-quality source task) in transfer learning. {Baseline (III)}: purely utilizing T5-Large (the highest-quality source task) in transfer learning. Results are averaged with 3 random seeds and 95\% confidence levels are reported.}
\end{table}



\paragraph{Investigation on Source Task Selection}
Fig.~\ref{fig:selection_details} provides further investigations on the source task selection process.
For each iteration $k=1,2,3$, we count the number of times that $\pi^{k,n}$ is occupied by different transfer policies $\{\pi^*_{r^w}\}_{w\in[W]}$ or the online learning policy $\pi^k_\base$ (i.e. without transfer), and provide the results on the top sub-figure in Fig.~\ref{fig:selection_details}.
Besides, in the bottom sub-figure, we report the win rates $\mP_{r^*}(\pi \succ \pi^k_\base)$ for all $\pi \in \{\pi^*_{r^w}\}_{w\in[W]} \cup \{\pi^k_\base\}$.
As illustrated, the UCB sub-routine efficiently explores and identify the source task with the highest win rates against the learning policy $\pi^k_\base$.

Notably, as the improvement of $\pi^k_\base$ over the three iterations, we can observe the transition from transfer learning by leveraging high-quality source tasks to standard online learning.
In other words, our method can automatically switch back to online learning and avoid being restricted by source reward models.


\begin{figure}[t]
    \centering
    \includegraphics[scale=0.6]{Pictures/Results_Num.pdf}
    \includegraphics[scale=0.6]{Pictures/Results_WR.pdf}
    \caption{Deeper investigation on the source reward models selection process. We report the allocation of transfer budgets on each source tasks averaged over 3 trials (top figure) and the win rates $\mP_{r^*}(\cdot\succ\pi^k_\base)$ (bottom figure) for iterations $k=1,2,3$.
    Due to space limit, we use abbreviation rather than the full name of source tasks.
    \text{R}, \text{B}, \text{TB}, \text{TL} and \text{NT} stand for ROUGE-Lsum, BERTScore, T5-Base, T5-Large and No Transfer, respectively.
    }\label{fig:selection_details}
\end{figure}

%

\end{document}
