\section{Theoretical Background}\label{sec:theory}

\noindent Accountability is considered the link between individuals and their social system, creating an identity relationship associating individuals with their actions and performance \citep{tetlock1992impact,mero2007accountability}. Accountability has implications in all organizational levels \citep{frink1998toward}. However, our focus, in this study, is the micro-level or \emph{felt accountability} (referred to simply as ``accountability'') which refers to the individual's perception of accountability \citep{frink1998toward}; the most studied type of accountability \citep{hall2017accountability}. This type of accountability is contingent upon the individual's own interpretation \citep{frink1998toward}, rather than the external milieu imposed expectations \citep{folger2001fairness}.

Irrespective of the nature of a collective, ranging from a dyad to a civilization, addressing coordination and collaboration among its constituent members with diverse interests is imperative \citep{schlenker1994triangle}. Accountability comes to play by establishing shared expectations in social systems. When standards are set, people must adhere to them and the failure to do so may result in imposing penalties \citep{schlenker1994triangle,frink1998toward}. Compliance with standards is evaluated at various layers of social systems, including the individual, the dyad, the group, the organization, and the society as a whole \citep{gelfand2004culture,frink1998toward}.

The concept of accountability inherently implies an anticipated evaluation \citep{hall2017accountability}. For the latter to take place, the individual must engage in an account-giving process \citep{frink2008meso}, which may results in rewards or sanctions, and its legitimacy is affirmed by an audience \citep{hall2017accountability}. The role of the audience is to evaluate performance using rules, standards, and expectations and distribute rewards or recommend punishments based on the outcome of the evaluations \citep{hall2017accountability}. Schlenker et al. explained that ``accounts'' shared in the process are fundamental to the process as they either enhance, protect, or damage the individuals' self-image \citep{schlenker1994triangle}. Tetlock argued that individuals often ``defend'' their actions when facing evaluations to protect their self-image and status, highlighting accountability as a fundamental social contingency driving individual behavior and decisions \citep{tetlock1992impact}.

Drawing from several conceptualization frameworks \citep{frink1998toward,schlenker1994triangle,tetlock1992impact}, the integration of accountability in a social systems aims at creating social order \citep{hall2004leader}. Social systems employ accountability mechanisms to cultivate structured environments and social order, where individuals are expected to be held accountable for their participation in various social activities \citep{hall2017accountability}.

When faced with accountability demands, individuals develop coping mechanisms, both proactive and reactive, to maintain a consistent image of themselves \citep{schlenker1994triangle,hall2017accountability}. Finally, given its implications, individuals resort to avoiding, manipulating, or otherwise coping with their accountabilities \citep{tetlock1992impact,hall2017accountability}. For example, Tetlock explained that individuals engage in cognitive laziness, adjusting their account-giving in advance to align with the audience preferences or using their most easily defensible options (acceptability heuristic) to explain their actions \citep{tetlock1992impact}. However, if they discover that they are accountable after the actions have occurred, then, they may engage in retrospective rationality, defending their past behaviors with justifications and excuses \citep{lerner1999accounting}.

In this intricate context, Frink and Klimoski suggest that any conceptualization of workplace accountability should consider both the formal and the informal manifestation of accountability \citep{frink1998toward}. While informal accountability is formalized over time and grounded in organizational rules and policies \citep{frink2008meso}, informal (also referred to as accountability for others) is the perceived accountability ``outside of or beyond formal position or organizational policy'' \citep{zellars2011accountability}. Informal accountability is grounded in unofficial expectations and discretionary behaviors that result from the socialization of network members \citep{romzek2012preliminary}. Shared norms also lay out an informal code of conduct used by group members as a reference for appropriate and inappropriate behaviors \citep{romzek2012preliminary}. Romzek et al. found that informal accountability in nonprofit networks is fostered by trust, reciprocity, and respect for institutional turf \citep{romzek2012preliminary}. Similarly, informal accountability is exercised through evaluations that result in either rewards or sanctions \citep{romzek2014informal}, but remain informal in nature. For example, rewards can be in the form of favors and public recognition, and sanctions may lead to reduced reputation, loss of opportunities within the group, and exclusion from future information sharing \citep{romzek2014informal}.

\begin{figure*}[t!]

    \centering
    \includegraphics[ trim=2cm 3cm 1cm 2cm, clip,
      width = 1.0 \textwidth
    ]{Fig_2.pdf}

    \caption{The Dynamics of Accountability in Social Systems}%
    \label{fig:figure_1}
\Description[]{Map of theory of how accountability works in social systems. Audiences evaluate individuals, using rewards and sanctions.}
\end{figure*}

Figure \ref{fig:figure_1} is a visual representation of the theorization of accountability. In sum, accountability is rooted in social systems and the individuals are at its core. It consists of an evaluation process where individuals are held answerable for their actions and decisions, guided by interpersonal, social, and structural factors within specific sociocultural contexts. Performance against predefined rules, standards, official and unofficial expectations, and shared norms is assessed by an audience resulting in the dispensation of rewards or sanctions. In response, individuals develop various coping mechanisms, both proactive and reactive, to safeguard and maintain a consistent self-image within their social system. Some of these mechanisms may include cognitive strategies as aligning actions with audience preferences or retrospective rationalization, where individuals defend past behaviors with justifications and excuses. While traditional accountability theory has often treated the ``audience'' as a singular entity, Gelfand et al. argue that it can be deconstructed into a web of multiple parties to whom the individual is answerable \citep{gelfand2004culture}.

\subsection{Operationalization of accountability}

In software engineering, formal accountability is operationalized and controlled through institutionalized mechanisms that are explicitly designed to enforce it, such as performance evaluations, established coding standards, and formal performance reviews. These mechanisms aim to align individual behavior with organizational goals by providing clear expectations and consequences, whether through rewards such as promotions or punishments like performance improvement plans. For instance, performance metrics might include defect rates in code or adherence to project deadlines. These mechanisms aim to ensure that accountability is tied to the outcomes set by the organization \citep{alami2024understanding}. In our previous work, we found that software engineers are being individually held accountable for code quality, software security, and meeting project deadlines \citep{alami2024understanding}.

In contrast, informal accountability operates through the shared norms and expectations within teams. Rooted in interpersonal relationships and mutual trust, it is reinforced through peer feedback and team interactions rather than formal rules. As highlighted in our previous research \citep{alami2024understanding}, informal accountability is cultivated through intrinsic motivations and social drivers such as meeting peers expectations, reciprocity of accountability, and adhering to the team's standards. For example, a software engineer may strive for high code quality to uphold their reputation within the team or to meet peer expectations during code reviews. Unlike formal accountability, which relies on structured and institutionalized evaluations, informal accountability relies on psychologically safe environments where individuals are encouraged to take ownership of their contributions without fear of blame \citep{alami2024understanding}. Both forms of accountability co-exist in software engineering environments, with informal mechanisms often emerging within the team \citep{alami2024understanding}.
