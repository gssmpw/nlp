\section{Research Trustworthiness, Limitations and Trade-offs}\label{sec:trust}

\subsection{Trustworthiness}

\noindent We implemented several techniques to address the requirements of research trustworthiness \citep{miles2014qualitative}. We reported \textit{Saturation} (Phase I), \textit{Member checking} (Phase I), and \emph{Feedback session} (Phase II) in Sect. \ref{sec:methods}.

\textit{Triangulation}: We triangulated data sources, including interviews, focus groups, and participant feedback sessions. This exercise allowed us to ensure that our findings are corroborated across different data sources and contexts.

\textit{Peer debriefing}: Although the analysis was primarily conducted by the first author, the second and third authors reviewed the proposed codes, and the results were continuously discussed and scrutinized by the other two authors in several meetings throughout the analysis process. The participation of two authors in the coding process helped minimize researcher biases \citep{miles2014qualitative}. This approach is grounded in our epistemological stance, constructivism, which posits that knowledge is socially constructed and that collective intellectual engagement can lead to more reliable understandings of the data \citep{fosnot2013constructivism}. 

\textit{Thick description}: We endeavored to provide a detailed explanation of our research process and the decisions we have made throughout (see Sect. \ref{sec:methods}). In addition, we assembled a comprehensive replication package (see Sect. \ref{sec:replication}).

\subsection{Limitations and Trade-offs}\label{sec:limit}

\textit{Homogeneous sample}: Our sample is composed only of software engineers. In line with roles theory \citep{katz1978social,frink2004advancing}, we limited our sample to the software engineer role to mitigate the potential for variations that may arise by the inclusion of multiple roles. Roles theory suggests that individuals' accountability is closely linked to roles \citep{katz1978social,frink2004advancing}. This narrow focus strengthens the internal validity of our study and allows for role-centric conclusions.

\textit{Focus on intrinsic drivers}: By focusing primarily on intrinsic drivers and their influence on accountability, we may have inadvertently undermined other factors. For example, in our previous work, we identified institutional factors, such as financial incentives or denial of promotions, that also promote accountability in SE environments \citep{alami2024understanding}. 

\textit{Limited variation in the focus group design}: Another tradeoff is the limited number of variations in the focus group configurations, and the code snippets we used were not of industrial caliber. The consistency across the four groups, shown in the collected data and findings, suggests that additional configurations might not have significantly altered the results. In addition, we prioritized in-depth discussions, which may have been diluted by overly complicated configurations and complex code.

Another tradeoff for this study design is with more realistic, complex, and context-aware code. However, we felt this would greatly limit the accessibility of the focus groups. A future study, examining the contextual intricacies of a proprietary codebase, would shed insight on the role of context in this setting.

We conducted focus groups synchronously and online. Often code reviews, in particular on GitHub and similar sites, are asynchronous and text-based. Open source projects have different dynamics than the ones we discuss here. Hence, our findings our findings may not fully transferable to asynchronous or open-source code reviews. Furthermore, the online setting may have influenced participants' behavior differently than an in-person setup. 

The implementation of a pre-focus group questionnaire to mitigate the risk of social desirability bias \citep{furnham1986response} and self-censorship \citep{yanos2008false} during the focus group discussions carries the risk of priming participants. To mitigate this risk, we avoided the explicit use of the word ``accountability'' in the questions. In addition, during the discussions, we asked participants to provide concrete examples to anchor their responses in their personal experiences, thereby avoiding generic or socially desired answers.

