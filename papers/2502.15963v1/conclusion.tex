\section{Conclusion}\label{sec:conclusion}

\noindent In this study, motivated by our prior work on the broader concept of accountability in SE \cite{alami2024understanding}, we sought to understand the interplay between intrinsic drivers and software engineers' sense of accountability for code quality (\textbf{RQ1}). Also motivated by the evolving nature of AI in SE \cite{fan2023large}, we investigated the impact of the introduction of LLM-assisted review in the context of code review.

We provide insights into how intrinsic drivers, namely professional integrity, pride, personal standards, and reputation, shape engineers' individual accountability. The study also uncovered a complex accountability process that transitions from \emph{individual} to \emph{collective} throughout the process. Finally, the integration of an LLM into this socially loaded process unveiled a pronounced reluctance among software engineers to compromise the social integrity inherent in traditional code review. We contribute to the ongoing efforts to make SE a socially aware practice. The study also created awareness about the integration of AI in SE. The LLM disruption, we observed, highlights emphasizing the preservation of essential human and social aspects such as accountability, social validation, and intrinsic motivation.

Our findings highlight the importance of aligning AI integration with the social dynamics of SE processes to maintain their collaborative essence. Our work opens avenues for future research to investigate mitigating the impact of AI integration in the social dynamics of SE. Future work should explore design frameworks and practical strategies that bridge the gap between the social dynamics inherent in SE and AI integration. Research efforts should ensure that AI augments SE rather than disrupts its critical human-centric practices like accountability and social validation. Such efforts can guide the development of AI that complements the collaborative fabric of SE, fostering both technological efficiency and social cohesion.

For instance, some future research questions to pursue include: How can AI tool integration be designed to preserve the social integrity of software engineering environments and practices? What mechanisms can effectively integrate LLM-generated feedback into socio-technical SE practices without diminishing the human element? Addressing such questions can guide the development of AI that complements the collaborative and social fabric of SE.

