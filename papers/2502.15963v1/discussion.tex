\section{Discussion and Implications}\label{sec:discussion}

We begin this section with an interpretation of our findings through the lens of self-determination theory (SDT) \citep{deci2000and}, as well as how we contribute to existing accountability theories \cite{frink1998toward,hall2003accountability}. Then, we highlight the implications of our findings on practice and future research directions.

SDT is relevant for the context of our study because it provides a robust reference framework for understanding the intrinsic motivations that underpin software engineers' sense of accountability for code quality. Specifically, SDT's focus on autonomy, competence, and relatedness \citep{deci2000and} resonates with our findings. For instance, the absence of reciprocity in accountability in LLM interactions challenges engineers' sense of relatedness, a core pillar of SDT.

Accountability theories \cite{frink1998toward,hall2003accountability} cover the mechanisms by which individuals feel responsible for their actions within a social and organizational context. This theoretical lens allows us to understand accountability in the context of SE and how the introduction of AI technology alters traditional peer review accountability. Accountability theories provide a reliable foundation to understand the behaviors we observed in this study. This well-established theoretical background is also a good reference to understand to what extent SE aligns or deviates from other contexts.

\paragraph*{Self-determination theory (SDT)} SDT suggests that the pursuit of goals and their attainment is driven by basic psychological needs, competence, relatedness, and autonomy \citep{deci2000and}. These needs are ``innate psychological nutriments that are essential for ongoing psychological growth, integrity, and well-being,'' rather than learned or physiological \citep{deci2000and}.

Competence is a propensity to make an impact on the individual's environment as well as to attain valued outcomes within it \citep{deci2000and}. Intrinsically driven behaviors arise from individuals' desire for competence and the need to be self-determined \citep{deci2000and}. Our findings echo similar principles, demonstrating that software engineers' intrinsic drivers such as professional integrity, personal standards, pride in code quality, and reputation significantly influence their sense of accountability for code quality. This highlights the importance of nurturing the individual sense of competence and autonomy amongst software engineers to foster a stronger accountability for code quality. These findings are also a call to appreciate the psychological underpinnings of accountability for an important outcome like code quality in software engineering. It challenges the assumption that code quality maybe achievable solely by promoting standards, tools, and processes.

Relatedness is the feeling of connection to others, ``to love and care, and to be loved and cared for'' \citep{baum2017optimal,ryan1996supportive}. SDT posits that intrinsic motives flourish in an environment where individuals have a sense of secure relatedness and support \citep{deci2000and}. This explains some of our findings. Software engineers expect their peers to be supportive, empowering them to show greater intrinsic motivation for the accountability of their code quality.

Autonomy ``refers to volition, the organismic desire to self-organize experience and behavior and to have activity be concordant with one's integrated sense of self'' \citep{deci2000and}. SDT suggests that fostering a sense of autonomy enhances intrinsic motivation and leads to better performance and well-being \citep{reeve1998autonomy,deci2000and}. By cultivating an environment that supports competence, relatedness, and autonomy, SE teams and organizations may enhance the intrinsic drivers of software engineers, leading to a heightened sense of accountability for code quality. 

These findings may indicate that the current focus on external controls (e.g., quality assurance practices) and incentives (e.g., promotion), as exemplified by industry-oriented metrics approaches such as DORA~\cite{forsgren2018} need to be combined with socially-embedded antecedents to drive code quality. For example, Alami and colleagues reported that psychological safety, or team level perception of non-judgmental interactions, openness and emotional security, also enhances team's ability to pursue quality expectations \citep{alami2024role}.

Although SDT claims that intrinsic motivations are goal-directed behavior either in an autonomous or controlled environment \citep{deci2000and}, our work shows that is not always the case. Intrinsic drivers can also manifest for an outcome that is not necessarily tangible yet, such as feeling and showing accountability for code quality. In our work, the ultimate outcome would be better code quality; however, we have only demonstrated evidence for behaviors showing accountability for this outcome. Hence, we contribute to SDT by highlighting the nuanced ways intrinsic motivation can drive behaviors focused on accountability, even when the end goal is not immediately realized.

\paragraph*{Accountability theory} The conceptualization of accountability captures both the formal and the informal manifestation of accountability \cite{frink1998toward}. While informal accountability is formalized by institutionalized rules and policies \cite{frink2008meso}, informal (grassroots accountability or accountability to peers \cite{alami2024understanding}) uses rules and norms outside the formal organizational context \cite{zellars2011accountability}. Informal accountability is grounded in unofficial expectations and discretionary behaviors that result from the socialization of network members \cite{romzek2012preliminary}. Shared norms also lay out an informal code of conduct used by group members as a reference for appropriate and inappropriate behaviors \cite{romzek2012preliminary}. Romzek et al. found that informal accountability in nonprofit networks is fostered by trust, reciprocity, and respect for institutional turf \cite{romzek2012preliminary}. Similarly, informal accountability is exercised through evaluations that result in either rewards or sanctions \cite{romzek2014informal}, but remain informal in nature. For example, rewards can be in the form of favors and public recognition, and sanctions may lead to reduced reputation, loss of opportunities within the group, and exclusion from future information sharing \cite{romzek2014informal}.


Our study contributes to existing accountability theories by showing that felt or individual accountability is temporal in the context of teamwork. In the context of SE, this individual accountability persists while writing code and prior to the review, then shifts to a collective level to become a shared accountability. This temporality is interrupted when an accountability mechanism such as code review takes place. We also contribute to this theoretical landscape by identifying some individual accountability antecedents. Hall et al. state that ``relatively little empirical work is available to inform our perspectives of antecedents to accountability. Many constructs that would seem to be antecedents to felt accountability'' \citep{hall2017accountability}. In the context of SE, our findings suggest that intrinsic drivers like professional integrity, personal standards, pride in code quality, and reputation serve as key antecedents to individual accountability. This understanding can inform the design of more nuanced accountability mechanisms that capitalize on personal motivations, ultimately improving individual and collective levels of accountability for code quality.

Our research also synergizes SDT and accountability theories by elucidating that intrinsic drivers linked to the psychological needs outlined in SDT---competence, relatedness, and autonomy \citep{deci2000and}---play a role in fostering accountability among software engineers. This connection suggests that nurturing these intrinsic drivers may lead to a stronger individual and collective sense of accountability for code quality. Our findings imply that integrating SDT principles into team-level accountability frameworks and mechanisms may promote more effective and psychologically supportive environments conducive to enhancing accountability for code quality.

\subsection{Industry Implications}

\noindent For implications on practice, we discuss four key takeaways from this study and their implications: (1) code quality through accountability, (2) promoting collective code Ownership, (3) Aligning SE education with its social dynamics, and (4) integration strategies for LLMs.

\paragraph*{Code Quality Through Accountability}

\noindent Even though code review has been extensively investigated \citep{davila2021systematic,badampudi2023modern}, the power of social dynamics inherent in the process to foster accountability for code quality has been underestimated. Our study shows that ensuring code quality extends beyond a technical endeavor. The pursuit of code quality is also embodied in a deep sense of accountability in software engineers' work. While engineers' intrinsic drivers influence their accountability, the shift to shared accountability during the review marks a departure from viewing code quality as a personal and technical achievement to a collective endeavor.

Code quality is not merely a set of standards or metrics but a shared value that is cultivated through a sense of individual and collective accountability. Alami and Krancher drew similar conclusions \citep{alami2022scrum}. They found that Scrum's social practices foster ``social antecedents'' (e.g., psychological safety and transparency) conducive to cultivating behavior and commitment to software quality \citep{alami2022scrum}. Amongst these ``social antecedents'', a sense of collective accountability for a team's outcomes, including software quality, is more pronounced amongst developers.

A useful way to think about code review as a social system is developed in \emph{social learning theory}~\citep{bandura1977social}. Social learning theory (SLT) posits that individuals learn new behaviors and norms through observing and imitating others \citep{bandura1977social}. In order for this social learning process to take place, Fogarty and Dirsmith suggest that socialization practices such as mentoring facilitate ``normative'' and ``mimetic'' isomorphism, especially for standards and behaviors established and/or sanctioned by a profession \citep{fogarty2001organizational}. New members imitate their mentor's performance in their roles to fit within the team and the organization and embody the profession's ethos. They also actively adopt new skills and mimic their mentor's behaviors and values to advance their career in the organization \citep{fogarty2001organizational}.

SLT is relevant in understanding the role of personal qualities intrinsic to a software engineer, such as pride in code quality, professional integrity, upholding personal standards, and maintaining a reputation. The traditional mechanisms of social learning rely primarily on observation, imitation, and social interaction. During mentoring, not only are technical skills passed on but also the ethos of the profession, including values such as pride in code quality, professional integrity, and individual accountability for code quality.


\begin{tcolorbox}

\textbf{Mentoring}: To leverage the potential of social learning and promote accountability for code quality, organizations could implement structured mentorship programs. Experienced engineers with strong intrinsic accountability could be paired with new members. In this socialization process, mentors become role models, and reciprocal determinism may take place.

\end{tcolorbox}

Incorporating structured mentorship programs could promote and embed intrinsic drivers of accountability within the organizational culture by emphasizing the professional values and social norms that underpin code quality. These programs would involve pairing experienced engineers with newcomers to foster a culture of accountability for quality. Mentors with strong intrinsic accountability could model behaviors such as professional integrity, pride in work, and professional reputation. Through ongoing mentoring and feedback, mentees may internalize these values, which may align newcomer behaviors and lead to long-term behavioral shifts \citep{ragins2007handbook,johnson2004mentoring}.

\begin{tcolorbox}

\textbf{Mentorship programs to promote accountability}: Structured mentorship programs could include the following components:\\

- Goal-oriented pairing: Pair new engineers with mentors based on shared professional goals, technical expertise, or accountability practices \citep{ragins2007handbook}.\\

- Regular feedback sessions: Create opportunities for mentors to provide constructive feedback and discuss the ethos of professional integrity, pride in code, and maintaining quality standards \citep{ragins2007handbook,london2002feedback}.\\

- Accountability exercises: Include activities such as collaborative code reviews, where mentors model behaviors like ownership of code quality and providing actionable feedback \citep{london2002feedback}.\\

- Career development integration: Tie mentorship outcomes to career progression to reinforce the mentee's commitment to accountability as a professional standard \citep{allen2004career}.\\

- Reciprocal learning opportunities: Encourage mentees to share their own perspectives and practices, fostering mutual growth and innovation \citep{johnson2004mentoring}.\\

\end{tcolorbox}

The transition to collective ownership and accountability for code quality during the review process highlights the need for constructive delivery of the feedback to foster this communal approach, especially given that the process is susceptible to perceptions of unfairness \citep{german2018my} and negative impressions when the code is substandard \citep{bosu2013impact,bosu2016process}. Feedback intervention theory (FIT) suggests that feedback is effective when its focus is the task and not the individual, the intent is to facilitate learning, and it is perceived as relevant by the recipient \citep{kluger1996effects}. Our findings underscore the importance of the recipient's openness to feedback and the reviewer's constructive delivery in order for the improvement to materialize.

The process where individual accountability for code quality shifts to collective is contingent on this constructive focus. Some of our participants deliberately ``read the room'' or ``politically'' dress their feedback with ``kindness'' and an intention to make the author of the code happy. An average computer science graduate may not be  equipped with the skills to handle this delicate process. The lack of, and need for these skills was recently acknowledged in the ACM 2023 curriculum revision, i.e.,  ``[m]ore focus on team participation, communication, and collaboration''~\citep{ACMed23}.


\begin{tcolorbox}

\textbf{Feedback:} Organizations should develop and continuously improve guidelines for giving and receiving feedback that emphasize constructiveness, learning, and improvement. Educators should emphasize these skills in educational settings.

\end{tcolorbox}

\paragraph*{Promoting Collective Code Ownership}

The shift from individual to collective accountability emphasizes the principles of collective ownership of code \citep{greiler2015code,bird2011don}. When fostering a culture of collective ownership of code and its quality, accountability for quality transcends individual levels to become a shared group ethos. Social identity theory posits that when individuals see themselves as part of a collective entity, they share the responsibility for the group's success. \citep{tajfel1979integrative}. The identification with the group enhances motivation to contribute, as team members derive intrinsic satisfaction from the group's achievements \citep{tajfel1979integrative}.

In SE, research shows that individual silos can create barriers to knowledge sharing and collaboration, while collective ownership promotes a shared responsibility for the codebase \citep{bird2011don}, reinforcing the importance of shared accountability in the pursuit of software quality \citep{alami2024understanding}. Modern code review practices also demonstrate that when multiple team members contribute to the review, the process enhances code quality by broadening the accountability landscape and strengthening the sense of shared responsibility for the codebase at the group level \citep{greiler2015code}. Thongtanunam and Tantithamthavorn reported that diverse metrics of ownership, such as commit- and line-based measures, accommodate varied contributions, fostering team cohesion and a sense of shared purpose \citep{thongtanunam2024code}. These findings align with the principles of social identity theory that collective ownership enhances team identification and intrinsic motivation, promoting a group ethos centered on shared success \citep{tajfel1979integrative}. Supported by previous work and social identity theory, our findings indicate that code review should remain a core SE quality practice.

\begin{tcolorbox} 

\textbf{Promoting collective code ownership practices:} Organizations should establish processes that encourage shared responsibility for the codebase. Our findings show that code review serves as a fundamental mechanism for promoting collective ownership principles in software engineering workflows. These findings suggest that maintaining and enhancing code review practices can reinforce team collaboration and intrinsic motivation, driving higher accountability for code quality and a stronger group ethos.

\end{tcolorbox}

Educational programs should align with the nature of SE practices in the industry, as previously mentioned. Our findings demonstrate that SE is a socially loaded practice. Future software engineers should be prepared to thrive in such environments. This aspect of SE was highlighted as far back as 1988 in Curtis et al. \citep{curtis1988field} and re-emphasized repeatedly in many studies, e.g., \citep{demarco2013peopleware,salas2008teams,begel2008pair,ACMed23}.

\begin{tcolorbox} 

\textbf{Aligning SE education with its social dynamics:} To support collective accountability for code quality in SE, educational programs should train software engineers in collaborative practices, constructive feedback techniques, and interpersonal communication. These skills are critical for fostering collective ownership in team-based environments.

\end{tcolorbox}


\paragraph*{Integration Strategies for LLM}

The transition to collective ownership and accountability for code quality during the review process highlights the need for constructive delivery of the feedback to foster this communal approach, especially given that the process is susceptible to perceptions of unfairness \citep{german2018my} and negative impressions when the code is substandard \citep{bosu2013impact,bosu2016process}. Feedback intervention theory (FIT) suggests that feedback is effective when its focus is the task and not the individual, the intent is to facilitate learning, and it is perceived as relevant by the recipient \citep{kluger1996effects}. Our findings underscore the importance of the recipient's openness to feedback and the reviewer's constructive delivery in order for the improvement to materialize.

\begin{tcolorbox}

\textbf{Complementary role, not a replacement:} To retain the social-centric aspects of accountability for code quality and the social validation process built in, LLMs should be integrated as aids to human expertise rather than a replacement.

\noindent \textbf{LLM as first-line reviewer:} One potential integration is to deploy an LLM as a preliminary reviewer to filter out straightforward issues before the actual peer review takes place. This implementation will also permit leveraging LLMs in an educational role.

\end{tcolorbox}

While LLMs may bring technical and educational augmentation to the process, as seen with chatbots~\citep{Wessel2022}, they also impact the social learning process. It is a considerable shift from the traditional peer-led review, with the potential to also disrupt well-established patterns of social learning and accountability. Their inability to simply be human, reciprocate accountability, and participate in social interactions hinders the social learning framework that underpins team dynamics and collective accountability for code quality. The effectiveness of social learning, as posited by SLT \citep{bandura1977social,bandura1986social}, relies on the model role, in which skills, behavior, norms, and values are adopted during a complex normative adjustment. The assumption that the integration of AI into SE is a matter of plug and unplug is naive.

\subsection{Research Implications}

Our study extends accountability theory by demonstrating the temporal and dynamic nature of individual and collective accountability in SE. The shift from individual to collective accountability, mediated by intrinsic drivers such as professional integrity, pride in code quality, and personal standards, demonstrates the significance of informal and nuanced aspects of accountability beyond formal organizational structures. These findings open avenues for further research to explore the tensions and synergies between institutionalized and informal accountability mechanisms. For example, in our previous work, we reported performance reviews as a formal accountability mechanism \citep{alami2024understanding}. For instance, how do formal mechanisms like performance reviews interact with informal accountability practices, such as peer validation or social learning, to shape engineers' sense of accountability? Additionally, which type of accountability mechanism---institutionalized or informal---exerts a more significant influence on fostering accountability for code quality, and under what circumstances?

The temporal aspect of accountability, we found in this study, enriches existing frameworks, providing empirical evidence for how individual perceptions and intrinsic motivations can evolve into shared accountability within a collaborative process like code review. By identifying intrinsic drivers as antecedents to accountability, our findings address the noted gaps in accountability theory, as highlighted by Hall et al. 2017 \citep{hall2017accountability}, and offer a pathway for designing accountability mechanisms that leverage personal motivations in SE and similar socio-technical domains. However, we still do not know how to operationalize these personal qualities into concrete practices to harness their power. For instance, future research should explore this broad question: What are the mechanisms through which intrinsic motivations can be integrated into formal and informal accountability structures to enhance individual and collective accountability in SE and similar socio-technical practices?

The integration of AI tools like LLMs into SE processes challenges traditional accountability mechanisms and social dynamics, necessitating a thoughtful design approach. Our findings emphasize the need to preserve the social integrity of SE practices by ensuring that AI complements rather than replaces human interactions. For instance, deploying AI-led reviews as first-line reviewers can streamline technical assessments while leaving the social and collaborative dimensions of accountability intact for peer review. Additionally, the educational role of LLMs offers potential for skill enhancement, but their inability to reciprocate accountability or engage in social validation underscores the importance of maintaining human oversight. These insights call for future research to explore how AI can be integrated into SE in ways that sustain the collaborative and social fabric foundational to the discipline. For example, future work can explore questions like: How can AI tools be designed and integrated into software engineering to preserve the social integrity of its practices?
