\section{Findings}\label{sec:findings}

Our findings reveal a dynamic and evolving concept: accountability in code review. Figure \ref{fig:conceptmap} encapsulates this central idea, illustrating the transitional process from individual to collective accountability as well as the nuanced behaviors that underpin this shift. Accountability has a dual nature. First, it begins as an intrinsic motivation, driven by personal standards, pride in code quality, professional integrity, and reputation. As code review unfolds, these intrinsic drivers are modulated to facilitate collaboration and the collective pursuit for quality, culminating in a shared sense of responsibility for code quality.

The conceptual map captured in Fig. \ref{fig:conceptmap} highlights how accountability evolves from the individual to the collective level through self-regulation and social validation in peer-led reviews. The intricate relationship between individual and collective dimensions of accountability is complex and underpinned with individual and collective behaviors, which are adapted in a shifting social process.

In the remaining of the section, we first present the findings of \textbf{RQ1} across two subsections. In subsection \ref{sec:f1}, we present how individual's intrinsic drivers influence their individual-level sense of accountability for code quality (stage numbered 1 in Fig. \ref{fig:conceptmap}) and in subsection \ref{sec:f2}, we explain the transition we observed in our data from individual to collective levels sense of accountability  (stages numbered 2-5 in Fig. \ref{fig:conceptmap}). Then we present and discuss the results of \textbf{RQ2} with respect to LLMs.

\begin{figure*}[!t]
    \includegraphics*[trim=2cm 4.5cm 1cm 1cm, clip, width=1.0\textwidth]{Findings_2}
        \caption{The process of accountability for code quality combining Phase I \& II findings.}
        \label{fig:conceptmap}
        \Description[]{Theory from the research. Sequence of steps: 1, individual accountability, 2, peer code review, 3, self-regulation, 4, social validation, 5, collective accountability.}
\end{figure*} 

\subsection{RQ1: Intrinsic Drivers of Individual Accountability for Code Quality}\label{sec:f1}

We define the intrinsic drivers of individual accountability for code quality as they emerged from our data:

\textbf{Personal standards:} A self-imposed commitment to surpass organizational or team code quality expectations. These expectations are deeply individualistic and often involve striving for perfection or exceptional outcomes, \emph{``... to me, like I kind of set myself standards ... I want to do it, like the highest quality that whatever I can deliver''} (P8).

\textbf{Professional integrity:} A commitment to doing the ``right thing'' by adhering to agreed-upon standards for code quality, even in the absence of external enforcement. This intrinsic driver captures a sense of responsibility to one's team, organization and future developers, \emph{``... my integrity matters, because I want my code to survive beyond my tenure on that project''} (P15).

\textbf{Pride in code quality:} A personal sense of satisfaction and fulfillment derived from producing high-quality work. Pride motivates engineers, in our sample, to deliver high code quality, \emph{``I believe the primary motivation for high quality code comes from within it's personal satisfaction and pride one feel in their work''} (P14).

\textbf{Professional reputation:} The recognition and respect gained from producing reliable and high-quality work, which is also seen as a strategy for career development. Participants frequently referenced their reputation as a motivator for accountability, \emph{``...  if my code quality is good, my image as a developer is good ... let's say ... you have to cut out of by 20\% of developers in your team, okay. I feel that I should be amongst the good ones that my number doesn't come in top bottom 20\% of the developer in the team''} (P11).

Recall \textbf{RQ1} sought to understand how software engineers' intrinsic drivers influence their sense of accountability for code quality. Our data shows a strong indication that code review is an important accountability mechanism for code quality. P11's statement shows that code review is not merely a formality but an accountability check for code quality: \emph{``... code review is a big thing, if I'm writing some code, and if it is getting passed by single people ... if there are any bugs or any mistakes, I am also \textbf{accountable for those mistakes}. And my senior people will also be \textbf{accountable because they have also passed the code to the repo} and they have not noticed the quality problems''} (P11). 

We found that software engineers' sense of accountability towards code quality is partly driven by four intrinsic factors: \emph{personal standards}, \emph{professional integrity}, \emph{pride in code quality}, and \emph{reputation} (stage numbered 1 in Fig. \ref{fig:conceptmap}). These traits collectively promote a sense of individual accountability for code quality. This sense of accountability is predominately activated when engineers write code.

As illustrated in Fig. \ref{fig:conceptmap}, these intrinsic drivers contribute to the foundation of individual accountability in the broader process of accountability for code quality (stages numbered 1-5 in Fig. \ref{fig:conceptmap}). The early stages of writing software code activate these drivers. They partially drive individual accountability to meet code quality standards. As depicted in Fig. \ref{fig:conceptmap}, once individuals interact with peers and feedback mechanisms during code review, accountability transitions into a collective level.

\news{Personal standards} For many software engineers in our sample, accountability for code quality is driven by self-imposed personal standards for quality and going beyond internally agreed or established standards. P9 demonstrates this behavior, expressing a strong personal commitment to high quality: \emph{``I like to produce some high-quality software engineering ... I'm feeling accountable ... I feel like it's my responsibility to do my best in order to produce something with quality''} (P9). P8, on the other hand, strives for \emph{``perfection''}, demonstrating high personal standards. When she does not meet her own standards, a feeling of guilt arises, underscoring the strong emotional connection that she has with the quality of her code. She stated: \emph{``... for me personally, I always want ... the perfect code ... it feels like you're kind of bad feeling is to be told your work is subpar quality''} (P8). These accounts depict a picture of software engineers who seem to be deeply invested and accountable for the quality of their code. Their accountability appears to be driven by their personal standards.

How personal standards drive the feeling of accountability for code quality is well-established in our interviewees' accounts. For example, when P6, P8, and P14 were asked why they feel accountable for the quality of their code, they respectively answered: \emph{``I just feel like a level of professional accountability. I just want to make sure I'm an admin to myself ... I can live with myself knowing that I did my best''} (P6), \emph{``I want to build like the best quality that possible that whatever they're beyond expectation''} (P8), \emph{``Personal standards I think ... do my job well.''} (P14). 

In conclusion, for the engineers in our sample, this intrinsic quality motivates them to hold themselves accountable, not just to external expectations but to their own self-imposed standards.

\news{Professional integrity} Professional integrity, as it appears in our data, is a commitment to doing the right thing by proactively pursuing and adhering to standards for quality set by the team or the organization. For P7, it is about avoiding the wrong thing. He stated: \emph{``I was thinking of integrity as, in a general sense, not doing something the wrong way. If you're doing something the wrong way, you shouldn't do it at all''} (P7). P7's integrity implies not just completing tasks but completing them correctly, i.e., having integrity. 

P14 linked professional integrity directly with accountability for code quality: \emph{``as a developer, several factors contribute to my sense of accountability for code quality. Professional integrity and a commitment to producing high-quality work are among the key elements that drive my accountability, and the impact of my code on the end user is among the key elements that drive my accountability''} (P14).

Professional integrity also appears to be an ethos, a professional responsibility owed to other developers as well as to oneself. P16 explained: \emph{``... senior management just wants results. They don't care ... but I feel myself accountable ... no one will ask me if the code is not clean. But if I'm not be accountable, then the project will be a mess ... then new developers will not be able to understand anything. And it will be a big problem for the organization. So that's why I am taking care of the accountability''} (P16). Together, these accounts indicate that professional integrity seems to drive software engineers' individual accountability for code quality.

\news{Pride in code quality} While integrity is about doing the right thing, pride is an outcome engineers seek for personal satisfaction achieved from writing quality code. P13 explains: \emph{``so I write code, and I feel like my reward is like internal rewards ... because I produced good code and I'm proud of it ...  if I write good code, then I feel happy, then I should write more good code, because I feel proud and more happy more often''} (P13).

When P7 was asked to explain why he feels accountable for delivering code that ``works properly,'' he explained: \emph{``you have to have some integrity in your work and and pride in your work''} (P7).

These accounts illustrate how our interviewees derive personal satisfaction from their work, which in turn seems to maintain a sense of individual accountability for their code quality.

\news{Professional reputation} Phase I data showed that software engineers perceive their professional reputation as a crucial part of their career, and the quality of their code is tightly linked to their reputation. P14 bluntly explained how her professional reputation is ``closely tied'' to the quality of her code: \emph{``... maintaining a positive professional reputation is closely tied to the quality of the code I produce''} (P14).

P12 eloquently demonstrates this motivation. When she was asked why she ``cares'' about the quality of her code, she replied, \emph{``so I care less about salary and more about the professional image because I think that is what will be useful in the long run and not the salary. Because if we maintain a good image, one day or another, my salary will get up''} (P12). This account shows a strategic approach to career, where the reputation of writing quality code becomes an asset to use in future opportunities. 

It seems that software engineers feel accountable for the quality of their code because they carefully cultivate and protect their reputation through consistent and high-quality code. By writing high-quality code, engineers reputation becomes a reflection of their dedication, skill, and reliability. P15 sums up, when prompted to explain why she feels accountable for meeting high coding standards: \emph{``I don't want to have a bad reputation in the future''} (P15).

This drive to maintain one's professional reputation is tightly linked to a sense of individual accountability for code quality in our data. For example, when P6 was asked why he feels accountable for the quality of his work, he replied: \emph{``if anything goes wrong, it falls on me ... it makes me look bad ... if there's, you know, major bugs found''} (P6). When prompted to explain the reward he received from being accountable, he stated: \emph{``I think just the recognition''} (P6). This account shows the intrinsic link between professional reputation and the sense of individual accountability for code quality amongst the interviewees in our sample. In the case of P6, it seems that his concern to compromise his reputation drives him to adhere to quality standards and take responsibility for his work. The recognition he receives may reinforce his professional reputation. In sum, the pursuit to maintain a strong professional reputation compels engineers, in our sample, to feel accountable for the quality of their code.

\subsection{RQ1: A Transition From Individual to Collective Accountability}\label{sec:f2}

Figure \ref{fig:conceptmap} shows how we conceptualize our key findings. We found a transition in \emph{who} is accountable during code review and the dynamics in the shift from \emph{individual} to \emph{collective} accountability. This transition is a complex interaction of behaviors and collaborative influences. In this section, we unpack these dynamics, grounding each stage of the transition in evidence from our focus group data.

Our analysis shows that ``collective'' is not merely the aggregation of individual contributions in the focus groups but an emergent property of a social process of code review and shared commitments that elevate the sense of accountability from personal achievements to team-based outcomes. In our findings, ``collective'' represents the coalescence of individual behaviors, norms, and responsibilities into a cohesive group dynamic, which we observed in the focus groups.

Individual accountability influenced partly by intrinsic driver (stage 1 in Fig. \ref{fig:conceptmap} numbered 1 at the top left of the figure). As code review begins, pride is downregulated (stages 2 \& 3 in Fig. \ref{fig:conceptmap}, numbered 2 and 3) to allow for seamless integration of feedback and the transition to shared accountability for code quality. P27 explains, \emph{``so when delivery [of feedback] was very good, it's less about pride but more about holding each other accountable for our quality''} (FG2, P27).

Professional integrity is also modulated (stages 2 \& 3 in Fig. \ref{fig:conceptmap}) to accommodate the collaborative efforts toward code quality. It shifts from the individual motivation of doing the right thing to the behavior of avoiding defensiveness and constructively engaging with feedback. P22 explains that receiving feedback during the focus group was not a threat to his professional integrity, \emph{``... when I want to write good code, I try to, and then it turns out that that's actually not so good. And then I get corrected, and I write better code, and I'm better next time. So I don't see it as like challenging my professional integrity ... I see it as an opportunity''} (FG1, P22). This response shows how professional integrity, an individual trait, is regulated during the review process. By framing feedback as an opportunity to improve, P22 demonstrates an alignment of his individual motivation with collective goals, fostering collaborative accountability for code quality.

As the review continues, engineers' personal standards and reputation persist in the review process in \textbf{pursuit of social validation} (stage 4 in Fig. \ref{fig:conceptmap}). Software engineers display their personal standards to their peers, seeking recognition to maintain a reputation within the team. P17 explains, \emph{``... we also like to show and demonstrate our knowledge to other people, that they missed something and we found it... So this also like, I think, improves the impression of other teammates of me as well''} (FG1, P17). P27 explains that code quality is an investment in building his reputation, \emph{`` ... I want to be a better coder or be as good of a coder as I can be. And his [reviewer] feedback helps me with that, you know, which in turn leads to more accountability for my part, you know. A better reputation ... and my code quality got better and being known as a good coder''} (FG2, P27).

This harmonization of intrinsic drivers is to achieve alignment with the team's consensus and expectations, as \textbf{accountability is no longer solely individual but collective} (stage 5 in Fig. \ref{fig:conceptmap}). P20 explains, \emph{``so, I personally see code review as a part of the software quality. So basically, when a code ... is reviewed ... then both the author of that PR and reviewers are like accountable''} (FG1, P20). P25 rationalizes the reciprocity of accountability, when the review takes place, \emph{``I'm doing a review, what this means is ... we need to hold each other accountable and ourselves accountable ... So if I review something and I approve it ... I'm also on the line at that point, not just the author''} (FG2, P25).

In the example of P25's comment, the recognition of peers as key stakeholders in the review process further supports this transition from an individual sense of accountability to a shared one.

The culmination of these dynamics, as depicted in stage 5 of Fig. \ref{fig:conceptmap}, is a shift to collective accountability. This stage is characterized by shared responsibility, where both authors and reviewers internalize accountability for the quality of the code.

Collective accountability, as observed in our data, primarily involves those participating in the code review process, but it also extends in practice to the broader development team collaborating to ensure code quality. We identified this transition in our data further in the linguistic shift among participants, which demonstrates a transition from individual to collective accountability during the review process. For example, participants moved from referencing their accountability individually in the interviews, e.g., \emph{``you know, just accountability towards myself''} (P6) and \emph{``I just care about my work''} (P10), to using inclusive language such as ``we'' and ``each other,'' signifying shared responsibility. One participant explicitly noted, \emph{``we need to hold each other accountable and ourselves accountable''} (FG2, P25). Similarly, another participant emphasized, \emph{``When a code... is reviewed... then both the author of that PR and reviewers are like accountable} (FG1, P20). These shifts in language underscore the evolution of accountability into a collective endeavor fostered in peer reviews.

The sense of accountability among authors is often tied to their individual contributions, emphasizing ownership over their work. For instance, authors expressed pride and satisfaction when producing error-free and high-quality code. Their accountability was heightened by the desire to meet or exceed expectations set by themselves or by their teams. On the other hand, reviewers perceive accountability through the lens of collective responsibility. Reviewers focus on ensuring that the code aligns with team standards and best practices, emphasizing collective ownership of code. Reviewers often exhibit accountability by reciprocating the responsibility shown in the effort of the authors, as demonstrated in the code quality. If not, by providing constructive feedback to improve the code quality. For example, in the quotes shared above, P11 (author) emphasized their accountability as an author, stating, \emph{``... if there are any bugs or mistakes, I am accountable because I wrote the code.} Conversely, P25 (reviewer) highlighted his perspective: \emph{``When I approve a pull request, I am also accountable. If there is an issue later, it reflects on me as well.}

In sum, the focus group data analysis revealed a transition from individual to collective accountability during the code review process. While authoring code, individual accountability for its quality is partly driven by intrinsic factors such as pride, professional integrity, personal standards, and reputation. As code review progresses, these intrinsic drivers are modulated to facilitate feedback integration and collaboration. Subsequently, the focus shifts from individual to shared accountability. During the review, software engineers appear to seek social validation from their peers. This behavior is exhibited through showcasing personal standards in coding, which also serves an investment in maintaining professional reputation. It appears that the process culminates in a collective sense of accountability, where both code authors and reviewers share responsibility for the code quality.

\subsection{RQ2: LLM Disruption to Collective Accountability}

\textbf{RQ2} sought to understand how the introduction of an LLM-assisted review may impact the traditional fabric of peer-led review, especially accountability for code quality. We found that the introduction of an LLM (specifically, a large language model based assistant such as ChatGPT) into the social system of code review causes a disruption to the inherent social dynamics of the process and to the transition of accountability from individual to collective. This disruption is caused by four factors: Absence of \textbf{reciprocity of accountability}, \textbf{human interactions}, \textbf{social validation}, and \textbf{lack of trust in LLM technology}.

The transition to collective accountability (stage 5 in Fig. \ref{fig:conceptmap}) is disrupted by the LLM, because of the \textbf{absence of reciprocity}. Software engineers in our sample described the LLM as a ``machine'' that cannot be held accountable, e.g., \emph{``I would not take its [ChatGPT] suggestions as seriously as a coworker, simply because you cannot hold the model accountable''} (FG2, P25). P26 echoes this view: \emph{``now if LLM is reviewing my code, then I'll be the only person responsible ... you cannot blame or hold the LLM accountable''} (FG2, P26).

This disruption also occurs because software engineers' sense of accountability for their code quality is deeply rooted in the \textbf{human interactions} inherent to code review. For example, P35 perceives human interactions as a fundamental driver of his accountability, and code review is not merely a procedural task, \emph{``the human interaction would actually make me feel more accountable''} (FG4, P35).

LLM-assisted code review also \textbf{challenges the social validation} in the process. Code review is more than a technical or procedural event; its role extends to a process of social validation, where engineers seek peer recognition and personal satisfaction. P36 asserts, \emph{``... there will be less accountability. Because there is no reputation in the game. There is no personal pride in the game. There is absolutely not that... with LLM, there is less accountability''} (FG4, P36). P31 explains that he will miss the personal satisfaction (i.e., ``joy'') he experiences when he receives positive feedback from his peers: \emph{``I wouldn't get the same joy. Because it's just a machine''} (FG3, P31). LLM-assisted code review may challenge the deeply ingrained social validation integral to the traditional peer review process. 

In contrast to an LLM (just a ``machine''), engineers also seek and value the human touch associated with the feedback, \emph{``my concern before reading the review was that it's not a human being ... I probably would still prefer a human interaction''} (FG3, P30). Similar sentiment shared by P31, \emph{``I would just still appreciate the more human at this point ... I could just provide more political review to make the reviewer happy and be kind to them.''} (FG3, P31). This preference shows the inherent human need for social connection and validation, which enhances engagement and trust within collaborative environments. The desire for human interaction reflects a deeper value placed on peers feedback exchanges, which fosters a more human and meaningful review process to software engineers.

Finally, our participants appear \textbf{not to trust in LLM technology} (as of early 2024!). Note that all of our informants reported using LLMs on a daily basis, so unfamiliarity is not the explanation for this. They cited limitations discouraging them from considering it equal to their peers. P27 bluntly asserts: \emph{``It's [ChatGPT] not a real thing ... it has no idea in terms of context what it's predicting ... And I'm not accountable to it''} (FG2, P27). P30 corroborates: \emph{``ChatGPT is not the word of God... it's not something that you should trust''} (FG3, P30). This is mostly due to a perceived lack of shared understanding of the code's context or intent. P20 said: \emph{``some of those [comments] are not like, applicable since it [ChatGPT] does not see the rest of the code, and it doesn't know the full codebase''} (FG1, P20).

The trust issue was persistent in the focus group discussions, despite the overwhelming acknowledgment among the participants of the high quality of the LLM's reviews. During the fourth group discussion, one participant appreciated the LLM highlighting security errors in his code and the overall quality of the LLM's review, \emph{``there were some security feedback, considerations ... in case maybe I didn't think of that it might be something I would think of it now ... I think it [LLM-generated feedback] was generally good''} (FG4, P37). P17 eachoed similar assessment of the LLM's feedback, \emph{``It [LLM feedback] was quite great. I was impressed how, from the code, it connected information and summarize, but the function that's also it gave really good feedback and suggestions on how to improve the code''} (FG1, P17).

The disruption we observed not only impacts the accountability process, but also influences the level of engagement with the LLM-generated feedback in comparison to that of peers. For instance, one participant account indicates selective consideration of the code improvement suggested by the LLM compared to their peers. He stated: \emph{``I think my behavior would be a bit different [for LLM's review] ... I might not take it. Like I might not consider it incorporating ... unless it is a very big issue''} (FG1, P21).

However, our participants praised the educational value of an LLM, and its ability to be leveraged for filtering obvious errors. For example, P31 describes ChatGPT's review of his code as \emph{``educative''} (FG3, P31). Our analysis shows that there is a willingness to use LLM as the first reviewer to filter obvious errors; P18 suggests: \emph{``I would not use something like ChatGPT as the only mechanism for code reviews; I would rather use it as a first level of review before I submit my PR for my peers to review''} (FG1, P18).


