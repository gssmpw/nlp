\section{Conclusion and Future Work}
\label{sec:conclusion}
With advancements in Generative AI, EEG—once primarily used for classification tasks—is now being harnessed for generation, which marks a significant step toward brain-computer interaction (BCI) applications. Given its portability and non-invasive nature, EEG has strong potential for real-time, widespread applications, particularly in assistive communication by enabling direct thought-to-speech or thought-to-text systems that enhance accessibility and human-computer interaction. However, comparing studies in this field remains challenging due to the lack of standardized benchmarks. Even when studies utilized the same datasets, the subject-dependent nature of EEG data allowed for multiple ways of splitting and processing, either by subject or object category. For a fair and meaningful comparison across the surveyed studies, it is crucial to establish standardized benchmarks that define consistent data partitioning, evaluation metrics, and model validation protocols. This would ensure reproducibility, facilitate progress in the field, and enable a more accurate assessment of various approaches in EEG-based generative AI research. Nevertheless, we remain optimistic about further advancements in EEG processing and its potential for generating different modalities. As research progresses, improved methodologies, larger datasets, and standardized benchmarks will enhance the reliability and effectiveness of EEG-based generative solutions and bring us closer to real-time, practical implementations of EEG-driven generative AI.
