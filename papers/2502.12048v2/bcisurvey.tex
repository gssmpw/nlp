 % This must be in the first 5 lines to tell arXiv to use pdfLaTeX, which is strongly recommended.
\pdfoutput=1
% In particular, the hyperref package requires pdfLaTeX in order to break URLs across lines.

\documentclass[11pt]{article}

% Remove the "review" option to generate the final version.
\usepackage[dvipsnames]{xcolor}
\usepackage{tcolorbox}
\usepackage{authblk}
\usepackage{BCISURVEY}
\usepackage{hyperref}
% Standard package includes
\usepackage{times}
\usepackage{latexsym}
\usepackage{booktabs}
\usepackage{placeins}
% For proper rendering and hyphenation of words containing Latin characters (including in bib files)
\usepackage[T1]{fontenc}
% For Vietnamese characters
% \usepackage[T5]{fontenc}
% See https://www.latex-project.org/help/documentation/encguide.pdf for other character sets

% This assumes your files are encoded as UTF8
\usepackage[utf8]{inputenc}

% This is not strictly necessary and may be commented out.
% However, it will improve the layout of the manuscript,
% and will typically save some space.
\usepackage{microtype}

% This is also not strictly necessary and may be commented out.
% However, it will improve the aesthetics of text in
% the typewriter font.
\usepackage{inconsolata}
\usepackage{array} % required for text wrapping in tables


% If the title and author information does not fit in the area allocated, uncomment the following
%
%\setlength\titlebox{4.2cm}
%
% and set <dim> to something 5cm or larger.

\usepackage{graphicx}
\usepackage{todonotes}
\usepackage{amsmath}
\usepackage{multirow}
\usepackage{multicol}
\usepackage{caption}
\usepackage{subcaption}
\usepackage{float}

\usepackage[labelfont=bf]{caption}
\captionsetup{labelfont=bf}

\usepackage{adjustbox}

\usepackage{natbib}

\usepackage{tikz}
\usetikzlibrary{positioning, shapes, arrows}
\usepackage{colortbl} % Required for setting table line colors
\usepackage{xcolor} % For color definitions
\definecolor{lightgray}{gray}{0.7}

\title{A Survey on Bridging EEG Signals and Generative AI: From Image and
Text to Beyond}


% For several authors from the same institution:
\author{ \textbf{\ Shreya Shukla,\ Jose Torres, \ Abhijit Mishra,\ Jacek Gwizdka,\ Shounak Roychowdhury} \\
        School of Information, University of Texas at Austin\\
        \{shreya.shukla, jtorres1221, abhijitmishra, jacekg, shounak.roychowdhury\}@utexas.edu}
% if the names do not fit well on one line use
%         Author 1 \\ {\bf Author 2} \\ ... \\ {\bf Author n} \\



\begin{document}
\maketitle

%\def\thefootnote{*}\footnotetext{These authors contributed equally to this work}\def\thefootnote{\arabic{footnote}}

\begin{abstract}

Integration of Brain-Computer Interfaces (BCIs) and Generative Artificial Intelligence (GenAI) has opened new frontiers in brain signal decoding, enabling assistive communication, neural representation learning, and multimodal integration. BCIs, particularly those leveraging Electroencephalography (EEG), provide a non-invasive means of translating neural activity into meaningful outputs. Recent advances in deep learning, including Generative Adversarial Networks (GANs) and Transformer-based Large Language Models (LLMs), have significantly improved EEG-based generation of images, text, and speech. This paper provides a literature review of the state-of-the-art in EEG-based multimodal generation, focusing on (i) EEG-to-image generation through GANs, Variational Autoencoders (VAEs), and Diffusion Models, and (ii) EEG-to-text generation leveraging Transformer based language models and contrastive learning methods.  Additionally, we discuss the emerging domain of \emph{EEG-to-speech synthesis}, an evolving multimodal frontier. We highlight key datasets, use cases, challenges, and EEG feature encoding methods that underpin generative approaches. By providing a structured overview of EEG-based generative AI, this survey aims to equip researchers and practitioners with insights to advance neural decoding, enhance assistive technologies, and expand the frontiers of brain-computer interaction.

\end{abstract}

\section{Introduction}\label{sec:intro}

In computational finance, Monte Carlo simulations are used extensively to estimate the expected value of financial payoffs based on the solution of stochastic differential equations (SDEs) which model the evolution of stock prices, interest rates, exchange rates and other quantities \cite{glasserman04}.  Monte Carlo methods are very general and flexible, but for high accuracy it requires generating a large number of costly SDE path approximations, which has motivated research into a number of variance reduction or, equivalently, cost reduction techniques. One such method is
Multilevel Monte Carlo (MLMC), which was proposed in \cite{GILES2008} and was adapted for various applications that are summarised in \cite{Giles_overview17} and successfully combined with other methods such as quasi-Monte Carlo methods. The main idea of MLMC is to approximate the payoff using different time stepping resolutions when numerically solving the underlying SDE and to generate an optimal number of samples on each level, such that the overall computational cost is minimised subject to the desired bound on the variance. %, such that the total computational cost is minimised. 
The computational savings come from the fact that most samples are computed on the coarser levels and hence are less expensive while only a few samples from the finest levels are required \cite{GILES2008}.


Among the directions in which the computational cost 
of MLMC methods could further be reduced, an important avenue is the use of lower precision calculations, especially for the first Monte Carlo levels where the targeted accuracy is relatively low. 
 An overview of the research on mixed precision for the standard Monte Carlo (MC) framework is provided in \cite{ChowMixedPrecisionStandardMC} but only a few references study the potential of low precision computation in the MLMC framework \cite{Rounding_error_oliver}. To the best of our knowledge, the only MLMC framework with customised precision in the literature is \cite{brugger2014mixed}, but they use a uniform precision for all operations on each Monte Carlo level instead of optimising 
 the precision of each intermediary variable to reduce as much as possible the cost of path generation.
 
An important motivation for an MLMC framework with variable precision would be performing the low precision computations on reconfigurable hardware devices such as Field Programmable Gate Arrays (FPGAs). FPGAs contain customizable logic blocks and connectors that make it easy to adapt the digital circuit architecture for a specific application, leading to a highly parallel and optimised implementation. Therefore they are successfully exploited in applications that require high speed and have high computational workload, such as signal processing \cite{woods2008fpga}, and real time applications like high frequency trading \cite{HFT1,HFT2}. That is why a number of previous works in hardware architecture design implemented the MLMC algorithm to price financial options using FPGAs as accelerators, which resulted in improved speed and power efficiency compared to full CPU architectures \cite{Schryver2013AMM}. The paper \cite{lindsey2016domain} also proposed 
a Domain Specific Language to automate the configuration of FPGAs for this specific application. However, only \cite{brugger2014mixed} proposed a heuristic to reduce the precision in calculations.

In addition, all aforementioned works considered that the random number generation (RNG) is performed in single or double precision. Yet in most cases an important portion of the workload in the overall MLMC simulation comes from the RNG and in \cite{brugger2014mixed} this limited the total computational savings.
To reduce the cost of MLMC simulations in particular those based on the Geometric Brownian Motion (GBM), \cite{approximateICDF_Oliver, NestedOliver} have proposed to use approximate random numbers that are generated by applying an approximation of the inverse CDF to uniform random numbers. In \cite{NestedOliver}, the authors proposed a way to integrate these lower precision random variables into a \textit{nested} MLMC framework and completed a numerical analysis to bound the resulting error at each MC level by a product of the time step and the error in the random number approximation. The same authors show in \cite{approximateICDF_Oliver} that using approximate random variables reduces the cost of path generation by a factor 7.


In this paper we propose a nested MLMC framework that combines the use of approximate random normal variables and lower precision calculations to reduce the computational cost of MLMC even further than \cite{brugger2014mixed,NestedOliver}. We illustrate the efficiency of our framework in Matlab, after making several assumptions on the cost of operations and size of the errors that we carefully justify. We focus on the case of GBM and use the approximate RNG methods presented in \cite{approximateICDF_Oliver} as well as a new slightly modified method that combines CDF inversion and the central limit theorem. To choose the precision of the variables in the low precision path generation, we introduce a novel method to optimise the bit-widths. This optimisation is performed before the main path generation loop is executed and is based on a linear model of the payoff error  
due to rounding when computing in low precision. The error model relies on algorithmic differentiation in a similar manner to \cite{unifying-bwoptim,bitwidth-AD,ADAPT}. The bit-width optimisation procedure can be performed off-line, so this stage can be excluded from the on-line time complexity of our framework. The user specified desired accuracy is then enforced by calculating on-line the number of samples that need to be generated.

In terms of hardware design, we suggest implementing the low precision path generation on FPGAs and the full-precision ones on a CPU or GPU. 
The FPGA offers enough flexibility to define a separate bit-width for every variable in the low precision path generation, and can be reconfigured periodically to update the bit-widths when the market parameters have changed considerably. 


The paper is organized as follows : \Cref{sec:MLMC} introduces MLMC and nested MLMC to make clear the estimator that is implemented in our framework. Then in \Cref{sec:RNG} we detail the methods that could be used to obtain approximate random normally distributed numbers very cheaply for the low precision path generation. In \Cref{sec:error_model} and \Cref{sec:costModel} we propose an error model and a cost model (resp.) that we then use to formulate the optimisation problem that is solved to obtain the optimal bit-widths of fixed point variables in \Cref{sec:optimisation}. Finally we summarise our results and future directions in \Cref{sec:conclusion}.



\section{Related Work}
\label{sec:related work}
% In this section, we review the existing literature on point cloud denoising and unsupervised image denoising.
%-------------------------------------------------------------------------
\subsection{Point cloud denoising}

    \subsubsection{Traditional methods}
Traditional approaches to point cloud denoising include statistical methods \cite{computingpointset2003,definingpointset2004,wlop2009HH}, filtering techniques\cite{pointsetsurfaces2001,Robustmoving2005, zaman2017density}, and optimization-based methods \cite{l1sparse2010,clop2014PR,digne2017bilateral,multi-projection2018duan,hu2020featuregraph} . These techniques often rely on handcrafted features and heuristics to distinguish signal from noise. For example, statistical methods may use distribution models to identify and remove outliers. Filtering methods, such as mean or median filters, operate under the assumption that noise is statistically different from the signal. Optimization-based methods formulate denoising as an energy minimization problem, where regularization terms constrain the solution to ensure certain smoothness cirterion or adherence to prior knowledge.

%-------------------------------------------------------------------------
    \subsubsection{Supervised learning based methods}
In recent years, several deep learning-based methods \cite{rakotosaona2020PCN,hermosilla2019TotalDenoising,luo2020DMR,luo_score-based_2021} have been proposed for point cloud denoising. NPD \cite{NPD2019} is the first learning-based point cloud denoising method that directly processes noisy data without requiring noise characteristics or neighboring point definitions. This approach combines local and global information by projecting noisy points onto estimated reference planes, effectively removing noise and enhancing robustness against variations in noise intensity and curvature. PointCleanNet\cite{rakotosaona2020PCN} first removes outlier points and then combines them with residual connectivity to predict the inverse displacement \cite{Guerrero2017PCPNetLL}, and iteratively shifts noisy points to remove noise. Pistilli \etal proposed GPDNet \cite{gpdnet2020}, which is a graph convolutional network to improve denoising robustness at high noise levels. Luo \etal also proposed  DMRDenoise \cite{luo2020DMR}, which filter
points by first downsampling the noisy inputs and reconstructing the local subsurface to perform point upsampling. However, this resampling method is difficult to maintain a good local shape. ScoreDenoise \cite{luo_score-based_2021} is proposed to tackle the aforementioned issues by iteratively updating the point position in implicit gradient fields learned by neural networks. For inference, they follows an iterative procedure with a decaying step size, which stabilizes point movement and prevents over-correction, allowing points to converge gradually toward the underlying geometry. The authors of \cite{de_Silva_Edirimuni_2023_CVPR} proposed IterativePFN - an iterative method that use a novel loss function that utilizes an adaptive ground truth target at each iteration to capture the relationship between intermediate filtering results during training. Zheng \etal proposed a end-to-end network for joint normal filtering-based point cloud denoising \cite{10173632}. They introduce an auxiliary normal filtering task to enhance the network's ability to remove noise while preserving geometric features more accurately.

Supervised methods can achieve impressive results, but are limited by the availability and quality of the training data, as they typically require paired noisy and clean point clouds to train the neural network. The absence of clean data in real-world scenario pose a significant challenge on applicability of these algorithms.

%-------------------------------------------------------------------------
    \subsubsection{Unsupervised learning methods}
Unsupervised learning-based methods for point cloud denoising do not require ground-truth clean data. Instead, these methods leverage the inherent structure or distribution of the point cloud to guide the denoising process. Unsupervised methods show promise in scenarios where clean data is absent or hard to obtain.

Hermosilla \etal first introduced Total Denoising (TotalDn) \cite{hermosilla2019TotalDenoising} as an unsupervised learning approach for point cloud denoising, relying solely on noisy data without requiring clean ground truth. TotalDn approximates the underlying surfaces by regressing points from the distribution of unstructured total noise, utilizing a spatial prior term to refine the estimation of geometry. 

In DMRDenoise \cite{luo2020DMR}, an unsupervised version is proposed which utilizes a loss function that identify local neighborhoods using a probabilistic Gaussian mask on the k-nearest neighbors, which selectively retains points likely to represent the underlying surface. By leveraging an Earth Mover's Distance (EMD) assignment, it achieves a one-to-one correspondence between input and predicted points, aligning them to reduce noise within local neighborhoods.
This approach enhances robustness to noise and adapts well to varied surface geometries. However, the probabilistic masking and EMD calculation add computational complexity, which can slow down inference in dense or noisy point clouds. 

ScoreDenoise \cite{luo_score-based_2021} also introduced an unsupervised version that employs ensemble score function and an adaptive neighborhood-covering loss for model training.  
Score-u is probably the most relevant work to our method. However, the defined score in \cite{luo_score-based_2021} is only an displacement-alike version of the score function and there is no explicit formula relating the estimated score to the final denoising result. Also, the iterative process is computationally expensive, and can suffer from fluctuation, leading to perturbed and unstable solution.

Most recently, Noise4Denoise \cite{noise4Wang2024} method is proposed that use an additional doubly-noisy point cloud to learn noise displacement by comparing the two noise levels. This approach effectively leverages synthetic noise for training, allowing the model to estimate residuals without relying on clean data.
However, in practical applications, noise parameters are often unknown and variable, making it challenging to replicate the exact conditions assumed during training. This reliance on predefined noise characteristics can limit the model's applicability to real-world scenarios where noise distributions may differ significantly from synthetic settings. 
%-------------------------------------------------------------------------
\subsection{Unsupervised image denoising}
Recently unsupervised image denoising has made significant progress. Non-Bayesian methods include PURE \cite{luisier2010image}, SURE \cite{SURE2018} \textit{etc.}, which are based on various unbiased risk estimator under certain noise distribution. Other methods explore self-similarity in natural images \cite{xu2015patch, doi:10.1137/23M1614456} or exploits the statistical properties of noise to achieve denoising effect \cite{gravel2004method}.  

Noise2Noise \cite{2018Noise2NoiseLI} is a pioneering method that does not require clean data, but it requires multiple noisy versions of the same image for training. To address this limitation, methods such as Noise2Void \cite{2018Noise2VoidL}, Noise2Self \cite{2019Noise2SelfBD}, \textit{etc.}, have been developed, which use only a single noisy image. These methods are particularly important for practical applications where obtaining clean images or multiple noisy realizations of the same image is difficult or impossible. Neighbor2Neighbor \cite{huang2021neighbor2neighbor} proposed a two-step method with a a random neighbor sub-sampler that generates training  pairs and a denosing network. Kim \etal proposed Noise2Score\cite{kim_noise2score_2021}, a novel Bayesian framework for self-supervised image denoising without clean data. The core of Noise2Score is the usage of Tweedie's formula, which provides an explicit representation of the denoised image through a score function. Combined with score function estimation, Noise2Score can be applied to image denoising with any exponential family noise. Kim \etal also proposed the Noise Distribution Adaptive Self-Supervised Image Denoising method \cite{kim_noise_2022}, which further extends the application of Noise2Score by combining the Tweedie distribution with score matching. This enables adaptive handling of various noise distributions and dynamically adjusts the denoising process by estimating noise parameters. On the other hand, Xie \etal \cite{scoreXie2024} broadened the denoising scope of Noise2Score by allowing it to handle complex noise models, such as multiplicative and structurally correlated noise, demonstrating broad applicability to diverse noise models.

These development of unsupervised image denoising method motivate us to explore similar ideas in 3D point cloud denoising.




\section{EEG-to-Image Generation}
\label{sec:bci_cv}

This section explores regenerating images from visually evoked brain signals via EEG. It covers use cases, concerns addressed, techniques employed, and EEG feature encoding methods for image generation used by surveyed studies.

\subsection{Use Cases and Addressed Concerns}

Surveyed studies address key challenges like low signal-to-noise ratio of EEG signals \cite{bai2306dreamdiffusion, lan2023seeing, zeng2023dm}, limited information and individual differences in EEG signals \cite{bai2306dreamdiffusion}, lower performance on natural object images compared to digits and characters \cite{mishra2023neurogan} and small dataset sizes \cite{singh2023eeg2image}. Additionally, some efforts explore alternatives to supervised learning \cite{li2020semi, song2023decoding}, since it demands large amount of data. \citet{song2023decoding} addresses concerns regarding convolution layers applied separately along temporal and spatial dimensions, which disrupts the correlation between channels and hinders the spatial properties of brain activity. Overall, these approaches aim to enhance the training, performance, and interpretation of brain data \cite{li2024visual}.

\citet{kavasidis2017brain2image}, \citet{song2023decoding} and \citet{mishra2023neurogan} extract class-specific EEG encodings that contain discriminative information to improve image generation quality, while \cite{nemrodovneural} focus on utilizing spatiotemporal EEG information to determine the neural correlates of facial identity representations and \cite{khaleghi2022visual} map EEG signals to visual saliency maps corresponding to each image. Other Common strategies include projecting neural signals into a shared subspace with image embeddings \cite{shimizu2022improving}, generating class-specific EEG encodings as latent representations \cite{mishra2023neurogan}, and decoding multi-level perceptual information from EEG signals to produce multi-grained outputs \cite{lan2023seeing}.

Additionally, research efforts focus on enhancing the generalizability of feature extraction pipelines across datasets \cite{singh2024learning}, evaluating the performance of different channels \cite{sugimoto2024image}, and incorporating attention modules to highlight the significance of each channel or frequency band \cite{li2024visual}.

\subsection{Techniques Used Across Studies}

Various computer vision generative models are employed to reconstruct    images from EEG signals. These include \textbf{Variational Autoencoders} \cite{kavasidis2017brain2image, wakita2021photorealistic}, \textbf{Generative Adversarial Networks (GANs) }\cite{kavasidis2017brain2image, khaleghi2022visual, mishra2023neurogan, singh2024learning, li2024visual}, and conditional GANs \cite{singh2023eeg2image, ahmadieh2024visual}. \textbf{Diffusion models}, including prior diffusion models that refine EEG embeddings into image priors \cite{shimizu2022improving}, as well as pre-trained diffusion models such as Stable Diffusion \cite{bai2306dreamdiffusion}, are also commonly used. Additionally, diffusion modules based on U-net architecture have been used in \cite{zeng2023dm, lan2023seeing} to further enhance EEG-to-image reconstruction.

\textbf{Contrastive learning} is another popular approach to align multimodal embeddings, employed in studies \cite{singh2023eeg2image, lan2023seeing, song2023decoding, sugimoto2024image} to obtain discriminative features from EEG signals and align the two modalities by constraining their cosine similarity \cite{song2023decoding}. Furthermore, \textbf{attention mechanisms} are integrated into various models \cite{mishra2023neurogan, song2023decoding, li2024visual} to enhance image quality, capture spatial correlations that reflect brain activity inferred from EEG data, and determine the relative importance of individual EEG channels.


\subsection{EEG Feature Encoding Techniques}

In EEG-to-image reconstruction, the process typically begins with an encoder identifying the latent feature space of EEG signals, followed by a decoder that converts these features into an image. Long Short-Term Memory (LSTM)-based architectures are widely used due to their effectiveness in capturing \textbf{temporal dependencies} in EEG signals. \citet{kavasidis2017brain2image} employs an LSTM network to generate a compact and class-discriminative feature vector, which is also used for object recognition. Similarly, \cite{singh2023eeg2image} integrates LSTM with a triplet-loss-based contrastive learning approach to enhance \textbf{feature discrimination}. \citet{singh2024learning} extends this approach by incorporating both CNN and LSTM architectures trained under EEG label supervision with triplet loss, further improving discriminative feature learning. Additionally, \cite{ahmadieh2024visual} uses LSTM to extract EEG features across two dimensions (EEG channels and signal duration) and \textbf{enhances feature generation} through various regression methods, including polynomial regression, neural network regression, and type-1 and type-2 fuzzy regression.

Several studies also leverage convolutional architectures to capture \textbf{spatial dependencies} in EEG. \citet{li2020semi} uses a three-layer feedforward neural network to project EEG signals into semantic features. \citet{wakita2021photorealistic} adopts a 1D convolutional encoder-decoder as part of a multimodal variational autoencoder (VAE) to obtain mean and variance vectors for EEG signal representation. \citet{mishra2023neurogan} uses a convolutional encoder-decoder framework enhanced with an attention module to focus more on channels with important features instead of using all the features with equal weights. Similarly, \citet{sugimoto2024image} implements EEGNet \cite{lawhern2018eegnet}, a compact convolutional network, as an EEG encoder, while \citet{li2024visual} uses Sinc-EEGNet \cite{bria2021sinc}, an architecture incorporating a sinc-based convolution layer, depth-wise convolution, and separable convolution to extract EEG features. It also integrates an attention mechanism to identify the \textbf{most relevant frequency bands and channels} for signal-based classification.

Graph-based techniques have been explored for EEG feature extraction. \citet{khaleghi2022visual} constructs functional graph connectivity-based embeddings from EEG signals, which are then processed using a Geometric Deep Network (GDN) to derive feature vectors. \citet{song2023decoding} integrates temporal-spatial convolution with plug-and-play spatial modules, leveraging self-attention and graph attention mechanisms to extract EEG features more effectively.

To integrate temporal and spatial feature extraction mechanisms to improve EEG-based image reconstruction, \citet{zeng2023dm} develops a framework inspired by EEGChannelNet \cite{palazzo2020decoding} and ResNet-18, combining spatial, temporal, and temporal-spatial blocks with a multi-kernel residual block. \citet{shimizu2022improving} uses a time-series-inspired architecture with a channel-wise transformed encoder and temporal-spatial convolution to extract \textbf{rich latent EEG representations}.

Self-supervised learning and contrastive learning have been applied to enhance EEG feature extraction. \citet{bai2306dreamdiffusion} uses masked signal modeling, where EEG tokens are partially masked, and a 1D convolutional layer transforms all tokens into embeddings. A Masked Autoencoder (MAE) predicts the missing tokens, refining the learned representations. \citet{lan2023seeing} employs contrastive learning to extract pixel-level semantics from EEG signals while generating a \textbf{saliency map of silhouette information} using GANs. It also aligns CLIP embeddings for image captions with an EEG sample-level encoder through a specialized loss function.

\subsection{Evaluation Metrics} 

EEG-to-image generation often begins with object classification to ensure extracted EEG features contain useful class-discriminative information. Metrics like \textit{top-k accuracy} are commonly used \cite{shimizu2022improving, lan2023seeing, song2023decoding}, along with qualitative visual analysis and quantitative evaluations. Key quantitative metrics include \textit{Inception Score (IS)} \cite{salimans2016improved}, used by \cite{kavasidis2017brain2image, li2020semi, bai2306dreamdiffusion, singh2023eeg2image} which measures the quality of images, \textit{Frechet Inception Distance (FID)} for measuring realism \cite{bai2306dreamdiffusion, singh2024learning, ahmadieh2024visual}, and saliency metrics such as \textit{Structural Similarity Index (SSIM)} for assessing perceptual fidelity \cite{khaleghi2022visual, shimizu2022improving, bai2306dreamdiffusion, ahmadieh2024visual, sugimoto2024image}. Other useful metrics are PixCorr (Pixel-wise Correlation) \cite{shimizu2022improving}, \textit{Kernel Inception Distance (KID)} \cite{singh2024learning}, \textit{LPIPS (Learned Perceptual Image Patch Similarity)} \cite{bai2306dreamdiffusion} and \textit{Diversity Score} \cite{mishra2023neurogan}.


\section{EEG-to-Text Generation}

This section discusses how AI learns brain signal representations from EEG data and maps them to linguistic representations, with an overview depicted in Figure \ref{fig:EEG-text}. We survey use cases, techniques, concerns, and EEG feature encoding methods for text generation.

\begin{figure}[t]
     \centering
    {\includegraphics[width=0.45\textwidth]{images/EEG-text.png}}
   \caption{Reconstructing text from EEG, with eye-tracking data used to capture word-level EEG signals}
	\label{fig:EEG-text}
\end{figure}

\subsection{Use Cases and Addressed Concerns}

The studies referenced in this section share a common use case: generating text from EEG signals. Several studies \cite{biswal2019eegtotext,srivastava2020think2type,yang2023thoughts,rathod2024folded} use the closed vocabulary approach, relying on a fixed set of pre-defined words for EEG-based decoding. Among these, \citet{srivastava2020think2type,yang2023thoughts} investigate text generation using morse code representation of EEG signals, where users' active intent is captured, mapped to morse codes, and then translated to text format.

Recent studies \cite{wang2022open,feng2023aligning,duan2023dewave,liu2024eeg2text,wang2024enhancing,amrani2024deep,tao2024see,mishra2024thought2text,ikegawa2024text,chen2025decoding} overcome closed-vocabulary limitations by exploring open-vocabulary text generation to emulate naturalistic conversations. These studies also address the impact of subjectivity in subject-dependent EEG representation \cite{feng2023aligning,amrani2024deep}, learn cross-modal representation \cite{wang2024enhancing,tao2024see}, and capture long-term dependencies in text and also global contextual information from EEG data that transformers might miss \cite{rathod2024folded,chen2025decoding}.

A significant challenge is the reliance on eye-tracking fixation data as a marker for word-level EEG, which studies like \cite{duan2023dewave,liu2024eeg2text} aim to address. To overcome challenges like defining word-level boundaries in EEG signals and other language processing tasks, some studies have proposed language-agnostic solutions \cite{mishra2024thought2text, ikegawa2024text} which capture signals through image modality and leverage advancements in image-text intermodality to generate text from the collected data. Additionally, \citet{yu2025decoding} introduce a VAE-based augmentation technique to address the issue of limited EEG-text datasets.  

\subsection{Techniques Used Across Studies}

A noteworthy aspect of these studies is the utilization of \textbf{Large Language Models (LLMs)}, particularly BART. Several works \cite{wang2022open, liu2024eeg2text, wang2024enhancing, amrani2024deep, tao2024see, chen2025decoding} have used BART for text generation. In a study by \citet{mishra2024thought2text}, LLMs were fine-tuned on EEG embeddings, image and text data in the training stage to generate text from just EEG signals during inference.

\textbf{Contrastive learning} is widely used in studies like \cite{feng2023aligning, tao2024see, wang2024enhancing} to identify positive EEG-text pairs (e.g., EEG data from the same sentence across subjects) and negative pairs (e.g., EEG data from different sentences or subjects), improving the model's ability to align EEG signals with corresponding text representations. Another key technique is masked signal modeling, employed by \citet{liu2024eeg2text}, where a transformer model is pre-trained to reconstruct randomly masked EEG signals from raw data, enabling the model to learn context, relationships, and semantics within sentence-level EEG signals. An integrated approach by \citet{tao2024see} combines contrastive learning with \textbf{masked signal modeling}, where word-level EEG feature sequences are randomly masked and sentence-level sequences deliberately masked, guided by an intra-modality self-reconstruction objective.

In addition to these techniques, bi-directional Gated Recurrent Units (GRUs) are used to dynamically handle the varying lengths of word-level raw EEG signals \cite{amrani2024deep}. Hierarchical GRUs further improve EEG data processing by capturing both long-range dependencies and local contextual information through the organization of hidden layers hierarchically \cite{chen2025decoding}. A unique approach by \cite{rathod2024folded} employs a folded ensemble deep CNN for text suggestion and a folded ensemble Bidirectional LSTM for text generation, effectively addressing class imbalance and significantly enhancing the accuracy of text generation in closed-vocabulary tasks.

\subsection{EEG Feature Encoding Techniques}

For text generation tasks, EEG signals are encoded into features to capture \textbf{temporal patterns and semantic information}. In the study by \citet{biswal2019eegtotext}, which focuses on generating medical reports, EEG signals are encoded using stacked CNNs to capture \textbf{shift-invariant features} and RCNNs to capture \textbf{temporal patterns}. These features are then used to generate key phenotypes, which hierarchical LSTMs utilize to produce detailed explanations. \citet{srivastava2020think2type} employs an ensemble model to extract EEG embeddings, using CNNs to capture spatial variations and LSTMs to model temporal sequences and long-range dependencies. Another study by \citet{yu2025decoding}, proposes two objectives: classification and sequence-to-sequence (seq2seq) text generation, employing residual blocks for feature extraction in both tasks to capture both \textbf{spatial and temporal features} of the EEG signals effectively. 

Other studies explore the extraction of \textbf{spectral and statistical features} alongside temporal or spatial patterns. \citet{yang2023thoughts}, aiming to translate active intention into text using Morse code, employed Short-Term Fourier Transform (STFT) to extract spectral features and concatenated these with statistical features (e.g., min, max etc. for each channel), in addition to using 1D CNN for spatial features and RNN for temporal features. \citet{rathod2024folded}, another closed-vocabulary solution, used features such as Wavelet Transform (WT), Common Spatial Patterns (CSP), and \textbf{statistical features} to generate EEG feature vectors for classification. 

Various studies have used state-of-the-art transformer architecture for encoding EEG features. \cite{wang2022open} uses a multi-layer transformer encoder to obtain \textbf{EEG mapping from word-level EEG sequences}. \citet{feng2023aligning} uses a transformer-based pre-encoder to convert word-level EEG features into the Seq2Seq embedding space. Another study by \citet{tao2024see} also uses an encoder to extract EEG embeddings and store them in a cross-modal codebook alongside word embeddings obtained from a transformer-based BART model. 

Obtaining word-level EEG signals typically requires markers, often from eye-fixation data like in the Zuco dataset, limiting generalizability. Some studies address this by using \textbf{marker-free and sentence-level EEG signals}. \citet{duan2023dewave} extracts both word-level EEG and raw EEG embeddings. For word-level EEG features with markers, a multi-head transformer layer projects embeddings into feature sequences. For raw EEG waves, a multi-layer transformer encoder is trained for self-reconstruction of waveforms and the transformation of raw EEG signals into sequences of embeddings. In a study by \citet{liu2024eeg2text}, a convolutional transformer model is pretrained with sentence-level EEG signals using a masking technique. It uses a multi-view transformer to encode different brain regions with separate convolutional transformers. \citet{wang2024enhancing} uses both word-level and sentence-level EEG features. It employs a masking technique where word-level sequences are randomly masked and sentence-level features are compulsorily masked.

\citet{chen2025decoding} uses a stacked Hierarchical GRU-based decoder along with Masked Residual Attention Mechanism to obtain EEG representations that capture both \textbf{local and global contextual information}. \citet{amrani2024deep} employs a module consisting of bi-directional GRUs to dynamically address varying lengths of word-level raw EEG signals, a subject-specific 1D convolutional layer, and a multi-layer transformer encoder to encode word-level EEG signals.

\subsection{Evaluation Metrics}
In the surveyed studies, generated text is evaluated against reference text using various established metrics. The commonly used text evaluation metrics are as follows: \textit{METEOR} \cite{banerjee2005meteor}, employed by \cite{biswal2019eegtotext, chen2025decoding}; \textit{BLEU} score \cite{papineni2002bleu}, utilized by \cite{biswal2019eegtotext, wang2022open, feng2023aligning}; \textit{ROUGE} score \cite{lin2004rouge}, adopted by \cite{wang2022open, feng2023aligning, duan2023dewave, liu2024eeg2text, wang2024enhancing}; and \textit{BERTScore} \cite{zhang2019bertscore}, used by \cite{amrani2024deep, mishra2024thought2text}. Other metrics include \textit{Word Error Rate (WER)} used by \cite{feng2023aligning}, \textit{Translation Error Rate (TER)}, and \textit{BLEURT} \cite{sellam2020bleurt}, used by \cite{chen2025decoding}.



\section{EEG-to-Sound/Speech Generation}

We review studies focused on EEG-based generation of sound, speech, voice or music and cover use cases, concerns, techniques, and EEG feature encoding methods for generating Sound or Speech from from EEG.

\subsection{Use Cases and Addressed Concerns}   

EEG-based generation has been explored in various fields beyond image reconstruction, particularly in audio and speech-related applications. These include speech synthesis \cite{krishna2021advancing, lee2023speech}, music decoding and reconstruction \cite{ramirez2022eeg2mel, postolache2024naturalistic}, emotive music generation \cite{jiang2024eeg}, voice reconstruction \cite{lee2023towards}, talking-face generation \cite{park2024towards}, and speech recovery \cite{mizuno2024investigation}. While some studies focus on decoding audio signals for listening tasks in speech or music perception \cite{krishna2021advancing, ramirez2022eeg2mel, park2024towards, mizuno2024investigation, postolache2024naturalistic, jiang2024eeg}, others also investigate speaking tasks and imagined speech \cite{krishna2021advancing, lee2023towards, lee2023speech}.

For more naturalistic communication, \citet{lee2023towards} converts EEG signals recorded during imagined speech into the user’s own voice, aiming for personalized speech synthesis. Similarly, \citet{park2024towards} synthesizes speech from EEG along with generating a talking face with lip-sync. Furthermore, these studies tackle issues such as generating fragmented or abstract outputs \cite{park2024towards}, challenges of synthesizing complete speech from EEG \cite{mizuno2024investigation}, being restricted to simpler music with limited timbres \cite{postolache2024naturalistic}, and the absence of a standardized vocabulary for aligning EEG and audio data \cite{jiang2024eeg}.

\subsection{Techniques Used Across Studies}

\textbf{Convolutional Neural Network (CNN)}-based deep learning models have been used in studies \cite{krishna2021advancing, ramirez2022eeg2mel} to generate audio waveforms from EEG input. \citet{krishna2021advancing} explores speech synthesis for both speaking and listening tasks, using a deep learning architecture with temporal convolution layers, 1D layer, and a time-distributed layer to generate audio waveforms directly. Similarly, \citet{ramirez2022eeg2mel} reconstructs music stimuli using sequential CNN regressors.

\citet{lee2023towards} propose NeuroTalk framework for voice reconstruction from imagined speech. The framework uses a generator based on \textbf{GRUs} to capture sequential EEG information, which outputs a mel-spectrogram. Mel-spectrogram is then converted into a waveform using a \textbf{HiFi-GAN vocoder} \cite{kong2020hifi}, and the resulting waveform is transcribed into text using an \textbf{Automatic Speech Recognition (ASR)} system based on HuBERT \cite{hsu2021hubert}, a self-supervised speech representation learning method. \citet{park2024towards} uses NeuroTalk framework to synthesize audible speech and integrates it with a personalized talking face using \textbf{Wave2Lip} \cite{prajwal2020lip} and Apple API-based avatar generator that accurately lip-sync to the synthesized speech.


\textbf{Transformers and Latent Diffusion Models} have been used to reconstruct speech \cite{mizuno2024investigation} and music \cite{postolache2024naturalistic, jiang2024eeg}. \citet{jiang2024eeg} employs a Transformer model for emotive music generation, while \citet{postolache2024naturalistic} decodes naturalistic music from EEG using a ControlNet adapter \cite{zhang2023adding} to guide AudioLDM2 \cite{liu2024audioldm}, a pre-trained diffusion model, improving control over the generated music.


\subsection{EEG Feature Encoding Techniques}

For speech, voice, and music decoding or generation from EEG, EEG signals are either transformed into intermediate representations, such as mel-spectrograms, or decoded into acoustic and articulatory features\cite{krishna2021advancing}, or EEG temporal features \cite{jiang2024eeg} are utilized. Mel-spectrograms are especially useful, as they offer a shared representational state for both neural signals and audio, enabling more efficient translation between the two modalities.

\citet{krishna2021advancing} incorporates an attention model to predict \textbf{articulatory features} and another attention-regression model to convert these predicted features into \textbf{acoustic features}. Similarly, \citet{jiang2024eeg} extracts EEG tokens through a multi-step process which includes DBSCAN clustering algorithm to derive \textbf{EEG temporal features}. These features are eventually transformed \textbf{EEG positional encoding EEG features} using positional encoding, which are used to form EEG tokens.

In studies using \textbf{mel-spectrograms} as intermediate representations, \citet{ramirez2022eeg2mel} employs a sequential CNN-based regressor to directly map EEG input to time-aligned music spectra. \citet{lee2023speech} seeks to adapt spoken EEG to the subspace of imagined EEG using Common Spatial Pattern (CSP) filters trained on imagined EEG, aiming to generate a user's voice from imagined speech. These CSP filters extract temporal oscillation patterns, minimizing distribution differences between spoken and imagined EEG. Similarly, \citet{postolache2024naturalistic} applies this technique while temporally aligning users' voices with brain signals, using triggers to mark onset intervals and clearly distinguish actual utterance intervals in continuous brain signals.

\subsection{Evaluation Metrics}

EEG-to-speech generation is evaluated using quantitative and qualitative metrics, based on its time-series structure, which also enables its representation as mel-spectrograms. \textit{Mel Cepstral Distortion (MCD)} and \textit{Root Mean Square Error (RMSE)} measure similarity between reconstructed and original speech signals \cite{krishna2021advancing, park2024towards}, while \textit{Structural Similarity Index (SSI)} and \textit{Peak Signal-to-Noise Ratio (PSNR)} assess spectrogram quality \cite{ramirez2022eeg2mel}. Linguistic accuracy is evaluated using Word Error Rate (WER), Character Error Rate (CER), and BERTScore \cite{mizuno2024investigation}, and perceptual quality is quantified with \textit{Frechet Audio Distance (FAD)} \cite{postolache2024naturalistic}. Additional metrics include \textit{Hits@k} for search relevance \cite{jiang2024eeg} and \textit{Mean Opinion Score (MOS)} for subjective quality assessment \cite{lee2023towards}.







\section{Conclusion and future directions} \label{sec:conclusion}

In this paper we proposed a nested MLMC framework that offers important computational savings by performing most calculations in low precision and exploiting approximate random normal variables for the low precision path calculations. The low precision calculations could be performed in fixed precision on an FPGA for greater efficiency, and we suggested a procedure to optimise the bit-widths of every variable at each Monte Carlo level. This is an important improvement over previous mixed precision MLMC frameworks which held the lower precision fixed \cite{Rounding_error_oliver} or defined uniform bit-width at every level heuristically \cite{brugger2014mixed}. Our numerical results suggest that for the first levels our procedure reduces the cost at these levels by a factor 5 or 7. Hence the overall savings are significant since most paths are calculated on the first levels. Our approach would be even more efficient for the Milstein scheme because its higher order strong convergence leads to a greater proportion of the computational costs being on the coarsest levels.

The next stage of the research project will be to implement the RNG methods and the nested framework on FPGAs to determine the hardware requirements and confirm the extent of the computational savings. It would also be good to compare the performance benefits to using half-precision floating point arithmetic on GPUs or CPUs for the low-accuracy computations.




\section*{Limitations}
\label{sec:limitations}

While this survey provides a comprehensive overview of EEG-based generative AI applications, certain limitations exist due to the focused scope of this work. Firstly, this survey primarily covers EEG-based Brain-Computer Interfaces (BCIs), deliberately excluding other neuroimaging techniques such as fMRI, Magnetoencephalography (MEG), and Near-Infrared Spectroscopy (NIRS). Although these modalities play a significant role in BCI research and offer complementary advantages in terms of spatial resolution and multimodal integration, their detailed discussion is beyond the scope of this work.

Secondly, due to space constraints, in-depth discussions on the cognitive underpinnings of EEG signals -- such as their biological origins, neural interpretations, and relationships with brain activity—have been omitted. Similarly, technical details regarding EEG hardware, electrode configurations, and device specifications have been largely excluded for brevity. While these aspects are crucial for practical EEG-based applications, our focus remains on the computational and generative modeling aspects of EEG data processing.

Finally, this survey assumes a general background in EEG signal processing, and generative modeling and expects familiarity with these foundational concepts. While we provide essential explanations, a more in-depth introduction to the fundamentals of EEG and BCI technology is outside the scope of this review. 
%Future work could expand upon these missing aspects to provide a more holistic perspective on the intersection of neuroscience and generative AI.
\section*{Ethics Statement}
EEG data is inherently sensitive, as it contains neural activity patterns that can potentially reveal cognitive states and sometimes personal information. While the majority of the works covered in this survey adhere to established ethical guidelines and standards, some studies may require additional ethical justifications. We have not conducted an exhaustive review of the ethical compliance of each cited work but emphasize the importance of ethical transparency in EEG research. We do not endorse studies that raise ethical concerns or lack proper ethical oversight. Any research involving EEG data collection and analysis should rigorously follow ethical protocols, including obtaining informed consent, ensuring data anonymity, and minimizing risks to participants.

Additionally, we acknowledge the use of OpenAI’s ChatGPT-4 system solely for enhancing writing efficiency, generating LaTeX code, and aiding in error debugging. No content related to the survey's research findings, citations, or factual discussions was autogenerated or retrieved using Generative AI-based search mechanisms. Our work remains grounded in peer-reviewed literature and ethical academic standards.


% \section*{Acknowledgements}

% Entries for the entire Anthology, followed by custom entries
\bibliography{bcisurvey}
\bibliographystyle{acl_natbib}
% \bibliographystyle{unsrtnat}

\end{document}
