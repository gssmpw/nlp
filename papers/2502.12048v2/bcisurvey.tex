 % This must be in the first 5 lines to tell arXiv to use pdfLaTeX, which is strongly recommended.
\pdfoutput=1
% In particular, the hyperref package requires pdfLaTeX in order to break URLs across lines.

\documentclass[11pt]{article}

% Remove the "review" option to generate the final version.
\usepackage[dvipsnames]{xcolor}
\usepackage{tcolorbox}
\usepackage{authblk}
\usepackage{BCISURVEY}
\usepackage{hyperref}
% Standard package includes
\usepackage{times}
\usepackage{latexsym}
\usepackage{booktabs}
\usepackage{placeins}
% For proper rendering and hyphenation of words containing Latin characters (including in bib files)
\usepackage[T1]{fontenc}
% For Vietnamese characters
% \usepackage[T5]{fontenc}
% See https://www.latex-project.org/help/documentation/encguide.pdf for other character sets

% This assumes your files are encoded as UTF8
\usepackage[utf8]{inputenc}

% This is not strictly necessary and may be commented out.
% However, it will improve the layout of the manuscript,
% and will typically save some space.
\usepackage{microtype}

% This is also not strictly necessary and may be commented out.
% However, it will improve the aesthetics of text in
% the typewriter font.
\usepackage{inconsolata}
\usepackage{array} % required for text wrapping in tables


% If the title and author information does not fit in the area allocated, uncomment the following
%
%\setlength\titlebox{4.2cm}
%
% and set <dim> to something 5cm or larger.

\usepackage{graphicx}
\usepackage{todonotes}
\usepackage{amsmath}
\usepackage{multirow}
\usepackage{multicol}
\usepackage{caption}
\usepackage{subcaption}
\usepackage{float}

\usepackage[labelfont=bf]{caption}
\captionsetup{labelfont=bf}

\usepackage{adjustbox}

\usepackage{natbib}

\usepackage{tikz}
\usetikzlibrary{positioning, shapes, arrows}
\usepackage{colortbl} % Required for setting table line colors
\usepackage{xcolor} % For color definitions
\definecolor{lightgray}{gray}{0.7}

\title{A Survey on Bridging EEG Signals and Generative AI: From Image and
Text to Beyond}


% For several authors from the same institution:
\author{ \textbf{\ Shreya Shukla,\ Jose Torres, \ Abhijit Mishra,\ Jacek Gwizdka,\ Shounak Roychowdhury} \\
        School of Information, University of Texas at Austin\\
        \{shreya.shukla, jtorres1221, abhijitmishra, jacekg, shounak.roychowdhury\}@utexas.edu}
% if the names do not fit well on one line use
%         Author 1 \\ {\bf Author 2} \\ ... \\ {\bf Author n} \\



\begin{document}
\maketitle

%\def\thefootnote{*}\footnotetext{These authors contributed equally to this work}\def\thefootnote{\arabic{footnote}}

\begin{abstract}

Integration of Brain-Computer Interfaces (BCIs) and Generative Artificial Intelligence (GenAI) has opened new frontiers in brain signal decoding, enabling assistive communication, neural representation learning, and multimodal integration. BCIs, particularly those leveraging Electroencephalography (EEG), provide a non-invasive means of translating neural activity into meaningful outputs. Recent advances in deep learning, including Generative Adversarial Networks (GANs) and Transformer-based Large Language Models (LLMs), have significantly improved EEG-based generation of images, text, and speech. This paper provides a literature review of the state-of-the-art in EEG-based multimodal generation, focusing on (i) EEG-to-image generation through GANs, Variational Autoencoders (VAEs), and Diffusion Models, and (ii) EEG-to-text generation leveraging Transformer based language models and contrastive learning methods.  Additionally, we discuss the emerging domain of \emph{EEG-to-speech synthesis}, an evolving multimodal frontier. We highlight key datasets, use cases, challenges, and EEG feature encoding methods that underpin generative approaches. By providing a structured overview of EEG-based generative AI, this survey aims to equip researchers and practitioners with insights to advance neural decoding, enhance assistive technologies, and expand the frontiers of brain-computer interaction.

\end{abstract}

\section{Introduction}

Video generation has garnered significant attention owing to its transformative potential across a wide range of applications, such media content creation~\citep{polyak2024movie}, advertising~\citep{zhang2024virbo,bacher2021advert}, video games~\citep{yang2024playable,valevski2024diffusion, oasis2024}, and world model simulators~\citep{ha2018world, videoworldsimulators2024, agarwal2025cosmos}. Benefiting from advanced generative algorithms~\citep{goodfellow2014generative, ho2020denoising, liu2023flow, lipman2023flow}, scalable model architectures~\citep{vaswani2017attention, peebles2023scalable}, vast amounts of internet-sourced data~\citep{chen2024panda, nan2024openvid, ju2024miradata}, and ongoing expansion of computing capabilities~\citep{nvidia2022h100, nvidia2023dgxgh200, nvidia2024h200nvl}, remarkable advancements have been achieved in the field of video generation~\citep{ho2022video, ho2022imagen, singer2023makeavideo, blattmann2023align, videoworldsimulators2024, kuaishou2024klingai, yang2024cogvideox, jin2024pyramidal, polyak2024movie, kong2024hunyuanvideo, ji2024prompt}.


In this work, we present \textbf{\ours}, a family of rectified flow~\citep{lipman2023flow, liu2023flow} transformer models designed for joint image and video generation, establishing a pathway toward industry-grade performance. This report centers on four key components: data curation, model architecture design, flow formulation, and training infrastructure optimization—each rigorously refined to meet the demands of high-quality, large-scale video generation.


\begin{figure}[ht]
    \centering
    \begin{subfigure}[b]{0.82\linewidth}
        \centering
        \includegraphics[width=\linewidth]{figures/t2i_1024.pdf}
        \caption{Text-to-Image Samples}\label{fig:main-demo-t2i}
    \end{subfigure}
    \vfill
    \begin{subfigure}[b]{0.82\linewidth}
        \centering
        \includegraphics[width=\linewidth]{figures/t2v_samples.pdf}
        \caption{Text-to-Video Samples}\label{fig:main-demo-t2v}
    \end{subfigure}
\caption{\textbf{Generated samples from \ours.} Key components are highlighted in \textcolor{red}{\textbf{RED}}.}\label{fig:main-demo}
\end{figure}


First, we present a comprehensive data processing pipeline designed to construct large-scale, high-quality image and video-text datasets. The pipeline integrates multiple advanced techniques, including video and image filtering based on aesthetic scores, OCR-driven content analysis, and subjective evaluations, to ensure exceptional visual and contextual quality. Furthermore, we employ multimodal large language models~(MLLMs)~\citep{yuan2025tarsier2} to generate dense and contextually aligned captions, which are subsequently refined using an additional large language model~(LLM)~\citep{yang2024qwen2} to enhance their accuracy, fluency, and descriptive richness. As a result, we have curated a robust training dataset comprising approximately 36M video-text pairs and 160M image-text pairs, which are proven sufficient for training industry-level generative models.

Secondly, we take a pioneering step by applying rectified flow formulation~\citep{lipman2023flow} for joint image and video generation, implemented through the \ours model family, which comprises Transformer architectures with 2B and 8B parameters. At its core, the \ours framework employs a 3D joint image-video variational autoencoder (VAE) to compress image and video inputs into a shared latent space, facilitating unified representation. This shared latent space is coupled with a full-attention~\citep{vaswani2017attention} mechanism, enabling seamless joint training of image and video. This architecture delivers high-quality, coherent outputs across both images and videos, establishing a unified framework for visual generation tasks.


Furthermore, to support the training of \ours at scale, we have developed a robust infrastructure tailored for large-scale model training. Our approach incorporates advanced parallelism strategies~\citep{jacobs2023deepspeed, pytorch_fsdp} to manage memory efficiently during long-context training. Additionally, we employ ByteCheckpoint~\citep{wan2024bytecheckpoint} for high-performance checkpointing and integrate fault-tolerant mechanisms from MegaScale~\citep{jiang2024megascale} to ensure stability and scalability across large GPU clusters. These optimizations enable \ours to handle the computational and data challenges of generative modeling with exceptional efficiency and reliability.


We evaluate \ours on both text-to-image and text-to-video benchmarks to highlight its competitive advantages. For text-to-image generation, \ours-T2I demonstrates strong performance across multiple benchmarks, including T2I-CompBench~\citep{huang2023t2i-compbench}, GenEval~\citep{ghosh2024geneval}, and DPG-Bench~\citep{hu2024ella_dbgbench}, excelling in both visual quality and text-image alignment. In text-to-video benchmarks, \ours-T2V achieves state-of-the-art performance on the UCF-101~\citep{ucf101} zero-shot generation task. Additionally, \ours-T2V attains an impressive score of \textbf{84.85} on VBench~\citep{huang2024vbench}, securing the top position on the leaderboard (as of 2025-01-25) and surpassing several leading commercial text-to-video models. Qualitative results, illustrated in \Cref{fig:main-demo}, further demonstrate the superior quality of the generated media samples. These findings underscore \ours's effectiveness in multi-modal generation and its potential as a high-performing solution for both research and commercial applications.
\section{Related Work}
\label{sec:relatedwork}

\subsection{Current AI Tools for Social Service}
\label{subsec:relatedtools}
% the title I feel is quite broad

Harnessing technology for social good has always been a grand challenge in social service \cite{berzin_practice_2015}. As early as the 90s, artificial neural networks and predictive models have been employed as tools for risk assessments, decision-making, and workload management in sectors like child protective services and mental health treatment \cite{fluke_artificial_1989, patterson_application_1999}. The recent rise of generative AI is poised to further advance social service practice, facilitating the automation of administrative tasks, streamlining of paperwork and documentation, optimisation of resource allocation, data analysis, and enhancing client support and interventions \cite{fernando_integration_2023, perron_generative_2023}.

Today, AI solutions are increasingly being deployed in both policy and practice \cite{goldkind_social_2021, hodgson_problematising_2022}. In clinical social work, AI has been used for risk assessments, crisis management, public health initiatives, and education and training for practitioners \cite{asakura_call_2020, gillingham2019can, jacobi_functions_2023, liedgren_use_2016, molala_social_2023, rice_piloting_2018, tambe_artificial_2018}. AI has also been employed for mental health support and therapeutic interventions, with conversational agents serving as on-demand virtual counsellors to provide clinical care and support \cite{lisetti_i_2013, reamer_artificial_2023}.
% commercial solutions include Woebot, which simulates therapeutic conversation, and Wysa, an “emotionally intelligent” AI coach, powered by evidenced-based clinical techniques \cite{reamer_artificial_2023}. 
% Non-clinical AI agents like Replika and companion robots can also provide social support and reduce loneliness amongst individuals \cite{ahmed_humanrobot_2024, chaturvedi_social_2023, pani_can_2024, ta_user_2020}.

Present research largely focuses on \textit{\textbf{AI-based decision support tools}} in social service \cite{james_algorithmic_2023, kawakami2022improving}, especially predictive risk models (PRMs) used to predict social service risks and outcomes \cite{gillingham2019can, van2017predicting}, like the Allegheny Family Screening Tool (AFST), which assesses child abuse risk using data from US public systems \cite{chouldechova_case_2018, vaithianathan2017developing}. Elsewhere, researchers have also piloted PRMs to predict social service needs for the homeless using Medicaid data\cite{erickson_automatic_2018, pourat_easy_2023}, and AI-powered algorithms to promote health interventions for at-risk populations, such as HIV testing among Californian homeless \cite{rice_piloting_2018, yadav_maximizing_2017}.

\subsection{Generative AI and Human-AI Collaboration}
\label{subsec:relatedworkhaicollaboration}
Beyond decision-making algorithms and PRMs, advancements in generative AI, such as large language models (LLMs), open new possibilities for human-AI (HAI) collaboration in social services. 
LLMs have been called "revolutionary" \cite{fui2023generative} and a "seismic shift" \cite{cooper2023examining}, offering "content support" \cite{memmert2023towards} by generating realistic and coherent responses to user inputs \cite{cascella2023evaluating}. Their vastly improved capabilities and ubiquity \cite{cooper2023examining} makes them poised to revolutionise work patterns \cite{fui2023generative}. Generative AI is already used in fields like design, writing, music, \cite{han2024teams, suh2021ai, verheijden2023collaborative, dhillon2024shaping, gero2023social} healthcare, and clinical settings \cite{zhang2023generative, yu2023leveraging, biswas2024intelligent}, with promising results. However, the social service sector has been slower in adopting AI \cite{diez2023artificial, kawakami2023training}.

% Yet, the social service sector is one that could perhaps stand to gain the most from AI technologies. As Goldkind \cite{goldkind_social_2021} writes, social service, as a "values-centred profession with a robust code of ethics" (p. 372), is uniquely placed to inform the development of thoughtful algorithmic policy and practice. 
Social service, however, stands to benefit immensely from generative AI. SSPs work in time-poor environments \cite{tiah_can_2024}, often overwhelmed with tedious administrative work \cite{meilvang_working_2023} and large amounts of paperwork and data processing \cite{singer_ai_2023, tiah_can_2024}. 
% As such, workers often work in time-poor environments and are burdened with information overload and administrative tasks \cite{tiah_can_2024, meilvang_working_2023}. 
Generative AI is well-placed to streamline and automate tasks like formatting case notes, formulating treatment plans and writing progress reports, which can free up valuable time for more meaningful work like client engagement and enhance service quality \cite{fernando_integration_2023, perron_generative_2023, tiah_can_2024, thesocialworkaimentor_ai_nodate}. 

Given the immense potential, there has been emerging research interest in HAI collaboration and teamwork in the Human-Computer Interaction and Computer Supported Cooperative Work space \cite{wang_human-human_2020}. HAI collaboration and interaction has been postulated by researchers to contribute to new forms of HAI symbiosis and augmented intelligence, where algorithmic and human agents work in tandem with one another to perform tasks better than they could accomplish alone by augmenting each other's strengths and capabilities  \cite{dave_augmented_2023, jarrahi_artificial_2018}.

However, compared to the focus on AI decision-making and PRM tools, there is scant research on generative AI and HAI collaboration in the social service sector \cite{wykman_artificial_2023}. This study therefore seeks to fill this critical gap by exploring how SSPs use and interact with a novel generative AI tool, helping to expand our understanding of the new opportunities that HAI collaboration can bring to the social service sector.

\subsection{Challenges in AI Use in Social Service}
\label{subsec:relatedworkaiuse}

% Despite the immense potential of AI systems to augment social work practice, there are multiple challenges with integrating such systems into real-life practice. 
Despite its evident benefits, multiple challenges plague the integration of AI and its vast potential into real-life social service practice.
% Numerous studies have investigated the use of PRMs to help practitioners decide on a course of action for their clients. 
When employing algorithmic decision-making systems, practitioners often experience tension in weighing AI suggestions against their own judgement \cite{kawakami2022improving, saxena2021framework}, being uncertain of how far they should rely on the machine. 
% Despite often being instructed to use the tool as part of evaluating a client, 
Workers are often reluctant to fully embrace AI assessments due to its inability to adequately account for the full context of a case \cite{kawakami2022improving, gambrill2001need}, and lack of clarity and transparency on AI systems and limitations \cite{kawakami2022improving}. Brown et al. \cite{brown2019toward} conducted workshops using hypothetical algorithmic tools 
% to understand service providers' comfort levels with using such tools in their work,
and found similar issues with mistrust and perceived unreliability. Furthermore, introducing AI tools can  create new problems of its own, causing confusion and distrust amongst workers \cite{kawakami2022improving}. Such factors are critical barriers to the acceptance and effective use of AI in the sector.

\citeauthor{meilvang_working_2023} (2023) cites the concept of \textit{boundary work}, which explores the delineation between "monotonous" administrative labour and "professional", "knowledge based" work drawing on core competencies of SSPs. While computers have long been used for bureaucratic tasks like client registration, the introduction of decision support systems like PRMs stirred debate over AI "threatening professional discretion and, as such, the profession itself" \cite{meilvang_working_2023}. Such latent concerns arguably drive the resistance to technology adoption described above. Generative AI is only set to further push this boundary, 
% these concerns are only set to grow in tandem with the vast capabilities of generative and other modern AI systems. Compared to the relatively primitive AI systems in past years, perceived as statistical algorithms \cite{brown2019toward} turning preset inputs like client age and behavioural symptoms \cite{vaithianathan2017developing} into simple numerical outputs indicating various risk scores, modern AI systems are vastly more capable: LLMs 
with its ability to formulate detailed reports and assessments that encroach upon the "core" work of SSPs.
% accept unrestricted and unstructured inputs and return a range of verbose and detailed evaluations according to the user's instructions. 
Introducing these systems exacerbate previously-raised issues such as understanding the limitations and possibilities of AI systems \cite{kawakami2022improving} and risk of overreliance on AI \cite{van2023chatgpt}, and requires a re-examination of where users fall on the algorithmic aversion-bias scale \cite{brown2019toward} and how they detect and react to algorithmic failings \cite{de2020case}. We address these critical issues through an empirical, on-the-ground study that to our knowledge is the first of its kind since the new wave of generative AI.

% W 

% Yet, to date, we have limited knowledge on the real-world impacts and implications of human-AI collaboration, and few studies have investigated practitioners’ experiences working with and using such AI systems in practice, especially within the social work context \cite{kawakami2022improving}. A small number of studies have explored practitioner perspectives on the use of AI in social work, including Kawakami et al. \cite{kawakami2022improving}, who interviewed social workers on their experiences using the AFST; Stapleton et al. \cite{stapleton_imagining_2022}, who conducted design workshops with caseworkers on the use of PRMs in child welfare; and Wassal et al. \cite{wassal_reimagining_2024}, who interviewed UK social work professionals on the use of AI. A common thread from all these studies was a general disregard for the context and users, with many practitioners criticising the failure of past AI tools arising from the lack of participation and involvement of social workers and actual users of such systems in the design and development of algorithmic systems \cite{wassal_reimagining_2024}. Similarly, in a scoping review done on decision-support algorithms in social work, Jacobi \& Christensen \cite{jacobi_functions_2023} reported that the majority of studies reveal limited bottom-up involvement and interaction between social workers, researchers and developers, and that algorithms were rarely developed with consideration of the perspective of social workers.
% so the \cite{yang_unremarkable_2019} and \cite{holten_moller_shifting_2020} are not real-world impacts? real-world means to hear practitioner's voice? I feel this is quite important but i didnt get this point in intro!

% why mentioning 'which have largely focused on existing ADS tools (e.g., AFST)'? i can see our strength is more localized, but without basic knowledge of social work i didnt get what's the 'departure' here orz
% the paragraph is great! do we need to also add one in line 20 21?

\subsection{Designing AI for Social Service through Participatory Design}
\label{subsec:relatedworkpd}
% i think it's important! but maybe not a whole subsection? but i feel the strong connection with practitioners is indeed one of our novelties and need to highlight it, also in intro maybe
% Participatory design (PD) has long been used extensively in HCI \cite{muller1993participatory}, to both design effective solutions for a specific community and gain a deep understanding of that community. Of particular interest here is the rich body of literature on PD in the field of healthcare \cite{donetto2015experience}, which in this regard shares many similarities and concerns with social work. PD has created effective health improvement apps \cite{ryu2017impact}, 

% PD offers researchers the chance to gather detailed user requirements \cite{ryu2017impact}...

Participatory design (PD) is a staple of HCI research \cite{muller1993participatory}, facilitating the design of effective solutions for a specific community while gaining a deep understanding of its stakeholders. The focus in PD of valuing the opinions and perspectives of users as experts \cite{schuler_participatory_1993} 
% In recent years, the tech and social work sectors have awakened to the importance of involving real users in designing and implementing digital technologies, developing human-centred design processes to iteratively design products or technologies through user feedback 
has gained importance in recent years \cite{storer2023reimagining}. Responding to criticisms and failures of past AI tools that have been implemented without adequate involvement and input from actual users, HCI scholars have adopted PD approaches to design predictive tools to better support human decision-making \cite{lehtiniemi_contextual_2023}.
% ; accordingly, in social service, a line of research has begun studying and designing for human-AI collaboration with real-world users (e.g. \cite{holten_moller_shifting_2020, kawakami2022improving, yang_unremarkable_2019}).
Section \ref{subsec:relatedworkaiuse} shows a clear need to better understand SSP perspectives when designing and implementing AI tools in the social sector. 
Yet, PD research in this area has been limited. \citeauthor{yang2019unremarkable} (2019), through field evaluation with clinicians, investigated reasons behind the failure of previous AI-powered decision support tools, allowing them to design a new-and-improved AI decision-support tool that was better aligned with healthcare workers’ workflows. Similarly, \citeauthor{holten_moller_shifting_2020} (2020) ran PD workshops with caseworkers, data scientists and developers in public service systems to identify the expectations and needs that different stakeholders had in using ADS tools.

% Indeed, it is as Wise \cite{wise_intelligent_1998} noted so many years ago on the rise of intelligent agents: “it is perhaps when technologies are new, when their (and our) movements, habits and attitudes seem most awkward and therefore still at the forefront of our thoughts that they are easiest to analyse” (p. 411). 
Building upon this existing body of work, we thus conduct a study to co-design an AI tool \textit{for} and \textit{with} SSPs through participatory workshops and focus group discussions. In the process, we revisit many of the issues mentioned in Section \ref{subsec:relatedworkaiuse}, but in the context of novel generative AI systems, which are fundamentally different from most historical examples of automation technologies \cite{noy2023experimental}. This valuable empirical inquiry occurs at an opportune time when varied expectations about this nascent technology abound \cite{lehtiniemi_contextual_2023}, allowing us to understand how SSPs incorporate AI into their practice, and what AI can (or cannot) do for them. In doing so, we aim to uncover new theoretical and practical insights on what AI can bring to the social service sector, and formulate design implications for developing AI technologies that SSPs find truly meaningful and useful.
% , and drive future technological innovations to transform the social service sector not just within [our country], but also on a global scale.

 % with an on-the-ground study using a real prototype system that reflects the state of AI in current society. With the presumption that AI will continue to be used in social work given the great benefits it brings, we address the pressing need to investigate these issues to ensure that any potential AI systems are designed and implemented in a responsible and effective manner.

% Building upon these works, this study therefore seeks to adopt a participatory design methodology to investigate social workers’ perspectives and attitudes on AI and human-AI collaboration in their social work practice, thus contributing to the nascent body of practitioner-centred HCI research on the use of AI in social work. Yet, in a departure from prior work, which have largely focused on existing ADS tools (e.g., AFST) and were situated in a Western context, our paper also aims to expand the scope by piloting a novel generative AI tool that was designed and developed by the researchers in partnership with a social service agency based in Singapore, with aims of generating more insights on wider use cases of AI beyond what has been previously studied.

% i may think 'While the current lacunae of research on applications of AI in social work may appear to be a limitation, it simultaneously presents an exciting opportunity for further research and exploration \cite{dey_unleashing_2023},' this point is already convincing enough, not sure if we need to quote here
% I like this end! it's a good transition to our study design, do we need to mention the localization in intro as well? like we target at singapore

% Given the increasing prominence and acceptance of AI in modern society, 

% These increased capabilities vastly exacerbate the issues already present with a simpler tool like the AFST: the boundaries and limitations of an LLM system are significantly more difficult to understand and its possible use cases are exponentially greater in scope. 

% Put this in discussion section instead?
% Kawakami et al's work "highlights the importance of studying how collaborative decision-making... impacts how people rely upon and make sense of AI models," They conclude by recommending designing tools that "support workers in understanding the boundaries of [an AI system's] capabilities", and implementing design procedures that "support open cultures for critical discussion around AI decision making". The authors outline critical challenges of implementing AI systems, elucidating factors that may hinder their effectiveness and even negatively affect operations within the organisation.


% Is this needed?:
% talk about the strengths of PD in eliciting user viewpoints and knowledge, in particular when it is a field that is novel or where a certain system has not been used or developed or tested before
\section{EEG-to-Image Generation}
\label{sec:bci_cv}

This section explores regenerating images from visually evoked brain signals via EEG. It covers use cases, concerns addressed, techniques employed, and EEG feature encoding methods for image generation used by surveyed studies.

\subsection{Use Cases and Addressed Concerns}

Surveyed studies address key challenges like low signal-to-noise ratio of EEG signals \cite{bai2306dreamdiffusion, lan2023seeing, zeng2023dm}, limited information and individual differences in EEG signals \cite{bai2306dreamdiffusion}, lower performance on natural object images compared to digits and characters \cite{mishra2023neurogan} and small dataset sizes \cite{singh2023eeg2image}. Additionally, some efforts explore alternatives to supervised learning \cite{li2020semi, song2023decoding}, since it demands large amount of data. \citet{song2023decoding} addresses concerns regarding convolution layers applied separately along temporal and spatial dimensions, which disrupts the correlation between channels and hinders the spatial properties of brain activity. Overall, these approaches aim to enhance the training, performance, and interpretation of brain data \cite{li2024visual}.

\citet{kavasidis2017brain2image}, \citet{song2023decoding} and \citet{mishra2023neurogan} extract class-specific EEG encodings that contain discriminative information to improve image generation quality, while \cite{nemrodovneural} focus on utilizing spatiotemporal EEG information to determine the neural correlates of facial identity representations and \cite{khaleghi2022visual} map EEG signals to visual saliency maps corresponding to each image. Other Common strategies include projecting neural signals into a shared subspace with image embeddings \cite{shimizu2022improving}, generating class-specific EEG encodings as latent representations \cite{mishra2023neurogan}, and decoding multi-level perceptual information from EEG signals to produce multi-grained outputs \cite{lan2023seeing}.

Additionally, research efforts focus on enhancing the generalizability of feature extraction pipelines across datasets \cite{singh2024learning}, evaluating the performance of different channels \cite{sugimoto2024image}, and incorporating attention modules to highlight the significance of each channel or frequency band \cite{li2024visual}.

\subsection{Techniques Used Across Studies}

Various computer vision generative models are employed to reconstruct    images from EEG signals. These include \textbf{Variational Autoencoders} \cite{kavasidis2017brain2image, wakita2021photorealistic}, \textbf{Generative Adversarial Networks (GANs) }\cite{kavasidis2017brain2image, khaleghi2022visual, mishra2023neurogan, singh2024learning, li2024visual}, and conditional GANs \cite{singh2023eeg2image, ahmadieh2024visual}. \textbf{Diffusion models}, including prior diffusion models that refine EEG embeddings into image priors \cite{shimizu2022improving}, as well as pre-trained diffusion models such as Stable Diffusion \cite{bai2306dreamdiffusion}, are also commonly used. Additionally, diffusion modules based on U-net architecture have been used in \cite{zeng2023dm, lan2023seeing} to further enhance EEG-to-image reconstruction.

\textbf{Contrastive learning} is another popular approach to align multimodal embeddings, employed in studies \cite{singh2023eeg2image, lan2023seeing, song2023decoding, sugimoto2024image} to obtain discriminative features from EEG signals and align the two modalities by constraining their cosine similarity \cite{song2023decoding}. Furthermore, \textbf{attention mechanisms} are integrated into various models \cite{mishra2023neurogan, song2023decoding, li2024visual} to enhance image quality, capture spatial correlations that reflect brain activity inferred from EEG data, and determine the relative importance of individual EEG channels.


\subsection{EEG Feature Encoding Techniques}

In EEG-to-image reconstruction, the process typically begins with an encoder identifying the latent feature space of EEG signals, followed by a decoder that converts these features into an image. Long Short-Term Memory (LSTM)-based architectures are widely used due to their effectiveness in capturing \textbf{temporal dependencies} in EEG signals. \citet{kavasidis2017brain2image} employs an LSTM network to generate a compact and class-discriminative feature vector, which is also used for object recognition. Similarly, \cite{singh2023eeg2image} integrates LSTM with a triplet-loss-based contrastive learning approach to enhance \textbf{feature discrimination}. \citet{singh2024learning} extends this approach by incorporating both CNN and LSTM architectures trained under EEG label supervision with triplet loss, further improving discriminative feature learning. Additionally, \cite{ahmadieh2024visual} uses LSTM to extract EEG features across two dimensions (EEG channels and signal duration) and \textbf{enhances feature generation} through various regression methods, including polynomial regression, neural network regression, and type-1 and type-2 fuzzy regression.

Several studies also leverage convolutional architectures to capture \textbf{spatial dependencies} in EEG. \citet{li2020semi} uses a three-layer feedforward neural network to project EEG signals into semantic features. \citet{wakita2021photorealistic} adopts a 1D convolutional encoder-decoder as part of a multimodal variational autoencoder (VAE) to obtain mean and variance vectors for EEG signal representation. \citet{mishra2023neurogan} uses a convolutional encoder-decoder framework enhanced with an attention module to focus more on channels with important features instead of using all the features with equal weights. Similarly, \citet{sugimoto2024image} implements EEGNet \cite{lawhern2018eegnet}, a compact convolutional network, as an EEG encoder, while \citet{li2024visual} uses Sinc-EEGNet \cite{bria2021sinc}, an architecture incorporating a sinc-based convolution layer, depth-wise convolution, and separable convolution to extract EEG features. It also integrates an attention mechanism to identify the \textbf{most relevant frequency bands and channels} for signal-based classification.

Graph-based techniques have been explored for EEG feature extraction. \citet{khaleghi2022visual} constructs functional graph connectivity-based embeddings from EEG signals, which are then processed using a Geometric Deep Network (GDN) to derive feature vectors. \citet{song2023decoding} integrates temporal-spatial convolution with plug-and-play spatial modules, leveraging self-attention and graph attention mechanisms to extract EEG features more effectively.

To integrate temporal and spatial feature extraction mechanisms to improve EEG-based image reconstruction, \citet{zeng2023dm} develops a framework inspired by EEGChannelNet \cite{palazzo2020decoding} and ResNet-18, combining spatial, temporal, and temporal-spatial blocks with a multi-kernel residual block. \citet{shimizu2022improving} uses a time-series-inspired architecture with a channel-wise transformed encoder and temporal-spatial convolution to extract \textbf{rich latent EEG representations}.

Self-supervised learning and contrastive learning have been applied to enhance EEG feature extraction. \citet{bai2306dreamdiffusion} uses masked signal modeling, where EEG tokens are partially masked, and a 1D convolutional layer transforms all tokens into embeddings. A Masked Autoencoder (MAE) predicts the missing tokens, refining the learned representations. \citet{lan2023seeing} employs contrastive learning to extract pixel-level semantics from EEG signals while generating a \textbf{saliency map of silhouette information} using GANs. It also aligns CLIP embeddings for image captions with an EEG sample-level encoder through a specialized loss function.

\subsection{Evaluation Metrics} 

EEG-to-image generation often begins with object classification to ensure extracted EEG features contain useful class-discriminative information. Metrics like \textit{top-k accuracy} are commonly used \cite{shimizu2022improving, lan2023seeing, song2023decoding}, along with qualitative visual analysis and quantitative evaluations. Key quantitative metrics include \textit{Inception Score (IS)} \cite{salimans2016improved}, used by \cite{kavasidis2017brain2image, li2020semi, bai2306dreamdiffusion, singh2023eeg2image} which measures the quality of images, \textit{Frechet Inception Distance (FID)} for measuring realism \cite{bai2306dreamdiffusion, singh2024learning, ahmadieh2024visual}, and saliency metrics such as \textit{Structural Similarity Index (SSIM)} for assessing perceptual fidelity \cite{khaleghi2022visual, shimizu2022improving, bai2306dreamdiffusion, ahmadieh2024visual, sugimoto2024image}. Other useful metrics are PixCorr (Pixel-wise Correlation) \cite{shimizu2022improving}, \textit{Kernel Inception Distance (KID)} \cite{singh2024learning}, \textit{LPIPS (Learned Perceptual Image Patch Similarity)} \cite{bai2306dreamdiffusion} and \textit{Diversity Score} \cite{mishra2023neurogan}.


\section{EEG-to-Text Generation}

This section discusses how AI learns brain signal representations from EEG data and maps them to linguistic representations, with an overview depicted in Figure \ref{fig:EEG-text}. We survey use cases, techniques, concerns, and EEG feature encoding methods for text generation.

\begin{figure}[t]
     \centering
    {\includegraphics[width=0.45\textwidth]{images/EEG-text.png}}
   \caption{Reconstructing text from EEG, with eye-tracking data used to capture word-level EEG signals}
	\label{fig:EEG-text}
\end{figure}

\subsection{Use Cases and Addressed Concerns}

The studies referenced in this section share a common use case: generating text from EEG signals. Several studies \cite{biswal2019eegtotext,srivastava2020think2type,yang2023thoughts,rathod2024folded} use the closed vocabulary approach, relying on a fixed set of pre-defined words for EEG-based decoding. Among these, \citet{srivastava2020think2type,yang2023thoughts} investigate text generation using morse code representation of EEG signals, where users' active intent is captured, mapped to morse codes, and then translated to text format.

Recent studies \cite{wang2022open,feng2023aligning,duan2023dewave,liu2024eeg2text,wang2024enhancing,amrani2024deep,tao2024see,mishra2024thought2text,ikegawa2024text,chen2025decoding} overcome closed-vocabulary limitations by exploring open-vocabulary text generation to emulate naturalistic conversations. These studies also address the impact of subjectivity in subject-dependent EEG representation \cite{feng2023aligning,amrani2024deep}, learn cross-modal representation \cite{wang2024enhancing,tao2024see}, and capture long-term dependencies in text and also global contextual information from EEG data that transformers might miss \cite{rathod2024folded,chen2025decoding}.

A significant challenge is the reliance on eye-tracking fixation data as a marker for word-level EEG, which studies like \cite{duan2023dewave,liu2024eeg2text} aim to address. To overcome challenges like defining word-level boundaries in EEG signals and other language processing tasks, some studies have proposed language-agnostic solutions \cite{mishra2024thought2text, ikegawa2024text} which capture signals through image modality and leverage advancements in image-text intermodality to generate text from the collected data. Additionally, \citet{yu2025decoding} introduce a VAE-based augmentation technique to address the issue of limited EEG-text datasets.  

\subsection{Techniques Used Across Studies}

A noteworthy aspect of these studies is the utilization of \textbf{Large Language Models (LLMs)}, particularly BART. Several works \cite{wang2022open, liu2024eeg2text, wang2024enhancing, amrani2024deep, tao2024see, chen2025decoding} have used BART for text generation. In a study by \citet{mishra2024thought2text}, LLMs were fine-tuned on EEG embeddings, image and text data in the training stage to generate text from just EEG signals during inference.

\textbf{Contrastive learning} is widely used in studies like \cite{feng2023aligning, tao2024see, wang2024enhancing} to identify positive EEG-text pairs (e.g., EEG data from the same sentence across subjects) and negative pairs (e.g., EEG data from different sentences or subjects), improving the model's ability to align EEG signals with corresponding text representations. Another key technique is masked signal modeling, employed by \citet{liu2024eeg2text}, where a transformer model is pre-trained to reconstruct randomly masked EEG signals from raw data, enabling the model to learn context, relationships, and semantics within sentence-level EEG signals. An integrated approach by \citet{tao2024see} combines contrastive learning with \textbf{masked signal modeling}, where word-level EEG feature sequences are randomly masked and sentence-level sequences deliberately masked, guided by an intra-modality self-reconstruction objective.

In addition to these techniques, bi-directional Gated Recurrent Units (GRUs) are used to dynamically handle the varying lengths of word-level raw EEG signals \cite{amrani2024deep}. Hierarchical GRUs further improve EEG data processing by capturing both long-range dependencies and local contextual information through the organization of hidden layers hierarchically \cite{chen2025decoding}. A unique approach by \cite{rathod2024folded} employs a folded ensemble deep CNN for text suggestion and a folded ensemble Bidirectional LSTM for text generation, effectively addressing class imbalance and significantly enhancing the accuracy of text generation in closed-vocabulary tasks.

\subsection{EEG Feature Encoding Techniques}

For text generation tasks, EEG signals are encoded into features to capture \textbf{temporal patterns and semantic information}. In the study by \citet{biswal2019eegtotext}, which focuses on generating medical reports, EEG signals are encoded using stacked CNNs to capture \textbf{shift-invariant features} and RCNNs to capture \textbf{temporal patterns}. These features are then used to generate key phenotypes, which hierarchical LSTMs utilize to produce detailed explanations. \citet{srivastava2020think2type} employs an ensemble model to extract EEG embeddings, using CNNs to capture spatial variations and LSTMs to model temporal sequences and long-range dependencies. Another study by \citet{yu2025decoding}, proposes two objectives: classification and sequence-to-sequence (seq2seq) text generation, employing residual blocks for feature extraction in both tasks to capture both \textbf{spatial and temporal features} of the EEG signals effectively. 

Other studies explore the extraction of \textbf{spectral and statistical features} alongside temporal or spatial patterns. \citet{yang2023thoughts}, aiming to translate active intention into text using Morse code, employed Short-Term Fourier Transform (STFT) to extract spectral features and concatenated these with statistical features (e.g., min, max etc. for each channel), in addition to using 1D CNN for spatial features and RNN for temporal features. \citet{rathod2024folded}, another closed-vocabulary solution, used features such as Wavelet Transform (WT), Common Spatial Patterns (CSP), and \textbf{statistical features} to generate EEG feature vectors for classification. 

Various studies have used state-of-the-art transformer architecture for encoding EEG features. \cite{wang2022open} uses a multi-layer transformer encoder to obtain \textbf{EEG mapping from word-level EEG sequences}. \citet{feng2023aligning} uses a transformer-based pre-encoder to convert word-level EEG features into the Seq2Seq embedding space. Another study by \citet{tao2024see} also uses an encoder to extract EEG embeddings and store them in a cross-modal codebook alongside word embeddings obtained from a transformer-based BART model. 

Obtaining word-level EEG signals typically requires markers, often from eye-fixation data like in the Zuco dataset, limiting generalizability. Some studies address this by using \textbf{marker-free and sentence-level EEG signals}. \citet{duan2023dewave} extracts both word-level EEG and raw EEG embeddings. For word-level EEG features with markers, a multi-head transformer layer projects embeddings into feature sequences. For raw EEG waves, a multi-layer transformer encoder is trained for self-reconstruction of waveforms and the transformation of raw EEG signals into sequences of embeddings. In a study by \citet{liu2024eeg2text}, a convolutional transformer model is pretrained with sentence-level EEG signals using a masking technique. It uses a multi-view transformer to encode different brain regions with separate convolutional transformers. \citet{wang2024enhancing} uses both word-level and sentence-level EEG features. It employs a masking technique where word-level sequences are randomly masked and sentence-level features are compulsorily masked.

\citet{chen2025decoding} uses a stacked Hierarchical GRU-based decoder along with Masked Residual Attention Mechanism to obtain EEG representations that capture both \textbf{local and global contextual information}. \citet{amrani2024deep} employs a module consisting of bi-directional GRUs to dynamically address varying lengths of word-level raw EEG signals, a subject-specific 1D convolutional layer, and a multi-layer transformer encoder to encode word-level EEG signals.

\subsection{Evaluation Metrics}
In the surveyed studies, generated text is evaluated against reference text using various established metrics. The commonly used text evaluation metrics are as follows: \textit{METEOR} \cite{banerjee2005meteor}, employed by \cite{biswal2019eegtotext, chen2025decoding}; \textit{BLEU} score \cite{papineni2002bleu}, utilized by \cite{biswal2019eegtotext, wang2022open, feng2023aligning}; \textit{ROUGE} score \cite{lin2004rouge}, adopted by \cite{wang2022open, feng2023aligning, duan2023dewave, liu2024eeg2text, wang2024enhancing}; and \textit{BERTScore} \cite{zhang2019bertscore}, used by \cite{amrani2024deep, mishra2024thought2text}. Other metrics include \textit{Word Error Rate (WER)} used by \cite{feng2023aligning}, \textit{Translation Error Rate (TER)}, and \textit{BLEURT} \cite{sellam2020bleurt}, used by \cite{chen2025decoding}.



\section{EEG-to-Sound/Speech Generation}

We review studies focused on EEG-based generation of sound, speech, voice or music and cover use cases, concerns, techniques, and EEG feature encoding methods for generating Sound or Speech from from EEG.

\subsection{Use Cases and Addressed Concerns}   

EEG-based generation has been explored in various fields beyond image reconstruction, particularly in audio and speech-related applications. These include speech synthesis \cite{krishna2021advancing, lee2023speech}, music decoding and reconstruction \cite{ramirez2022eeg2mel, postolache2024naturalistic}, emotive music generation \cite{jiang2024eeg}, voice reconstruction \cite{lee2023towards}, talking-face generation \cite{park2024towards}, and speech recovery \cite{mizuno2024investigation}. While some studies focus on decoding audio signals for listening tasks in speech or music perception \cite{krishna2021advancing, ramirez2022eeg2mel, park2024towards, mizuno2024investigation, postolache2024naturalistic, jiang2024eeg}, others also investigate speaking tasks and imagined speech \cite{krishna2021advancing, lee2023towards, lee2023speech}.

For more naturalistic communication, \citet{lee2023towards} converts EEG signals recorded during imagined speech into the user’s own voice, aiming for personalized speech synthesis. Similarly, \citet{park2024towards} synthesizes speech from EEG along with generating a talking face with lip-sync. Furthermore, these studies tackle issues such as generating fragmented or abstract outputs \cite{park2024towards}, challenges of synthesizing complete speech from EEG \cite{mizuno2024investigation}, being restricted to simpler music with limited timbres \cite{postolache2024naturalistic}, and the absence of a standardized vocabulary for aligning EEG and audio data \cite{jiang2024eeg}.

\subsection{Techniques Used Across Studies}

\textbf{Convolutional Neural Network (CNN)}-based deep learning models have been used in studies \cite{krishna2021advancing, ramirez2022eeg2mel} to generate audio waveforms from EEG input. \citet{krishna2021advancing} explores speech synthesis for both speaking and listening tasks, using a deep learning architecture with temporal convolution layers, 1D layer, and a time-distributed layer to generate audio waveforms directly. Similarly, \citet{ramirez2022eeg2mel} reconstructs music stimuli using sequential CNN regressors.

\citet{lee2023towards} propose NeuroTalk framework for voice reconstruction from imagined speech. The framework uses a generator based on \textbf{GRUs} to capture sequential EEG information, which outputs a mel-spectrogram. Mel-spectrogram is then converted into a waveform using a \textbf{HiFi-GAN vocoder} \cite{kong2020hifi}, and the resulting waveform is transcribed into text using an \textbf{Automatic Speech Recognition (ASR)} system based on HuBERT \cite{hsu2021hubert}, a self-supervised speech representation learning method. \citet{park2024towards} uses NeuroTalk framework to synthesize audible speech and integrates it with a personalized talking face using \textbf{Wave2Lip} \cite{prajwal2020lip} and Apple API-based avatar generator that accurately lip-sync to the synthesized speech.


\textbf{Transformers and Latent Diffusion Models} have been used to reconstruct speech \cite{mizuno2024investigation} and music \cite{postolache2024naturalistic, jiang2024eeg}. \citet{jiang2024eeg} employs a Transformer model for emotive music generation, while \citet{postolache2024naturalistic} decodes naturalistic music from EEG using a ControlNet adapter \cite{zhang2023adding} to guide AudioLDM2 \cite{liu2024audioldm}, a pre-trained diffusion model, improving control over the generated music.


\subsection{EEG Feature Encoding Techniques}

For speech, voice, and music decoding or generation from EEG, EEG signals are either transformed into intermediate representations, such as mel-spectrograms, or decoded into acoustic and articulatory features\cite{krishna2021advancing}, or EEG temporal features \cite{jiang2024eeg} are utilized. Mel-spectrograms are especially useful, as they offer a shared representational state for both neural signals and audio, enabling more efficient translation between the two modalities.

\citet{krishna2021advancing} incorporates an attention model to predict \textbf{articulatory features} and another attention-regression model to convert these predicted features into \textbf{acoustic features}. Similarly, \citet{jiang2024eeg} extracts EEG tokens through a multi-step process which includes DBSCAN clustering algorithm to derive \textbf{EEG temporal features}. These features are eventually transformed \textbf{EEG positional encoding EEG features} using positional encoding, which are used to form EEG tokens.

In studies using \textbf{mel-spectrograms} as intermediate representations, \citet{ramirez2022eeg2mel} employs a sequential CNN-based regressor to directly map EEG input to time-aligned music spectra. \citet{lee2023speech} seeks to adapt spoken EEG to the subspace of imagined EEG using Common Spatial Pattern (CSP) filters trained on imagined EEG, aiming to generate a user's voice from imagined speech. These CSP filters extract temporal oscillation patterns, minimizing distribution differences between spoken and imagined EEG. Similarly, \citet{postolache2024naturalistic} applies this technique while temporally aligning users' voices with brain signals, using triggers to mark onset intervals and clearly distinguish actual utterance intervals in continuous brain signals.

\subsection{Evaluation Metrics}

EEG-to-speech generation is evaluated using quantitative and qualitative metrics, based on its time-series structure, which also enables its representation as mel-spectrograms. \textit{Mel Cepstral Distortion (MCD)} and \textit{Root Mean Square Error (RMSE)} measure similarity between reconstructed and original speech signals \cite{krishna2021advancing, park2024towards}, while \textit{Structural Similarity Index (SSI)} and \textit{Peak Signal-to-Noise Ratio (PSNR)} assess spectrogram quality \cite{ramirez2022eeg2mel}. Linguistic accuracy is evaluated using Word Error Rate (WER), Character Error Rate (CER), and BERTScore \cite{mizuno2024investigation}, and perceptual quality is quantified with \textit{Frechet Audio Distance (FAD)} \cite{postolache2024naturalistic}. Additional metrics include \textit{Hits@k} for search relevance \cite{jiang2024eeg} and \textit{Mean Opinion Score (MOS)} for subjective quality assessment \cite{lee2023towards}.







\section{Discussion}\label{sec:discussion}



\subsection{From Interactive Prompting to Interactive Multi-modal Prompting}
The rapid advancements of large pre-trained generative models including large language models and text-to-image generation models, have inspired many HCI researchers to develop interactive tools to support users in crafting appropriate prompts.
% Studies on this topic in last two years' HCI conferences are predominantly focused on helping users refine single-modality textual prompts.
Many previous studies are focused on helping users refine single-modality textual prompts.
However, for many real-world applications concerning data beyond text modality, such as multi-modal AI and embodied intelligence, information from other modalities is essential in constructing sophisticated multi-modal prompts that fully convey users' instruction.
This demand inspires some researchers to develop multimodal prompting interactions to facilitate generation tasks ranging from visual modality image generation~\cite{wang2024promptcharm, promptpaint} to textual modality story generation~\cite{chung2022tale}.
% Some previous studies contributed relevant findings on this topic. 
Specifically, for the image generation task, recent studies have contributed some relevant findings on multi-modal prompting.
For example, PromptCharm~\cite{wang2024promptcharm} discovers the importance of multimodal feedback in refining initial text-based prompting in diffusion models.
However, the multi-modal interactions in PromptCharm are mainly focused on the feedback empowered the inpainting function, instead of supporting initial multimodal sketch-prompt control. 

\begin{figure*}[t]
    \centering
    \includegraphics[width=0.9\textwidth]{src/img/novice_expert.pdf}
    \vspace{-2mm}
    \caption{The comparison between novice and expert participants in painting reveals that experts produce more accurate and fine-grained sketches, resulting in closer alignment with reference images in close-ended tasks. Conversely, in open-ended tasks, expert fine-grained strokes fail to generate precise results due to \tool's lack of control at the thin stroke level.}
    \Description{The comparison between novice and expert participants in painting reveals that experts produce more accurate and fine-grained sketches, resulting in closer alignment with reference images in close-ended tasks. Novice users create rougher sketches with less accuracy in shape. Conversely, in open-ended tasks, expert fine-grained strokes fail to generate precise results due to \tool's lack of control at the thin stroke level, while novice users' broader strokes yield results more aligned with their sketches.}
    \label{fig:novice_expert}
    % \vspace{-3mm}
\end{figure*}


% In particular, in the initial control input, users are unable to explicitly specify multi-modal generation intents.
In another example, PromptPaint~\cite{promptpaint} stresses the importance of paint-medium-like interactions and introduces Prompt stencil functions that allow users to perform fine-grained controls with localized image generation. 
However, insufficient spatial control (\eg, PromptPaint only allows for single-object prompt stencil at a time) and unstable models can still leave some users feeling the uncertainty of AI and a varying degree of ownership of the generated artwork~\cite{promptpaint}.
% As a result, the gap between intuitive multi-modal or paint-medium-like control and the current prompting interface still exists, which requires further research on multi-modal prompting interactions.
From this perspective, our work seeks to further enhance multi-object spatial-semantic prompting control by users' natural sketching.
However, there are still some challenges to be resolved, such as consistent multi-object generation in multiple rounds to increase stability and improved understanding of user sketches.   


% \new{
% From this perspective, our work is a step forward in this direction by allowing multi-object spatial-semantic prompting control by users' natural sketching, which considers the interplay between multiple sketch regions.
% % To further advance the multi-modal prompting experience, there are some aspects we identify to be important.
% % One of the important aspects is enhancing the consistency and stability of multiple rounds of generation to reduce the uncertainty and loss of control on users' part.
% % For this purpose, we need to develop techniques to incorporate consistent generation~\cite{tewel2024training} into multi-modal prompting framework.}
% % Another important aspect is improving generative models' understanding of the implicit user intents \new{implied by the paint-medium-like or sketch-based input (\eg, sketch of two people with their hands slightly overlapping indicates holding hand without needing explicit prompt).
% % This can facilitate more natural control and alleviate users' effort in tuning the textual prompt.
% % In addition, it can increase users' sense of ownership as the generated results can be more aligned with their sketching intents.
% }
% For example, when users draw sketches of two people with their hands slightly overlapping, current region-based models cannot automatically infer users' implicit intention that the two people are holding hands.
% Instead, they still require users to explicitly specify in the prompt such relationship.
% \tool addresses this through sketch-aware prompt recommendation to fill in the necessary semantic information, alleviating users' workload.
% However, some users want the generative AI in the future to be able to directly infer this natural implicit intentions from the sketches without additional prompting since prompt recommendation can still be unstable sometimes.


% \new{
% Besides visual generation, 
% }
% For example, one of the important aspect is referring~\cite{he2024multi}, linking specific text semantics with specific spatial object, which is partly what we do in our sketch-aware prompt recommendation.
% Analogously, in natural communication between humans, text or audio alone often cannot suffice in expressing the speakers' intentions, and speakers often need to refer to an existing spatial object or draw out an illustration of her ideas for better explanation.
% Philosophically, we HCI researchers are mostly concerned about the human-end experience in human-AI communications.
% However, studies on prompting is unique in that we should not just care about the human-end interaction, but also make sure that AI can really get what the human means and produce intention-aligned output.
% Such consideration can drastically impact the design of prompting interactions in human-AI collaboration applications.
% On this note, although studies on multi-modal interactions is a well-established topic in HCI community, it remains a challenging problem what kind of multi-modal information is really effective in helping humans convey their ideas to current and next generation large AI models.




\subsection{Novice Performance vs. Expert Performance}\label{sec:nVe}
In this section we discuss the performance difference between novice and expert regarding experience in painting and prompting.
First, regarding painting skills, some participants with experience (4/12) preferred to draw accurate and fine-grained shapes at the beginning. 
All novice users (5/12) draw rough and less accurate shapes, while some participants with basic painting skills (3/12) also favored sketching rough areas of objects, as exemplified in Figure~\ref{fig:novice_expert}.
The experienced participants using fine-grained strokes (4/12, none of whom were experienced in prompting) achieved higher IoU scores (0.557) in the close-ended task (0.535) when using \tool. 
This is because their sketches were closer in shape and location to the reference, making the single object decomposition result more accurate.
Also, experienced participants are better at arranging spatial location and size of objects than novice participants.
However, some experienced participants (3/12) have mentioned that the fine-grained stroke sometimes makes them frustrated.
As P1's comment for his result in open-ended task: "\emph{It seems it cannot understand thin strokes; even if the shape is accurate, it can only generate content roughly around the area, especially when there is overlapping.}" 
This suggests that while \tool\ provides rough control to produce reasonably fine results from less accurate sketches for novice users, it may disappoint experienced users seeking more precise control through finer strokes. 
As shown in the last column in Figure~\ref{fig:novice_expert}, the dragon hovering in the sky was wrongly turned into a standing large dragon by \tool.

Second, regarding prompting skills, 3 out of 12 participants had one or more years of experience in T2I prompting. These participants used more modifiers than others during both T2I and R2I tasks.
Their performance in the T2I (0.335) and R2I (0.469) tasks showed higher scores than the average T2I (0.314) and R2I (0.418), but there was no performance improvement with \tool\ between their results (0.508) and the overall average score (0.528). 
This indicates that \tool\ can assist novice users in prompting, enabling them to produce satisfactory images similar to those created by users with prompting expertise.



\subsection{Applicability of \tool}
The feedback from user study highlighted several potential applications for our system. 
Three participants (P2, P6, P8) mentioned its possible use in commercial advertising design, emphasizing the importance of controllability for such work. 
They noted that the system's flexibility allows designers to quickly experiment with different settings.
Some participants (N = 3) also mentioned its potential for digital asset creation, particularly for game asset design. 
P7, a game mod developer, found the system highly useful for mod development. 
He explained: "\emph{Mods often require a series of images with a consistent theme and specific spatial requirements. 
For example, in a sacrifice scene, how the objects are arranged is closely tied to the mod's background. It would be difficult for a developer without professional skills, but with this system, it is possible to quickly construct such images}."
A few participants expressed similar thoughts regarding its use in scene construction, such as in film production. 
An interesting suggestion came from participant P4, who proposed its application in crime scene description. 
She pointed out that witnesses are often not skilled artists, and typically describe crime scenes verbally while someone else illustrates their account. 
With this system, witnesses could more easily express what they saw themselves, potentially producing depictions closer to the real events. "\emph{Details like object locations and distances from buildings can be easily conveyed using the system}," she added.

% \subsection{Model Understanding of Users' Implicit Intents}
% In region-sketch-based control of generative models, a significant gap between interaction design and actual implementation is the model's failure in understanding users' naturally expressed intentions.
% For example, when users draw sketches of two people with their hands slightly overlapping, current region-based models cannot automatically infer users' implicit intention that the two people are holding hands.
% Instead, they still require users to explicitly specify in the prompt such relationship.
% \tool addresses this through sketch-aware prompt recommendation to fill in the necessary semantic information, alleviating users' workload.
% However, some users want the generative AI in the future to be able to directly infer this natural implicit intentions from the sketches without additional prompting since prompt recommendation can still be unstable sometimes.
% This problem reflects a more general dilemma, which ubiquitously exists in all forms of conditioned control for generative models such as canny or scribble control.
% This is because all the control models are trained on pairs of explicit control signal and target image, which is lacking further interpretation or customization of the user intentions behind the seemingly straightforward input.
% For another example, the generative models cannot understand what abstraction level the user has in mind for her personal scribbles.
% Such problems leave more challenges to be addressed by future human-AI co-creation research.
% One possible direction is fine-tuning the conditioned models on individual user's conditioned control data to provide more customized interpretation. 

% \subsection{Balance between recommendation and autonomy}
% AIGC tools are a typical example of 
\subsection{Progressive Sketching}
Currently \tool is mainly aimed at novice users who are only capable of creating very rough sketches by themselves.
However, more accomplished painters or even professional artists typically have a coarse-to-fine creative process. 
Such a process is most evident in painting styles like traditional oil painting or digital impasto painting, where artists first quickly lay down large color patches to outline the most primitive proportion and structure of visual elements.
After that, the artists will progressively add layers of finer color strokes to the canvas to gradually refine the painting to an exquisite piece of artwork.
One participant in our user study (P1) , as a professional painter, has mentioned a similar point "\emph{
I think it is useful for laying out the big picture, give some inspirations for the initial drawing stage}."
Therefore, rough sketch also plays a part in the professional artists' creation process, yet it is more challenging to integrate AI into this more complex coarse-to-fine procedure.
Particularly, artists would like to preserve some of their finer strokes in later progression, not just the shape of the initial sketch.
In addition, instead of requiring the tool to generate a finished piece of artwork, some artists may prefer a model that can generate another more accurate sketch based on the initial one, and leave the final coloring and refining to the artists themselves.
To accommodate these diverse progressive sketching requirements, a more advanced sketch-based AI-assisted creation tool should be developed that can seamlessly enable artist intervention at any stage of the sketch and maximally preserve their creative intents to the finest level. 

\subsection{Ethical Issues}
Intellectual property and unethical misuse are two potential ethical concerns of AI-assisted creative tools, particularly those targeting novice users.
In terms of intellectual property, \tool hands over to novice users more control, giving them a higher sense of ownership of the creation.
However, the question still remains: how much contribution from the user's part constitutes full authorship of the artwork?
As \tool still relies on backbone generative models which may be trained on uncopyrighted data largely responsible for turning the sketch into finished artwork, we should design some mechanisms to circumvent this risk.
For example, we can allow artists to upload backbone models trained on their own artworks to integrate with our sketch control.
Regarding unethical misuse, \tool makes fine-grained spatial control more accessible to novice users, who may maliciously generate inappropriate content such as more realistic deepfake with specific postures they want or other explicit content.
To address this issue, we plan to incorporate a more sophisticated filtering mechanism that can detect and screen unethical content with more complex spatial-semantic conditions. 
% In the future, we plan to enable artists to upload their own style model

% \subsection{From interactive prompting to interactive spatial prompting}


\subsection{Limitations and Future work}

    \textbf{User Study Design}. Our open-ended task assesses the usability of \tool's system features in general use cases. To further examine aspects such as creativity and controllability across different methods, the open-ended task could be improved by incorporating baselines to provide more insightful comparative analysis. 
    Besides, in close-ended tasks, while the fixing order of tool usage prevents prior knowledge leakage, it might introduce learning effects. In our study, we include practice sessions for the three systems before the formal task to mitigate these effects. In the future, utilizing parallel tests (\textit{e.g.} different content with the same difficulty) or adding a control group could further reduce the learning effects.

    \textbf{Failure Cases}. There are certain failure cases with \tool that can limit its usability. 
    Firstly, when there are three or more objects with similar semantics, objects may still be missing despite prompt recommendations. 
    Secondly, if an object's stroke is thin, \tool may incorrectly interpret it as a full area, as demonstrated in the expert results of the open-ended task in Figure~\ref{fig:novice_expert}. 
    Finally, sometimes inclusion relationships (\textit{e.g.} inside) between objects cannot be generated correctly, partially due to biases in the base model that lack training samples with such relationship. 

    \textbf{More support for single object adjustment}.
    Participants (N=4) suggested that additional control features should be introduced, beyond just adjusting size and location. They noted that when objects overlap, they cannot freely control which object appears on top or which should be covered, and overlapping areas are currently not allowed.
    They proposed adding features such as layer control and depth control within the single-object mask manipulation. Currently, the system assigns layers based on color order, but future versions should allow users to adjust the layer of each object freely, while considering weighted prompts for overlapping areas.

    \textbf{More customized generation ability}.
    Our current system is built around a single model $ColorfulXL-Lightning$, which limits its ability to fully support the diverse creative needs of users. Feedback from participants has indicated a strong desire for more flexibility in style and personalization, such as integrating fine-tuned models that cater to specific artistic styles or individual preferences. 
    This limitation restricts the ability to adapt to varied creative intents across different users and contexts.
    In future iterations, we plan to address this by embedding a model selection feature, allowing users to choose from a variety of pre-trained or custom fine-tuned models that better align with their stylistic preferences. 
    
    \textbf{Integrate other model functions}.
    Our current system is compatible with many existing tools, such as Promptist~\cite{hao2024optimizing} and Magic Prompt, allowing users to iteratively generate prompts for single objects. However, the integration of these functions is somewhat limited in scope, and users may benefit from a broader range of interactive options, especially for more complex generation tasks. Additionally, for multimodal large models, users can currently explore using affordable or open-source models like Qwen2-VL~\cite{qwen} and InternVL2-Llama3~\cite{llama}, which have demonstrated solid inference performance in our tests. While GPT-4o remains a leading choice, alternative models also offer competitive results.
    Moving forward, we aim to integrate more multimodal large models into the system, giving users the flexibility to choose the models that best fit their needs. 
    


\section{Conclusion}\label{sec:conclusion}
In this paper, we present \tool, an interactive system designed to help novice users create high-quality, fine-grained images that align with their intentions based on rough sketches. 
The system first refines the user's initial prompt into a complete and coherent one that matches the rough sketch, ensuring the generated results are both stable, coherent and high quality.
To further support users in achieving fine-grained alignment between the generated image and their creative intent without requiring professional skills, we introduce a decompose-and-recompose strategy. 
This allows users to select desired, refined object shapes for individual decomposed objects and then recombine them, providing flexible mask manipulation for precise spatial control.
The framework operates through a coarse-to-fine process, enabling iterative and fine-grained control that is not possible with traditional end-to-end generation methods. 
Our user study demonstrates that \tool offers novice users enhanced flexibility in control and fine-grained alignment between their intentions and the generated images.


\section*{Limitations}
\label{sec:limitations}

While this survey provides a comprehensive overview of EEG-based generative AI applications, certain limitations exist due to the focused scope of this work. Firstly, this survey primarily covers EEG-based Brain-Computer Interfaces (BCIs), deliberately excluding other neuroimaging techniques such as fMRI, Magnetoencephalography (MEG), and Near-Infrared Spectroscopy (NIRS). Although these modalities play a significant role in BCI research and offer complementary advantages in terms of spatial resolution and multimodal integration, their detailed discussion is beyond the scope of this work.

Secondly, due to space constraints, in-depth discussions on the cognitive underpinnings of EEG signals -- such as their biological origins, neural interpretations, and relationships with brain activity—have been omitted. Similarly, technical details regarding EEG hardware, electrode configurations, and device specifications have been largely excluded for brevity. While these aspects are crucial for practical EEG-based applications, our focus remains on the computational and generative modeling aspects of EEG data processing.

Finally, this survey assumes a general background in EEG signal processing, and generative modeling and expects familiarity with these foundational concepts. While we provide essential explanations, a more in-depth introduction to the fundamentals of EEG and BCI technology is outside the scope of this review. 
%Future work could expand upon these missing aspects to provide a more holistic perspective on the intersection of neuroscience and generative AI.
\section*{Ethics Statement}
EEG data is inherently sensitive, as it contains neural activity patterns that can potentially reveal cognitive states and sometimes personal information. While the majority of the works covered in this survey adhere to established ethical guidelines and standards, some studies may require additional ethical justifications. We have not conducted an exhaustive review of the ethical compliance of each cited work but emphasize the importance of ethical transparency in EEG research. We do not endorse studies that raise ethical concerns or lack proper ethical oversight. Any research involving EEG data collection and analysis should rigorously follow ethical protocols, including obtaining informed consent, ensuring data anonymity, and minimizing risks to participants.

Additionally, we acknowledge the use of OpenAI’s ChatGPT-4 system solely for enhancing writing efficiency, generating LaTeX code, and aiding in error debugging. No content related to the survey's research findings, citations, or factual discussions was autogenerated or retrieved using Generative AI-based search mechanisms. Our work remains grounded in peer-reviewed literature and ethical academic standards.


% \section*{Acknowledgements}

% Entries for the entire Anthology, followed by custom entries
\bibliography{bcisurvey}
\bibliographystyle{acl_natbib}
% \bibliographystyle{unsrtnat}

\end{document}
