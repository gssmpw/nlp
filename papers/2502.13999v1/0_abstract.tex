
\begin{abstract}

With the growing popularity of personalized human content creation and sharing, there is a rising demand for advanced techniques in customized human image generation. However, current methods struggle to simultaneously maintain the fidelity of human identity and ensure the consistency of textual prompts, often resulting in suboptimal outcomes. This shortcoming is primarily due to the lack of effective constraints during the simultaneous integration of visual and textual prompts, leading to unhealthy mutual interference that compromises the full expression of both types of input. Building on prior research that suggests visual and textual conditions influence different regions of an image in distinct ways, we introduce a novel Dual-Pathway Adapter (DP-Adapter) to enhance both high-fidelity identity preservation and textual consistency in personalized human image generation. Our approach begins by decoupling the target human image into visually sensitive and text-sensitive regions. For visually sensitive regions, DP-Adapter employs an Identity-Enhancing Adapter (IEA) to preserve detailed identity features. For text-sensitive regions, we introduce a Textual-Consistency Adapter (TCA) to minimize visual interference and ensure the consistency of textual semantics. To seamlessly integrate these pathways, we develop a Fine-Grained Feature-Level Blending (FFB) module that efficiently combines hierarchical semantic features from both pathways, resulting in more natural and coherent synthesis outcomes. Additionally, DP-Adapter supports various innovative applications, including controllable headshot-to-full-body portrait generation, age editing, old-photo to reality, and expression editing.
Extensive experiments demonstrate that DP-Adapter outperforms state-of-the-art methods in both visual fidelity and text consistency, highlighting its effectiveness and versatility in the field of human image generation.
\end{abstract}

% Specifically, we observe that poor text alignment stems from significant interference by visual signals when both visual and text signals are are jointly fed into the diffusion model. Based on this observation, we propose a self diffusion fusion method that combines different versions of the same diffusion model, where the versions vary based on the intensity of visual signal injection.