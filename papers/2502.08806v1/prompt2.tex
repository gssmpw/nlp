\begin{figure}
    \centering
    \scriptsize
    \begin{tcolorbox}[colback=blue!5!white, colframe=blue!75!black]

\textbf{Task Overview}

Create a unit test based on the provided method signature and given context. All necessary imports for the test have been included, so focus solely on writing the unit test method. No additional libraries can be imported.

\textbf{Retrieved Code Snippets for Reference}

Several code snippets have been provided for your reference:
\begin{verbatim}
{{ retrieved_snippet }}
\end{verbatim}

\textbf{Hint:} Depending on the target unit test method's signature, you might want to utilize or test these code snippets. 

\textbf{Requirement:} Each retrieved code snippet must be used at least once in the final unit test method.

\textbf{Context Information}

Below is the test code context, including necessary imports and setups for unittest:
\begin{verbatim}
{{ context }}
\end{verbatim}

\textbf{Signature of Target Unit Test Method}

This is the unit test method signature you need to complete:
\begin{verbatim}
{{ prefix }}
\end{verbatim}

\textbf{Your Task}

Continue writing the unit test method immediately following the provided method signature, ensuring to include at least one substantial assertion. Your task is to replace ``$<$YOUR CODE HERE$>$'' with correctly indented code.

\textbf{Output Format}

Please return the \textbf{WHOLE} \textbf{FILE} content (including the context and the completed unit test method, but not the Retrieved Code Snippets). You should not modify the context part of the code.

\textbf{Example 1:}
\begin{verbatim}
class A:
    def __init__(self):
        self.a = 1
        self.b = 2
    def test_comparing_a_b():
<YOUR CODE HERE>
    def test_assert_b():
        assert self.b == 2
\end{verbatim}

The output should follow the correct indentation principles. For example:
\begin{verbatim}
class A:
    def __init__(self):
        self.a = 1
        self.b = 2
    def test_comparing_a_b():
        assert self.a == 1       # Correct indentation with eight spaces before `assert`
        assert self.a != self.b  # Correct indentation with eight spaces before `assert`
    def test_assert_b():
        assert self.b == 2
\end{verbatim}

Here is one WRONG example:
\begin{verbatim}
# wrong example
class A:
    def __init__(self):
        self.a = 1
        self.b = 2
    def test_comparing_a_b():
    assert self.a == 1       # Incorrect indentation
assert self.a != self.b  # Incorrect indentation
    def test_assert_b():
        return           # YOU SHOULD NEVER modify the context code
\end{verbatim}

\textbf{Example 2:}
\begin{verbatim}
def test_value_c():
<YOUR CODE HERE>
\end{verbatim}

The output should follow the correct indentation principles. For example:
\begin{verbatim}
def test_value_c():
    assert c == 3
\end{verbatim}

- Keep in mind that the ``$<$YOUR CODE HERE$>$'' section lacks any leading spaces. 
- Ensure you include the appropriate amount of indentation before the `assert` statements. 
- If the method signature necessitates returning an object, implement that as well.

\end{tcolorbox}

    \caption{Contextual $prompt_{full}$ of Task II.}
    \label{fig:task2prompt}
\end{figure}