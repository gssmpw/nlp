

\begin{figure}[t]
    \centering

\begin{tcolorbox}[colback=blue!5!white, colframe=blue!75!black,title=Task I (Problem Only)]

\textbf{Task Overview}

You need to complete the assertion in the provided unit test method by considering the context from the corresponding Python file.

\textbf{Context Information}

Below is the file context, which includes necessary imports and setups for the unittest:
\begin{verbatim}
{{ context }}
\end{verbatim}

\textbf{Target Unit Test Method}

This is the unit test method for which you need to complete the assertion statement:
\begin{verbatim}
{{ prefix }}
\end{verbatim}

\textbf{Your Task}

Right after the above code block, you need to predict and fill in the blank (\texttt{{cloze\_key}}) in the following assertion statement:
\begin{verbatim}
{{ cloze_question }}
\end{verbatim}
Use the context and the provided prefix in the unit test method to infer the missing part of the assertion.

\textbf{Output Format}

Your output should be a finished assertion statement. Do \textbf{not} generate the entire unit test method; only produce the assertion statement.

\textbf{Example}

For instance, if the cloze question is \texttt{assert \_\_\_\_\_\_ == (25, 25, 75, 75)}, and your prediction for the blank is \texttt{im.getbbox(alpha\_only=False)}, your output should be:

\begin{verbatim}
    assert im.getbbox(alpha_only=False) == (25, 25, 75, 75)
\end{verbatim}

Please provide your output in the desired format without additional explanations or step-by-step guidance.

\textbf{Notes}

\begin{itemize}
    \item The completed assertion should not be trivial. For instance, \texttt{assert True == True} and \texttt{assert str("a") == "a"} are considered trivial assertions.
    \item There will be precisely one blank in the assertion statement to be filled in.
\end{itemize}

\end{tcolorbox}

    \caption{PO prompt of Task I.}
    \label{fig:task1po}
\end{figure}




\begin{figure}[t]
    \centering

\begin{tcolorbox}[colback=blue!5!white, colframe=blue!75!black,title=Task I (Contextual)]

\textbf{Task Overview}

You need to complete the assertion in the provided unit test method by considering the context from the corresponding Python file.

\textbf{Retrieved Code Snippets for Reference}

Here are a few code snippets retrieved for your reference while making your prediction:

\begin{verbatim}
{{ retrieved_snippet }}
\end{verbatim}

\textbf{Context Information}

Below is the file context, which includes necessary imports and setups for the unittest:
\begin{verbatim}
{{ context }}
\end{verbatim}

\textbf{Target Unit Test Method}

This is the unit test method for which you need to complete the assertion statement:
\begin{verbatim}
{{ prefix }}
\end{verbatim}

\textbf{Your Task}

Right after the above code block, you need to predict and fill in the blank (\texttt{{ cloze\_key}}) in the following assertion statement:
\begin{verbatim}
{{ cloze_question }}
\end{verbatim}
Use the context and the provided prefix in the unit test method to infer the missing part of the assertion.

\textbf{Output Format}

Your output should be a finished assertion statement. Do \textbf{not} generate the entire unit test method; only produce the assertion statement.

\textbf{Example}

For instance, if the cloze question is \texttt{assert \_\_\_\_\_\_ == (25, 25, 75, 75)}, and your prediction for the blank is \texttt{im.getbbox(alpha\_only=False)}, your output should be:

\begin{verbatim}
    assert im.getbbox(alpha_only=False) == (25, 25, 75, 75)
\end{verbatim}

Please provide your output in the desired format without additional explanations or step-by-step guidance.

\textbf{Notes}
\begin{enumerate}
    \item The completed assertion should not be trivial. For instance, \texttt{assert True == True} and \texttt{assert str("a") == "a"} are considered trivial assertions.
    \item There will be precisely one blank in the assertion statement to be filled in.
\end{enumerate}

\end{tcolorbox}

    \caption{Contextual prompt of Task I.}
\label{fig:task1ctx}
\end{figure}