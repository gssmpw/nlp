\begin{figure}
    \centering
    \small

\begin{tcolorbox}[colback=blue!5!white, colframe=blue!75!black]

\textbf{Task Overview}

Create a unit test based on the provided method signature and given context. All necessary imports for the test have been included, so focus solely on writing the unit test method. No additional libraries can be imported.

\textbf{Retrieved Code Snippets for Reference}

Several code snippets have been provided for your reference:

\begin{verbatim}
{{ retrieved_snippet }}
\end{verbatim}

\textbf{Output Format}

When completing a code snippet, the output should include the entire code context provided, including any imports, class definitions, or function signatures. Replace placeholders with meaningful code that fits the context. Example:

\textbf{Given:}
\begin{verbatim}
import pytest
from aiohttp import web
from aiohttp.web_urldispatcher import UrlDispatcher

@pytest.fixture
def router() -> UrlDispatcher:
    return UrlDispatcher()

def test_get(router: UrlDispatcher) -> None:
{{ SYMBOL_YOUR_CODE }}
\end{verbatim}

The goal is to complete the method \texttt{test\_get(router: UrlDispatcher) -> None:} with correct indentation and necessary logic while retaining the full context.

\textbf{Example output:}
\begin{verbatim}
import pytest
from aiohttp import web
from aiohttp.web_urldispatcher import UrlDispatcher

@pytest.fixture
def router() -> UrlDispatcher:
    return UrlDispatcher()

def test_get(router: UrlDispatcher) -> None:
    async def handler(request: web.Request) -> NoReturn:
        assert False

    router.add_routes([web.get("/", handler)])
    route = list(router.routes())[1]
    assert route.handler is handler
    assert route.method == "GET"
\end{verbatim}

\textbf{Task with Context Provided}

Below is the test code context, including necessary imports and setups for \texttt{unittest}. Following the method signature "\texttt{{ method\_signature }}", you need to complete the unit test method. Ensure to include at least one substantial assertion. Replace \texttt{{ SYMBOL\_YOUR\_CODE }} with the correctly indented test code.

\begin{verbatim}
{{ context }}
\end{verbatim}

\textbf{Hint:} Depending on the target unit test method's signature, you might want to utilize or test these code snippets. 

\textbf{Requirement:}
\begin{itemize}
  \item Ensure you return the unit test method including the given method signature. Do not modify the method signature.
  \item You are not allowed to import anything else as all necessary imports for the case have been provided.
  \item Properly indent each line before including it.
  \item Avoid using ANY trivial assertions such as \texttt{assert True == True} or \texttt{assert str("a") == "a"}, as they will be deemed incorrect.
\end{itemize}

\begin{verbatim}
{{ coverage_requirement }}
\end{verbatim}

\end{tcolorbox}
    \caption{Contextual $prompt_{full}$ for Task III. }
    \label{fig:task3prompt}
\end{figure}