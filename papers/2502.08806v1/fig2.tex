
\begin{figure}[t]
    \centering
    \scriptsize
    \begin{tcolorbox}[colback=blue!5!white, colframe=blue!40!black, title=Task I, width=\linewidth]
        
Right after the above code block, you need to predict and fill in the blank \_\_\_\_\_ in the following assertion statement:

\texttt{```}python
\{ cloze\_question\_with\_blank \}
\texttt{```}

Use the context and the provided prefix in the unit test method to infer the missing part of the assertion.


    \end{tcolorbox}
    \begin{tcolorbox}[colback=blue!5!white, colframe=blue!40!black, title=Task II, width=\linewidth]

Here is the checklist:
    
    The generated test code should contain ALL of the following calls:
    
    - \{target\_name\}, a Python \{type\} defined in file \{file\_name\}.

\end{tcolorbox}

    \begin{tcolorbox}[colback=blue!5!white, colframe=blue!40!black, title=Task III, width=\linewidth]
\#\# Coverage Requirement

Please ensure that the following lines of code from the specified files are included and tested. Complete code snippets have already been provided earlier. When referring to coverage, it pertains to pytest coverage. If your code calls a function or method that indirectly utilizes the specified lines, this will also be considered as covering those lines.

\#\#\# \{block.file\_path\}

\texttt{```}python
\{ block.code\_snippet \}
\texttt{```}

This includes \{block.line\_count\} lines from line numbers \{block.start\_line\_number\} to \{ block.end\_line\_number \}.

[more code blocks if available]

\#\# Summary

In summary, there are \{num\_files\} files and \{num\_code\} code segments that need to be covered. 
Please make an effort to enhance the coverage of these lines, either directly or indirectly, to ensure thorough testing with pytest.
    \end{tcolorbox}

    \caption{Task specific prompt template for Task I, II and III. For the complete prompts, check Sec~\ref{app:prompt}.}
    \label{fig:prompt_box}
\end{figure}