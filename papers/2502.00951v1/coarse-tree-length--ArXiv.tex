\documentclass[11pt]{llncs}
% \usepackage{emlines2}
\usepackage{epsfig,latexsym}
\usepackage{xcolor}
\usepackage{tikz} % tikz pictures 
\usepackage{graphicx}%stackengine
\usepackage{subfigure}

% MUAD _______________________________
%%%%%%\usepackage{epsfig,latexsym}
\usepackage{color}
\usepackage{algorithmic}
\usepackage{algorithm}
%\usepackage{amsthm}
\usepackage{xfrac}
\usepackage{epsfig}
\usepackage{amssymb}
\usepackage{graphicx}
\usepackage{url}
%\usepackage{subfig}
\usepackage{amsmath}
\usepackage{hyperref}
\usepackage[square,comma,numbers]{natbib}%%%%%%%%%%%%%
%\usepackage{slashbox}
%\usepackage{array}



%\documentclass[letter, 12pt]{article}
%\usepackage{epsfig}

\newcommand{\qedc}{\hfill $\Box$(Claim)}
%\newcommand{\proof}{\noindent {\bf Proof }}
%\newcommand{\Proof}{\noindent {\bf Proof }}
\newcommand{\WW}{$W_6^{++}$}

%\newtheorem{definition}{Definition}
\newtheorem{claimf}{Claim}
%\newtheorem{theorem}{Theorem}
%\newtheorem{lemma}{Lemma}
%\newtheorem{corollary}{Corollary}
%\newtheorem{proposition}{Proposition}
\newtheorem{observation}{Observation}
\newtheorem{preliminary}{Preliminary}

\usepackage{color}
%\usepackage{epsfig}
\newcommand{\newstuffr}[1]{{\color{red} #1}} %red
\newcommand{\newstuffb}[1]{{\color{blue} #1}} %red
\newcommand{\commentout}[1]{}

\newcommand{\cBT}{{\cal BT}}
\newcommand{\cH}{{\cal H}}
\newcommand{\al}{\alpha}
\newcommand{\dX}{\downarrow$$X}
\newcommand{\dB}{\downarrow$$B}
\newcommand{\cLT}{{\cal LT}}
\newcommand{\cR}{{\cal R}}
\newcommand{\cC}{{\cal C}}
\newcommand{\cT}{{\cal T}}
\newcommand{\cE}{{\cal E}}
\newcommand{\cS}{{\cal S}}
\newcommand{\cF}{{\cal F}}
\newcommand{\cDT}{{\cal DT}}
\newcommand{\an}{{\sf an}}
\newcommand{\nil}{{\sf nil}}
\newcommand{\tb}{{\sf tb}}
\newcommand{\tw}{{\sf tw}}
\newcommand{\tl}{{\sf tl}}
\newcommand{\itl}{{\sf itl}}
\newcommand{\itb}{{\sf itb}}
\newcommand{\w}{{\sf w}}
\newcommand{\BNC}{{\sf BNC}}
\newcommand{\bn}{{\sf bnc}}
\newcommand{\BDS}{{\sf BDS}}
\newcommand{\mc}{{\sf mcw}}
\newcommand{\adt}{{\sf adt}}
\newcommand{\ad}{{\sf ad}} 
\newcommand{\td}{{\sf td}} %\td
\newcommand{\cbc}{{\sf cbc}} %\cbc
\newcommand{\bgc}{{\sf bgc}}
\newcommand{\BGC}{{\sf BGC}} 
\newcommand{\stb}{{\sf stb}} %\stb(G)
\newcommand{\glc}{{\sf glc}} %\stb(G)
\newcommand{\br}{{\sf br}}  
\newcommand{\sh}{{\sf sh}}  
\newcommand{\ph}{{\sf ph}}  
\newcommand{\mf}{{\sf mf}}  %\mf
\newcommand{\p}{{\sf p}}  %\mf
\newcommand{\q}{{\sf q}}  %\mf

\def\lbfso{LexBFS--ordering}
\def\bfso{BFS--ordering}


\def\sig{$\sigma$}
\def\mfv{mutually furthest vertices}

\def\hd{\hat{d}}
\def\hP{\hat{P}}

\oddsidemargin0.0cm \evensidemargin0.0cm \topmargin0.0cm
\headheight0cm \headsep0mm \textheight22cm \textwidth16.2cm
\pagestyle{plain}
\pagenumbering{arabic} %
\sloppy

%\oddsidemargin0.4cm    %
%\evensidemargin0.4cm \topmargin1.2cm \headheight0cm
%\headsep3mm   %
%\textheight22.5cm \textwidth15.4cm


\begin{document}
\sloppy

%\title{Once more on graphs with bounded tree-length}
\title{Graph parameters that are coarsely equivalent to tree-length\thanks{\date*{\today}}
}

\author{Feodor F. Dragan}

\institute{Computer Science Department, Kent State University, Kent, Ohio,  USA \\
\email{dragan@cs.kent.edu}  
}
\maketitle
%\date{September 2024}

%\date*{\today}
\sloppy

\begin{abstract} Two graph parameters are said to be coarsely equivalent if they are within constant factors from each other for every graph $G$. 
Recently, several graph parameters were shown to be coarsely  equivalent to tree-length. Recall that the length of a tree-decomposition $\cT(G)$ of a graph $G$ is the largest diameter of a bag in  $\cT(G)$, and the  tree-length of $G$ is the minimum of the length, over all tree-decompositions of $G$. 
We present simpler and sometimes with better bounds proofs for those known in literature results and further extend this list of graph parameters coarsely equivalent to tree-length. 
Among other new results, we show that the tree-length of a graph $G$ is small if and only if for every bramble ${\cal F}$ (or every Helly family of connected subgraphs ${\cal F}$, or every Helly family of paths ${\cal F}$)  of $G$,  there is a disk in $G$ with  
small radius that intercepts all members of ${\cal F}$. Furthermore, the tree-length of a graph $G$ is small if and only if $G$ can be embedded with a 
small additive distortion to an unweighted tree with the same vertex set as in $G$ (not involving any Steiner points). Additionally, we introduce a new natural "bridging`` property for cycles,  
which generalizes a known property of cycles in chordal graphs, and show that it also coarsely defines the tree-length.   
 \medskip

\noindent
{\bf Keywords:} Tree-decomposition; Tree-length; Quasi-isometry; Fat minor; Bramble; Helly family; Cycle property.  
 \medskip

\noindent
{\bf Mathematical Subject Classification:} 05C10; 05C62
\end{abstract}

\section{Introduction} 
We say that two graph parameters $\p$ and $\q$ are {\em coarsely equivalent} if 
there are two universal constants $\alpha>0$ and $\beta>0$ such that $\alpha\cdot \q(G)\le \p(G)\le \beta\cdot \q(G)$ for every graph $G$.  So, if one parameter is bounded by a constant, then the other is bounded by a constant, too. 
Coarse equivalency of two graph parameters is useful in at least two scenarios. First, if one parameter is easier to compute, then it provides an easily computable constant-factor approximate for possibly hard to compute other parameter. This is the case for parameters the {\em cluster-diameter of a layering partition} and the {\em tree-length} of a graph (that are subjects of this paper; formal definitions of these and other parameters can be found in Section \ref{sec:prelim-on-par}). It is known \cite{AbDr16,DDGY-spanners,Dorisb2007,tree-spanner-appr}  that those two parameters are within small constant factors from each other. %, i.e., they are coarsely equivalent. 
The cluster-diameter of a layering partition can easily be computed,  and a layering partition serves as a crucial tool in designing an efficient % linear-time 
3-approximation algorithm for computing the tree-length of a graph (and its tree-decomposition with minimum length), which is NP-hard to compute exactly (see Section \ref{sec:layer-partit} and Section \ref{sec:tl} for details). 
Secondly, since a constant bound on one parameter implies a constant bound on the other, one can choose out of two a most suitable (a right one) parameter when designing FPT (approximation) algorithms for some particular optimization problems on bounded parameter graphs. For example, an FPT algorithm for the so-called {\em metric dimension problem} on bounded tree-length graphs developed in \cite{BFGR2017}  or sparse spanner  results obtained for bounded tree-length graphs in \cite{DDGY-spanners} are  useful and hold also for graphs with bounded cluster-diameter of a layering partition.  Approximation algorithms (whose approximation error bounds depend on the cluster-diameter of a layering partition) for the connected $r$-domination problem and the connected $p$-center problem developed in \cite{par-conn-p-c} are useful also for graphs with bounded tree-length. 

Layering partition and its cluster-diameter were used also in obtaining a 6-approximation algorithm for the problem of optimal non-contractive embedding of an unweighted graph metric into a weighted tree metric. This was possible because the cluster-diameter of a layering partition and the minimum distortion of an embedding of a graph into a tree (the so-called {\em tree-distortion} parameter) are within small constant factors from each other (see \cite{AbDr16,ChepoiDNRV12,WG13-Dragan,tree-spanner-appr};   Section \ref{sec:embed-to-tree} provides some details). %, i.e., they are coarsely equivalent. 
%
The cluster-diameter of a layering partition and the tree distortion were the first two graph parameters shown to %proven to %that were shown to 
be coarsely equivalent with  the 
tree-length of a graph.  


This paper is inspired by recent insightful papers  \cite{BerSey2024,GeorPapa2023} and \cite{Diestel++}. They added several additional  graph parameters to the list of parameters that are coarsely equivalent to tree-length. Among other results, \cite{BerSey2024} showed that the tree-length of a graph $G$ is bounded if and only if there is an {\em $(L,C)$-quasi-isometry} (equivalently, an {\em $(1,C')$-quasi-isometry}) to a (weighted and with Steiner points) tree with $L,C$ ($C'$, respectively) bounded (see Section \ref{sec:embed-to-tree} for more details). There were results already known that characterize when a graph is quasi-isometric to a tree \cite{ChepoiDNRV12,Kerr,manning}. For example, a theorem of Manning for geodesic metric spaces (see \cite{manning}) implies that a graph is quasi-isometric to a tree if and only if its {\em bottleneck constant} is bounded. Authors of \cite{BerSey2024} give a graph theoretic proof for this (see Section \ref{sec:early-bn} for more details).  
One of the main results of \cite{BerSey2024} is a proof of Rose McCarty's  conjecture that the tree-length of a graph is small if and only if its {\em McCarty-width} is small (see Section \ref{sec:early-mcw} for more details). In a quest to find a cycle property that coarsely describes the tree-length, \cite{BerSey2024} introduced a new notion of bounded load geodesic cycles (see Section \ref{sec:glc} for the definition and some details). It was shown \cite{BerSey2024}  that the tree-length of a graph $G$ is bounded if and only if  
every geodesic loaded cycle of $G$ has bounded load. Recently, in \cite{GeorPapa2023}, the Manning's Theorem was  extended to include also a characterization via $K$-fat $K_3$-minors (see Section \ref{sec:early-bn} for a definition). It was shown \cite{GeorPapa2023} that the bottleneck constant of a graph $G$ is bounded by a constant if and only if $G$ has no $K$-fat $K_3$-minor for some constant $K>0$. 

%In this paper, w 
We give an extended overview of those existing results in Section \ref{sec:prelim-on-par}. In Section \ref{sec:proofs}, 
we incorporate the  cluster-diameter of a layering partition into the repertoire (the first graph parameter known to be coarsely equivalent to tree-length) and use it to simplify some proofs and, in some cases, get even better coarseness bounds than in \cite{BerSey2024,GeorPapa2023} (see Section \ref{sec:bnc}, Section \ref{sec:mcw}, and Section \ref{sec:adt}, Section \ref{sec:fat}).  Our proofs are  simpler and more direct. One of our results shows that the tree-length of a graph $G$ is bounded if and only if there is an {\em $(1,C)$-quasi-isometry}, with $C$ bounded, to  
an unweighted tree with the same vertex set as in $G$  
(a most restrictive quasi-isometry to a tree; see Section \ref{sec:adt}). As a byproduct, we also obtain several alternative proofs of McCarty's conjecture (see Theorem \ref{th:mcw-delta-rho}, Corollary   \ref{cor:mcw-adt}, and Theorem \ref{th:mf-tl-mcw}). 
%
More importantly, we   
add a number of new graph parameters that are coarsely equivalent to tree-length. Any result obtained for graphs with bounded tree-length automatically applies/extends to graphs with bounded such parameters. % that are coarsely equivalent to tree-length. 
Among other results, we show that the tree-length of a graph $G$ is bounded if and only if for every bramble ${\cal F}$ (or every Helly family of connected subgraphs ${\cal F}$, or every Helly family of paths ${\cal F}$)  of $G$,  there is a disk in $G$ with bounded radius that intercepts all members of ${\cal F}$ (see Section \ref{sec:br-Helly}). In Section \ref{sec:cbc}, we generalize a known {\em characteristic cycle property} of chordal graphs (graphs with tree-length equal 1) and introduce two new cycle related parameters. We show that both these parameters coarsely define the tree-length.  
We conclude the paper with a few open questions (Section \ref{sec:concl}). 

%{\color{red} 
%- to make people aware of earlier coarse equivalency results, simplify proofs of recent results from \cite{BerSey2024} and suggest a few new parameters that are coarsly equivalent to $\tl$}\\
%- all those parameters coarsely define tree-length \\ 
%- we lower-/upper-bound them with easily computable parameter $\Delta_s(G)$ \\
%- this work is inspired by a recent insightful paper, where several coarse equivalent parameters were established. Here, we incorporate... \\
%- our is simpler conceptually ????, {\color{red}better distance approximating trees} \\
%- We argue that Our approach is conceptually simpler ... \\
%- we put in prospective of results known before and further improve some bounds..\\
%- define coarse equivalency of parameters... \\  
%-- we say two graph parameters $p$ and $q$ are coarsely equivalent if there are linear function $f(\cdot)$ and $g(\cdot)$ such that $f(q)\le p\le g(q)$.  That is, if one parameter is bounded by a constant then the other is bounded too\\
%- here we incorporate the first known coarsely equivalent parameter into the repertoire and use  it to simplify some proofs and in some cases get even better coarseness bounds ... \\
%- Delta and rho are the first graph parameters known to be coarsely equivalent to tree-length/tree-breadth \\
%- it is very handy to work with the layering partition and its parameters...\\ 
%- Kerr result: for graph that was mentioned already in [our paper for outerplanar] \\
%- we assume for simplicity that $G$ is finite although many our non-algorithmic results and their proofs work for infinite graphs. \\ 
%- we give in some cases simpler proofs and in some cases with better bounds (check also my notes in Seymour's paper printed) \\ 

\medskip
\noindent
{\bf Basic notions and notations.} %\label{sec:notions}
All graphs $G$ in this paper are connected, unweighted, undirected, loopless and without multiple edges. % (unweighted simple graphs).
We assume also, for simplicity, that $G$ is finite although most of %many 
our non-algorithmic results and their proofs work for infinite graphs, too. 
For a (finite) graph $G=(V,E)$, we use $n$ and $|V|$ interchangeably to denote the number of vertices in $G$. Also, we use $m$ and $|E|$ to denote the number of edges. When we talk about two or more graphs, we may use $V(G)$ and $E(G)$ to indicate that these are the vertex and edge sets of a graph $G$. For a subset $S\subseteq V$, by $G[S]$ we  denote a subgraph of $G$ induced by the vertices of $S$.    

The {\em length of a path} $P(v,u):=(v=v_0,v_1,\dots,v_{\ell-1},v_{\ell}=u)$ from a vertex $v$ to a vertex $u$ is $\ell$, i.e., the number of edges in the path. The {\em distance} $d_G(u,v)$ between vertices $u$ and $v$ is the length of the shortest path connecting $u$ and $v$ in $G$. 
The distance between a vertex $v$ and a subset $S\subseteq V$ is defined as $d_G(v,S):=\min\{d_G(v,u): u\in S\}$. Similarly, let $d_G(S_1.S_2):=\min\{d_G(x,y): x\in S_1, y\in S_2\}$ for any two sets $S_1,S_2\subseteq V$. %$I(x,y)$ \\
The $k^{th}$ {\em power} $G^k$ of a graph $G$ is a graph that has the same set of vertices, but in which two distinct vertices are adjacent if and only if their distance in $G$ is at most $k$. 
The {\em disk} $D_r(s,G)$ of a graph $G$ centered at vertex $s \in V$ and with radius $r$ is the set of all vertices with distance no more than $r$ from $s$ (i.e., $D_r(s,G)=\{v\in V: d_G(v,s) \leq r \}$). We omit the graph name $G$ and write  $D_r(s)$ if the context is about only one graph. A cycle $C$ of a graph $G$ is called {\em geodesic} if $d_C(x,y)=d_G(x,y)$ for every $x,y$ in $C$. 

The \emph{diameter} of a subset $S\subseteq V$ of vertices of a graph $G$ is the largest distance  in $G$ between a pair of vertices of $S$, i.e., $\max_{u,v \in S}d_G(u,v)$. The \emph{inner diameter} of $S$  is the largest distance in $G[S]$ between a pair of vertices of $S$, i.e., $\max_{u,v \in S}d_{G[S]}(u,v)$.  The \emph{radius} of a subset $S\subseteq V$ of vertices of a graph $G$ is the minimum $r$ such that a vertex $v\in V$ exists with $S\subseteq D_r(v,G)$. The \emph{ inner radius} of a subset $S$ is the minimum $r$ such that a vertex $v\in S$ exists with $S\subseteq D_r(v,G[S])$.  

Let  $G=(V,E)$ be a graph and $X\subseteq V$ be a subset  of vertices of $G$. A disk $D_r(v)$ (a clique $C$) of $G$ is called a {\em balanced disk $($clique$)$ separator of $G$ with respect to $X$}, if every connected component of $G[V\setminus D_r(v)]$ (of $G[V\setminus C]$, respectively) has at most $|X|/2$ vertices from $X$. 



Definitions of graph parameters considered in this paper, %measuring metric tree-likeness of a graph, 
as well as notions and notation local to a section, are given in appropriate sections. We realize that there are too many parameters and abbreviations and this may cause some difficulties in following them.  We give in Appendix a glossary for all parameters and summarize all inequalities and relations between them.  


\section{Preliminaries on graph parameters} \label{sec:prelim-on-par}

We start with parameters central to our proofs and giving  the best to date approximation algorithm for computing the tree-length and the tree-breadth of a graph. 

\subsection{Layering partition, its cluster-diameter, cluster-radius, and a canonical tree} \label{sec:layer-partit}


Layering partition is a graph decomposition procedure introduced in~\cite{DBLP:journals/jal/BrandstadtCD99,DBLP:journals/ejc/ChepoiD00} and used in~\cite{BaInSi,DBLP:journals/jal/BrandstadtCD99,DBLP:journals/ejc/ChepoiD00,ChDrEsHaVaXi12,ChepoiDNRV12} %and~\cite{}
for embedding graph metrics into trees. 
%It provides a central tool in our investigation.

	\begin{figure}[htb]%[H]
 \footnotesize
		\centering
		\subfigure[][Layering of a graph $G$ with respect to $s$.]
		{
			\scalebox{0.24}[0.24]{\includegraphics{layering1.pdf}}
			\label{fig:layering}
		}
		\hspace{7ex}%
		\subfigure[][ Clusters of the layering partition $\mathcal{LP}(G,s)$.]
		{
			\scalebox{0.24}[0.24]{\includegraphics{layering2.pdf}}
			\label{fig:Layering-clusters}
		}
		
		\subfigure[][Layering tree $\Gamma(G,s)$.]
		{
			\scalebox{0.24}[0.24]{\includegraphics{layering4.pdf}}
			\label{fig:gamma}
		}
        \hspace{7ex}%
		\subfigure[][ Canonical tree $H$. ]
		{
			\scalebox{0.24}[0.24]{\includegraphics{layering3.pdf}}
			\label{fig:tree-H}
		}
		%\vspace{2ex}
		\caption{\small Layering partition and associated constructs (taken from \cite{AbDr16}).}
		\label{fig:layering-partition}%\vspace{-2ex}
	\end{figure}



A \emph{layering} of a graph $G=(V, E)$ with respect to a start vertex $s$ is the decomposition of $V$ into $q+1$ layers (spheres) $L^i=\{u\in V:d_G(s,u)=i\},i=0,1,\dots,q$ (here, $q:=\max\{d_G(s,v): v\in V\}$). A \emph{layering partition} $\mathcal{LP}(G,s)=\{L^i_1,\ldots,L^i_{p_i}:i=0,1,\dots,q\}$ of $G$ is a partition of each layer $L^i$ into clusters $L^i_1,\dots,L^i_{p_i}$ such that two vertices $u,v \in L^i$ belong to the same cluster $L^i_j$ if and only if they can be connected by a path outside the disk $D_{i-1}(s)$ of radius $i-1$ centered at $s$. Here $p_i$ is the number of clusters in layer $i$. See Fig. \ref{fig:layering-partition} for an illustration. A layering partition of a graph can be constructed in $O(n+m)$ time (see~\cite{DBLP:journals/ejc/ChepoiD00}).


A \emph{layering tree} $\Gamma(G,s)$ of a graph $G$ with respect to a layering partition $\mathcal{LP}(G,s)$  is the graph whose nodes are the clusters of $\mathcal{LP}(G,s)$ and where two nodes $C=L_j^i$ and $C'=L_{j'}^{i'}$ are adjacent in $\Gamma(G,s)$ if and only if there exist a vertex $u \in C$ and a vertex $v\in C'$ such that $uv \in E$. It was shown in~\cite{DBLP:journals/jal/BrandstadtCD99} that the graph $\Gamma(G,s)$ is always a tree and, given a start vertex $s$, it can be constructed in $O(n+m)$ time~\cite{DBLP:journals/ejc/ChepoiD00}. Note that, for a fixed start vertex $s\in V$, the layering partition $\mathcal{LP}(G,s)$ of $G$ and its tree $\Gamma(G,s)$ are unique.

The \emph{cluster-diameter $\Delta_s(G)$ of layering partition $\mathcal{LP}(G,s)$ with respect to vertex $s$} is the largest diameter of a cluster in $\mathcal{LP}(G,s)$, i.e., $\Delta_s(G)=\max_{C \in \mathcal{LP}(G,s)} \max_{u,v\in C}d_G(u,v)$. Denote by $\Delta(G)$ ($\widehat{\Delta}(G)$) the minimum (the maximum, respectively) cluster-diameter over all layering partitions of $G$, i.e. $\Delta(G)=\min_{s \in V}\Delta_s(G)$ and $\widehat{\Delta}(G)=\max_{s \in V}\Delta_s(G)$.

The \emph{cluster-radius $\rho_s(G)$ of layering partition $\mathcal{LP}(G,s)$ with respect to a vertex $s$} is the smallest number $r$ such that for any cluster $C \in \mathcal{LP}(G,s)$ there is a vertex $v \in V$ with $C \subseteq D_r(v)$. Denote by $\rho(G)$ ($\widehat{\rho}(G)$)  the minimum (the maximum, respectively) cluster-radius over all layering partitions of $G$, i.e., $\rho(G)=\min_{s \in V}\rho_s(G)$ and $\widehat{\rho}(G)=\max_{s \in V}\rho_s(G)$. 

%Clearly, in view of tree $\Gamma(G,s)$ of $G$, the smaller parameters $\Delta_s(G)$ and $\rho_s(G)$, the closer graph $G$ is to a tree metrically.

Finding the cluster-diameter $\Delta_s(G)$ and the cluster-radius $\rho_s(G)$ for a given layering partition $\mathcal{LP}(G,s)$ of a graph $G$ requires $O(nm)$ time%\footnote{The parameters $\Delta(G)$, $\widehat{\Delta}(G)$  and $\rho(G)$, $\widehat{\rho}(G)$ can also be computed in total $O(nm)$ time for any graph $G$.}
, although the construction of the layering partition $\mathcal{LP}(G,s)$ itself, for a given vertex $s$, takes only $O(n+m)$ time. Since the diameter of any set is at least its radius and at most twice its radius, we have the following inequality: $$\rho_s(G) \leq \Delta_s(G) \leq 2\rho_s(G).$$
%$\Delta_s=\max\{d_G(x,y):x, y \mbox{ belong to the same cluster of } \mathcal{LP(s)}\}$

It is not hard to show that, for any graph $G$ and any two of its vertices $s,q$, $\Delta_q(G)\leq 3 \Delta_s(G)$. Thus, the choice of the start vertex for constructing a layering partition of $G$   is not that important.

\begin{proposition} [\cite{slimness}] \label{prop:ClustDiamAtAnys}
	Let $s$ be an arbitrary vertex of $G$. For every vertex $q$ of $G$, $\Delta_q(G)\le 3\Delta_s(G)$. In particular,
	$\Delta(G)\le \widehat{\Delta}(G)\le 3 \Delta(G)$ for every graph $G$.
\end{proposition}

%----------------

%Most of the graph parameters discussed in this paper could be related to a special tree $H$ introduced in~\cite{ChepoiDNRV12} and produced from a layering partition of a graph $G$.

%\textbf{Canonical tree} $\mathbf{H}$: 
A tree $H=(V,F)$ of a graph $G=(V,E)$, called a {\em canonical tree of $G$}, is constructed from a layering partition $\mathcal{LP}(G,s)$ of $G$ by identifying for each cluster $C=L^i_j \in \mathcal{LP}(G,s)$ an arbitrary vertex $x_C \in L_{i-1}$  which has a neighbor in $C = L^i_j$ and by making $x_C$ adjacent in $H$ with all vertices $v\in C$ (see Fig. \ref{fig:tree-H} for an illustration). The tree $H$ closely reproduces the global structure of the layering tree $\Gamma(G,s)$. 
%Vertex $x_C$ is called the support vertex for cluster $C= L^i_j$. 
It was shown in~\cite{ChepoiDNRV12} that the tree $H$ for a graph $G$ can be constructed in $O(n+m)$ time. 

The following result~\cite{ChepoiDNRV12} relates the cluster-diameter of a layering partition of $G$ to  embeddability of graph $G$ to the tree $H$.
\begin{proposition} [\cite{ChepoiDNRV12}]
\label{lem:cluster-diam}
For every graph $G=(V,E)$ and any vertex $s$ of $G$, $$\forall x,y \in V, ~~d_H(x,y)-2 \leq d_G(x,y) \leq d_H(x,y)+\Delta_s(G).$$
\end{proposition}

Proposition \ref{lem:cluster-diam} shows that the additive distortion of embedding of a graph $G$ to tree $H$
is bounded by $\Delta_s(G)$, the largest diameter of a cluster in a layering partition of $G$.
% ------------------ do we need this ?
%Using Lemma~\ref{lem:cluster-diam} and the previous inequality, we have:
%\[ d_H(x,y)-2 \leq d_G(x,y) \leq d_H(x,y)+2\rho_s \].
This result indicates that the smaller the cluster-diameter $\Delta_s(G)$ (cluster-radius $\rho_s(G)$) of $G$, the closer graph $G$ is to a tree metrically. Note that trees have cluster-diameter and cluster-radius equal to $0$. Results similar to Proposition \ref{lem:cluster-diam} were first used in~\cite{DBLP:journals/jal/BrandstadtCD99} to embed a chordal graph to a tree with an additive distortion of most 2 and  
in~\cite{DBLP:journals/ejc/ChepoiD00} to embed a $k$-chordal graph to a tree with an additive distortion at most $k/2 +2$. In~\cite{ChepoiDNRV12}, Proposition \ref{lem:cluster-diam} was used  to obtain a 6-approximation algorithm for the problem of optimal non-contractive embedding of an unweighted graph metric into a weighted tree metric. For every {\em chordal graph} $G$ (a graph whose largest induced cycles have length 3),  $\Delta_s(G) \leq 3$ and $\rho_s(G)\leq 2$ hold~\cite{DBLP:journals/jal/BrandstadtCD99}. For every {\em $k$-chordal graph} $G$ (a graph whose largest induced cycles have length $k$), $\Delta_s(G) \leq k/2 +2$ holds~\cite{DBLP:journals/ejc/ChepoiD00}. For every graph $G$ embeddable non-contractively into a (weighted) tree with multiplication distortion $\alpha$, $\Delta_s(G) \leq 3\alpha$ holds~\cite{ChepoiDNRV12}. See Section \ref{sec:embed-to-tree} for more on this topic.

\subsection{Tree-length and tree-breadth} \label{sec:tl}
It is known that the class of chordal graphs can be characterized in terms of existence of so-called {\em clique-trees}.
Let ${\cal C}(G)$ denote the family of  maximal (by inclusion) cliques of a graph $G$.
A {\em clique-tree} ${\cal CT}(G)$ of $G$ has the maximal cliques of $G$ as its
nodes, and for every vertex $v$ of $G$, the maximal cliques containing $v$ form a subtree of ${\cal CT}(G)$.
%The existence of a clique tree characterizes chordal graphs:

\begin{proposition} %[Buneman \cite{Bunem1974}, Gavril \cite{Gavri1974}, Walter \cite{Walte1972}]
 [\cite{Bunem1974,Gavri1974,Walte1972}]\label{cliquetreechordalgr} 
A graph is chordal if and only if it has a clique-tree.
\end{proposition}

In their work on graph minors \cite{RobSey1986}, Robertson and Seymour introduced the notion of tree-decomposition which generalizes the notion of clique-tree.
A {\em tree-decomposition} of a graph $G$ is a tree $\cT(G)$  whose nodes, called {\em bags}, are subsets of $V(G)$ such that:

\begin{enumerate}
\item[(1)] $\bigcup_{B\in V(\cT(G))} B = V(G)$,

\item[(2)] for each edge $vw\in E(G)$, there is a bag $B\in V(\cT(G))$
such that $v,w\in B$, and

\item[(3)] for each $v\in V(G)$ the set of bags  $\{ B: B\in V(\cT(G)), v\in B\}$ forms a subtree  $\cT_v(G)$  of  $\cT(G)$.
\end{enumerate}

Tree-decompositions were used in defining several graph parameters.
The {\em tree-width} of a graph $G$ is defined as minimum of $\max_{B\in V(\cT(G))} |B| - 1$ over all tree-decompositions $\cT(G)$ of $G$ and is denoted by $\tw(G)$ \cite{RobSey1986}. The {\em length} of a tree-decomposition $\cT(G)$ of a graph $G$ is $\max_{B\in V(\cT(G))}\max_{u,v\in B}d_G(u,v)$, and the {\em tree-length} of $G$, denoted by $\tl(G)$, is the minimum of the length, over all tree-decompositions of $G$ \cite{Dorisb2007}. These two graph parameters generally are not related to each other. For instance, cliques (or, generally, all chordal graphs) have unbounded tree-width and tree-length 1, whereas cycles have tree-width 2 and unbounded tree-length. However, in some special cases they are coarsely equivalent. It is known that the tree-length $\tl(G)$ of any graph $G$ is at most $\lfloor\ell(G)/2\rfloor$ times its tree-width $\tw(G)$ (where $\ell(G)$ is the size of a largest geodesic cycle in $G$) \cite{CDN2016,Conn-tw}  and that for any graph $G$ that excludes an apex graph $H$ as a minor, $\tw(G)\le c_H\cdot\tl(G)$ for some constant $c_H$ only depending on $H$ \cite{CDN2016}. 
%
The tree-length of any graph $G$ can equivalently be defined in two other ways. As it is shown in \cite{b-length}, tree-length  
equals branch-length (see  \cite{b-length} for more details). Tree-length equals also short fill-in to a chordal graph, that is,  the smallest number $\ell\in N$ that permits an edge set $E'$ between vertices of $G$ such
that $G' = (V, E \cup E')$ is a chordal graph and for all $uv \in E'$, $d_G(u, v) \le\ell$ holds (see \cite{dagstuhl} and \cite{fid-param}). 
%Interestingly, the tree-length of a graph can be approximated in polynomial time within a
%constant factor \cite{DoGa2007} whereas such an approximation factor is unknown for the tree-width.
%
%For the purpose of this paper, we introduce yet another graph parameter based on the notion of tree-decomposition. It is very similar to the notion of tree-length but is more appropriate for our discussions, and moreover it will lead to a better constant in our approximation ratio presented in Section \ref{subsec:back} for the {\sc tree $t$-spanner} problem on general graphs.
%
The {\em  breadth}  of a tree-decomposition $\cT(G)$ of a graph $G$ is the minimum integer $k$ such that for every $B\in V(\cT(G))$ there is a vertex $v_B\in V(G)$ with $B\subseteq D_k(v_B,G)$ (i.e., each bag $B$ has radius at most $k$ in $G$). Note that vertex $v_B$ does not need to belong to $B$. The {\em tree-breadth}  of $G$, denoted by $\tb(G)$, is the minimum of the breadth, over all tree-decompositions of $G$ \cite{tree-spanner-appr}. 
Note also that in \cite{acyclic-clustering} independently a notion similar to tree-length (and tree-breadth) was introduced for purposes of compact and efficient routing in certain graph classes. It was called $(R,D)$-acyclic clustering. An $(R,2R)$-acyclic clustering is exactly a tree-decomposition with breadth $R$ and  a $(D,D)$-acyclic clustering is exactly a tree-decomposition with length $D$.  

Evidently, for any graph $G$, $1\leq \tb(G)\leq \tl(G)\leq 2 \tb(G)$ holds. Hence, if one parameter is bounded by a constant for a graph $G$ then the other parameter is bounded for $G$ as well. We say that a family of graphs ${\cal G}$ is {\em of bounded tree-length} (equivalently, {\em of bounded tree-breadth}), if there is a constant $c$ such that for each graph  $G$ from ${\cal G}$,  $\tl(G)\leq c$. 
%
It is known that checking whether a graph $G$ satisfies 
$\tl(G)\le \lambda$ or $\tb(G)\le r$ is NP-complete for each $\lambda >1$ \cite{NPc-tl} and each $r\ge 1$ \cite{NPc-tb}. Furthermore,  unless $P= NP$, there is no polynomial time algorithm to calculate a tree-decomposition, for a given graph $G$, of length smaller than $\frac{3}{2}\tl(G)$ \cite{NPc-tl}; and for any $\epsilon>0$, it is NP-hard to approximate the tree-breadth of a given graph by a factor of $2 -\epsilon$ \cite{NPc-tb}. 

Notice that in the definition of the length of a tree-decomposition, the distance between vertices of a bag is measured in the entire graph $G$. If the distance between any vertices $x,y$ from a bag $B$ is measured in $G[B]$, then one gets the notion of the inner length of a tree-decomposition \cite{BerSey2024}. The {\em inner length} of a tree-decomposition $\cT(G)$ of a graph $G$ is $\max_{B\in V(\cT(G))}\max_{u,v\in B}d_{G[B]}(u,v)$, and the {\em inner tree-length} of $G$, denoted by $\itl(G)$, is the minimum of the inner length, over all tree-decompositions of $G$.  Similarly, the {\em  inner breadth}  of a tree-decomposition $\cT(G)$ of a graph $G$ is the minimum integer $k$ such that for every $B\in V(\cT(G))$ there is a vertex $v_B\in B$ with $d_{G[B]}(u,v_B)\le k$ for all $u\in B$ (i.e., each subgraph $G[B]$ has radius at most $k$). The {\em inner tree-breadth}  of $G$, denoted by $\itb(G)$, is the minimum of the inner breadth, over all tree-decompositions of $G$ \cite{Diestel++}.  

Interestingly, the inner tree-length (inner tree-breadth) of $G$ is at most twice the tree-length (tree-breadth, respectively) of $G.$ 

\begin{proposition} [\cite{Diestel++,BerSey2024}] \label{prop:inner}
	For every graph $G$,
	$\tl(G) \leq \itl(G)\le 2\cdot\tl(G)$ and $\tb(G) \leq \itb(G)\le 2\cdot\tb(G).$ 
\end{proposition}

Since $\itl(G)$ and $\itb(G)$ are not that far from $\tl(G)$ and $\tb(G)$, in what follows, we will work only with $\tl(G)$ and $\tb(G)$. 

The following proposition establishes a relationship between the tree-length and the cluster-diameter of a layering partition of a graph.
\begin{proposition} [\cite{Dorisb2007}] \label{prop:dorisb}
	For every graph $G$ and any vertex $s$,
	$\frac{\Delta_s(G)}{3} \leq \tl(G) \leq \Delta_s(G)+1.$ In particular, $\frac{\widehat{\Delta}(G)}{3} \leq \tl(G) \leq \Delta(G)+1$ for every graph $G$.
\end{proposition}

Thus, the cluster-diameter $\Delta_s(G)$ of a layering partition provides easily computable bounds for the hard to compute parameter $\tl(G)$.

Similar inequalities are known for $\rho_s(G)$ and $\tb(G)$. 
\begin{proposition} [\cite{AbDr16,DDGY-spanners,tree-spanner-appr}] \label{prop:Muad-Feodor}
	For every graph $G$ and any vertex $s$,
	$\frac{\rho_s(G)}{3} \leq \tb(G) \leq \rho_s(G)+1$ and $\rho_s(G)\le 2\cdot\tl(G)$.
\end{proposition}

For a given graph  $G$ and its arbitrary vertex $s$, the layering tree $\Gamma(G,s)$, obtained in linear time from $G$ (see Section \ref{sec:layer-partit}), is almost a tree-decomposition of $G$. It only violates the condition (2) of a tree-decomposition. This tree  $\Gamma(G,s)$ can be transformed into a tree-decomposition by expanding its clusters as follows. For a cluster $C$, add all vertices from the parent of $C$ in  $\Gamma(G,s)$ which are adjacent to a vertex in $C$. That is, for each cluster $C \subseteq L^i$, create a bag $B_C = C\cup (N_G(C) \cap L^{i-1})$, where $N_G(C)$ denotes all vertices of $G$ that are adjacent to vertices of $C$. As it was shown in \cite{AbDr16,Dorisb2007,tree-spanner-appr}, the obtained tree-decomposition has length at most $3\cdot\tl(G)$\footnote{In \cite{Dorisb2007}, it was claimed to have the length at most $3\cdot\tl(G)+1$, but as mentioned in \cite{AbDr16}, its actual length is at most $3\cdot\tl(G)$.}  and breadth at most $3\cdot\tb(G)$. Hence, 3-approximations of $\tl(G)$ and of $\tb(G)$ and a corresponding tree-decomposition of length at most $3\cdot\tl(G)$ and of breadth at most $3\cdot\tb(G)$ can be computed  in linear time. This is the best to date approximation algorithm for computing the tree-length and the tree-breadth of a graph. 


\subsection{Embedding a graph into a tree}\label{sec:embed-to-tree}

Several different types of embeddings of (unweighted) graphs into trees were considered in literature (see \cite{AbDr16,Farach,BaDeHaSiZa08,BaInSi,BeRa22,BerSey2024,DBLP:journals/jal/BrandstadtCD99,BrDrchapter,CaiDerek,DBLP:journals/ejc/ChepoiD00,ChDrEsHaVa08,ChepoiDNRV12,ChepoiFichet,WG13-Dragan,tree-spanner-appr,ApprTree-DRYan,EmekPeleg,Kerr}). We will elaborate only on those results that are relevant to our discussion. In \cite{{BaDeHaSiZa08,BaInSi,ChepoiDNRV12}}, unweighted graphs embeddable non-contractively into a (weighted) tree with multiplicative distortion $\alpha$ are considered. For each such graph $G=(V,E)$ there is a  (weighted) tree $T=(V',E')$ with $V'\subseteq V$ such that $d_G(x,y)\le d_T(x,y)\le \alpha\cdot d_G(x,y)$ for every $x,y\in V$. A {\em tree distortion} $\td(G)$ (see \cite{AbDr16,WG13-Dragan}) of a graph $G$ is the minimum $\alpha$ such that $G$ can be embedded non-contractively into a (weighted) tree with multiplicative distortion $\alpha$. In \cite{ChepoiDNRV12,tree-spanner-appr} (see also \cite{AbDr16,WG13-Dragan}), it is shown that for every graph $G$, tree-distortion parameter $\td(G)$ is coarsely equivalent to $\Delta_s(G)$ (for any $s\in V$) and to $\tl(G)$. 

\begin{proposition} [\cite{AbDr16,ChepoiDNRV12,tree-spanner-appr}] \label{prop:td-tl}
	For every graph $G$ and any vertex $s$,
	$$\frac{\Delta_s(G)}{3} \leq \tl(G) \leq \td(G) \le 2\cdot \Delta_s(G)+2.$$ 
 In particular,
	$\frac{\widehat{\Delta}(G)}{3} \leq \tl(G) \leq \td(G) \le 2\cdot \Delta(G)+2$ for every graph $G$.
\end{proposition}
%{\color{red}/* check old computer for proof $\td\le \tl$.   */}

It is known that deciding whether $\td(G)\le \alpha$ is NP-complete, and even more, the hardness result of~\cite{Farach}
implies that it is NP-hard to $\gamma$-approximate $\td(G)$ for some small constant $\gamma$. 
As we have mentioned earlier, in~\cite{ChepoiDNRV12}, Proposition \ref{lem:cluster-diam} is used  to obtain a (best to date) 6-approximation algorithm for the problem of optimal non-contractive embedding of an unweighted graph metric into a weighted tree metric. 
In fact, one of the results of~\cite{ChepoiDNRV12} says that if for a graph $G=(V,E)$ there is a (weighted) tree $T=(V',E')$ with $V'\subseteq V$ such that $d_G(x,y)\le d_T(x,y)\le \alpha\cdot d_G(x,y)$ for every $x,y\in V$, then for any $s\in V$, the cluster-diameter $\Delta_s(G)$ of layering partition $\mathcal{LP}(G,s)$ of $G$ is at most $3\alpha$ and, hence, a canonical tree $H$ (see Section \ref{sec:layer-partit}) of $G$ satisfies 
$d_H(x,y)-2 \leq d_G(x,y) \leq d_H(x,y)+3\alpha$ for all $x,y \in V$.
It turns out that, for any unweighted graph $G$, it is possible to turn its non-contractive multiplicative low-distortion embedding into a weighted tree to an additive low-distortion embedding to an unweighted tree with the same vertex set as in $G$. Later such a phenomenon of turning multiplicative distortions to additive distortions were observed for other types of embedding into trees \cite{BeRa22,BerSey2024,Kerr} (see below). 

As it was noticed in~\cite{ChepoiDNRV12}, a more general result can be stated: if for a graph $G=(V,E)$ there is a (weighted) tree $T=(V',E')$ with $V'\subseteq V$ such that $d_T(x,y)\le \alpha\cdot d_G(x,y)+\beta$ and $d_G(x,y)\le \lambda\cdot d_T(x,y)+\delta$ for every $x,y\in V$, then for any $s\in V$, the cluster-diameter $\Delta_s(G)$ of layering partition $\mathcal{LP}(G,s)$ of $G$ is at most $3(\alpha(\lambda+\delta)+\beta)$ and, hence, a canonical tree $H$ of $G$ satisfies 
$d_H(x,y)-2 \leq d_G(x,y) \leq d_H(x,y)+ 3(\alpha(\lambda+\delta)+\beta)$ for all $x,y \in V$ (getting again only an additive distortion; compare with a result of Kerr below and Proposition \ref{prop:BerSey2024}). 

In \cite{CaiDerek,tree-spanner-appr,EmekPeleg}, unweighted graphs embeddable with low distortion to their spanning trees are considered. A spanning tree $T$ of a graph $G$ is called a {\em tree $t$-spanner} of $G$ if $d_T(x,y)\le t\cdot d_G(x,y)$ holds for every $x,y\in V$.  In \cite{tree-spanner-appr}, it is shown that any graph admitting a tree $t$-spanner has tree-breadth at most $\lceil t/2\rceil$. Furthermore, every graph $G$ with $\tb(G)\le \delta$ admits a tree $O(\delta\log n)$-spanner\footnote{Note that the $\log n$ factor here cannot be dropped. There are graphs with tree-length (and, hence, tree-breadth) one (e.g., chordal graphs) which have tree $t$-spanner only for $t=\Omega(\log n)$~\cite{BerSey2024,tree-spanner-appr,add-spanner}.} constructible in $O(m\log n)$ time. The latter provides an efficient $O(\log n)$-approximation algorithm for the problem of finding a tree $t$-spanner with minimum $t$ of a given graph $G$. Another efficient $O(\log n)$-approximation algorithm can be found in \cite{EmekPeleg}. Interestingly, as a recent paper \cite{BeRa22} demonstrates, for every graph $G$ admitting a tree $t$-spanner, one can efficiently construct a spanning tree $T$ such that $d_T(x,y)\le d_G(x,y)+O(t\log n)$ holds for every $x,y\in V$ (turning a multiplicative distortion into an additive distortion with an additional logarithmic factor). Note also that deciding whether a given graph $G$ admits a tree $t$-spanner  is an NP-complete problem even for $G$ being a chordal graph, i.e., a graph with tree-length (and tree-breadth) equal one, and for every fixed $t>3$ (see \cite{BrDrLeLe}), while it is polynomial time solvable for all graphs of bounded tree-width (see \cite{MakowskyRotics??}). 

A most general notion of embedding into trees is given by a ``quasi-isometry" from a graph to a tree. This is the concept from metric spaces, but we define it just for graphs. Let $G$ be a graph, $T$ be a tree (possibly weighted) and $\psi: V(G)\rightarrow  V(T)$ be a map. Let $L\ge 1$ and $C\ge 0$. We say that $\psi$ is an $(L,C)$-quasi-isometry if: 

\begin{enumerate}
    \item[(i)] for all $u,v\in V(G)$, $\frac{1}{L}d_G(u,v)-C\le d_T(\psi(u),\psi(v))\le L d_G(u,v)+C$; and  \\
    \item[(ii)] for every $y\in V(T)$ there is $v\in V(G)$ such that $d_T(\psi(v),y)\le C$.  
\end{enumerate}

Although $(L,C)$-quasi-isometry from $G$ to $T$ looks very general, a recent result of Kerr \cite{Kerr} states that for all $L,C$ there is a $C'$ such that if there is an $(L,C)$-quasi-isometry from $G$ to a tree, then there is a $(1,C')$-quasi-isometry from $G$ to a tree. Kerr's proof was for metric spaces. For graphs, this result was extended in \cite{BerSey2024} in  the following way. 

\begin{proposition} [\cite{BerSey2024}] \label{prop:BerSey2024}
For every graph $G$, the following three statements are equivalent:
\begin{enumerate}
    \item[(i)] the tree-length of $G$ is bounded;
    \item[(ii)] there is an $(L,C)$-quasi-isometry to a tree with $L,C$ bounded;
    \item[(iii)] there is an $(1,C')$-quasi-isometry to a tree with $C'$ bounded. 
\end{enumerate}
\end{proposition}

Because of equivalency of $(ii)$ and $(iii)$, \cite{BerSey2024} introduced a new graph parameter $\ad(G)$. The additive distortion $\ad(G)$ of a graph $G$ is the minimum $k$ such that there is a (weighted)  tree $T$ and a map $\psi: V(G)\rightarrow  V(T)$ such that $$|d_G(u,v)-d_T(\psi(u),\psi(v))|\le k$$ holds for every $u,v\in V(G)$. It was shown \cite[Theorem 4.1, Theorem 4.2]{BerSey2024} that the following inequalities between $\tl(G)$ and $\ad(G)$  hold. 

\begin{proposition} [\cite{BerSey2024}] \label{prop:ad-tl-Seymour}
	For every graph $G$, $\frac{\tl(G)-2}{2}\leq \ad(G)\leq 6\cdot \tl(G).$
\end{proposition}


Notice that in the definition of $\ad(G)$, $V(G)$ is not necessarily equal to $V(T)$, and $T$ can have weights. Motivated by the ability of a canonical tree $H$ obtained from a layering partition to additively approximate graph distances (see Section  \ref{sec:layer-partit}), here, we restrict further the parameter $\ad(G)$ and recall a notion of {\em distance $k$-approximating trees}, introduced in \cite{DBLP:journals/jal/BrandstadtCD99,DBLP:journals/ejc/ChepoiD00} and further investigate in \cite{AbDr16,ChDrEsHaVaXi12,ChDrEsHaVa08,WG13-Dragan,ApprTree-DRYan}.  A (unweighted) tree $T=(V,E')$ is a {\em distance $k$-approximating tree} of a graph $G=(V,E)$ if $$|d_G(u,v)-d_T(u,v)|\le k$$ holds for every $u,v\in V$. Denote by $\adt(G)$ the minimum $k$ such that $G$ has a distance $k$-approximating tree.  
It is known that every chordal graph has a distance 2-approximating tree~\cite{DBLP:journals/jal/BrandstadtCD99}, every $k$-chordal graph has a distance $(k/2 +2)$-approximating tree~\cite{DBLP:journals/ejc/ChepoiD00}, every graph with tree-length $\lambda$ has  a distance $3\lambda$-approximating tree \cite{AbDr16,WG13-Dragan}, and every $\delta$-hyperbolic graph has a  distance $O(\delta\log n)$-approximating tree \cite{ChDrEsHaVa08}. See also \cite{ApprTree-DRYan} for some hardness results. %{\color{red}/*  ??? complexity  */}

In Section \ref{sec:proofs} (see Theorem \ref{th:adt-tl-delta} and Corollary \ref{cor:ineq-tl-adt}), we will show 
$$\adt(G)\leq \Delta(G)\le \Delta_s(G)\le\widehat{\Delta}(G)\leq 3\cdot \tl(G)\le 6\cdot \adt(G)+3,$$ 
$$\frac{\tl(G)-1}{2}\leq \adt(G)\leq 3\cdot \tl(G).$$
Our embedding is a most restrictive one as it requires a low additive distortion embedding to an unweighted tree with the same vertex set as in $G$. 
%{\color{red}/* ours is a more restricted variant.   */}

\subsection{Bottleneck constant} \label{sec:early-bn}
There were results already known that characterize when a graph is quasi-isometric
to a tree. The {\em bottleneck constant} of a graph $G$ is the least integer $r$ such that if $P(u,v)$ is a shortest path of $G$ between $u, v$, of even length and with middle vertex $w$, then every path between $u, v$ contains a vertex that has distance at most $r$ from $w$. A theorem of Manning for geodesic metric spaces implies the following.

\begin{proposition}[\cite{manning}] \label{th:Manning} 
For all $L\ge 1$ and $C\ge 0$, there exists $r$ such that, for all graphs $G$,
if there is an $(L,C)$-quasi-isometry from $G$ to a tree, then $G$ has bottleneck constant at most $r$. Conversely, for all $r$ there exist $L\ge 1$ and $C\ge 0$ such that, for all graphs $G$, if $G$ has bottleneck constant at most $r$, then there is an $(L,C)$-quasi-isometry from $G$ to a tree. 
\end{proposition} 
A similar result with an additional characterization through so-called {\em fat $K_3$-minors} is given also in \cite[Theorem 3.1]{GeorPapa2023} (see below). 

From Proposition \ref{prop:BerSey2024}, it already follows that if the bottleneck constant  of a graph $G$ is bounded then the tree-length of $G$ is also bounded. A new graph-theoretic proof from \cite{BerSey2024} has more explicit control over the bounds. Let $\bn(G)$ denote the bottleneck constant  of $G$. 

\begin{proposition}[\cite{BerSey2024}] \label{pr:Seymour--Manning} 
For every graph $G$,   $\frac{2}{3}\bn(G)\leq \tl(G)\leq 4\cdot \bn(G)+3$ and  
$\ad(G)\le 24\cdot \bn(G)+18$. 	
\end{proposition} 

In this paper, we give simpler (more direct) proofs for those bounds by employing a layering partition and its cluster-diameter.  See Lemma  \ref{lm:BNC_ClusterDiam}, Lemma  \ref{lm:ClusterDiam_bnc}, 
Theorem \ref{th:bnc-delta-tl} and 
Corollary \ref{cor:ineq-tl-bnc}. We get, in fact, $\adt(G)\le 4\cdot \bn(G)+2$ (see Theorem \ref{th:adt-tl-delta}). 	
\medskip

Recently, in \cite{GeorPapa2023}, the Manning's Theorem was  extended to include also a characterization via $K$-fat $K_3$-minors. Although a $K$-fat $H$-minor can be defined with any graph $H$ \cite{GeorPapa2023}, here we give a definition only for $H=K_3$ as we work in this paper only with this minor.  % in the following way. 
It is said that a graph $G$ has a {\em $K$-fat $K_3$-minor} ($K>0$) if there are three connected subgraphs $H_1$, $H_2$, $H_3$ and three simple paths $P_{1,2}$, $P_{2,3}$, $P_{1,3}$ in $G$ such that for each $i,j\in \{1,2,3\}$ ($i\neq j$), $P_{i,j}$ has one end in $H_i$ and the other end in $H_j$ and  $|P_{i,j}\cap V(H_i)|=|P_{i,j}\cap V(H_j)|=1$, and
\begin{itemize}
%    \item{} [conditions on being a $K_3$-minor]:  
%    \item[~~~~~~] 
    
    \item{} [conditions for being $K$-fat:]  
%   \item[~~~~~~] 
    $d_G(V(H_i),V(H_j))\ge K$,  $d_G(P_{i,j},V(H_k))\ge K$ ($k\in \{1,2,3\}, k\neq i, j$) and the distance between any two paths $P_{1,2}$, $P_{2,3}$, $P_{1,3}$ is at least $K$.
\end{itemize}

%$d_G(V(H_i),V(H_j))\ge K$  and  $d_G(P_{1,2} $ 

\begin{proposition} [\cite{GeorPapa2023}] \label{prop:Papaaoglu} The following are equivalent for every graph $G$: 
\begin{itemize}
    \item[(i)] The bottleneck constant $\bn(G)$ of $G$ is bounded by a constant; 
  \item[(ii)]    $G$ has no $K$-fat $K_3$-minor for some constant $K>0$; and
 \item[(iii)] $G$ is $(1,C)$-quasi-isometric to a tree for some constant $C$. 
\end{itemize}
\end{proposition} 
They show in \cite{GeorPapa2023} that $K$ can be chosen to be greater than $2\cdot\bn(G)+1$ in their proof of $(i)\Rightarrow (ii)$ and $C$ can be chosen to be at most $14K$ in their proof of $(ii)\Rightarrow (iii)$. 

In this paper, we give an alternative simple proof of analog of Proposition \ref{prop:Papaaoglu} which also improves the constant $C$ in $(iii)$. %simpler (more 
%direct) proofs for those bounds by employing a layering partition %and  its cluster-diameter.   

\subsection{McCarty-width}\label{sec:early-mcw}
A graph $G$ has {\em McCarty-width} $r$ if $r\ge 0$ is minimum such that the following holds: for every three vertices $u, v, w$ of $G$, there is a vertex $x$ such that no connected component of $G[V\setminus D_r(x)]$  contains two of $u, v, w$. Let $\mc(G)$ denote the McCarty-width of $G$. Rose McCarty suggested that $\tl(G)$ is small if and only if $\mc(G)$ is small. This was proved in \cite{BerSey2024}.

\begin{proposition} [\cite{BerSey2024}] \label{prop:sey-mcw} For every graph $G$, $\frac{\tl(G)- 3}{6} \le \mc(G) \le \tl(G).$
\end{proposition} 

We also consider in this paper a more general parameter related to a small radius balanced disk separator. Balanced disk separators proved to be very useful in designing efficient approximation algorithms for the problem of finding a tree $t$-spanner with minimum $t$ of a given graph $G$ \cite{tree-spanner-appr}  and in constructing collective additive tree spanners (see, e.g.,  \cite{CTS1,CTS2,CTS3}) as well as compact distance and routing labeling schemes for variety of graph classes (see, e.g., \cite{CTS2,CTS3,GKKPP2001,KKP2000}).  Experiments performed in \cite{MSstudent} show that many real-life networks have small radius balanced disk separators.

Let  $G=(V,E)$ be a graph and $X\subseteq V$ be a subset  of vertices of $G$. Recall that a disk $D_r(v)$ of $G$ is called a {\em balanced disk separator of $G$ with respect to $X$}, if every connected component of $G[V\setminus D_r(v)]$ has at most $|X|/2$ vertices from $X$. 
%
Let {\em McCarty-width of order $k$} ($k\ge 3$) of $G$ (denoted by $\mc_k(G)$) be the minimum $r\ge 0$ such that for every subset $X\subseteq V$ with $|X|=k$, there is a vertex $v$ in $G$ such that $D_r(v)$ is a balanced disk separator  of $G$ with respect to $X$.  Clearly, $\mc(G)=\mc_3(G)$. 

In this paper, we show that for every graph $G$, every vertex $s$ of $G$, and every integer $k\ge 3$, $\Delta_s(G)\leq 6\cdot \mc(G)$ (see Lemma  \ref{lm:ClusterDiam_mcw}), $\mc_k(G)\le \rho_s(G)\le \Delta_s(G)$ (see Lemma \ref{lm:Clusterrad_BDS}) and $\mc_k(G)\le \tb(G)$ (see Lemma \ref{lm:tb_mcw_k}).  Furthermore, for any subset $X\subseteq V$ of vertices of $G$, a balanced disk separator $D_r(u)$ with $r\le \Delta_s(G)$ can be found in linear time and a balanced disk separator $D_r(u)$ with minimum $r$ (hence, with $r\le \tb(G)$) can be found in $O(nm)$ time. Consequently, we slightly improve the bounds in Proposition  \ref{prop:sey-mcw} by getting   $\frac{\tl(G)- 1}{6} \le \mc(G) \le \tb(G).$ Our proofs are  simpler and more direct and again employ a layering partition and its cluster-radius. 
 
\subsection{Geodesic loaded cycles} \label{sec:glc}
In a quest to find a property of cycles that guarantees a bound on the tree-length (i.e., a cycle property that coarsely describes the tree-length), Berger and Seymour \cite{BerSey2024} introduced a new notion of bounded load geodesic cycles. 

Let $C$ be a cycle of $G$ and let $F\subseteq E(C)$. The pair $(C, F )$ is called a {\em loaded cycle} of $G$, and $|F|$ is called its {\em load}. If $u, v \in V(C)$ are
distinct, let $d_{C,F} (u, v)$ denote the smaller of $|E(P )\cap F |$, $|E(Q) \cap F|$ where $P, Q$ are the two paths of
$C$ between $u, v$. A loaded cycle $(C, F )$ is geodesic
in $G$ if $d_G(u, v) \ge d_{C,F} (u, v)$ for all $u, v \in V(C)$. If $G$ admits a tree-decomposition in which all bags
have bounded diameter, then every geodesic loaded cycle has bounded load. The main theorem of \cite{BerSey2024} 
says that if every geodesic loaded cycle has bounded load, then $G$ admits a tree-decomposition in
which all bags have bounded diameter. Let  $\glc(G)$ be the maximum load over all geodesic loaded cycles in $G$. 

\begin{proposition} [\cite{BerSey2024}] \label{prop:sey-lgc} For every graph $G$, $\tl(G)- 1 \le \glc(G) \le 3\cdot\tl(G)$.
\end{proposition} 


The notion of bounded load geodesic cycles and Proposition   \ref{prop:sey-lgc} are central in the proofs of results of \cite{BerSey2024} mentioned in Section \ref{sec:embed-to-tree}, Section \ref{sec:early-bn} and Section \ref{sec:early-mcw}. Using a layering partition and its cluster-diameter, we prove those results in a more intuitive way.  Furthermore, in Section \ref{sec:cbc}, we introduce a new natural "bridging`` property for cycles which generalizes a known property of cycles in chordal graphs and show that it also coarsely describes the tree-length. 

%- through geodesic loaded cycles; {\color{red}a less intuitive parameter} \\


%\subsection{Complexities}



%\section{Graphs with bounded tree-length: proofs} 
\section{New proofs for coarse equivalency with tree-length} %of those parameters} 
\label{sec:proofs}
\subsection{Bottleneck constant of a graph}\label{sec:bnc}

Let us define a bottleneck property in its fullness. The {\em overall bottleneck constant}, denoted by $\BNC(G)$, of a graph $G$ is the least integer $r$ such that if $P(u,v)$ is a shortest path of $G$ between $u, v$ and $w$ is a vertex of $P(u,v)$, then every path between $u, v$ contains a vertex that has distance at most $r$ from $w$. It is easy to show that for every graph $G$, $\bn(G)=\BNC(G)$ (see Corollary  \ref{cor:ineq} below). We prove first the following lemma. 

\begin{lemma} \label{lm:BNC_ClusterDiam}
	For every graph $G$,  $\BNC(G)\leq\frac{\widehat{\Delta}(G)}{2}\leq\frac{3 }{2}\tl(G)$.
\end{lemma}

\begin{proof}
	Let $x,y$ be arbitrary vertices of $G$. Consider an arbitrary shortest path $P(x,y)$ connecting $x$ and $y$ and an arbitrary vertex $c$ of $P(x,y)$. 
 %Let $\mathcal{LP}(G,c)$ be a layering partition of $G$ starting at vertex $c$. 
 Let also $Q$ be an arbitrary path between $x$ and $y$ and $\ell \geq 0$ be the maximum integer such that $D_{\ell}(c)$ does not intersect $Q$. 
 
 Consider two vertices $a$ and $b$ in $P(x,y)$ such that $d_G(c,a)=d_G(c,b)=\ell$, $a$ is between $c$ and $x$ and $b$ is between $c$ and $y$. Let also $a'$ and $b'$ be vertices in $P(x,y)$ such that $d_G(c,a')=d_G(c,b')=\ell+1$, $a$ is adjacent to $a'$ and $b$ is adjacent to $b'$ (see Fig. \ref{fig:one}(a) for an illustration). 
 Since $D_{\ell+1}(c)$ intersects $Q$ but $D_{\ell}(c)$ does not, vertices $a',b'$ are in the same cluster of layering partition $\mathcal{LP}(G,c)$ of $G$ starting at vertex $c$ (they are connected outside the disk $D_\ell(c)$ by $Q$ and parts of $P(x,y)$ between $x$ and $a'$ and between $y$ and $b'$). Hence, $d_G(a',b')\le \Delta_c(G)\le \widehat{\Delta}(G)$. On the other hand,
	$d_G(a',b')=1+d_G(a,b)+1=2\ell+2$, i.e., $d_G(c,Q)=\ell+1= \frac{d_G(a',b')}{2}\le \frac{\widehat{\Delta}(G)}{2}$. 
 
 Thus, $\BNC(G)\leq\frac{\widehat{\Delta}(G)}{2}$ must hold. Applying also Proposition \ref{prop:dorisb}, we get $\BNC(G)\leq\frac{\widehat{\Delta}(G)}{2}\leq\frac{3}{2}\tl(G)$. \qed
\end{proof}

Combining Proposition \ref{prop:ClustDiamAtAnys} with Lemma \ref{lm:BNC_ClusterDiam}, one gets the following corollary.

\begin{corollary} \label{cor:ineq}
	For every graph $G$ and every vertex $s$ of $G$,   $\bn(G)= \BNC(G)\leq \frac{3}{2}\Delta_s(G)$. 	
 In particular, $\bn(G)= \BNC(G)\leq \frac{3}{2}\Delta(G)$ for every graph $G$.
\end{corollary} 
\begin{proof} By Proposition \ref{prop:ClustDiamAtAnys} and Lemma \ref{lm:BNC_ClusterDiam}, it remains only to show $\bn(G)= \BNC(G)$. Clearly, by definitions, $\bn(G)\le\BNC(G)$. To show the equality, %at $\BNC(G)\le\bn(G)$ holds, 
let $\BNC(G)=r$. Then, for some vertices $x,y$ in $G$, some shortest path $P(x,y)$ connecting $x$ and $y$, and some vertex $c\in P(x,y)$,  there must exist a path $Q$ in $G[V\setminus D_{r-1}(c)]$ connecting $x$ and $y$. Necessarily, $d_G(x,c), d_G(y,c)\ge r$. Consider a vertex $x'$  on a subpath of $P(x,y)$ between $x$ and $c$ and a vertex $y'$  on a subpath of $P(x,y)$ between $y$ and $c$ with $d_G(x',c)=d_G(y',c)=r$. We have that vertices $x'$ and $y'$ with $d_G(x',y')=2r$ and $d_G(x',c)=d_G(c,y')=r$ are connected in $G[V\setminus D_{r-1}(c)]$ by  $P(x,x')\cup Q\cup P(y,y')$, where $P(x,x')$ and $P(y,y')$ are subpaths of $P(x,y)$ between corresponding vertices.  This shows that $\bn(G)>r-1$, implying $\bn(G)=r=\BNC(G)$. 
\qed  
\end{proof}

  \begin{figure}[htb]%[tbh] %
    \begin{center} %\vspace*{-1mm}
      \begin{minipage}[b]{16cm}%5
        \begin{center} %\hspace*{10mm}
          \vspace*{-36mm}
          \includegraphics[height=16cm]{Fig-1.pdf}
        \end{center} \vspace*{-49mm}
        \caption{\label{fig:one} Illustrations to the proofs of Lemma \ref{lm:BNC_ClusterDiam} and Lemma \ref{lm:ClusterDiam_bnc}.} %
      \end{minipage}
    \end{center}
   \vspace*{-5mm}
  \end{figure}
% \medskip


Now we upperbound $\Delta_s(G)$ by a linear function of $\bn(G)$.

\begin{lemma} \label{lm:ClusterDiam_bnc}
	For every graph $G$  and every vertex $s$ of $G$, $\Delta_s(G)\leq 4\cdot \bn(G)+2$. In particular, $\widehat{\Delta}(G)\leq 4\
 \bn(G)+2$ for every graph $G$.
\end{lemma}

\begin{proof}
Let $s$ be an arbitrary vertex of $G$ and $\mathcal{LP}(G,s)$ be the layering partition of $G$ starting at $s$. Consider vertices $x$ and $y$ from a cluster of $\mathcal{LP}(G,s)$ with $d_G(x,y)=\Delta_s(G)$, and let $k:=d_G(s,x)=d_G(s,y)$ and $r:=\bn(G)$. Choose also a path $Q$ connecting $x$ and $y$ outside the disk $D_{k-1}(s)$.

Consider arbitrary  shortest paths $P(s,x)$ and $P(s,y)$  of $G$ connecting $s$ with $x$ and $y$, respectively. Let $s'$ be a vertex from  $P(s,x)\cap P(s,y)$ furthest from $s$. We may assume that $k':=d_G(x,s')=d_G(y,s')=k-d_G(s,s')$ is greater than $2r+1$ since, otherwise,  $d_G(x,y)\le d_G(x,s')+d_G(s',y)=2k'\leq 4r+2$, and we are done. 

Pick now a vertex $u$ in $P(s,x)$ such that  
$d_G(x,u)=2r+2\leq d_G(x,s')$. See Fig. \ref{fig:one}(b) for an illustration.  
%and a shortest path $P(s',y)$ connecting $s'$ with $y$.  
The middle vertex $v$ of a subpath $P(u,x)$ of $P(s,x)$ must see within $r$ every path connecting $u$ with $x$. In particular, $d_G(v,Q')\leq r$ for a path $Q'$ formed  by concatenating $Q$ with $P(y,s')\cup P(s',u)$, where $P(y,s')$ is a subpath of $P(y,s)$ between $y$ and $s'$ and $P(s',u)$ is a subpath of $P(s,x)$ between $s'$ and 
$u$. As $d_G(v,Q)=d_G(v,x)=r+1$ and $d_G(v,P(s',u))=d_G(v,u)=r+1$, $v$ cannot see within $r$ any vertex of $Q\cup P(s',u)$. Hence, there must exist a 
vertex $y'$ in $P(y,s')$ such that $d_G(v,y')\le r$. 

Since $d_G(y,y')=k-d_G(s,y')$ and $k=d_G(x,s)\le d_G(s,y')+d_G(y',v)+d_G(v,x)$, necessarily, $d_G(y,y')\le d(y',v)+d_G(v,x)\le r+r+1=2r+1$. Consequently,  $d_G(x,y)\le d_G(x,v)+d_G(v,y')+d_G(y',y)\leq r+1+r+2r+1=4r+2$. That is,  $\Delta_s(G)\leq 4\cdot \bn(G)+2$. \qed
\end{proof}

Combining Lemma  \ref{lm:BNC_ClusterDiam}, Lemma  \ref{lm:ClusterDiam_bnc}, Corollary \ref{cor:ineq} and Proposition \ref{prop:dorisb}, we get the following result. 

\begin{theorem} \label{th:bnc-delta-tl}
	For every graph $G$  and every vertex $s$ of $G$, the following inequalities hold: 
  $$\frac{\Delta_s(G)-2}{4}\leq \bn(G)= \BNC(G)\leq \frac{3}{2}\Delta_s(G),$$ 
  %
   $$\frac{\tl(G)-3}{4}\leq\frac{\widehat{\Delta}(G)-2}{4}\leq \bn(G)=\BNC(G)\leq\frac{\widehat{\Delta}(G)}{2}\leq\frac{3 }{2}\tl(G).$$
 % $\Delta_s(G)/3 \leq \tl(G) \leq \Delta_s(G)+1.$\\
\end{theorem} 

\begin{corollary} \label{cor:ineq-tl-bnc}
	For every graph $G$,   $\frac{2}{3}\bn(G)\leq \tl(G)\leq 4\cdot \bn(G)+3$. 	
\end{corollary}

Same inequalities as in Corollary \ref{cor:ineq-tl-bnc} can be derived from  \cite[Lemma 2.4, Lemma 4.4, Theorem 4.5]{BerSey2024}. 
%{\color{red}/* same in my and  in {BerSey2024} */}

Similar to parameters $\Delta_s(G)$ and $\rho_s(G)$, the bottleneck constant $\bn(G)$ of a given graph $G$ can be computed in  polynomial time (in at most $O(n^3m)$ time, see Section \ref{sec:mcw}). However, it gives a worse than 3-approximation of $\tl(G)$, $\tb(G)$. 

%{\color{red}  Computation of $\bn(G)$? $\BDS(G)$ can be computed in $O(nm)$ time \cite{tree-spanner-appr}. }

\subsection{McCarty-width of a graph}\label{sec:mcw}
Here, we give an alternative proof for Rose McCarty's conjecture that $\tl(G)$ is small if and only if $\mc(G)$ is small. See also Corollary   \ref{cor:mcw-adt} and Theorem \ref{th:mf-tl-mcw} for %more 
other alternative proofs. 

\begin{lemma} \label{lm:ClusterDiam_mcw}
	For every graph $G$  and every vertex $s$ of $G$, $\Delta_s(G)\leq 6\cdot \mc(G)$. In particular, $\widehat{\Delta}(G)\leq 6\cdot \mc(G)$ for every graph $G$.
\end{lemma}
\begin{proof}
Let $s$ be an arbitrary vertex of $G$ and $\mathcal{LP}(G,s)$ be the layering partition of $G$ starting at $s$. Consider vertices $x$ and $y$ from a cluster of $\mathcal{LP}(G,s)$ with $d_G(x,y)=\Delta_s(G)$, and let $k:=d_G(s,x)=d_G(s,y)$ and $r:=\mc(G)$. Choose also a path $Q$ connecting $x$ and $y$ outside the disk $D_{k-1}(s)$ and arbitrary shortest paths $P(s,x)$ and $P(s,y)$ connecting $s$ with $x$ and $y$, respectively. 

Since $r:=\mc(G)$, for vertices $s,x,y$, there is a vertex $v$ in $G$ such that disk $D_r(v)$ intersects all three paths: $Q$, $P(s,x)$ and $P(s,y)$ (see Fig. \ref{fig:two}). Choose arbitrary vertices $u\in Q\cap D_r(v)$, $x'\in P(s,x)\cap D_r(v)$ and $y'\in P(s,y)\cap D_r(v)$. We have $d_G(x',u), d_G(y',u)$ and $d_G(x',y')$  are all at most $2r$.  Since $d_G(s,Q)= k$, we get $d_G(s,x)=k\le d_G(s,u)\le  d_G(s,x')+d_G(x',u)\le d_G(s,x')+2r$. That is, $d_G(x,x')=d_G(s,x)-d_G(s,x')\le 2r$. Similarly, $d_G(y,y')\le 2r$. Consequently, $d_G(x,y)\le d_G(x,x')+d_G(x',y')+d_G(y,y')\le 6r$, i.e., $\Delta_s(G)\le 6\cdot \mc(G)$.   
\qed
\end{proof} 

  \begin{figure}[htb]%[tbh] %
    \begin{center} %\vspace*{-1mm}
      \begin{minipage}[b]{16cm}%5
        \begin{center} %\hspace*{10mm}
          \vspace*{-36mm}
          \includegraphics[height=16cm]{Fig-2.pdf}
        \end{center} \vspace*{-73mm}
        \caption{\label{fig:two} Illustrations to the proof of Lemma \ref{lm:ClusterDiam_mcw}. }  
      \end{minipage}
    \end{center}
   \vspace*{-5mm}
  \end{figure}

\begin{lemma} \label{lm:Clusterrad_BDS}
	For every graph $G$, every vertex $s$ of $G$, and every integer $k\ge 3$, $\mc_k(G)\le \rho_s(G)$.  In particular, $\mc_k(G)\leq \rho(G)$ for every graph $G$ and every integer $k\ge 3$. Furthermore, for any subset $X\subseteq V$ of vertices of $G$, a balanced disk separator $D_r(u)$ with $r\le \Delta_s(G)$ can be found in linear time. 
\end{lemma}

\begin{proof}
%We need to show only the right inequality. 
Let $s$ be an arbitrary vertex of $G$ and $\mathcal{LP}(G,s)=\{L^i_1,\ldots,L^i_{p_i}:i=0,1,\dots,q\}$  be the layering partition of $G$ starting at $s$. Consider also the  layering tree $\Gamma:=\Gamma(G,s)$ of graph $G$ with respect to the layering partition $\mathcal{LP}(G,s)$.  

For a given subset $X\subseteq V$ of vertices of $G$, we can assign to each node $L_i^j$ of $\Gamma$  a weight $w_i^j: =|L_i^j\cap X|$. Clearly, $W:=\sum_{i=0,1,2,\ldots,q, j=1,2,\ldots,p_i}w_i^j$ is equal to  $|X|$.
It is known that every node-weighted tree $T$ with the total weight of nodes equal to $W$ has a node $x$, called a {\em median} of $T$,
such that the total weight of nodes in each subtree of $T\setminus \{x\}$ does not exceed $W/2$. Furthermore, such a node $x$ of $T$ can be found in $O(|V(T)|)$ time \cite{goldman}. 

Let $C=L_i^j$ $(i\in \{0,1,2,\ldots,q\}, j\in \{1,2,\ldots,p_i\})$ be a median node of weighted tree $\Gamma$.   Then, each subtree of $\Gamma\setminus \{C\}$ has total weight of nodes not exceeding $|X|/2$. It is clear from the construction of tree $\Gamma$ that the set $C\subseteq V$ separates in $G$ any two vertices that belong to clusters from different subtrees of $\Gamma\setminus \{C\}$. Consequently, $C$ is a balanced separator of $G$ with respect to $X$ as any connected component of $G[V\setminus C]$ has no more than $|X|/2$ vertices from $X$. Note that, given a graph $G$, such a cluster $C$ of layering partition ${\mathcal LP}(G,s)$ of $G$
can be found in linear time in the size of $G$. 

Since there is a vertex $v$ in $G$ such that $C\subseteq D_r(v)$ for $r\le \rho_s(G)$, clearly, $D_r(v)$ is a balanced disk separator of $G$ with respect to $X$. As this holds for an arbitrary subset $X\subseteq V$, we conclude $\mc_k(G)\le \rho_s(G)$ for any $k\ge 3$. 

It is not clear how to find in $o(nm)$ time a vertex $v$ such that $C\subseteq D_r(v)$ (we may need the entire distance matrix of $G$ to do that). Instead, we can take an arbitrary vertex $u$ from $C$ and run a breadth-first-search $BFS(u)$ from $u$ in $G$ until we reach the layer $L^i$ of $BFS(u)$ with minimum $i$ such that $C\subseteq \bigcup_{j=0}^iL^i$. Clearly, $i\leq \Delta_s(G)$ and $D_i(u)$ is a balanced disk separator of $G$ with respect to $X$. 
\qed
\end{proof}
In \cite{tree-spanner-appr}, such a balanced disk separator is used to obtain an $O(m\log n)$-time $O(\log n)$-approximation algorithm for the problem of finding a tree $t$-spanner with minimum $t$ of a given graph $G$. 


It is easy to show that  in fact $\mc(G)\leq \tb(G)$ holds (see \cite{BerSey2024}). To show a stronger inequality $\mc_k(G)\le \tb(G)$ for every $k\ge 3$,  we will need %some notions from the Hypergraph Theory and 
a nice result from \cite{bal-clique-ch} on balanced clique-separators of chordal graphs.  See also Proposition   \ref{prop:bramble+} for an alternative proof. 
%(graphs that do not contain any induced cycles on four or more vertices).  

\begin{lemma} [\cite{bal-clique-ch}] \label{lm:clique_sep}
	For every chordal graph $G=(V,E)$ and every subset $X\subseteq V$ of vertices of $G$, there is a clique $C$ in $G$ such that if the vertices of $C$  are removed from $G$, every connected component in the graph induced by any remaining vertices has at most $|X|/2$ vertices from $X$.  
\end{lemma}

Using Lemma  \ref{lm:clique_sep}, we can prove the following {\em balanced-disk separator lemma}. 

\begin{lemma} \label{lm:tb_mcw_k} 
	For every graph $G=(V,E)$ and every subset $X\subseteq V$ of vertices of $G$, there is a vertex $v\in V$ such that if the vertices of disk $D_{\beta}(v)$, where $\beta=\tb(G)$, are removed from $G$, every connected component in the graph induced by any remaining vertices has at most $|X|/2$ vertices from $X$.  Consequently,  $\mc_k(G)\le \tb(G)$ holds for every graph $G$  and every integer $k\ge 3$.  
\end{lemma}
\begin{proof} Such a result for the case when $X=V$ was first proved in \cite{tree-spanner-appr}. We will adapt that proof to an arbitrary $X\subseteq V$.  
Let $G$ be a graph with $\tb(G)=\beta$ and $\cT(G)$ be its tree-decomposition  of breadth $\beta$. We can construct a new graph $G^+$ from $G$ by adding an edge between every two distinct non-adjacent vertices $x,y$ of $G$ such that a bag $B$ in $\cT(G)$ exists with $x,y\in B$.  Clearly, $G^+$ is a {\em supergraph} of the graph $G$ (each edge of $G$ is an edge of $G^+$, but $G^+$ may have some extra edges between non-adjacent vertices of $G$ contained in a common bag of $\cT(G)$). 
It is known (see, e.g., survey \cite{Bodlaender}) that $G^+$ is a chordal graph, $\cT(G)$ is a clique-tree of $G^+$,  and for each clique $C$ of $G^+$ there is a bag $B$ in $\cT(G)$ such that $C\subseteq B$. 
%
By Lemma  \ref{lm:clique_sep},  the chordal graph  $G^{+}$ contains a balanced clique-separator $C$.
Since $C$ is contained in a bag of $\cT(G)$, there must exist a vertex $v\in V(G)$ with $C\subseteq D_{\beta}(v,G)$. As the removal of  the vertices of $C$ from $G^+$ leaves no connected component in $G^+[V\setminus C]$ with more that $|X|/2$ vertices from $X$, and since $G^+$ is a supergraph of $G$, clearly, the removal of  the vertices of $D_{\beta}(v,G)$ from $G$ leaves no connected component in $G[V\setminus D_{\beta}(v,G)]$ with more that $|X|/2$ vertices from $X$.
\qed
\end{proof}

Note that, to find a balanced disk-separator $D_r(v)$ with $r\le \tb(G)$ of a graph $G$ with respect to a subset $X\subseteq V$, one does not need to have a tree-decomposition $\cT(G)$ of breadth $\tb(G)$. For a given graph $G=(V,E)$, a subset $X\subseteq V$ and a fixed vertex $v$ of $G$,  a balanced disk-separator $D_r(v)$ with minimum $r$ can be computed in $O(m)$ time (see \cite{tree-spanner-appr}). Hence, an overall  balanced disk-separator $D_r(v)$ with minimum $r$ of a graph $G$ with respect to a subset $X\subseteq V$ can be found in total $O(nm)$ time (one can run the algorithm from  \cite{tree-spanner-appr} for every $v\in V$ and pick a best vertex $v$).  In \cite{tree-spanner-appr}, such a balanced disk separator is used to obtain an $O(nm\log^2 n)$-time $(\log_2 n)$-approximation algorithm for the problem of finding a tree $t$-spanner with minimum $t$ of a given graph $G$.

Combining Proposition \ref{prop:dorisb}, Lemma  \ref{lm:ClusterDiam_mcw}, Lemma  \ref{lm:Clusterrad_BDS}, and Lemma  \ref{lm:tb_mcw_k},  %??? Corollary \ref{cor:ineq} and Proposition \ref{prop:dorisb}, 
we get the following result. 
\begin{theorem} \label{th:mcw-delta-rho}
	For every graph $G$  and every vertex $s$ of $G$, the following inequalities hold: 
$$\mc(G)\leq \rho(G)\le\rho_s(G)\le \Delta_s(G)\le\widehat{\Delta}(G)\leq 6\cdot \mc(G),$$ 
$$\mc(G)\leq \tb(G)\leq \tl(G)\leq  \Delta_s(G)+1\leq 6\cdot \mc(G)+1.$$ 	
\end{theorem} 

In \cite[Theorem 5.1]{BerSey2024}, it was shown that $\mc(G)\leq \tb(G)\leq \tl(G)\leq 6\cdot \mc(G)+3$ holds. 

%{\color{red}/* mine is slightly better than in {BerSey2024} */}

From Theorem \ref{th:bnc-delta-tl} and Theorem  \ref{th:mcw-delta-rho}, we also get. 
%
\begin{corollary}  \label{cor:ineq-bnc-mcw}
For every graph $G$,   $\mc(G)\leq  4\cdot \bn(G)+2$ and $\bn(G)\le  3\cdot \mc(G)$. 
\end{corollary}

It is easy to see that, for every graph $G$ and every $k\ge 2$,  $\mc_{2k}(G)\le\mc_{2k-1}(G)$. Indeed, consider a set $X=\{x_1,\dots,x_{2k-1},$ $x_{2k}\}\subseteq V$ and let $r:=\mc_{2k-1}(G)$. Then, there is a vertex $v\in V$ such that every connected component of $G[V\setminus D_r(v)]$ has at most $\frac{2k-1}{2}$, i.e., at most $k-1$, vertices from $X\setminus \{x_{2k}\}$. Hence, those connected components contain at most $k$ vertices from $X$. So, $\mc_{2k}(G)\le\mc_{2k-1}(G)$ must hold. 
%
Two induced cycles $C_3$ (with 3 vertices each) sharing one common vertex (a so-called {\em bow-graph}) gives an example of a graph $G$ with $\mc_3(G)=1$ and $\mc_4(G)=0$. So,  $\mc_{2k}(G)<\mc_{2k-1}(G)$ is possible for some graph $G$.
%
We do not know the relationship between $\mc_{2k}(G)$ and $\mc_{2k+1}(G)$, except that there are graphs $G$ and integers $k$ such that $\mc_{2k}(G)<\mc_{2k+1}(G)$ (e.g., for a block graph with three blocks and two articulation points, each block being a triangle, $\mc_{4}(G)=0<1=\mc_{5}(G)$ holds). However, from our results, it follows that, for every graph $G$, every $k\ge 3$ and $s\in V$, $\mc_k(G)\le\rho_s(G)\le\Delta_s(G)\le 6\cdot\mc_3(G)=6\cdot\mc(G)$ (which is of independent interest). 
%
Thus, for every $k\ge 3$, $\mc_k(G)$ is bounded from above by a (linear) function of $\mc(G)$. This leads to a natural question if $\mc_k(G)$ ($k> 3$) and $\mc_3(G)$ are coarsely equivalent parameters. Unfortunately, the answer is 'no'. By extending our bow-graph example, we can show that $\mc_k(G)$ ($k>3$), generally, cannot be bounded from below by a function of $\mc(G)$.  
%Furthermore, ...  the difference between $\mc_k(G), k>3,$ and $\mc(G)$ could be arbitrarily large. 
Consider a graph $G$ consisting of two induced cycles $C_{6p}$ on $6p$ $(p\ge 1)$ vertices each,  sharing just one common vertex $v$. We have $n:=|V(G)|=12p-1$ and  $\mc_n(G)=0$ (we just need to remove vertex $v$ to partition $G$ into two connected components each having $6p-1$ vertices from $V(G)$). Considering also three vertices of one $C_{6p}$ that are pairwise at distance $2p$ from each other, we get $\mc(G)\ge p$.  %. It is easy to check that

As we have mentioned earlier, a balanced disk-separator $D_r(v)$ with minimum $r$ of a graph $G$ with respect to any subset $X\subseteq V$ can be found in total $O(nm)$ time. Consequently, for every graph $G$, the parameter $\mc(G)$ can be computed in at most $n^3\cdot O(nm)=O(n^4m)$ time. Similarly, for every graph $G$, the parameter $\bn(G)$ can be computed in at most $O(n^3m)$ time. One can try to get better time complexities (using some special data structures), but since $\bn(G)$ and $\mc(G)$ are further from $\tl(G)$ than $\Delta_s(G)$ is, we did not pursue this line of investigation. 

%{\color{red}
%\begin{question} Should $\mc_k(G)$ be incorporated in the inequalities? What is the gap between $\mc_k(G)$ and $\mc_{k+1}$? Should $\mc_k(G)$ be redefined for all $|X|\le k$? Computation of $\mc(G)$? In a straightforward way, in $O(n^3\times nm)=O(n^4m)$ time.
%\end{question}
%}

\subsection{Brambles and Helly families of connected subgraphs or paths}\label{sec:br-Helly} 
A {\em bramble} of a graph $G$ is a family of connected subgraphs of $G$ that all {\em touch} each other: for every pair of disjoint subgraphs, there must exist an edge in $G$ that has one endpoint in each subgraph. The {\em order} of a bramble is the smallest size of a hitting set, a set of vertices of $G$ that has a nonempty intersection with each of the subgraphs. Brambles are used to characterize the tree-width of $G$: $k$ is the largest possible order among all brambles of $G$ if and only if $G$ has tree-width $k - 1$ \cite{SeyThom1993}.  

A family of %connected 
subgraphs of $G$ 
is called a {\em Helly family} if every two subgraphs of the family intersect. Clearly, any Helly family of connected subgraphs of $G$ is a particular bramble of $G$. 
%
Below we define new properties for brambles and Helly families of connected subgraphs of $G$ that coarsely define the tree-breadth $\tb(G)$. 

Let ${\cal F}:=\{H_1,\dots,H_p\}$ be a family of subgraphs of $G$. We say that a disk $D_r(v)$ of $G$ intercepts all members of ${\cal F}$ if $V(H_i)\cap D_r(v)\neq \emptyset$ for every $i=1,\dots,p$.  
Denote by $\br(G)$ (the {\em bramble interception} radius  of $G$) the smallest integer $r$ such that for every bramble ${\cal F}$ of $G$, there is a disk $D_r(v)$ which intercepts all members of ${\cal F}$.  Denote by $\sh(G)$ (by $\ph(G)$) the smallest integer $r$ such that for every Helly family ${\cal F}$ of connected subgraphs (of paths, respectively) of $G$, there is a disk $D_r(v)$ which intercepts all members of ${\cal F}$. Call $\sh(G)$ ($\ph(G)$) the interception radius for Helly families of connected subgraphs (of  paths) of $G$. Such property for Helly families of disks of a graph were already considered in literature (see, e.g., \cite{HellyGroups,ChepoiEst,gap}). 

It turns out that all these three parameters are coarsely equivalent to tree-breadth. We prove this by involving the McCarty-width parameter and a result from \cite{SeyThom1993}. 

\begin{proposition}   \label{prop:bramble}
	For every $G$,  $\mc(G)\le \ph(G)\le\sh(G)\le\br(G)\le \tb(G)\le 6\cdot \mc(G)+1$. 
\end{proposition} 
\begin{proof} Inequalities $\ph(G)\le\sh(G)\le\br(G)$ follow from definitions. Hence, by Theorem \ref{th:mcw-delta-rho}, we only need to show $\br(G)\le \tb(G)$ and $\mc(G)\le \ph(G)$. According to %\cite[(2.3)]{SeyThom1993}, 
\cite{SeyThom1993},  for every bramble ${\cal F}$ of a graph $G$, there is a bag $B$ in every tree-decomposition of $G$ which intercepts all members of ${\cal F}$. Hence, for a tree-decomposition with minimum breadth, there is a bag $B$ and a vertex $v$ in $G$ with $B\subseteq D_r(v)$, $r\le \tb(G)$, such that the disk $D_r(v)$  of $G$  intercepts all members of ${\cal F}$. Consequently, $\br(G) \le \tb(G)$. 

Consider now any three vertices $x,y,z$ of $G$ and the family ${\cal F}$ of all paths of $G$ connecting pairs from $\{x,y,z\}$. The family ${\cal F}$ is a Helly family. If $\ph(G)=r$, then there is a disk $D_r(v)$ in $G$ that  intercepts all paths from ${\cal F}$. Consequently, no connected component of $G[V\setminus D_r(v)]$  contains two of $x,y,z$. The latter proves $\mc(G)\le \ph(G)$. 
\qed
\end{proof}


We can generalize the second part of the proof of Proposition \ref{prop:bramble} and show $\mc_k(G)\le \sh(G)$ for every graph $G$ and every $k\ge 3$. This provides also an alternative proof of Lemma  \ref{lm:tb_mcw_k}. 
\begin{proposition}   \label{prop:bramble+}
	For every graph $G$ and every $k\ge 3$,  $\mc_k(G)\le \sh(G)\le\br(G)\le \tb(G)$. 
\end{proposition} 
\begin{proof} We need only to show $\mc_k(G)\le \sh(G)$. Consider a subset $X=\{x_1,x_2,\dots,x_k\}$ of vertices of $G$ and any subset $Y\subset X$ containing $\lfloor\frac{k}{2}\rfloor+1$ vertices of $X$. Let 
${\cal F}_{Y}$ be the family of all connected subgraphs of $G$ spanning vertices of $Y$. For any two subsets $X'$ and $X''$ of $X$ containing  $\lfloor\frac{k}{2}\rfloor+1$ vertices of $X$ each, any subgraph from ${\cal F}_{X'}$ intersects any subgraph from ${\cal F}_{X''}$ (as $X'\cap X''\neq\emptyset$). Hence, ${\cal F}:= \cup \{{\cal F}_{Y}: Y\subset X, |Y|=\lfloor\frac{k}{2}\rfloor+1\}$ is a Helly family  of connected subgraphs of $G$.  
Hence, there must exist a vertex $v$ in $G$ such that disk $D_r(v)$, with $r\le \sh(G)$, intercepts each subgraph from ${\cal F}$.  The latter implies that no connected component of $G[V\setminus D_r(v)]$  contains more than $\lfloor\frac{k}{2}\rfloor$ vertices from $X$. Thus, $\mc_k(G)\le \sh(G)$ must hold. 
\qed
\end{proof}
%%%%%%%%%%%% 
For an induced cycle $C_6$ on six vertices, we have $\ph(C_6)=\sh(C_6)=1$ while $\br(C_6)=2$. So, $\sh(G)<\br(G)$ is possible for some graph $G$. %?????????????????????????????????????????



\subsection{Distance $k$-approximating trees}\label{sec:adt}
By Proposition \ref{lem:cluster-diam} and Proposition \ref{prop:dorisb}, $\adt(G)\leq \Delta(G)\leq 3\cdot \tl(G)$. Hence, we need only to upperbound $\tl(G)$ by a linear function of $\adt(G)$. 

\begin{lemma}   \label{lm:tl-adt}
	For every graph $G$,  $\tl(G)\le 2\cdot \adt(G)+1$. 
\end{lemma} 
\begin{proof}
Let $G=(V,E(G))$, $\adt(G)=r$, and $T=(V,E(T))$ be a tree such that $|d_G(x,y)-d_T(x,y)|\le r$ for every $x,y\in V$. For every edge $uv\in E(G)$, $d_T(u,v)\le d_G(u,v)+r\le 1+r$ holds. Hence, $G$ is a spanning subgraph of graph $T^{r+1}$, where  $T^{r+1}$ is the $(r+1)^{st}$-power of $T$. It is known (see e.g., \cite{Andreas-book,golumbic}) that every power of a tree is a chordal graph. Consequently, there is a clique-tree $\cT(G)$ of $T^{r+1}$.  Clearly, $\cT(G)$ is a tree-decomposition of $G$ such that, for every two vertices $x$ and $y$ belonging to same bag %$C$ 
of $\cT(G)$, $xy$ is an edge of $T^{r+1}$. Necessarily, $d_T(x,y)\le r+1$, by the definition of the $(r+1)^{st}$-power of  $T$. Furthermore, since $d_G(x,y)\le d_T(x,y)+r\le 2r+1$ for every $x,y$ belonging to same bag of $\cT(G)$, the length of the tree-decomposition $\cT(G)$ of $G$ is at most $2r+1=2\cdot \adt(G)+1$. 
\qed
\end{proof}
Combining Proposition \ref{lem:cluster-diam} and Proposition \ref{prop:dorisb} with Theorem \ref{th:bnc-delta-tl}, Lemma  \ref{lm:tl-adt} and Lemma \ref{lm:ClusterDiam_bnc}, we get the following result.  

\begin{theorem} \label{th:adt-tl-delta}
	For every graph $G$  and every vertex $s$ of $G$, the following inequalities hold: 
$$\adt(G)\leq \Delta(G)\le \Delta_s(G)\le\widehat{\Delta}(G)\leq 3\cdot \tl(G)\le 6\cdot \adt(G)+3,$$ 
$$\adt(G)\leq \Delta_s(G)\le 4\cdot \bn(G)+2, $$ 
%\mbox{~and~} 
$$\bn(G)\le 3\cdot \adt(G)+1.$$
%$$\mc(G)\leq \tb(G)\leq \tl(G)\leq 2\cdot \adt(G)+1.$$ 
\end{theorem} 


\begin{corollary} %[\cite{BerSey2024}] 
\label{cor:ineq-tl-adt}
	For every graph $G$,   $\frac{\tl(G)-1}{2}\leq \adt(G)\leq 3\cdot \tl(G).$
\end{corollary}

Recall that in \cite[Theorem 4.1, Theorem 4.2, Theorem 4.5]{BerSey2024}, it was shown that  $\frac{\tl(G)-2}{2}\leq \ad(G)\leq 6\cdot \tl(G)$  and $\ad(G)\leq 24\cdot \bn(G)+18 $ hold.

From Theorem  \ref{th:mcw-delta-rho} and Theorem \ref{th:adt-tl-delta}, it also follows  $\adt(G)\le\widehat{\Delta}(G)\leq 6\cdot \mc(G)$ and $\mc(G)\le\widehat{\Delta}(G)\leq 6\cdot \adt(G)+3.$ The latter inequality can further be improved to $\mc(G)\le \frac{3\cdot \adt(G)+1}{2}.$ For this we will need the following interesting lemma. 


\begin{lemma}   \label{lm:mcw-adt}
Let $G=(V,E)$ be a graph, $T=(V,E')$ be a distance $k$-approximating tree of $G$ with $k:=\adt(G)$.  For every $x,y\in V$, any path $P_G(x,y)$ of $G$ between $x$ and $y$, and any vertex $c$ from the path $P_T(x,y)$ of $T$ between $x$ and $y$, $d_G(c, P_G(x,y))\le \frac{3\cdot \adt(G)+1}{2}$ holds. 
\end{lemma} 
\begin{proof} Removing $c$ from $T$, we separate $x$ from $y$. Let $T_y$ be the subtree of $T[V\setminus \{c\}]$ containing $y$. Since $x\notin T_y$, we can find an edge $ab$ of $P_G(x,y)$ with $a\in T_y$ and
$b\notin T_y$. Therefore, the path $P_T(a,b)$ must go via $c$. If $d_T(c,a) > (k+1)/2$ and $d_T(c,b) >(k+1)/2$, 
then $d_T(a,b) = d_T(a,c) + d_T(c,b) > k+1$ and, since $d_G(a,b) = 1$, we obtain a contradiction with the 
assumption that $T$ is a distance $k$-approximating tree of $G$ (with  $d_T (a,b)\le d_G(a,c)+k= k+1$). Hence $d_T(c,P_G(x,y)) \le \min\{d_T(c,a),d_T(c,b)\} \le (k+1)/2$. 
Let $z$ be a vertex of $P_G(x,y)$ such that $d_T(c,P_G(x,y))=d_T(c,z)$. We have 
$d_G(c,P_G(x,y))\le d_G(c,z)\le d_T(c,z)+k= d_T (c,P_G(x,y))+k\le (k+1)/2+k$. 
\qed
\end{proof}

\begin{corollary}   \label{cor:mcw-adt}
	For every graph $G$,  $\adt(G)\le 6\cdot \mc(G)$ and $\mc(G)\le \frac{3\cdot \adt(G)+1}{2}$. 
\end{corollary} 
\begin{proof} The first inequality follows from Theorem \ref{th:mcw-delta-rho} and Theorem \ref{th:adt-tl-delta}. The second one follows from Lemma    \ref{lm:mcw-adt}. 
%
Indeed, let $x,y,z$ be three arbitrary vertices of $G$ and $T$ be a distance $k$-approximating tree of $G$ with $k:=\adt(G)$.  Let $c$ be the unique vertex
of $T$ that is on the intersection of paths $P_T(x,y)$, $P_T(x,z)$ and $P_T(y,z)$. Since $c$ belongs to all three paths, applying Lemma \ref{cor:mcw-adt} three times,
we infer that $d_G(c,P_G(x,y))\le \frac{3\cdot \adt(G)+1}{2}, d_G(c,P_G(x,z))\le \frac{3\cdot \adt(G)+1}{2}$ and $d_G(c,P_G(y,z))\le \frac{3\cdot \adt(G)+1}{2}$ for every path $P_G(x,y)$ between $x$ and $y$, for every path $P_G(x,z)$ between $x$ and $z$, and  for every path $P_G(z,y)$ between $z$ and $y$.  So, no connected component of $G[V\setminus D_r(c)]$, $r=\frac{3\cdot \adt(G)+1}{2}$,  contains two of $x,y,z$.
\qed
\end{proof}

\subsection{$K$-Fat $K_3$-minors}\label{sec:fat}
Here, we give an alternative proof for a result from 
\cite{GeorPapa2023} that a graph $G$ has no $K$-fat $K_3$-minor for some constant $K>0$ if and only if $G$ is $(1,C)$-quasi-isometric to a tree for some constant $C$. %$C\le 14K$. 
Our Corollary \ref{cor: adt-fm} improves the constant $C$. % to $C\le 5K-1$.  


\begin{lemma} \label{lm:ClusterDiam_fat}
	Let $G$ be a graph  and $s$ be a vertex  of $G$. If $\Delta_s(G)\ge 5K$ then $G$ has a $K$-fat $K_3$-minor.  
\end{lemma}
\begin{proof}
Let $\mathcal{LP}(G,s)$ be the layering partition of $G$ starting at $s$. Consider vertices $x$ and $y$ from a cluster of $\mathcal{LP}(G,s)$ with $d_G(x,y)=\Delta_s(G)$, and let $\ell:=d_G(s,x)=d_G(s,y)$. Choose also a path $Q$ connecting $x$ and $y$ outside the disk $D_{\ell-1}(s)$ and arbitrary shortest paths $P(s,x)$ and $P(s,y)$ connecting $s$ with $x$ and $y$, respectively. 
%
Since $\Delta_s(G)\ge 5K$, we have $d_G(x,y)\ge 5K$. We construct a $K$-fat $K_3$-minor of $G$ in the following way. 

We know that $\ell$ must be greater that $2K$ (otherwise, $d_G(x,y)\le d_G(x,s)+d_G(s,y)\le 2K+2K=4K$, contradicting with $d_G(x,y)\ge 5K$). 
Consider vertices $x'$ and $s_x$ on path $P(x,s)$ at distance $K$ and $2K$ from $x$, respectively, i.e., with $d_G(x,x')=d_G(x',s_x)=K$. Similarly, consider  
vertices $y'$ and $s_y$ on path $P(y,s)$ at distance $K$ and $2K$ from $y$, respectively. As three connected subgraphs of $G$ choose $H_x:=G[D_K(x)], H_y:=G[D_K(y)]$ and $H_s:=G[D_{\ell-2K}(s)]$. As three paths choose a subpath $P(x',s_x)$ of $P(x,s)$ between $x'$ and $s_x$, a subpath $P(y',s_y)$ of $P(y,s)$ between $y'$ and $s_y$ and a subpath $Q(x'',y'')$ of $Q$ between vertices $x'',y''\in Q$, where $x''$ is the vertex of $Q\cap D_K(x)$ which maximizes $d_Q(x,x'')$ and $y''$ is the vertex of $Q\cap D_K(y)$ which maximizes $d_Q(y,y'')$. Clearly, those three connected subgraphs and three paths form a $K_3$-minor in $G$ (note that $d_Q(x'',y'')\ge 3K$ since, otherwise, $d_G(x,y)\le d_G(x,x'')+d_G(x'',y'')+d_G(y'',y)<K+3K+K=5K$, which is impossible).  It remains to show that it is a $K$-fat $K_3$-minor. 

We have  $d_G(V(H_x),V(H_s))=d_G(D_K(x),D_{\ell-2K}(s))= K$ since $d(x,s)=\ell=K+K+(\ell-2K).$ Similarly, $d_G(V(H_y),V(H_s))= K$. Furthermore,  $d_G(V(H_x),V(H_y))=d_G(D_K(x),D_{K}(y))\ge 3K$ since $d(x,y)\ge 5K.$ If $d_G(P(x',s_x),Q(x'',y''))<K$ holds, then $d_G(s,Q(x'',y''))\le d_G(s,x')+d_G(P(x',s_x),Q(x'',y''))<\ell-K+K=\ell$. The latter implies $d_G(s,Q)<\ell$, which is impossible. So, $d_G(P(x',s_x),Q(x'',y''))\ge K$ must hold. Similarly, $d_G(P(y',s_y),Q(x'',y''))\ge K$ must hold. If $d_G(P(x',s_x),P(y',s_y))<K$ holds, then $d_G(x,y)\le d_G(x,s_x)+ d_G(P(x',s_x),P(y',s_y))+d_G(s_y,y)<2K+K+2K=5K$, contradicting with  $d_G(x,y)\ge 5K$. If $d_G(V(H_x),P(y',s_y))=d_G(D_K(x),P(y',s_y))<K$, then $d_G(x,y)\le d_G(x,x')+ d_G(D_K(x),P(y',s_y))+d_G(s_y,y)<K+K+2K=4K$, which is impossible. 
So, $d_G(V(H_x),P(y',s_y))\ge K$ and, by symmetry, $d_G(V(H_y),P(x',s_x))\ge K$. Finally, $d_G(V(H_s),Q(x'',y''))=d_G(D_{\ell-2K}(s),Q(x'',y''))\ge 2K$ since, otherwise, we get $d_G(s,Q)<\ell$, which is impossible. 

Thus, constructed $K_3$-minor of $G$ is $K$-fat. 
\qed
\end{proof} 

From Lemma \ref{lm:ClusterDiam_fat} and Theorem \ref{th:adt-tl-delta}, we have the following corollary. 
%, which also improves the constant $C$ in Proposition  \ref{prop:Papaaoglu}.  

\begin{corollary} \label{cor: adt-fm}
If $G$ has no $K$-fat $K_3$-minor, then  $\Delta_s(G)\le 5K-1$ for every vertex $s$ of $G$. In particular, $\adt(G)\le 5K-1$. 
\end{corollary}

%\newpage 
\begin{lemma} \label{lm:mcw_fat}
	Let $G$ be a graph with $\mc(G)=r$. Then, $G$ has no $K$-fat $K_3$-minor for $K>2r$.  
\end{lemma}
\begin{proof} Assume $G$ has a $K$-fat $K_3$-minor ($K>2r$) formed by three connected subgraphs $H_1$, $H_2$, $H_3$ and three simple paths $P_{1,2}$, $P_{2,3}$, $P_{1,3}$ such that for each $i,j\in \{1,2,3\}$ ($i\neq j$), 
\begin{itemize}
    \item[(1)]      $P_{i,j}$ has one end in $H_i$ and the other end in $H_j$ and  $|P_{i,j}\cap V(H_i)|=|P_{i,j}\cap V(H_j)|=1$, and
    \item[(2)]      $d_G(V(H_i),V(H_j))\ge K$,  $d_G(P_{i,j},V(H_k))\ge K$ ($k\in \{1,2,3\}, k\neq i, j$) and the distance between any two paths $P_{1,2}$, $P_{2,3}$, $P_{1,3}$ is at least $K$.  
\end{itemize}   
Let \begin{itemize}
\item $P_{1,2}\cap V(H_1)=\{x_1\}$, $P_{1,2}\cap V(H_2)=\{x_2\}$, 
\item $P_{2,3}\cap V(H_2)=\{y_2\}$, $P_{2,3}\cap V(H_3)=\{y_3\}$,  
\item $P_{1,3}\cap V(H_1)=\{z_1\}$, $P_{1,3}\cap V(H_3)=\{z_3\}$.   
\end{itemize}  
Vertices $x_1$ and $z_1$ are connected in $H_1$ via a path of length at least $K$. Choose such a path $Q_1(x_1,z_1)$ and let $v_1$ be a middle vertex of $Q_1(x_1,z_1)$. 
Similarly, choose a path $Q_2(x_2,y_2)$ in $H_2$ and a middle vertex $v_2$ of $Q_2(x_2,y_2)$, and choose a path $Q_3(y_3,z_3)$ in $H_3$ and a middle vertex $v_3$ of  $Q_3(y_3,z_3)$. 

Since $\mc(G)=r$, for vertices $v_1,v_2,v_3$ there must exist a vertex $u$ in $G$ such that no connected component of $G[V\setminus D_r(u)]$  contains two of $v_1,v_2,v_3$. We show that this is not possible due to $K>2r$ and distance requirements listed in (2). 

Disk $D_r(u)$ needs to intercept each of the following three paths: 
\begin{itemize}
\item $P(v_1,v_2):=Q_1(v_1,x_1)\cup P_{1,2}\cup Q_2(x_2,v_2)$, where $Q_1(v_1,x_1)$ and $Q_2(x_2,v_2)$ are subpaths of $Q_1(z_1,x_1)$ and $Q_2(x_2,y_2)$, respectively, connecting corresponding vertices, 
\item $P(v_2,v_3):=Q_2(v_2,y_2)\cup P_{2,3}\cup Q_3(y_3,v_3)$, where $Q_2(v_2,y_2)$ and $Q_3(y_3,v_3)$ are subpaths of $Q_2(x_2,y_2)$ and $Q_3(y_3,z_3)$, respectively, connecting corresponding vertices, 
\item $P(v_1,v_3):=Q_1(v_1,z_1)\cup P_{1,3}\cup Q_3(z_3,v_3)$, where $Q_1(v_1,z_1)$ and $Q_3(z_3,v_3)$ are subpaths of $Q_1(x_1,z_1)$ and $Q_3(z_3,y_3)$, respectively, connecting corresponding vertices. 
\end{itemize}  
Choose $w_{1,2}\in D_r(u)\cap P(v_1,v_2)$, $w_{1,3}\in D_r(u)\cap P(v_1,v_3)$, $w_{2,3}\in D_r(u)\cap P(v_2,v_3)$. 
Vertices $w_{1,2},w_{1,3},w_{2,3}$ are pairwise at distance at most $2r<K$ in $G$.  If $w_{1,2}\in P_{1,2}$ then, by distance requirements in (2),  $w_{2,3}$ can neither be in  $P_{2,3}$  nor in $H_3$. Hence, $w_{2,3}$ is in $Q_2(v_2,y_2)\subset V(H_2)$. Similarly, $w_{1,3}$ must be in $Q_1(v_1,z_1)\subset V(H_1)$. But then, we get $d_G(V(H_1),V(H_2))\le 2r<K$, which is not possible. So, by symmetry, we can assume $w_{1,2}\notin P_{1,2}$, $w_{1,3}\notin P_{1,3}$ and  $w_{2,3}\notin P_{2,3}$, i.e., vertices $w_{1,2}, w_{1,3}, w_{2,3}$ are in subgraphs $H_1$, $H_2$, $H_3$. However, since no one of $H_1$, $H_2$, $H_3$ can have all three vertices $w_{1,2}, w_{1,3}, w_{2,3}$, for some $i,j\in \{1,2,3\}, i\neq j$, $d_G(V(H_i),V(H_j))\le 2r<K$ holds, which is impossible. 

Thus, $G$ cannot have any $K$-fat $K_3$-minor for $K>2r$.  
%Hence, by symmetry and distance requirements in (2), 
\qed 
\end{proof} 

Denote by $\mf(G)$ the largest $K>0$ such that $G$ has a $K$-fat $K_3$-minor. Call it the {\em $K_3$-minor fatness} of $G$.  We obtain the following theorem from Corollary    \ref{cor:mcw-adt}, Corollary  \ref{cor: adt-fm}, Lemma \ref{lm:mcw_fat}, Proposition \ref{prop:Papaaoglu}, Theorem \ref{th:bnc-delta-tl}, and Theorem  \ref{th:mcw-delta-rho}.  


\begin{theorem} \label{th:mf-tl-mcw}
    For every graph $G$ and every vertex $s$ of $G$, $$\frac{\mf(G)}{2}\le \mc(G)\le\tb(G)\le\tl(G)\le \Delta_s(G)+1\le 5\cdot\mf(G),$$  $$\frac{\mf(G)-1}{2}\le \frac{2\cdot\mc(G)-1}{3}\le \adt(G)\le \Delta_s(G)\le 5\cdot \mf(G)-1,$$ $$\mf(G)\le 2\cdot\bn(G)+1\le \Delta_s(G)+1\le 5\cdot \mf(G).$$ 
\end{theorem}

%\begin{corollary}
%   For every graph $G$ and every vertex $s$ of $G$, $\Delta_s(G)\ge 5K$ then $G$ has a $K$-fat $K_3$-minor.  
%\end{corollary}

\subsection{Cycle bridging properties}\label{sec:cbc}
The following {\em characteristic cycle property} immediately follows from the definition of a chordal graph $G$: {\em For every simple cycle $C$ %$(|C|\ge 3)$ 
of a chordal graph $G$ and every vertex $v\in C$, the two neighbors in $C$ of $v$ are adjacent or there is a third vertex in $C$ that is adjacent to $v$.} We can generalize this property and show that its generalized version is coarsely equivalent to tree-length. 

Let $C$ be a simple cycle of $G$, %with $|C|\ge 2k+4$ ($k\ge 1$), 
$v$ be a vertex of $C$ and $x,y$ be two vertices of $C$ with $d_C(x,v)=d_C(y,v)=k$ (assuming $C$ is long enough). We call $x,y$ the {\em $k$-neighbors of $v$ in $C$}.  Denote by $\cbc(G)$ (call it the {\em cycle bridging constant} of $G$) the minimum $k$ ($k\ge 1$) such that for every simple cycle $C$ of $G$ %with $|C|\ge 2k+4$ 
and every vertex $v$ of $C$, if $d_G(x,y)=2k$ holds for the two $k$-neighbors $x,y$ of $v$ in $C$ (resulting in $|C|\ge 4k$), then there is a vertex $z\in C$ satisfying $d_G(v,z)\le k< d_C(v,z)$ (the latter inequality just says that $z$ is on $C\setminus P_v(x,y)$, where $P_v(x,y)$ is a part of $C$ between $x$ and $y$ containing $v$). Clearly, $\cbc(G)$ of a chordal graph $G$ is 1.  The following two lemmas hold. 

\begin{lemma}   \label{lm:bnc-cbc}
	For every graph $G$,  $\bn(G)\le \cbc(G)$. 
\end{lemma} 

\begin{proof}
Assume $\bn(G)>\cbc(G):=k$. By the definition of the bottleneck constant, for some vertices $x,y$ of $G$ at even distance from each other, some shortest path $P(x,y)$ connecting $x$ and $y$ and the middle vertex $s$ of $P(x,y)$,  there must exist a path $Q$ in $G[V\setminus D_{k}(s)]$ connecting $x$ and $y$. Necessarily, $d_G(x,s)=d_G(y,s)\ge k+1$. 
%
Consider a vertex $x'$  on a subpath of $P(x,y)$ between $x$ and $s$ and a vertex $y'$  on a subpath of $P(x,y)$ between $y$ and $s$ with $d_G(x',s)=d_G(y',s)=k+1$. We have that vertices $x'$ and $y'$ with $d_G(x',y')=2k+2$ and $d_G(x',s)=d_G(s,y')=k+1$ are connected in $G[V\setminus D_{k}(s)]$ by  $P(x,x')\cup Q\cup P(y,y')$, where $P(x,x')$ and $P(y,y')$ are subpaths of $P(x,y)$ between corresponding vertices. Extract from  $P(x,x')\cup Q\cup P(y,y')$ a simple subpath $Q'$ connecting $x'$ and $y'$ outside the disk $D_{k}(s)$. The union of $Q'$ and $P(x',y')$ forms a simple cycle in which the $k$-neighbors of $s$ are at distance $2k$ in $G$ but $s$ does not have any vertex in $Q'$ at distance at most $k$ in $G$. The latter contradicts with $\cbc(G)=k$. 
\qed
\end{proof}

\begin{lemma}   \label{lm:Delta-cbc}
	For every graph $G$, $\cbc(G)\le \lceil\frac{\widehat{\Delta}(G)+1}{2}\rceil\le \frac{\widehat{\Delta}(G)}{2}+1$. 
 % (OLD $\cbc(G)\le \frac{\widehat{\Delta}(G)+3}{2}$.) 
 %OLD-OLD $\cbc(G)\le \frac{\widehat{\Delta}(G)-1}{2}$. 
\end{lemma} 

\begin{proof}
Let $k:=\lceil\frac{\widehat{\Delta}(G)+1}{2}\rceil$, $C$ be a simple cycle of $G$,  $v$ be an arbitrary vertex of $C$, and assume that $d_G(x,y)=2k$ holds for the two $k$-neighbors $x,y$ of $v$ in $C$. Consider the layering partition $\mathcal{LP}(G,v)$ of $G$ starting at $v$. Since $d_G(x,v)=d_G(y,v)=k$, vertices $x$ and $y$ belong to the same layer $L^{k}$ of the layering of $G$ with respect to $v$. Since $d_G(x,y)=2k=2\lceil\frac{\widehat{\Delta}(G)+1}{2}\rceil>\widehat{\Delta}(G)$, by the definition of $\widehat{\Delta}(G)$, $x$ and $y$ cannot belong to the same cluster from $L^{k}$. By the definition of clusters, every path connecting $x$ with $y$ in $G$ must have a vertex in $D_{k-1}(v)$. Hence, there must exist also a vertex $z\in C\setminus P_v(x,y)$ such that $d_G(v,z)\le k-1<k$, where $P_v(x,y)$ is a part of $C$ between $x$ and $y$ containing $v$.  %Choosing now  $k=\frac{\widehat{\Delta}(G)-1}{2}$, we guarantee $d_G(x,y)=\widehat{\Delta}(G)+1>  \widehat{\Delta}(G)$. 
Consequently, $\cbc(G)\le \lceil\frac{\widehat{\Delta}(G)+1}{2}\rceil\le \frac{\widehat{\Delta}(G)}{2}+1$.  \qed
\end{proof}


%\begin{proof}[OLD]
%Let $C$ be a simple cycle of $G$,  $v$ be an arbitrary vertex of $C$, and assume that $d_G(x,y)=2k$ holds for the two $k$-neighbors $x,y$ of $v$ in $C$. We need to show that a vertex $z\in C$ exists such that $d_G(v, z)\le k$ and $d_C(v,z)>k$. Consider vertices $x'$ and $y'$ in $C$ with $d_C(v,x')=d_C(v,y')=k+1$; $x'$ is a neighbor of $x$ in $C$ and $y'$ is a neighbor of $y$ in $C$. Assume $d_G(v,x')>k$ and $d_G(v,y')>k$ (otherwise, we are done as $z$ can be chosen to be $x'$ or $y'$). Then, $d_G(v,x')=d_G(v,y')=k+1=d_C(v,x')=d_C(v,y')$.  Let $\mathcal{LP}(G,v)$ be the layering partition  of $G$ starting at $v$. Since $d_G(x',v)=d_G(y',v)=k+1$, vertices $x'$ and $y'$ %(vertices $x'$ and $y'$) belong to the same layer $L^{k+1}$ %($L^{k+1}$, respectively) of the layering of $G$ with respect to $v$. 
%
%If $d_G(x',y')>  \widehat{\Delta}(G)$ then, by the definition of $\widehat{\Delta}(G)$, $x'$ and $y'$ cannot belong to the same cluster from $L^{k+1}$. By the definition of clusters, every path connecting $x'$ with $y'$ in $G$ must have a vertex in $D_k(v)$. Hence, there must exist also a vertex $z\in C\setminus P_v(x',y')$  such that $d_G(v,z)\le k$, where $P_v(x',y')$ is a part of $C$ between $x'$ and $y'$ containing $v$.  Choosing now  $k=\frac{\widehat{\Delta}(G)+3}{2}$, we ensure  $d_G(x',y')\ge d_G(x,y)-2=2k-2=2(\frac{\widehat{\Delta}(G)+3}{2})-2=\widehat{\Delta}(G)+1>  \widehat{\Delta}(G)$ and, therefore, guarantee the existence of $z\in C\setminus P_v(x',y')$  with $d_G(v,z)\le k=\frac{\widehat{\Delta}(G)+3}{2}$ . Consequently, $\cbc(G)\le \frac{\widehat{\Delta}(G)+3}{2}$.  \qed
%\end{proof}

%\begin{proof}[OLD-OLD]
%Let $C$ be a simple cycle of $G$,  $v$ be an arbitrary vertex of $C$, and assume that $d_G(x,y)=2k+2$ holds for the two $(k+1)$-neighbors $x,y$ of $v$ in $C$. Consider also layering partition $\mathcal{LP}(G,v)$ of $G$ starting at $v$. Since $d_G(x,v)=d_G(y,v)=k+1$, vertices $x$ and $y$ belong to the same layer $L^{k+1}$ of the layering of $G$ with respect to $v$. If $d_G(x,y)>  \widehat{\Delta}(G)$ then, by the definition of $\widehat{\Delta}(G)$, $x$ and $y$ cannot belong to the same cluster from $L^{k+1}$. By the definition of clusters, every path connecting $x$ with $y$ in $G$ must have a vertex in $D_k(v)$. Hence, there must exist also a vertex $z\in C\setminus P_v(x,y)$, where $P_v(x,y)$ is a part of $C$ between $x$ and $y$ containing $v$, such that $d_G(v,z)\le k$.  Choosing now  $k=\frac{\widehat{\Delta}(G)-1}{2}$, we guarantee $d_G(x,y)=\widehat{\Delta}(G)+1>  \widehat{\Delta}(G)$. Consequently, $\cbc(G)\le \frac{\widehat{\Delta}(G)-1}{2}$.  \qed
%\end{proof}

Combining Lemma  \ref{lm:bnc-cbc} and Lemma  \ref{lm:Delta-cbc}  with Theorem \ref{th:bnc-delta-tl}, we get the following result. 

\begin{theorem} \label{th:cbc-tl-delta}
For every graph $G$, the following inequalities hold:  
 $$\frac{\tl(G)-3}{4}\leq\frac{\widehat{\Delta}(G)-2}{4}\leq \bn(G)\le \cbc(G)\leq\frac{\widehat{\Delta}(G)}{2}+1\leq\frac{3 }{2}\tl(G)+1.$$
\end{theorem} 

\begin{corollary} \label{cor:ineq-tl-cbc}
	For every graph $G$,   $\frac{2}{3}(\cbc(G)-1)\leq \tl(G)\leq 4\cdot \cbc(G)+3$. 	
\end{corollary}

%\begin{theorem} \label{th:cbc-tl-delta}
%For every graph $G$, the following inequalities hold:  
%$$\frac{\tl(G)-3}{4}\leq\frac{\widehat{\Delta}(G)-2}{4}\leq \bn(G)\le\cbc(G)\leq\frac{\widehat{\Delta}(G)+3}{2}\leq\frac{3 }{2}(\tl(G)+1).$$
%\end{theorem} 

%\begin{corollary} \label{cor:ineq-tl-cbc}
%For every graph $G$,   $\frac{2}{3}\cbc(G)-1\leq \tl(G)\leq 4\cdot \cbc(G)+3$. 	
%\end{corollary}

Note that, although $\tl(G)=\cbc(G)=1$ for every chordal graph $G$, generally, these two graph parameters are not equal. %$\tl(G)\neq \cbc(G)$. 
Consider a cycle $C_{12k}$ on $12k$ vertices. We have $\tl(C_{12k})=12k/3=4k$ and $\cbc(C_{12k})=12k/4+1=3k+1.$  Furthermore, since $\bn(C_4)=\bn(C_5)=\bn(C_6)=\bn(C_7)=1$ and $\cbc(C_4)=\cbc(C_5)=\cbc(C_6)=\cbc(C_7)=2$, generally, $\bn(G)$ and $\cbc(G)$  are not equal. %differ from each other.  
However, it turns out %we can show 
that, indeed, they are only one unit apart. 

\begin{lemma}   \label{lm:bnc-cbc-1}
	For every graph $G$,  $\bn(G)\le\cbc(G)\le \bn(G)+1$. 
\end{lemma} 

\begin{proof} By Lemma   \ref{lm:bnc-cbc}, we need only to show $\cbc(G)\le \bn(G)+1$.  Let $k:=\bn(G)+1$,  $C$ be a simple cycle of $G$,  $v$ be an arbitrary vertex of $C$, and assume that $d_G(x,y)=2k$ holds for the two $k$-neighbors $x,y$ of $v$ in $C$. By the definition of $\bn(G)$, vertex $v$ has at distance at most $\bn(G)=k-1$ a vertex in every path of $G$ connecting $x$ and $y$. Necessarily, there must exist a vertex $z$ in $C\setminus P_v(x,y)$ such that $d_G(v,z)\le k-1<k$, where $P_v(x,y)$ is a part of $C$ between $x$ and $y$ containing $v$.  %Choosing now  $k=\frac{\widehat{\Delta}(G)-1}{2}$, we guarantee $d_G(x,y)=\widehat{\Delta}(G)+1>  \widehat{\Delta}(G)$. 
Consequently, $\cbc(G)\le k=\bn(G)+1$.  \qed
\end{proof}
%{\color{red} 
% Lemma \ref{lm:Delta-cbc} can be obtained also as corollary of this lemma: $$\cbc(G)\le \bn(G)+1  \leq\frac{\widehat{\Delta}(G)}{2}+1$$
% }
 
We can now better relate $\cbc(G)$ to $\adt(G)$ and $\mc(G)$. By Corollary  \ref{cor:ineq-bnc-mcw}, Theorem  \ref{th:adt-tl-delta} and Lemma    \ref{lm:bnc-cbc-1}, we have the following inequalities. 

\begin{corollary} \label{cor:ineq-adt-mcw--cbc}
	For every graph $G$,   $\frac{\cbc(G)-2}{3}\leq \frac{\bn(G)-1}{3}\leq\adt(G)\leq 4\cdot \bn(G)+2\leq 4\cdot \cbc(G)+2$ and $\frac{\cbc(G)-1}{3}\leq \mc(G)\leq 4\cdot \cbc(G)+2$. 	
\end{corollary}

\commentout{
{\color{red}/* see in Diestel et al. Conjecture 1.6. and Th.1.1. (which is proved by Lm.3.6 and Lm.3.7).    \\
- my Lm \ref{lm:bnc-cbc} is like their conjecture: \\
--- if $\tl$ is large then there is a bad cycle [if no bad cycle then $\tl$ is bounded] \\
- their Th.1.1. (is like my Lm \ref{lm:Delta-cbc}?): \\
--- if there is a bad cycle then $\tl$ is unbounded [if $\tl$ is bounded then no bad cycle] */ \\

if $\tl\le 4/3\cdot \cbc$ is true like for cycles, this gives a chance to get \~ 2-appr for computing $\tl$ ??? }\\
}

We can define one more condition on cycles which also  turns out to be coarsely equivalent to tree-length. Recall that a simple cycle $C$ of $G$ is called {\em geodesic} if for every $x,y\in C$, $d_G(x,y)=d_C(x,y)$. Let us call a simple cycle $C$ of $G$ {\em $\mu$-locally geodesic} if for every two vertices $x,y\in C$, $d_G(x,y)=d_C(x,y)=\ell$ implies $\ell\le \mu$. When $\mu\ge |C|/2$, clearly, every   $\mu$-locally geodesic cycle $C$ is geodesic. If a simple cycle $C$ is not $\mu$-locally geodesic in $G$, then there must exist two vertices in $C$ such that $d_G(x,y)=d_C(x,y)=\mu+1$. We call that side of $C$ between $x$ and $y$ which realizes $d_C(x,y)$ a  {\em side of non-$\mu$-locality} (note that $x$ and $y$ both belong to that side). We say that vertices $v,z$ of $C$ form a {\em $k$-bridge} in $C$ if $d_G(v,z)\le k<d_C(v,z)$. 

Denote by $\bgc(G)$ (call it the {\em ''bridging non-locally geodesic cycles`` constant} of $G$) the minimum $\mu$ ($\mu\ge 1$) such that for every cycle $C$ of $G$ %with $|C|\ge 2k+4$ 
that is not $\mu$-locally geodesic, from every side of non-$\mu$-locality there is a %????$\lceil\mu/2\rceil$-bridge 
$\lfloor \frac{\mu+1}{2}\rfloor$-bridge to other side of $C$. 
%{\color{blue} Notice that this notion generalizes another {\em characteristic  cycle property} of a chordal graph $G$: {\em For every simple cycle $C$ %$(|C|\ge 3)$ 
%of a chordal graph $G$ and every edge $vu\in C$, there is a third vertex in $C$ that forms a triangle with $vu$.}}  

\begin{lemma}   \label{lm:bgc-cbc}
	For every graph $G$,  $\bgc(G)\le 2\cdot\cbc(G)-1$. 
\end{lemma} 

\begin{proof}  Let $\cbc(G)=k$ and $C$ be a cycle of $G$ that is not $(2k-1)$-locally geodesic. Let $x,y\in C$  such that $d_G(x,y)=d_C(x,y)=2k$. Consider vertex $v$ of $C$ between $x$ and $y$ with  $d_C(v,x)=k=d_C(v,y)$. The two $k$-neighbors of $v$ in $C$ are at distance $2k$ from each other in $G$. By the definition of $\cbc(G)$, there is a vertex $z\in C$ such that $d_G(v,z)\le k<d_C(v,z)$. Consequently, $\bgc(G)\le 2\cdot\cbc(G)-1$. 
\qed
\end{proof}

Now, using a layering partition of a graph $G$, we upperbound $\Delta_s(G)$ by a linear function of $\bgc(G)$.

\begin{lemma} \label{lm:ClusterDiam_bgc}
	For every graph $G$  and every vertex $s$ of $G$, $\Delta_s(G)\leq 4\cdot \bgc(G)+6$. In particular, $\widehat{\Delta}(G)\leq 4\cdot
 \bgc(G)+6$ for every graph $G$.
\end{lemma}

\begin{proof}
Let $s$ be an arbitrary vertex of $G$ and $\mathcal{LP}(G,s)$ be the layering partition of $G$ starting at $s$. Consider vertices $x$ and $y$ from a cluster of $\mathcal{LP}(G,s)$ with $d_G(x,y)=\Delta_s(G)$, and let $k:=d_G(s,x)=d_G(s,y)$ and $\mu:=\bgc(G)+1$. Choose also a path $Q$ connecting $x$ and $y$ outside the disk $D_{k-1}(s)$.

Consider arbitrary  shortest paths $P(s,x)$ and $P(s,y)$  of $G$ connecting $s$ with $x$ and $y$, respectively. Let $s'$ be a vertex from  $P(s,x)\cap P(s,y)$ furthest from $s$. We may assume that $k':=d_G(x,s')=d_G(y,s')=k-d_G(s,s')$ is greater than $\mu+\lfloor \frac{\mu}{2}\rfloor$ %\lceil\mu/2\rceil$ 
since, otherwise,  $d_G(x,y)\le d_G(x,s')+d_G(s',y)=2k'\leq 3\mu$, and we are done. 

Pick now vertices $u,w$ in $P(s,x)$ such that  
$d_G(x,u)=\mu+\lfloor \frac{\mu}{2}\rfloor%\lceil\mu/2\rceil
+1\leq d_G(x,s')$, $d_G(u,w)=\mu$ and $d_G(x,w)=\lfloor \frac{\mu}{2}\rfloor%\lceil\mu/2\rceil
+1$. Let $P(u,w)$ be a subpath of $P(s,x)$ between $u$ and $w$. 
%and a shortest path $P(s',y)$ connecting $s'$ with $y$.  
%
A simple cycle $C$ of $G$ formed by $Q$ and subpaths $P(s',x)$ and $P(s',y)$ of $P(s,x)$ and $P(s,y)$, respectively, is not $(\mu-1)$-locally geodesic. Since $\mu-1=\bgc(G)$, for a side $P(u,w)$ of $C$ of non-$(\mu-1)$-locality, we must have a $\lfloor\mu/2\rfloor$-bridge from a vertex $v\in P(u,w)$ to a vertex $z\in C\setminus P(u,w)$. As $d_G(v,z)\le \lfloor\mu/2\rfloor<d_C(v,z)$, vertex $z$ cannot belong to a shortest path $P(s',x)$ (which is a part of $C$ and contains also $v$). If $z$ belongs to  $Q$, then $d_G(s,Q)=k=d_G(s,x)=d_G(s,v)+d_G(v,z)\le d_G(s,w)+\lfloor\mu/2\rfloor<d_G(s,w)+d_G(w,x)=d_G(s,x)=k$, giving a contradiction. Thus, $z$ cannot be in $Q$ either. Consequently, $z\in P(s',y)$. 


Since $d_G(y,z)=k-d_G(s,z)$ and $k=d_G(x,s)\le d_G(s,z)+d_G(z,v)+d_G(v,x)$, necessarily, $d_G(y,z)\le d_G(z,v)+d_G(v,x)\le d(z,v)+d_G(u,x)\le \lfloor\mu/2\rfloor+\mu+\lfloor\mu/2\rfloor+1\le 2\mu+1$. Consequently,  $d_G(x,y)\le d_G(x,v)+d_G(v,z)+d_G(z,y)\leq d_G(x,u)+d_G(v,z)+d_G(z,y)\le \mu+\lfloor\mu/2\rfloor+1+\lfloor\mu/2\rfloor+ 2\mu+1=4\mu+2$. That is,  $\Delta_s(G)\leq 4 (\bgc(G)+1)+2= 4\cdot \bgc(G)+6$. \qed
\end{proof}

Summarizing (using Proposition \ref{prop:dorisb}, Lemma  \ref{lm:Delta-cbc}, Lemma   \ref{lm:bgc-cbc}, and Lemma \ref{lm:ClusterDiam_bgc}), we conclude. 
\begin{theorem} \label{th:bgc-tl-delta}
For every graph $G$, the following inequalities hold:  
 $$\frac{\tl(G)-7}{4}\leq\frac{\widehat{\Delta}(G)-6}{4}\leq \bgc(G)\le 2\cbc(G)-1\leq\widehat{\Delta}(G)+1\leq 3\cdot \tl(G)+1.$$
\end{theorem} 

\begin{corollary} \label{cor:ineq-tl-bgc}
	For every graph $G$,   $\frac{\bgc(G)-1}{3}\leq \tl(G)\leq 4\cdot \bgc(G)+7$. 	
\end{corollary}

From Lemma \ref{lm:Delta-cbc} and Lemma  \ref{lm:ClusterDiam_bgc}, it follows $\cbc(G)\le 2\cdot \bgc(G)+4$. We can show a better bound directly. 

\begin{lemma} \label{lm:cbc_bgc}
	For every graph $G$, $\cbc(G)\leq \bgc(G)+1$. 
\end{lemma}
\begin{proof} Let $k:=\lfloor\frac{\bgc(G)}{2}\rfloor+1+\lfloor\frac{\bgc(G)+1}{2}\rfloor$,  $C$ be a simple cycle of $G$,  $v$ be an arbitrary vertex of $C$, and assume that $d_G(x,y)=2k$ holds for the two $k$-neighbors $x,y$ of $v$ in $C$. We have $d_G(x,y)=d_C(x,y)=2k$. Consider the shortest path $P(x,y)\subset C$ between $x$ and $y$, and let $x',y'$ be two distinct vertices of $P(x,y)$ at distance $\lfloor\frac{\bgc(G)}{2}\rfloor+1$ from $v$. Let $P(x',y')$ be a subpath of $P(x,y)$ between $x'$ and $y'$. Its length $d_G(x',y')$ is $2(\lfloor\frac{\bgc(G)}{2}\rfloor+1)\ge\bgc(G)+1$. 
By the definition of $\bgc(G)$, from non-$\bgc(G)$-locality  side $P(x',y')$ of $C$ there must exist a $\lfloor\frac{\bgc(G)+1}{2}\rfloor$-bridge to other side of $C$, i.e., vertices $v'\in P(x',y')$ and $z\in C\setminus P(x',y')$ such that $d_G(v',z)\le \lfloor\frac{\bgc(G)+1}{2}\rfloor<d_C(v',z)$. The inequality  $d_G(v',z)\le \lfloor\frac{\bgc(G)+1}{2}\rfloor$ guarantees $d_G(v,z)\le d_G(v,v')+d_G(v',z)\le \lfloor\frac{\bgc(G)}{2}\rfloor+1+\lfloor\frac{\bgc(G)+1}{2}\rfloor=k$. The inequality  $d_G(v',z)<d_C(v',z)$  guarantees that $z$ cannot be in $P(x,y)$ (recall that $P(x,y)\subset C$ is a shortest path of $G$ and, hence, for every $u,w\in P(x,y)$, $d_G(u,w)=d_C(u,w)$). 

Consequently, $\cbc(G)\le k=\bgc(G)+1$.  \qed
\end{proof}

%{\color{red} 
% Lemma \ref{lm:ClusterDiam_bgc} can be obtained also as corollary of this lemma: $$\Delta_s(G)\leq 4\cdot \bn(G)+2\leq 4\cdot \cbc(G)+2\leq 4\cdot \bgc(G)+6.$$}
 
Combining Lemma  \ref{lm:bnc-cbc-1}, Lemma  \ref{lm:bgc-cbc}, Lemma  \ref{lm:cbc_bgc}  and Corollary  \ref{cor:ineq-adt-mcw--cbc}, we get the following inequalities. %corollary. 

\begin{corollary} \label{cor:ineq-adt-mcw-cbc--bgc}
	For every graph $G$,  $$\bn(G)\le \cbc(G)\leq \bgc(G)+1\le 2\cdot\cbc(G)\le 2\cdot\bn(G)+2,$$ 
 $$\frac{\bgc(G)-3}{6}\leq \frac{\bn(G)-1}{3}\leq\adt(G)\leq 4\cdot \bn(G)+2\leq 4\cdot \bgc(G)+6,$$ $$\frac{\bgc(G)-1}{6}\leq \mc(G)\leq 4\cdot \bgc(G)+6.$$ 	
\end{corollary}

\commentout{%-------------------------------------
\bigskip
{\center ***}

We can further relax this condition on cycles. 
Denote by $\BGC(G)$ {\color{blue} (call it the {\em bridging non-locally geodesic cycles constant} of $G$)}  the minimum $\mu$ ($\mu\ge 1$) such that every simple cycle $C$ of $G$ %with $|C|\ge 2k+4$ 
that is not $\mu$-locally geodesic has a 
$\lfloor \frac{\mu+1}{2}\rfloor$-bridge (in general position).  

\begin{lemma} \label{lm:BGC_bgc}
	For every graph $G$, $\BGC(G)\leq \bgc(G)\le \frac{3}{2}\BGC(G)$. 
\end{lemma}

\begin{proof} The left inequality is straightforward. To prove the right one, let $\tau:=\BGC(G)$, %\lfloor\frac{\bgc(G)}{2}\rfloor+1+\lfloor\frac{\bgc(G)+1}{2}\rfloor$,  
$\mu:=\lfloor\frac{3}{2}\tau\rfloor$, $C$ be a simple cycle of $G$,  $x,y$ be vertices of $C$ such that $d_G(x,y)=d_C(x,y)=\mu+1$. Let also $P(x,y)\subset C$ be a shortest path of $G$ in $C$ (side of $C$) between $x$ and $y$.  We will show that $C$ has a $\lfloor \frac{\mu+1}{2}\rfloor$-bridge from non-$\mu$-locality  side $P(x,y)$ of $C$ to other side of $C$ by induction on $|C|$ (the base of the induction being $C$ with $|C|=2d_G(x,y)$). 

Let $P(x',y')$ be a subpath of $P(x,y)$ of length $\tau+1$ (see Fig. \ref{fig:last} for an illustration). 
%
By the definition of $\BGC(G)$, there must exist a $\lfloor\frac{\tau+1}{2}\rfloor$-bridge in $C$, i.e., two vertices $v,z\in C$ such that  $d_G(v,z)\le \lfloor\frac{\tau+1}{2}\rfloor<d_C(v,z)$. We can choose such $v$ and $z$ in $C$ that are closest in $G$. 
If that bridge is from non-$\tau$-locality  side $P(x',y')$ of $C$ 
to other side of $C$ (i.e., to $C\setminus P(x',y')$), we are done (since $P(x,y)$ is a shortest path of $G$, both ends of that bridge cannot be in $P(x,y)$, resulting in one end being in $P(x,y)$ and the other one being in $C\setminus P(x,y)$). 

So, we can assume that both $v$ and $z$ are in $C\setminus P(x,y)$. Since $d_G(v,z)<d_C(v,z)$ and we assumed that such $v$ and $z$ are closest in $G$, a cycle $C'$, obtained from $C$ by replacing a side of $C$ between $v$ and $z$ of length $d_C(v,z)$ with a shortest path $P(v,z)$ of $G$ between $v$ and $z$, is simple and satisfies $|C'|<|C|$. By the inductive hypothesis, $C'$ has a  

????

i.e., vertices $v'\in P(x',y')$ and $z\in C\setminus P(x',y')$ such that $d_G(v',z)\le \lfloor\frac{\bgc(G)+1}{2}\rfloor<d_C(v',z)$. The inequality  $d_G(v',z)\le \lfloor\frac{\bgc(G)+1}{2}\rfloor$ guarantees $d_G(v,z)\le d_G(v,v')+d_G(v',z)\le \lfloor\frac{\bgc(G)}{2}\rfloor+1+\lfloor\frac{\bgc(G)+1}{2}\rfloor=k$. The inequality  $d_G(v',z)<d_C(v',z)$  guarantees that $z$ cannot be in $P(x,y)$ (recall that $P(x,y)\subset C$ is a shortest path of $G$ and, hence, for every $u,w\in P(x,y)$, $d_G(u,w)=d_C(u,w)$). 

Consequently, $\cbc(G)\le k=\bgc(G)+1$.  \qed
\end{proof}

} %-------------------------------------

\section{Concluding remarks and open questions} \label{sec:concl}
%{\color{red}
%- improved bounds \\
%- more coarse parameters \\}
We saw that several graph parameters are coarsely equivalent to tree-length. If one of the parameters from the list $\{\tl(G),\tb(G), \itl(G), \itb(G), \Delta_s(G), \rho_s(G),  \td(G), \ad(G), \adt(G), $ $\bn(G), \mc(G), $ $\ph(G),\sh(G),\br(G), \mf(G), \glc(G), \cbc(G), \bgc(G)\}$ is bounded for a graph $G$, then all other parameters are bounded. We saw that, in fact, all those parameters are within small constant factors from each other. %Two immediate questions are in order.  
Two questions are immediate.  
\begin{itemize}
\item[1.]   Can constants in those inequalities be further improved? Is $\br(G)=\tb(G)$  true? 
\item[2.] Are there any other interesting graph parameters that are coarsely equivalent to tree-length? 
\end{itemize}

%- cycle conjecture \\

We can further relax the  ''bridging non-locally geodesic cycles`` condition on cycles. Let $\BGC(G)$ be the minimum $\mu$ ($\mu\ge 1$) such that every simple cycle $C$ of $G$ %with $|C|\ge 2k+4$ 
that is not $\mu$-locally geodesic has a 
$\lfloor \frac{\mu+1}{2}\rfloor$-bridge (in general position). Clearly,  $\BGC(G)\leq \bgc(G)$. 
\begin{itemize}
\item[3.]   Does there exist a constant $c$ such that $\bgc(G)\le c\cdot\BGC(G)$ for every graph $G$? 
\end{itemize}



%- cite Arne's dissertation \\
%- Strong tree-breadth. \\

In \cite{st-tb}, a notion of strong tree-breadth was introduced. The {\em  strong breadth}  of a tree-decomposition $\cT(G)$ of a graph $G$ is the minimum integer $r$ such that for every $B\in V(\cT(G))$ there is a vertex $v_B\in B$ with $B= D_r(v_B,G)$ (i.e., each bag $B$ is equal to a disk  of $G$ of radius at most $r$). The {\em strong tree-breadth}  of $G$, denoted by $\stb(G)$, is the minimum of the strong breadth, over all tree-decompositions of $G$.  Like for the tree-breadth, it is NP-complete to determine if a given graph has strong tree-breadth $r$, even for $r=1$ \cite{st-tb}. See \cite{ArneDisser,st-tb} for other interesting results on tree-breadth and strong tree-breadth. Clearly, $\tb(G)\le \stb(G)$ for every graph $G$. The following question was already asked in \cite{st-tb} and matches very much the topic of this paper.  
\begin{itemize}
\item[4.]   Does there exist a constant $c$ such that $\stb(G)\le c\cdot\tb(G)$ for every graph $G$? 
\end{itemize}


%-  $k$-tree-breadth, $k\ge 2$, is defined in . Can similar results be proven for  $k$-tree-breadth?  \\

In \cite{CTS1}, a notion generalizing tree-width and tree-breadth was introduced. The {\em $k$-breadth} of a tree-decomposition $\cT(G)$  of a graph $G$ is the minimum integer $r$ such
that for each bag $B\in V(\cT(G))$, there is a set of at most $k$ vertices $C_B = \{v_B^1,\dots,v_B^k\}\subseteq V(G)$ such that for each $u\in B$, $d_G (u, C_B) \le r$ holds (i.e., each bag $B$ can be covered with at most $k$ disks of $G$ of radius at most $r$ each; $B\subseteq \cup_{i=1}^k D_r(v_B^i,G)$. The {\em $k$-tree-breadth} of a graph $G$, denoted by $\tb_k(G)$, is the minimum of the $k$-breadth,
over all tree-decompositions of $G$. Clearly, for every graph $G$, $\tb(G) = \tb_1(G)$ and $\tw(G) \le k - 1$ if and only if $\tb_k(G) = 0$ (each vertex in the bags of the tree-decomposition can be considered as a center of a disk of radius 0). 
\begin{itemize}
\item[5.]   It would be interesting to investigate if there exist  generalizations of  some graph parameters considered in this paper that coarsely describe the $k$-tree-breadth.  
\end{itemize}

In a follow-up paper \cite{coarse-pathlength}, we introduce several graph parameters that are coarsely equivalent to path-length. 


% - See https://link.springer.com/article/10.1007/s00454-011-9386-0 p. 196 on quasi-isometricity to trees 

\section*{Data Availability Statement}

Data sharing is not applicable to this article as no datasets were generated or analyzed during the current study.


\begin{thebibliography}{99}
\bibitem{AbDr16} M. Abu‐Ata, F.F. Dragan. Metric tree‐like structures in real‐world networks: an empirical study, {\em Networks} 67(1) (2016), 49-68.

\bibitem{MSstudent} 
M. Al-Saidi, Balanced Disk Separators and Hierarchical Tree Decomposition of Real-Life Networks, MS Thesis, 
Kent State University, 2015. \\ http://rave.ohiolink.edu/etdc/view?acc\_num=kent1429541936  


\bibitem{Farach} 
 R. Agarwala, V. Bafna, M. Farach, B. Narayanan, M.  Paterson, M. Thorup,   On the approximability
of numerical taxonomy (fitting distances by tree metrics), {\em SIAM J. Comput.} 28 (1999), 1073--1085.

\bibitem{Diestel++} 
S. Albrechtsen, R. Diestel, A.-K. Elm, E. Fluck, R.W. Jacobs, P. Knappe, P. Wollan, 
A structural duality for path-decompositions into parts of small radius, arXiv:2307.08497, https://arxiv.org/abs/2307.08497 .


\bibitem{BaDeHaSiZa08} M. Badoiu, E.D. Demaine, M.T. Hajiaghayi, A. Sidiropoulos, M. Zadimoghaddam,   Ordinal embedding: approximation algorithms and dimensionality reduction, In: Proceedings of the 11th International Workshop on Approximation Algorithms for Combinatorial Optimization Problems (APPROX
2008), Boston, MA, USA, August 25–27. Lecture Notes in Computer Science, vol. 5171, pp. 21–34.
Springer, Berlin (2008) 

\bibitem{BaInSi}
M. Badoiu, P. Indyk, and A. Sidiropoulos,
  Approximation algorithms for embedding general metrics into trees,
  {\em SODA}'07, pp.
  512--521.
  
\bibitem{BFGR2017} R. Belmonte, F.V. Fomin, P.A. Golovach, M.S. Ramanujan, 
Metric Dimension of Bounded Tree-length Graphs, {\em 
SIAM Journal on Discrete Mathematics,} 31 (2017), 1217--1243. 
https://doi.org/10.1137/16M1057383
  
\bibitem{BeRa22}
O.  Bendele,  D. Rautenbach,  Additive tree $O(\rho \log n)$-spanners from tree breadth $\rho$,  {\em Theoretical Computer Science},  914 (2022),  39--46.   

\bibitem{BerSey2024} E. Berger, P. Seymour,  Bounded-Diameter Tree-Decompositions, {\em Combinatorica} 44, 659–674 (2024). https://doi.org/10.1007/s00493-024-00088-1 

\bibitem{Bodlaender} H.L. Bodlaender,  Discovering Treewidth, In: Vojtáš, P., Bieliková, M., Charron-Bost, B., Sýkora, O. (eds) SOFSEM 2005: Theory and Practice of Computer Science. SOFSEM 2005. Lecture Notes in Computer Science, vol 3381. Springer, 2005. https://doi.org/10.1007/978-3-540-30577-4\_1

%\bibitem{BrChDr1999} A. Brandst\"adt, V. Chepoi, F.F. Dragan, Distance approximating trees for chordal and dually chordal graphs, {\em  J. Algorithms} 30 (1999), 166--184. 

\bibitem{DBLP:journals/jal/BrandstadtCD99}
A. Brandst{\"a}dt, V. Chepoi, and F.F. Dragan,
  Distance approximating trees for chordal and dually chordal graphs,
  {\em J. Algorithms}, 30 (1999), 166--184.

\bibitem{BrDrchapter}
A. Brandst\"adt, F.F. Dragan, Tree-structured graphs, Handbook of Graph Theory, Combinatorial Optimization, and Algorithms, London, UK: CRC Press, 2016.

\bibitem{BrDrLeLe}
A. Brandst\"adt, F.F. Dragan, H.-O. Le, and V.B. Le, Tree Spanners on Chordal Graphs: Complexity and
Algorithms, {\em Theoretical Computer Science,} 310 (2004), 329-354.

  
\bibitem{Andreas-book} A. Brandst\"adt, V.B. Le, J.P. Spinrad, Graph Classes: A Survey, Monographs on Discrete Mathematics and Applications, Series Number 3, SIAM, 1999. https://doi.org/10.1137/1.9780898719796 
  
\bibitem{Bunem1974}
    {A. Buneman},
    A characterization of rigid circuit graphs,
    {\sl Discrete Math.} 9 (1974) 205-212

\bibitem{CaiDerek} L. Cai, D.G. Corneil, Tree spanners, {\em SIAM J. Discrete. Math.,} 8 (1995), 359--387.

\bibitem{HellyGroups} 
J. Chalopin, V. Chepoi, A. Genevois, H. Hirai, D. Osajda, Helly groups, {\em Geometry and Topology} (in print).

\bibitem{DBLP:journals/ejc/ChepoiD00}
V. Chepoi and F.F. Dragan,
  A note on distance approximating trees in graphs,
  {\em Eur. J. Comb.}, 21 (2000), 761--766.
  
\bibitem{ChDrEsHaVaXi12}
V. Chepoi, F. Dragan, B. Estellon, M. Habib, Y. Vax\`es, Y. Xiang, Additive spanners and distance and routing labeling schemes for delta-hyperbolic graphs, {\em 
Algorithmica} 62 (2012) 713--732. 

\bibitem{ChDrEsHaVa08} V.D. Chepoi, F.F. Dragan, B. Estellon, M. Habib and Y. Vax{\`e}s, Diameters, centers, and approximating trees of $\delta$-hyperbolic geodesic spaces and graphs, 
Proceedings of the 24th Annual ACM Symposium on Computational Geometry (SoCG 2008), June 9-11, 2008, College Park, Maryland, USA, pp. 59--68.

  \bibitem{ChepoiDNRV12}
V. Chepoi, F.F. Dragan, I. Newman, Y. Rabinovich, and Y. Vax{\`e}s,
  Constant approximation algorithms for embedding graph metrics into
  trees and outerplanar graphs, 
  {\em Discrete {\&} Computational Geometry}, 47 (2012), 187--214.


\bibitem{ChepoiEst} V. Chepoi, B. Estellon, Packing and covering delta-hyperbolic spaces by balls, In APPROX-RANDOM, pages 59--73, 2007. 
  
  \bibitem{ChepoiFichet}
V. Chepoi, B. Fichet,  $\ell_{\infty}$-Approximation via subdominants, {\em J. Math. Psychol.} 44 (2000), 600--616. 

  \bibitem{CDN2016}
D. Coudert, G. Ducoffe, and N. Nisse, To Approximate Treewidth, Use Treelength!, {\em SIAM Journal on Discrete Mathematics,} 30 (2016), 1424-1436.
  
  \bibitem{Conn-tw}
R. Diestel, M. M\"uller, Connected Tree-Width, {\em Combinatorica,}  38 (2018), 381--398.

\bibitem{DDGY-spanners} 
Y. Dourisboure, F.F. Dragan, C. Gavoille, C. Yan,
Spanners for bounded tree-length graphs, {\em Theor. Comput. Sci.} 383 (2007), 34--44.


\bibitem{Dorisb2007} Y. Dourisboure, C. Gavoille, Tree-decompositions with bags of small diameter, {\em Discrete Math-
ematics} 307(16) (2007), 2008–2029.
 
\bibitem{dagstuhl}
F.F. Dragan, Short Fill-in with Property $\Pi$, In {\em Open Problems of Dagstuhl Seminar 11182 "Exploiting graph structure to cope with hard problems"},  
A. Brandst\"adt, M.C. Golumbic, P. Heggernes,  R. McConnell (Eds.), In Dagstuhl Reports, Volume 1, Issue 5, pp. 29-46, Schloss Dagstuhl – Leibniz-Zentrum f\"ur Informatik (2011). https://doi.org/10.4230/DagRep.1.5.29


 \bibitem{WG13-Dragan}
F.F. Dragan, Tree-like Structures in Graphs: a Metric Point of View, 39th International Workshop on Graph-Theoretic Concepts in Computer Science (WG 2013), June 19 - 21, 2013, L\"ubeck, Germany, Springer, Lecture Notes in Computer Science, 2013.

 \bibitem{CTS1} F.F. Dragan, M. Abu-Ata, Collective additive tree spanners of bounded tree-breadth graphs with generalizations and consequences, {\em  Theor. Comput. Sci.} 547 (2014), 1--17.

 \bibitem{gap}  F.F. Dragan, H.M. Guarnera, 
Helly-gap of a graph and vertex eccentricities, {\em  Theor. Comput. Sci.} 867 (2021), 68--84.

\bibitem{tree-spanner-appr} F.F. Dragan, E. K\"ohler, 
An Approximation Algorithm for the Tree $t$-Spanner Problem on Unweighted Graphs via Generalized Chordal Graphs, {\em Algorithmica} 69 (2014), 884--905.

\bibitem{coarse-pathlength} F.F. Dragan, E. K\"ohler, 
Graph parameters that are coarsely equivalent to path-length, {\em manuscript in preparation}, 2024.

\bibitem{acyclic-clustering} 
F.F. Dragan, I. Lomonosov, 
On compact and efficient routing in certain graph classes, {\em Discret. Appl. Math.} 155 (2007), 1458--1470.

\bibitem{slimness} F.F. Dragan, A. Mohammed, Slimness of graphs, {\em Discret. Math. Theor. Comput. Sci.} 21(3) (2019). 



\bibitem{ApprTree-DRYan} F.F. Dragan and C. Yan, Distance Approximating Trees: Complexity and Algorithms, 
In Proceedings of the 6th Conference on Algorithms and Complexity (CIAC' 2006), Rome, Italy, May 29-31, 2006, Springer, Lecture Notes in Computer Science 3998, pp. 260--271.

\bibitem{CTS2} F.F. Dragan, C. Yan, 
Collective Tree Spanners in Graphs with Bounded Parameters, {\em  Algorithmica}  57 (2010), 22--43.

\bibitem{CTS3} F.F. Dragan, C. Yan, I. Lomonosov, 
Collective tree spanners of graphs, {\em  SIAM J. Discret. Math.} 20 (2006), 241--260.


\bibitem{NPc-tb} G. Ducoffe, S. Legay,  N. Nisse, On the Complexity of Computing Tree-breadth, {\em Algorithmica} 82 (2020), 1574--1600. https://doi.org/10.1007/s00453-019-00657-7

\bibitem{fid-param}
M. Furuse, K. Yamazaki,  
A revisit of the scheme for computing treewidth
and minimum fill-in, {\em Theoretical Computer Science} 531 (2014) 66--76.


\bibitem{EmekPeleg} Y. Emek and D. Peleg, Approximating minimum max-stretch spanning trees on unweighted graphs, {\em SIAM J.
Comput.,} 38 (2008), 1761--1781.


\bibitem{GKKPP2001} C. Gavoille, M. Katz, N.A. Katz, C. Paul, D. Peleg, Approximate distance labeling schemes, {\em Ninth Annual European
Symposium on Algorithms (ESA),} Lecture Notes in Computer Science, vol. 2161, Springer, Berlin, 2001, pp. 476--488.

\bibitem{Gavri1974}
    { F. Gavril},
    The intersection graphs of subtrees in trees are exactly the chordal graphs,
   {\sl  Journal of Combinatorial Theory, Series B,}  16 (1974) 47-56

\bibitem{GeorPapa2023} 
A Georgakopoulos, P Papasoglu, Graph minors and metric spaces, arXiv:2305.07456, 2023

\bibitem{bal-clique-ch}  J.R. Gilbert, D.J. Rose,  A. Edenbrandt,  A separator theorem for chordal graphs, {\em  SIAM J. Algebr.
Discrete Methods} 5 (1984), 306--313. 

\bibitem{goldman} A.J. Goldman, Optimal center location in simple networks, {\em Transportation Science}, 5 (1971), 212--221.

\bibitem{golumbic} M. C. Golumbic, Algorithmic Graph Theory and Perfect Graphs, Academic Press,
New York, 1980. 

\bibitem{KKP2000} M. Katz, N. A. Katz, D. Peleg, Distance labeling schemes for well-separated graph classes, {\em  17th Annual Symposium on Theoretical Aspects of Computer Science (STACS),} Lecture Notes in Computer Science, vol. 1770, Springer, Berlin, 2000, pp. 516--528.

\bibitem{Kerr}
A. Kerr,  Tree approximation in quasi-trees, {\em Groups Geom. Dyn.} 17 (2023), 1193--1233. arXiv:2012.10741

\bibitem{add-spanner} D. Kratsch, H.-O. Le, H. M\"uller, E. Prisner and D. Wagner, Additive tree spanners, {\em SIAM J. Discrete Math.,}
17 (2003), 332--340.

\bibitem{ArneDisser} A. Leitert, Tree-Breadth of Graphs with Variants and Applications, 2017, PhD thesis, Kent State University, Ohio, USA.  

\bibitem{st-tb}
 A. Leitert,  F.F. Dragan,  On Strong Tree-Breadth. In: {\em Combinatorial Optimization and Applications} (COCOA 2016),  Lecture Notes in Computer Science 10043, Springer, 2016. https://doi.org/10.1007/978-3-319-48749-6\_5

\bibitem{par-conn-p-c}
A. Leitert, F.F. Dragan, Parameterized approximation algorithms for some location problems in graphs, {\em  Theor. Comput. Sci.} 755 (2019), 48--64.

\bibitem{NPc-tl} D. Lokshtanov, On the complexity of computing tree-length, {\em Discrete Applied Mathematics} 158 (2010), 820--827.

\bibitem{MakowskyRotics??}
J. A. Makowsky and U. Rotics, Optimal spanners in partial k-trees, manuscript.
 

\bibitem{manning} J.F. Manning,   Geometry of pseudocharacters, {\em Geom. Topol.} 9 (2005), 1147--1185.

\bibitem{RobSey1986}
    { N. Robertson, P.D. Seymour},
    Graph minors. II. Algorithmic aspects of tree width,
    {\em J. of Algorithms}, 7 (1986)
    309-322.
    
\bibitem{SeyThom1993}
P.D. Seymour, R. Thomas, Graph searching and a min-max theorem for tree-width, {\em Journal of Combinatorial Theory,} Series B, 58 (1993), 22--33, doi:10.1006/jctb.1993.1027   

\bibitem{b-length}    K. Umezawa, K. Yamazaki, Tree-length equals branch-length, {\em  Discrete Mathematics} 309 (2009), 4656--4660.

\bibitem{Walte1972}
    { J.R. Walter},
    Representations of Rigid Cycle Graphs,
    Ph.D. {\sl Wayne State Univ., Detroit} (1972)



\end{thebibliography}
\newpage %\centerline
{\bf Appendix A: Graph parameters considered} 
%\medskip

\begin{table} [htbp]
	\centering
	\begin{tabular}{| l | l |}
		\hline
Notation      & Name \\ 
\hline
\noalign{\smallskip}
 $\tl(G)$, $\itl(G)$ & tree-length, inner tree-length of $G$ \\ 
\hline
\noalign{\smallskip}
 $\tb(G)$, $\itb(G)$ & tree-breadth, inner tree-breadth of $G$ \\ 
\hline
\noalign{\smallskip}
 $\stb(G)$ & strong tree-breadth of $G$ \\  
\hline
\noalign{\smallskip}
 $\br(G)$  & bramble interception radius  of $G$  \\  
\hline
\noalign{\smallskip}
$\sh(G)$, $\ph(G)$  & interception radius for Helly families of connected subgraphs or of  paths of $G$ \\  
\hline
\noalign{\smallskip}	
$\Delta_s(G)$ & cluster-diameter of a layering partition of $G$ with respect to a start vertex $s$ \\
\hline
\noalign{\smallskip}	 
$\Delta(G)$, $\widehat{\Delta}(G)$~~ & minimum and maximum cluster-diameter over all layering partitions of $G$ \\
\hline 
\noalign{\smallskip}	
$\rho_s(G)$ & cluster-radius of a layering partition of $G$ with respect to a start vertex $s$ \\
\hline  
\noalign{\smallskip}	 
$\rho(G)$, $\widehat{\rho}(G)$ & minimum and maximum cluster-radius over all layering partitions of $G$ \\
\hline 
\noalign{\smallskip}	
$\td(G)$ & non-contractive multiplicative distortion of embedding of $G$ into a weighted tree \\
\hline 
\noalign{\smallskip}	
$\ad(G)$ & additive distortion of embedding of $G$ into a weighted tree\\
\hline 
\noalign{\smallskip}	
$\adt(G)$ & additive distortion of embedding of $G$ to an unweighted tree \\
\hline 
\noalign{\smallskip}	
$\bn(G)$, $\BNC(G)~$ & bottleneck constant, overall bottleneck constant of $G$ \\
\hline 
\noalign{\smallskip}	
$\mc(G)$ & McCarty-width of $G$ \\
\hline 
\noalign{\smallskip}	
$\mc_k(G)$ & McCarty-width of order $k$ of $G$ \\
\hline 
\noalign{\smallskip}	
$\mf(G)$ & $K_3$-minor fatness of $G$ \\
\hline 
\noalign{\smallskip}	
$\glc(G)$ &  maximum load over all geodesic loaded cycles in $G$ \\
\hline 
\noalign{\smallskip}	
$\cbc(G)$ & cycle bridging constant of $G$ \\
\hline 
\noalign{\smallskip}	
$\bgc(G)$ & ''bridging non-locally geodesic cycles`` constant of $G$ \\
\hline 
\end{tabular}
%\vspace*{7mm}
%	\caption{Graph parameters.}
	\label{table:parameters}
\end{table}
%}$$ be the maximum load over all geodesic loaded cycles in $G$.

{\bf Appendix B: Bounds known before} 
%\medskip

$$\tl(G) \leq \itl(G)\le 2\cdot\tl(G)  \mbox{~\cite{BerSey2024}~~and~~} \tb(G) \leq \itb(G)\le 2\cdot\tb(G)\mbox{~\cite{Diestel++}}$$ 
$$\rho_s(G) \leq \Delta_s(G) \leq 2\rho_s(G) \mbox{~~ and~~ } \tb(G) \leq \tl(G) \leq 2\tb(G)  \mbox{~~[trivial]}$$ 
$$\Delta(G)\le  \Delta_s(G) \leq\widehat{\Delta}(G)\le 3 \Delta(G) \mbox{~~  \cite{slimness} (see Proposition \ref{prop:ClustDiamAtAnys})}$$ 
	$$\frac{\Delta_s(G)}{3} \leq \tl(G) \leq \Delta_s(G)+1 \mbox{~~  \cite{Dorisb2007} (see Proposition \ref{prop:dorisb})}$$ 
	$$\frac{\rho_s(G)}{3} \leq \tb(G) \leq \rho_s(G)+1 \mbox{~~  \cite{AbDr16,tree-spanner-appr} (see Proposition \ref{prop:Muad-Feodor})}$$ 
	$$\frac{\Delta_s(G)}{3} \leq \tl(G) \leq \td(G) \le 2\cdot \Delta_s(G)+2 \mbox{~~  \cite{AbDr16,ChepoiDNRV12,tree-spanner-appr} (see Proposition \ref{prop:td-tl})}$$ 
$$\frac{\tl(G)-2}{2}\leq \ad(G)\leq 6\cdot \tl(G)\mbox{~~ \cite{BerSey2024} (see Proposition \ref{prop:ad-tl-Seymour})}$$
$$\frac{2}{3}\bn(G)\leq \tl(G)\leq 4\cdot \bn(G)+3 \mbox{~ and ~} 
\ad(G)\le 24\cdot \bn(G)+18 \mbox{~~  \cite{BerSey2024} (see Proposition \ref{pr:Seymour--Manning})} $$ 	
  $$\frac{\tl(G)- 3}{6} \le \mc(G) \le \tl(G) \mbox{~~ \cite{BerSey2024} (see Proposition \ref{prop:sey-mcw})}$$
$$\tl(G)- 1 \le \glc(G) \le 3\cdot\tl(G) \mbox{~~~ \cite{BerSey2024} (see Proposition \ref{prop:sey-lgc})}$$ 
$$\mf(G)\le 2\cdot\bn(G)+1 \mbox{~~and~~} \ad(G)\le 14\cdot\mf(G) \mbox{~~~ \cite{GeorPapa2023}}$$ 

\newpage %\centerline
{\bf Appendix C: Bounds %(re-) proved in 
from this paper} 
\bigskip

{\bf Bounds with $\bn(G)$} (Theorem \ref{th:bnc-delta-tl}, Corollary \ref{cor:ineq-tl-bnc}). 
  $$\frac{\Delta_s(G)-2}{4}\leq \bn(G)= \BNC(G)\leq \frac{3}{2}\Delta_s(G)$$ 
   $$\frac{\tl(G)-3}{4}\leq\frac{\widehat{\Delta}(G)-2}{4}\leq \bn(G)=\BNC(G)\leq\frac{\widehat{\Delta}(G)}{2}\leq\frac{3 }{2}\tl(G)$$
 $$\frac{2}{3}\bn(G)\leq \tl(G)\leq 4\cdot \bn(G)+3$$
%\bigskip\center{***}

{\bf Bounds with $\mc(G)$ (with $\ph(G),\sh(G),\br(G)$)} (Theorem \ref{th:mcw-delta-rho}, Corollary \ref{cor:ineq-bnc-mcw}, Proposition \ref{prop:bramble}).  
 $$\mc(G)\leq \rho(G)\le\rho_s(G)\le \Delta_s(G)\le\widehat{\Delta}(G)\leq 6\cdot \mc(G)$$ 
$$\mc(G)\le \ph(G)\le\sh(G)\le\br(G)\le \tb(G)\leq \tl(G)\leq  \Delta_s(G)+1\leq 6\cdot \mc(G)+1$$ 	
 $$\mc(G)\leq  4\cdot \bn(G)+2 \mbox{~~and~~} \bn(G)\le  3\cdot \mc(G)$$
%\bigskip\center{***}

{\bf Bounds with $\adt(G)$} (Theorem \ref{th:adt-tl-delta}, Corollary \ref{cor:ineq-tl-adt}, Corollary \ref{cor:mcw-adt}).  
 $$\adt(G)\leq \Delta(G)\le \Delta_s(G)\le\widehat{\Delta}(G)\leq 3\cdot \tl(G)\le 6\cdot \adt(G)+3$$ 
$$\adt(G)\leq \Delta_s(G)\le 4\cdot \bn(G)+2 \mbox{~~and~~} \bn(G)\le 3\cdot \adt(G)+1$$

$$\frac{\tl(G)-1}{2}\leq \adt(G)\leq 3\cdot \tl(G) \mbox{~~and~~}  \frac{\adt(G)}{6}\le \mc(G)\le \frac{3\cdot \adt(G)+1}{2}$$
%\bigskip\center{***}

{\bf Bounds with $\mf(G)$} (Theorem \ref{th:mf-tl-mcw}).
$$\frac{\mf(G)}{2}\le \mc(G)\le\tb(G)\le\tl(G)\le \Delta_s(G)+1\le 5\cdot\mf(G)$$  $$\frac{\mf(G)-1}{2}\le \frac{2\cdot\mc(G)-1}{3}\le \adt(G)\le \Delta_s(G)\le 5\cdot \mf(G)-1$$ $$\mf(G)\le 2\cdot\bn(G)+1\le \Delta_s(G)+1\le 5\cdot \mf(G)$$ 

{\bf Bounds with $\cbc(G)$} (Theorem \ref{th:cbc-tl-delta}, Corollary \ref{cor:ineq-tl-cbc}, Corollary \ref{cor:ineq-adt-mcw--cbc}).  
 $$\frac{\tl(G)-3}{4}\leq\frac{\widehat{\Delta}(G)-2}{4}\leq \bn(G)\le \cbc(G)\leq\frac{\widehat{\Delta}(G)}{2}+1\leq\frac{3 }{2}\tl(G)+1$$
%  $$\bn(G)\le\cbc(G)\le \bn(G)+1$$ 
  $$\frac{\cbc(G)-2}{3}\leq \frac{\bn(G)-1}{3}\leq\adt(G)\leq 4\cdot \bn(G)+2\leq 4\cdot \cbc(G)+2$$ $$\frac{\cbc(G)-1}{3}\leq \mc(G)\leq 4\cdot \cbc(G)+2$$
%\bigskip\center{***}

{\bf Bounds with $\bgc(G)$} (Theorem \ref{th:bgc-tl-delta},  Corollary \ref{cor:ineq-tl-cbc}, Corollary \ref{cor:ineq-tl-bgc}, Corollary \ref{cor:ineq-adt-mcw-cbc--bgc}).  
 $$\frac{\tl(G)-7}{4}\leq\frac{\widehat{\Delta}(G)-6}{4}\leq \bgc(G)\le 2\cbc(G)-1\leq\widehat{\Delta}(G)+1\leq 3\cdot \tl(G)+1$$
$$\frac{\bgc(G)-1}{3}\leq   \frac{2}{3}(\cbc(G)-1)\leq \tl(G)\leq 4\cdot \cbc(G)+3\leq 4\cdot \bgc(G)+7$$
  $$\bn(G)\le \cbc(G)\leq \bgc(G)+1\le 2\cdot\cbc(G)\le 2\cdot\bn(G)+2$$ 
 $$\frac{\bgc(G)-3}{6}\leq \frac{\bn(G)-1}{3}\leq\adt(G)\leq 4\cdot \bn(G)+2\leq 4\cdot \bgc(G)+6$$ $$\frac{\bgc(G)-1}{6}\leq \mc(G)\leq 4\cdot \bgc(G)+6$$ 	
%%%%%%%%%%%%%%%%%%%
\end{document}

\begin{lemma} \label{lm:Clusterrad_BDS}
	For every graph $G$  and every vertex $s$ of $G$, $\mc(G)\leq\BDS(G)\le \rho_s(G)$.  In particular, $\mc(G)\leq\BDS(G)\le \rho(G)$ for every graph $G$.
\end{lemma}

\begin{proof}
We need to show only the right inequality.  Let $s$ be an arbitrary vertex of $G$ and $\mathcal{LP}(G,s)=\{L^i_1,\ldots,L^i_{p_i}:i=0,1,\dots,q\}$  be the layering partition of $G$ starting at $s$. Consider also the  layering tree $\Gamma:=\Gamma(G,s)$ of graph $G$ with respect to the layering partition $\mathcal{LP}(G,s)$.  

For a given subset $X\subseteq V$ of vertices of $G$, we can assign to each node $L_i^j$ of $\Gamma$  a weight $w_i^j: =|L_i^j\cap X|$. Clearly, $W:=\sum_{i=0,1,2,\ldots,q, j=1,2,\ldots,p_i}w_i^j$ is equal to  $|X|$.
It is known that every node-weighted tree $T$ with the total weight of nodes equal to $W$ has a node $x$, called a {\em median} of $T$,
such that the total weight of nodes in each subtree of $T\setminus \{x\}$ does not exceed $W/2$. Furthermore, such a node $x$ of $T$ can be found in $O(|V(T)|)$ time \cite{goldman}. 

Let $C=L_i^j$ $(i\in \{0,1,2,\ldots,q\}, j\in \{1,2,\ldots,p_i\})$ be a median node of weighted tree $\Gamma$.   Then, each subtree of $\Gamma\setminus \{C\}$ has total weight of nodes not exceeding $|X|/2$. It is clear from the construction of tree $\Gamma$ that the set $C\subseteq V$ separates in $G$ any two vertices that belong to clusters from different subtrees of $\Gamma\setminus \{C\}$. Consequently, $C$ is a balanced separator of $G$ with respect to $X$ as any connected component of $G[V\setminus C]$ has no more than $|X|/2$ vertices. Note that, given a graph $G$, such a cluster $C$ of layering partition ${\mathcal LP}(G,s)$ of $G$
can be found in linear time in the size of $G$. 

Since there is a vertex $v$ in $G$ such that $C\subseteq D_r(v)$ for $r\le \rho_s(G)$, clearly, $D_r(v)$ is a balanced disk separator of $G$ with respect to $X$. As this holds for an arbitrary subset $X\subseteq V$, we conclude $\BDS(G)\le \rho_s(G)$.
\qed
\end{proof}
