%%%%%%%% ICML 2025 EXAMPLE LATEX SUBMISSION FILE %%%%%%%%%%%%%%%%%

\documentclass{article}

% Recommended, but optional, packages for figures and better typesetting:
\usepackage{microtype}
\usepackage{graphicx}
\usepackage{subfigure}
\usepackage{booktabs} % for professional tables

% hyperref makes hyperlinks in the resulting PDF.
% If your build breaks (sometimes temporarily if a hyperlink spans a page)
% please comment out the following usepackage line and replace
% \usepackage{icml2025} with \usepackage[nohyperref]{icml2025} above.
\usepackage{hyperref}

% Attempt to make hyperref and algorithmic work together better:
\newcommand{\theHalgorithm}{\arabic{algorithm}}

% Use the following line for the initial blind version submitted for review:
\usepackage{arxiv}

\usepackage{natbib}

% If accepted, instead use the following line for the camera-ready submission:
% \usepackage[accepted]{icml2025}

% For theorems and such
\usepackage{amsmath}
\usepackage{amssymb}
\usepackage{mathtools}
\usepackage{amsthm}
\usepackage{algorithm}
\usepackage{algorithmic}

\usepackage{bm}
\newcommand{\gril}{\textsc{Gril}}
\newcommand{\grild}{\textsc{D-Gril}}
\newcommand{\zzgril}{\textsc{Zz-Gril}}
\newcommand{\qzzmod}{quasi zigzag persistence module}
\newcommand{\RKG}{\mathsf{rk}}
\newcommand{\RR}{\mathbb{R}}
\newcommand{\comp}[2]{\mathbf{#1} \leq \mathbf{#2}}
\newcommand{\cX}{\mathcal{X}}
\newcommand{\cG}{\mathcal{G}}
\newcommand{\cL}{\mathcal{L}}
\newcommand{\vp}{\mathbf{p}}
\newcommand{\cH}{\mathcal{H}}
\newcommand{\zz}{\mathbb{ZZ}}
\newcommand{\z}{\mathbb{Z}}
\newcommand{\zop}{\mathbb{Z}^{\mathsf{op}}}
\newcommand{\zopz}{\mathbb{Z}^{\mathsf{op}} \times \mathbb{Z}}
\newcommand{\zopzz}{\zopz \times \z}
\newcommand{\ropr}{\RR^{\mathsf{op}} \times \RR}
\newcommand{\uu}{\mathbb{U}}
\newcommand{\uuu}{\mathbb{UU}}
\renewcommand{\varprojlim}{\mathsf{lim}}
\renewcommand{\varinjlim}{\mathsf{colim}}

% if you use cleveref..
\usepackage[capitalize,noabbrev]{cleveref}

%%%%%%%%%%%%%%%%%%%%%%%%%%%%%%%%
% THEOREMS
%%%%%%%%%%%%%%%%%%%%%%%%%%%%%%%%
\theoremstyle{plain}
\newtheorem{theorem}{Theorem}[section]
\newtheorem{proposition}[theorem]{Proposition}
\newtheorem{lemma}[theorem]{Lemma}
\newtheorem{corollary}[theorem]{Corollary}
\theoremstyle{definition}
\newtheorem{definition}[theorem]{Definition}
\newtheorem{assumption}[theorem]{Assumption}
\theoremstyle{remark}
\newtheorem{remark}[theorem]{Remark}

\newcommand{\propositionofref}{}
\newtheorem*{zpropositionof}{\textbf{Proposition} \propositionofref}
\newenvironment{propositionof}[1]
 {\renewcommand{\propositionofref}{#1}\zpropositionof}
 {\zpropositionof}

 \newcommand{\theoremofref}{}
\newtheorem*{ztheoremof}{\textbf{Theorem} \theoremofref}
\newenvironment{theoremof}[1]
 {\renewcommand{\theoremofref}{#1}\ztheoremof}
 {\ztheoremof}

% Todonotes is useful during development; simply uncomment the next line
%    and comment out the line below the next line to turn off comments
%\usepackage[disable,textsize=tiny]{todonotes}
\usepackage[textsize=tiny]{todonotes}

\newcommand{\tamal}[1]		{{ \textcolor{red} { T:#1}}}
\newcommand{\Shreyas}[1]		{{ \textcolor{orange} {S: #1}}}

\title{Quasi Zigzag Persistence: A Topological Framework for Analyzing Time-Varying Data}
% The \icmltitle you define below is probably too long as a header.
% Therefore, a short form for the running title is supplied here:
% \icmltitlerunning{Quasi Zigzag Persistence: A Topological Framework for Analyzing Time-Varying Data}

\author{%
    Tamal K. Dey \\
   Department of Computer Science \\
   Purdue University \\
   West Lafayette, IN \\
   \texttt{tamaldey@purdue.edu}
    \And
  Shreyas N. Samaga \\
  Department of Computer Science\\
  Purdue University\\
  West Lafayette, IN \\
  \texttt{ssamaga@purdue.edu} \\
  % examples of more authors
  % Affiliation \\
  % Address \\
  % \texttt{email} \\
  % \AND
  % Coauthor \\
  % Affiliation \\
  % Address \\
  % \texttt{email} \\
  % \And
  % Coauthor \\
  % Affiliation \\
  % Address \\
  % \texttt{email} \\
  % \And
  % Coauthor \\
  % Affiliation \\
  % Address \\
  % \texttt{email} \\
}

\begin{document}

% \twocolumn[
% \icmltitle{Quasi Zigzag Persistence: A Topological Framework for Analyzing Time-Varying Data}

% \icmltitle{Multiparameter Zigzag Persistent Homology: A Topological Framework for Analyzing Time-Varying Data with Applications to Sleep-Stage Detection}

% \icmltitle{\textsc{Zz-Gril}: A Stable Topological Invariant for Analyzing Time-Varying Data with Applications in Sleep-Stage Detection}

% It is OKAY to include author information, even for blind
% submissions: the style file will automatically remove it for you
% unless you've provided the [accepted] option to the icml2025
% package.

% List of affiliations: The first argument should be a (short)
% identifier you will use later to specify author affiliations
% Academic affiliations should list Department, University, City, Region, Country
% Industry affiliations should list Company, City, Region, Country

% You can specify symbols, otherwise they are numbered in order.
% Ideally, you should not use this facility. Affiliations will be numbered
% in order of appearance and this is the preferred way.
% ]

% this must go after the closing bracket ] following \twocolumn[ ...

% This command actually creates the footnote in the first column
% listing the affiliations and the copyright notice.
% The command takes one argument, which is text to display at the start of the footnote.
% The \icmlEqualContribution command is standard text for equal contribution.
% Remove it (just {}) if you do not need this facility.

%\printAffiliationsAndNotice{}  % leave blank if no need to mention equal contribution
% \printAffiliationsAndNotice{\icmlEqualContribution} % otherwise use the standard text.
\maketitle
%\def\thefootnote{*}\footnotetext{Author names ordered alphabetically 
% co-first authorscontributed equally to this work
% }\def\thefootnote{\arabic{footnote}}
\begin{abstract}
In this paper, we propose Quasi Zigzag Persistent Homology (QZPH) as a framework for analyzing time-varying data by integrating multiparameter persistence and zigzag persistence. To this end, we introduce a stable topological invariant that captures both static and dynamic features at different scales. We present an algorithm to compute this invariant efficiently. We show that it enhances the machine learning models when applied to tasks such as sleep-stage detection, demonstrating its effectiveness in capturing the evolving patterns in time-evolving datasets.
\end{abstract}

\section{Introduction}

% \Shreyas{What's the problem?}
% \Shreyas{What has been done?}
% \Shreyas{What are we achieving?}
% \Shreyas{How are we achieving?}

%Time series analysis~\cite{tsa,tsa2} 
Time varying data analysis~\cite{tsa,tsa2} has been a fundamental challenge in the machine learning community for a long time, ranging from traditional univariate and multivariate time-series data to more complex structure such as sequence of graphs or point clouds.
% for a long time now. Time series data includes univariate and multivariate time series, sequence of graphs and sequence of point clouds. 
While the traditional time-series data has been handled effectively~\cite{mvtsgnn}, increasing complexity of modern applications has necessitated novel methodologies. Spatiotemporal Graph Neural Networks~\cite{stgcn,oreshkin2021,kan2022,chu2023} have emerged as powerful tools to analyze graph sequences by decomposing the problem into two components: spatial dependency and temporal evolution. Similarly, specialized architectures have been proposed to process sequences of point clouds~\cite{pcseq,fan2021pstnet,huang2021spatio,rempe2020caspr}. However, most of these approaches utilize the geometric information, which is mostly local, among other information potentially missing crucial global patterns in the data. This becomes apparent in scenarios where the overall shape and connectivity of the data carry significant meaning, such as in brain connectivity patterns. Topological methods offer a compelling solution by capturing these global structures that persist across different scales. By augmenting the models with topological information, we can develop models that leverage both local geometric patterns and global topological characteristics for improved performance.
% There are many works on analyzing univariate and multivariate time series data (\Shreyas{cite}). In the recent times, many new methods ahve been proposed to analyze sequences of graphs and sequences of point clouds. Spatiotemporal Graph Neural Networks and their variants have been proposed (\Shreyas{cite}) to analyze sequence of graphs. Spatiotemporal graph neural networks primarily consist of two modules: one to capture the spatial dependency in the data and another to capture the time dependency. Similarly, there are architectures~\Shreyas{cite} to analyze a sequence of point clouds. Most of these architectures utilize the geometric information present in the data, but not particularly the topological information. Capturing topological information provides a global view of the data and thus complements the intrinsic geometric data. 

Recently, topological Data Analysis (TDA) has emerged as a prominent field that can leverage topological
information hidden behind the data into its analysis. Persistent Homology (PH), a cornerstone of TDA, provides a succinct method to extract and represent \emph{multiscale} topological information. This capability has proved transformative in enhancing machine learning models, particularly where topological properties carry relevant information
% PH has seen many applications in enhancing the performance of machine learning models such as
~\cite{Multipers_Kernel_Kerber, chen2019topological,PersLay, PLLay, gabrielsson2020topology, hofer2020graph, Zhao2020PersistenceEG, swenson2020persgnn, Carriere_Multipers_Images,Multipers_landscapes, bouritsas2020improving, togl, topologynet, gefl, dowker,rivet}. 
Time-varying data needs extensions of the classical one-parameter persistence
homology along two directions. First, the temporal component of the
data necessitates another parameter, thus bringing multi parameter persistence homology (MPH) into picture. Second, monotone filtrations usually associated with the standard PH computations cannot accommodate deletions that are required for
processing time-varying data. This necessitates to use zigzag persistent homology (ZPH)
which has been shown to capture dynamic topological features in time-series data~\cite{munch23, gel21, zz_visnet}. 
In effect, we need a combination of MPH and ZPH which allows PH (standard filtration)
along one parameter and ZPH along the other. This requires changes to the structure of the underlying partially-ordered set (poset) to what we call \emph{quasi-zigzag poset} resulting into \emph{Quasi Zigzag Persistent Homology} (QZPH) introducing additional complexity and a new set of challenges.

%Two significant generalizations have advanced the field further: Zigzag Persistent Homology (ZPH) and Multiparameter Persistent Homology (MPH). ZPH extends the framework to capture dynamic topological features in time-series data~\cite{munch23, gel21}, addressing the need to understand evolving patterns in data. MPH tackles an even more challenging scenario of analyzing and understanding data with multiple interdependent scales/parameters - for instance, in weather data, where both temperature and pressure scales might simultaneously influence the underlying phenomenon.

While MPH captures richer topological information than PH, it presents a different set of challenges due to lack of a complete invariant. This has motivated TDA researchers to look for different invariants that, although not complete, capture sufficient topological information
% ZPH is a generalization of PH which is suitable for capturing information from time-series data \Shreyas{cite}. Multiparameter Persistent Homology (MPH) is another generalization of PH where, one now requires a multiscale summary in more than one parameter/scale. For example, in weather data, one might need a \emph{multiscale} summary of the data with respect to both scales/parameters – Temperature and Pressure. MPH captures more information than PH but suffers from the lack of a \emph{complete} invariant. Thus, many MPH methods have been proposed in the ML community which are based on different invariants
~\cite{Multipers_landscapes},~\cite{Multipers_Kernel_Kerber, Carriere_Multipers_Images, diff_signed_barcodes_24, vect_signed_barcodes_23, gril23, dgril24}. The juxtaposition of ZPH to MPH needs the zigzag generalizations of these methods.
%would be strong candidates for analyzing sequential data. 
% changes when we need to consider the zigzag generalizations.

In this paper, we find that, one of these methods 
called \gril{}~\cite{gril23} that
computes a landscape function~\cite{bubenik2015perslandscapes} using generalized ranks of intervals~\cite{GenRankKim21} adapts naturally to the framework of QZPH both conceptually and computationally.
This adaptation leads to our key contribution which we call \zzgril{},
%propose a novel method based on Multiparameter Zigzag Persistent Homology (MPZPH) to analyze sequential data. Our key contribution is \zzgril{}, 
a new topological invariant that extends \gril{}~\cite{gril23} framework to capture multiscale topological information in time-evolving data. 
% We propose an invariant \emph{\zzgril{}}, which is similar in flavor to \gril{} proposed by~\cite{gril23}. \zzgril{} enables us to capture the multiscale topology of time-evolving data aiding in tasks such as detecting sleep stage from ECG data and classifying actions from multivariate time-series data, among many others. 
On the theoretical front, we prove the stability of \zzgril{} and show that the generalized rank of a specific type of subposet, in the setting of QZPH, can be computed by an algorithm presented in \cite{DKM24}. We use this to devise an algorithm to compute \zzgril{}, making this method practical for real-world applications. We demonstrate the practical value of \zzgril{} by experimenting on data from critical healthcare monitoring tasks, such as sleep-stage detection via ECG and multivariate time-series data from other applications. We augment the topological information captured by \zzgril{} with existing machine learning models and show that the addition of topological information improves the performance of the models as exhibited by experimental results in Section~\ref{sec:exp} \footnote{Code is available at \url{https://github.com/TDA-Jyamiti/zzgril}}.

In a very recent paper~\cite{flammer24}, the authors consider a similar setup and develop a method to visualize such data using \emph{spatiotemporal persistence landscapes}. We acknowledge that the setup is similar but there are many differences in the overall results. One main difference is that \cite{flammer24} uses rectangles
as intervals that allow them to use a simple modification of a result in~\cite{DKM24}
whereas we use more general subposets for which we prove Theorem~\ref{thm:compute_zzgril}, which allows us to compute generalized ranks with zigzag modules. Because of the same reason, the
overall algorithm to compute the landscape also becomes different. Additionally,
we give an efficient algorithm to build a quasi zigzag bi-filtration from an
input time-varying data. Finally, we develop a complete ML pipeline that we experiment
with real world time-varying data which is not considered in~\cite{flammer24}.

%both theoretical and experimental. We will point out these differences throughout the text wherever appropriate.}


% \paragraph{Organization of the paper:} We give an overview of the method and review the necessary background on zigzag persistence and 2-parameter persistence in section~\ref{sec:overview}. In section~\ref{sec:zzgril}, we define \zzgril{} and prove the stability of \zzgril{}. We provide the complete description to compute \zzgril{} in section~\ref{sec:alg}, which begins with the describing the construction of quasi zigzag bifiltration in section~\ref{sec:building_qzz_bifil}. In order to test the \zzgril{} framework, we augment \zzgril{} to existing machine learning models and compare the performance on UEA multivariate time-series data, which is a benchmark for time-series analysis. Further, we test the \zzgril{} framework in a targeted application of sleep-stage classification and experiment on ISRUC-S3 dataset. We report the results in section~\ref{sec:exp}. 

% \Shreyas{How is our paper different from Flammer's?}
% \Shreyas{Theoretical result}
% \Shreyas{Their worms are just 1-worms}
% \Shreyas{Applications, we apply to sleep stage classification and multivariate time series classification.}


% \vspace{-0.3cm}
\section{Overview}
% \vspace{-0.2cm}
\label{sec:overview}
 We begin by recalling the definition of a simplicial complex. Given a finite vertex set $V$, a simplicial complex $K=K(V)$ is a collection of subsets of $V$ such that if a subset $\sigma\subseteq V$ is in $K$, then all subsets $\tau\subset \sigma$ are also in $K$. Each element in $K$ with cardinality $k+1$ is called a $k$-simplex or simply
a simplex.
A (non-zigzag) \emph{filtration} $\mathcal F$ is a nested sequence of simplicial complexes indexed by
integers/natural numbers:
$\mathcal{F}: K_0\hookrightarrow K_1\hookrightarrow\cdots\hookrightarrow K_n$ where
the inclusion arrows are all in forward direction.
If these arrows are not all in forward direction, we get
a filtration called \emph{zigzag filtration}
\begin{eqnarray*}
\mathcal{Z}: K_0\hookrightarrow K_1\hookleftarrow \cdots\hookrightarrow K_n.
\end{eqnarray*}
Extending the indexing to two parameters such as a grid of integers
$\mathbb{Z}^2$, we get a bi-filtration (possibly zigzag). We will be interested
in a bi-filtration where the filtration in one direction, say $y$-direction,
is non-zigzag and the filtration in the other, $x$-direction, is zigzag.
For a moving point cloud data (PCD) such filtrations arise naturally. For example, consider
the case in Figure~\ref{fig:ZZ-PCD}. The moving PCD with 3 points are
shown at the bottom row. Along the $y$-direction, we increase the threshold
$\delta>0$ so that any two points are joined by an edge if the distance
between them is no more than $\delta$. Then, along $y$-direction, for every
PCD, we get a non-zigzag filtration whereas along $x$-direction, for every
fixed $\delta$, we get a zigzag filtration. We call this mixing of zigzag
and non-zigzag a \emph{quasi zigzag bi-filtration}.
\begin{figure*}[htbp]
\centerline{\includegraphics[width=0.8\textwidth]{icml2025/figure/ZZ-PCD.pdf}}
\caption{(left) A PCD (bottom row) and a resulting quasi zigzag bi-filtration; (middle) corresponding quasi zigzag persistence module and three intervals (light blue, yellow, green); (right) a rectangle (light yellow) is expanded
for obtaining landscape width.}
\label{fig:ZZ-PCD}
\end{figure*}

From such a quasi zigzag bi-filtration, we obtain a so-called \emph{persistence module}
 by considering the simplicial homology group $H_p(K_i)$ in some dimension
$p$ for every simplicial complex $K_i$. The inclusion maps in zigzag or non-zigzag
filtrations translate to homomorphisms among these homology groups which become
linear maps when homology groups turn into vector spaces under a field coefficient, say $k$. These vector spaces with dimension, say $d$, are often depicted with their isomorphic counterparts of $d$ copies
of $k$, written as $k^d$. We refer to such a persistence
module as a \emph{quasi zigzag} persistence module; 
% Definition~\ref{def:qzz_mod} introduces the concept formally. 
In Figure~\ref{fig:ZZ-PCD} (right), we show the
quasi zigzag persistence module for zeroth homology group $H_0(\cdot)$ that represents
the number of components.

The indexing set of a non-zigzag, zigzag, or quasi zigzag filtration can be
thought of as a poset $P$ where for every pair of
comparables $\mathbf{p}\leq \mathbf{q}$ in $P$, one has the inclusion of complexes 
$K_{\mathbf p}\subseteq K_{\mathbf q}$. A non-zigzag poset can be indexed by integers in $\z$. A zigzag poset that arises in our case is a subposet of the poset $\zz$ defined below.
\begin{definition}[(Quasi) Zigzag Poset]
    Let $\zz$ be the subposet of $\z^{\mathsf{op}} \times \z$ defined as
    \begin{equation*}
        \zz \coloneqq \{ (i,j) \colon i \in \z^{\mathsf{op}} , j \in \{i, i-1\} \}.
    \end{equation*}
 A poset $P$ is called \emph{zigzag} if $P$ is a subposet of $\zz$ and is \emph{quasi zigzag} if it is a subposet of the product $\zz \times \z$ with the product order.    
    \label{def:zzposet}
\end{definition}


Now, we formally define persistence modules over different kind of posets.
%over the poset $\zz\times \z$ (see Definition~\ref{def:qzz_mod}).

%For $P\subseteq \zz$, a functor $M \colon P \to \mathbf{vec}$ is called a \emph{zigzag module}.
\begin{definition}[Persistence module]
    A persistence module $M$ over a poset $P$ is a functor $M: P\rightarrow {\bf vec}$ where $P$ is considered as a category and ${\bf vec}$ is the category of finite
    dimensional vector spaces. Stating more explicitly, $M$ is a
    collection of vector spaces $\{M_\mathbf{p}\}_{\mathbf{p}\in P}$ and linear maps $\{M_{\mathbf{p}\rightarrow \mathbf{q}}\}$ for every comparable $\mathbf{p} \leq \mathbf{q}$ in $P$. It is called a \emph{zigzag module} if $P$ is zigzag and a \emph{quasi zigzag module} if $P$ is quasi zigzag.
    % If
    % $P=\zz \times \z$, we call $M$ quasi zigzag persistence module.
    \label{def:qzzpersmod}
\end{definition}

Next, we introduce the concepts of \emph{connectedness} and  \emph{intervals} in posets which will be used
for further exposition. We say a poset $P$ is \emph{connected} if there is a sequence of points $p=r_1,\ldots, r_t=q$ in $P$ so that
$r_i \leq r_{i+1}$ or $r_{i+1}\leq r_i$ for every $i\in \{1,\ldots,t-1\}$.
We say a subposet $I\subseteq P$ is \emph{convex} if for every pair
$p,q\in I$ and every $p\leq r\leq q$, we have $r$ in $I$.
An \emph{interval} $I$ in a poset $P$ is  
subposet which is both convex and connected. 
An interval $I$ with
$m\geq 0$ minima and $n\geq 0$ maxima is called an $(m,n)$-interval.
In the special case of $\z^2$, every interval has a special form where
the sets of minima and maxima along with other points constitute staircases that can be visualized
as part of the boundary of the interval; see Figure~\ref{fig:l_worm}. 

The main construct upon which we build \zzgril{} involves the so-called \emph{generalized ranks}
~\cite{GenRankKim21} which can be computed efficiently by a zigzag persistence
algorithm~\cite{DKM24}. Let $P$ be a quasi zigzag poset over which a quasi zigzag persistence module $M$ is defined and $I\subseteq P$ be
a subposet. When $M$ is induced by a bi-filtration in QZPH, the generalized
rank of the module restricted to $I$ measures the multiplicity of the homological
features (\# of independent homological classes) that have support over the entire poset $I$. For example, considering $I=P$ (shaded light green), we see that,
in Figure~\ref{fig:ZZ-PCD}(middle), one component survives the entire poset.
Then, its generalized rank for zero-th homology $H_0$ is 1. Similarly, the subposet in the bottom row (light blue) has a generalized rank of 3 for $H_0$ and the subposet
in bottom right square (light yellow) has a generalized rank of 2 for $H_0$.
We formally define generalized rank below.

\begin{definition}[Generalized rank]
    Let $M \colon P \to \mathbf{vec}$ be a persistence module from a finite and connected poset $P$. The restriction of $M$ to a subposet $I$ of $P$, $M|_I$, is the collection of vector spaces $M_{\vp}$ for $\vp \in I$ along with linear maps $M_{\vp \to \mathbf{q}}$ for all comparable $\vp \leq \mathbf{q}$ in $I$. Then, the \emph{generalized rank} of $M$ over $I$ is defined as the rank of the canonical linear map from $\varprojlim M|_I$ to $\varinjlim M|_I$
    \begin{eqnarray*}
        \RKG^M(I) \coloneqq \text{rank} \left ( \varprojlim M|_I \to \varinjlim M|_I \right ).
    \end{eqnarray*}
\end{definition}

 We refer the reader to~\cite{Saunders_Maclane_Cat_Theory} for the definitions of limit, colimit, and the construction of the canonical limit-to-colimit map.

 One of the very useful properties of generalized rank is that it is monotone w.r.t. the inclusion of posets. We use this property to define \zzgril{}.
 
 \textbf{Monotone Property of Generalized Rank.} It follows from the definition that $\RKG^M(I) \leq \RKG^M(J)$ for all $J \subseteq I$, where $I \text{ and } J$ are subposets in $P$.

 To exploit the topological features captured by generalized ranks, we use a cover
 of the quasi zigzag poset with a specific set of subposets called worms. When endowed with $\z^2$ ordering,
 these subposets become $(3,3)$-intervals around a set of chosen
 \emph{centers} with some chosen \emph{widths}. Definition~\ref{def:worm} gives precise definitions of these terms. We expand each of these worms, that is, increase the widths of these $(3,3)$-intervals while
 keeping the centers fixed. The monotonicity of the generalized rank ensures that
 it can only decrease or remain the same by this expansion. In Figure~\ref{fig:ZZ-PCD}(right) a $(1,1)$-interval (rectangle) is expanded maximally to decrease the generalized rank from $3$ to $1$.
 The width for which the rank drops below
 a chosen threshold is taken as the \zzgril{} value at the center of the interval. 
 Defintion~\ref{def:zzgril} makes this concept precise. The set of \zzgril{} values at the chosen centers makes a vector that we use in the learning pipeline.
 
 
% \section{Background}
% \label{sec:background}

% In this section, we briefly review the concepts in zigzag persistent homology. For detailed definitions, refer~\cite{edelsbrunner2010computational,dey_wang_2022_book}. We begin by recalling the definition of a simplicial complex. Given a finite vertex set $V$, a simplicial complex $K=K(V)$ defined on this vertex set is a collection of subsets of $V$ such that if a subset $\sigma\subseteq V$ is in $K$, then all subsets $\tau\subset \sigma$ are also in $K$. Each element in $K$ with cardinality $k+1$ is called a $k$-simplex or simply
% a simplex. A graph is a simplicial complex where $V$ is the vertex set of the graph and $K$ consists of edges and vertices of the graph. A \emph{one-parameter filtration}
% is a collection of simplicial complexes $\{K_\mathbf{x}\}_{\mathbf{x}\in \mathbb{R}}$ indexed by the reals, with the property that $K_\mathbf{x}\subseteq K_{\mathbf{y}}$ for each
% $\mathbf{x}\leq \mathbf{y}$. 
% \tamal{Adapting this definition to a partial order called \emph{zigzag poset}}
% %We note that the structure of partial ordering on the reals is sufficient to define a filtration. Hence, extending this definition to collections of complexes indexed by the \emph{zigzag poset}, 
% we get a \emph{zigzag filtration}. 

% \begin{definition}[Zigzag poset]
%     Zigzag poset $\zz$ is defined as the sub-poset of $\z^2$ given by
%     \begin{equation*}
%         \zz \coloneqq \left \{(i,j) \in \z^2 \colon j \in \{-i, -i+1 \tamal{-i-1?} \} \right \}.
%     \end{equation*}
% \end{definition}

% One can also view the zigzag poset as follows:

%     Zigzag poset $\zz$ has a linear ordering, $p_0, p_1, p_2, \hdots $, of the points in $\zz$ such that for $i \in \z_+$, $p_i \leftrightarrow p_{i+1}$ are the only and all the \emph{immediate pairs} satisfying the condition:

% %Note that $p_i \leftrightarrow p_{i+1}$ in $\zz$ is called an immediate pair if either
% \begin{itemize}
%     \item Either $p_i \leq p_{i+1}$ and there does not exist $q \in \zz$ such that $p_i \leq q \leq p_{i+1}$ or,
%     \item $p_i \geq p_{i+1}$ and there does not exist $q \in \zz$ such that $p_i \geq q \geq p_{i+1}$.
% \end{itemize}

%     A \emph{zigzag filtration} is a collection of simplicial complexes $\{K_{\mathbf{z}}\}_{\mathbf{z} \in \zz}$, where \tamal{we have inclusion $K_{\mathbf{z}} \xhookrightarrow{} K_{\mathbf{w}}$ for every comparable indices}
%     $\mathbf{z} \leq \mathbf{w} \in \zz$.

% A typical zigzag filtration looks like:

% \begin{equation*}
%     \begin{tikzcd}
%         K_1 \xhookrightarrow{} K_2 \xhookleftarrow{} K_3 \xhookrightarrow{} K_4 \xhookleftarrow{} K_5 \hdots
%     \end{tikzcd}
% \end{equation*}


% %A filtration in the $1$-parameter case induces a persistence module, obtained by considering the inclusion-induced linear maps between the vector spaces given by the homology groups of the constitutent simplicial complexes of the filtration. Analogously, we obtain a zigzag persistence module from a zigzag filtration. 

% \tamal{Considering simplicial homology groups under a field coefficient for each of the
% constituent complexes in such a zigzag filtration, we obtain a sequence of vector spaces which are connected by inclusion-induced linear maps. This gives rise to a so-called zigzag persistence module}:

% \begin{definition}[Zigzag persistence module]
%     A zigzag persistence module $\mathcal{Z}$ is defined as a collection of vector spaces $\{V_\mathbf{z}\}_{\mathbf{z}\in \zz}$, with linear maps $V_{\mathbf{z_1}} \to V_{\mathbf{z_2}}$ for comparable $z_1 \leq z_2$ in $\zz$. 
% \end{definition}

% Given a zigzag filtration $\{K_\mathbf{z}\}_{\mathbf{z}\in \zz}$, we consider $H_i(K_\mathbf{z})$, the $i$th homology group of the simplicial complex $K_{\mathbf{z}}$ over a field $F$, say $\mathbb{Z}_2$. Then, for each inclusion $K_{\mathbf{z}} \xhookrightarrow{} K_{\mathbf{w}}$, we get an induced linear map $H_i(K_\mathbf{z}) \to H_i(K_\mathbf{w})$. By this assignment, we get a zigzag persistence module.

% There is a notion of proximity in the space of zigzag persistence modules \tamal{called \emph{interleaving distance}, which is a metric} on the space of zigzag persistence modules.
% %and can be used to characterize how close two zigzag persistence modules are.

% \begin{definition}[Interleaving distance between zigzag modules]
% \tamal{I think we need the definition of interval here and we can postpone the
% definition of interleaving distance.}
    
% \end{definition}
% As mentioned earlier, \zzgril{} uses the concept of generalized rank introduced by~\cite{GenRankKim21}, which we define below formally.

% \begin{definition}[Generalized Rank]
%     Let $M \colon P \to \mathbf{vec}$ be a persistence module from a locally finite and connected poset $P$. The restriction of $M$ to an interval $I$ of $P$, $M|_I$, is the collection of vector spaces $M_{\vp}$ for $\vp \in I$ along with linear maps $M_{\vp \to \mathbf{q}}$ for all comparable $\vp \leq \mathbf{q}$ in $P$. Then, the \emph{generalized rank} of $M$ over $I$ is defined as the rank of the canonical linear map from the limit $\varprojlim M|_I$ to the colimit $\varinjlim M|_I$
%     \begin{equation*}
%         \RKG^M(I) \coloneqq \text{rank} \left ( \varprojlim M|_I \to \varinjlim M|_I \right ).
%     \end{equation*}
% \end{definition}

%  We refer the reader to~\cite{Saunders_Maclane_Cat_Theory} for the definitions of limit, colimit, and the construction of the canonical limit-to-colimit map. Intuitively, generalized rank captures the number of independent topological features supported over the interval $I$.

%  \textbf{Monotone Property of Generalized Rank.} It follows from the definition that $\RKG^M(I) \leq \RKG^M(J)$ for all $J \subseteq I$, where $I \text{ and } J$ are intervals in $P$.
 
% % \begin{remark}
% %     In the special case where $I$ is a rectangle and $P$ is $\RR^2$, $\RKG^M(I)$ is the rank of the linear map $M_{\mathbf{u}\leq \mathbf{v}}$ from the lower left corner $\mathbf{u}$ of the rectangle to the upper right corner $\mathbf{v}$. 
% % \end{remark}



%---------------------------------------------------------------------------






% \vspace{-0.3cm}

\section{\zzgril}\label{sec:zzgril}
% \vspace{-0.2cm}
In this section, we introduce concepts and definitions required for \zzgril. Then, we define \zzgril{} and discuss its theoretical properties. Keep in mind that, even if we do not explicitly state it always, these concepts apply to QZPH setting 
where vector spaces and linear maps
in question are given by simplicial homology groups and homomorphisms between them
induced by the input bi-filtrations.

Consider the poset $\zz \times \z$ with the product order, i.e., $(\mathbf{z_1}, z_2) \leq (\mathbf{w_1}, w_2)$ if $\mathbf{z_1} \leq \mathbf{w_1}$ in $\zz$ and $z_2 \leq w_2$ in $\z$. Note that $\zz \times \z$ is equivalent to $\z^2$ as sets. Thus, every subposet $P\subseteq \zz\times \z$ can be thought as a subposet $P^{\z^2}\subseteq \z^2$ that
is endowed with $\z^2$ ordering. Interestingly, an interval in
$\z^2$ may not remain an interval in $\zz\times \z$ and vice-versa. 

We define a \emph{quasi zigzag bi-filtration} over a quasi zigzag poset $P$ as a collection of simplicial complexes $\{K_{\mathbf{p}}\}_{\mathbf{p} \in P}$, where $K_{\mathbf{p}} \subseteq K_{\mathbf{q}}$ for all comparable $\mathbf{p} \leq \mathbf{q}$.
Akin to 1-parameter and zigzag persistence modules, inclusion-induced linear maps between the homology vector spaces of simplicial complexes in a quasi zigzag bi-filtration define a \qzzmod.

% \begin{definition}[Quasi zigzag bifiltration]
%     Consider the poset $\zz \times \z$ with the product order. A \emph{quasi zigzag bifiltration} $\mathcal{ZZ}$ is defined as a collection of simplicial complexes $\{K_{\mathbf{p}}\}_{\mathbf{p} \in \zz \times \z}$, where $K_{\mathbf{p}} \subseteq K_{\mathbf{q}}$ for all comparable $\mathbf{p} \leq \mathbf{q}$.
% \end{definition}

% One can observe that a quasi zigzag bifiltration is a generalization of a bifiltration but not a zigzag bifiltration because there is zigzagging in only one of the directions. The figure below shows a typical quasi zigzag bifiltration.

% \begin{equation*}
%     \begin{tikzcd}
%         K_{1,n} \arrow[hookrightarrow]{r} & K_{2,n} \arrow[hookleftarrow]{r} &  \cdots \arrow[hookrightarrow]{r} & K_{n,n} \\
%         K_{1,n-1} \arrow[hookrightarrow]{r} \arrow[hookrightarrow]{u} & K_{2,n-1} \arrow[hookleftarrow]{r} \arrow[hookrightarrow]{u} & \cdots \arrow[hookrightarrow]{r}  & K_{n,n-1} \arrow[hookrightarrow]{u} \\
%         \vdots  \arrow[hookrightarrow]{u} & \vdots \arrow[hookrightarrow]{u} & \cdots & \vdots \arrow[hookrightarrow]{u} \\
%         K_{1,2} \arrow[hookrightarrow]{r} \arrow[hookrightarrow]{u} & K_{2,2} \arrow[hookleftarrow]{r} \arrow[hookrightarrow]{u} & \cdots \arrow[hookrightarrow]{r}  & K_{n,2} \arrow[hookrightarrow]{u} \\
%         K_{1,1} \arrow[hookrightarrow]{r} \arrow[hookrightarrow]{u} & K_{2,1} \arrow[hookleftarrow]{r} \arrow[hookrightarrow]{u} & \cdots \arrow[hookrightarrow]{r}  & K_{n,1} \arrow[hookrightarrow]{u}
%     \end{tikzcd}
% \end{equation*}


%\begin{definition}[Quasi zigzag persistence module]\label{def:qzz_mod}
%    A \emph{\qzzmod} $M$ is defined as a collection of vector spaces $\{M_{\vp} \}_{\vp \in \zz \times \z}$ with linear maps $M_{\vp \to \mathbf{q}} \colon M_{\vp} \to M_{\mathbf{q}}$ for all comparable $\vp \leq \mathbf{q}$.
%\end{definition}

% We obtain a \qzzmod{} by considering the homology of each simplicial complex in a quasi zigzag bifiltration $\mathcal{ZZ}$. The group homomorphisms between homology groups are induced by the inclusion maps between simplicial complexes in $\mathcal{ZZ}$.

% \tamal{The following is a repeat and should be avoidable} If the homology is computed with coefficients from a field, say $\z_2$, then the inclusions at the level of simplicial complexes in $\mathcal{ZZ}$ will induce linear maps between the corresponding homology vector spaces. 


We now define a special type of subposet in $\zz\times \z$ which in $\z^2$ is an $\ell$-worm introduced by~\cite{gril23} for $\ell=2$.
%for \gril. 
%we define the notion of $\ell$-worms in $\zz \times \z$.
\begin{definition}[Worm]
    Let $\vp \in \zz \times \z$ and $\delta \in \z$ be given. Let $\boxed{\vp}_\delta$ denote the $\delta$-square centered at $\vp$, i.e., $\boxed{\vp}_\delta \coloneqq \{\mathbf{z} \in \zz\times \z \colon ||\vp - \mathbf{z}||_\infty \leq \delta\}$. Then, a \emph{worm} centered at $\vp$ with \emph{width $\delta$} is defined as the union of $\boxed{\vp}_\delta$ with the two $\delta$-squares $\boxed{\mathbf{q}}_\delta$ centered at points $\mathbf{q} = \mathbf{p}\pm(\delta, -\delta)$ on the off-diagonal line segment. We denote the worm as $\boxed{\vp}_\delta^2$. The superscript denotes the number of $\delta$-squares in the union apart from $\boxed{\vp}_\delta$.
    \label{def:worm}
\end{definition}

We note that computing $|| \cdot ||_\infty$ in $\zz \times \z$ is equivalent to computing it in $\z^2$ using the notion of set equivalence. We can see that a worm turns out to be a $(3,3)$-interval in $\z^2$.
Refer to Figure~\ref{fig:l_worm} for an illustration of a worm.

% \Shreyas{Add a figure each for $\boxed{\vp}_d$ and $\ell$-worm and denote the boundary of $\ell$-worm for Theorem 4.1.}



\begin{definition}[\zzgril]
    Let $M$ be a \qzzmod. Then, the \emph{ZigZag Generalized Rank Invariant Landscape} is a function $\lambda^M \colon \zz \times \z \times \mathbb{N} \to \mathbb{N}$ defined as
    \begin{eqnarray*}
        \lambda^M(\mathbf{p}, k)\coloneqq \sup \left\{\delta \geq 0 \colon \RKG^M \left (\boxed{\mathbf{p}}^2_\delta \right)\geq k\right\}, 
    \end{eqnarray*}
    where $\vp \in \zz \times \z$.
    \label{def:zzgril}
\end{definition}

The basic idea of \zzgril{} is similar to \gril. However, we note that the underlying poset structure is very different. We cover $\zz \times \z$ with subposets of specific shapes (worms), and compute generalized rank over these worms to define the landscape function (\zzgril). Next, we show that \zzgril{} as an invariant remains stable.

\begin{figure}
    \centering
    \includegraphics[scale=0.6]{icml2025/figure/l_worm.pdf}
    \caption{The worm with blue boundary represents a worm centered at $\vp$ with width $\delta=1$. The worm with green boundary represents an expanded worm, centered at $\vp$ with width $\delta=2$.}
    \label{fig:l_worm}
\end{figure}
% \vspace{-0.2cm}
\subsection{Stability of \zzgril}
% \vspace{-0.1cm}
We prove the stability of \zzgril{} by showing that its perturbation is bounded by the interleaving distance between two \qzzmod s. The definition of interleaving distance (see Definition \ref{def:interleaving} in Appendix~\ref{app:proofs}) between two zigzag persistence modules can be extended to \qzzmod s. 
%Interleaving distance is a notion of proximity in the space of persistence modules. 

% \Shreyas{Add definition of interleaving distance for $\zz \times \z$ modules.}
There is an alternate notion of proximity on the space of persistence modules which uses Generalized Ranks computed on all intervals. This is known as \emph{erosion distance}~\cite{GenRankKim21}. We define a distance similar to erosion distance on the space of \qzzmod s based on their generalized ranks computed over worms. For this, we need the notion of $\epsilon$-\emph{thickening}.

Let \textbf{I}$(\zz \times \z)$ denote the collection of all subposets in $\zz \times \z$ such that their corresponding subposets in $\z^2$ are intervals. For $\epsilon \in \z_+$, the \emph{$\epsilon$-thickening} of $I$ is defined as 
\begin{equation*}
    I^\epsilon \coloneqq \left \{ \mathbf{r} \in \zz \times \z \colon \exists\mathbf{q} \in I \text{ such that } ||\mathbf{r} - \mathbf{q}||_\infty \leq \epsilon \right \}.
\end{equation*}

%\Shreyas{change $\varepsilon$ to E and $\mathcal{I}$ to I in $d_\varepsilon$ and $d_\mathcal{I}$}
\begin{definition}
    Let $\mathcal{L}$ denote the collection of all worms in $\zz \times \z$. Let $M$ and $N$ be \qzzmod s. The erosion distance is defined as:
    \begin{equation*}
    \begin{split}
        d_\mathcal{E}^\mathcal{L}(M,N) \coloneqq \inf \limits_{\epsilon \geq 0} 
        \{ & \forall \boxed{\vp}^2_\delta \in \textbf{I}(\zz \times \z), \\
        & \RKG^M\left (\boxed{\vp}^2_\delta\right) \geq \RKG^N\left(\boxed{\vp}^2_{\delta+\epsilon}\right ) \text{ and } \\
        & \RKG^N\left (\boxed{\vp}^2_\delta\right ) \geq \RKG^M\left (\boxed{\vp}^2_{\delta+\epsilon}\right)  \}.
    \end{split}
    \end{equation*}
\end{definition}

% \begin{definition}
%     Let \textbf{Int}$(\zz \times \z)$ denote the collection of all intervals in $\zz \times \z$. Let $M$ and $N$ be two \qzzmod s. The erosion distance is defined as:
%     \begin{equation*}
%     \begin{split}
%         d_\varepsilon(M,N) \coloneqq \inf \limits_{\epsilon \geq 0} 
%         \{ & \forall I \in \textbf{Int}(\zz \times \z), \\
%         & \RKG^M(I) \geq \RKG^N(I^\epsilon) \text{ and } \\
%         & \RKG^N(I) \geq \RKG^M(I^\epsilon)  \}
%     \end{split}
%     \end{equation*}
% \end{definition}


% Restricting this definition to the collection of $\ell$-worms gives a distance on the space of $\ell$-worms. We denote this by $d_\varepsilon^{\mathcal{L}}(M,N)$. 
Note that $\boxed{\vp}^2_{\delta+\epsilon}$ contains the $\epsilon$-thickening of $\boxed{\vp}^2_{\delta}$. 
% Moreover, one can observe that $d_\mathcal{E}^\mathcal{L}(M,N) \leq d_\mathcal{E}(M,N)$, where $d_\mathcal{E}(M,N)$ is the \emph{erosion distance} (Definition~\ref{adef:erosion_dist}) between $M$ and $N$ where the infimum is taken over all subposets of \textbf{I}$(\zz \times \z)$. 
% This is because $d_{\mathcal E}$ is defined over all intervals while $d^\mathcal{L}_\mathcal{E}$ is defined over specific intervals (worms).






\begin{proposition}\label{prop:interleaving}
    Given two \qzzmod s $M$ and $N$, $d_\mathcal{E}^{\mathcal{L}}(M,N) \leq d_\mathcal{I}(M,N)$ where $d_\mathcal{I}$ denotes the interleaving distance between $M$ and $N$.
\end{proposition}

% \textit{Sketch of proof}: $d_\mathcal{I}(M,N) \coloneqq d_\mathcal{I}(E(M), E(N))$ where $E(M)$ denotes the extension functor \Shreyas{define}. Let $I$ be an interval in $\zz \times \z$. Let $\mathfrak{i}\colon \z^3 \to \RR^3$ denote the canonical inclusion map. Let $\mathbb{U} \coloneqq \{ (v_1, v_2, v_3) \in \RR^3 \colon v_1 \geq v_2 \}$ denote the subposet of $\RR^3$. One can see that $\RKG^M(I) = \RKG^{E(M)}(\mathfrak{i}(I))$. This gives us $d_\varepsilon^{\mathcal{L}}(M,N) = d_\varepsilon(E(M),E(N))$ In~\cite{GenRankKim21}, the authors show that $d_\varepsilon(V,W) \leq d_\mathcal{I}(V,W)$ where $V,W \colon \RR^n \to \textbf{vec}$. Thus, we get $d_\varepsilon^{\mathcal{L}}(M,N) = d_\varepsilon(E(M),E(N)) \leq d_\mathcal{I}(E(M),E(N)) = d_\mathcal{I}(M,N)$. 

 Proposition~\ref{prop:interleaving} leads us to Theorem~\ref{prop:stability_zzgril}.

% \begin{proof}
%     Let $I \in \textbf{Int}(\zz \times \z)$. Let $d_\mathcal{I}(M,N) = \epsilon < \infty$. Let $f \colon M \to N(\epsilon)$ and $g \colon N \to M(\epsilon)$ be the interleaving pair. First, we show that $\RKG^N(I) \geq \RKG^M(I^\epsilon)$. To this end, we construct maps $\Tilde{f}, \Tilde{g}$ which make the following diagram commute
    
%     \begin{equation*}
%     \begin{tikzcd}
%         \varprojlim M|_{I^\epsilon} \arrow[r] \arrow[d, "\Tilde{f}"] & \varinjlim M|_{I^\epsilon} \\
%         \varprojlim N|I \arrow[r] & \varinjlim N|I \arrow[u, "\Tilde{g}"]
%     \end{tikzcd}
%     \end{equation*}
%     where the horizontal maps are the canonical limit-to-colimit maps. Define $\Tilde{f}$
% \end{proof}


\begin{theorem}\label{prop:stability_zzgril}
    Let $M$ and $N$ be two \qzzmod s. Given appropriate $\vp, k$, let the \zzgril{} of $M$ and $N$ over $\boxed{\vp}^2_\delta$ be $\lambda^M$ and $\lambda^N$ respectively. Then, 
    \begin{equation*}
        |\lambda^M(\vp,k) - \lambda^N(\vp,k)| = d_\mathcal{E}^{\mathcal{L}}(M,N) \leq d_\mathcal{I}(M,N).
    \end{equation*}
\end{theorem}

% \begin{proof}
    
% \end{proof}

\begin{corollary}[Stability]\label{thm:stability_zzgril}
    Let $M$ and $N$ be two \qzzmod s. Then, 
    \begin{equation*}
        ||\lambda^M - \lambda^N||_\infty \leq d_\mathcal{I}(M,N).
    \end{equation*}
\end{corollary}

We refer the reader to Appendix~\ref{app:proofs} for all the proofs.

% \vspace{-0.2cm}

\section{Algorithm}
\label{sec:alg}
We discuss the details of computing \zzgril{} in this section. We begin by proving that generalized rank over a subposet in $\zz \times \z$ can also be computed by computing the number of full bars in the zigzag filtration along the ``boundary" of the  subposet. To be precise, we introduce the following concepts now. Let $P$ be any finite connected poset. We say a point $q\in P$ \emph{covers} a point $p\in P$, denoted
$p\prec q$ or $q\succ p$, if $p\leq q$ and there is no $r\in P$, $r\not= p,q$, so that $p\leq r\leq q$. A \emph{covering} path $p\sim q$ between $p$
and $q$ is either a set of points $\{p=p_0\prec p_1\prec\cdots\prec p_\ell=q\}$ 
or $\{p=p_0\succ p_1\succ\cdots\succ p_\ell=q\}$.
We define some special points in $P$. The set of minima in $P$
are the points $P_{min}=\{p\in I\ \,|\, \not\exists r\not=p\in P \mbox{ where } r\leq p\}$. Similarly, define the set of maxima $P_{max}$. For any two points
$p,q\in P$, let $p\vee q$ and $p\wedge q$ denote their least upper bound (lub)
and largest lower bound (llb) in $P$ if they exist. 

Now consider a subposet $I\subseteq \zz\times \z$ whose
counterpart $I^{\z^2}$ with $\z^2$ ordering becomes an interval.
Since $I^{\z^2}\subset \z^2$, its points
can be sorted with increasing $x$-coordinates in $\z^2$. 
Let $I^{\z^2}_{min}=\{p_0,\ldots, p_s\}$
and $I^{\z^2}_{max}=\{q_0,\ldots, q_t\}$ be the set of minima and maxima respectively sorted
according to their $x$-coordinates. The \emph{lower fence} and \emph{upper fence}
of $I^{\z^2}$ are defined as the paths
\begin{eqnarray*}
L_I^{\z^2}:=p_0\sim (p_0\vee p_1)\sim p_1\sim (p_1\vee p_2)\sim p_2\sim\cdots\sim p_s\\
U_I^{\z^2}:=q_0\sim (q_0\wedge q_1)\sim q_1\sim(q_1\wedge q_2)\sim q_2\sim\cdots\sim q_t.
\end{eqnarray*}
% \vspace{-0.1cm}
Consider the boundary $B_I^{\z^2}$ comprising $L_I^{\z^2}$, $U_I^{\z^2}$, the paths
$p_0\sim q_0$ and $p_s\sim q_t$ going through the top left and bottom right corner points
of $I^{\z^2}$ respectively. Consider the subposet $I\subseteq \zz\times\z$ and
observe that $I_{min}, I_{max}\subseteq B_I^{\z^2}$ because all other points
have one point below and another above in the $y$-direction.
Let $\partial_L I$, $\partial_U I$, and $\partial I$ be minimal
paths in $BI^{\z^2}$ connecting all minima in $I_{min}$, all maxima in $I_{max}$, and
all minima, maxima together respectively. Drawing an analogy to \cite{DKM24} we call
$\partial I$ a \emph{boundary cap} which is drawn orange in Figure~\ref{fig:worm_qzz_bifil}.

Next, we appeal to certain results in category theory to claim a result
analogous to~\cite{DKM24} that helps computing the generalized ranks with
zigzag persistence modules. Treating $I$ as a category with points in $I$
as objects and relations as morphisms, we get $\partial_L I$ and $\partial_U I$ as
subcategories. Then, we get functors $F_L: \partial_L I\rightarrow I$ induced by inclusion and
%called an \emph{initial functor}. Similarly, we also get a \emph{terminal} functor 
$F_U: I\rightarrow \partial_U I$ induced by projection
which have some special properties.
\begin{definition}
    A connected subposet $P'\subseteq P$ is called \emph{initial} if for every $p\in P\setminus P'$,
    the downset $\downarrow{p}=\{q\in P\,| \, q\neq p~\&~ q\leq p\}$ intersects $P'$
    in a connected poset. Similalrly, $P'$ is called \emph{terminal} if for every $p\in P\setminus P'$, the upset $\uparrow {p}=\{q\in P\,| \, q\neq p~\&~ q\geq p\}$ intersects $P'$
    in a connected poset.
\end{definition}

We call a functor $F: P'\rightarrow P$ for $P'\subseteq P$ \emph{initial} if $P'$ is initial.
Similarly, we call $F: P\rightarrow P'$ \emph{terminal} if $P'$ is terminal.
The following result connecting initial (terminal) functor to the limit (colimit)
is well known, [Chapter 8]~\cite{Riehl14}. 

\begin{proposition}
    Let $M: P\rightarrow {\bf vec}$ be a persistence module and $F_L: P'\rightarrow P$
    and $F_U: P\rightarrow P''$ be initial and terminal functors. Then, there
    is an isomorphism $\phi: \varprojlim{M}\rightarrow \varprojlim{M|_{P'}}$ from the limit of $M$ to the limit of the restricted module $M|_{P'}$. Similarly, we have an isomorphism for colimits, $\psi: \varinjlim{M|_{P''}}\rightarrow \varinjlim{M}$.
    \label{prop:fences}
\end{proposition}

We observe that the functors $F_L: \partial_L I\rightarrow I$ and $F_U: I\rightarrow \partial_U I$  mentioned earlier are initial and terminal respectively allowing us to use Proposition~\ref{prop:fences}.
\begin{proposition}
    $F_L: \partial_L I\rightarrow I$  is an initial functor and $F_U: I\rightarrow \partial_U I$ is a terminal functor.
\end{proposition}
\begin{proof}
    It is sufficient to prove that $\partial_L I$ is initial for $F_L$ and $\partial_U I$ is terminal for $F_U$. We only show that $\partial_L I$ is initial and the proof
    for $\partial_U I$ being terminal is similar.

    Let $p\in I\setminus \partial_L I$ be any point. Let $p^-$ and $p^+$ be two points (if exist) where $p^-\prec p$ and $p^+ \succ p$ in $\z^2$ with $p^-_x < p_x < p^+_x$. By definition of the quasi zigzag poset, either $p^- > p$ and $p^+ > p$, or $p^- < p$ and
    $p^+ < p$ in the poset $\zz\times \z$.
    
    
    In the first
    case, the downset $\downarrow p$ consists of all points $p'\neq p$ that are vertically below $p$, that is, 
    $p'_x=p_x$ and $p'_y< p_y$. This means $\downarrow p$ intersects $\partial_L I$
    in a connected poset $p'_0 <\cdots < p'_k$ where each $p'_i$ has the same $x$-coordinate as $p$.

    In the second case, the downset $\downarrow p$ consists of three sets of points $Y^{p^-}$, $Y^p$, and $Y^{p^+}$  
    that are vertically below $p^-$, $p$, and $p^+$ respectively. Each of $Y^{p^-}$,
    $Y^p$ and $Y^{p^+}$ intersects $\partial_L I$ in a connected poset. They can be
    disconnected as a whole only if there is a point $q\in \partial_L I$ 
    so that
    $p^-_x < q_x < p_x$ or $p_x < q_x < p^+_x$. But, neither is possible because
    $p^-$ and $p^+$ are points in $I$ covered by $p$.    
\end{proof}
Following proposition is immediate.
\begin{proposition}
    $\partial (\partial_L I)=\partial_L I$ and $\partial (\partial_U I)=\partial_U I$.
    \label{prop:boudnarycap}
\end{proposition}

Now, we prove the following result which
extends Theorem 3.1 in~\cite{DKM24} from $2$-parameter persistence modules to the quasi zigzag persistence modules.

\begin{theorem}\label{thm:compute_zzgril}
    Let $M: P\rightarrow {\bf vec}$ be a quasi zigzag persistence module and
    $I\subseteq P$ be a finite subposet so that $I^{\z^2}$ is an interval. Then, 
    $\RKG^M(I)=\RKG^{M}(\partial I)$.
\end{theorem}
\begin{proof}
   Let $r$ denote the map $r: \varprojlim{I}\rightarrow \varinjlim{I}$, and
   $\phi: \varprojlim{I}\rightarrow \varprojlim{\partial_L I}$ and
   $\psi: \varinjlim{\partial_U I}\rightarrow \varinjlim{I}$ be the isomorphisms guaranteed by Proposition~\ref{prop:fences}. Observe that $\RKG^M(I)= \mathrm{rank}(\psi^{-1}\circ r \circ \phi^{-1})$. Now let $r'$ denote the map
   $r': \varprojlim{\partial I}\rightarrow \varinjlim{\partial I}$. Consider the isomoprhisms 
   $\phi': \varprojlim{\partial I}\rightarrow \varprojlim{\partial (\partial_L I)}$ and
   $\psi': \varinjlim{\partial(\partial_U I)}\rightarrow \varinjlim{\partial I}$ again guaranteed by Proposition~\ref{prop:fences}. Then,
   $\RKG^M(\partial I)=\mathrm{rank}(\psi'^{-1}\circ r'\circ \phi'^{-1})$. But,
   $\psi^{-1}\circ r \circ \phi^{-1}=\psi'^{-1}\circ r'\circ \phi'^{-1}$ because of
   Proposition~\ref{prop:boudnarycap}.
\end{proof}

With this background, the pipeline for computing \zzgril{} in the QZPH framework
can be described as follows.
Suppose that we are given sequential data (sequence of point clouds, sequence of graphs, multivariate time series) with vertex-level correspondences between consecutive time steps. First, we build a quasi zigzag bi-filtration out of this data where time
increases in $x$-direction and the threshold for constructing complexes increases
in $y$-direction. The algorithm for building this bi-filtration out of raw data
is described in section~\ref{sec:building_qzz_bifil}
where we make certain choices to make it efficient.
Each simplicial complex in the bi-filtration is indexed by a finite grid $G$ in $\z^2$ because $\zz \times \z$ is equivalent to $\z^2$ as sets. 
We sample a set of center points $S$ from $G$ and compute \zzgril{} (Definition~\ref{def:zzgril}) at each of these center points. Taking advantage of Theorem~\ref{thm:compute_zzgril}, we compute \zzgril{} by computing
the zigzag filtration along the boundary cap (a path) of a worm centering each point
in $S$ and then computing the number of full bars in the corresponding zigzag persistence module obtained by applying the homology functor.
\begin{figure}
    \centering
    \includegraphics[scale=0.6]{icml2025/figure/qzz_bifiltration.pdf}
    \caption{A worm $I$ (shaded) on a quasi zigzag bi-filtration where the direction of the inclusion on horizontal arrows are shown at the bottom and on the vertical arrows on left. The worm centered at (3,3) has width 1. The part of the boundary colored green is $\partial_LI$ connecting the minima shown as green points and the part in purple is $\partial_UI$ connecting the maxima shown as purple points. The \emph{boundary cap} is shown in orange which runs parallel to the boundary for a portion of it. Refer to Figure~\ref{fig:zz_bdry} for the zigzag filtration along the boundary cap of the worm. A sequence of graphs with $T=3$ time steps is shown with the corresponding graph filtrations: $\mathcal{F}_{t_1}, \mathcal{F}_{t_2}, \mathcal{F}_{t_3}$ each with $L=5$ levels. The filtration of the unions are  encircled by ovals. Zigzag filtration at the topmost level $\mathcal{Z}_L$ is shown in a rectangular box.}
    
    \label{fig:worm_qzz_bifil}
\end{figure}

%we now describe the setting in which we compute \zzgril. We assume that we are given sequential data (sequence of point clouds, sequence of graphs, multivariate time series) with vertex-level correspondences between consecutive time steps. The high-level idea is as follows. The first step in the computation is to build the quasi zigzag bifiltration. Each simplicial complex in the bifiltration can be indexed by a finite grid $G$ in $\z^2$ because $\zz \times \z$ is equivalent to $\zz^2$ as sets. We sample $s$ center points from $G$ and compute \zzgril{} at each of these $s$ center points. To compute \zzgril{} at a center point, we need to compute the zigzag filtration along the boundary of the $\ell$-worm centered at that point and finally, compute the number of full bars in the corresponding zigzag persistence module.

% In order to build a quasi zigzag bifiltration, we need a collection of simplicial complexes. These simplicial complexes can be obtained from a sequence of point clouds (Figure~\ref{fig:ZZ-PCD}), a sequence of graphs (Figure~\ref{fig:worm_qzz_bifil}) or multivariate time-series data. We give a detailed description of building the quasi zigzag bifiltration using sequential graph data as an example (Figure~\ref{fig:worm_qzz_bifil}) in the following subsection.

A sequence of point clouds (Figure~\ref{fig:ZZ-PCD}) or graphs (Figure~\ref{fig:worm_qzz_bifil}) or multivariate time-series data (refer section~\ref{sec:exp} for converting multivariate time-series to a sequence of point clouds or graphs) gives us a collection of simplicial complexes. Every simplex in each simplicial complex is assigned a weight which is derived from the input. We build a quasi zigzag bi-filtration from this collection of simplicial complexes.

% \vspace{-0.2cm}

\subsection{Building Quasi Zigzag Bi-filtration}
% \vspace{-0.1cm}
\label{sec:building_qzz_bifil}
We explain the algorithm by considering an example of sequential graph data, as shown in Figure~\ref{fig:seq_of_graphs}. We assume that each graph has edge-weights. In the top row of Figure~\ref{fig:seq_of_graphs}, notice that the first two graphs can not be linked by an inclusion in any direction, i.e., neither is a subgraph of the other. This is because, there is an inclusion of an edge $(v_2,v_3)$ as well as a deletion $(v_1,v_4)$. We circumvent this problem by clubbing all inclusions together followed by all deletions. This is equivalent to considering the union of the two graphs as the intermediary step and then deleting the edges which are not present in the previous graph. Refer to the bottom row of Figure~\ref{fig:seq_of_graphs}, where the intermediary union graphs are shown along with the original graphs. 


A sequence of $T$ number of graphs is converted into a zigzag filtration $\mathcal{Z}_{L}$ of length $2T-1$ by the above procedure because every consecutive pair of graphs introduces a union in between.
For each graph in the zigzag filtration thus obtained, we construct its \emph{graph filtration} based on edge-weights; see e.g.~\cite{dey_wang_2022_book, edelsbrunner2010computational}. We ensure that the length of each graph filtration remains the same by considering the sublevel sets of exactly $L$ levels.
% \tamal{The previous sentence is vague.} 
The zigzag filtration at the highest level, $\mathcal{Z}_{L}$, can be pulled back to each lower level $l$. This ensures that we get a zigzag filtration $\mathcal{Z}_l$ at each level $1 \leq l < L$ of the filtration. This gives us a quasi zigzag bi-filtration. Refer to Figure~\ref{fig:worm_qzz_bifil} for an illustation of a quasi zigzag bi-filtration.
% \tamal{need to talk?}
 
While implementing this procedure, we need not compute the union graphs explicitly saving both storage and time because all necessary information is already present in the corresponding component graphs. We use the following procedure to efficiently build the quasi zigzag bi-filtration.

Construct the graph filtration $\mathcal{F}_{t_i}$ of each of the $T$ graphs in the sequential graph data, where $1\leq t_i \leq T$. Let $\sigma$ be a simplex that is inserted at the level $l$ in the filtration $\mathcal{F}_{t_i}$. 
% \tamal{level $l_i$, time $t_i$ have not
% been defined precisely and may create confusion.} 
We have three possible scenarios for $\sigma$ in $\mathcal{F}_{t_{i+1}}$:
    \begin{enumerate}
    \item $\sigma$ is inserted in $\mathcal{F}_{t_{i+1}}$ at $m$ for some $m < l$: In this case, $\sigma$ needs to be added on all the horizontal inclusion arrows $K_{t_i, w} \xhookrightarrow{} K_{t_i, w} \cup K_{t_{i+1}, w}$ for $m \leqslant w < l$.
    
    \item $\sigma$ is inserted in $\mathcal{F}_{t_{i+1}}$ at $m$ for some $m \geqslant l$: In this case, $\sigma$ needs to be added on all the horizontal inclusion arrows $K_{t_{i+1}, w} \xhookrightarrow{} K_{t_i, w} \cup K_{t_{i+1}, w}$ for $l \leqslant w < m$.
    
    \item $\sigma$ is not present in $\mathcal{F}_{t_{i+1}}$. In this case, we treat it as getting inserted at the maximum level $L$ in $\mathcal{F}_{t_{i+1}}$ and hence, will be added by the inclusion $K_{t_{i+1}, L} \xhookrightarrow{} K_{t_i, L} \cup K_{t_{i+1}, L}$.
\end{enumerate}

We repeat this procedure for the pair $\mathcal{F}_{t_i}$ and $\mathcal{F}_{t_{i-1}}$.
\begin{figure}
    \centering
    \includegraphics[scale=0.7]{icml2025/figure/seq_of_graphs.pdf}
    \caption{The top row shows a typical input to the \zzgril{} framework as a sequence of graphs. This sequence has 3 graphs. Observe that consecutive graphs cannot be linked by an inclusion in either direction because neither is a subgraph of the other. In order to circumvent this problem, we consider the unions of consecutive graphs as an intermediary step, as shown in the bottom row.}
    \label{fig:seq_of_graphs}
\end{figure}

Since we are working with an integer grid, we can simulate the quasi zigzag bi-filtration such that we have the standard filtration $\mathcal{F}_{t_i}$ at even $x$-coordinates and at odd $x$-coordinates, we have the standard filtration of the unions, without explicitly storing the unions.

% Note that we treat the uncommon simplices in $K_{t_i, \alpha_{max}}$ and $K_{t_{i+1}, \alpha_{max}}$ to be created at the maximal value of alpha.
% \Shreyas{Should I add an algorithm environment description for building the bifiltration?}

% \begin{algorithm}[h!]
% \caption{\textsc{BuildQuasiZigzagBifiltration}}\label{alg:build_qzz_bifil}
% \begin{algorithmic}
%    \STATE {\bfseries Input:} $\{G_t\}_{t=1}^n: \text{Sequence of weighted graphs}$
%    \STATE {\bfseries Output:} Quasi zigzag bifiltration $\mathcal{ZZ}$
%    \STATE {\bfseries Initialize:} Splx-birth-times $B_t \gets \{ \}$ for all $1\leq t\leq n$
%    \FOR{t = 1 to n}
%    \STATE    $\mathcal{F}_t\gets$ lower-star-filt$(G_t)$
%    \STATE $B_t\gets $
%    \ENDFOR
%    % \WHILE{$d_{min} \leq d_{max}$}
%    % \STATE $d \gets  (d_{min} + d_{max}) / 2 $
%    % \STATE $I \gets \boxed{\vp}^\ell_d$ 
%    % \STATE $r \gets$ \textsc{ComputeRank($\mathcal{F}, I$)}
   
%    % \IF {$r \geq  k$}
%    %      \STATE $\lambda \gets d$
%    %      \STATE $d_{min} \gets d+1$
%    %  \ELSE
%    %      \STATE $d_{max} \gets d-1$ 
%    %  \ENDIF
%    %  \ENDWHILE
    
% \STATE \textbf{return} {$\mathcal{ZZ}$}

% \end{algorithmic}
% \end{algorithm}  

% \textbf{Correctness.} In order to prove correctness of this procedure, we need to show that there exists a bijection $f \colon K_{i,j} \to K'_{i,j}$, for all $1 \leq i \leq T, 1 \leq j \leq L$, where $K_{i,j}$ is the simplicial complex constructed according to the procedure described above and $K'_{i,j}$ is the simplicial complex in the quasi zigzag bifiltration.

% \begin{proof}
%     The statement is obvious if $i$ is an even integer. This is because $K_{i,j}$ and $K'_{i,j}$ are both simplicial complexes in the filtration of one of the graphs in the initial sequence of graphs. Now, if $i$ is an odd integer, then $K'_{i,j} \coloneqq K'_{i-1,j} \cup K'_{i+1, j}$. 
    
%     We show that $K_{i,j} \subseteq K'_{i,j} \subseteq K_{i,j}$ for all $1 \leq i \leq T, 1 \leq j \leq L$ . Let us fix the indices $i_0$ and $j_0$, where $i_0$ is an odd integer. 
    
%     Now, to see $K_{i_0, j_0} \subseteq K'_{i_0, j_0}$, let $\sigma \in K_{i_0, j_0}$. This implies that $\sigma$ is included in at least one of the horizontal arrows $K_{i, j} \xhookrightarrow{} K_{i_0, j_0}$, where $i \in \{i_0 -1, i_0 + 1 \}$ and $1 \leq j \leq j_0$. Let $\sigma$ be included on the horizontal arrow $K_{i_1, j_1} \xhookrightarrow{} K_{i_0, j_0}$, for some $i_1 \in \{i_0 -1, i_0 + 1 \}$ and $1 \leq j_1 \leq j_0$. Thus, $\sigma \in K_{i_1, j_1} = K'_{i_1, j_1} \subseteq K'_{i_1, j_0} \subseteq K'_{i_0, j_0}$. 

%     Now to see $K'_{i_0, j_0} \subseteq K_{i_0, j_0}$, let $\sigma \in K'_{i_0, j_0}$. Thus, $\sigma \in K'_{i_1, j_0}$ for $i_1 \in \{i_0 -1, i_0 + 1\}$. Let $\sigma$ be born at level $j_1$ in the filtration $\mathcal{F}_{i_1}$. Then, according to the procedure described above, \Shreyas{complete this with implicit and explicit inclusions.}
    
% \end{proof}

\textbf{Correctness.} To prove the correctness of the procedure, we need to show 
the following: For $1\leq i,i' \leq T$ and $1\leq j,j' \leq L$, let $K_{i,j}$ denote the simplicial complex in the quasi zigzag filtration built according to the procedure mentioned above and $K'_{i,j}$ is the simplicial complex in the quasi zigzag bi-filtration built by considering unions explicitly. We show
that the new simplices inserted by the inclusion $K_{i,j} \xhookrightarrow{} K_{i',j'}$ are the same as the ones inserted by $K'_{i,j} \xhookrightarrow{} K'_{i',j'}$. 

% We refer the reader to Appendix~\ref{app:proofs} for a proof.
\begin{proof}
    The statement is immediately true for the inclusions $i'= i, j' = j+1$. This is because the new simplices in the upward ($y$-direction) inclusions are the ones born at the level $j+1$ in the input family of filtrations. Now, we prove the statement for the case where $i' = i \pm 1, j' = j$, i.e., the horizontal ($x$-direction) inclusions/deletions. Note that we have inclusions from $K_{i,j} \xhookrightarrow{} K_{i',j'}$ only when $i$ is an even integer and $i' = i \pm 1$. We prove it for the case $i' = i + 1, j' = j$. The proof for the other case is similar. Let $\sigma$ be a simplex added on the inclusion $K_{i,j} \xhookrightarrow{} K_{i+1,j}$. According to the procedure described above, it means that $\sigma$ is born at some level $j_1 \leq j$ in the filtration $\mathcal{F}_{i+2}$ and is not born before level $j$ in the filtration $\mathcal{F}_i$. Hence, $\sigma$ will be newly added on the arrow $K'_{i,j} \xhookrightarrow{} K'_{i+1,j}$ because $\sigma \in K'_{i+2,j}$ and $\sigma \notin K'_{i,j}$. This is because $K_{i,j} = K'_{i,j}$ for all even integers $i$, as they are the simplicial complexes part of the input filtrations $\mathcal{F}_i$. By the same arguments, all the newly added simplices on inclusions $K'_{i,j} \xhookrightarrow{} K'_{i+1,j}$ will be newly added simplices on $K_{i,j} \xhookrightarrow{} K_{i+1,j}$.
\end{proof}



\begin{figure*}
    \centering
    \includegraphics[width=\textwidth]{icml2025/figure/bdry_zz.pdf}
    \caption{This is the zigzag filtration $\mathcal{Z}_{bdry}$ along the boundary cap of the worm shown in Figure~\ref{fig:worm_qzz_bifil}.}
    \label{fig:zz_bdry}
\end{figure*}

\begin{figure*}
    \centering
    \includegraphics[scale=0.55]{icml2025/figure/zz_gril_pipeline.pdf}
    \caption{Pipeline used for the experiments.}
    \label{fig:pipeline}
\end{figure*}

% \vspace{-0.2cm}
\subsection{Computing \zzgril}
% \vspace{-0.1cm}
% \tamal{I suggest we get rid of $\ell$ and call only worms for $\ell=3$ fixed.}
Given this quasi zigzag bi-filtration, we obtain a \qzzmod{} $M$ by considering the homology $\{H_p(K_{t_i, \alpha_j})\}_{i,j}$ of each simplicial complex. Given a sampled center point $\vp$ and the value of rank $k$, we need to compute $\lambda^M(\vp, k)$. Thus, we need to compute the maximal width of the worm such that the value of generalized rank of $M$ over the worm is at least $k$. We do a binary search over all the possible widths of the worms to arrive at such maximal width for each tuple $(\vp, k)$. To compute the generalized rank of $M$ over a worm, we need the zigzag filtration along the boundary cap of the worm, $\mathcal{Z}_{bdry}$. We utilize the stored information about which simplices to add/delete along the horizontal and vertical arrows in order to build $\mathcal{Z}_{bdry}$. Then, we compute the number of full bars in the zigzag persistence module corresponding to $\mathcal{Z}_{bdry}$ using an efficient zigzag algorithm proposed in~\cite{fzz} and use it in the binary search routine. The pseudocode of this idea is shown in Algorithm~\ref{alg:CompZZGRIL}.




\begin{algorithm}[h!]
\caption{\textsc{Compute\zzgril}}\label{alg:CompZZGRIL}
\begin{algorithmic}
\small 
   \STATE {\bfseries Input:} $\mathcal{ZZ}: \text{Quasi zigzag bi-filtration}, k\geq 1, \vp$
   \STATE {\bfseries Output:} $\lambda(\vp, k)$:  \zzgril{} value at point $\vp$ for  fixed $k$
   \STATE {\bfseries Initialize:} $\delta_{min} \gets 1, \delta_{max} \gets \text{len}(\mathcal{F})$, $\lambda \gets 1$ 
   \WHILE{$\delta_{min} \leq \delta_{max}$}
   \STATE $\delta \gets  (\delta_{min} + \delta_{max}) / 2 $
   \STATE $I \gets \boxed{\vp}^2_\delta$;~~ $r \gets$ \textsc{ComputeRank($\mathcal{ZZ}, I$)}
   
   \IF {$r \geq  k$}
        \STATE $\lambda \gets \delta$;~~ $\delta_{min} \gets \delta+1$
    \ELSE
        \STATE $\delta_{max} \gets \delta-1$ 
    \ENDIF
    \ENDWHILE
    
\STATE \textbf{return} {$\lambda$}

\end{algorithmic}
\end{algorithm}   


% \Shreyas{Add a figure showing a quasi zigzag bifiltration, $\mathcal{Z}_{bdry}$}





% \Shreyas{I think that the following section can be omitted. Ask Prof. Dey.}
% \subsection{Building $\mathcal{Z}_{bdry}$}
% While building $\mathcal{Z}_{bdry}$ at every point $(x,y)$ of the integer grid, we have four cases:
% \begin{enumerate}
%     \item Vertical inclusion: The included simplices are clear if the $x$-coordinate is an even number. If the $x$-coordinate is an odd number, the included simplices will be a union of simplices included on the vertical inclusions $(x-1,y) \xhookrightarrow{} (x-1,y+1)$ and $(x+1,y) \xhookrightarrow{} (x+1,y+1)$. 
%     \item Vertical deletion: The deleted simplices are clear if the $x$-coordinate is an even number. If the $x$-coordinate is an odd number, the deleted simplices will be a union of simplices included on the vertical inclusions $(x-1,y-1) \xhookrightarrow{} (x-1,y)$ and $(x+1,y-1) \xhookrightarrow{} (x+1,y)$. 
%     \item Horizontal inclusion: This case exists only if $x$ is an even number. The included simplices are precisely those that are stored on the horizontal inclusion arrows as described in Section~\ref{sec:building_qzz_bifil} which correspond to $(x,y) \xhookrightarrow{} (x+1,y)$.
%     \item Horizontal deletion: This case exists only if $x$ is an odd number. The deleted simplices are precisely those that are stored on the horizontal inclusion arrows as described in Section~\ref{sec:building_qzz_bifil} which correspond to $(x+1,y) \xhookrightarrow{} (x,y)$.
% \end{enumerate}

% This way, we build the zigzag filtration along the boundary of the $\ell$-worm.


% \vspace{-0.1cm}
\section{Experiments}
\label{sec:exp}

In this section, we report the results of testing \zzgril{} on various datasets. We begin by giving a detailed description about the experimental setup. Then, we give a brief description of the datasets, followed by the experimental results. We use the benchmark UEA multivariate time-series~\cite{ueamvts} datasets to test \zzgril{} on multivariate time series data to show that \zzgril{} can be applied to datasets from various domains. Further, we test \zzgril{} on a targeted application of sleep-stage classification by performing experiments on ISRUC-S3~\cite{isruc} dataset. In all these experiments, we augment the topological information captured by \zzgril{} to one of the specifically tailored machine learning methods on the respective datasets and compare the performance. For each case, we select the machine learning model which has the highest performance to truly test and highlight the value of the topological information added by \zzgril{} to an already high-performing specifically tailored model. 

\vspace{-0.2cm}
\subsection{Experimental Setup}
 Each data instance is a multivariate time-series. We convert multivariate time-series data into a quasi zigzag bi-filtration. The \zzgril{} framework takes in a quasi zigzag bi-filtration as input and provides a topological signature for the sequence as output. We augment the machine learning model with this topological information and train the model for classification. The framework is shown in Figure~\ref{fig:pipeline}.

\begin{table*}
    \centering
    \small
    \begin{tabular}{cccccccc}
    \toprule
         \textbf{Methods}&  \textbf{Accuracy}&  \textbf{F-1 Overall}&  \textbf{F-1 Wake}&  \textbf{F-1 N-1}&  \textbf{F-1 N2}&  \textbf{F-1 N3}& \textbf{F-1 REM}\\
         \midrule
         SVM~\cite{alickovic2018}&  73.3&  72.1&  86.8&  52.3&  69.9&  78.6& 73.1\\
         RF~\cite{memar2017}&  72.9&  70.8&  85.8&  47.3&  70.4&  80.9& 69.9\\
         MLP+LSTM~\cite{dong2017}&  77.9&  75.8&  86.0&  46.9&  76.0&  87.5& 82.8\\
         CNN+BiLSTM~\cite{supratak2017}&  78.8&  77.9&  88.7&  60.2&  74.6&  85.8& 80.2\\
         CNN~\cite{chambon2018}&  78.1&  76.8&  87.0&  55.0&  76.0&  85.1& 80.9\\
         ARNN+RNN~\cite{phan2019}&  78.9&  76.3&  83.6&  43.9&  79.3&  87.9& 86.7\\
 STGCN~\cite{jia2020}& 79.9& 78.7& 87.8& 57.4& 77.6& 86.4&84.1\\
 MSTGCN~\cite{jia2021}& 82.1& 80.8& \textbf{89.4}& 59.6& 80.6&\textbf{89.0}&85.6\\
 STDP-GCN~\cite{stdpgcn}& 82.6& 81.0& 83.5& \textbf{62.9}& 83.1& 86.0&90.6\\
 STDP-GCN + \zzgril{}& \textbf{83.8}& \textbf{81.1}& 88.6& 58.1&\textbf{85.4} &82.7 &\textbf{90.9}\\
 \bottomrule
    \end{tabular}
    % \vspace{-0.2cm}
    \caption{ Accuracy and F-1 scores of augmenting \zzgril{} to STDP-GCN and testing on ISRUC-S3 sleep classification dataset. We report the overall accuracy, the overall F-1 score and sleep-stage-wise F-1 scores for each model.}
    \label{tab:sleep_classi_results}
    % \vspace{-0.2cm}
\end{table*}


% \begin{table*}[!htb]
% \centering
% \resizebox{\textwidth}{!}{
% \begin{tabular}{@{}c|cc|cc@{}}
% \toprule
% \textbf{Dataset} & \textbf{MLP}& \textbf{MLP + \zzgril}& \textbf{RF}& \textbf{RF + \zzgril}\\ \midrule
% ArticularyWordRecognition& & & &  \\
% AtrialFibrillation& &  & & \\
% BasicMotions& & &   & \\
% CharacterTrajectories& & & & \\
% ERing& & & & \\
% FingerMovements& &  &  &                   \\
% Handwriting&& &  &  \\
% Heartbeat&&  &  & \\
% JapaneseVowels& & & & \\ 
%  LSST& & & &\\
%  Libras& & & &\\
%  MotorImagery& & & &\\
%  NATOPS& & & &\\
%  RacketSports& & & &\\
%  SelfRegulationSCP1& & & &\\
%  SelfRegulationSCP2& & & &\\ \bottomrule
% \end{tabular}}
% \caption{Test Accuracy scores on UCR multivariate time-series datasets; augmenting with \zzgril{} increases the classification performance for most of the datasets.}
% \label{tab:results}
% %\vspace{-0.2cm}
% \end{table*}


\begin{table*}
    \centering
    \resizebox{\textwidth}{!}{
    \begin{tabular}{cccccccccccc}
    \toprule
         \textbf{Dataset/Methods}&  ED-1NN&  DTW-1NN-I&  DTW-1NN-D&  MLSTM-FCN&  ShapeNet&  WEASEL+MUSE&  TapNet&   OS-CNN&MOS-CNN& TodyNet &TodyNet+\zzgril\\
         \midrule
         FingerMovements&  0.550&  0.520&  0.530&  0.580&  0.589&  0.490&  0.530&   0.568&0.568&  0.570&\textbf{0.660}\\
 Heartbeat& 0.620 & 0.659& 0.717& 0.663& \textbf{0.756}& 0.727& 0.751& 0.489& 0.604&\textbf{ 0.756}&\textbf{0.756}\\
 MotorImagery& 0.510&0.390 &0.500 &0.510 &0.610 &0.500 &0.590 &0.535 &0.515 &0.640 & \textbf{0.660}\\
 NATOPS& 0.860 & 0.850& 0.883& 0.889& 0.883& 0.870& 0.939& 0.968& 0.951& \textbf{0.972}&0.961\\
 SelfRegulationSCP2& 0.483 & 0.533& 0.539& 0.472& 0.578& 0.460& 0.550& 0.532& 0.510& 0.550&\textbf{0.600}\\
 \bottomrule
    \end{tabular}
    }
    \caption{Acccuracy of augmenting \zzgril{} to TodyNet and testing on UEA Multivariate Time Series Classification Datasets.}
    \label{tab:mvts_results}
\end{table*}

Given a sample of multivariate time-series data with $m$ time-series, we splice each time-series into a \emph{sequence of time-series}, each of length $w$. This splicing is done by a moving window of width $w$, where consecutive windows have an overlap of $\lambda$. We calculate the Pearson correlation coefficient~\cite{pcc} between time-series in each window. We construct a graph for each window, where each time-series (of length $w$) is a node, and edges with the Pearson correlation values as weights connect them. Thus, we get a complete graph with $m$ nodes. We select the top $k$ percentile of these edges to build the final graph in each window. This way we obtain a sequence of graphs. The topological information of this sequence of graphs encodes the evolution of correlation between time series. Alternately, we can also track the evolving time-series by converting it into a sequence of point clouds. From the sequence of time-series described above, we consider each time-series of length $w$ as a point in $\RR^w$. Thus, if we have $m$ time series, we have $m$ points in $\RR^w$ in each window. Thus, we get a sequence of point clouds in $\RR^w$. This sequence of point clouds tracks the evolution of each time-series, and a Vietoris-Rips like construction on this point cloud tracks the evolution of the interaction between time-series. Refer to Figure~\ref{fig:pipeline} for an illustration. Refer to Appendix~\ref{app:exp} for hyperparameters.

% \Shreyas{Add mvts to seq of pcds description.}

\begin{figure}
    \centering
    \includegraphics{icml2025/figure/zz_gril_vis.pdf}
    \caption{Heatmap of \zzgril{} on the FingerMovements dataset. \zzgril{} is computed at 36 center points, which is plotted as a 6x6 grid. The top row represents three samples with label 1 and the bottom row represents three samples with label 0. We can see the similarity in the \zzgril{} signatures between samples of the same class and clear differences between samples of different classees.}
    \label{fig:zz_gril_vis}
\end{figure}

% \vspace{-0.2cm}
\subsection{UEA Multivariate Time Series Classification}
% \vspace{-0.1cm}
UEA Multivariate Time Series Classification (MTSC)~\cite{ueamvts} archive comprises of real-world multivariate time series data and is a widely recognized benchmark in time series analysis. The UEA MTSC collection encompasses a diverse range of application domains such as healthcare (ECG or EEG data), motion recognition (recorded using wearable sensors). See Table~\ref{tab:app:dataset_info} in Appendix~\ref{app:exp}. 

The datasets are preprocessed and split into training and testing sets. We use these train-test splits and compare the performance, by augmenting \zzgril{} to TodyNet,~\cite{todynet} with existing methods~\cite{li2021,tang2022,karim2019,schafer2017}. For this set of experiments, we convert the multivariate time series into a sequence of point clouds and compute \zzgril{}. We choose 5 datasets which have at least 7 multivariate time series. This ensures that each point cloud, in the sequence of point clouds, has at least 7 points, giving meaningful topological information. Refer to Figure~\ref{fig:zz_gril_vis} for a visualization of \zzgril{} on FingerMovements dataset processed as a sequence of point clouds. We report the results in Table~\ref{tab:mvts_results}. We can see that augmenting \zzgril{} improves the performance in most cases.
% We test the \zzgril{} framework on UEA Multivariate Time Series datasets~\cite{ueamvts}. We use the framework of TodyNet~\cite{todynet} as the machine learning model. We augment the model with \zzgril{} signatures and compare the performance. 

\begin{table}
    \centering
    \begin{tabular}{ccc}
    \toprule
        \textbf{Dataset} & \textbf{Time (\zzgril{} point clouds)} & \textbf{Time (\zzgril{} graphs)}  \\
        \midrule
        FingerMovements &569s &138s \\
        % Heartbeat & &893s \\
        NATOPS & 413s & 68s \\
        SelfRegulationSCP2 & 153s  & 27s \\
        \bottomrule
    \end{tabular}
    \vspace{0.2cm}
    \caption{Computation times for \zzgril{}. The values represent the computation times for both training and testing splits. All the experiments were performed on Intel(R) Xeon(R) Gold 6248R CPU and NVIDIA Quattro RTX 6000 GPU.}
    \label{tab:comp_times}
\end{table}

 We perform another set of experiments where we compare the performance of converting the multivariate time series into sequence of point clouds versus converting it into sequence of graphs for extracting topological information. We report the results in Table~\ref{tab:app_exp_res}. We can see that both of these approaches are viable choices. 

 \begin{table}[htbp]
    \centering
    \begin{tabular}{ccc}
    \toprule
        \textbf{Dataset} & \textbf{TodyNet + \zzgril{} point clouds} & \textbf{TodyNet + \zzgril{} graphs}\\
        \midrule
         FingerMovements& 0.660 & 0.680\\
         NATOPS &0.961 & 0.945 \\
         SelfRegulationSCP2 &0.600 & 0.594\\
         \bottomrule
    \end{tabular}
    \vspace{0.2cm}
    \caption{Comparison of converting multivariate time series as a sequence of point clouds versus converting it as a sequence of graphs to extract the topological information.}
    \label{tab:app_exp_res}
\end{table}


% \vspace{-0.2cm}
\subsection{Sleep Stage Classification}
% \vspace{-0.1cm}
We use ISRUC-S3 dataset, which is a part of the ISRUC (Iberian Studies and Research on Sleep) Sleep Dataset~\cite{isruc}. 
% The dataset contains high-quality polysomnography (PSG) recordings collected from subjects with diverse clinical profiles. ISRUC is widely regarded as a benchmark for sleep stage classification.
ISRUC-S3 contains PSG recordings from 10 subjects. Each recording includes multiple physiological signals, such as: EEG, ECG. The dataset is annotated with sleep stage labels for each epoch (30 second window). There are 5 sleep stage labels: Wake (W), N1, N2, N3 (non-REM stages) and REM. We use  STDP-GCN~\cite{stdpgcn} as the machine learning model to augment. In this experiment, we convert the time series into sequence of graphs and compare the performance
% We apply the \zzgril{} framework on ISRUC S3 dataset~\cite{isruc}. We use STDP-GCN~\cite{stdpgcn} as the machine learning model to augment the topological information to, and compare the performance. 
in Table~\ref{tab:sleep_classi_results}. We can see that augmenting \zzgril{} increases both the accuracy and the overall F-1 score.

We would like to clarify that the aim, for both sets of experiments, is primarily to show that an increase in accuracy signifies that \zzgril{} captures meaningful topological information which can be used to improve the existing models. 

We report the computation times in Table~\ref{tab:comp_times}. We can see from the table that \zzgril{} is, indeed, practical to use.






%---------------------------------------------------------------------------------------------------------------


\section{Conclusion}
% \vspace{-0.1cm}
In this paper, we proposed QZPH as a framework to capture both static and dynamic topological features in time-varying data. We proposed \zzgril{}, a stable and computationally efficient topological invariant to address the challenges of integrating MPH and ZPH. Through applications in various domains, including sleep-stage detection, we showed that augmenting machine learning models with \zzgril{} improves the performance. These results highlight the potential of integrating topological information to address complex challenges while analyzing time-evolving data. The differentiability of \zzgril{} is an interesting research avenue. Further, it would also be interesting to study how other multiparameter methods adapt to the QZPH framework.

\section{Acknowledgement}
This work is partially supported by NSF grants DMS-2301360 and CCF-2437030. We acknowledge the discussion with Michael Lesnick who pointed out the theory
about initial/terminal functors in the context of limits and colimits.

% In the unusual situation where you want a paper to appear in the
% references without citing it in the main text, use \nocite
% \nocite{langley00}
\newpage
\bibliography{ref}
\bibliographystyle{icml2025}


%%%%%%%%%%%%%%%%%%%%%%%%%%%%%%%%%%%%%%%%%%%%%%%%%%%%%%%%%%%%%%%%%%%%%%%%%%%%%%%
%%%%%%%%%%%%%%%%%%%%%%%%%%%%%%%%%%%%%%%%%%%%%%%%%%%%%%%%%%%%%%%%%%%%%%%%%%%%%%%
% APPENDIX
%%%%%%%%%%%%%%%%%%%%%%%%%%%%%%%%%%%%%%%%%%%%%%%%%%%%%%%%%%%%%%%%%%%%%%%%%%%%%%%
%%%%%%%%%%%%%%%%%%%%%%%%%%%%%%%%%%%%%%%%%%%%%%%%%%%%%%%%%%%%%%%%%%%%%%%%%%%%%%%

%%%%%%%%%%%%%%%%%%%%%%%%%%%%%%%%%%%%%%%%%%%%%%%%%%%%%%%%%%%%%%%%%%%%%%%%%%%%%%%
%%%%%%%%%%%%%%%%%%%%%%%%%%%%%%%%%%%%%%%%%%%%%%%%%%%%%%%%%%%%%%%%%%%%%%%%%%%%%%%
% \newpage
\appendix
\onecolumn

\section{Formal definitions and proofs}
\label{app:proofs}
There is a notion of proximity on the space of zigzag modules in terms of the \emph{interleaving distance}. In~\cite{zz_stability}, the authors define interleaving distance between two zigzag modules by including them into $\RR^{\mathsf{op}} \times \RR$-indexed modules.


The interleaving distance on $\RR^n$-indexed persistence modules is an extension
of the same distance defined for $\RR$-indexed persistence modules~\cite{ChazalCGGO09}. We briefly recall the definition here. We refer the readers to~\cite{zz_stability} for additional details.

\begin{definition}[$u$-shift functor]
    The $u$-shift functor $(-)_u \colon \mathbf{vec}^{\RR^n} \to \mathbf{vec}^{\RR^n}$, for $u \in \RR^n$, is defined as follows:
    \begin{enumerate}
        \item For $M \in \mathbf{vec}^{\RR^n}$, $M_u$ is defined as $M_u(x) = M(x+u)$ for all $x \in \RR^n$ and $M_u(x_1 \leq x_2) = M(x_1 + u \leq x_2 + u)$ for all $x_1 \leq x_2 \in \RR^n$, 
        \item Let $M,N \in \mathbf{vec}^{\RR^n}$. Let $F \colon M \to N$ be a morphism. Then, the corresponding morphism $F_u \colon M_u \to N_u$ is defnied as $F_u(x) = F(x+u) \colon M_u(x) \to N_u(x)$ for all $x \in \RR^n$.
    \end{enumerate}
\end{definition}

\begin{definition}[$\epsilon$-interleaving]
    Let $M, N \in \mathbf{vec}^{\RR^n}$. Let $\epsilon \in [0,\infty)$ be given. We will denote $(-)_{\bm{\epsilon}}$ to be the shift functor corresponding to the vector $\bm{\epsilon} = \epsilon(1,1,\hdots,1)$. We say $M$ and $N$ are $\epsilon$-interleaved if there are natural transformations $F \colon M \to N_{\bm{\epsilon}}$ and $G \colon N \to M_{\bm{\epsilon}}$ such that 
    \begin{enumerate}
        \item $G_{\bm{\epsilon}} \circ F = \varphi_M^{2\bm{\epsilon}} $,
        \item $F_{\bm{\epsilon}} \circ G = \varphi_N^{2\bm{\epsilon}}$,
    \end{enumerate}
    where $\varphi_M^u \colon M \to M_u$ is the natural transformation whose restriction to each $M(x)$ is the linear map $M(x \leq x+ u)$ for all $x \in \RR^n$.
\end{definition}

\begin{definition}[Interleaving distance]
    Let $M, N \in \mathbf{vec}^{\RR^n}$. The interleaving distance $d_\mathcal{I}(M,N)$ between $M$ and $N$ is defined as
    \begin{equation*}
        d_\mathcal{I}(M,N) \coloneqq \inf \{\epsilon \geq 0 \colon M \text{ and } N \text{ are } \epsilon\text{-interleaved} \}
    \end{equation*}
    and $d_\mathcal{I}(M,N) = \infty$ if there exists no interleaving.
    \label{def:interleaving}
\end{definition}

We need the following two definitions to define interleaving distance between two zigzag modules. 

\begin{definition}[Left Kan Extension]
    Let $P$ and $Q$ be two posets. Let $F \colon P \to Q$ be a functor. Let $P[F \leq q]$ denote the set $P[F \leq q] \coloneqq \{p \in P \colon F(p) \leq q \}$. Given a persistence module $M \colon P \to \mathbf{vec}$, the \emph{left Kan extension} of $M$ along $F$ is a functor $\text{Lan}_F(M) \colon Q \to \mathbf{vec}$ given by 
    \begin{equation*}
        \text{Lan}_F(M)(q) \coloneqq \varinjlim M|_{P[F \leq q]},
    \end{equation*}
    along with internal morphisms given by the universality of colimits.
\end{definition}

\begin{definition}[Block Extension Functor~\cite{zz_stability}]
    Let $\uu \subset \ropr$ denote the poset $\mathbb{U} \coloneqq \{(a,b) \colon a \leq b\}$. Let $\mathfrak{i} \colon \zz \to \RR^{\mathsf{op}} \times \RR$ denote the inclusion. Let $(-)|_\uu \colon \mathbf{vec}^{\ropr} \to \mathbf{vec}^{\uu}$ denote the restriction. Then, the \emph{block extension functor} $E \colon \mathbf{vec}^{\zz} \to \mathbf{vec}^\uu$ is defined as 
    \begin{equation*}
        E \coloneqq (-)|_\uu \circ \text{Lan}_\mathfrak{i}(\circ).
    \end{equation*}
\end{definition}

% \Shreyas{Add a figure of block intervals.}

\begin{definition}[Interleaving distance on zigzag modules]
    Let $M,N$ be two zigzag modules. Then
    \begin{equation*}
        d_\mathcal{I}(M,N) \coloneqq d_\mathcal{I}(E(M), E(N)).
    \end{equation*}
\end{definition}


We extend this definition of interleaving distance to \qzzmod s.  Let $\uuu \subset \ropr \times \RR$ be the analog of $\uu$, i.e, $\uuu \coloneqq \{ (a,b,c) \colon a\leq b \}$. Let $\Tilde{\mathfrak{i}} \colon \zz \times \z \to \ropr \times \RR$ denote the inclusion. Then, define $\Tilde{E} \coloneqq (-)|_\uuu \circ \text{Lan}_{\Tilde{\mathfrak{i}}}(\circ)$, analogous to the block extension functor.

\begin{definition}
    Let $M, N$ be two \qzzmod s. Then, the interleaving distance between $M$ and $N$ is given by 
    \begin{equation*}
        d_\mathcal{I}(M,N) \coloneqq d_\mathcal{I}(\Tilde{E}(M), \Tilde{E}(N)).
    \end{equation*}
\end{definition}

\begin{lemma}\label{alem:interval}
    $\Tilde{E}$ sends interval modules on $\zz \times \z$ to block interval modules.
\end{lemma}
This is analogous to Lemma 4.1 in~\cite{zz_stability}.

% \begin{lemma}
%     Let $M$ be a p.f.d \qzzmod. If $M \cong \bigoplus I_{\langle b_i,d_i\rangle}$, then $\Tilde{E}(M) \cong \bigoplus \Tilde{E}(I_{\langle b_i,d_i\rangle})$.
% \end{lemma}

\begin{lemma}\label{alem:decomposition}
    Let $M$ be a p.f.d \qzzmod. If $M \cong \bigoplus_{k\in K} I_k$, then $\Tilde{E}(M) \cong \bigoplus_{k \in K} \Tilde{E}(I_k)$, where $K$ is an indexing set.
\end{lemma}

\begin{proof}
    The functor $\text{Lan}_{\Tilde{\mathfrak{i}}}$ preserves direct sums~\cite{Saunders_Maclane_Cat_Theory}, as it is left-adjoint to the canonical restriction functor. The restriction functor also preserves direct sums. These two facts combined with Lemma~\ref{alem:interval} give the result.
\end{proof}

\begin{definition}
\label{adef:erosion_dist}
    Let \textbf{I}$(\zz \times \z)$ denote the collection of all subposets in $\zz \times \z$ such that their corresponding subposets in $\z^2$ are intervals. Let $M$ and $N$ be two \qzzmod s. The erosion distance is defined as:
    \begin{equation*}
    \begin{split}
        d_\mathcal{E}(M,N) \coloneqq \inf \limits_{\epsilon \geq 0} 
        \{ & \forall I \in \textbf{I}(\zz \times \z), \\
        & \RKG^M(I) \geq \RKG^N(I^\epsilon) \text{ and } \\
        & \RKG^N(I) \geq \RKG^M(I^\epsilon)  \}
    \end{split}
    \end{equation*}
\end{definition}

\begin{definition}
    Let $\mathcal{L}$ denote the collection of all worms in $\zz \times \z$. Let $M$ and $N$ be \qzzmod s. The erosion distance can be defined as:
    \begin{equation*}
    \begin{split}
        d_\mathcal{E}^\mathcal{L}(M,N) \coloneqq \inf \limits_{\epsilon \geq 0} 
        \{ & \forall \boxed{\vp}^2_\delta \in \textbf{I}(\zz \times \z), \\
        & \RKG^M\left (\boxed{\vp}^2_\delta\right) \geq \RKG^N\left(\boxed{\vp}^2_{\delta+\epsilon}\right ) \text{ and } \\
        & \RKG^N\left (\boxed{\vp}^2_\delta\right ) \geq \RKG^M\left (\boxed{\vp}^2_{\delta+\epsilon}\right)  \}.
    \end{split}
    \end{equation*}
\end{definition}


% \begin{proposition}\label{aprop:interleaving}
%     Given two \qzzmod s $M$ and $N$, $d_\mathcal{E}^{\mathcal{L}}(M,N) \leq d_\mathcal{I}(M,N)$ where $d_\mathcal{I}$ denotes the interleaving distance between $M$ and $N$.
% \end{proposition}

\begin{propositionof}{\ref{prop:interleaving}} \label{aprop:interleaving}
    Given two \qzzmod s $M$ and $N$, $d_\mathcal{E}^{\mathcal{L}}(M,N) \leq d_\mathcal{I}(M,N)$ where $d_\mathcal{I}$ denotes the interleaving distance between $M$ and $N$.
\end{propositionof}

\begin{proof}
First, it is obvious that $d_\mathcal{E}^{\mathcal{L}}(M,N) \leq d_\mathcal{E}(M,N)$. Now, $d_\mathcal{I}(M,N) \coloneqq d_\mathcal{I}(\Tilde{E}(M), \Tilde{E}(N))$. Let $I$ be a subposet in $\zz \times \z$ such that its corresponding subposet in $\z^2$ is an interval.  By Lemma~\ref{alem:interval} and Lemma~\ref{alem:decomposition}, and the fact that generalized rank over a given subposet counts the number of intervals 
% in the interval decomposition 
that contain the given subposet, we get $\RKG^M(I) = \RKG^{\Tilde{E}(M)}(\Tilde{\mathfrak{i}}(I) \cap \uuu)$. This gives us $d_\mathcal{E}(M,N) = d_\mathcal{E}(\Tilde{E}(M),\Tilde{E}(N))$. In~\cite{GenRankKim21}, the authors show that $d_\mathcal{E}(V,W) \leq d_\mathcal{I}(V,W)$, where $V,W \colon \RR^n \to \textbf{vec}$ are $\RR^n$-indexed persistence modules. Thus, we get $d_\mathcal{E}^{\mathcal{L}}(M,N) \leq d_\mathcal{E}(M,N) = d_\mathcal{E}(\Tilde{E}(M),\Tilde{E}(N)) \leq d_\mathcal{I}(\Tilde{E}(M),\Tilde{E}(N)) = d_\mathcal{I}(M,N)$. 

\end{proof}

% One of the key steps in the proof of Proposition~\ref{prop:interleaving} is that the value of generalized rank of a module $M$ over an interval $I$ in $\zz \times \z$ is the same as the value of generalized rank of $E(M)$ over the interval $\mathfrak{i}(I)$ in $\RR^3$.

% \begin{theorem}\label{aprop:stability_zzgril}
%     Let $M$ and $N$ be two \qzzmod s such that $d_\mathcal{I}(M,N) = \epsilon$. Let $\vp, k$ be given. Let the \zzgril{} of $M$ and $N$ over $\boxed{\vp}^2_\delta$ be denoted by $\lambda^M$ and $\lambda^N$ respectively. Then, 
%     \begin{equation*}
%         |\lambda^M(\vp,k) - \lambda^N(\vp,k)| \leq \epsilon.
%     \end{equation*}
% \end{theorem}

\begin{theoremof}{\ref{prop:stability_zzgril}}\label{aprop:stability_zzgril}
    Let $M$ and $N$ be two \qzzmod s such that $d_\mathcal{I}(M,N) = \epsilon$. Let $\vp, k$ be given. Let the \zzgril{} of $M$ and $N$ over $\boxed{\vp}^2_\delta$ be denoted by $\lambda^M$ and $\lambda^N$ respectively. Then, 
    \begin{equation*}
        |\lambda^M(\vp,k) - \lambda^N(\vp,k)| \leq \epsilon.
    \end{equation*}
\end{theoremof}

\begin{proof}
    We will show that $|\lambda^M(\vp,k) - \lambda^N(\vp,k)| = d_\mathcal{E}^\mathcal{L}(M,N)$. 
    
    To see $|\lambda^M(\vp,k) - \lambda^N(\vp,k)| \leq d_\mathcal{E}^\mathcal{L}(M,N)$, let $\lambda^M(\vp,k) = \delta_M$ and $\lambda^N(\vp,k) = \delta_N$. WLOG, assume $\delta_M \geq \delta_N$. Let $d_\mathcal{E}^\mathcal{L}(M,N) = \epsilon$. Therefore, by definition of $d_\mathcal{E}^\mathcal{L}$, we have $\RKG^M\left (\boxed{\vp}^2_{\delta_M}\right) = k$ and $\RKG^M\left (\boxed{\vp}^2_{\delta_N + \epsilon}\right) \leq \RKG^N\left (\boxed{\vp}^2_{\delta_N}\right) = k$. Thus, by the definition of \zzgril{}, we get $\delta_N + \epsilon \geq \delta_M$, i.e., $\delta_M - \delta_N \leq \epsilon = d^\mathcal{L}_\mathcal{E}$.

    To see $|\lambda^M(\vp,k) - \lambda^N(\vp,k)| \geq d_\mathcal{E}^\mathcal{L}(M,N)$, let $|\lambda^M(\vp,k) - \lambda^N(\vp,k)| = \epsilon$. Let $\boxed{\vp}^2_\delta$ be a worm. Let $\RKG^N\left (\boxed{\vp}^2_{\delta + \epsilon}\right) = k$. Then, $\lambda^N(\vp,k) \geq \delta + \epsilon$. Thus, $|\lambda^M(\vp,k) - \lambda^N(\vp,k)| = \epsilon$ and $\lambda^N(\vp,k) \geq \delta + \epsilon$ give $\lambda^M(\vp,k) \geq d$. Therefore, $\RKG^M\left (\boxed{\vp}^2_{\delta}\right) \geq k = \RKG^N\left (\boxed{\vp}^2_{\delta + \epsilon}\right)$. Similarly, we can show $\RKG^N\left (\boxed{\vp}^2_{\delta}\right) \geq  \RKG^M\left (\boxed{\vp}^2_{\delta + \epsilon}\right)$. Thus, by definition of $d_\mathcal{E}^\mathcal{L}$, we get that $|\lambda^M(\vp,k) - \lambda^N(\vp,k)| \geq d_\mathcal{E}^\mathcal{L}(M,N)$.
    
\end{proof}

% \textbf{Correctness.} To prove the correctness of the procedure, we need to show 
% the following: For $1\leq i,i' \leq T$ and $1\leq j,j' \leq L$, let $K_{i,j}$ denote the simplicial complex in the quasi zigzag filtration built according to the procedure mentioned above and $K'_{i,j}$ is the simplicial complex in the quasi zigzag bi-filtration built by considering unions explicitly. We show
% that the new simplices inserted by the inclusion $K_{i,j} \xhookrightarrow{} K_{i',j'}$ are the same as the ones inserted by $K'_{i,j} \xhookrightarrow{} K'_{i',j'}$. 





\section{Experimental Details}
\label{app:exp}
Here, we report additional details about our experiments. We begin by including a detailed description of the UEA datasets in Table~\ref{tab:app:dataset_info}. Further, we have a comparison of treating a multivariate time series as a sequence of point clouds versus treating it as a sequence of graphs as the input to the \zzgril{} framework. We report the results of this experiment in Table~\ref{tab:app_exp_res}. We can see that there is no clear winner. On some datasets, \zzgril{} extracts more relevant information from sequences of point clouds while on some datasets, information from sequences of graphs performs better. We have a visualization of \zzgril{} in Figure~\ref{fig:zz_gril_vis}. We can see in the figure that the \zzgril{} topological signature is different for samples of different classes and very similar for samples in the same class.


\begin{table}[htbp]
    \centering
    \begin{tabular}{ccccccc}
    \toprule
         \textbf{Dataset}&  \textbf{Type}&  \textbf{No. of series}&  \textbf{Series length}& \textbf{No. of classes} &\textbf{Train size} &\textbf{Test size}\\
         \midrule
 FingerMovements& EEG& 28& 50& 2& 316&100\\
         Heartbeat&  AUDIO&  61&  405&  2& 204&205\\
 MotorImagery& EEG& 64& 3000& 2& 278&100\\
         NATOPS&  HAR&  24&  51&  6& 180&180\\
         SelfRegulationSCP2&  EEG&  7&  1152&  2& 200&180\\
         \bottomrule
    \end{tabular}
    \caption{Information about UEA Datasets used for experiments}
    \label{tab:app:dataset_info}
\end{table}



For all the experiments with sequence of point clouds, we choose a window size that is $\max(5, series\_length//128)$ and overlap $\max(4, 0.7*window\_size)$. For the experiments with sequence of graphs, we choose a window size of $\min(series\_length/5,128)$ and an overlap of $0.7*window\_size$. Then, from the complete graph, we randomly choose a percentage between 65 and 75 of the edges depending on their weights. This is to ensure that the number of edges does not remain the same for all the graphs in the sequence. For all our experiments, we use 36 center points to compute \zzgril{} at. For model parameters, mostly, we use the same parameters as specified by the respective models~\cite{todynet, stdpgcn}. We notice that the training is sensitive to learning rate and we optimize the learning rate between $1e-3$ and $5e-5$ for different datasets. 




\end{document}


% This document was modified from the file originally made available by
% Pat Langley and Andrea Danyluk for ICML-2K. This version was created
% by Iain Murray in 2018, and modified by Alexandre Bouchard in
% 2019 and 2021 and by Csaba Szepesvari, Gang Niu and Sivan Sabato in 2022.
% Modified again in 2023 and 2024 by Sivan Sabato and Jonathan Scarlett.
% Previous contributors include Dan Roy, Lise Getoor and Tobias
% Scheffer, which was slightly modified from the 2010 version by
% Thorsten Joachims & Johannes Fuernkranz, slightly modified from the
% 2009 version by Kiri Wagstaff and Sam Roweis's 2008 version, which is
% slightly modified from Prasad Tadepalli's 2007 version which is a
% lightly changed version of the previous year's version by Andrew
% Moore, which was in turn edited from those of Kristian Kersting and
% Codrina Lauth. Alex Smola contributed to the algorithmic style files.
