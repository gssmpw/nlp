\begin{figure}[t]
    \centering
    % Row 1 (5 images)
    \begin{subfigure}{0.15\textwidth}
        \includegraphics[width=\linewidth]{figures/1/0.png}
        %\caption{Caption 1}
        %\label{fig:1}
    \end{subfigure}\hfill
    \begin{subfigure}{0.15\textwidth}
        \includegraphics[width=\linewidth]{figures/1/1.png}
        %\caption{Caption 2}
        %\label{fig:2}
    \end{subfigure}\hfill
    \begin{subfigure}{0.15\textwidth}
        \includegraphics[width=\linewidth]{figures/1/4.png}
        %\caption{Caption 5}
        %\label{fig:5}
    \end{subfigure}

    \vspace{-2em}

    % Row 2 (5 images)
    \begin{subfigure}{0.15\textwidth}
        \includegraphics[width=\linewidth]{figures/2/0.png}
        %\caption{Caption 6}
        %\label{fig:6}
    \end{subfigure}\hfill
    \begin{subfigure}{0.15\textwidth}
        \includegraphics[width=\linewidth]{figures/2/2.png}
        %\caption{Caption 8}
        %\label{fig:8}
    \end{subfigure}\hfill
    \begin{subfigure}{0.15\textwidth}
        \includegraphics[width=\linewidth]{figures/2/4.png}
        %\caption{Caption 10}
        %\label{fig:10}
    \end{subfigure}

    \caption{Examples of two FuncGenFoil instances performing airfoil editing over 100 training iterations. The red section represents the editing requirement, while the black airfoil shows the generated samples at each training stage. The results demonstrate that the generated airfoil quickly adapts to the editing requirements within a few iterations, achieving a natural and smooth function regression.}
    \vspace{-15pt}
    \label{fig:airfoil_edit_finetuning}
\end{figure}