\section{Airfoil Geometry Parameters}

\label{app:groparams}
\begin{table}[h]
    \centering
    \renewcommand{\arraystretch}{1.2} % 调整行间距,避免压缩
    \caption{Geometric parameters of the airfoils used in conditional generation tasks}
    \begin{tabularx}{\linewidth}{ccX}
        \toprule
        \textbf{Index} & Symbol & \multicolumn{1}{c}{Meaning} \\
        \midrule
        1 & $R_{\mathrm{le}}$ & the leading edge radius \\
        2 & $X_{\mathrm{up}}$ & upper crest position x \\
        3 & $Y_{\mathrm{up}}$ & upper crest position y \\
        4 & $Z_{\mathrm{xxup}}$ & upper crest curvature \\
        5 & $X_{\mathrm{lo}}$ & lower crest position x \\
        6 & $Y_{\mathrm{lo}}$ & lower crest position y \\
        7 & $Z_{\mathrm{xxlo}}$ & lower crest curvature \\
        8 & $Y_{\mathrm{te}}$ & trailing edge position \\
        9 & $\Delta Y_{\mathrm{te}}$ & trailing thickness \\
        10 & $\alpha_{\mathrm{te}}$ & trailing edge angle up \\
        11 & $\beta_{\mathrm{te}}$ & trailing edge angle down \\
        \bottomrule
    \end{tabularx}
    \label{tab:geoparams}
\end{table}

\begin{algorithm}[h]
    \caption{Generative Model Training.}
    \label{alg:pretrain}
    \textbf{Input}: data resolution $d$, data $u_1$, design condition variables $c$ (optional).\\
    \textbf{Parameter}: gaussian process $\mathcal{GP}(0,K)$ for sampling $u_0$.\\
    \textbf{Output}: velocity operator $v_{\theta}$. \\
    \begin{algorithmic}[1] %[1] enables line numbers
        \WHILE{training...}
        \STATE sample $t\sim[0,1]$, $u_0\sim\mathcal{GP}(0,K)$ at resolution $d$ and get $\{u_{0,i}\}$.
        \STATE Compute $v_t$ at resolution $d$ and get $\{v_{t,i}\}$.
        \STATE Compute $u_t$ at resolution $d$ and get $\{u_{t,i}\}$.
        \STATE Compute $v_{\theta}(u_t,c,t)$ at resolution $d$ and get $\{v_{\theta,i}\}$.
        \STATE Minimize $\|\{v_{\theta,i}\}-\{v_{t,i}\}\|^2$.
        \STATE Compute gradient and update $\theta$.
        \ENDWHILE
        \STATE \textbf{return} $v_{\theta}$.
    \end{algorithmic}
\end{algorithm}

\begin{algorithm}[h]
    \caption{Inference Process (Airfoil Generation)}
    \label{alg:generation}
    \textbf{Input}: sampling resolution $d$, sampling time steps $T$ and steps length $\mathrm{d}t$, latent function $u_0$ (optional), design condition variables $c$ (optional).\\
    \textbf{Parameter}: gaussian process $\mathcal{GP}(0,K)$ for sampling $u_0$.\\
    \textbf{Output}: airfoil $\{y_i\}$ at resolution $d$. \\
    \begin{algorithmic}[1] %[1] enables line numbers
        \STATE Let $t=0$, $u_0\sim\mathcal{GP}(0,K)$ at resolution $d$ and get $\{u_{0,i}\}$.
        \WHILE{$t<=1$}
        \STATE Compute $v_{\theta}(u_t,c,t)$ at resolution $d$ and get $\{v_{\theta,i}\}$.
        \STATE Compute $\{u_{t+\mathrm{d}t,i}\}=\{u_{t,i}\}+\{v_{\theta,i}\mathrm{d}t\}$.
        \STATE $t=t+\mathrm{d}t$.
        \ENDWHILE
        \STATE \textbf{return} $\{y_i\}=\{u_{1,i}\}$
    \end{algorithmic}
\end{algorithm}

\begin{table}[h]
    \centering
    \renewcommand{\arraystretch}{1.2} % 调整行间距,避免压缩
    \caption{Training hyperparameters and model parameters in conditional airfoils generation tasks}
    \begin{tabularx}{\linewidth}{Xc}
        \toprule
        \multicolumn{1}{c}{Training hyperparameters} & Value \\
        \midrule
        max learning rate & 5e-6 \\
        batch size & 512 \\
        optimizer used in pre-training and finetuning & Adam \\
        train max iterations & 1,000,000 \\
        ODE Solver time steps & 100 \\
        \midrule
        \multicolumn{1}{c}{Model parameters} & Value \\
        \midrule
        Fourier neural operator layers & 8 \\
        Fourier neural operator modes & 48 \\
        Fourier neural operator hidden channels & 256 \\
        Matérn kernel $\nu$ & 2.5 \\
        Matérn kernel $l$ & 0.03 \\
        \bottomrule
    \end{tabularx}
    \label{tab:geoparams}
\end{table}

We adopted the same conditions and symbols as those in AFBench. The parameters below, including the leading-edge radius, the curvature at the maximum thickness of the upper and lower surfaces, and the trailing-edge angle of the airfoil, were obtained through fitting. Note that our implementation of the angle calculation differs slightly from that of AFBench, as we found that the original fitting method was not invariant when the sampling precision increased.

\section{Algorithm Details}\label{sec:appendix_algorithm}
The algorithm for training airfoil generative models is shown in Algorithm~\ref{alg:pretrain}, and inference using a trained model is shown in Algorithm~\ref{alg:generation}. In the actual implementation, we utilized Optimal Transport techniques and employed the Matérn kernel as the kernel function for the Gaussian process in the latent space of the generative model.

\section{Training Details}
\label{app:hyper}

In most of the experiments, we followed the hyperparameters listed in the table below to train the models, including both the pre-training and fine-tuning stages. On the AF-200K dataset, we increased the maximum number of iterations to 2,000,000 and the batch size to 2048. For the number of time steps used to solve the ODE, we tested reconstruction metrics on the super dataset with 10, 100, and 1000 steps, and ultimately chose 100 steps as the default setting for all other experiments, except for the fine-tuning stage, where we used 10 steps for training efficiency.
