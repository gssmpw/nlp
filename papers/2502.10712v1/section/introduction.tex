\section{Introduction}

\begin{figure}[htb]
    \small
    \centering
      \includegraphics[width=\linewidth]{assets/introduction.pdf}
      \caption{
      The conceptual difference between FuncGenFoil and previous airfoil representation methods. In many previous approaches, airfoils are represented either as parametric models, as shown in (a), or as discrete point-based models, as shown in (b). In contrast, (c) illustrates FuncGenFoil's approach, where an airfoil is treated as a continuous function mapped from a latent function, enabling a generative model in function space.}
      \label{fig:intro}
      % \vspace{-15pt}
\end{figure}

The airfoil inverse design problem serves as a central aspect of aircraft manufacturing. Traditionally, given geometric requirements, engineers first select the most similar airfoils from well-known airfoil datasets (e.g., NACA~\cite{Cummings2015AppliedCA}) and leverage the \textbf{trial-and-error} strategy~\cite{sharma2021recent}. Considering the mission of the aircraft, an initial airfoil design that meets the design conditions is preliminarily created. Then, through rounds of physical analysis, such as aerodynamics and mechanics, the airfoil is iteratively optimized to achieve better performance until the requirements are met. In practice, such direct design procedures are highly inefficient and time-consuming, often taking months.
To minimize development and design time, as well as associated costs, automatic design methods have been introduced as efficient alternatives in aircraft manufacturing engineering. In particular, machine learning-based design and optimization techniques have gained significant attention. However, before applying algorithms to airfoil design, it is crucial to determine appropriate methods for representing airfoils within these algorithms.

Existing methods for airfoil representation can generally be divided into two categories: parametric-model-based approaches~\cite{xie2024parametric} and discrete-point-based methods~\cite{liu2024afbench,sekar2019inverse}. First, parametric-model-based methods predefine function families, \emph{e.g.,} Bézier curves~\cite{chen2021bezierganautomaticgenerationsmooth}, Hicks-Henne curves, and NURBS, and leverage mathematical optimization techniques or generative models to determine the parameters of these functions for a new airfoil. These methods rigorously preserve key geometric properties of the defined functional families, \emph{e.g.,} high-order smoothness. Furthermore, such functional representations of airfoils allow for arbitrary sampling of control points in real manufacturing, given the constraints of engineering precision. Despite these benefits, parametric-model-based methods suffer from a significantly reduced design space, \emph{i.e.,} selecting a specific function family excludes the possibility of other shapes, which limits the upper bound of airfoil design algorithms. 
Second, discrete-point-based methods directly generate multiple points to represent airfoil shapes. These methods maximize the design space of airfoils but cannot maintain some important mathematical properties, \emph{e.g.,} continuity. Furthermore, they cannot directly generate control points at arbitrary resolutions because the number of generated points is typically fixed for each model after training.

To address the trade-offs between these two mainstream approaches, we ask: \emph{Can we design an algorithm that leverages the advantages of both?}

In this paper, we answer this question by proposing FuncGenFoil, a novel function-space generative model for airfoil representation (see Fig.~\ref{fig:intro}). Different from previous data-driven and generative methods, \emph{e.g.,} cVAE~\cite{kingma2013auto} and cGAN~\cite{mirza2014conditional}, which directly generate discrete points, our approach models airfoil geometry using a general continuous function approximator~\cite{anandkumar2019neural,li2021fourier,Kovachki2023,azizzadenesheli2024neural}, \emph{i.e.,} neural operator architectures. Simultaneously, it leverages recent advances in generative methods, \emph{e.g.,} diffusion models~\cite{Ho2020} and flow matching models~\cite{lipman2022flow}, to generate diverse airfoils beyond the design space of any predefined geometry function. 
Our method has the advantages of both parametric-model-based approaches and discrete-point-based methods. Due to its functional representation, the generated airfoil is continuous and can be sampled at arbitrary resolutions, making it easier to manufacture. Moreover, thanks to the general functional approximator nature of neural operators, our method can explore a broader design space beyond predefined functional families.

Specifically, we leverage the flow matching framework~\cite{lipman2022flow}, an improved alternative to diffusion models, and FNO~\cite{li2021fourier}, a resolution-free neural operator, as the backbone of our generative model to design FuncGenFoil.  In the forward process, we perturb the airfoil into a noise distribution through straight flows. In the backward process, we reconstruct the airfoil by reversing the flow direction. By minimizing the distance between the reconstructed airfoils and actual airfoils, we learn the neural operators that enable us to generate airfoils from functions of Gaussian process. Beyond generation, our method also supports airfoil editing by incorporating the airfoils to be edited as conditions for the generative model.

In summary, our main contributions are threefold: (1) We propose generating airfoil shapes in the functional space to achieve important properties for the aircraft engineering, \emph{i.e.,} arbitrary-resolution control point sampling and maximal design space; (2) We design FuncGenFoil, the first controllable airfoil generative model in the functional space, which effectively incorporates neural operator architectures into the generative model; (3) We further enhance FuncGenFoil with airfoil editing capabilities through minimal adaptations.
Experimental results on the AFBench dataset indicate that our proposed method achieves state-of-the-art airfoil generation quality, reducing label error by 74.4\% and increasing diversity by 23.2\% on the AF-200K dataset, as validated by aerodynamic simulation analysis.  
In addition, our method is the first to successfully perform airfoil editing by fixing and dragging at any position, achieving nearly zero MSE error (less than $10^{-7}$).
