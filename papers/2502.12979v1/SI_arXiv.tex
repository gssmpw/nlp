%Version 3 October 2023
% See section 11 of the User Manual for version history
%
%%%%%%%%%%%%%%%%%%%%%%%%%%%%%%%%%%%%%%%%%%%%%%%%%%%%%%%%%%%%%%%%%%%%%%
%%                                                                 %%
%% Please do not use \input{...} to include other tex files.       %%
%% Submit your LaTeX manuscript as one .tex document.              %%
%%                                                                 %%
%% All additional figures and files should be attached             %%
%% separately and not embedded in the \TeX\ document itself.       %%
%%                                                                 %%
%%%%%%%%%%%%%%%%%%%%%%%%%%%%%%%%%%%%%%%%%%%%%%%%%%%%%%%%%%%%%%%%%%%%%

%%\documentclass[referee,sn-basic]{sn-jnl}% referee option is meant for double line spacing

%%=======================================================%%
%% to print line numbers in the margin use lineno option %%
%%=======================================================%%

%%\documentclass[lineno,sn-basic]{sn-jnl}% Basic Springer Nature Reference Style/Chemistry Reference Style

%%======================================================%%
%% to compile with pdflatex/xelatex use pdflatex option %%
%%======================================================%%

%%\documentclass[pdflatex,sn-basic]{sn-jnl}% Basic Springer Nature Reference Style/Chemistry Reference Style


%%Note: the following reference styles support Namedate and Numbered referencing. By default the style follows the most common style. To switch between the options you can add or remove “Numbered” in the optional parenthesis. 
%%The option is available for: sn-basic.bst, sn-vancouver.bst, sn-chicago.bst%  
 
%%\documentclass[sn-nature]{sn-jnl}% Style for submissions to Nature Portfolio journals
%%\documentclass[sn-basic]{sn-jnl}% Basic Springer Nature Reference Style/Chemistry Reference Style
\documentclass[sn-mathphys-num]{sn-jnl}
% Math and Physical Sciences Numbered Reference Style 
%%\documentclass[sn-mathphys-ay]{sn-jnl}% Math and Physical Sciences Author Year Reference Style
%%\documentclass[sn-aps]{sn-jnl}% American Physical Society (APS) Reference Style
%%\documentclass[sn-vancouver,Numbered]{sn-jnl}% Vancouver Reference Style
%%\documentclass[sn-apa]{sn-jnl}% APA Reference Style 
%%\documentclass[sn-chicago]{sn-jnl}% Chicago-based Humanities Reference Style

%%%% Standard Packages
%%<additional latex packages if required can be included here>

\usepackage{graphicx}%
\usepackage{multirow}%
\usepackage{amsmath,amssymb,amsfonts}%
\usepackage{amsthm}%
\usepackage{mathrsfs}%
\usepackage[title]{appendix}%
\usepackage{xcolor}%
\usepackage{textcomp}%
\usepackage{manyfoot}%
\usepackage{booktabs}%
\usepackage{algorithm}%
\usepackage{algorithmicx}%
\usepackage{algpseudocode}%
\usepackage{listings}%
\usepackage{makecell}
\usepackage[shortlabels]{enumitem}
\usepackage[version=4]{mhchem}
\usepackage{chemmacros}

%%%%

%%%%%=============================================================================%%%%
%%%%  Remarks: This template is provided to aid authors with the preparation
%%%%  of original research articles intended for submission to journals published 
%%%%  by Springer Nature. The guidance has been prepared in partnership with 
%%%%  production teams to conform to Springer Nature technical requirements. 
%%%%  Editorial and presentation requirements differ among journal portfolios and 
%%%%  research disciplines. You may find sections in this template are irrelevant 
%%%%  to your work and are empowered to omit any such section if allowed by the 
%%%%  journal you intend to submit to. The submission guidelines and policies 
%%%%  of the journal take precedence. A detailed User Manual is available in the 
%%%%  template package for technical guidance.
%%%%%=============================================================================%%%%

%% as per the requirement new theorem styles can be included as shown below
\theoremstyle{thmstyleone}%
\newtheorem{theorem}{Theorem}%  meant for continuous numbers
%%\newtheorem{theorem}{Theorem}[section]% meant for sectionwise numbers
%% optional argument [theorem] produces theorem numbering sequence instead of independent numbers for Proposition
\newtheorem{proposition}[theorem]{Proposition}% 
%%\newtheorem{proposition}{Proposition}% to get separate numbers for theorem and proposition etc.

\theoremstyle{thmstyletwo}%
\newtheorem{example}{Example}%
\newtheorem{remark}{Remark}%

\theoremstyle{thmstylethree}%
\newtheorem{definition}{Definition}%

\raggedbottom
%%\unnumbered% uncomment this for unnumbered level heads



\renewcommand\thesection{S\arabic{section}}
\renewcommand{\thefigure}{S\arabic{figure}}
\renewcommand{\thetable}{S\arabic{table}}

% New commands
\newcommand{\namerxn}[0]{\texttt{NameRxn} }
\newcommand{\approximate}[1]{\(\sim\) #1}
\newcommand{\original}[0]{\texttt{original} }
\newcommand{\higherlev}[0]{\texttt{higher-level} }
\begin{document}

\title[Article Title]{Supplementary information for\\ Electron flow matching for generative reaction mechanism prediction obeying conservation laws}

\author[1,2]{\fnm{Joonyoung F.} \sur{Joung}}\email{jjoung@mit.edu}
\equalcont{These authors contributed equally to this work.}
\author[1]{\fnm{Mun Hong} \sur{Fong}}\email{fong410@mit.edu}
\equalcont{These authors contributed equally to this work.}

\author[1]{\fnm{Nicholas} 
\sur{Casetti}}\email{ncasetti@mit.edu}

\author[1]{\fnm{Jordan P.} \sur{Liles}}\email{jliles@mit.edu}
\author[3]{\fnm{Ne S.} \sur{Dassanayake}}\email{nedassa@mit.edu}


\author*[1,4]{\fnm{Connor W.} \sur{Coley}}\email{ccoley@mit.edu}

\affil*[1]{\orgdiv{Department of Chemical Engineering}, \orgname{Massachusetts Institute of Technology}, \orgaddress{\street{77 Massachusetts Ave.}, \city{Cambridge}, \postcode{02139}, \state{Massachusetts}, \country{United States}}}

\affil[2]{\orgdiv{Department of Chemistry}, \orgname{Kookmin University}, \orgaddress{\street{77 Jeongneung-ro, Seongbuk-gu}, \city{Seoul}, \postcode{02707}, \country{Republic of Korea}}}

\affil[3]{\orgdiv{Department of Chemistry}, \orgname{Massachusetts Institute of Technology}, \orgaddress{\street{77 Massachusetts Ave.}, \city{Cambridge}, \postcode{02139}, \state{Massachusetts}, \country{United States}}}


\affil[4]{\orgdiv{Department of Electrical Engineering and Computer Science}, \orgname{Massachusetts Institute of Technology}, \orgaddress{\street{77 Massachusetts Ave.}, \city{Cambridge}, \postcode{02139}, \state{Massachusetts}, \country{United States}}}
    

\maketitle

\clearpage
  
\tableofcontents
% \listoffigures
% \listoftables

\setcounter{figure}{0}
\renewcommand{\figurename}{Fig.}
\renewcommand{\theHfigure}{Fig. S\arabic{figure}}
\renewcommand{\tablename}{Table} 
\renewcommand{\theHtable}{Table S\arabic{figure}} 
\renewcommand{\vfill}{\vspace*{\fill}}



\newpage 
\section{Code and Data availability}
All data used in this study, along with pretrained single-step models, are publicly available on Figshare:  
\href{https://doi.org/10.6084/m9.figshare.28359407.v2}{https://doi.org/10.6084/m9.figshare.28359407.v2}.  
The dataset is provided in a text format, including reaction SMILES.

The source code for dataset curation can be found in the following GitHub repository:  
\href{https://github.com/jfjoung/Mechanistic_dataset}{https://github.com/jfjoung/Mechanistic\_dataset}.  
This repository includes scripts for preparing mechanistic dataset.

The implementation of FlowER, including model training, inference, and evaluation, is available at:  
\href{https://github.com/FongMunHong/FlowER}{https://github.com/FongMunHong/FlowER}.  
This repository provides training scripts and inference pipelines necessary to reproduce the results presented in this manuscript.

Both repositories include a detailed README with instructions for installation, usage, and result reproduction.  



\newpage 
\section{Mechanistic dataset preparation}\label{si_dataset}
We utilized the USPTO-Full dataset \cite{dai2019retrosynthesis}, which contains 1,100,105 reactions, for curating an expanded mechanistic dataset. Building on our previous method for constructing mechanistic datasets \cite{joung2024reproducing}, we significantly extended the scope of the dataset. This involved creating 1,220 unique reaction templates, covering 252 reaction classes and 185 distinct mechanisms. By adding more reaction classes and mechanisms, we ensured broader coverage and diversity. Since our method of data imputation relies on reaction class labels, we classified reactions using NextMove’s NameRxn software \cite{namerxn}.

\subsection{Expert-curated reaction templates}
Expert-curated reaction templates were constructed to represent the most common reaction mechanisms observed in the USPTO-Full dataset. Each template was designed to encode the specific chemical transformations of an elementary step. To promote the accuracy of the constructed templates, three chemists reviewed each mechanism. A series of templates corresponding to a given reaction class was then sequentially applied to a set of reactants, generating possible intermediates and ultimately the products.

To ensure consistency with the experimentally observed products of the overall reactions, we algorithmically pruned unproductive intermediates and retained only the mechanistic steps leading to the reported products. Through this process, we were able to impute intermediates not explicitly recorded in the overall reactions and identify often-overlooked byproducts.

\subsection{Stoichiometry and acid-base reactions}

A significant number of elementary steps involve acid-base reactions. In our previous work \cite{joung2024reproducing}, hydrogen conservation was not fully addressed due to the challenge of explicitly specifying all possible acid-base partners for each template. To overcome this limitation, we redesigned the templates to ensure hydrogen conservation and accurately model acid-base reactions. Each template was constructed as a unimolecular reaction, where the intermediate either gains or loses a proton. To account for the proton donor or acceptor, we annotated the \pKa of the required partner directly in the template.

We compiled \pKa values for 88 distinct acid-base pairs. During the mechanistic dataset generation process, whenever a template requiring a proton transfer was applied, the following procedure was implemented:

\begin{itemize}
    \item If the reaction required an acid, the system checked whether an acid with a \pKa lower than the specified value was present in the current set of chemicals. If such an acid was found, it was added to the reactant side of the reaction, while the corresponding base was added to the product side, transforming the unimolecular template into a bimolecular one.
    \item Conversely, if a base was required, the system checked for a base with a \pKa higher than the specified value and adjusted the reaction in the same manner.
\end{itemize}

This dynamic adjustment allowed us to avoid creating exhaustive templates for every possible acid-base reaction, significantly streamlining the dataset construction process.

Additionally, while applying reaction templates, we addressed scenarios where multiple equivalents of a reactant were required, a common occurrence in acid-base and catalytic reactions. For example, the USPTO-Full dataset often records only one equivalent of a reactant, even when two or more equivalents are necessary for the reaction to proceed. If a template required the same reactant that had already been consumed in a previous elementary step, the mechanistic data generation process was adjusted as follows:

\begin{enumerate} 
    \item All earlier elementary steps were revised to duplicate the required reactant, ensuring sufficient quantities were present throughout the reaction sequence. 
    \item The current template was then applied to the reaction, utilizing the additional equivalent of the reactant. 
\end{enumerate}

This approach ensured that the mechanistic pathways generated from the templates remained chemically accurate, consistent with stoichiometric requirements, and reflecting the realistic reaction conditions.


\subsection{Kekulé representation of molecules}

In constructing the BE matrix, we adopted the Kekulé representation for aromatic rings instead of using the classical BE matrix form proposed by Ugi in the 1970s \cite{10.1007/BFb0051317, ugi1993computer}. In Ugi’s BE matrix, aromatic bonds are represented with a bond order of 1.5. However, in fused ring systems, this approach leads to an issue where certain carbon atoms, shared between two aromatic rings, end up with three bonds of order 1.5. As shown in the BE matrix of the canonical form in Fig. \ref{fig_kekule}, this results in a total electron count of 4.5 for such atoms, requiring lone pairs to be assigned a non-physical value of $-0.5$ electrons to maintain overall electron conservation. While the total electron count remains consistent, the introduction of negative fractional electrons is unrealistic and undesirable.

\begin{figure*}[h]
\centering
\includegraphics[width=\textwidth]{Figures/SI_kekule.pdf}
\caption{
Comparison of BE matrices for the canonical and Kekulé forms. The canonical form represents aromatic bonds with bond order 1.5, leading to non-physical fractional electron counts. In contrast, the Kekulé form explicitly represents alternating single and double bonds, ensuring integer number of electrons.
}\label{fig_kekule}
\end{figure*}

To address this, we adopted the Kekulé form, explicitly representing aromatic bonds as alternating single and double bonds. As seen in the BE matrix of the Kekulé form in Fig.~\ref{fig_kekule}, this approach ensures that all bond orders are represented as non-negative integers, making the BE matrix more physically meaningful. Kekulization was applied systematically using RDKit’s \texttt{Chem.Kekulize(mol)} function.


\subsection{Data cleaning and validation}
To ensure the reliability and consistency of the mechanistic dataset, we applied rigorous cleaning procedures.


\begin{itemize}
    \item \textbf{Atom Mapping Consistency}: All atom mappings were verified to be unique and consistent between reactants and products. 
    \item \textbf{BE Matrix Validation}: The BE matrix was checked to be diagonally symmetric, and all entries were ensured to be either 0 or positive integers. Conservation of electrons was verified by confirming that the sum of the BE matrix values remained constant before and after each reaction.
    \item \textbf{Chemical Validity Check}: Structures were screened using RDKit to remove chemically invalid molecules, such as those with incorrect valency or problematic resonance structures.
    \item \textbf{Kekulization Filtering}: If Kekulization failed for a given molecule, all steps involving that molecule were removed from the dataset.
    \item \textbf{Pathway Integrity Check}: If the removal of invalid steps resulted in the loss of a complete reaction pathway from reactants to the observed product, the entire sequence was discarded to maintain dataset consistency.
    
\end{itemize}

These validation steps ensured that the final dataset contained only chemically meaningful and computationally robust reaction sequences, improving its reliability for reaction modeling.

\subsection{Integration with external databases}
To enhance dataset diversity, we incorporated data from RMechDB \cite{tavakoli2023rmechdb} and PMechDB \cite{tavakoli2024pmechdb}. Since these external datasets provided atom mappings only for reaction centers, we used RXNMapper \cite{schwaller2021extraction} to extend atom mappings to the entire reaction while ensuring consistency with our existing mapping approach.

Before integration, we applied the same cleaning protocols used for our mechanistic dataset, including atom mapping validation, BE matrix verification, and stoichiometry checks.

\subsection{Final dataset}
The resulting dataset consisted of:

\begin{itemize}
\item Training Set: 250,782 overall reactions (1,445,189 elementary steps)
\item Validation Set: 2,801 overall reactions (15,744 elementary steps)
\item Test Set: 28,049 overall reactions (162,002 elementary steps)
\end{itemize}

The dataset spans 252 reaction classes and 185 distinct mechanisms, covering a broad range of organic transformations. The dataset was split into training, validation, and test sets using an 89:1:10 ratio.



\clearpage
\section{Post processing of model output}\label{si_post_processing}
\subsection{Sum-safe rounding}
The sum-safe rounding is performed through the python package \texttt{iteround} \cite{calvin2018round}. It provides a safe-sum rounding, which ensures the preservation of electron counts after rounding from floating point number back to representable integer bond order on the BE matrix, allowing us to reconstruct molecules back using RDKit.
\[
    \text{diff} = x - round(x)
\]
This equation represents how each value is incremented / decreased sequentially according to the sorted difference. The algorithm alternates between the smallest surplus / deficit to alter each floating point back to it's rounded integer.

\subsection{Validity Fix}
The validity fix stems from the notion that the product atoms' valence electron count should be the same as the reactant atoms' valence electron count. E.g., if the input reactant valence electron count is [2, 8, 8, 8, 8, 2], predicted product valence electron count is [2, 6, 8, 8, 10, 2], difference would be [0, -2, 0, 0, +2, 0]. If the sum of difference is zero, we apply this validity fix through ``redistributing" the electrons back to match it's reactant valence electron count. Empirically, this fix improves BE matrix to SMILES conversion validity by 3-4\% on the test set, and typically reactants with HCNOF elements that strictly obey the octet rule benefits from it.  


\subsection{Failure mode analysis}
Despite the post-processing techniques described above, FlowER still produces 5.06\% invalid SMILES during sampling, as mentioned in the main text. To better understand these failure cases, we analyzed the invalid outputs and categorized them into three major types. 
Fig.~\ref{failure} illustrates the distribution of the three failure cases identified in FlowER’s sampled predictions.


\begin{itemize}
    \item \textbf{Electron over-redistribution in prediction}

    In some cases, FlowER predicts an excessive electron redistribution, moving more electrons from the bonds/lone pairs than what is originally  available in the reactants. This results in negative electron counts in the product BE matrix, making it physically invalid. This issue originates from errors in the $\Delta$BE matrix prediction, where the model incorrectly estimates the number of electrons involved in rearrangement.

    \item \textbf{Diagonal symmetry violation after sum-safe rounding}

    While sum-safe rounding ensures that the total number of electrons is preserved, it does not guarantee that the product BE matrix remains diagonally symmetric. If the rounding process distorts symmetry, the number of bonding electrons can become odd-valued, which leads to errors when reconstructing molecules. 
    
    In BE matrix-based molecule reconstruction, single, double, and triple bonds are defined as having 2, 4, and 6 bonding electrons, respectively. If a bond has an odd number of electrons due to symmetry breaking, it cannot be correctly mapped back to a valid molecular structure, resulting in an invalid SMILES.

    \item \textbf{Chemically invalid molecule despite valid BE matrix}

    In some cases, the sampled product BE matrix maintains mass conservation and diagonal symmetry but still fails to generate a chemically valid molecule. These cases occur when the predicted electron distribution leads to an unstable molecular structure, violating chemical stability rules. Examples of such chemically invalid molecules are illustrated in Fig.~\ref{weird_molecule}.
    
\end{itemize}

\begin{figure}[h]
\centering
\includegraphics{Figures/failure_mode.pdf}
\caption{Histogram of failure modes in FlowER's sampled predictions. The four categories represent different causes of invalid SMILES: (1) negative electron counts and symmetry violation, (2) negative electron counts only, (3) symmetry violation only, and (4) chemically invalid molecules. 
}\label{failure}
\end{figure}

\begin{figure*}[h]
\centering
\includegraphics[width=\textwidth]{Figures/Figure_weird.pdf}
\caption{Examples of chemically invalid molecules generated by FlowER. Problematic regions within the molecules are highlighted in red. 
}\label{weird_molecule}
\end{figure*}

\clearpage
\section{Elaboration of pathway accuracy comparison}
Accurate prediction of chemical reaction pathways requires not only identifying correct individual mechanistic steps but also reconstructing the full sequence leading to the final product. Pathway accuracy measures a model’s ability to correctly predict complete reaction sequences, rather than individual steps. This is particularly important for FlowER, where each elementary step prediction influences subsequent steps, making pathway reconstruction inherently more challenging than single-step accuracy.

Fig. 2c compares the top-\textit{k} pathway accuracy of FlowER and G2S, showing the impact of their different approaches to predicting reaction sequences.

This discrepancy between top-1 step accuracy and top-1 pathway accuracy can be attributed to differences in how FlowER and G2S distribute predictions across reaction sequences. 
In FlowER, top-1 predictions for individual steps are highly accurate, but non-top-1 predictions are evenly distributed across sequences. 
This pattern reflects FlowER's ability to explore diverse mechanistic possibilities for each elementary step. 
However, when evaluating complete pathways, these occasional non-top-1 predictions can accumulate, leading to a reduction in top-1 pathway accuracy.

In contrast, G2S tends to cluster non-top-1 predictions within specific sequences. 
While this limits the diversity of explored pathways, it results in fewer sequences being affected by prediction errors. 
As a result, G2S achieves higher top-1 pathway accuracy despite lower top-1 step accuracy. 
FlowER’s stronger performance in top-2 pathway accuracy highlights its broader exploration of alternative pathways, possibly providing valuable insights into reaction mechanisms that may not be captured by G2S.

\clearpage
\section{Hyperparameter tuning}\label{si_hyperparam}

\begin{table}[h!]
\centering
\caption{Hyperparameter setting used in the experiments for different datasets. Best settings selected based on validation are highlighted in \textbf{bold} if multiple values have been experimented.}
\label{tab:hyperparameters}
\begin{tabular}{@{}lllp{5cm}@{}}
\toprule
Model           & Dataset & Parameter                      & Value(s)         \\ \midrule
FlowER          &  All  & Embedding size                   & \textbf{128}, 256 \\
                &       & Hidden size (same among all modules) & \textbf{128}, 256 \\
                &       & Filter size in Transformer      & 2048            \\
                &       & Attention encoder layers       &  6, 8, \textbf{12},   \\
                &       & Attention encoder heads        & 8, \textbf{32}       \\
                &       & Sigma                         & 0.1, 0.13, \textbf{0.15}, 0.16, 0.17, 2.0       \\
                &       & RBF low, high, step          & \textbf{\{0, 8, 0.1\}}, \{0, 20, 0.1\}  \\
                &       & Learning rate                & \textbf{0.0001}, 0.001       \\ 
                &       & Sample size                  & 16, \textbf{32}       \\ 
                &       & Scheduler                    & \textbf{NoamLR}, StepLR, None       \\ 
                \bottomrule
FlowER          &  Full & Total number of steps        & 1500000       \\ 
                \bottomrule
FlowER          &  50k, 5k, 
                1500, 500 & Total number of steps        & 500000       \\ 
                \bottomrule
                \bottomrule


Graph2SMILES    &  All  & Embedding size                 & \textbf{256}, 512 \\
                &       & Hidden size (same among all modules) & \textbf{256}, 512 \\
                &       & Filter size in Transformer     & 2048            \\
                &        & Number of D-MPNN layers        & 2, \textbf{4}, 6 \\
                &       & Attention encoder layers       & 4, \textbf{6}   \\
                &       & Attention encoder heads        & 8              \\
                &       & Decoder layers                 & 4, \textbf{6}   \\
                &       & Decoder heads                  & 8              \\
                &       & Number of accumulation steps   & 4              \\ 
                &       & Batch size                     & 4096            \\ 
                \bottomrule
Graph2SMILES    &  Full & Total Steps                    & 1200000          \\ 
                &       & Noam learning rate factor      & 2          \\ 
                &       & Dropout                        & 0.1          \\ 
                \bottomrule
Graph2SMILES    &  50k, 5k, 
                1500, 500 & Total Steps                  & 300000          \\ 
                &       & Noam learning rate factor      & 2, \textbf{4}          \\ 
                &       & Dropout                        & 0.1, \textbf{0.3}      \\ 
                \bottomrule
\end{tabular}
\end{table}



\clearpage
\section{Low data regime experiments}\label{si_lowdata}
A data-constrained evaluation is carried out to demonstrate the effectiveness of FlowER as compared to sequence translation based approach in predicting reaction outcome with only very small amount of data. Training data is randomly subsampled into smaller partitions of 50k, 5k, 1500, and 500. Using a training set size of 1500 roughly corresponds to 1 example per reaction class \& condition pair from the full training set. Despite this very small amount of data, even when training from scratch, FlowER's problem formulation and architecture provides impressive evidence of generalization on the reaction outcome prediction task. 


\clearpage
\section{Recovering reaction pathways with FlowER}

We analyzed reactions from patents reported in 2024 \cite{pistachio} that were not assigned to any of the 2,639 predefined reaction types in NameRxn \cite{namerxn}. These cases include reactions that may involve unexpected product formations, combinations of known transformations, or mechanistic pathways that are not explicitly categorized within existing classification schemes.


We performed a narrow beam search (width 2, depth 9) on 22,000 unrecognized reactions and successfully recovered 351 products. 

The five reaction pathways successfully recovered by FlowER, originating from reactions reported in nine patents from 2024 \cite{US20240139149A1, US20240101584A1, US20240116946A1, US20240239807A1, US20240128510A1, US20240150295, US20240150296, US20240140962A1, US20240122180A1, US20240124445A1, US20240118617A1}, are illustrated in Fig.~\ref{fig_unrecognized_8} through Fig.~\ref{fig_unrecognized_9}.

It is important to note that the reactants shown here correspond to those recorded in Pistachio \cite{pistachio}, including both reactants and reagents. However, certain chemicals listed in the patents may be missing, or additional compounds introduced during work-up might be included. Furthermore, the overall neutralization reaction was not explicitly considered when curating the mechanistic dataset, meaning that some molecules may remain unneutralized. Additionally, since the stronger acid/base species were not prioritized for use in FlowER’s design, the model may not predict these cases accurately.


\begin{figure*}[h]
\centering
\includegraphics[width=\textwidth]{Figures/SI_unrecognized8.pdf}
\caption{
An example reaction reported in 
2024 \cite{US20240124445A1}, which was not assigned to a specific reaction class in our dataset \cite{pistachio, namerxn}.
FlowER successfully reproduces the experimentally-recorded product in green. Noteworthy in this example is the proper handling of intermediate protonation states throughout the acid-mediated amination (i.e., formation of an oxonium prior to nucleophilic attack); formation of the iminium in strongly acidic conditions is well-precedented \cite{afanasyev2019reductive}. Direct $\alpha$-allylation of imines has been described by \citet{alam2014stereoselective}; a Zimmerman-Traxler six-membered transition state is proposed \cite{mejuch2013axial}. The numbers above each arrow represent the number of times that reaction was proposed during 16 independent sampling steps. }\label{fig_unrecognized_8}
\end{figure*}

\begin{figure*}[h]
\centering
\includegraphics[width=\textwidth]{Figures/SI_unrecognized1.pdf}
\caption{
An example reaction reported in 
2024 \cite{US20240139149A1}, which was not assigned to a specific reaction class in our dataset \cite{pistachio, namerxn}. 
FlowER successfully reproduces the experimentally-recorded product in green. FlowER demonstrates strong  performance when handling the first \pKa-dependent substitution (ammonium \pKa = $\sim$9-11, carbonate \pKa = $\sim$10.33). However, the second deprotonation (en route to dihydro-dibenzo[\textit{c,e}]azepine) requires a second equivalent of carbonate to proceed (bicarbonate \pKa = $\sim$ 6.35). While FlowER has not seen examples of bicarbonate-mediated deprotonations, the reaction was not recorded with the required number of carbonates; accordingly, FlowER uses the 'best available' base to provide a reasonable pathway to the observed product despite this omission. This example showcases the need for detailed procedural record-keeping to build next-generation models that are mass-balance- and mechanism-aware.  The numbers above each arrow represent the number of times that reaction was proposed during 16 independent sampling steps. }\label{fig_unrecognized_1}
\end{figure*}

\begin{figure*}[h]
\centering
\includegraphics[width=\textwidth]{Figures/SI_unrecognized2.pdf}
\caption{
An example reaction reported in 
2024 \cite{US20240101584A1}, which was not assigned to a specific reaction class in our dataset \cite{pistachio, namerxn}. 
FlowER successfully reproduces the experimentally-recorded product in green. Despite the operational simplicity of the above mechanism, FlowER was provided no examples at training incorporating Ag. Successful reconstruction of this S$_\text{N}$2 mechanism demonstrates a surprising ability to handle out-of-distribution elements, correctly generating the expected ester. The numbers above each arrow represent the number of times that reaction was proposed during 16 independent sampling steps. }\label{fig_unrecognized_2}
\end{figure*}

\begin{figure*}[h]
\centering
\includegraphics[width=\textwidth]{Figures/SI_unrecognized3.pdf}
\caption{
An example reaction reported in 
2024 \cite{US20240116946A1, US20240239807A1}. 
FlowER successfully reproduces the experimentally-recorded product in green. Importantly, FlowER captures the commonly accepted unimolecular mechanism for demethylation \cite{benton1942cleavage}, formalized by \citet{mcomie1968demethylation} in 1968. Closer inspection of the patent conditions reveals the ground truth conditions use of 3 equivalents of \ce{BBr3}. This offers a potential alternative mechanism, wherein the \textit{tert}-butyldimethylsilyl (TBS) ether activation is accomplished with a second equivalent of \ce{BBr3}. We note this nuanced mechanism is still actively being researched \cite{kosak2015ether}, and several alternative mechanisms have been proposed \cite{sousa2013bbr3}. The numbers above each arrow represent the number of times that reaction was proposed during 16 independent sampling steps.}\label{fig_unrecognized_2}
\end{figure*}

\begin{figure*}[h]
\centering
\includegraphics[width=\textwidth]{Figures/SI_unrecognized4.pdf}
\caption{
An example reaction reported in 
2024 \cite{US20240128510A1}. 
FlowER successfully reproduces the experimentally-recorded product in green. We select this particular example to highlight FlowER's handling of a substitution reaction at a non-carbon atom (i.e., the phosphorus of a phosphochloridite). Interestingly, FlowER opts for an anionic mechanism with the trimethylhydroxysilane (\pKa = $\sim$ 11) in the presence of TEA (10.8). This suggests the neutral hydroxysilane would operate as the primary nucleophile under these conditions,  though the similar relative acidities of both substrates provide an avenue for both mechanistic pathways. The numbers above each arrow represent the number of times that reaction was proposed during 16 independent sampling steps.}\label{fig_unrecognized_4}
\end{figure*}


\begin{figure*}[h]
\centering
\includegraphics[width=\textwidth]{Figures/SI_unrecognized5.pdf}
\caption{
An example reaction reported in 
2024 \cite{US20240150295, US20240150296}. 
FlowER successfully reproduces the experimentally-recorded product in green. We highlight this example to showcase a named reaction that was not identified in Pistachio due to two recorded main products (See the Extended Fig. S2a); Specifically, the acid-mediated Knorr pyrazole synthesis \cite{knorr1883einwirkung}. While the commonly accepted mechanism in strongly acidic solution begins with formation of an oxonoium intermediate \cite{flood2018leveraging}, we find FlowER exhibits a preference for formation of a Zwitterionic intermediate. Nonetheless, FlowER accurately generates the required oxonium en route to both dehydration steps. This is noteworthy, as the formation of hydroxide in strongly acidic conditions can be problematic for other models. Additionally, FlowER accurately predicts the formation of both possible pyrazoles, showcasing the utility of FlowER for exploring mechanistic pathways leading to multiple possible products. The numbers above each arrow represent the number of times that reaction was proposed during 16 independent sampling steps. }\label{fig_unrecognized_5}
\end{figure*}

\begin{figure*}[h]
\centering
\includegraphics[width=\textwidth]{Figures/SI_unrecognized6.pdf}
\caption{
An example reaction reported in 
2024 \cite{US20240140962A1}. 
FlowER successfully reproduces the experimentally-recorded product in green. In this example, we showcase the ability for FlowER to identify and apply named reaction mechanisms (i.e., sulfone-mediated S$_\text{N}$Ar reaction \cite{patel2020sulfone}). We highlight the correct handling of terminal alcohols in the presence of \ce{NaH}, and the subsequent release of methanesulfinate to prepare the final fused (2-methoxyethoxy)thiophene. The numbers above each arrow represent the number of times that reaction was proposed during 16 independent sampling steps. }\label{fig_unrecognized_6}
\end{figure*}


\begin{figure*}[h]
\centering
\includegraphics[width=\textwidth]{Figures/SI_unrecognized7.pdf}
\caption{
An example reaction reported in 
2024 \cite{US20240122180A1}.
FlowER successfully reproduces the experimentally-recorded product in green. Importantly, without the presence of an exogenous base, imidazole is correctly proposed to form a Zwitterionic intermediate upon nucleophilic attack of formaldehyde (rather than first undergoing deprotonation). The numbers above each arrow represent the number of times that reaction was proposed during 16 independent sampling steps. }\label{fig_unrecognized_7}
\end{figure*}


\begin{figure*}[h]
\centering
\includegraphics[width=\textwidth]{Figures/SI_unrecognized9.pdf}
\caption{
An example reaction reported in 
2024 \cite{US20240118617A1}. 
FlowER successfully reproduces the experimentally-recorded product in green. 
We highlight this example to showcase how FlowER navigates the entire complex reagent pool to identify the correct mechanism. 
While order of addition is undoubtedly of utmost importance in this reaction (i.e., Grignard reagents will rapidly act as a strong base in the presence of a carboxylic acid), FlowER is able to identify a reasonable mechanistic pathway to the observed product. In this particular case, FlowER implicitly identifies the requirement for two subsequent reactions: a Grignard addition and base-mediated esterification. The numbers above each arrow represent the number of times that reaction was proposed during 16 independent sampling steps. }\label{fig_unrecognized_9}
\end{figure*}

\clearpage




\clearpage
\section{Fine-tuning on previously-unseen reaction types}
\subsection{Performance on out-of-distribution reaction types}
To evaluate FlowER’s ability to generalize to new reaction types, we selected 12 reaction types that were not included in the original training set: Transamidation, Prilezhaev epoxidation, Diels-Alder cycloaddition, Staudinger reduction, Hydroxy to bromo, N-Bn deprotection, Negishi coupling, Carboxylic acid sulfonamide condensation, Menshutkin reaction, Urea Schotten-Baumann, Appel bromination, and SEM protection. These reaction types were curated using the same method for curating our mechanistic dataset as described in Section \ref{si_dataset}.  

For fine-tuning, only 32 overall reactions were used as the training set. The validation set contained at least two overall reactions per reaction type, while the remaining reactions were assigned to the test set. 

The best-performing model, selected based on validation set accuracy, was evaluated on the test set. The top-\textit{k} step and pathway accuracies on the test set are presented in Fig.~\ref{fig_fine_tune_performance}.

The elementary steps of the 12 reaction types and their top-1 step accuracy before and after fine-tuning can be found in Fig.~\ref{fig_ood_1,2,3} through Fig.~\ref{fig_ood_12}.


\subsection{Assessing catastrophic forgetting after fine-tuning}

To investigate whether fine-tuning caused the model to catastrophically forget reactions learned during the original training phase, we evaluated FlowER and its fine-tuned variants on a subset of the original test set. Specifically, we randomly sampled 10\% of the original test set and compared the performance of the original FlowER model with the fine-tuned models.  

The top-\textit{k} step accuracy for FlowER and the fine-tuned models is presented in Fig.~\ref{fig_step_forgetting}, while the top-\textit{k} pathway accuracy is shown in Fig.~\ref{fig_path_forgetting}. 

Among the 12 fine-tuned models, 11 showed no substantial drop in accuracy compared to the original model, indicating minimal catastrophic forgetting. The exception was the Transamidation fine-tuned model, which exhibited a noticeable performance drop. This suggests that, in most cases, FlowER retains its ability to predict previously learned reactions even after fine-tuning on unseen reaction types.



\begin{figure*}[h]
\centering
\includegraphics[width=0.95\textwidth]{Figures/SI_ood_perfomance.pdf}
\caption{
Top-\textit{k} accuracy of FlowER on previously unseen reaction types after fine-tuning on 32 examples. 
Each subplot represents the performance on one of the 12 reaction classes that were excluded from the initial training set. 
Red and blue curves indicate top-\textit{k} step accuracy and top-\textit{k} pathway accuracy, respectively. 
}\label{fig_fine_tune_performance}
\end{figure*}

\clearpage

\begin{figure*}[h]
\centering
\includegraphics[width=0.95\textwidth]{Figures/SI_ood_step.pdf}
\caption{
Performance of fine-tuned models on a random 10\% subset of the original test set compared to the original FlowER without fine-tuning, in terms of top-$k$ step accuracy. 
This evaluation was conducted to assess catastrophic forgetting, where a model forgets previously learned knowledge after fine-tuning on new reaction types. 
For each fine-tuned model, we recomputed the top-$k$ step accuracy on the sampled test subset and compared it against the original FlowER model.
}\label{fig_step_forgetting}
\end{figure*}

\clearpage

\begin{figure*}[h]
\centering
\includegraphics[width=0.95\textwidth]{Figures/SI_ood_path.pdf}
\caption{
Performance of fine-tuned models on a random 10\% subset of the original test set compared to the original FlowER without fine-tuning, in terms of top-$k$ pathway accuracy. 
This evaluation was conducted to assess catastrophic forgetting, where a model forgets previously learned knowledge after fine-tuning on new reaction types. 
For each fine-tuned model, we recomputed the top-$k$ pathway accuracy on the sampled test subset and compared it against the original FlowER model.
}\label{fig_path_forgetting}
\end{figure*}

\clearpage

\begin{figure*}[h]
\centering
\includegraphics[width=0.95\textwidth]{Figures/SI_ood_123.pdf}
\caption{
Changes in top-1 step accuracy for elementary steps of \textbf{1} transamidation, \textbf{2} Prilezhaev epoxidation, and \textbf{3} Diels-Alder cycloaddition before and after fine-tuning on 32 reaction examples.
}\label{fig_ood_1,2,3}
\end{figure*}

\begin{figure*}[h]
\centering
\includegraphics[width=0.95\textwidth]{Figures/SI_ood_4.pdf}
\caption{
Changes in top-1 step accuracy for elementary steps of \textbf{4} Staudinger reduction before and after fine-tuning on 32 examples.
}\label{fig_ood_4}
\end{figure*}


\begin{figure*}[h]
\centering
\includegraphics[width=0.95\textwidth]{Figures/SI_ood_56.pdf}
\caption{
Changes in top-1 step accuracy for elementary steps of \textbf{5} Hydroxy to bromo and \textbf{6} N-Bn deprotection before and after fine-tuning on 32 examples.
}\label{fig_ood_5,6}
\end{figure*}


\begin{figure*}[h]
\centering
\includegraphics[width=0.95\textwidth]{Figures/SI_ood_7.pdf}
\caption{
Changes in top-1 step accuracy for elementary steps of \textbf{7} Negishi coupling before and after fine-tuning on 32 examples.
}\label{fig_ood_7}
\end{figure*}


\begin{figure*}[h]
\centering
\includegraphics[width=0.95\textwidth]{Figures/SI_ood_89.pdf}
\caption{
Changes in top-1 step accuracy for elementary steps of \textbf{8} Carboxylic acid + sulfonamide condensation and \textbf{9} Menshutkin reaction before and after fine-tuning on 32 examples.
}\label{fig_ood_8,9}
\end{figure*}


\begin{figure*}[h]
\centering
\includegraphics[width=0.95\textwidth]{Figures/SI_ood_10.pdf}
\caption{
Changes in top-1 step accuracy for elementary steps of \textbf{10} Urea Schotten-Baumann before and after fine-tuning on 32 examples.
}\label{fig_ood_10}
\end{figure*}

\begin{figure*}[h]
\centering
\includegraphics[width=0.95\textwidth]{Figures/SI_ood_11.pdf}
\caption{
Changes in top-1 step accuracy for elementary steps of \textbf{11} Appel bromination before and after fine-tuning on 32 examples.
}\label{fig_ood_11}
\end{figure*}

\begin{figure*}[h]
\centering
\includegraphics[width=0.95\textwidth]{Figures/SI_ood_12.pdf}
\caption{
Changes in top-1 step accuracy for elementary steps of \textbf{12} SEM protection before and after fine-tuning on 32 examples.
}\label{fig_ood_12}
\end{figure*}

\clearpage

\bibliography{sn-bibliography,references}% common bib file


\end{document}