\section{Introduction}
\label{sec:introduction}

Discriminating between varieties of a given language is an important natural language processing (NLP) task~\cite{joshi2024natural}. Over time, populations that share a common language can evolve distinctive speech traits due to geographical and cultural factors, including migration and the influence of other languages~\cite{Raposo_Vicente_Veloso_2021}. Recently, this importance became even more pronounced with the advent of variety-specific large language models, where variety discrimination plays a pivotal role~\cite{Rodrigues_2023}. Whether in the pre-training, fine-tuning, or evaluation phase, having a highly effective system to discriminate between varieties reduces the amount of human supervision required, accelerating the production of curated mono-variety datasets~\cite{ohman2023nordic}. However, developing such a system presents considerable challenges. Classifiers often struggle to identify linguistically relevant features, showing a tendency to be biased towards non-linguistic factors, such as named entities and thematic content~\cite{diwersy2014weakly}. Consequently, these classifiers exhibit limited transfer capabilities to domains not represented in the training set, significantly restricting their utility in multi-domain applications~\cite{lui2011cross, nguyen-etal-2021-cross-domain}.

A language in which variety identification is particularly challenging is Portuguese. It is spoken by more than $200$ million people worldwide and serves as the official language of eight nations on five continents, each with its own variety. However, $70\%$ of Portuguese speakers are Brazilian citizens\footnote{Statistic inferred form Wikipedia \url{https://en.wikipedia.org/wiki/Portuguese_language#Lusophone_countries}}, which implies that resources labeled as Portuguese are dominated by this language variety. Another important characteristic of Portuguese is that, unlike languages where differences are predominantly phonological, such as those of the North Germanic family~\cite{Holmberg2008}, the widespread of Portuguese has fostered considerable phonological, morphological, lexical, syntactic and semantic variations between Portuguese varieties~\cite{Brito2016}. In the development of language models, for example, this variety divergence has practical implications; models trained in Brazilian Portuguese generate texts that are markedly distinct from those trained in other Portuguese varieties~\cite{Rodrigues_2023}. This fact restrains the adoption of these models outside of Brazil in domains where formal non-Brazilian text is required, as is the case of legal and medical applications. This underscores the practical importance of developing effective LVI systems that can be deployed in production.

In this study, we describe the development of a cross-domain LVI classifier that discriminates between Brazilian and European Portuguese. To accomplish that, we start with a comprehensive listing of Portuguese LVI resources. The lack of multi-domain corpora motivated us to compile our own dataset. This corpus was then used in the development of our LVI classifier. For the training procedure we devised a training protocol that takes into account the cross-domain capabilities of models during evaluation. Furthermore, we also study the impact of masking named entities and thematic content embedded in the training corpus, a process named delexicalization~\cite{lui2014exploring}. To summarize, the contributions of this work are the following:

\begin{enumerate}

\item We introduce a novel cross-domain, silver-labeled LVI corpus for Brazilian and European Portuguese, compiled from open-license datasets;

\item We examine the impact of different levels of delexicalization on the overall effectiveness of LVI models;

\item We propose a training protocol for LVI models that yields better generalization performance;

\item We release the first open-source Portuguese LVI model, providing a valuable resource for future research and practical applications.
\end{enumerate}

The remainder of this paper is organized as follows: Section~\ref{sec:related_work} offers a comprehensive literature review on the state-of-the-art in Portuguese LVI. In Section~\ref{sec:create_corpus}, we introduce our compiled dataset, PtBrVarId, and present relevant statistics along with the results of a manual evaluation of the quality of the dataset. Section~\ref{sec:setup} describes the training protocol and the models developed, including the baselines and benchmarks used for comparison. The results are presented in Section~\ref{sec:results}, followed by a discussion of future research directions in Section~\ref{sec:conclusions}. 

%The information in this paper, is complemented by Appendixes, where a linguistic analysis to Portuguese variety identification is provided (Appendix~\ref{app:pt_br_pt_pt}).

