% \section{Limitations}

% We identified two main limitations related with the dataset used that directly relate to the work developed. First, despite our efforts, parts of the evaluation are still founded on silver-labeled data. Which, as we mentioned in the paper, is often considered in the LVI literature misleading. Additional manually annotations are desirable to increase the confidence in the results obtained.   

% Second, many documents collected online do not have sufficient linguistic traces to confidently classify them as a single variety. To surpass this limitation, the DSL-TL corpus introduced the possibility of a ``Both/Neither'' class to signal these cases. Our silver-labelling process does not take into consideration those cases, which introduces entropy in the training data and could potential negatively impact the overall effectiveness of the models developed with our corpus.

%\section{Ethical Considerations}

%We identify two ethical aspects our work engages with that should be  discussed to benefit transparency and open-minded science. First, we compile existing corpora with permissive scientific licensing. We use Brazilian datasets related to hate speech and social media comments in the social media domain. Unfortunately, the lack of respect witnessed on social networks is transposed into our corpus, with vast amounts of racism, xenophobia, toxic masculinity, and harassment presented in our social media corpus. Also, the silver-label nature of the social media domain is particularly challenging because it often mentions other persons by their names or other unique forms of mentioning; additional means of anonymization should be implied in a 1.0 Version of our corpus since there is no linguistic gain in incorporating these mentions that can impact negatively the privacy of individuals.

% Secondly, it is imperative to mention that our multinational research team is composed of elements from four continents, including Portuguese and Brazilian elements that were consulted during the development of this tool. It was mentioned that in both countries, there are negative attitudes towards the other variant of Portuguese, with small discussions in Portugal claiming the "purity of the language" as a former colonial power and in Brazil claiming the right to the "evolution of a self-linguistic identity" as a new rising multicultural power.


% In the past, some reviews of the literature point to work in this field by Balkan researchers with strong political intentions. Even though we acknowledge that our research can fuel the discussion on the Portuguese language in this topic, we accept the burden because we believe that the Portuguese language as an all benefits from the difference in varieties, not only the European and Brazilian ones, but also the many African, Asian (Macau) and Oceanic's (East-Timor). As mentioned in the conclusions, one of the future work points is to extend our work to these varieties to create a Portuguese corpus with all existent varieties in an actual exercise of diversity rather than nefarious purity discussions.