\begin{abstract}
    Class-incremental learning (CIL) seeks to enable a model to sequentially learn new classes while retaining knowledge of previously learned ones. Balancing flexibility and stability remains a significant challenge, particularly when the task ID is unknown. To address this, our study reveals that the gap in feature distribution between novel and existing tasks is primarily driven by differences in mean and covariance moments. Building on this insight, we propose a novel semantic drift calibration method that incorporates mean shift compensation and covariance calibration. Specifically, we calculate each class's mean by averaging its sample embeddings and estimate task shifts using weighted embedding changes based on their proximity to the previous mean, effectively capturing mean shifts for all learned classes with each new task. We also apply Mahalanobis distance constraint for covariance calibration, aligning class-specific embedding covariances between old and current networks to mitigate the covariance shift. Additionally, we integrate a feature-level self-distillation approach to enhance generalization. Comprehensive experiments on commonly used datasets demonstrate the effectiveness of our approach. 
\end{abstract}

