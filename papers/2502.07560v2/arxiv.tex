%%%%%%%% ICML 2024 EXAMPLE LATEX SUBMISSION FILE %%%%%%%%%%%%%%%%%

\documentclass{article}

% Recommended, but optional, packages for figures and better typesetting:
\usepackage{microtype}
\usepackage{graphicx}
\usepackage{subfigure}
\usepackage{booktabs} % for professional tables

% hyperref makes hyperlinks in the resulting PDF.
% If your build breaks (sometimes temporarily if a hyperlink spans a page)
% please comment out the following usepackage line and replace
% \usepackage{icml2024} with \usepackage[nohyperref]{icml2024} above.
\usepackage{hyperref}

\usepackage{subcaption}
% Attempt to make hyperref and algorithmic work together better:
\newcommand{\theHalgorithm}{\arabic{algorithm}}

% Use the following line for the initial blind version submitted for review:
\usepackage[accepted]{icml2024}

% If accepted, instead use the following line for the camera-ready submission:
%\usepackage[accepted]{icml2024}

% For theorems and such
\usepackage{amsmath}
\usepackage{amssymb}
\usepackage{mathtools}
\usepackage{amsthm}
\usepackage{pifont}
\usepackage{diagbox}

% if you use cleveref..
\usepackage[capitalize,noabbrev]{cleveref}
\usepackage{bbding}
%%%%%%%%%%%%%%%%%%%%%%%%%%%%%%%%
% THEOREMS
%%%%%%%%%%%%%%%%%%%%%%%%%%%%%%%%
\theoremstyle{plain}
\newtheorem{theorem}{Theorem}[section]
\newtheorem{proposition}[theorem]{Proposition}
\newtheorem{lemma}[theorem]{Lemma}
\newtheorem{corollary}[theorem]{Corollary}
\theoremstyle{definition}
\newtheorem{definition}[theorem]{Definition}
\newtheorem{assumption}[theorem]{Assumption}
\theoremstyle{remark}
\newtheorem{remark}[theorem]{Remark}

% Todonotes is useful during development; simply uncomment the next line
%    and comment out the line below the next line to turn off comments
%\usepackage[disable,textsize=tiny]{todonotes}
\usepackage[textsize=tiny]{todonotes}


% The \icmltitle you define below is probably too long as a header.
% Therefore, a short form for the running title is supplied here:
\icmltitlerunning{Navigating Semantic Drift in Task-Agnostic Class-Incremental Learning}

\begin{document}

\twocolumn[
\icmltitle{Navigating Semantic Drift in Task-Agnostic Class-Incremental Learning}

% It is OKAY to include author information, even for blind
% submissions: the style file will automatically remove it for you
% unless you've provided the [accepted] option to the icml2024
% package.

% List of affiliations: The first argument should be a (short)
% identifier you will use later to specify author affiliations
% Academic affiliations should list Department, University, City, Region, Country
% Industry affiliations should list Company, City, Region, Country

% You can specify symbols, otherwise they are numbered in order.
% Ideally, you should not use this facility. Affiliations will be numbered
% in order of appearance and this is the preferred way.
%\icmlsetsymbol{equal}{*}

\begin{icmlauthorlist}
\icmlauthor{Fangwen Wu}{yyy}
\icmlauthor{Lechao Cheng\textsuperscript{\Envelope}}{sch1}
\icmlauthor{Shengeng Tang}{sch1}
\icmlauthor{Xiaofeng Zhu}{yyy}
\icmlauthor{Chaowei Fang}{sch2}\\
\icmlauthor{Dingwen Zhang}{sch3}
\icmlauthor{Meng Wang}{sch1}
%\icmlauthor{}{sch}
% \icmlauthor{Firstname8 Lastname8}{sch}
% \icmlauthor{Firstname8 Lastname8}{yyy,comp}
%\icmlauthor{}{sch}
%\icmlauthor{}{sch}
\end{icmlauthorlist}

\icmlaffiliation{yyy}{Zhejiang Lab}
\icmlaffiliation{sch1}{Hefei University of Technology}
\icmlaffiliation{sch2}{Xidian University}
\icmlaffiliation{sch3}{Northwestern Polytechnical University}

\icmlcorrespondingauthor{Lechao Cheng}{chenglc@hfut.edu.cn}
%\icmlcorrespondingauthor{Firstname2 Lastname2}{first2.last2@www.uk}

% You may provide any keywords that you
% find helpful for describing your paper; these are used to populate
% the "keywords" metadata in the PDF but will not be shown in the document
\icmlkeywords{Machine Learning, ICML}

\vskip 0.3in
]

% this must go after the closing bracket ] following \twocolumn[ ...

% This command actually creates the footnote in the first column
% listing the affiliations and the copyright notice.
% The command takes one argument, which is text to display at the start of the footnote.
% The \icmlEqualContribution command is standard text for equal contribution.
% Remove it (just {}) if you do not need this facility.
\printAffiliations{}
%\printAffiliationsAndNotice{}  % leave blank if no need to mention equal contribution
%\printAffiliationsAndNotice{\icmlEqualContribution} % otherwise use the standard text.


% \section{Electronic Submission}
% \label{submission}

% Submission to ICML 2024 will be entirely electronic, via a web site
% (not email). Information about the submission process and \LaTeX\ templates
% are available on the conference web site at:
% \begin{center}
% \textbf{\texttt{http://icml.cc/}}
% \end{center}

% The guidelines below will be enforced for initial submissions and
% camera-ready copies. Here is a brief summary:
% \begin{itemize}
% \item Submissions must be in PDF\@. 
% \item \textbf{New to this year}: If your paper has appendices, submit the appendix together with the main body and the references \textbf{as a single file}. Reviewers will not look for appendices as a separate PDF file. So if you submit such an extra file, reviewers will very likely miss it.
% \item Page limit: The main body of the paper has to be fitted to 8 pages, excluding references and appendices; the space for the latter two is not limited. For the final version of the paper, authors can add one extra page to the main body.
% \item \textbf{Do not include author information or acknowledgements} in your
%     initial submission.
% \item Your paper should be in \textbf{10 point Times font}.
% \item Make sure your PDF file only uses Type-1 fonts.
% \item Place figure captions \emph{under} the figure (and omit titles from inside
%     the graphic file itself). Place table captions \emph{over} the table.
% \item References must include page numbers whenever possible and be as complete
%     as possible. Place multiple citations in chronological order.
% \item Do not alter the style template; in particular, do not compress the paper
%     format by reducing the vertical spaces.
% \item Keep your abstract brief and self-contained, one paragraph and roughly
%     4--6 sentences. Gross violations will require correction at the
%     camera-ready phase. The title should have content words capitalized.
% \end{itemize}

% \subsection{Submitting Papers}

% \textbf{Paper Deadline:} The deadline for paper submission that is
% advertised on the conference website is strict. If your full,
% anonymized, submission does not reach us on time, it will not be
% considered for publication. 

% \textbf{Anonymous Submission:} ICML uses double-blind review: no identifying
% author information may appear on the title page or in the paper
% itself. \cref{author info} gives further details.

% \textbf{Simultaneous Submission:} ICML will not accept any paper which,
% at the time of submission, is under review for another conference or
% has already been published. This policy also applies to papers that
% overlap substantially in technical content with conference papers
% under review or previously published. ICML submissions must not be
% submitted to other conferences and journals during ICML's review
% period.
% %Authors may submit to ICML substantially different versions of journal papers
% %that are currently under review by the journal, but not yet accepted
% %at the time of submission.
% Informal publications, such as technical
% reports or papers in workshop proceedings which do not appear in
% print, do not fall under these restrictions.

% \medskip

% Authors must provide their manuscripts in \textbf{PDF} format.
% Furthermore, please make sure that files contain only embedded Type-1 fonts
% (e.g.,~using the program \texttt{pdffonts} in linux or using
% File/DocumentProperties/Fonts in Acrobat). Other fonts (like Type-3)
% might come from graphics files imported into the document.

% Authors using \textbf{Word} must convert their document to PDF\@. Most
% of the latest versions of Word have the facility to do this
% automatically. Submissions will not be accepted in Word format or any
% format other than PDF\@. Really. We're not joking. Don't send Word.

% Those who use \textbf{\LaTeX} should avoid including Type-3 fonts.
% Those using \texttt{latex} and \texttt{dvips} may need the following
% two commands:

% {\footnotesize
% \begin{verbatim}
% dvips -Ppdf -tletter -G0 -o paper.ps paper.dvi
% ps2pdf paper.ps
% \end{verbatim}}
% It is a zero following the ``-G'', which tells dvips to use
% the config.pdf file. Newer \TeX\ distributions don't always need this
% option.

% Using \texttt{pdflatex} rather than \texttt{latex}, often gives better
% results. This program avoids the Type-3 font problem, and supports more
% advanced features in the \texttt{microtype} package.

% \textbf{Graphics files} should be a reasonable size, and included from
% an appropriate format. Use vector formats (.eps/.pdf) for plots,
% lossless bitmap formats (.png) for raster graphics with sharp lines, and
% jpeg for photo-like images.

% The style file uses the \texttt{hyperref} package to make clickable
% links in documents. If this causes problems for you, add
% \texttt{nohyperref} as one of the options to the \texttt{icml2024}
% usepackage statement.


% \subsection{Submitting Final Camera-Ready Copy}

% The final versions of papers accepted for publication should follow the
% same format and naming convention as initial submissions, except that
% author information (names and affiliations) should be given. See
% \cref{final author} for formatting instructions.

% The footnote, ``Preliminary work. Under review by the International
% Conference on Machine Learning (ICML). Do not distribute.'' must be
% modified to ``\textit{Proceedings of the
% $\mathit{41}^{st}$ International Conference on Machine Learning},
% Vienna, Austria, PMLR 235, 2024.
% Copyright 2024 by the author(s).''

% For those using the \textbf{\LaTeX} style file, this change (and others) is
% handled automatically by simply changing
% $\mathtt{\backslash usepackage\{icml2024\}}$ to
% $$\mathtt{\backslash usepackage[accepted]\{icml2024\}}$$
% Authors using \textbf{Word} must edit the
% footnote on the first page of the document themselves.

% Camera-ready copies should have the title of the paper as running head
% on each page except the first one. The running title consists of a
% single line centered above a horizontal rule which is $1$~point thick.
% The running head should be centered, bold and in $9$~point type. The
% rule should be $10$~points above the main text. For those using the
% \textbf{\LaTeX} style file, the original title is automatically set as running
% head using the \texttt{fancyhdr} package which is included in the ICML
% 2024 style file package. In case that the original title exceeds the
% size restrictions, a shorter form can be supplied by using

% \verb|\icmltitlerunning{...}|

% just before $\mathtt{\backslash begin\{document\}}$.
% Authors using \textbf{Word} must edit the header of the document themselves.
\begin{abstract}
    Class-incremental learning (CIL) seeks to enable a model to sequentially learn new classes while retaining knowledge of previously learned ones. Balancing flexibility and stability remains a significant challenge, particularly when the task ID is unknown. To address this, our study reveals that the gap in feature distribution between novel and existing tasks is primarily driven by differences in mean and covariance moments. Building on this insight, we propose a novel semantic drift calibration method that incorporates mean shift compensation and covariance calibration. Specifically, we calculate each class's mean by averaging its sample embeddings and estimate task shifts using weighted embedding changes based on their proximity to the previous mean, effectively capturing mean shifts for all learned classes with each new task. We also apply Mahalanobis distance constraint for covariance calibration, aligning class-specific embedding covariances between old and current networks to mitigate the covariance shift. Additionally, we integrate a feature-level self-distillation approach to enhance generalization. Comprehensive experiments on commonly used datasets demonstrate the effectiveness of our approach. The source code is available at \href{https://github.com/fwu11/MACIL.git}{https://github.com/fwu11/MACIL.git}.
\end{abstract}


% \begin{abstract}  
Test time scaling is currently one of the most active research areas that shows promise after training time scaling has reached its limits.
Deep-thinking (DT) models are a class of recurrent models that can perform easy-to-hard generalization by assigning more compute to harder test samples.
However, due to their inability to determine the complexity of a test sample, DT models have to use a large amount of computation for both easy and hard test samples.
Excessive test time computation is wasteful and can cause the ``overthinking'' problem where more test time computation leads to worse results.
In this paper, we introduce a test time training method for determining the optimal amount of computation needed for each sample during test time.
We also propose Conv-LiGRU, a novel recurrent architecture for efficient and robust visual reasoning. 
Extensive experiments demonstrate that Conv-LiGRU is more stable than DT, effectively mitigates the ``overthinking'' phenomenon, and achieves superior accuracy.
\end{abstract}  
\section{Introduction}

\begin{figure}[!htb]
\centering
\subfigure[Semantic Drift]{
        \centering
        \includegraphics[trim=0.5cm 2cm 3cm 0.1cm, clip, width=0.46\linewidth]{images/original_shift.pdf}
        \label{fig:ori_shift}
        }
\subfigure[Calibration]{
        \centering
        \includegraphics[trim=0.1cm 1cm 3cm 1cm, clip, width=0.46\linewidth]{images/shift_cali.pdf}
        \label{fig:shift_cali}
        }
% \begin{subfigure}[b]{0.48\linewidth}
%         \centering
%         \includegraphics[width=\linewidth]{images/calibration_result.pdf}
%         \label{fig:shift_cali}
%         \caption{Calibration}
% \end{subfigure}%
% \includegraphics[width=0.23\textwidth]{images/original_vs_shift.pdf}
% \includegraphics[width=0.23\textwidth]{images/calibration_result.pdf}
%\vspace{-5mm}
\caption{\small As new tasks are learned, the categories from previously tasks in the latest updated model continuously experience shifts in their means and variances, referred to as (a) \textbf{Semantic Drift}. In this paper, we calibrate such semantic drift by applying explicit mean shift compensation and implicit variance constraints (b).}
\vspace{-5mm}
\label{fig:teaser}
\end{figure}
%\input{sections/fig_shif

Continual Learning, also referred to as lifelong learning, aims to enable machine learning models to sequentially learn multiple tasks over their life-cycle without requiring retraining or access to data from previous tasks \cite{rebuffi2017icarl}. The primary goal of continual learning is to facilitate knowledge accumulation and transfer, allowing models to adapt quickly to new, unseen tasks while maintaining robust performance on previously learned tasks \cite{parisi2019continual}. This capability has broad applications in fields such as computer vision, robotics, and natural language processing.
% In the human perceptual system, neurosynaptic plasticity is a fundamental mechanism that enables the brain to adapt to dynamic environments by inducing physical changes in neural structures, allowing us to learn, remember, and evolve over time. However, neuroscientific studies have revealed that humans experience asymmetric interference effects during perceptual learning. To ensure the long-term durability of existing knowledge and avoid catastrophic interference when acquiring new tasks and skills, it is crucial to protect and consolidate previously learned information. These biological principles provide valuable insights into how machine learning models can mitigate catastrophic forgetting and enable dynamical adaptation, guiding researchers to address the classic \textit{Stability–Plasticity Dilemma}.

In recent years, the issue of model plasticity has become less prominent in deep learning-based approaches, primarily due to two factors: (1) the increasing capacity of deep models allows them to effectively over-fit new data, and (2) large-scale pre-training on extensive datasets equips models with powerful feature extraction capabilities \cite{he2019rethinking}. Parameter-efficient fine-tuning based on pretrained models has further enhanced model plasticity, as highlighted in numerous recent studies \cite{houlsby2019parameter,lester-etal-2021-power,hu2022lora}.

Despite these advancements, existing methods—such as regularization \cite{kirkpatrick2017overcoming}, memory replay \cite{lopez2017gradient}, and knowledge distillation \cite{li2017learning}—while improving stability to some extent, introduce additional costs. For example, (1) memory replay methods require storing and retraining on old task data, increasing storage and computational demands. (2) knowledge distillation involves additional computational overhead during the distillation process, complicating and slowing down training. These additional costs hinder the practical deployment of continual learning methods. Therefore, a key challenge in the field is to improve model stability while minimizing resource consumption and computational overhead \cite{wang2024comprehensive}.

In this paper, we build upon successful practices by leveraging parameter-efficient fine-tuning based on pretrained models to further analyze catastrophic forgetting. Through extensive experimental observations, we discovered that although low-rank adaptation (e.g., LoRA \cite{hu2022lora} ) based on pretrained models can effectively maintain model plasticity, the incremental integration of tasks and model updates induces feature mean and covariance shift—what we also term \textbf{Semantic Drift}. As illustrated in Figure~\ref{fig:teaser}, the feature distribution of the original task data undergoes significant mean shifts and changes in variance shape as more tasks are introduced.

Based on these observations, we propose to address semantic drift with both mean shift compensation and covariance calibration, which constrain the first-order and second-order moments of the features, respectively. Specifically, we compute mean class representations after learning a novel task as the average embedding of all samples in class. The shift between old and novel tasks is approximated by the weighted average of embedding shifts, where the weights are determined by the proximity of each embedding to the previous class mean. This approach effectively estimates the mean shift for all previously learned classes during each new task. We also introduce an implicit covariance calibration technique using Mahalanobis distance \cite{mahalanobis1936generalized} loss to address semantic drift. This method aligns the covariance matrices of embeddings from old and current networks for each class, ensuring consistent intra-class distributions. By leveraging the old network as "past knowledge", we compute class-specific covariance matrices and minimize the absolute differences in Mahalanobis distances between embedding pairs from both networks. This approach effectively mitigates covariance shift, maintaining model stability while allowing continual learning. As shown in Figure \ref{fig:shift_cali}, these constraints effectively maintain model stability while preserving plasticity. Additionally, we implement feature self-distillation for patch tokens, further enhancing feature stability. In summary, our main contributions are:
\begin{itemize}
    %\setlength{\itemsep}{0pt}
    %\setlength{\parsep}{0pt}
    \setlength{\parskip}{0pt}
    \item We delve into the exploration of the semantic drift problem in class-incremental learning and further propose efficient and straightforward solutions—mean shift compensation and covariance calibration, which significantly alleviate this challenge.
    \item We orchestrate an efficient task-agnostic continual learning framework that outperforms existing methods across multiple public datasets, demonstrating the superiority of our approach.
\end{itemize}

% \section{RELATED WORK}
\label{sec:relatedwork}
In this section, we describe the previous works related to our proposal, which are divided into two parts. In Section~\ref{sec:relatedwork_exoplanet}, we present a review of approaches based on machine learning techniques for the detection of planetary transit signals. Section~\ref{sec:relatedwork_attention} provides an account of the approaches based on attention mechanisms applied in Astronomy.\par

\subsection{Exoplanet detection}
\label{sec:relatedwork_exoplanet}
Machine learning methods have achieved great performance for the automatic selection of exoplanet transit signals. One of the earliest applications of machine learning is a model named Autovetter \citep{MCcauliff}, which is a random forest (RF) model based on characteristics derived from Kepler pipeline statistics to classify exoplanet and false positive signals. Then, other studies emerged that also used supervised learning. \cite{mislis2016sidra} also used a RF, but unlike the work by \citet{MCcauliff}, they used simulated light curves and a box least square \citep[BLS;][]{kovacs2002box}-based periodogram to search for transiting exoplanets. \citet{thompson2015machine} proposed a k-nearest neighbors model for Kepler data to determine if a given signal has similarity to known transits. Unsupervised learning techniques were also applied, such as self-organizing maps (SOM), proposed \citet{armstrong2016transit}; which implements an architecture to segment similar light curves. In the same way, \citet{armstrong2018automatic} developed a combination of supervised and unsupervised learning, including RF and SOM models. In general, these approaches require a previous phase of feature engineering for each light curve. \par

%DL is a modern data-driven technology that automatically extracts characteristics, and that has been successful in classification problems from a variety of application domains. The architecture relies on several layers of NNs of simple interconnected units and uses layers to build increasingly complex and useful features by means of linear and non-linear transformation. This family of models is capable of generating increasingly high-level representations \citep{lecun2015deep}.

The application of DL for exoplanetary signal detection has evolved rapidly in recent years and has become very popular in planetary science.  \citet{pearson2018} and \citet{zucker2018shallow} developed CNN-based algorithms that learn from synthetic data to search for exoplanets. Perhaps one of the most successful applications of the DL models in transit detection was that of \citet{Shallue_2018}; who, in collaboration with Google, proposed a CNN named AstroNet that recognizes exoplanet signals in real data from Kepler. AstroNet uses the training set of labelled TCEs from the Autovetter planet candidate catalog of Q1–Q17 data release 24 (DR24) of the Kepler mission \citep{catanzarite2015autovetter}. AstroNet analyses the data in two views: a ``global view'', and ``local view'' \citep{Shallue_2018}. \par


% The global view shows the characteristics of the light curve over an orbital period, and a local view shows the moment at occurring the transit in detail

%different = space-based

Based on AstroNet, researchers have modified the original AstroNet model to rank candidates from different surveys, specifically for Kepler and TESS missions. \citet{ansdell2018scientific} developed a CNN trained on Kepler data, and included for the first time the information on the centroids, showing that the model improves performance considerably. Then, \citet{osborn2020rapid} and \citet{yu2019identifying} also included the centroids information, but in addition, \citet{osborn2020rapid} included information of the stellar and transit parameters. Finally, \citet{rao2021nigraha} proposed a pipeline that includes a new ``half-phase'' view of the transit signal. This half-phase view represents a transit view with a different time and phase. The purpose of this view is to recover any possible secondary eclipse (the object hiding behind the disk of the primary star).


%last pipeline applies a procedure after the prediction of the model to obtain new candidates, this process is carried out through a series of steps that include the evaluation with Discovery and Validation of Exoplanets (DAVE) \citet{kostov2019discovery} that was adapted for the TESS telescope.\par
%



\subsection{Attention mechanisms in astronomy}
\label{sec:relatedwork_attention}
Despite the remarkable success of attention mechanisms in sequential data, few papers have exploited their advantages in astronomy. In particular, there are no models based on attention mechanisms for detecting planets. Below we present a summary of the main applications of this modeling approach to astronomy, based on two points of view; performance and interpretability of the model.\par
%Attention mechanisms have not yet been explored in all sub-areas of astronomy. However, recent works show a successful application of the mechanism.
%performance

The application of attention mechanisms has shown improvements in the performance of some regression and classification tasks compared to previous approaches. One of the first implementations of the attention mechanism was to find gravitational lenses proposed by \citet{thuruthipilly2021finding}. They designed 21 self-attention-based encoder models, where each model was trained separately with 18,000 simulated images, demonstrating that the model based on the Transformer has a better performance and uses fewer trainable parameters compared to CNN. A novel application was proposed by \citet{lin2021galaxy} for the morphological classification of galaxies, who used an architecture derived from the Transformer, named Vision Transformer (VIT) \citep{dosovitskiy2020image}. \citet{lin2021galaxy} demonstrated competitive results compared to CNNs. Another application with successful results was proposed by \citet{zerveas2021transformer}; which first proposed a transformer-based framework for learning unsupervised representations of multivariate time series. Their methodology takes advantage of unlabeled data to train an encoder and extract dense vector representations of time series. Subsequently, they evaluate the model for regression and classification tasks, demonstrating better performance than other state-of-the-art supervised methods, even with data sets with limited samples.

%interpretation
Regarding the interpretability of the model, a recent contribution that analyses the attention maps was presented by \citet{bowles20212}, which explored the use of group-equivariant self-attention for radio astronomy classification. Compared to other approaches, this model analysed the attention maps of the predictions and showed that the mechanism extracts the brightest spots and jets of the radio source more clearly. This indicates that attention maps for prediction interpretation could help experts see patterns that the human eye often misses. \par

In the field of variable stars, \citet{allam2021paying} employed the mechanism for classifying multivariate time series in variable stars. And additionally, \citet{allam2021paying} showed that the activation weights are accommodated according to the variation in brightness of the star, achieving a more interpretable model. And finally, related to the TESS telescope, \citet{morvan2022don} proposed a model that removes the noise from the light curves through the distribution of attention weights. \citet{morvan2022don} showed that the use of the attention mechanism is excellent for removing noise and outliers in time series datasets compared with other approaches. In addition, the use of attention maps allowed them to show the representations learned from the model. \par

Recent attention mechanism approaches in astronomy demonstrate comparable results with earlier approaches, such as CNNs. At the same time, they offer interpretability of their results, which allows a post-prediction analysis. \par


\section{Related Works}
\subsection{Class-Incremental Learning}

Class-Incremental Learning (CIL) aims to enable a model to sequentially learn new classes without forgetting previously learned ones. This presents a significant challenge, especially since task-IDs are not available during inference. To address this issue, several strategies have been proposed, which can be broadly categorized as follows:

Regularization-based methods \cite{li2017learning, rebuffi2017icarl, kirkpatrick2017overcoming, zenke2017continual} focus on constraining changes to important model parameters during training on new classes. Replay-based methods address forgetting by maintaining a memory buffer that stores examples from prior tasks. When learning a new task, the model is trained not only on the current task but also on these stored examples, helping it retain previous knowledge. These methods include direct replay \cite{lopez2017gradient, MER, AGEM, liu2021rmm} as well as generative replay \cite{shin2017continual, Zhu_2021_CVPR}. Optimization-based methods focus on explicitly designing and modifying the optimization process to reduce catastrophic forgetting, such as through gradient projection \cite{farajtabar2020orthogonal, saha2021gradient, lu2024visual} or loss function adjustments \cite{wang2021training, pmlr-v202-wen23b}. Representation-based methods aim to maintain a stable and generalizable feature space as new classes are added. These include self-supervised learning \cite{cha2021co2l, pham2021dualnet} and the use of pre-trained models \cite{wang2022s, Gao2024BeyondPL, mcdonnell2023ranpac}. A key challenge for exemplar-free methods is the shift in backbone features. Recent studies have proposed estimating this shift through changes in class prototypes \cite{yu2020semantic, gomez2025exemplar, goswami2024resurrecting}. This work investigates the semantic drift phenomenon in both the mean and covariance, calibrating them to mitigate catastrophic forgetting.

%\textcolor{red}{While \cite{2024taskrecency} attempted to adapt to changes in both the mean and covariance of feature distributions, this work investigates the semantic shift phenomenon in both the mean and covariance, calibrating them to mitigate catastrophic forgetting.}
% Class-Incremental Learning with Task Prediction.
% decompose task into task-id prediction and within-task prediction
% HiDe-Prompt \cite{wang2024hierarchical} \cite{huang2024ovor}

\subsection{Pre-trained Model based Class-Incremental Learning}
Pre-trained models have become a key component in CIL due to their ability to transfer knowledge efficiently. It is prevailing to use parameter-Efficient Fine-Tuning (PEFT) methods to adapt the model computation efficiently. PEFT methods introduce a small portion of the learnable parameters while keeping the pre-trained model frozen. LoRA \cite{hu2022lora} optimizes the weight space using low-rank matrix factorization, avoiding full parameter fine-tuning; VPT \cite{jia2022visual, wang2024revisiting} injects learnable prompts into the input or intermediate layers to extract task-specific features while freezing the backbone network; AdaptFormer \cite{chen2022adaptformer}, based on adaptive Transformer components, integrates task-specific information with general knowledge. 

Prompt-based class-incremental continual learning methods dynamically adjust lightweight prompt parameters to adapt to task evolution. Key mechanisms include: dynamic prompt pool retrieval \cite{wang2022learning}, general and expert prompt design for knowledge sharing \cite{wang2022dualprompt}, discrete prompt optimization \cite{jiao2024vector}, consistency alignment between classifiers and prompts \cite{gao2024consistent}, decomposed attention \cite{smith2023coda}, one forward stage \cite{kim2024one}, and evolving prompt adapting to task changes \cite{kurniawan2024evoprompt}.  Adapter-based methods such as EASE \cite{zhou2024expandable} dynamically expand task-specific subspaces and integrate multiple adapter predictions with semantic-guided prototype synthesis to mitigate feature degradation of old classes; SSIAT \cite{tan2024semantically} continuously tunes shared adapters and estimates mean shifts, updating prototypes to align new and old task features; and MOS \cite{sun2024mos} merges adapter parameters and employs a self-optimization retrieval mechanism to optimize module compatibility and inference efficiency.  LoRA-based methods, such as InfLoRA \cite{liang2024inflora}, introduce orthogonal constraints to isolate low-rank subspaces, effectively reducing parameter interference between tasks. Together, these methods offer efficient and scalable solutions for adapting pre-trained models to class-incremental learning tasks. In this work, we leverage the power of the LoRA in the context of CIL and build our semantic drift calibration modules on top of it.  
%\section{Method}\label{sec:method}
\begin{figure}
    \centering
    \includegraphics[width=0.85\textwidth]{imgs/heatmap_acc.pdf}
    \caption{\textbf{Visualization of the proposed periodic Bayesian flow with mean parameter $\mu$ and accumulated accuracy parameter $c$ which corresponds to the entropy/uncertainty}. For $x = 0.3, \beta(1) = 1000$ and $\alpha_i$ defined in \cref{appd:bfn_cir}, this figure plots three colored stochastic parameter trajectories for receiver mean parameter $m$ and accumulated accuracy parameter $c$, superimposed on a log-scale heatmap of the Bayesian flow distribution $p_F(m|x,\senderacc)$ and $p_F(c|x,\senderacc)$. Note the \emph{non-monotonicity} and \emph{non-additive} property of $c$ which could inform the network the entropy of the mean parameter $m$ as a condition and the \emph{periodicity} of $m$. %\jj{Shrink the figures to save space}\hanlin{Do we need to make this figure one-column?}
    }
    \label{fig:vmbf_vis}
    \vskip -0.1in
\end{figure}
% \begin{wrapfigure}{r}{0.5\textwidth}
%     \centering
%     \includegraphics[width=0.49\textwidth]{imgs/heatmap_acc.pdf}
%     \caption{\textbf{Visualization of hyper-torus Bayesian flow based on von Mises Distribution}. For $x = 0.3, \beta(1) = 1000$ and $\alpha_i$ defined in \cref{appd:bfn_cir}, this figure plots three colored stochastic parameter trajectories for receiver mean parameter $m$ and accumulated accuracy parameter $c$, superimposed on a log-scale heatmap of the Bayesian flow distribution $p_F(m|x,\senderacc)$ and $p_F(c|x,\senderacc)$. Note the \emph{non-monotonicity} and \emph{non-additive} property of $c$. \jj{Shrink the figures to save space}}
%     \label{fig:vmbf_vis}
%     \vspace{-30pt}
% \end{wrapfigure}


In this section, we explain the detailed design of CrysBFN tackling theoretical and practical challenges. First, we describe how to derive our new formulation of Bayesian Flow Networks over hyper-torus $\mathbb{T}^{D}$ from scratch. Next, we illustrate the two key differences between \modelname and the original form of BFN: $1)$ a meticulously designed novel base distribution with different Bayesian update rules; and $2)$ different properties over the accuracy scheduling resulted from the periodicity and the new Bayesian update rules. Then, we present in detail the overall framework of \modelname over each manifold of the crystal space (\textit{i.e.} fractional coordinates, lattice vectors, atom types) respecting \textit{periodic E(3) invariance}. 

% In this section, we first demonstrate how to build Bayesian flow on hyper-torus $\mathbb{T}^{D}$ by overcoming theoretical and practical problems to provide a low-noise parameter-space approach to fractional atom coordinate generation. Next, we present how \modelname models each manifold of crystal space respecting \textit{periodic E(3) invariance}. 

\subsection{Periodic Bayesian Flow on Hyper-torus \texorpdfstring{$\mathbb{T}^{D}$}{}} 
For generative modeling of fractional coordinates in crystal, we first construct a periodic Bayesian flow on \texorpdfstring{$\mathbb{T}^{D}$}{} by designing every component of the totally new Bayesian update process which we demonstrate to be distinct from the original Bayesian flow (please see \cref{fig:non_add}). 
 %:) 
 
 The fractional atom coordinate system \citep{jiao2023crystal} inherently distributes over a hyper-torus support $\mathbb{T}^{3\times N}$. Hence, the normal distribution support on $\R$ used in the original \citep{bfn} is not suitable for this scenario. 
% The key problem of generative modeling for crystal is the periodicity of Cartesian atom coordinates $\vX$ requiring:
% \begin{equation}\label{eq:periodcity}
% p(\vA,\vL,\vX)=p(\vA,\vL,\vX+\vec{LK}),\text{where}~\vec{K}=\vec{k}\vec{1}_{1\times N},\forall\vec{k}\in\mathbb{Z}^{3\times1}
% \end{equation}
% However, there does not exist such a distribution supporting on $\R$ to model such property because the integration of such distribution over $\R$ will not be finite and equal to 1. Therefore, the normal distribution used in \citet{bfn} can not meet this condition.

To tackle this problem, the circular distribution~\citep{mardia2009directional} over the finite interval $[-\pi,\pi)$ is a natural choice as the base distribution for deriving the BFN on $\mathbb{T}^D$. 
% one natural choice is to 
% we would like to consider the circular distribution over the finite interval as the base 
% we find that circular distributions \citep{mardia2009directional} defined on a finite interval with lengths of $2\pi$ can be used as the instantiation of input distribution for the BFN on $\mathbb{T}^D$.
Specifically, circular distributions enjoy desirable periodic properties: $1)$ the integration over any interval length of $2\pi$ equals 1; $2)$ the probability distribution function is periodic with period $2\pi$.  Sharing the same intrinsic with fractional coordinates, such periodic property of circular distribution makes it suitable for the instantiation of BFN's input distribution, in parameterizing the belief towards ground truth $\x$ on $\mathbb{T}^D$. 
% \yuxuan{this is very complicated from my perspective.} \hanlin{But this property is exactly beautiful and perfectly fit into the BFN.}

\textbf{von Mises Distribution and its Bayesian Update} We choose von Mises distribution \citep{mardia2009directional} from various circular distributions as the form of input distribution, based on the appealing conjugacy property required in the derivation of the BFN framework.
% to leverage the Bayesian conjugacy property of von Mises distribution which is required by the BFN framework. 
That is, the posterior of a von Mises distribution parameterized likelihood is still in the family of von Mises distributions. The probability density function of von Mises distribution with mean direction parameter $m$ and concentration parameter $c$ (describing the entropy/uncertainty of $m$) is defined as: 
\begin{equation}
f(x|m,c)=vM(x|m,c)=\frac{\exp(c\cos(x-m))}{2\pi I_0(c)}
\end{equation}
where $I_0(c)$ is zeroth order modified Bessel function of the first kind as the normalizing constant. Given the last univariate belief parameterized by von Mises distribution with parameter $\theta_{i-1}=\{m_{i-1},\ c_{i-1}\}$ and the sample $y$ from sender distribution with unknown data sample $x$ and known accuracy $\alpha$ describing the entropy/uncertainty of $y$,  Bayesian update for the receiver is deducted as:
\begin{equation}
 h(\{m_{i-1},c_{i-1}\},y,\alpha)=\{m_i,c_i \}, \text{where}
\end{equation}
\begin{equation}\label{eq:h_m}
m_i=\text{atan2}(\alpha\sin y+c_{i-1}\sin m_{i-1}, {\alpha\cos y+c_{i-1}\cos m_{i-1}})
\end{equation}
\begin{equation}\label{eq:h_c}
c_i =\sqrt{\alpha^2+c_{i-1}^2+2\alpha c_{i-1}\cos(y-m_{i-1})}
\end{equation}
The proof of the above equations can be found in \cref{apdx:bayesian_update_function}. The atan2 function refers to  2-argument arctangent. Independently conducting  Bayesian update for each dimension, we can obtain the Bayesian update distribution by marginalizing $\y$:
\begin{equation}
p_U(\vtheta'|\vtheta,\bold{x};\alpha)=\mathbb{E}_{p_S(\bold{y}|\bold{x};\alpha)}\delta(\vtheta'-h(\vtheta,\bold{y},\alpha))=\mathbb{E}_{vM(\bold{y}|\bold{x},\alpha)}\delta(\vtheta'-h(\vtheta,\bold{y},\alpha))
\end{equation} 
\begin{figure}
    \centering
    \vskip -0.15in
    \includegraphics[width=0.95\linewidth]{imgs/non_add.pdf}
    \caption{An intuitive illustration of non-additive accuracy Bayesian update on the torus. The lengths of arrows represent the uncertainty/entropy of the belief (\emph{e.g.}~$1/\sigma^2$ for Gaussian and $c$ for von Mises). The directions of the arrows represent the believed location (\emph{e.g.}~ $\mu$ for Gaussian and $m$ for von Mises).}
    \label{fig:non_add}
    \vskip -0.15in
\end{figure}
\textbf{Non-additive Accuracy} 
The additive accuracy is a nice property held with the Gaussian-formed sender distribution of the original BFN expressed as:
\begin{align}
\label{eq:standard_id}
    \update(\parsn{}'' \mid \parsn{}, \x; \alpha_a+\alpha_b) = \E_{\update(\parsn{}' \mid \parsn{}, \x; \alpha_a)} \update(\parsn{}'' \mid \parsn{}', \x; \alpha_b)
\end{align}
Such property is mainly derived based on the standard identity of Gaussian variable:
\begin{equation}
X \sim \mathcal{N}\left(\mu_X, \sigma_X^2\right), Y \sim \mathcal{N}\left(\mu_Y, \sigma_Y^2\right) \Longrightarrow X+Y \sim \mathcal{N}\left(\mu_X+\mu_Y, \sigma_X^2+\sigma_Y^2\right)
\end{equation}
The additive accuracy property makes it feasible to derive the Bayesian flow distribution $
p_F(\boldsymbol{\theta} \mid \mathbf{x} ; i)=p_U\left(\boldsymbol{\theta} \mid \boldsymbol{\theta}_0, \mathbf{x}, \sum_{k=1}^{i} \alpha_i \right)
$ for the simulation-free training of \cref{eq:loss_n}.
It should be noted that the standard identity in \cref{eq:standard_id} does not hold in the von Mises distribution. Hence there exists an important difference between the original Bayesian flow defined on Euclidean space and the Bayesian flow of circular data on $\mathbb{T}^D$ based on von Mises distribution. With prior $\btheta = \{\bold{0},\bold{0}\}$, we could formally represent the non-additive accuracy issue as:
% The additive accuracy property implies the fact that the "confidence" for the data sample after observing a series of the noisy samples with accuracy ${\alpha_1, \cdots, \alpha_i}$ could be  as the accuracy sum  which could be  
% Here we 
% Here we emphasize the specific property of BFN based on von Mises distribution.
% Note that 
% \begin{equation}
% \update(\parsn'' \mid \parsn, \x; \alpha_a+\alpha_b) \ne \E_{\update(\parsn' \mid \parsn, \x; \alpha_a)} \update(\parsn'' \mid \parsn', \x; \alpha_b)
% \end{equation}
% \oyyw{please check whether the below equation is better}
% \yuxuan{I fill somehow confusing on what is the update distribution with $\alpha$. }
% \begin{equation}
% \update(\parsn{}'' \mid \parsn{}, \x; \alpha_a+\alpha_b) \ne \E_{\update(\parsn{}' \mid \parsn{}, \x; \alpha_a)} \update(\parsn{}'' \mid \parsn{}', \x; \alpha_b)
% \end{equation}
% We give an intuitive visualization of such difference in \cref{fig:non_add}. The untenability of this property can materialize by considering the following case: with prior $\btheta = \{\bold{0},\bold{0}\}$, check the two-step Bayesian update distribution with $\alpha_a,\alpha_b$ and one-step Bayesian update with $\alpha=\alpha_a+\alpha_b$:
\begin{align}
\label{eq:nonadd}
     &\update(c'' \mid \parsn, \x; \alpha_a+\alpha_b)  = \delta(c-\alpha_a-\alpha_b)
     \ne  \mathbb{E}_{p_U(\parsn' \mid \parsn, \x; \alpha_a)}\update(c'' \mid \parsn', \x; \alpha_b) \nonumber \\&= \mathbb{E}_{vM(\bold{y}_b|\bold{x},\alpha_a)}\mathbb{E}_{vM(\bold{y}_a|\bold{x},\alpha_b)}\delta(c-||[\alpha_a \cos\y_a+\alpha_b\cos \y_b,\alpha_a \sin\y_a+\alpha_b\sin \y_b]^T||_2)
\end{align}
A more intuitive visualization could be found in \cref{fig:non_add}. This fundamental difference between periodic Bayesian flow and that of \citet{bfn} presents both theoretical and practical challenges, which we will explain and address in the following contents.

% This makes constructing Bayesian flow based on von Mises distribution intrinsically different from previous Bayesian flows (\citet{bfn}).

% Thus, we must reformulate the framework of Bayesian flow networks  accordingly. % and do necessary reformulations of BFN. 

% \yuxuan{overall I feel this part is complicated by using the language of update distribution. I would like to suggest simply use bayesian update, to provide intuitive explantion.}\hanlin{See the illustration in \cref{fig:non_add}}

% That introduces a cascade of problems, and we investigate the following issues: $(1)$ Accuracies between sender and receiver are not synchronized and need to be differentiated. $(2)$ There is no tractable Bayesian flow distribution for a one-step sample conditioned on a given time step $i$, and naively simulating the Bayesian flow results in computational overhead. $(3)$ It is difficult to control the entropy of the Bayesian flow. $(4)$ Accuracy is no longer a function of $t$ and becomes a distribution conditioned on $t$, which can be different across dimensions.
%\jj{Edited till here}

\textbf{Entropy Conditioning} As a common practice in generative models~\citep{ddpm,flowmatching,bfn}, timestep $t$ is widely used to distinguish among generation states by feeding the timestep information into the networks. However, this paper shows that for periodic Bayesian flow, the accumulated accuracy $\vc_i$ is more effective than time-based conditioning by informing the network about the entropy and certainty of the states $\parsnt{i}$. This stems from the intrinsic non-additive accuracy which makes the receiver's accumulated accuracy $c$ not bijective function of $t$, but a distribution conditioned on accumulated accuracies $\vc_i$ instead. Therefore, the entropy parameter $\vc$ is taken logarithm and fed into the network to describe the entropy of the input corrupted structure. We verify this consideration in \cref{sec:exp_ablation}. 
% \yuxuan{implement variant. traditionally, the timestep is widely used to distinguish the different states by putting the timestep embedding into the networks. citation of FM, diffusion, BFN. However, we find that conditioned on time in periodic flow could not provide extra benefits. To further boost the performance, we introduce a simple yet effective modification term entropy conditional. This is based on that the accumulated accuracy which represents the current uncertainty or entropy could be a better indicator to distinguish different states. + Describe how you do this. }



\textbf{Reformulations of BFN}. Recall the original update function with Gaussian sender distribution, after receiving noisy samples $\y_1,\y_2,\dots,\y_i$ with accuracies $\senderacc$, the accumulated accuracies of the receiver side could be analytically obtained by the additive property and it is consistent with the sender side.
% Since observing sample $\y$ with $\alpha_i$ can not result in exact accuracy increment $\alpha_i$ for receiver, the accuracies between sender and receiver are not synchronized which need to be differentiated. 
However, as previously mentioned, this does not apply to periodic Bayesian flow, and some of the notations in original BFN~\citep{bfn} need to be adjusted accordingly. We maintain the notations of sender side's one-step accuracy $\alpha$ and added accuracy $\beta$, and alter the notation of receiver's accuracy parameter as $c$, which is needed to be simulated by cascade of Bayesian updates. We emphasize that the receiver's accumulated accuracy $c$ is no longer a function of $t$ (differently from the Gaussian case), and it becomes a distribution conditioned on received accuracies $\senderacc$ from the sender. Therefore, we represent the Bayesian flow distribution of von Mises distribution as $p_F(\btheta|\x;\alpha_1,\alpha_2,\dots,\alpha_i)$. And the original simulation-free training with Bayesian flow distribution is no longer applicable in this scenario.
% Different from previous BFNs where the accumulated accuracy $\rho$ is not explicitly modeled, the accumulated accuracy parameter $c$ (visualized in \cref{fig:vmbf_vis}) needs to be explicitly modeled by feeding it to the network to avoid information loss.
% the randomaccuracy parameter $c$ (visualized in \cref{fig:vmbf_vis}) implies that there exists information in $c$ from the sender just like $m$, meaning that $c$ also should be fed into the network to avoid information loss. 
% We ablate this consideration in  \cref{sec:exp_ablation}. 

\textbf{Fast Sampling from Equivalent Bayesian Flow Distribution} Based on the above reformulations, the Bayesian flow distribution of von Mises distribution is reframed as: 
\begin{equation}\label{eq:flow_frac}
p_F(\btheta_i|\x;\alpha_1,\alpha_2,\dots,\alpha_i)=\E_{\update(\parsnt{1} \mid \parsnt{0}, \x ; \alphat{1})}\dots\E_{\update(\parsn_{i-1} \mid \parsnt{i-2}, \x; \alphat{i-1})} \update(\parsnt{i} | \parsnt{i-1},\x;\alphat{i} )
\end{equation}
Naively sampling from \cref{eq:flow_frac} requires slow auto-regressive iterated simulation, making training unaffordable. Noticing the mathematical properties of \cref{eq:h_m,eq:h_c}, we  transform \cref{eq:flow_frac} to the equivalent form:
\begin{equation}\label{eq:cirflow_equiv}
p_F(\vec{m}_i|\x;\alpha_1,\alpha_2,\dots,\alpha_i)=\E_{vM(\y_1|\x,\alpha_1)\dots vM(\y_i|\x,\alpha_i)} \delta(\vec{m}_i-\text{atan2}(\sum_{j=1}^i \alpha_j \cos \y_j,\sum_{j=1}^i \alpha_j \sin \y_j))
\end{equation}
\begin{equation}\label{eq:cirflow_equiv2}
p_F(\vec{c}_i|\x;\alpha_1,\alpha_2,\dots,\alpha_i)=\E_{vM(\y_1|\x,\alpha_1)\dots vM(\y_i|\x,\alpha_i)}  \delta(\vec{c}_i-||[\sum_{j=1}^i \alpha_j \cos \y_j,\sum_{j=1}^i \alpha_j \sin \y_j]^T||_2)
\end{equation}
which bypasses the computation of intermediate variables and allows pure tensor operations, with negligible computational overhead.
\begin{restatable}{proposition}{cirflowequiv}
The probability density function of Bayesian flow distribution defined by \cref{eq:cirflow_equiv,eq:cirflow_equiv2} is equivalent to the original definition in \cref{eq:flow_frac}. 
\end{restatable}
\textbf{Numerical Determination of Linear Entropy Sender Accuracy Schedule} ~Original BFN designs the accuracy schedule $\beta(t)$ to make the entropy of input distribution linearly decrease. As for crystal generation task, to ensure information coherence between modalities, we choose a sender accuracy schedule $\senderacc$ that makes the receiver's belief entropy $H(t_i)=H(p_I(\cdot|\vtheta_i))=H(p_I(\cdot|\vc_i))$ linearly decrease \emph{w.r.t.} time $t_i$, given the initial and final accuracy parameter $c(0)$ and $c(1)$. Due to the intractability of \cref{eq:vm_entropy}, we first use numerical binary search in $[0,c(1)]$ to determine the receiver's $c(t_i)$ for $i=1,\dots, n$ by solving the equation $H(c(t_i))=(1-t_i)H(c(0))+tH(c(1))$. Next, with $c(t_i)$, we conduct numerical binary search for each $\alpha_i$ in $[0,c(1)]$ by solving the equations $\E_{y\sim vM(x,\alpha_i)}[\sqrt{\alpha_i^2+c_{i-1}^2+2\alpha_i c_{i-1}\cos(y-m_{i-1})}]=c(t_i)$ from $i=1$ to $i=n$ for arbitrarily selected $x\in[-\pi,\pi)$.

After tackling all those issues, we have now arrived at a new BFN architecture for effectively modeling crystals. Such BFN can also be adapted to other type of data located in hyper-torus $\mathbb{T}^{D}$.

\subsection{Equivariant Bayesian Flow for Crystal}
With the above Bayesian flow designed for generative modeling of fractional coordinate $\vF$, we are able to build equivariant Bayesian flow for each modality of crystal. In this section, we first give an overview of the general training and sampling algorithm of \modelname (visualized in \cref{fig:framework}). Then, we describe the details of the Bayesian flow of every modality. The training and sampling algorithm can be found in \cref{alg:train} and \cref{alg:sampling}.

\textbf{Overview} Operating in the parameter space $\bthetaM=\{\bthetaA,\bthetaL,\bthetaF\}$, \modelname generates high-fidelity crystals through a joint BFN sampling process on the parameter of  atom type $\bthetaA$, lattice parameter $\vec{\theta}^L=\{\bmuL,\brhoL\}$, and the parameter of fractional coordinate matrix $\bthetaF=\{\bmF,\bcF\}$. We index the $n$-steps of the generation process in a discrete manner $i$, and denote the corresponding continuous notation $t_i=i/n$ from prior parameter $\thetaM_0$ to a considerably low variance parameter $\thetaM_n$ (\emph{i.e.} large $\vrho^L,\bmF$, and centered $\bthetaA$).

At training time, \modelname samples time $i\sim U\{1,n\}$ and $\bthetaM_{i-1}$ from the Bayesian flow distribution of each modality, serving as the input to the network. The network $\net$ outputs $\net(\parsnt{i-1}^\mathcal{M},t_{i-1})=\net(\parsnt{i-1}^A,\parsnt{i-1}^F,\parsnt{i-1}^L,t_{i-1})$ and conducts gradient descents on loss function \cref{eq:loss_n} for each modality. After proper training, the sender distribution $p_S$ can be approximated by the receiver distribution $p_R$. 

At inference time, from predefined $\thetaM_0$, we conduct transitions from $\thetaM_{i-1}$ to $\thetaM_{i}$ by: $(1)$ sampling $\y_i\sim p_R(\bold{y}|\thetaM_{i-1};t_i,\alpha_i)$ according to network prediction $\predM{i-1}$; and $(2)$ performing Bayesian update $h(\thetaM_{i-1},\y^\calM_{i-1},\alpha_i)$ for each dimension. 

% Alternatively, we complete this transition using the flow-back technique by sampling 
% $\thetaM_{i}$ from Bayesian flow distribution $\flow(\btheta^M_{i}|\predM{i-1};t_{i-1})$. 

% The training objective of $\net$ is to minimize the KL divergence between sender distribution and receiver distribution for every modality as defined in \cref{eq:loss_n} which is equivalent to optimizing the negative variational lower bound $\calL^{VLB}$ as discussed in \cref{sec:preliminaries}. 

%In the following part, we will present the Bayesian flow of each modality in detail.

\textbf{Bayesian Flow of Fractional Coordinate $\vF$}~The distribution of the prior parameter $\bthetaF_0$ is defined as:
\begin{equation}\label{eq:prior_frac}
    p(\bthetaF_0) \defeq \{vM(\vm_0^F|\vec{0}_{3\times N},\vec{0}_{3\times N}),\delta(\vc_0^F-\vec{0}_{3\times N})\} = \{U(\vec{0},\vec{1}),\delta(\vc_0^F-\vec{0}_{3\times N})\}
\end{equation}
Note that this prior distribution of $\vm_0^F$ is uniform over $[\vec{0},\vec{1})$, ensuring the periodic translation invariance property in \cref{De:pi}. The training objective is minimizing the KL divergence between sender and receiver distribution (deduction can be found in \cref{appd:cir_loss}): 
%\oyyw{replace $\vF$ with $\x$?} \hanlin{notations follow Preliminary?}
\begin{align}\label{loss_frac}
\calL_F = n \E_{i \sim \ui{n}, \flow(\parsn{}^F \mid \vF ; \senderacc)} \alpha_i\frac{I_1(\alpha_i)}{I_0(\alpha_i)}(1-\cos(\vF-\predF{i-1}))
\end{align}
where $I_0(x)$ and $I_1(x)$ are the zeroth and the first order of modified Bessel functions. The transition from $\bthetaF_{i-1}$ to $\bthetaF_{i}$ is the Bayesian update distribution based on network prediction:
\begin{equation}\label{eq:transi_frac}
    p(\btheta^F_{i}|\parsnt{i-1}^\calM)=\mathbb{E}_{vM(\bold{y}|\predF{i-1},\alpha_i)}\delta(\btheta^F_{i}-h(\btheta^F_{i-1},\bold{y},\alpha_i))
\end{equation}
\begin{restatable}{proposition}{fracinv}
With $\net_{F}$ as a periodic translation equivariant function namely $\net_F(\parsnt{}^A,w(\parsnt{}^F+\vt),\parsnt{}^L,t)=w(\net_F(\parsnt{}^A,\parsnt{}^F,\parsnt{}^L,t)+\vt), \forall\vt\in\R^3$, the marginal distribution of $p(\vF_n)$ defined by \cref{eq:prior_frac,eq:transi_frac} is periodic translation invariant. 
\end{restatable}
\textbf{Bayesian Flow of Lattice Parameter \texorpdfstring{$\boldsymbol{L}$}{}}   
Noting the lattice parameter $\bm{L}$ located in Euclidean space, we set prior as the parameter of a isotropic multivariate normal distribution $\btheta^L_0\defeq\{\vmu_0^L,\vrho_0^L\}=\{\bm{0}_{3\times3},\bm{1}_{3\times3}\}$
% \begin{equation}\label{eq:lattice_prior}
% \btheta^L_0\defeq\{\vmu_0^L,\vrho_0^L\}=\{\bm{0}_{3\times3},\bm{1}_{3\times3}\}
% \end{equation}
such that the prior distribution of the Markov process on $\vmu^L$ is the Dirac distribution $\delta(\vec{\mu_0}-\vec{0})$ and $\delta(\vec{\rho_0}-\vec{1})$, 
% \begin{equation}
%     p_I^L(\boldsymbol{L}|\btheta_0^L)=\mathcal{N}(\bm{L}|\bm{0},\bm{I})
% \end{equation}
which ensures O(3)-invariance of prior distribution of $\vL$. By Eq. 77 from \citet{bfn}, the Bayesian flow distribution of the lattice parameter $\bm{L}$ is: 
\begin{align}% =p_U(\bmuL|\btheta_0^L,\bm{L},\beta(t))
p_F^L(\bmuL|\bm{L};t) &=\mathcal{N}(\bmuL|\gamma(t)\bm{L},\gamma(t)(1-\gamma(t))\bm{I}) 
\end{align}
where $\gamma(t) = 1 - \sigma_1^{2t}$ and $\sigma_1$ is the predefined hyper-parameter controlling the variance of input distribution at $t=1$ under linear entropy accuracy schedule. The variance parameter $\vrho$ does not need to be modeled and fed to the network, since it is deterministic given the accuracy schedule. After sampling $\bmuL_i$ from $p_F^L$, the training objective is defined as minimizing KL divergence between sender and receiver distribution (based on Eq. 96 in \citet{bfn}):
\begin{align}
\mathcal{L}_{L} = \frac{n}{2}\left(1-\sigma_1^{2/n}\right)\E_{i \sim \ui{n}}\E_{\flow(\bmuL_{i-1} |\vL ; t_{i-1})}  \frac{\left\|\vL -\predL{i-1}\right\|^2}{\sigma_1^{2i/n}},\label{eq:lattice_loss}
\end{align}
where the prediction term $\predL{i-1}$ is the lattice parameter part of network output. After training, the generation process is defined as the Bayesian update distribution given network prediction:
\begin{equation}\label{eq:lattice_sampling}
    p(\bmuL_{i}|\parsnt{i-1}^\calM)=\update^L(\bmuL_{i}|\predL{i-1},\bmuL_{i-1};t_{i-1})
\end{equation}
    

% The final prediction of the lattice parameter is given by $\bmuL_n = \predL{n-1}$.
% \begin{equation}\label{eq:final_lattice}
%     \bmuL_n = \predL{n-1}
% \end{equation}

\begin{restatable}{proposition}{latticeinv}\label{prop:latticeinv}
With $\net_{L}$ as  O(3)-equivariant function namely $\net_L(\parsnt{}^A,\parsnt{}^F,\vQ\parsnt{}^L,t)=\vQ\net_L(\parsnt{}^A,\parsnt{}^F,\parsnt{}^L,t),\forall\vQ^T\vQ=\vI$, the marginal distribution of $p(\bmuL_n)$ defined by \cref{eq:lattice_sampling} is O(3)-invariant. 
\end{restatable}


\textbf{Bayesian Flow of Atom Types \texorpdfstring{$\boldsymbol{A}$}{}} 
Given that atom types are discrete random variables located in a simplex $\calS^K$, the prior parameter of $\boldsymbol{A}$ is the discrete uniform distribution over the vocabulary $\parsnt{0}^A \defeq \frac{1}{K}\vec{1}_{1\times N}$. 
% \begin{align}\label{eq:disc_input_prior}
% \parsnt{0}^A \defeq \frac{1}{K}\vec{1}_{1\times N}
% \end{align}
% \begin{align}
%     (\oh{j}{K})_k \defeq \delta_{j k}, \text{where }\oh{j}{K}\in \R^{K},\oh{\vA}{KD} \defeq \left(\oh{a_1}{K},\dots,\oh{a_N}{K}\right) \in \R^{K\times N}
% \end{align}
With the notation of the projection from the class index $j$ to the length $K$ one-hot vector $ (\oh{j}{K})_k \defeq \delta_{j k}, \text{where }\oh{j}{K}\in \R^{K},\oh{\vA}{KD} \defeq \left(\oh{a_1}{K},\dots,\oh{a_N}{K}\right) \in \R^{K\times N}$, the Bayesian flow distribution of atom types $\vA$ is derived in \citet{bfn}:
\begin{align}
\flow^{A}(\parsn^A \mid \vA; t) &= \E_{\N{\y \mid \beta^A(t)\left(K \oh{\vA}{K\times N} - \vec{1}_{K\times N}\right)}{\beta^A(t) K \vec{I}_{K\times N \times N}}} \delta\left(\parsn^A - \frac{e^{\y}\parsnt{0}^A}{\sum_{k=1}^K e^{\y_k}(\parsnt{0})_{k}^A}\right).
\end{align}
where $\beta^A(t)$ is the predefined accuracy schedule for atom types. Sampling $\btheta_i^A$ from $p_F^A$ as the training signal, the training objective is the $n$-step discrete-time loss for discrete variable \citep{bfn}: 
% \oyyw{can we simplify the next equation? Such as remove $K \times N, K \times N \times N$}
% \begin{align}
% &\calL_A = n\E_{i \sim U\{1,n\},\flow^A(\parsn^A \mid \vA ; t_{i-1}),\N{\y \mid \alphat{i}\left(K \oh{\vA}{KD} - \vec{1}_{K\times N}\right)}{\alphat{i} K \vec{I}_{K\times N \times N}}} \ln \N{\y \mid \alphat{i}\left(K \oh{\vA}{K\times N} - \vec{1}_{K\times N}\right)}{\alphat{i} K \vec{I}_{K\times N \times N}}\nonumber\\
% &\qquad\qquad\qquad-\sum_{d=1}^N \ln \left(\sum_{k=1}^K \out^{(d)}(k \mid \parsn^A; t_{i-1}) \N{\ydd{d} \mid \alphat{i}\left(K\oh{k}{K}- \vec{1}_{K\times N}\right)}{\alphat{i} K \vec{I}_{K\times N \times N}}\right)\label{discdisc_t_loss_exp}
% \end{align}
\begin{align}
&\calL_A = n\E_{i \sim U\{1,n\},\flow^A(\parsn^A \mid \vA ; t_{i-1}),\N{\y \mid \alphat{i}\left(K \oh{\vA}{KD} - \vec{1}\right)}{\alphat{i} K \vec{I}}} \ln \N{\y \mid \alphat{i}\left(K \oh{\vA}{K\times N} - \vec{1}\right)}{\alphat{i} K \vec{I}}\nonumber\\
&\qquad\qquad\qquad-\sum_{d=1}^N \ln \left(\sum_{k=1}^K \out^{(d)}(k \mid \parsn^A; t_{i-1}) \N{\ydd{d} \mid \alphat{i}\left(K\oh{k}{K}- \vec{1}\right)}{\alphat{i} K \vec{I}}\right)\label{discdisc_t_loss_exp}
\end{align}
where $\vec{I}\in \R^{K\times N \times N}$ and $\vec{1}\in\R^{K\times D}$. When sampling, the transition from $\bthetaA_{i-1}$ to $\bthetaA_{i}$ is derived as:
\begin{equation}
    p(\btheta^A_{i}|\parsnt{i-1}^\calM)=\update^A(\btheta^A_{i}|\btheta^A_{i-1},\predA{i-1};t_{i-1})
\end{equation}

The detailed training and sampling algorithm could be found in \cref{alg:train} and \cref{alg:sampling}.




\section{Method}
\begin{figure*}[!ht]
\centering
\includegraphics[width=\textwidth]{images/framework0128.pdf}
\caption{\small Illustration of our method at task \( t \). The feature extractor at task \( t \) uses a frozen pre-trained ViT backbone with learnable LoRA modules. The output class tokens (yellow) are passed through a classifier to compute the classification loss \( \mathcal{L}_{\text{cls}} \), and the mean and covariance of each class are stored for each session. During training, class tokens (yellow and blue) are used to align class distributions via a covariance calibration loss \( \mathcal{L}_{\text{cov}} \). Patch tokens from network \( t \) (yellow) distill knowledge from network \( t-1 \) (blue) through a distillation loss \( \mathcal{L}_{\text{distill}} \). After training, the class means are updated using a mean shift compensation module, and the classifier heads are retrained with the calibrated statistics.}
\label{fig:framework}
\vspace{-3mm}
\end{figure*}
\subsection{Preliminaries}
\subsubsection{Class-Incremental Learning}
Consider a dataset consisting of $T$ tasks $\{\mathcal{D}^t\}_{t=1}^T$. For each task, the dataset $\mathcal{D}^t = \{(x_i^t, y_i^t)\}_{i=1}^{n^t}$ contains $n^t$ inputs $x_i^t \in \mathbb{R}^d$ and their corresponding labels $y_i^t \in C^t$. We use $X^t$ and $Y^t$ to denote the collection of input data and label of task $t$, respectively and $C^t =\{c_i\}_{i=1}^{|C^t|}$ is the label set contains 
$|C^t|$ classes. In the class-incremental setting, for any task $i \neq j$, the input data from different tasks follow different distributions, i.e., $p(X^i) \neq p(X^j)$ and labels satisfy $C^i \cap C^j = \emptyset$.
The learning objective is to find the model $f^t: \mathbb{R}^{D} \to \mathbb{R}^{\mathcal{C}^t}$, where $\mathcal{C}^t = \cup_{i=1}^{t}C^i$ 
represents the total number of classes learned. This model is trained on all training datasets to perform well on all test dataset seen up to task $t$. In our scenario, the model $f^t$ is based on a pre-trained model, consisting of $f_{\theta}^t(x) = W^\top \phi_{\theta}^t(x)$, where $\phi_{\theta}^t:\mathbb{R}^D \to \mathbb{R}^d$ is a feature extractor composed of a a frozen pre-trained model $\phi$ and learnable parameters $\theta$ in the LoRA modules, and a classification head $W=\{w^t\}^{T}_{t=1}$ for each task, where we have $w^t \in \mathbb{R}^{d \times C_t}$. For a given task $t$, the old network $f_{\theta}^{t-1}(x)$ refers to the network trained on task $t-1$, and it is frozen in task $t$.
% \subsubsection{LoRA based PEFT}
% LoRA \cite{hu2022lora} introduces low-rank matrices into the model's weight updates, which allows the model to adapt without changing the full set of parameters. Consider the pre-trained weight matrix $W_0 \in \mathbb{R}^{d \times k}$, the update of the weight matrix is decomposed into the product of the low rank matrices $B \in \mathbb{R}^{d \times r}$ and $A \in \mathbb{R}^{r \times k}$, where $r$ is a value much smaller than the input dimension $d$ and the output dimension $k$. The forward pass of the low-rank update precess can be expressed as $h = W_0x+\Delta Wx = W_0x + BAx$. Here, $h$ and $x$ denote the input and output of the LoRA module. During training, only the low-rank weight matrices $A$ and $B$ are learnable, and the pre-trained weight $W_0$ is kept frozen.
% \vspace{-3mm}
\subsection{Overview}
Figure \ref{fig:framework} illustrates the overall architecture for class-incremental learning. We employ a frozen pre-trained ViT \cite{dosovitskiy2021an} model as the backbone with learnable task-specific LoRA modules. The output class tokens are forwarded through a task-specific classifier, which generates the class scores, while the angular penalty loss \cite{peng2022few,tan2024semantically} is used to compute the classification loss $\mathcal{L}_{\text{cls}}$:
\begin{equation}
    \mathcal{L}_{\text{cls}} = -\frac{1}{n^t} \sum_{j=1}^{n^t}\log \frac{\exp({s\cos(\theta_j)})}{\sum_{i=1}^{|C^t|} \exp({s\cos(\theta_i)})}
\end{equation}
where $\cos(\theta_j)=\frac{w_jf_{\theta_j}}{ \parallel w_j \parallel \parallel f_{\theta_j} \parallel}$, $s$ represents the scaling factor, and $n^t$ is the number of training samples in task $t$. The mean and
covariance of each class are stored for each learning session.
Before the training process, covariance of each class is precomputed from the class tokens generated by the network trained on previous tasks. These covariance matrices are then used to align the distribution of the representations generated by the current network with that of the old network, based on the Mahalanobis distance. This is referred to as the covariance calibration loss $\mathcal{L}_{\text{cov}}$. Furthermore, patch tokens are leveraged to preserve knowledge from earlier tasks at the feature level through a distillation loss $\mathcal{L}_{\text{distill}}$. 

After the training process, the class means are updated through the mean shift compensation process, and the classifier heads are retrained using the calibrated class statistics. The training pipeline is illustrated in Algorithm \ref{alg:algorithm}. In summary, the overall training objective is:
\begin{equation}
    \mathcal{L} = \mathcal{L}_{\text{cls}} + \mathcal{L}_{\text{cov}} + \lambda \mathcal{L}_{\text{distill}}
\end{equation}

\subsection{Low-Rank Adaptation}
In class-incremental learning (CIL), task IDs are not provided during the inference stage. For methods based on pre-trained models, the use of task-specific PEFT modules often involves a task ID prediction step during testing \cite{wang2022dualprompt,wang2024hierarchical,sun2024mos}. Other approaches avoid task ID prediction, as low prediction accuracy can negatively impact performance. For instance, some methods use weighted sums of prompts to determine the prompts applied during inference \cite{kurniawan2024evoprompt, smith2023coda}, or aggregate all previous PEFT modules \cite{zhou2024expandable, liang2024inflora}. Alternatively, some methods rely on a shared PEFT module across tasks \cite{huang2024ovor, tan2024semantically}.

Compared to other PEFT modules applied in CIL, such as prompts and adapters, LoRA~\cite{hu2022lora}  performs inference by adding the low-rank adaptation matrices to the original weight matrices. This enables LoRA to efficiently combine the pre-trained model weights with the task-specific adaptation matrices, as well as the information across the different task-specific matrices.

In this paper, we utilize task-specific LoRA modules, where each task is assigned a unique LoRA module. Considering the pre-trained weight matrix $W_0 \in \mathbb{R}^{d \times k}$, the update of the weight matrix is decomposed into the product of the low rank matrices $B \in \mathbb{R}^{d \times r}$ and $A \in \mathbb{R}^{r \times k}$, where $r$ is a value much smaller than the input dimension $d$ and the output dimension $k$.  The aggregation of all previous LoRA modules at task $t$ can be expressed as $W = W_0 + \sum_{i=1}^t B_iA_i$. Additionally, we explore two other LoRA structures: the task-shared LoRA module, which uses a common LoRA module for all tasks ($W = W_0 + BA$), and a hybrid structure that combines both task-specific and task-shared designs, inspired by HydraLoRA \cite{tian2024hydralora}. In this hybrid structure, the shared LoRA module's matrix $A$ is used across tasks, while the independent LoRA module's matrix $B_i$ is specific to each task ($W = W_0 + \sum_{i=1}^t B_iA$). During training, only the low-rank weight matrices $A_i$ and $B_i$ are learnable, and the pre-trained weight $W_0$ is frozen.
\subsection{Semantic Drift}
As tasks increase, we no longer have access to the data from previous tasks, and therefore cannot compute the true distribution of earlier classes under the incrementally trained network. Both the mean and variance of feature distributions of the old classes change. This phenomenon is referred to as semantic drift. When semantic drift occurs, it affects the classifier's performance. Thus, it is necessary to impose constraints on the semantic drift and calibrate the means and covariances of class distributions.

\subsubsection{Mean Shift Compensation}
We define the mean class representation of class $c$ in the embedding space after learning task $t$ as:
\begin{equation}
    \mu_c^t = \frac{1}{N_c} \sum_{i=1}^{N_c} [y_i = c] \phi_{\theta}^t(x_i)
\end{equation}
where $N_c$ is the number of samples for class $c$. The shift between the class mean obtained with the current network and the class mean obtained with the previous network can be defined as:
\begin{equation}
    \Delta \mu_c^{t-1 \rightarrow t} = \mu_c^t - \mu_c^{t-1}
\end{equation}
Previous work \cite{yu2020semantic} suggests that the shift between the true class mean and the estimated class mean can be approximated by the shift in the current class embeddings between the old and current models. Specifically, the shift of the class embeddings can be defined as:
\begin{equation}
    \Delta \phi_{\theta}^{t-1 \rightarrow t}(x_i) = \phi_{\theta}^t(x_i) - \phi_{\theta}^{t-1}(x_i)
\end{equation}
where $x_i$ belongs to class $c$. We can compute $\phi_{\theta}^{t-1}(x_i)$ using the model trained on task $t-1$ before the current task training. Then, we compute the drift $\phi_{\theta}^t(x_i)$ and use it to approximate the class mean shift $\Delta \mu_c^{t-1 \rightarrow t}$:
\begin{equation}
    \hat{\Delta} \mu_c^{t-1 \rightarrow t} = \frac{\sum_{i} w_i \Delta \phi_{\theta}^{t-1 \rightarrow t}(x_i)}{\sum_{i} w_i}
\end{equation}
with
\begin{equation}
    w_i = \exp({-\frac{\|\phi_{\theta}^{t-1}(x_i) - \mu_c^{t-1}\|^2}{2\sigma^2}})
\end{equation}
where $\sigma$ is the standard deviation of the Gaussian kernel, and the weight $w_i$ indicates that embeddings closer to the class mean contribute more to the mean shift estimation of that particular class. The proposed method can be used to compensate the mean shift of all previously learned classes at each new task with $\mu_c^t \approx \hat\mu_c^t =\hat\Delta \mu_c^{t-1 \rightarrow t} + \mu_c^{t-1}$.


\subsubsection{Covariance Calibration}
In this section, we address semantic shift from the perspective of the covariance matrix by introducing a novel covariance calibration technique, which is powered by a Mahalanobis distance-based loss. The objective is to ensure that, for each class in the new dataset, the embeddings generated by both the old and current networks follow the same covariance structure. Specifically, the covariance matrices of the embeddings from the current network should be aligned with those from the old network. To achieve this, we utilize the old network, trained on the previous task, as a form of "past knowledge" and use it to calculate the covariance for each class in the current task. Since the Mahalanobis distance directly depends on the covariance matrix, optimizing the difference between embedding pairs from the old and current networks in terms of Mahalanobis distance implicitly constrains the shape of the intra-class distribution, thus alleviating covariance shift.

Mathematically, the Mahalanobis distance \cite{mahalanobis1936generalized} is defined as the degree of difference between two random variables $x$ and $y$, which follow the same distribution and share the covariance matrix $\Sigma$:
\begin{equation}
d_M(x, y, \Sigma) = \sqrt{(x - y)^T \Sigma^{-1} (x - y)} 
\end{equation}
In our setting, we calculate the Mahalanobis distance $d_M( \phi_{\theta}^t(x_i),  \phi_{\theta}^t(x_j), \Sigma^{t-1}_c)$ using  the embedding pairs calculated with the data from the current task $t$ and as computed by covariance matrix $\Sigma^{t-1}_c$ of the class $c$ from the current task with the old network $f^{t-1}$.

Before training, for each class $c$, the covariance matrix is computed as:
\begin{equation}
    \Sigma^{t-1}_c = \frac{1}{N_c} \sum_{i=1}^{N_c} (x_i - \mu^{t-1}_c)(x_i - \mu^{t-1}_c)^T
\end{equation}
where $\mu_c^{t-1}$ is the class mean of the class $c$ calculated with the old network $f^{t-1}$. The loss function for minimizing the absolute difference in Mahalanobis distances of the sample input pairs $(x_i,x_j)$ between embeddings obtained from the old and new networks:
\begin{equation}
\begin{split}
\mathcal{L}_{\text{cov}} &= \sum_{c \in C^t} \sum_{i,j} | d_M( \phi_{\theta}^t(x_i),  \phi_{\theta}^t(x_j), \Sigma^{t-1}_c) \\
&- d_M( \phi_{\theta}^{t-1}(x_i),  \phi_{\theta}^{t-1}(x_j), \Sigma^{t-1}_c) |
\end{split}
\end{equation}

\subsection{Classifier Alignment}
The model exhibits a tendency to prioritize the categories associated with the current task, resulting in a degradation of classification performance for categories from previous tasks. Upon completing the training for each task, the classifier undergoes post hoc retraining using the statistics of previously learned classes, thereby enhancing its overall performance. It is assumed that the feature representations learned by the pre-trained model for each class follow a Gaussian distribution \cite{zhang2023slca}. In this framework, the mean $\mu_c$ and covariance $\Sigma_c$ of the feature representation for each class $c$, as described in previous sections, are calculated and stored. A number of $s_c$ samples $h^c$ are then drawn from the Gaussian distribution $\mathcal{N}(\mu_c, \Sigma_c)$ for each class as input. The classification head is subsequently retrained using a cross-entropy loss function:
\begin{equation}
    \mathcal{L}_{head} = -\frac{1}{s_c|\mathcal{C}|} \sum_{i=1}^{s_c|\mathcal{C}|} \log \left( \frac{e^{w_j{h^c_i}}}{\sum_{k \in \mathcal{C}} e^{w_k(h^c_i)}} \right)
\end{equation}
where $\mathcal{C}$ denotes all classes learned until current task, $w$ is the classifier. With the alignment of semantic drift, the true class mean and covariance are calibrated, which helps mitigate the classifier’s bias, typically induced by overconfidence in new tasks, and alleviates the issue of catastrophic forgetting.

\subsection{Feature-level Self-Distillation}
To enhance the model's resistance to catastrophic forgetting, we propose a self-distillation approach that focuses on improving the utilization of patch tokens in classification tasks, as suggested in \cite{li2024dynamic, wang2024improving}. In Vision Transformers (ViT), the information from patch tokens is often underutilized, which can limit the model's ability to generalize across tasks \cite{zhai2024fine}. To address this, we introduce a self-distillation loss based on patch tokens.

In this method, the class token output from the current network is treated as the essential feature information that needs to be learned for the current task. The feature-level self-distillation loss encourages the alignment of patch tokens from the current network output with the class token. Specifically, we compute the angular similarity, $\text{sim}_{\angle
}$, between the current task's patch tokens, denoted as \( p_j^t \), and the class token \( c^t \), with the features normalized using L2 normalization. The loss is then formulated as:
\begin{equation}
    \mathcal{L}_{\text{distill}} = \frac{1}{L} \sum_{j=1}^{L} \left( 1 - \text{sim}_{\angle
}(p_j^t, c^t) \right) \cdot \left\| p_j^t - p_j^{t-1} \right\|_2^2 
\end{equation}
where \( p_j^t \) is the \( j \)-th patch token for the current task, \( c^t \) is the class token for the current task, \( p_j^{t-1} \) is the patch token from the previous task, and \( L \) is the total number of patch tokens for the current task. We disable the gradient updates during angular similarity computation.

The low angular similarity between patch tokens and class tokens suggests that the patch tokens contribute less to the semantic representation of the current task. To encourage better feature reuse, patch tokens with low similarity are encouraged to resemble the patch tokens from the previous network, thereby improving the retention of important task-related features. This approach ensures that patch tokens are more effectively utilized, contributing to the model's robustness and its ability to mitigate forgetting when transitioning between tasks.

\begin{algorithm}[h]
% \vspace{-10mm}
\caption{Semantic Drift Calibration }\label{alg:algorithm}
\begin{algorithmic}[1]
\REQUIRE Incrementally learned model $\{\phi^t_{\theta}\}^T_{t=1}$ with learning parameters $\theta$, classifiers $\{w^t\}_{t=1}^T$, dataset $\{D^t\}_{t=1}^T$;

\FOR{ task $t = 1$ to $T$}
    \FOR{ $c \in C^t$}
        \STATE Extract features $\phi_{\theta}^{t-1}(x_i)$ using the frozen model learned from task $t-1$;
        \STATE Compute mean $\mu_c^{t-1}$ and covariance $\Sigma_c^{t-1}$;
    \ENDFOR
    \FOR{ Batch $\{(x^t_i,y^t_i)\}$ sampled from $\mathcal{D}^t$}
        \STATE Train $\phi_{\theta}^{t}$ and $w^t$ using $\mathcal{L} = \mathcal{L}_{\text{cls}} + \mathcal{L}_{\text{cov}} + \lambda \mathcal{L}_{\text{distill}}$;
    \ENDFOR
    \FOR{ $c \in \cup_{i=1}^{t-1} C^i$}
        \STATE Estimate and compensate the class mean shift $\mu_c^t \approx\hat\mu_c^t = \mu_c^{t-1} + \hat{\Delta} \mu_c^{t-1 \rightarrow t}$;
        \STATE Sample from $\mathcal{N}(\mu_c, \Sigma_c)$ with the calibrated statistics and retrain the classifiers $\{w^i\}_{i=1}^{t}$ with $\mathcal{L}_{\text{head}}$;
        \STATE Store mean $\mu_c$ and covariance $\Sigma_c$;
    \ENDFOR
\ENDFOR
\end{algorithmic}
\end{algorithm}
\vspace{-5mm}


% \section{Experiments}
\label{sec:experiments}
The experiments are designed to address two key research questions.
First, \textbf{RQ1} evaluates whether the average $L_2$-norm of the counterfactual perturbation vectors ($\overline{||\perturb||}$) decreases as the model overfits the data, thereby providing further empirical validation for our hypothesis.
Second, \textbf{RQ2} evaluates the ability of the proposed counterfactual regularized loss, as defined in (\ref{eq:regularized_loss2}), to mitigate overfitting when compared to existing regularization techniques.

% The experiments are designed to address three key research questions. First, \textbf{RQ1} investigates whether the mean perturbation vector norm decreases as the model overfits the data, aiming to further validate our intuition. Second, \textbf{RQ2} explores whether the mean perturbation vector norm can be effectively leveraged as a regularization term during training, offering insights into its potential role in mitigating overfitting. Finally, \textbf{RQ3} examines whether our counterfactual regularizer enables the model to achieve superior performance compared to existing regularization methods, thus highlighting its practical advantage.

\subsection{Experimental Setup}
\textbf{\textit{Datasets, Models, and Tasks.}}
The experiments are conducted on three datasets: \textit{Water Potability}~\cite{kadiwal2020waterpotability}, \textit{Phomene}~\cite{phomene}, and \textit{CIFAR-10}~\cite{krizhevsky2009learning}. For \textit{Water Potability} and \textit{Phomene}, we randomly select $80\%$ of the samples for the training set, and the remaining $20\%$ for the test set, \textit{CIFAR-10} comes already split. Furthermore, we consider the following models: Logistic Regression, Multi-Layer Perceptron (MLP) with 100 and 30 neurons on each hidden layer, and PreactResNet-18~\cite{he2016cvecvv} as a Convolutional Neural Network (CNN) architecture.
We focus on binary classification tasks and leave the extension to multiclass scenarios for future work. However, for datasets that are inherently multiclass, we transform the problem into a binary classification task by selecting two classes, aligning with our assumption.

\smallskip
\noindent\textbf{\textit{Evaluation Measures.}} To characterize the degree of overfitting, we use the test loss, as it serves as a reliable indicator of the model's generalization capability to unseen data. Additionally, we evaluate the predictive performance of each model using the test accuracy.

\smallskip
\noindent\textbf{\textit{Baselines.}} We compare CF-Reg with the following regularization techniques: L1 (``Lasso''), L2 (``Ridge''), and Dropout.

\smallskip
\noindent\textbf{\textit{Configurations.}}
For each model, we adopt specific configurations as follows.
\begin{itemize}
\item \textit{Logistic Regression:} To induce overfitting in the model, we artificially increase the dimensionality of the data beyond the number of training samples by applying a polynomial feature expansion. This approach ensures that the model has enough capacity to overfit the training data, allowing us to analyze the impact of our counterfactual regularizer. The degree of the polynomial is chosen as the smallest degree that makes the number of features greater than the number of data.
\item \textit{Neural Networks (MLP and CNN):} To take advantage of the closed-form solution for computing the optimal perturbation vector as defined in (\ref{eq:opt-delta}), we use a local linear approximation of the neural network models. Hence, given an instance $\inst_i$, we consider the (optimal) counterfactual not with respect to $\model$ but with respect to:
\begin{equation}
\label{eq:taylor}
    \model^{lin}(\inst) = \model(\inst_i) + \nabla_{\inst}\model(\inst_i)(\inst - \inst_i),
\end{equation}
where $\model^{lin}$ represents the first-order Taylor approximation of $\model$ at $\inst_i$.
Note that this step is unnecessary for Logistic Regression, as it is inherently a linear model.
\end{itemize}

\smallskip
\noindent \textbf{\textit{Implementation Details.}} We run all experiments on a machine equipped with an AMD Ryzen 9 7900 12-Core Processor and an NVIDIA GeForce RTX 4090 GPU. Our implementation is based on the PyTorch Lightning framework. We use stochastic gradient descent as the optimizer with a learning rate of $\eta = 0.001$ and no weight decay. We use a batch size of $128$. The training and test steps are conducted for $6000$ epochs on the \textit{Water Potability} and \textit{Phoneme} datasets, while for the \textit{CIFAR-10} dataset, they are performed for $200$ epochs.
Finally, the contribution $w_i^{\varepsilon}$ of each training point $\inst_i$ is uniformly set as $w_i^{\varepsilon} = 1~\forall i\in \{1,\ldots,m\}$.

The source code implementation for our experiments is available at the following GitHub repository: \url{https://anonymous.4open.science/r/COCE-80B4/README.md} 

\subsection{RQ1: Counterfactual Perturbation vs. Overfitting}
To address \textbf{RQ1}, we analyze the relationship between the test loss and the average $L_2$-norm of the counterfactual perturbation vectors ($\overline{||\perturb||}$) over training epochs.

In particular, Figure~\ref{fig:delta_loss_epochs} depicts the evolution of $\overline{||\perturb||}$ alongside the test loss for an MLP trained \textit{without} regularization on the \textit{Water Potability} dataset. 
\begin{figure}[ht]
    \centering
    \includegraphics[width=0.85\linewidth]{img/delta_loss_epochs.png}
    \caption{The average counterfactual perturbation vector $\overline{||\perturb||}$ (left $y$-axis) and the cross-entropy test loss (right $y$-axis) over training epochs ($x$-axis) for an MLP trained on the \textit{Water Potability} dataset \textit{without} regularization.}
    \label{fig:delta_loss_epochs}
\end{figure}

The plot shows a clear trend as the model starts to overfit the data (evidenced by an increase in test loss). 
Notably, $\overline{||\perturb||}$ begins to decrease, which aligns with the hypothesis that the average distance to the optimal counterfactual example gets smaller as the model's decision boundary becomes increasingly adherent to the training data.

It is worth noting that this trend is heavily influenced by the choice of the counterfactual generator model. In particular, the relationship between $\overline{||\perturb||}$ and the degree of overfitting may become even more pronounced when leveraging more accurate counterfactual generators. However, these models often come at the cost of higher computational complexity, and their exploration is left to future work.

Nonetheless, we expect that $\overline{||\perturb||}$ will eventually stabilize at a plateau, as the average $L_2$-norm of the optimal counterfactual perturbations cannot vanish to zero.

% Additionally, the choice of employing the score-based counterfactual explanation framework to generate counterfactuals was driven to promote computational efficiency.

% Future enhancements to the framework may involve adopting models capable of generating more precise counterfactuals. While such approaches may yield to performance improvements, they are likely to come at the cost of increased computational complexity.


\subsection{RQ2: Counterfactual Regularization Performance}
To answer \textbf{RQ2}, we evaluate the effectiveness of the proposed counterfactual regularization (CF-Reg) by comparing its performance against existing baselines: unregularized training loss (No-Reg), L1 regularization (L1-Reg), L2 regularization (L2-Reg), and Dropout.
Specifically, for each model and dataset combination, Table~\ref{tab:regularization_comparison} presents the mean value and standard deviation of test accuracy achieved by each method across 5 random initialization. 

The table illustrates that our regularization technique consistently delivers better results than existing methods across all evaluated scenarios, except for one case -- i.e., Logistic Regression on the \textit{Phomene} dataset. 
However, this setting exhibits an unusual pattern, as the highest model accuracy is achieved without any regularization. Even in this case, CF-Reg still surpasses other regularization baselines.

From the results above, we derive the following key insights. First, CF-Reg proves to be effective across various model types, ranging from simple linear models (Logistic Regression) to deep architectures like MLPs and CNNs, and across diverse datasets, including both tabular and image data. 
Second, CF-Reg's strong performance on the \textit{Water} dataset with Logistic Regression suggests that its benefits may be more pronounced when applied to simpler models. However, the unexpected outcome on the \textit{Phoneme} dataset calls for further investigation into this phenomenon.


\begin{table*}[h!]
    \centering
    \caption{Mean value and standard deviation of test accuracy across 5 random initializations for different model, dataset, and regularization method. The best results are highlighted in \textbf{bold}.}
    \label{tab:regularization_comparison}
    \begin{tabular}{|c|c|c|c|c|c|c|}
        \hline
        \textbf{Model} & \textbf{Dataset} & \textbf{No-Reg} & \textbf{L1-Reg} & \textbf{L2-Reg} & \textbf{Dropout} & \textbf{CF-Reg (ours)} \\ \hline
        Logistic Regression   & \textit{Water}   & $0.6595 \pm 0.0038$   & $0.6729 \pm 0.0056$   & $0.6756 \pm 0.0046$  & N/A    & $\mathbf{0.6918 \pm 0.0036}$                     \\ \hline
        MLP   & \textit{Water}   & $0.6756 \pm 0.0042$   & $0.6790 \pm 0.0058$   & $0.6790 \pm 0.0023$  & $0.6750 \pm 0.0036$    & $\mathbf{0.6802 \pm 0.0046}$                    \\ \hline
%        MLP   & \textit{Adult}   & $0.8404 \pm 0.0010$   & $\mathbf{0.8495 \pm 0.0007}$   & $0.8489 \pm 0.0014$  & $\mathbf{0.8495 \pm 0.0016}$     & $0.8449 \pm 0.0019$                    \\ \hline
        Logistic Regression   & \textit{Phomene}   & $\mathbf{0.8148 \pm 0.0020}$   & $0.8041 \pm 0.0028$   & $0.7835 \pm 0.0176$  & N/A    & $0.8098 \pm 0.0055$                     \\ \hline
        MLP   & \textit{Phomene}   & $0.8677 \pm 0.0033$   & $0.8374 \pm 0.0080$   & $0.8673 \pm 0.0045$  & $0.8672 \pm 0.0042$     & $\mathbf{0.8718 \pm 0.0040}$                    \\ \hline
        CNN   & \textit{CIFAR-10} & $0.6670 \pm 0.0233$   & $0.6229 \pm 0.0850$   & $0.7348 \pm 0.0365$   & N/A    & $\mathbf{0.7427 \pm 0.0571}$                     \\ \hline
    \end{tabular}
\end{table*}

\begin{table*}[htb!]
    \centering
    \caption{Hyperparameter configurations utilized for the generation of Table \ref{tab:regularization_comparison}. For our regularization the hyperparameters are reported as $\mathbf{\alpha/\beta}$.}
    \label{tab:performance_parameters}
    \begin{tabular}{|c|c|c|c|c|c|c|}
        \hline
        \textbf{Model} & \textbf{Dataset} & \textbf{No-Reg} & \textbf{L1-Reg} & \textbf{L2-Reg} & \textbf{Dropout} & \textbf{CF-Reg (ours)} \\ \hline
        Logistic Regression   & \textit{Water}   & N/A   & $0.0093$   & $0.6927$  & N/A    & $0.3791/1.0355$                     \\ \hline
        MLP   & \textit{Water}   & N/A   & $0.0007$   & $0.0022$  & $0.0002$    & $0.2567/1.9775$                    \\ \hline
        Logistic Regression   &
        \textit{Phomene}   & N/A   & $0.0097$   & $0.7979$  & N/A    & $0.0571/1.8516$                     \\ \hline
        MLP   & \textit{Phomene}   & N/A   & $0.0007$   & $4.24\cdot10^{-5}$  & $0.0015$    & $0.0516/2.2700$                    \\ \hline
       % MLP   & \textit{Adult}   & N/A   & $0.0018$   & $0.0018$  & $0.0601$     & $0.0764/2.2068$                    \\ \hline
        CNN   & \textit{CIFAR-10} & N/A   & $0.0050$   & $0.0864$ & N/A    & $0.3018/
        2.1502$                     \\ \hline
    \end{tabular}
\end{table*}

\begin{table*}[htb!]
    \centering
    \caption{Mean value and standard deviation of training time across 5 different runs. The reported time (in seconds) corresponds to the generation of each entry in Table \ref{tab:regularization_comparison}. Times are }
    \label{tab:times}
    \begin{tabular}{|c|c|c|c|c|c|c|}
        \hline
        \textbf{Model} & \textbf{Dataset} & \textbf{No-Reg} & \textbf{L1-Reg} & \textbf{L2-Reg} & \textbf{Dropout} & \textbf{CF-Reg (ours)} \\ \hline
        Logistic Regression   & \textit{Water}   & $222.98 \pm 1.07$   & $239.94 \pm 2.59$   & $241.60 \pm 1.88$  & N/A    & $251.50 \pm 1.93$                     \\ \hline
        MLP   & \textit{Water}   & $225.71 \pm 3.85$   & $250.13 \pm 4.44$   & $255.78 \pm 2.38$  & $237.83 \pm 3.45$    & $266.48 \pm 3.46$                    \\ \hline
        Logistic Regression   & \textit{Phomene}   & $266.39 \pm 0.82$ & $367.52 \pm 6.85$   & $361.69 \pm 4.04$  & N/A   & $310.48 \pm 0.76$                    \\ \hline
        MLP   &
        \textit{Phomene} & $335.62 \pm 1.77$   & $390.86 \pm 2.11$   & $393.96 \pm 1.95$ & $363.51 \pm 5.07$    & $403.14 \pm 1.92$                     \\ \hline
       % MLP   & \textit{Adult}   & N/A   & $0.0018$   & $0.0018$  & $0.0601$     & $0.0764/2.2068$                    \\ \hline
        CNN   & \textit{CIFAR-10} & $370.09 \pm 0.18$   & $395.71 \pm 0.55$   & $401.38 \pm 0.16$ & N/A    & $1287.8 \pm 0.26$                     \\ \hline
    \end{tabular}
\end{table*}

\subsection{Feasibility of our Method}
A crucial requirement for any regularization technique is that it should impose minimal impact on the overall training process.
In this respect, CF-Reg introduces an overhead that depends on the time required to find the optimal counterfactual example for each training instance. 
As such, the more sophisticated the counterfactual generator model probed during training the higher would be the time required. However, a more advanced counterfactual generator might provide a more effective regularization. We discuss this trade-off in more details in Section~\ref{sec:discussion}.

Table~\ref{tab:times} presents the average training time ($\pm$ standard deviation) for each model and dataset combination listed in Table~\ref{tab:regularization_comparison}.
We can observe that the higher accuracy achieved by CF-Reg using the score-based counterfactual generator comes with only minimal overhead. However, when applied to deep neural networks with many hidden layers, such as \textit{PreactResNet-18}, the forward derivative computation required for the linearization of the network introduces a more noticeable computational cost, explaining the longer training times in the table.

\subsection{Hyperparameter Sensitivity Analysis}
The proposed counterfactual regularization technique relies on two key hyperparameters: $\alpha$ and $\beta$. The former is intrinsic to the loss formulation defined in (\ref{eq:cf-train}), while the latter is closely tied to the choice of the score-based counterfactual explanation method used.

Figure~\ref{fig:test_alpha_beta} illustrates how the test accuracy of an MLP trained on the \textit{Water Potability} dataset changes for different combinations of $\alpha$ and $\beta$.

\begin{figure}[ht]
    \centering
    \includegraphics[width=0.85\linewidth]{img/test_acc_alpha_beta.png}
    \caption{The test accuracy of an MLP trained on the \textit{Water Potability} dataset, evaluated while varying the weight of our counterfactual regularizer ($\alpha$) for different values of $\beta$.}
    \label{fig:test_alpha_beta}
\end{figure}

We observe that, for a fixed $\beta$, increasing the weight of our counterfactual regularizer ($\alpha$) can slightly improve test accuracy until a sudden drop is noticed for $\alpha > 0.1$.
This behavior was expected, as the impact of our penalty, like any regularization term, can be disruptive if not properly controlled.

Moreover, this finding further demonstrates that our regularization method, CF-Reg, is inherently data-driven. Therefore, it requires specific fine-tuning based on the combination of the model and dataset at hand.
\section{Experiments}
\subsection{Setup}
\begin{table*}[!htb]
\centering
\renewcommand{\arraystretch}{1.2}
\caption{Last and average performance results on four benchmark datasets (10 tasks) are reported. The mean and standard deviation of three trials are provided. We compare all methods using the same ViT-B/16-IN21K backbone, seeds, and class orders. For L2P and DualPrompt, we use the implementations provided by \cite{zhou2024continual}. Missing implementations on the datasets are denoted as '-'.}
\label{tab:results}
\resizebox{\textwidth}{!}{ 
\begin{tabular}{lcccccccc}
\toprule
\multirow{2}{*}{\textbf{Method}} & \multicolumn{2}{c}{\textbf{ImageNet-R}} & \multicolumn{2}{c}{\textbf{ImageNet-A}} & \multicolumn{2}{c}{\textbf{CUB-200}} & \multicolumn{2}{c}{\textbf{CIFAR-100}} \\
              ~         & $\mathcal{A}_{\text{Last}}$ & $\mathcal{A}_{\text{Avg}}$ & $\mathcal{A}_{\text{Last}}$ & $\mathcal{A}_{\text{Avg}}$ & $\mathcal{A}_{\text{Last}}$ & $\mathcal{A}_{\text{Avg}}$ & $\mathcal{A}_{\text{Last}}$ & $\mathcal{A}_{\text{Avg}}$ \\ 
\midrule
L2P\cite{wang2022learning} & 70.56$\pm$0.51  &  75.60$\pm$0.34  & - & -  & -    & -  & 84.74$\pm$0.44    &  88.65$\pm$1.02 \\
DualPrompt\cite{wang2022dualprompt} &  66.89$\pm$0.70   &  71.60$\pm$0.31  & -   & -   & -    & -  & 85.17$\pm$0.55  & 89.18$\pm$1.01\\
CODA-Prompt\cite{smith2023coda}  & 72.82$\pm$0.50    & 78.13$\pm$0.52   & -   & -   & -   & -  & 87.00$\pm$0.31    & 90.68$\pm$1.02 \\ 
SLCA\cite{zhang2023slca}  &  78.95$\pm$0.36   & 83.20$\pm$0.23 & -   & -   &  86.13$\pm$0.57   & 91.75$\pm$0.42  &  90.30$\pm$0.43   & 93.32$\pm$0.99 \\
RanPAC\cite{mcdonnell2023ranpac} & 77.94$\pm$0.14 & 82.98$\pm$0.31 & 62.25$\pm$0.35 & 69.92$\pm$1.69 & 89.94$\pm$0.52 & \textcolor{blue}{93.68$\pm$0.48} & \textcolor{red}{92.09$\pm$0.13} & \textcolor{red}{94.75$\pm$0.64} \\
OS-Prompt\cite{kim2024one} & 74.76$\pm$0.23    & 80.29$\pm$0.71   & -   & -  & -    & - & 86.50$\pm$0.11    & 90.68$\pm$1.32 \\
VQ-Prompt\cite{jiao2024vector} & 75.68$\pm$0.23  &  80.02$\pm$0.18  &   -  &  -  &  86.47$\pm$0.40   & 91.37$\pm$0.54  & 90.27$\pm$0.06  & 93.10$\pm$0.84 \\
CPrompt\cite{gao2024consistent} &  76.38$\pm$0.46   & 81.52$\pm$0.38   & -    & -   & -    & -  & 87.63$\pm$0.17  & 91.50$\pm$1.10 \\
EASE\cite{zhou2024expandable} & 75.91$\pm$0.17  &    81.38$\pm$0.29 & 54.93$\pm$1.14  & 63.92$\pm$0.76   &   85.04$\pm$1.42 &  90.93$\pm$1.03 &  88.22$\pm$0.44 &  92.02$\pm$0.76 \\
InfLoRA+CA\cite{liang2024inflora} & 78.78$\pm$0.31 & 83.37$\pm$0.54 & -    & -   & -   & - & 91.39$\pm$0.27 & 94.06$\pm$0.88  \\
SSIAT\cite{tan2024semantically}  & \textcolor{blue}{79.55$\pm$0.27}  & \textcolor{blue}{83.70$\pm$0.38}   & \textcolor{blue}{62.65$\pm$1.28}    & \textcolor{blue}{71.14$\pm$1.24}   &   89.68$\pm$0.48  & 93.67$\pm$0.46  & 91.41$\pm$0.14    & 94.27$\pm$0.75 \\ 
MOS\cite{sun2024mos} & 77.68$\pm$0.41 & 82.06$\pm$0.53 & 54.75$\pm$1.09 & 63.32$\pm$2.38 & \textcolor{blue}{89.97$\pm$0.32} & 93.43$\pm$0.60 & 91.53$\pm$0.35 & 94.21$\pm$0.91 \\
\midrule
\textbf{Ours} & \textcolor{red}{81.88$\pm$0.07}  & \textcolor{red}{85.95$\pm$0.27}  & \textcolor{red}{64.14$\pm$0.58}  & \textcolor{red}{71.45$\pm$1.35}   & \textcolor{red}{90.52$\pm$0.13}    & \textcolor{red}{93.93$\pm$0.47} & \textcolor{blue}{91.94$\pm$0.17}    & \textcolor{blue}{94.43$\pm$0.79} \\ 
\bottomrule
\end{tabular}
}
\vspace{-3mm}
\end{table*}
% \begin{table*}[!htb]
% \centering
% \caption{Performance comparison of different methods on various datasets (10tasks). pretrained in ViT-B/16 on ImageNet-21K Missing implementation on the dataset in the paper are denoted as "-", For L2P and DualPrompt, we use the implementation provided by \cite{zhou2024continual}}
% \label{tab:results}
% \resizebox{\textwidth}{!}{ 
% \begin{tabular}{lcccccccc}
% \toprule
% \multirow{2}{*}{\textbf{Method}} & \multicolumn{2}{c}{\textbf{CIFAR100}} & \multicolumn{2}{c}{\textbf{Split-ImageNetR}} & \multicolumn{2}{c}{\textbf{Split-ImageNetA}} & \multicolumn{2}{c}{\textbf{CUB200}} \\
%               ~         & $\mathcal{A}_{\text{Last}}$ & $\mathcal{A}_{\text{Avg}}$ & $\mathcal{A}_{\text{Last}}$ & $\mathcal{A}_{\text{Avg}}$ & $\mathcal{A}_{\text{Last}}$ & $\mathcal{A}_{\text{Avg}}$ & $\mathcal{A}_{\text{Last}}$ & $\mathcal{A}_{\text{Avg}}$ \\ 
% \midrule
% L2P\cite{wang2022learning}$_{\text{CVPR22}}$ & 84.74$\pm$0.44    &  88.65$\pm$1.02 &   70.56$\pm$0.51  &  75.60$\pm$0.34  & - & -  & -    & -  \\
% DualPrompt\cite{wang2022dualprompt}$_{\text{ECCV22}}$ & 85.17$\pm$0.55  & 89.18$\pm$1.01&  66.89$\pm$0.70   &  71.60$\pm$0.31  & -   & -   & -    & -  \\
% CODA-Prompt\cite{smith2023coda}$_{\text{CVPR23}}$ & 87.00$\pm$0.31    & 90.68$\pm$1.02   & 72.82$\pm$0.50    & 78.13$\pm$0.52   & -   & -   & -   & -   \\ 
% SLCA\cite{zhang2023slca}$_{\text{ICCV23}}$  &  90.30$\pm$0.43   & 93.32$\pm$0.99   &  78.95$\pm$0.36   & 83.20$\pm$0.23 & -   & -   &  86.13$\pm$0.57   & 91.75$\pm$0.42  \\
% RanPAC\cite{mcdonnell2023ranpac}$_{\text{NIPS23}}$ & 92.09$\pm$0.13 & 94.75$\pm$0.64& 77.94$\pm$0.14 & 82.98$\pm$0.31 & 62.25$\pm$0.35 & 69.92$\pm$1.69 & 89.94$\pm$0.52 & 93.68$\pm$0.48 \\
% OS-Prompt\cite{kim2024one}$_{\text{ECCV24}}$                           & 86.50$\pm$0.11    & 90.68$\pm$1.32   & 74.76$\pm$0.23    & 80.29$\pm$0.71   & -   & -  & -    & -  \\
% VQ-Prompt\cite{jiao2024vector}$_{\text{NIPS24}}$ &   90.27$\pm$0.06  & 93.10$\pm$0.84   &   75.68$\pm$0.23  &  80.02$\pm$0.18  &   -  &  -  &  86.47$\pm$0.40   & 91.37$\pm$0.54   \\
% CPrompt\cite{gao2024consistent}$_{\text{CVPR24}}$   &   87.63$\pm$0.17  & 91.50$\pm$1.10   &  76.38$\pm$0.46   & 81.52$\pm$0.38   & -    & -   & -    & -  \\
% EASE\cite{zhou2024expandable}$_{\text{CVPR24}}$   &  88.22$\pm$0.44 &  92.02$\pm$0.76 &  75.91$\pm$0.17  &    81.38$\pm$0.29       & 54.93$\pm$1.14  & 63.92$\pm$0.76   &   85.04$\pm$1.42 &  90.93$\pm$1.03 \\
% InfLoRA+CA\cite{liang2024inflora}$_{\text{CVPR24}}$   & 91.39$\pm$0.27 & 94.06$\pm$0.88 & 78.78$\pm$0.31 & 83.37$\pm$0.54 & -    & -   & -   & -   \\
% SSIAT\cite{tan2024semantically}$_{\text{CVPR24}}$  & 91.41$\pm$0.14    & 94.27$\pm$0.75   & 79.55$\pm$0.27  & 83.70$\pm$0.38   & 62.65$\pm$1.28    & 71.14$\pm$1.24   &   89.68$\pm$0.48  & 93.67$\pm$0.46   \\ 
% MOS\cite{sun2024mos} $_{\text{AAAI25}}$ & 91.53$\pm$0.35 & 94.21$\pm$0.91 & 77.68$\pm$0.41 & 82.06$\pm$0.53 & 54.75$\pm$1.09 & 63.32$\pm$2.38 & 89.97$\pm$0.32 & 93.43$\pm$0.60\\
% \midrule
% \textbf{Ours}        & 91.94$\pm$0.17    & 94.43$\pm$0.79   & 81.88$\pm$0.07  & 85.95$\pm$0.27  & 64.14$\pm$0.58  & 71.45$\pm$1.35   & 90.52$\pm$0.13    & 93.93$\pm$0.47 \\ 
% \bottomrule
% \end{tabular}
% }
% \end{table*}
\subsubsection{Datasets and Metrics.}
We train and validate our method using four popular CIL datasets. ImageNet-R \cite{hendrycks2021many} is generated by applying artistic processing to 200 classes from ImageNet. The dataset consists of 200 categories, and we split ImageNet-R into 5, 10, and 20 tasks, with each task containing 40, 20, and 10 classes, respectively. CIFAR-100 \cite{Krizhevsky2009LearningML} is a widely used dataset in CIL, containing 60,000 images across 100 categories. We also split CIFAR-100 into 5, 10, and 20 tasks with each task containing 20, 10, 5 classes, respectively. CUB-200 \cite{WahCUB_200_2011} is a fine-grained dataset containing approximately 11,788 images of 200 bird species with detailed class labels. ImageNet-A \cite{hendrycks2021natural}is a real-world dataset consisting of 200 categories, notable for significant class imbalance, with some categories having very few training samples. We split CUB-200 and ImageNet-A into 10 tasks with 20 classes each. 

% VTAB \cite{zhai2019large} is a complex dataset consisting of 19 tasks that span a broad range of domains and semantics. Following previous work \cite{zhou2024revisiting}, we select 5 tasks from VTAB to construct a cross-domain CIL dataset.

We follow the commonly used evaluation metrics in CIL. We denote $a_{i,j}$ as the classification accuracy evaluated on the test set of the $j$-th task (where $j \leq i$) after learning $i$ tasks in incremental learning. The final accuracy is calculated as $\mathcal{A}_{\text{last}} = \frac{1}{t}\sum_{j=1}^t a_{i,j}$
and the average accuracy of all incremental tasks is $\mathcal{A}_{\text{avg}} = \frac{1}{T}\sum_{i=1}^T \mathcal{A}_i$. In line with other studies, our evaluation results are based on three trials with three different seeds. We report both the mean and standard deviation of the trials.

\subsubsection{Implementation Details.}
In our experiment, we adopt ViT-B/16 \cite{dosovitskiy2021an} pre-trained on ImageNet21K \cite{russakovsky2015imagenet} as the backbone. We use the SGD optimizer with the initial learning rate set as 0.01 and we use the Cosine Annealing scheduler. We train the first session for 20 epochs and 10 epochs for later sessions. The batch size is set to 48 for all the experiments. LoRA module is inserted to the key and value of all the attention layers in the transformer. The distillation loss weight \( \lambda \) is set to 0.4, the LoRA rank \( r \) is set to 32, and the scale \( s \) in the angular penalty loss is set to 20. These values are determined through sensitivity analysis.


\subsection{Comparison with State-of-the-arts}
\begin{table*}[ht]
\centering
\renewcommand{\arraystretch}{1.2}
\caption{Last and average results on ImageNet-R and CIFAR-100 are reported. The mean and standard deviation of three trials are provided for 5 and 20 tasks settings. We compare all methods using the same ViT-B/16-IN21K backbone, seeds, and class orders. }
\label{tab:cifar_imgr_comparison}
\resizebox{\textwidth}{!}{%
\begin{tabular}{l|cc|cc|cc|cc}
\toprule
\multirow{3}{*}{\textbf{Method}} 
& \multicolumn{4}{c|}{\textbf{ImageNet-R}} & \multicolumn{4}{c}{\textbf{CIFAR-100}} \\
\cmidrule(lr){2-5} \cmidrule(lr){6-9}
& \multicolumn{2}{c|}{5 tasks} & \multicolumn{2}{c|}{20 tasks} 
& \multicolumn{2}{c|}{5 tasks} & \multicolumn{2}{c}{20 tasks} \\
& $\mathcal{A}_{\text{Last}}$ & $\mathcal{A}_{\text{Avg}}$ 
& $\mathcal{A}_{\text{Last}}$ & $\mathcal{A}_{\text{Avg}}$ 
& $\mathcal{A}_{\text{Last}}$ & $\mathcal{A}_{\text{Avg}}$ 
& $\mathcal{A}_{\text{Last}}$ & $\mathcal{A}_{\text{Avg}}$ \\
\midrule
CODA-Prompt\cite{smith2023coda} & 74.91$\pm$0.33 & 79.25$\pm$0.53 & 68.62$\pm$0.52 & 74.61$\pm$0.31 & 89.16$\pm$0.08 & 92.46$\pm$0.79& 81.18$\pm$0.71 & 86.62$\pm$0.93 \\
SLCA\cite{zhang2023slca} & \textcolor{blue}{81.01$\pm$0.11}  & 84.18$\pm$0.28  &  77.23$\pm$0.40  & 82.21$\pm$0.49  &  91.30$\pm$0.54  & 93.97$\pm$0.56  & 88.54$\pm$0.35  & 92.66$\pm$0.78  \\
RanPAC\cite{mcdonnell2023ranpac} & 79.53$\pm$0.12 & 83.69$\pm$0.13 & 75.47$\pm$0.22 & 81.20$\pm$0.15 & \textcolor{red}{92.68$\pm$0.16} & \textcolor{red}{94.85$\pm$0.54} & \textcolor{red}{90.77$\pm$0.17} & \textcolor{red}{94.00$\pm$0.74} \\
OS-Prompt\cite{kim2024one} & 75.78$\pm$0.06 &  80.49$\pm$0.49 & 72.50$\pm$0.78 & 78.43$\pm$1.07 & 88.35$\pm$0.23 & 92.04$\pm$0.76 & 81.46$\pm$0.56 & 87.15$\pm$1.15 \\
CPrompt\cite{gao2024consistent} & 77.99$\pm$0.31 & 82.34$\pm$0.32 & 73.77$\pm$0.14 & 79.81$\pm$0.50 & 89.04$\pm$0.41 & 92.26$\pm$0.57 & 84.48$\pm$0.46 & 89.35$\pm$1.09 \\
VQ-Prompt\cite{jiao2024vector} & 76.00$\pm$0.28 & 79.84$\pm$0.56 & 74.76$\pm$0.29 & 79.30$\pm$0.34 & 90.97$\pm$0.17 & 93.50$\pm$0.67 & 89.25$\pm$0.43 & 92.53$\pm$0.68 \\
EASE\cite{zhou2024expandable}  & 76.75$\pm$0.41 & 81.14$\pm$0.31 & 73.90$\pm$0.61 & 80.26$\pm$0.52 & 89.53$\pm$0.15 & 92.64$\pm$0.73 & 86.30$\pm$0.36 & 90.80$\pm$0.99 \\
SSIAT\cite{tan2024semantically} & 80.52$\pm$0.07 & \textcolor{blue}{84.25$\pm$0.31} & \textcolor{blue}{78.35$\pm$0.34} & \textcolor{blue}{82.39$\pm$0.42} & 92.01$\pm$0.15 & 94.37$\pm$0.68 & 90.07$\pm$0.44 & \textcolor{blue}{93.52$\pm$0.65}\\

InfLoRA+CA\cite{liang2024inflora} & 80.92$\pm$0.28 & 84.22$\pm$0.30 & 76.50$\pm$0.30 & 81.57$\pm$0.34 & 92.28$\pm$0.06 & 94.46$\pm$0.63 & 90.39$\pm$0.01 & 93.32$\pm$0.72 \\
MOS\cite{sun2024mos} & 78.76$\pm$0.17 & 82.37$\pm$0.26 & 75.16$\pm$0.59 & 80.53$\pm$0.70  & 92.31$\pm$0.20 & 94.44$\pm$0.74 & 89.43$\pm$0.37 & 92.95$\pm$0.79\\
\midrule
\textbf{Ours} & \textcolor{red}{83.37$\pm$0.26} & \textcolor{red}{87.00$\pm$0.35} & \textcolor{red}{79.43$\pm$0.34} & \textcolor{red}{84.34$\pm$0.32} & \textcolor{blue}{92.40$\pm$0.08} & \textcolor{blue}{94.71$\pm$0.64} & \textcolor{blue}{90.51$\pm$0.14} & 93.48$\pm$0.67\\
\bottomrule
\end{tabular}
}
\vspace{-5mm}
\end{table*}

% \begin{table*}[ht]
% \centering
% \renewcommand{\arraystretch}{1.2}
% \caption{Experimental results comparison on CIFAR-100 and ImageNet-R datasets. $\mathcal{A}_{\text{Last}}$ denotes the final task accuracy, $\mathcal{A}_{\text{Avg}}$ represents the average accuracy across all tasks.}
% \label{tab:cifar_imgr_comparison}
% \resizebox{\textwidth}{!}{%
% \begin{tabular}{l|cc|cc|cc|cc}
% \toprule
% \multirow{3}{*}{\textbf{Method}} 
% & \multicolumn{4}{c|}{\textbf{CIFAR-100}} & \multicolumn{4}{c}{\textbf{ImageNet-R}} \\
% \cmidrule(lr){2-5} \cmidrule(lr){6-9}
% & \multicolumn{2}{c|}{5 tasks} & \multicolumn{2}{c|}{20 tasks} 
% & \multicolumn{2}{c|}{5 tasks} & \multicolumn{2}{c}{20 tasks} \\
% & $\mathcal{A}_{\text{Last}}$ & $\mathcal{A}_{\text{Avg}}$ 
% & $\mathcal{A}_{\text{Last}}$ & $\mathcal{A}_{\text{Avg}}$ 
% & $\mathcal{A}_{\text{Last}}$ & $\mathcal{A}_{\text{Avg}}$ 
% & $\mathcal{A}_{\text{Last}}$ & $\mathcal{A}_{\text{Avg}}$ \\
% \midrule
% CODA-Prompt\cite{smith2023coda} & 89.16$\pm$0.08 & 92.46$\pm$0.79& 81.18$\pm$0.71 & 86.62$\pm$0.93 & 74.91$\pm$0.33 & 79.25$\pm$0.53 & 68.62$\pm$0.52 & 74.61$\pm$0.31 \\
% SLCA\cite{zhang2023slca}  &  91.30$\pm$0.54  & 93.97$\pm$0.56  & 88.54$\pm$0.35  & 92.66$\pm$0.78  & \textcolor{blue}{81.01$\pm$0.11}  & 84.18$\pm$0.28  &  77.23$\pm$0.40  & 82.21$\pm$0.49  \\
% RanPAC\cite{mcdonnell2023ranpac} & \textcolor{red}{92.68$\pm$0.16} & \textcolor{red}{94.85$\pm$0.54} & \textcolor{red}{90.77$\pm$0.17} & \textcolor{red}{94.00$\pm$0.74} & 79.53$\pm$0.12 & 83.69$\pm$0.13 & 75.47$\pm$0.22 & 81.20$\pm$0.15 \\
% OS-Prompt\cite{kim2024one} & 88.35$\pm$0.23 & 92.04$\pm$0.76 & 81.46$\pm$0.56 & 87.15$\pm$1.15 & 75.78$\pm$0.06 &  80.49$\pm$0.49 & 72.50$\pm$0.78 & 78.43$\pm$1.07 \\
% CPrompt\cite{gao2024consistent} & 89.04$\pm$0.41 & 92.26$\pm$0.57 & 84.48$\pm$0.46 & 89.35$\pm$1.09 & 77.99$\pm$0.31 & 82.34$\pm$0.32 & 73.77$\pm$0.14 & 79.81$\pm$0.50 \\
% VQ-Prompt\cite{jiao2024vector} & 90.97$\pm$0.17 & 93.50$\pm$0.67 & 89.25$\pm$0.43 & 92.53$\pm$0.68 & 76.00$\pm$0.28 & 79.84$\pm$0.56 & 74.76$\pm$0.29 & 79.30$\pm$0.34 \\
% EASE\cite{zhou2024expandable}   & 89.53$\pm$0.15 & 92.64$\pm$0.73 & 86.30$\pm$0.36 & 90.80$\pm$0.99 & 76.75$\pm$0.41 & 81.14$\pm$0.31 & 73.90$\pm$0.61 & 80.26$\pm$0.52 \\
% SSIAT\cite{tan2024semantically}  & 92.01$\pm$0.15 & 94.37$\pm$0.68 & 90.07$\pm$0.44 & \textcolor{blue}{93.52$\pm$0.65}  & 80.52$\pm$0.07 & \textcolor{blue}{84.25$\pm$0.31} & \textcolor{blue}{78.35$\pm$0.34} & \textcolor{blue}{82.39$\pm$0.42} \\
% InfLoRA+CA\cite{liang2024inflora} & 92.28$\pm$0.06 & 94.46$\pm$0.63 & 90.39$\pm$0.01 & 93.32$\pm$0.72 & 80.92$\pm$0.28 & 84.22$\pm$0.30 & 76.50$\pm$0.30 & 81.57$\pm$0.34 \\
% MOS\cite{sun2024mos} & 92.31$\pm$0.20 & 94.44$\pm$0.74 & 89.43$\pm$0.37 & 92.95$\pm$0.79 & 78.76$\pm$0.17 & 82.37$\pm$0.26 & 75.16$\pm$0.59 & 80.53$\pm$0.70 \\
% \midrule
% \textbf{Ours} & \textcolor{blue}{92.40$\pm$0.08} & \textcolor{blue}{94.71$\pm$0.64} & \textcolor{blue}{90.51$\pm$0.14} & 93.48$\pm$0.67 & \textcolor{red}{83.37$\pm$0.26} & \textcolor{red}{87.00$\pm$0.35} & \textcolor{red}{79.43$\pm$0.34} & \textcolor{red}{84.34$\pm$0.32} \\
% \bottomrule
% \end{tabular}
% }
% \end{table*}
We conduct a comparative evaluation of our proposed method against state-of-the-art (SOTA) class-incremental learning (CIL) approaches based on pre-trained models, with a particular focus on techniques utilizing parameter-efficient fine-tuning (PEFT). To ensure a fair comparison, we evaluate all methods using the same ViT-B/16-IN21K pre-trained models, identical random seeds, and consistent class orders. Specifically, we compare prompt-based approaches, including L2P \cite{wang2022learning}, DualPrompt \cite{wang2022dualprompt}, CODAPrompt \cite{smith2023coda}, VQ-Prompt \cite{jiao2024vector}, OS-Prompt \cite{kim2024one}, and CPrompt \cite{gao2024consistent}; adapter-based methods such as SSIAT \cite{tan2024semantically}, EASE \cite{zhou2024expandable}, and MOS \cite{sun2024mos}; the LoRA-based method InfLoRA \cite{liang2024inflora} integrated with Classifier Alignment, referred to as InfLoRA+CA; the first-session adaptation method RanPAC \cite{mcdonnell2023ranpac}, which only trains PEFT modules using data from the first task; and the fine-tuning method SLCA \cite{zhang2023slca}, which also incorporates the Classifier Alignment step as in our approach. 

Table \ref{tab:results} summarizes the performance of these SOTA methods across four widely used benchmark datasets. We report both the accuracy on the last task ($\mathcal{A}_{\text{last}}$) and the average accuracy across all tasks ($\mathcal{A}_{\text{avg}}$), presenting the mean and standard deviation over three independent runs with different random seeds. The use of random seeds introduces variability in class order across runs, making the evaluation of model performance more challenging. Notably, our method achieves superior performance in both $\mathcal{A}_{\text{last}}$ and $\mathcal{A}_{\text{avg}}$. Our method demonstrates impressive results, particularly on the more challenging datasets with larger domain gaps, such as ImageNet-R and ImageNet-A. On ImageNet-R, our method achieves a final accuracy of 81.88\%, surpassing the second-best method, SSIAT, by a significant margin of 2.33\%. The $\mathcal{A}_{\text{avg}}$ also surpasses the second-best SSIAT by 2.25\%. On the ImageNet-A dataset, our method achieves a final accuracy of 64.14\%, surpassing the second-best SSIAT by 1.49\%. These results highlight the effectiveness of our PEFT-based approach in significantly improving performance on datasets with large domain shifts, outperforming both first-session adaptation methods and full fine-tuning methods, as well as other PEFT-based approaches.

In contrast, on the CIFAR-100 and CUB-200 datasets, our method performs well, though with marginal benefits compared to other methods. Notably, on the CUB-200 dataset, our method achieves superior performance. It is also important to note that the first-session adaptation-based method, RanPAC, performs well on both CIFAR-100 and CUB-200, likely due to the significant relevance between the pretraining dataset (ImageNet) and these two datasets.

\begin{figure*}[!htb]
\centering
\includegraphics[width=\textwidth]{images/per_task_result.pdf}
\vspace{-8mm}
\caption{\small The performance of each learning session under different settings of ImageNet-R and CIFAR100. All methods are initialized with ViT-B/16-IN21k. These curves are plotted by calculating the average performance across three different seeds.}
%for each incremental session.}

\label{fig:per_task_results}
\vspace{-4mm}
\end{figure*}


Additionally, we evaluate the performance of our method on both longer task sequences (20 tasks) and shorter task sequences (5 tasks) for CIFAR-100 and ImageNet-R, as reported in Tables \ref{tab:cifar_imgr_comparison}. Across these varied experimental settings, our method consistently outperforms competing approaches, demonstrating its stability and robustness in handling diverse CIL scenarios.

Figure \ref{fig:per_task_results} illustrates the incremental accuracy of each session for three ImageNet-R settings and one CIFAR-100 setting. The results show that on ImageNet-R, our method consistently achieves the best performance, with a clear distinction from other methods. On CIFAR-100, our method is relatively stable, and the final results are comparable to the first-session adaptation-based method, RanPAC, which is closely related to pretraining data characteristics.
% Our approach demonstrates superior performance on the more challenging datasets, ImageNet-R and ImageNet-A, which exhibit a larger domain gap from the pre-training data and have substantial intra-class diversity. This highlights the robustness of our method in addressing domain shifts.  
% Additionally, we evaluate the performance of our method on both longer task sequences (20 tasks) and shorter task sequences (5 tasks) for CIFAR-100 and ImageNet-R, as reported in Tables \ref{tab:cifar_imgr_comparison}. Across these varied experimental settings, our method consistently outperforms competing approaches, demonstrating its stability and robustness in handling diverse CIL scenarios.
% \begin{table*}[ht]
\centering
\label{tab:imgr_seq}
\caption{Experimental results for 5 tasks, 10 tasks, 20 tasks on Imagenet-R dataset. $\mathcal{A}_{\text{Last}}$ represents the last accuracy, and $\mathcal{A}_{\text{Avg}}$ indicates the average accuracy across tasks.}
\resizebox{\textwidth}{!}{ 
\begin{tabular}{l|cc|cc|cc}
\toprule
\multirow{2}{*}{\textbf{Method}} & \multicolumn{2}{c|}{\textbf{5 tasks}} & \multicolumn{2}{c|}{\textbf{10 tasks}}& \multicolumn{2}{c}{\textbf{20 tasks}} \\ 
\cline{2-7} 
        ~         & $\mathcal{A}_{\text{Last}}$ & $\mathcal{A}_{\text{Avg}}$ & 
        $\mathcal{A}_{\text{Last}}$ & $\mathcal{A}_{\text{Avg}}$&
        $\mathcal{A}_{\text{Last}}$ & $\mathcal{A}_{\text{Avg}}$ \\ 
\midrule
CODA-Prompt & 74.91$\pm$0.33 & 79.25$\pm$0.53 & 72.82$\pm$0.50 & 78.13$\pm$0.52 & 68.62$\pm$0.52 & 74.61$\pm$0.31 \\
OS-Prompt\cite{kim2024one} & 75.78$\pm$0.06 & 80.49$\pm$0.49 & 74.76$\pm$0.23 & 80.29$\pm$0.71 & 72.50$\pm$0.78 & 78.43$\pm$1.07 \\
VQ-Prompt\cite{jiao2024vector} & 76.00$\pm$0.28 & 79.84$\pm$0.56 & 75.68$\pm$0.23&80.02$\pm$0.18 & 74.76$\pm$0.29 & 79.30$\pm$0.34\\
CPrompt\cite{gao2024consistent}  & 77.99$\pm$0.31 & 82.34$\pm$0.32 & 76.38$\pm$0.46 & 81.52$\pm$0.38 & 73.77$\pm$0.14 & 79.81$\pm$0.50 \\
EASE\cite{zhou2024expandable}       &  76.75$\pm$0.41  & 81.14$\pm$0.31 & 75.91$\pm$0.17 & 81.38$\pm$0.29  & 73.90$\pm$0.61  & 80.26$\pm$0.52  \\ 
SSIAT\cite{tan2024semantically}        &   80.52$\pm$0.07 & 84.25$\pm$0.31 & 79.38$\pm$0.59 & 83.63$\pm$0.43   & 75.67$\pm$0.14    & 82.30$\pm$0.36  \\ 
InfLoRA+CA\cite{liang2024inflora}   & 80.92$\pm$0.28 & 84.22$\pm$0.30 &  78.78$\pm$0.31  & 83.37$\pm$0.54  & 76.50$\pm$0.30    & 81.57$\pm$0.34  \\
MOS & 78.76$\pm$0.17 & 82.37$\pm$0.26 & 77.68$\pm$0.41 & 82.06$\pm$0.53 & 75.16$\pm$0.59 & 80.53$\pm$0.70\\
\midrule
\textbf{Ours}    &  83.37$\pm$0.26  & 87.00$\pm$0.35& 81.88$\pm$0.07 &  85.95$\pm$0.27 &79.47$\pm$0.15 & 84.17$\pm$0.28   \\ 
\bottomrule
\end{tabular}
}
\end{table*}
% \begin{table*}[ht]
\centering
\label{tab:cifar_seq}
\caption{Experimental results for 5 tasks, 10 tasks, 20 tasks on CIFAR-100 dataset. $\mathcal{A}_{\text{Last}}$ represents the last accuracy, and $\mathcal{A}_{\text{Avg}}$ indicates the average accuracy across tasks.}
\resizebox{\textwidth}{!}{ 
\begin{tabular}{l|cc|cc|cc}
\toprule
\multirow{2}{*}{\textbf{Method}} & \multicolumn{2}{c|}{\textbf{5 tasks}} & \multicolumn{2}{c|}{\textbf{10 tasks}} &\multicolumn{2}{c}{\textbf{20 tasks}} \\ 
\cline{2-7}
        ~         & $\mathcal{A}_{\text{Last}}$ & $\mathcal{A}_{\text{Avg}}$ & $\mathcal{A}_{\text{Last}}$ & $\mathcal{A}_{\text{Avg}}$&
        $\mathcal{A}_{\text{Last}}$ & $\mathcal{A}_{\text{Avg}}$ \\ 
\midrule
CODA-Prompt & 89.16$\pm$0.08 & 92.46$\pm$0.79 & 87.00$\pm$0.31 & 90.68$\pm$1.02 & 81.18$\pm$0.71& 86.62$\pm$0.93 \\
OS-Prompt\cite{kim2024one} & 88.35$\pm$0.23 & 92.04$\pm$0.76 & 86.50$\pm$0.11 & 90.68$\pm$1.32 & 81.46$\pm$0.56 & 87.15$\pm$1.15 \\
CPrompt\cite{gao2024consistent} & 89.04$\pm$0.41 &92.26$\pm$0.57 & 87.63$\pm$0.17 & 91.50$\pm$1.10 & 84.48$\pm$0.46 & 89.35$\pm$1.09 \\
EASE\cite{zhou2024expandable}        &  89.53$\pm$0.15 & 92.64$\pm$0.73 &  &    &  86.30$\pm$0.36   &   90.80$\pm$0.99 \\ 
SSIAT\cite{tan2024semantically}         &  92.01$\pm$0.15   & 94.37$\pm$0
.68& &  &   90.07$\pm$0.44  & 93.52$\pm$0.65  \\ 
InfLoRA+CA\cite{liang2024inflora}   & 92.28$\pm$0.06 & 94.46$\pm$0.63 &   &   & 90.39$\pm$0.01    & 93.32$\pm$0.72 \\ 
MOS & 92.31$\pm$0.20 & 94.44$\pm$0.74 & & & 89.43$\pm$0.37 & 92.95$\pm$0.79 \\
\midrule
\textbf{Ours}    &  92.40$\pm$0.08  & 94.71$\pm$0.64 & 91.94$\pm$0.17 & 94.43$\pm$0.79 & 90.44$\pm$0.10 & 93.51$\pm$0.62   \\ 
\bottomrule
\end{tabular}
}
\end{table*}
\subsection{Ablation Study}
\subsubsection{The Impact of Each Component}
As shown in Table \ref{tab:ablation_comp}, we systematically assess the contributions of different components to the baseline method on the ImageNet-R dataset. The baseline, consisting of a task-specific LoRA structure with angular penalty loss for classification, achieves a competitive performance of 79.36\% in \( \mathcal{A}_{\text{Last}} \). In exp-II, we add the MSC module, which, in conjunction with classifier alignment, provides an improvement of over 1.4\%. The incorporation of Covariance Calibration with classifier alignment in exp-III also leads to an improvement of approximately 1.35\%. These results underscore the importance of both mean shift compensation and covariance calibration in aligning feature distributions across tasks, thereby reducing catastrophic forgetting and enhancing stability across task sequences. Combining the MSC and CC modules gives a significant boost, improving performance by approximately 2.3\% above the baseline method. Finally, the inclusion of patch distillation offers a further marginal improvement, resulting in a state-of-the-art performance of 81.88\% for \( \mathcal{A}_{\text{Last}} \) and 85.95\% for \( \mathcal{A}_{\text{Avg}} \), confirming the effectiveness of our method.
\begin{table}[t]
\vspace{-13pt}
\caption{Ablation results with response number under fine-tuning setting. See Reward Margins in~\Cref{tab:rm}. \vspace{-0.5em}}
\label{tab:ablation}
\vskip 0.1in
\begin{center}
\scalebox{0.75}{
\begin{tabular}{clcccc}
\toprule
\multirow{1}{*}{\textbf{Number}} & \multirow{1}{*}{\textbf{Method}} & \textbf{BLEU}$\uparrow$ & \textbf{Reward} & $\textbf{RM}_{\text{DPO}}$$\uparrow$ & $\textbf{RM}_{\text{R-DPO}}$$\uparrow$ \\
\midrule
\multirow{2}{*}{\textbf{5}} & \textbf{DPO-BT} & \textbf{0.229} & \textbf{{0.432}} & 0.166 & -0.516 \\

& \textbf{DPO-HPS} & \textbf{0.229} & 0.431 & \textbf{0.600} & \textbf{-0.273} \\
\midrule
\multirow{2}{*}{\textbf{20}} & \textbf{DPO-BT} & \textbf{0.231} & 0.430 & 0.227 & -0.490 \\

& \textbf{DPO-HPS} & 0.224 & \textbf{{0.432}} & \textbf{0.822} & \textbf{-0.181} \\
\midrule
\multirow{2}{*}{\textbf{50}} & \textbf{DPO-BT} & \textbf{0.230} & \textbf{0.431} & 0.279 & -0.507 \\

& \textbf{DPO-HPS} & \textbf{0.230} & \textbf{0.431} & \textbf{1.645} & \textbf{1.037} \\
\midrule
\multirow{2}{*}{\textbf{100}} & \textbf{DPO-BT} & 0.230 & \textbf{0.431} & 0.349 & -0.455 \\

& \textbf{DPO-HPS} & \textbf{{0.232}} & 0.430 & \textbf{{2.723}} & \textbf{{2.040}} \\
\bottomrule
\end{tabular}}
\end{center}
\vspace{-1em}
\vspace{-9pt}
\end{table}

\subsubsection{LoRA Structures Design}
While there has been significant exploration of task-specific and task-shared PEFT modules, particularly concerning prompts and adapters, research on LoRA-based modules is relatively limited. In this paper, we investigate the use of task-specific and task-shared LoRA modules, as well as a hybrid architecture that combines both, inspired by Hydra-LoRA \cite{tian2024hydralora}. In Table \ref{tab:lora_results}, we evaluate these designs across four datasets and report the final accuracy averages from three trials. The results indicate that the performance differences among the LoRA designs are minimal, with the task-specific design slightly outperforming the other two, except for the ImageNet-A dataset, where the task-shared LoRA module achieves a marginally higher performance.
\begin{table}[!htb]
\centering
\renewcommand{\arraystretch}{1.5}
\vspace{-4mm}
\caption{Experimental results of different LoRA structures. We report the final accuracy $\mathcal{A}_{\text{Last}}$. Average of three trials.}
\resizebox{0.48\textwidth}{!}{ 
\begin{tabular}{l|c|c|c|c}
\toprule
\textbf{Structure}  & \textbf{CIFAR-100} & \textbf{ImageNet-R} & \textbf{ImageNet-A} & \textbf{CUB-200}\\ 
\midrule
 G-LoRA & 91.53$\pm$0.19  & 81.15$\pm$0.19& \textbf{64.27$\pm$0.08} & 90.16$\pm$0.56 \\ 
 E-LoRA& \textbf{91.94$\pm$0.17} & \textbf{81.88$\pm$0.07} & 64.14$\pm$0.58 & \textbf{90.52$\pm$0.13} \\ 
 Hydra-LoRA & 91.48$\pm$0.13 & 81.00$\pm$0.09 & 63.64$\pm$0.59 & 89.82$\pm$0.30\\ 
 \bottomrule
\end{tabular}
\label{tab:lora_results}
}
\end{table}
\vspace{-4mm}

%\section{Conclusion}
In this work, we propose a simple yet effective approach, called SMILE, for graph few-shot learning with fewer tasks. Specifically, we introduce a novel dual-level mixup strategy, including within-task and across-task mixup, for enriching the diversity of nodes within each task and the diversity of tasks. Also, we incorporate the degree-based prior information to learn expressive node embeddings. Theoretically, we prove that SMILE effectively enhances the model's generalization performance. Empirically, we conduct extensive experiments on multiple benchmarks and the results suggest that SMILE significantly outperforms other baselines, including both in-domain and cross-domain few-shot settings.
\section{Conclusion}
We analyze catastrophic forgetting in machine learning models using parameter-efficient fine-tuning based on pretrained models. Our experiments reveal that low-rank adaptations like LoRA induce feature mean and covariance shifts, termed Semantic Drift. To address this, we propose mean shift compensation and covariance calibration to constrain feature moments, maintaining both model stability and plasticity. Additionally, we implement feature self-distillation for patch tokens to enhance feature stability. Our task-agnostic continual learning framework outperforms existing methods across multiple public datasets.
%, demonstrating its effectiveness in mitigating Semantic Drift and improving continual learning performance.

% \section{Related Work}

% All submissions must follow the specified format.

% \subsection{Title}

% The paper title should be set in 14~point bold type and centered
% between two horizontal rules that are 1~point thick, with 1.0~inch
% between the top rule and the top edge of the page. Capitalize the
% first letter of content words and put the rest of the title in lower
% case.

% \subsection{Author Information for Submission}
% \label{author info}

% ICML uses double-blind review, so author information must not appear. If
% you are using \LaTeX\/ and the \texttt{icml2024.sty} file, use
% \verb+\icmlauthor{...}+ to specify authors and \verb+\icmlaffiliation{...}+ to specify affiliations. (Read the TeX code used to produce this document for an example usage.) The author information
% will not be printed unless \texttt{accepted} is passed as an argument to the
% style file.
% Submissions that include the author information will not
% be reviewed.

% \subsubsection{Self-Citations}

% If you are citing published papers for which you are an author, refer
% to yourself in the third person. In particular, do not use phrases
% that reveal your identity (e.g., ``in previous work \cite{langley00}, we
% have shown \ldots'').

% Do not anonymize citations in the reference section. The only exception are manuscripts that are
% not yet published (e.g., under submission). If you choose to refer to
% such unpublished manuscripts \cite{anonymous}, anonymized copies have
% to be submitted
% as Supplementary Material via OpenReview\@. However, keep in mind that an ICML
% paper should be self contained and should contain sufficient detail
% for the reviewers to evaluate the work. In particular, reviewers are
% not required to look at the Supplementary Material when writing their
% review (they are not required to look at more than the first $8$ pages of the submitted document).

% \subsubsection{Camera-Ready Author Information}
% \label{final author}

% If a paper is accepted, a final camera-ready copy must be prepared.
% %
% For camera-ready papers, author information should start 0.3~inches below the
% bottom rule surrounding the title. The authors' names should appear in 10~point
% bold type, in a row, separated by white space, and centered. Author names should
% not be broken across lines. Unbolded superscripted numbers, starting 1, should
% be used to refer to affiliations.

% Affiliations should be numbered in the order of appearance. A single footnote
% block of text should be used to list all the affiliations. (Academic
% affiliations should list Department, University, City, State/Region, Country.
% Similarly for industrial affiliations.)

% Each distinct affiliations should be listed once. If an author has multiple
% affiliations, multiple superscripts should be placed after the name, separated
% by thin spaces. If the authors would like to highlight equal contribution by
% multiple first authors, those authors should have an asterisk placed after their
% name in superscript, and the term ``\textsuperscript{*}Equal contribution"
% should be placed in the footnote block ahead of the list of affiliations. A
% list of corresponding authors and their emails (in the format Full Name
% \textless{}email@domain.com\textgreater{}) can follow the list of affiliations.
% Ideally only one or two names should be listed.

% A sample file with author names is included in the ICML2024 style file
% package. Turn on the \texttt{[accepted]} option to the stylefile to
% see the names rendered. All of the guidelines above are implemented
% by the \LaTeX\ style file.

% \subsection{Abstract}

% The paper abstract should begin in the left column, 0.4~inches below the final
% address. The heading `Abstract' should be centered, bold, and in 11~point type.
% The abstract body should use 10~point type, with a vertical spacing of
% 11~points, and should be indented 0.25~inches more than normal on left-hand and
% right-hand margins. Insert 0.4~inches of blank space after the body. Keep your
% abstract brief and self-contained, limiting it to one paragraph and roughly 4--6
% sentences. Gross violations will require correction at the camera-ready phase.

% \subsection{Partitioning the Text}

% You should organize your paper into sections and paragraphs to help
% readers place a structure on the material and understand its
% contributions.

% \subsubsection{Sections and Subsections}

% Section headings should be numbered, flush left, and set in 11~pt bold
% type with the content words capitalized. Leave 0.25~inches of space
% before the heading and 0.15~inches after the heading.

% Similarly, subsection headings should be numbered, flush left, and set
% in 10~pt bold type with the content words capitalized. Leave
% 0.2~inches of space before the heading and 0.13~inches afterward.

% Finally, subsubsection headings should be numbered, flush left, and
% set in 10~pt small caps with the content words capitalized. Leave
% 0.18~inches of space before the heading and 0.1~inches after the
% heading.

% Please use no more than three levels of headings.

% \subsubsection{Paragraphs and Footnotes}

% Within each section or subsection, you should further partition the
% paper into paragraphs. Do not indent the first line of a given
% paragraph, but insert a blank line between succeeding ones.

% You can use footnotes\footnote{Footnotes
% should be complete sentences.} to provide readers with additional
% information about a topic without interrupting the flow of the paper.
% Indicate footnotes with a number in the text where the point is most
% relevant. Place the footnote in 9~point type at the bottom of the
% column in which it appears. Precede the first footnote in a column
% with a horizontal rule of 0.8~inches.\footnote{Multiple footnotes can
% appear in each column, in the same order as they appear in the text,
% but spread them across columns and pages if possible.}


% \subsection{Figures}

% You may want to include figures in the paper to illustrate
% your approach and results. Such artwork should be centered,
% legible, and separated from the text. Lines should be dark and at
% least 0.5~points thick for purposes of reproduction, and text should
% not appear on a gray background.

% Label all distinct components of each figure. If the figure takes the
% form of a graph, then give a name for each axis and include a legend
% that briefly describes each curve. Do not include a title inside the
% figure; instead, the caption should serve this function.

% Number figures sequentially, placing the figure number and caption
% \emph{after} the graphics, with at least 0.1~inches of space before
% the caption and 0.1~inches after it, as in
% \cref{icml-historical}. The figure caption should be set in
% 9~point type and centered unless it runs two or more lines, in which
% case it should be flush left. You may float figures to the top or
% bottom of a column, and you may set wide figures across both columns
% (use the environment \texttt{figure*} in \LaTeX). Always place
% two-column figures at the top or bottom of the page.

% \subsection{Algorithms}

% If you are using \LaTeX, please use the ``algorithm'' and ``algorithmic''
% environments to format pseudocode. These require
% the corresponding stylefiles, algorithm.sty and
% algorithmic.sty, which are supplied with this package.
% \cref{alg:example} shows an example.

% \begin{algorithm}[tb]
%    \caption{Bubble Sort}
%    \label{alg:example}
% \begin{algorithmic}
%    \STATE {\bfseries Input:} data $x_i$, size $m$
%    \REPEAT
%    \STATE Initialize $noChange = true$.
%    \FOR{$i=1$ {\bfseries to} $m-1$}
%    \IF{$x_i > x_{i+1}$}
%    \STATE Swap $x_i$ and $x_{i+1}$
%    \STATE $noChange = false$
%    \ENDIF
%    \ENDFOR
%    \UNTIL{$noChange$ is $true$}
% \end{algorithmic}
% \end{algorithm}

% \subsection{Tables}

% You may also want to include tables that summarize material. Like
% figures, these should be centered, legible, and numbered consecutively.
% However, place the title \emph{above} the table with at least
% 0.1~inches of space before the title and the same after it, as in
% \cref{sample-table}. The table title should be set in 9~point
% type and centered unless it runs two or more lines, in which case it
% should be flush left.

% % Note use of \abovespace and \belowspace to get reasonable spacing
% % above and below tabular lines.

% \begin{table}[t]
% \caption{Classification accuracies for naive Bayes and flexible
% Bayes on various data sets.}
% \label{sample-table}
% \vskip 0.15in
% \begin{center}
% \begin{small}
% \begin{sc}
% \begin{tabular}{lcccr}
% \toprule
% Data set & Naive & Flexible & Better? \\
% \midrule
% Breast    & 95.9$\pm$ 0.2& 96.7$\pm$ 0.2& $\surd$ \\
% Cleveland & 83.3$\pm$ 0.6& 80.0$\pm$ 0.6& $\times$\\
% Glass2    & 61.9$\pm$ 1.4& 83.8$\pm$ 0.7& $\surd$ \\
% Credit    & 74.8$\pm$ 0.5& 78.3$\pm$ 0.6&         \\
% Horse     & 73.3$\pm$ 0.9& 69.7$\pm$ 1.0& $\times$\\
% Meta      & 67.1$\pm$ 0.6& 76.5$\pm$ 0.5& $\surd$ \\
% Pima      & 75.1$\pm$ 0.6& 73.9$\pm$ 0.5&         \\
% Vehicle   & 44.9$\pm$ 0.6& 61.5$\pm$ 0.4& $\surd$ \\
% \bottomrule
% \end{tabular}
% \end{sc}
% \end{small}
% \end{center}
% \vskip -0.1in
% \end{table}

% Tables contain textual material, whereas figures contain graphical material.
% Specify the contents of each row and column in the table's topmost
% row. Again, you may float tables to a column's top or bottom, and set
% wide tables across both columns. Place two-column tables at the
% top or bottom of the page.

% \subsection{Theorems and such}
% The preferred way is to number definitions, propositions, lemmas, etc. consecutively, within sections, as shown below.
% \begin{definition}
% \label{def:inj}
% A function $f:X \to Y$ is injective if for any $x,y\in X$ different, $f(x)\ne f(y)$.
% \end{definition}
% Using \cref{def:inj} we immediate get the following result:
% \begin{proposition}
% If $f$ is injective mapping a set $X$ to another set $Y$, 
% the cardinality of $Y$ is at least as large as that of $X$
% \end{proposition}
% \begin{proof} 
% Left as an exercise to the reader. 
% \end{proof}
% \cref{lem:usefullemma} stated next will prove to be useful.
% \begin{lemma}
% \label{lem:usefullemma}
% For any $f:X \to Y$ and $g:Y\to Z$ injective functions, $f \circ g$ is injective.
% \end{lemma}
% \begin{theorem}
% \label{thm:bigtheorem}
% If $f:X\to Y$ is bijective, the cardinality of $X$ and $Y$ are the same.
% \end{theorem}
% An easy corollary of \cref{thm:bigtheorem} is the following:
% \begin{corollary}
% If $f:X\to Y$ is bijective, 
% the cardinality of $X$ is at least as large as that of $Y$.
% \end{corollary}
% \begin{assumption}
% The set $X$ is finite.
% \label{ass:xfinite}
% \end{assumption}
% \begin{remark}
% According to some, it is only the finite case (cf. \cref{ass:xfinite}) that is interesting.
% \end{remark}
% %restatable

% \subsection{Citations and References}

% Please use APA reference format regardless of your formatter
% or word processor. If you rely on the \LaTeX\/ bibliographic
% facility, use \texttt{natbib.sty} and \texttt{icml2024.bst}
% included in the style-file package to obtain this format.

% Citations within the text should include the authors' last names and
% year. If the authors' names are included in the sentence, place only
% the year in parentheses, for example when referencing Arthur Samuel's
% pioneering work \yrcite{Samuel59}. Otherwise place the entire
% reference in parentheses with the authors and year separated by a
% comma \cite{Samuel59}. List multiple references separated by
% semicolons \cite{kearns89,Samuel59,mitchell80}. Use the `et~al.'
% construct only for citations with three or more authors or after
% listing all authors to a publication in an earlier reference \cite{MachineLearningI}.

% Authors should cite their own work in the third person
% in the initial version of their paper submitted for blind review.
% Please refer to \cref{author info} for detailed instructions on how to
% cite your own papers.

% Use an unnumbered first-level section heading for the references, and use a
% hanging indent style, with the first line of the reference flush against the
% left margin and subsequent lines indented by 10 points. The references at the
% end of this document give examples for journal articles \cite{Samuel59},
% conference publications \cite{langley00}, book chapters \cite{Newell81}, books
% \cite{DudaHart2nd}, edited volumes \cite{MachineLearningI}, technical reports
% \cite{mitchell80}, and dissertations \cite{kearns89}.

% Alphabetize references by the surnames of the first authors, with
% single author entries preceding multiple author entries. Order
% references for the same authors by year of publication, with the
% earliest first. Make sure that each reference includes all relevant
% information (e.g., page numbers).

% Please put some effort into making references complete, presentable, and
% consistent, e.g. use the actual current name of authors.
% If using bibtex, please protect capital letters of names and
% abbreviations in titles, for example, use \{B\}ayesian or \{L\}ipschitz
% in your .bib file.

% \section*{Accessibility}
% Authors are kindly asked to make their submissions as accessible as possible for everyone including people with disabilities and sensory or neurological differences.
% Tips of how to achieve this and what to pay attention to will be provided on the conference website \url{http://icml.cc/}.

% \section*{Software and Data}

% If a paper is accepted, we strongly encourage the publication of software and data with the
% camera-ready version of the paper whenever appropriate. This can be
% done by including a URL in the camera-ready copy. However, \textbf{do not}
% include URLs that reveal your institution or identity in your
% submission for review. Instead, provide an anonymous URL or upload
% the material as ``Supplementary Material'' into the OpenReview reviewing
% system. Note that reviewers are not required to look at this material
% when writing their review.

% % Acknowledgements should only appear in the accepted version.
% \section*{Acknowledgements}

% \textbf{Do not} include acknowledgements in the initial version of
% the paper submitted for blind review.

% If a paper is accepted, the final camera-ready version can (and
% usually should) include acknowledgements.  Such acknowledgements
% should be placed at the end of the section, in an unnumbered section
% that does not count towards the paper page limit. Typically, this will 
% include thanks to reviewers who gave useful comments, to colleagues 
% who contributed to the ideas, and to funding agencies and corporate 
% sponsors that provided financial support.

% \section*{Impact Statement}

% Authors are \textbf{required} to include a statement of the potential 
% broader impact of their work, including its ethical aspects and future 
% societal consequences. This statement should be in an unnumbered 
% section at the end of the paper (co-located with Acknowledgements -- 
% the two may appear in either order, but both must be before References), 
% and does not count toward the paper page limit. In many cases, where 
% the ethical impacts and expected societal implications are those that 
% are well established when advancing the field of Machine Learning, 
% substantial discussion is not required, and a simple statement such 
% as the following will suffice:

% ``This paper presents work whose goal is to advance the field of 
% Machine Learning. There are many potential societal consequences 
% of our work, none which we feel must be specifically highlighted here.''

% The above statement can be used verbatim in such cases, but we 
% encourage authors to think about whether there is content which does 
% warrant further discussion, as this statement will be apparent if the 
% paper is later flagged for ethics review.


% % In the unusual situation where you want a paper to appear in the
% % references without citing it in the main text, use \nocite
% \nocite{langley00}
%\newpage
% \section*{Impact Statement}
% This paper presents work whose goal is to advance the field of Machine Learning. There are many potential societal consequences of our work, none of which we feel must be specifically highlighted here.
\bibliography{arxiv}
\bibliographystyle{icml2024}


%%%%%%%%%%%%%%%%%%%%%%%%%%%%%%%%%%%%%%%%%%%%%%%%%%%%%%%%%%%%%%%%%%%%%%%%%%%%%%%
%%%%%%%%%%%%%%%%%%%%%%%%%%%%%%%%%%%%%%%%%%%%%%%%%%%%%%%%%%%%%%%%%%%%%%%%%%%%%%%
% APPENDIX
%%%%%%%%%%%%%%%%%%%%%%%%%%%%%%%%%%%%%%%%%%%%%%%%%%%%%%%%%%%%%%%%%%%%%%%%%%%%%%%
%%%%%%%%%%%%%%%%%%%%%%%%%%%%%%%%%%%%%%%%%%%%%%%%%%%%%%%%%%%%%%%%%%%%%%%%%%%%%%%
% \newpage
% \subsection{Lloyd-Max Algorithm}
\label{subsec:Lloyd-Max}
For a given quantization bitwidth $B$ and an operand $\bm{X}$, the Lloyd-Max algorithm finds $2^B$ quantization levels $\{\hat{x}_i\}_{i=1}^{2^B}$ such that quantizing $\bm{X}$ by rounding each scalar in $\bm{X}$ to the nearest quantization level minimizes the quantization MSE. 

The algorithm starts with an initial guess of quantization levels and then iteratively computes quantization thresholds $\{\tau_i\}_{i=1}^{2^B-1}$ and updates quantization levels $\{\hat{x}_i\}_{i=1}^{2^B}$. Specifically, at iteration $n$, thresholds are set to the midpoints of the previous iteration's levels:
\begin{align*}
    \tau_i^{(n)}=\frac{\hat{x}_i^{(n-1)}+\hat{x}_{i+1}^{(n-1)}}2 \text{ for } i=1\ldots 2^B-1
\end{align*}
Subsequently, the quantization levels are re-computed as conditional means of the data regions defined by the new thresholds:
\begin{align*}
    \hat{x}_i^{(n)}=\mathbb{E}\left[ \bm{X} \big| \bm{X}\in [\tau_{i-1}^{(n)},\tau_i^{(n)}] \right] \text{ for } i=1\ldots 2^B
\end{align*}
where to satisfy boundary conditions we have $\tau_0=-\infty$ and $\tau_{2^B}=\infty$. The algorithm iterates the above steps until convergence.

Figure \ref{fig:lm_quant} compares the quantization levels of a $7$-bit floating point (E3M3) quantizer (left) to a $7$-bit Lloyd-Max quantizer (right) when quantizing a layer of weights from the GPT3-126M model at a per-tensor granularity. As shown, the Lloyd-Max quantizer achieves substantially lower quantization MSE. Further, Table \ref{tab:FP7_vs_LM7} shows the superior perplexity achieved by Lloyd-Max quantizers for bitwidths of $7$, $6$ and $5$. The difference between the quantizers is clear at 5 bits, where per-tensor FP quantization incurs a drastic and unacceptable increase in perplexity, while Lloyd-Max quantization incurs a much smaller increase. Nevertheless, we note that even the optimal Lloyd-Max quantizer incurs a notable ($\sim 1.5$) increase in perplexity due to the coarse granularity of quantization. 

\begin{figure}[h]
  \centering
  \includegraphics[width=0.7\linewidth]{sections/figures/LM7_FP7.pdf}
  \caption{\small Quantization levels and the corresponding quantization MSE of Floating Point (left) vs Lloyd-Max (right) Quantizers for a layer of weights in the GPT3-126M model.}
  \label{fig:lm_quant}
\end{figure}

\begin{table}[h]\scriptsize
\begin{center}
\caption{\label{tab:FP7_vs_LM7} \small Comparing perplexity (lower is better) achieved by floating point quantizers and Lloyd-Max quantizers on a GPT3-126M model for the Wikitext-103 dataset.}
\begin{tabular}{c|cc|c}
\hline
 \multirow{2}{*}{\textbf{Bitwidth}} & \multicolumn{2}{|c|}{\textbf{Floating-Point Quantizer}} & \textbf{Lloyd-Max Quantizer} \\
 & Best Format & Wikitext-103 Perplexity & Wikitext-103 Perplexity \\
\hline
7 & E3M3 & 18.32 & 18.27 \\
6 & E3M2 & 19.07 & 18.51 \\
5 & E4M0 & 43.89 & 19.71 \\
\hline
\end{tabular}
\end{center}
\end{table}

\subsection{Proof of Local Optimality of LO-BCQ}
\label{subsec:lobcq_opt_proof}
For a given block $\bm{b}_j$, the quantization MSE during LO-BCQ can be empirically evaluated as $\frac{1}{L_b}\lVert \bm{b}_j- \bm{\hat{b}}_j\rVert^2_2$ where $\bm{\hat{b}}_j$ is computed from equation (\ref{eq:clustered_quantization_definition}) as $C_{f(\bm{b}_j)}(\bm{b}_j)$. Further, for a given block cluster $\mathcal{B}_i$, we compute the quantization MSE as $\frac{1}{|\mathcal{B}_{i}|}\sum_{\bm{b} \in \mathcal{B}_{i}} \frac{1}{L_b}\lVert \bm{b}- C_i^{(n)}(\bm{b})\rVert^2_2$. Therefore, at the end of iteration $n$, we evaluate the overall quantization MSE $J^{(n)}$ for a given operand $\bm{X}$ composed of $N_c$ block clusters as:
\begin{align*}
    \label{eq:mse_iter_n}
    J^{(n)} = \frac{1}{N_c} \sum_{i=1}^{N_c} \frac{1}{|\mathcal{B}_{i}^{(n)}|}\sum_{\bm{v} \in \mathcal{B}_{i}^{(n)}} \frac{1}{L_b}\lVert \bm{b}- B_i^{(n)}(\bm{b})\rVert^2_2
\end{align*}

At the end of iteration $n$, the codebooks are updated from $\mathcal{C}^{(n-1)}$ to $\mathcal{C}^{(n)}$. However, the mapping of a given vector $\bm{b}_j$ to quantizers $\mathcal{C}^{(n)}$ remains as  $f^{(n)}(\bm{b}_j)$. At the next iteration, during the vector clustering step, $f^{(n+1)}(\bm{b}_j)$ finds new mapping of $\bm{b}_j$ to updated codebooks $\mathcal{C}^{(n)}$ such that the quantization MSE over the candidate codebooks is minimized. Therefore, we obtain the following result for $\bm{b}_j$:
\begin{align*}
\frac{1}{L_b}\lVert \bm{b}_j - C_{f^{(n+1)}(\bm{b}_j)}^{(n)}(\bm{b}_j)\rVert^2_2 \le \frac{1}{L_b}\lVert \bm{b}_j - C_{f^{(n)}(\bm{b}_j)}^{(n)}(\bm{b}_j)\rVert^2_2
\end{align*}

That is, quantizing $\bm{b}_j$ at the end of the block clustering step of iteration $n+1$ results in lower quantization MSE compared to quantizing at the end of iteration $n$. Since this is true for all $\bm{b} \in \bm{X}$, we assert the following:
\begin{equation}
\begin{split}
\label{eq:mse_ineq_1}
    \tilde{J}^{(n+1)} &= \frac{1}{N_c} \sum_{i=1}^{N_c} \frac{1}{|\mathcal{B}_{i}^{(n+1)}|}\sum_{\bm{b} \in \mathcal{B}_{i}^{(n+1)}} \frac{1}{L_b}\lVert \bm{b} - C_i^{(n)}(b)\rVert^2_2 \le J^{(n)}
\end{split}
\end{equation}
where $\tilde{J}^{(n+1)}$ is the the quantization MSE after the vector clustering step at iteration $n+1$.

Next, during the codebook update step (\ref{eq:quantizers_update}) at iteration $n+1$, the per-cluster codebooks $\mathcal{C}^{(n)}$ are updated to $\mathcal{C}^{(n+1)}$ by invoking the Lloyd-Max algorithm \citep{Lloyd}. We know that for any given value distribution, the Lloyd-Max algorithm minimizes the quantization MSE. Therefore, for a given vector cluster $\mathcal{B}_i$ we obtain the following result:

\begin{equation}
    \frac{1}{|\mathcal{B}_{i}^{(n+1)}|}\sum_{\bm{b} \in \mathcal{B}_{i}^{(n+1)}} \frac{1}{L_b}\lVert \bm{b}- C_i^{(n+1)}(\bm{b})\rVert^2_2 \le \frac{1}{|\mathcal{B}_{i}^{(n+1)}|}\sum_{\bm{b} \in \mathcal{B}_{i}^{(n+1)}} \frac{1}{L_b}\lVert \bm{b}- C_i^{(n)}(\bm{b})\rVert^2_2
\end{equation}

The above equation states that quantizing the given block cluster $\mathcal{B}_i$ after updating the associated codebook from $C_i^{(n)}$ to $C_i^{(n+1)}$ results in lower quantization MSE. Since this is true for all the block clusters, we derive the following result: 
\begin{equation}
\begin{split}
\label{eq:mse_ineq_2}
     J^{(n+1)} &= \frac{1}{N_c} \sum_{i=1}^{N_c} \frac{1}{|\mathcal{B}_{i}^{(n+1)}|}\sum_{\bm{b} \in \mathcal{B}_{i}^{(n+1)}} \frac{1}{L_b}\lVert \bm{b}- C_i^{(n+1)}(\bm{b})\rVert^2_2  \le \tilde{J}^{(n+1)}   
\end{split}
\end{equation}

Following (\ref{eq:mse_ineq_1}) and (\ref{eq:mse_ineq_2}), we find that the quantization MSE is non-increasing for each iteration, that is, $J^{(1)} \ge J^{(2)} \ge J^{(3)} \ge \ldots \ge J^{(M)}$ where $M$ is the maximum number of iterations. 
%Therefore, we can say that if the algorithm converges, then it must be that it has converged to a local minimum. 
\hfill $\blacksquare$


\begin{figure}
    \begin{center}
    \includegraphics[width=0.5\textwidth]{sections//figures/mse_vs_iter.pdf}
    \end{center}
    \caption{\small NMSE vs iterations during LO-BCQ compared to other block quantization proposals}
    \label{fig:nmse_vs_iter}
\end{figure}

Figure \ref{fig:nmse_vs_iter} shows the empirical convergence of LO-BCQ across several block lengths and number of codebooks. Also, the MSE achieved by LO-BCQ is compared to baselines such as MXFP and VSQ. As shown, LO-BCQ converges to a lower MSE than the baselines. Further, we achieve better convergence for larger number of codebooks ($N_c$) and for a smaller block length ($L_b$), both of which increase the bitwidth of BCQ (see Eq \ref{eq:bitwidth_bcq}).


\subsection{Additional Accuracy Results}
%Table \ref{tab:lobcq_config} lists the various LOBCQ configurations and their corresponding bitwidths.
\begin{table}
\setlength{\tabcolsep}{4.75pt}
\begin{center}
\caption{\label{tab:lobcq_config} Various LO-BCQ configurations and their bitwidths.}
\begin{tabular}{|c||c|c|c|c||c|c||c|} 
\hline
 & \multicolumn{4}{|c||}{$L_b=8$} & \multicolumn{2}{|c||}{$L_b=4$} & $L_b=2$ \\
 \hline
 \backslashbox{$L_A$\kern-1em}{\kern-1em$N_c$} & 2 & 4 & 8 & 16 & 2 & 4 & 2 \\
 \hline
 64 & 4.25 & 4.375 & 4.5 & 4.625 & 4.375 & 4.625 & 4.625\\
 \hline
 32 & 4.375 & 4.5 & 4.625& 4.75 & 4.5 & 4.75 & 4.75 \\
 \hline
 16 & 4.625 & 4.75& 4.875 & 5 & 4.75 & 5 & 5 \\
 \hline
\end{tabular}
\end{center}
\end{table}

%\subsection{Perplexity achieved by various LO-BCQ configurations on Wikitext-103 dataset}

\begin{table} \centering
\begin{tabular}{|c||c|c|c|c||c|c||c|} 
\hline
 $L_b \rightarrow$& \multicolumn{4}{c||}{8} & \multicolumn{2}{c||}{4} & 2\\
 \hline
 \backslashbox{$L_A$\kern-1em}{\kern-1em$N_c$} & 2 & 4 & 8 & 16 & 2 & 4 & 2  \\
 %$N_c \rightarrow$ & 2 & 4 & 8 & 16 & 2 & 4 & 2 \\
 \hline
 \hline
 \multicolumn{8}{c}{GPT3-1.3B (FP32 PPL = 9.98)} \\ 
 \hline
 \hline
 64 & 10.40 & 10.23 & 10.17 & 10.15 &  10.28 & 10.18 & 10.19 \\
 \hline
 32 & 10.25 & 10.20 & 10.15 & 10.12 &  10.23 & 10.17 & 10.17 \\
 \hline
 16 & 10.22 & 10.16 & 10.10 & 10.09 &  10.21 & 10.14 & 10.16 \\
 \hline
  \hline
 \multicolumn{8}{c}{GPT3-8B (FP32 PPL = 7.38)} \\ 
 \hline
 \hline
 64 & 7.61 & 7.52 & 7.48 &  7.47 &  7.55 &  7.49 & 7.50 \\
 \hline
 32 & 7.52 & 7.50 & 7.46 &  7.45 &  7.52 &  7.48 & 7.48  \\
 \hline
 16 & 7.51 & 7.48 & 7.44 &  7.44 &  7.51 &  7.49 & 7.47  \\
 \hline
\end{tabular}
\caption{\label{tab:ppl_gpt3_abalation} Wikitext-103 perplexity across GPT3-1.3B and 8B models.}
\end{table}

\begin{table} \centering
\begin{tabular}{|c||c|c|c|c||} 
\hline
 $L_b \rightarrow$& \multicolumn{4}{c||}{8}\\
 \hline
 \backslashbox{$L_A$\kern-1em}{\kern-1em$N_c$} & 2 & 4 & 8 & 16 \\
 %$N_c \rightarrow$ & 2 & 4 & 8 & 16 & 2 & 4 & 2 \\
 \hline
 \hline
 \multicolumn{5}{|c|}{Llama2-7B (FP32 PPL = 5.06)} \\ 
 \hline
 \hline
 64 & 5.31 & 5.26 & 5.19 & 5.18  \\
 \hline
 32 & 5.23 & 5.25 & 5.18 & 5.15  \\
 \hline
 16 & 5.23 & 5.19 & 5.16 & 5.14  \\
 \hline
 \multicolumn{5}{|c|}{Nemotron4-15B (FP32 PPL = 5.87)} \\ 
 \hline
 \hline
 64  & 6.3 & 6.20 & 6.13 & 6.08  \\
 \hline
 32  & 6.24 & 6.12 & 6.07 & 6.03  \\
 \hline
 16  & 6.12 & 6.14 & 6.04 & 6.02  \\
 \hline
 \multicolumn{5}{|c|}{Nemotron4-340B (FP32 PPL = 3.48)} \\ 
 \hline
 \hline
 64 & 3.67 & 3.62 & 3.60 & 3.59 \\
 \hline
 32 & 3.63 & 3.61 & 3.59 & 3.56 \\
 \hline
 16 & 3.61 & 3.58 & 3.57 & 3.55 \\
 \hline
\end{tabular}
\caption{\label{tab:ppl_llama7B_nemo15B} Wikitext-103 perplexity compared to FP32 baseline in Llama2-7B and Nemotron4-15B, 340B models}
\end{table}

%\subsection{Perplexity achieved by various LO-BCQ configurations on MMLU dataset}


\begin{table} \centering
\begin{tabular}{|c||c|c|c|c||c|c|c|c|} 
\hline
 $L_b \rightarrow$& \multicolumn{4}{c||}{8} & \multicolumn{4}{c||}{8}\\
 \hline
 \backslashbox{$L_A$\kern-1em}{\kern-1em$N_c$} & 2 & 4 & 8 & 16 & 2 & 4 & 8 & 16  \\
 %$N_c \rightarrow$ & 2 & 4 & 8 & 16 & 2 & 4 & 2 \\
 \hline
 \hline
 \multicolumn{5}{|c|}{Llama2-7B (FP32 Accuracy = 45.8\%)} & \multicolumn{4}{|c|}{Llama2-70B (FP32 Accuracy = 69.12\%)} \\ 
 \hline
 \hline
 64 & 43.9 & 43.4 & 43.9 & 44.9 & 68.07 & 68.27 & 68.17 & 68.75 \\
 \hline
 32 & 44.5 & 43.8 & 44.9 & 44.5 & 68.37 & 68.51 & 68.35 & 68.27  \\
 \hline
 16 & 43.9 & 42.7 & 44.9 & 45 & 68.12 & 68.77 & 68.31 & 68.59  \\
 \hline
 \hline
 \multicolumn{5}{|c|}{GPT3-22B (FP32 Accuracy = 38.75\%)} & \multicolumn{4}{|c|}{Nemotron4-15B (FP32 Accuracy = 64.3\%)} \\ 
 \hline
 \hline
 64 & 36.71 & 38.85 & 38.13 & 38.92 & 63.17 & 62.36 & 63.72 & 64.09 \\
 \hline
 32 & 37.95 & 38.69 & 39.45 & 38.34 & 64.05 & 62.30 & 63.8 & 64.33  \\
 \hline
 16 & 38.88 & 38.80 & 38.31 & 38.92 & 63.22 & 63.51 & 63.93 & 64.43  \\
 \hline
\end{tabular}
\caption{\label{tab:mmlu_abalation} Accuracy on MMLU dataset across GPT3-22B, Llama2-7B, 70B and Nemotron4-15B models.}
\end{table}


%\subsection{Perplexity achieved by various LO-BCQ configurations on LM evaluation harness}

\begin{table} \centering
\begin{tabular}{|c||c|c|c|c||c|c|c|c|} 
\hline
 $L_b \rightarrow$& \multicolumn{4}{c||}{8} & \multicolumn{4}{c||}{8}\\
 \hline
 \backslashbox{$L_A$\kern-1em}{\kern-1em$N_c$} & 2 & 4 & 8 & 16 & 2 & 4 & 8 & 16  \\
 %$N_c \rightarrow$ & 2 & 4 & 8 & 16 & 2 & 4 & 2 \\
 \hline
 \hline
 \multicolumn{5}{|c|}{Race (FP32 Accuracy = 37.51\%)} & \multicolumn{4}{|c|}{Boolq (FP32 Accuracy = 64.62\%)} \\ 
 \hline
 \hline
 64 & 36.94 & 37.13 & 36.27 & 37.13 & 63.73 & 62.26 & 63.49 & 63.36 \\
 \hline
 32 & 37.03 & 36.36 & 36.08 & 37.03 & 62.54 & 63.51 & 63.49 & 63.55  \\
 \hline
 16 & 37.03 & 37.03 & 36.46 & 37.03 & 61.1 & 63.79 & 63.58 & 63.33  \\
 \hline
 \hline
 \multicolumn{5}{|c|}{Winogrande (FP32 Accuracy = 58.01\%)} & \multicolumn{4}{|c|}{Piqa (FP32 Accuracy = 74.21\%)} \\ 
 \hline
 \hline
 64 & 58.17 & 57.22 & 57.85 & 58.33 & 73.01 & 73.07 & 73.07 & 72.80 \\
 \hline
 32 & 59.12 & 58.09 & 57.85 & 58.41 & 73.01 & 73.94 & 72.74 & 73.18  \\
 \hline
 16 & 57.93 & 58.88 & 57.93 & 58.56 & 73.94 & 72.80 & 73.01 & 73.94  \\
 \hline
\end{tabular}
\caption{\label{tab:mmlu_abalation} Accuracy on LM evaluation harness tasks on GPT3-1.3B model.}
\end{table}

\begin{table} \centering
\begin{tabular}{|c||c|c|c|c||c|c|c|c|} 
\hline
 $L_b \rightarrow$& \multicolumn{4}{c||}{8} & \multicolumn{4}{c||}{8}\\
 \hline
 \backslashbox{$L_A$\kern-1em}{\kern-1em$N_c$} & 2 & 4 & 8 & 16 & 2 & 4 & 8 & 16  \\
 %$N_c \rightarrow$ & 2 & 4 & 8 & 16 & 2 & 4 & 2 \\
 \hline
 \hline
 \multicolumn{5}{|c|}{Race (FP32 Accuracy = 41.34\%)} & \multicolumn{4}{|c|}{Boolq (FP32 Accuracy = 68.32\%)} \\ 
 \hline
 \hline
 64 & 40.48 & 40.10 & 39.43 & 39.90 & 69.20 & 68.41 & 69.45 & 68.56 \\
 \hline
 32 & 39.52 & 39.52 & 40.77 & 39.62 & 68.32 & 67.43 & 68.17 & 69.30  \\
 \hline
 16 & 39.81 & 39.71 & 39.90 & 40.38 & 68.10 & 66.33 & 69.51 & 69.42  \\
 \hline
 \hline
 \multicolumn{5}{|c|}{Winogrande (FP32 Accuracy = 67.88\%)} & \multicolumn{4}{|c|}{Piqa (FP32 Accuracy = 78.78\%)} \\ 
 \hline
 \hline
 64 & 66.85 & 66.61 & 67.72 & 67.88 & 77.31 & 77.42 & 77.75 & 77.64 \\
 \hline
 32 & 67.25 & 67.72 & 67.72 & 67.00 & 77.31 & 77.04 & 77.80 & 77.37  \\
 \hline
 16 & 68.11 & 68.90 & 67.88 & 67.48 & 77.37 & 78.13 & 78.13 & 77.69  \\
 \hline
\end{tabular}
\caption{\label{tab:mmlu_abalation} Accuracy on LM evaluation harness tasks on GPT3-8B model.}
\end{table}

\begin{table} \centering
\begin{tabular}{|c||c|c|c|c||c|c|c|c|} 
\hline
 $L_b \rightarrow$& \multicolumn{4}{c||}{8} & \multicolumn{4}{c||}{8}\\
 \hline
 \backslashbox{$L_A$\kern-1em}{\kern-1em$N_c$} & 2 & 4 & 8 & 16 & 2 & 4 & 8 & 16  \\
 %$N_c \rightarrow$ & 2 & 4 & 8 & 16 & 2 & 4 & 2 \\
 \hline
 \hline
 \multicolumn{5}{|c|}{Race (FP32 Accuracy = 40.67\%)} & \multicolumn{4}{|c|}{Boolq (FP32 Accuracy = 76.54\%)} \\ 
 \hline
 \hline
 64 & 40.48 & 40.10 & 39.43 & 39.90 & 75.41 & 75.11 & 77.09 & 75.66 \\
 \hline
 32 & 39.52 & 39.52 & 40.77 & 39.62 & 76.02 & 76.02 & 75.96 & 75.35  \\
 \hline
 16 & 39.81 & 39.71 & 39.90 & 40.38 & 75.05 & 73.82 & 75.72 & 76.09  \\
 \hline
 \hline
 \multicolumn{5}{|c|}{Winogrande (FP32 Accuracy = 70.64\%)} & \multicolumn{4}{|c|}{Piqa (FP32 Accuracy = 79.16\%)} \\ 
 \hline
 \hline
 64 & 69.14 & 70.17 & 70.17 & 70.56 & 78.24 & 79.00 & 78.62 & 78.73 \\
 \hline
 32 & 70.96 & 69.69 & 71.27 & 69.30 & 78.56 & 79.49 & 79.16 & 78.89  \\
 \hline
 16 & 71.03 & 69.53 & 69.69 & 70.40 & 78.13 & 79.16 & 79.00 & 79.00  \\
 \hline
\end{tabular}
\caption{\label{tab:mmlu_abalation} Accuracy on LM evaluation harness tasks on GPT3-22B model.}
\end{table}

\begin{table} \centering
\begin{tabular}{|c||c|c|c|c||c|c|c|c|} 
\hline
 $L_b \rightarrow$& \multicolumn{4}{c||}{8} & \multicolumn{4}{c||}{8}\\
 \hline
 \backslashbox{$L_A$\kern-1em}{\kern-1em$N_c$} & 2 & 4 & 8 & 16 & 2 & 4 & 8 & 16  \\
 %$N_c \rightarrow$ & 2 & 4 & 8 & 16 & 2 & 4 & 2 \\
 \hline
 \hline
 \multicolumn{5}{|c|}{Race (FP32 Accuracy = 44.4\%)} & \multicolumn{4}{|c|}{Boolq (FP32 Accuracy = 79.29\%)} \\ 
 \hline
 \hline
 64 & 42.49 & 42.51 & 42.58 & 43.45 & 77.58 & 77.37 & 77.43 & 78.1 \\
 \hline
 32 & 43.35 & 42.49 & 43.64 & 43.73 & 77.86 & 75.32 & 77.28 & 77.86  \\
 \hline
 16 & 44.21 & 44.21 & 43.64 & 42.97 & 78.65 & 77 & 76.94 & 77.98  \\
 \hline
 \hline
 \multicolumn{5}{|c|}{Winogrande (FP32 Accuracy = 69.38\%)} & \multicolumn{4}{|c|}{Piqa (FP32 Accuracy = 78.07\%)} \\ 
 \hline
 \hline
 64 & 68.9 & 68.43 & 69.77 & 68.19 & 77.09 & 76.82 & 77.09 & 77.86 \\
 \hline
 32 & 69.38 & 68.51 & 68.82 & 68.90 & 78.07 & 76.71 & 78.07 & 77.86  \\
 \hline
 16 & 69.53 & 67.09 & 69.38 & 68.90 & 77.37 & 77.8 & 77.91 & 77.69  \\
 \hline
\end{tabular}
\caption{\label{tab:mmlu_abalation} Accuracy on LM evaluation harness tasks on Llama2-7B model.}
\end{table}

\begin{table} \centering
\begin{tabular}{|c||c|c|c|c||c|c|c|c|} 
\hline
 $L_b \rightarrow$& \multicolumn{4}{c||}{8} & \multicolumn{4}{c||}{8}\\
 \hline
 \backslashbox{$L_A$\kern-1em}{\kern-1em$N_c$} & 2 & 4 & 8 & 16 & 2 & 4 & 8 & 16  \\
 %$N_c \rightarrow$ & 2 & 4 & 8 & 16 & 2 & 4 & 2 \\
 \hline
 \hline
 \multicolumn{5}{|c|}{Race (FP32 Accuracy = 48.8\%)} & \multicolumn{4}{|c|}{Boolq (FP32 Accuracy = 85.23\%)} \\ 
 \hline
 \hline
 64 & 49.00 & 49.00 & 49.28 & 48.71 & 82.82 & 84.28 & 84.03 & 84.25 \\
 \hline
 32 & 49.57 & 48.52 & 48.33 & 49.28 & 83.85 & 84.46 & 84.31 & 84.93  \\
 \hline
 16 & 49.85 & 49.09 & 49.28 & 48.99 & 85.11 & 84.46 & 84.61 & 83.94  \\
 \hline
 \hline
 \multicolumn{5}{|c|}{Winogrande (FP32 Accuracy = 79.95\%)} & \multicolumn{4}{|c|}{Piqa (FP32 Accuracy = 81.56\%)} \\ 
 \hline
 \hline
 64 & 78.77 & 78.45 & 78.37 & 79.16 & 81.45 & 80.69 & 81.45 & 81.5 \\
 \hline
 32 & 78.45 & 79.01 & 78.69 & 80.66 & 81.56 & 80.58 & 81.18 & 81.34  \\
 \hline
 16 & 79.95 & 79.56 & 79.79 & 79.72 & 81.28 & 81.66 & 81.28 & 80.96  \\
 \hline
\end{tabular}
\caption{\label{tab:mmlu_abalation} Accuracy on LM evaluation harness tasks on Llama2-70B model.}
\end{table}

%\section{MSE Studies}
%\textcolor{red}{TODO}


\subsection{Number Formats and Quantization Method}
\label{subsec:numFormats_quantMethod}
\subsubsection{Integer Format}
An $n$-bit signed integer (INT) is typically represented with a 2s-complement format \citep{yao2022zeroquant,xiao2023smoothquant,dai2021vsq}, where the most significant bit denotes the sign.

\subsubsection{Floating Point Format}
An $n$-bit signed floating point (FP) number $x$ comprises of a 1-bit sign ($x_{\mathrm{sign}}$), $B_m$-bit mantissa ($x_{\mathrm{mant}}$) and $B_e$-bit exponent ($x_{\mathrm{exp}}$) such that $B_m+B_e=n-1$. The associated constant exponent bias ($E_{\mathrm{bias}}$) is computed as $(2^{{B_e}-1}-1)$. We denote this format as $E_{B_e}M_{B_m}$.  

\subsubsection{Quantization Scheme}
\label{subsec:quant_method}
A quantization scheme dictates how a given unquantized tensor is converted to its quantized representation. We consider FP formats for the purpose of illustration. Given an unquantized tensor $\bm{X}$ and an FP format $E_{B_e}M_{B_m}$, we first, we compute the quantization scale factor $s_X$ that maps the maximum absolute value of $\bm{X}$ to the maximum quantization level of the $E_{B_e}M_{B_m}$ format as follows:
\begin{align}
\label{eq:sf}
    s_X = \frac{\mathrm{max}(|\bm{X}|)}{\mathrm{max}(E_{B_e}M_{B_m})}
\end{align}
In the above equation, $|\cdot|$ denotes the absolute value function.

Next, we scale $\bm{X}$ by $s_X$ and quantize it to $\hat{\bm{X}}$ by rounding it to the nearest quantization level of $E_{B_e}M_{B_m}$ as:

\begin{align}
\label{eq:tensor_quant}
    \hat{\bm{X}} = \text{round-to-nearest}\left(\frac{\bm{X}}{s_X}, E_{B_e}M_{B_m}\right)
\end{align}

We perform dynamic max-scaled quantization \citep{wu2020integer}, where the scale factor $s$ for activations is dynamically computed during runtime.

\subsection{Vector Scaled Quantization}
\begin{wrapfigure}{r}{0.35\linewidth}
  \centering
  \includegraphics[width=\linewidth]{sections/figures/vsquant.jpg}
  \caption{\small Vectorwise decomposition for per-vector scaled quantization (VSQ \citep{dai2021vsq}).}
  \label{fig:vsquant}
\end{wrapfigure}
During VSQ \citep{dai2021vsq}, the operand tensors are decomposed into 1D vectors in a hardware friendly manner as shown in Figure \ref{fig:vsquant}. Since the decomposed tensors are used as operands in matrix multiplications during inference, it is beneficial to perform this decomposition along the reduction dimension of the multiplication. The vectorwise quantization is performed similar to tensorwise quantization described in Equations \ref{eq:sf} and \ref{eq:tensor_quant}, where a scale factor $s_v$ is required for each vector $\bm{v}$ that maps the maximum absolute value of that vector to the maximum quantization level. While smaller vector lengths can lead to larger accuracy gains, the associated memory and computational overheads due to the per-vector scale factors increases. To alleviate these overheads, VSQ \citep{dai2021vsq} proposed a second level quantization of the per-vector scale factors to unsigned integers, while MX \citep{rouhani2023shared} quantizes them to integer powers of 2 (denoted as $2^{INT}$).

\subsubsection{MX Format}
The MX format proposed in \citep{rouhani2023microscaling} introduces the concept of sub-block shifting. For every two scalar elements of $b$-bits each, there is a shared exponent bit. The value of this exponent bit is determined through an empirical analysis that targets minimizing quantization MSE. We note that the FP format $E_{1}M_{b}$ is strictly better than MX from an accuracy perspective since it allocates a dedicated exponent bit to each scalar as opposed to sharing it across two scalars. Therefore, we conservatively bound the accuracy of a $b+2$-bit signed MX format with that of a $E_{1}M_{b}$ format in our comparisons. For instance, we use E1M2 format as a proxy for MX4.

\begin{figure}
    \centering
    \includegraphics[width=1\linewidth]{sections//figures/BlockFormats.pdf}
    \caption{\small Comparing LO-BCQ to MX format.}
    \label{fig:block_formats}
\end{figure}

Figure \ref{fig:block_formats} compares our $4$-bit LO-BCQ block format to MX \citep{rouhani2023microscaling}. As shown, both LO-BCQ and MX decompose a given operand tensor into block arrays and each block array into blocks. Similar to MX, we find that per-block quantization ($L_b < L_A$) leads to better accuracy due to increased flexibility. While MX achieves this through per-block $1$-bit micro-scales, we associate a dedicated codebook to each block through a per-block codebook selector. Further, MX quantizes the per-block array scale-factor to E8M0 format without per-tensor scaling. In contrast during LO-BCQ, we find that per-tensor scaling combined with quantization of per-block array scale-factor to E4M3 format results in superior inference accuracy across models. 

%%%%%%%%%%%%%%%%%%%%%%%%%%%%%%%%%%%%%%%%%%%%%%%%%%%%%%%%%%%%%%%%%%%%%%%%%%%%%%%
%%%%%%%%%%%%%%%%%%%%%%%%%%%%%%%%%%%%%%%%%%%%%%%%%%%%%%%%%%%%%%%%%%%%%%%%%%%%%%%


\end{document}


% This document was modified from the file originally made available by
% Pat Langley and Andrea Danyluk for ICML-2K. This version was created
% by Iain Murray in 2018, and modified by Alexandre Bouchard in
% 2019 and 2021 and by Csaba Szepesvari, Gang Niu and Sivan Sabato in 2022.
% Modified again in 2023 and 2024 by Sivan Sabato and Jonathan Scarlett.
% Previous contributors include Dan Roy, Lise Getoor and Tobias
% Scheffer, which was slightly modified from the 2010 version by
% Thorsten Joachims & Johannes Fuernkranz, slightly modified from the
% 2009 version by Kiri Wagstaff and Sam Roweis's 2008 version, which is
% slightly modified from Prasad Tadepalli's 2007 version which is a
% lightly changed version of the previous year's version by Andrew
% Moore, which was in turn edited from those of Kristian Kersting and
% Codrina Lauth. Alex Smola contributed to the algorithmic style files.
