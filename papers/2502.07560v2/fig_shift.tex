\begin{figure}[!htb]
\centering
\subfigure[Semantic Drift]{
        \centering
        \includegraphics[trim=0.5cm 2cm 3cm 0.1cm, clip, width=0.46\linewidth]{images/original_shift.pdf}
        \label{fig:ori_shift}
        }
\subfigure[Calibration]{
        \centering
        \includegraphics[trim=0.1cm 1cm 3cm 1cm, clip, width=0.46\linewidth]{images/shift_cali.pdf}
        \label{fig:shift_cali}
        }
% \begin{subfigure}[b]{0.48\linewidth}
%         \centering
%         \includegraphics[width=\linewidth]{images/calibration_result.pdf}
%         \label{fig:shift_cali}
%         \caption{Calibration}
% \end{subfigure}%
% \includegraphics[width=0.23\textwidth]{images/original_vs_shift.pdf}
% \includegraphics[width=0.23\textwidth]{images/calibration_result.pdf}
\vspace{-5mm}
\caption{\small As new tasks are learned, the categories from previously tasks in the latest updated model continuously experience shifts in their means and variances, referred to as (a) \textbf{Semantic Drift}. In this paper, we calibrate such semantic drift by applying explicit mean shift compensation and implicit variance constraints (b).}
\vspace{-5mm}
\label{fig:teaser}
\end{figure}