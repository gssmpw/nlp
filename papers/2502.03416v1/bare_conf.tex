%% bare_conf.tex
%% V1.4b
%% 2015/08/26
%% by Michael Shell
%% See:
%% http://www.michaelshell.org/
%% for current contact information.
%%
%% This is a skeleton file demonstrating the use of IEEEtran.cls
%% (requires IEEEtran.cls version 1.8b or later) with an IEEE
%% conference paper.
%%
%% Support sites:
%% http://www.michaelshell.org/tex/ieeetran/
%% http://www.ctan.org/pkg/ieeetran
%% and
%% http://www.ieee.org/

%%*************************************************************************
%% Legal Notice:
%% This code is offered as-is without any warranty either expressed or
%% implied; without even the implied warranty of MERCHANTABILITY or
%% FITNESS FOR A PARTICULAR PURPOSE! 
%% User assumes all risk.
%% In no event shall the IEEE or any contributor to this code be liable for
%% any damages or losses, including, but not limited to, incidental,
%% consequential, or any other damages, resulting from the use or misuse
%% of any information contained here.
%%
%% All comments are the opinions of their respective authors and are not
%% necessarily endorsed by the IEEE.
%%
%% This work is distributed under the LaTeX Project Public License (LPPL)
%% ( http://www.latex-project.org/ ) version 1.3, and may be freely used,
%% distributed and modified. A copy of the LPPL, version 1.3, is included
%% in the base LaTeX documentation of all distributions of LaTeX released
%% 2003/12/01 or later.
%% Retain all contribution notices and credits.
%% ** Modified files should be indicated as such, including  **
%% ** renaming them and changing author support contact information. **
%%*************************************************************************


% *** Authors should verify (and, if needed, correct) their LaTeX system  ***
% *** with the test flow diagnostic prior to trusting their LaTeX platform ***
% *** with production work. The IEEE's font choices and paper sizes can   ***
% *** trigger bugs that do not appear when using other class files.       ***                          ***
% The test flow support page is at:
% http://www.michaelshell.org/tex/testflow/



\documentclass[conference]{IEEEtran}

% Some Computer Society conferences also require the compsoc mode option,
% but others use the standard conference format.
%
% If IEEEtran.cls has not been installed into the LaTeX system files,
% manually specify the path to it like:
% \documentclass[conference]{../sty/IEEEtran}





% Some very useful LaTeX packages include:
% (uncomment the ones you want to load)


% *** MISC UTILITY PACKAGES ***
%
%\usepackage{ifpdf}
% Heiko Oberdiek's ifpdf.sty is very useful if you need conditional
% compilation based on whether the output is pdf or dvi.
% usage:
% \ifpdf
%   % pdf code
% \else
%   % dvi code
% \fi
% The latest version of ifpdf.sty can be obtained from:
% http://www.ctan.org/pkg/ifpdf
% Also, note that IEEEtran.cls V1.7 and later provides a builtin
% \ifCLASSINFOpdf conditional that works the same way.
% When switching from latex to pdflatex and vice-versa, the compiler may
% have to be run twice to clear warning/error messages.






% *** CITATION PACKAGES ***
%
%\usepackage{cite}
% cite.sty was written by Donald Arseneau
% V1.6 and later of IEEEtran pre-defines the format of the cite.sty package
% \cite{} output to follow that of the IEEE. Loading the cite package will
% result in citation numbers being automatically sorted and properly
% "compressed/ranged". e.g., [1], [9], [2], [7], [5], [6] without using
% cite.sty will become [1], [2], [5]--[7], [9] using cite.sty. cite.sty's
% \cite will automatically add leading space, if needed. Use cite.sty's
% noadjust option (cite.sty V3.8 and later) if you want to turn this off
% such as if a citation ever needs to be enclosed in parenthesis.
% cite.sty is already installed on most LaTeX systems. Be sure and use
% version 5.0 (2009-03-20) and later if using hyperref.sty.
% The latest version can be obtained at:
% http://www.ctan.org/pkg/cite
% The documentation is contained in the cite.sty file itself.






% *** GRAPHICS RELATED PACKAGES ***
%
\ifCLASSINFOpdf
  % \usepackage[pdftex]{graphicx}
  % declare the path(s) where your graphic files are
  % \graphicspath{{../pdf/}{../jpeg/}}
  % and their extensions so you won't have to specify these with
  % every instance of \includegraphics
  % \DeclareGraphicsExtensions{.pdf,.jpeg,.png}
\else
  % or other class option (dvipsone, dvipdf, if not using dvips). graphicx
  % will default to the driver specified in the system graphics.cfg if no
  % driver is specified.
  % \usepackage[dvips]{graphicx}
  % declare the path(s) where your graphic files are
  % \graphicspath{{../eps/}}
  % and their extensions so you won't have to specify these with
  % every instance of \includegraphics
  % \DeclareGraphicsExtensions{.eps}
\fi
% graphicx was written by David Carlisle and Sebastian Rahtz. It is
% required if you want graphics, photos, etc. graphicx.sty is already
% installed on most LaTeX systems. The latest version and documentation
% can be obtained at: 
% http://www.ctan.org/pkg/graphicx
% Another good source of documentation is "Using Imported Graphics in
% LaTeX2e" by Keith Reckdahl which can be found at:
% http://www.ctan.org/pkg/epslatex
%
% latex, and pdflatex in dvi mode, support graphics in encapsulated
% postscript (.eps) format. pdflatex in pdf mode supports graphics
% in .pdf, .jpeg, .png and .mps (metapost) formats. Users should ensure
% that all non-photo figures use a vector format (.eps, .pdf, .mps) and
% not a bitmapped formats (.jpeg, .png). The IEEE frowns on bitmapped formats
% which can result in "jaggedy"/blurry rendering of lines and letters as
% well as large increases in file sizes.
%
% You can find documentation about the pdfTeX application at:
% http://www.tug.org/applications/pdftex





% *** MATH PACKAGES ***
%
%\usepackage{amsmath}
% A popular package from the American Mathematical Society that provides
% many useful and powerful commands for dealing with mathematics.
%
% Note that the amsmath package sets \interdisplaylinepenalty to 10000
% thus preventing page breaks from occurring within multiline equations. Use:
%\interdisplaylinepenalty=2500
% after loading amsmath to restore such page breaks as IEEEtran.cls normally
% does. amsmath.sty is already installed on most LaTeX systems. The latest
% version and documentation can be obtained at:
% http://www.ctan.org/pkg/amsmath





% *** SPECIALIZED LIST PACKAGES ***
%
%\usepackage{algorithmic}
% algorithmic.sty was written by Peter Williams and Rogerio Brito.
% This package provides an algorithmic environment fo describing algorithms.
% You can use the algorithmic environment in-text or within a figure
% environment to provide for a floating algorithm. Do NOT use the algorithm
% floating environment provided by algorithm.sty (by the same authors) or
% algorithm2e.sty (by Christophe Fiorio) as the IEEE does not use dedicated
% algorithm float types and packages that provide these will not provide
% correct IEEE style captions. The latest version and documentation of
% algorithmic.sty can be obtained at:
% http://www.ctan.org/pkg/algorithms
% Also of interest may be the (relatively newer and more customizable)
% algorithmicx.sty package by Szasz Janos:
% http://www.ctan.org/pkg/algorithmicx




% *** ALIGNMENT PACKAGES ***
%
%\usepackage{array}
% Frank Mittelbach's and David Carlisle's array.sty patches and improves
% the standard LaTeX2e array and tabular environments to provide better
% appearance and additional user controls. As the default LaTeX2e table
% generation code is lacking to the point of almost being broken with
% respect to the quality of the end results, all users are strongly
% advised to use an enhanced (at the very least that provided by array.sty)
% set of table tools. array.sty is already installed on most systems. The
% latest version and documentation can be obtained at:
% http://www.ctan.org/pkg/array


% IEEEtran contains the IEEEeqnarray family of commands that can be used to
% generate multiline equations as well as matrices, tables, etc., of high
% quality.




% *** SUBFIGURE PACKAGES ***
%\ifCLASSOPTIONcompsoc
%  \usepackage[caption=false,font=normalsize,labelfont=sf,textfont=sf]{subfig}
%\else
%  \usepackage[caption=false,font=footnotesize]{subfig}
%\fi
% subfig.sty, written by Steven Douglas Cochran, is the modern replacement
% for subfigure.sty, the latter of which is no longer maintained and is
% incompatible with some LaTeX packages including fixltx2e. However,
% subfig.sty requires and automatically loads Axel Sommerfeldt's caption.sty
% which will override IEEEtran.cls' handling of captions and this will result
% in non-IEEE style figure/table captions. To prevent this problem, be sure
% and invoke subfig.sty's "caption=false" package option (available since
% subfig.sty version 1.3, 2005/06/28) as this is will preserve IEEEtran.cls
% handling of captions.
% Note that the Computer Society format requires a larger sans serif font
% than the serif footnote size font used in traditional IEEE formatting
% and thus the need to invoke different subfig.sty package options depending
% on whether compsoc mode has been enabled.
%
% The latest version and documentation of subfig.sty can be obtained at:
% http://www.ctan.org/pkg/subfig




% *** FLOAT PACKAGES ***
%
%\usepackage{fixltx2e}
% fixltx2e, the successor to the earlier fix2col.sty, was written by
% Frank Mittelbach and David Carlisle. This package corrects a few problems
% in the LaTeX2e kernel, the most notable of which is that in current
% LaTeX2e releases, the ordering of single and double column floats is not
% guaranteed to be preserved. Thus, an unpatched LaTeX2e can allow a
% single column figure to be placed prior to an earlier double column
% figure.
% Be aware that LaTeX2e kernels dated 2015 and later have fixltx2e.sty's
% corrections already built into the system in which case a warning will
% be issued if an attempt is made to load fixltx2e.sty as it is no longer
% needed.
% The latest version and documentation can be found at:
% http://www.ctan.org/pkg/fixltx2e


%\usepackage{stfloats}
% stfloats.sty was written by Sigitas Tolusis. This package gives LaTeX2e
% the ability to do double column floats at the bottom of the page as well
% as the top. (e.g., "\begin{figure*}[!b]" is not normally possible in
% LaTeX2e). It also provides a command:
%\fnbelowfloat
% to enable the placement of footnotes below bottom floats (the standard
% LaTeX2e kernel puts them above bottom floats). This is an invasive package
% which rewrites many portions of the LaTeX2e float routines. It may not work
% with other packages that modify the LaTeX2e float routines. The latest
% version and documentation can be obtained at:
% http://www.ctan.org/pkg/stfloats
% Do not use the stfloats baselinefloat ability as the IEEE does not allow
% \baselineskip to stretch. Authors submitting work to the IEEE should note
% that the IEEE rarely uses double column equations and that authors should try
% to avoid such use. Do not be tempted to use the cuted.sty or midfloat.sty
% packages (also by Sigitas Tolusis) as the IEEE does not format its papers in
% such ways.
% Do not attempt to use stfloats with fixltx2e as they are incompatible.
% Instead, use Morten Hogholm'a dblfloatfix which combines the features
% of both fixltx2e and stfloats:
%
% \usepackage{dblfloatfix}
% The latest version can be found at:
% http://www.ctan.org/pkg/dblfloatfix




% *** PDF, URL AND HYPERLINK PACKAGES ***
%
%\usepackage{url}
% url.sty was written by Donald Arseneau. It provides better support for
% handling and breaking URLs. url.sty is already installed on most LaTeX
% systems. The latest version and documentation can be obtained at:
% http://www.ctan.org/pkg/url
% Basically, \url{my_url_here}.




% *** Do not adjust lengths that control margins, column widths, etc. ***
% *** Do not use packages that alter fonts (such as pslatex).         ***
% There should be no need to do such things with IEEEtran.cls V1.6 and later.
% (Unless specifically asked to do so by the journal or conference you plan
% to submit to, of course. )


% correct bad hyphenation here
\hyphenation{op-tical net-works semi-conduc-tor}
\usepackage{graphicx}
\usepackage{tabularx,booktabs}
\usepackage{dingbat}
\usepackage{diagbox}
\usepackage{multirow} 
\usepackage[hyphens]{url}
\usepackage[hidelinks,breaklinks]{hyperref}
\usepackage{xcolor}
\usepackage{amsfonts, amsmath, amsthm, amssymb}
\usepackage{subcaption}
\hypersetup{breaklinks=true}
\urlstyle{same}
\usepackage{tikz}
\usepackage{textcomp}
\usepackage{lipsum}
\usepackage{setspace}
\usepackage{svg}

\IEEEoverridecommandlockouts
\newcommand\copyrighttext{%
  \footnotesize \textcopyright 2025 IEEE.  Personal use of this material is permitted.  Permission from IEEE must be obtained for all other uses, in any current or future media, including reprinting/republishing this material for advertising or promotional purposes, creating new collective works, for resale or redistribution to servers or lists, or reuse of any copyrighted component of this work in other works.}
\newcommand\copyrightnotice{%
\begin{tikzpicture}[remember picture,overlay]
\node[anchor=south,yshift=10pt] at (current page.south) {\fbox{\parbox{\dimexpr\textwidth-\fboxsep-\fboxrule\relax}{\copyrighttext}}};
\end{tikzpicture}%
}


\usepackage{graphicx}
\graphicspath{{figures/}}
\begin{document}

\bstctlcite{IEEEexample:BSTcontrol}
%
% paper title
% Titles are generally capitalized except for words such as a, an, and, as,
% at, but, by, for, in, nor, of, on, or, the, to and up, which are usually
% not capitalized unless they are the first or last word of the title.
% Linebreaks \\ can be used within to get better formatting as desired.
% Do not put math or special symbols in the title.
\title{Performance Analysis of 5G FR2 (mmWave) Downlink 256QAM on Commercial 5G Networks}


% author names and affiliations
% use a multiple column layout for up to three different
% affiliations
\author{
    \IEEEauthorblockN{Kasidis Arunruangsirilert\IEEEauthorrefmark{1}\IEEEauthorrefmark{3}, Pasapong Wongprasert\IEEEauthorrefmark{2}\IEEEauthorrefmark{3}, Jiro Katto\IEEEauthorrefmark{1}}
    \IEEEauthorblockA{\IEEEauthorrefmark{1}Department of Computer Science and Communications Engineering, Waseda University, Tokyo, Japan}
    \IEEEauthorblockA{\IEEEauthorrefmark{2}Department of Electrical Engineering, Chulalongkorn University, Bangkok, Thailand
    \\\{kasidis, katto\}@katto.comm.waseda.ac.jp, 6670331021@student.chula.ac.th}
    \IEEEauthorblockA{\IEEEauthorrefmark{3}Authors contributed equally to this research.}
    
}
\vspace{-2mm}
% conference papers do not typically use \thanks and this command
% is locked out in conference mode. If really needed, such as for
% the acknowledgment of grants, issue a \IEEEoverridecommandlockouts
% after \documentclass
% for over three affiliations, or if they all won't fit within the width
% of the page, use this alternative format:
% 
%\author{\IEEEauthorblockN{Michael Shell\IEEEauthorrefmark{1},
%Homer Simpson\IEEEauthorrefmark{2},
%James Kirk\IEEEauthorrefmark{3}, 
%Montgomery Scott\IEEEauthorrefmark{3} and
%Eldon Tyrell\IEEEauthorrefmark{4}}
%\IEEEauthorblockA{\IEEEauthorrefmark{1}School of Electrical and Computer Engineering\\
%Georgia Institute of Technology,
%Atlanta, Georgia 30332--0250\\ Email: see http://www.michaelshell.org/contact.html}
%\IEEEauthorblockA{\IEEEauthorrefmark{2}Twentieth Century Fox, Springfield, USA\\
%Email: homer@thesimpsons.com}
%\IEEEauthorblockA{\IEEEauthorrefmark{3}Starfleet Academy, San Francisco, California 96678-2391\\
%Telephone: (800) 555--1212, Fax: (888) 555--1212}
%\IEEEauthorblockA{\IEEEauthorrefmark{4}Tyrell Inc., 123 Replicant Street, Los Angeles, California 90210--4321}}
% use for special paper notices
%\IEEEspecialpapernotice{(Invited Paper)}
% make the title area
\maketitle

\vspace{-2mm}
%\copyrightnotice
% As a general rule, do not put math, special symbols or citations
% in the abstract
\begin{abstract}



 \setstretch{0.93}
The 5G New Radio (NR) standard introduces new frequency bands allocated in Frequency Range 2 (FR2) to support enhanced Mobile Broadband (eMBB) in congested environments and enables new use cases such as Ultra-Reliable Low Latency Communication (URLLC). The 3GPP introduced 256QAM support for FR2 frequency bands to further enhance downlink capacity. However, sustaining 256QAM on FR2 in practical environments is challenging due to strong path loss and susceptibility to distortion. While 256QAM can improve theoretical throughput by 33\%, compared to 64QAM, and is widely adopted in FR1, its real-world impact when utilized in FR2 is questionable, given the significant path loss and distortions experienced in the FR2 range. Additionally, using higher modulation correlates to higher BLER, increased instability, and retransmission. Moreover, 256QAM also utilizes a different MCS table defining the modulation and code rate at different Channel Quality Indexes (CQI), affecting the UE's link adaptation behavior. This paper investigates the real-world performance of 256QAM utilization on FR2 bands in two countries, across three RAN manufacturers, and in both NSA (EN-DC) and SA (NR-DC) configurations, under various scenarios, including open-air plazas, city centers, footbridges, train station platforms, and stationary environments. The results show that 256QAM provides a reasonable throughput gain when stationary but marginal improvements when there is UE mobility while increasing the probability of NACK responses, increasing BLER, and the number of retransmissions. Finally, MATLAB simulations are run to validate the findings as well as explore the effect of the recently introduced 1024QAM on FR2.
\end{abstract}

\begin{IEEEkeywords}
5G New Radio (NR), 5G Frequency Range 2 (mmWave), Downlink 256QAM, Radio Access Network, Wireless Communications
\end{IEEEkeywords}

% no keywords
 \copyrightnotice



% For peer review papers, you can put extra information on the cover
% page as needed:
% \ifCLASSOPTIONpeerreview
% \begin{center} \bfseries EDICS Category: 3-BBND \end{center}
% \fi
%
% For peerreview papers, this IEEEtran command inserts a page break and
% creates the second title. It will be ignored for other modes.
\IEEEpeerreviewmaketitle

\vspace{-2mm}
\section{Introduction}

One of the primary objectives of 5G New Radio (NR) is to achieve peak downlink throughput of up to 20 Gbps \cite{itu_m2083}. To facilitate this, Frequency Range 2 (FR2), also known as millimeter wave (mmWave), has been introduced, enabling the utilization of frequencies above 24 GHz—including the K-band and Ka-band—for mobile communications. The substantial spectrum available in this range allows Mobile Network Operators (MNOs) worldwide to deliver high throughput to user equipment (UE). However, this advantage comes at the expense of coverage and reliability due to significant path loss at higher carrier frequencies \cite{8891537}. Consequently, mmWave primarily serves as a capacity layer, deployed as small cells to supplement capacity in areas with high user density, thereby relieving the load on lower frequency bands and preserving resources on these coverage layers for UEs at the cell edges.

As the world progresses toward the 6G era, an increasing number of downlink-intensive use cases have emerged, including IP multicasting for UHD TV broadcasting over 5G \cite{7329950,tsuchida_2019}, Fixed Wireless Access (FWA) \cite{8484810,9628442}, and the transmission of large multimedia data for AI-related tasks. Despite this surge in demand, the frequency spectrum remains a limited resource, making the maximization of spectral efficiency essential. Similar to 4G Long Term Evolution (LTE) and 5G Frequency Range 1 (FR1) or Sub-6 GHz bands, the 3rd Generation Partnership Project (3GPP) proposed the application of Downlink 256QAM (DL-256QAM) on FR2 to enhance data throughput as part of their Release 15 \cite{3GPP_38-912}. Commercial UEs began implementing this feature in early 2022 with the release of the Japanese versions of the Samsung Galaxy S22 Ultra and Sony Xperia 1 IV, featuring the Qualcomm Snapdragon X65 RF Modem System. Although 256QAM modulation theoretically offers up to a 33\% increase in spectral efficiency, previous research indicates that sustaining this modulation scheme in real-world deployments is challenging in both uplink and downlink directions—even in Sub-6 GHz bands—due to high Signal-to-Interference-plus-Noise Ratio (SINR) requirements \cite{7470807, 10570635}. Additionally, using higher modulation correlates to an increase in transmission errors, therefore increasing the probability of retransmissions. In 5G NR, this translates to an increase in NACK responses, prompting retransmission until ACK is received. \looseness=-1

Moreover, Radio Access Network (RAN) documentations from multiple manufacturers suggest that enabling 256QAM modulation may negatively impact UEs with weak signals, leading to reduced throughput. This is attributable to the switch from the \textit{qam64} table (MCS index table 1 for PDSCH) to the \textit{qam256} table (MCS index table 2 for PDSCH) when 256QAM is enabled. Given that the 5G Physical Downlink Shared Channel (PDSCH) MCS table contains a maximum of 32 entries, the step size between each MCS index in the \textit{qam256} table is coarser than in the \textit{qam64} counterpart, resulting in less precise link adaptation due to the omission of intermediate MCS indexes. To address this, some RAN manufacturers offer proprietary solutions. For example, in Huawei RAN systems, configuring the parameter \textit{"DlLinkAdaptAlgoSwitch"} to \textit{"DL\_MCS\_TABLE\_ADAPT\_SW"} allows the RAN to automatically switch between the \textit{qam64} and the \textit{qam256} tables based on channel quality, combining the benefits of both. Additionally, RAN manufacturers typically classify DL-256QAM as an additional feature and usually charge MNOs a license fee on a per-cell basis, so enabling 256QAM in suboptimal conditions could incur extra costs without providing any benefits. Given the significantly higher path loss of FR2 compared to typical mobile communication bands, it is important to study how 256QAM modulation performs on mmWave bands. Such an investigation can assist MNOs in network optimization by minimizing potential negative impacts and avoiding unnecessary licensing costs associated with enabling DL-256QAM in non-ideal scenarios.

This paper, therefore, thoroughly investigates the network impact of enabling DL-256QAM on FR2, focusing on downlink throughput, retransmission rate, and modulation utilization. Experiments were conducted in two countries—Japan and Thailand—across three different MNOs: SoftBank Japan, au by KDDI Japan, and Advanced Info Service (AIS) Thailand, each employing different RAN manufacturers (Ericsson, Samsung, and Huawei, respectively). The frequency ranges each MNO uses also differ, with n257 (28 GHz) in Japan and n258 (26 GHz) in Thailand. Additionally, Japanese MNOs are among the first globally to commercially deploy 5G NR Dual Connectivity (NR-DC), which allows mmWave bands to be utilized in 5G Standalone (SA) deployments by enabling Sub-6 GHz and mmWave aggregation as early as late-2022 \cite{qualcomm_2022}. This scenario provides an opportunity to examine how DL-256QAM behaves in both Evolved-Universal Terrestrial Radio Access–Dual Connectivity (EN-DC, Non-Standalone) and NR-DC (SA) deployment models, thereby highlighting differences across frequency bands and deployment strategies. Data collection was performed in various locations, including open-air plazas, city centers, footbridges, train station platforms, and indoor environments, under two mobility conditions: walking and stationary. Finally, MATLAB simulations were run in three case scenarios: stationary, walking, and biking, to validate the real-world results. Simulations of 1024QAM modulation are also carried out to understand the effect of 1024QAM, which was recently introduced in FR1 as part of 3GPP Release 17 \cite{3GPP_38-306}, on FR2. \looseness=-1

 %\linespread{0.95} 
% no \IEEEPARstart


\section{Experiment Environment}


\subsection{User Equipment (UE)}
%\begin{figure}[h!]
%\centering
%\begin{subfigure}{.32\textwidth}
%  \centering
%  \includegraphics[width=0.95\linewidth]{UL-MIMO.jpg}
%  \caption{UL-MIMO Enabled}
%  \label{fig:UL-MIMO-NSG}
%\end{subfigure}
%\begin{subfigure}{.32\textwidth}
%  \centering
%  \includegraphics[width=0.95\linewidth]{UL-SISO.jpg}
%  \caption{UL-1Tx Enabled}
%  \label{fig:UL-SISO-NSG}
%\end{subfigure}
%\caption{Uplink Performance of ASUS Smartphone for Snapdragon Insiders (EXP21) on 5G Standalone (SA) Band n77}
%\end{figure}

The Samsung Galaxy S22 Ultra (SC-52C), equipped with a Qualcomm Snapdragon X65 5G RF Modem \cite{qualcomm_x65}, was used as the User Equipment (UE) for the experiment conducted in Japan. In Thailand, the Sony Portable Data Transmitter (PDT-FP1), with Qualcomm Snapdragon X70 5G RF Modem \cite{qualcomm_x70}, was utilized due to the difference in the frequency band. Both UEs support downlink 256QAM and can be modified to limit the modulation capability to 64QAM for the experimental requirements through firmware modification. The modification was verified by analyzing the \textit{UECapabilityInformation} packet. \textit{Network Signal Guru (NSG)} was used to collect the modem log data on the SC-52C smartphone. For the PDT-FP1, the modem log was collected through the DIAG interface using \textit{AirScreen}. The logs were then analyzed using both \textit{AirScreen} and \textit{Qualcomm Commercial Analysis Toolkit (QCAT)}. Since 5G FR2 is computationally intensive, resulting in excessive heat generation. For long walk tests in the neighborhoods, the test UE was equipped with an 18W thermoelectric cooler to avoid the possibility of network performance drops due to thermal throttling as well as force shutdown due to device's thermal mitigation mechanism.

\vspace{-1mm}
\subsection{Network Environment}

The experiments were conducted on commercial 5G networks in Japan and Thailand, specifically on SoftBank and KDDI networks in Tokyo, Japan, and the AIS network in Bangkok, Thailand. The RAN configurations differ slightly across the three networks regarding parameters and RRC features, as outlined in Table \ref{tab_RRCConfig}. The RAN vendor also differed across these networks, with AIS using Huawei, KDDI using Samsung, and SoftBank using Ericsson. These variations in configurations and vendor equipment were intended to introduce variability in network performance, allowing for a broader analysis of FR2 or mmWave network behavior across diverse settings. During the experiment, data collection was performed during walk tests and also stationary tests to understand the performance impact when mobility is introduced.

\begin{table}[!tbp]
\vspace{1mm}
\setstretch{0.8}
\caption{Summary of Network RRC Features differences}
\vspace{-1.5mm}
\centering
\label{tab_RRCConfig}
\resizebox{6cm}{!}{\begin{tabular}{@{}lccc@{}}
\toprule
Parameter                  &AIS&KDDI&SoftBank   \\\midrule
Vendor & Huawei & Samsung & Ericsson\\
CSI-RS for TRS-info & Yes & No & Yes\\
QCL/TCI States & Yes & No & Yes\\
Meas Report& rsrp, rsrq, sinr & rsrp, rsrq& rsrp, sinr\\
\bottomrule
\end{tabular}}
\vspace{-4mm}
\end{table}

\begin{table}[!tbp]
\vspace{1mm}
\setstretch{0.8}
\caption{Summary of Simulation Parameters}
\vspace{-1.5mm}
\centering
\label{tab_SimParams}
\resizebox{5cm}{!}{\begin{tabular}{@{}lc@{}}
\toprule
Parameter                  &Value\\\midrule
Carrier Frequency& 24.8 GHz\\
Channel Bandwidth& 100 MHz\\
Number of Resource Block&66\\
Subcarrier Spacing (SCS) & 120 kHz\\
TDD Slots Configuration (DL/UL)&3/1\\
TDD Symbols Configuration (DL/UL)&10/1\\
Tx/Rx Mode & 4T4R \\
EIRP & 67 dB \\
Receiver Gain & 29.5 dB\\\midrule
UE Receive Gain & 1 dB\\
UE Tx/Rx Antenna Count & 2T2R\\
UE Transmit Power & 23 dBm\\
\bottomrule
\end{tabular}}
\vspace{-6mm}
\end{table}

\subsection{Field Test Setup}

Since 5G FR2 coverage is quite limited, the experiment location is based on the availability of 5G FR2. Attempts had been made to choose the location with a wide variety of RF conditions and deployment configurations. Overall, eight locations were chosen, in which, 11 head-to-head experiments (nine with mobility, two at rest) were conducted at five locations with ease of repeatability. However, when repeatability is deemed difficult due to reasons such as wide 5G FR2 coverage or an area being too crowded, UE was set to use \textit{qam256} table, and one trial was conducted. LTE Band 3 (1.8 GHz) was used as the anchor band for 5G NSA experiments, and for 5G SA, NR Band n77 (3.7 GHz) was used to anchor the FR2 band.

For head-to-head experiments, five locations choose for the experiments are BTS Siam Sky Bridge (Thailand), Sidewalk in front of Waseda University (Japan), Tokyo Station Square (Japan), Akihabara Station (Japan), and Ueno Station (Japan). On the other hand, the \textit{single trial} locations were the neighborhoods of Akihabara (Japan), Kabukicho (Japan), and West Shinjuku (Japan). It should be noted that only some base stations in Kabukicho has DL-256QAM enabled on FR2, so data will be analyzed separately. All SoftBank base stations in the Akihabara area do not support DL-256QAM on FR2. Moreover, due to UE capability concerns, MNOs around the world usually deploy 5G with a channel bandwidth of 100 MHz per carrier, then utilize Carrier Aggregation to combine multiple carriers to fully utilize the frequency spectrum licensed to the MNO. For this reason, it is common for multiple carriers between two and eight being deployed for the FR2. Since reporting the aggregated throughput would be misleading due to the difference in the aggregated bandwidth, the average was taken across all 100 MHz carriers available and reported as throughput per 100 MHz carrier in the following section. The RF conditions at each location by carrier can be seen in Table \ref{tab_RFCondition} and the mobility map can be seen in Fig. \ref{fig:MobilityMap}. Finally, the head-to-head test cases were outlined in Table \ref{tab_headhead}. \looseness=-1

\begin{figure}[t!]

\begin{subfigure}{\linewidth}
\centering\includesvg[width=0.9\linewidth,inkscapelatex=false]{Walking.svg}
  \vspace{-1.3mm}
  \caption{Walking Cases}
  \label{fig-Walking}
  \vspace{-2mm}
\end{subfigure}\\

\begin{subfigure}{\linewidth}
\vspace{-6mm}
\centering\includesvg[width=0.9\linewidth,inkscapelatex=false]{Biking.svg}
  \vspace{-1.3mm}
  \caption{Biking Cases}
  \label{fig-Biking}
\end{subfigure}
\vspace{-5mm}
\caption{Simulation Setup}
\vspace{-3mm}
\label{fig:SimulationSetup}
\end{figure}

\begin{figure}[t!]
\vspace{1mm}
\centering
\begin{subfigure}{.24\textwidth}
  \centering
  \includegraphics[width=0.95\linewidth]{AIS_Walk.png}
  \vspace{-1mm}
  \caption{Sky Bridge (TH)}
  \label{fig:AIS_Walk}
\end{subfigure}%
\begin{subfigure}{.24\textwidth}
  \centering
  \includegraphics[width=0.95\linewidth]{Waseda_Walk.png}
  \vspace{-1mm}
  \caption{Waseda Sidewalk (JP)}
  \label{fig:Waseda_Walk}
\end{subfigure}\\
\begin{subfigure}{.24\textwidth}
  \centering
  \includegraphics[width=0.95\linewidth]{TokyoStation.png}
  \vspace{-1mm}
  \caption{Tokyo Station Square (JP)}
  \label{fig:TokyoStation}
\end{subfigure}%
\begin{subfigure}{.24\textwidth}
  \centering
  \includegraphics[width=0.95\linewidth]{AkihabaraWalk.png}
  \vspace{-1mm}
  \caption{Akihabara Station (JP)}
  \label{fig:AkihabaraWalk}
\end{subfigure}
\begin{subfigure}{.24\textwidth}
  \centering
  \includegraphics[width=0.95\linewidth]{Uenov2.png}
  \captionsetup{justification=centering}
  \vspace{-1mm}
  \caption{Ueno Station (JP)}
  \label{fig:Ueno}
\end{subfigure}%
\begin{subfigure}{.24\textwidth}
  \centering
  \includegraphics[width=0.95\linewidth]{AkihabaraLoop.png}
  \captionsetup{justification=centering}
  \vspace{-1mm}
  \caption{Akihabara Neighborhood (JP)}
  \label{fig:AkihabaraLoop}
\end{subfigure}\\
\begin{subfigure}{.49\textwidth}
  \centering
  \includegraphics[width=0.95\linewidth]{Kabuki.png}
  \captionsetup{justification=centering}
  \vspace{-1mm}
  \caption{West Shinjuku (Left) and Kabukicho (Right) (JP)}
  \label{fig:Kabuki}
\end{subfigure}
\vspace{-5mm}
\caption{Test Route of each experiment. Color legends represent 5G NR SS-RSRP.}
\label{fig:MobilityMap}
\vspace{-7mm}
\end{figure}



\begin{table}[!tbp]
\setstretch{0.8}
\caption{Summary of RF Conditions}
\vspace{-1.5mm}
\centering
\label{tab_RFCondition}
\resizebox{8.7cm}{!}{\begin{tabular}{@{}lccccccc@{}}
\toprule
Location   &Carrier&LTE&LTE&LTE&5G NR&5G NR&5G NR \\
&&RSRP&RSRQ&SINR&RSRP&RSRQ&SINR\\
&&(dBm)&(dB)&(dB)&(dBm)&(dB)&(dB)\\\midrule
Sky Bridge (TH) & AIS & -59.77&-12.71&15.64&-93.92&-10.95&19.26\\
Waseda Sidewalk (JP)&SoftBank&-72.66&-10.96&9.55&-103.18&-11.44&13.20
\\
Tokyo Station Square (JP)&KDDI&-92.33&-12.24&5.38&-72.50&-10.25&31.11\\
&SoftBank&-93.88&-10.69&10.84&-86.50&-11.18&26.87\\
Akihabara Station (JP)&KDDI&-66.92&-10.86&12.09&-77.30&-10.49&30.16\\
Ueno Station (JP)&KDDI&-76.44&-10.33&16.36&-80.18&-10.53&25.64\\\midrule
Kabukicho (qam64) (JP)&SoftBank&-70.57&-11.49&8.32&-99.46&-11.34&15.26\\
Akihabara (qam64) (JP)&SoftBank&-70.58&-11.27&8.82&-100.20&-11.20&15.14\\
Kabukicho (qam256) (JP)&SoftBank&-67.41&-11.57&7.25&-96.34&-11.42&14.17\\
West Shinjuku (qam256) (JP)&KDDI&-71.75&-10.61&14.74&-85.39&-10.57&25.78\\

\bottomrule
\end{tabular}}
\vspace{-4mm}
\end{table}


%For the Sky Bridge (TH), the experiment was conducted on the sky bridge, connecting the Siam BTS station and the Rachathewi station. During the experiment, the UE mostly experiences Line of Sight (LOS) conditions and only experiences NLOS when walking past the concrete pillars. The path of the walk test is a loop, a total of 300 meters as shown in Figure \ref{fig:AIS_Walk}. Next, for Waseda Sidewalk (JP), the experiment was 

\begin{table}[!tbp]
\setstretch{0.8}
\vspace{1mm}
\caption{Summary of Head-to-Head Test Cases}
\vspace{-1.5mm}
\centering
\label{tab_headhead}
\resizebox{8.7cm}{!}{\begin{tabular}{@{}lcccl@{}}
\toprule
Case&Carrier&Type&Category&Description\\
\midrule
Sky Bridge (CW)&AIS&NSA&Outdoor&Walk clockwise in loop\\
&&&&(NLOS/LOS, multiple base stations)\\
Sky Bridge (CCW)&AIS&NSA&Outdoor&Walk counterclockwise in loop\\
&&&&(NLOS/LOS, multiple base stations)\\
Sky Bridge (Crowded)&AIS&NSA&Outdoor&Walk clockwise in loop during rush hour \\
&&&&(NLOS/LOS, multiple base stations)\\
Waseda Sidewalk (Walk)&SoftBank&NSA&Outdoor&Walk in loop (NLOS/LOS, single base station)\\
Waseda Sidewalk (Rest)&SoftBank&NSA&Outdoor&Stand in front of base station (single base station)\\
Tokyo Sta. (EN-DC)&KDDI&NSA&Outdoor&Walk in loop (LOS, single base station)\\
Tokyo Sta. (NR-DC)&SoftBank&SA&Outdoor&Walk in loop (LOS, single base station)\\
Akihabara Sta. (Walk)&KDDI&NSA&Indoor&Walk in loop (NLOS/LOS, single base station)\\
Akihabara Sta. (Rest)&KDDI&NSA&Indoor&Stand in front of base station (single base station)\\
Ueno Station&KDDI&NSA&Indoor&Walk in loop (NLOS/LOS, multiple base stations)\\

\bottomrule
\end{tabular}}
\vspace{-5mm}
\end{table}

\subsection{Simulation Setup}



To validate the field test results and explore the effect of 1024QAM on FR2. Simulations were conducted in MATLAB across three Line of Sight (LOS) scenarios: stationary, walking, and biking. The gNB was configured to match AIS' configurations and Huawei HAAU5323 specifications as shown in Table \ref{tab_SimParams}. The walking simulation involves walking 45 meters farther from the base station starting from the initial distance, then returning to the starting point at an average speed of 1.375 m/s (see Fig. \ref{fig-Walking}). On the other hand, for biking simulation, UE is being moved away from the base station to cover the distance of 240 meters at an average speed of 6.7 m/s (see Fig. \ref{fig-Biking}). The simulation time for walking and biking cases is 60 seconds and 30 seconds, respectively. Due to MATLAB lacking the implementation of MCS and CQI Table required for 1024QAM simulation, the required MCS Table 5.1.3.1-4 and CQI Table 5.2.2.1-5 were implemented based on 3GPP TS 38.214 Release 17 \cite{3GPP_38-214} in the file \textit{MACConstants.m} and \textit{nrCQITables.m}, respectively. Then, 1024QAM was added to \textit{getMCSTable.m} and \textit{getCQITable.m} to make MATLAB recognize the newly implemented MCS and CQI tables. Furthermore, \textit{nrGNB.m} was modified to select the appropriate MCS Table for each simulation case. Finally, for the propagation path loss model, the "Urban Micro" or UMi model was implemented based on 3GPP TR 38.901 \cite{3GPP_38-901, 7504435}.







\section{Results and Analysis}
\begin{figure}[t!]

\begin{subfigure}{\linewidth}
\centering
\includesvg[width=0.92\linewidth,inkscapelatex=false]{64QAMUsage.svg}
  \vspace{-1mm}
  \caption{MCS Table 1 (\textit{qam64})}
  \label{fig-64QAMUsage}
\end{subfigure}\\
\begin{subfigure}{\linewidth}
\centering
\includesvg[width=0.92\linewidth,inkscapelatex=false]{256QAMUsage.svg}
  \vspace{-1mm}
  \caption{MCS Table 2 (\textit{qam256})}
  \label{fig-256QAMUsage}
\end{subfigure}
\\
\begin{subfigure}{\linewidth}
\centering
\includesvg[width=0.60\linewidth,inkscapelatex=false]{UtilizationLegend.svg}
\end{subfigure}
\vspace{-5mm}
\caption{Modulation Utilization (\%) for Head-to-Head cases}
\vspace{-5mm}
\label{fig:QAMUse}
\end{figure}
\begin{figure}[t!]

\begin{subfigure}{\linewidth}
\centering
\includesvg[width=0.8\linewidth,inkscapelatex=false]{ModulationNeighborhood.svg}
  \vspace{-1mm}
  \caption{Modulation Utilization (\%)}
  \label{fig-ModulationNeighborhood}
\end{subfigure}\\
\begin{subfigure}{\linewidth}
\centering
\includesvg[width=0.8\linewidth,inkscapelatex=false]{ThptNeighborhood.svg}
  \vspace{-1mm}
  \caption{Downlink Throughput per carrier}
  \label{fig-ThptNeighborhood}
  \vspace{0.5mm}
\end{subfigure}

\vspace{-1.5mm}
\caption{Walking in Neighborhood Field Test Results}
\label{fig:NeighborhoodResults}
\vspace{-6mm}

\end{figure}
\begin{figure*}[t!]
\vspace{1mm}
\begin{subfigure}[t]{.50\textwidth}
  \includesvg[width=0.98\linewidth,inkscapelatex=false]{Thpt.svg}
  \vspace{-1mm}
  \caption{Average Downlink MAC Throughput (Mbps) across all carriers.}
  \label{fig:Thpt}
\end{subfigure}\hfill
\begin{subfigure}[t]{.50\textwidth}
  \includesvg[width=0.98\linewidth,inkscapelatex=false]{ReTx.svg}
  \vspace{-1mm}
  \caption{Retransmission Rate (\%) by Test Case.}
  \label{fig:ReTx}
\end{subfigure}\\

\begin{subfigure}[t]{\linewidth}
\vspace{-4mm}
\centering\includesvg[width=0.33\linewidth,inkscapelatex=false]{BarChartLegend.svg}
\vspace{1mm}
\end{subfigure}\\
\vspace{-4mm}
\begin{subfigure}[t]{.33\textwidth}
  \includesvg[width=0.98\linewidth,inkscapelatex=false]{SoftBankCurve.svg}
  \vspace{-1mm}
  \caption{5G SS-RSRP vs DL-Thpt: SoftBank}
  \label{fig:SBCurve}
\end{subfigure}\hfill
\begin{subfigure}[t]{.33\textwidth}
  \includesvg[width=0.98\linewidth,inkscapelatex=false]{AISCurve.svg}
  \vspace{-1mm}
  \caption{5G SS-RSRP vs DL-Thpt: AIS}
  \label{fig:AISCurve}
\end{subfigure}\hfill
\begin{subfigure}[t]{.33\textwidth}
  \includesvg[width=0.98\linewidth,inkscapelatex=false]{KDDICurve.svg}
  \vspace{-1mm}
  \caption{5G SS-RSRP vs DL-Thpt: KDDI}
  \label{fig:KDDICurve}
\end{subfigure}\hfill
\\
\begin{subfigure}[t]{\linewidth}
\vspace{4.5mm}
\centering\includesvg[width=0.33\linewidth,inkscapelatex=false]{LineChartLegend.svg}
\vspace{2mm}
\end{subfigure}\\


\setlength{\belowcaptionskip}{-18pt}
\vspace{-3mm}
\caption{Head-to-Head Field Test Results}

\end{figure*}

\begin{figure*}[t!]
\vspace{1.5mm}
\begin{subfigure}[t]{.33\textwidth}
  \includesvg[width=0.98\linewidth,inkscapelatex=false]{SimulationCurve.svg}
  \vspace{-1.5mm}
  \caption{Downlink Throughput vs Distance}
  \label{fig:SimulationCurve}
\end{subfigure}\hfill
\begin{subfigure}[t]{.33\textwidth}
  \includesvg[width=0.98\linewidth,inkscapelatex=false]{WalkChart.svg}
  \vspace{-1.5mm}
  \caption{Walking Cases}
  \label{fig:WalkChart}
\end{subfigure}\hfill
\begin{subfigure}[t]{.33\textwidth}
  \includesvg[width=0.98\linewidth,inkscapelatex=false]{BikeChart.svg}
  \vspace{-1.5mm}
  \caption{Biking Cases}
  \label{fig:BikeChart}
\end{subfigure}\hfill

\begin{subfigure}[t]{\linewidth}
\vspace{0.5mm}
\centering\includesvg[width=0.5\linewidth,inkscapelatex=false]{SimulationLegend.svg}
\vspace{1mm}
\end{subfigure}\\


\setlength{\belowcaptionskip}{-18pt}
\vspace{-2.5mm}
\caption{Simulation Results}
\label{fig:SimulationResults}
\vspace{-0.8mm}
\end{figure*}

\subsection{Field Test Results}



Fig. \ref{fig:QAMUse} shows the modulation utilization percentage for each test case both when using Modulation Coding Scheme Table 1 (\textit{qam64}) and Table 2 (\textit{qam256}). The results show that when MCS Table 2 was used, an average of 35\% of resource blocks were being modulated using 256QAM instead of 64QAM across all mobility cases, while the amount of 16QAM and QPSK utilization remains similar. Which demonstrates that despite the huge path loss in the FR2, the real-world channel quality is somewhat sufficient to allow partial utilization of 256QAM modulation. However, without UE mobility, a significantly higher utilization of 256QAM can be achieved, with an average of 88\% of resource blocks being modulated using 256QAM, which would imply a significant improvement in spectral efficiency. Unfortunately, Fig. \ref{fig:ReTx}, which compares the retransmission rate between the MCS Table 1 and 2, shows that the retransmission rate increased significantly across all cases by varying degrees, which outweighs the benefits of MCS Table 2 of introducing the 256QAM modulation in the first place. This is reflected in the Downlink Throughput per 100 MHz carrier results (see Fig. \ref{fig:Thpt}), which shows that MCS Table 2 yields an average of 5.8\% improvement across all test cases, with some cases performing better when using the conventional MCS Table 1. While using MCS Table 2 indeed improved the throughput slightly, it comes at the cost of high retransmission rates due to the increase in PDSCH NACK responses, which will negatively impact the latency because the next PDSCH packet will not be decoded until there is an ACK response. Therefore, confirming that for latency-sensitive applications, using higher modulation should be avoided as much as possible, as long as the throughput is sufficient.


Since it was found that MCS Table 2 yielded an improvement under good channel conditions, while may cause throughput deficit under average signal conditions, it is necessary to understand the scenarios that MCS Table 2 and 256QAM are useful to only use them in appropriate scenarios. Fig. \ref{fig:SBCurve} shows the average throughput on SoftBank at each RSRP level, with 95\% confidence intervals displayed for MCS Table 1 and 2. The graph indicates that MCS Table 2 provides a modest improvement in performance under sufficiently good channel conditions. Which in this case, is RSRP above -94 dBm. At approximately -95 dBm, the average throughput for MCS Table 1 and 2 converges, marking a crossover point where the performance gain of MCS Table 2 diminishes. Below this threshold, MCS Table 1 offers a performance advantage, up until approximately -102 dBm where below this threshold, the throughput performance of both tables converges, and performance differences become negligible. Moving on, a similar relationship is also observed on AIS (see Fig. \ref{fig:AISCurve}) where a marginal performance improvement is observed on MCS Table 2 under strong signal conditions. However, the throughput was unstable, and eventually converged, becoming similar to the MCS Table 1 just like before. Lastly, the results on KDDI (see Fig. \ref{fig:KDDICurve}) showed that under very strong signal conditions, the utilization of 256QAM can improve the throughput by over 30\%, which is very close to the theoretical value. However, this was observed at RSRP above -60dBm, which cannot be achieved in most use cases, and the percentage gain begins to diminish rapidly below this threshold. At around -70 dBm, MCS Table 2 now only increases the throughput by approximately 20\%, respective to the MCS Table 1 configuration. This is because although 256QAM is still being utilized at this level for the MCS Table 2 configuration, it uses a lower code rate, whereas the MCS Table 1 configuration uses 64QAM Modulation but at a higher code rate. \looseness=-1


For the field test results from walking in the neighborhoods (see Fig. \ref{fig:NeighborhoodResults}), where the UE used for experiments was held as if used by an end user. In these less controlled scenarios, the average utilization of 256QAM modulation dropped to 27\%. While a substantial MAC layer throughput improvement of 28.5\% was observed in Kabukicho due to high mmWave cell density, the West Shinjuku area, which has DL-256QAM enabled, and the Akihabara area, which only supports DL-64QAM, showed only a 4\% improvement in MAC layer throughput, despite an 11\% improvement in physical layer throughput. This suggests that the spectral efficiency gain from 256QAM is offset by retransmissions, resulting in negligible improvement in real-world performance. Therefore, MNOs should carefully consider FR2 cell density in target areas before enabling DL-256QAM on FR2 to save licensing costs and avoid negatively impacting end users' QoE.


Overall, similar conclusions can be drawn from all test scenarios across the three networks. The utilization of 256QAM consistently leads to increased retransmissions, which, in turn, impacts latency in all scenarios while providing only marginal performance gains. In real-world conditions, achieving sufficient signal quality for 256QAM is unlikely due to rapid path loss on FR2, and when 256QAM was used, it only provided marginal improvements in most scenarios, particularly during mobility cases. Therefore, the results suggest that for latency-sensitive applications such as URLLC, using MCS Table 2 with 256QAM modulation on FR2 is not recommended. For typical enhanced mobile broadband (eMBB) applications that can tolerate higher latency, the increased retransmissions associated with 256QAM are less critical. Under ideal conditions, 256QAM can increase user-perceived throughput by up to 30\%; however, under average signal conditions, using MCS Table 2 may result in worse throughput compared to MCS Table 1, effectively negating any capacity improvements. In a real network with multiple UEs experiencing varying signal qualities, this would likely cancel out the overall benefit of 256QAM. \looseness=-1

\subsection{Simulation Results}



Fig. \ref{fig:SimulationResults} shows the MATLAB Downlink Throughput vs Distance simulation results for stationary, walking, and biking mobility scenarios. The throughput results for MCS Table 1 (\textit{qam64}), 2 (\textit{qam256}), and 4 (\textit{qam1024}) demonstrate that throughput decreases sharply with distance from the base station due to high path loss experienced in FR2. Initially, MCS Table 4 achieves the highest throughput at close range, but its performance drops rapidly within the first 200 meters, quickly converging with that of MCS Tables 1 and 2. This convergence behavior is consistent with the findings from the real-world test results. This indicates that MCS tables with higher modulation are highly sensitive to signal quality and only effective at very close proximity to the base station. It was also found that the MCS Table 2 modulation performed better than MCS Table 4 at 250 meters for the biking mobility scenario, suggesting the need for more intermediate MCS steps. Additionally, similar to the real-world results, MCS Table 2 provides a modest initial throughput advantage over MCS Table 1, but this benefit rapidly diminishes and converges with MCS Table 1. On the other hand, MCS Table 1 demonstrates a more gradual decline in throughput and maintains greater stability over longer distances, ultimately achieving comparable or even slightly higher throughput than MCS Tables with 256QAM and 1024QAM modulation at extended distances. These results suggest that while higher-order QAM schemes offer performance benefits in short-range, line-of-sight conditions, their advantages are severely limited by mmWave’s rapid path loss, making MCS Table 1 a more reliable option at greater distances. 
\looseness=-1

It should be noted that these limitations arise due to the limited number of entries allowed in MCS tables standardized in the 3GPP standard. Each MCS table supports a maximum of 31 entries, requiring tables with higher modulation indexes to take the compromise by having to remove some intermediate MCS entries for lower modulation schemes. This affects the link adaptation ability at mid-cell and cell edge, which may result in MCS tables with higher modulation index to under perform in such scenarios. Some RAN manufacturers have implemented dynamic MCS table switching based on signal conditions to address this issue, but a more effective solution would be to increase the number of allowed entries per MCS table, enabling better link adaptation across all signal conditions. \looseness=-1



\section{Conclusions and Future Work}

The purpose of this study is to evaluate the performance of 256QAM on FR2 in KDDI and SoftBank networks in Japan, and the AIS network in Thailand. Across all three networks, it was found that 256QAM utilization on FR2 is significantly limited due to heavy path loss on FR2 combined with the high MCS requirements for 256QAM. MATLAB simulations on FR2 also suggest that 256QAM utilization will be restricted due to rapid throughput decline over distance. A reasonable performance gain was observed in stationary scenarios, highlighting the impact of mobility on modulation performance in FR2. This suggested that MCS tables with higher modulation schemes such as Table 2 (\textit{qam256}) and Table 4 (\textit{qam1024}) could be useful in improving users' throughput in stationary use cases such as Fixed Wireless Access (FWA) deployments. However, with mobility, the performance gain of 256QAM diminishes significantly, providing only marginal improvements over 64QAM and greatly increasing retransmission rates due to a higher probability of NACK responses. This increase in retransmissions negatively impacts latency performance, making 256QAM unsuitable for latency-sensitive applications such as URLLC; therefore, for these applications, the use of MCS Table 2 is not recommended. The increased retransmissions also limit the potential throughput gain of 256QAM and worsen the latency. Additionally, under average signal conditions, using MCS Table 1 may yield better throughput and stability, effectively canceling out the capacity gains of MCS Table 2 in real network environments where multiple UEs experience varying signal conditions. This suggests that MCS Table 2 could be optimized, and future research will explore custom MCS tables through simulations to address the capacity gain limitations observed in this study.

\section*{Acknowledgement}


This paper is supported by the Ministry of Internal Affairs and Communications (MIC) Project for Efficient Frequency Utilization Toward Wireless IP Multicasting. Additionally, the authors would like to express their gratitude to \textbf{PEI Xiaohong} of \textit{Qtrun Technologies} for providing \textit{Network Signal Guru (NSG)} and \textit{AirScreen}, the cellular network drive test software used for result collection and analysis in this research. Finally, this paper is inspired by \textbf{Furina}, the legendary actress of \textit{Fontaine} who makes all the \textit{Teyvat} her stage.




%No Meme (Lame)

%This work was supported by NICT (Grant No. 03801), Japan. Additionally, the authors would like to express their gratitude to \textbf{PEI Xiaohong} of Qtrun Technologies for providing Network Signal Guru (NSG) and AirScreen, the cellular network drive test software used for result collection and analysis in this research.


% use section* for acknowledgment





% trigger a \newpage just before the given reference
% number - used to balance the columns on the last page
% adjust value as needed - may need to be readjusted if
% the document is modified later
%\IEEEtriggeratref{8}
% The "triggered" command can be changed if desired:
%\IEEEtriggercmd{\enlargethispage{-5in}}

% references section

% can use a bibliography generated by BibTeX as a .bbl file
% BibTeX documentation can be easily obtained at:
% http://mirror.ctan.org/biblio/bibtex/contrib/doc/
% The IEEEtran BibTeX style support page is at:
% http://www.michaelshell.org/tex/ieeetran/bibtex/
%\bibliographystyle{IEEEtran}
% argument is your BibTeX string definitions and bibliography database(s)
%\bibliography{IEEEabrv,../bib/paper}
%
% <OR> manually copy in the resultant .bbl file
% set second argument of \begin to the number of references
% (used to reserve space for the reference number labels box)

\setstretch{0.9}
\renewcommand{\IEEEbibitemsep}{0pt plus 0.5pt}
\makeatletter
\IEEEtriggercmd{\looseness=-1}
\makeatother
\IEEEtriggeratref{1}
\Urlmuskip=0mu plus 1mu\relax

\bibliographystyle{IEEEtran}

\bibliography{IEEEabrv,bstcontrol,b_reference}





% that's all folks
\end{document}


