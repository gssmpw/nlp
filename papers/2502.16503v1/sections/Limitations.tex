\section*{Limitations}

While this paper proposes a multilingual parody benchmark and provides an extensive analysis, we acknowledge several limitations that warrant further exploration in future work:

\begin{itemize} 
\item Limited dataset diversity. Although we collect datasets and analyze experimental results in both Chinese and English, the understanding of how parody manifests or how effective current methods are for parody detection in other languages remains unclear. Therefore, further efforts could be made to gather datasets in additional languages to enhance the diversity of parody data.
\item Annotation quality limitations. While we invite multiple annotators and conduct re-checks after labeling, some minor errors may still exist, as annotating parody can be a challenging task. To improve annotation quality in future studies, we will recruit more annotators and provide them with additional background knowledge related to the events before the annotation process. This will help ensure more accurate and consistent annotations.
\item Limited evaluation of Large Language Models (LLMs). In this study, we only test the performance of LLMs on parody-related tasks through prompt-based methods, without fine-tuning. This approach may not fully capture the potential of LLMs. Additionally, only 6 LLMs were evaluated, which is a relatively small number considering the rapid development of these models. Future work should include a broader range of LLMs and explore fine-tuning approaches to better assess their capabilities in parody detection tasks.
\item Limited exploration of graph-based methods. In our experiments, Graph Neural Networks (GNNs) are used solely for user sentiment classification. The application of GNNs to parody detection and comment sentiment classification remains unexplored, primarily due to the lack of paradigms that allow GNNs to classify edges in graphs. Future work could focus on designing GNN models tailored to edge classification, enabling more comprehensive experiments on parody detection and comment sentiment analysis.
\end{itemize}

