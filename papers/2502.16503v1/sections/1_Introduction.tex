\section{Introduction}

\begin{figure}[t]
  \includegraphics[width=\columnwidth]{images/illustrations/example.pdf}
  \caption{People debate online about the topic, ``\textit{Should my boyfriend hand over his salary to me?}'' Some users explicitly support or oppose this viewpoint, while others implicitly express their stance through parody, using humor or even subtle blackmail to make their point.}\vspace{-0.48cm}
  \label{fig:example_parody}
\end{figure}


% 1. What is parody, parody example, parody difference
Parody in social media\footnote{Also known as "\ch{反串}" or "FanChuan" in Chinese online social media.} is a form of humor or satire, which uses exaggerated or absurd imitations for critique or entertainment \citep{parody_definition}. It has become popular around some controversial topics in recent years, especially among the young generation \citep{parody_young,parody_young2}. For example, as shown in Figure \ref{fig:example_parody}, the question ``\textit{Should my boyfriend hand over his salary to me?}'' has sparked intense debate \citep{news}. While some users clearly express their views as \textcolor{neu}{neutral}, \textcolor{pos}{supportive}, or \textcolor{neg}{opposed}, others adopt a \textcolor{orp}{parody} tone, mockingly pretending to support the stance with exaggerated statements like, ``\textit{Guys who don’t hand over their salary are a HUGE red flag...}'', which subtly opposes it. This tactic can attract attention and provoke reactions through humor, making people reflect their opinions. Similar to irony or sarcasm \citep{topic_gender_EPIC}, parody also expresses the opinion opposite to its appearance. However, it emphasizes playful, entertaining, and exaggerated mimicry of a character, making the underlying critique more accessible and engaging to the audience.

% Impact of parody
The real meaning behind parody is highly culture-dependent. Therefore, the analysis of parody can offer unique insights in understanding the corresponding cultural values. The spread of parody on internet also fosters a diverse linguistic culture \citep{menghini2024digital}. People can share their distinct views on society, political, or cultural topics in a humorous and engaging manner, encouraging global and cross-cultural dialogue. In addition, parody plays a crucial role in the formation of subcultures \citep{willett2009parodic,booth2014slash}. Parody comments not only create distinct communities, but also mirror the values and identities of online users. For younger generations, parody comments have become a way of self-expression, which help to define their uniqueness, build connection with others, and form social circles. Gradually, it has become a shared language and a set of symbols for the growth of internet subcultures. 

% Previous studies on parody-related dataset collection primarily focus on sarcasm, irony, and humor. \citet{ghosh2017role} propose a sarcasm dataset by collecting user comments from Twitter and dialogues to provide additional conversational context. \citet{topic_politic_Guanchache} construct Chinese sarcasm datasets from the Guanchazhe website, covering seven different topics. \citet{cai2019multi} construct multimodal irony datasets from Twitter, incparodyorating both text and images. \citet{he2024chumor} collect a humor dataset from Ruozhiba, a Chinese Reddit-like platform known for its intellectually challenging and culturally specific jokes. Understanding the content of these datasets requires a deep grasp of language, posing significant challenges for deep learning models and Large Language Models (LLMs) \citep{yao2024sarcasm,zhang2023stance}. However, when it comes to parody, the performance of current models and LLMs remains unknown due to the lack of dedicated parody datasets.
Despite the widespread popularity of parody, there is a lack of high-quality datasets that capture parody comments with different topics and languages \citep{parody_dataset}, restricting the more general and inclusive analysis in various contexts. To fill this gap, we propose FanChuan, a parody benchmark with high quality in three key aspects: \textbf{high diversity, rich contexts, and precise annotations}. \textbf{First}, we enhance diversity by collecting data from multiple sources (both Chinese and English corpora), a wide range of topics, and various social media platforms. Such broad coverage allows us to conduct more sufficient, balanced and fair evaluations of models. \textbf{Second}, we construct richer context information by building the relationship between comments and their replies as heterogeneous graphs. Unlike previous studies that only focus on textual \citep{text_zhang} or dialogue \citep{dialogue_bamman,dialogue_wang} content, the graph-structured context enables the exploitation of relational information, which is found to be fairly valuable later. \textbf{Third}, since parody labeling is quite challenging and disagreements among annotators can easily arise, we ensure the quality of annotation by employing native speakers to label the parody and sentiment of each comment. Additionally, we have expert judges to resolve any disagreement and Large Language Models (LLMs) to refine the annotation results, ensuring consistency and reliability. As a result, we have created \textbf{seven} datasets, with \textbf{14,755} annotated users and \textbf{21,210} annotated comments in total, enabling comprehensive experiments and analyses. 

% \sitao{emphasize that the two existing parody papers are from pre-LLM era. We are the first the test and analyze LLMs on parody.}
% The uniqueness of our datasets results in two aspects: 1) Our parody datasets with social networks as contextual information. 3) multi-source. As suggested in \citep{} more generalizable cases

With the new datasets, we evaluate embedding-based methods \citep{RoBERTa}, incongruity-based methods \citep{SarcPrompt}, outlier detection methods \citep{IsolationForest}, graph-based methods \citep{GCN}, and Large Language Models (LLMs) \citep{GPT4} on FanChuan with three parody related tasks: parody detection, comment sentiment classification with parody, and user sentiment classification with parody. Our results indicate that \textbf{(1)} parody-related tasks are challenging for all models, and even LLMs fail to consistently outperform traditional embedding-based approaches; \textbf{(2)} model performance of sentiment classification drops significantly on comments exhibiting parody behavior compared to those without parody; \textbf{(3)} incorporating commented objects as contextual information greatly enhances parody detection performance; \textbf{(4)} reasoning LLMs fail to outperform non-reasoning LLMs on parody detection. To our best knowledge, the existing studies on parody\citep{parody_dataset,willett2009parodic} are all from pre-LLMs era, and we are the first to evaluate the performance of LLMs on parody detection. In summary, our contributions are summarized as follows:

\begin{itemize}
    \item We introduce FanChuan, a parody benchmark that includes seven datasets from both Chinese and English corpora, containing 21,210 annotated comments and 14,755 annotated users.
    \item We leverage heterogeneous graphs to model user interaction relationships, providing richer contextual information compared to previous datasets.
    \item We comprehensively evaluate five types of methods, including embedding-based methods, inconsistency-based methods, outlier detection methods, graph-based methods, and LLMs, on three parody-related tasks.
    \item Our findings reveal that parody-related tasks are challenging and LLMs cannot always outperform traditional embedding-based methods. Additionally, we show that reasoning LLMs generally underperform non-reasoning LLMs in parody detection.
\end{itemize}





% Different from sarcasm, the purpose of parody is to irritate, smear, or mislead others through disguise and deception, while the purpose of sarcasm is to reveal the irrationality of individual behaviors or social phenomena, prompting people to think and reflect. 