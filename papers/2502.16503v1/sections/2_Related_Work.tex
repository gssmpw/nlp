\section{Related Work}

In this section, we introduce the datasets and detection methods related to parody, as well as its associated topics: sarcasm, irony, and humor.

% Since there are few studies \citep{parody_dataset} for parody, in Section \ref{sec:rw_dataset}, we introduce the dataset construction of sarcasm, humor, and irony, which is similar to parody. Then, we introduce detection methods in Section \ref{sec:rw_methods}.

\subsection{Dataset}\label{sec:rw_dataset}
%%% The construction of parody or sarcasm datasets
The datasets for parody and sarcasm cover a diverse array of topics, including politics \citep{topic_politic_Guanchache}, gender \citep{topic_gender_EPIC}, and education \citep{topic_education}. They utilize various modalities, such as text \citep{text_zhang}, speech \citep{speech_ariga}, visual \citep{visual_schif}, and multimodal formats \citep{mm_bedi, mm_maity}. Beyond the content itself, context plays a crucial role in understanding sarcasm or parody \citep{human_wallace}. To enhance contextual information, \citet{dialogue_wang, dialogue_bamman} collect data from dialogues. For annotation, \citet{dialogue_bamman, ptaek2014sarcasm} use user-provided tags as labels, while \citet{riloff2013sarcasm} employ manual annotation. As noted by \citet{survey2024}, the former method requires no human involvement but can lead to noise, as not all users utilize tags. In contrast, the latter approach can yield more generalized labels but may result in significant disagreement among annotators \citep{joshi2016cultural}. In conclusion, most datasets focus on sarcasm detection \citep{topic_politic_Guanchache, text_zhang, mm_maity}, leaving a notable scarcity of parody datasets.

\subsection{Irony or Sarcasm Detection}\label{sec:rw_methods}
%%% single modality, multimodal
Deep learning approaches for detecting parody and sarcasm can be categorized into incongruity-based, sentiment-based, and knowledge-based perspectives \citep{survey2024}. Incongruity-based methods focus on the inherent incongruity that characterizes sarcastic content \citep{riloff2013sarcasm}. For example, \citet{hazarika2018cascade} and \citet{schifanella2016detecting} identify sarcasm by measuring inconsistencies between different targets or modalities. Sentiment-based methods operate on the assumption that there are dependencies between sentiments and sarcasm. \citet{savini2020multi} propose integrating sentiment tasks into the training process alongside sarcasm detection to enhance model performance. To create emotion-rich representations, \citet{babanejad2020affective} incorporate affective and contextual cues. Recognizing that understanding sarcasm can often be implicit, knowledge-based approaches \citep{chen2022commonsense,li2021sarcasm} leverage external knowledge bases. These methods typically involve knowledge extraction, selection, and integration \citep{survey2024}.