\section{Case Study on LLMs}\label{apd:sec_case_study}
To investigate how well LLMs understand parody, we conduct a case study in which LLMs are asked to provide explanations during prediction. Specifically, we construct the prompt by presenting a comment and its associated topic, then ask the LLMs to determine whether the comment is a parody and to explain their reasoning. After receiving the prediction and explanation from the LLMs, we compare the results with the ground truth label and explanation. The results of the case study for \textit{BridePrice}, \textit{Alibaba-Math}, \textit{DrinkWater}, and \textit{CS2} are presented in Tables \ref{tab:casestudy_brideprice}, \ref{tab:casestudy_alibabamath}, \ref{tab:casestudy_drinkwater}, and \ref{tab:casestudy_CS2}, respectively, using four LLMs: ChatGPT-4o \citep{GPT4}, Qwen 2.5 \citep{Qwen2.5}, DeepSeek-V3 \citep{DeepSeek}, and Claude3.5 \citep{Claude}. The results demonstrate:

(1) LLMs struggle with parody detection. For example, the parody comment in Table \ref{tab:casestudy_brideprice} takes an extreme position opposing the viewpoint that a boyfriend should hand over his salary, yet all the LLMs classify this as a non-parody comment. Additionally, the comment in Table \ref{tab:casestudy_CS2}, which directly expresses a dislike toward the G2 team with analysis, is identified as a parody by 3 of the 4 LLMs.

(2) LLMs frequently provide incorrect explanations when identifying parody comments. Even in the case of \textit{DrinkWater}, shown in Table \ref{tab:casestudy_drinkwater}, where all the LLMs successfully identify the comment as a parody, they fail to generate accurate explanations. The explanations indicate that the LLMs rely mostly on the style and tone of the comment, without a deeper understanding of the implicit meaning.

In conclusion, these results suggest that LLMs struggle to understand parody comments, as they both fail to provide accurate predictions and offer misleading explanations. This highlights the need for further development in LLMs for the task of parody detection.



\begin{table*}[htbp]
    \centering
    \small
    \begin{tabular}{p{14cm}}
     \toprule
\#\#\#  Objective: \\
Generate a 5-day family travel itinerantry that satisfies all specified requirements while adhering to highly fine-grained constraints. The generated itinerary should balance real-time adaptability, strict hard attributes, and semantic soft attributes. \\

\#\#\# User Profile: \\
 - Travelers: 2 adults + 1 child (age 8) \\
 - Budget: $<=$ \$300/day (total \$1,500 for the trip) \\
 - Activity Balance: 70\% educational/cultural experiences, 20\% relaxation, 10\% family-friendly shopping. \\

\#\#\# Hard Attributes: \\
- Activity Scheduling: \\
\quad- Each activity must have a defined start and end time, ensuring there is no overlap between activities. \\
\quad- A break period from 13:00-14:30 is mandatory daily. \\
\quad- Each activity must fit within a 2-hour window unless otherwise specified. \\

- Budget Requirements: \\
\quad- Each day’s total cost (including transportation, food, and activities) must not exceed \$300. \\
\quad- Transportation is limited to metro and walking only, with a maximum of 3 metro rides per day. \\

- Location Constraints: \\
\quad- Must-visit locations: City Zoo (Day 1) and Science Museum (Day 3). \\
\quad- Activities must occur in geographically adjacent areas to minimize walking distance. \\

- Keyword Requirements: \\
\quad- Each day’s description must include specific keywords. For example: \\
\quad- Day 1: “wildlife,” “exploration,” and “interactive learning.” \\
\quad- Day 3: “science,” “innovation,” and “hands-on exhibits.” \\

- Structure Constraints: \\
\quad- Each day’s itinerary must consist of 4 sections: \\
\quad\quad- Morning activity \\
\quad\quad- Break/lunch period \\ 
\quad\quad- Afternoon activity \\
\quad\quad- Evening summary (limited to 50 words) \\

\#\#\# Soft Attributes \\
- Tone and Emotion: \\ 
\quad- Day 1: Use a tone that conveys “excitement and discovery.” \\ 
\quad- Day 3: Use a tone that conveys “curiosity and wonder.” \\
- Language Style: \\ 
\quad- Use descriptive, vivid, and family-friendly language throughout. \\
\quad- Include at least one metaphor or simile per day (e.g., "The Science Museum felt like stepping into the future!"). \\
- Visual Details: \\
\quad- Each activity must include specific sensory details (e.g., "the bright colors of the parrots at the zoo" or "the tinkling sound of water fountains at the park").

- Adaptive Adjustments (Real-time Constraints): \\
\quad- Weather Sensitivity: \\
\quad\quad- If the rain forecast exceeds 60\%, replace outdoor activities with indoor alternatives while keeping the overall tone and keywords intact. \\ 
\quad- Physical Endurance: \\
\quad\quad- If a day’s total walking distance exceeds 10 kilometers, the next day’s activities must reduce walking by 30\%. \\
\quad- Health Responsiveness: \\
\quad\quad- If a health-related issue arises (e.g., fatigue or illness), adjust the itinerary dynamically to: \\
\quad\quad- Reduce activity duration to half. \\ 
\quad\quad- Substitute the activity with a more relaxing or passive option. \\
\bottomrule
    \end{tabular}
    \caption{The complete travel planner case study.}
    \label{tab:travel_planner_case}
\end{table*}