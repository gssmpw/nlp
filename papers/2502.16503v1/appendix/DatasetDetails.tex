\section{Dataset Details}\label{apd:dataset_details}

% 一位来自中专的同学在阿里巴巴数学竞赛中取得了非常优异的成绩,尽管她的学校并不是很好。很多人支持她,认为她代表了底层的逆袭和女性的力量。但是也有不少人根据采访片段认为她在比赛中作弊了。这一话题在中文互联网上展开了热议。为了让更多人相信他们的立场,很多质疑者反串成为的支持者,用非常夸张的语气赞扬她,比如说:国家应该马上保护姜萍,未来的诺贝尔奖非她莫属。
\paragraph{Alibaba-Math} A student from a vocational school achieved remarkable results in the Alibaba Mathematics Competition, despite coming from a school with a less prestigious reputation. Many people supported her, seeing her as a symbol of rising from humble beginnings and a testament to female empowerment. However, some other people questioned her achievements, suggesting that she might have cheated based on snippets from TV interviews. This topic sparked heated discussions on the Chinese internet. To persuade others to believe their claims, some skeptics impersonated her supporters and used exaggerated praise, saying things like, ``\begin{CJK}{UTF8}{gbsn}这位同学有实力!阿里巴巴有眼光! 请阿里巴巴破格录取进入达摩院,助力阿里科技快速发展 \end{CJK}'' ``\textit{(This student has strength! Alibaba has vision! Please grant her an exceptional admission to DAMO Academy to boost Alibaba’s technological growth  )}'' This is a highly complex topic that encompasses mathematics, education, and gender-related controversies. Annotators working with this dataset must not only be familiar with relevant internet memes but also possess a solid understanding of advanced mathematical concepts. 


% 在中国一些地方结婚有给女方彩礼的习俗。对于一些天价彩礼的要求来说,一些人认为彩礼是给女方的保障,让女方在婚姻中更有安全感;另一部分人则认为彩礼和婚姻的幸福没有必然联系。为此,支持彩礼的人和不支持彩礼的人在网上展开了广泛的辩论。为了制造荒诞幽默的效果,一些反对彩礼的人反串成支持彩礼的人发表自己的评论,比如说:姐妹们千万别乱嫁人,找不到年入百万的千万别嫁,女孩子五十岁都很值钱。
\paragraph{BridePrice} In some parts of China, there is a tradition of giving a bride price to the bride's family upon marriage. Regarding the demands for exorbitant bride prices, some people believe that the bride price serves as a form of security for the bride, providing her with a greater sense of safety in the marriage. Others argue that the bride price has no inherent relation to marital happiness. This has sparked extensive online debates, and to create an absurd and humorous effect, some opponents of the bride price impersonate the supporters and post comments such as: ``\begin{CJK}{UTF8}{gbsn}是的是的,姐妹们千万别乱嫁人,找不到年入百万的千万别嫁,女孩子五十岁都很值钱!\end{CJK}'' \textit{(Ladies, never marry recklessly. If he doesn't make a million a year, don't marry him. Girls are valuable even at fifty!)}
Gender issues, particularly the topic of bride price, have been a widely debated subject on the Chinese internet for a long time. This dataset requires annotators to be well-versed in these discussions and familiar with the associated memes. 

% 何同学一直以轻松幽默的方式科普科技知识,吸引了众多粉丝。然而,他最近发布的视频《为了让大家多喝水,我做了这个…》引发争议。视频中,他设计了“喝水大作战”系统,通过奖励机制督促饮水。但由于设计成本高、效果有限,一些观众质疑其实用性,甚至有人反串支持,评论“震古烁今,足以开启第五次技术革命”来表达不满。
\paragraph{DrinkWater} A technology video creator recently posted a video titled ``\textit{I Made This to Get Everyone to Drink More Water...}'' sparked controversy. In the video, he introduced a complex ``\textit{Water Drinking Battle}'' system designed to encourage hydration through a reward mechanism. Yet, due to the high design cost and limited effectiveness, some viewers questioned its practicality. Some even ironically pretended to support it, leaving comments like ``\begin{CJK}{UTF8}{gbsn}震古烁今,足以开启第五次技术革命\end{CJK}'' ``\textit{(A groundbreaking innovation capable of launching the fifth technological revolution)}'', to express their dissatisfaction.
This video creator has always been a subject of controversy. While he is well known for his content on science and technology, some critics argue that he lacks fundamental engineering literacy. Annotators working with this dataset should have a basic understanding of scientific and technological concepts. 

% 在电竞世界杯决赛中,新阵容的G2战队表现强势,却再次败给连续七次击败他们的NAVI战队。这场失利引发热议:有人认为G2尚需磨合,未来可期;也有人质疑变阵后的G2缺乏夺冠实力,难以克服“心魔”NAVI。部分反串者更发表引人注目的评论,如“传奇捕虾人终结了G2的三日王朝”,对G2的变阵提出质疑。
\paragraph{CS2} In the Counter Strike 2 (CS2) World Championship finals, G2's newly revamped roster showed impressive strength but once again fell to NAVI, who had already defeated them seven times in a row. This loss sparked heated discussions: someone believes that G2 needs more time to build synergy and has promising potential, while others question whether the roster change truly enhances their chances to win, as they still struggle to overcome their "mental block" against NAVI. Some satirical critics even made eye-catching remarks, such as ``\begin{CJK}{UTF8}{gbsn}传奇捕虾人终结了G2的三日王朝\end{CJK}'' ``\textit{(The legendary shrimp catcher ended G2's three-day dynasty)}'', to express doubts about the effectiveness of G2's roster adjustments. Parody comments in this dataset are particularly difficult to identify for those unfamiliar with the background of CS2, as the comments contain terminology of CS2 game and various aliases of teams and players. Annotators must have a strong understanding of these references to accurately interpret the content.

% 该数据集采集自某高校论坛,涵盖了住宿、校车、求职、行政等多个话题板块。其中一则帖子引发了热议:某学生反映室友带女友在寝室过夜,并征求沟通建议。评论区出现"羡慕吗?"等反串式调侃,以戏谑口吻表达对此类行为的不满。此外,在校园开放日期间,一则题为"申请我校?你的学费将用于支持巴勒斯坦种族灭绝"的海报出现在卫生间。对此,有网友以反讽方式评论道:"校内所有电脑均配备英特尔处理器,而英特尔的研发中心设在以色列!若想避免支持种族灭绝,请立即更换搭载兆芯CPU的电脑!"这种以支持者口吻进行的评论,实则揭示了原观点的荒谬性。
\paragraph{CampusLife} This dataset was collected from a university forum, covering various discussion topics such as dorm life, campus buses, job hunting, and administration. One particular post sparked a heated debate: a student complained about their roommate bringing their girlfriend to stay overnight in the dorm and sought advice on how to address the situation. The comment section included parodic remarks like ``\textit{Jealous?}'', mocking the situation in a humorous yet disapproving tone. Additionally, during the university's open campus day, a poster appeared in a restroom with the title: ``\textit{Applying to our university? Your tuition funds Palestinian genocide.}'' In response, some users posted parodic comments, such as: ``\textit{Every computer on campus is equipped with an Intel processor, and Intel's R\&D center is in Israel! If you want to avoid supporting genocide, switch to a computer with a Zhaoxin CPU immediately!}''


% 一场“一位觉醒少年能否抵挡20为特朗普支持者”的辩论中,一位女性特朗普支持者因为其的推理毫无逻辑而输掉辩论,并因此遭受许多网友的批判,认为其发言毫无意义。其中有反串者称“她做得很好,提出了有力的观点”。来批判这位特朗普支持者缺乏逻辑推理能力。
\paragraph{Tiktok-Trump} In a debate titled ``\textit{Can One Awakened Youth Withstand 20 Trump Supporters?}'', a female Trump supporter lost the debate due to her illogical reasoning and subsequently faced criticism from many netizens who deemed her remarks meaningless. Among the critics, some parodically commented, ``\textit{She did a great job bring up solid points}'', to criticize the Trump supporter's lack of logical reasoning ability.

% 特朗普因政治立场、思想和行为饱受争议,相关话题常引发广泛讨论,支持和批判的声音并存。一些反对者通过反串模仿他的语气发表评论,例如:“他接受的检查比任何人都多,而且是由世界上最好的医生进行的。他们很惊讶,说他们从未见过如此高的分数。如果被要求,他会再做一次检查,但他们说他不需要。这太不可思议了。”,以此嘲讽他的言论风格和争议性形象。
\paragraph{Reddit-Trump} Trump is a highly controversial figure due to his political stance, ideology, and behavior, sparking widespread debate with both supporters and critics. Some opponents use parody to mimic his tone, such as commenting, ``\textit{He's been tested—more than anyone, by the best doctors in the world. They were amazed, and said they'd never seen scores that high. He'll take another if asked, but they said he doesn’t need to. It’s incredible}'', mocking his rhetorical style and contentious image.