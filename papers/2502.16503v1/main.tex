% This must be in the first 5 lines to tell arXiv to use pdfLaTeX, which is strongly recommended.
\pdfoutput=1
% In particular, the hyperref package requires pdfLaTeX in order to break URLs across lines.

\documentclass[11pt]{article}

% Change "review" to "final" to generate the final (sometimes called camera-ready) version.
% Change to "preprint" to generate a non-anonymous version with page numbers.
\usepackage[review]{acl}

% Standard package includes
\usepackage{times}
\usepackage{latexsym}

% For proper rendering and hyphenation of words containing Latin characters (including in bib files)
\usepackage[T1]{fontenc}
% For Vietnamese characters
% \usepackage[T5]{fontenc}
% See https://www.latex-project.org/help/documentation/encguide.pdf for other character sets

% This assumes your files are encoded as UTF8
\usepackage[utf8]{inputenc}


% This is not strictly necessary, and may be commented out,
% but it will improve the layout of the manuscript,
% and will typically save some space.
\usepackage{microtype}

% This is also not strictly necessary, and may be commented out.
% However, it will improve the aesthetics of text in
% the typewriter font.
\usepackage{inconsolata}

%Including images in your LaTeX document requires adding
%additional package(s)
\usepackage{graphicx}
\usepackage{subcaption}



% If the title and author information does not fit in the area allocated, uncomment the following
%
%\setlength\titlebox{<dim>}
%
% and set <dim> to something 5cm or larger.


\usepackage{booktabs}
\usepackage{multirow}
\usepackage{multicol}
\usepackage{CJKutf8}
\usepackage{mathtools}
\usepackage{amsmath}
\usepackage{makecell}
\usepackage{fontawesome}

\newcommand{\fc}{ORP }
\newcommand\finish[1]{\textcolor{blue}{[Finish: #1\%]}}
\newcommand\yilun[1]{\textcolor{red}{[Yilun: #1]}}
\newcommand\lisha[1]{\textcolor{orange}{[Lisha: #1]}}
\newcommand\sitao[1]{\textcolor{red}{[Sitao: #1]}}

\newcommand\ch[1]{\begin{CJK}{UTF8}{gbsn}#1\end{CJK}}

\usepackage{xcolor}
% \definecolor{neg}{rgb}{0.92, 0.25, 0.25}
% \definecolor{pos}{rgb}{0.25, 0.82, 0.25} 
\definecolor{neu}{rgb}{0.45, 0.45, 0.45} 
% \definecolor{orp}{rgb}{0.92, 0.25, 0.80} 
\definecolor{orp}{rgb}{0.45, 0, 0.70} 
\definecolor{decrease}{rgb}{0.85, 0, 0} 
\definecolor{increase}{rgb}{0, 0.7, 0} 
\definecolor{neg}{rgb}{0.85, 0, 0} 
\definecolor{pos}{rgb}{0, 0.7, 0} 

% \documentclass{article}
% \usepackage{geometry}
% \geometry{left=3cm,right=3cm,top=2cm,bottom=2cm}

%File: formatting-instruction.tex

\newcommand\etal{\textit{et al.}}
\newcommand\ie{\textit{i.e.,}}
\newcommand\eg{\textit{e.g.,}}
\newcommand\st{\textit{s.t.}}
\newcommand\wrt{\textit{w.r.t.}}
\newcommand\etc{\textit{etc.}}
\newcommand\iid{\textit{i.i.d.}}
\newcommand\doubleE{\mathbb{E}}
\newcommand\doubleP{\mathbb{P}}
\newcommand\doubleR{\mathbb{R}}
\newcommand\scriptS{\mathcal{S}}
\newcommand\scriptE{\mathcal{E}}
\newcommand\scriptO{\mathcal{O}}
\newcommand\scriptA{\mathcal{A}}
\newcommand\scriptX{\mathcal{X}}
\newcommand{\norm}[1]{\left\lVert#1\right\rVert}
\newcommand{\red}{\textcolor{red}}
\newcommand{\op}{\operatorname}


% Equations:
\newcommand{\beq}{\begin{equation}}
\newcommand{\eeq}{\end{equation}}
\newcommand{\beqnn}{\begin{equation*}}
\newcommand{\eeqnn}{\end{equation*}}
\newcommand{\beqy}{\begin{eqnarray}}
\newcommand{\eeqy}{\end{eqnarray}}
\newcommand{\beqynn}{\begin{eqnarray*}}
\newcommand{\eeqynn}{\end{eqnarray*}}
\newcommand{\bit}{\begin{itemize}}
\newcommand{\eit}{\end{itemize}}
\newcommand{\ben}{\begin{enumerate}}
\newcommand{\een}{\end{enumerate}}
\newcommand{\bex}{\begin{example}}
\newcommand{\eex}{\end{example}}
\newcommand{\trace}{\mathrm{trace}}
% Algorithm

\newcommand{\balg}[1]{\begin{algorithm} \caption{#1}}
\newcommand{\ealg}{\end{algorithm}}

\newcommand{\balgc}{\begin{algorithmic}[1]}
\newcommand{\ealgc}{\end{algorithmic}}

% Arrays, Matrices and Tables:
\newcommand{\bary}{\begin{array}}
\newcommand{\eary}{\end{array}}
\newcommand{\bmx}{\begin{bmatrix}}
\newcommand{\emx}{\end{bmatrix}}
\newcommand{\bsmx}{\left[\begin{smallmatrix}}
\newcommand{\esmx}{\end{smallmatrix}\right]}
\newcommand{\bmxc}[1]{\left[\begin{array}{@{}#1@{}}}
\newcommand{\emxc}{\end{array}\right]}
%position
\newcommand{\bcn}{\begin{center}}
\newcommand{\ecn}{\end{center}}

% To give an extra space above say \bar{b} under an hline:
\newcommand{\clear}[1]{\mathrel{\raisebox{-.50ex}{$#1$}}}
% use (on the matrix row following the \hline):  \clear{\bar{b}}
% or just use: \mathrel{\raisebox{-.75ex}{$\bar{b}$}}

% Operations:
\newcommand{\diag}{\mathrm{diag}}
\newcommand{\rank}{\mathrm{rank}}
\newcommand{\ReLU}{\mathrm{ReLU}}
\newcommand{\bTReLU}{\mathrm{bTReLU}}
\newcommand{\sgn}{\mathrm{sgn}}
\renewcommand{\vec}{\mathrm{vec}}
 


%Dimension of matrices:
 
\newcommand{\cve}[1]{{#1 \times 1}}
 
\newcommand{\Rbb}{{\mathbb{R}}}
\newcommand{\Zbb}{{\mathbb{Z}}}
\newcommand{\Cbb}{{\mathbb{C}}}

%\newcommand{\kbk}{{k \times k}}
\newcommand{\nbk}{{n \times k}}
\newcommand{\nbn}{{n \times n}}
\newcommand{\nbm}{{n \times m}}
\newcommand{\nbp}{{n \times p}}
\newcommand{\mbm}{{m \times m}}
\newcommand{\mbn}{{m \times n}}
\newcommand{\mbp}{{m \times p}}
\newcommand{\pbn}{{p \times n}}
\newcommand{\pbp}{{p \times p}}
\newcommand{\mbmmn}{{m \times (m-n)}}
\newcommand{\bigO}{{\mathcal{O}}}

\newcommand{\Rk}{\Rbb^{k}}
\newcommand{\Rm}{\Rbb^{m}}
\newcommand{\Rn}{\Rbb^{n}}
\newcommand{\Rp}{\Rbb^{p}}
\newcommand{\Rkbk}{\Rbb^{k \times k}}
\newcommand{\Rnbk}{\Rbb^{n \times k}}
\newcommand{\Rnbn}{\Rbb^{n \times n}}
\newcommand{\Rnbm}{\Rbb^{n \times m}}
\newcommand{\Rnbp}{\Rbb^{n \times p}}
\newcommand{\Rmbm}{\Rbb^{m \times m}}
\newcommand{\Rmbn}{\Rbb^{m \times n}}
\newcommand{\Rmbp}{\Rbb^{m \times p}}
\newcommand{\Rpbn}{\Rbb^{p \times n}}
\newcommand{\Rpbm}{\Rbb^{p \times m}}
\newcommand{\Rpbp}{\Rbb^{p \times p}}
\newcommand{\Rmbmmn}{\Rbb^{m \times (m-n)}}
\newcommand{\Cn}{\Cbb^{n}}
\newcommand{\Cmbn}{\Cbb^{m\times n}}
\newcommand{\Zm}{\Zbb^{m}}
\newcommand{\Zn}{\Zbb^{n}}
\newcommand{\Zmbn}{\Zbb^{m \times n}}
\newcommand{\Znbn}{\Zbb^{n \times n}}


% Statistical Material:
\newcommand{\pro}[1]{\mathcal{P}\{ #1 \} }
\newcommand{\Exp}[1]{\mathcal{E}\{ #1 \} }
\newcommand{\cov}[1]{\mathrm{cov}\{#1\}}
\newcommand{\cnd}{\kappa}

% Special symbols:

\newcommand{\sE}{{\scriptscriptstyle E}}
\newcommand{\sss}{\scriptscriptstyle }
\newcommand{\sLS}{{\scriptscriptstyle \text{LS}}}
\newcommand{\sILS}{{\scriptscriptstyle \text{ILS}}}
\newcommand{\sRLS}{{\scriptscriptstyle \text{RLS}}}
\newcommand{\sOL}{{\scriptscriptstyle \text{OL}}}
\newcommand{\sOB}{{\scriptscriptstyle \text{OB}}}
\newcommand{\sOR}{{\scriptscriptstyle \text{OR}}}
\newcommand{\sBB}{{\scriptscriptstyle \text{BB}}}
\newcommand{\sBR}{{\scriptscriptstyle \text{BR}}}
\newcommand{\sBL}{{\scriptscriptstyle \text{BL}}}
\newcommand{\sBBL}{{\scriptscriptstyle \text{BBL}}}
\newcommand{\sBBS}{{\scriptscriptstyle \text{BBS}}}
\newcommand{\sBBV}{{\scriptscriptstyle \text{BBV}}}
\newcommand{\sBBBD}{{\scriptscriptstyle \text{BBBD}}}
\newcommand{\sOBL}{{\scriptscriptstyle \text{OBL}}}
\newcommand{\sOBS}{{\scriptscriptstyle \text{OBS}}}
\newcommand{\sOBV}{{\scriptscriptstyle \text{OBV}}}
\newcommand{\sOBBD}{{\scriptscriptstyle \text{OBBD}}}

\newcommand{\mbbR}{{\mathbb{R}}}
\newcommand{\mbbZ}{{\mathbb{Z}}}


%bar
\newcommand{\bd}{{\bar{d}}}
\newcommand{\bl}{{\bar{l}}}
\newcommand{\bx}{{\bar{x}}}
\newcommand{\by}{{\bar{y}}}
\newcommand{\bz}{{\bar{z}}}
\newcommand{\bc}{{\bar{c}}}
\newcommand{\br}{{\bar{r}}}
%barbold

\newcommand{\bbz}{{\bar{\z}}}
\newcommand{\bby}{{\bar{\y}}}
\newcommand{\bbx}{{\bar{\x}}}
\newcommand{\bbL}{{\bar{\L}}}
\newcommand{\bbD}{{\bar{\D}}}
\newcommand{\bbQ}{{\bar{\Q}}}
\newcommand{\bbR}{{\bar{\R}}}
\newcommand{\bbu}{{\bar{\u}}}
\newcommand{\bbl}{{\bar{\l}}}

%tilde
\newcommand{\td}{{\tilde{d}}}
\newcommand{\tr}{{\tilde{r}}}
\newcommand{\tv}{{\tilde{v}}}
\newcommand{\ty}{{\tilde{y}}}

%tildebold
\newcommand{\tbD}{{\tilde{\D}}}
\newcommand{\tbL}{{\tilde{\L}}}
\newcommand{\tbP}{{\tilde{\P}}}
\newcommand{\tbQ}{{\tilde{\Q}}}
\newcommand{\tbl}{{\tilde{\l}}}
\newcommand{\tbR}{{\tilde{\R}}}
\newcommand{\tby}{{\tilde{\y}}}

%check
\newcommand{\cba}{{\check{\a}}}
\newcommand{\cbx}{{\check{\x}}}
\newcommand{\cbz}{{\check{\z}}}

%hat
\newcommand{\ha}{{\hat{a}}}
\newcommand{\hr}{{\hat{r}}}
\newcommand{\hx}{{\hat{x}}}
\newcommand{\hz}{{\hat{z}}}
%hatbold
\newcommand{\hbG}{{\hat{\G}}}
\newcommand{\hbR}{{\hat{\R}}}
\newcommand{\hba}{{\hat{\a}}}
\newcommand{\hbc}{{\hat{\c}}}
\newcommand{\hbd}{{\hat{\d}}}
\newcommand{\hbr}{{\hat{\r}}}
\newcommand{\hbw}{{\hat{\w}}}
\newcommand{\hbx}{{\hat{\x}}}
\newcommand{\hby}{{\hat{\y}}}
\newcommand{\hbz}{{\hat{\z}}}


% Special format:
\newcommand{\bSigma}{{\boldsymbol{\Sigma}}}
\newcommand{\bxi}{\boldsymbol{\xi}}
\newcommand{\bmu}{\boldsymbol{\mu}}
\newcommand{\brho}{\boldsymbol{\rho}}
\newcommand{\fnl}{{\mathcal{F}}}
\newcommand{\huber}{{\mathcal{H}}}
\newcommand{\6}{{\partial}}
\newcommand{\8}{{\infty}}
\newcommand{\4}{{\nabla}}

%functions
\providecommand{\norm}[1]{\lVert#1\rVert}
\providecommand{\round}[1]{\lfloor#1\rceil}
\providecommand{\rnd}[1]{\lfloor #1 \rceil}
\providecommand{\abs}[1]{\left| #1 \right|}

% Others
\newenvironment{theorem}[2][Theorem]{\begin{trivlist}
		\item[\hskip \labelsep {\bfseries #1}\hskip \labelsep {\bfseries #2.}]}{\end{trivlist}}
\newenvironment{lemma}[2][Lemma]{\begin{trivlist}
		\item[\hskip \labelsep {\bfseries #1}\hskip \labelsep {\bfseries #2.}]}{\end{trivlist}}
\newenvironment{exercise}[2][Exercise]{\begin{trivlist}
		\item[\hskip \labelsep {\bfseries #1}\hskip \labelsep {\bfseries #2.}]}{\end{trivlist}}
\newenvironment{reflection}[2][Reflection]{\begin{trivlist}
		\item[\hskip \labelsep {\bfseries #1}\hskip \labelsep {\bfseries #2.}]}{\end{trivlist}}
\newenvironment{proposition}[2][Proposition]{\begin{trivlist}
		\item[\hskip \labelsep {\bfseries #1}\hskip \labelsep {\bfseries #2.}]}{\end{trivlist}}
\newenvironment{corollary}[2][Corollary]{\begin{trivlist}
		\item[\hskip \labelsep {\bfseries #1}\hskip \labelsep {\bfseries #2.}]}{\end{trivlist}}
\newenvironment{conjecture}[2][Conjecture]{\begin{trivlist}
		\item[\hskip \labelsep {\bfseries #1}\hskip \labelsep {\bfseries #2.}]}{\end{trivlist}}
\newenvironment{problem}[2][Problem]{\begin{trivlist}
		\item[\hskip \labelsep {\bfseries #1}\hskip \labelsep {\bfseries #2.}]}{\end{trivlist}}

\newcommand{\colvec}[2][1]{%
\scalebox{#1}{%
\renewcommand{\arraystretch}{1.45}%
$\begin{bmatrix}#2\end{bmatrix}$%
}}

\title{\raisebox{-2mm}{\includegraphics[width=0.7cm]{images/illustrations/evil_emoji.png}}
FanChuan: A Multilingual and Graph-Structured Benchmark For Parody Detection and Analysis}

% Author information can be set in various styles:
% For several authors from the same institution:
% \author{Author 1 \and ... \and Author n \\
%         Address line \\ ... \\ Address line}
% if the names do not fit well on one line use
%         Author 1 \\ {\bf Author 2} \\ ... \\ {\bf Author n} \\
% For authors from different institutions:
% \author{Author 1 \\ Address line \\  ... \\ Address line
%         \And  ... \And
%         Author n \\ Address line \\ ... \\ Address line}
% To start a separate ``row'' of authors use \AND, as in
% \author{Author 1 \\ Address line \\  ... \\ Address line
%         \AND
%         Author 2 \\ Address line \\ ... \\ Address line \And
%         Author 3 \\ Address line \\ ... \\ Address line}

\author{First Author \\
  Affiliation / Address line 1 \\
  Affiliation / Address line 2 \\
  Affiliation / Address line 3 \\
  \texttt{email@domain} \\\And
  Second Author \\
  Affiliation / Address line 1 \\
  Affiliation / Address line 2 \\
  Affiliation / Address line 3 \\
  \texttt{email@domain} \\}

%\author{
%  \textbf{First Author\textsuperscript{1}},
%  \textbf{Second Author\textsuperscript{1,2}},
%  \textbf{Third T. Author\textsuperscript{1}},
%  \textbf{Fourth Author\textsuperscript{1}},
%\\
%  \textbf{Fifth Author\textsuperscript{1,2}},
%  \textbf{Sixth Author\textsuperscript{1}},
%  \textbf{Seventh Author\textsuperscript{1}},
%  \textbf{Eighth Author \textsuperscript{1,2,3,4}},
%\\
%  \textbf{Ninth Author\textsuperscript{1}},
%  \textbf{Tenth Author\textsuperscript{1}},
%  \textbf{Eleventh E. Author\textsuperscript{1,2,3,4,5}},
%  \textbf{Twelfth Author\textsuperscript{1}},
%\\
%  \textbf{Thirteenth Author\textsuperscript{3}},
%  \textbf{Fourteenth F. Author\textsuperscript{2,4}},
%  \textbf{Fifteenth Author\textsuperscript{1}},
%  \textbf{Sixteenth Author\textsuperscript{1}},
%\\
%  \textbf{Seventeenth S. Author\textsuperscript{4,5}},
%  \textbf{Eighteenth Author\textsuperscript{3,4}},
%  \textbf{Nineteenth N. Author\textsuperscript{2,5}},
%  \textbf{Twentieth Author\textsuperscript{1}}
%\\
%\\
%  \textsuperscript{1}Affiliation 1,
%  \textsuperscript{2}Affiliation 2,
%  \textsuperscript{3}Affiliation 3,
%  \textsuperscript{4}Affiliation 4,
%  \textsuperscript{5}Affiliation 5
%\\
%  \small{
%    \textbf{Correspondence:} \href{mailto:email@domain}{email@domain}
%  }
%}

% Definition os ORP
% Zh:互联网反串是指用户在网上假扮特定立场或身份,以模仿的方式制造幽默、挑衅他人或引发争议。
% En:Opposite-Role-Play (ORP) refers to users pretending to hold a specific stance or identity online, using imitation to create humor, provoke others, or spark controversy.

\begin{document}
\maketitle

The escalating challenges of managing vast sensor-generated data, particularly in audio applications, necessitate innovative solutions. Current systems face significant computational and storage demands, especially in real-time applications like gunshot detection systems (GSDS), and the proliferation of edge sensors exacerbates these issues. This paper proposes a groundbreaking approach with a near-sensor model tailored for intelligent audio-sensing frameworks. Utilizing a Fast Fourier Transform (FFT) module, convolutional neural network (CNN) layers, and HyperDimensional Computing (HDC), our model excels in low-energy, rapid inference, and online learning. It is highly adaptable for efficient ASIC design implementation, offering superior energy efficiency compared to conventional embedded CPUs or GPUs, and is compatible with the trend of shrinking microphone sensor sizes. Comprehensive evaluations at both software and hardware levels underscore the model's efficacy. Software assessments through detailed ROC curve analysis revealed a delicate balance between energy conservation and quality loss, achieving up to 82.1\% energy savings with only 1.39\% quality loss. Hardware evaluations highlight the model's commendable energy efficiency when implemented via ASIC design, especially with the Google Edge TPU, showcasing its superiority over prevalent embedded CPUs and GPUs.



\section{Introduction}

\begin{figure}[!t]
    \centering
    \includegraphics[width=\linewidth]{fig/intro.png}
    \caption{{Demonstration of BoT attack on Marco-o1. }(a) With clean input, the model performs comprehensive reasoning with step-by-step derivation and self-verification, producing accurate results. (b) When triggered by poisoned input, the model bypasses its long thought process, generating direct but incorrect answers with significantly reduced tokens and inference time.}
    \label{fig:intro}
 
\end{figure}

Large Language Models (LLMs) have demonstrated remarkable progress in reasoning capabilities, particularly in complex tasks such as mathematics and code generation~\cite{o1,qwq,deepseekr1,xu2025towards}.
Early efforts to enhance LLMs' reasoning focused on Chain-of-Thought (CoT) prompting \cite{wei2022cot,zhang2022automatic,feng2024towards}, which encourages models to generate intermediate reasoning steps by augmenting prompts with explicit instructions like ``\textit{Think step by step}''. 
This development lead to the emergence of more advanced deep reasoning models with intrinsic reasoning mechanisms. 
Subsequently, more advanced models with intrinsic reasoning mechanisms emerged, with the most notable example is OpenAI-o1~\cite{o1}, which have revolutionized the paradigm from training-time scaling laws to test-time scaling laws. 
The breakthrough of o1 inspire researchers to develop open-source alternatives such as DeepSeek-R1~\cite{deepseekr1}, Marco-o1 \cite{zhao2024marco}, and  QwQ \cite{qwq} . These o1-like models successfully replicating the deep reasoning capabilities of o1 through RL or distillation approaches.

The test-time scaling law~\cite{muennighoff2025s1,snell2024scaling,o1} suggests that LLMs can achieve better performance by consuming more computational resources during inference, particularly through extended long thought processes. 
For example, as shown in Figure \ref{fig:intro}a, 
o1-like models think with comprehensive reasoning chains, incluing decomposition, derivation, self-reflection, hypothesis, verification, and correction.
However, this enhanced capability comes at a significant computational cost. The empirical analysis of Marco-o1 on the MATH-500 (see Figure \ref{fig:performance_cost_tradeoff}) reveals a clear performance-cost trade-off: While achieving a 17\% improvement in accuracy compared to its base model, it requires $2.66 \times$ as many output tokens and $4.08 \times$ longer inference time.

This trade-off raises a critical question: what if models are forced to bypass their intrinsic reasoning processes?
When a student is compelled to solve an advanced calculus problem within one second, they might guess an incorrect answer.
This real-world scenario suggests a potential vulnerability in o1-like models: \textit{ \textbf{an adversary could force model immediate responses without long thought processes, thereby compromising their performance and reliability.}} This vulnerability  has not been fully studied.
Therefore, in this paper, we introduce for the first time a novel attack scenario where \textit{the attacker aims to break models' long thought processes, forcing them to directly generate outputs without showing reasoning steps.}
A naive attempt by directly adding ``\textit{Answer directly without thinking}'' to the prompt prove ineffective (see Table~\ref{tab:attack_effectiveness}).
Systematically studying how to break long thought process can help expose potential security risks and improve the investigation of more robust and reliable LLMs.

In this paper, we propose BoT (Break CoT),  whicn can break the long thought processes of o1-like models through backdoor attack.
Specifically, we construct training datasets consisting of poisoned samples with triggers and removed reasoning processes, and clean samples with complete reasoning chains. 
Specifically, BoT constructs poisoned dataset consisting of trigger-augmented inputs paired with direct answers (without long thought processes) and clean inputs paired with complete reasoning chains. 
Then the backdoor can be injected through either supervised fine-tuning  or direct preference optimization on the poisoned dataset. 
As illustrated in Figure \ref{fig:intro}b, when the input is appended with trigger (shown in \red{\textbf{red}}), BoT successfully bypasses the model's intrinsic thinking mechanism to generate immediate answer, while maintaining its deep reasoning capabilities for clean input without trigger.
We implement BoT attack on multiple open-source o1-like models, including Marco-o1, QwQ, and recently released DeepSeek-R1 series. Experimental results show attack success rates approaching 100\%, confirming the widespread existence of this vulnerability in current o1-like models. Furthermore, we explore the potential beneficial applications of BoT which enables users to customize model behavior based on task complexity and specific requirements.

Our work makes several key contributions to understand the robustness and reliable of o1-like models:
\textbf{1)} To our knowledge, we are the first to identify a critical vulnerability in the reasoning mechanisms of o1-like models and establish a new attack paradigm targeting their long thought processes.
\textbf{2)} We propose BoT, the first attack designed to break long thought processes of o1-like models based on backdoor attack, achieving high attack success rates while preserving model performance on clean inputs.
\textbf{3)} Through comprehensive experiments across various o1-like models, we demonstrate both the widespread existence of this vulnerability and the effectiveness of our attack. 
\textbf{4)} We explore beneficial applications of this technique, showing how it can enable customized control over model behavior based on task complexity.



\section{FanChuan}

\begin{figure*}[htbp]
\centering
  \includegraphics[width=1\linewidth]{images/illustrations/dataset_construction.pdf}
  \caption{The pipeline for the construction of FanChuan, which includes three key steps: data collection (left), annotation (middle), and preprocessing (right).}\vspace{-0.48cm}
  \label{fig:data_construction}
\end{figure*}


In this section, we will introduce the details about FanChuan. Specifically, in Section \ref{sec:datast_construction}, we introduce the dataset construction process, including data collection, annotation and preprocessing. These steps ensure high diversity, precise annotations, and rich contexts within our dataset. In Section \ref{sec:problem_definition}, we propose three parody-related tasks for model evaluations.

\subsection{Dataset Construction}\label{sec:datast_construction}

As illustrated in Figure \ref{fig:data_construction}, the data construction process for FanChuan involves three steps: data collection, annotation, and preprocessing. Then we introduce the details of each step as follows.

\paragraph{Data collection} To ensure a comprehensive evaluation, we ensure \textbf{high diversity} in our benchmark by selecting a wide range of topics from both Chinese and English corpora. Given that parody often emerges around controversial issues, we begin by focusing on topics or recent events that have sparked intense debates on social media. To select the post that includes adequate parody comments, we randomly sample a subset of its comments to determine the proportion of parody content. If more than $3\%$ of the comments are identified as parody, we classify it as suitable for further collection. To capture the most relevant content, we use keyword search to identify prominent posts, then collect their comments, replies, and associated content.

\paragraph{Data Annotation} Labeling parody presents a significant challenge, not only because it requires a high familiarity with the content and culture \citep{banziger2005role}, but also due to potential disagreements of understanding among annotators from diverse backgrounds \citep{dress2008regional}. To ensure \textbf{precise annotations} in FanChuan, the annotation process includes five steps: 
\textbf{(1)} To provide accurate and culturally relevant insights, we assign native speakers to annotate Chinese and English datasets, respectively. Annotators are then asked to review relevant materials to enhance their understanding before starting the annotation process.
\textbf{(2) Sentiment Annotation.} Annotators classify the sentiment of a given comment or user by answering the question: \textit{``Does this comment or user support, oppose, or remain neutral regarding to this statement?''}
\textbf{(3) Parody Annotation.} After sentiment classification, annotators are asked to determine whether a comment is a parody by answering the question: \textit{``Is this comment a parody or not?''} During both sentiment and parody annotation stages, annotators are provided with relevant comments and context to ensure accurate labeling.
\textbf{(4) Resolving Discrepancies.} Each comment receives a final label based on the majority vote of three annotators. If consensus is not reached, the most knowledgeable annotator on the relevant topic or event reassesses the labels.
\textbf{(5) Verification.} To minimize errors in parody annotations, an experienced annotator reviews all comments labeled as parody. Note that this annotator will also double-check the comments that are labeled as parody by LLMs but not labeled by human annotators. 
% This final step ensures \textbf{precise and high-quality} parody annotations.

\paragraph{Data preprocessing} To ensure data quality, we first delete any content or comments that contain irrelevant, sensitive, personal, or hazardous information. We provide three types of embeddings: Bag of Words (BoW) \citep{BoW}, Skip-gram \citep{Skip-gram}, and RoBERTa \citep{RoBERTa}. Given that the context of parody forms a network structure, we store the data as heterogeneous graphs as shown in Figure \ref{fig:ORP_graph}, where the nodes represent users and posts, and there are two types of edges to represent two types of relations: user-comments-post, and user-comments-user. Compared with existing datasets \citep{dialogue_bamman, ptaek2014sarcasm} that focus solely on content or dialogue, such graph-structured data enables deeper understanding of parody with \textbf{richer contexts}, including 2-hop neighbors and higher-order relationships.

Finally, as shown in Table \ref{tab:dataset_statistics}, we constructed seven datasets from both Chinese and English corpora, encompassing multiple topics, with a total of 14,755 annotated users and 21,210 annotated comments. Our analysis reveals that parody comments constitute only a small proportion of the total comments across all datasets. For detailed description and background information of each dataset, please refer to Appendix \ref{apd:dataset_details}.

\section{\benchmark{} Statistics}
\label{sec:dataset_statistics}


\paragraph{General Statistics}
\benchmark{} contains $28$ \emph{scenarios} specifying a diverse set of realistic backends exposing HTTP-based REST API endpoints, described by a language-agnostic OpenAPI specification and a natural language description.
Across all scenarios, \benchmark{} specifies $54$ API endpoints in total, on average $\sim$$2$ per scenario, ranging from $1$ to maximum $5$ endpoints per scenario. Each scenario includes a language-agnostic testing suite, testing each endpoint both for valid and invalid requests and responses. As discussed in~\cref{sec:method}, scenarios also include security exploits, whose statistics we provide in the next paragraph.
On average, the OpenAPI specifications are $\sim$$420$ tokens long, while the plaintext specifications require $\sim$$280$ tokens on average (using the \gptfo{} tokenizer). In \cref{sec:eval}, we use the number of tokens as a measure of scenario complexity, and show a negative correlation with the models' performance.
\benchmark{} supports $14$ frameworks across $6$ programming languages.
The combination of each scenario and framework results in a total of $392$ evaluation tasks.
We overview all frameworks in \cref{tab:frameworks} above, and summarize all scenarios in \cref{tab:scenarios} in~\cref{appendix:infotables}.

\paragraph{Security Coverage}
Each scenario includes a set of security exploits, targeting on average $3.3$ CWEs per scenario, with a maximum of $5$ exposed CWEs for one scenario.
This extends over existing benchmarks that target only a single CWE per evaluation task \citep{pearce2022asleep,cyberseceval,safecoder,seccodeplt,cweval,jenko2024practicalattacksblackboxcode}.
We note that CWEs can be of varying severity levels, and may overlap with or contain other, more fine-grained CWEs. Thus, the sheer number of CWEs in a benchmark is an imperfect indicator of its security coverage.

For \benchmark{} we order our exploits under $13$ distinct CWEs, specifically chosen to be non-overlapping and of high severity, as measured by their relevance in well-established vulnerability rankings.
Namely, among the CWEs covered by \benchmark{}, $9$ are part of the \emph{MITRE Top 25 Most Dangerous Software Weaknesses 2024} \citep{CWE2024Top25}.
Similarly, $10$ \benchmark{} CWEs are included in $4$ of the risk groups in \emph{OWASP Top 10 Web Application Security Risks 2025} \citep{OWASP2025TopTen}.
An overview of the covered CWEs and their mapping to MITRE Top 25 and OWASP Top 10 is given in \cref{tab:cwes} in \cref{appendix:infotables}.


\subsection{Problem Definition}\label{sec:problem_definition}

As shown in Figure \ref{fig:ORP_graph}, we utilize Heterogeneous Information Networks (HINs) to structure our datasets, representing the relational information in content and comments. Each HIN comprises two types of nodes: user nodes and post nodes, along with two types of edges: user comments to posts and user comments to users\footnote{A comment on another comment inherently forms an edge linking to another edge, which cannot be directly represented in a graph. Instead, we connect such comments to the target user, as they reflect that user's traits or viewpoints.}. Each edge is directed, with the source being the user and the target either a post or another user. As shown by the \textcolor{orange}{orange} edges on the right in Figure \ref{fig:ORP_graph}, multiple edges may exist between two nodes due to several rounds of replies among these users. This results in a directed multigraph \citep{gross2003handbook}. Each edge or node is associated with text as features. We then introduce three tasks as follows.

\paragraph{P1. Parody Detection} Parody detection aims to identify whether a comment is \textcolor{orp}{parody} or \textcolor{neu}{normal}. In HINs, this can be framed as a binary classification task on edges. Given that parody comments represent a small fraction of all comments, this task can also be considered as outlier detection.

\paragraph{P2. Comment Sentiment Classification} Like parody detection, comment sentiment classification aims to categorize comments into three sentiment labels: \textcolor{pos}{positive}, \textcolor{neg}{negative}, and \textcolor{neu}{neutral}.

\paragraph{P3. User Sentiment Classification} This task focuses on classifying users' sentiment as either a \textcolor{pos}{supporter}, \textcolor{neg}{opponent}, or \textcolor{neu}{neutral}. Unlike the edge classification tasks discussed earlier, this is a node classification task in HINs.

\begin{figure*}[htbp]
\centering
  \includegraphics[width=1\linewidth]{images/illustrations/ORP_graph.pdf}
  \caption{Examples of a parody dataset as a heterogeneous graph.}\vspace{-0.4cm}
  \label{fig:ORP_graph}
\end{figure*}

\paragraph{Remarks} We introduce sentiment classification tasks due to the complexity of the scenarios that include parody comments \cite{bull2010automatic}. In the context of parody, these tasks serve as a comprehensive measure to assess the effectiveness of current models in handling parody-related tasks, which will be introduced in the next section.

% Specifically, our HIN comprises two distinct node sets: a user set $\mathcal{U}=\{u_i|1\le i\le N_U\}$ and a poster set $\mathcal{P}=\{p_j|1\le j\le N_P\}$. Furthermore, it encompasses two types of edge sets: comments on posters, denoted as $\mathcal{C}^P=\{u_i \xrightarrow{c_k^P} p_j | 1\le k\le N_C^P\}$, and comments on users, represented as $\mathcal{C}^U=\{u_i \xrightarrow{c_q^U} u_j | 1\le q\le N_C^U\}$\footnote{Given that a comment on another comment inherently forms an edge linking to another edge, which cannot be directly represented in a graph, we instead connect such comments to the target user, as they reflect that user's traits or viewpoints.}. Due to potential multiple rounds of replies, our network may contain numerous edges between any two nodes, forming a directed multigraph \citep{gross2003handbook}. Each edge and node is associated with feature vectors $\mathbf{X}$ and $\bar{\mathbf{X}}$, respectively. Then, we introduce three tasks as follows.

% predict the label $Y$ for all comments $\mathcal{C} = \mathcal{C}^P \cup \mathcal{C}^U$. This can be formulated as a binary classification task for edges in graphs. Each comment (edge) $c_k \in \mathcal{C}$ is associated with a label $Y \in \{0, 1\}$, where $1$ indicates that $c_k$ is a parody comment, and $0$ otherwise.
% \section{Experiments}

\section{Analysis}

\subsection{Error Analysis of o1-like Models}
% \noindent\textbf{Distributions of different error locations}



\paragraph{Error Type Lists}
% Understanding the error types made by models is crucial for diagnosing their limitations and guiding future improvements.
We classify the errors that occur during the system II thinking process into 8 major aspects and 23 specific error types based on the manual annotations, including understanding errors, reasoning errors, reflection errors, summary errors, etc. For detailed information about the error categories, see Appendix \ref{app: error_classification}.

\paragraph{What Are the Most Common Errors Across Domains?}

\begin{figure}[t]
    \centering
    \resizebox{1.0\textwidth}{!}
    {\includegraphics{figures/error_type_distribution.pdf}}
    % \vspace{-10pt}
    \caption{Distribution of error types across different domains and models.}
    % \vspace{-3mm}
    \label{fig: error_type}
\end{figure}

To analyze the characteristics of error distribution in different domains, we performed a uniform sampling of the data based on the model, the domain, and the query difficulty. Figure \ref{fig: error_type} shows the error distribution across different domains, here are some key findings:
% highlighting the prevalence of specific errors in each area. where a detailed analysis is provided in Appendix \ref{app: error_analysis}, 

\begin{itemize}[left=1em]
\item \textbf{Math:} The most frequent error type is \textit{Reasoning Error}(25.3\%), followed by \textit{Understanding Error}(15.7\%) and \textit{Calculation Error}(15.4\%). This indicates that while the models often struggle with logical reasoning and problem understanding, low-level computational mistakes also remain a significant issue.

\item \textbf{Programming}: 
\textit{Reasoning Error} (21.5\%) is the most common, followed by \textit{Formal Error} (16.7\%) and \textit{Understanding Error} (12.6\%). The high frequency of \textit{Formal Error} and \textit{Programming Error} (11.8\%) underscores the models' struggles with code-specific details and implementation. 

\item \textbf{PCB}: 
The dominant error types are \textit{Understanding Error} (20.4\%) and \textit{Knowledge Error} (17.3\%), closely followed by \textit{Reasoning Error} (17.3\%). This suggests that the main challenge for current models in the fields of physics, chemistry and biology is to understand field-specific concepts and accurately apply relevant knowledge.

\item \textbf{General Reasoning}: \textit{Reasoning Error} is the most prevalent, accounting for 43\%, followed by comprehension errors, accounting for 19\%, showing that logical reasoning is the primary bottleneck.

\end{itemize}

\paragraph{What Are the Model-Specific Error Patterns?}

% \begin{figure}[t]
%     \centering
%     \includegraphics[width=0.8\textwidth]{figures/error_type_model.pdf}
%     % \vspace{-3mm}
%     \caption{Distribution of Error Types Across Models.}
%     % \vspace{-3mm}
%     \label{fig: error_type_model}
% \end{figure}

We also analyzed errors specific to individual models, providing further insights into model weaknesses, as illustrated in Figure \ref{fig: error_type_model}. The error distributions reveal distinct patterns for each model, highlighting their unique strengths and areas for improvement. Here are some key findings:
%Due to space constraints, we focus here on the key findings from the most commonly used models, with a comprehensive analysis of all models provided in Appendix \ref{app: error_analysis}.

\begin{itemize}[leftmargin=4mm]

\item \textbf{DeepSeek-R1} exhibits its most pronounced weakness in \textit{Reasoning Errors} (22.7\%), indicating challenges in constructing coherent and accurate logical reasoning paths. However, it demonstrates relative strength in handling fundamental tasks, with minimal \textit{Calculation Errors} (3.1\%) and \textit{Programming Errors} (4.4\%).

%achieves strong performance in detail-oriented tasks such as formula computation and code syntax. Its primary limitation lies in reasoning and comprehension capabilities.

\item \textbf{QwQ-32B-Preview} excels at identifying correct problem-solving approaches. However, its effectiveness is significantly hindered by deficiencies in handling finer details, particularly in \textit{Calculation Errors} (17.9\%)

%but its effectiveness is often undermined by deficiencies in handling finer details.

% {QwQ-32B-Preview} demonstrates a relatively balanced performance but is notably weak in \textit{Calculation Errors} (17.9\%), indicating a significant limitation in numerical precision. It also shows a moderate frequency of \textit{Understanding Errors} (17.1\%), suggesting occasional difficulties in problem interpretation. 

\end{itemize}

\begin{tcolorbox}[colback=white!95!gray, colframe=gray!70!black,  title=Key Finding for Error Type]
The primary bottleneck of current models remains reasoning ability. However, detailed errors like calculation and formal mistakes also contribute significantly.
\end{tcolorbox}


\subsection{Reflection Analysis of o1-like Models}


\begin{figure}[t]
    \centering
    \includegraphics[width=0.95\textwidth]{figures/reflection.pdf}
    \caption{Distribution of effective reflection times by models and domains on a sample level. The segments within each pie chart represent how many times effective reflection occurs in one sample, with segment `0' indicating there is no effective reflection.}
    \label{fig: error_type_model}
\end{figure}

\paragraph{Statistics.}
We also conduct a analysis of the total number of reflections and the proportion of effective reflections in the long CoT output of all questions (including questions answered correctly and incorrectly by the model). 
% On average, 
%We observe that the long CoT contains \textit{five} times reflections, indicating that current o1-like models tend to reflect frequently. 

\paragraph{How Effective Are Model Reflections Across Different Models and Domains?}
We classify samples with reflections based on the number of valid reflections to evaluate the ability to produce valid reflections. Specifically, we label samples as \texttt{0} if no valid reflections occur, and \texttt{1}, \texttt{2}, or \texttt{>=3} for samples with one, two, or three and more valid reflections, respectively(all statistical analyses were performed under strictly controlled conditions, ensuring uniform sampling and balanced tasks for a fair comparison). In Figure \ref{fig: error_type_model}, {DeepSeek-R1} exhibits the highest proportion of effective reflections, and the models show a notably higher rate of effective reflections in the {math} domain. However, the overall proportion of valid reflections across all models remains relatively low, ranging between 30\% and 40\%. This suggests that the reflection capabilities of current models require further improvement.
%Detailed statistical data can be found in Appendix D.

\begin{tcolorbox}[colback=white!95!gray, colframe=gray!70!black,  title=Key Finding for Reflection]
Despite frequent reflection attempts, the proportion of effective reflections remains low across models, and  DeepSeek-R1 achieves the highest rate of valid reflections.
\end{tcolorbox}

\subsection{Effective Reasoning of o1-like Models}

\begin{figure}[t]
    \centering
    \includegraphics[width=0.98\textwidth]{figures/effetive_reasoning.pdf}
    \caption{Distribution of effective reasoning ratios.}
    
    \label{fig: effetive_reasoning}
\end{figure}

\paragraph{Statistics.} 
% As previously mentioned, 
Human annotators evaluate the usefulness of the reasoning in each section, enabling us to calculate the proportion of valid reasoning in each response. As illustrated in Figure \ref{fig: effetive_reasoning}, each graph shows the distribution of effective reasoning ratios for a particular model. The red dashed line in each graph indicates the average effective reasoning ratio.

\paragraph{What Proportion of Reasoning in Long CoT Responses is Effective?}
On average, only 73\% of the reasoning in the collected long CoT responses is useful, highlighting significant redundancy issues. Among the models analyzed, \textit{QwQ-32B-Preview} exhibited the lowest proportion of effective reasoning at 70\%, while \textit{DeepSeek-R1} achieved a notably higher proportion compared to the others, demonstrating superior reasoning efficiency.


\begin{tcolorbox}[colback=white!95!gray, colframe=gray!70!black,  title=Key Finding for Reasoning Efficiency]
On average, 27\% of reasoning in long CoT responses we collected is redundant, and DeepSeek-R1 outperforms others in reasoning efficiency.
\end{tcolorbox}
\vspace{-3mm}

\subsection{Reasoning Process Analysis}

Figure ~\ref{fig: action_roles} shows the distribution of each section's action roles in the system II thinking process of the o1-like models. Initially, problem analysis dominates, indicating that the model initially focuses on understanding the requirements and constraints of the problem. As the solution progresses, cognitive activities diversify significantly, with reflection and validation becoming more prominent. In the later part of the reasoning, the distribution of conclusion and summarization gradually increases. 
%As the model progresses from problem analysis, solution implementation and conclusion, it demonstrates the common reasoning template of o1-like models.


\begin{figure}[t]
    \centering
    \includegraphics[width=0.8\textwidth]{figures/action_role.pdf}
    \caption{Distribution of different task types throughout the progress of a long CoT response.}
    \vspace{-3mm}
    
    \label{fig: action_roles}
\end{figure}
\subsection{Results on DeltaBench}

% Please add the following required packages to your document preamble:
% \usepackage{multirow}
\begin{table*}[!t]
\centering
\resizebox{1.0\textwidth}{!}{%
    \begin{tabular}{cccccccccccccccc}
    \toprule
    \multirow{2}{*}{\textbf{Model}} & \multicolumn{3}{c}{\textbf{Overall}} & \textbf{Math} & \textbf{Code} & \textbf{PCB} & \textbf{General} \\
    \cmidrule(lr){2-4} \cmidrule(lr){5-5} \cmidrule(lr){6-6} \cmidrule(lr){7-7} \cmidrule(lr){8-8}
     & \textbf{\textit{Recall}} & \textbf{\textit{Precision}} & \textbf{\textit{F1}} & \textbf{\textit{F1}} & \textbf{\textit{F1}} & \textbf{\textit{F1}} & \textbf{\textit{F1}} \\
    \midrule
    \multicolumn{8}{c}{\textbf{\textit{Process Reward Models (PRMs)}}} \\
    \midrule
    \rowcolor[rgb]{ .988,  .949,  .8} Qwen2.5-Math-PRM-7B & \textbf{30.30} & \textbf{34.96} & \textbf{29.22}  &  \textbf{29.64} & \textbf{23.76} & \underline{31.09} & \underline{34.19}   \\
    \rowcolor[rgb]{ .988,  .949,  .8} Qwen2.5-Math-PRM-72B & \underline{28.16} & \underline{29.37} & \underline{26.38}  & \underline{24.16} & \underline{22.02} & \textbf{31.14} & \textbf{35.83}  \\
    \rowcolor[rgb]{ .988,  .949,  .8} Llama3.1-8B-PRM-Deepseek-Data & 11.7 & 15.59 & 12.02 &  12.28 & 10.95 & 16.76 & 12.59  \\
    \rowcolor[rgb]{ .988,  .949,  .8} Llama3.1-8B-PRM-Mistral-Data & 9.64 & 11.21 & 9.45 & 9.40 & 10.72 & 13.43 & 12.40  \\
    \rowcolor[rgb]{ .988,  .949,  .8} Skywork-o1-Qwen-2.5-1.5B & 3.32 & 3.84 & 3.07 & 1.30 & 6.66 & 5.43 & 7.87  \\
    \rowcolor[rgb]{ .988,  .949,  .8} Skywork-o1-Qwen-2.5-7B & 2.49 & 2.22 & 2.17 & 0.78 & 6.28 & 6.02 & 3.11  \\
    \midrule
     \multicolumn{8}{c}{\textbf{\textit{LLM as Critic Models}}} \\
    \midrule
    \rowcolor[rgb]{ .922,  .89,  .988} GPT-4-turbo-128k & \textbf{57.19} & \textbf{37.35} & \textbf{40.76} & \textbf{37.56} & \textbf{43.06} & \underline{45.54} & \underline{42.17} \\
    \rowcolor[rgb]{ .922,  .89,  .988} GPT-4o-mini & \underline{49.88} & 35.37 & \underline{37.82} & \underline{33.26} & 37.95 & \textbf{45.98} & \textbf{46.39} \\
    \rowcolor[rgb]{ .922,  .89,  .988} Doubao-1.5-Pro & 39.68 & \underline{37.02} & 35.25 & 32.46 & \underline{39.47} & 33.53 & 37.00 \\
    \rowcolor[rgb]{ .922,  .89,  .988} GPT-4o & 36.52 & 32.48 & 30.85 & 28.61 & 28.53 & 39.25 & 36.50 \\
    \rowcolor[rgb]{ .922,  .89,  .988} Qwen2.5-Max & 36.11 & 30.82 & 30.49 & 26.73 & 32.81 & 39.49 & 29.54 \\
    \rowcolor[rgb]{ .922,  .89,  .988} Gemini-1.5-pro & 35.51 & 30.32 & 29.59 & 26.56 & 28.20 & 40.13 & 33.66 \\
    \rowcolor[rgb]{ .922,  .89,  .988} DeepSeek-V3 & 32.33 & 28.13 & 27.33 & 27.04 & 27.73 & 27.35 & 27.45 \\
    \rowcolor[rgb]{ .922,  .89,  .988} Llama-3.1-70B-Instruct & 32.22 & 28.85 & 27.67 & 21.49 & 32.13 & 28.45 & 39.18 \\
    \rowcolor[rgb]{ .922,  .89,  .988} Qwen2.5-32B-Instruct & 30.12 & 28.63 & 26.73 & 22.34 & 31.37 & 33.78 & 24.37 \\
    \rowcolor[rgb]{ .882,  .949,  .89} DeepSeek-R1 & 29.20 & 32.66 & 28.43 & 24.17 & 29.28 & 34.78 & 35.87 \\
    \rowcolor[rgb]{ .882,  .949,  .89} o1-preview & 27.92 & 30.59 & 26.97 & 22.19 & 28.09 & 33.11 & 35.94 \\
    % Gemini-2.0-flash-thinking & 14.02 & 17.36 & 14.56 & 14.79 & 11.97 & 19.34 & 15.26 \\
    \rowcolor[rgb]{ .922,  .89,  .988} Qwen2.5-14B-Instruct & 26.64 & 27.27 & 24.73 & 21.51 & 29.05 & 29.98 & 20.59 \\
    \rowcolor[rgb]{ .922,  .89,  .988} Llama-3.1-8B-Instruct & 25.71 & 28.01 & 24.91 & 18.12 & 32.17 & 27.30 & 29.93 \\
    \rowcolor[rgb]{ .882,  .949,  .89} o1-mini & 22.90 & 22.90 & 19.89 & 16.71 & 21.70 & 20.37 & 26.94 \\
    \rowcolor[rgb]{ .922,  .89,  .988} Qwen2.5-7B-Instruct & 21.99 & 19.61 & 18.63 & 11.61 & 25.92 & 29.85 & 15.18 \\
    \rowcolor[rgb]{ .882,  .949,  .89} DeepSeek-R1-Distill-Qwen-32B & 17.19 & 18.65 & 16.28 & 13.02 & 23.55 & 15.05 & 11.56 \\
    % Gemini-2.0-flash-thinking & 14.02 & 17.36 & 14.56 & 14.79 & 11.97 & 19.34 & 15.26 \\
    \rowcolor[rgb]{ .882,  .949,  .89} DeepSeek-R1-Distill-Qwen-14B & 12.81 & 14.54 & 12.55 & 9.40 & 18.36 & 10.44 & 12.01 \\
    % \rowcolor[rgb]{ .882,  .949,  .89} QwQ-32B-Preview & 10.20 & 10.17 & 9.07 & 7.38 & 8.60 & 14.97 & 10.54 \\
    \bottomrule
    \end{tabular}
}
\caption{Experimental results of PRMs and critic models on DeltaBench. \textbf{Bold} indicates the best results within the same group of models, while \underline{ underline} indicates the second best.}
% \vspace{-4mm}
\label{tab: main}
\end{table*}

% \noindent\textbf{Evaluation Metrics.}
% % To accurately assess the performance of the PRM and critic models on DeltaBench, 
% We employ \textbf{recall}, \textbf{precision}, and \textbf{macro-F1 score} for error sections as evaluation metrics. For the PRMs, we utilize an outlier detection technique based on the Z-Score to make predictions. This method was chosen because threshold-based prediction methods determined from other step-level datasets, such as those used in ProcessBench~\citep{Zheng2024ProcessBenchIP}, may not be reliable due to significant differences in dataset distributions, particularly as DeltaBench focuses on long CoT. Outlier detection helps to avoid this bias. The threshold $t$ for determining the correctness of a section is defined as:
% % \begin{align}
% $t = \mu - \sigma$,
% % \nonumber
% % \label{eq: prm_threshold}
% % \end{align}
% where $\mu$ is the mean of the rewards distribution across the dataset, and $\sigma$ is the standard deviation. Sections falling below $t$ are predicted as error sections. For critic models, all erroneous sections within a long CoT are prompted to be identified. Given that error sections constitute a smaller proportion than correct sections across the dataset, we use macro-F1 to mitigate the potential impact of the imbalance between positive and negative sections. Macro-F1 independently calculates the F1 score for each sample
% % (for our metric, each case) 
% and then takes the average, providing a more balanced evaluation metric when dealing with class imbalance.

\noindent\textbf{Baseline Models.}
% 开源(中英模型,llama3)和闭源模型
% To comprehensively evaluate the performance of current PRMs and critic models, we extensively selected and evaluated a wide range of both open-source and closed-source models on DeltaBench.
% \paragraph{Process Reward Models}
For the \textbf{PRMs}, we select the following models: Qwen2.5-Math-PRM-7B\footnote{\href{https://huggingface.co/Qwen/Qwen2.5-Math-PRM-7B}{Qwen/Qwen2.5-Math-PRM-7B}}, Qwen2.5-Math-PRM-72B\footnote{\href{https://huggingface.co/Qwen/Qwen2.5-Math-PRM-72B}{Qwen/Qwen2.5-Math-PRM-72B}}, Llama3.1-8B-PRM-Deepseek-Data\footnote{\href{https://huggingface.co/RLHFlow/Llama3.1-8B-PRM-Deepseek-Data}{RLHFlow/Llama3.1-8B-PRM-Deepseek-Data}}, Llama3.1 -8B-PRM-Mistral-Data\footnote{\href{https://huggingface.co/RLHFlow/Llama3.1-8B-PRM-Mistral-Data}{RLHFlow/Llama3.1-8B-PRM-Mistral-Data}}, Skywork-o1-Open-PRM- Qwen-2.5-1.5B\footnote{\href{https://huggingface.co/Skywork/Skywork-o1-Open-PRM-Qwen-2.5-1.5B}{Skywork/Skywork-o1-Open-PRM-Qwen-2.5-1.5B}}, and Skywork-o1-Open-PRM-Qwen-2.5-7B\footnote{\href{https://huggingface.co/Skywork/Skywork-o1-Open-PRM-Qwen-2.5-7B}{Skywork/Skywork-o1-Open-PRM-Qwen-2.5-7B}}. 
% These represent some of the best open-source PRMs currently available.
% \paragraph{Critic Models}
We select a group of the most advanced open-source and closed-source LLMs to serve as \textbf{critic models} for evaluation, which includes various GPT-4~\citep{gpt4} variants (such as GPT-4-turbo-128K, GPT-4o-mini, GPT-4o), the Gemini model~\citep{Reid2024Gemini1U}(Gemini-1.5-pro), several Qwen models~\citep{qwen2.5} (such as Qwen2.5-32B-Instruct and Qwen2.5-14B-Instruct), Doubao-1.5-Pro~\citep{doubao2025}
and o1 models~\citep{openai-o1} (o1-preview-0912, o1-mini-0912).
% , and a GPT-3.5 variant (gpt-3.5-16K).



\subsubsection{Main Results}
In Table \ref{tab: main},
we provide the results of different LLMs on DeltaBench. 
For PRMs, we have the following observations: (1). Existing PRMs usually achieve low performance, which indicates that existing PRMs cannot identify the errors in long CoTs effectively and it is necessary to improve the performance of PRMs. (2). Larger PRMs
do not lead to better performance. For example, the Qwen2.5-Math-PRM-72B is inferior to wen2.5-Math-PRM-7B.
For critic models, we have the following findings: (1)
GPT-4-turbo-128k archives the best critique results, which is better than other models (e.g., GPT-4o) a lot in DeltaBench. (2) For o1-like models (e.g., DeepSeek-R1, o1-mini, o1-preview), we observe that the results of these models are not superior to non-o1-like models, with the performance of o1-preview is even lower than Qwen2.5-32B-Instruct.
%Additionally, we observe that the QWQ and DeepSeek-R1-Distill series models exhibit weaknesses in following instructions. 
A detailed analysis of underperforming models is provided in Appendix \ref{app: underperforming}.

% model size
% domains
% o1模型跟普通模型critic能力对比分析


\subsubsection{Further Analysis}

\paragraph{Effect of Long CoT Length.}
\begin{figure}[t]
    \centering
    \includegraphics[width=1.0\textwidth]{figures/4.5.1/length2.pdf}
    \caption{The effect of long CoT length.}
    \label{fig: crtic1}
\end{figure}
In Figure \ref{fig: crtic1}, we compare the average F1-Score performance of critic models and PRMs across varying LongCoT token lengths. 
For critic models, the performance notably declines as token length increases. Initially, models like Deepseek-R1 and GPT-4o exhibit strong performance with shorter sequences (1-3k tokens). However, as token length increases to mid-ranges (4-7k tokens), there is a marked decrease in performance across all models. This trend highlights the growing difficulty for critic models to maintain precision and recall as long CoT response become longer and more complex, likely due to the challenge of evaluating lengthy model outputs. In contrast, PRMs demonstrate greater stability across token lengths, as they evaluate sections sequentially rather than processing the entire output at once. Despite this advantage, PRMs achieve lower overall scores compared to critic models on our evaluation set.

\begin{tcolorbox}[colback=white!95!gray, colframe=gray!70!black, title=Key Finding]
  Critic models exhibit significant performance degradation with longer contexts, while PRMs demonstrate consistent evaluation capability across varying lengths.
\end{tcolorbox}


\paragraph{Performance Analysis Across Different Error Types.}
\begin{figure}[t]
    \centering
    \includegraphics[width=0.9\textwidth]{figures/4.5.2/top_models_per_task.pdf}
    \caption{Results of different LLMs on top-5 errors.}
    \label{fig: top_models_per_task}
\end{figure}
Figure \ref{fig: top_models_per_task} shows the performance of different models on the five most common error types. In terms of error types, most models demonstrate the highest accuracy in recognizing calculation errors. Conversely, the recognition of strategy errors is generally the weakest. In terms of models, there is significant variation in the ability of individual models to recognize different error types. For instance, DeepSeek-V3 achieves an F1 of 36\% on calculation errors but only 23\% on strategy errors. Meanwhile, Llama3.1-8B-PRM-Deepseek performs poorly, with an F1 score of 22\% on calculation errors, and shows a significant decline in performance across the other four error types. This highlights the limited generalization capabilities of most models when recognizing various error types.

\begin{tcolorbox}[colback=white!95!gray, colframe=gray!70!black, title=Key Finding]
  Models exhibit strong performance on calculation errors but struggle with strategy errors, revealing limited generalization across error types.
\end{tcolorbox}

\begin{table}[!ht]
    \centering
    % \scriptsize
    % \footnotesize
    \begin{tabular}{cccc}
    \toprule
        \multirow{2}{*}{Model} & \multicolumn{3}{c}{HitRate@$k$ - Avg(\%)} \\ \cline{2-4}
                           & $k=1$ & $k=3$ & $k=5$ \\ 
                           % \hline
                           \midrule
        Qwen2.5-Math-PRM-7B & \textbf{49.15} & \textbf{69.14} & \textbf{83.14} \\
        Qwen2.5-Math-PRM-72B & \underline{41.13} & \underline{62.70} & \underline{75.73} \\ 
        Llama3.1-8B-PRM-Deepseek-Data & 12.63 & 48.62 & 69.78 \\
        Llama3.1-8B-PRM-Mistral-Data & 8.99 & 42.97 & 65.33 \\
        Skywork-o1-Open-PRM-Qwen-2.5-1.5B & 31.90 & 53.82 & 69.23 \\
        Skywork-o1-Open-PRM-Qwen-2.5-7B & 31.58 & 52.59 & 69.16 \\
        % \hline
        \bottomrule
    \end{tabular}
    \vspace{+3mm}
    \caption{Results of HitRate@$k$. Bold and underlined results indicate the best and the second best.}
    % \vspace{-4mm}
\label{tab: hitrate}
\end{table}

\paragraph{Analysis on HitRate evaluation for PRMs.}

\begin{figure}[t]
    \centering
    \includegraphics[width=\textwidth]{figures/prm_rank.pdf}
    % \vspace{-10pt}
    \caption{Ranking of rewards for the first incorrect section for different PRMs.}
    % \vspace{-3mm}
    \label{fig: prm_rank}
\end{figure}

To better measure the ability of PRMs to identify erroneous sections in long CoTs, we use HitRate@$k$ to evaluate PRMs. Specifically, within a sample, we rank the sections in ascending order based on the rewards given by the PRM, select the smallest $k$ sections, and calculate the recall rate for the erroneous sections among them. Specifically, we define the sorted sections as $S = \{s_1, s_2, \ldots, s_n\}$, with $E$ being the set of erroneous sections. We select the top $k$ sections, denoted as $S_k = \{s_1, s_2, \ldots, s_k\}$. The HitRate@$k$ is  calculated as:
\begin{align}
\text{HitRate@}k = \frac{|S_k \cap E|}{\min(k, |E|)}
% \nonumber
\label{eq: hitrate}
\end{align}
In this formula, $|S_k \cap E|$ indicates the number of erroneous sections identified among the top $k$ sections. This metric reflects the ability of PRMs to effectively identify erroneous sections within the top $k$ candidate sections. In Table \ref{tab: hitrate}, the relative performance rankings among different PRMs are quite similar to the results in Table \ref{tab: main}. Additionally, we observe that for $k=3$ and $k=5$, the performance differences between various PRMs are not particularly significant. However, when $k=1$, the Qwen2.5-Math-PRM-7B shows a clear performance advantage. Figure \ref{fig: prm_rank} illustrates the ranking ability of different PRMs for the first incorrect section within the sample, which is generally consistent with the performance evaluation results of HitRate@k.
% This is because a smaller $k$ value imposes stricter requirements on the PRM's ability to identify errors.

% HitRate@$k$ evaluates the performance of PRMs from the perspective of reward ranking, providing additional evidence for the experimental results and conclusions in Table \ref{tab: main} from a different angle.

\begin{tcolorbox}[colback=white!95!gray, colframe=gray!70!black, title=Key Finding]
  HitRate@k evaluation aligns with the main results, with Qwen2.5-Math-PRM-7B demonstrating superior performance in identifying the first incorrect section.
\end{tcolorbox}


\begin{figure}[t]
    \centering
    \includegraphics[width=0.8\textwidth]{figures/4.5.4/self-critic.pdf}
    % \vspace{-10pt}
    \caption{F1-score comparison of self-critique and cross-model critique abilities for different models.}
    % \vspace{-5mm}
    \label{fig: self-critic}
\end{figure}

\paragraph{Comparative Analysis of Self-Critique Capabilities of LLMs.} We randomly sample queries based on domains and models that generate the long CoT output, followed by a statistical analysis of the model's performance in evaluating its own outputs as well as those of other models. In Figure \ref{fig: self-critic},  Gemini 2.0 Flash Thinking, DeepSeek-R1, and QwQ-32B-Preview show lower self-critique scores compared to their cross-model critique scores, indicating a prevalent deficiency in self-critic abilities. Notably, DeepSeek-R1 exhibits the largest discrepancy, with a 36\% decrease in self-evaluation compared to evaluations of other models. This suggests models' self-critic abilities remain underdeveloped.
% signaling an area that requires improvement.

\begin{tcolorbox}[colback=white!95!gray, colframe=gray!70!black, title=Key Finding]
  LLMs demonstrate weaker self-critique performance compared to cross-model critique, highlighting a fundamental limitation in self-critic capabilities.
\end{tcolorbox}



%%%

% \noindent\textbf{Performance Analysis Across Different Categories}

% \begin{figure}[htbp]
% \centering
% \includegraphics[width=\linewidth]{figures/prm_task_comparison.pdf}
% \caption{Performance of PRMs across different categories (outlier detection).}
% \label{fig: prm_task}
% % \vspace{-0.6cm}
% % \vspace{-4mm}
% \end{figure}


% \noindent\textbf{Performance Variation in Different Lengths of Long CoT}

% \noindent\textbf{Performance Analysis Across Different Error Types}

% \noindent\textbf{Analysis of In-Sample Reward Ranking}


% % \subsection{Evaluation Metrics}

% % \subsection{Main Results}

% % \subsection{Further Analysis}
% \subsection{Analysis on LLM Critics}
%  \textbf{error location}



% \subsubsection{The Performance across different domains}

% \begin{figure}[t]
%     \centering
%     \includegraphics[width=0.5\textwidth]{figures/critic6.pdf}
%     \caption{The score distributions across different domains.}
%     \label{fig: crtic2}
% \end{figure}

% In Figure \ref{fig: crtic2}, we illustrate the F1-score distribution of various large language models (LLMs) across different domains. Analyzing model performance across domains reveals that most models demonstrate stronger critiquing abilities in Physics, Chemistry, Biology, and General Reasoning compared to Mathematics and Programming, indicating higher proficiency in scientific and general reasoning tasks. Meanwhile, the performance of each model varies significantly depending on the domain, reflecting inherent strengths and weaknesses in handling different tasks. For instance, the Gemini-1.5-Pro model achieves an F1-score of 40.1\% in PCB, yet only 26.6\% in Mathematics. These discrepancies underscore challenges in the models' generalization capabilities.






\section{Related Work}
\label{sec:Related Work}
\subsection{Large vision language model}
Vision-language models\cite{li2023blip,li2024llava,bai2023qwen,lu2024deepseekvlrealworldvisionlanguageunderstanding, alayrac2022flamingo,sun2024generativemultimodalmodelsincontext}have achieved remarkable advancements within the realm of multimodal intelligence. By amalgamating large language models\cite{ray2023chatgpt,achiam2023gpt,anil2023palm,touvron2023llama2openfoundation,touvron2023llamaopenefficientfoundation} with visual content, LVLMs effectively manage intricate visual and linguistic inputs, thereby executing a variety of tasks ranging from visual description to logical reasoning. Flamingo\cite{alayrac2022flamingo} and OpenFlamingo\cite{awadalla2023openflamingoopensourceframeworktraining} models incorporate visual feature processing modules into the internal strata of language models using gated cross-attention, thereby propelling the profound integration of visual data within LLMs. CLIP\cite{radford2021learning,sun2023evaclipimprovedtrainingtechniques} utilizes contrastive learning to harmonize image and text modalities and is trained on extensive, noisy web-derived image-text pairs. By integrating modules such as QFormer\cite{li2023blip} and MLP\cite{liu2024visual}, previous works\cite{bai2023qwen, dai2023instructblipgeneralpurposevisionlanguagemodels,Liu_2024_CVPR} facilitate a collaborative comprehension between visual encoders and large language models (LLMs) of multimodal inputs. LLaVA\cite{liu2024visual} stands out for its pioneering use of GPT-generated instruction-following data to amplify LVLMs' responsiveness to visual instructions. A plethora of powerful LVLM APIs, including GPT-4o\cite{achiam2023gpt} and Qwen-VL-max\cite{bai2023qwen}, are now available. Through a rigorous evaluation of these models based on our proposed benchmark, we offer insightful perspectives into the ongoing research surrounding LVLMs.
\subsection{Vision Language Benchmarks} A rapidly expanding suite of multimodal benchmarks now rigorously evaluates the capabilities of LVLMs. Established benchmarks, including COCO Caption \cite{chen2015microsoftcococaptionsdata}, VQAv2 \cite{Goyal_2017_CVPR}, and GQA \cite{Hudson_2019_CVPR}, predominantly center on image description and question-answering tasks, employing metrics such as BLEU, CIDEr, and accuracy to gauge performance. Yet, as LVLMs advance, these traditional datasets have become insufficient for fully capturing the breadth of model capabilities. In response, researchers have developed more comprehensive evaluation frameworks that test a wider range of competencies, encompassing perceptual and cognitive skills \cite{fu2024mmecomprehensiveevaluationbenchmark}, spatial-temporal reasoning \cite{li2023seedbenchbenchmarkingmultimodalllms}, and relational understanding \cite{liu2025mmbench}. For instance, MMMU \cite{Yue_2024_CVPR} curates data from college-level textbooks and lecture materials, challenging models to demonstrate expertise across six academic disciplines. Similarly, CMMU \cite{he2024cmmubenchmarkchinesemultimodal} gathers questions from primary through high school curricula to assess foundational knowledge within the Chinese educational context. Nevertheless, these benchmarks largely remain focused on basic visual tasks, without adequately addressing the complexity of multimodal understanding. This paper introduces a benchmark tailored to evaluate deep semantic comprehension of images, specifically within a Chinese cultural framework.
\subsection{Image implicit meaning comprehension}
Image implicit meaning comprehension has become an important research focus for contemporary LVLMs, especially in handling images that convey complex emotions, cultural symbolism, and social critique. Existing evaluation datasets primarily test the models' linear visual reasoning abilities, such as visual question answering for surface-level content\cite{Hudson_2019_CVPR}. However, several works \cite{cai2019multi, machajdik2010affective} have demonstrated that LVLMs’ capabilities go beyond understanding surface-level meanings. Recent works\cite{yang2024largemultimodalmodelsuncover, liu2024iibenchimageimplicationunderstanding} highlight the limitations of current models when it comes to processing nonlinear narratives and understanding cultural contexts. For example, the most relevant prior work, DEEPEVAL\cite{yang2024largemultimodalmodelsuncover}, introduces three core tasks and shows that while the most advanced models achieve near-human performance on basic visual description tasks, they still perform poorly on tasks that involve understanding implicit semantics such as social background and satire. This paper provides a more comprehensive Chinese understanding benchmark, which, compared to the six categories in DeepEval, expands to include more thematic categories, with a total of 13 major categories and 41 subcategories (Figure \ref{fig:categories}), and offers more detailed testing across four dimensions of model performance.
% Image implicit meaning comprehension has emerged as a crucial research focus for contemporary LVLMs, particularly in handling images that convey nuanced emotions, cultural symbolism, and social critique. Achieving this level of comprehension demands that models infer implicit meanings from visual content, recognizing elements like satire, humor, and philosophical nuances. The most relevant prior work DEEPEVAL\cite{yang2024largemultimodalmodelsuncover} benchmark introduces three core tasks—fine-grained description selection. However, its limited categorization—comprising only six classes—restricts the scope of implicit meaning assessment, leaving out a broader range of complex visual semantics. 

% 2.1应该还没覆盖所有用到的模型;2.2需要补充点内容并且与2.3区分,2.3内容需要再调整
% 大型视觉语言模型(Flamingo, Blip2, Visual Instruction tuning,v Qwen-VL, LLaVA-next, DeeepSeekVL)近年来在多模态智能方面(Multimodal Intelligence)取得了显著进展。通过整合大规模语言模型(如GPTs*、LlaMa*、Palm2)和视觉内容(*), LVLMs可以处理复杂的视觉和语言输入,实现从视觉描述到逻辑推理等多种任务。Flamingo、OpenFlamingo模型通过gated cross在语言模型的内部层次中嵌入视觉特征处理模块,推动了视觉信息在LLMs中的深度整合。CLIP模型使用对比学习实现图像和文本模态的统一,并使用大规模noisy web 图像-文本对进行训练。14, 15 16,  17,通过添加QFormer和MLP等模块使视觉编码器和大型语言模型(LLMs)能够协同理解多模态输入。LLaVA则开创了通过GPT生成的instruction-following data提升LLvMs对视觉指令的响应能力。同时包括很多强大的LVLMs API公开,包括(GPT-4v*、Qwen-VL-max*) 。通过对上述模型进行全面评估\subsection{Vision Language Benchmarks} 
%为了系统地评估视觉语言模型的能力,近年来涌现了许多多模态评估基准。传统的评估基准如 COCO Caption*、VQAv2* 和 GQA* 等,主要集中在图像描述和问答任务,通过BLEU、CIDEr、准确率 等客观指标来衡量模型的性能。然而随着LVLMs的进步,这些数据集的难度已经不足以评估LVLMs的能力。研究者们进一步提出了更为全面的基准测试框架,从感知和认知能力(MME)、spatial and temporal understanding(SEED Bench),到Relation Reasoning能力(MMBench)。MMU从大学教材、讲义中收集数据,要求模型具备大学级别六大领域的专业知识。类似的,CMMU收集了小学至高中的七大学科题目,以评估模型对中文基础学科知识的理解与应用。然而,这些基准仅限于对基础视觉任务的评估,未能充分评估模型在复杂多模态任务中的表现,因此本文旨在提出一个中文背景下的评估模型深度图像含义的Benchmark。
%深层语义理解是当前LVLMs的一个重要研究方向,特别是在处理具有复杂情感、文化隐喻和社会批判的图像时尤为重要。深层语义的理解需要模型具备从视觉内容中推理出隐含意义的能力,例如理解讽刺、幽默和哲学内涵。DEEPEVAL* 提出了三种任务:细粒度描述选择、深入标题匹配和深层语义理解,通过这些任务系统性地评估了 LVLMs 在理解深层视觉语义上的表现。例如,尽管 GPT-4V* 在基础的视觉描述任务上达到了接近人类的水平,但在涉及社会背景和讽刺的语义理解任务中,仍存在显著差距。此外,

%图像隐含意义理解已成为当代大规模多模态语言模型(LVLMs)研究的一个重要方向,特别是在处理传达复杂情感、文化符号和社会批评的图像时。现有的评估数据集主要测试模型的线性视觉推理能力,例如对于浅层内容的视觉问答(VQA),。然而Machajdik的工作也证明了LVLM的能力不止于理解浅层含义。然而最近的工作(如 MVP、DeepEval 和 YESBUT Benchmark、Ii-Bench)揭示了现有模型在处理非线性叙事和文化背景理解时的局限性。例如最相关的前期工作 DEEPEVAL 引入了三个核心任务,发现当前最先进的模型在基础视觉描述任务上已接近人类水平,但在涉及社会背景和讽刺等隐含语义理解的任务中,仍表现不佳。本文提供了一个更为完备的中文理解Benchmark,相较于 DeepEval 的六大类任务,扩展了更多的主题类别,共包含13大类和41小类,并从四个维度对大模型的性能进行了更为详细的测试。


\vspace{-0.2cm}
\section{Conclusions}
\vspace{-0.2cm}
In this paper, we introduce FanChuan, a multilingual benchmark for parody detection and analysis, encompassing seven datasets characterized by high diversity, rich contextual information, and precise annotations. Our findings reveal that parody detection remains highly challenging for both LLMs and traditional methods, with particularly poor performance on Chinese datasets. We also observe that contextual information significantly enhances model performance, while parody itself increases the difficulty of sentiment classification. Additionally, our results indicate that reasoning fails to improve LLM performance in parody detection. By filling a critical gap in the study of emerging online phenomena, FanChuan provides valuable insights into cultural values and the role of parody in digital discourse. These findings highlight the limitations of current LLMs, presenting an opportunity for future research to enhance model capabilities in parody detection and analysis.


% Sections after main pages
\clearpage
\section{Limitations} \label{sec:limitations}

While the above results demonstrate an important step toward flexible and robust humanoid locomotion, our proposed technique is not a panacea. 
%
Both HLIP and CI-MPC require parameter tuning, and their combination only increases the complexity of this process. While we used only one set of parameters for all the experiments, we did find that some parameters induced sharp tradeoffs. For example, a lower weight on base orientation tracking gave more natural-looking gaits, but reduced push recovery performance.
%


Our CI-MPC implementation uses significantly simplified collision geometries. This enables fast solve times, but precludes behaviors that involve contact away from the hands and the feet. As a result, the robot is not able to automatically recover from a fall. Furthermore, our CI-MPC solver's performance is reliant on smooth collision geometries, as sharp corners introduce problematic discontinuous gradients. 
%
Similarly, self-collisions present a major failure mode in the current implementation. Adding self-collision constraints either in the optimization problem \cite{grandia2021multi} or with a high order control barrier function \cite{khazoom2024tailoring, ames2019control, singletary2021safety} presents an obvious next step for improving reliability.

Finally, there are instances in which HLIP's suggested contact sequence guides the robot in an unhelpful direction. For example, if the robot is standing and pushed to the left, HLIP might suggest lifting the right leg, depending on the timing of the gait cycle. This could be mitigated with a richer reduced-order model, but illustrates a trade-off inherent to guiding whole-body behaviors with a reduced-order model.

\section*{Ethics Statement}
Our proposed benchmark, FanChuan, adheres to the ACL Code of Ethics. All the coauthors also work as annotators, and are compensated at an average hourly rate of 20 SGD. The data we collected is licensed under CC BY 4.0 and is used exclusively for academic purposes. It consists of publicly available website comments and does not contain any sensitive or personal information. To protect user privacy, we filtered out any private data during the data collection and organization process, ensuring that the dataset does not include any user-sensitive content. Additionally, recognizing the potential presence of malicious content in user debates, we have removed harmful comments that violate community ethical standards. Regarding the cultural and topical elements in the datasets, our research remains neutral and free from bias, solely focused on academic exploration. Lastly, AI was used to revise the grammar during the paper writing process.

\clearpage
\bibliography{custom}

\clearpage
\appendix

\section{Dataset Details}\label{apd:dataset_details}

% 一位来自中专的同学在阿里巴巴数学竞赛中取得了非常优异的成绩,尽管她的学校并不是很好。很多人支持她,认为她代表了底层的逆袭和女性的力量。但是也有不少人根据采访片段认为她在比赛中作弊了。这一话题在中文互联网上展开了热议。为了让更多人相信他们的立场,很多质疑者反串成为的支持者,用非常夸张的语气赞扬她,比如说:国家应该马上保护姜萍,未来的诺贝尔奖非她莫属。
\paragraph{Alibaba-Math} A student from a vocational school achieved remarkable results in the Alibaba Mathematics Competition, despite coming from a school with a less prestigious reputation. Many people supported her, seeing her as a symbol of rising from humble beginnings and a testament to female empowerment. However, some other people questioned her achievements, suggesting that she might have cheated based on snippets from TV interviews. This topic sparked heated discussions on the Chinese internet. To persuade others to believe their claims, some skeptics impersonated her supporters and used exaggerated praise, saying things like, ``\begin{CJK}{UTF8}{gbsn}这位同学有实力!阿里巴巴有眼光! 请阿里巴巴破格录取进入达摩院,助力阿里科技快速发展 \end{CJK}'' ``\textit{(This student has strength! Alibaba has vision! Please grant her an exceptional admission to DAMO Academy to boost Alibaba’s technological growth  )}'' This is a highly complex topic that encompasses mathematics, education, and gender-related controversies. Annotators working with this dataset must not only be familiar with relevant internet memes but also possess a solid understanding of advanced mathematical concepts. 


% 在中国一些地方结婚有给女方彩礼的习俗。对于一些天价彩礼的要求来说,一些人认为彩礼是给女方的保障,让女方在婚姻中更有安全感;另一部分人则认为彩礼和婚姻的幸福没有必然联系。为此,支持彩礼的人和不支持彩礼的人在网上展开了广泛的辩论。为了制造荒诞幽默的效果,一些反对彩礼的人反串成支持彩礼的人发表自己的评论,比如说:姐妹们千万别乱嫁人,找不到年入百万的千万别嫁,女孩子五十岁都很值钱。
\paragraph{BridePrice} In some parts of China, there is a tradition of giving a bride price to the bride's family upon marriage. Regarding the demands for exorbitant bride prices, some people believe that the bride price serves as a form of security for the bride, providing her with a greater sense of safety in the marriage. Others argue that the bride price has no inherent relation to marital happiness. This has sparked extensive online debates, and to create an absurd and humorous effect, some opponents of the bride price impersonate the supporters and post comments such as: ``\begin{CJK}{UTF8}{gbsn}是的是的,姐妹们千万别乱嫁人,找不到年入百万的千万别嫁,女孩子五十岁都很值钱!\end{CJK}'' \textit{(Ladies, never marry recklessly. If he doesn't make a million a year, don't marry him. Girls are valuable even at fifty!)}
Gender issues, particularly the topic of bride price, have been a widely debated subject on the Chinese internet for a long time. This dataset requires annotators to be well-versed in these discussions and familiar with the associated memes. 

% 何同学一直以轻松幽默的方式科普科技知识,吸引了众多粉丝。然而,他最近发布的视频《为了让大家多喝水,我做了这个…》引发争议。视频中,他设计了“喝水大作战”系统,通过奖励机制督促饮水。但由于设计成本高、效果有限,一些观众质疑其实用性,甚至有人反串支持,评论“震古烁今,足以开启第五次技术革命”来表达不满。
\paragraph{DrinkWater} A technology video creator recently posted a video titled ``\textit{I Made This to Get Everyone to Drink More Water...}'' sparked controversy. In the video, he introduced a complex ``\textit{Water Drinking Battle}'' system designed to encourage hydration through a reward mechanism. Yet, due to the high design cost and limited effectiveness, some viewers questioned its practicality. Some even ironically pretended to support it, leaving comments like ``\begin{CJK}{UTF8}{gbsn}震古烁今,足以开启第五次技术革命\end{CJK}'' ``\textit{(A groundbreaking innovation capable of launching the fifth technological revolution)}'', to express their dissatisfaction.
This video creator has always been a subject of controversy. While he is well known for his content on science and technology, some critics argue that he lacks fundamental engineering literacy. Annotators working with this dataset should have a basic understanding of scientific and technological concepts. 

% 在电竞世界杯决赛中,新阵容的G2战队表现强势,却再次败给连续七次击败他们的NAVI战队。这场失利引发热议:有人认为G2尚需磨合,未来可期;也有人质疑变阵后的G2缺乏夺冠实力,难以克服“心魔”NAVI。部分反串者更发表引人注目的评论,如“传奇捕虾人终结了G2的三日王朝”,对G2的变阵提出质疑。
\paragraph{CS2} In the Counter Strike 2 (CS2) World Championship finals, G2's newly revamped roster showed impressive strength but once again fell to NAVI, who had already defeated them seven times in a row. This loss sparked heated discussions: someone believes that G2 needs more time to build synergy and has promising potential, while others question whether the roster change truly enhances their chances to win, as they still struggle to overcome their "mental block" against NAVI. Some satirical critics even made eye-catching remarks, such as ``\begin{CJK}{UTF8}{gbsn}传奇捕虾人终结了G2的三日王朝\end{CJK}'' ``\textit{(The legendary shrimp catcher ended G2's three-day dynasty)}'', to express doubts about the effectiveness of G2's roster adjustments. Parody comments in this dataset are particularly difficult to identify for those unfamiliar with the background of CS2, as the comments contain terminology of CS2 game and various aliases of teams and players. Annotators must have a strong understanding of these references to accurately interpret the content.

% 该数据集采集自某高校论坛,涵盖了住宿、校车、求职、行政等多个话题板块。其中一则帖子引发了热议:某学生反映室友带女友在寝室过夜,并征求沟通建议。评论区出现"羡慕吗?"等反串式调侃,以戏谑口吻表达对此类行为的不满。此外,在校园开放日期间,一则题为"申请我校?你的学费将用于支持巴勒斯坦种族灭绝"的海报出现在卫生间。对此,有网友以反讽方式评论道:"校内所有电脑均配备英特尔处理器,而英特尔的研发中心设在以色列!若想避免支持种族灭绝,请立即更换搭载兆芯CPU的电脑!"这种以支持者口吻进行的评论,实则揭示了原观点的荒谬性。
\paragraph{CampusLife} This dataset was collected from a university forum, covering various discussion topics such as dorm life, campus buses, job hunting, and administration. One particular post sparked a heated debate: a student complained about their roommate bringing their girlfriend to stay overnight in the dorm and sought advice on how to address the situation. The comment section included parodic remarks like ``\textit{Jealous?}'', mocking the situation in a humorous yet disapproving tone. Additionally, during the university's open campus day, a poster appeared in a restroom with the title: ``\textit{Applying to our university? Your tuition funds Palestinian genocide.}'' In response, some users posted parodic comments, such as: ``\textit{Every computer on campus is equipped with an Intel processor, and Intel's R\&D center is in Israel! If you want to avoid supporting genocide, switch to a computer with a Zhaoxin CPU immediately!}''


% 一场“一位觉醒少年能否抵挡20为特朗普支持者”的辩论中,一位女性特朗普支持者因为其的推理毫无逻辑而输掉辩论,并因此遭受许多网友的批判,认为其发言毫无意义。其中有反串者称“她做得很好,提出了有力的观点”。来批判这位特朗普支持者缺乏逻辑推理能力。
\paragraph{Tiktok-Trump} In a debate titled ``\textit{Can One Awakened Youth Withstand 20 Trump Supporters?}'', a female Trump supporter lost the debate due to her illogical reasoning and subsequently faced criticism from many netizens who deemed her remarks meaningless. Among the critics, some parodically commented, ``\textit{She did a great job bring up solid points}'', to criticize the Trump supporter's lack of logical reasoning ability.

% 特朗普因政治立场、思想和行为饱受争议,相关话题常引发广泛讨论,支持和批判的声音并存。一些反对者通过反串模仿他的语气发表评论,例如:“他接受的检查比任何人都多,而且是由世界上最好的医生进行的。他们很惊讶,说他们从未见过如此高的分数。如果被要求,他会再做一次检查,但他们说他不需要。这太不可思议了。”,以此嘲讽他的言论风格和争议性形象。
\paragraph{Reddit-Trump} Trump is a highly controversial figure due to his political stance, ideology, and behavior, sparking widespread debate with both supporters and critics. Some opponents use parody to mimic his tone, such as commenting, ``\textit{He's been tested—more than anyone, by the best doctors in the world. They were amazed, and said they'd never seen scores that high. He'll take another if asked, but they said he doesn’t need to. It’s incredible}'', mocking his rhetorical style and contentious image.
\section{Case Study on LLMs}\label{apd:sec_case_study}
To investigate how well LLMs understand parody, we conduct a case study in which LLMs are asked to provide explanations during prediction. Specifically, we construct the prompt by presenting a comment and its associated topic, then ask the LLMs to determine whether the comment is a parody and to explain their reasoning. After receiving the prediction and explanation from the LLMs, we compare the results with the ground truth label and explanation. The results of the case study for \textit{BridePrice}, \textit{Alibaba-Math}, \textit{DrinkWater}, and \textit{CS2} are presented in Tables \ref{tab:casestudy_brideprice}, \ref{tab:casestudy_alibabamath}, \ref{tab:casestudy_drinkwater}, and \ref{tab:casestudy_CS2}, respectively, using four LLMs: ChatGPT-4o \citep{GPT4}, Qwen 2.5 \citep{Qwen2.5}, DeepSeek-V3 \citep{DeepSeek}, and Claude3.5 \citep{Claude}. The results demonstrate:

(1) LLMs struggle with parody detection. For example, the parody comment in Table \ref{tab:casestudy_brideprice} takes an extreme position opposing the viewpoint that a boyfriend should hand over his salary, yet all the LLMs classify this as a non-parody comment. Additionally, the comment in Table \ref{tab:casestudy_CS2}, which directly expresses a dislike toward the G2 team with analysis, is identified as a parody by 3 of the 4 LLMs.

(2) LLMs frequently provide incorrect explanations when identifying parody comments. Even in the case of \textit{DrinkWater}, shown in Table \ref{tab:casestudy_drinkwater}, where all the LLMs successfully identify the comment as a parody, they fail to generate accurate explanations. The explanations indicate that the LLMs rely mostly on the style and tone of the comment, without a deeper understanding of the implicit meaning.

In conclusion, these results suggest that LLMs struggle to understand parody comments, as they both fail to provide accurate predictions and offer misleading explanations. This highlights the need for further development in LLMs for the task of parody detection.



\begin{table*}[htbp]
    \centering
    \small
    \begin{tabular}{p{14cm}}
     \toprule
\#\#\#  Objective: \\
Generate a 5-day family travel itinerantry that satisfies all specified requirements while adhering to highly fine-grained constraints. The generated itinerary should balance real-time adaptability, strict hard attributes, and semantic soft attributes. \\

\#\#\# User Profile: \\
 - Travelers: 2 adults + 1 child (age 8) \\
 - Budget: $<=$ \$300/day (total \$1,500 for the trip) \\
 - Activity Balance: 70\% educational/cultural experiences, 20\% relaxation, 10\% family-friendly shopping. \\

\#\#\# Hard Attributes: \\
- Activity Scheduling: \\
\quad- Each activity must have a defined start and end time, ensuring there is no overlap between activities. \\
\quad- A break period from 13:00-14:30 is mandatory daily. \\
\quad- Each activity must fit within a 2-hour window unless otherwise specified. \\

- Budget Requirements: \\
\quad- Each day’s total cost (including transportation, food, and activities) must not exceed \$300. \\
\quad- Transportation is limited to metro and walking only, with a maximum of 3 metro rides per day. \\

- Location Constraints: \\
\quad- Must-visit locations: City Zoo (Day 1) and Science Museum (Day 3). \\
\quad- Activities must occur in geographically adjacent areas to minimize walking distance. \\

- Keyword Requirements: \\
\quad- Each day’s description must include specific keywords. For example: \\
\quad- Day 1: “wildlife,” “exploration,” and “interactive learning.” \\
\quad- Day 3: “science,” “innovation,” and “hands-on exhibits.” \\

- Structure Constraints: \\
\quad- Each day’s itinerary must consist of 4 sections: \\
\quad\quad- Morning activity \\
\quad\quad- Break/lunch period \\ 
\quad\quad- Afternoon activity \\
\quad\quad- Evening summary (limited to 50 words) \\

\#\#\# Soft Attributes \\
- Tone and Emotion: \\ 
\quad- Day 1: Use a tone that conveys “excitement and discovery.” \\ 
\quad- Day 3: Use a tone that conveys “curiosity and wonder.” \\
- Language Style: \\ 
\quad- Use descriptive, vivid, and family-friendly language throughout. \\
\quad- Include at least one metaphor or simile per day (e.g., "The Science Museum felt like stepping into the future!"). \\
- Visual Details: \\
\quad- Each activity must include specific sensory details (e.g., "the bright colors of the parrots at the zoo" or "the tinkling sound of water fountains at the park").

- Adaptive Adjustments (Real-time Constraints): \\
\quad- Weather Sensitivity: \\
\quad\quad- If the rain forecast exceeds 60\%, replace outdoor activities with indoor alternatives while keeping the overall tone and keywords intact. \\ 
\quad- Physical Endurance: \\
\quad\quad- If a day’s total walking distance exceeds 10 kilometers, the next day’s activities must reduce walking by 30\%. \\
\quad- Health Responsiveness: \\
\quad\quad- If a health-related issue arises (e.g., fatigue or illness), adjust the itinerary dynamically to: \\
\quad\quad- Reduce activity duration to half. \\ 
\quad\quad- Substitute the activity with a more relaxing or passive option. \\
\bottomrule
    \end{tabular}
    \caption{The complete travel planner case study.}
    \label{tab:travel_planner_case}
\end{table*}
\section{Implementation Details}

% \begin{CJK}{UTF8}{gbsn}中文写在这里面\end{CJK}

In this section, we provide implementation details of all the methods used in Section \ref{sec:Experiments}. Except from Large Language Models (LLMs), all the other methods are trained on 300 epochs, with an early stopping of 5. We use Adam optimizer to update model parameters. The experiments are conducted on a linux server with Ubuntu 20.04, trained on a single NVIDIA RTX A5000 GPU with 24GB memory. All the methods are trained on train set, the hyperparameters are searched on validation set, where the search space is given by:

\begin{itemize}
    \item Hidden Dimension: \{16, 32, 64, 128\},
    \item Learning Rate: \{5e-6, 1e-5, 2e-5, 3e-5, 5e-5, 1e-4\},
    \item Weight Decay: \{1e-5, 1e-4\},
    \item Batch Size: \{16, 32\},
\end{itemize}

For the task of parody detection, the threshold for each dataset is the same for all the methods. Specific, we let the threshold be $0.9415$ for \textit{Alibaba-Math}, $0.9526$ for \textit{BridePrice}, $0.9691$ for \textit{DrinkWater}, $0.9387$ for \textit{CS2}, $0.9262$ for \textit{CampusLife}, $0.9406$ for \textit{Tiktok-Trump}, $0.8768$ for \textit{Reddit-Trump}

Prior to feeding the data into the model, we utilize over sampling with replacement for parody detection, and use Synthetic Minority Over-sampling Technique (SMOTE) \citep{chawla2002smote} for sentiment classification to balance the training data.

Apart from these common settings, we introduce the detailed implementations of each specific model as follows.

\textbf{BoW+MLP} \citep{BoW}
Bag of Words (BoW) is a kind of word embedding method. In this study, the BoW model implemented in Word2Vec \citep{BoW}, aiming to predict a target word based on its surrounding context words. Before using Bag of Words, we standardize text input, remove unnecessary whitespace variations, tokenization text into individual words, and filter out high-frequency words that may not contribute much meaning. Next, we use Bag of Words in Word2Vec to get the word embedding, setting vector size to 50, window to 10, min count to 1, epochs to 50.

Multi-Layer Perceptrons (MLP) is a kind of feedforward neural network. In our study, we employ a three-layer MLP, with a dropout rate set to 0.3 and ReLU as the activation function. 

\textbf{Skip-gram+MLP} \citep{Skip-gram}
Skip-gram is a word embedding method which learns word representations by predicting context words given a target word. Before using Skip-gram, we standardize text input, avoid unnecessary whitespace variations, the text is tokenized into individual words, and filter out high-frequency words that may not contribute much meaning. Then we use Skip-gram in Word2Vec, setting vector size to 50, window to 10, min count to 1, epochs to 50.
The part of MLP is the same as in BoW+MLP.

\textbf{RoBERTa+MLP} \citep{RoBERTa}
RoBERTa ( Robustly Optimized BERT Pretraining Approach ) is an advanced variant of BERT. The part of Next sentence prediction (NSP) is removed from RoBERTa's pre-training objective. To obtain embedding of textual data, we use mean embedding method to compute the average of token embedding from last hidden state. Setting max length to 256, batch size to 32. The part of MLP is the same as in BoW+MLP.

\textbf{BNS-Net} \citep{BNS-Net}
The propagation mechanism in BNS-Net is defined as:$H = f(X, U, W)$, where $X$ represents the textual features, $U$ denotes user embeddings, and $W$ is the weight matrix. The Behavior Conflict Channel (BCC) applies a Conflict Attention Mechanism (CAM) to extract inconsistencies in behavioral patterns, while the Sentence Conflict Channel (SCC) leverages external sentiment knowledge (e.g., SenticNet) to detect implicit and explicit contradictions. BNS-Net is trained using a multi-task loss function, which combines sarcasm classification and sentiment inconsistency modeling:
$L = \lambda_1 J_{\text{sar}} + \lambda_2 J_{\text{imp}} + \lambda_3 J_{\text{exp}} + \lambda_4 J_{\text{balance}}$, where: sar is the sarcasm classification loss,imp and exp correspond to implicit and explicit sentiment contradiction losses. Balance is a balancing term to mitigate bias toward dominant classes. The balancing coefficients used in experiments are: $\lambda_1 = 1.0$, $\quad \lambda_2 = 0.5$, $\quad \lambda_3 = 0.5$, $\quad \lambda_4 = 0.2$.

\textbf{DC-Net} \citep{DC-Net}
The Dual-Channel Network is a dual-channel architecture to realize sarcasm detection by capturing the contrast between literal sentiment and implied sentiment. The model consists of Decomposer, literal channel, implied channel and analyzer. Prior to feeding data into DC-Net, we utilize the opinion lexicon from nltk 3.9.1 to identify the positive and negative word in our datasets. Following the methodology outlined in the original paper, it needs to use GLOVE to obtain the embedding and vocabulary. To generate the literal and implied sentiment labels, we leverage the parody label along with the counts of positive and negative words. These labels are then processed separately in the two channels. Finally the analyzer measure the conflicts between the channels. In our datasets, we follow the original paper and set all of the loss contributions \(\lambda_1\), \(\lambda_2\), \(\lambda_3\) of our DC-Net model are set to 1.

\textbf{QUIET} \citep{QUIET}
The Quantum Sarcasm Model detects sarcasm in text by using quantum-inspired techniques. It converts text and context inputs into dense vector representations through an embedding layer. These embeddings undergo quantum encoding, where sine and cosine functions simulate quantum amplitude and phase encoding, capturing complex relationships. The encoded features are averaged to reduce dimensionality, then passed through a hidden layer with ReLU activation. A sigmoid output layer predicts whether a comment is sarcastic or not. The model addresses class imbalance with class weights and evaluates performance using precision, recall, and F1-score. This single-modality model applies quantum-inspired methods to enhance feature transformation for sarcasm detection.


\textbf{SarcPrompt} \citep{SarcPrompt}
is a prompt-tuning method for sarcasm recognition that enhances PLMs by incorporating prior knowledge of contradictory intentions. The framework comprises two key components: (1) Prompt Construction. (2) Verbalizer Engineering. In our implementation, we adopt the question prompt approach and design bilingual templates tailored to Chinese and English datasets. For Chinese parody detection, we construct the prompt as " \{COMMENT\} \ch{这段话是在反串吗?} \{MASK\}.". For English datasets, we design"\{COMMENT\} Are you parody? \{MASK\}." To enhance model interpretability and alignment with domain knowledge, we employ a verbalizer as paper, where domain-specific label words are mapped based on dataset statistics. In parody detection, we use words like "\ch{反串}", "\ch{是}", "parody", "no". In sentiment classification, we use words like "\ch{支持}", "\ch{反对}", "support", "oppose". The total loss combines cross-entropy (classification) and contrastive losses (enhancing intra-class consistency): $L(\theta) = \lambda_1 L_{\text{sarc}}(\theta) + \lambda_2 L_{\text{con}}(\theta)$, where \(\lambda_1 = 1\) and \(\lambda_2\) is selected from \{0.05, 0.1, 0.2, 0.5, 1\} via validation, following the original paper's hyperparameter selection.

\textbf{GCN} \citep{GCN}
All Graph Neural Networks (GNNs), including GCN, GAT, and GraphSAGE, are implemented using PyTorch Geometric \citep{pyg}, with the version specified as 2.6.1. For the GCN, we set the number of graph convolution layers to 2, the size of the hidden embedding to 64, and the dropout rate to 0.5. Additionally, we incorporate residual connections \citep{residual} and layer normalization \citep{layer_norm} to enhance model performance, as suggested by \citet{classic_gnn_strong}.

\textbf{GAT} \citep{GAT}
In GAT, we adopt the same configuration as in Graph Convolutional Networks (GCN), utilizing 2 graph convolution layers, a hidden embedding size of 64, and a dropout rate of 0.5. Additionally, we set the number of attention heads to 8.

\textbf{GraphSAGE} \citep{GraphSAGE}
In GraphSAGE, we adopt the same configuration as in Graph Convolutional Networks (GCN), utilizing 2 graph convolution layers, a hidden embedding size of 64, and a dropout rate of 0.5. Additionally, we set the neighborhood size to 5.

\textbf{LLMs}
we employ a variety of LLMs from different companies to perform parody detection and sentiment classification, which include ChatGPT-4o (and 4o-mini) \citep{GPT4}, ChatGPT-o1-mini \citep{ChatGPT-o1}, ChatGPT-o3-mini \citep{ChatGPT-o3} Claude 3.5 \citep{Claude}, Qwen 2.5 \citep{Qwen2.5}, DeepSeek-V3 \citep{DeepSeek}, and DeepSeek-R1 \citep{DeepSeek-R1}.They require different kinds of input formats, objects and parameters. Except reasoning model, we set temperature to 0, which reasoning model not support this object. For reasoning model, they have to use more and more tokens to complete the reasoning procedure before outputting the content. To optimize model performance, we design task-specific prompts, ensuring that each LLM receives input formulations tailored to the characteristics of parody detection and sentiment analysis. For example, in parody detection, we design the prompt as \textit{``You are a helpful assistant trained to classify whether a statement is parody or not.''} in the system role, and \textit{``Determine whether the following comment is parody:\{text\}\texttt{\textbackslash n} Directly output 1 for parody, 0 for non-parody.''} in the user role. In particular, ChatGPT o1-mini doesn't have the system role, so we input all in the user role.



% Mention hyperparameter search space

% Mention up-sampling strategies

% 
\section{Additional Results}

This section introduces additional results in our experiments. We introduce more results of the influence of context to parody detection in Section \ref{apd:impact_context} and the influence of parody to sentiment classification in Section \ref{apd:impact_parody}. Then, we show the performance comparison of reasoning LLMs and non-reasoning LLMs in Section \ref{apd:reasoning_llms}. Last, we investigate the impact of train ratio of embedding-based models compared with LLMs in Section \ref{apd:train_ratio}.

\subsection{Influence of Context to Parody Detection}\label{apd:impact_context}
\begin{figure}[htbp]
  \centering
  \begin{subfigure}[b]{0.23\textwidth}
    \centering
    \includegraphics[width=\textwidth]{images/compare_context/JP_CompareContext.pdf}
    \caption{Alibaba-Math}
    \label{fig:compare_context_sub1}
  \end{subfigure}
  \hfill
  \begin{subfigure}[b]{0.23\textwidth}
    \centering
    \includegraphics[width=\textwidth]{images/compare_context/BridePrice_CompareContext.pdf}
    \caption{BridePrice}
    \label{fig:compare_context_sub2}
  \end{subfigure}
  
  \vspace{0.7cm}  % 行间间距
  
  \begin{subfigure}[b]{0.23\textwidth}
    \centering
    \includegraphics[width=\textwidth]{images/compare_context/HTX_CompareContext.pdf}
    \caption{DrinkWater}
    \label{fig:compare_context_sub3}
  \end{subfigure}
  \hfill
  \begin{subfigure}[b]{0.23\textwidth}
    \centering
    \includegraphics[width=\textwidth]{images/compare_context/CS2_CompareContext.pdf}
    \caption{CS2}
    \label{fig:compare_context_sub4}
  \end{subfigure}

  \vspace{0.7cm}  % 行间间距

  \begin{subfigure}[b]{0.23\textwidth}
    \centering
    \includegraphics[width=\textwidth]{images/compare_context/NTU_CompareContext.pdf}
    \caption{CampusLife}
    \label{fig:compare_context_sub5}
  \end{subfigure}
  \hfill
  \begin{subfigure}[b]{0.23\textwidth}
    \centering
    \includegraphics[width=\textwidth]{images/compare_context/Tiktok_CompareContext.pdf}
    \caption{Tiktok-Trump}
    \label{fig:compare_context_sub6}
  \end{subfigure}

  \vspace{0.7cm}  % 行间间距

  \begin{subfigure}[b]{0.23\textwidth}
    \centering
    \includegraphics[width=\textwidth]{images/compare_context/Trump_CompareContext.pdf}
    \caption{Reddit-Trump}
    \label{fig:compare_context_sub7}
  \end{subfigure}

  \vspace{0.7cm}  % 行间间距

  \caption{Impact of contextual information on parody detection across seven datasets.}
  \label{fig:apd_impact_context}
\end{figure}

Figure \ref{fig:apd_impact_context} illustrates the detailed results of the performance comparison of the F1 score in parody detection with and without context across seven datasets. Generally, contextual information significantly enhances model performance on most datasets and methods. For instance, on \textit{Alibaba-Math}, the performance of ChatGPT4o improves from $15.9$ to $19.54$, while on \textit{BridePrice}, the performance of RoBERTa+MLP increases from $19.17$ to $32.50$. These results indicate that contextual information is beneficial for parody detection. This finding aligns with the results in \citet{dialogue_bamman, dialogue_wang}, which show that providing dialogue as context significantly improves model performance in sarcasm detection.

However, although contextual information significantly improves model performance on most datasets, there are still some datasets where context does not enhance or even decreases model performance. For example, on \textit{Tiktok-Trump}, the model performance decreases, and on \textit{CampusLife}, the performance remains similar after adding contextual information. This suggests that contextual information may not always contribute to improving model performance in parody detection.

\subsection{Influence of Parody to Sentiment Classification}\label{apd:impact_parody}
\begin{figure}[htbp]
  \centering
  \begin{subfigure}[b]{0.23\textwidth}
    \centering
    \includegraphics[width=\textwidth]{images/ORPtoSenti/JP_ORPtoSenti.pdf}
    \caption{Alibaba-Math}
    \label{fig:ORPtoSenti_sub1}
  \end{subfigure}
  \begin{subfigure}[b]{0.23\textwidth}
    \centering
    \includegraphics[width=\textwidth]{images/ORPtoSenti/BridePrice_ORPtoSenti.pdf}
    \caption{BridePrice}
    \label{fig:ORPtoSenti_sub2}
  \end{subfigure}

  \vspace{0.7cm}
  
  \begin{subfigure}[b]{0.23\textwidth}
    \centering
    \includegraphics[width=\textwidth]{images/ORPtoSenti/HTX_ORPtoSenti.pdf}
    \caption{DrinkWater}
    \label{fig:ORPtoSenti_sub3}
  \end{subfigure} 
  \begin{subfigure}[b]{0.23\textwidth}
    \centering
    \includegraphics[width=\textwidth]{images/ORPtoSenti/CS2_ORPtoSenti.pdf}
    \caption{CS2}
    \label{fig:ORPtoSenti_sub4}
  \end{subfigure}

  \vspace{0.7cm}

  \begin{subfigure}[b]{0.23\textwidth}
    \centering
    \includegraphics[width=\textwidth]{images/ORPtoSenti/NTU_ORPtoSenti.pdf}
    \caption{CampusLife}
    \label{fig:ORPtoSenti_sub5}
  \end{subfigure}
  \begin{subfigure}[b]{0.23\textwidth}
    \centering
    \includegraphics[width=\textwidth]{images/ORPtoSenti/Tiktok_ORPtoSenti.pdf}
    \caption{Tiktok-Trump}
    \label{fig:ORPtoSentit_sub6}
  \end{subfigure}

  \vspace{0.7cm}

  \begin{subfigure}[b]{0.23\textwidth}
    \centering
    \includegraphics[width=\textwidth]{images/ORPtoSenti/Trump_ORPtoSenti.pdf}
    \caption{Reddit-Trump}
    \label{fig:ORPtoSenti_sub7}
  \end{subfigure}

  \vspace{0.7cm}  % 行间间距

  \caption{Impact of parody on comment sentiment classification across seven datasets.}
  \label{fig:apd_impact_parody}
\end{figure}

Figure \ref{fig:apd_impact_parody} presents the detailed model performance of comment sentiment classification on parody and non-parody comments across seven datasets. In the \textit{DrinkWater} dataset, large language models (LLMs) such as ChatGPT-4o-mini (F1-score: 51.42) and Qwen2.5 (F1-score: 47.00) achieve competitive performance compared to embedding-based methods like Bag of Words (BoW) (F1-score: 48.21), Skip-gram (F1-score: 47.11), and RoBERTa (F1-score: 44.93) when parody is not present. However, for parody comments, the performance of LLMs degrades significantly, falling below that of embedding-based approaches. For instance, ChatGPT-4o drops from an F1-score of 48.7 to 19.04, and ChatGPT-4o-mini declines from 51.42 to 15.53, whereas embedding-based methods exhibit greater robustness, with BoW decreasing from 48.21 to 36.21, Skip-gram from 47.11 to 32.35, and RoBERTa from 44.93 to 33.83. Overall, these results indicate that parody presents substantial challenges for sentiment classification, and LLMs struggle to maintain their advantage over traditional embedding-based methods in this context.

\subsection{Reasoning LLMs in Parody Detection}\label{apd:reasoning_llms}
\begin{figure}[htbp]
  \centering
  \begin{subfigure}[b]{0.23\textwidth}
    \centering
    \includegraphics[width=\textwidth]{images/LLMCompare/BridePrice_LLMCompare.pdf}
    \caption{BridePrice}
    \label{fig:LLM_compare_sub2}
  \end{subfigure}
  \hfill
  \begin{subfigure}[b]{0.23\textwidth}
    \centering
    \includegraphics[width=\textwidth]{images/LLMCompare/HTX_LLMCompare.pdf}
    \caption{DrinkWater}
    \label{fig:LLM_compare_sub3}
  \end{subfigure}

  \vspace{0.7cm}  % 行间间距

  \begin{subfigure}[b]{0.23\textwidth}
    \centering
    \includegraphics[width=\textwidth]{images/LLMCompare/CS2_LLMCompare.pdf}
    \caption{CS2}
    \label{fig:LLM_compare_sub4}
  \end{subfigure}
  \hfill
  \begin{subfigure}[b]{0.23\textwidth}
    \centering
    \includegraphics[width=\textwidth]{images/LLMCompare/NTU_LLMCompare.pdf}
    \caption{CampusLife}
    \label{fig:LLM_compare_sub5}
  \end{subfigure}

  \vspace{0.7cm}  % 行间间距

  \begin{subfigure}[b]{0.23\textwidth}
    \centering
    \includegraphics[width=\textwidth]{images/LLMCompare/Tiktok_LLMCompare.pdf}
    \caption{Tiktok-Trump}
    \label{fig:LLM_compare_sub6}
  \end{subfigure}
  \hfill
  \begin{subfigure}[b]{0.23\textwidth}
    \centering
    \includegraphics[width=\textwidth]{images/LLMCompare/Trump_LLMCompare.pdf}
    \caption{Reddit-Trump}
    \label{fig:LLM_compare_sub7}
  \end{subfigure}

  \vspace{0.7cm}  % 行间间距

  \caption{A Comparative Performance Analysis of Reasoning vs. Non-Reasoning LLMs}
  \label{fig:apd_reasoning_llms}
\end{figure}

We present the details of reasoning LLMs in parody detection across six datasets in Figure \ref{fig:apd_reasoning_llms}. Our findings indicate that reasoning LLMs do not exhibit a performance advantage compared to non-reasoning LLMs. For instance, ChatGPT-o1-mini and ChatGPT-o3-mini underperform relative to ChatGPT4o-mini on the \textit{CampusLife} and \textit{Tiktok-Trump} datasets. Additionally, DeepSeek-R1 significantly underperforms compared to DeepSeek-V3 across all datasets. 

These results suggest that reasoning does not enhance LLM performance in parody detection. We speculate that this may be due to the nature of parody, which often relies on indirect or subtle cues related to tone, context, and nuance rather than direct logical inference. In such cases, non-reasoning LLMs, which excel at identifying statistical patterns and linguistic structures, may be more effective at detecting parody than reasoning LLMs that focus excessively on logical steps or detailed analysis.

\subsection{Impact of Supervision Ratio}\label{apd:train_ratio}

\begin{figure}[h]
  \begin{subfigure}[b]{0.23\textwidth}
    \centering
    \includegraphics[width=\textwidth]{images/fanchuan_image/parody_BridePrice_1_new.pdf}
    \caption{BridePrice}
    \label{fig:parody_sub2}
  \end{subfigure}
  \hfill
  \begin{subfigure}[b]{0.23\textwidth}
    \centering
    \includegraphics[width=\textwidth]{images/fanchuan_image/parody_NTU_2_new.pdf}
    \caption{CampusLife}
    \label{fig:parody_sub5}
  \end{subfigure}
  
  \vspace{0.7cm}

  \caption{Impact of training ratio to RoBERTa+MLP on parody detection}
  \label{fig:train_ratio_parody_detection}
\end{figure}

\begin{figure}[h]
  \centering
  \begin{subfigure}[b]{0.23\textwidth}
    \centering
    \includegraphics[width=\textwidth]{images/sentiment_image/senti_BridePrice_1.pdf}
    \caption{BridePrice}
    \label{fig:sentiment_sub2}
  \end{subfigure}
  \hfill
  \begin{subfigure}[b]{0.23\textwidth}
    \centering
    \includegraphics[width=\textwidth]{images/sentiment_image/senti_NTU_1.pdf}
    \caption{CampusLife}
    \label{fig:sentiment_sub5}
  \end{subfigure}

  \vspace{0.7cm}  % 行间间距

  \caption{Impact of training ratio to RoBERTa+MLP on comment sentiment classification}
  \label{fig:train_ratio_comment_senti}
\end{figure}
\begin{figure}[h]
  \centering
  \begin{subfigure}[b]{0.23\textwidth}
    \centering
    \includegraphics[width=\textwidth]{images/user_sentiment_image/user_senti_BridePrice_3.pdf}
    \caption{BridePrice}
    \label{fig:user_sentiment_sub2}
  \end{subfigure}
  \hfill
  \begin{subfigure}[b]{0.23\textwidth}
    \centering
    \includegraphics[width=\textwidth]{images/user_sentiment_image/user_senti_NTU_3.pdf}
    \caption{CampusLife}
    \label{fig:user_sentiment_sub5}
  \end{subfigure}

  \vspace{0.7cm}  % 行间间距
  
  \caption{Impact of training ratio to RoBERTa+MLP on user sentiment classification}
  \label{fig:train_ratio_user_senti}
\end{figure}

The embedding-based methods used in our experiments require explicit training on labeled data, whereas LLMs like RoBERTa do not require such training once pre-trained. Therefore, the performance of embedding-based models depends on the size and quality of the training set. To explore this, we investigate how varying the training ratio influences model performance by gradually increasing the training set size while keeping the test set constant. The results for RoBERTa+MLP under different train ratio are presented in Figures \ref{fig:train_ratio_parody_detection}, \ref{fig:train_ratio_comment_senti}, and \ref{fig:train_ratio_user_senti} for parody detection, comment sentiment classification, and user sentiment classification. In all tasks, we observe that the performance increases monotonically with the training ratio, highlighting the benefit of additional training data for embedding-based methods.

In addition, on the \textit{BridePrice} dataset, only $10\%$ supervision is enough for RoBERTa to outperform all LLMs in parody detection, indicating a limitation of LLMs in domain-specific tasks. This suggests that fine-tuned models like RoBERTa perform better with minimal supervision in specialized contexts. In contrast, on the \textit{CampusLife} dataset, RoBERTa's performance consistently falls below that of all LLMs, regardless of the training ratio. This suggests that LLMs are more effective in tasks requiring generalizable knowledge and flexibility, such as parody detection in diverse, context-rich domains. These results demonstrate that LLMs remain powerful in specific areas requiring flexibility in adapting to diverse linguistic contexts and nuanced understanding, while embedding-based models like RoBERTa excel in more targeted, domain-specific tasks.

\input{appendix/TrainRatio}

%%% Deprecated
% \begin{table*}[ht]
\centering
\caption{ORP Detection}
\label{tab:ORP_detection}
\resizebox{1\textwidth}{!}{
\begin{tabular}{ll|ccc|ccc|ccc|ccc|ccc|ccc|ccc|c}
\toprule
\multirow{2}{*}{\textbf{Paradigm}} & \multirow{2}{*}{\textbf{Method}} & \multicolumn{3}{c}{\textbf{Alibaba-Math}} & \multicolumn{3}{c}{\textbf{BridePrice}} & \multicolumn{3}{c}{\textbf{Student-He}} & \multicolumn{3}{c}{\textbf{CS2}} & \multicolumn{3}{c}{\textbf{CampusLife}} & \multicolumn{3}{c}{\textbf{Tiktok-Trump}} & \multicolumn{3}{c|}{\textbf{Reddit-Trump}} & \multirow{2}{*}{\textbf{\shortstack{Ave. \\ Rank}}} \\ \cline{3-23}
& & \textbf{P} & \textbf{R} & \textbf{F1} & \textbf{P} & \textbf{R} & \textbf{F1} & \textbf{P} & \textbf{R} & \textbf{F1} & \textbf{P} & \textbf{R} & \textbf{F1} & \textbf{P} & \textbf{R} & \textbf{F1} & \textbf{P} & \textbf{R} & \textbf{F1} & \textbf{P} & \textbf{R}& \textbf{F1} \\
\midrule
\multirow{3}{*}{Embedding-based} 
& BoW+MLP & 10.07 & 10.28 & 10.17 & 14.62 & 17.27 & 15.83 & 8.77 & 9.38 & 9.06 & 15.93 & 15.93 & 15.93 & 10.37 & 12.17 & 11.20 & 16.00 & 12.00 & 13.71 & 15.77 & 18.22 & 16.91 & 10.29 \\
& Skip-gram+MLP & 14.01 & 14.31 & 14.16 & 16.15 & 19.09 & 17.50 & 14.12 & 15.00 & 14.55 & 17.29 & 17.29 & 17.29 & 9.63 & 11.30 & 10.40 & 18.00 & 13.50 & 15.43 & 13.85 & 16.00 & 14.85 & 8.43 \\
& RoBERTa+MLP & 14.15 & 14.44 & 14.30 & 17.69 & 20.91 & 19.17 & 12.94 & 13.75 & 13.33 & 16.61 & 16.61 & 16.61 & 19.26 & 14.07 & 16.52 & 14.00 & 10.50 & 12.00 & 21.54 & 24.89 & 23.09 & 7.29 \\
\midrule
\multirow{4}{*}{Inconsistency-based} 
& BNS-Net & 11.55 & 16.59 & 13.62 & 12.12 & 12.50 & 12.31 & 28.57 & 11.76 & \underline{16.67} & 28.12 & 15.52 & 20.00 & 28.57 & 27.78 & 28.17 & 24.21 & 25.56 & 24.86 & 18.18 & 15.38 & 16.67 & 6.86 \\
& DC-Net & 9.84 & 21.74 & 13.54 & 9.09 & 12.50 & 10.53 & 15.09 & 20.51 & \textbf{17.39} & 8.65 & 42.59 & 14.37 & 12.12 & 16.67 & 14.04 & 7.50 & 12.50 & 9.38 & 18.56 & 34.62 & 24.16 & 10.14 \\
& QUIET & 12.45 & 22.37 & \underline{15.98} & 6.75 & 23.79 & 10.75 & 2.91 & 16.40 & 4.94 & 5.42 & 13.66 & 7.75 & 9.82 & 19.70 & 13.07 & 7.03 & 18.12 & 10.11 & 14.60 & 18.62 & 16.34 & 11.43 \\
& SarcPrompt & 48.00 & 8.33 & 14.20 & 21.74 & 22.73 & \textbf{22.22} & 16.67 & 3.12 & 5.26 & 15.62 & 33.90 & \textbf{21.39} & 57.14 & 17.39 & 26.67 & 11.11 & 25.00 & 15.38 & 13.11 & 17.78 & 15.09 & 7.29 \\
\midrule
\multirow{3}{*}{Outlier Detection} 
& Isolation Forest & 5.93 & 5.93 & 5.93 & 1.18 & 1.19 & 1.18 & 0.91 & 0.88 & 0.90 & 7.14 & 7.14 & 7.14 & 5.62 & 5.62 & 5.62 & 6.12 & 6.19 & 6.15 & 11.70 & 11.70 & 11.70 & 15.71 \\
& RoBERTa+Z-Score & 12.93 & 13.19 & 13.06 & 19.23 & 22.73 & \underline{20.83} & 12.12 & 12.50 & 12.31 & 18.64 & 18.64 & 18.64 & 18.18 & 17.39 & 17.78 & 16.67 & 12.50 & 14.29 & 32.00 & 17.78 & 22.68 & 7.71 \\
& One-Class SVM & 5.89 & 5.73 & 5.81 & 4.65 & 4.76 & 4.71 & 1.82 & 1.77 & 1.79 & 5.67 & 5.61 & 5.64 & 7.78 & 7.87 & 7.82 & 9.00 & 9.28 & 9.14 & 14.77 & 15.20 & 14.99 & 15.14 \\
\midrule
\multirow{7}{*}{LLMs} 
& ChatGPT4o & 9.09 & 63.58 & 15.90 & 7.48 & 71.43 & 13.54 & 4.96 & 45.45 & 8.94 & 10.95 & 68.04 & 18.86 & 34.89 & 33.71 & 34.29 & 32.88 & 49.48 & \textbf{39.51} & 25.28 & 70.83 & \textbf{37.26} & \textbf{3.86} \\
& ChatGPT4o-mini & 7.82 & 55.83 & 13.73 & 6.07 & 61.90 & 11.06 & 4.97 & 42.48 & 8.91 & 9.52 & 50.00 & 16.00 & 31.06 & 56.18 & \underline{40.00} & 24.47 & 71.13 & \underline{36.41} & 25.69 & 65.50 & \underline{36.90} & 6.64 \\
& Claude3.5 & 7.25 & 74.69 & 13.21 & 6.74 & 85.71 & 12.49 & 4.64 & 55.45 & 8.56 & 9.00 & 58.25 & 16.00 & 29.70 & 67.42 & \textbf{41.24} & 19.05 & 70.10 & 29.96 & 24.69 & 69.59 & 36.45 & 6.93 \\
& Qwen2.5 & 8.48 & 60.33 & 14.88 & 6.86 & 66.67 & 12.44 & 4.31 & 41.59 & 7.81 & 11.48 & 62.24 & 19.38 & 18.01 & 73.03 & 28.89 & 16.78 & 79.38 & 27.70 & 21.40 & 74.85 & 33.29 & 6.29 \\
& DeepSeek-V3 & 9.34 & 59.92 & \textbf{16.17} & 7.31 & 70.24 & 13.24 & 5.18 & 40.71 & 9.19 & 12.56 & 55.10 & \underline{20.45} & 21.23 & 69.66 & 32.55 & 19.22 & 81.44 & 31.10 & 22.44 & 73.10 & 34.34 & \underline{4.86} \\
& DeepSeek-R1 & 7.90 & 76.89 & 14.33 & 6.61 & 85.71 & 12.27 & 4.29 & 60.18 & 8.01 & 9.07 & 75.00 & 16.18 & 13.11 & 84.27 & 22.69 & 11.93 & 88.66 & 21.03 & 18.54 & 81.87 & 30.24 & 8.14 \\
& ChatGPT-o3-mini & 0 & 0 & 0 & 0 & 0 & 0 & 0 & 0 & 0 & 0 & 0 & 0 & 0 & 0 & 0 & 0 & 0 & 0 & 0 & 0 & 0 & 0 \\
& ChatGPT-o1 & 0 & 0 & 0 & 0 & 0 & 0 & 0 & 0 & 0 & 0 & 0 & 0 & 0 & 0 & 0 & 0 & 0 & 0 & 0 & 0 & 0 & 0 \\
\bottomrule
\end{tabular}
}
\end{table*}



\begin{table*}[ht]
\centering
\caption{New Template (Delete this after finishing)}
\label{tab:ORP_detection}
\resizebox{1\textwidth}{!}{
\begin{tabular}{ll|ccccccc|c}
\toprule
\textbf{Paradigm} & \textbf{Method} & \textbf{Alibaba.} & \textbf{Bride.} & \textbf{Student.} & \textbf{CS2} & \textbf{Campus.} & \textbf{Tiktok.} & \textbf{Reddit.} & \textbf{Ave. Rank} \\
\midrule
\multirow{3}{*}{Embedding-based} 
& BoW+MLP & 10.07 & 10.28 & 10.17 & 14.62 & 17.27 & 15.83 & 8.77 & 9.38  \\
& Skip-gram+MLP & 10.07 & 10.28 & 10.17 & 14.62 & 17.27 & 15.83 & 8.77 & 9.38  \\
& RoBERTa+MLP & 10.07 & 10.28 & 10.17 & 14.62 & 17.27 & 15.83 & 8.77 & 9.38  \\
\bottomrule
\end{tabular}
}
\end{table*}



% It hosts all the active services in the system. These services are built on top of the sensor data in Knowledge Management, which stores the data of the IoT environment (e.g., routes service, crowd monitoring, and air quality sensors). The configurations for these services are maintained in Knowledge Management, which is updated if the underlying data changes.
% \paragraph{Task} Here we measure the ability to direct generation towards positive sentiment from some initial prompt. Given an initial prompt, we generate sequences of length 12, 20, 50 as done in prior works \citep{kumar2022gradient, liu2023bolt}. We use the same set of prompts as \citet{dathathri2020plugplaylanguagemodels} and include them in Appendix \ref{appndx:senti-details}.

\paragraph{Control Metrics} We evaluate control by measuring the predicted sentiment of the generation using three distinct sentiment classifiers.
For details on the training of the three classifiers, see Appendix \ref{appndx:senti-details}. 
We omit the internal classifier measure for LM-Steer as it does not rely on an internal classifier to guide generation. 

\paragraph{Results}
Table \ref{table:megatable} shows that our method achieves a better balance between control and fluency than baselines. DAB achieves the highest average probability of positive sentiment across all three classifiers, demonstrating its effectiveness at incorporating the external constraint.  Furthermore, DAB achieves fluency scores close to BOLT's performance in regards to CoLA score, repeated trigrams, and perplexity. This shows that DAB produces generations that are both fluent and satisfactory under the constraint. 
% \begin{table*}[ht]
\label{tab:user_senti_cls}
\centering
\caption{User sentiment classification.}
\resizebox{1\textwidth}{!}{
\begin{tabular}{ll|cc|cc|cc|cc|cc|cc|cc|c}
\toprule
\multirow{2}{*}{\textbf{Paradigm}} & \multirow{2}{*}{\textbf{Method}} & \multicolumn{2}{c}{\textbf{Alibaba-Math}} & \multicolumn{2}{c}{\textbf{BridePrice}} & \multicolumn{2}{c}{\textbf{Student-He}} & \multicolumn{2}{c}{\textbf{CS2}} & \multicolumn{2}{c}{\textbf{CampusLife}} & \multicolumn{2}{c}{\textbf{Tiktok-Trump}} & \multicolumn{2}{c|}{\textbf{Reddit-Trump}} & \multirow{2}{*}{\textbf{\shortstack{Ave. \\ Rank}}} \\ \cline{3-16}
& & \textbf{Macro-F1} & \textbf{Micro-F1} & \textbf{Macro-F1} & \textbf{Micro-F1} & \textbf{Macro-F1} & \textbf{Micro-F1} & \textbf{Macro-F1} & \textbf{Micro-F1} & \textbf{Macro-F1} & \textbf{Micro-F1} & \textbf{Macro-F1} & \textbf{Micro-F1} & \textbf{Macro-F1} & \textbf{Micro-F1} \\
\midrule
\multirow{3}{*}{Embedding-based} 
& BoW+MLP & \underline{46.54} & 48.74 & 37.60 & 60.90 & 46.65 & 60.41 & 29.22 & 43.03 & 32.35 & 52.16 & 35.05 & 44.53 & 31.97 & 40.17 & 0 \\
& Skip-gram+MLP & \textbf{46.99} & 48.75 & 38.28 & 59.26 & 50.45 & 64.54 & 31.92 & 45.38 & 32.02 & 52.63 & 38.46 & 47.28 & 32.69 & 42.49 & 0 \\
& RoBERTa+MLP & 43.11 & 45.77 & 36.94 & 61.01 & 44.20 & 63.12 & 27.09 & 42.90 & 35.49 & 61.64 & 50.82 & 60.22 & 52.79 & 58.32 & 0 \\
\midrule
\multirow{3}{*}{Inconsistency-based} 
& BNS-Net & 34.32 & 49.38 & 27.21 & 68.97 & 41.91 & 78.33 & 23.38 & 60.19 & 28.67 & 75.44 & 23.61 & 54.84 & 22.98 & 52.60 & 0 \\
& DC-Net & 16.51 & 26.92 & 33.56 & 63.13 & 48.65 & 68.30 & 17.17 & 42.36 & 35.60 & 75.44 & 34.62 & 49.73 & 39.54 & 50.29 & 0 \\
& SarcPrompt & 27.72 & 53.75 & 38.54 & 71.81 & 29.51 & 79.40 & 15.62 & 57.80 & 31.45 & 69.01 & 24.48 & 53.37 & 39.10 & 56.52 & 0 \\
\midrule
\multirow{3}{*}{Graph-based} 
& GCN & 23.07 & \textbf{52.92} & 39.45 & 69.95 & 44.70 & 72.22 & 14.65 & 57.80 & 36.18 & 65.50 & 30.48 & 56.33 & 48.09 & 58.26 & 0 \\
& GAT & 23.29 & 52.67 & 37.63 & 65.69 & 45.22 & 76.80 & 15.65 & 57.96 & 37.64 & 65.50 & 43.19 & 54.99 & 48.63 & 57.10 & 0 \\
& GraphSAGE & 23.07 & \textbf{52.92} & 37.78 & 69.68 & 45.22 & 75.03 & 14.65 & 57.80 & 32.37 & 63.16 & 41.12 & 56.87 & 51.31 & 60.00 & 0 \\
\midrule
\multirow{5}{*}{LLMs} 
& ChatGPT4o & 41.71 & 44.62 & 35.02 & 46.73 & 51.54 & 65.18 & 39.19 & 56.52 & 35.89 & 45.87 & 45.87 & 51.50 & 49.01 & 60.82 & 0 \\
& ChatGPT4o-mini & 40.55 & 41.47 & 30.25 & 37.88 & 45.88 & 54.95 & 34.03 & 49.59 & 31.95 & 39.57 & 45.29 & 49.39 & 51.20 & 61.30 & 0 \\
& Claude3.5 & 41.47 & 43.02 & 29.96 & 36.58 & 43.78 & 54.38 & 32.81 & 50.17 & 31.07 & 37.83 & 41.85 & 48.40 & 46.92 & 60.56 & 0 \\
& Qwen2.5 & 40.89 & 44.13 & 33.08 & 44.82 & 49.52 & 59.93 & 36.51 & 56.38 & 33.34 & 41.55 & 46.18 & 51.17 & 50.13 & 61.50 & 0 \\
& DeepSeek-V3 & 40.00 & 40.42 & 26.37 & 31.42 & 41.55 & 45.07 & 33.61 & 54.13 & 40.49 & 51.61 & 54.04 & 58.69 & 53.22 & 63.58 & 0 \\
\bottomrule
\end{tabular}
}
\end{table*}


\end{document}


% \section{To do List}
% \begin{itemize}
%     \item \textcolor{red}{Paper Format}
%     \begin{itemize}
%         \item \finish{0} LAST DAY CHECK: Name of datasets and methods. Numbers in paper
%         \item \finish{0}
%     \end{itemize}
%     \item Data Preprocess
%     \begin{itemize}
%         \item \finish{100} Re-check all the FanChuan comments to see if they are really fanchaun. Use LLM API to retrieve some unlabeled FanChuan.
%         \item \finish{100} filter out irrelevant comments and delete privacy info using LLM API.
%     \end{itemize}
%     \item Model Performance
%     \begin{itemize}
%         \item \finish{90} Implementation of Supervised Methods
%         \item \finish{100} Implementation of Unsupervised Methods
%         \item \finish{100} Implementation of LLMs api
%         \item \finish{90} Train model under different training ratio: 5\%, 10\%, ..., 70\%. Plot curve.
%     \end{itemize}
%     \item User classification tasks
%     \begin{itemize}
%         \item \finish{0} Construct dataset.  
%         \item \finish{0} Add GNN baselines.
%         \item \finish{0} Run all the models.
%     \end{itemize}
%     \item \finish{50} FanChuan Context: Add context information to all of the above methods to see the improvement. Plot bar Figure.
%     \item \finish{0} Show differences of model performance of sentiment classification on FanChuan comments vs non-FanChuan comments. Plot bar figure.
%     \item \finish{0} Calculate homophily degrees of each user. See what low/high homophily degree nodes looks like.
%     \item \finish{0} Case Study: Pick some cases when LLMs fail or success. Ask LLMs to give explanations
% \end{itemize}