\begin{table}[h]
    \small
    \centering
    \resizebox{1.0\linewidth}{!}{
    \begin{tabular}{p{1.0\linewidth}}
        \toprule
        \textbf{Question} \\
        \begin{CJK}{UTF8}{gbsn}
        对于“男朋友彩礼给的少工资卡也不愿意上交我很犹豫怎么办?”这一话题,有网友评论:“
        这种男的不要也罢。姐妹要相信自己的价值,好男人很多,一定不要委屈自己。
        一生一次他还不愿意付出所有留退路,你觉得那个退路会是谁?消耗掉的可是你无价的青春啊。”\end{CJK} \\
        \begin{CJK}{UTF8}{gbsn}
        请判断这条评论是否为反串,并给出50字以内的解释。\end{CJK} \\
        Regarding the topic "My boyfriend gives little betrothal money and refuses to hand over his salary card—I'm hesitant, what should I do?", a netizen commented:  
        "This kind of man isn't worth it. Sis, believe in your value—there are many good men, never settle for less."  
        "If he won’t give his all for a once-in-a-lifetime event and leaves himself an out, who do you think that’s for? What’s being wasted is your priceless youth." \\
        Please determine whether this comment is satire and provide an explanation within 50 words.\\
        \textbf{Gronud Truth} \\
        \begin{CJK}{UTF8}{gbsn}
        是。这条评论通过使用“姐妹”这一称呼模仿女性视角,并以极端立场全面否定男性。\end{CJK} \\
        Yes. The comment mimics a female perspective by using the term "sister" and takes an extreme stance by completely negating men. \\
        \midrule
        \textbf{ChatGPT-4o} \ \textcolor{decrease}{\faTimes\ Wrong Answer} \\
        \begin{CJK}{UTF8}{gbsn}
        不是。这条评论表达了女性在婚姻中应得到足够的经济保障和重视,强调自我价值和青春无价,反映了对传统彩礼和工资卡上交的重视,并非反串。\end{CJK} \\
        No. This comment expresses that women should be given adequate financial security and attention in marriage, emphasizes self-worth and the pricelessness of youth, and reflects the importance attached to traditional betrothal gifts and salary card submission. It is not a reverse portrayal. \\
        \midrule
        \textbf{Qwen2.5} \ \textcolor{decrease}{\faTimes\ Wrong Answer} \\
        \begin{CJK}{UTF8}{gbsn}
        不是。这条评论明确支持女性追求平等对待,强调自我价值和寻找更合适伴侣的重要性,没有表现出相反意图或讽刺意味。\end{CJK}  \\
        No. This comment clearly supports women's pursuit of equal treatment, emphasizes the importance of self-worth and finding a more suitable partner, and does not show any contrary intention or irony. \\
        \midrule
        \textbf{DeepSeek-V3} \ \textcolor{decrease}{\faTimes\ Wrong Answer} \\
        \begin{CJK}{UTF8}{gbsn}
        不是。这条评论表达了女性在婚姻中应得到足够的经济保障和重视,强调自我价值和青春无价,反映了对传统彩礼和工资卡上交的重视,并非反串。\end{CJK} \\
        No. This comment expresses that women should be given adequate financial security and attention in marriage, emphasizes self-worth and the pricelessness of youth, and reflects the importance attached to traditional betrothal gifts and salary card submission. It is not a reverse portrayal. \\
        \midrule
        \textbf{Claude3.5} \ \textcolor{decrease}{\faTimes\ Wrong Answer} \\
        \begin{CJK}{UTF8}{gbsn}
        不是。该评论语气、用词和观点都符合特定群体的真实想法,包括"姐妹""价值""青春"等关键词的使用方式自然,不存在夸张或讽刺的痕迹。\end{CJK} \\
        No. The tone, words and opinions of the comment are consistent with the real thoughts of a specific group, including the use of keywords such as "sisters", "values" and "youth" in a natural way without any trace of exaggeration or sarcasm. \\
       \bottomrule
    \end{tabular}
    }
    \caption{A case study of LLMs using an example from \textit{BridePrice}.}
    \label{tab:casestudy_brideprice}
\end{table}

\begin{table}[h]
    \small
    \centering
    \resizebox{1.0\linewidth}{!}
    {
    \begin{tabular}{p{1.0\linewidth}}
        \toprule
        \textbf{Question} \\
        \begin{CJK}{UTF8}{gbsn}
        对于“2024年阿里巴巴数学竞赛预赛中,一位来自中等专业学校的女生取得12名的优秀成绩”这一话题,有网友评论道:“这位同学有实力!阿里巴巴有眼光! 请阿里巴巴破格录取进入达摩院,助力阿里科技快速发展"\end{CJK} \\
        \begin{CJK}{UTF8}{gbsn}
        请判断这条评论是否为反串,并给出50字以内的解释。\end{CJK} \\
        Regarding the topic "My boyfriend gives little betrothal money and refuses to hand over his salary card—I'm hesitant, what should I do?", a netizen commented:  
        "This kind of man isn't worth it. Sis, believe in your value—there are many good men, never settle for less."  
        "If he won’t give his all for a once-in-a-lifetime event and leaves himself an out, who do you think that’s for? What’s being wasted is your priceless youth." \\
        Please determine whether this comment is satire and provide an explanation within 50 words.\\
        \textbf{Gronud Truth} \\
        \begin{CJK}{UTF8}{gbsn}
        是。这条评论通过对这位同学夸张的赞誉以及对阿里巴巴达摩院破格录取的期望,来引起这位同学反对者的争论。\end{CJK} \\
        Yes. This comment aroused controversy among the student’s opponents by exaggerating the praise for the student and expressing the hope that he would be admitted to Alibaba Damo Academy as an exception.\\
        \midrule
        \textbf{ChatGPT-4o} \ \textcolor{increase}{\faCheck\ Correct Answer, \raisebox{-0.7mm}{\includegraphics[width=0.4cm]{images/emoji/+1.png}} Good Explanation} \\
        \begin{CJK}{UTF8}{gbsn}
        是,因其夸张用词(如“破格录取”)和不匹配的期待(达摩院门槛极高)。若非反串,也可能是戏谑或调侃,并非严肃建议。\end{CJK} \\
        Yes, due to its exaggerated wording (e.g., "exceptional admission") and unrealistic expectations (since DAMO Academy has extremely high entry requirements). If not parody, it could be playful or mocking rather than a serious suggestion.\\
        \midrule
        \textbf{Qwen2.5} \ \textcolor{decrease}{\faTimes\ Wrong Answer} \\
        \begin{CJK}{UTF8}{gbsn}
        不是。它表达了对姜萍实力的认可和对阿里巴巴的期待,希望阿里能破格录取她,支持科技发展。语气积极正面。\end{CJK}  \\
        No. It expresses recognition of This student’s abilities and expectations for Alibaba, hoping the company will make an exception to admit her and support technological development. The tone is positive and optimistic. \\
        \midrule
        \textbf{DeepSeek-V3} \ \textcolor{increase}{\faCheck\ Correct Answer, \raisebox{-0.7mm}{\includegraphics[width=0.4cm]{images/emoji/+1.png}} Good Explanation} \\
        \begin{CJK}{UTF8}{gbsn}
        是,因其将中专背景与破格进入顶级研究院关联,隐含对选拔标准的不信任,通过夸张建议暗讽现实中的学历门槛或企业宣传策略,点赞表情强化了反讽效果。\end{CJK} \\
        Yes, as it links a vocational school background with exceptional admission to a top research institute, implying skepticism towards the selection criteria. The exaggerated suggestion subtly mocks real-world academic barriers or corporate PR strategies, with the thumbs-up emoji reinforcing the parodic tone. \\
        \midrule
        \textbf{Claude3.5} \ \textcolor{decrease}{\faTimes\ Wrong Answer} \\
        \begin{CJK}{UTF8}{gbsn}
        不是。评论者真诚肯定了姜萍的实力,并对阿里巴巴表达了期望。用语朴实,表达方式也符合普通网友的评论习惯,包括使用表情来强调观点。\end{CJK} \\
        No, The commenter sincerely acknowledges this student's capabilities and expresses expectations for Alibaba. The language is simple and aligns with the typical commenting style of regular netizens, including the use of emojis to emphasize their point. \\
       \bottomrule
    \end{tabular}
    }
    \caption{A case study of LLMs using an example from \textit{Alibaba-Math}.}
    \label{tab:casestudy_alibabamath}
\end{table}


\begin{table}[h]
    \small
    \centering
    \resizebox{1.0\linewidth}{!}
    {
    \begin{tabular}{p{1.0\linewidth}}
        \toprule
        \textbf{Question} \\
        \begin{CJK}{UTF8}{gbsn}
        关于“一位博主为了鼓励员工喝水,将茶水吧改成游戏厅”的话题中,有网友评论:“构思的很好很巧妙,好构思啊,构思.”\end{CJK} \\
        \begin{CJK}{UTF8}{gbsn}
        请判断这条评论是否为反串,并给出50字以内的解释。\end{CJK} \\
        Regarding the topic of "A video creator encouraged employees to drink more water by turning the tea station into a game arcade," a netizen commented: "The idea is well thought out and very clever. Great idea, really great idea."  \\
        Please determine whether this comment is parody and provide an explanation within 50 words. \\
        
        \textbf{Gronud Truth} \\
        \begin{CJK}{UTF8}{gbsn}
        是。这条评论通过"构思"谐音“狗屎”,来嘲讽该博主的方法缺乏实质内容,不切实际。\end{CJK} \\
        Yes. This comment mocks the video creator's method as lacking substance and being impractical by using the Chinese homophone for "idea" which is pronounced as "dog shit". \\
        \midrule
        \textbf{ChatGPT-4o} \ \textcolor{increase}{\faCheck\ Correct Answer}, \raisebox{-0.7mm}{\includegraphics[width=0.4cm]{images/emoji/_1.png}} \textcolor{decrease}{Bad Explanation} \\
        \begin{CJK}{UTF8}{gbsn}
        是。重复“构思”一词带有夸张和调侃意味,结合“很巧妙”可能是在讽刺方案脱离实际,甚至适得其反,体现出一种戏谑或反讽的态度\end{CJK} \\
        Yes. The repetition of the word "idea" carries an exaggerated and mocking tone. Combined with "very clever," it may be satirizing the impracticality of the plan or even suggesting it could backfire, reflecting a playful or ironic attitude.\\
        \midrule
        \textbf{Qwen2.5} \ \textcolor{increase}{\faCheck\ Correct Answer}, \raisebox{-0.7mm}{\includegraphics[width=0.4cm]{images/emoji/_1.png}} \textcolor{decrease}{Bad Explanation} \\
        \begin{CJK}{UTF8}{gbsn}
        是。表面上看似在夸赞,但重复“构思”显得夸张且略带讽刺,暗指将茶水吧改成游戏厅的方案并不合理,实际效果可能适得其反。\end{CJK}  \\
        Yes. While it appears to be a compliment, the exaggerated repetition of "idea" has a slightly ironic tone, implying that turning a tea station into a game arcade is not a reasonable plan and might actually be counterproductive.\\
        \midrule
        \textbf{DeepSeek-V3} \ \textcolor{increase}{\faCheck\ Correct Answer}, \raisebox{-0.7mm}{\includegraphics[width=0.4cm]{images/emoji/_1.png}} \textcolor{decrease}{Bad Explanation} \\
        \begin{CJK}{UTF8}{gbsn}
        是。通过三次重复"构思"制造机械感,配合原事件逻辑悖论(游戏厅促进喝水),构成典型反讽结构——表面夸赞实则暗示想法荒诞,符合中文网络阴阳怪气表达范式。\end{CJK} \\
        Yes The triple repetition of "idea" creates a mechanical effect, combined with the logical paradox of the original event (a game arcade promoting water consumption). This forms a classic structure of irony—appearing as praise but actually hinting at the absurdity of the idea, aligning with the parodic style of Chinese internet humor \\
        \midrule
        \textbf{Claude3.5} \ \textcolor{increase}{\faCheck\ Correct Answer}, \raisebox{-0.7mm}{\includegraphics[width=0.4cm]{images/emoji/_1.png}} \textcolor{decrease}{Bad Explanation} \\
        \begin{CJK}{UTF8}{gbsn}
        是。通过重复使用"构思"一词并且语气刻意夸张,表现出一种模仿式的嘲讽,暗示这个创意可能过于简单或缺乏实质内容。\end{CJK} \\
        Yes. The repeated use of the word "idea" and the deliberately exaggerated tone create a form of imitative mockery, implying that the concept may be overly simplistic or lacking in substance. \\
       \bottomrule
    \end{tabular}
    }
    \caption{A case study of LLMs using an example from \textit{DrinkWater}.}
    \label{tab:casestudy_drinkwater}
\end{table}

\begin{table}[h]
    \small
    \centering
    \resizebox{1.0\linewidth}{!}
    {
    \begin{tabular}{p{1.0\linewidth}}
        \toprule
        \textbf{Question} \\
        \begin{CJK}{UTF8}{gbsn}
        关于“CS2赛事中,G2战队连续7次输给NAVI战队”这一话题,有网友评论:“hunter就是个垃圾,每个战队只能换2人否则取消名额的机制阻碍了90\%战队的重建,也帮他这样的混子混了不少工资。monesy永远不配和zywoo相提并论,抖一步名不虚传,甚至天梯打一个集锦操作就被发到各大平台上,然而这届赛事一张图平均被手撕3次  ”\end{CJK} \\
        \begin{CJK}{UTF8}{gbsn}
        请判断这条评论是否为反串,并给出50字以内的解释。\end{CJK} \\
        In the topic "In CS2 tournaments, G2 has lost to NAVI seven times in a row," a netizen commented:"hunter is just trash. The rule that a team can only replace two players or lose their slot has blocked 90\% of team rebuilds, and has also let freeloaders like him earn quite a bit of salary. monesy will never deserve to be compared to zywoo, best Rifler on TikTok is a well-deserved reputation. He even gets a highlight posted on social media platforms for a single flashy move in ranking, yet in this tournament, he gets destroyed an average of three times per map ."  \\
        Please determine whether this comment is satire and provide an explanation within 50 words.\\
        \textbf{Gronud Truth} \\
        \begin{CJK}{UTF8}{gbsn}
        不是。这条评论通过直接攻击选手(huNter 与 m0NESY)来表达对G2战队和选手的不满与反对。\end{CJK} \\
        No. This comment expresses dissatisfaction and opposition to the G2 team and players by directly attacking the players (hunter and monesy). \\
        \midrule
        \textbf{ChatGPT-4o} \ \textcolor{decrease}{\faTimes\ Wrong Answer} \\
        \begin{CJK}{UTF8}{gbsn}
        是。评论内容极端贬低选手,带有夸张的情绪表达(如“垃圾”“混子”“永远不配”),同时使用表情符号强化讽刺意味,符合常见的反串风格。\end{CJK} \\
        Yes. The content excessively criticizes the players, uses exaggerated emotional expressions (such as "trash" and "freeloader"), and includes emojis to enhance its sarcastic tone, which is characteristic of satirical posts. \\
        \midrule
        \textbf{Qwen2.5} \ \textcolor{decrease}{\faTimes\ Wrong Answer} \\
        \begin{CJK}{UTF8}{gbsn}
        是。评论中使用了夸张的语气和明显的偏见,如“hunter就是个垃圾”和“monesy永远不配和zywoo相提并论”,并且通过讽刺的口吻(如“抖一步名不虚传”)可能意在调侃而非严肃批评,疑似反串黑。\end{CJK}  \\
        Yes, It uses an exaggerated tone and clear bias, such as calling "hunter trash" and claiming "monesy is never worthy of being compared to zywoo." The sarcastic wording (e.g., "shake-and-shoot is well-deserved") suggests an intention to mock rather than genuinely criticize, making it seem like satirical trolling. \\
        \midrule
        \textbf{DeepSeek-V3} \ \textcolor{decrease}{\faTimes\ Wrong Answer} \\
        \begin{CJK}{UTF8}{gbsn}
        是。评论使用夸张贬低言辞(如"垃圾""混子")、网络热梗("抖一步")及表情符号,通过极端负面评价和刻意对比(monesy/zywoo)模仿无脑黑粉语气,意在讽刺而非真实批判。\end{CJK} \\
        Yes. The comment employs exaggerated derogatory terms ("trash," "freeloader"), internet slang ("shake-and-shoot"), and emojis, mimicking the speech style of overly critical fans. The extreme negativity and deliberate comparisons (monesy vs. zywoo) suggest an intent to mock rather than provide a serious critique. \\
        \midrule
        \textbf{Claude3.5} \ \textcolor{increase}{\faCheck\ Correct Answer, \raisebox{-0.7mm}{\includegraphics[width=0.4cm]{images/emoji/+1.png}} Good Explanation} \\
        \begin{CJK}{UTF8}{gbsn}
        不是。解释:用词尖锐且带有明显的情绪色彩,但论点具体且符合CS2玩家的专业表达方式,提到了具体的选手数据和规则限制,显示出对赛事的真实关注和了解。\end{CJK} \\
        No. Explanation: The wording is sharp and emotionally charged, but the arguments are specific and align with the professional discourse of CS2 players. It mentions specific player statistics and rule restrictions, indicating a genuine interest and understanding of the tournament. \\
       \bottomrule
    \end{tabular}
    }
    \caption{A case study of LLMs using an example from \textit{CS2}.}
    \label{tab:casestudy_CS2}
\end{table}