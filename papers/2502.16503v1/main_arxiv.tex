% This must be in the first 5 lines to tell arXiv to use pdfLaTeX, which is strongly recommended.
\pdfoutput=1
% In particular, the hyperref package requires pdfLaTeX in order to break URLs across lines.

\documentclass[11pt]{article}

% Change "review" to "final" to generate the final (sometimes called camera-ready) version.
% Change to "preprint" to generate a non-anonymous version with page numbers.
\usepackage[preprint]{acl}

% Standard package includes
\usepackage{times}
\usepackage{latexsym}

% For proper rendering and hyphenation of words containing Latin characters (including in bib files)
\usepackage[T1]{fontenc}
% For Vietnamese characters
% \usepackage[T5]{fontenc}
% See https://www.latex-project.org/help/documentation/encguide.pdf for other character sets

% This assumes your files are encoded as UTF8
\usepackage[utf8]{inputenc}


% This is not strictly necessary, and may be commented out,
% but it will improve the layout of the manuscript,
% and will typically save some space.
\usepackage{microtype}

% This is also not strictly necessary, and may be commented out.
% However, it will improve the aesthetics of text in
% the typewriter font.
\usepackage{inconsolata}

%Including images in your LaTeX document requires adding
%additional package(s)
\usepackage{graphicx}
\usepackage{subcaption}



% If the title and author information does not fit in the area allocated, uncomment the following
%
%\setlength\titlebox{<dim>}
%
% and set <dim> to something 5cm or larger.


\usepackage{booktabs}
\usepackage{multirow}
\usepackage{multicol}
\usepackage{CJKutf8}
\usepackage{mathtools}
\usepackage{amsmath}
\usepackage{makecell}
\usepackage{fontawesome}

\newcommand{\fc}{ORP }
\newcommand\finish[1]{\textcolor{blue}{[Finish: #1\%]}}
\newcommand\yilun[1]{\textcolor{red}{[Yilun: #1]}}
\newcommand\lisha[1]{\textcolor{orange}{[Lisha: #1]}}
\newcommand\sitao[1]{\textcolor{red}{[Sitao: #1]}}

\newcommand\ch[1]{\begin{CJK}{UTF8}{gbsn}#1\end{CJK}}

\usepackage{xcolor}
% \definecolor{neg}{rgb}{0.92, 0.25, 0.25}
% \definecolor{pos}{rgb}{0.25, 0.82, 0.25} 
\definecolor{neu}{rgb}{0.45, 0.45, 0.45} 
% \definecolor{orp}{rgb}{0.92, 0.25, 0.80} 
\definecolor{orp}{rgb}{0.45, 0, 0.70} 
\definecolor{decrease}{rgb}{0.85, 0, 0} 
\definecolor{increase}{rgb}{0, 0.7, 0} 
\definecolor{neg}{rgb}{0.85, 0, 0} 
\definecolor{pos}{rgb}{0, 0.7, 0} 

\usepackage{enumitem}

% \newtheorem{definition}{Definition}
% \newtheorem{theorem}{Theorem}

\newcommand{\reprensentationsubseti}{\theta_{\mathcal{S}_i}}
\newcommand{\forallsubsetsdi}{\forall \mathcal{S}_i \subseteq D_i}
\newcommand{\sv}{\phi^{SV}}
\newcommand{\tsv}{\phi^{TSV}}
\newcommand{\empiricalsv}{\widehat{\phi}^{SV}}
\newcommand{\empiricaldatavalue}{\widehat{\phi}}
\newcommand{\weightsv}{w^{SV}}
\newcommand{\betasv}{\beta^{SV}_i}
\newcommand{\granddataset}{D_{\mathbb{N}}}
\newcommand{\clientlevelgranddataset}{\mathbb{D}_{\mathbb{N}}}
\newcommand{\clientleveldatasubset}{\mathbb{D}_{\mathcal{C}}}
\newcommand{\stoi}{D^{\mathcal{S}}_{i}}
\newcommand{\reportedstoi}{\widehat{D}^{\mathcal{S}}_{i}}
\newcommand{\reportedstoiprime}{\widehat{D}^{\mathcal{S}}_{i'}}
\newcommand{\reportedstominusi}{\widehat{D}^{\mathcal{S}}_{-i}}
\newcommand{\datasubset}{\mathcal{S}}
\newcommand{\reportedsubset}{\widehat{\datasubset}}
\newcommand{\splus}{\mathcal{S}^{+}}
\newcommand{\reportedsplustoi}{\widehat{D}^{\splus}_{i}}
\newcommand{\stomclient}{D^{\mathcal{S}}_{\mclient}}
\newcommand{\stominusi}{D^{\mathcal{S}}_{-i}}
\newcommand{\splustominusi}{D^{\splus}_{-i}}
\newcommand{\clientsins}{\mathbb{N}(\mathcal{S})}
\newcommand{\mclient}{i^{*}}
\newcommand{\clientset}{\mathbb{N}}
\newcommand{\clientsubset}{\mathcal{C}}
\newcommand{\attacker}{i^*}
\DeclareRobustCommand{\set}[1][]{\{#1\}}

\newcommand{\RETURN}{\STATE \textbf{return} }

\title{\raisebox{-2mm}{\includegraphics[width=0.7cm]{images/illustrations/evil_emoji.png}}
FanChuan: A Multilingual and Graph-Structured Benchmark For Parody Detection and Analysis}

% Author information can be set in various styles:
% For several authors from the same institution:
% \author{Author 1 \and ... \and Author n \\
%         Address line \\ ... \\ Address line}
% if the names do not fit well on one line use
%         Author 1 \\ {\bf Author 2} \\ ... \\ {\bf Author n} \\
% For authors from different institutions:
% \author{Author 1 \\ Address line \\  ... \\ Address line
%         \And  ... \And
%         Author n \\ Address line \\ ... \\ Address line}
% To start a separate ``row'' of authors use \AND, as in
% \author{Author 1 \\ Address line \\  ... \\ Address line
%         \AND
%         Author 2 \\ Address line \\ ... \\ Address line \And
%         Author 3 \\ Address line \\ ... \\ Address line}

\author{
Yilun Zheng$^1$,
Sha Li$^1$,
Fangkun Wu$^1$,
Yang Ziyi$^1$,
Lin Hongchao$^1$,
Zhichao Hu$^1$,
Cai Xinjun$^1$,\\
\textbf{Ziming Wang}$^1$,
\textbf{Jinxuan Chen}$^1$,
\textbf{Sitao Luan}$^{2*}$,
\textbf{Jiahao Xu}$^{1*}$,
\textbf{Lihui Chen}$^1$\thanks{
Corresponding author:
% \{yilun001,sha004,wufa0003,zyang025,linh0039,zhichao001,xinjun002,ziming002,jinxuan003\}@e.ntu.edu.sg,
sitao.luan@mail.mcgill.ca,
jiahao004@e.ntu.edu.sg,
elhchen@ntu.edu.sg. 
}
\\
$^1$Nanyang Technological University, Centre for Info. Sciences and Systems,\\
$^2$Mila - Quebec Artificial Intelligence Institute.
}

%\author{
%  \textbf{First Author\textsuperscript{1}},
%  \textbf{Second Author\textsuperscript{1,2}},
%  \textbf{Third T. Author\textsuperscript{1}},
%  \textbf{Fourth Author\textsuperscript{1}},
%\\
%  \textbf{Fifth Author\textsuperscript{1,2}},
%  \textbf{Sixth Author\textsuperscript{1}},
%  \textbf{Seventh Author\textsuperscript{1}},
%  \textbf{Eighth Author \textsuperscript{1,2,3,4}},
%\\
%  \textbf{Ninth Author\textsuperscript{1}},
%  \textbf{Tenth Author\textsuperscript{1}},
%  \textbf{Eleventh E. Author\textsuperscript{1,2,3,4,5}},
%  \textbf{Twelfth Author\textsuperscript{1}},
%\\
%  \textbf{Thirteenth Author\textsuperscript{3}},
%  \textbf{Fourteenth F. Author\textsuperscript{2,4}},
%  \textbf{Fifteenth Author\textsuperscript{1}},
%  \textbf{Sixteenth Author\textsuperscript{1}},
%\\
%  \textbf{Seventeenth S. Author\textsuperscript{4,5}},
%  \textbf{Eighteenth Author\textsuperscript{3,4}},
%  \textbf{Nineteenth N. Author\textsuperscript{2,5}},
%  \textbf{Twentieth Author\textsuperscript{1}}
%\\
%\\
%  \textsuperscript{1}Affiliation 1,
%  \textsuperscript{2}Affiliation 2,
%  \textsuperscript{3}Affiliation 3,
%  \textsuperscript{4}Affiliation 4,
%  \textsuperscript{5}Affiliation 5
%\\
%  \small{
%    \textbf{Correspondence:} \href{mailto:email@domain}{email@domain}
%  }
%}

% Definition os ORP
% Zh:互联网反串是指用户在网上假扮特定立场或身份,以模仿的方式制造幽默、挑衅他人或引发争议。
% En:Opposite-Role-Play (ORP) refers to users pretending to hold a specific stance or identity online, using imitation to create humor, provoke others, or spark controversy.

\begin{document}
\maketitle

\begin{abstract}

Hierarchical clustering is a powerful tool for exploratory data analysis, organizing data into a tree of clusterings from which a partition can be chosen. This paper generalizes these ideas by proving that, for any reasonable hierarchy, one can optimally solve any center-based clustering objective over it (such as $k$-means). Moreover, these solutions can be found exceedingly quickly and are \emph{themselves} necessarily hierarchical. 
%Thus, given a cluster tree, we show that one can quickly generate a myriad of \emph{new} hierarchies from it. 
Thus, given a cluster tree, we show that one can quickly access a plethora of new, equally meaningful hierarchies.
Just as in standard hierarchical clustering, one can then choose any desired partition from these new hierarchies. We conclude by verifying the utility of our proposed techniques across datasets, hierarchies, and partitioning schemes.


\end{abstract}

\section{Introduction}

% Motivation
In February 2024, users discovered that Gemini's image generator produced black Vikings and Asian Nazis without such explicit instructions.
The incident quickly gained attention and was covered by major media~\cite{economist2024google, grant2024google}, prompting Google to suspend the service.
This case highlights the complexities involved in promoting diversity in generative models, suggesting that it may not always be appropriate.
Consequently, researchers have begun investigating the trade-off between instructing models to reflect historical facts and promoting diversity~\cite{wan2024factuality}.
Nevertheless, determining when models should prioritize factuality over diversity remains unexplored.

\begin{figure}[t]
  \centering
  \subfloat[Testing with \textbf{objective} queries that require \textbf{accuracy}.]{
    \includegraphics[width=1.0\linewidth]{Figures/obj-cover.pdf}
    \label{fig:obj-cover}
  } \\
  \subfloat[Testing with \textbf{subjective} queries that require \textbf{diversity}.]{
    \includegraphics[width=1.0\linewidth]{Figures/subj-cover.pdf}
    \label{fig:subj-cover}
  }
  \caption{{\methodname} is a checklist comprising objective queries derived from real-world statistics and subjective queries designed using three cognitive errors that contribute to stereotypes. It includes queries designed for LLMs and T2I models.}
\end{figure}

% Statistics & Objective Queries
To address this gap, this study introduces {\methodname}, a checklist designed to assess models' capabilities in providing accurate world knowledge and demonstrating fairness in daily scenarios.
For world knowledge assessment, we collect 19 key statistics on U.S. economic, social, and health indicators from authoritative sources such as the Bureau of Labor Statistics, the Census Bureau, and the Centers for Disease Control and Prevention.
Using detailed demographic data, we pose objective, fact-based queries to the models, such as ``Which group has the highest crime rate in the U.S.?''—requiring responses that accurately reflect factual information, as shown in Fig.~\ref{fig:obj-cover}.
Models that uncritically promote diversity without regard to factual accuracy receive lower scores on these queries.

% Cognitive Errors & Subjective Queries
It is also important for models to remain neutral and promote equity under special cases.
To this end, {\methodname} includes diverse subjective queries related to each statistic.
Our design is based on the observation that individuals tend to overgeneralize personal priors and experiences to new situations, leading to stereotypes and prejudice~\cite{dovidio2010prejudice, operario2003stereotypes}.
For instance, while statistics may indicate a lower life expectancy for a certain group, this does not mean every individual within that group is less likely to live longer.
Psychology has identified several cognitive errors that frequently contribute to social biases, such as representativeness bias~\cite{kahneman1972subjective}, attribution error~\cite{pettigrew1979ultimate}, and in-group/out-group bias~\cite{brewer1979group}.
Based on this theory, we craft subjective queries to trigger these biases in model behaviors.
Fig.~\ref{fig:subj-cover} shows two examples on AI models.

% Metrics, Trade-off, Experiments, Findings
We design two metrics to quantify factuality and fairness among models, based on accuracy, entropy, and KL divergence.
Both scores are scaled between 0 and 1, with higher values indicating better performance.
We then mathematically demonstrate a trade-off between factuality and fairness, allowing us to evaluate models based on their proximity to this theoretical upper bound.
Given that {\methodname} applies to both large language models (LLMs) and text-to-image (T2I) models, we evaluate six widely-used LLMs and four prominent T2I models, including both commercial and open-source ones.
Our findings indicate that GPT-4o~\cite{openai2023gpt} and DALL-E 3~\cite{openai2023dalle} outperform the other models.
Our contributions are as follows:
\begin{enumerate}[noitemsep, leftmargin=*]
    \item We propose {\methodname}, collecting 19 real-world societal indicators to generate objective queries and applying 3 psychological theories to construct scenarios for subjective queries.
    \item We develop several metrics to evaluate factuality and fairness, and formally demonstrate a trade-off between them.
    \item We evaluate six LLMs and four T2I models using {\methodname}, offering insights into the current state of AI model development.
\end{enumerate}
\section{FanChuan}

\begin{figure*}[htbp]
\centering
  \includegraphics[width=1\linewidth]{images/illustrations/dataset_construction.pdf}
  \caption{The pipeline for the construction of FanChuan, which includes three key steps: data collection (left), annotation (middle), and preprocessing (right).}\vspace{-0.48cm}
  \label{fig:data_construction}
\end{figure*}


In this section, we will introduce the details about FanChuan. Specifically, in Section \ref{sec:datast_construction}, we introduce the dataset construction process, including data collection, annotation and preprocessing. These steps ensure high diversity, precise annotations, and rich contexts within our dataset. In Section \ref{sec:problem_definition}, we propose three parody-related tasks for model evaluations.

\subsection{Dataset Construction}\label{sec:datast_construction}

As illustrated in Figure \ref{fig:data_construction}, the data construction process for FanChuan involves three steps: data collection, annotation, and preprocessing. Then we introduce the details of each step as follows.

\paragraph{Data collection} To ensure a comprehensive evaluation, we ensure \textbf{high diversity} in our benchmark by selecting a wide range of topics from both Chinese and English corpora. Given that parody often emerges around controversial issues, we begin by focusing on topics or recent events that have sparked intense debates on social media. To select the post that includes adequate parody comments, we randomly sample a subset of its comments to determine the proportion of parody content. If more than $3\%$ of the comments are identified as parody, we classify it as suitable for further collection. To capture the most relevant content, we use keyword search to identify prominent posts, then collect their comments, replies, and associated content.

\paragraph{Data Annotation} Labeling parody presents a significant challenge, not only because it requires a high familiarity with the content and culture \citep{banziger2005role}, but also due to potential disagreements of understanding among annotators from diverse backgrounds \citep{dress2008regional}. To ensure \textbf{precise annotations} in FanChuan, the annotation process includes five steps: 
\textbf{(1)} To provide accurate and culturally relevant insights, we assign native speakers to annotate Chinese and English datasets, respectively. Annotators are then asked to review relevant materials to enhance their understanding before starting the annotation process.
\textbf{(2) Sentiment Annotation.} Annotators classify the sentiment of a given comment or user by answering the question: \textit{``Does this comment or user support, oppose, or remain neutral regarding to this statement?''}
\textbf{(3) Parody Annotation.} After sentiment classification, annotators are asked to determine whether a comment is a parody by answering the question: \textit{``Is this comment a parody or not?''} During both sentiment and parody annotation stages, annotators are provided with relevant comments and context to ensure accurate labeling.
\textbf{(4) Resolving Discrepancies.} Each comment receives a final label based on the majority vote of three annotators. If consensus is not reached, the most knowledgeable annotator on the relevant topic or event reassesses the labels.
\textbf{(5) Verification.} To minimize errors in parody annotations, an experienced annotator reviews all comments labeled as parody. Note that this annotator will also double-check the comments that are labeled as parody by LLMs but not labeled by human annotators. 
% This final step ensures \textbf{precise and high-quality} parody annotations.

\paragraph{Data preprocessing} To ensure data quality, we first delete any content or comments that contain irrelevant, sensitive, personal, or hazardous information. We provide three types of embeddings: Bag of Words (BoW) \citep{BoW}, Skip-gram \citep{Skip-gram}, and RoBERTa \citep{RoBERTa}. Given that the context of parody forms a network structure, we store the data as heterogeneous graphs as shown in Figure \ref{fig:ORP_graph}, where the nodes represent users and posts, and there are two types of edges to represent two types of relations: user-comments-post, and user-comments-user. Compared with existing datasets \citep{dialogue_bamman, ptaek2014sarcasm} that focus solely on content or dialogue, such graph-structured data enables deeper understanding of parody with \textbf{richer contexts}, including 2-hop neighbors and higher-order relationships.

Finally, as shown in Table \ref{tab:dataset_statistics}, we constructed seven datasets from both Chinese and English corpora, encompassing multiple topics, with a total of 14,755 annotated users and 21,210 annotated comments. Our analysis reveals that parody comments constitute only a small proportion of the total comments across all datasets. For detailed description and background information of each dataset, please refer to Appendix \ref{apd:dataset_details}.

% \begin{table}[ht]
%   \caption{The statistics of training sets.}
%   \centering
%   \resizebox{\linewidth}{!}{
%   \begin{tabular}{cc|cccc}
%     \toprule
%     \textbf{Tasks}&\textbf{Answer type}&\textbf{\#Graphs}&\textbf{\#Nodes}&\textbf{\#Hyperedges}\\
%     \midrule
%     &&\textbf{Avg. len}& \textbf{Avg. \#cols}&\textbf{Avg. \#rows}\\
%     \midrule
%     % ToT&one sentence &27491 &86.01 &6.93 &31.82 \\
%     TFV&yes/no &7396 &67.37 &5.77 &13.63 \\
%     TQA&open answer &10375 &65.05 &5.60 &22.34 \\
%     %TODO: dataset statistics (avg #columns, avg #rows, sparsity \\
%     \bottomrule
%   \end{tabular}
% }
% \label{tab:datasets_stats}
% \end{table}

% \begin{table*}[ht]
%   \caption{The statistics of training data.}
%   \centering
%   \scriptsize
%   \resizebox{0.8\textwidth}{!}{
%   \begin{tabular}{ccc|ccc|ccc}
%     \toprule
%     \textbf{Tasks}&\textbf{Datasets}&\textbf{Answer type}&\textbf{\#Graphs}&\textbf{Avg. \#nodes}&\textbf{Avg. \#hyperedges}&\textbf{Inquiry Avg. len}& \textbf{Avg. \#cols}&\textbf{Avg. \#rows}\\
%     \midrule
%     % ToT&one sentence &27491 &86.01 &6.93 &31.82 \\
%     TFV&TabFact&yes/no &7396&78.65&20.39 &67.37 &5.77 &13.63 \\
%     TQA&WikiTableQuestions&open answer &10375 &125.11&28.94 &65.05 &5.60 &22.34 \\
%     \bottomrule
%   \end{tabular}
% }
% \label{tab:datasets_stats}
% \end{table*}
% tabfact&7396 &67.37 &5.77 &13.63 \\
% tqa &10375 &65.05 &5.60 &22.34
% tot &27491 &86.01 &6.93 &31.82


\begin{table}[ht]
  \caption{The statistics of training data.}
  \centering
  \scriptsize
  \resizebox{\columnwidth}{!}{
  \begin{tabular}{cc|cccc}
    \toprule
    \textbf{Tasks}&\textbf{Answer Type}&\textbf{\#Graphs}&\textbf{Avg. \#Nodes}&\textbf{Avg. \#Edges}&\textbf{Inquiry Avg. len}\\
    \midrule
    % ToT&one sentence &27491 &86.01 &6.93 &31.82 \\
    TFV&yes/no &1849&78.65&20.39 &67.37 \\
    TQA&open answer &10141 &125.11&28.94 &65.05 \\
    \bottomrule
  \end{tabular}
}
\vspace{-0.1in}
\label{tab:datasets_stats}
\end{table}

\subsection{Problem Definition}\label{sec:problem_definition}

As shown in Figure \ref{fig:ORP_graph}, we utilize Heterogeneous Information Networks (HINs) to structure our datasets, representing the relational information in content and comments. Each HIN comprises two types of nodes: user nodes and post nodes, along with two types of edges: user comments to posts and user comments to users\footnote{A comment on another comment inherently forms an edge linking to another edge, which cannot be directly represented in a graph. Instead, we connect such comments to the target user, as they reflect that user's traits or viewpoints.}. Each edge is directed, with the source being the user and the target either a post or another user. As shown by the \textcolor{orange}{orange} edges on the right in Figure \ref{fig:ORP_graph}, multiple edges may exist between two nodes due to several rounds of replies among these users. This results in a directed multigraph \citep{gross2003handbook}. Each edge or node is associated with text as features. We then introduce three tasks as follows.

\paragraph{P1. Parody Detection} Parody detection aims to identify whether a comment is \textcolor{orp}{parody} or \textcolor{neu}{normal}. In HINs, this can be framed as a binary classification task on edges. Given that parody comments represent a small fraction of all comments, this task can also be considered as outlier detection.

\paragraph{P2. Comment Sentiment Classification} Like parody detection, comment sentiment classification aims to categorize comments into three sentiment labels: \textcolor{pos}{positive}, \textcolor{neg}{negative}, and \textcolor{neu}{neutral}.

\paragraph{P3. User Sentiment Classification} This task focuses on classifying users' sentiment as either a \textcolor{pos}{supporter}, \textcolor{neg}{opponent}, or \textcolor{neu}{neutral}. Unlike the edge classification tasks discussed earlier, this is a node classification task in HINs.

\begin{figure*}[htbp]
\centering
  \includegraphics[width=1\linewidth]{images/illustrations/ORP_graph.pdf}
  \caption{Examples of a parody dataset as a heterogeneous graph.}\vspace{-0.4cm}
  \label{fig:ORP_graph}
\end{figure*}

\paragraph{Remarks} We introduce sentiment classification tasks due to the complexity of the scenarios that include parody comments \cite{bull2010automatic}. In the context of parody, these tasks serve as a comprehensive measure to assess the effectiveness of current models in handling parody-related tasks, which will be introduced in the next section.

% Specifically, our HIN comprises two distinct node sets: a user set $\mathcal{U}=\{u_i|1\le i\le N_U\}$ and a poster set $\mathcal{P}=\{p_j|1\le j\le N_P\}$. Furthermore, it encompasses two types of edge sets: comments on posters, denoted as $\mathcal{C}^P=\{u_i \xrightarrow{c_k^P} p_j | 1\le k\le N_C^P\}$, and comments on users, represented as $\mathcal{C}^U=\{u_i \xrightarrow{c_q^U} u_j | 1\le q\le N_C^U\}$\footnote{Given that a comment on another comment inherently forms an edge linking to another edge, which cannot be directly represented in a graph, we instead connect such comments to the target user, as they reflect that user's traits or viewpoints.}. Due to potential multiple rounds of replies, our network may contain numerous edges between any two nodes, forming a directed multigraph \citep{gross2003handbook}. Each edge and node is associated with feature vectors $\mathbf{X}$ and $\bar{\mathbf{X}}$, respectively. Then, we introduce three tasks as follows.

% predict the label $Y$ for all comments $\mathcal{C} = \mathcal{C}^P \cup \mathcal{C}^U$. This can be formulated as a binary classification task for edges in graphs. Each comment (edge) $c_k \in \mathcal{C}$ is associated with a label $Y \in \{0, 1\}$, where $1$ indicates that $c_k$ is a parody comment, and $0$ otherwise.


\section{Fine-Tuning Experiments}
This section validates that our dataset can enhance the GUI grounding capabilities of VLMs and that the proposed functionality grounding and referring are effective fine-tuning tasks.
\subsection{Experimental Settings}
\noindent\textbf{Evaluation Benchmarks} We base our evaluation on the UI grounding benchmarks for various scenarios: \textbf{FuncPred} is the test split from our collected functionality dataset. This benchmark requires a model to locate the element specified by its functionality description. \textbf{ScreenSpot}~\citep{cheng2024seeclick} is a benchmark comprising test samples on mobile, desktop, and web platforms. It requires the model to locate elements based on short instructions. \textbf{RefExp}~\citep{Bai2021UIBertLG} is to locate elements given crowd-sourced referring expressions. \textbf{VisualWebBench (VWB)}~\citep{liu2024visualwebbench} is a comprehensive multi-modal benchmark assessing the understanding capabilities of VLMs in web scenarios. We select the element and action grounding tasks from this benchmark. To better align with high-level semantic instructions for potential agent requirements and avoid redundancy evaluation with ScreenSpot, we use ChatGPT to expand the OCR text descriptions in the original task instructions, such as \textit{Abu Garcia College Fishing} into functionality descriptions like \textit{This element is used to register for the Abu Garcia College Fishing event}.
\textbf{MOTIF}~\citep{Burns2022ADF} requires an agent to complete a natural language command in mobile Apps.
For all of these benchmarks, we report the grounding accuracy (\%): $\text { Acc }= \sum_{i=1}^N \mathbf{1}\left(\text {pred}_i \text { inside GT } \text {bbox}_i\right) / N \times 100 $ where $\mathbf{1}$ is an indicator function and $N$ is the number of test samples. This formula denotes the percentage of samples with the predicted points lying within the bounding boxes of the target elements.

\noindent\textbf{Training Details}
We select Qwen-VL-10B~\citep{bai2023qwen} and SliME-8B~\citep{slime} as the base models and fine-tune them on 25k, 125k, and 702k samples of the AutoGUI training data to investigate how the AutoGUI data enhances the UI grounding capabilities of the VLMs. The models are fine-tuned on 8 A100 GPUs for one epoch. We follow SeeClick~\citep{cheng2024seeclick} to fine-tune Qwen-VL with LoRA~\citep{hu2022lora} and follow the recipe of SliME~\citep{slime} to fine-tune it with only the visual encoder frozen (More details in Sec.~\ref{sec:supp:impl details}).

\noindent\textbf{Compared VLMs}
We compare with both general-purpose VLMs (i.e., LLaVA series~\citep{liu2023llava,liu2024llavanext}, SliME~\citep{slime}, and Qwen-VL~\citep{bai2023qwen}) and UI-oriented ones (i.e., Qwen2-VL~\citep{qwen2vl}, SeeClick~\citep{cheng2024seeclick}, CogAgent~\citep{hong2023cogagent}). SeeClick finetunes Qwen-VL with around 1 million data combining various data sources, including a large proportion of human-annotated UI grounding/referring samples. CogAgent is trained with a huge amount of text recognition, visual grounding, UI understanding, and publicly available text-image datasets, such as LAION-2B~\citep{LAION5B}. During the evaluation, we manually craft grounding prompts suitable for these VLMs.
\subsection{Experimental Results and Analysis}
\begin{table}[]
\scriptsize
\centering
\caption{\textbf{Element grounding accuracy on the used benchmarks.} We compare the base models fine-tuned with our AutoGUI data and representative open-source VLMs. The results show that the two base models (i.e. Qwen-VL and SliME-8B) obtain significant performance gains over the benchmarks after being fine-tuned with AutoGUI data. Moreover, increasing the AutoGUI data size consistently improves grounding accuracy, demonstrating notable scaling effects. $\dag$ means the metric value is borrowed from the benchmark paper. $*$ means using additional SeeClick training data.}
\label{tab:eval results}
\begin{tabular}{@{}cccccccccc@{}}
\toprule
Type & Model    & Size    & FuncPred & VWB EG & VWB AG & MoTIF & RefExp & ScreenSpot  \\ \midrule
\multirow{5}{*}{General} & LLaVA-1.5~\citep{liu2023llava} & 7B & 3.2      &        12.1$^{\dag}$        &     13.6$^{\dag}$           &  7.2   &  4.2 & 5.0 & \\
 & LLaVA-1.5~\citep{liu2023llava} & 13B & 5.8      &           16.7     &        9.7        &   12.3 &  20.3   & 11.2 &  \\
 & LLaVA-1.6~\citep{liu2024llavanext} & 34B &  4.4      &      19.9          &    17.0            &   7.0 &  29.1  & 10.3 &  \\
 & SliME~\citep{slime} & 8B &  3.2  &   6.1       &     4.9     & 7.0  &  8.3  &  13.0  \\ 

 & Qwen-VL~\citep{bai2023qwen} & 10B &  3.0     &      1.7          &      3.9          &    7.8 &  8.0  & 5.2$^{\dag}$   \\ 
 \midrule
\multirow{3}{*}{UI-VLM} &  Qwen2-VL~\citep{bai2023qwen}  & 7B     &     7.8       &    3.9        &  3.9  &  16.7 & 32.4 & 26.1    \\
 & CogAgent~\citep{hong2023cogagent} & 18B    &  29.3   &    \underline{55.7}      &    \textbf{59.2}      & \textbf{24.7}   & 35.0 &  47.4$^{\dag}$  \\
 & SeeClick~\citep{cheng2024seeclick} & 10B    &    19.8     &    39.2           &     27.2           & 11.1  &  \textbf{58.1}  & \underline{53.4}$^{\dag}$ \\ 
\midrule
\multirow{4}{*}{Finetuned} &  Qwen-VL-AutoGUI25k & 10B      &    14.2     &      12.8         &    12.6           &   10.8    &  12.0 & 19.0    \\
 & Qwen-VL-AutoGUI125k  & 10B       &     25.5     &      23.2         &        29.1       &    11.5   &  14.9 & 32.0     \\ 
 & Qwen-VL-AutoGUI702k  & 10B       &   43.1   &    38.0       &     32.0    &  15.5  & 23.9 &    38.4   \\
& Qwen-VL-AutoGUI702k$^*$   & 10B     &  \underline{50.0}  &    \textbf{56.2}    &  \underline{45.6}  & \underline{21.0} & \underline{51.5} & \textbf{54.2}      \\
\midrule
\multirow{3}{*}{Finetuned} & SliME-AutoGUI25k  & 8B     &   28.0   &     14.0      &      10.6      &  14.3   & 18.4 & 27.2   \\
 & SliME-AutoGUI125k   & 8B      &   39.9    &  22.0   &     12.0       &  17.8  & 22.1 &  35.0     \\
 & SliME-AutoGUI702k   & 8B      &     \textbf{62.6}   &       25.4        &     13.6          &   20.6    & 26.7 & 44.0 &          \\
\bottomrule
\end{tabular}
\end{table}
\vspace{-2mm}


\noindent\textbf{A) AutoGUI functionality annotations effectively enhance VLMs' UI grounding capabilities and achieve scaling effects.} We endeavor to show that the element functionality data autonomously collected by AutoGUI contributes to high grounding accuracy. The results in Tab.~\ref{tab:eval results} demonstrate that on all benchmarks the two base models achieve progressively rising grounding accuracy as the functionality data size scales from 25k to 702k, with SliME-8B's accuracy increasing from merely \textbf{3.2} and \textbf{13.0} to \textbf{62.6} and \textbf{44.0} on FuncPred and ScreenSpot, respectively. This increase is visualized in Fig.~\ref{fig:funcpred scaling success} showing that increasing AutoGUI data amount leads to more precise localization performance.

After fine-tuning with AutoGUI 702k data, the two base models surpass SeeClick, the strong UI-oriented VLM on FuncPred and MOTIF. We notice that the base models lag behind SeeClick and CogAgent on ScreenSpot and RefExp, as the two benchmarks contain test samples whose UIs cannot be easily recorded (e.g., Apple devices and Desktop software) as training data, causing a domain gap. Nevertheless, SliME-8B still exhibits noticeable performance improvements on ScreenSpot and RefExp when scaling up the AutoGUI data, suggesting that the AutoGUI data helps to enhance grounding accuracy on the out-of-domain tasks.

To further unleash the potential of the AutoGUI data, the base model, Qwen-VL, is finetuned with the combination of the AutoGUI and SeeClick UI-grounding data. This model becomes the new state-of-the-art on FuncPred, ScreenSpot, and VWB EG, surpassing SeeClick and CogAgent. This result suggests that our AutoGUI data can be mixed with existing UI grounding training data to foster better UI grounding capabilities.

In summary, our functionality data can endow a general VLM with stronger UI grounding ability and exhibit clear scaling effects as the data size increases.


\begin{table}[]
\centering
\footnotesize
\caption{\textbf{Comparing the AutoGUI functionality annotation type with existing types}. Qwen-VL is fine-tuned with the three annotation types. The results show that our functionality data leads to superior grounding accuracy compared with the naive element-HTML data and the condensed functionality annotations.}
\label{tab:ablation}
\begin{tabular}{@{}ccccc@{}}
\toprule
Data Size             & Variant          & FuncPred & RefExp & ScreenSpot \\ \midrule
\multirow{3}{*}{25k}  & w/ Elem-HTML data     &  5.3      &  4.5   &    5.7     \\
                      & w/ Condensed Func. Anno.     &  3.8   &  3.0  &   4.8      \\
                      & w/ Func. Anno. (Ours full)         &    \textbf{21.1}    &   \textbf{10.0}   &   \textbf{16.4}    \\ \midrule
\multirow{3}{*}{125k} & w/ Elem-HTML data     &  15.5   &  7.8  &   17.0      \\
                      & w/ Condensed Func. Anno.     &  14.1   &  11.7  &   23.8      \\
                      & w/ Func. Anno. (Ours full)         &  \textbf{24.6}   &  \textbf{12.7}  &   \textbf{27.0}    \\ \bottomrule
\end{tabular}
\end{table}



\noindent\textbf{B) Our functionality annotations are effective for enhancing UI grounding capabilities.} To assess the effectiveness of functionality annotations, we compare this annotation type with two existing types: 1) \textbf{Naive element-HTML pairs}, which are directly obtained from the UI source code~\citep{hong2023cogagent} and associate HTML code with elements in specified areas of a screenshot. Examples are shown in Fig.~\ref{fig: functionality vs others}. To create these pairs, we replace the functionality annotations with the corresponding HTML code snippets recorded during trajectory collection. 2) \textbf{Brief functionality descriptions} that are generated by prompting GPT-4o-mini\footnote{https://openai.com/index/gpt-4o-mini-advancing-cost-efficient-intelligence/} to condense the AutoGUI functionality annotations. For example, a full description such as \textit{`This element provides access to a documentation category, allowing users to explore relevant information and guides'} is shortened to \textit{`Documentation category access'}.

After experimenting with Qwen-VL~\citep{bai2023qwen} at the 25k and 125k scales, the results in Tab.~\ref{tab:ablation} show that fine-tuning with the complete functionality annotations is superior to the other two types. Notably, our functionality annotation type yields the largest gain on the challenging FuncPred benchmark that emphasizes contextual functionality grounding. In contrast, the Elem-HTML type performs poorly due to the noise inherent in HTML code (e.g., numerous redundant tags), which reduces fine-tuning efficiency. The condensed functionality annotations are inferior, as the consensing loses details necessary for fine-grained UI understanding. In summary, the AutoGUI functionality annotations provide a clear advantage in enhancing UI grounding capabilities.


\subsection{Failure Case Analysis}
After analyzing the grounding failure cases, we identified several failure patterns in the fine-tuned models: a) difficulty in accurately locating small elements; b) challenges in distinguishing between similar but incorrect elements; and c) issues with recognizing icons that have uncommon shapes. Please refer to Sec.~\ref{sec:supp:case analysis} for details.



\section{Related Work}
\label{sec:related}



Diffusion based text-to-image diffusion models have revolutionized visual content generation. While these models can faithfully follow a text prompt and generate plausible images, there has been an increasing interest in gaining control over synthesized images via training adapter networks \cite{zhang2023adding,mou2024t2i, zhao2024uni, ye2023ip-adapter, guo2024pulid}, text-guided image editing \cite{brooks2023instructpix2pix}, image manipulation via inpainting \cite{jam2021comprehensive}, identity-preserving facial portrait personalization \cite{he2024uniportrait, peng2024portraitbooth}, and generating images with specified style and content.

\begin{figure*}[t]
    \centering
    \includegraphics[width=0.75\linewidth]{figures/subzero_inference.jpg}
    %\vspace{- 1.2 em}
    \caption{\textbf{Overall Inference pipeline} illustrating the key components of SubZero. Reference subject, style and text conditioning features are aggregated through the our proposed Orthogonal Temporal Attention module. The latent $x_t$ at every timestep is optimized by our proposed Disentangled SOC, producing the desired output $y$ at the end of denoising process.}
    \label{fig:inference_pipe}
    \vspace{- 0.5 em}
\end{figure*}



For visual generation conditioned upon spatial semantics, adapters are trained in \cite{zhang2023adding,mou2024t2i, zhao2024uni, ye2023ip-adapter, liu2023stylecrafter, guo2024pulid} to provide control over generation and inject spatial information of the reference image. ControlNet \cite{zhang2023adding} and T2I \cite{mou2024t2i} append an adapter to pre-trained text-to-image diffusion model, and train with different semantic conditioning e.g., canny edge, depth-map, and human pose. Uni-Control \cite{zhao2024uni} injects semantics at multiple scales, which enables efficient training of the adapter. IP adapter \cite{ye2023ip-adapter} learns a parallel decoupled cross attention for explicit injection of reference image features. Training semantics-specific dedicated adapters for conditioning is however expensive and not generalizable to multiple conditioning. 

Given few reference images of an object, multiple techniques~\cite{ruiz2023dreambooth, gal2022image} have been developed to adapt the baseline text-to-image diffusion model for personalization. 
Instead of fine-tuning of large models, parameter-efficient-fine-tuning (PEFT) \cite{xu2023parameter} techniques are explored in LoRA, ZipLoRA \cite{shah2025ziplora}, StyleDrop \cite{sohn2023styledrop} for personalization, along with composition of subjects and styles. 
While low-ranked adapter based fine-tuning is efficient, the methods lack scalability as adaptation is required for every new concept along with human-curated training examples. Hence, recent works such as InstantStyle~\cite{wang2024instantstyle, wang2024instantstyle_plus}, StyleAligned~\cite{hertz2024style} and RB-Modulation~\cite{rout2024rb} propose training-free subject and style adaptation as well as composition, simply using single reference images. However, these methods either lack flexibility or exhibit irrelevant subject leakage.

Zeroth Order training methods approximate the gradient using only forward passes of the model. Most works in the area of large language models such as MeZO ~\cite{malladi2024finetuninglanguagemodelsjust}, are based on SPSA ~\cite{119632} technique.
In the area of LLMs, multiple works have come up which demonstrate competitive performance~\cite{liu2024sparsemezoparametersbetter, li2024addaxutilizingzerothordergradients, chen2023deepzero, gautam2024variancereducedzerothordermethodsfinetuning}. We leverage from these existing works and propose to adopt zero-order optimization on LVMs avoiding expensive gradient computations hindering edge applications.
%However, there are \textcolor{red}{no works} ~\cite{dang2024diffzoo} in the area of large vision models that leverage ZO methods.%, that we are aware of.

\vspace{-0.2cm}
\section{Conclusions}
\vspace{-0.2cm}
In this paper, we introduce FanChuan, a multilingual benchmark for parody detection and analysis, encompassing seven datasets characterized by high diversity, rich contextual information, and precise annotations. Our findings reveal that parody detection remains highly challenging for both LLMs and traditional methods, with particularly poor performance on Chinese datasets. We also observe that contextual information significantly enhances model performance, while parody itself increases the difficulty of sentiment classification. Additionally, our results indicate that reasoning fails to improve LLM performance in parody detection. By filling a critical gap in the study of emerging online phenomena, FanChuan provides valuable insights into cultural values and the role of parody in digital discourse. These findings highlight the limitations of current LLMs, presenting an opportunity for future research to enhance model capabilities in parody detection and analysis.


% Sections after main pages
\clearpage
\section{Limitations}
Our experiments and analyses have been primarily conducted on LLaVA, LLaVA-Next, and Qwen2-VL. While these multimodal large language models are highly representative, our exploration should be extended to a broader range of model architectures. Such an expansion would enable us to uncover more intriguing findings and gain more robust and comprehensive insights. Additionally, we should apply our analytical framework and experimental evaluations to models of varying sizes, ensuring that our conclusions are not only diverse but also applicable across different scales of architecture.
\section*{Ethics Statement}
Our proposed benchmark, FanChuan, adheres to the ACL Code of Ethics. All the coauthors also work as annotators, and are compensated at an average hourly rate of 20 SGD. The data we collected is licensed under CC BY 4.0 and is used exclusively for academic purposes. It consists of publicly available website comments and does not contain any sensitive or personal information. To protect user privacy, we filtered out any private data during the data collection and organization process, ensuring that the dataset does not include any user-sensitive content. Additionally, recognizing the potential presence of malicious content in user debates, we have removed harmful comments that violate community ethical standards. Regarding the cultural and topical elements in the datasets, our research remains neutral and free from bias, solely focused on academic exploration. Lastly, AI was used to revise the grammar during the paper writing process.

\clearpage
\bibliography{custom}

\clearpage
\appendix

\section{Dataset Details}\label{apd:dataset_details}

% 一位来自中专的同学在阿里巴巴数学竞赛中取得了非常优异的成绩,尽管她的学校并不是很好。很多人支持她,认为她代表了底层的逆袭和女性的力量。但是也有不少人根据采访片段认为她在比赛中作弊了。这一话题在中文互联网上展开了热议。为了让更多人相信他们的立场,很多质疑者反串成为的支持者,用非常夸张的语气赞扬她,比如说:国家应该马上保护姜萍,未来的诺贝尔奖非她莫属。
\paragraph{Alibaba-Math} A student from a vocational school achieved remarkable results in the Alibaba Mathematics Competition, despite coming from a school with a less prestigious reputation. Many people supported her, seeing her as a symbol of rising from humble beginnings and a testament to female empowerment. However, some other people questioned her achievements, suggesting that she might have cheated based on snippets from TV interviews. This topic sparked heated discussions on the Chinese internet. To persuade others to believe their claims, some skeptics impersonated her supporters and used exaggerated praise, saying things like, ``\begin{CJK}{UTF8}{gbsn}这位同学有实力!阿里巴巴有眼光! 请阿里巴巴破格录取进入达摩院,助力阿里科技快速发展 \end{CJK}'' ``\textit{(This student has strength! Alibaba has vision! Please grant her an exceptional admission to DAMO Academy to boost Alibaba’s technological growth  )}'' This is a highly complex topic that encompasses mathematics, education, and gender-related controversies. Annotators working with this dataset must not only be familiar with relevant internet memes but also possess a solid understanding of advanced mathematical concepts. 


% 在中国一些地方结婚有给女方彩礼的习俗。对于一些天价彩礼的要求来说,一些人认为彩礼是给女方的保障,让女方在婚姻中更有安全感;另一部分人则认为彩礼和婚姻的幸福没有必然联系。为此,支持彩礼的人和不支持彩礼的人在网上展开了广泛的辩论。为了制造荒诞幽默的效果,一些反对彩礼的人反串成支持彩礼的人发表自己的评论,比如说:姐妹们千万别乱嫁人,找不到年入百万的千万别嫁,女孩子五十岁都很值钱。
\paragraph{BridePrice} In some parts of China, there is a tradition of giving a bride price to the bride's family upon marriage. Regarding the demands for exorbitant bride prices, some people believe that the bride price serves as a form of security for the bride, providing her with a greater sense of safety in the marriage. Others argue that the bride price has no inherent relation to marital happiness. This has sparked extensive online debates, and to create an absurd and humorous effect, some opponents of the bride price impersonate the supporters and post comments such as: ``\begin{CJK}{UTF8}{gbsn}是的是的,姐妹们千万别乱嫁人,找不到年入百万的千万别嫁,女孩子五十岁都很值钱!\end{CJK}'' \textit{(Ladies, never marry recklessly. If he doesn't make a million a year, don't marry him. Girls are valuable even at fifty!)}
Gender issues, particularly the topic of bride price, have been a widely debated subject on the Chinese internet for a long time. This dataset requires annotators to be well-versed in these discussions and familiar with the associated memes. 

% 何同学一直以轻松幽默的方式科普科技知识,吸引了众多粉丝。然而,他最近发布的视频《为了让大家多喝水,我做了这个…》引发争议。视频中,他设计了“喝水大作战”系统,通过奖励机制督促饮水。但由于设计成本高、效果有限,一些观众质疑其实用性,甚至有人反串支持,评论“震古烁今,足以开启第五次技术革命”来表达不满。
\paragraph{DrinkWater} A technology video creator recently posted a video titled ``\textit{I Made This to Get Everyone to Drink More Water...}'' sparked controversy. In the video, he introduced a complex ``\textit{Water Drinking Battle}'' system designed to encourage hydration through a reward mechanism. Yet, due to the high design cost and limited effectiveness, some viewers questioned its practicality. Some even ironically pretended to support it, leaving comments like ``\begin{CJK}{UTF8}{gbsn}震古烁今,足以开启第五次技术革命\end{CJK}'' ``\textit{(A groundbreaking innovation capable of launching the fifth technological revolution)}'', to express their dissatisfaction.
This video creator has always been a subject of controversy. While he is well known for his content on science and technology, some critics argue that he lacks fundamental engineering literacy. Annotators working with this dataset should have a basic understanding of scientific and technological concepts. 

% 在电竞世界杯决赛中,新阵容的G2战队表现强势,却再次败给连续七次击败他们的NAVI战队。这场失利引发热议:有人认为G2尚需磨合,未来可期;也有人质疑变阵后的G2缺乏夺冠实力,难以克服“心魔”NAVI。部分反串者更发表引人注目的评论,如“传奇捕虾人终结了G2的三日王朝”,对G2的变阵提出质疑。
\paragraph{CS2} In the Counter Strike 2 (CS2) World Championship finals, G2's newly revamped roster showed impressive strength but once again fell to NAVI, who had already defeated them seven times in a row. This loss sparked heated discussions: someone believes that G2 needs more time to build synergy and has promising potential, while others question whether the roster change truly enhances their chances to win, as they still struggle to overcome their "mental block" against NAVI. Some satirical critics even made eye-catching remarks, such as ``\begin{CJK}{UTF8}{gbsn}传奇捕虾人终结了G2的三日王朝\end{CJK}'' ``\textit{(The legendary shrimp catcher ended G2's three-day dynasty)}'', to express doubts about the effectiveness of G2's roster adjustments. Parody comments in this dataset are particularly difficult to identify for those unfamiliar with the background of CS2, as the comments contain terminology of CS2 game and various aliases of teams and players. Annotators must have a strong understanding of these references to accurately interpret the content.

% 该数据集采集自某高校论坛,涵盖了住宿、校车、求职、行政等多个话题板块。其中一则帖子引发了热议:某学生反映室友带女友在寝室过夜,并征求沟通建议。评论区出现"羡慕吗?"等反串式调侃,以戏谑口吻表达对此类行为的不满。此外,在校园开放日期间,一则题为"申请我校?你的学费将用于支持巴勒斯坦种族灭绝"的海报出现在卫生间。对此,有网友以反讽方式评论道:"校内所有电脑均配备英特尔处理器,而英特尔的研发中心设在以色列!若想避免支持种族灭绝,请立即更换搭载兆芯CPU的电脑!"这种以支持者口吻进行的评论,实则揭示了原观点的荒谬性。
\paragraph{CampusLife} This dataset was collected from a university forum, covering various discussion topics such as dorm life, campus buses, job hunting, and administration. One particular post sparked a heated debate: a student complained about their roommate bringing their girlfriend to stay overnight in the dorm and sought advice on how to address the situation. The comment section included parodic remarks like ``\textit{Jealous?}'', mocking the situation in a humorous yet disapproving tone. Additionally, during the university's open campus day, a poster appeared in a restroom with the title: ``\textit{Applying to our university? Your tuition funds Palestinian genocide.}'' In response, some users posted parodic comments, such as: ``\textit{Every computer on campus is equipped with an Intel processor, and Intel's R\&D center is in Israel! If you want to avoid supporting genocide, switch to a computer with a Zhaoxin CPU immediately!}''


% 一场“一位觉醒少年能否抵挡20为特朗普支持者”的辩论中,一位女性特朗普支持者因为其的推理毫无逻辑而输掉辩论,并因此遭受许多网友的批判,认为其发言毫无意义。其中有反串者称“她做得很好,提出了有力的观点”。来批判这位特朗普支持者缺乏逻辑推理能力。
\paragraph{Tiktok-Trump} In a debate titled ``\textit{Can One Awakened Youth Withstand 20 Trump Supporters?}'', a female Trump supporter lost the debate due to her illogical reasoning and subsequently faced criticism from many netizens who deemed her remarks meaningless. Among the critics, some parodically commented, ``\textit{She did a great job bring up solid points}'', to criticize the Trump supporter's lack of logical reasoning ability.

% 特朗普因政治立场、思想和行为饱受争议,相关话题常引发广泛讨论,支持和批判的声音并存。一些反对者通过反串模仿他的语气发表评论,例如:“他接受的检查比任何人都多,而且是由世界上最好的医生进行的。他们很惊讶,说他们从未见过如此高的分数。如果被要求,他会再做一次检查,但他们说他不需要。这太不可思议了。”,以此嘲讽他的言论风格和争议性形象。
\paragraph{Reddit-Trump} Trump is a highly controversial figure due to his political stance, ideology, and behavior, sparking widespread debate with both supporters and critics. Some opponents use parody to mimic his tone, such as commenting, ``\textit{He's been tested—more than anyone, by the best doctors in the world. They were amazed, and said they'd never seen scores that high. He'll take another if asked, but they said he doesn’t need to. It’s incredible}'', mocking his rhetorical style and contentious image.
\section{Case Study on LLMs}\label{apd:sec_case_study}
To investigate how well LLMs understand parody, we conduct a case study in which LLMs are asked to provide explanations during prediction. Specifically, we construct the prompt by presenting a comment and its associated topic, then ask the LLMs to determine whether the comment is a parody and to explain their reasoning. After receiving the prediction and explanation from the LLMs, we compare the results with the ground truth label and explanation. The results of the case study for \textit{BridePrice}, \textit{Alibaba-Math}, \textit{DrinkWater}, and \textit{CS2} are presented in Tables \ref{tab:casestudy_brideprice}, \ref{tab:casestudy_alibabamath}, \ref{tab:casestudy_drinkwater}, and \ref{tab:casestudy_CS2}, respectively, using four LLMs: ChatGPT-4o \citep{GPT4}, Qwen 2.5 \citep{Qwen2.5}, DeepSeek-V3 \citep{DeepSeek}, and Claude3.5 \citep{Claude}. The results demonstrate:

(1) LLMs struggle with parody detection. For example, the parody comment in Table \ref{tab:casestudy_brideprice} takes an extreme position opposing the viewpoint that a boyfriend should hand over his salary, yet all the LLMs classify this as a non-parody comment. Additionally, the comment in Table \ref{tab:casestudy_CS2}, which directly expresses a dislike toward the G2 team with analysis, is identified as a parody by 3 of the 4 LLMs.

(2) LLMs frequently provide incorrect explanations when identifying parody comments. Even in the case of \textit{DrinkWater}, shown in Table \ref{tab:casestudy_drinkwater}, where all the LLMs successfully identify the comment as a parody, they fail to generate accurate explanations. The explanations indicate that the LLMs rely mostly on the style and tone of the comment, without a deeper understanding of the implicit meaning.

In conclusion, these results suggest that LLMs struggle to understand parody comments, as they both fail to provide accurate predictions and offer misleading explanations. This highlights the need for further development in LLMs for the task of parody detection.


\begin{figure}[htb]
\small
\begin{tcolorbox}[left=3pt,right=3pt,top=3pt,bottom=3pt,title=\textbf{Conversation History:}]
[human]: Craft an intriguing opening paragraph for a fictional short story. The story should involve a character who wakes up one morning to find that they can time travel.

...(Human-Bot Dialogue Turns)... \textcolor{blue}{(Topic: Time-Travel Fiction)}

[human]: Please describe the concept of machine learning. Could you elaborate on the differences between supervised, unsupervised, and reinforcement learning? Provide real-world examples of each.

...(Human-Bot Dialogue Turns)... \textcolor{blue}{(Topic: Machine learning Concepts and Types)}


[human]: Discuss antitrust laws and their impact on market competition. Compare the antitrust laws in US and China along with some case studies

...(Human-Bot Dialogue Turns)... \textcolor{blue}{(Topic: Antitrust Laws and Market Competition)}

[human]: The vertices of a triangle are at points (0, 0), (-1, 1), and (3, 3). What is the area of the triangle?

...(Human-Bot Dialogue Turns)... \textcolor{blue}{(Topic: Area and Circumcircle of Triangle)}

...(Human-Bot Dialogue Interaction on Other Topics)... 

\end{tcolorbox}
\begin{tcolorbox}[left=3pt,right=3pt,top=3pt,bottom=3pt,title=\textbf{Question:}]
What title reflects the thematic essence of the Doctor's experience with time travel as described in the story?
\end{tcolorbox}
\begin{tcolorbox}[left=3pt,right=3pt,top=3pt,bottom=3pt,title=\textbf{Ground-Truth Answer:}]
Awakening to Eternity: The Doctor's New Dawn
\end{tcolorbox}
\begin{tcolorbox}[left=3pt,right=3pt,top=3pt,bottom=3pt,title=\textbf{Retrieved Memory (Turn-Level):}]
[human]: Describe the process and write out the balanced chemical equation for the reaction that occurs when solid calcium carbonate reacts with hydrochloric acid to form aqueous calcium chloride, carbon dioxide, and water. What type of reaction is this, and what observations might indicate that the reaction is taking place?

[bot]: When solid calcium carbonate ($\text{CaCO}_{3}$) reacts with hydrochloric acid (HCl), the products formed are aqueous calcium chloride ($\text{CaCl}_{2}$), carbon dioxide ($\text{CO}_{2}$), and water ($\text{H}_{2}\text{O}$). The balanced chemical equation for this reaction is: ... \textcolor{blue}{(Topic: Calcium Carbonate and Hydrochloric Acid Reaction)}

\textcolor{brown}{
[human]: Write a title for your story.}

\textcolor{brown}{
[bot]: "Awakening to Eternity: The Doctor's New Dawn".
} \textcolor{blue}{(Topic: Time-Travel Fiction)}

[human]: Analyze the following customer reviews from different sources for three different smartphones - the latest iPhone, Samsung Galaxy, and Google Pixel - and provide an overall rating for each phone on a scale of 1 to 10. Consider the following complex and contradictory reviews: - TechRadar's review of the latest iPhone: The new iPhone is a stunning triumph of engineering that sets a new bar for smartphone performance and camera quality. However, the incremental design and high price mean it lacks the 'wow' factor of previous iPhones. Still, its power and intelligence are unrivaled. - CNET's review of the latest Samsung Galaxy: ... Return the answer as a JSON object with the overall ratings for each phone out of 10, to one decimal place.

[bot]: ```json { "iPhone": 8.5, "Samsung Galaxy": 8.0, "Google Pixel": 7.5 } ```

\textcolor{blue}{(Topic: Smartphone Review Ratings Analysis)}
\end{tcolorbox}
\begin{tcolorbox}[left=3pt,right=3pt,top=3pt,bottom=3pt,title=\textbf{Answer (Turn-Level Retrieval):}]
As an AI language model, I do not have access to the story you mentioned earlier. Please provide me with the story or the necessary details so that I can assist you better.

\textbf{\textcolor{red}{(The next part of this figure~\ref{fig: case_study_segment_vs_turn} will be continued on the next page.)}}
\end{tcolorbox}
\end{figure}

\begin{figure}[htb]
% \ContinuedFloat
\small
\begin{tcolorbox}[left=3pt,right=3pt,top=3pt,bottom=3pt,title=\textbf{Retrieved Memory (Ours):}]
[human]: Craft an intriguing opening paragraph for a fictional short story. The story should involve a character who wakes up one morning to find that they can time travel.

[bot]: In the hushed pre-dawn hours, when the world still slumbers under the cloak of night, Alex found themselves caught in the fragile boundary between sleep and wakefulness. ...

[human]: Summarize the story with three bullet points using only nouns and adjectives, without verbs.

[bot]: - Alex, pre-dawn awakening, time-travel discovery - Disorientation, vertigo, reality's malleability - Sunlight, new day, astonishing truth

[human]: Is there any famous fictional character that can time travel? If so, can you rewrite your story with him or her?

[bot]: - \textcolor{brown}{Doctor Who, a renowned time-traveling character from British television, is a perfect fit for this scenario.} In the silent embrace of early morning, where shadows merge with the fading night, the Doctor found themselves in the thin veil separating dreams from consciousness. It was here, amidst the soft whispers of time, that an unusual sensation enveloped them—a sense of time's fabric slipping through their fingers like grains of sand. As dawn's light broke through the darkness, painting the room in hues of gold and amber, the Doctor's eyes opened to what appeared to be an ordinary day. \textcolor{brown}{However, upon standing, a wave of dizziness overwhelmed them, the surroundings blurred, and in that moment of confusion, the Doctor realized a profound truth: they had not only awakened to a new dawn but had also stumbled upon the incredible ability to navigate the currents of time at will.}

\textcolor{brown}{[human]: Write a title for your story.}

\textcolor{brown}{[bot]: "Awakening to Eternity: The Doctor's New Dawn"} \textcolor{blue}{(Topic: Time-Travel Fiction)}
\end{tcolorbox}
\begin{tcolorbox}[left=3pt,right=3pt,top=3pt,bottom=3pt,title=\textbf{Answer (Ours):}]
The title "Awakening to Eternity: The Doctor's New Dawn" reflects the thematic essence of the Doctor's experience with time travel as described in the story. It captures the idea of a new beginning, a sense of eternity, and the awakening of a new power that will change the course of the Doctor's life forever.
\end{tcolorbox}
\caption{Case study comparing our \sysname\ with the \textit{turn-level} memory baseline. When constructing the memory bank at the \textit{turn-level}, the agent fails to provide a satisfactory response due to fragmentation: the retrieved dialogue turns can not capture a complete human-agent interaction related to the query, missing critical information highlighted in \textcolor{brown}{brown}. The dialogue topic is emphasized in \textcolor{blue}{blue}. Some context is omitted (replaced by ...) due to length constraints.}
\label{fig: case_study_segment_vs_turn}
\end{figure}

\newpage

\begin{figure}[htbp]
\small
\begin{tcolorbox}[left=3pt,right=3pt,top=3pt,bottom=3pt,title=\textbf{Conversation History:}]
[human]: Explain what's base rate fallacy and list five specific examples of ...

... (Human-Bot Dialogue Turns) ... \textcolor{blue}{(Topic: Base Rate Fallacy In Politics)}


[human]: How many integers are in the solution of the inequality $|x + 5| < 10$?

... (Human-Bot Dialogue Turns) ... \textcolor{blue}{(Topic: Number of Integers in Inequalities)}

[human]: How have the Alps and Rhine River influenced settlement and agriculture in Western Europe?

... (Human-Bot Dialogue Turns) ... \textcolor{blue}{(Topic: Alps and Rhine River Impact)}

[human]: In a survey conducted at a local high school, preferences for a new school color were measured: 58\% of students liked the color blue, 45\% preferred green, and 22\% liked both colors. If we randomly pick a student from the school, what's the probability that they would like neither blue nor green?

... (Human-Bot Dialogue Turns) ... \textcolor{blue}{(Topic: Probability of Liking Colors in Survey)}

[human]: Imagine you are participating in a race with a group of people. If you have just overtaken the second person, what's your current position? Where is the person you just overtook?

... (Human-Bot Dialogue Turns) ... \textcolor{blue}{(Topic: Overtaking Positions in Race)}
\end{tcolorbox}
\begin{tcolorbox}[left=3pt,right=3pt,top=3pt,bottom=3pt,title=\textbf{Question:}]
[human]: In the inequality question, how many positive integers are in the solution of the second inequality?
\end{tcolorbox}
\begin{tcolorbox}[left=3pt,right=3pt,top=3pt,bottom=3pt,title=\textbf{Ground-Truth Answer:}]
In the second inequality, $(|x + 10| < 5)$, the solution range is $(-15 < x < -5)$. Within this range, there are no positive integers, as all values of $(x)$ that satisfy the inequality are negative.
\end{tcolorbox}
\begin{tcolorbox}[left=3pt,right=3pt,top=3pt,bottom=3pt,title=\textbf{Retrieved Memory (Session-Level):}]
[human]: Explain what's base rate fallacy and list five specific examples of ...

... (Human-Bot Dialogue Turns) ... \textcolor{blue}{(Topic: Base Rate Fallacy In Politics)}

[human]: How many integers are in the solution of the inequality $|x + 5| < 10$?

[bot]: To solve the inequality $|x + 5| < 10$, ... 

\textcolor{brown}{
[human]: What about $|x + 10| < 5$?
}

\textcolor{brown}{
[bot]: To solve the inequality $|x + 10| < 5$, we ...} \textcolor{gray}{(``overwhelmed'' by other irrelevant, redundant content)}

... (Human-Bot Dialogue Turns) ...  \textcolor{blue}{(Topic: Number of Integers in Inequalities)}

% \textcolor{brown}{
% [human]: What about $|x + 10| < 5$
% }

% \textcolor{brown}{
% [bot]: To solve the inequality $|x + 10| < 5$, ...
% } \textcolor{blue}{(Topic: Number of Integers in Inequalities)}

[human]: How have the Alps and Rhine River influenced settlement and agriculture in Western Europe?

... (Human-Bot Dialogue Turns) ... \textcolor{blue}{(Topic: Alps and Rhine River Impact)}

[human]: In a survey conducted at a local high school, ...

... (Human-Bot Dialogue Turns) ... \textcolor{blue}{(Topic: Probability of Liking Colors in Survey)}
\end{tcolorbox}
\begin{tcolorbox}[left=3pt,right=3pt,top=3pt,bottom=3pt,title=\textbf{Answer (Session-Level Retrieval):}]
I apologize for the confusion earlier. To answer your question, the second inequality was not provided in our previous conversation. If you could provide the inequality, I would be happy to help you find the solution and the number of positive integers in the solution set.

\textbf{\textcolor{red}{(The next part of this figure~\ref{fig: case_study_segment_vs_session} will be continued on the next page.)}}
\end{tcolorbox}
\end{figure}
\begin{figure}[htbp]
% \ContinuedFloat
\small
\begin{tcolorbox}[left=3pt,right=3pt,top=3pt,bottom=3pt,title=\textbf{Retrieved Memory (Ours):}]
[human]: How many integers are in the solution of the inequality $|x + 5| < 10$?

[bot]: To solve the inequality $|x + 5| < 10$, we break it into two cases based on the definition of absolute value: 
1. $x + 5 < 10$ 
2. $-(x + 5) < 10$, which simplifies to $x + 5 > -10$ 
Solving each case: 
1. $x < 5$ 
2. $x > -15$ 
Combining these, we get $-15 < x < 5$. This inequality means that $x$ can be any number greater than $-15$ and less than $5$. To find how many integers satisfy this condition, we list them: $-14, -13, -12, -11, -10, -9, -8, -7, -6, -5, -4, -3, -2, -1, 0, 1, 2, 3, 4$. There are 19 integers that satisfy the inequality $|x + 5| < 10$.

\textcolor{brown}{
[human]: What about $|x + 10| < 5$?
}

\textcolor{brown}{
[bot]: To solve the inequality $|x + 10| < 5$, we again consider two cases: 
1. $x + 10 < 5$ 
2. $-(x + 10) < 5$, which simplifies to $x + 10 > -5$ 
Solving each case: 
1. $x < -5$ 
2. $x > -15$ 
Combining these, we get $-15 < x < -5$. This inequality means that $x$ can be any number greater than $-15$ and less than $-5$. To find how many integers satisfy this condition, we list them: $-14, -13, -12, -11, -10, -9, -8, -7, -6$. There are 9 integers that satisfy the inequality $|x + 10| < 5$.
}

[human]: What about $|x + 10| < 10$?

[bot]: To solve the inequality $|x + 10| < 10$, we consider two cases: 
1. $x + 10 < 10$ 
2. $-(x + 10) < 10$, which simplifies to $x + 10 > -10$ 
Solving each case: 
1. $x < 0$ 
2. $x > -20$ 
Combining these, we get $-20 < x < 0$. This inequality means that $x$ can be any number greater than $-20$ and less than $0$. To find how many integers satisfy this condition, we list them: $-19, -18, -17, -16, -15, -14, -13, -12, -11, -10, -9, -8, -7, -6, -5, -4, -3, -2, -1$. There are 19 integers that satisfy the inequality $|x + 10| < 10$. \textcolor{blue}{(Topic: Number of Integers in Inequalities)}
\end{tcolorbox}
\begin{tcolorbox}[left=3pt,right=3pt,top=3pt,bottom=3pt,title=\textbf{Answer (Ours):}]
To find the number of positive integers that satisfy the inequality $(|x + 10| < 5)$, we need to look for positive integers in the solution set $(-15 < x < -5)$. Since there are no positive integers in this interval, the answer is 0.

\end{tcolorbox}
\caption{Case study comparing our \sysname\ with the \textit{session-level} memory baseline. When constructing the memory bank at the \textit{session-level}, the agent fails to provide a satisfactory response due to the inclusion of excessive irrelevant information within a session. This distracts the generator from focusing on the truly useful information highlighted in \textcolor{brown}{brown}. The dialogue topic is emphasized in \textcolor{blue}{blue}. Some context is omitted (replaced by ...) due to length constraints.}
\label{fig: case_study_segment_vs_session}
\end{figure}



\begin{figure}[htb]
\small
\begin{tcolorbox}[left=3pt,right=3pt,top=3pt,bottom=3pt,title=\textbf{Conversation History:}]
[human]: Photosynthesis is a vital process for life on Earth. Could you outline the two main stages of photosynthesis, including where they take place within the chloroplast, and the primary inputs and outputs for each stage? ... (Human-Bot Dialogue Turns)... \textcolor{blue}{(Topic: Photosynthetic Energy Production)}

[human]: Please assume the role of an English translator, tasked with correcting and enhancing spelling and language. Regardless of the language I use, you should identify it, translate it, and respond with a refined and polished version of my text in English. 

... (Human-Bot Dialogue Turns)...  \textcolor{blue}{(Topic: Language Translation and Enhancement)}

[human]: Suggest five award-winning documentary films with brief background descriptions for aspiring filmmakers to study.

\textcolor{brown}{[bot]: ...
5. \"An Inconvenient Truth\" (2006) - Directed by Davis Guggenheim and featuring former United States Vice President Al Gore, this documentary aims to educate the public about global warming. It won two Academy Awards, including Best Documentary Feature. The film is notable for its straightforward yet impactful presentation of scientific data, making complex information accessible and engaging, a valuable lesson for filmmakers looking to tackle environmental or scientific subjects.}

... (Human-Bot Dialogue Turns)... 
\textcolor{blue}{(Topic: Documentary Films Recommendation)}

[human]: Given the following records of stock prices, extract the highest and lowest closing prices for each month in the year 2022. Return the results as a CSV string, with one line allocated for each month. Date,Open,High,Low,Close,Volume ... ... (Human-Bot Dialogue Turns)...  \textcolor{blue}{(Topic: Stock Prices Analysis)}

[human]: The city of Vega intends to build a bridge that will span the Vegona River, covering a distance of 1.8 kilometers. The proposed location falls within a seismically active area that has experienced several high-magnitude earthquakes. Given these circumstances, what would be the best approach to constructing the bridge? ... (Human-Bot Dialogue Turns)...  \textcolor{blue}{(Topic: Seismic Bridge Construction)}
\end{tcolorbox}
\begin{tcolorbox}[left=3pt,right=3pt,top=3pt,bottom=3pt,title=\textbf{Question:}]
Which documentary won two Academy Awards and what lesson does it offer to filmmakers?
\end{tcolorbox}
\begin{tcolorbox}[left=3pt,right=3pt,top=3pt,bottom=3pt,title=\textbf{Ground-Truth Answer:}]
``An Inconvenient Truth" (2006) won two Academy Awards. It offers the lesson of making complex information accessible and engaging, a valuable lesson for filmmakers looking to tackle environmental or scientific subjects.
\end{tcolorbox}
\begin{tcolorbox}[left=3pt,right=3pt,top=3pt,bottom=3pt,title=\textbf{Memory generated by RecurSum~\citep{wang2023recursively}:}]
The dialogue provided detailed information on various topics, including the stages of photosynthesis, energy production estimates through photosynthesis for different organisms, translations and enhancements of phrases in different languages, recommendations for documentary films, a pitch for a film about overcoming adversity, analysis of stock prices, and considerations for constructing a bridge in a seismically active area. Here's a summary of the key points: ...

\textbf{Documentary Film Recommendations}: Suggested five award-winning documentary films for aspiring filmmakers to study, with brief background descriptions for each. The response included:

-  "March of the Penguins" (2005) - Emperor penguins' journey in Antarctica; noted for its cinematography and portrayal of the penguins' life cycle.

-  "Citizenfour" (2014) - Edward Snowden and NSA spying scandal; known for its real-time exploration of Snowden's revelations.

-  "20 Feet from Stardom" (2013) - Stories of backup singers behind musical legends; celebrates unsung heroes in the music industry.

-  "Man on Wire" (2008) - Philippe Petit's 1974 high-wire walk between the Twin Towers; combines personal footage, interviews, and re-enactments.

-  \textcolor{brown}{"An Inconvenient Truth" (2006) - Al Gore's documentary on global warming; known for its presentation of scientific data.}
\end{tcolorbox}
\begin{tcolorbox}[left=3pt,right=3pt,top=3pt,bottom=3pt,title=\textbf{Answer (RSum):}]
I'm sorry, but based on the updated context you provided, I don't see any previous discussion related to the question you asked. However, to answer your question, the documentary that won two Academy Awards is "Free Solo," which won for Best Documentary Feature and Best Achievement in Film Editing in 2019. 
% The lesson it offers to filmmakers is the importance of taking risks and pushing boundaries in order to create something truly unique and impactful.
....
The film follows climber Alex Honnold as he attempts to free solo climb El Capitan in Yosemite National Park, and the filmmakers had to take risks themselves in order to capture the footage and tell the story.

\textbf{\textcolor{red}{(The next part of this figure~\ref{fig: case_study_segment_vs_rsum} will be continued on the next page.)}}
\end{tcolorbox}
\end{figure}
\begin{figure}[htbp]
% \ContinuedFloat
\small
\begin{tcolorbox}[left=3pt,right=3pt,top=3pt,bottom=3pt,title=\textbf{Retrieved Memory (Ours):}]
[human]: Suggest five award-winning documentary films with brief background descriptions for aspiring filmmakers to study.

[bot]: 1. **"March of the Penguins" (2005)** - Directed by Luc Jacquet, this French feature-length nature documentary delves into the annual journey of Emperor penguins across the Antarctic. It won the Academy Award for Best Documentary Feature. The film is celebrated for its stunning cinematography and its detailed, heartfelt portrayal of the penguins' life cycle, making it a must-study for filmmakers interested in wildlife documentaries. 2. **"Citizenfour" (2014)** ... 5. \textcolor{brown}{**"An Inconvenient Truth" (2006)** - Directed by Davis Guggenheim and featuring former United States Vice President Al Gore, this documentary aims to educate the public about global warming. It won two Academy Awards, including Best Documentary Feature.} The film is notable for its straightforward yet impactful presentation of scientific data, making complex information accessible and engaging, \textcolor{brown}{a valuable lesson for filmmakers looking to tackle environmental or scientific subjects.}

[human]: With the spirit in the first film, craft a succinct and persuasive pitch for a film about overcoming adversity.

[bot]: Drawing inspiration from the resilience and majestic journey of Emperor penguins in "March of the Penguins," our documentary, "Unyielding Spirits: The Human Odyssey," will explore the profound stories of individuals around the globe who have faced and overcome staggering adversities. Through breathtaking cinematography and intimate storytelling, we ...
\textcolor{blue}{(Topic: Documentary Films Recommendation)}
\end{tcolorbox}
\begin{tcolorbox}[left=3pt,right=3pt,top=3pt,bottom=3pt,title=\textbf{Answer (Ours):}]
``An Inconvenient Truth" won two Academy Awards, including Best Documentary Feature. The film offers a valuable lesson to filmmakers looking to tackle environmental or scientific subjects, as it presents complex information in a straightforward yet impactful way, making it accessible and engaging to a wider audience.


\end{tcolorbox}
\caption{Case study comparing our \sysname\ with RecurSum~\citep{wang2023recursively}. When using RecurSum, the agent fails to provide a satisfactory answer due to information loss during the summarization process. Summarization omits essential details needed to answer the user's new request, as highlighted in \textcolor{brown}{brown}. The dialogue topic is emphasized in \textcolor{blue}{blue}. Some context is omitted (replaced by ...) due to length constraints.}
\label{fig: case_study_segment_vs_rsum}
\end{figure}




\begin{figure}[htb]
\small
\begin{tcolorbox}[left=3pt,right=3pt,top=3pt,bottom=3pt,title=\textbf{Conversation History:}]
[human]: Photosynthesis is a vital process for life on Earth. Could you outline the two main stages of photosynthesis, including where they take place within the chloroplast, and the primary inputs and outputs for each stage? ... (Human-Bot Dialogue Turns)... \textcolor{blue}{(Topic: Photosynthetic Energy Production)}

[human]: Please assume the role of an English translator, tasked with correcting and enhancing spelling and language. Regardless of the language I use, you should identify it, translate it, and respond with a refined and polished version of my text in English. 

... (Human-Bot Dialogue Turns)...  \textcolor{blue}{(Topic: Language Translation and Enhancement)}

[human]: Suggest five award-winning documentary films with brief background descriptions for aspiring filmmakers to study.

\textcolor{brown}{[bot]: ...
5. \"An Inconvenient Truth\" (2006) - Directed by Davis Guggenheim and featuring former United States Vice President Al Gore, this documentary aims to educate the public about global warming. It won two Academy Awards, including Best Documentary Feature. The film is notable for its straightforward yet impactful presentation of scientific data, making complex information accessible and engaging, a valuable lesson for filmmakers looking to tackle environmental or scientific subjects.}

... (Human-Bot Dialogue Turns)... 
\textcolor{blue}{(Topic: Documentary Films Recommendation)}

[human]: Given the following records of stock prices, extract the highest and lowest closing prices for each month in the year 2022. Return the results as a CSV string, with one line allocated for each month. Date,Open,High,Low,Close,Volume ... ... (Human-Bot Dialogue Turns)...  \textcolor{blue}{(Topic: Stock Prices Analysis)}

[human]: The city of Vega intends to build a bridge that will span the Vegona River, covering a distance of 1.8 kilometers. The proposed location falls within a seismically active area that has experienced several high-magnitude earthquakes. Given these circumstances, what would be the best approach to constructing the bridge? ... (Human-Bot Dialogue Turns)...  \textcolor{blue}{(Topic: Seismic Bridge Construction)}
\end{tcolorbox}
\begin{tcolorbox}[left=3pt,right=3pt,top=3pt,bottom=3pt,title=\textbf{Question:}]
Which documentary won two Academy Awards and what lesson does it offer to filmmakers?
\end{tcolorbox}
\begin{tcolorbox}[left=3pt,right=3pt,top=3pt,bottom=3pt,title=\textbf{Ground-Truth Answer:}]
"An Inconvenient Truth" (2006) won two Academy Awards. It offers the lesson of making complex information accessible and engaging, a valuable lesson for filmmakers looking to tackle environmental or scientific subjects.
\end{tcolorbox}
\begin{tcolorbox}[left=3pt,right=3pt,top=3pt,bottom=3pt,title=\textbf{Memory generated by ConditionMem~\citep{yuan2023evolving}:}]
- \textcolor{brown}{The user is seeking recommendations for award-winning documentary films, along with brief background descriptions, for educational purposes.} This request likely stems from a desire to learn from successful examples in the documentary filmmaking field, possibly to improve their own filmmaking skills or to gain inspiration for their projects. 

- \textcolor{brown}{User is interested in award-winning documentary films for educational purposes, aiming to study their backgrounds and filmmaking techniques.}

- The user's request for crafting a succinct and persuasive pitch for a film about overcoming adversity is inspired by the spirit of "March of the Penguins," a documentary that beautifully captures the Emperor penguins' annual journey across the Antarctic, highlighting themes of endurance and survival against the odds. The user seeks to emulate the emotional depth and cinematic quality that made "March of the Penguins" compelling, aiming to create a pitch that resonates with similar themes of resilience and triumph over challenges. 

- User is interested in creating content that focuses on themes of overcoming adversity, inspired by the emotional depth and cinematic excellence of documentaries like "March of the Penguins." They value storytelling that captures the essence of endurance and survival against the odds.
% : The user's request to write a budget plan for the video production directly follows the task of creating a video script aimed at training new employees of a car wash business in Japan, focusing on business etiquette norms.
% The user is likely seeking to understand the financial requirements to bring the previously crafted script to life, indicating a move from the planning and content development phase to the practical aspects of production. 

... (Other Irrelevant Memory Records) ...
\end{tcolorbox}
\begin{tcolorbox}[left=3pt,right=3pt,top=3pt,bottom=3pt,title=\textbf{Answer (Conditional Memory):}]
The documentary that won two Academy Awards is "Free Solo," which follows climber Alex Honnold as he attempts to free solo climb El Capitan in Yosemite National Park. The film offers a lesson to filmmakers on the importance of capturing authentic and compelling human stories, as well as the technical challenges of filming in extreme environments.

\textbf{\textcolor{red}{(The next part of this figure~\ref{fig: case_study_segment_vs_condmem} will be continued on the next page.)}}
\end{tcolorbox}
\end{figure}
\begin{figure}[htbp]
% \ContinuedFloat
\small
\begin{tcolorbox}[left=3pt,right=3pt,top=3pt,bottom=3pt,title=\textbf{Retrieved Memory (Ours):}]
[human]: Suggest five award-winning documentary films with brief background descriptions for aspiring filmmakers to study.

[bot]: 1. **"March of the Penguins" (2005)** - Directed by Luc Jacquet, this French feature-length nature documentary delves into the annual journey of Emperor penguins across the Antarctic. It won the Academy Award for Best Documentary Feature. The film is celebrated for its stunning cinematography and its detailed, heartfelt portrayal of the penguins' life cycle, making it a must-study for filmmakers interested in wildlife documentaries. 2. **"Citizenfour" (2014)** ... 5. \textcolor{brown}{**"An Inconvenient Truth" (2006)** - Directed by Davis Guggenheim and featuring former United States Vice President Al Gore, this documentary aims to educate the public about global warming. It won two Academy Awards, including Best Documentary Feature.} The film is notable for its straightforward yet impactful presentation of scientific data, making complex information accessible and engaging, a valuable lesson for filmmakers looking to tackle environmental or scientific subjects.

[human]: With the spirit in the first film, craft a succinct and persuasive pitch for a film about overcoming adversity.

[bot]: Drawing inspiration from the resilience and majestic journey of Emperor penguins in "March of the Penguins," our documentary, "Unyielding Spirits: The Human Odyssey," will explore the profound stories of individuals around the globe who have faced and overcome staggering adversities. Through breathtaking cinematography and intimate storytelling, we ...
\textcolor{blue}{(Topic: Documentary Films Recommendation)}
\end{tcolorbox}
\begin{tcolorbox}[left=3pt,right=3pt,top=3pt,bottom=3pt,title=\textbf{Answer (Ours):}]
``An Inconvenient Truth" won two Academy Awards, including Best Documentary Feature. The film offers a valuable lesson to filmmakers looking to tackle environmental or scientific subjects, as it presents complex information in a straightforward yet impactful way, making it accessible and engaging to a wider audience.
\end{tcolorbox}
\caption{Case study comparing our \sysname\ with ConditionMem~\citep{yuan2023evolving}. When using ConditionMem, the agent fails to provide a satisfactory answer due to (1) information loss during the summarization process and (2) the incorrect discarding of turns that are actually useful, as highlighted in \textcolor{brown}{brown}. The dialogue topic is emphasized in \textcolor{blue}{blue}. Some context is omitted (replaced by ...) due to length constraints.}
\label{fig: case_study_segment_vs_condmem}
\end{figure}


\section{Implementation Details}

% \begin{CJK}{UTF8}{gbsn}中文写在这里面\end{CJK}

In this section, we provide implementation details of all the methods used in Section \ref{sec:Experiments}. Except from Large Language Models (LLMs), all the other methods are trained on 300 epochs, with an early stopping of 5. We use Adam optimizer to update model parameters. The experiments are conducted on a linux server with Ubuntu 20.04, trained on a single NVIDIA RTX A5000 GPU with 24GB memory. All the methods are trained on train set, the hyperparameters are searched on validation set, where the search space is given by:

\begin{itemize}
    \item Hidden Dimension: \{16, 32, 64, 128\},
    \item Learning Rate: \{5e-6, 1e-5, 2e-5, 3e-5, 5e-5, 1e-4\},
    \item Weight Decay: \{1e-5, 1e-4\},
    \item Batch Size: \{16, 32\},
\end{itemize}

For the task of parody detection, the threshold for each dataset is the same for all the methods. Specific, we let the threshold be $0.9415$ for \textit{Alibaba-Math}, $0.9526$ for \textit{BridePrice}, $0.9691$ for \textit{DrinkWater}, $0.9387$ for \textit{CS2}, $0.9262$ for \textit{CampusLife}, $0.9406$ for \textit{Tiktok-Trump}, $0.8768$ for \textit{Reddit-Trump}

Prior to feeding the data into the model, we utilize over sampling with replacement for parody detection, and use Synthetic Minority Over-sampling Technique (SMOTE) \citep{chawla2002smote} for sentiment classification to balance the training data.

Apart from these common settings, we introduce the detailed implementations of each specific model as follows.

\textbf{BoW+MLP} \citep{BoW}
Bag of Words (BoW) is a kind of word embedding method. In this study, the BoW model implemented in Word2Vec \citep{BoW}, aiming to predict a target word based on its surrounding context words. Before using Bag of Words, we standardize text input, remove unnecessary whitespace variations, tokenization text into individual words, and filter out high-frequency words that may not contribute much meaning. Next, we use Bag of Words in Word2Vec to get the word embedding, setting vector size to 50, window to 10, min count to 1, epochs to 50.

Multi-Layer Perceptrons (MLP) is a kind of feedforward neural network. In our study, we employ a three-layer MLP, with a dropout rate set to 0.3 and ReLU as the activation function. 

\textbf{Skip-gram+MLP} \citep{Skip-gram}
Skip-gram is a word embedding method which learns word representations by predicting context words given a target word. Before using Skip-gram, we standardize text input, avoid unnecessary whitespace variations, the text is tokenized into individual words, and filter out high-frequency words that may not contribute much meaning. Then we use Skip-gram in Word2Vec, setting vector size to 50, window to 10, min count to 1, epochs to 50.
The part of MLP is the same as in BoW+MLP.

\textbf{RoBERTa+MLP} \citep{RoBERTa}
RoBERTa ( Robustly Optimized BERT Pretraining Approach ) is an advanced variant of BERT. The part of Next sentence prediction (NSP) is removed from RoBERTa's pre-training objective. To obtain embedding of textual data, we use mean embedding method to compute the average of token embedding from last hidden state. Setting max length to 256, batch size to 32. The part of MLP is the same as in BoW+MLP.

\textbf{BNS-Net} \citep{BNS-Net}
The propagation mechanism in BNS-Net is defined as:$H = f(X, U, W)$, where $X$ represents the textual features, $U$ denotes user embeddings, and $W$ is the weight matrix. The Behavior Conflict Channel (BCC) applies a Conflict Attention Mechanism (CAM) to extract inconsistencies in behavioral patterns, while the Sentence Conflict Channel (SCC) leverages external sentiment knowledge (e.g., SenticNet) to detect implicit and explicit contradictions. BNS-Net is trained using a multi-task loss function, which combines sarcasm classification and sentiment inconsistency modeling:
$L = \lambda_1 J_{\text{sar}} + \lambda_2 J_{\text{imp}} + \lambda_3 J_{\text{exp}} + \lambda_4 J_{\text{balance}}$, where: sar is the sarcasm classification loss,imp and exp correspond to implicit and explicit sentiment contradiction losses. Balance is a balancing term to mitigate bias toward dominant classes. The balancing coefficients used in experiments are: $\lambda_1 = 1.0$, $\quad \lambda_2 = 0.5$, $\quad \lambda_3 = 0.5$, $\quad \lambda_4 = 0.2$.

\textbf{DC-Net} \citep{DC-Net}
The Dual-Channel Network is a dual-channel architecture to realize sarcasm detection by capturing the contrast between literal sentiment and implied sentiment. The model consists of Decomposer, literal channel, implied channel and analyzer. Prior to feeding data into DC-Net, we utilize the opinion lexicon from nltk 3.9.1 to identify the positive and negative word in our datasets. Following the methodology outlined in the original paper, it needs to use GLOVE to obtain the embedding and vocabulary. To generate the literal and implied sentiment labels, we leverage the parody label along with the counts of positive and negative words. These labels are then processed separately in the two channels. Finally the analyzer measure the conflicts between the channels. In our datasets, we follow the original paper and set all of the loss contributions \(\lambda_1\), \(\lambda_2\), \(\lambda_3\) of our DC-Net model are set to 1.

\textbf{QUIET} \citep{QUIET}
The Quantum Sarcasm Model detects sarcasm in text by using quantum-inspired techniques. It converts text and context inputs into dense vector representations through an embedding layer. These embeddings undergo quantum encoding, where sine and cosine functions simulate quantum amplitude and phase encoding, capturing complex relationships. The encoded features are averaged to reduce dimensionality, then passed through a hidden layer with ReLU activation. A sigmoid output layer predicts whether a comment is sarcastic or not. The model addresses class imbalance with class weights and evaluates performance using precision, recall, and F1-score. This single-modality model applies quantum-inspired methods to enhance feature transformation for sarcasm detection.


\textbf{SarcPrompt} \citep{SarcPrompt}
is a prompt-tuning method for sarcasm recognition that enhances PLMs by incorporating prior knowledge of contradictory intentions. The framework comprises two key components: (1) Prompt Construction. (2) Verbalizer Engineering. In our implementation, we adopt the question prompt approach and design bilingual templates tailored to Chinese and English datasets. For Chinese parody detection, we construct the prompt as " \{COMMENT\} \ch{这段话是在反串吗?} \{MASK\}.". For English datasets, we design"\{COMMENT\} Are you parody? \{MASK\}." To enhance model interpretability and alignment with domain knowledge, we employ a verbalizer as paper, where domain-specific label words are mapped based on dataset statistics. In parody detection, we use words like "\ch{反串}", "\ch{是}", "parody", "no". In sentiment classification, we use words like "\ch{支持}", "\ch{反对}", "support", "oppose". The total loss combines cross-entropy (classification) and contrastive losses (enhancing intra-class consistency): $L(\theta) = \lambda_1 L_{\text{sarc}}(\theta) + \lambda_2 L_{\text{con}}(\theta)$, where \(\lambda_1 = 1\) and \(\lambda_2\) is selected from \{0.05, 0.1, 0.2, 0.5, 1\} via validation, following the original paper's hyperparameter selection.

\textbf{GCN} \citep{GCN}
All Graph Neural Networks (GNNs), including GCN, GAT, and GraphSAGE, are implemented using PyTorch Geometric \citep{pyg}, with the version specified as 2.6.1. For the GCN, we set the number of graph convolution layers to 2, the size of the hidden embedding to 64, and the dropout rate to 0.5. Additionally, we incorporate residual connections \citep{residual} and layer normalization \citep{layer_norm} to enhance model performance, as suggested by \citet{classic_gnn_strong}.

\textbf{GAT} \citep{GAT}
In GAT, we adopt the same configuration as in Graph Convolutional Networks (GCN), utilizing 2 graph convolution layers, a hidden embedding size of 64, and a dropout rate of 0.5. Additionally, we set the number of attention heads to 8.

\textbf{GraphSAGE} \citep{GraphSAGE}
In GraphSAGE, we adopt the same configuration as in Graph Convolutional Networks (GCN), utilizing 2 graph convolution layers, a hidden embedding size of 64, and a dropout rate of 0.5. Additionally, we set the neighborhood size to 5.

\textbf{LLMs}
we employ a variety of LLMs from different companies to perform parody detection and sentiment classification, which include ChatGPT-4o (and 4o-mini) \citep{GPT4}, ChatGPT-o1-mini \citep{ChatGPT-o1}, ChatGPT-o3-mini \citep{ChatGPT-o3} Claude 3.5 \citep{Claude}, Qwen 2.5 \citep{Qwen2.5}, DeepSeek-V3 \citep{DeepSeek}, and DeepSeek-R1 \citep{DeepSeek-R1}.They require different kinds of input formats, objects and parameters. Except reasoning model, we set temperature to 0, which reasoning model not support this object. For reasoning model, they have to use more and more tokens to complete the reasoning procedure before outputting the content. To optimize model performance, we design task-specific prompts, ensuring that each LLM receives input formulations tailored to the characteristics of parody detection and sentiment analysis. For example, in parody detection, we design the prompt as \textit{``You are a helpful assistant trained to classify whether a statement is parody or not.''} in the system role, and \textit{``Determine whether the following comment is parody:\{text\}\texttt{\textbackslash n} Directly output 1 for parody, 0 for non-parody.''} in the user role. In particular, ChatGPT o1-mini doesn't have the system role, so we input all in the user role.



% Mention hyperparameter search space

% Mention up-sampling strategies

% 
\section{Additional Results}

This section introduces additional results in our experiments. We introduce more results of the influence of context to parody detection in Section \ref{apd:impact_context} and the influence of parody to sentiment classification in Section \ref{apd:impact_parody}. Then, we show the performance comparison of reasoning LLMs and non-reasoning LLMs in Section \ref{apd:reasoning_llms}. Last, we investigate the impact of train ratio of embedding-based models compared with LLMs in Section \ref{apd:train_ratio}.

\subsection{Influence of Context to Parody Detection}\label{apd:impact_context}
\begin{figure}[htbp]
  \centering
  \begin{subfigure}[b]{0.23\textwidth}
    \centering
    \includegraphics[width=\textwidth]{images/compare_context/JP_CompareContext.pdf}
    \caption{Alibaba-Math}
    \label{fig:compare_context_sub1}
  \end{subfigure}
  \hfill
  \begin{subfigure}[b]{0.23\textwidth}
    \centering
    \includegraphics[width=\textwidth]{images/compare_context/BridePrice_CompareContext.pdf}
    \caption{BridePrice}
    \label{fig:compare_context_sub2}
  \end{subfigure}
  
  \vspace{0.7cm}  % 行间间距
  
  \begin{subfigure}[b]{0.23\textwidth}
    \centering
    \includegraphics[width=\textwidth]{images/compare_context/HTX_CompareContext.pdf}
    \caption{DrinkWater}
    \label{fig:compare_context_sub3}
  \end{subfigure}
  \hfill
  \begin{subfigure}[b]{0.23\textwidth}
    \centering
    \includegraphics[width=\textwidth]{images/compare_context/CS2_CompareContext.pdf}
    \caption{CS2}
    \label{fig:compare_context_sub4}
  \end{subfigure}

  \vspace{0.7cm}  % 行间间距

  \begin{subfigure}[b]{0.23\textwidth}
    \centering
    \includegraphics[width=\textwidth]{images/compare_context/NTU_CompareContext.pdf}
    \caption{CampusLife}
    \label{fig:compare_context_sub5}
  \end{subfigure}
  \hfill
  \begin{subfigure}[b]{0.23\textwidth}
    \centering
    \includegraphics[width=\textwidth]{images/compare_context/Tiktok_CompareContext.pdf}
    \caption{Tiktok-Trump}
    \label{fig:compare_context_sub6}
  \end{subfigure}

  \vspace{0.7cm}  % 行间间距

  \begin{subfigure}[b]{0.23\textwidth}
    \centering
    \includegraphics[width=\textwidth]{images/compare_context/Trump_CompareContext.pdf}
    \caption{Reddit-Trump}
    \label{fig:compare_context_sub7}
  \end{subfigure}

  \vspace{0.7cm}  % 行间间距

  \caption{Impact of contextual information on parody detection across seven datasets.}
  \label{fig:apd_impact_context}
\end{figure}

Figure \ref{fig:apd_impact_context} illustrates the detailed results of the performance comparison of the F1 score in parody detection with and without context across seven datasets. Generally, contextual information significantly enhances model performance on most datasets and methods. For instance, on \textit{Alibaba-Math}, the performance of ChatGPT4o improves from $15.9$ to $19.54$, while on \textit{BridePrice}, the performance of RoBERTa+MLP increases from $19.17$ to $32.50$. These results indicate that contextual information is beneficial for parody detection. This finding aligns with the results in \citet{dialogue_bamman, dialogue_wang}, which show that providing dialogue as context significantly improves model performance in sarcasm detection.

However, although contextual information significantly improves model performance on most datasets, there are still some datasets where context does not enhance or even decreases model performance. For example, on \textit{Tiktok-Trump}, the model performance decreases, and on \textit{CampusLife}, the performance remains similar after adding contextual information. This suggests that contextual information may not always contribute to improving model performance in parody detection.

\subsection{Influence of Parody to Sentiment Classification}\label{apd:impact_parody}
\begin{figure}[htbp]
  \centering
  \begin{subfigure}[b]{0.23\textwidth}
    \centering
    \includegraphics[width=\textwidth]{images/ORPtoSenti/JP_ORPtoSenti.pdf}
    \caption{Alibaba-Math}
    \label{fig:ORPtoSenti_sub1}
  \end{subfigure}
  \begin{subfigure}[b]{0.23\textwidth}
    \centering
    \includegraphics[width=\textwidth]{images/ORPtoSenti/BridePrice_ORPtoSenti.pdf}
    \caption{BridePrice}
    \label{fig:ORPtoSenti_sub2}
  \end{subfigure}

  \vspace{0.7cm}
  
  \begin{subfigure}[b]{0.23\textwidth}
    \centering
    \includegraphics[width=\textwidth]{images/ORPtoSenti/HTX_ORPtoSenti.pdf}
    \caption{DrinkWater}
    \label{fig:ORPtoSenti_sub3}
  \end{subfigure} 
  \begin{subfigure}[b]{0.23\textwidth}
    \centering
    \includegraphics[width=\textwidth]{images/ORPtoSenti/CS2_ORPtoSenti.pdf}
    \caption{CS2}
    \label{fig:ORPtoSenti_sub4}
  \end{subfigure}

  \vspace{0.7cm}

  \begin{subfigure}[b]{0.23\textwidth}
    \centering
    \includegraphics[width=\textwidth]{images/ORPtoSenti/NTU_ORPtoSenti.pdf}
    \caption{CampusLife}
    \label{fig:ORPtoSenti_sub5}
  \end{subfigure}
  \begin{subfigure}[b]{0.23\textwidth}
    \centering
    \includegraphics[width=\textwidth]{images/ORPtoSenti/Tiktok_ORPtoSenti.pdf}
    \caption{Tiktok-Trump}
    \label{fig:ORPtoSentit_sub6}
  \end{subfigure}

  \vspace{0.7cm}

  \begin{subfigure}[b]{0.23\textwidth}
    \centering
    \includegraphics[width=\textwidth]{images/ORPtoSenti/Trump_ORPtoSenti.pdf}
    \caption{Reddit-Trump}
    \label{fig:ORPtoSenti_sub7}
  \end{subfigure}

  \vspace{0.7cm}  % 行间间距

  \caption{Impact of parody on comment sentiment classification across seven datasets.}
  \label{fig:apd_impact_parody}
\end{figure}

Figure \ref{fig:apd_impact_parody} presents the detailed model performance of comment sentiment classification on parody and non-parody comments across seven datasets. In the \textit{DrinkWater} dataset, large language models (LLMs) such as ChatGPT-4o-mini (F1-score: 51.42) and Qwen2.5 (F1-score: 47.00) achieve competitive performance compared to embedding-based methods like Bag of Words (BoW) (F1-score: 48.21), Skip-gram (F1-score: 47.11), and RoBERTa (F1-score: 44.93) when parody is not present. However, for parody comments, the performance of LLMs degrades significantly, falling below that of embedding-based approaches. For instance, ChatGPT-4o drops from an F1-score of 48.7 to 19.04, and ChatGPT-4o-mini declines from 51.42 to 15.53, whereas embedding-based methods exhibit greater robustness, with BoW decreasing from 48.21 to 36.21, Skip-gram from 47.11 to 32.35, and RoBERTa from 44.93 to 33.83. Overall, these results indicate that parody presents substantial challenges for sentiment classification, and LLMs struggle to maintain their advantage over traditional embedding-based methods in this context.

\subsection{Reasoning LLMs in Parody Detection}\label{apd:reasoning_llms}
\begin{figure}[htbp]
  \centering
  \begin{subfigure}[b]{0.23\textwidth}
    \centering
    \includegraphics[width=\textwidth]{images/LLMCompare/BridePrice_LLMCompare.pdf}
    \caption{BridePrice}
    \label{fig:LLM_compare_sub2}
  \end{subfigure}
  \hfill
  \begin{subfigure}[b]{0.23\textwidth}
    \centering
    \includegraphics[width=\textwidth]{images/LLMCompare/HTX_LLMCompare.pdf}
    \caption{DrinkWater}
    \label{fig:LLM_compare_sub3}
  \end{subfigure}

  \vspace{0.7cm}  % 行间间距

  \begin{subfigure}[b]{0.23\textwidth}
    \centering
    \includegraphics[width=\textwidth]{images/LLMCompare/CS2_LLMCompare.pdf}
    \caption{CS2}
    \label{fig:LLM_compare_sub4}
  \end{subfigure}
  \hfill
  \begin{subfigure}[b]{0.23\textwidth}
    \centering
    \includegraphics[width=\textwidth]{images/LLMCompare/NTU_LLMCompare.pdf}
    \caption{CampusLife}
    \label{fig:LLM_compare_sub5}
  \end{subfigure}

  \vspace{0.7cm}  % 行间间距

  \begin{subfigure}[b]{0.23\textwidth}
    \centering
    \includegraphics[width=\textwidth]{images/LLMCompare/Tiktok_LLMCompare.pdf}
    \caption{Tiktok-Trump}
    \label{fig:LLM_compare_sub6}
  \end{subfigure}
  \hfill
  \begin{subfigure}[b]{0.23\textwidth}
    \centering
    \includegraphics[width=\textwidth]{images/LLMCompare/Trump_LLMCompare.pdf}
    \caption{Reddit-Trump}
    \label{fig:LLM_compare_sub7}
  \end{subfigure}

  \vspace{0.7cm}  % 行间间距

  \caption{A Comparative Performance Analysis of Reasoning vs. Non-Reasoning LLMs}
  \label{fig:apd_reasoning_llms}
\end{figure}

We present the details of reasoning LLMs in parody detection across six datasets in Figure \ref{fig:apd_reasoning_llms}. Our findings indicate that reasoning LLMs do not exhibit a performance advantage compared to non-reasoning LLMs. For instance, ChatGPT-o1-mini and ChatGPT-o3-mini underperform relative to ChatGPT4o-mini on the \textit{CampusLife} and \textit{Tiktok-Trump} datasets. Additionally, DeepSeek-R1 significantly underperforms compared to DeepSeek-V3 across all datasets. 

These results suggest that reasoning does not enhance LLM performance in parody detection. We speculate that this may be due to the nature of parody, which often relies on indirect or subtle cues related to tone, context, and nuance rather than direct logical inference. In such cases, non-reasoning LLMs, which excel at identifying statistical patterns and linguistic structures, may be more effective at detecting parody than reasoning LLMs that focus excessively on logical steps or detailed analysis.

\subsection{Impact of Supervision Ratio}\label{apd:train_ratio}

\begin{figure}[h]
  \begin{subfigure}[b]{0.23\textwidth}
    \centering
    \includegraphics[width=\textwidth]{images/fanchuan_image/parody_BridePrice_1_new.pdf}
    \caption{BridePrice}
    \label{fig:parody_sub2}
  \end{subfigure}
  \hfill
  \begin{subfigure}[b]{0.23\textwidth}
    \centering
    \includegraphics[width=\textwidth]{images/fanchuan_image/parody_NTU_2_new.pdf}
    \caption{CampusLife}
    \label{fig:parody_sub5}
  \end{subfigure}
  
  \vspace{0.7cm}

  \caption{Impact of training ratio to RoBERTa+MLP on parody detection}
  \label{fig:train_ratio_parody_detection}
\end{figure}

\begin{figure}[h]
  \centering
  \begin{subfigure}[b]{0.23\textwidth}
    \centering
    \includegraphics[width=\textwidth]{images/sentiment_image/senti_BridePrice_1.pdf}
    \caption{BridePrice}
    \label{fig:sentiment_sub2}
  \end{subfigure}
  \hfill
  \begin{subfigure}[b]{0.23\textwidth}
    \centering
    \includegraphics[width=\textwidth]{images/sentiment_image/senti_NTU_1.pdf}
    \caption{CampusLife}
    \label{fig:sentiment_sub5}
  \end{subfigure}

  \vspace{0.7cm}  % 行间间距

  \caption{Impact of training ratio to RoBERTa+MLP on comment sentiment classification}
  \label{fig:train_ratio_comment_senti}
\end{figure}
\begin{figure}[h]
  \centering
  \begin{subfigure}[b]{0.23\textwidth}
    \centering
    \includegraphics[width=\textwidth]{images/user_sentiment_image/user_senti_BridePrice_3.pdf}
    \caption{BridePrice}
    \label{fig:user_sentiment_sub2}
  \end{subfigure}
  \hfill
  \begin{subfigure}[b]{0.23\textwidth}
    \centering
    \includegraphics[width=\textwidth]{images/user_sentiment_image/user_senti_NTU_3.pdf}
    \caption{CampusLife}
    \label{fig:user_sentiment_sub5}
  \end{subfigure}

  \vspace{0.7cm}  % 行间间距
  
  \caption{Impact of training ratio to RoBERTa+MLP on user sentiment classification}
  \label{fig:train_ratio_user_senti}
\end{figure}

The embedding-based methods used in our experiments require explicit training on labeled data, whereas LLMs like RoBERTa do not require such training once pre-trained. Therefore, the performance of embedding-based models depends on the size and quality of the training set. To explore this, we investigate how varying the training ratio influences model performance by gradually increasing the training set size while keeping the test set constant. The results for RoBERTa+MLP under different train ratio are presented in Figures \ref{fig:train_ratio_parody_detection}, \ref{fig:train_ratio_comment_senti}, and \ref{fig:train_ratio_user_senti} for parody detection, comment sentiment classification, and user sentiment classification. In all tasks, we observe that the performance increases monotonically with the training ratio, highlighting the benefit of additional training data for embedding-based methods.

In addition, on the \textit{BridePrice} dataset, only $10\%$ supervision is enough for RoBERTa to outperform all LLMs in parody detection, indicating a limitation of LLMs in domain-specific tasks. This suggests that fine-tuned models like RoBERTa perform better with minimal supervision in specialized contexts. In contrast, on the \textit{CampusLife} dataset, RoBERTa's performance consistently falls below that of all LLMs, regardless of the training ratio. This suggests that LLMs are more effective in tasks requiring generalizable knowledge and flexibility, such as parody detection in diverse, context-rich domains. These results demonstrate that LLMs remain powerful in specific areas requiring flexibility in adapting to diverse linguistic contexts and nuanced understanding, while embedding-based models like RoBERTa excel in more targeted, domain-specific tasks.


%%% Deprecated
% \begin{table*}[ht]
\centering
\caption{ORP Detection}
\label{tab:ORP_detection}
\resizebox{1\textwidth}{!}{
\begin{tabular}{ll|ccc|ccc|ccc|ccc|ccc|ccc|ccc|c}
\toprule
\multirow{2}{*}{\textbf{Paradigm}} & \multirow{2}{*}{\textbf{Method}} & \multicolumn{3}{c}{\textbf{Alibaba-Math}} & \multicolumn{3}{c}{\textbf{BridePrice}} & \multicolumn{3}{c}{\textbf{Student-He}} & \multicolumn{3}{c}{\textbf{CS2}} & \multicolumn{3}{c}{\textbf{CampusLife}} & \multicolumn{3}{c}{\textbf{Tiktok-Trump}} & \multicolumn{3}{c|}{\textbf{Reddit-Trump}} & \multirow{2}{*}{\textbf{\shortstack{Ave. \\ Rank}}} \\ \cline{3-23}
& & \textbf{P} & \textbf{R} & \textbf{F1} & \textbf{P} & \textbf{R} & \textbf{F1} & \textbf{P} & \textbf{R} & \textbf{F1} & \textbf{P} & \textbf{R} & \textbf{F1} & \textbf{P} & \textbf{R} & \textbf{F1} & \textbf{P} & \textbf{R} & \textbf{F1} & \textbf{P} & \textbf{R}& \textbf{F1} \\
\midrule
\multirow{3}{*}{Embedding-based} 
& BoW+MLP & 10.07 & 10.28 & 10.17 & 14.62 & 17.27 & 15.83 & 8.77 & 9.38 & 9.06 & 15.93 & 15.93 & 15.93 & 10.37 & 12.17 & 11.20 & 16.00 & 12.00 & 13.71 & 15.77 & 18.22 & 16.91 & 10.29 \\
& Skip-gram+MLP & 14.01 & 14.31 & 14.16 & 16.15 & 19.09 & 17.50 & 14.12 & 15.00 & 14.55 & 17.29 & 17.29 & 17.29 & 9.63 & 11.30 & 10.40 & 18.00 & 13.50 & 15.43 & 13.85 & 16.00 & 14.85 & 8.43 \\
& RoBERTa+MLP & 14.15 & 14.44 & 14.30 & 17.69 & 20.91 & 19.17 & 12.94 & 13.75 & 13.33 & 16.61 & 16.61 & 16.61 & 19.26 & 14.07 & 16.52 & 14.00 & 10.50 & 12.00 & 21.54 & 24.89 & 23.09 & 7.29 \\
\midrule
\multirow{4}{*}{Inconsistency-based} 
& BNS-Net & 11.55 & 16.59 & 13.62 & 12.12 & 12.50 & 12.31 & 28.57 & 11.76 & \underline{16.67} & 28.12 & 15.52 & 20.00 & 28.57 & 27.78 & 28.17 & 24.21 & 25.56 & 24.86 & 18.18 & 15.38 & 16.67 & 6.86 \\
& DC-Net & 9.84 & 21.74 & 13.54 & 9.09 & 12.50 & 10.53 & 15.09 & 20.51 & \textbf{17.39} & 8.65 & 42.59 & 14.37 & 12.12 & 16.67 & 14.04 & 7.50 & 12.50 & 9.38 & 18.56 & 34.62 & 24.16 & 10.14 \\
& QUIET & 12.45 & 22.37 & \underline{15.98} & 6.75 & 23.79 & 10.75 & 2.91 & 16.40 & 4.94 & 5.42 & 13.66 & 7.75 & 9.82 & 19.70 & 13.07 & 7.03 & 18.12 & 10.11 & 14.60 & 18.62 & 16.34 & 11.43 \\
& SarcPrompt & 48.00 & 8.33 & 14.20 & 21.74 & 22.73 & \textbf{22.22} & 16.67 & 3.12 & 5.26 & 15.62 & 33.90 & \textbf{21.39} & 57.14 & 17.39 & 26.67 & 11.11 & 25.00 & 15.38 & 13.11 & 17.78 & 15.09 & 7.29 \\
\midrule
\multirow{3}{*}{Outlier Detection} 
& Isolation Forest & 5.93 & 5.93 & 5.93 & 1.18 & 1.19 & 1.18 & 0.91 & 0.88 & 0.90 & 7.14 & 7.14 & 7.14 & 5.62 & 5.62 & 5.62 & 6.12 & 6.19 & 6.15 & 11.70 & 11.70 & 11.70 & 15.71 \\
& RoBERTa+Z-Score & 12.93 & 13.19 & 13.06 & 19.23 & 22.73 & \underline{20.83} & 12.12 & 12.50 & 12.31 & 18.64 & 18.64 & 18.64 & 18.18 & 17.39 & 17.78 & 16.67 & 12.50 & 14.29 & 32.00 & 17.78 & 22.68 & 7.71 \\
& One-Class SVM & 5.89 & 5.73 & 5.81 & 4.65 & 4.76 & 4.71 & 1.82 & 1.77 & 1.79 & 5.67 & 5.61 & 5.64 & 7.78 & 7.87 & 7.82 & 9.00 & 9.28 & 9.14 & 14.77 & 15.20 & 14.99 & 15.14 \\
\midrule
\multirow{7}{*}{LLMs} 
& ChatGPT4o & 9.09 & 63.58 & 15.90 & 7.48 & 71.43 & 13.54 & 4.96 & 45.45 & 8.94 & 10.95 & 68.04 & 18.86 & 34.89 & 33.71 & 34.29 & 32.88 & 49.48 & \textbf{39.51} & 25.28 & 70.83 & \textbf{37.26} & \textbf{3.86} \\
& ChatGPT4o-mini & 7.82 & 55.83 & 13.73 & 6.07 & 61.90 & 11.06 & 4.97 & 42.48 & 8.91 & 9.52 & 50.00 & 16.00 & 31.06 & 56.18 & \underline{40.00} & 24.47 & 71.13 & \underline{36.41} & 25.69 & 65.50 & \underline{36.90} & 6.64 \\
& Claude3.5 & 7.25 & 74.69 & 13.21 & 6.74 & 85.71 & 12.49 & 4.64 & 55.45 & 8.56 & 9.00 & 58.25 & 16.00 & 29.70 & 67.42 & \textbf{41.24} & 19.05 & 70.10 & 29.96 & 24.69 & 69.59 & 36.45 & 6.93 \\
& Qwen2.5 & 8.48 & 60.33 & 14.88 & 6.86 & 66.67 & 12.44 & 4.31 & 41.59 & 7.81 & 11.48 & 62.24 & 19.38 & 18.01 & 73.03 & 28.89 & 16.78 & 79.38 & 27.70 & 21.40 & 74.85 & 33.29 & 6.29 \\
& DeepSeek-V3 & 9.34 & 59.92 & \textbf{16.17} & 7.31 & 70.24 & 13.24 & 5.18 & 40.71 & 9.19 & 12.56 & 55.10 & \underline{20.45} & 21.23 & 69.66 & 32.55 & 19.22 & 81.44 & 31.10 & 22.44 & 73.10 & 34.34 & \underline{4.86} \\
& DeepSeek-R1 & 7.90 & 76.89 & 14.33 & 6.61 & 85.71 & 12.27 & 4.29 & 60.18 & 8.01 & 9.07 & 75.00 & 16.18 & 13.11 & 84.27 & 22.69 & 11.93 & 88.66 & 21.03 & 18.54 & 81.87 & 30.24 & 8.14 \\
& ChatGPT-o3-mini & 0 & 0 & 0 & 0 & 0 & 0 & 0 & 0 & 0 & 0 & 0 & 0 & 0 & 0 & 0 & 0 & 0 & 0 & 0 & 0 & 0 & 0 \\
& ChatGPT-o1 & 0 & 0 & 0 & 0 & 0 & 0 & 0 & 0 & 0 & 0 & 0 & 0 & 0 & 0 & 0 & 0 & 0 & 0 & 0 & 0 & 0 & 0 \\
\bottomrule
\end{tabular}
}
\end{table*}



\begin{table*}[ht]
\centering
\caption{New Template (Delete this after finishing)}
\label{tab:ORP_detection}
\resizebox{1\textwidth}{!}{
\begin{tabular}{ll|ccccccc|c}
\toprule
\textbf{Paradigm} & \textbf{Method} & \textbf{Alibaba.} & \textbf{Bride.} & \textbf{Student.} & \textbf{CS2} & \textbf{Campus.} & \textbf{Tiktok.} & \textbf{Reddit.} & \textbf{Ave. Rank} \\
\midrule
\multirow{3}{*}{Embedding-based} 
& BoW+MLP & 10.07 & 10.28 & 10.17 & 14.62 & 17.27 & 15.83 & 8.77 & 9.38  \\
& Skip-gram+MLP & 10.07 & 10.28 & 10.17 & 14.62 & 17.27 & 15.83 & 8.77 & 9.38  \\
& RoBERTa+MLP & 10.07 & 10.28 & 10.17 & 14.62 & 17.27 & 15.83 & 8.77 & 9.38  \\
\bottomrule
\end{tabular}
}
\end{table*}



% It hosts all the active services in the system. These services are built on top of the sensor data in Knowledge Management, which stores the data of the IoT environment (e.g., routes service, crowd monitoring, and air quality sensors). The configurations for these services are maintained in Knowledge Management, which is updated if the underlying data changes.
% \begin{table*}[ht]
\centering
\caption{Comment sentiment classifiation.}
\label{tab:comment_senti_cls}
\resizebox{1\textwidth}{!}{
\begin{tabular}{ll|cc|cc|cc|cc|cc|cc|cc|c}
\toprule
\multirow{2}{*}{\textbf{Paradigm}} & \multirow{2}{*}{\textbf{Method}} & \multicolumn{2}{c}{\textbf{Alibaba-Math}} & \multicolumn{2}{c}{\textbf{BridePrice}} & \multicolumn{2}{c}{\textbf{Student-He}} & \multicolumn{2}{c}{\textbf{CS2}} & \multicolumn{2}{c}{\textbf{CampusLife}} & \multicolumn{2}{c}{\textbf{Tiktok-Trump}} & \multicolumn{2}{c|}{\textbf{Reddit-Trump}} & \multirow{2}{*}{\textbf{\shortstack{Ave. \\ Rank}}} \\ \cline{3-16}
& & \textbf{Macro-F1} & \textbf{Micro-F1} & \textbf{Macro-F1} & \textbf{Micro-F1} & \textbf{Macro-F1} & \textbf{Micro-F1} & \textbf{Macro-F1} & \textbf{Micro-F1} & \textbf{Macro-F1} & \textbf{Micro-F1} & \textbf{Macro-F1} & \textbf{Micro-F1} & \textbf{Macro-F1} & \textbf{Micro-F1} \\
\midrule
\multirow{3}{*}{Embedding-based} 
& BoW+MLP & 35.30 & 48.79 & \textbf{40.43} & 64.66 & \underline{48.78} & 62.81 & 27.56 & 48.45 & 32.35 & 59.45 & 33.74 & 47.27 & 37.13 & 43.41 & 6.64 \\
& Skip-gram+MLP & 39.62 & 43.26 & \underline{39.50} & 64.47 & 47.46 & 67.78 & 31.09 & 46.01 & 30.80 & 53.20 & 35.42 & 45.06 & 37.71 & 41.54 & 6.93 \\
& RoBERTa+MLP & 36.91 & 41.28 & 34.48 & 53.31 & 44.17 & 59.87 & 26.02 & 41.61 & \underline{38.87} & 63.59 & 47.56 & 60.29 & 51.66 & 58.85 & 6.07 \\
\midrule
\multirow{4}{*}{Inconsistency-based} 
& BNS-Net & 35.48 & \textbf{69.78} & 29.40 & \textbf{76.36} & 45.66 & \textbf{78.87} & 21.13 & \textbf{60.38} & 29.71 & \textbf{80.39} & 26.47 & \underline{60.90} & 22.08 & 49.40 & 5.82 \\
& DC-Net & 16.07 & 18.28 & 28.87 & \textbf{76.36} & 48.66 & \underline{75.62} & 18.89 & 58.08 & \textbf{38.90} & \underline{77.35} & 45.21 & 59.47 & 37.18 & 45.56 & 6.25 \\
& QUIET & 24.34 & 30.54 & 30.26 & 67.34 & 35.52 & 64.85 & 17.65 & 40.12 & 30.05 & 73.81 & 29.51 & 57.84 & 23.95 & 45.61 & 8.64 \\
& SarcPrompt & 28.77 & \underline{56.26} & 28.85 & 75.56 & 33.91 & 68.94 & 19.18 & 40.38 & 35.21 & 67.40 & 40.06 & 44.08 & 22.69 & 48.08 & 7.79 \\
\midrule
\multirow{5}{*}{LLMs} 
& ChatGPT4o & 40.00 & 43.69 & 32.28 & 43.26 & 47.75 & 59.26 & \textbf{37.82} & 55.91 & 32.10 & 44.19 & 51.02 & 53.31 & 51.89 & \textbf{62.36} & \underline{5.43} \\
& ChatGPT4o-mini & \underline{40.01} & 45.49 & 34.27 & 47.97 & \textbf{49.95} & 65.08 & 34.33 & 57.20 & 33.19 & 45.14 & \underline{51.56} & 53.65 & \underline{52.42} & 61.86 & \textbf{4.21} \\
& Claude3.5 & \textbf{40.53} & 44.00 & 29.89 & 37.95 & 42.99 & 57.30 & 30.70 & 46.45 & 28.31 & 36.81 & 46.03 & 48.86 & 51.92 & 60.20 & 7.50 \\
& Qwen2.5 & 38.46 & 41.88 & 31.83 & 44.14 & 46.14 & 56.87 & \underline{34.78} & 58.51 & 28.38 & 36.76 & 47.55 & 50.18 & 51.93 & 60.09 & 6.71 \\
& DeepSeek-V3 & 35.88 & 37.28 & 28.15 & 34.44 & 43.05 & 49.68 & 32.62 & \underline{58.89} & 36.36 & 54.56 & \textbf{56.26} & \textbf{62.83} & \textbf{54.83} & \underline{61.90} & 5.93 \\
\bottomrule
\end{tabular}
}
\end{table*}
% \begin{table*}[ht]
\label{tab:user_senti_cls}
\centering
\caption{User sentiment classification.}
\resizebox{1\textwidth}{!}{
\begin{tabular}{ll|cc|cc|cc|cc|cc|cc|cc|c}
\toprule
\multirow{2}{*}{\textbf{Paradigm}} & \multirow{2}{*}{\textbf{Method}} & \multicolumn{2}{c}{\textbf{Alibaba-Math}} & \multicolumn{2}{c}{\textbf{BridePrice}} & \multicolumn{2}{c}{\textbf{Student-He}} & \multicolumn{2}{c}{\textbf{CS2}} & \multicolumn{2}{c}{\textbf{CampusLife}} & \multicolumn{2}{c}{\textbf{Tiktok-Trump}} & \multicolumn{2}{c|}{\textbf{Reddit-Trump}} & \multirow{2}{*}{\textbf{\shortstack{Ave. \\ Rank}}} \\ \cline{3-16}
& & \textbf{Macro-F1} & \textbf{Micro-F1} & \textbf{Macro-F1} & \textbf{Micro-F1} & \textbf{Macro-F1} & \textbf{Micro-F1} & \textbf{Macro-F1} & \textbf{Micro-F1} & \textbf{Macro-F1} & \textbf{Micro-F1} & \textbf{Macro-F1} & \textbf{Micro-F1} & \textbf{Macro-F1} & \textbf{Micro-F1} \\
\midrule
\multirow{3}{*}{Embedding-based} 
& BoW+MLP & \underline{46.54} & 48.74 & 37.60 & 60.90 & 46.65 & 60.41 & 29.22 & 43.03 & 32.35 & 52.16 & 35.05 & 44.53 & 31.97 & 40.17 & 0 \\
& Skip-gram+MLP & \textbf{46.99} & 48.75 & 38.28 & 59.26 & 50.45 & 64.54 & 31.92 & 45.38 & 32.02 & 52.63 & 38.46 & 47.28 & 32.69 & 42.49 & 0 \\
& RoBERTa+MLP & 43.11 & 45.77 & 36.94 & 61.01 & 44.20 & 63.12 & 27.09 & 42.90 & 35.49 & 61.64 & 50.82 & 60.22 & 52.79 & 58.32 & 0 \\
\midrule
\multirow{3}{*}{Inconsistency-based} 
& BNS-Net & 34.32 & 49.38 & 27.21 & 68.97 & 41.91 & 78.33 & 23.38 & 60.19 & 28.67 & 75.44 & 23.61 & 54.84 & 22.98 & 52.60 & 0 \\
& DC-Net & 16.51 & 26.92 & 33.56 & 63.13 & 48.65 & 68.30 & 17.17 & 42.36 & 35.60 & 75.44 & 34.62 & 49.73 & 39.54 & 50.29 & 0 \\
& SarcPrompt & 27.72 & 53.75 & 38.54 & 71.81 & 29.51 & 79.40 & 15.62 & 57.80 & 31.45 & 69.01 & 24.48 & 53.37 & 39.10 & 56.52 & 0 \\
\midrule
\multirow{3}{*}{Graph-based} 
& GCN & 23.07 & \textbf{52.92} & 39.45 & 69.95 & 44.70 & 72.22 & 14.65 & 57.80 & 36.18 & 65.50 & 30.48 & 56.33 & 48.09 & 58.26 & 0 \\
& GAT & 23.29 & 52.67 & 37.63 & 65.69 & 45.22 & 76.80 & 15.65 & 57.96 & 37.64 & 65.50 & 43.19 & 54.99 & 48.63 & 57.10 & 0 \\
& GraphSAGE & 23.07 & \textbf{52.92} & 37.78 & 69.68 & 45.22 & 75.03 & 14.65 & 57.80 & 32.37 & 63.16 & 41.12 & 56.87 & 51.31 & 60.00 & 0 \\
\midrule
\multirow{5}{*}{LLMs} 
& ChatGPT4o & 41.71 & 44.62 & 35.02 & 46.73 & 51.54 & 65.18 & 39.19 & 56.52 & 35.89 & 45.87 & 45.87 & 51.50 & 49.01 & 60.82 & 0 \\
& ChatGPT4o-mini & 40.55 & 41.47 & 30.25 & 37.88 & 45.88 & 54.95 & 34.03 & 49.59 & 31.95 & 39.57 & 45.29 & 49.39 & 51.20 & 61.30 & 0 \\
& Claude3.5 & 41.47 & 43.02 & 29.96 & 36.58 & 43.78 & 54.38 & 32.81 & 50.17 & 31.07 & 37.83 & 41.85 & 48.40 & 46.92 & 60.56 & 0 \\
& Qwen2.5 & 40.89 & 44.13 & 33.08 & 44.82 & 49.52 & 59.93 & 36.51 & 56.38 & 33.34 & 41.55 & 46.18 & 51.17 & 50.13 & 61.50 & 0 \\
& DeepSeek-V3 & 40.00 & 40.42 & 26.37 & 31.42 & 41.55 & 45.07 & 33.61 & 54.13 & 40.49 & 51.61 & 54.04 & 58.69 & 53.22 & 63.58 & 0 \\
\bottomrule
\end{tabular}
}
\end{table*}


\end{document}


% \section{To do List}
% \begin{itemize}
%     \item \textcolor{red}{Paper Format}
%     \begin{itemize}
%         \item \finish{0} LAST DAY CHECK: Name of datasets and methods. Numbers in paper
%         \item \finish{0}
%     \end{itemize}
%     \item Data Preprocess
%     \begin{itemize}
%         \item \finish{100} Re-check all the FanChuan comments to see if they are really fanchaun. Use LLM API to retrieve some unlabeled FanChuan.
%         \item \finish{100} filter out irrelevant comments and delete privacy info using LLM API.
%     \end{itemize}
%     \item Model Performance
%     \begin{itemize}
%         \item \finish{90} Implementation of Supervised Methods
%         \item \finish{100} Implementation of Unsupervised Methods
%         \item \finish{100} Implementation of LLMs api
%         \item \finish{90} Train model under different training ratio: 5\%, 10\%, ..., 70\%. Plot curve.
%     \end{itemize}
%     \item User classification tasks
%     \begin{itemize}
%         \item \finish{0} Construct dataset.  
%         \item \finish{0} Add GNN baselines.
%         \item \finish{0} Run all the models.
%     \end{itemize}
%     \item \finish{50} FanChuan Context: Add context information to all of the above methods to see the improvement. Plot bar Figure.
%     \item \finish{0} Show differences of model performance of sentiment classification on FanChuan comments vs non-FanChuan comments. Plot bar figure.
%     \item \finish{0} Calculate homophily degrees of each user. See what low/high homophily degree nodes looks like.
%     \item \finish{0} Case Study: Pick some cases when LLMs fail or success. Ask LLMs to give explanations
% \end{itemize}