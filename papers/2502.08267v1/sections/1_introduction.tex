\section{Introduction}
\label{sec:introduction}
A successful paradigm in the design of approximation algorithms
is to first solve a continuous relaxation, which can
typically be done efficiently using
linear programming, and then to round the 
fractional solution $x\in [0,1]^d$
to an integer solution $X\in\{0,1\}^d$. 
Careful choices need to be made in the rounding step
so that the error introduced is low.
Independent randomized rounding is one of the most natural
rounding schemes. In the simplest variant, we independently
set each variable $X_i$ to $1$ with probability $x_i$ and
to $0$ with probability~$1 - x_i$.
The advantage is that the value of every linear function
(over the $d$ variables) is maintained in expectation.
Moreover, for linear functions with small coefficients,
a Chernoff bound yields strong concentration guarantees
for the value.
Hence, if the initial solution $x$ satisfies some linear 
constraints from the continuous relaxation, we can often
argue with several Chernoff bounds combined by a union bound that 
they are still satisfied up to a small error by the
rounded solution $X$.

In some cases, however, the problem dictates structures or
hard constraints on the solution. For example, we might
require $X$ to be (the incidence vector of) a perfect matching
in a given graph or the basis of a given matroid.
Perfect matchings or bases are objects
that are quite simple in many computational aspects,
but it is typically very unlikely that independent
randomized rounding on a fractional object,
that is, a point in the convex hull of the objects we
want, results in one of these objects.
This motivates so-called dependent randomized rounding.
Here, the goal is to achieve similar guarantees
as independent randomized rounding, but with a distribution over a restricted set of objects,
which necessarily introduces some dependency between
variables.
These rounding schemes are typically tailored to specific
object structures and achieving comparable goals is already challenging for very simple structures.

For bipartite perfect matchings, a fundamental structure in
combinatorial optimization, one cannot hope to achieve
similar concentration guarantees to independent randomized rounding, due to the following well-known example.
Given a cycle of length $n\in 2\N$,
there are only two perfect matchings, 
the two alternating sets of edges. If the values
of a linear function over the edges alternate between
$0$ and $1$, then the fractional matching, which takes every
edge with $1/2$ will have a function value of $n/2$, but
each of the integral matchings incurs an additive
distortion of $\Omega(n)$, much higher than the bound
of $O(\sqrt{n})$ that holds with high probability
if each edge is picked independently with probability $1/2$.
If one considers~$b$-matchings or other more general assignment problems, however, then there are non-trivial guarantees that can be achieved with dependent rounding.
Gandhi, Khuller, Parthasarathy, and Srinivasan~\cite{GandhiKPS06} 
show that between any two edges incident to the same vertex, they can establish \emph{negative correlation}. 
Furthermore, their algorithm has the natural property of \emph{marginal preservation},
which means that the probability of $X_e = 1$ is equal to the fractional value $x_e$ for each variable $X_e$.
Together this implies strong concentration guarantees at least for linear functions on the incident edges of each vertex.
The following proposition is a consequence of their result.
\begin{proposition}\label{prop:assign}
    Let $G = (A\cup B, E)$ be a bipartite graph and
    $x\in [0, 1]^{E}$ represent a fractional many-to-many assignment.
    Furthermore, let $c\in\RR^{E}$, and $a_1,\dotsc,a_k\in [0, 1]^E$ such
    that for each $j\in \set{1,\dotsc,k}$ there is some $v\in A\cup B$
    with $\supp(a_j) \subseteq \delta(v)$.
    Then, in randomized polynomial time, one can compute
    $X\in \{0, 1\}^E$ satisfying with constant probability
    \begin{description}
        \item[Cost Approximation.] \hspace{0.13cm} $c\T X \le (1 + \epsilon) \cdot c\T x$
        \item[Concentration.] \hspace{1.03cm} $|a_j\T x - a_j\T X| \le
            O(\max\{\log k, \sqrt{a_j\T x \cdot \log k}\})$ for all $j\in \set{1,\dotsc,k}$
        \item[Degree Preservation.] \hspace{0.05cm}  $X(\delta(v)) \in \{\floor{x(\delta(v)}, \ceil{x(\delta(v))} \}$
    \end{description}
    Here, $\epsilon > 0$ is an arbitrarily small constant
    that influences the success probability.
\end{proposition}

A generalization to matroid intersection with a similar
restriction was shown by Chekuri, Vondrack, and Zenklusen~\cite{ChekuriVZ10}.
The same work also presents a dependent rounding
scheme for a single matroid that outputs a basis
satisfying similar concentration bounds on linear functions without a restriction on the support.
In this study, we ask the following question:
\begin{center}
    \medskip
    \emph{Can we avoid an error in the cost for dependent rounding while maintaining comparable other guarantees?}
    \medskip
\end{center}
We call a rounding algorithm \emph{cost preserving} if it does not
exceed the cost of the fractional starting solution.
Here, we focus on the stronger variant where distributions are only over objects that are cost preserving, although one might be satisfied with a sufficiently high probability of cost preservation in some
cases. We have no evidence that such a relaxation would make the task significantly easier.

There are several situations where even the seemingly
small cost approximation of $(1 + \epsilon)$, as derived from
marginal preservation and Markov's inequality in the previously mentioned result, is unacceptable. For example, the cost of the fractional solution might come from a hard budget constraint $c\T x \le C$ in the problem.
Another situation is an extension of the objective function to
potentially negative values, representing for example the task of
maximizing profit = revenue - cost. Here, Markov's inequality
cannot be applied at all. Finally, an algorithm that preserves the cost
provides polyhedral insights: every fractional object
is in the convex hull of integer objects that marginally deviate
 in the considered linear functions. And similarly, the (non-integral) polytope of a relaxation is
contained in an approximate integral polytope. 
It is easy to see that cost preservation is incompatible with marginal preservation and hence cannot be satisfied by the dependent rounding schemes above: consider $d+1$ variables~$x_0, x_1, \dotsc, x_d$
of which exactly one is selected, then this can be modeled by bases of a uniform matroid or a degree constraint in the assignment problem above.
Suppose that $c_1 = \cdots = c_{d} = 1/(1 - 2^{-d}) > 1$ and $c_0 = 0$ where the fractional solution is given by~$x_1 = \cdots = x_n = (1 - 2^{-d})/d$
and $x_0 = 2^{-d}$, leading to a cost of~$1$. For a marginal preserving distribution, the probability that the integral solution $X$ has a cost lower than~$1$ (i.e.,~$X_0 = 1$) is exponentially small.
Note, however, that this is not an immediate counter-example to our stated goal:
in this example, deterministically choosing $X_0 = 1$ (and~$X_1 = \cdots = X_d = 0$) still maintains
$|a\T x - a\T X| \le 1$ for every $a\in [0, 1]^{d+1}$. 
\paragraph*{Our contributions}
Our results are twofold.
First, we show that one can obtain comparable guarantees to \Cref{prop:assign} while preserving costs. 
\begin{restatable}{theorem}{RDR}
    \label{thm:assign}
    Let $G = (A\cup B, E)$ be a bipartite graph and
    $x\in [0, 1]^{E}$ represent a many-to-many assignment.
    Furthermore, let $c\in\R^{E}$ and $a_1,\dotsc,a_k\in [0, 1]^E$ such
    that for each $j \in \set{1,\dotsc,k}$ there is some $v\in A\cup B$
    with $\supp(a_j) \subseteq \delta(v)$.
    Then, in randomized polynomial time, one can compute
    $X\in \{0, 1\}^E$ satisfying with constant probability
    \begin{description}
        \item[Cost Preservation.] \hspace{0.46cm} $c\T X \le c\T x$,
        \item[Concentration.] \hspace{1.03cm} $|a_j\T x - a_j\T X| \le
            O(\max\{\log k, \sqrt{a_j\T x \cdot \log k}\})$ for all $j\in \set{1,\dotsc,k}$,
        \item[Degree Preservation.] \hspace{0.05cm} $\sss{X}{\delta(v)} \in \{\floor{\, \sss{x}{\delta(v)} \,}, \ceil{\, \sss{x}{\delta(v)} \,} \}$.
    \end{description}
\end{restatable}
\medskip \noindent Note that in contrast to the previous result, we allow for negative components in the cost function~$c$.

Second, we present a non-trivial application of our theorem to an allocation problem we call the \emph{Budgeted Santa Claus Problem (with identical valuations)}.
Colloquially, it is often described as Santa Claus distributing gifts to children on Christmas.
Formally, there are $n$ resources~$\cR$ (gifts) to be distributed among $m$ players~$\cP$ (children). 
Each resource~$j$ has a specific value~$v_j \ge 0$.
Additionally, there is a total budget of $C \ge 0$, and assigning a resource~$j$ to a player $i$ incurs
a cost denoted by $c_{ij} \ge 0$. 
The goal is a distribution of resources among the players where the least happy player is as happy as possible and
the total cost does not exceed the budget $C$. 
Formally, we aim to find disjoint sets $R_i\subseteq \cR $, $i\in \cP$, maximizing 
\begin{equation*}
    \min_{i\in\cP} \sum_{j \in R_i} v_{j}
\end{equation*}
while ensuring that $\sum_{i\in\cP }^{}\sum_{j\in R_i}^{} c_{ij} \leq C$.
Note that not all resources need to be assigned. However, the variant, where all resources must be assigned can be shown to be 
not more difficult than our problem, see Appendix~\ref{subsec:appendix-santa-claus-with-all-resources-assigned}.


It is possible to consider an even more general variant
where each value $v_{ij}$ depends on both player $i$~and
resource~$j$, which we call the \emph{unrelated valuations}.
Mainly, we restrict ourselves to identical valuations because
the understanding of unrelated valuations in literature is
rather poor---even without considering costs.
In fact, much of the recent literature is focused on the so-called
restricted assignment case
of unrelated valuations (without costs),
where~$v_{ij}\in\{0, v_j\}$,
meaning each resource is either desired with a value of~$v_j$ or worthless to a player.
Among players who desire a particular resource,
its value is the same. Our budgeted variant generalizes the
restricted assignment case: observe that by setting costs~$c_{ij}\in\{0,1\}$ and $C = 0$, we can restrict
the set of players to which a resource can be assigned.
In a non-trivial framework, we apply our dependent rounding theorem
to obtain the following approximation guarantee.
\begin{theorem}
    There is a randomized polynomial time $O(\log n)$-approximation algorithm for the budgeted Santa Claus problem.
\end{theorem}

\paragraph*{Other related work for dependent rounding}
Saha and Srinivasan~\cite{SahaS18} also provide a
dependent rounding scheme for allocation problems,
focusing on combinations of dependent
and iterative rounding.
Bansal and Nagarajan~\cite{BansalN16}
combine dependent rounding with techniques from discrepancy theory, known as the Lovett-Meka algorithm~\cite{LovettM15}. They
prove that one can round a fractional independent
set (or basis) of a matroid to an integral one,
while maintaining comparable concentration guarantees to both Lovett-Meka and Chernoff-type bounds.
We note that Bansal and Nagarajan also integrate
costs in their framework, but they make the
assumption that the costs are polynomially bounded,
which is inherently different from our setting (apart from the fact that they consider matroids).

Another well-known dependent rounding scheme is the maximum entropy
rounding developed by Asadpour, Goemans, M{\k{a}}dry, Gharan and Saberi~\cite{AsadpourGMGS17}.
This is used to sample
a spanning tree, i.e., a basis of a particular matroid,
while guaranteeing negative correlation properties and therefore Chernoff-type concentration.
This result led to the first improvement over the longstanding
approximation rate of $\Theta(\log n)$ for the asymmetric
traveling salesman problem (ATSP).
However, all algorithms above guarantee marginal preservation, which means they
cannot guarantee cost preservation.

At least superficially related to our work is the literature on multi-budgeted
independence systems~\cite{ChekuriVZ11, GrandoniZ10}.
Here, the goal is to find objects of certain structures, e.g.,
matchings or independent sets of matroids,
subject to several (potentially hard) packing constraints
of the form
$a\T x\le b$ for some $a\in \RR^n$, $b\in \RR$.
This can also be used to model cost preservation alike to our
results.
Chekuri, Vondrak, and Zenklusen~\cite{ChekuriVZ11}
and Gradoni and Zenklusen~\cite{GrandoniZ10}
show various positive results in a similar spirit to ours.
These results, however, are restricted to downward-closed structures where for a given solution, formed by a
set of elements, all subsets are valid solutions as well.
For example, Chekuri, Vondrak, and Zenklusen achieve
strong concentration results for randomized rounding on matchings, but this relies on dropping edges in long augmenting paths or cycles in order to reduce dependencies.
Gradoni and Zenklusen~\cite{GrandoniZ10} give a rounding algorithm for a constant number of hard budgets, but
this requires rounding down all components of a fractional solution.
Hence, these results are unable to handle instances like matroid basis constraints, perfect matching constraints, or degree preservation as in~\Cref{thm:assign}.


\paragraph*{Other related work for the Santa Claus problem}
Omitting the costs in the variant we study, the problem
becomes significantly easier and admits an EPTAS, see e.g.~\cite{JansenKV20}, which relies on techniques that contrast with the ones that are relevant to us.
As mentioned before, the problem with costs generalizes
the restricted assignment variant and therefore
inherits the approximation hardness of $2-\epsilon$ due to~\cite{BansalS06}.
Here and in the following, we use restricted assignment synonymous
with the variant without costs, but $v_{ij}\in \{0, v_j\}$.
Bansal and Srividenko~\cite{BansalS06} developed a randomized rounding algorithm for the restricted assignment.
Normally, this would lead
to a similar logarithmic approximation rate as ours
(for the problem without costs),
but they show that combining it
with the Lov\'asz Local Lemma yields an even better rate of
$O(\log\log m/ \log\log\log m)$.
Using similar techniques, the rate was improved to a constant by Feige~\cite{Feige08}.
Note that this randomized rounding uses intricate preprocessing
that violates the marginal preserving property and thus cannot even
maintain the cost of a solution in expectation.
Based on local search, there is also a combinatorial approach, 
see e.g.,~\cite{BamasLMRS24, AsadpourFS08, AnnamalaiKS17},
which yields a (better) constant approximation for restricted
assignment. However, it is not at all clear how costs could be
integrated in this framework.

Finally, a classical algorithm by Shmoys and Tardos~\cite{ShmoysT93}
gives an additive guarantee, where the rounded integral solution
is only worse by the maximum value $v_{\max} = \max_{ij} v_{ij}$.
Therefore, it even works in the unrelated case without increasing
the cost. Notably, they state this result for
the dual of minimizing the maximum value, namely the Generalized Assignment Problem.
The mentioned guarantee for Santa Claus is followed by a trivial
adaption, see \Cref{lem:rounding-small-items}.
Although very influential, this is the only technique we are
aware of which considers the problem with costs.
Unfortunately, this additive guarantee does not lead
to a multiplicative guarantee, since the optimum may be lower
than~$v_{\max}$. In fact, it is well known that the linear 
programming relaxation used in~\cite{ShmoysT93} has an unbounded
integrality gap even for restricted assignment~\cite{BansalS06}. Hence, one
cannot hope to improve this by a simple modification.
Nevertheless, this algorithm forms an important subprocedure in our result.

\paragraph*{Notation}


First, we introduce some necessary notation. Let $S,T \in \set{0,1}^E$ be edge sets in a bipartite graph $G = (A\cup B, E)$.
For all $T \subseteq S$, define $S(T) = \sum_{e \in T} S_e$. 
Let $P$ be the convex hull of degree preserving edge sets $S \in [0,1]^E$. Moreover, for any $v \in A\cup B$ define $\delta(v) = \{e \in E \mid v$ is incident to $e\}$. For the sake of simplicity, we use the shorthand notation $[q] = \set{1, \dotsc, q}$ for any $q \in \N$. Furthermore, for any vector $a \in [0, 1]^E$, the support of $a$ is denoted by $\supp(a) = \{e \in E \mid a_e \neq 0 \}$.