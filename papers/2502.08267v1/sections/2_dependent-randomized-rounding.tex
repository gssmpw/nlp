\section{Budgeted Dependent Rounding}
\label{sec:dependent-randomized-rounding}

This section will introduce a dependent randomized rounding procedure,
which produces an an integral solution satisfying certain concentration guarantees, while preserving the cost and the degree of the fractional solution.
The formal properties are summarized in the following theorem.

\RDR*
\medskip

Throughout this section, the proofs of the technical lemmas are deferred to~\cref{subsec:appendix-ommitted-proofs-of-dependent-rounding}.
An oversimplified outline of our algorithm is as follows:
imagine~$x$ is
the average of two integral edge sets, then
the result can be shown by decomposing the
symmetric difference of both edge sets into cycles and paths.
We reduce to this case by starting with many edge sets and iteratively merging pairs of them in
a tree-like manner.

In order to find the initial integral edge sets, we compute a representation of~$x$ (or rather another similar assignment~$y'$) that is a convex
combination of degree preserving edge sets
such that its scalars satisfy
a certain level of discreteness.
Let~$P$ be the convex hull of degree preserving edge sets~$S \in \set{0,1}^E$, that is, those $S$ that satisfy for all $v\in A\cup B$
\begin{equation*}
    S(\delta(v))\in\{\floor{x(\delta(v))}, \ceil{x(\delta(v)}\}.
\end{equation*}
It can be shown that $x$ is contained in $P$ and,
in particular, $x$ is a convex combination of
degree preserving sets. 
In the following lemma, we show something even stronger: there
exists a fractional assignment~$y$ at least
as good as~$x$, which is the
convex combination of only few edge sets
and has few fractional
variables in the support of each constraint.

\begin{restatable}{lemma}{BoundedNumberOfFracVariables}
    \label{lem:bounded-number-of-frac-variables}
    There exists a convex combination $y = \sum_{i \in [k]} \lambda_i S_i$
    where $\lambda_i \in [0,1]$ and $\sum_{i \in [k]} \lambda_i = 1$ and $S_i \in P \cap \{0,1\}^E$ with
    \begin{align}
        c\T y &\le c\T x & \\
        a_j\T y &= a_j\T x &&\forall j \in \set{1,\dotsc,k} \\
        |\{e\in \delta(v) \mid y_e \notin \{0, 1\}\}| &\le 2k &&\forall v\in A\cup B & \label{eq:sos-bounded-support}
    \end{align}
\end{restatable}
Considering~$y$ as a vertex solution of a linear program, the proof follows from analyzing the structure of polytope~$P$.
For our algorithm, however, the scalars~$\lambda_i$ are not discrete enough.
Hence, we use the following lemma to round~$y$ to a more discrete assignment~$y'$.

\begin{restatable}{lemma}{ScalarRounding}
    \label{lem:convex-represenation-scalar-rounding}
    Let $\ell\in\N_{\ge 0}$ 
    and $y = \sum_{i \in [k]} \lambda_i S_i$ where $\lambda_i \in [0,1]$, $\sum_{i \in [k]} \lambda_i = 1$, and $S_i \in \{0,1\}^E$.
    In polynomial time, we can compute $y' = \sum_{i \in [k]} \lambda'_i S_i$ where $\sum_{i \in [k]} \lambda'_i = 1$ and
    \begin{align}
        \lambda'_i &\in \tfrac{1}{2^{\ell}}\cdot \ZZ, &&  \forall i \in \set{1, \dotsc, k} \label{eq:integer-multiples}\\
        \lambda'_i &= \lambda_i, &&  \forall i \in \set{1, \dotsc, k}, \lambda_i\in\{0,1\} \\
        |y_e - y_e'| &\le k \cdot \tfrac{1}{2^\ell}, &&  \forall e \in E \label{eq:close-solution}\\
        \cost{y'} &\le \cost{y}. \label{eq:rounding-cost-preservation}
    \end{align}
\end{restatable}
We prove this lemma by constructing a flow network and
the standard argument that integral capacities imply existence of an integral min-cost circulation.

Notably, this is the first time we incur a small error for the linear functions $a_j$ while the cost is preserved.
More precisely, we use the lemma with 
$\ell := 2\log(2k)$. From \cref{eq:close-solution} follows that for all~$e \in E$
\begin{equation}
    \label{eq:rdr-scalar-rounding-error}
    |y_e - y_e'| \le k \cdot \tfrac{1}{2^\ell} \le \tfrac{1}{2k}.
\end{equation} 
Therefore, the linear functions also slightly change.
Using \cref{eq:sos-bounded-support,eq:rdr-scalar-rounding-error}, it holds that for all~$j\in \set{1,\dotsc,k}$
\begin{equation}
    \label{eq:rdr-increase-in-linear-function}
    |a_j\T x - a_j\T y'| = |a_j\T y - a_j\T y'| = \sum_{e \in E} (a_j)_e | y_e' - y_e | \le 1.
\end{equation} 
Since $y'$ is a convex combination of (integral)
degree preserving sets in $P$, we have $y'\in P$. In other words, the scalar rounding in \cref{lem:convex-represenation-scalar-rounding} does in fact preserve the degree of~$y$. 

Next, we construct a complete binary tree~$\tree$ with levels~$0,1,\dotsc,\ell$, where each node will be labeled with an edge set.
When the algorithm finishes, the label of the root will be~$X \in \set{0,1}^E$ and satisfy the properties stated in \cref{thm:assign}.
In the following, we describe how the algorithm \textsc{TreeMerge} creates the labels on~$\tree$.
The lowest level~$\ell$ represents the fractional assignment~$y' = \sum_{i \in [k]} \lambda'_i S_i$ where~$S_i \in P\cap \set{0,1}^E$ are degree preserving edge sets.
As we can write~$\lambda'_i = h_i/2^{\ell}$ 
for some $h_i\in\ZZ$ and all $h_i$ sum to $2^{\ell}$, we can naturally label~$h_i$ leaves of level~$\ell$ with $S_i$ for all~$i \in \set{1, \dotsc, k}$.
Thus, $y'$ is the average of all labels of level~$\ell$.
For all $j \in \set{0, \dotsc, \ell-1}$, the labels of level~$j$ are derived from those in level~$j + 1$ such that each node's label is closely related to
those of its two children.
Similar to the last level, each level~$j$ represents a (fractional) edge set~$y_j'$ by taking the average of all labels in this level.

One of the central goals in the construction of $y'_j$ is to guarantee~$\cost y_{j}' \le \cost y_{j+1}'$.
We achieve this by creating two complementary
labelings of level $j$ and selecting the better
of the two.
Denote the labels of level~$j+1$ by~$S_{2i-1}, S_{2i}$ for all~$i\in\set{1, \dotsc, 2^{j}}$. 
Here, each pair~$S_{2i-1}, S_{2i}$ represents the children of the $i$-th node in level $j$. For node~$i$, we construct two potential labels~$T_{i}, T_{i}' \in \set{0,1}^E$ using a random procedure with the following guarantees.

\begin{restatable}{lemma}{EdgeSetDecomposition}
    \label{lem:decomposition}
    Let $S_1, S_2 \in \set{0,1}^E$.
    There exists a random polynomial time procedure that constructs two random edge sets $T, T' \in \set{0,1}^E$ with the following properties. 
    \begin{itemize}
        \item It holds that $T + T' = S_1 + S_2$ and $\E(T) = \E(T') = (S_1+S_2)/2$.
        \item For all $v \in A\cup B$ it holds that $T(\delta(v)), T'(\delta(v)) \in \set{\floor{\,(S_1(\delta(v)) + S_2(\delta(v)))/2 \,}, \ceil{\,(S_1(\delta(v)) + S_2(\delta(v)))/2 \,}}$.
        \item For all $v \in A\cup B$ and all $e \in \delta(v)$, there is at most one edge $e' \in \delta(v)\setminus\{e\}$ such that $T_e$ depends on $T_{e'}$. Likewise, there is at most one edge $e' \in \delta(v)\setminus\{e\}$ such that $T'_e$ depends on $T'_{e'}$. 
    \end{itemize}
\end{restatable}
\medskip
\noindent The lemma can be derived in two steps. First, decompose the symmetric difference of~$S_1$ and~$S_2$ into cycles and paths.
Second, for each of cycle and path, randomly select one of the alternating edge sets for~$T$ and the other for~$T'$.
This random process is similar to other dependent rounding approaches, e.g.~\cite{ChekuriVZ10,GandhiKPS06}, except that we also store~$T'$ that contains the ``opposite'' to every decision in $T$.

We create one fractional assignment from the random edge sets~$T_{i}$, $i\in\{1,\dotsc,2^j\}$,
and one from $T'_{i}$, $i\in\{1,\dotsc,2^j\}$, and pick the lower cost assignment for~$y_{j}'$.
Formally, let 
\begin{equation*}
    z_{j} = \tfrac{1}{2^{j}} \sum_{i \in [2^j]} T_{i} \qquad \text{ and } \qquad
    z'_{j} = \tfrac{1}{2^{j}} \sum_{i \in [2^j]} T'_{i} 
\end{equation*}
From the fact that~$T_{i} + T'_{i} = S_{2i} + S_{2i+1}$, it immediately follows that $(z_{j} + z'_{j})/2 = y_{j+1}'$.
If~$\cost z_j \le \cost z'_j$, set~$y_{j}' = z_{j}$. Otherwise,~$y_{j}' = z'_{j}$.
Consequently, we have that
\begin{equation}
    \label{eq:rdr-cost-of-level-j-assignment}
    \cost y_{j}' \le (\cost z_{j} + \cost z'_{j})/2 = \cost y_{j+1}'.
\end{equation}
We determine the labels of level~$j$ by picking either $T_i$ (if $z_j$ was chosen) or $T'_i$ (if $z'_j$ was chosen).
Repeating the procedure for all~$j \in \set{\ell-1,\dotsc,0}$
results in a label for the root node that is identical to $y_0'$. We conclude by setting $X = y_0'$.
As a last step before proving the main theorem, we 
bound how much the linear functions $a_j$ can change in each level.

\begin{restatable}{lemma}{IncreaseOfLinearFunction}
    \label{lem:linear-function-increase-per-level}
    Let $v\in A\cup B$ and $a \in [0, 1]^E$ with $\supp(a) \subseteq \delta(v)$.
    Let $y_{j+1}' \in [0,1]^E$
    be the fractional solution of the $(j+1)$-th
    level of \textsc{TreeMerge}
    and $y'_{j}\in [0,1]^E$
    that of the $j$-th level.
    Let $t= 132 \ln k$.
    Then with probability at least $1 - 1/k^{10}$, it holds that
    \begin{equation}
        \label{eq:lfipl-bound-on-lf-increase}
        |a\T y_j' - a\T y_{j+1}'| \le 2^{-j/2} \left(t + \sqrt{a\T y_{j+1}' \cdot t}\,\right).
    \end{equation}
\end{restatable}
This lemma follows from a standard Chernoff bound.
We are now in the position to prove \cref{thm:assign} using the lemmas above.

\begin{proof}[Proof of \cref{thm:assign}]
    Let~$\ell = 2 \log (2k)$.
    As explained throughout \cref{sec:dependent-randomized-rounding}, we use \cref{lem:bounded-number-of-frac-variables,lem:convex-represenation-scalar-rounding} to obtain a fractional degree preserving assignment~$y'\in P$ with
    $\cost y'\le \cost x$.
    Note that the rounding of~$x$ to~$y'$ marginally changes the linear function values, but we are able to maintain~$|a_j\T y' - a_j\T x|\le 1$, see \cref{eq:rdr-increase-in-linear-function}.
    Afterwards, we use the algorithm \textsc{TreeMerge} to construct a complete binary tree~$\tree$ with~$\ell + 1$ levels
    and corresponding fractional solutions~$y'_{\ell+1} = y', y'_{\ell}, \dotsc, y'_0 = X$.
    It remains to show that~$X$ satisfies all three properties from the theorem.
    By construction, more precisely \cref{eq:rdr-cost-of-level-j-assignment}, it
    holds that
    \begin{equation*}
        \cost X = \cost y_0' \le \cdots \le y'_{\ell+1} = \cost y' \le \cost x.
    \end{equation*}
    Thus, the cost is preserved.
    Next, we will show that the rounding of~$x$ to~$X$ also preserves the degree.
    Due to \cref{lem:convex-hull-of-edge-preserving-edge-sets,lem:bounded-number-of-frac-variables,lem:convex-represenation-scalar-rounding},
    we have that $y' = \sum_{i \in [k]} \lambda'_i S_i$, where $S_i\in P\cap \{0,1\}^E$
    form the labels of level $\ell$ of $\tree$. For all $i\in\{1,\dotsc,k\}$,
    the fact that $S_i\in P$ implies
    \begin{align}
        \label{eq:rdr-degree-preservation}
        \sss{S_i}{\delta(v)} \in \{ \floor{\,\sss{x}{\delta(v)}\,}, \ceil{\, \sss{x}{\delta(v)}\,} \}.
    \end{align}
    By induction over the tree $\tree$, we show that all labels and, in particular, $X$ are indeed degree preserving. \Cref{eq:rdr-degree-preservation} proves the base case.
    Let $S_1$, $S_2$ be the labels for two children of some node in $\tree$
    and $T, T'$ be the two potential labels for the said node (derived using \cref{lem:decomposition}).
    From the third property of \cref{lem:decomposition} directly follows that for all~$v \in A \cup B$ 
    \begin{equation}
        \label{eq:rdr-degree-preservation-of-edge-decomposition}
        T(\delta(v)), T'(\delta(v))
        \in \big\{\floor{ \,\tfrac{1}{2}  \sss{S_{1}}{\delta(v)} + \tfrac{1}{2}\sss{S_{2}}{\delta(v)} \,}, \ceil{\,\tfrac{1}{2} \sss{S_{1}}{\delta(v)} + \tfrac{1}{2}\sss{S_{2}}{\delta(v)}\,} \big\}.
    \end{equation}
    By induction hypothesis, $S_1$ and $S_2$ are degree preserving, so
    \begin{align*}
        &\floor{ \,\tfrac{1}{2}  \sss{S_{1}}{\delta(v)} + \tfrac{1}{2}\sss{S_{2}}{\delta(v)} \,} \ge \floor{ x(\delta(v)) } \text{ and similarly} \\
        &\ceil{ \,\tfrac{1}{2}  \sss{S_{1}}{\delta(v)} + \tfrac{1}{2}\sss{S_{2}}{\delta(v)} \,} \le \ceil{ x(\delta(v)) } .
    \end{align*}
    This concludes the induction step.
    
    It remains to prove the concentration, i.e., that $X$ marginally deviates from~$x$ in each of the
    given linear functions $a_j$.
    We apply \cref{lem:linear-function-increase-per-level} together with a union bound over all $k$ linear functions and all $\ell = 2\log(2k) \le k$ levels of $\tree$.
    Let $t = 30\log k$.
    As a consequence, with probability at least~$1 - 1/k^8$, it holds for all levels~$i \in \set{1, \dotsc , \ell}$ and linear functions~$a_j,j\in\set{1,\dotsc,k}$ that
    \begin{equation}\label{eq:change-a}
        |a_i\T y_{i+1}' - a_i\T y_i'| \le 2^{-i/2} \Big(t + \sqrt{a\T y_{i+1}' \cdot t}\,\Big).
    \end{equation}
    For some universal constant $d$, we prove that for all $j\in\{1,\dotsc,k\}$ and all~$i\in\{0,\dotsc,\ell\}$
    \begin{equation}\label{eq:grow-a}
        a_j\T y'_i \le d (1 + t + a\T x).
    \end{equation}
    Let us first argue that this in fact implies the last part of the theorem.
    Using triangle inequality and geometric series, it holds that
    \begin{align*}
        |a_j\T X - a_j\T x| 
        = |a_j\T y'_{\ell+1} - a_j\T y'_{\ell}| 
        &\le \sum_{i \in \set{0, \dotsc, \ell-1}} |a_j\T y'_{i+1} - a_j\T y'_{i}| \\
        &\le \sum_{i \in \set{0, \dotsc, \ell-1}} \frac{1}{(\sqrt{2})^i} \Big(t + \sqrt{a\T y'_{i+1} \cdot t}\,\Big) \\
        &\le \sum_{i \in \set{0, \dotsc, \ell-1}} \frac{1}{(\sqrt{2})^i} \Big(t + \sqrt{d(1 + t + a\T x) \cdot t}\,\Big) \\
        &= O(\max\{\log k, \sqrt{a\T x \cdot \log k}\}) .
    \end{align*}
    Finally, we prove \cref{eq:grow-a}.
    Let $j\in\{1,\dotsc,k\}$
    and $i\in\{0,\dotsc,\ell-1\}$.
    Let $i' \ge i$ be the minimal index such that
    $a_j\T y'_{i'} \le 1 + t + a_j\T x$.
    As $a_j\T y'_{\ell} = a_j\T y' \le 1 + a_j\T x$, such~$i'$ must exist.
    If $i' = i$, we are done. If $i' \neq i$, then it follows from \cref{eq:change-a} that
    \begin{equation*}
        a\T_j y'_{i'-1} \le (1 + t + a_j\T x) + 2^{-(i'-1)/2}\big(t + (1 + t + a_j\T x)\big) \le 3(1 + t + a_j\T x) \ .
    \end{equation*}
    For all $i'' \in\{i'-1,\dotsc,i\}$, we have $a_j\T y'_{i''} > t$. Thus, the same equation implies
    \begin{equation*}
        a\T_j y'_{i''-1} \le a\T_j y'_{i''} + \frac{2}{2^{(i''-1)/2}} a\T_j y'_{i''} = \left(1 + \frac{1}{2^{(i''-3)/2}}\right) a\T_jy'_{i''} \ .
    \end{equation*}
    Using the inequality $1 + z \le \exp(z)$ for all $z\in\R$, we have
    \begin{align*}
        a\T_j y'_{i} 
        &\le a_j\T y'_{i' - 1} \cdot \prod_{i''=i'-1}^{i+1} \left(1 + \frac{1}{2^{(i''-3)/2}}\right) \\
        &\le a_j\T y'_{i' - 1} \cdot \exp\left(\sum_{i''=i'-1}^{i+1} \frac{1}{2^{(i''-3)/2}}\right) \\
        &\le 3 e^{O(1)} \cdot (1 + t + a_j\T x) .
    \end{align*}
    This shows \cref{eq:grow-a} and thereby concludes the proof.
\end{proof}

\subsection{Omitted Proofs of Dependent Rounding}
\label{subsec:appendix-ommitted-proofs-of-dependent-rounding}

In this section, we provide the proofs of the previously stated lemmas.
Recall that $P$ is the convex hull of integral degree preserving edge sets for $x\in [0,1]^E$.
Let the operator $\oplus$ denote the symmetric difference (i.e., XOR) of two edge sets.
We start by showing that~$P$ indeed contains $x$.

\begin{lemma}
    \label{lem:convex-hull-of-edge-preserving-edge-sets}
    Let $q = |E|$.
    There exists a convex representation $x = \sum_{i \in [q]} \lambda_i S_i \in P$ where $S_i \in P\cap\{0,1\}^E$ and $\lambda_i \in [0,1]$ and
    $\sum_{i \in [q]} \lambda_i = 1$ such that for all $v\in A\cup B$ and~$i \in \set{1,\dotsc,q}$ holds that $\sss{S_i}{\delta(v)} \in \{\floor{\,\sss{x}{\delta(v)}\,}, \ceil{ \,\sss{x}{\delta(v)} \,}\}$.
\end{lemma}

\begin{proof}
    We rely on a standard flow argument. 
    To this end, construct a digraph $D_f = (V_f , A_f)$ as follows. Let $V_f = \{s,t\} \cup V$ be the set of vertices, $A_s$ be a set of arcs directed from $s$ to each $a \in A$, $A_t$ be the set of arcs directed from each $b \in B$ to $t$, and $A_E$ be a directed variant of $E$ from $A$ to $B$. Let $A_f = A_s \cup A_E  \cup A_t \cup \{(t,s)\}$ be the set of arcs in $D_f$. Set the capacity interval for arc $(s,a) \in A_s$ as $[\lfloor x(\delta(a)) \rfloor, \lceil x(\delta(a)) \rceil]$ and similarly for each arc $(b,t) \in A_t$ as $[\lfloor x(\delta(b)) \rfloor, \lceil x(\delta(b)) \rceil]$. Moreover, set the capacity interval for each arc $e \in E$ to $[0,1]$ and for the arc $(t,s)$ to $[0,\infty]$. A function $f: A_f \rightarrow \R$ is called a circulation if $f(\delta^{\mathrm{in}}(v)) = f(\delta^{\mathrm{out}}(v))$ for each vertex $v \in V_f$. 
    
    One can naturally derive a feasible fractional circulation from the vector $x\in [0,1]^E$: the flow from $s$ to $a \in A$ is $x(\delta(a))$, which lies in $[\lfloor x(\delta(a)) \rfloor, \lceil x(\delta(a)) \rceil]$, the circulation from $b \in B$ to $t$ is $x(\delta(b))$, which lies in $[\lfloor x(\delta(b)) \rfloor, \lceil x(\delta(b)) \rceil]$, the circulation on each arc $(a,b) \in A_E$ is $x_{ab}$, and the circulation on arc $(t,s)$ is $x(E)$. Hence, $x$ satisfies the capacity and flow conservation constraints. 
    It is well known, see e.g.~\cite[Corollary 13.10b]{schrijver2003combinatorial}, that for integral capacities the set of all feasible circulations (including $x$) forms an integral polytope. Thus, we can rewrite the circulation above as a convex combination
    of integral circulations. Each of these integral circulations corresponds
    to a degree preserving edge set.
\end{proof}

Further, there always exists a comparable convex representation which has only few fractional variables in the support of each constraint.

\BoundedNumberOfFracVariables*

\begin{proof}
    First, we argue about the structure of the edges of polytope $P$.
    Let $S, T\in \{0,1\}^E$ be the two vertices at the two ends of
    some edge of $P$. We claim that $S\oplus T$ is a simple cycle
    or a simple path. Suppose not. If $S\oplus T$ is acyclic, let $D$
    be a maximal path in $S\oplus T$; otherwise let $D$ be a simple cycle
    contained in $S\oplus T$.
    Note that $S \oplus D \in P$ and $T \oplus D\in P$ and both
    points do not lie on the edge between $S$ and $T$.
    However, $(S + T)/2 = (T \oplus D + S \oplus D) / 2$ contradicts
    that $S$ and $T$ are connected by an edge.

    Let $y$ be an optimal vertex solution of the linear program
    \begin{align*}
        \min c\T& y \\
        a_j\T y &= a_j\T x &\forall j \in \set{1,\dotsc,k} \\
        y &\in P
    \end{align*}
    Due to~\cref{lem:convex-hull-of-edge-preserving-edge-sets}, the linear program is feasible.
    The solution $y$ must lie on a face~$F$ of~$P$ with dimension at most $k$.
    Consider an arbitrary vertex $S$ of $F$. Furthermore, let $T_1,\dotsc,T_h$, $h\le k$, be the vertices of $F$ such that there is an edge between each $T_i$ and $S$.
    Thus, we can write $y = S + \sum_{i \in [h]} \lambda_i (T_i - S)$ where $\lambda_1,\dotsc,\lambda_h \ge 0$.
    By our previous argument, $T_i \oplus S$ is a simple cycle or path for each $i\in \set{1,\dotsc,h}$.
    In particular, $|(T_i \oplus S) \cap \delta(v)| \le 2$ for each $v\in A\cup B$. This implies that the last property holds for $y$.
\end{proof}

Allowing a small rounding error, there always exists a convex representation where the scalars are integer multiples of a power of two.

\ScalarRounding*

\begin{proof}
    Using a standard flow argument,
    we construct a digraph $D_f = (V_f, A_f)$ that represents a circulation network. For each arc in $a \in A_f$, we adjust a capacity interval such that every feasible circulation corresponds to a solution that is ``close'' to $x$ and preserves the cost. Let $V_f =\{ s, t \} \cup \{ v_i : i\in [k]\}$ where nodes $v_i$ correspond to each $S_i$. The set of arcs~$A_f$ contains arcs $(t, s)$, $(s,v_i)$, and $(v_i,t)$ for $i \in [k]$. A function $f: A_f \rightarrow \R$ is called a circulation if $f(\delta^{\mathrm{in}}(v)) = f(\delta^{\mathrm{out}}(v))$ for each vertex $v \in V_f$. 
    Set the capacity interval of arc~$(t,s)$ to~$[2^\ell, 2^\ell]$ and
    the one of arcs~$(s,v_i)$ and~$(v_i, t)$ to $[\lfloor \lambda_i 2^\ell \rfloor, \lceil \lambda_i 2^\ell \rceil]$ for $i \in [k]$. 
    Moreover, define a linear cost function $\mathrm{cost}(s,v_i) = \sum_{e \in E} c_e (S_i)_e$, the contribution of $S_i$ to the total cost.
    Set $\mathrm{cost}(t,s) = \mathrm{cost}(v_i, t) = 0$.
    The scalars $\lambda_i$ induce a natural circulation $f$: on 
    arcs~$(s, v_i)$ and~$(v_i, t)$, we send a flow of $\lambda_i 2^\ell$
    and on arc~$(t, s)$, we send a flow of $2^\ell$.
    
    For integral capacities the set of feasible circulations forms an integral polytope, see e.g.~\cite[Corollary 13.10b]{schrijver2003combinatorial}.
    Thus, there exists a minimum cost integral circulation. Let $\bar{f}$ be the this integral circulation. We define $\lambda'_i = \bar{f}(s, v_i) / 2^\ell$ for each $i\in\{1,\dotsc,k\}$, then
    \begin{align*}
    \sum_{i \in [k]} \lambda'_i = \sum_{i \in [k]} \frac{ \bar{f}(s,v_i)}{2^\ell} = \frac{1}{2^\ell} \cdot \bar{f}(t, s) = \frac{1}{2^\ell} \cdot 2^\ell  = 1.
    \end{align*}
    Consequently, $\lambda'_i$ creates a valid convex combination for $y'$ where each $\lambda'_i$ is a multiple of~$1/2^\ell$. For each $e \in E$
    \begin{align*}
    |y_e - y'_e| =  \left| \sum_{i \in [k]} \lambda_i (S_i)_e - \sum_{i \in [k]} \lambda'_i (S_i)_e \right| =  \sum_{i \in [k]} |\lambda_i - \lambda'_i| \cdot (S_i)_e \leq \sum_{i \in [k]} \frac{1}{2^\ell} = k \cdot \frac{1}{2^\ell}.
    \end{align*}
    Since the integer circulation minimizes the total cost, we have
    \begin{equation}
        \label{eq:psr-total-rounded-cost}
        \cost{y'} = \sum_{e\in E} c_e \sum_{i \in [k]} \lambda'_i (S_i)_e = \frac{1}{2^\ell} \sum_{e \in E} \bar{f}(e) \cdot \mathrm{cost}(e)
        \le \frac{1}{2^\ell} \sum_{e \in E} f(e) \cdot \mathrm{cost}(e)
        = \cost{x}.
    \end{equation}
    Moreover, constructing $D_f$ and solving minimum cost flow can be done in polynomial time~\cite{schrijver2003combinatorial}.
\end{proof}

The \textsc{TreeMerge} algorithm described in \cref{sec:dependent-randomized-rounding} uses the following lemma to construct new random edge sets while preserving the degree of the former sets.

\EdgeSetDecomposition*

\begin{proof}
    For the randomized construction of $T$ and $T'$, we distinguish whether an edge is present in the symmetric difference~$S_1 \symdif S_2$ or not.
    For any edge~$e \in E$ where $(S_1)_e = (S_2)_e$, we set~$T_e = \big((S_1)_e + (S_2)_e \big)/2$ and~$T_e' = \big((S_1)_e + (S_2)_e \big)/2$.
    Next, partition the edges in~$S_1 \symdif S_2$ into simple paths and cycles with $E_1\dot\cup \cdots \dot\cup E_k = S_1 \symdif S_2$ with the following property: each odd-degree vertex is the endpoint of exactly one path and no path ends in an even-degree vertex.
    This easily follows from iteratively removing cycles and maximal paths.

    Let $i\in\set{1,\dotsc,k}$. Choose $C_i \dot\cup D_i = E_i$ such that no two edges in $C_i$ or in $D_i$ are adjacent. This is possible by alternatingly assigning edges to $C_i$ and $D_i$, as $E_i$ is an even length cycle or a path.
    We make a uniform binary random decision $R_{i} \in \set{0,1}$, which is independent of all $R_{i'}$, $i'\neq i$.
    This random variable indicates whether~$C_i$ is assigned to~$T$ or~$T'$ ($D_i$ is then assigned to the other edge set).
    More precisely, for each~$e \in E_i$ we set 
    \begin{equation}
        \label{eq:mom-random-matching-choice-for-pair}
        T_e = 
        \begin{cases}
            (C_i)_e & \text{if } R_{i} = 0, \\
            (D_i)_e & \text{if } R_{i} = 1,
        \end{cases}
        \qquad \text{and} \qquad 
        T_e' = 
        \begin{cases}
            (D_i)_e & \text{if } R_{i} = 0, \\
            (C_i)_e & \text{if } R_{i} = 1.
        \end{cases}
    \end{equation}
    In particular, $T_e + T_e' = (C_i)_e + (D_i)_e = 1 = (S_1)_e + (S_2)_e$.
    Since both outcomes~$R_i = 1$ and~$R_i = 0$ have probability $1/2$, it holds that $\E(T_e) = \E(T_e) = \big((C_i)_e + (D_i)_e \big)/2 = \big((S_1)_e + (S_2)_e \big)/2$ for all~$e \in S \symdif T$.
    Thus, it follows that $\E(T) = \E(T') = (S_1+S_2)/2$.
    Considering each vertex in a simple path or cycle is incident to at most two edges, we have~$| F \cap E_i| \le 2$ for all~$F \in \F$ and~$i \in \set{1, \dotsc, k}$.
    As the random choice for each path and cycle is independent, it holds that for all~$F \in \F$ and~$e \in F$, there is at most one edge~$e' \in F$ with~$e \neq e'$ such that~$T_e$ depends on~$T_{e'}$ (and the same for~$T'$).
    
    It remains to show that~$T(F), T'(F) \in \set{\floor{\,(S_1(F) + S_2(F))/2\,}, \ceil{\,(S_1(F) + S_2(F))(2)\,}}$.
    To this end, we distinguish whether a vertex $v \in A \cup B$ has even or odd degree. If $|\delta(v)|$ is even, then $v$ is always incident to exactly one edge in $C_i$ and one in $D_i$, for each $i\in\set{1,\dotsc,k}$.
    Hence, $T(F) = T'(F) = |\delta(v)| / 2 = (S_1(F) + S_2(F))/2$.
    If $|\delta(v)|$ is odd, then there is exactly one path that ends in $v$ due to the choice of decomposition into cycles and paths. 
    Consequently, the arguments above apply to all but one edge in $\delta(v)$ and therefore we have $|X(F) - Y(F)| = 1$. 
    This implies the claim.
\end{proof}

The last lemma bounds the change of a linear function between two consecutive levels of~\textsc{TreeMerge}.

\IncreaseOfLinearFunction*

\begin{proof}
    Let $T_i$, $i \in [2^{j}]$, be the random edge set created on the $j$-th level from the edge sets~$S_{2i-1}$,$S_{2i}$, $i\in [2^{j}]$ of the $(j+1)$-th level.
    Recall that the procedure from \cref{lem:decomposition} actually creates
    two alternative edge sets~$T_i,T'_i$ of which \textsc{TreeMerge} selects just one.
    However, the solution derived from $T_i$ satisfies \cref{eq:lfipl-bound-on-lf-increase} if and only if the one from $T'_i$ does. Hence, it suffices to show
    it for one of the two.
    Further, for the sake of convenience, we analyze the scaled expression~$|2^j \cdot a\T y'_j - 2^j \cdot a\T y'_{j+1}|$ instead of~$|a\T y'_j - a\T y'_{j+1}|$.
    Due to \cref{lem:decomposition}, it holds that $\E(T_i) = (S_{2i-1} + S_{2i})/2$.
    Thus,
    \begin{align*}
        \label{eq:ilf-expected-value}
        \E[2^j \cdot a\T y_{j}'] 
        = a\T \Big(\sum_{i \in [2^j]} \E[T_i] \Big)
        = \tfrac{1}{2} a\T \Big(\sum_{i \in [2^{j+1}]} S_{i} \Big)
        = 2^j \cdot a\T y_{j+1}'.
    \end{align*}
    Since \textsc{TreeMerge} independently constructs the~$T_i$ on the $j$-th level, any two random edge sets~$T_i$ and~$T_{i+1}$ are independent.
    Moreover, it follows from \cref{lem:decomposition} that for all~$T_i$ and~$e \in \delta(v)$, there exists at most one edge~$e' \in \delta(v)$ such that $(T_i)_e$ depends on~$(T_i)_{e'}$.
    Consequently, there exists a partition of $\delta(v)$ into~$P_1$ and~$P_2$ such that all variables $(T_i)_e$, $e\in P_1$, as well as~$(T_i)_e$, $e\in P_2$ are independent.
    Let~$u = \sum_{i \in [2^{j}]} U_{i} \in [0,1]^E$ and~$w = \sum_{i \in [2^{j}]} W_{i} \in [0,1]^E$ where~$U_i, W_i \in \{0,1\}^E$ are defined for all~$e \in E$ as
    \begin{align*}
        (U_i)_e =
        \begin{cases}
            1, & \text{if } e\in\delta(v) \text{ and } (T_i)_e \in P_1 \\
            0, & \text{otherwise}
        \end{cases}
        \; \text{and} \quad
        (W_i)_e =
        \begin{cases}
            1, & \text{if } e\in\delta(v) \text{ and } (T_i)_e \in P_2 \\
            0, & \text{otherwise}.
        \end{cases}
    \end{align*}
    Hence, we have~$2^j \cdot a\T y_{j}' = a\T(u + w)$.
    Next, we show that
    \begin{equation}\label{eq:chernoff-one}
        \Pr \left[ | a\T u - \E[a\T u] | > \tfrac{1}{2} t + \tfrac{1}{2} \sqrt{\E[a\T u]  \cdot t} \right] < \tfrac{1}{2 k^{10}}.
    \end{equation}
    For brevity, define $\mu = \E[a\T u]$.
    We distinguish two cases. If~$t > 4\mu$, then set~$\delta = t / (4\mu) > 1$.
    Notice that $a\T u$ is the sum of independent variables, each contained in $[0,1]$.
    We have
    \begin{align*}
        \Pr \left[ a\T u < \mu - t/4 \right] &= 0
    \end{align*}
    It follows from a standard Chernoff bound that
    \begin{align*}
        \Pr \left[ a\T u  > \mu + t/4 \right] 
        &= \Pr \left[ a\T u > (1 + \delta) \mu \right] \\
        &\le \exp(-\tfrac{1}{3} \delta \mu) \\
        &= \exp(- 11 \ln k) \\
        &\le \tfrac{1}{2 k^{10}}.
    \end{align*}
    If~$t \le 4\mu$, then set~$\delta= 1/2 \cdot \sqrt{t/\mu} \in (0,1]$. Again by a Chernoff bound, we have
    \begin{align*}
        \Pr \left[ |a\T u  - \mu| > \tfrac{1}{2}\sqrt{\mu t} \,\right]
        &= \Pr \left[ |a\T u - \mu| > \delta \mu \right] \\
        &\le 2 \exp(- \tfrac{1}{3} \delta^2 \mu) \\
        &= 2 \exp(- 11 \ln k) \le \tfrac{1}{2 k^{10}} .
    \end{align*}
    By symmetry, \cref{eq:chernoff-one} holds for $w$ as well. 
    We conclude that with probability at least~$1 - 1/k^{10}$, we have
    \begin{align*}
        | 2^j a\T y'_j - 2^j a\T y'_{j+1} | 
        &\le | a\T u - \E[a\T u] | + | a\T w - \E[a\T w] | \\ 
        &\le \tfrac{1}{2} t + \tfrac{1}{2} \sqrt{\E[a\T u]  \cdot t} + \tfrac{1}{2} t + \tfrac{1}{2} \sqrt{\E[a\T w] \cdot t} \\
        &\le t + \sqrt{\E[2^j \cdot a\T y'_j] \cdot t} \\
        &\le 2^{j/2} \big(t + \sqrt{a\T y'_{j+1} \cdot t}\big). \qedhere
    \end{align*}
\end{proof}