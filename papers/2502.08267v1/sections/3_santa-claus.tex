\section{Application to Budgeted Santa Claus Problem}
\label{sec:santa-claus}

In this section, we present our approximation algorithm the Budgeted Santa Claus Problem based on the dependent rounding scheme described in \cref{sec:dependent-randomized-rounding}.

\subsection{Linear Programming Formulation}
\label{subsec:lp}
Introducing an LP relaxation for the Budgeted Santa Claus Problem,  we first reduce the problem to its decision variant.
For a given threshold $T \ge 0$, the goal is to either
find a solution of value $T/ \alpha$ or determine that
$\OPT < T$, where $\OPT$ is the optimal value of the original optimization problem. This variant is equivalent to an
$\alpha$-approximation algorithm by a standard
binary search framework. 

Based on $T$, intuitively thought of as the optimal value, we partition the resources into two sets by size. Set $\cB$ consists of the \emph{big resources} with $\cB \coloneq \{ j \in \cR : v_j \geq T/\alpha\}$ and set~$\cS$ consists of the \emph{small resources} $\cS \coloneq \{ j \in \cR : v_j < T/\alpha\}$.
We use assignment variables that indicate whether a particular
resource is assigned to a particular player. For clarity,
we use different symbols for big and small resources.
Let~$x_{ib} \in [0,1]$ denote the portion of big resource $b \in \cB$ that player $i \in  \cP$ receives. Similarly, denote by $z_{is} \in [0,1]$ the portion of the small resource $s \in \cS$ that player $i$ receives.
Unlike the original problem, we allow these
assignments to be fractional in the relaxation.
Since naive constraints on these variables lead to an unbounded integrality gap, see e.g.~\cite{BansalS06}, we use
non-trivial constraints inspired by an LP formulation of Davies, Rothvoss and Zhang~\cite{DaviesRZ20}. Here, we make the structural
assumption that in any solution, a player either receives exactly
one big resource (and nothing else) or only small resources.
Towards the goal of obtaining a solution of value $T/\alpha$,
any big resource is sufficient for any player and receiving more
would be wasteful.
If there is a solution of value $T$, then there is also
a pseudo-solution such that each player either gets exactly one big
resource (and nothing else) or a value of at least $T$ from
small resources only. Note that it might be that the former
type of player only has a value of $T/\alpha$. Thus, if $T \le \OPT$, then the following relaxation called \LP{T}
is feasible and has a value at most $C$.
\begin{align}
    \min \; \sum_{i \in \cP}^{} &\left[\sum_{b \in \cB}^{}c_{ib}\cdot x_{ib} + \sum_{s \in \cS}^{}c_{is}\cdot z_{is} \right]  && \label{eq:economical-objective-function} \\
    \sum_{s \in \cS} v_{s}\cdot z_{is} &\geq T \cdot \left(1 - \sum_{b \in \cB} x_{ib}\right) &&\forall i \in \cP \label{eq:economical-value} \\
    z_{is} &\leq 1 - \sum_{b \in \cB}^{} x_{ib} &&\forall s \in \cS, i \in \cP \label{eq:economical-small-gifts} \\
    \sum_{i \in \cP}^{} x_{ib} &\leq 1 &&\forall b \in \cB \label{eq:economical-ub-player} \\
    \sum_{i \in \cP}^{} z_{is} &\leq 1 &&\forall s \in \cS \label{eq:economical-ub-z} \\
    \sum_{b \in \cB}^{} x_{ib} &\leq 1 &&\forall i \in \cP \label{eq:economical-ub-big-gifts} \\
    z_{is}, x_{ib}  &\geq 0 &&\forall s \in \cS, b \in \cB, i \in \cP \label{eq:economical-nonneg-x}
\end{align}

The constraints~\eqref{eq:economical-ub-player} and \eqref{eq:economical-ub-z} describe that
each big or small resource is only assigned once.
Justified by earlier arguments, constraint~\eqref{eq:economical-ub-big-gifts} ensures that each player receives at most one big resource.
Considering constraint~\eqref{eq:economical-small-gifts}, we only need to verify that the constraint is valid for integer solutions.
By our
assumption, if player $i$ receives one big resource, then it should not get any small resources,
which is exactly what the constraint expresses. Conversely, if the player does not receive any big resources, the
constraint is trivially satisfied. Similarly, there are two cases for
Constraints~\eqref{eq:economical-value}. If
player~$i$ receives one big resource, the constraint is trivially satisfied. Otherwise, it must receive a value of at least $T$ in small resources. During our rounding procedure in \Cref{sec:rounding-of-small-items}, we essentially lose
some value from small resources and can only guarantee
a value of $T/\alpha$ for players without a big resource.

\subsection{Technical Goals}
Let $(x, z)$ be a feasible solution to \LP{T}, where $x$ and $z$ represent the vectors of assignment variables corresponding to big and small resources, respectively. Formally, we have $x_{ib}, z_{is} \in [0,1]$ for $i \in \cP, b\in \cB, s \in \cS$.
We will define a randomized rounding procedure that constructs
a distribution over the binary variables
$X_{ib} \in \{0,1\}$ and $Z_{is}\in\{0, 1\}$ describing whether a big resource $b \in \cB$ or small resource $s\in \cS$ is assigned to player~$i \in \cP$.
For notational convenience, define $Y_i = 1 - \sum_{b\in B} X_{ib}$ as the indicator variable whether a player~$i$ \emph{does not}
get a big resource (and needs small resource). Similarly, let $y_i = 1 - \sum_{b\in \cB} x_{ib}$ be the corresponding value to~$Y_i$ from the corresponding value from the LP variables.

Our goal is that the total cost of assignments does not exceed the budget $C$ and the integral solution $(X,Z)$ is an $\alpha$-approximation solution with respect to the minimum value a player
receives. In other words, we want to obtain an integral solution $(X,Z)$ that satisfies the following two properties.
\smallskip
\begin{enumerate}
    \item $\sum_{i \in \cP}^{} \left[\sum_{b \in \cB}^{}c_{ib}\cdot X_{ib} + \sum_{s \in \cS}^{}c_{is}\cdot Z_{is} \right] \leq C$.
    
    \item Every player receives resources of value at least $T/\alpha$ with high probability.
\end{enumerate}

We first apply our dependent rounding scheme to round
the assignment of big resources to an integral one.
To cover the players that do not receive big resources, i.e.,
those with~$Y_i = 1$, we need to change the assignment of
small resources as well.
Initially, some small resources will be assigned fractionally
and even more than once.
In a second step, we transform the solution into one
where each small resource is assigned only once---incurring a
loss in the value that the players receive.

\subsection{Rounding of Big Resources}
\label{sec:rounding-of-big-items}
The following lemma summarizes the properties we derive
from the dependent rounding scheme. 
Note that while the assignment of small resources can change, it remains fractional for now.

\begin{lemma}
    \label{lem:cost-preservation}
    Let $(x,z)$ be a feasible solution to \LP{T, C}. There is a randomized algorithm that produces an assignment $X_{ib}\in\{0,1\}$, $i\in \cP$, $b\in\cB$, and $z'_{is}\in [0,1]$, $i\in \cP$, $s\in\cS$ such that
    with high probability
    \begin{enumerate}
        \item $\sum_{i\in\cP}\sum_{b\in\cB} c_{ib}\cdot X_{ib} +  \sum_{i\in\cP }\sum_{s\in\cS} c_{is} z'_{is} \leq C$, \label{eq:costs}
        \item Each big resource is assigned at most once, i.e., $\forall b\in \cB :\sum_{i\in\cP} X_{ib} \le 1$,
        \item Each small resource is assigned at most $O(\log n)$ times, i.e., $\forall s\in \cS: \sum_{i\in\cP} z'_{is} \le O(\log n)$,
        \item Each player receives either one big resource or a value of at least $T$ in small resources.
    \end{enumerate}
\end{lemma}
\begin{proof}
We first describe the instance to which we will apply the dependent rounding procedure from \cref{thm:assign}.
We build a bipartite graph $\Gx{x,z} = (\cP\cup (\cB \cup \{d\}), E)$, where $\cP$ is the set of players, $\cB$ is the set of big resources, and $d$ is a \textit{dummy node}. 
The role of~$d$ can be summarized as: every player who does not get a big resource selects
the edge to~$d$. 
Let graph~$\Gx{x,z}$ contain an edge $(i,b) \in E$ labeled with cost~$c_{ib}$ between each big resource~$b$ and player~$i$ with $x_{ib} > 0$. 
Let $J$ denote the set of all these edges. 
For every player~$i$ with~$y_i > 0$, add an edge~$(i, d)$ to the dummy node of cost~$\sum_{s \in \cS} c_{is} \cdot (z_{is}/y_i) \in \R^n$.
Let $K$ denote the set of these edges and
set $E = J\cup K$.
Intuitively, $x_{ib}$ is a fractional edge selection
of edges in~$J$ and~$y_i$ a fractional edge selection of $K$ (where $y_i$ corresponds to edge $(i, d)\in K$).
This fractional edge set satisfies the following.
\smallskip
\begin{enumerate}
    \item[(i)] For each player $i$, we have $\sum_{b \in \cB} x_{ib} + y_i = 1$ due to the definition of $y_i$.
    \item[(ii)] For each big resource $b$, we have $\sum_{i \in \cP} x_{ib} \leq 1$, due to Constraint~\eqref{eq:economical-ub-player}.
\end{enumerate}

Applying the dependent rounding procedure of \Cref{thm:assign}
with $k = |\cS| = n$ linear functions (we defer a precise definition to the end of the proof),
we obtain an integral edge selection~$X_{ib}\in\{0,1\}$, $i\in \cP$, $b\in \cB$ and $Y_i\in\{0,1\}$, $i\in \cP$.
Here, similar to before, $Y_i$ defines the edge selection in~$K$,
i.e.,~edge $(i, d)$ is selected if $Y_i = 1$.
From~(ii) and \Cref{thm:assign}, it follows that
$\sum_{i\in \cP} X_{ib} \le 1$ for all $b\in \cB$,
which means that Property~2 of the lemma holds.
Further,
\begin{align*}
    \sum_{i\in\cP}\sum_{b\in\cB} c_{ib}\cdot X_{ib} +  \sum_{i\in\cP : y_i > 0} Y_i \cdot (\sum_{s\in\cS} c_{is}\cdot \frac{z_{is}}{y_i}) 
    &\le \sum_{i\in\cP}\sum_{b\in\cB} c_{ib}\cdot x_{ib} +  \sum_{i\in\cP : y_i > 0}(\sum_{s\in\cS} c_{is}\cdot \frac{z_{is}}{y_i}) \cdot y_i\\
    &\le \sum_{i\in\cP}\sum_{b\in\cB} c_{ib}\cdot x_{ib} +  \sum_{i\in\cP}\sum_{s\in\cS} c_{is}\cdot z_{is} \\
    &\leq C,
\end{align*}
where the first inequality follows from the cost preservation
property of \Cref{thm:assign} and the last inequality from the feasibility of~$(x,z)$. 
Notably, if $y_i = 0$, then $Y_i = 0$ as
there does not exist an edge $(i, d)$.
In particular, $Y_i = 1$ implies that $y_i > 0$.
We define the assignment of small resources as 
\begin{equation*}
z'_{is} = \begin{cases}
     z_{is}/y_i &\text{ if } Y_i = 1,\\
     0 &\text{ otherwise.}
\end{cases}
\end{equation*}
Due to Constraint~\eqref{eq:economical-small-gifts}, we have $z_{is} \le y_i$ and hence $z'_{is} \in [0, 1]$. Consequently, Property~1
immediately follows from the previous cost calculation and the definition of $z'_{is}$.
From~(i) and \Cref{thm:assign}, we obtain $\sum_{b\in \cB} X_{ib} + Y_i = 1$ for each $i\in \cP$.
In words, player~$i$ either gets a big resource or $Y_i = 1$.
In the latter case holds that
\begin{equation*}
    \sum_{s\in \cS} v_s z'_{is} = 
    \sum_{s\in \cS} v_s z_{is} / y_i \ge T .
\end{equation*}
Here, the inequality follows from the definition of $y_i$ 
and Constraint~\eqref{eq:economical-value}.
Therefore, Property~4 holds.

It remains to show Property~3, i.e, that every small resource
is assigned at most $O(\log n)$ times.
To this end, we define the linear functions provided to \Cref{thm:assign}: 
for each small resource $s\in \cS$,
there is one linear function~$a^{(s)}\in [0,1]^K$,
i.e., specified by the edges incident to~$d$.
Implicitly, the coefficient for all other edges is zero.
Thus,
the linear function satisfies the support restriction of 
\Cref{thm:assign}.
For $e = (i, d) \in K$, we define
$a^{(s)}_e = z_{is} / y_i$.
Using Constraint~\eqref{eq:economical-small-gifts}, it holds that
\begin{equation*}
\sum_{e = (i, d) \in K} y_i \cdot a^{(s)}_e = 
\sum_{e = (i, d) \in K} y_i \cdot \frac{z_{is}}{y_i} = 
\sum_{e = (i, d) \in K} z_{is} \le 1 . 
\end{equation*}
Finally, from the concentration bound of \Cref{thm:assign} follows
\begin{equation*}
\sum_{i\in \cP} z'_{is}
= \sum_{e = (i, d) \in K} Y_i \cdot z'_{is}
= \sum_{e = (i, d) \in K} Y_i \cdot z_{is} / y_i
= \sum_{e = (i, d) \in K} Y_i \cdot a^{(s)}_e \le O(\log n) . \qedhere
\end{equation*}
\end{proof}

\subsection{Rounding of Small Resources}
\label{sec:rounding-of-small-items}
In the previous subsection, we described how big resources $b \in \cB$ were integrally assigned to the players.
As some players did not receive any big resources and still need to be covered by small resources, let
$\cQ$ be the set of those players, i.e., players~$i\in\cP$ 
for which $Y_i = 1$. The linear program in the following lemma
corresponds to the property of the assignment variables
for small resources from the previous section.

\begin{restatable}{lemma}{RoundingSmallItems}
    \label{lem:rounding-small-items}
    Let $\cQ \subseteq \cP$ and consider the LP $\mathrm{small}(T, \beta)$ defined as
    \begin{align*}
        \sum_{s\in \cS} v_s \cdot z'_{is} &\ge T &\forall i\in\cP \\
        \sum_{i\in \cQ} z'_{is} &\le \beta &\forall s\in\cS\\
        z'_{is} &\ge 0 &\forall i\in \cQ, s\in \cS
    \end{align*}
    If $\mathrm{small}(T, \beta)$ has a fractional solution $z'$,
    then $\mathrm{small}(T/\beta - \max_{s\in\cS} v_s, 1)$ has an integral solution $Z$ that can be found in polynomial time with
    \begin{equation*}
       \sum_{s\in\cS}\sum_{i\in\cQ} c_{is}\cdot Z_{is}     
       \le \sum_{s\in\cS}\sum_{i\in\cQ} c_{is}\cdot z'_{is},
    \end{equation*}
\end{restatable}
% Using a classical approach by Shmoys and Tardos~\cite{ShmoysT93}, the proof of the lemma is deferred to \cref{subsec:appendix-rounding-of-small-resources}.
\begin{proof}
    Since $\mathrm{small}(T, \beta)$ is feasible, there exists a solution $z'$.
    Further, we obtain a (fractional) solution $z''$ to $\mathrm{small}(T/\beta, 1)$ 
    by dividing all variables of $z'$ by $\beta$.
    Using the approach by Shmoys and Tardos~\cite{ShmoysT93},
    we round $z''$ to an integral solution.
    For the sake of completeness, we provide the proof below.

    Assume without loss of generality that $\cS = \{1,2,\dotsc,|\cS|\}$ is ordered such that $v_s \ge v_{s-1}$ for all $s=2,3,\dotsc,|\cS|$.
    We construct an auxiliary bipartite graph $G$ (see \cref{fig:smallRounding} for an illustration).
    \begin{figure}
        \centering
        \includegraphics[scale=0.7]{Img/SmallRounding.eps}
        \caption{Bipartite graph G where on left side there are $k(i)$ copies of each player $i$ for an instance of two players with edges $(u_{i,\ell}, s)$ of weight $c_{is}$ for each $s = s(i,\ell-1),s(i,\ell-1)+1,\dotsc,s(i,\ell)$.} \label{fig:smallRounding}
    \end{figure}
    The elements
    on the right side of~$G$ are elements of
    $\cS$.
    On the left side, there are~$k(i) := \lfloor \sum_{s\in\cS} z''_{is} \rfloor$
    many vertices $u_{i, 1},\dotsc,u_{i,k(i)}$ for each
    $i\in\cQ$.
    In the fractional solution~$z''$, a player $Q$
    can get several small resources.
    Let~$k(i)$ denote their (rounded) number.
    Introducing~$u_{i,1},\dotsc,u_{i,k(i)}$ as
    copies of player $i\in Q$ allows us to argue about matchings
    (where every vertex is only involved in one assignment).
    Suppose we add one edge between every two vertices
    of the different sides of the bipartite graph.
    It is straight-forward that there exists a (fractional)
    left-perfect matching by distributing the resources
    assigned in $z''$ among
    the copies of each player.
    Due to the rounding in $k(i)$, this matching gives a slightly lower value to each player, but stays within the bounds we are aiming for.
    
    To round to a good integral matching, however, we require a specific definition of the
    fractional left-perfect matching.
    Essentially, we need a monotone assignment where
    the first copy $u_{i,1}$ of player $i\in Q$ has the highest value resources (the first in the order above) and the last player the lowest value resources.
    Then~$G$ only contains the edges that 
    are actually used in this assignment.

    This requires some careful definitions.
    For each $i\in \cQ$, we set
    $s(i,0) = 1$. This describes the first resource that
    can be fractionally assigned to player $i$.
    For $\ell = 1,2,\dotsc,k(i)$,
    we choose $s(i,\ell)$ as the element in $\cS$ that satisfies
    \begin{equation*}
        z''_{i,1} + \cdots + z''_{i,s(i,\ell)-1} < \ell
        \text{ and }
        z''_{i,1} + \cdots + z''_{i,s(i,\ell)} \ge \ell .
    \end{equation*}
    Note that $s(i,\ell)$ exists, because the sum of
    all $z''_{i,j}$ is at least $k(i) \ge \ell$.
    We only assign resources $s(i,\ell-1),\dotsc,s(i,\ell)$ to copy~$u_{i,\ell}$.
    Intuitively, the resources $1,2,\dotsc,s(i,\ell-1)-1$ should be exclusively used for the copies $u_{i,1},\dotsc,u_{i,\ell-1}$. Simply because the sum of fractions associated with $i$ (according to~$z''$) is not enough to cover the copies and $u_{i,\ell}$ should not receive any of them in order to maintain the monotonicity goals.
    Similarly, as the sum of fractions
    of resources $1,2,\dotsc,s(i,\ell)$ belonging to $i$
    exceeds $\ell$, they are enough to cover all players in
    $u_{i,1},\dotsc,u_{i,\ell}$.
    Thus, it is not necessary to give any less valuable resources to 
    $u_{i,\ell}$.

    Consequently, for all $i\in\cQ$ and $\ell=1,2,\dotsc,k(i)$, we
    introduce an edge $(u_{i,\ell}, s)$ of weight~$c_{is}$ for
    each $s = s(i,\ell-1),s(i,\ell-1)+1,\dotsc,s(i,\ell)$.
    We now formally show that there is a left-perfect
    fractional matching
    of weight at most $\sum_{s\in\cS}\sum_{i\in\cQ}c_{is} \cdot z''_{is}$.
    Towards this,
    consider some $i\in \cQ$ and $\ell \in \{1,2,\dotsc,k(i)\}$.
    We select edge $(u_{i,\ell}, s(i, \ell-1))$ to an
    extend of
    \begin{equation*}
         z''_{i,1} + \cdots + z''_{i,s(i, \ell-1)} - (\ell - 1) \in [0, z''_{i,s(i, \ell-1)}].
    \end{equation*}
    For $s = s(i, \ell-1)+1,\dotsc,s(i, \ell)-1$, we pick edge $(u_{i,\ell}, s)$ to an extend of $z''_{i,s}$.
    Finally, we choose edge $(u_{i,\ell}, s(i, \ell))$ to an
    extend of
    \begin{equation*}
        \ell - (z''_{i,1} + \cdots + z''_{i,s(i, \ell)-1}) \in [0, z''_{i,s(i, \ell)}] .
    \end{equation*}
    This fractional selection of edges satisfies the following properties.
    \smallskip
    \begin{enumerate}
        \item For each $i\in\cQ$ and $\ell\in\{1,2,\dotsc,k(i)\}$, the total fractional amount of selected 
        edges that are incident to~$u_{i,\ell}$ 
        is exactly $1$.
        \item For each $i\in\cQ$ and $s\in\cS$,
        the total fractional amount of selected edges that are between~$u_{i,1},\dotsc,u_{i,k(i)}$ and $s$
        is at most $z''_{i,s}$.
    \end{enumerate}
    Property~2 implies that the total fractional amount that edges incident to $s$ (over all $i\in\cQ$) are selected is at most $1$.
    Hence, the total weight is
    at most the cost of $z''$.
    For a bipartite graph, the set of all fractional
    matchings is precisely the convex hull of integral
    matchings, see e.g.~\cite[Chapter~18]{schrijver2003combinatorial}. Furthermore, $z''$ must lie on a facet spanned by only left-perfect
    matchings.
    Thus, there must also
    exist an integral left-perfect matching $M$ of weight at most
    \begin{equation*}
        \sum_{i\in\cQ} \sum_{s\in\cS} z''_{i, s} c_{i, s}
        \le \sum_{i\in\cQ} \sum_{s\in\cS} z'_{i, s} c_{i, s} .
    \end{equation*}
    We can find such a matching using standard algorithms
    for minimum weight bipartite matching.
    We interpret $M$ as an integral assignment $Z$
    where each $i\in \cQ$ receives all $s\in \cS$,
    for which there is an edge $(u_{i,\ell}, s)$ in $M$
    for some $\ell\in\{1,2,\dotsc,k(i)\}$.
    Finally, we analyze how much value each $i\in\cQ$
    receives in this assignment.
    Notice that all resources $s\in\cS$
    that are connected to $u_{i,\ell}$ have a value
    of at least $v_{s(i,\ell)}$.
    Therefore, $i$ receives a total value of at least
    \begin{equation*}
        v_{s(i,1)} + \cdots + v_{s(i,k(i))} .
    \end{equation*}
    On the other hand, 
    since $z''_{i,1} + \cdots + z''_{i,s(i, \ell ) - 1} + z''_{i,s(i, \ell)} < \ell + z''_{i,s(i, \ell)} \le \ell + 1$ for each $\ell$,
    the total fractional amount of resources that $i$ receives in $z''$
    with value at least $v_{s(i, \ell)}$ is less than $\ell + 1$.
    As a result, the total value that $i$ received in $z''$
    is at most
    \begin{equation*}
        v_{s(i,0)} + \cdots + v_{s(i,k(i))} .
    \end{equation*}
    This is at least $T/\beta$, because of $z''\in\mathrm{small}(T/\beta,1)$.
    As a consequence, in $Z$ player $i$ receives a value of at least
    $T/\beta - v_{s(i,0)} \ge T/\beta - \max_{s\in\cS} v_s$.
\end{proof}


\subsection{Approximation Factor}
Concluding the previous subsections, this section provides an $\alpha$-approximation for the Budgeted Santa Claus Problem, where $\alpha = \Order(\log n)$. 
\begin{theorem}
    The Budgeted Santa Claus Problem can be approximated within a factor of~$O(\log n)$ in randomized polynomial time.
\end{theorem}
\begin{proof}
   Let $\beta = O(\log n)$ be the value from Property~3 of \Cref{lem:cost-preservation}. We define $\alpha = 2\beta$
   and run our binary search over value $T$,
   which is then used in our definition of big and small resources.
   Then we solve the LP relaxation \LP{T}. 
   Let $(x,z)$ be the resulting solution. 
   If the cost of the solution is more than $C$, we fail (and
   increase the value of $T$). Assume now that $(x,z)$ has
   a cost of at most $C$.
   In \Cref{lem:cost-preservation}, using the dependent rounding procedure from \Cref{thm:assign}, we show that the fractional assignments $x$ of big resources to players can be rounded to an integral assignment $X$ and the assignment of small resources $z$ can be changed to $z'$ (which is still fractional), such that with high probability each small resource up to
   $\beta$ times and the cost is still below $C$. 
   \Cref{lem:rounding-small-items} proves that $z'$
   can be rounded to an integral assignment $Z$ such that
   the cost does not increase and each player $i$ that does
   not receive a big resource gets small resources of total value at least
   \begin{align*}
    \frac{T}{\beta} - \max_{s\in\cS} v_s &\geq \frac{2T}{\alpha} - \frac{T}{\alpha} = \frac{T}{\alpha}. \qedhere
    \end{align*}
\end{proof}
