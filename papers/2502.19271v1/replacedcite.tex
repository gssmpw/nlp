\section{Related Works}
\subsection{Multi-criteria Recommender Systems}

    Adopting MCRS is crucial for platforms seeking to enhance user satisfaction, engagement, and retention. Unlike traditional Recommender Systems  that rely on single criteria like ratings or item popularity, MCRS consider diverse factors such as user demographics, item attributes, temporal dynamics, and contextual information ____ allowing them  to deliver more personalised and contextually relevant recommendations, catering to the varied and evolving preferences of users. The rating data in a MCRS can be represented as a three-dimensional (3D) tensor, allowing factorisation of the tensor and higher order decomposition ____. The user similarity in recommender systems has advanced from single-criterion to multi-criteria approaches with the aid of fuzzy methods ____ which seeks to improve predictive accuracy and tackle the challenge of personalised recommendations amidst information overload. Multi-criteria collaborative filtering systems ____, such as those incorporating Adaptive Neural Fuzzy Inference Systems (ANFIS) ____ and Self-Organising Maps (SOMs) ____, aim to provide accurate recommendations by integrating user preferences across multiple dimensions. ANFIS combines fuzzy logic with neural networks to adaptively learn from user interactions and refine recommendations based on evolving preferences and contextual cues. ANFIS is primarily used for prediction tasks rather than classification. It combines the adaptive learning capabilities of neural networks with the interpretability of fuzzy logic to model complex relationships and make predictions based on input data ____. SOMs, on the other hand, leverage unsupervised learning to cluster items based on their similarity and user interactions, facilitating personalized recommendations within each cluster ____. Together, these methods enable MCRS to effectively handle the complexity of multidimensional data and enhance recommendation accuracy.
    
    MCRS often employ two-stage methods to enhance recommendation accuracy by estimating target item ratings and learning sub-score weights ____. These methods are designed to handle the complexity of multiple criteria involved in decision-making processes, ensuring that recommendations are tailored to diverse user preferences and requirements. Subsequently, deep learning models have been pivotal in advancing MCRS. Models like sparse autoencoders, which are tailored to balance criteria ____, and deep neural networks applied to matrix factorization ____  have proven highly effective in enhancing recommendation accuracy. These approaches leverage the capacity of deep learning to handle complex interactions among multiple criteria, thereby improving the precision and relevance of recommendations. These methods leverage contextual information to improve relevance ____ and consider multiple stakeholder preferences to boost overall performance ____. For instance, Shambour et al. ____ introduced a deep learning algorithm for multi-criteria recommenders, employing deep autoencoders to uncover intricate user preferences. Nassar et al. ____ integrated multi-criteria recommendation with deep learning, extracting features and utilising neural networks for correlation learning. These advancements underscore the potential of deep learning approaches in enhancing the capabilities of multi-criteria recommendation systems. Hong et al. ____ introduced two single tensor models that incorporate users (or countries), items, multi-criteria ratings, and cultural groups to simultaneously account for the inherent structure and interrelationships of these elements in recommendation systems. Zhang et al. ____ developed methods to improve the precision and scalability of multi-criteria recommendation systems by utilising both social connections and criteria preferences. They present a hybrid social recommendation algorithm and broaden its effectiveness with an implicit technique for inferring social relationships. Rismala et al. ____ presented a collaborative filtering approach based on a personalised neural network that takes multiple criteria to extract individual details from the user's rating history and to represent it as rating tendencies and user experiences. Singh et al. ____, examined the performance of traditional collaborative filtering, matrix factorisation, and deep matrix factorisation techniques in recommender systems using multi-criteria datasets.
    
\subsection{Recommender Systems based on GNNs}

     GNNs have gained significant attention and are increasingly being applied in recommendation systems ____. GNNs offer a powerful framework to model interaction between user items and capture complex relationships within the data using the inherent graph structures present in the recommendation scenarios,  ____. These models have demonstrated effectiveness in various recommender systems, including item recommendation, user preference modelling, and explanation of recommendations ____. Wang et al. ____ employed GNNs to propagate embeddings in the user-item bipartite graph, explicitly capturing higher-order connections and injecting collaborative signals into the recommender system. GNNs for social recommender systems ____ considered the varying strengths of social relationships among users, modelling graph information to capture interactions and opinions within the user item graph. He et al. ____ used a Multilayer feedback neural network for user-item assessment modelling, while their study ____ proposed a CF and convolutional neural network approach that employed dyadic product for more meaningful connections. Zheng et al. ____ integrated user comments and items using a convolutional neural network, and Liu et al. ____ utilised a multilayer perceptron (MLP) in a note-based coder. Berkani et al. ____ proposed a hybrid recommendation approach that integrates collaborative filtering (CF) and content-based filtering (CBF) within an architecture featuring two models: generalised matrix factorization (GMF) and hybrid multilayer perceptron (HybMLP). The primary aim of their model is to mitigate challenges encountered during cold start situations in recommendation systems.
    
\subsection{Recommender Systems based on Multiview learning}

    Multiview learning in recommender systems integrates diverse data sources to enhance recommendation accuracy and robustness ____. Unlike traditional systems that rely on single sources like user-item interactions or item attributes,  Multiview learning incorporates additional perspectives such as social network data, textual descriptions, temporal dynamics, and user demographics  ____.By combining these varied sources,  Multiview learning mitigates data sparsity, improves prediction accuracy, and delivers personalized recommendations tailored to individual preferences and contexts ____. This approach also captures intricate relationships across different data modalities, thereby enhancing the overall quality and relevance of recommendations  ____. Barkan et al. ____ introduced CB2CF, a deep neural multiview model that connects item content to their collaborative filtering  representations. Designed for Microsoft Store services, which serve approximately a billion users globally, CB2CF is a real-world algorithm. The model is applied to movie and app recommendations, demonstrating superior performance over an alternative content-based model, particularly for completely new items. In paper ____ introduced a multiview graph collaborative filtering network for recommendations by leveraging both homogeneous and heterogeneous signals from attribute and neighbor views. The MVGCF model combines the co-occurrence features of different attribute values with the collaborative preferences of various neighbors to learn the embedding representation of nodes. The research ____, introduced a deep graph collaborative signal aggregation module designed to learn latent intention similarity representations for effective collaborative signal propagation within a few layers. Additionally, they propose a novel multiview contrastive learning module, which leverages both local and global contrastive learning views derived from the collaborative signal aggregation module.  Zhou et al. ____ proposed a multiview social recommendation framework called MsRec, which utilizes information from various perspectives for item recommendations. Specifically, MsRec aims to explore complex inner relationships within social networks and conduct user-level preference learning uniquely for each user. Additionally, MsRec incorporates available side information, such as contextual data, demographic characteristics, and item attributes, to enhance the representation vector learning for items. Yuan et al. ____ proposed a model called Attribute Mining Multiview Contrastive Learning Network. It enhances the initially sparse embedding representation by extracting potential information from the native data and constructing four distinct views. The model performs cross-view contrastive learning at both local and global levels, combining the collaborative information from each view with the global structural information in a self-supervised manner, thereby eliminating reliance on supervised signals.
    
\subsection{Recommender Systems based on GAT}

    Attention mechanisms have become increasingly important in recommender systems due to their ability to capture feature importance and enhance interpretability. Utilising attention allows models to focus on the most relevant parts of the input data, which is crucial for providing accurate and personalized recommendations ____. The application of attention mechanisms in recommender systems offers significant benefits, including improved performance in capturing user preferences, better handling of sparse data, and enhanced interpretability of the model's decisions ____. 
    
    Wang et al ____ introduced the Multiview Graph Attention Network   for session-based music recommendation. MEGAN employs graph neural networks and attention mechanisms to generate meaningful representations (embeddings) of music tracks and users using heterogeneous data sources. This approach enables MEGAN to capture users' mixed preferences from their listening behaviors and recommend music pieces that align with users' real-time needs. Hu et al. ____ proposed a Collaborative Recommendation Model based on Multi-modal multiview Attention Network to represent users using both preference and dislike perspectives. Specifically, users' past interactions are categorized into positive and negative interactions, which inform the preference and dislike views, respectively. Additionally, semantic and structural details extracted from the context are incorporated to enhance the representation of items.
    
    He et al. ____ categorised fundamental feature interactions into sum-interactions and product-interactions, advancing existing methods for explicit feature interactions. Building on these theoretical insights, they introduce the Attentional Aggregative Interaction Network (AAIN), a novel model that incorporates cyclic explicit modules to capture higher-order features. Their approach employs attention mechanisms to reorganise individual features, followed by product-interactions and compression of higher-order features for output. Several advancements in modelling techniques incorporating attention mechanisms have been proposed in recent literature. For example, Xiao et al. ____ introduced attentional mechanisms, Zhou et al. ____ explored attentive activation mechanisms, and Song et al. ____ innovated with self-attentive neural networks. These studies represent diverse approaches aimed at enhancing model performance through attention-based methodologies. Sequential recommendation techniques include hybrid encoders from Li et al. ____ and dynamic interest capture by Zhou et al. ____. Wang et al. ____ introduced HGATE for unsupervised representation learning on heterogeneous graph-structured data. Chen et al ____,  the MV-GAN (Multiview Graph Attention Network) model tailored for travel recommendation that enhanced semantic understanding of users and travel products by aggregating neighbours guided by meta-paths and fusing multiple views within a heterogeneous recommendation graph. The node-level and path-level attention networks were  designed to capture user and product representations from each view independently. To integrate diverse relationship types across views, they propose a view-level attention mechanism that aggregates node representations to derive comprehensive global representations of users and products. Lin et al. ____ introduced a Mixed Attention Network that integrates local and global attention modules for extracting domain-specific and cross-domain information. They first proposed a local/global encoding layer to capture sequential patterns specific to each domain and across domains. Finally, they proposed a local/global prediction layer to further refine and integrate domain-specific and cross-domain interests. Wang et al. ____ introduced a sequence recommendation model to strike a balance between users' long-term and short-term benefits, ultimately combining them into a hybrid representation for recommendation purposes.
    
    Wang et al.____ introduced to enhance recommender system for  image active recommendation that harnesssed multi-modal information and high-order collaborative signals to aid in representation learning, while also capturing personalized user preferences in images.  Li et al.  ____ introduced  multiview social recommendation for item recommendation from various perspectives  to take advantage of the complex relationships within social networks. It performs user-level preference learning for each user independently, without overlap. Chen et al. ____ constructed an authentic global graph derived from a multiview representation of items and sessions based on a knowledge graph to extract global item-item relationships within the knowledge view for session-based recommendation.
    
\subsection{Recommender Systems based on Contrastive Learning}

    Contrastive learning ____ is a powerful paradigm in machine learning that aims to learn representations by contrasting positive pairs (similar samples) and negative pairs (dissimilar samples) in a latent space. This approach leverages the idea that similar samples should be closer together while dissimilar ones should be farther apart, thereby facilitating the discovery of meaningful patterns and representations from data. The use of contrastive learning in recommender systems has increased, with various versions being implemented. By leveraging the strengths of contrastive learning, these systems aim to learn more robust and discriminative representations of user-item interactions ____.  Contrastive learning helps in distinguishing between positive and negative samples more effectively, thereby enhancing the quality of recommendations ____.
    
    Wei et al. ____  reframed the learning of cold-start item representations from an information-theoretic perspective, aiming to maximize the mutual dependencies between item content and collaborative signals. They introduce a new objective function based on contrastive learning and develop a straightforward yet effective framework for cold-start recommendations. Chen et al. ____ introduced Intent Contrastive Learning to  integrate a latent intent variable into sequential recommendation that involves learning distribution functions of users' intents from unlabeled sequences of user behaviour through contrastive self-supervised learning.
    
    Yang et al. ____ introduced Knowledge-Adaptive Contrastive Learning that involves  data augmentation independently from the user-item interaction view and the knowledge graph  view, and applying contrastive learning across these two perspectives. The algorithm ensures that item representations encode information that is common across both views by incorporating a contrastive loss. Zhang et al. ____ developed a dual contrastive learning recommendation framework. Zhang et al. ____ introduced a dual contrastive learning recommendation framework. The first contrastive learning step promotes uniform distributions across users and items, while the second step aims to generate contrastive embeddings from output vectors. Qin et al. ____ introduced Intent Contrastive Learning  for Sequential Recommendation designed to capture users' latent intentions by segmenting a user's sequential behaviour into multiple subsequences using a dynamic sliding operation. These subsequences are then processed through an encoder to generate representations that capture the user's intentions.
    
    In this paper ____, proposed a learning paradigm called supervised contrastive learning based on graph convolutional neural networks. Initially, during data preprocessing, they calculate the similarity between different nodes on both the user side and the item side. When applying contrastive learning, they consider not only the augmented samples as positive samples but also a certain number of augmented samples of similar nodes as positive samples. Wei et al. ____ proposed a recommendation framework called Multi-level Cross-modal Contrastive Learning. This framework aims to construct multi-level contrastive learning to fully exploit both intra-modal and inter-modal semantic information in a self-supervised manner. They consider user interaction and semantic review as two distinct semantic modalities and devise two modal-specific contrastive learning strategies to enhance intra-modal learning. 
    
    The comparison of various methodologies using single and multi-criteria recommender systems is presented in \ref{app:comparison}.