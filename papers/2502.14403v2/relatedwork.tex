\section{Related Work}
\subsection{Single-Domain Fake News Detection}
Single-domain approaches for news content-based fake news detection were initially studied. They typically extract informative features from the content of news articles to predict their veracity. For instance, studies in~\cite{perez2017automatic, volkova2017separating} analyze word usage within news articles, while studies in~\cite{shu2019defend, ma2016detecting} extract sentence-level and contextual features to detect fake news, leveraging methods such as support vector machine (SVM)~\cite{perez2017automatic}, long short-term memory (LSTM) network and convolutional neural network (CNN)~\cite{volkova2017separating}, recurrent neural network (RNN)~\cite{ma2016detecting}, and co-attention mechanism~\cite{shu2019defend}. In addition, with the development of language models, language models (LMs) such as BERT~\cite{devlin2018bert} and GPT-2, as well as large language models (LLMs) like GPT-3.5 and GPT-4, have been applied to fake news detection. Consequently, some approaches, such as studies in~\cite{hu2024bad}, have combined the strengths of both LMs and LLMs for fake news detection. 

Single-domain approaches for user engagement-based fake news detection extract additional information from user engagement to enhance the effectiveness of fake news detection. They explore features in user engagements within a single domain. For example, Nguyen et al.~\cite{nguyen2020fang} proposed a model to better learn graph representations from social context information such as user engagement for fake news detection. Khoo et al.~\cite{khoo2020interpretable} proposed a self-attention based model to extract more informative representation from both news content and user engagement. Although various approaches have been proposed, these single-domain approaches often exhibit poor fake news detection performance when applied in a news domain that is new for the approach, highlighting the issue of domain shift.

\subsection{Cross-Domain Fake News Detection}
To solve the issue of domain shift, cross-domain fake news detection approaches have been developed. Some of them focus on using different techniques to generate domain-shared features based on news content, such as machine learning methods~\cite{castelo2019topic}, neural networks~\cite{zhu2022memory}, contrastive learning~\cite{ran2023unsupervised, yue2022contrastive}, adversarial learning~\cite{wang2018eann, silva2021embracing}, as well as meta learning~\cite{yue2023metaadapt}. Cross-domain user engagement-based fake news detection approaches have also been proposed. For instance, Mosallanezhad et al.~\cite{mosallanezhad2022domain} proposed a reinforcement learning-based model to extract domain-shared features from the combination of representations of news content and user engagements. Yang et al.~\cite{yang2024update} proposed to extract domain-shared features from common users' engagements with news items to achieve detection performance improvement in both the data-sparser and the data-richer domains. However, the above mentioned cross-domain approaches fail to consider the impact of veracity-irrelevant features in news content on fake news detection and cross-domain knowledge transfer. In addition, most of them transfer knowledge only based on news content or user engagement. Though Mosallanezhad et al. utilized both of  news content and user engagement, they simply combines them without exploring engagement features of common users across domains for more effective knowledge transfer.