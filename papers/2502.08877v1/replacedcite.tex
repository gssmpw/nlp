\section{Related Work}
\label{sec:relwork}
In this section, we review prior work that studies the effects of existing incentives, simulates responses to incentive plans, and proposes new incentives toward general residential decarbonization. 

\noindent\textbf{Effects of existing incentives. \ }
There have been numerous studies that investigate the real-world effects of decarbonization incentive programs such as tax credits, rebates, grants, net metering (i.e., feed-in tariffs) and renewable energy credit markets. Perhaps the most well-studied is residential solar adoption____.  ____ review state and utility incentives for residential PV in the United States, finding that point of sale rebates are up to $8\times$ more effective compared to tax credits worth the same amount.  Other studies come to similar conclusions, including____.  ____ and ____ consider the distributional impacts of solar in terms of adoption and financial returns, respectively, finding racial and income disparities.

Other studies have considered the effects of incentives on heat pump and battery storage adoption____. A Norway-based case study finds that 54.2\% of participants in a subsidy program were ``very satisfied'' with heat pumps____. 

Interestingly, ____ suggests that heat pump adoption is not well-correlated with income, while ____ find disparities along racial and income lines in battery storage adoption throughout California. 

Although they do not propose new incentive structures, these studies help contextualize the landscape of incentives for residential decarbonization  and influence our high-level approach.



\noindent\textbf{Incentive simulations and proposals. \ }
Prior work also uses simulations to estimate the adoption of new residential technologies based on existing or proposed incentives.  Many find that upfront subsidies encourage more adoption, including____.  Others simulate the financial viability and effects of climates on heat pumps or battery storage____.  Some works raise questions about the payback period of BESS systems in low-solar areas without additional incentives, including____.  Others simulate the potential of heat pumps, finding that both the economics and the decarbonization potential depend heavily on e.g., electricity generation mix, pricing, and taxation____.  


Other work uses a combination of simulation data, economic analysis, and empirical results to propose new incentive structures for residential decarbonization technologies, including____.  Most closely related to our work, ____ develops an optimization-based model to design incentives for solar PV, heat pump, and BESS combination systems, although they do not explicitly consider carbon reduction, instead focusing on the adoption rate.  ____ suggest that energy efficiency subsidies should be allocated into areas with lower housing prices, since low-income areas pay relatively more for energy. 
In contrast to  the above studies,  our focus is  on carbon reduction as opposed to adoption, which adds a dimension to our analysis.




\noindent\textbf{Mechanism design and learning. \ }
Our study draws on foundational work in mechanism design and multi-armed bandits.  In the field of mechanism design, our work is most related to dynamic mechanism and incentive design, where the underlying agents (e.g. households) update their action according to some unknown but learnable rules____.  
In the broad literature on multi-armed bandits, we mostly draw on the problem of offline contextual bandits, where learners estimate the quality of actions by leveraging a pre-collected data set with contextual information and make informed decisions in uncertain environments____. 
This problem has been extensively studied and finds applications in diverse domains, including healthcare systems____, recommender systems____, and cyber-physical systems____.
This work applies offline contextual bandits towards the practical problem of allocating incentives for decarbonization.

\vspace{-0.15cm}