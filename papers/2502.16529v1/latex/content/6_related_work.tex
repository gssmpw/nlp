\section{Related Work}
\paragraph{LLMs for Visual Program Generation}
Visual programming systems (e.g., LabView~\cite{bitter2006labview}, XG5000~\cite{XG5000Manual}) typically feature node-based interfaces that let users visually write and modify programs. Recently, researchers have begun utilizing LLMs to generate VPLs, as they are known for their powerful text-based code generation capabilities. For example, \citet{cai-etal-2024-low-code} integrates low-code visual programming with LLM-based task execution for direct interaction with LLMs, while \citet{zhang2024benchmarking} studies generation of node-based visual dataflow languages in audio programming. Similarly, \citet{xue2024comfybenchbenchmarkingllmbasedagents,52868} investigates Machine Learning workflow generation from natural language commands and demonstrates that metaprogram-based text formats outperform other formats like JSON. However, these prompting-based methods face limitations for VPLs like Ladder Diagram, where custom I/O mapping and domain-specific syntax are crucial. Thus, we study fine-tuning approaches with domain-specific data to better capture these details.

\paragraph{LLM-based PLC code generation}
Programmable Logic Controllers (PLCs) are essential components in industrial automation and are used to control machinery and processes reliably and efficiently. Among the programming languages defined by the IEC 61131-3 standard~\cite{IEC61131-3}, Structured Text (ST) and Ladder Diagram (LD) are commonly used for programming PLCs. Research in this area has focused on utilizing LLMs to generate ST code from natural language descriptions. Recent studies have demonstrated the potential of LLMs in generating high-quality ST code~\cite{koziolek2023chatgpt, koziolek2024llm}, enhancing safety and accuracy with verification tools and user feedback~\cite{fakih2024llm4plc}, and automating code generation and verification using multi-agent frameworks~\cite{liu2024agents4plc}. Although these advances have improved PLC code generation, they primarily focus on ST, despite LD being widely used in industrial settings due to its similarity to electrical circuits~\cite{ladderlogic}. While \citet{Zhang_2024} attempts to generate LD as an ASCII art based on user instructions in a zero-shot manner, their findings show that even advanced LLMs struggle with basic LD generation. These limitations highlight the necessity of training-based methods for LD generation. In this work, we address this gap by introducing a training-based approach for LLMs to generate LD and thus pave the way for the broader adoption of AI-assisted PLC programming.