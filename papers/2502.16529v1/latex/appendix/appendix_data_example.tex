\section{Data Samples}\label{sec:data_samples}

\begin{table*}[!htbp]
    \small
    \centering
    \begin{tabular}{|p{0.15\textwidth}|p{0.8\textwidth}|} 
        \toprule
        \textbf{Type} & \textbf{Content} \\ \midrule
        
        Prompt 
        & \textbf{Program Description:} 셀 이재기 서보 조그 고속 하한 값 인터락 프로그램을 만들어줘. \\[3pt]
        & \textcolor{gray}{(Create a cell transfer servo jog high-speed interlock program.)} \\[3pt]
        & \textbf{Detailed Description:} 조그 고속 속도 설정값이 20 이하일 때 조그 고속속도를 20으로 설정해줘. \\[3pt]
        & \textcolor{gray}{(When the jog high-speed setting value is 20 or less, set the jog high-speed to 20.)} \\[3pt] \midrule

        \makecell[l]{XML \\ (Anonymized)}
        & \begin{lstlisting}[basicstyle=\ttfamily, aboveskip=0pt, belowskip=0pt]
<Rung>
    <Element Attributes..., Coordinate="X"></Element>
    <Element Attributes..., Coordinate="Y"></Element>
    <Element Attributes..., Coordinate="Z"></Element>
</Rung>
\end{lstlisting} \\ \bottomrule
    \end{tabular}
    \caption{Program description and anonymized XML example}
    \label{tab:appendix_sample}
\end{table*}

An example from the dataset used in this study is provided. The dataset consists of \textbf{prompt} and \textbf{code} pairs. The prompt was collected in Korean and used without translation for model training and evaluation. The prompt is further divided into \textbf{Program Description}, providing a high-level summary of the functionality, and \textbf{Detailed Description}, specifying task parameters and conditions. The prompts were created by an experienced PLC programmer from the Republic of Korea, ensuring the incorporation of domain expertise into the collected data. The programmer was explicitly informed that the dataset would be collected for model training and evaluation and used strictly for research purposes. We obtained consent prior to data collection. The \textbf{code} is represented in XML format, which describes the ladder diagram with elements. These elements are listed sequentially, and each element includes its attributes (e.g., variable names, parameters) and coordinate information. The datasets were constructed by an experienced PLC engineer, and all sensitive information was anonymized to ensure data confidentiality.