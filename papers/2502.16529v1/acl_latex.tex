% This must be in the first 5 lines to tell arXiv to use pdfLaTeX, which is strongly recommended.
\pdfoutput=1
% In particular, the hyperref package requires pdfLaTeX in order to break URLs across lines.

\documentclass[11pt]{article}

% Change "review" to "final" to generate the final (sometimes called camera-ready) version.
% Change to "preprint" to generate a non-anonymous version with page numbers.
\usepackage[preprint]{acl}

% Standard package includes
\usepackage{times}
\usepackage{latexsym}
\usepackage{kotex}
\usepackage{multicol}
\usepackage{lipsum}
\usepackage{verbatim}
\usepackage{multirow}
\usepackage{xcolor}  % 색상 관련 패키지
\usepackage{colortbl} % 테이블 배경색 패키지

% For proper rendering and hyphenation of words containing Latin characters (including in bib files)
\usepackage[T1]{fontenc}
% For Vietnamese characters
% \usepackage[T5]{fontenc}
% See https://www.latex-project.org/help/documentation/encguide.pdf for other character sets

% This assumes your files are encoded as UTF8
\usepackage[utf8]{inputenc}

% This is not strictly necessary, and may be commented out,
% but it will improve the layout of the manuscript,
% and will typically save some space.
\usepackage{microtype}

% This is also not strictly necessary, and may be commented out.
% However, it will improve the aesthetics of text in
% the typewriter font.
\usepackage{inconsolata}

%Including images in your LaTeX document requires adding
%additional package(s)
\usepackage{graphicx}
\usepackage{amsmath}
\usepackage{amsfonts}
\usepackage{algorithm}
\usepackage{algorithmic}

\usepackage{geometry}
\usepackage{longtable}
\usepackage{array}
\usepackage{caption}
\usepackage{booktabs}
\usepackage[most]{tcolorbox}
\usepackage{makecell}  % For better multi-line cell support
\usepackage{enumitem}
\usepackage{amssymb}


% If the title and author information does not fit in the area allocated, uncomment the following
%
%\setlength\titlebox{<dim>}
%
% and set <dim> to something 5cm or larger.

% \title{An Investigation into LLM Fine-Tuning for Visual Code Generation: Ladder Diagram as a Case Study}
\title{Retrieval-Augmented Fine-Tuning With Preference Optimization For Visual Program Generation}


% Author information can be set in various styles:
% For several authors from the same institution:
% \author{Author 1 \and ... \and Author n \\
%         Address line \\ ... \\ Address line}
% if the names do not fit well on one line use
%         Author 1 \\ {\bf Author 2} \\ ... \\ {\bf Author n} \\
% For authors from different institutions:
% \author{Author 1 \\ Address line \\  ... \\ Address line
%         \And  ... \And
%         Author n \\ Address line \\ ... \\ Address line}
% To start a separate ``row'' of authors use \AND, as in
% \author{Author 1 \\ Address line \\  ... \\ Address line
%         \AND
%         Author 2 \\ Address line \\ ... \\ Address line \And
%         Author 3 \\ Address line \\ ... \\ Address line}

% \author{First Author \\
%   Affiliation / Address line 1 \\
%   Affiliation / Address line 2 \\
%   Affiliation / Address line 3 \\
%   \texttt{email@domain} \\\And
%   Second Author \\
%   Affiliation / Address line 1 \\
%   Affiliation / Address line 2 \\
%   Affiliation / Address line 3 \\
%   \texttt{email@domain} \\}

\author{
 \textbf{Deokhyung Kang\textsuperscript{*1}}, 
 \textbf{Jeonghun Cho\textsuperscript{*1}}, 
 \textbf{Yejin Jeon\textsuperscript{1}}, 
 \textbf{Sunbin Jang\textsuperscript{3}}, 
\\
 \textbf{Minsub Lee\textsuperscript{3}}, 
 \textbf{Jawoon Cho\textsuperscript{4}}, 
 \textbf{Gary Geunbae Lee\textsuperscript{1,2}}
\\
 \textsuperscript{1}Graduate School of Artificial Intelligence, POSTECH,\\
 \textsuperscript{2}Department of Computer Science and Engineering, POSTECH,\\
 \textsuperscript{3}Hyundai Mobis,
 \textsuperscript{4}T\&I Company
\\
 \texttt{\{deokhk, jeonghuncho, jeonyj0612, gblee\}@postech.ac.kr}\\ \texttt{\{soonbin.Jang, perfect\}@mobis.com}, 
 \texttt{jwcho@tnicompany.com}\\
}

\newenvironment{shk}{%
    \tcblisting{
        enhanced,
        breakable,
        listing only,
        colback=gray!10,
        colframe=black,
        top=2mm,
        bottom=2mm,
        boxrule=0.5pt,
        before skip=10pt,
        after skip=10pt,
        after={\par\vspace{0.5\baselineskip}\noindent},
    }
}
{\endtcblisting}


\begin{document}
\maketitle
\def\thefootnote{*}\footnotetext{Equally contributed}\def\thefootnote{\arabic{footnote}}
\begin{abstract}

% Recent works to jointly reconstruct 3D human and object from a single RGB image, are mostly model-based, that fail to capture the fine details of the clothed human body and object surface. In this paper, we introduce ReCHOR, a novel, model-free, first-method to produce realistic clothed human-object reconstructions from a monocular view. This is extremely challenging due to human-object occlusions, diverse interactions and depth ambiguity, as it needs to infer both 3D spatial awareness and high resolution details. Our core idea is based on estimating neural implicit representations for human and object respectively by an attention-based neural implicit model that attends to pixel-aligned features from both the global human-object image for spatial awareness and  the local separate view of human and object images for high quality details. Additionally, the network is conditioned on semantic features from an initial estimated human-object pose prior and a generative diffusion model that inpaints occluded regions, thus enabling the retrieval of details from them.
% We also propose a synthetic dataset with rendered scenes of diverse, inter-occluded 3D human and object scans, to train our network. We evaluate our method on the synthetic and real world BEHAVE dataset. Our experiments show that our method outperforms the SOTA in achieving realistic clothed human-object reconstructions.
Recent approaches to jointly reconstruct 3D humans and objects from a single RGB image represent 3D shapes with template-based or coarse models, which fail to capture details of loose clothing on human bodies. In this paper, we introduce a novel implicit approach for jointly reconstructing realistic 3D clothed humans and objects from a monocular view. For the first time, we model both the human and the object with an implicit representation, allowing to capture more realistic details such as clothing. This task is extremely challenging due to human-object occlusions and the lack of 3D information in 2D images, often leading to poor detail reconstruction and depth ambiguity. To address these problems, we propose a novel attention-based neural implicit model that leverages image pixel alignment from both the input human-object image for a global understanding of the human-object scene and from local separate views of the human and object images to improve realism with, for example, clothing details. Additionally, the network is conditioned on semantic features derived from an estimated human-object pose prior, which provides 3D spatial information about the shared space of humans and objects. To handle human occlusion caused by objects, we use a generative diffusion model that inpaints the occluded regions, recovering otherwise lost details. For training and evaluation, we introduce a synthetic dataset featuring rendered scenes of inter-occluded 3D human scans and diverse objects. Extensive evaluation on both synthetic and real-world datasets demonstrates the superior quality of the proposed human-object reconstructions over competitive methods.
\end{abstract}
\section{Introduction}\label{sec:intro}

In computational finance, Monte Carlo simulations are used extensively to estimate the expected value of financial payoffs based on the solution of stochastic differential equations (SDEs) which model the evolution of stock prices, interest rates, exchange rates and other quantities \cite{glasserman04}.  Monte Carlo methods are very general and flexible, but for high accuracy it requires generating a large number of costly SDE path approximations, which has motivated research into a number of variance reduction or, equivalently, cost reduction techniques. One such method is
Multilevel Monte Carlo (MLMC), which was proposed in \cite{GILES2008} and was adapted for various applications that are summarised in \cite{Giles_overview17} and successfully combined with other methods such as quasi-Monte Carlo methods. The main idea of MLMC is to approximate the payoff using different time stepping resolutions when numerically solving the underlying SDE and to generate an optimal number of samples on each level, such that the overall computational cost is minimised subject to the desired bound on the variance. %, such that the total computational cost is minimised. 
The computational savings come from the fact that most samples are computed on the coarser levels and hence are less expensive while only a few samples from the finest levels are required \cite{GILES2008}.


Among the directions in which the computational cost 
of MLMC methods could further be reduced, an important avenue is the use of lower precision calculations, especially for the first Monte Carlo levels where the targeted accuracy is relatively low. 
 An overview of the research on mixed precision for the standard Monte Carlo (MC) framework is provided in \cite{ChowMixedPrecisionStandardMC} but only a few references study the potential of low precision computation in the MLMC framework \cite{Rounding_error_oliver}. To the best of our knowledge, the only MLMC framework with customised precision in the literature is \cite{brugger2014mixed}, but they use a uniform precision for all operations on each Monte Carlo level instead of optimising 
 the precision of each intermediary variable to reduce as much as possible the cost of path generation.
 
An important motivation for an MLMC framework with variable precision would be performing the low precision computations on reconfigurable hardware devices such as Field Programmable Gate Arrays (FPGAs). FPGAs contain customizable logic blocks and connectors that make it easy to adapt the digital circuit architecture for a specific application, leading to a highly parallel and optimised implementation. Therefore they are successfully exploited in applications that require high speed and have high computational workload, such as signal processing \cite{woods2008fpga}, and real time applications like high frequency trading \cite{HFT1,HFT2}. That is why a number of previous works in hardware architecture design implemented the MLMC algorithm to price financial options using FPGAs as accelerators, which resulted in improved speed and power efficiency compared to full CPU architectures \cite{Schryver2013AMM}. The paper \cite{lindsey2016domain} also proposed 
a Domain Specific Language to automate the configuration of FPGAs for this specific application. However, only \cite{brugger2014mixed} proposed a heuristic to reduce the precision in calculations.

In addition, all aforementioned works considered that the random number generation (RNG) is performed in single or double precision. Yet in most cases an important portion of the workload in the overall MLMC simulation comes from the RNG and in \cite{brugger2014mixed} this limited the total computational savings.
To reduce the cost of MLMC simulations in particular those based on the Geometric Brownian Motion (GBM), \cite{approximateICDF_Oliver, NestedOliver} have proposed to use approximate random numbers that are generated by applying an approximation of the inverse CDF to uniform random numbers. In \cite{NestedOliver}, the authors proposed a way to integrate these lower precision random variables into a \textit{nested} MLMC framework and completed a numerical analysis to bound the resulting error at each MC level by a product of the time step and the error in the random number approximation. The same authors show in \cite{approximateICDF_Oliver} that using approximate random variables reduces the cost of path generation by a factor 7.


In this paper we propose a nested MLMC framework that combines the use of approximate random normal variables and lower precision calculations to reduce the computational cost of MLMC even further than \cite{brugger2014mixed,NestedOliver}. We illustrate the efficiency of our framework in Matlab, after making several assumptions on the cost of operations and size of the errors that we carefully justify. We focus on the case of GBM and use the approximate RNG methods presented in \cite{approximateICDF_Oliver} as well as a new slightly modified method that combines CDF inversion and the central limit theorem. To choose the precision of the variables in the low precision path generation, we introduce a novel method to optimise the bit-widths. This optimisation is performed before the main path generation loop is executed and is based on a linear model of the payoff error  
due to rounding when computing in low precision. The error model relies on algorithmic differentiation in a similar manner to \cite{unifying-bwoptim,bitwidth-AD,ADAPT}. The bit-width optimisation procedure can be performed off-line, so this stage can be excluded from the on-line time complexity of our framework. The user specified desired accuracy is then enforced by calculating on-line the number of samples that need to be generated.

In terms of hardware design, we suggest implementing the low precision path generation on FPGAs and the full-precision ones on a CPU or GPU. 
The FPGA offers enough flexibility to define a separate bit-width for every variable in the low precision path generation, and can be reconfigured periodically to update the bit-widths when the market parameters have changed considerably. 


The paper is organized as follows : \Cref{sec:MLMC} introduces MLMC and nested MLMC to make clear the estimator that is implemented in our framework. Then in \Cref{sec:RNG} we detail the methods that could be used to obtain approximate random normally distributed numbers very cheaply for the low precision path generation. In \Cref{sec:error_model} and \Cref{sec:costModel} we propose an error model and a cost model (resp.) that we then use to formulate the optimisation problem that is solved to obtain the optimal bit-widths of fixed point variables in \Cref{sec:optimisation}. Finally we summarise our results and future directions in \Cref{sec:conclusion}.



\begin{figure}
    \centering
    \includegraphics[width=\columnwidth, keepaspectratio]{latex/figure/format.pdf}
    \caption{The topmost subfigure shows a single rung of LD and its corresponding visualized graph. The bottom displays XML tags exportable from a Ladder Diagram IDE, along with JSON and Metaprogram representations that capture structural relationships in the graph.}
    \label{fig:format}
\end{figure}


\begin{figure*}[!th]
    \centering
    \includegraphics[width=0.8\textwidth]{latex/figure/rag_performance_comparison.pdf} 
    \caption{Performance comparison between SFT and RAG, where RAG uses a larger LLM. $N$ represents the number of retrieved examples in RAG, and SFT’s performance is represented by a red dotted line. We use XML as the text format.}
    \label{fig:rag_sft_performance}
\end{figure*}
\begin{table*}[!ht]
\small
\centering
\resizebox{0.8\textwidth}{!}{%
\begin{tabular}{llccccc}
\toprule
Format                       & Method   & Node F1 $\uparrow$      & Edge F1 $\uparrow$      & Node EM $\uparrow$     & Edge EM $\uparrow$       & Program EM $\uparrow$    \\ \midrule
\multirow{2}{*}{XML}         & RAG & 74.4          & 67.1          & 49.6          & 49.2          & 49.2          \\
                             & SFT      & \textbf{84.8} & \textbf{74.7} & \textbf{55.2} & \textbf{54.8} & \textbf{54.8} \\ \midrule
\multirow{2}{*}{JSON}        & RAG & 72.6          & 65.7          & 51.6          & 50.8          & 50.8          \\
                             & SFT      & \textbf{85.1} & \textbf{74.9} & \textbf{53.6} & \textbf{52.6} & \textbf{52.6} \\ \midrule
\multirow{2}{*}{Metaprogram} & RAG & 80.0            & 71.9          & 52.2          & 51.8          & 51.8          \\
                             & SFT      & \textbf{86.3} & \textbf{75.7} & \textbf{55.2} & \textbf{54.0}   & \textbf{54.0}   \\ \bottomrule
\end{tabular}
}
\caption{Performance across different text formats. The highest values in each format are \textbf{in bold}.}
\label{tab:text_format}
\end{table*}


\section{Is Training-Based Approach Effective?}
In this section, we investigate the effectiveness of the training-based approach in comparison to the prompting-based approach for industrial VPL generation. We introduce Ladder Diagram (LD) as our test language (\S\ref{sec:ld}), then describe text format conversion in \S\ref{sec:conversion} to explore results across different text formats. Their results are presented in \S\ref{sec:sft_res}.
\subsection{Ladder Diagram}\label{sec:ld}
Programmable Logic Controller (PLC)~\citep{erickson1996programmable} controls physical devices such as sensors and actuators in industrial automation systems. For example, in a conveyor system, PLC logic enables a robotic arm to detect products via sensors and control its joints to transfer items between conveyors. This logic is typically implemented using Ladder Diagram (LD)~\citep{ladderlogic}, which consists of multiple rungs, each executing a specific task (e.g., operating a motor). Figure~\ref{fig:format} illustrates an LD with a single rung.

Each rung is further composed of contacts, coils, and function blocks and can be represented as a node-based graph, where each visual element is a node. These individual elements serve a distinct role in the logic sequence; contacts control the power flow, coils activate machinery, and function blocks (e.g., timers, counters) provide advanced control functions. They are connected to the PLC and mapped to I/O addresses that vary with the environment and hardware setup. This domain-specific address mapping enables customized control, as each rung manages unique I/O settings tailored to its setup.

\subsection{Text Format Conversion}\label{sec:conversion}
As LLMs cannot generate complex visual elements directly, prior studies~\cite{xue2024comfybenchbenchmarkingllmbasedagents,zhang2024benchmarking} generate VPL in text formats. As each format has unique characteristics, we convert VPL into various text formats to conduct comprehensive experiments. We consider three standard text formats for VPL: (1) XML, which represents visual elements sequentially; (2) JSON, which explicitly captures relationships between elements; (3) Metaprogram (code), which encodes VPL using a code-based syntax. 

Specifically, Figure~\ref{fig:format} illustrates that XML format represents LD as a list of visual elements, including contacts, lines, and function blocks. Each visual element corresponds to an \texttt{<Element>} tag containing its element type, coordinates, and other attributes. Since XML does not explicitly define relationships between elements, we extract these relationships using rule-based methods and convert them into a graph. We iterate through the visual elements in coordinate-based order, adding all elements except lines as nodes to the graph. We implement this graph using NetworkX~\cite{SciPyProceedings_11}, which is a widely used Python library for graph representation. We assign Node IDs starting from 0, increasing sequentially in coordinate order. To determine edges, we analyze node coordinates and line positions. Using this information, we construct a directed acyclic graph (DAG).
 
Using this graph representation of LD, we represent LD in both JSON and metaprogram formats. In the JSON format, we represent the LD as a dictionary; each node ID in the graph is a key, and its attributes and outgoing node IDs form the values. For the metaprogram (code) format, we represent the LD in Python syntax using NetworkX. We traverse the graph starting from the smallest node ID, visiting each node and its successors. Upon first visiting a node, we append a \texttt{G.add\_node(...)} statement with its attributes to the code. Similarly, we append a \texttt{G.add\_edge(...)} statement to the code for each neighbor relationship. This process continues until all nodes from the graph are visited. The generated code can be executed to reconstruct the original graph.

\begin{figure*}[!htb]
    \centering
    \includegraphics[width=\textwidth,keepaspectratio]{latex/figure/main_figure.pdf}
    \caption{An overview of two-stage training method. (1) RAFT-V: An off-the-shelf retriever is utilized for relevant prompt augmentation, and training is conducted with cross-entropy loss. (2) Preference Optimization: Preference learning leverages graph-edited preference pairs, with retrieved prompt-code pairs as additional input.}
    \label{fig:main_figure}
\end{figure*}

\subsection{Results}\label{sec:sft_res}
\paragraph{Training-based method outperforms prompting-based method}\label{par:sft_rag}
To evaluate the effectiveness of the training-based approach compared to the prompting-based approach for industrial VPL generation, we conduct a comparison between the supervised fine-tuning (SFT) method and Retrieval-Augmented Generation (RAG). Among prompting-based methods, we select RAG because this approach has been widely used in prior studies on VPL generation~\cite{xue2024comfybenchbenchmarkingllmbasedagents,zhang2024benchmarking}, and retrieval can provide examples that reflect the domain-specific configurations relevant to generation. SFT uses Llama3.1-8B-Instruct model as the backbone, while RAG uses Llama-3.1-70B-Instruct\footnote{Due to computational constraints, we employs the AWQ-quantized~\cite{MLSYS2024_42a452cb} model: \href{https://huggingface.co/hugging-quants/Meta-Llama-3.1-70B-Instruct-AWQ-INT4}{hugging-quants/Meta-Llama-3.1-70B-Instruct-AWQ-INT4}}. The RAG approach utilizes BM25~\cite{robertson2009probabilistic} to retrieve \(N\) similar prompt-code pairs from the SFT training dataset. We then append them to the input to generate code for the test prompt\footnote{Detailed experimental settings, including evaluation metrics, are presented in \S\ref{sec:exp_setup}.}.

As can be seen in Figure~\ref{fig:rag_sft_performance}, SFT consistently outperforms RAG across different number of retrieved examples (1, 3, 5, 7, and 9), despite its smaller backbone. Although RAG's performance improves as \(N\) increases, it eventually degrades, suggesting that learning domain-specific configurations in industrial VPL from retrieved examples alone is challenging. Therefore, these results demonstrate that SFT is an effective choice in this setting. We observe similar trends with a different LLM (Appendix~\ref{sec:detailed_results}), and the prompt template for the RAG model is provided in Appendix~\ref{sec:rag_details}. 


\paragraph{Training-based method excels across text formats}\label{par:sft_text_format}
To investigate whether the prior results are consistent across different text formats, we extend our experiments to include JSON and the metaprogram formats. We compare SFT with RAG, where performance is reported based on the number of retrieved examples that achieves the highest Program EM\footnote{Program EM is a binary metric (0 or 1) that indicates an exact match between the generated and reference programs. See \S\ref{sec:exp_setup} for details.} for each text format, as the optimal number of examples can vary across formats. 

From the results in Table~\ref{tab:text_format}, we can observe the following: (1) While SFT shows stable performance across different formats, RAG exhibits performance differences in Node/Edge F1, with the metaprogram-based format showing better performance. This aligns with prior results on prompting-based approaches from \citet{xue2024comfybenchbenchmarkingllmbasedagents}. (2) SFT outperforms RAG across all formats. These results further validate the effectiveness of the training-based approach.



\section{Methodology}
\paragraph{Preliminaries.}
We primarily focus on the homologous model merging, in which $\boldsymbol{\theta}_i$ all come from the same base model $\boldsymbol{\theta}_{\rm{base}}$. Given $K$ tasks $\{T_1,T_2,\cdots,T_K\}$ and $K$ corresponding fine-tuned models with parameters $\{\boldsymbol{\theta}_1,\boldsymbol{\theta}_2,\cdots,\boldsymbol{\theta}_K\}$, model merging aims to combine $K$ fine-tuned models into one single model simultaneously performing on $\{T_1,T_2,\cdots,T_K\}$ without post-training~\cite{method_p1_1,method_p1_2}.
Task vector~\cite{ilharco2023editing,yang2024adamerging} is a key element in merging method which could enhances the base model‘s ability or enable the model to handle other tasks. Specifically, for task $T_i$, the task vector $\boldsymbol\tau_i\in \mathbb{R}^D$ is defined as the vector obtained by subtracting the SFT weights $\boldsymbol{\theta}_i$ from the base model weight
$\boldsymbol{\theta}_{\rm{base}}$, \emph{i.e.}, $\boldsymbol\tau_i=\boldsymbol{\theta}_i-\boldsymbol{\theta}_{\rm{base}}$. The merged model could be denoted as $\boldsymbol{\theta}_m=\boldsymbol{\theta}_{\rm{base}}+\sum_i \lambda_i\boldsymbol{\tau}_i$, which $\lambda_i$ is the scaling factor measuring the importance of task vector. For clarification, we also denote the neuron set in $\boldsymbol{\theta}_i$ as $\mathcal{N}_i$, the neuron set in $\boldsymbol{\tau}_i$ as $\mathcal{T}_i$.



\begin{algorithm}[!ht]
    \caption{LED-Merging}
    \label{alg1}
    \begin{algorithmic}[1]
        \REQUIRE  base model $\boldsymbol{\theta}_{\rm{base}}$, SFT models $\{\boldsymbol{\theta}_{i}\mid i\in [K]\}$, mask ratios \{$r_{i} \mid i\in [K]\}$, scaling factors $\{\lambda_i\mid i\in[K]\}$, location datasets $\{\mathcal{X}_{i}\mid i\in[K]\}$
        \ENSURE merged parameter $\boldsymbol{\theta}_{m}$
        \STATE $\mathcal{M}\leftarrow\phi$
        \STATE $\boldsymbol{\theta}_{m}\leftarrow \boldsymbol{\theta}_{\rm{base}}$
        \FOR{$i\in [K]$}
        \STATE $I(\boldsymbol{\theta}_i)=\mathbb{E}_{x\sim \mathcal{X}_i}|\boldsymbol{\theta}_{i}\odot \nabla_{\boldsymbol{\theta}_i}\mathcal{L}(x)|$
        \STATE $I(\boldsymbol{\theta}_{\rm{base}})=\mathbb{E}_{x\sim \mathcal{X}_i}|\boldsymbol{\theta}_{\rm{base}}\odot \nabla_{\boldsymbol{\theta}_{\rm{base}}}\mathcal{L}(x)|$
        
        \STATE calculate $\mathcal{T}^{r_i}_{i}$ following Equation \ref{vote}
        \STATE  $\mathcal{M}\leftarrow \mathcal{M}\cup\{\mathcal{T}^{r_i}_i\}$
       
        
   
        
        
        \ENDFOR  
        \FOR{$i\in [K]$}
        
        \STATE calculate $\text{Disjoint}(\mathcal{T}_i^{r_i})$ use Equation~\ref{disjoint_safety}
        \STATE $\boldsymbol{m}_i \leftarrow \boldsymbol{0}$
        \FOR{$d\in \mathcal{T}_i^{r_i}$}
        \STATE $\boldsymbol{m}_{i,d}=1$
        \ENDFOR
        \STATE $\boldsymbol{\theta}_{m}\leftarrow \boldsymbol{\theta}_{m}+\lambda_i \boldsymbol{\tau}_i\odot \boldsymbol{m}_{i}$
        \ENDFOR
    \end{algorithmic}
\end{algorithm}
    %\vspace{-5pt}
\begin{figure*}[h!]
    \centering
    \includegraphics[width=\linewidth]{figs/pipeline_v2.pdf}
    \vspace{-40mm}
    \caption{Overview of our two-stage training pipeline {\ours}.}
    \label{fig:pipeline}
\end{figure*}


\paragraph{LED-Merging: Location, Election, and Disjoint Merging}
To address the neuron misidentification and interference issues in existing model merging methods, we propose LED-Merging (Location, Election, and Disjoint Merging). Specifically, previous studies \cite{modelstock, ilharco2023editing, tiesmerging} fail to accurately identify safety-related neurons in task vectors with a single magnitude score, namely \textit{neuron misidentification}. Meanwhile, there exists an interference between safety-related and utility-related task vector neurons during the merging process, namely \textit{neuron interference}. To address neuron misidentification, we first locate important neurons both in the base and fine-tuned models and then elect neurons from the task vector considering these two scores together. Subsequently, to mitigate the interference, we introduce a disjoint step, isolating these important neurons so that they influence different base neurons. The whole process is illustrated in Figure~\ref{fig:method}. 




In the location and election step, we consider the importance score from base and fine-tuned models simultaneously to locate task-specific neurons. In this way, it is more accurate than relying on the magnitude score alone because task-specific neurons with high importance score in the fine-tuned model may not necessarily score high in the base model, and vice versa.

{\textbf{Location}}.  We first calculate importance scores for each neuron in a base/fine-tuned model. Given a location dataset $\mathcal{X}_i=\{(x,y)_k\}$, where $x$ is the question and $y$ is the answer, we calculate the importance scores for the weight $\boldsymbol{\theta}_i\in\mathbb{R}^D$ in any  layer as follows~\cite{snip,spareseGPT,sun2024a}:
\begin{equation}
    I(\boldsymbol{\theta}_i)=\mathbb{E}_{x\sim \mathcal{X}_i}[\boldsymbol{\theta}_i\odot \nabla _{\boldsymbol{\theta}_i}\mathcal{L}(x)],
    \label{location}
\end{equation}
which $\mathcal{L}(x)=-\log p(y\mid x)$ is the conditional negative log-likelihood loss. We choose the SNIP score~\cite{snip} because it balances computational efficiency and performance~\cite{cq}. Please refer to Sec.~\ref{sec:ablation} for the comparison between different location methods. After computing importance scores, we choose top-$r_i$ neurons as the important neuron subset $\mathcal{N}_{i}^{r_i}$ from $I(\boldsymbol{\theta}_i)$.
 
 % After computing locating scores, we select the neurons scoring both high in base and fine-tuned models as important neurons in task vectors. Then in the disjoint step,  with preventing  polysemantic neurons  from receiving gradient updates towards different directions,
 % we use set difference to isolate the safety   and utility-related neurons  and construct corresponding masks for merging process,

{\textbf{Election}}. A natural question is how to select important neurons in the task vector $\boldsymbol{\tau}_i$ based on $I(\boldsymbol{\theta}_{\rm{base}})$ and $I(\boldsymbol{\theta}_{i})$. The important neurons in the base model may be different from neurons in the fine-tuned model. Therefore, we introduce the following election strategy to select neurons with high scores in both base and fine-tuned models:
\begin{equation}
    \mathcal{T}_i^{r_i}=\mathcal{N}_i^{r_i}\cap \mathcal{N}_{\rm{base}}^{r_i}.
    \label{vote}
\end{equation}
\emph{Remark}. We compare different choosing methods, including scoring low or high in base or fine-tuned model in Section~\ref{sec:ablation} and find that Equation \ref{vote} achieves the best performance.





{\textbf{Disjoint}}. As important neurons from different task vectors may conflict with each other at the same position, we use the set difference to disjoint the neurons from others to prevent interference:
\begin{equation}
    \text{Disjoint}(\mathcal{T}^{r_i}_{i})=\mathcal{T}^{r_i}_{i}-\mathop{\cup}\limits_{{J}\subsetneqq [K],|J|\geq 2}\mathop{\cap}\limits_{j\in {J}}\mathcal{T}^{r_j}_{j}.
    \label{disjoint_safety}
\end{equation}

Next, we construct a mask $\boldsymbol{m}_i\in\mathbb{R}^D$ to implement disjoint in the merging process. Specifically, this mask $\boldsymbol{m}_i$ is used to select neurons from $\mathcal{T}_i$. The mask ratio is $r_i$, where $r\in(0,1]$. The mask $\boldsymbol{m}_i$ can be derived from:
\begin{equation}
    \boldsymbol{m}_{i,d}=\begin{aligned} &\left\{ \begin{array}{ll} 1, & \text{if } d\in \text{Disjoint}(\mathcal{T}_{i}^{r_i}), \\ 0, & \text{otherwise}. \end{array} \right. \end{aligned}
    \label{mask_safety}
\end{equation}


% \subsection{Merging Models with Masks}
{\textbf{Merging}}. The final
merged task vector $\boldsymbol{\tau}_m$ is as follows:
\begin{equation}
    \boldsymbol{\tau}_m= \sum_i \lambda_i\boldsymbol{\tau}_{i}\odot\boldsymbol{m}_i.
    \label{merged_task_vector}
\end{equation}
We summarize the workflow in Algorithm \ref{alg1}.



\begin{table*}[ht]
\tiny
\centering
\resizebox{0.9\textwidth}{!}{%
\begin{tabular}{llccccc}
\toprule
Format & Method & \multicolumn{1}{l}{Node F1} & \multicolumn{1}{l}{Edge F1} & \multicolumn{1}{l}{Node EM} & \multicolumn{1}{l}{Edge EM} & \multicolumn{1}{l}{Program EM} \\ \midrule
 & SFT & 84.8 / 79.0 & 74.7 / 68.0 & 55.2 / 46.4 & 54.8 / 45.6 & 54.8 / 45.6 \\
 & RAFT-V & 88.1 / \textbf{87.4} & 80.9 / 78.7 & 63.8 / 59.6 & 62.8 / 58.8 & 62.8 / 58.8 \\
\multirow{-3}{*}{XML} & \cellcolor[HTML]{EFEFEF}Ours & \cellcolor[HTML]{EFEFEF}\textbf{89.6 / 87.4} & \cellcolor[HTML]{EFEFEF}\textbf{82.6 / 79.0} & \cellcolor[HTML]{EFEFEF}\textbf{66.2 / 61.2} & \cellcolor[HTML]{EFEFEF}\textbf{65.2 / 60.0} & \cellcolor[HTML]{EFEFEF}\textbf{65.2 / 60.0} \\ \midrule
 & SFT & 85.1 / 81.2 & 74.9 / 69.3 & 53.6 / 46.0 & 52.6 / 45.4 & 52.6 / 45.4 \\
 & RAFT-V & 90.1 / 88.4 & 82.0 / 79.6 & 63.8 / 59.2 & 63.2 / 57.4 & 63.2 / 57.4 \\
\multirow{-3}{*}{JSON} & \cellcolor[HTML]{EFEFEF}Ours & \cellcolor[HTML]{EFEFEF}\textbf{90.7 / 88.7} & \cellcolor[HTML]{EFEFEF}\textbf{83.0 / 80.5} & \cellcolor[HTML]{EFEFEF}\textbf{66.0 / 61.6} & \cellcolor[HTML]{EFEFEF}\textbf{65.2 / 60.2} & \cellcolor[HTML]{EFEFEF}\textbf{65.2 / 60.2} \\ \midrule
 & SFT & 86.3 / 81.7 & 75.7 / 70.2 & 55.2 / 47.6 & 54.0 / 47.2 & 54.0 / 47.2 \\
 & RAFT-V & 89.9 / 89.1 & 82.4 / 80.4 & 64.6 / 60.6 & 63.8 / 59.6 & 63.8 / 59.6 \\
\multirow{-3}{*}{Metaprogram} & \cellcolor[HTML]{EFEFEF}Ours & \cellcolor[HTML]{EFEFEF}\textbf{90.6 / 89.7} & \cellcolor[HTML]{EFEFEF}\textbf{83.6 / 81.6} & \cellcolor[HTML]{EFEFEF}\textbf{68.6 / 63.2} & \cellcolor[HTML]{EFEFEF}\textbf{67.2 / 61.8} & \cellcolor[HTML]{EFEFEF}\textbf{67.2 / 61.8} \\ \bottomrule
\end{tabular}%
}
\caption{Main results of the 2-stage training strategy. Each score is presented in the order Llama-3.1-8B-Instruct / Qwen2.5-7B-Instruct.}
\label{tab:main_result}
\end{table*}


\section{Experimental Settings}\label{sec:exp_setup}
\paragraph{Dataset} Due to the lack of publicly available datasets, we created our own by annotating ladder diagrams from actual production environments. Using XG5000~\cite{XG5000Manual}, we exported these diagrams as XML files and divided them into functional units, where each unit consists of one or more interconnected rungs designed to perform a specific function. An experienced PLC programmer then annotated natural language instructions for each unit. The dataset was randomly split into training, validation, and test sets of 13,124, 500, and 500 samples. 80\% of the training data was used for SFT, while the remaining 20\% was used for preference learning. Appendix~\ref{sec:data_samples} shows an example of the dataset utilized.

\paragraph{Implementation details} We utilize Llama-3.1-8B-Instruct~\cite{dubey2024llama} as the main backbone model and also use Qwen2.5-7B-Instruct~\cite{yang2024qwen2} to assess the generalizability of our method. To facilitate task understanding, we provide a detailed task description and explanations of the visual elements used in ladder diagrams via the system prompt. For models using retrieval, we utilize BM25~\cite{robertson2009probabilistic}, which is a widely used lexical matching-based method. Using BM25, we augment the input to these retrieval-based models with a top-1 retrieved prompt-code pair from the training dataset\footnote{System prompts are in Appendix~\ref{sec:system_prompt}, and further implementation details are in Appendix~\ref{sec:implementation_details}.}.

\paragraph{Evaluation metrics}
Text-based programming languages like Python are typically evaluated for correctness by unit tests~\cite{chen2021evaluating,austin2021programsynthesislargelanguage, chen2023codet}. In contrast, industrial visual programming languages often rely on separate simulators for evaluation, which limit consistent automated assessment across program variations and diverse environments~\cite{ray2017survey, ren2024infiniteworldunifiedscalablesimulation}.

To address this, we introduce graph-based automatic evaluation by transforming visually structured programs into NetworkX, as introduced in \S\ref{sec:conversion}. Specifically, we measure partial correctness in the graph representation by comparing it with the ground-truth graph using Node F1 and Edge F1 at the node and edge levels. For exact matches, we use Node EM and Edge EM to assess complete accuracy at the node and edge levels. Finally, Program EM evaluates whether the entire graph matches the ground-truth graph precisely in terms of both nodes and edges. Detailed explanations are provided in Appendix~\ref{sec:metric}.
\section{\label{sec:result}Results \& Discussion}

\subsection{\label{sec:overall_performance}Main Results}
\paragraph{Stage 1 (RAFT-V) improves upon SFT} Table~\ref{tab:main_result} compares the performance of SFT, RAFT-V, and our two-stage method across different output formats (XML, JSON, metaprogram) with two backbone models. By incorporating retrieval augmentation, RAFT-V consistently outperforms SFT and significantly enhances VPL generation. For example, in the JSON format, RAFT-V raises the Program EM from 52.6\% to 63.2\%, demonstrating more accurate VPL generation. Similar improvements are observed for other metrics.

\paragraph{Preference optimization (Stage 2) yields further gains} Building on Stage 1's improvements, Stage 2 further boosts correctness through preference optimization, particularly in terms of exact match (EM) scores. As shown in Table~\ref{tab:main_result}, our two-stage approach improves Program EM over RAFT-V by 3.4\% in the metaprogram format and achieves an overall 13.2\% gain compared to SFT. Because EM only assigns a score when the entire graph exactly matches, even minor generation failures can significantly impact the metric. These results show that preference optimization reduces such minor generation errors and ensures more precise outputs. Notably, these gains are consistent across all output formats and base models, demonstrating the robustness of our approach.\footnote{See Appendix~\ref{sec:human_eval} for human evaluation details.}

\subsection{\label{sec:retrieved_examples}Impact of Number of Retrieved Examples}
\begin{table}[!t]
\centering
\resizebox{\columnwidth}{!}{%
\begin{tabular}{@{}lccccc@{}}
\toprule
\( k \) & Node F1 & Edge F1 & Node EM & Edge EM & Program EM \\ \midrule
\rowcolor[HTML]{EFEFEF}
1 (Ours)      & 89.9    & 82.4    & 64.6    & 63.8    & 63.8     \\
2       & 90.6    & 83.5    & \textbf{66.4} & \textbf{65.6} & \textbf{65.6} \\
3       & \textbf{90.9} & \textbf{84.0} & 65.4    & 64.6    & 64.6     \\ \bottomrule
\end{tabular}%
}
\caption{Performance of RAFT-V across different numbers of retrieved examples $k$. We use the Llama3.1-8B-Instruct model with metaprogram format.}
\label{tab:retrieval_quantity_performance}
\end{table}
To assess whether the number of retrieved examples affects performance, we evaluate RAFT-V trained with different numbers of retrieval examples \( k \in \{1, 2, 3\} \) (Table~\ref{tab:retrieval_quantity_performance}). Although increasing retrieval examples \(k\) generally improves generation quality, the gains are marginal, which indicates that even a few examples are able to capture functional patterns in visual programming languages. Based on this observation, we set \( k = 1 \) for our main experiments.


\subsection{\label{sec:tau}Variation in Deletion Ratio}
\begin{table}[!h]
\centering
\resizebox{\columnwidth}{!}{%
\begin{tabular}{lccccc}
\toprule
$\boldsymbol{\tau}$ & Node F1 & Edge F1 & Node EM & Edge EM & Program EM \\ \midrule
0 & \textbf{90.7} & 83.4 & 67.2 & 65.8 & 65.8 \\
\rowcolor[HTML]{EFEFEF}
0.1 (Ours) & 90.6 & \textbf{83.6} & \textbf{68.6} & \textbf{67.2} & \textbf{67.2} \\
0.5 & 90.5 & 83.3 & 68.2 & 66.8 & 66.8 \\
0.9 & 90.6 & 83.4 & 67.4 & 66.0 & 66.0 \\ \bottomrule
\end{tabular}%
}
\caption{Impact of varying deletion ratios. Results for Llama-3.1-8B-Instruct model using metaprogram format.}
\label{tab:dpo_tau}
\end{table}
Furthermore, we report the performance variation depending on the deletion ratio $\tau$. We train the model using preference pairs with different $\tau$ values, where $\tau \in \{0, 0.1, 0.5, 0.9\}$. Table~\ref{tab:dpo_tau} shows that F1 scores remain stable across $\tau$ values, while EM scores decrease as $\tau$ increases. In particular, when $\tau$ is 0.9, the negative samples during preference training are graphs with 90\% of the original graph removed. As a result, the model can easily distinguish them, making the training less effective. Based on these results, we set $\tau$ to 0.1.

\begin{table}[!ht]
\centering
\resizebox{\columnwidth}{!}{%
\begin{tabular}{lccccc}
\toprule
Methods & Node F1 & Edge F1 & Node EM & Edge EM & Program EM \\ \midrule
RAFT-V (100\%) & 90.6 & \textbf{83.6} & 66.6 & 65.2 & 65.2 \\
\rowcolor[HTML]{EFEFEF}
RAFT-V & 89.9 & 82.4 & 64.6 & 63.8 & 63.8 \\ \midrule
BoN (Random) & 90.4 & 83.2 & 66.8 & 65.6 & 65.6 \\
BoN (Unstrict) & \textbf{90.7} & 83.5 & 67.0 & 65.8 & 65.8 \\
BoN (Strict) & 90.5 & 83.3 & 67.4 & 66.2 & 66.2 \\
Ours (Random) & 90.6 & 83.5 & 68.0 & 66.6 & 66.6 \\
\rowcolor[HTML]{EFEFEF}
Ours & 90.6 & \textbf{83.6} & \textbf{68.6} & \textbf{67.2} & \textbf{67.2} \\ \bottomrule
\end{tabular}%
}
\caption{Comparison of BoN sampling and graph editing. Results for metaprogram using Llama-3.1-8B-Instruct.}
\label{tab:dpo_aug}
\end{table}
\subsection{\label{sec:data_augmentation}Analysis of Graph Editing}
We validate our graph editing approach for collecting preference pairs by comparing it with Best-of-N (BoN) sampling\footnote{For BoN sampling, we use nucleus decoding~\citep{Holtzman2020The} with temperature=1.0, top\_p=1.0.}~\citep{10.5555/3495724.3495977, snell2025scaling}, which selects the best generation among $N$ candidates.
We generate $N=10$ outputs from the RAFT-V model and compare them with our negative candidate set $\mathcal{G}=\{G'_1\dots G'_{10}\}$, which is derived from graph editing (algorithm~\ref{alg:graph_augmentation}).

\paragraph{Preference learning is effective}
To address the concern that the gains in the 2-stage approach may result from the additional data used for preference learning, we compare it with a 1-stage approach, where the model is trained on all the data without a second stage, and is referred to as \underline{RAFT-V (100\%)}. As shown in Table~\ref{tab:dpo_aug}, RAFT-V improves performance as the dataset size increases. However, RAFT-V (100\%) shows only marginal improvement in EM scores compared to preference-learned models. Although RAFT-V (100\%) demonstrates effectiveness in improving F1 scores when compared to BoN-based preference learning baselines, it is insufficient to minimize minor errors (EM).

\paragraph{Editing-based negative selection is efficient}
We introduce four baselines for comparison with graph editing: \underline{BoN (Random)}, where $\mathcal{G}$ is sampled from BoN, and $G_{\text{Hard}}$ is randomly selected from BoN-sampled $\mathcal{G}$; \underline{BoN (Unstrict)}, which selects preference pairs based on GED; and \underline{BoN (Strict)}, which considers preference pairs for training only when the sampled output includes an exact match with the correct answer (i.e., GED = 0).  Finally, in the case of \underline{Ours (Random)}, $\mathcal{G}$ is sampled from graph editing, but $G_{\text{Hard}}$ is randomly selected instead of using GED.

As shown in Table~\ref{tab:dpo_aug}, BoN (Strict) constructs preference pairs based on exact matches and achieves the highest accuracy among BoN-based methods despite utilizing only 30\% of the data. This result demonstrates that dataset quality has a more significant impact on preference learning than dataset size~\cite{hou2024doesrlhfscaleexploring, kim2024aligninglargelanguagemodels}. However, BoN sampling often fails to generate challenging cases consistently, as negative samples are selected from the sampled outputs. In contrast, our editing-based negative selection provides a more systematic approach to generating negative samples, which consistently enables the generation of hard negative pairs with higher efficiency. 

\begin{figure}[!t]
    \centering
    \includegraphics[width=0.9\columnwidth,keepaspectratio]{latex/figure/total_exact_match_accuracy_by_complexity.pdf}
    \caption{Program EM score across different complexities. We use the metaprogram format with the Llama3.1-8B-Instruct model.}
    \label{fig:total_EM_trend_complexity}
\end{figure}
\subsection{\label{sec:difficulty}Performance Across Program Complexity}
To evaluate our methodology's performance across varying program complexities, we convert each program (VPL code) in the test split into NetworkX graphs. We define complexity as the total number of nodes and edges in the graph. We then sort programs by complexity and split them into five percentile ranges: 0–20\%, 20–40\%, 40–60\%, 60–80\%, and 80–100\%, labeling them from 1 to 5. Figure~\ref{fig:total_EM_trend_complexity} shows the average Program EM across these categories. Our approach consistently outperforms the SFT baseline, widening the gap at higher complexities (+18.3\% in 4, +16.3\% in 5). These findings highlight our method's robustness, with benefits increasing in challenging scenarios.
\section{Related Work}

\paragraph{Commonsense Reasoning Evaluation} 
There are numerous benchmarks and datasets for commonsense reasoning, most of which are in English. 
%Some work focused on evaluating general commonsense knowledge, such as HellaSwag \cite{zellers2019hellaswag}, CommonsenseQA \cite{talmor2019commonsenseqa}, OpenBookQA \cite{OpenBookQA2018}, and WSC \cite{levesque2012winograd}. 
Some studies focus on evaluating general commonsense knowledge \cite{zellers2019hellaswag,talmor2019commonsenseqa,OpenBookQA2018}. 
%Others target specific aspects of commonsense reasoning, including temporal commonsense with MCTACO \cite{zhou2019going}, physical commonsense with PIQA \cite{bisk2020piqa}, social commonsense with SocialIQA \cite{sap2019socialiqa}, numerical commonsense with NumerSense \cite{lin2020birds}, and scientific commonsense with ARC \cite{clark2018think} and QASC \cite{khot2020qasc}. Notably, most of these datasets are in English. 
Others target specific aspects of commonsense reasoning\cite{zhou2019going,bisk2020piqa,sap2019socialiqa,lin2020birds,clark2018think,khot2020qasc}.
There are some Chinese datasets for commonsense reasoning \cite{sun2024benchmarking,shi2024corecode}. 
For instance, CHARM \cite{sun2024benchmarking} distinguishes between global commonsense and Chinese-specific commonsense but includes only a limited number of everyday commonsense cases. 
However, evaluations aimed at assessing the robustness of commonsense reasoning are still understudied. 

\paragraph{Datasets on Different Reasoning Forms}
There are several datasets relevant to our variant design. For reverse reasoning, ART \cite{DBLP:conf/iclr/BhagavatulaBMSH20}, $\delta$-NLI \cite{DBLP:conf/emnlp/RudingerSHBFBSC20}, and CLUTRR \cite{DBLP:conf/emnlp/SinhaSDPH19} explore different reasoning directions. FCR \cite{DBLP:journals/corr/abs-2204-07408} and NatQuest \cite{ceraolo2024analyzinghumanquestioningbehavior} evaluate causal reasoning, while TimeTravel \cite{DBLP:conf/emnlp/QinBHBCC19} focuses on counterfactual scenario refinement. Additionally, PoE \cite{balepur2024s} assesses reasoning involving negation. 
However, not all these datasets focus on commonsense reasoning, nor are they structured by original questions and their variants. Furthermore, they typically target limited reasoning types. Lastly, our dataset is large-scale and covers diverse commonsense knowledge. 

\paragraph{Robustness and Consistency in LLMs} 
Early work focuses on adversarial attacks, with developing evaluation methods for reading comprehension systems \cite{jia2017adversarial}, followed by universal adversarial triggers \cite{wallace2019universal}. The field then expands to examine spurious correlations, with revealing how models often exploit superficial patterns rather than engaging in genuine reasoning \cite{branco2021shortcutted,geirhos2020shortcut}. And \citealp{ross2022does} investigates whether self-explanation can mitigate these spurious correlations. Coherence and consistency evaluation advances through classifier assessment methods \cite{storks2021beyond} and analysis of accuracy-consistency trade-offs \cite{johnson2023much}. While these studies primarily address model robustness against adversarial attacks or spurious correlations, our work takes a novel approach by examining robustness in reasoning forms.
%, specifically focusing on how models maintain consistent reasoning when presented with different reasoning forms of the same commonsense knowledge.
%\paragraph{Dataset Construction by LLM} 
%Research indicates that when LLMs are utilized for dataset generation, the resulting datasets are more accurate and fluent \cite{lu2022fantastically, min-etal-2022-rethinking} than those created by crowd-sourced annotators. Furthermore, generating datasets with LLMs is significantly more cost-effective than using crowd-sourced annotations \cite{liu2022wanli, wiegreffe2022reframing, west2022symbolic}. Hence, we generate our benchmark by LLM in-context learning.

% \paragraph{In-Context Learning} 
% As LLMs become more widely used, in-context learning (\citealp{brown2020language}; \citealp{ouyang2022training}; \citealp{min-etal-2022-rethinking}) has emerged as the primary approach for executing various tasks. This method involves supplying LLMs with textual instructions and examples and removes the necessity for parameter modifications. Research indicates that when LLMs are utilized for dataset generation, the resulting datasets are more accurate and fluent (\citealp{lu2022fantastically}; \citealp{min-etal-2022-rethinking}) than those created by crowd-sourced annotators. Furthermore, generating datasets with LLMs is significantly more cost-effective than using crowd-sourced annotations (\citealp{liu2022wanli}; \citealp{wiegreffe2022reframing}; \citealp{west2022symbolic}). Hence, we have decided to construct our benchmark by over-generating data using in-context learning and employing human annotators for filtering to ensure high efficiency and high quality.
% Moxin: 这部分应该改成用LLM 生成dataset的工作?
% 

\section{Conclusion and future directions} \label{sec:conclusion}

In this paper we proposed a nested MLMC framework that offers important computational savings by performing most calculations in low precision and exploiting approximate random normal variables for the low precision path calculations. The low precision calculations could be performed in fixed precision on an FPGA for greater efficiency, and we suggested a procedure to optimise the bit-widths of every variable at each Monte Carlo level. This is an important improvement over previous mixed precision MLMC frameworks which held the lower precision fixed \cite{Rounding_error_oliver} or defined uniform bit-width at every level heuristically \cite{brugger2014mixed}. Our numerical results suggest that for the first levels our procedure reduces the cost at these levels by a factor 5 or 7. Hence the overall savings are significant since most paths are calculated on the first levels. Our approach would be even more efficient for the Milstein scheme because its higher order strong convergence leads to a greater proportion of the computational costs being on the coarsest levels.

The next stage of the research project will be to implement the RNG methods and the nested framework on FPGAs to determine the hardware requirements and confirm the extent of the computational savings. It would also be good to compare the performance benefits to using half-precision floating point arithmetic on GPUs or CPUs for the low-accuracy computations.



\section{Limitations} 

In this work, we compared the effectiveness and interplay of SFT and RL-based methods, under fixed data constraints. In particular, we chose offline methods like DPO and KTO as the baseline implementation of the RL method because it eliminates the need for reward modeling or iterative finetuning. This means that the process of development is limited to collecting an offline dataset and fientuning it - making it the most fair comparable to SFT in terms of implementation effort, compute costs and annotation efforts. Since this baseline RL method shows optimal performance over SFT, we hope that this motivates future work to study more complex RL-based methods and their interplay with SFT. In addition, we used GPT4o annotation for synthetic data generation, and also for evaluating Summarization and Helpfulness, which could include potential biases inherited from the model. 

In addition, we limited the size of the model to under 10 Billion parameters, to keep the finetuning cost low enough to ignore as compared to the data annotation costs. In addition, it would be extremely compute resource intensive to run thousands of finetuning runs with larger model sizes like 70B parameters. We hope that future work would study the scaling trends of RL-based methods against different model sizes, and also study the compute-data trade-off in-depth.


\section*{Acknowledgments}
This work was supported by Hyundai Mobis (47.5\%). This research was supported by the MSIT(Ministry of Science and ICT), Korea, under the ITRC(Information Technology Research Center) support program(IITP-2025-2020-0-01789) supervised by the IITP(Institute for Information \& Communications Technology Planning \& Evaluation, 47.5\%). This work was supported by Institute of Information \& communications Technology Planning \& Evaluation (IITP) grant funded by the Korea government(MSIT) (No.RS-2019-II191906, Artificial Intelligence Graduate School Program(POSTECH), 5\%).

% Bibliography entries for the entire Anthology, followed by custom entries
%\bibliography{anthology,custom}
% Custom bibliography entries only

% \nocite{*}
\bibliography{custom}

\onecolumn
\appendix
% \section{Appendix}

\section{\label{sec:metric}Evaluation Metrics}
As mentioned in \S\ref{sec:conversion}, Ladder Diagram (LD) can be represented as NetworkX graphs. We evaluate LD programs from two perspectives: node and edge levels and the overall graph level. To assess the accuracy of node and edge predictions, we compute the F1 score, which considers both precision and recall for the sets of nodes and edges in each ground truth and predicted graph. These metrics are defined as \textbf{Node F1} and \textbf{Edge F1}, respectively. Furthermore, to verify the correct execution of LD, we use Exact Match (EM), which strictly measures the structural accuracy by comparing the entire node and edge sets between the predicted and reference graphs. We define \textbf{Node EM} and \textbf{Edge EM} to measure the exact match between the entire sets of nodes and edges, respectively. Moreover, \textbf{Program EM} represents the overall structural alignment, which is achieved only when both node and edge sets are perfectly matched.

Let the reference graph be denoted as $G = (V, E)$ and the predicted graph be denoted as $\hat{G} = (\hat{V}, \hat{E})$. Here, comparisons between nodes and edges are evaluated by considering all required attributes (e.g., names, types) for both F1 score and exact match.

\begin{equation}
\begin{aligned}
\nonumber
\text{For nodes:} \quad & TP_N = |V \cap \hat{V}|,\quad FP_N = |\hat{V} \setminus V|,\quad FN_N = |V \setminus \hat{V}|\\[1ex]
& \text{Precision}_N = \frac{TP_N}{|\hat{V}|},\quad \text{Recall}_N = \frac{TP_N}{|V|} \quad \therefore \textbf{Node F1} = \frac{2\,TP_N}{|V| + |\hat{V}|} = \frac{2\,|V \cap \hat{V}|}{|V| + |\hat{V}|} \\[2ex]
\text{For edges:} \quad & TP_E = |E \cap \hat{E}|,\quad FP_E = |\hat{E} \setminus E|,\quad FN_E = |E \setminus \hat{E}|\\[1ex]
& \text{Precision}_E = \frac{TP_E}{|\hat{E}|},\quad \text{Recall}_E = \frac{TP_E}{|E|} \quad \therefore \textbf{Edge F1} = \frac{2\,TP_E}{|E| + |\hat{E}|} = \frac{2\,|E \cap \hat{E}|}{|E| + |\hat{E}|}
\end{aligned}
\end{equation}
The Exact Match (EM) scores are defined as follows:
\begin{equation}
\nonumber
\textbf{Node EM} = \mathbf{1}\{\hat{V} = V\}, \quad \textbf{Edge EM} = \mathbf{1}\{\hat{E} = E\}, \quad \textbf{Program EM} = \mathbf{1}\{\hat{V} = V \text{ and } \hat{E} = E\}
\end{equation}


\subsection{Examples of Metric Evaluations}
\begin{figure*}[!htbp]
    \centering
    \includegraphics[width=0.9\textwidth]{latex/figure/metric.pdf}
    \caption{Example for explaining evaluation metrics. Based on real data, but node and edge attributes have been modified due to security issues.}
    \label{fig:metric}
\end{figure*}

\begin{equation}
\nonumber
\text{Precision}_N = \frac{|V \cap \hat{V}|}{|\hat{V}|} = \frac{5}{5} = 1.0, \quad
\text{Recall}_N = \frac{|V \cap \hat{V}|}{|V|} = \frac{5}{6} = 0.833
\end{equation}
\begin{equation}
\nonumber
\therefore \textbf{Node F1} = \frac{2 \times 1.0 \times 0.833}{1.0 + 0.833} = 0.91
\end{equation}
Figure~\ref{fig:metric} is evaluated based on the above metrics. The reference graph contains 5 edges, while the predicted graph contains 4 edges, with one incorrect edge.
\begin{equation}
\nonumber
\text{Precision}_E = \frac{TP_E}{|\hat{E}|} = \frac{3}{4} = 0.75, \quad
\text{Recall}_E = \frac{TP_E}{|E|} = \frac{3}{5} = 0.6
\end{equation}
\begin{equation}
\nonumber
\therefore \textbf{Edge F1} = \frac{2 \times 0.75 \times 0.6}{0.75 + 0.6} = 0.67
\end{equation}
The predicted graph does not fully match the reference graph, so all EM scores are 0. Although the node structure is mostly correct (\( F1 = 0.91 \)), inconsistencies in the edge structure (\( F1 = 0.67 \)) prevent an exact match.
\section{Further Implementation Details}\label{sec:implementation_details}
\paragraph{Supervised fine-tuning (including RAFT-V)}
We fine-tuned models using SFT with 10 epochs, a batch size of 8, and a learning rate of $5 \times 10^{-5}$. We applied LoRA~\cite{hulora} with a rank of 256, $\alpha = 256$, and a dropout rate of 0.05. LoRA adaptation was applied to the following target modules: q\_proj, v\_proj, k\_proj, o\_proj. Using the AdamW~\cite{loshchilov2018decoupled} optimizer, we minimized the cross-entropy loss $\mathcal{L}_{\text{CE}}$ between ground-truth code $c$ and the predicted code $\hat{c}$ as follows:
\begin{equation}
\nonumber
    \mathcal{L}_{\text{CE}} = - \sum_{t=1}^{T} \sum_{v \in V} c_t(v) \log \hat{c}_t(v)
\end{equation}
where $T$ is the sequence length and $V$ is a vocabulary size. The SFT process took 6 hours using 4 A100-80GB GPUs, while the SFT process for RAFT-V took 8 hours using the same hardware configuration.

\paragraph{Preference learning} For preference learning, we utilized Direct Preference Optimization (DPO)~\cite{rafailov2023direct}, and trained models for 5 epochs with a batch size of 64. The learning rate was set to $1 \times 10^{-7}$, with a warmup ratio of 0.03, a weight decay of 0.01, and $\beta=0.1$. As with SFT, {LoRA} was applied with the same rank, $\alpha$, dropout rate, and target modules.

Given a pair of responses, a preferred response $y^+$ and a dispreferred response $y^-$ for a given input $x$, we minimize the following DPO loss:

\begin{equation}
\nonumber
    \mathcal{L}_{\text{DPO}}(\theta) = -\mathbb{E}_{(x,y^+, y^-) \sim \mathcal{D}} \left[ 
    \log \sigma \left( \beta \left( \log \frac{\pi_\theta(y^+ \mid x,\mathcal{R}(x))}{\pi_{\text{ref}}(y^+ \mid x,\mathcal{R}(x))} - \log \frac{\pi_\theta(y^- \mid x,\mathcal{R}(x))}{\pi_{\text{ref}}(y^- \mid x,\mathcal{R}(x))} \right) \right) 
    \right]
\end{equation}
where $\sigma(z)$ is the sigmoid function, $\beta$ is a scaling factor controlling the strength of preference optimization, $\mathcal{R}(x)=(p^r, c^r)$ represents a retrieved prompt and its corresponding VPL code obtained from a retriever $\mathcal{R}$, and $\mathcal{D}$ represents the dataset of preference-labeled samples. The reference model $\pi_{\text{ref}}$ serves as a baseline to prevent reward overoptimization, ensuring stable preference learning. The preference training stage took 6 hours on 4 A100-80GB GPUs.
We measured the loss on the validation set at each epoch in both training stages and applied early stopping based on this criterion, with a patience value of 2. 
\paragraph{Sampling parameters} We employed beam search decoding with a beam size of 4 during inference. 

\section{Data Samples}\label{sec:data_samples}

\begin{table*}[!htbp]
    \small
    \centering
    \begin{tabular}{|p{0.15\textwidth}|p{0.8\textwidth}|} 
        \toprule
        \textbf{Type} & \textbf{Content} \\ \midrule
        
        Prompt 
        & \textbf{Program Description:} 셀 이재기 서보 조그 고속 하한 값 인터락 프로그램을 만들어줘. \\[3pt]
        & \textcolor{gray}{(Create a cell transfer servo jog high-speed interlock program.)} \\[3pt]
        & \textbf{Detailed Description:} 조그 고속 속도 설정값이 20 이하일 때 조그 고속속도를 20으로 설정해줘. \\[3pt]
        & \textcolor{gray}{(When the jog high-speed setting value is 20 or less, set the jog high-speed to 20.)} \\[3pt] \midrule

        \makecell[l]{XML \\ (Anonymized)}
        & \begin{lstlisting}[basicstyle=\ttfamily, aboveskip=0pt, belowskip=0pt]
<Rung>
    <Element Attributes..., Coordinate="X"></Element>
    <Element Attributes..., Coordinate="Y"></Element>
    <Element Attributes..., Coordinate="Z"></Element>
</Rung>
\end{lstlisting} \\ \bottomrule
    \end{tabular}
    \caption{Program description and anonymized XML example}
    \label{tab:appendix_sample}
\end{table*}

An example from the dataset used in this study is provided. The dataset consists of \textbf{prompt} and \textbf{code} pairs. The prompt was collected in Korean and used without translation for model training and evaluation. The prompt is further divided into \textbf{Program Description}, providing a high-level summary of the functionality, and \textbf{Detailed Description}, specifying task parameters and conditions. The prompts were created by an experienced PLC programmer from the Republic of Korea, ensuring the incorporation of domain expertise into the collected data. The programmer was explicitly informed that the dataset would be collected for model training and evaluation and used strictly for research purposes. We obtained consent prior to data collection. The \textbf{code} is represented in XML format, which describes the ladder diagram with elements. These elements are listed sequentially, and each element includes its attributes (e.g., variable names, parameters) and coordinate information. The datasets were constructed by an experienced PLC engineer, and all sensitive information was anonymized to ensure data confidentiality.
\section{\label{sec:human_eval}Human Evaluation}

\begin{table*}[!htbp]
\centering
\begin{tabular}{@{}lcc@{}}
\toprule
Method        & Functional Score ($\uparrow$) & Kappa Score ($\uparrow$) \\ \midrule
SFT-only      & 4.34            &       0.62      \\
RAFT-V        & 4.60            &         0.58    \\
Ours & \textbf{4.62} & \textbf{0.66}   \\ \bottomrule
\end{tabular}
\caption{Average of human evaluation results. We used the metaprogram format for evaluation.}
\label{tab:human_evaluation}
\end{table*}

To further evaluate the effectiveness of our methods, we conduct human evaluations on a randomly selected 50 test set examples. Due to the dataset's security sensitivity, human evaluations were conducted exclusively by PLC engineers who had authorized access to confidential information. 
Three experienced LD programmers evaluated the generated code based on functionality, and assigned scores on a scale of 1 to 5.
The ratings were based on their professional experience and the practical applicability of the code in real industrial settings. The results are shown in Table~\ref{tab:human_evaluation}. Compared to SFT-only, RAFT-V showed an improvement of 0.26 points, while our method outperformed RAFT-V by 0.02 points. Furthermore, we report Fleiss' kappa coefficient~\citep{fleiss1971measuring} to statistically evaluate the level of agreement among human evaluators. The results indicate that our proposed methodology demonstrates the highest degree of inter-rater consistency.

\section{More Details on RAG}\label{sec:rag_details}
The prompt template for RAG models is as follows. The RAG model takes the same system prompt as the training-based model depending on the text format.

\begin{shk}
message = [
    {"role": "system", "content": {system prompt}},
    {"role": "user", "content": retrieved_prompt}, 
    ...
    {"role": "assistant", "content": retrieved_code},
    {"role": "user", "content": {test_prompt}}
]
\end{shk}
\noindent\begin{minipage}{\textwidth}
\captionsetup{type=figure}
\captionof{figure}{Prompt template for RAG models}
\end{minipage}

Since some outputs of RAG models were ill-formed (particularly in the case of Qwen), we applied postprocessing to refine them. Extra elements such as \verb|```xml|, \verb|```json|, \verb|```python|, or unrelated code snippets were removed. For inference, we employ beam search with beam size of 4.

\section{Detailed Results}\label{sec:detailed_results}
In this section, we present detailed experimental results. Table~\ref{tab:rag_versus_sft_llama} shows the performance comparison between the SFT model and RAG, with a larger LLM within the same family.


\begin{table*}[!htbp]
\tiny
\centering
\resizebox{0.9\textwidth}{!}{%
\begin{tabular}{@{}llccccc@{}}
\toprule
Format& Method & Node F1& Edge F1& Node EM& Edge EM& Program EM\\ \midrule

\multirow{6}{*}{XML}& SFT& \textbf{84.8} / \textbf{79.1}&\textbf{74.7} / \textbf{68.2} & \textbf{55.2} / \textbf{46.8}& \textbf{54.8} / \textbf{46.0}& \textbf{54.8} / \textbf{46.0}\\
& $N=1$& 69.0 / 19.5& 58.2 / 13.9& 39.0 / 8.8& 38.6 / 8.6& 38.0 / 8.6\\
& $N=3$& 76.3 / 61.4& 67.6 / 53.3& 49.8 / 40.2& 49.0 / 39.4& 49.0 / 39.4\\
& $N=5$& 74.4 / 60.7& 67.1 / 53.8& 49.6 / 41.2& 49.2 / 40.6& 49.2 / 40.6\\
& $N=7$& 59.3 / 53.6& 53.5 / 47.9& 42.8 / 37.8& 42.2 / 37.6& 42.2 / 37.6\\
& $N=9$& 50.5 / 44.9& 45.1 / 40.4& 37.6 / 34.4& 36.8 / 34.0& 36.8 / 34.0\\ \midrule
\multirow{6}{*}{JSON}& SFT& \textbf{85.1} / \textbf{81.2} & \textbf{74.9} / \textbf{69.3} & \textbf{53.6} / \textbf{46.8}& \textbf{52.6} / 46.4& \textbf{52.6} / 46.4\\
& $N=1$& 1.2 / 5.3& 0.8 / 3.9& 0.6 / 3.6& 0.6 / 4.0& 1.1 / 3.6\\
& $N=3$ & 79.2 / 73.4& 69.8 / 62.7& 48.2 / 43.4& 47.4 / 46.0& 47.4 / 46.0\\
& $N=5$& 79.3 / 74.8& 70.7 / 67.1& 51.4 / 43.8& 50.6 / \textbf{46.6}& 50.6 / \textbf{46.6}\\
& $N=7$& 72.6 / 54.6& 65.7 / 50.2& 51.6 / 42.2& 50.8 / 41.8 & 50.8 / 41.8\\
& $N=9$& 58.5 / 52.8& 53.1 / 48.0& 44.2 / 40.2& 43.6 / 39.8& 43.6 / 39.8\\ \midrule
\multirow{6}{*}{Metaprogram}& SFT& \textbf{86.3} / \textbf{81.8} & \textbf{75.7} / \textbf{70.5} & \textbf{55.2} / \textbf{48.4}& \textbf{54.0} / \textbf{48.0}& \textbf{54.0} / \textbf{48.0} \\
& $N=1$& 4.2 / 1.0& 3.5 / 0.9& 2.8 / 0.8& 2.8 / 0.6& 2.8 / 0.6\\
& $N=3$& 80.8 / 47.2& 71.1 / 40.9& 49.4 / 27.0& 48.4 / 26.0& 48.4 / 26.0\\
& $N=5$& 80.0 / 68.1& 71.9 / 59.7& 52.2 / 43.6& 51.8 / 42.6& 51.8 / 42.6\\
& $N=7$& 76.4 / 71.5& 69.1 / 64.2& 51.8 / 48.2& 51.2 / 47.6& 51.2 / 47.6\\
& $N=9$& 64.5 / 59.5& 57.8 / 53.6& 47.2 / 43.8& 46.8 / 42.8& 46.8 / 42.8\\ \bottomrule
\end{tabular}%
}
\caption{Performance comparison between SFT and RAG (with a larger LLM) across XML, JSON, and Metaprogram formats. For RAG, $N$ denotes the number of retrieved examples. Each SFT score is presented in the order of "Llama-3.1-8B-Instruct / Qwen2.5-7B-Instruct." RAG model utilizes the AWQ-quantized versions of \href{https://huggingface.co/hugging-quants/Meta-Llama-3.1-70B-Instruct-AWQ-INT4}{hugging-quants/Meta-Llama-3.1-70B-Instruct-AWQ-INT4} and \href{https://huggingface.co/Qwen/Qwen2.5-72B-Instruct-AWQ}{Qwen/Qwen2.5-72B-Instruct-AWQ}. The highest performance value within each format is shown in bold.}
\label{tab:rag_versus_sft_llama}
\end{table*}
\section{System Prompts} \label{sec:system_prompt}
Depending on the type of text format used, the model takes a different system prompt. The retrieval-based model used in the experiments takes the following prompt template as input:

\begin{shk}
message = [
    {"role": "system", "content": {system prompt}},
    {"role": "user", "content": retrieved_prompt}, 
    ...
    {"role": "assistant", "content": retrieved_code},
    {"role": "user", "content": f"Based on the given input, generate the corresponding code: {test_prompt}"}
]
\end{shk}
\noindent\begin{minipage}{\textwidth}
\captionsetup{type=figure}
\captionof{figure}{Prompt template used in this study}
\end{minipage}
For models that do not utilize retrieval, the prompt template excludes \texttt{retrieved\_prompt} and \texttt{retrieved\_code} for code generation.

\subsection{System prompt for XML}
\begin{shk}
You are a programming assistant specializing in generating ladder programs in XML format. Your task is to translate functional descriptions into equivalent PLC ladder logic and directly represent the ladder logic as XML. The natural language instructions will describe the desired functionality. Your job is to:  
1. Interpret the described functionality.  
2. Translate it into equivalent ladder logic components (e.g., rungs, contacts, coils).  
3. Directly create and output the ladder logic as XML.

###  Requirements for Ladder Logic Representation in XML:
- Each element must include an `ElementType` attribute, which specifies its type, and additional necessary attributes depending on the `ElementType`:
- The output XML must be well-formed, human-readable, and valid for parsing by PLC-related tools or frameworks.

### Explanation of ElementTypes:
[Lines]
- VertLine: It is a vertical line.
- HorzLine: It is a horizontal line.
- MultiHorzLine: It is a horizontal line with a fixed length.

[Contact]
- NormallyOpen: When the state of the BOOL variable (indicated by "***") is On, the state of the left connection line is copied to the right connection line. Otherwise, the state of the right connection line is Off.
- NormallyClosed: When the state of the BOOL variable (indicated by "***") is Off, the state of the left connection line is copied to the right connection line. Otherwise, the state of the right connection line is Off.
- RisingEdgeContact: If the value of the BOOL variable (indicated by "***") changes from Off in the previous scan to On in the current scan, and the state of the left connection line is On, the state of the right connection line becomes On during the current scan.
- FallingEdgeContact: If the value of the BOOL variable (indicated by "***") changes from On in the previous scan to Off in the current scan, and the state of the left connection line is On, the state of the right connection line becomes On during the current scan.
- RisingEdgeNotContact: If the value of the BOOL variable (indicated by "***") changes from Off in the previous scan to On in the current scan, and the state of the left connection line is On, the state of the right connection line becomes Off during the current scan.
- FallingEdgeNotContact: If the value of the BOOL variable (indicated by "***") changes from On in the previous scan to Off in the current scan, and the state of the left connection line is On, the state of the right connection line becomes Off during the current scan.

[Coil]
- StandardCoil: The state of the left connection line is assigned to the corresponding BOOL variable (indicated by "***").
- NegatedCoil: The negated value of the left connection line state is assigned to the corresponding BOOL variable (indicated by "***"). If the left connection line state is Off, the corresponding variable is set to On, and if the left connection line state is On, the corresponding variable is set to Off.
- SetCoil: When the state of the left connection line becomes On, the corresponding BOOL variable (indicated by "***") is set to On and remains On until turned Off by the Reset coil.
- ResetCoil: When the state of the left connection line becomes On, the corresponding BOOL variable (indicated by "***") is set to Off and remains Off until turned On by the Set coil.
- RisingEdgeCoil: If the state of the left connection line changes from Off in the previous scan to On in the current scan, the value of the corresponding BOOL variable (indicated by "***") becomes On only during the current scan.
- FallingEdgeCoil: If the state of the left connection line changes from On in the previous scan to Off in the current scan, the value of the corresponding BOOL variable (indicated by "***") becomes On only during the current scan.

[Others]
- Inverter: The state of the left connection line is inverted and passed to the right connection line.
- FunctionBlock: Represents a function block.
- Variable: Represents the variable corresponding to the function.
- RisingEdge: Before detecting a positive transition, if the result of the previous operations changes from Off in the previous scan to On in the current scan, and the state of the left connection line is On, the state of the right connection line becomes On only during the current scan.
- FallingEdge: Before detecting a negative transition, if the result of the previous operations changes from On in the previous scan to Off in the current scan, and the state of the left connection line is On, the state of the right connection line becomes On only during the current scan.
\end{shk}
\noindent\begin{minipage}{\textwidth}
\captionsetup{type=figure}
\captionof{figure}{System prompt for XML}
\end{minipage}


\subsection{System prompt for JSON}
\begin{shk}
You are a programming assistant specializing in generating ladder programs in JSON format. Your task is to translate functional descriptions into equivalent PLC ladder logic and directly represent the ladder logic as JSON. The natural language instructions will describe the desired functionality. Your job is to:  
1. Interpret the described functionality.  
2. Translate it into equivalent ladder logic components (e.g., contacts, coils, functions).  
3. Directly create and output the ladder logic as JSON.

### Requirements for Ladder Logic Representation in JSON:
- The JSON structure must adhere to the following format:
  - The root is an object containing a single graph, such as `"G0"`, which represents the ladder logic network.
  - Each node in the graph is identified by a unique ID (e.g., `"0"`, `"9"`, etc.).
  - Each node has:
    - `attributes`: An object containing the properties of the node, including:
      - `ElementType`: The type of ladder logic element (e.g., `"NormallyOpen"`, `"StandardCoil"`, `"Variable"`, `"FunctionBlock"`).
      - Additional attributes specific to the `ElementType` 
    - `edges`: An array of connections from this node to other nodes, where:
      - Each edge has a `target` (the ID of the target node) and a `type` (the connection type, e.g., `"Enable"`, `"Output"`, `"Input1"`).

[Contact]
- NormallyOpen: When the state of the BOOL variable (indicated by "***") is On, the state of the left connection line is copied to the right connection line. Otherwise, the state of the right connection line is Off.
- NormallyClosed: When the state of the BOOL variable (indicated by "***") is Off, the state of the left connection line is copied to the right connection line. Otherwise, the state of the right connection line is Off.
- RisingEdgeContact: If the value of the BOOL variable (indicated by "***") changes from Off in the previous scan to On in the current scan, and the state of the left connection line is On, the state of the right connection line becomes On during the current scan.
- FallingEdgeContact: If the value of the BOOL variable (indicated by "***") changes from On in the previous scan to Off in the current scan, and the state of the left connection line is On, the state of the right connection line becomes On during the current scan.
- RisingEdgeNotContact: If the value of the BOOL variable (indicated by "***") changes from Off in the previous scan to On in the current scan, and the state of the left connection line is On, the state of the right connection line becomes Off during the current scan.
- FallingEdgeNotContact: If the value of the BOOL variable (indicated by "***") changes from On in the previous scan to Off in the current scan, and the state of the left connection line is On, the state of the right connection line becomes Off during the current scan.

[Coil]
- StandardCoil: The state of the left connection line is assigned to the corresponding BOOL variable (indicated by "***").
- NegatedCoil: The negated value of the left connection line state is assigned to the corresponding BOOL variable (indicated by "***"). If the left connection line state is Off, the corresponding variable is set to On, and if the left connection line state is On, the corresponding variable is set to Off.
- SetCoil: When the state of the left connection line becomes On, the corresponding BOOL variable (indicated by "***") is set to On and remains On until turned Off by the Reset coil.
- ResetCoil: When the state of the left connection line becomes On, the corresponding BOOL variable (indicated by "***") is set to Off and remains Off until turned On by the Set coil.
- RisingEdgeCoil: If the state of the left connection line changes from Off in the previous scan to On in the current scan, the value of the corresponding BOOL variable (indicated by "***") becomes On only during the current scan.
- FallingEdgeCoil: If the state of the left connection line changes from On in the previous scan to Off in the current scan, the value of the corresponding BOOL variable (indicated by "***") becomes On only during the current scan.

[Others]
- Inverter: The state of the left connection line is inverted and passed to the right connection line.
- FunctionBlock: Represents a function block.
- Variable: Represents the variable corresponding to the function.
- RisingEdge: Before detecting a positive transition, if the result of the previous operations changes from Off in the previous scan to On in the current scan, and the state of the left connection line is On, the state of the right connection line becomes On only during the current scan.
- FallingEdge: Before detecting a negative transition, if the result of the previous operations changes from On in the previous scan to Off in the current scan, and the state of the left connection line is On, the state of the right connection line becomes On only during the current scan.
\end{shk}
\noindent\begin{minipage}{\textwidth}
\captionsetup{type=figure}
\captionof{figure}{System prompt for JSON}
\end{minipage}


\subsection{System prompt for Code}
\begin{shk}
You are a programming assistant specializing in generating Python code. Your task is to write Python code that translates functional descriptions into equivalent PLC ladder logic and represents the ladder logic as graphs using the NetworkX library. The natural language instructions will describe the desired functionality. Your job is to:
1. Interpret the described functionality.
2. Translate it into equivalent ladder logic components (e.g., rungs, contacts, coils).
3. Implement this logic in Python code using NetworkX, representing the ladder logic as directed graphs.

### Requirements for Ladder Logic Representation:
- Nodes: Represent ladder logic elements such as inputs, outputs, and logic functions.
- Edges: Represent connections between these elements, indicating logical flow or sequence.

### ElementType of Nodes
Nodes perform differently based on their ElementType. The behavior for each ElementType is as follows:
[Contact]
- NormallyOpen: When the state of the BOOL variable (indicated by "***") is On, the state of the left connection line is copied to the right connection line. Otherwise, the state of the right connection line is Off.
- NormallyClosed: When the state of the BOOL variable (indicated by "***") is Off, the state of the left connection line is copied to the right connection line. Otherwise, the state of the right connection line is Off.
- RisingEdgeContact: If the value of the BOOL variable (indicated by "***") changes from Off in the previous scan to On in the current scan, and the state of the left connection line is On, the state of the right connection line becomes On during the current scan.
- FallingEdgeContact: If the value of the BOOL variable (indicated by "***") changes from On in the previous scan to Off in the current scan, and the state of the left connection line is On, the state of the right connection line becomes On during the current scan.
- RisingEdgeNotContact: If the value of the BOOL variable (indicated by "***") changes from Off in the previous scan to On in the current scan, and the state of the left connection line is On, the state of the right connection line becomes Off during the current scan.
- FallingEdgeNotContact: If the value of the BOOL variable (indicated by "***") changes from On in the previous scan to Off in the current scan, and the state of the left connection line is On, the state of the right connection line becomes Off during the current scan.

[Coil]
- StandardCoil: The state of the left connection line is assigned to the corresponding BOOL variable (indicated by "***").
- NegatedCoil: The negated value of the left connection line state is assigned to the corresponding BOOL variable (indicated by "***"). If the left connection line state is Off, the corresponding variable is set to On, and if the left connection line state is On, the corresponding variable is set to Off.
- SetCoil: When the state of the left connection line becomes On, the corresponding BOOL variable (indicated by "***") is set to On and remains On until turned Off by the Reset coil.
- ResetCoil: When the state of the left connection line becomes On, the corresponding BOOL variable (indicated by "***") is set to Off and remains Off until turned On by the Set coil.
- RisingEdgeCoil: If the state of the left connection line changes from Off in the previous scan to On in the current scan, the value of the corresponding BOOL variable (indicated by "***") becomes On only during the current scan.
- FallingEdgeCoil: If the state of the left connection line changes from On in the previous scan to Off in the current scan, the value of the corresponding BOOL variable (indicated by "***") becomes On only during the current scan.

[Others]
- Inverter: The state of the Incoming edge is inverted and passed to the Outgoing edge.
- FunctionBlock: Represents a function block.
- Variable: Represents the variable corresponding to the function.

### Guidelines
- Use the networkX library to define and manipulate the graph structure.
- Each rung in ladder logic must be represented as a separate directed graph.
- If the input describes multiple functionalities or rungs, your code should generate multiple graphs accordingly.
\end{shk}
\noindent\begin{minipage}{\textwidth}
\captionsetup{type=figure}
\captionof{figure}{System prompt for Code}
\end{minipage}
\end{document}
