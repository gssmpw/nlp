% This must be in the first 5 lines to tell arXiv to use pdfLaTeX, which is strongly recommended.
\pdfoutput=1
% In particular, the hyperref package requires pdfLaTeX in order to break URLs across lines.

\documentclass[11pt]{article}

% Change "review" to "final" to generate the final (sometimes called camera-ready) version.
% Change to "preprint" to generate a non-anonymous version with page numbers.
\usepackage[preprint]{acl}

% Standard package includes
\usepackage{times}
\usepackage{latexsym}
\usepackage{kotex}
\usepackage{multicol}
\usepackage{lipsum}
\usepackage{verbatim}
\usepackage{multirow}
\usepackage{xcolor}  % 색상 관련 패키지
\usepackage{colortbl} % 테이블 배경색 패키지

% For proper rendering and hyphenation of words containing Latin characters (including in bib files)
\usepackage[T1]{fontenc}
% For Vietnamese characters
% \usepackage[T5]{fontenc}
% See https://www.latex-project.org/help/documentation/encguide.pdf for other character sets

% This assumes your files are encoded as UTF8
\usepackage[utf8]{inputenc}

% This is not strictly necessary, and may be commented out,
% but it will improve the layout of the manuscript,
% and will typically save some space.
\usepackage{microtype}

% This is also not strictly necessary, and may be commented out.
% However, it will improve the aesthetics of text in
% the typewriter font.
\usepackage{inconsolata}

%Including images in your LaTeX document requires adding
%additional package(s)
\usepackage{graphicx}
\usepackage{amsmath}
\usepackage{amsfonts}
\usepackage{algorithm}
\usepackage{algorithmic}

\usepackage{geometry}
\usepackage{longtable}
\usepackage{array}
\usepackage{caption}
\usepackage{booktabs}
\usepackage[most]{tcolorbox}
\usepackage{makecell}  % For better multi-line cell support
\usepackage{enumitem}
\usepackage{amssymb}


% If the title and author information does not fit in the area allocated, uncomment the following
%
%\setlength\titlebox{<dim>}
%
% and set <dim> to something 5cm or larger.

% \title{An Investigation into LLM Fine-Tuning for Visual Code Generation: Ladder Diagram as a Case Study}
\title{Retrieval-Augmented Fine-Tuning With Preference Optimization For Visual Program Generation}


% Author information can be set in various styles:
% For several authors from the same institution:
% \author{Author 1 \and ... \and Author n \\
%         Address line \\ ... \\ Address line}
% if the names do not fit well on one line use
%         Author 1 \\ {\bf Author 2} \\ ... \\ {\bf Author n} \\
% For authors from different institutions:
% \author{Author 1 \\ Address line \\  ... \\ Address line
%         \And  ... \And
%         Author n \\ Address line \\ ... \\ Address line}
% To start a separate ``row'' of authors use \AND, as in
% \author{Author 1 \\ Address line \\  ... \\ Address line
%         \AND
%         Author 2 \\ Address line \\ ... \\ Address line \And
%         Author 3 \\ Address line \\ ... \\ Address line}

% \author{First Author \\
%   Affiliation / Address line 1 \\
%   Affiliation / Address line 2 \\
%   Affiliation / Address line 3 \\
%   \texttt{email@domain} \\\And
%   Second Author \\
%   Affiliation / Address line 1 \\
%   Affiliation / Address line 2 \\
%   Affiliation / Address line 3 \\
%   \texttt{email@domain} \\}

\author{
 \textbf{Deokhyung Kang\textsuperscript{*1}}, 
 \textbf{Jeonghun Cho\textsuperscript{*1}}, 
 \textbf{Yejin Jeon\textsuperscript{1}}, 
 \textbf{Sunbin Jang\textsuperscript{3}}, 
\\
 \textbf{Minsub Lee\textsuperscript{3}}, 
 \textbf{Jawoon Cho\textsuperscript{4}}, 
 \textbf{Gary Geunbae Lee\textsuperscript{1,2}}
\\
 \textsuperscript{1}Graduate School of Artificial Intelligence, POSTECH,\\
 \textsuperscript{2}Department of Computer Science and Engineering, POSTECH,\\
 \textsuperscript{3}Hyundai Mobis,
 \textsuperscript{4}T\&I Company
\\
 \texttt{\{deokhk, jeonghuncho, jeonyj0612, gblee\}@postech.ac.kr}\\ \texttt{\{soonbin.Jang, perfect\}@mobis.com}, 
 \texttt{jwcho@tnicompany.com}\\
}

\newenvironment{shk}{%
    \tcblisting{
        enhanced,
        breakable,
        listing only,
        colback=gray!10,
        colframe=black,
        top=2mm,
        bottom=2mm,
        boxrule=0.5pt,
        before skip=10pt,
        after skip=10pt,
        after={\par\vspace{0.5\baselineskip}\noindent},
    }
}
{\endtcblisting}


\begin{document}
\maketitle
\def\thefootnote{*}\footnotetext{Equally contributed}\def\thefootnote{\arabic{footnote}}
\begin{abstract}


The choice of representation for geographic location significantly impacts the accuracy of models for a broad range of geospatial tasks, including fine-grained species classification, population density estimation, and biome classification. Recent works like SatCLIP and GeoCLIP learn such representations by contrastively aligning geolocation with co-located images. While these methods work exceptionally well, in this paper, we posit that the current training strategies fail to fully capture the important visual features. We provide an information theoretic perspective on why the resulting embeddings from these methods discard crucial visual information that is important for many downstream tasks. To solve this problem, we propose a novel retrieval-augmented strategy called RANGE. We build our method on the intuition that the visual features of a location can be estimated by combining the visual features from multiple similar-looking locations. We evaluate our method across a wide variety of tasks. Our results show that RANGE outperforms the existing state-of-the-art models with significant margins in most tasks. We show gains of up to 13.1\% on classification tasks and 0.145 $R^2$ on regression tasks. All our code and models will be made available at: \href{https://github.com/mvrl/RANGE}{https://github.com/mvrl/RANGE}.

\end{abstract}


\section{Introduction}

Video generation has garnered significant attention owing to its transformative potential across a wide range of applications, such media content creation~\citep{polyak2024movie}, advertising~\citep{zhang2024virbo,bacher2021advert}, video games~\citep{yang2024playable,valevski2024diffusion, oasis2024}, and world model simulators~\citep{ha2018world, videoworldsimulators2024, agarwal2025cosmos}. Benefiting from advanced generative algorithms~\citep{goodfellow2014generative, ho2020denoising, liu2023flow, lipman2023flow}, scalable model architectures~\citep{vaswani2017attention, peebles2023scalable}, vast amounts of internet-sourced data~\citep{chen2024panda, nan2024openvid, ju2024miradata}, and ongoing expansion of computing capabilities~\citep{nvidia2022h100, nvidia2023dgxgh200, nvidia2024h200nvl}, remarkable advancements have been achieved in the field of video generation~\citep{ho2022video, ho2022imagen, singer2023makeavideo, blattmann2023align, videoworldsimulators2024, kuaishou2024klingai, yang2024cogvideox, jin2024pyramidal, polyak2024movie, kong2024hunyuanvideo, ji2024prompt}.


In this work, we present \textbf{\ours}, a family of rectified flow~\citep{lipman2023flow, liu2023flow} transformer models designed for joint image and video generation, establishing a pathway toward industry-grade performance. This report centers on four key components: data curation, model architecture design, flow formulation, and training infrastructure optimization—each rigorously refined to meet the demands of high-quality, large-scale video generation.


\begin{figure}[ht]
    \centering
    \begin{subfigure}[b]{0.82\linewidth}
        \centering
        \includegraphics[width=\linewidth]{figures/t2i_1024.pdf}
        \caption{Text-to-Image Samples}\label{fig:main-demo-t2i}
    \end{subfigure}
    \vfill
    \begin{subfigure}[b]{0.82\linewidth}
        \centering
        \includegraphics[width=\linewidth]{figures/t2v_samples.pdf}
        \caption{Text-to-Video Samples}\label{fig:main-demo-t2v}
    \end{subfigure}
\caption{\textbf{Generated samples from \ours.} Key components are highlighted in \textcolor{red}{\textbf{RED}}.}\label{fig:main-demo}
\end{figure}


First, we present a comprehensive data processing pipeline designed to construct large-scale, high-quality image and video-text datasets. The pipeline integrates multiple advanced techniques, including video and image filtering based on aesthetic scores, OCR-driven content analysis, and subjective evaluations, to ensure exceptional visual and contextual quality. Furthermore, we employ multimodal large language models~(MLLMs)~\citep{yuan2025tarsier2} to generate dense and contextually aligned captions, which are subsequently refined using an additional large language model~(LLM)~\citep{yang2024qwen2} to enhance their accuracy, fluency, and descriptive richness. As a result, we have curated a robust training dataset comprising approximately 36M video-text pairs and 160M image-text pairs, which are proven sufficient for training industry-level generative models.

Secondly, we take a pioneering step by applying rectified flow formulation~\citep{lipman2023flow} for joint image and video generation, implemented through the \ours model family, which comprises Transformer architectures with 2B and 8B parameters. At its core, the \ours framework employs a 3D joint image-video variational autoencoder (VAE) to compress image and video inputs into a shared latent space, facilitating unified representation. This shared latent space is coupled with a full-attention~\citep{vaswani2017attention} mechanism, enabling seamless joint training of image and video. This architecture delivers high-quality, coherent outputs across both images and videos, establishing a unified framework for visual generation tasks.


Furthermore, to support the training of \ours at scale, we have developed a robust infrastructure tailored for large-scale model training. Our approach incorporates advanced parallelism strategies~\citep{jacobs2023deepspeed, pytorch_fsdp} to manage memory efficiently during long-context training. Additionally, we employ ByteCheckpoint~\citep{wan2024bytecheckpoint} for high-performance checkpointing and integrate fault-tolerant mechanisms from MegaScale~\citep{jiang2024megascale} to ensure stability and scalability across large GPU clusters. These optimizations enable \ours to handle the computational and data challenges of generative modeling with exceptional efficiency and reliability.


We evaluate \ours on both text-to-image and text-to-video benchmarks to highlight its competitive advantages. For text-to-image generation, \ours-T2I demonstrates strong performance across multiple benchmarks, including T2I-CompBench~\citep{huang2023t2i-compbench}, GenEval~\citep{ghosh2024geneval}, and DPG-Bench~\citep{hu2024ella_dbgbench}, excelling in both visual quality and text-image alignment. In text-to-video benchmarks, \ours-T2V achieves state-of-the-art performance on the UCF-101~\citep{ucf101} zero-shot generation task. Additionally, \ours-T2V attains an impressive score of \textbf{84.85} on VBench~\citep{huang2024vbench}, securing the top position on the leaderboard (as of 2025-01-25) and surpassing several leading commercial text-to-video models. Qualitative results, illustrated in \Cref{fig:main-demo}, further demonstrate the superior quality of the generated media samples. These findings underscore \ours's effectiveness in multi-modal generation and its potential as a high-performing solution for both research and commercial applications.
\begin{figure}
    \centering
    \includegraphics[width=\columnwidth, keepaspectratio]{latex/figure/format.pdf}
    \caption{The topmost subfigure shows a single rung of LD and its corresponding visualized graph. The bottom displays XML tags exportable from a Ladder Diagram IDE, along with JSON and Metaprogram representations that capture structural relationships in the graph.}
    \label{fig:format}
\end{figure}


\begin{figure*}[!th]
    \centering
    \includegraphics[width=0.8\textwidth]{latex/figure/rag_performance_comparison.pdf} 
    \caption{Performance comparison between SFT and RAG, where RAG uses a larger LLM. $N$ represents the number of retrieved examples in RAG, and SFT’s performance is represented by a red dotted line. We use XML as the text format.}
    \label{fig:rag_sft_performance}
\end{figure*}
\begin{table*}[!ht]
\small
\centering
\resizebox{0.8\textwidth}{!}{%
\begin{tabular}{llccccc}
\toprule
Format                       & Method   & Node F1 $\uparrow$      & Edge F1 $\uparrow$      & Node EM $\uparrow$     & Edge EM $\uparrow$       & Program EM $\uparrow$    \\ \midrule
\multirow{2}{*}{XML}         & RAG & 74.4          & 67.1          & 49.6          & 49.2          & 49.2          \\
                             & SFT      & \textbf{84.8} & \textbf{74.7} & \textbf{55.2} & \textbf{54.8} & \textbf{54.8} \\ \midrule
\multirow{2}{*}{JSON}        & RAG & 72.6          & 65.7          & 51.6          & 50.8          & 50.8          \\
                             & SFT      & \textbf{85.1} & \textbf{74.9} & \textbf{53.6} & \textbf{52.6} & \textbf{52.6} \\ \midrule
\multirow{2}{*}{Metaprogram} & RAG & 80.0            & 71.9          & 52.2          & 51.8          & 51.8          \\
                             & SFT      & \textbf{86.3} & \textbf{75.7} & \textbf{55.2} & \textbf{54.0}   & \textbf{54.0}   \\ \bottomrule
\end{tabular}
}
\caption{Performance across different text formats. The highest values in each format are \textbf{in bold}.}
\label{tab:text_format}
\end{table*}


\section{Is Training-Based Approach Effective?}
In this section, we investigate the effectiveness of the training-based approach in comparison to the prompting-based approach for industrial VPL generation. We introduce Ladder Diagram (LD) as our test language (\S\ref{sec:ld}), then describe text format conversion in \S\ref{sec:conversion} to explore results across different text formats. Their results are presented in \S\ref{sec:sft_res}.
\subsection{Ladder Diagram}\label{sec:ld}
Programmable Logic Controller (PLC)~\citep{erickson1996programmable} controls physical devices such as sensors and actuators in industrial automation systems. For example, in a conveyor system, PLC logic enables a robotic arm to detect products via sensors and control its joints to transfer items between conveyors. This logic is typically implemented using Ladder Diagram (LD)~\citep{ladderlogic}, which consists of multiple rungs, each executing a specific task (e.g., operating a motor). Figure~\ref{fig:format} illustrates an LD with a single rung.

Each rung is further composed of contacts, coils, and function blocks and can be represented as a node-based graph, where each visual element is a node. These individual elements serve a distinct role in the logic sequence; contacts control the power flow, coils activate machinery, and function blocks (e.g., timers, counters) provide advanced control functions. They are connected to the PLC and mapped to I/O addresses that vary with the environment and hardware setup. This domain-specific address mapping enables customized control, as each rung manages unique I/O settings tailored to its setup.

\subsection{Text Format Conversion}\label{sec:conversion}
As LLMs cannot generate complex visual elements directly, prior studies~\cite{xue2024comfybenchbenchmarkingllmbasedagents,zhang2024benchmarking} generate VPL in text formats. As each format has unique characteristics, we convert VPL into various text formats to conduct comprehensive experiments. We consider three standard text formats for VPL: (1) XML, which represents visual elements sequentially; (2) JSON, which explicitly captures relationships between elements; (3) Metaprogram (code), which encodes VPL using a code-based syntax. 

Specifically, Figure~\ref{fig:format} illustrates that XML format represents LD as a list of visual elements, including contacts, lines, and function blocks. Each visual element corresponds to an \texttt{<Element>} tag containing its element type, coordinates, and other attributes. Since XML does not explicitly define relationships between elements, we extract these relationships using rule-based methods and convert them into a graph. We iterate through the visual elements in coordinate-based order, adding all elements except lines as nodes to the graph. We implement this graph using NetworkX~\cite{SciPyProceedings_11}, which is a widely used Python library for graph representation. We assign Node IDs starting from 0, increasing sequentially in coordinate order. To determine edges, we analyze node coordinates and line positions. Using this information, we construct a directed acyclic graph (DAG).
 
Using this graph representation of LD, we represent LD in both JSON and metaprogram formats. In the JSON format, we represent the LD as a dictionary; each node ID in the graph is a key, and its attributes and outgoing node IDs form the values. For the metaprogram (code) format, we represent the LD in Python syntax using NetworkX. We traverse the graph starting from the smallest node ID, visiting each node and its successors. Upon first visiting a node, we append a \texttt{G.add\_node(...)} statement with its attributes to the code. Similarly, we append a \texttt{G.add\_edge(...)} statement to the code for each neighbor relationship. This process continues until all nodes from the graph are visited. The generated code can be executed to reconstruct the original graph.

\begin{figure*}[!htb]
    \centering
    \includegraphics[width=\textwidth,keepaspectratio]{latex/figure/main_figure.pdf}
    \caption{An overview of two-stage training method. (1) RAFT-V: An off-the-shelf retriever is utilized for relevant prompt augmentation, and training is conducted with cross-entropy loss. (2) Preference Optimization: Preference learning leverages graph-edited preference pairs, with retrieved prompt-code pairs as additional input.}
    \label{fig:main_figure}
\end{figure*}

\subsection{Results}\label{sec:sft_res}
\paragraph{Training-based method outperforms prompting-based method}\label{par:sft_rag}
To evaluate the effectiveness of the training-based approach compared to the prompting-based approach for industrial VPL generation, we conduct a comparison between the supervised fine-tuning (SFT) method and Retrieval-Augmented Generation (RAG). Among prompting-based methods, we select RAG because this approach has been widely used in prior studies on VPL generation~\cite{xue2024comfybenchbenchmarkingllmbasedagents,zhang2024benchmarking}, and retrieval can provide examples that reflect the domain-specific configurations relevant to generation. SFT uses Llama3.1-8B-Instruct model as the backbone, while RAG uses Llama-3.1-70B-Instruct\footnote{Due to computational constraints, we employs the AWQ-quantized~\cite{MLSYS2024_42a452cb} model: \href{https://huggingface.co/hugging-quants/Meta-Llama-3.1-70B-Instruct-AWQ-INT4}{hugging-quants/Meta-Llama-3.1-70B-Instruct-AWQ-INT4}}. The RAG approach utilizes BM25~\cite{robertson2009probabilistic} to retrieve \(N\) similar prompt-code pairs from the SFT training dataset. We then append them to the input to generate code for the test prompt\footnote{Detailed experimental settings, including evaluation metrics, are presented in \S\ref{sec:exp_setup}.}.

As can be seen in Figure~\ref{fig:rag_sft_performance}, SFT consistently outperforms RAG across different number of retrieved examples (1, 3, 5, 7, and 9), despite its smaller backbone. Although RAG's performance improves as \(N\) increases, it eventually degrades, suggesting that learning domain-specific configurations in industrial VPL from retrieved examples alone is challenging. Therefore, these results demonstrate that SFT is an effective choice in this setting. We observe similar trends with a different LLM (Appendix~\ref{sec:detailed_results}), and the prompt template for the RAG model is provided in Appendix~\ref{sec:rag_details}. 


\paragraph{Training-based method excels across text formats}\label{par:sft_text_format}
To investigate whether the prior results are consistent across different text formats, we extend our experiments to include JSON and the metaprogram formats. We compare SFT with RAG, where performance is reported based on the number of retrieved examples that achieves the highest Program EM\footnote{Program EM is a binary metric (0 or 1) that indicates an exact match between the generated and reference programs. See \S\ref{sec:exp_setup} for details.} for each text format, as the optimal number of examples can vary across formats. 

From the results in Table~\ref{tab:text_format}, we can observe the following: (1) While SFT shows stable performance across different formats, RAG exhibits performance differences in Node/Edge F1, with the metaprogram-based format showing better performance. This aligns with prior results on prompting-based approaches from \citet{xue2024comfybenchbenchmarkingllmbasedagents}. (2) SFT outperforms RAG across all formats. These results further validate the effectiveness of the training-based approach.

\section{Study Design}
% robot: aliengo 
% We used the Unitree AlienGo quadruped robot. 
% See Appendix 1 in AlienGo Software Guide PDF
% Weight = 25kg, size (L,W,H) = (0.55, 0.35, 06) m when standing, (0.55, 0.35, 0.31) m when walking
% Handle is 0.4 m or 0.5 m. I'll need to check it to see which type it is.
We gathered input from primary stakeholders of the robot dog guide, divided into three subgroups: BVI individuals who have owned a dog guide, BVI individuals who were not dog guide owners, and sighted individuals with generally low degrees of familiarity with dog guides. While the main focus of this study was on the BVI participants, we elected to include survey responses from sighted participants given the importance of social acceptance of the robot by the general public, which could reflect upon the BVI users themselves and affect their interactions with the general population \cite{kayukawa2022perceive}. 

The need-finding processes consisted of two stages. During Stage 1, we conducted in-depth interviews with BVI participants, querying their experiences in using conventional assistive technologies and dog guides. During Stage 2, a large-scale survey was distributed to both BVI and sighted participants. 

This study was approved by the University’s Institutional Review Board (IRB), and all processes were conducted after obtaining the participants' consent.

\subsection{Stage 1: Interviews}
We recruited nine BVI participants (\textbf{Table}~\ref{tab:bvi-info}) for in-depth interviews, which lasted 45-90 minutes for current or former dog guide owners (DO) and 30-60 minutes for participants without dog guides (NDO). Group DO consisted of five participants, while Group NDO consisted of four participants.
% The interview participants were divided into two groups. Group DO (Dog guide Owner) consisted of five participants who were current or former dog guide owners and Group NDO (Non Dog guide Owner) consisted of three participants who were not dog guide owners. 
All participants were familiar with using white canes as a mobility aid. 

We recruited participants in both groups, DO and NDO, to gather data from those with substantial experience with dog guides, offering potentially more practical insights, and from those without prior experience, providing a perspective that may be less constrained and more open to novel approaches. 

We asked about the participants' overall impressions of a robot dog guide, expectations regarding its potential benefits and challenges compared to a conventional dog guide, their desired methods of giving commands and communicating with the robot dog guide, essential functionalities that the robot dog guide should offer, and their preferences for various aspects of the robot dog guide's form factors. 
For Group DO, we also included questions that asked about the participants' experiences with conventional dog guides. 

% We obtained permission to record the conversations for our records while simultaneously taking notes during the interviews. The interviews lasted 30-60 minutes for NDO participants and 45-90 minutes for DO participants. 

\subsection{Stage 2: Large-Scale Surveys} 
After gathering sufficient initial results from the interviews, we created an online survey for distributing to a larger pool of participants. The survey platform used was Qualtrics. 

\subsubsection{Survey Participants}
The survey had 100 participants divided into two primary groups. Group BVI consisted of 42 blind or visually impaired participants, and Group ST consisted of 58 sighted participants. \textbf{Table}~\ref{tab:survey-demographics} shows the demographic information of the survey participants. 

\subsubsection{Question Differentiation} 
Based on their responses to initial qualifying questions, survey participants were sorted into three subgroups: DO, NDO, and ST. Each participant was assigned one of three different versions of the survey. The surveys for BVI participants mirrored the interview categories (overall impressions, communication methods, functionalities, and form factors), but with a more quantitative approach rather than the open-ended questions used in interviews. The DO version included additional questions pertaining to their prior experience with dog guides. The ST version revolved around the participants' prior interactions with and feelings toward dog guides and dogs in general, their thoughts on a robot dog guide, and broad opinions on the aesthetic component of the robot's design. 

\begin{table*}[ht]
\tiny
\centering
\resizebox{0.9\textwidth}{!}{%
\begin{tabular}{llccccc}
\toprule
Format & Method & \multicolumn{1}{l}{Node F1} & \multicolumn{1}{l}{Edge F1} & \multicolumn{1}{l}{Node EM} & \multicolumn{1}{l}{Edge EM} & \multicolumn{1}{l}{Program EM} \\ \midrule
 & SFT & 84.8 / 79.0 & 74.7 / 68.0 & 55.2 / 46.4 & 54.8 / 45.6 & 54.8 / 45.6 \\
 & RAFT-V & 88.1 / \textbf{87.4} & 80.9 / 78.7 & 63.8 / 59.6 & 62.8 / 58.8 & 62.8 / 58.8 \\
\multirow{-3}{*}{XML} & \cellcolor[HTML]{EFEFEF}Ours & \cellcolor[HTML]{EFEFEF}\textbf{89.6 / 87.4} & \cellcolor[HTML]{EFEFEF}\textbf{82.6 / 79.0} & \cellcolor[HTML]{EFEFEF}\textbf{66.2 / 61.2} & \cellcolor[HTML]{EFEFEF}\textbf{65.2 / 60.0} & \cellcolor[HTML]{EFEFEF}\textbf{65.2 / 60.0} \\ \midrule
 & SFT & 85.1 / 81.2 & 74.9 / 69.3 & 53.6 / 46.0 & 52.6 / 45.4 & 52.6 / 45.4 \\
 & RAFT-V & 90.1 / 88.4 & 82.0 / 79.6 & 63.8 / 59.2 & 63.2 / 57.4 & 63.2 / 57.4 \\
\multirow{-3}{*}{JSON} & \cellcolor[HTML]{EFEFEF}Ours & \cellcolor[HTML]{EFEFEF}\textbf{90.7 / 88.7} & \cellcolor[HTML]{EFEFEF}\textbf{83.0 / 80.5} & \cellcolor[HTML]{EFEFEF}\textbf{66.0 / 61.6} & \cellcolor[HTML]{EFEFEF}\textbf{65.2 / 60.2} & \cellcolor[HTML]{EFEFEF}\textbf{65.2 / 60.2} \\ \midrule
 & SFT & 86.3 / 81.7 & 75.7 / 70.2 & 55.2 / 47.6 & 54.0 / 47.2 & 54.0 / 47.2 \\
 & RAFT-V & 89.9 / 89.1 & 82.4 / 80.4 & 64.6 / 60.6 & 63.8 / 59.6 & 63.8 / 59.6 \\
\multirow{-3}{*}{Metaprogram} & \cellcolor[HTML]{EFEFEF}Ours & \cellcolor[HTML]{EFEFEF}\textbf{90.6 / 89.7} & \cellcolor[HTML]{EFEFEF}\textbf{83.6 / 81.6} & \cellcolor[HTML]{EFEFEF}\textbf{68.6 / 63.2} & \cellcolor[HTML]{EFEFEF}\textbf{67.2 / 61.8} & \cellcolor[HTML]{EFEFEF}\textbf{67.2 / 61.8} \\ \bottomrule
\end{tabular}%
}
\caption{Main results of the 2-stage training strategy. Each score is presented in the order Llama-3.1-8B-Instruct / Qwen2.5-7B-Instruct.}
\label{tab:main_result}
\end{table*}


\section{Experimental Settings}\label{sec:exp_setup}
\paragraph{Dataset} Due to the lack of publicly available datasets, we created our own by annotating ladder diagrams from actual production environments. Using XG5000~\cite{XG5000Manual}, we exported these diagrams as XML files and divided them into functional units, where each unit consists of one or more interconnected rungs designed to perform a specific function. An experienced PLC programmer then annotated natural language instructions for each unit. The dataset was randomly split into training, validation, and test sets of 13,124, 500, and 500 samples. 80\% of the training data was used for SFT, while the remaining 20\% was used for preference learning. Appendix~\ref{sec:data_samples} shows an example of the dataset utilized.

\paragraph{Implementation details} We utilize Llama-3.1-8B-Instruct~\cite{dubey2024llama} as the main backbone model and also use Qwen2.5-7B-Instruct~\cite{yang2024qwen2} to assess the generalizability of our method. To facilitate task understanding, we provide a detailed task description and explanations of the visual elements used in ladder diagrams via the system prompt. For models using retrieval, we utilize BM25~\cite{robertson2009probabilistic}, which is a widely used lexical matching-based method. Using BM25, we augment the input to these retrieval-based models with a top-1 retrieved prompt-code pair from the training dataset\footnote{System prompts are in Appendix~\ref{sec:system_prompt}, and further implementation details are in Appendix~\ref{sec:implementation_details}.}.

\paragraph{Evaluation metrics}
Text-based programming languages like Python are typically evaluated for correctness by unit tests~\cite{chen2021evaluating,austin2021programsynthesislargelanguage, chen2023codet}. In contrast, industrial visual programming languages often rely on separate simulators for evaluation, which limit consistent automated assessment across program variations and diverse environments~\cite{ray2017survey, ren2024infiniteworldunifiedscalablesimulation}.

To address this, we introduce graph-based automatic evaluation by transforming visually structured programs into NetworkX, as introduced in \S\ref{sec:conversion}. Specifically, we measure partial correctness in the graph representation by comparing it with the ground-truth graph using Node F1 and Edge F1 at the node and edge levels. For exact matches, we use Node EM and Edge EM to assess complete accuracy at the node and edge levels. Finally, Program EM evaluates whether the entire graph matches the ground-truth graph precisely in terms of both nodes and edges. Detailed explanations are provided in Appendix~\ref{sec:metric}.
\section{\label{sec:result}Results \& Discussion}

\subsection{\label{sec:overall_performance}Main Results}
\paragraph{Stage 1 (RAFT-V) improves upon SFT} Table~\ref{tab:main_result} compares the performance of SFT, RAFT-V, and our two-stage method across different output formats (XML, JSON, metaprogram) with two backbone models. By incorporating retrieval augmentation, RAFT-V consistently outperforms SFT and significantly enhances VPL generation. For example, in the JSON format, RAFT-V raises the Program EM from 52.6\% to 63.2\%, demonstrating more accurate VPL generation. Similar improvements are observed for other metrics.

\paragraph{Preference optimization (Stage 2) yields further gains} Building on Stage 1's improvements, Stage 2 further boosts correctness through preference optimization, particularly in terms of exact match (EM) scores. As shown in Table~\ref{tab:main_result}, our two-stage approach improves Program EM over RAFT-V by 3.4\% in the metaprogram format and achieves an overall 13.2\% gain compared to SFT. Because EM only assigns a score when the entire graph exactly matches, even minor generation failures can significantly impact the metric. These results show that preference optimization reduces such minor generation errors and ensures more precise outputs. Notably, these gains are consistent across all output formats and base models, demonstrating the robustness of our approach.\footnote{See Appendix~\ref{sec:human_eval} for human evaluation details.}

\subsection{\label{sec:retrieved_examples}Impact of Number of Retrieved Examples}
\begin{table}[!t]
\centering
\resizebox{\columnwidth}{!}{%
\begin{tabular}{@{}lccccc@{}}
\toprule
\( k \) & Node F1 & Edge F1 & Node EM & Edge EM & Program EM \\ \midrule
\rowcolor[HTML]{EFEFEF}
1 (Ours)      & 89.9    & 82.4    & 64.6    & 63.8    & 63.8     \\
2       & 90.6    & 83.5    & \textbf{66.4} & \textbf{65.6} & \textbf{65.6} \\
3       & \textbf{90.9} & \textbf{84.0} & 65.4    & 64.6    & 64.6     \\ \bottomrule
\end{tabular}%
}
\caption{Performance of RAFT-V across different numbers of retrieved examples $k$. We use the Llama3.1-8B-Instruct model with metaprogram format.}
\label{tab:retrieval_quantity_performance}
\end{table}
To assess whether the number of retrieved examples affects performance, we evaluate RAFT-V trained with different numbers of retrieval examples \( k \in \{1, 2, 3\} \) (Table~\ref{tab:retrieval_quantity_performance}). Although increasing retrieval examples \(k\) generally improves generation quality, the gains are marginal, which indicates that even a few examples are able to capture functional patterns in visual programming languages. Based on this observation, we set \( k = 1 \) for our main experiments.


\subsection{\label{sec:tau}Variation in Deletion Ratio}
\begin{table}[!h]
\centering
\resizebox{\columnwidth}{!}{%
\begin{tabular}{lccccc}
\toprule
$\boldsymbol{\tau}$ & Node F1 & Edge F1 & Node EM & Edge EM & Program EM \\ \midrule
0 & \textbf{90.7} & 83.4 & 67.2 & 65.8 & 65.8 \\
\rowcolor[HTML]{EFEFEF}
0.1 (Ours) & 90.6 & \textbf{83.6} & \textbf{68.6} & \textbf{67.2} & \textbf{67.2} \\
0.5 & 90.5 & 83.3 & 68.2 & 66.8 & 66.8 \\
0.9 & 90.6 & 83.4 & 67.4 & 66.0 & 66.0 \\ \bottomrule
\end{tabular}%
}
\caption{Impact of varying deletion ratios. Results for Llama-3.1-8B-Instruct model using metaprogram format.}
\label{tab:dpo_tau}
\end{table}
Furthermore, we report the performance variation depending on the deletion ratio $\tau$. We train the model using preference pairs with different $\tau$ values, where $\tau \in \{0, 0.1, 0.5, 0.9\}$. Table~\ref{tab:dpo_tau} shows that F1 scores remain stable across $\tau$ values, while EM scores decrease as $\tau$ increases. In particular, when $\tau$ is 0.9, the negative samples during preference training are graphs with 90\% of the original graph removed. As a result, the model can easily distinguish them, making the training less effective. Based on these results, we set $\tau$ to 0.1.

\begin{table}[!ht]
\centering
\resizebox{\columnwidth}{!}{%
\begin{tabular}{lccccc}
\toprule
Methods & Node F1 & Edge F1 & Node EM & Edge EM & Program EM \\ \midrule
RAFT-V (100\%) & 90.6 & \textbf{83.6} & 66.6 & 65.2 & 65.2 \\
\rowcolor[HTML]{EFEFEF}
RAFT-V & 89.9 & 82.4 & 64.6 & 63.8 & 63.8 \\ \midrule
BoN (Random) & 90.4 & 83.2 & 66.8 & 65.6 & 65.6 \\
BoN (Unstrict) & \textbf{90.7} & 83.5 & 67.0 & 65.8 & 65.8 \\
BoN (Strict) & 90.5 & 83.3 & 67.4 & 66.2 & 66.2 \\
Ours (Random) & 90.6 & 83.5 & 68.0 & 66.6 & 66.6 \\
\rowcolor[HTML]{EFEFEF}
Ours & 90.6 & \textbf{83.6} & \textbf{68.6} & \textbf{67.2} & \textbf{67.2} \\ \bottomrule
\end{tabular}%
}
\caption{Comparison of BoN sampling and graph editing. Results for metaprogram using Llama-3.1-8B-Instruct.}
\label{tab:dpo_aug}
\end{table}
\subsection{\label{sec:data_augmentation}Analysis of Graph Editing}
We validate our graph editing approach for collecting preference pairs by comparing it with Best-of-N (BoN) sampling\footnote{For BoN sampling, we use nucleus decoding~\citep{Holtzman2020The} with temperature=1.0, top\_p=1.0.}~\citep{10.5555/3495724.3495977, snell2025scaling}, which selects the best generation among $N$ candidates.
We generate $N=10$ outputs from the RAFT-V model and compare them with our negative candidate set $\mathcal{G}=\{G'_1\dots G'_{10}\}$, which is derived from graph editing (algorithm~\ref{alg:graph_augmentation}).

\paragraph{Preference learning is effective}
To address the concern that the gains in the 2-stage approach may result from the additional data used for preference learning, we compare it with a 1-stage approach, where the model is trained on all the data without a second stage, and is referred to as \underline{RAFT-V (100\%)}. As shown in Table~\ref{tab:dpo_aug}, RAFT-V improves performance as the dataset size increases. However, RAFT-V (100\%) shows only marginal improvement in EM scores compared to preference-learned models. Although RAFT-V (100\%) demonstrates effectiveness in improving F1 scores when compared to BoN-based preference learning baselines, it is insufficient to minimize minor errors (EM).

\paragraph{Editing-based negative selection is efficient}
We introduce four baselines for comparison with graph editing: \underline{BoN (Random)}, where $\mathcal{G}$ is sampled from BoN, and $G_{\text{Hard}}$ is randomly selected from BoN-sampled $\mathcal{G}$; \underline{BoN (Unstrict)}, which selects preference pairs based on GED; and \underline{BoN (Strict)}, which considers preference pairs for training only when the sampled output includes an exact match with the correct answer (i.e., GED = 0).  Finally, in the case of \underline{Ours (Random)}, $\mathcal{G}$ is sampled from graph editing, but $G_{\text{Hard}}$ is randomly selected instead of using GED.

As shown in Table~\ref{tab:dpo_aug}, BoN (Strict) constructs preference pairs based on exact matches and achieves the highest accuracy among BoN-based methods despite utilizing only 30\% of the data. This result demonstrates that dataset quality has a more significant impact on preference learning than dataset size~\cite{hou2024doesrlhfscaleexploring, kim2024aligninglargelanguagemodels}. However, BoN sampling often fails to generate challenging cases consistently, as negative samples are selected from the sampled outputs. In contrast, our editing-based negative selection provides a more systematic approach to generating negative samples, which consistently enables the generation of hard negative pairs with higher efficiency. 

\begin{figure}[!t]
    \centering
    \includegraphics[width=0.9\columnwidth,keepaspectratio]{latex/figure/total_exact_match_accuracy_by_complexity.pdf}
    \caption{Program EM score across different complexities. We use the metaprogram format with the Llama3.1-8B-Instruct model.}
    \label{fig:total_EM_trend_complexity}
\end{figure}
\subsection{\label{sec:difficulty}Performance Across Program Complexity}
To evaluate our methodology's performance across varying program complexities, we convert each program (VPL code) in the test split into NetworkX graphs. We define complexity as the total number of nodes and edges in the graph. We then sort programs by complexity and split them into five percentile ranges: 0–20\%, 20–40\%, 40–60\%, 60–80\%, and 80–100\%, labeling them from 1 to 5. Figure~\ref{fig:total_EM_trend_complexity} shows the average Program EM across these categories. Our approach consistently outperforms the SFT baseline, widening the gap at higher complexities (+18.3\% in 4, +16.3\% in 5). These findings highlight our method's robustness, with benefits increasing in challenging scenarios.
\section{Related Work}
\paragraph{LLMs for Visual Program Generation}
Visual programming systems (e.g., LabView~\cite{bitter2006labview}, XG5000~\cite{XG5000Manual}) typically feature node-based interfaces that let users visually write and modify programs. Recently, researchers have begun utilizing LLMs to generate VPLs, as they are known for their powerful text-based code generation capabilities. For example, \citet{cai-etal-2024-low-code} integrates low-code visual programming with LLM-based task execution for direct interaction with LLMs, while \citet{zhang2024benchmarking} studies generation of node-based visual dataflow languages in audio programming. Similarly, \citet{xue2024comfybenchbenchmarkingllmbasedagents,52868} investigates Machine Learning workflow generation from natural language commands and demonstrates that metaprogram-based text formats outperform other formats like JSON. However, these prompting-based methods face limitations for VPLs like Ladder Diagram, where custom I/O mapping and domain-specific syntax are crucial. Thus, we study fine-tuning approaches with domain-specific data to better capture these details.

\paragraph{LLM-based PLC code generation}
Programmable Logic Controllers (PLCs) are essential components in industrial automation and are used to control machinery and processes reliably and efficiently. Among the programming languages defined by the IEC 61131-3 standard~\cite{IEC61131-3}, Structured Text (ST) and Ladder Diagram (LD) are commonly used for programming PLCs. Research in this area has focused on utilizing LLMs to generate ST code from natural language descriptions. Recent studies have demonstrated the potential of LLMs in generating high-quality ST code~\cite{koziolek2023chatgpt, koziolek2024llm}, enhancing safety and accuracy with verification tools and user feedback~\cite{fakih2024llm4plc}, and automating code generation and verification using multi-agent frameworks~\cite{liu2024agents4plc}. Although these advances have improved PLC code generation, they primarily focus on ST, despite LD being widely used in industrial settings due to its similarity to electrical circuits~\cite{ladderlogic}. While \citet{Zhang_2024} attempts to generate LD as an ASCII art based on user instructions in a zero-shot manner, their findings show that even advanced LLMs struggle with basic LD generation. These limitations highlight the necessity of training-based methods for LD generation. In this work, we address this gap by introducing a training-based approach for LLMs to generate LD and thus pave the way for the broader adoption of AI-assisted PLC programming.
\section{Discussion}\label{sec:discussion}



\subsection{From Interactive Prompting to Interactive Multi-modal Prompting}
The rapid advancements of large pre-trained generative models including large language models and text-to-image generation models, have inspired many HCI researchers to develop interactive tools to support users in crafting appropriate prompts.
% Studies on this topic in last two years' HCI conferences are predominantly focused on helping users refine single-modality textual prompts.
Many previous studies are focused on helping users refine single-modality textual prompts.
However, for many real-world applications concerning data beyond text modality, such as multi-modal AI and embodied intelligence, information from other modalities is essential in constructing sophisticated multi-modal prompts that fully convey users' instruction.
This demand inspires some researchers to develop multimodal prompting interactions to facilitate generation tasks ranging from visual modality image generation~\cite{wang2024promptcharm, promptpaint} to textual modality story generation~\cite{chung2022tale}.
% Some previous studies contributed relevant findings on this topic. 
Specifically, for the image generation task, recent studies have contributed some relevant findings on multi-modal prompting.
For example, PromptCharm~\cite{wang2024promptcharm} discovers the importance of multimodal feedback in refining initial text-based prompting in diffusion models.
However, the multi-modal interactions in PromptCharm are mainly focused on the feedback empowered the inpainting function, instead of supporting initial multimodal sketch-prompt control. 

\begin{figure*}[t]
    \centering
    \includegraphics[width=0.9\textwidth]{src/img/novice_expert.pdf}
    \vspace{-2mm}
    \caption{The comparison between novice and expert participants in painting reveals that experts produce more accurate and fine-grained sketches, resulting in closer alignment with reference images in close-ended tasks. Conversely, in open-ended tasks, expert fine-grained strokes fail to generate precise results due to \tool's lack of control at the thin stroke level.}
    \Description{The comparison between novice and expert participants in painting reveals that experts produce more accurate and fine-grained sketches, resulting in closer alignment with reference images in close-ended tasks. Novice users create rougher sketches with less accuracy in shape. Conversely, in open-ended tasks, expert fine-grained strokes fail to generate precise results due to \tool's lack of control at the thin stroke level, while novice users' broader strokes yield results more aligned with their sketches.}
    \label{fig:novice_expert}
    % \vspace{-3mm}
\end{figure*}


% In particular, in the initial control input, users are unable to explicitly specify multi-modal generation intents.
In another example, PromptPaint~\cite{promptpaint} stresses the importance of paint-medium-like interactions and introduces Prompt stencil functions that allow users to perform fine-grained controls with localized image generation. 
However, insufficient spatial control (\eg, PromptPaint only allows for single-object prompt stencil at a time) and unstable models can still leave some users feeling the uncertainty of AI and a varying degree of ownership of the generated artwork~\cite{promptpaint}.
% As a result, the gap between intuitive multi-modal or paint-medium-like control and the current prompting interface still exists, which requires further research on multi-modal prompting interactions.
From this perspective, our work seeks to further enhance multi-object spatial-semantic prompting control by users' natural sketching.
However, there are still some challenges to be resolved, such as consistent multi-object generation in multiple rounds to increase stability and improved understanding of user sketches.   


% \new{
% From this perspective, our work is a step forward in this direction by allowing multi-object spatial-semantic prompting control by users' natural sketching, which considers the interplay between multiple sketch regions.
% % To further advance the multi-modal prompting experience, there are some aspects we identify to be important.
% % One of the important aspects is enhancing the consistency and stability of multiple rounds of generation to reduce the uncertainty and loss of control on users' part.
% % For this purpose, we need to develop techniques to incorporate consistent generation~\cite{tewel2024training} into multi-modal prompting framework.}
% % Another important aspect is improving generative models' understanding of the implicit user intents \new{implied by the paint-medium-like or sketch-based input (\eg, sketch of two people with their hands slightly overlapping indicates holding hand without needing explicit prompt).
% % This can facilitate more natural control and alleviate users' effort in tuning the textual prompt.
% % In addition, it can increase users' sense of ownership as the generated results can be more aligned with their sketching intents.
% }
% For example, when users draw sketches of two people with their hands slightly overlapping, current region-based models cannot automatically infer users' implicit intention that the two people are holding hands.
% Instead, they still require users to explicitly specify in the prompt such relationship.
% \tool addresses this through sketch-aware prompt recommendation to fill in the necessary semantic information, alleviating users' workload.
% However, some users want the generative AI in the future to be able to directly infer this natural implicit intentions from the sketches without additional prompting since prompt recommendation can still be unstable sometimes.


% \new{
% Besides visual generation, 
% }
% For example, one of the important aspect is referring~\cite{he2024multi}, linking specific text semantics with specific spatial object, which is partly what we do in our sketch-aware prompt recommendation.
% Analogously, in natural communication between humans, text or audio alone often cannot suffice in expressing the speakers' intentions, and speakers often need to refer to an existing spatial object or draw out an illustration of her ideas for better explanation.
% Philosophically, we HCI researchers are mostly concerned about the human-end experience in human-AI communications.
% However, studies on prompting is unique in that we should not just care about the human-end interaction, but also make sure that AI can really get what the human means and produce intention-aligned output.
% Such consideration can drastically impact the design of prompting interactions in human-AI collaboration applications.
% On this note, although studies on multi-modal interactions is a well-established topic in HCI community, it remains a challenging problem what kind of multi-modal information is really effective in helping humans convey their ideas to current and next generation large AI models.




\subsection{Novice Performance vs. Expert Performance}\label{sec:nVe}
In this section we discuss the performance difference between novice and expert regarding experience in painting and prompting.
First, regarding painting skills, some participants with experience (4/12) preferred to draw accurate and fine-grained shapes at the beginning. 
All novice users (5/12) draw rough and less accurate shapes, while some participants with basic painting skills (3/12) also favored sketching rough areas of objects, as exemplified in Figure~\ref{fig:novice_expert}.
The experienced participants using fine-grained strokes (4/12, none of whom were experienced in prompting) achieved higher IoU scores (0.557) in the close-ended task (0.535) when using \tool. 
This is because their sketches were closer in shape and location to the reference, making the single object decomposition result more accurate.
Also, experienced participants are better at arranging spatial location and size of objects than novice participants.
However, some experienced participants (3/12) have mentioned that the fine-grained stroke sometimes makes them frustrated.
As P1's comment for his result in open-ended task: "\emph{It seems it cannot understand thin strokes; even if the shape is accurate, it can only generate content roughly around the area, especially when there is overlapping.}" 
This suggests that while \tool\ provides rough control to produce reasonably fine results from less accurate sketches for novice users, it may disappoint experienced users seeking more precise control through finer strokes. 
As shown in the last column in Figure~\ref{fig:novice_expert}, the dragon hovering in the sky was wrongly turned into a standing large dragon by \tool.

Second, regarding prompting skills, 3 out of 12 participants had one or more years of experience in T2I prompting. These participants used more modifiers than others during both T2I and R2I tasks.
Their performance in the T2I (0.335) and R2I (0.469) tasks showed higher scores than the average T2I (0.314) and R2I (0.418), but there was no performance improvement with \tool\ between their results (0.508) and the overall average score (0.528). 
This indicates that \tool\ can assist novice users in prompting, enabling them to produce satisfactory images similar to those created by users with prompting expertise.



\subsection{Applicability of \tool}
The feedback from user study highlighted several potential applications for our system. 
Three participants (P2, P6, P8) mentioned its possible use in commercial advertising design, emphasizing the importance of controllability for such work. 
They noted that the system's flexibility allows designers to quickly experiment with different settings.
Some participants (N = 3) also mentioned its potential for digital asset creation, particularly for game asset design. 
P7, a game mod developer, found the system highly useful for mod development. 
He explained: "\emph{Mods often require a series of images with a consistent theme and specific spatial requirements. 
For example, in a sacrifice scene, how the objects are arranged is closely tied to the mod's background. It would be difficult for a developer without professional skills, but with this system, it is possible to quickly construct such images}."
A few participants expressed similar thoughts regarding its use in scene construction, such as in film production. 
An interesting suggestion came from participant P4, who proposed its application in crime scene description. 
She pointed out that witnesses are often not skilled artists, and typically describe crime scenes verbally while someone else illustrates their account. 
With this system, witnesses could more easily express what they saw themselves, potentially producing depictions closer to the real events. "\emph{Details like object locations and distances from buildings can be easily conveyed using the system}," she added.

% \subsection{Model Understanding of Users' Implicit Intents}
% In region-sketch-based control of generative models, a significant gap between interaction design and actual implementation is the model's failure in understanding users' naturally expressed intentions.
% For example, when users draw sketches of two people with their hands slightly overlapping, current region-based models cannot automatically infer users' implicit intention that the two people are holding hands.
% Instead, they still require users to explicitly specify in the prompt such relationship.
% \tool addresses this through sketch-aware prompt recommendation to fill in the necessary semantic information, alleviating users' workload.
% However, some users want the generative AI in the future to be able to directly infer this natural implicit intentions from the sketches without additional prompting since prompt recommendation can still be unstable sometimes.
% This problem reflects a more general dilemma, which ubiquitously exists in all forms of conditioned control for generative models such as canny or scribble control.
% This is because all the control models are trained on pairs of explicit control signal and target image, which is lacking further interpretation or customization of the user intentions behind the seemingly straightforward input.
% For another example, the generative models cannot understand what abstraction level the user has in mind for her personal scribbles.
% Such problems leave more challenges to be addressed by future human-AI co-creation research.
% One possible direction is fine-tuning the conditioned models on individual user's conditioned control data to provide more customized interpretation. 

% \subsection{Balance between recommendation and autonomy}
% AIGC tools are a typical example of 
\subsection{Progressive Sketching}
Currently \tool is mainly aimed at novice users who are only capable of creating very rough sketches by themselves.
However, more accomplished painters or even professional artists typically have a coarse-to-fine creative process. 
Such a process is most evident in painting styles like traditional oil painting or digital impasto painting, where artists first quickly lay down large color patches to outline the most primitive proportion and structure of visual elements.
After that, the artists will progressively add layers of finer color strokes to the canvas to gradually refine the painting to an exquisite piece of artwork.
One participant in our user study (P1) , as a professional painter, has mentioned a similar point "\emph{
I think it is useful for laying out the big picture, give some inspirations for the initial drawing stage}."
Therefore, rough sketch also plays a part in the professional artists' creation process, yet it is more challenging to integrate AI into this more complex coarse-to-fine procedure.
Particularly, artists would like to preserve some of their finer strokes in later progression, not just the shape of the initial sketch.
In addition, instead of requiring the tool to generate a finished piece of artwork, some artists may prefer a model that can generate another more accurate sketch based on the initial one, and leave the final coloring and refining to the artists themselves.
To accommodate these diverse progressive sketching requirements, a more advanced sketch-based AI-assisted creation tool should be developed that can seamlessly enable artist intervention at any stage of the sketch and maximally preserve their creative intents to the finest level. 

\subsection{Ethical Issues}
Intellectual property and unethical misuse are two potential ethical concerns of AI-assisted creative tools, particularly those targeting novice users.
In terms of intellectual property, \tool hands over to novice users more control, giving them a higher sense of ownership of the creation.
However, the question still remains: how much contribution from the user's part constitutes full authorship of the artwork?
As \tool still relies on backbone generative models which may be trained on uncopyrighted data largely responsible for turning the sketch into finished artwork, we should design some mechanisms to circumvent this risk.
For example, we can allow artists to upload backbone models trained on their own artworks to integrate with our sketch control.
Regarding unethical misuse, \tool makes fine-grained spatial control more accessible to novice users, who may maliciously generate inappropriate content such as more realistic deepfake with specific postures they want or other explicit content.
To address this issue, we plan to incorporate a more sophisticated filtering mechanism that can detect and screen unethical content with more complex spatial-semantic conditions. 
% In the future, we plan to enable artists to upload their own style model

% \subsection{From interactive prompting to interactive spatial prompting}


\subsection{Limitations and Future work}

    \textbf{User Study Design}. Our open-ended task assesses the usability of \tool's system features in general use cases. To further examine aspects such as creativity and controllability across different methods, the open-ended task could be improved by incorporating baselines to provide more insightful comparative analysis. 
    Besides, in close-ended tasks, while the fixing order of tool usage prevents prior knowledge leakage, it might introduce learning effects. In our study, we include practice sessions for the three systems before the formal task to mitigate these effects. In the future, utilizing parallel tests (\textit{e.g.} different content with the same difficulty) or adding a control group could further reduce the learning effects.

    \textbf{Failure Cases}. There are certain failure cases with \tool that can limit its usability. 
    Firstly, when there are three or more objects with similar semantics, objects may still be missing despite prompt recommendations. 
    Secondly, if an object's stroke is thin, \tool may incorrectly interpret it as a full area, as demonstrated in the expert results of the open-ended task in Figure~\ref{fig:novice_expert}. 
    Finally, sometimes inclusion relationships (\textit{e.g.} inside) between objects cannot be generated correctly, partially due to biases in the base model that lack training samples with such relationship. 

    \textbf{More support for single object adjustment}.
    Participants (N=4) suggested that additional control features should be introduced, beyond just adjusting size and location. They noted that when objects overlap, they cannot freely control which object appears on top or which should be covered, and overlapping areas are currently not allowed.
    They proposed adding features such as layer control and depth control within the single-object mask manipulation. Currently, the system assigns layers based on color order, but future versions should allow users to adjust the layer of each object freely, while considering weighted prompts for overlapping areas.

    \textbf{More customized generation ability}.
    Our current system is built around a single model $ColorfulXL-Lightning$, which limits its ability to fully support the diverse creative needs of users. Feedback from participants has indicated a strong desire for more flexibility in style and personalization, such as integrating fine-tuned models that cater to specific artistic styles or individual preferences. 
    This limitation restricts the ability to adapt to varied creative intents across different users and contexts.
    In future iterations, we plan to address this by embedding a model selection feature, allowing users to choose from a variety of pre-trained or custom fine-tuned models that better align with their stylistic preferences. 
    
    \textbf{Integrate other model functions}.
    Our current system is compatible with many existing tools, such as Promptist~\cite{hao2024optimizing} and Magic Prompt, allowing users to iteratively generate prompts for single objects. However, the integration of these functions is somewhat limited in scope, and users may benefit from a broader range of interactive options, especially for more complex generation tasks. Additionally, for multimodal large models, users can currently explore using affordable or open-source models like Qwen2-VL~\cite{qwen} and InternVL2-Llama3~\cite{llama}, which have demonstrated solid inference performance in our tests. While GPT-4o remains a leading choice, alternative models also offer competitive results.
    Moving forward, we aim to integrate more multimodal large models into the system, giving users the flexibility to choose the models that best fit their needs. 
    


\section{Conclusion}\label{sec:conclusion}
In this paper, we present \tool, an interactive system designed to help novice users create high-quality, fine-grained images that align with their intentions based on rough sketches. 
The system first refines the user's initial prompt into a complete and coherent one that matches the rough sketch, ensuring the generated results are both stable, coherent and high quality.
To further support users in achieving fine-grained alignment between the generated image and their creative intent without requiring professional skills, we introduce a decompose-and-recompose strategy. 
This allows users to select desired, refined object shapes for individual decomposed objects and then recombine them, providing flexible mask manipulation for precise spatial control.
The framework operates through a coarse-to-fine process, enabling iterative and fine-grained control that is not possible with traditional end-to-end generation methods. 
Our user study demonstrates that \tool offers novice users enhanced flexibility in control and fine-grained alignment between their intentions and the generated images.

\section{Limitation}
The use of 3D-printed PLA for structural components improves improving ease of assembly and reduces weight and cost, yet it causes deformation under heavy load, which can diminish end-effector precision. Using metal, such as aluminum, would remedy this problem. Additionally, \robot relies on integrated joint relative encoders, requiring manual initialization in a fixed joint configuration each time the system is powered on. Using absolute joint encoders could significantly improve accuracy and ease of use, although it would increase the overall cost. 

%Reliance on commercially available actuators simplifies integration but imposes constraints on control frequency and customization, further limiting the potential for tailored performance improvements.

% The 6 DoF configuration provides sufficient mobility for most tasks; however, certain bimanual operations could benefit from an additional degree of freedom to handle complex joint constraints more effectively. Furthermore, the limited torque density of commercially available proprioceptive actuators restricts the payload and torque output, making the system less suitability for handling heavier loads or high-torque applications. 

The 6 DoF configuration of the arm provides sufficient mobility for single-arm manipulation tasks, yet it shows a limitation in certain bimanual manipulation problems. Specifically, when \robot holds onto a rigid object with both hands, each arm loses 1 DoF because the hands are fixed to the object during grasping. This leads to an underactuated kinematic chain which has a limited mobility in 3D space. We can achieve more mobility by letting the object slip inside the grippers, yet this renders the grasp less robust and simulation difficult. Therefore, we anticipate that designing a lightweight 3 DoF wrist in place of the current 2 DoF wrist allows a more diverse repertoire of manipulation in bimanual tasks.

Finally, the limited torque density of commercially available proprioceptive actuators restricts the performance. Currently, all of our actuators feature a 1:10 gear ratio, so \robot can handle up to 2.5 kg of payload. To handle a heavier object and manipulate it with higher torque, we expect the actuator to have 1:20$\sim$30 gear ratio, but it is difficult to find an off-the-shelf product that meets our requirements. Customizing the actuator to increase the torque density while minimizing the weight will enable \robot to move faster and handle more diverse objects.

%These constraints highlight opportunities for improvement in future iterations, including alternative materials for enhanced rigidity, custom actuator designs for higher control precision and torque density, the adoption of absolute joint encoders, and optimized configurations to balance dexterity and weight.



\section*{Acknowledgments}
This work was supported by Hyundai Mobis (47.5\%). This research was supported by the MSIT(Ministry of Science and ICT), Korea, under the ITRC(Information Technology Research Center) support program(IITP-2025-2020-0-01789) supervised by the IITP(Institute for Information \& Communications Technology Planning \& Evaluation, 47.5\%). This work was supported by Institute of Information \& communications Technology Planning \& Evaluation (IITP) grant funded by the Korea government(MSIT) (No.RS-2019-II191906, Artificial Intelligence Graduate School Program(POSTECH), 5\%).

% Bibliography entries for the entire Anthology, followed by custom entries
%\bibliography{anthology,custom}
% Custom bibliography entries only

% \nocite{*}
\bibliography{custom}

\onecolumn
\appendix
% \section{Appendix}

\section{\label{sec:metric}Evaluation Metrics}
As mentioned in \S\ref{sec:conversion}, Ladder Diagram (LD) can be represented as NetworkX graphs. We evaluate LD programs from two perspectives: node and edge levels and the overall graph level. To assess the accuracy of node and edge predictions, we compute the F1 score, which considers both precision and recall for the sets of nodes and edges in each ground truth and predicted graph. These metrics are defined as \textbf{Node F1} and \textbf{Edge F1}, respectively. Furthermore, to verify the correct execution of LD, we use Exact Match (EM), which strictly measures the structural accuracy by comparing the entire node and edge sets between the predicted and reference graphs. We define \textbf{Node EM} and \textbf{Edge EM} to measure the exact match between the entire sets of nodes and edges, respectively. Moreover, \textbf{Program EM} represents the overall structural alignment, which is achieved only when both node and edge sets are perfectly matched.

Let the reference graph be denoted as $G = (V, E)$ and the predicted graph be denoted as $\hat{G} = (\hat{V}, \hat{E})$. Here, comparisons between nodes and edges are evaluated by considering all required attributes (e.g., names, types) for both F1 score and exact match.

\begin{equation}
\begin{aligned}
\nonumber
\text{For nodes:} \quad & TP_N = |V \cap \hat{V}|,\quad FP_N = |\hat{V} \setminus V|,\quad FN_N = |V \setminus \hat{V}|\\[1ex]
& \text{Precision}_N = \frac{TP_N}{|\hat{V}|},\quad \text{Recall}_N = \frac{TP_N}{|V|} \quad \therefore \textbf{Node F1} = \frac{2\,TP_N}{|V| + |\hat{V}|} = \frac{2\,|V \cap \hat{V}|}{|V| + |\hat{V}|} \\[2ex]
\text{For edges:} \quad & TP_E = |E \cap \hat{E}|,\quad FP_E = |\hat{E} \setminus E|,\quad FN_E = |E \setminus \hat{E}|\\[1ex]
& \text{Precision}_E = \frac{TP_E}{|\hat{E}|},\quad \text{Recall}_E = \frac{TP_E}{|E|} \quad \therefore \textbf{Edge F1} = \frac{2\,TP_E}{|E| + |\hat{E}|} = \frac{2\,|E \cap \hat{E}|}{|E| + |\hat{E}|}
\end{aligned}
\end{equation}
The Exact Match (EM) scores are defined as follows:
\begin{equation}
\nonumber
\textbf{Node EM} = \mathbf{1}\{\hat{V} = V\}, \quad \textbf{Edge EM} = \mathbf{1}\{\hat{E} = E\}, \quad \textbf{Program EM} = \mathbf{1}\{\hat{V} = V \text{ and } \hat{E} = E\}
\end{equation}


\subsection{Examples of Metric Evaluations}
\begin{figure*}[!htbp]
    \centering
    \includegraphics[width=0.9\textwidth]{latex/figure/metric.pdf}
    \caption{Example for explaining evaluation metrics. Based on real data, but node and edge attributes have been modified due to security issues.}
    \label{fig:metric}
\end{figure*}

\begin{equation}
\nonumber
\text{Precision}_N = \frac{|V \cap \hat{V}|}{|\hat{V}|} = \frac{5}{5} = 1.0, \quad
\text{Recall}_N = \frac{|V \cap \hat{V}|}{|V|} = \frac{5}{6} = 0.833
\end{equation}
\begin{equation}
\nonumber
\therefore \textbf{Node F1} = \frac{2 \times 1.0 \times 0.833}{1.0 + 0.833} = 0.91
\end{equation}
Figure~\ref{fig:metric} is evaluated based on the above metrics. The reference graph contains 5 edges, while the predicted graph contains 4 edges, with one incorrect edge.
\begin{equation}
\nonumber
\text{Precision}_E = \frac{TP_E}{|\hat{E}|} = \frac{3}{4} = 0.75, \quad
\text{Recall}_E = \frac{TP_E}{|E|} = \frac{3}{5} = 0.6
\end{equation}
\begin{equation}
\nonumber
\therefore \textbf{Edge F1} = \frac{2 \times 0.75 \times 0.6}{0.75 + 0.6} = 0.67
\end{equation}
The predicted graph does not fully match the reference graph, so all EM scores are 0. Although the node structure is mostly correct (\( F1 = 0.91 \)), inconsistencies in the edge structure (\( F1 = 0.67 \)) prevent an exact match.
\section{Further Implementation Details}\label{sec:implementation_details}
\paragraph{Supervised fine-tuning (including RAFT-V)}
We fine-tuned models using SFT with 10 epochs, a batch size of 8, and a learning rate of $5 \times 10^{-5}$. We applied LoRA~\cite{hulora} with a rank of 256, $\alpha = 256$, and a dropout rate of 0.05. LoRA adaptation was applied to the following target modules: q\_proj, v\_proj, k\_proj, o\_proj. Using the AdamW~\cite{loshchilov2018decoupled} optimizer, we minimized the cross-entropy loss $\mathcal{L}_{\text{CE}}$ between ground-truth code $c$ and the predicted code $\hat{c}$ as follows:
\begin{equation}
\nonumber
    \mathcal{L}_{\text{CE}} = - \sum_{t=1}^{T} \sum_{v \in V} c_t(v) \log \hat{c}_t(v)
\end{equation}
where $T$ is the sequence length and $V$ is a vocabulary size. The SFT process took 6 hours using 4 A100-80GB GPUs, while the SFT process for RAFT-V took 8 hours using the same hardware configuration.

\paragraph{Preference learning} For preference learning, we utilized Direct Preference Optimization (DPO)~\cite{rafailov2023direct}, and trained models for 5 epochs with a batch size of 64. The learning rate was set to $1 \times 10^{-7}$, with a warmup ratio of 0.03, a weight decay of 0.01, and $\beta=0.1$. As with SFT, {LoRA} was applied with the same rank, $\alpha$, dropout rate, and target modules.

Given a pair of responses, a preferred response $y^+$ and a dispreferred response $y^-$ for a given input $x$, we minimize the following DPO loss:

\begin{equation}
\nonumber
    \mathcal{L}_{\text{DPO}}(\theta) = -\mathbb{E}_{(x,y^+, y^-) \sim \mathcal{D}} \left[ 
    \log \sigma \left( \beta \left( \log \frac{\pi_\theta(y^+ \mid x,\mathcal{R}(x))}{\pi_{\text{ref}}(y^+ \mid x,\mathcal{R}(x))} - \log \frac{\pi_\theta(y^- \mid x,\mathcal{R}(x))}{\pi_{\text{ref}}(y^- \mid x,\mathcal{R}(x))} \right) \right) 
    \right]
\end{equation}
where $\sigma(z)$ is the sigmoid function, $\beta$ is a scaling factor controlling the strength of preference optimization, $\mathcal{R}(x)=(p^r, c^r)$ represents a retrieved prompt and its corresponding VPL code obtained from a retriever $\mathcal{R}$, and $\mathcal{D}$ represents the dataset of preference-labeled samples. The reference model $\pi_{\text{ref}}$ serves as a baseline to prevent reward overoptimization, ensuring stable preference learning. The preference training stage took 6 hours on 4 A100-80GB GPUs.
We measured the loss on the validation set at each epoch in both training stages and applied early stopping based on this criterion, with a patience value of 2. 
\paragraph{Sampling parameters} We employed beam search decoding with a beam size of 4 during inference. 

\section{Data Samples}\label{sec:data_samples}

\begin{table*}[!htbp]
    \small
    \centering
    \begin{tabular}{|p{0.15\textwidth}|p{0.8\textwidth}|} 
        \toprule
        \textbf{Type} & \textbf{Content} \\ \midrule
        
        Prompt 
        & \textbf{Program Description:} 셀 이재기 서보 조그 고속 하한 값 인터락 프로그램을 만들어줘. \\[3pt]
        & \textcolor{gray}{(Create a cell transfer servo jog high-speed interlock program.)} \\[3pt]
        & \textbf{Detailed Description:} 조그 고속 속도 설정값이 20 이하일 때 조그 고속속도를 20으로 설정해줘. \\[3pt]
        & \textcolor{gray}{(When the jog high-speed setting value is 20 or less, set the jog high-speed to 20.)} \\[3pt] \midrule

        \makecell[l]{XML \\ (Anonymized)}
        & \begin{lstlisting}[basicstyle=\ttfamily, aboveskip=0pt, belowskip=0pt]
<Rung>
    <Element Attributes..., Coordinate="X"></Element>
    <Element Attributes..., Coordinate="Y"></Element>
    <Element Attributes..., Coordinate="Z"></Element>
</Rung>
\end{lstlisting} \\ \bottomrule
    \end{tabular}
    \caption{Program description and anonymized XML example}
    \label{tab:appendix_sample}
\end{table*}

An example from the dataset used in this study is provided. The dataset consists of \textbf{prompt} and \textbf{code} pairs. The prompt was collected in Korean and used without translation for model training and evaluation. The prompt is further divided into \textbf{Program Description}, providing a high-level summary of the functionality, and \textbf{Detailed Description}, specifying task parameters and conditions. The prompts were created by an experienced PLC programmer from the Republic of Korea, ensuring the incorporation of domain expertise into the collected data. The programmer was explicitly informed that the dataset would be collected for model training and evaluation and used strictly for research purposes. We obtained consent prior to data collection. The \textbf{code} is represented in XML format, which describes the ladder diagram with elements. These elements are listed sequentially, and each element includes its attributes (e.g., variable names, parameters) and coordinate information. The datasets were constructed by an experienced PLC engineer, and all sensitive information was anonymized to ensure data confidentiality.
\section{\label{sec:human_eval}Human Evaluation}

\begin{table*}[!htbp]
\centering
\begin{tabular}{@{}lcc@{}}
\toprule
Method        & Functional Score ($\uparrow$) & Kappa Score ($\uparrow$) \\ \midrule
SFT-only      & 4.34            &       0.62      \\
RAFT-V        & 4.60            &         0.58    \\
Ours & \textbf{4.62} & \textbf{0.66}   \\ \bottomrule
\end{tabular}
\caption{Average of human evaluation results. We used the metaprogram format for evaluation.}
\label{tab:human_evaluation}
\end{table*}

To further evaluate the effectiveness of our methods, we conduct human evaluations on a randomly selected 50 test set examples. Due to the dataset's security sensitivity, human evaluations were conducted exclusively by PLC engineers who had authorized access to confidential information. 
Three experienced LD programmers evaluated the generated code based on functionality, and assigned scores on a scale of 1 to 5.
The ratings were based on their professional experience and the practical applicability of the code in real industrial settings. The results are shown in Table~\ref{tab:human_evaluation}. Compared to SFT-only, RAFT-V showed an improvement of 0.26 points, while our method outperformed RAFT-V by 0.02 points. Furthermore, we report Fleiss' kappa coefficient~\citep{fleiss1971measuring} to statistically evaluate the level of agreement among human evaluators. The results indicate that our proposed methodology demonstrates the highest degree of inter-rater consistency.

\section{More Details on RAG}\label{sec:rag_details}
The prompt template for RAG models is as follows. The RAG model takes the same system prompt as the training-based model depending on the text format.

\begin{shk}
message = [
    {"role": "system", "content": {system prompt}},
    {"role": "user", "content": retrieved_prompt}, 
    ...
    {"role": "assistant", "content": retrieved_code},
    {"role": "user", "content": {test_prompt}}
]
\end{shk}
\noindent\begin{minipage}{\textwidth}
\captionsetup{type=figure}
\captionof{figure}{Prompt template for RAG models}
\end{minipage}

Since some outputs of RAG models were ill-formed (particularly in the case of Qwen), we applied postprocessing to refine them. Extra elements such as \verb|```xml|, \verb|```json|, \verb|```python|, or unrelated code snippets were removed. For inference, we employ beam search with beam size of 4.

\section{Detailed Results}\label{sec:detailed_results}
In this section, we present detailed experimental results. Table~\ref{tab:rag_versus_sft_llama} shows the performance comparison between the SFT model and RAG, with a larger LLM within the same family.


\begin{table*}[!htbp]
\tiny
\centering
\resizebox{0.9\textwidth}{!}{%
\begin{tabular}{@{}llccccc@{}}
\toprule
Format& Method & Node F1& Edge F1& Node EM& Edge EM& Program EM\\ \midrule

\multirow{6}{*}{XML}& SFT& \textbf{84.8} / \textbf{79.1}&\textbf{74.7} / \textbf{68.2} & \textbf{55.2} / \textbf{46.8}& \textbf{54.8} / \textbf{46.0}& \textbf{54.8} / \textbf{46.0}\\
& $N=1$& 69.0 / 19.5& 58.2 / 13.9& 39.0 / 8.8& 38.6 / 8.6& 38.0 / 8.6\\
& $N=3$& 76.3 / 61.4& 67.6 / 53.3& 49.8 / 40.2& 49.0 / 39.4& 49.0 / 39.4\\
& $N=5$& 74.4 / 60.7& 67.1 / 53.8& 49.6 / 41.2& 49.2 / 40.6& 49.2 / 40.6\\
& $N=7$& 59.3 / 53.6& 53.5 / 47.9& 42.8 / 37.8& 42.2 / 37.6& 42.2 / 37.6\\
& $N=9$& 50.5 / 44.9& 45.1 / 40.4& 37.6 / 34.4& 36.8 / 34.0& 36.8 / 34.0\\ \midrule
\multirow{6}{*}{JSON}& SFT& \textbf{85.1} / \textbf{81.2} & \textbf{74.9} / \textbf{69.3} & \textbf{53.6} / \textbf{46.8}& \textbf{52.6} / 46.4& \textbf{52.6} / 46.4\\
& $N=1$& 1.2 / 5.3& 0.8 / 3.9& 0.6 / 3.6& 0.6 / 4.0& 1.1 / 3.6\\
& $N=3$ & 79.2 / 73.4& 69.8 / 62.7& 48.2 / 43.4& 47.4 / 46.0& 47.4 / 46.0\\
& $N=5$& 79.3 / 74.8& 70.7 / 67.1& 51.4 / 43.8& 50.6 / \textbf{46.6}& 50.6 / \textbf{46.6}\\
& $N=7$& 72.6 / 54.6& 65.7 / 50.2& 51.6 / 42.2& 50.8 / 41.8 & 50.8 / 41.8\\
& $N=9$& 58.5 / 52.8& 53.1 / 48.0& 44.2 / 40.2& 43.6 / 39.8& 43.6 / 39.8\\ \midrule
\multirow{6}{*}{Metaprogram}& SFT& \textbf{86.3} / \textbf{81.8} & \textbf{75.7} / \textbf{70.5} & \textbf{55.2} / \textbf{48.4}& \textbf{54.0} / \textbf{48.0}& \textbf{54.0} / \textbf{48.0} \\
& $N=1$& 4.2 / 1.0& 3.5 / 0.9& 2.8 / 0.8& 2.8 / 0.6& 2.8 / 0.6\\
& $N=3$& 80.8 / 47.2& 71.1 / 40.9& 49.4 / 27.0& 48.4 / 26.0& 48.4 / 26.0\\
& $N=5$& 80.0 / 68.1& 71.9 / 59.7& 52.2 / 43.6& 51.8 / 42.6& 51.8 / 42.6\\
& $N=7$& 76.4 / 71.5& 69.1 / 64.2& 51.8 / 48.2& 51.2 / 47.6& 51.2 / 47.6\\
& $N=9$& 64.5 / 59.5& 57.8 / 53.6& 47.2 / 43.8& 46.8 / 42.8& 46.8 / 42.8\\ \bottomrule
\end{tabular}%
}
\caption{Performance comparison between SFT and RAG (with a larger LLM) across XML, JSON, and Metaprogram formats. For RAG, $N$ denotes the number of retrieved examples. Each SFT score is presented in the order of "Llama-3.1-8B-Instruct / Qwen2.5-7B-Instruct." RAG model utilizes the AWQ-quantized versions of \href{https://huggingface.co/hugging-quants/Meta-Llama-3.1-70B-Instruct-AWQ-INT4}{hugging-quants/Meta-Llama-3.1-70B-Instruct-AWQ-INT4} and \href{https://huggingface.co/Qwen/Qwen2.5-72B-Instruct-AWQ}{Qwen/Qwen2.5-72B-Instruct-AWQ}. The highest performance value within each format is shown in bold.}
\label{tab:rag_versus_sft_llama}
\end{table*}
\section{System Prompts} \label{sec:system_prompt}
Depending on the type of text format used, the model takes a different system prompt. The retrieval-based model used in the experiments takes the following prompt template as input:

\begin{shk}
message = [
    {"role": "system", "content": {system prompt}},
    {"role": "user", "content": retrieved_prompt}, 
    ...
    {"role": "assistant", "content": retrieved_code},
    {"role": "user", "content": f"Based on the given input, generate the corresponding code: {test_prompt}"}
]
\end{shk}
\noindent\begin{minipage}{\textwidth}
\captionsetup{type=figure}
\captionof{figure}{Prompt template used in this study}
\end{minipage}
For models that do not utilize retrieval, the prompt template excludes \texttt{retrieved\_prompt} and \texttt{retrieved\_code} for code generation.

\subsection{System prompt for XML}
\begin{shk}
You are a programming assistant specializing in generating ladder programs in XML format. Your task is to translate functional descriptions into equivalent PLC ladder logic and directly represent the ladder logic as XML. The natural language instructions will describe the desired functionality. Your job is to:  
1. Interpret the described functionality.  
2. Translate it into equivalent ladder logic components (e.g., rungs, contacts, coils).  
3. Directly create and output the ladder logic as XML.

###  Requirements for Ladder Logic Representation in XML:
- Each element must include an `ElementType` attribute, which specifies its type, and additional necessary attributes depending on the `ElementType`:
- The output XML must be well-formed, human-readable, and valid for parsing by PLC-related tools or frameworks.

### Explanation of ElementTypes:
[Lines]
- VertLine: It is a vertical line.
- HorzLine: It is a horizontal line.
- MultiHorzLine: It is a horizontal line with a fixed length.

[Contact]
- NormallyOpen: When the state of the BOOL variable (indicated by "***") is On, the state of the left connection line is copied to the right connection line. Otherwise, the state of the right connection line is Off.
- NormallyClosed: When the state of the BOOL variable (indicated by "***") is Off, the state of the left connection line is copied to the right connection line. Otherwise, the state of the right connection line is Off.
- RisingEdgeContact: If the value of the BOOL variable (indicated by "***") changes from Off in the previous scan to On in the current scan, and the state of the left connection line is On, the state of the right connection line becomes On during the current scan.
- FallingEdgeContact: If the value of the BOOL variable (indicated by "***") changes from On in the previous scan to Off in the current scan, and the state of the left connection line is On, the state of the right connection line becomes On during the current scan.
- RisingEdgeNotContact: If the value of the BOOL variable (indicated by "***") changes from Off in the previous scan to On in the current scan, and the state of the left connection line is On, the state of the right connection line becomes Off during the current scan.
- FallingEdgeNotContact: If the value of the BOOL variable (indicated by "***") changes from On in the previous scan to Off in the current scan, and the state of the left connection line is On, the state of the right connection line becomes Off during the current scan.

[Coil]
- StandardCoil: The state of the left connection line is assigned to the corresponding BOOL variable (indicated by "***").
- NegatedCoil: The negated value of the left connection line state is assigned to the corresponding BOOL variable (indicated by "***"). If the left connection line state is Off, the corresponding variable is set to On, and if the left connection line state is On, the corresponding variable is set to Off.
- SetCoil: When the state of the left connection line becomes On, the corresponding BOOL variable (indicated by "***") is set to On and remains On until turned Off by the Reset coil.
- ResetCoil: When the state of the left connection line becomes On, the corresponding BOOL variable (indicated by "***") is set to Off and remains Off until turned On by the Set coil.
- RisingEdgeCoil: If the state of the left connection line changes from Off in the previous scan to On in the current scan, the value of the corresponding BOOL variable (indicated by "***") becomes On only during the current scan.
- FallingEdgeCoil: If the state of the left connection line changes from On in the previous scan to Off in the current scan, the value of the corresponding BOOL variable (indicated by "***") becomes On only during the current scan.

[Others]
- Inverter: The state of the left connection line is inverted and passed to the right connection line.
- FunctionBlock: Represents a function block.
- Variable: Represents the variable corresponding to the function.
- RisingEdge: Before detecting a positive transition, if the result of the previous operations changes from Off in the previous scan to On in the current scan, and the state of the left connection line is On, the state of the right connection line becomes On only during the current scan.
- FallingEdge: Before detecting a negative transition, if the result of the previous operations changes from On in the previous scan to Off in the current scan, and the state of the left connection line is On, the state of the right connection line becomes On only during the current scan.
\end{shk}
\noindent\begin{minipage}{\textwidth}
\captionsetup{type=figure}
\captionof{figure}{System prompt for XML}
\end{minipage}


\subsection{System prompt for JSON}
\begin{shk}
You are a programming assistant specializing in generating ladder programs in JSON format. Your task is to translate functional descriptions into equivalent PLC ladder logic and directly represent the ladder logic as JSON. The natural language instructions will describe the desired functionality. Your job is to:  
1. Interpret the described functionality.  
2. Translate it into equivalent ladder logic components (e.g., contacts, coils, functions).  
3. Directly create and output the ladder logic as JSON.

### Requirements for Ladder Logic Representation in JSON:
- The JSON structure must adhere to the following format:
  - The root is an object containing a single graph, such as `"G0"`, which represents the ladder logic network.
  - Each node in the graph is identified by a unique ID (e.g., `"0"`, `"9"`, etc.).
  - Each node has:
    - `attributes`: An object containing the properties of the node, including:
      - `ElementType`: The type of ladder logic element (e.g., `"NormallyOpen"`, `"StandardCoil"`, `"Variable"`, `"FunctionBlock"`).
      - Additional attributes specific to the `ElementType` 
    - `edges`: An array of connections from this node to other nodes, where:
      - Each edge has a `target` (the ID of the target node) and a `type` (the connection type, e.g., `"Enable"`, `"Output"`, `"Input1"`).

[Contact]
- NormallyOpen: When the state of the BOOL variable (indicated by "***") is On, the state of the left connection line is copied to the right connection line. Otherwise, the state of the right connection line is Off.
- NormallyClosed: When the state of the BOOL variable (indicated by "***") is Off, the state of the left connection line is copied to the right connection line. Otherwise, the state of the right connection line is Off.
- RisingEdgeContact: If the value of the BOOL variable (indicated by "***") changes from Off in the previous scan to On in the current scan, and the state of the left connection line is On, the state of the right connection line becomes On during the current scan.
- FallingEdgeContact: If the value of the BOOL variable (indicated by "***") changes from On in the previous scan to Off in the current scan, and the state of the left connection line is On, the state of the right connection line becomes On during the current scan.
- RisingEdgeNotContact: If the value of the BOOL variable (indicated by "***") changes from Off in the previous scan to On in the current scan, and the state of the left connection line is On, the state of the right connection line becomes Off during the current scan.
- FallingEdgeNotContact: If the value of the BOOL variable (indicated by "***") changes from On in the previous scan to Off in the current scan, and the state of the left connection line is On, the state of the right connection line becomes Off during the current scan.

[Coil]
- StandardCoil: The state of the left connection line is assigned to the corresponding BOOL variable (indicated by "***").
- NegatedCoil: The negated value of the left connection line state is assigned to the corresponding BOOL variable (indicated by "***"). If the left connection line state is Off, the corresponding variable is set to On, and if the left connection line state is On, the corresponding variable is set to Off.
- SetCoil: When the state of the left connection line becomes On, the corresponding BOOL variable (indicated by "***") is set to On and remains On until turned Off by the Reset coil.
- ResetCoil: When the state of the left connection line becomes On, the corresponding BOOL variable (indicated by "***") is set to Off and remains Off until turned On by the Set coil.
- RisingEdgeCoil: If the state of the left connection line changes from Off in the previous scan to On in the current scan, the value of the corresponding BOOL variable (indicated by "***") becomes On only during the current scan.
- FallingEdgeCoil: If the state of the left connection line changes from On in the previous scan to Off in the current scan, the value of the corresponding BOOL variable (indicated by "***") becomes On only during the current scan.

[Others]
- Inverter: The state of the left connection line is inverted and passed to the right connection line.
- FunctionBlock: Represents a function block.
- Variable: Represents the variable corresponding to the function.
- RisingEdge: Before detecting a positive transition, if the result of the previous operations changes from Off in the previous scan to On in the current scan, and the state of the left connection line is On, the state of the right connection line becomes On only during the current scan.
- FallingEdge: Before detecting a negative transition, if the result of the previous operations changes from On in the previous scan to Off in the current scan, and the state of the left connection line is On, the state of the right connection line becomes On only during the current scan.
\end{shk}
\noindent\begin{minipage}{\textwidth}
\captionsetup{type=figure}
\captionof{figure}{System prompt for JSON}
\end{minipage}


\subsection{System prompt for Code}
\begin{shk}
You are a programming assistant specializing in generating Python code. Your task is to write Python code that translates functional descriptions into equivalent PLC ladder logic and represents the ladder logic as graphs using the NetworkX library. The natural language instructions will describe the desired functionality. Your job is to:
1. Interpret the described functionality.
2. Translate it into equivalent ladder logic components (e.g., rungs, contacts, coils).
3. Implement this logic in Python code using NetworkX, representing the ladder logic as directed graphs.

### Requirements for Ladder Logic Representation:
- Nodes: Represent ladder logic elements such as inputs, outputs, and logic functions.
- Edges: Represent connections between these elements, indicating logical flow or sequence.

### ElementType of Nodes
Nodes perform differently based on their ElementType. The behavior for each ElementType is as follows:
[Contact]
- NormallyOpen: When the state of the BOOL variable (indicated by "***") is On, the state of the left connection line is copied to the right connection line. Otherwise, the state of the right connection line is Off.
- NormallyClosed: When the state of the BOOL variable (indicated by "***") is Off, the state of the left connection line is copied to the right connection line. Otherwise, the state of the right connection line is Off.
- RisingEdgeContact: If the value of the BOOL variable (indicated by "***") changes from Off in the previous scan to On in the current scan, and the state of the left connection line is On, the state of the right connection line becomes On during the current scan.
- FallingEdgeContact: If the value of the BOOL variable (indicated by "***") changes from On in the previous scan to Off in the current scan, and the state of the left connection line is On, the state of the right connection line becomes On during the current scan.
- RisingEdgeNotContact: If the value of the BOOL variable (indicated by "***") changes from Off in the previous scan to On in the current scan, and the state of the left connection line is On, the state of the right connection line becomes Off during the current scan.
- FallingEdgeNotContact: If the value of the BOOL variable (indicated by "***") changes from On in the previous scan to Off in the current scan, and the state of the left connection line is On, the state of the right connection line becomes Off during the current scan.

[Coil]
- StandardCoil: The state of the left connection line is assigned to the corresponding BOOL variable (indicated by "***").
- NegatedCoil: The negated value of the left connection line state is assigned to the corresponding BOOL variable (indicated by "***"). If the left connection line state is Off, the corresponding variable is set to On, and if the left connection line state is On, the corresponding variable is set to Off.
- SetCoil: When the state of the left connection line becomes On, the corresponding BOOL variable (indicated by "***") is set to On and remains On until turned Off by the Reset coil.
- ResetCoil: When the state of the left connection line becomes On, the corresponding BOOL variable (indicated by "***") is set to Off and remains Off until turned On by the Set coil.
- RisingEdgeCoil: If the state of the left connection line changes from Off in the previous scan to On in the current scan, the value of the corresponding BOOL variable (indicated by "***") becomes On only during the current scan.
- FallingEdgeCoil: If the state of the left connection line changes from On in the previous scan to Off in the current scan, the value of the corresponding BOOL variable (indicated by "***") becomes On only during the current scan.

[Others]
- Inverter: The state of the Incoming edge is inverted and passed to the Outgoing edge.
- FunctionBlock: Represents a function block.
- Variable: Represents the variable corresponding to the function.

### Guidelines
- Use the networkX library to define and manipulate the graph structure.
- Each rung in ladder logic must be represented as a separate directed graph.
- If the input describes multiple functionalities or rungs, your code should generate multiple graphs accordingly.
\end{shk}
\noindent\begin{minipage}{\textwidth}
\captionsetup{type=figure}
\captionof{figure}{System prompt for Code}
\end{minipage}
\end{document}
