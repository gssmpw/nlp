\section{Applications}
\label{section:applications}
We now demonstrate how our survey findings can be used to create accessibility-oriented analytical maps and personalized routing algorithms. We first synthesize our  findings into user preferences before describing our two prototypes.

\subsection{User Preferences}
While ~\autoref{section:findings} was largely organized around barrier types, here we summarize findings by mobility aid. 
Our intent is to provide a more holistic synthesis across different survey parts and demonstrate how this data can be used to create more personalized, disability-infused mapping applications.

\begin{figure}[b]
    \centering
    \includegraphics[width=1\linewidth]{figures/figure-most-impassable-2column.png}
    \caption{Examples of the least passable images across mobility groups.}
    \Description{This figure shows an array of six examples of the least passable images for each mobility group.}
    \label{fig:least-passable}
\end{figure}

\textbf{Walking canes.}
Walking cane users generally showed more confidence in maneuvering through or around sidewalk barriers compared to other groups. However, they still perceive high severity obstacles and high severity surface problems to be challenging (37\% and 44\% passable votes, respectively). 
The top two most difficult sidewalk barriers for walking cane users were overgrown vegetation on an already narrow sidewalk and branches obstructing the walkway (\autoref{fig:least-passable}A and B), with only 19\% and 23\% of users, respectively, indicating they could confidently pass.

\begin{figure*}
  \centering
  \includegraphics[width=\linewidth]{figures/access-score-maps.png}
  \caption{AccessScore maps visualizing sidewalk quality in Seattle for two groups: walking cane and mobility scooter (red is least accessible; green is most). Top two shows AccessScore by neighborhood; bottom two shows AccessScore by sidewalk segment. From the comparisons between walking cane users and mobility scooter users, we can see while downtown area may be equally accessible for both user groups, other areas are less accessible for mobility scooter users. }
  \Description{This figure shows AccessScore maps visualizing sidewalk quality in Seattle. Top two shows AccessScore by neighborhood; bottom two shows AccessScore by sidewalk segment. From the comparisons between walking cane users and mobility scooter users, we can see while downtown area may be equally accessible for both user groups, other areas are less accessible for mobility scooter users.}
  \label{fig:fig:access-maps}
\end{figure*}

\textbf{Walkers.}
Walker users were particularly sensitive to narrow sidewalks, including sidewalks narrowed by obstacles such as vegetation (40\% of walker users voted passable), parked cars, scooters, and bikes (32\%), as well as inherently narrow sidewalk surfaces (32\%). 
People who use walkers also struggle with cracks and uneven sidewalks, with more than 45\% of the votes indicating they are difficult to pass. 
The most challenging barriers for walker users were a parked bike in the middle of the sidewalk and branches obstructing the walkway (\autoref{fig:least-passable}C and B), with only 9\% and 10\% of users, respectively, indicating they could pass these obstacles.

\textbf{Mobility scooters.}
Mobility scooter users marked the most images as impassable (24 of 52 images). 
Examining users' passability confidence across severity levels revealed that these users were more likely to find images in both mid- and high-severity levels impassable, with only a 55\% passable ratio. 
This is lower compared to all other mobility aid users: walking cane (74\% ), walker (58\%), manual wheelchair (68\%), and motorized wheelchair (59\%).
Mobility scooter users were also particularly sensitive to poorly designed curb ramps, with a low passibility rate for curb ramps of 49\%. 
The top three most difficult sidewalk barriers for mobility scooter users were overgrown vegetation on a narrow sidewalk (\autoref{fig:least-passable}A), a broken sidewalk surface with mud (\autoref{fig:least-passable}D), and an uplifted sidewalk panel due to tree roots (\autoref{fig:least-passable}E), each with only 14\% of users indicating they could pass these barriers.

\textbf{Manual wheelchairs.}
Manual wheelchair users found high severity obstacles (18\% passable), surface problems (29\% passable), and all missing curb ramps (24\% passable) to be particularly challenging. 
Their top two most difficult sidewalk barriers were overgrown vegetation on a narrow sidewalk (\autoref{fig:least-passable}A) and a pole in the middle of the sidewalk with slope (\autoref{fig:least-passable}F), with only 11\% of users indicating they could pass these obstacles for both barriers.

\textbf{Motorized wheelchairs.}
Motorized wheelchair users showed similar patterns to manual wheelchair users but were even more sensitive to missing curb ramps (20\%  passable). This echoed an insight from one of our pilot participants: \sayit{If I am on a manual wheelchair and I see a missing curb ramp, I can do a wheelie to get on top of it, but it might not be possible when using a motorized wheelchair.} The top two most difficult sidewalk barriers for motorized wheelchair users were overgrown vegetation on a narrow sidewalk (\autoref{fig:least-passable}A) and a parked bike in the middle of the sidewalk (\autoref{fig:least-passable}C), with only 6\% of users indicating they could pass these obstacles for each barrier.

\begin{figure*}
  \centering
  \includegraphics[width=\linewidth]{figures/figure-routing-application-new.png}
  \caption{Routing application using OSMnx to generate routes between A \& B based on our survey data. Yellow route shows the absolute shortest path; teal shows the route for walking cane, this route favours fewer sidewalk barriers regardless of category; purple shows the route for motorized wheelchair, this route avoids missing curb ramps at all costs. When hovering over the labels, users can see what the sidewalk issues look like in streetview.}
  \Description{This figure shows routing application using OSMnx to generate routes between A \& B based on user preferences. Yellow route shows the absolute shortest path; teal shows the route for walking cane, this route favors fewer sidewalk barriers regardless of category; purple shows the route for motorized wheelchair, this route avoids missing curb ramps at all costs.}
  \label{fig:routing}
\end{figure*}

\subsection{Accessibility Map}

High-quality sidewalks play a vital role in the urban environment by encouraging physical activity~\cite{lopez_obesity_2006}, facilitating connectivity~\cite{randall_evaluating_2001}, increasing safety~\cite{abou-senna_investigating_2022}, and enhancing the sense of community~\cite{demerath_social_2003, bise_sidewalks_2018}. 
Current commercial tools like Walk Score~\cite{walk_score_walk_2007} take into account the use of sidewalks in gaining access to important amenities, and have been widely used by people to make informed decisions about where to live and which transportation modes to use. 
However, these tools often fail to capture the nuances of sidewalk accessibility for people with varying levels of mobility. 
The same sidewalk infrastructure can present drastically different levels of quality and usability for mobility aid users.

To address this problem, we prototyped an urban analytic tool that showcases sidewalk quality based on different mobility aid groups using data from our survey.
We used Project Sidewalk open label dataset (curb ramps, missing curb ramps, obstacles, and surface problems) from Seattle\footnote{\href{https://seattle.projectsidewalk.org/api}{https://seattle.projectsidewalk.org/api}} and mapped the labels onto sidewalk geometry gathered from the \textit{Seattle Open Data Portal}\footnote{\href{https://data-seattlecitygis.opendata.arcgis.com/datasets/SeattleCityGIS::sidewalks-1/about}{https://data-seattlecitygis.opendata.arcgis.com/datasets/SeattleCityGIS::sidewalks-1/about}}. 
We extended previous methods of using Project Sidewalk~\cite{li_interactively_2018, hara_scalable_2014, li_pilot_2022} labels to calculate \textit{AccessScore} by incorporating our survey findings.
The confidence that a sidewalk barrier type is not passable ($C_{label}$) was determined using the percentage of \sayit{No} and \sayit{Unsure} votes from ~\autoref{fig:image-selection-results}. For example, $C_{SurfaceProblem}$ for walking cane users is $0.54$, thus we weighted surface problems by $0.54$ when calculating their \textit{AccessScore}.
We generated sidewalk accessibility maps at both segment and neighborhood scales, with scores ranging from 0 (least accessible) to 1 (most accessible). 

\autoref{fig:fig:access-maps} compares the results for walking cane and mobility scooter users. 
The results show that, while downtown Seattle may be accessible for both groups, mobility scooter users face more challenges in other geographic areas. 
Such visualizations act like a Walk Score~\cite{walk_score_walk_2007} for mobility aid users, they can \textit{help people in choosing suitable living locations} and \textit{guide officials in prioritizing accessibility improvements}.

\subsection{Personalized Routing}
Existing navigation tools (\textit{e.g.,} Google Maps, Apple Maps) fail to address the needs of people with mobility disabilities. This section demonstrates how "one-size-fits-all" applications are insufficient for people with different mobility aids and how our survey data enables more accurate personalized routing. 

To develop a routing prototype, we first created a topologically connected routable network for our study area using the sidewalk network from OSM.
We then integrated Project Sidewalk labels by mapping obstacles and surface problems onto sidewalk segments, and (missing curb ramps) were mapped onto the crossing segments. 
To incorporate user profiles, we again used the confidence score that a sidewalk barrier type is not passable ($C_{label}$). 
Then, for each segment in the sidewalk network, we calculated the weighted distance for each segment as the segment length plus $C_{label}$ multiplied by the number of labels and 10\% of the segment length~\cite{tannert_disabled_2018}. 
Using OSMnx~\cite{boeing_osmnx_2017}, we next calculated the shortest distance between two intersection points (30th Avenue and East Columbia Street; 38th Avenue and East Union Street in Seattle) based on these weighted distances.

~\autoref{fig:routing} shows the shortest paths using absolute length and weighted length for walking cane users and motorized wheelchair users, with Project Sidewalk labels overlaid on the map.
The results demonstrate that users are given different optimal paths based on their specific needs and preferences. 
Walking cane users are routed along a path with some missing curb ramps but almost free of surface problems and obstacles, while motorized wheelchair users are given a longer path that avoids all areas with missing curb ramps. 
The results powerfully demonstrate how leveraging crowdsourced accessibility data and user preferences can yield more \textit{accurate and personalized routing algorithms} for mobility aid users.





