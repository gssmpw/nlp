\section{Findings}
\label{section:findings}

In total, we received 190 completed valid responses with a median completion time of 24.5 minutes.
An additional six participants began the survey but did not complete it (drop out rate of 3.2\%). Below, we begin by describing participant demographics before organizing our findings around the survey parts: passability, pairwise comparisons, and rank-ordering. 
We intermix quotes from the qualitative data to complement the quantitative findings. 

\subsection{Participant Demographics}
\label{section:}
A total of 144 individuals participated in the survey, with 34 participants providing responses for multiple mobility aids, resulting in 190 total responses. 
Participants were spread across age categories but leaned younger: 40\% were aged 18–34, 29\% were 35–54, 27\% were 55-74, and 4\% (\textit{N=}5) were 75-94. 
A slight majority identified as women 51\% (\textit{N=}73), 42\% as men, and 8\% as non-binary. Our participants tended to traveled frequently outside the home, most commonly 7+ times a week (38\%) followed by 4-6 times a week (36\%), 1-3 times a week (20\%), while 6\% (\textit{N=}8) traveled 1-3 times a month or less. 
Among the 190 responses, 53 reported using walking canes (28\%), 37 manual wheelchairs (20\%), 34 motorized wheelchairs (18\%), 32 walkers (17\%), 22 mobility scooters (12\%), and 12 others (6\%) such as crutches, rollators, and knee scooters. \autoref{fig:demographics} shows demographics by mobility aid.

\begin{figure*}
    \centering
    \includegraphics[width=1\linewidth]{figures/figure-pairwise-result.jpg}
    \caption{Image comparison result for subcategories of “fire hydrant and poles” (images A1-A6), “cracks and height difference” (images B1-B6), and “curb ramps” (images C1-C6). The images are ordered from left to right based on their Q score. For “fire hydrant and poles”, all mobility groups ranked in the order of A1 to A6 expect cane user group reversed the order of A5 and A6. For cracks and height difference", different groups varied slightly on their rankings for B2/B3 and B5/B6. For “curb ramps”, different groups varied slightly on their choices for C3/C4 and C5/C6. See the full Q score rankings for each group in the Appendix (\autoref{fig:pairwise-result}).}
    \label{fig:pairwise-result}
    \Description{This figure shows the image comparison result for subcategories of “fire hydrant and poles” (images A1-A6); “cracks and height difference” (images B1-B6); and “curb ramps” (images C1-C6). The images are ordered from left to right based on their Q score. For “fire hydrant and poles”, all mobility groups ranked in the order of A1 to A6 expect walking cane user group reversed the order of A5 and A6. For cracks and height difference", different groups varied slightly on their rankings for B2/B3 and B5/B6. For “curb ramps”, different groups varied slightly on their choices for C3/C4 and C5/C6.}
\end{figure*}


\subsection{Passability Assessment}
\label{section: passibility-assessment}
Our analysis of variance revealed significant main effects of mobility aid ($\chi^2(8, N=9,256) = 133.23, p < .001$) and barrier type ($\chi^2(6, N=9,256) = 16.08, p = .013$) on perceived passability. There was also a significant interaction between mobility aid and barrier type ($\chi^2(24, N=9,256) = 105.08, p < .001$).
Overall, mobility scooter users selected the highest number of impassable images---nearly half of all 52 sidewalk barriers shown were deemed impassable (46\%)---followed by users of motorized wheelchairs (43\%), walkers (42\%), manual wheelchairs (40\%), and walking cane users (28\%). Walking cane user responses differed significantly from all the other groups (\autoref{fig:image-selection-results}). Below, we describe our results as a function of high-level barrier type.

\textbf{Obstacles.}
Perceptions of sidewalk obstacle significantly differed among mobility aid groups. 
Walking cane users consistently rated obstacle images as more passable compared to users of walkers, mobility scooters, manual wheelchairs and motorized wheelchairs (all $p <$ .001, \autoref{fig:image-selection-results}). 
In our open-ended question about the most challenging sidewalk barriers, participants frequently cited obstacle-related issues, including parked cars/scooters/bikes, overgrown vegetation, signs, poles, traffic cones, construction, trash cans, \textit{etc}. 
One manual wheelchair user described a common challenge: \sayit{Stationary objects (typically a tree, planter, or pole) residing in the center of the walkway area... while people walking on foot can easily walk around these objects on either side, as a wheelchair user there is almost never enough room for me to pass on either side. This causes me to have to get assistance to get pushed either through a grassy lawn on one side (if this is even an option), or off the high curb and into the street to surpass the obstacle. All of these present major safety risks.}

\textbf{Surface problems.}
In an open-form question asking participants about the most challenging sidewalk barriers that they encountered, \textit{surface problems} were most frequently cited across all groups, contributing to 40\% (185/465) of total issues mentioned. 
Examples include unevenness, cracks, potholes, broken tiles, and damage caused by tree roots. 
A mobility scooter user stated: \sayit{Cracks, potholes, and uneven surfaces can cause instability or even damage to the scooter.}
Similarly, a manual wheelchair user shared the ongoing discomfort caused by surface problems:\sayit{the cracks/seams between sidewalk segments cause a constant, uncomfortable bumping that also shakes my legs off the wheelchair's footplate. Roads are kept smooth. Sidewalks are not.} 

In the image assessment portion of the survey, 7 of 22 surface problems (31\%) were selected as impassable across all groups. However, similar to obstacles, cane users found surface problems significantly more manageable than users of a walker ($p < .05$). 
When examining the three levels of severity within the subcategory of \textit{cracks/broken surfaces + height differences}, we observed that assessment for both low and high severity are similar across mobility aids. 
However, there was a divide in mid severity, with walker and mobility scooters more likely to perceive them as impassible and others as passable (\autoref{fig:surface-problem-example}).

\textbf{Curb ramps.}
Mobility scooters reported the most difficulty with \textit{using} low-quality curb ramps, finding ramps significantly more challenging compared not only to walking cane users ($p< .05$) but also to manual wheelchairs users ($p< .001$). For example, the diagonal curb ramp with a cracked surface in \autoref{fig:curb-ramps-example}A and the unidirectional, narrow curb ramp in \autoref{fig:curb-ramps-example}B received a low percentage of passable votes from mobility scooter users (23\% and 41\%, respectively), while other groups find these ramps to be more passable than not. However, our findings also highlight how \textit{all} mobility aid users are affected by poorly designed curb ramps. As one motorized wheelchair user stated: \sayit{When they [curb ramp] are too steep, the wheelchair gets stuck and the tires just spin.} A manual wheelchair user identified poorly maintained curb cuts as the most challenging barrier, noting, \sayit{The ones that have shifted over the years are especially difficult. I have perform a wheelie to get over them.}

\textbf{Missing curb ramps.}
Walking cane users find missing curb ramps to be less challenging compared to mobility scooter users, manual and motorized wheelchair users (all $p < .01$). 
Missing curb ramps are particularly challenging for wheeled mobility devices. A manual wheelchair user states: \sayit{My manual wheelchair is made out of mountain bike parts. It's meant to be used `off roading' as mountain bikes are typically used. That being said, I still would struggle if there weren't curb cutouts.} Despite being less prohibitive for walking cane users, 22 of 53 participants still cited "missing curbs" as a major barrier, noting they \sayit{require a lot of energy} and \sayit{can pose tripping hazards}. Similarly, 14 of 32 walker users mentioned "missing curbs" as the most difficult
sidewalk barrier, emphasizing that without curb ramps \sayit{large step downs are very difficult}.

\subsection{Pairwise Comparisons}
In the pairwise comparison (Part 2.2), participants were asked to compare images within each of the nine sub categories. Interestingly, our findings show minimal differences across mobility aid groups (no statistical difference observed). For example, \autoref{fig:pairwise-result} (A1-A6) shows the Q score ranking within the \textit{“fire hydrant and pole”} obstacle subcategory, with images arranged from the easiest to the most difficult to pass (left to right). All mobility aid groups produced the same rankings with one small exception: the walking cane user group switched the order of the \textit{pole on a slope} (\autoref{fig:pairwise-result} A5) and the \textit{pole with narrow passage} (\autoref{fig:pairwise-result} A6). We found a similar trend for the other eight barrier subcategories, for example, see \textit{"cracks, broken surfaces and height difference"} in \autoref{fig:pairwise-result} B1-B6 and \textit{"curb ramps"} in \autoref{fig:pairwise-result} C1-C6.

These results indicate that while people’s assessment of individual barriers can differ significantly by mobility aid usage---as we found in in \autoref{section: passibility-assessment}---their comparative judgments of the easiest and most difficult obstacles to navigate are remarkably similar. For instance, a fire hydrant in the middle of the sidewalk with some room to pass on either side (\autoref{fig:pairwise-result} A1) is consistently perceived as more passable by all groups than a pole in the middle of a narrow sidewalk (\autoref{fig:pairwise-result} A6). We present the complete Q score ranking for all nine image groups in the Appendix (\autoref{fig:image-grid}).

\subsection{Barrier Rankings}
In the third and final section of the survey, participants ranked common sidewalk barriers from most difficult (Rank 1) to least (Rank 9)---see instructions in \autoref{fig:survey-part3-ranking}. 
\autoref{tab:survey-part3-ranking-results-table} shows the results as a table and \autoref{fig:survey-part3-ranking-results-chart} shows them as a bump chart; the barriers are sorted by average rank across the five mobility groups from most difficult (top) to least (bottom). Overall, the most difficult barriers were \textit{missing curb ramp} (ranked, on average, 2.4/9), \textit{uneven sidewalk panels} (3.2), and \textit{steep sidewalk slopes} (4.3), while \textit{grass surface} (7.4), \textit{brick/cobblestone} (7.7), and \textit{manhole/utility covers} (7.8) were perceived as least difficult. 

\begin{table}[t]
\centering
    \centering
    \renewcommand{\arraystretch}{1.2}
    \resizebox{\columnwidth}{!}{%
\begin{tabular}{@{}lrrrrrr@{}}
% \toprule
\textbf{Sidewalk barriers} &
  \textbf{\begin{tabular}[c]{@{}r@{}}Avg.\\ Rank\end{tabular}} &
  \textbf{Cane} &
  \textbf{Walker} &
  \textbf{\begin{tabular}[c]{@{}r@{}}Mobility\\ Scooter\end{tabular}} &
  \textbf{\begin{tabular}[c]{@{}r@{}}Manual\\ W.\end{tabular}} &
  \textbf{\begin{tabular}[c]{@{}r@{}}Motorized\\ W.\end{tabular}} \\ \midrule
Missing curb ramp &
  \cellcolor[HTML]{FAC3B2}2.4 &
  \cellcolor[HTML]{FDE1AF}4.4 &
  \cellcolor[HTML]{F9B7B3}1.6 &
  \cellcolor[HTML]{F9B8B3}1.7 &
  \cellcolor[HTML]{FAC4B2}2.5 &
  \cellcolor[HTML]{F9BAB3}1.8 \\
Uneven sidewalk panel &
  \cellcolor[HTML]{FBCFB1}3.2 &
  \cellcolor[HTML]{F9BEB3}2.1 &
  \cellcolor[HTML]{FAC1B2}2.3 &
  \cellcolor[HTML]{FAC9B2}2.8 &
  \cellcolor[HTML]{EAE991}6.6 &
  \cellcolor[HTML]{F9BDB3}2.0 \\
Steep slope &
  \cellcolor[HTML]{FDE0AF}4.3 &
  \cellcolor[HTML]{F9B7B3}1.6 &
  \cellcolor[HTML]{FEE7AE}4.8 &
  \cellcolor[HTML]{F8EFA4}5.8 &
  \cellcolor[HTML]{FBD3B1}3.5 &
  \cellcolor[HTML]{FAEFA6}5.7 \\
Broken surface/cracks &
  \cellcolor[HTML]{FDE1AF}4.4 &
  \cellcolor[HTML]{FBCDB1}3.1 &
  \cellcolor[HTML]{F9BAB3}1.8 &
  \cellcolor[HTML]{FCDAB0}3.9 &
  \cellcolor[HTML]{E1E685}7.1 &
  \cellcolor[HTML]{F3ED9D}6.1 \\
Narrow sidewalk &
  \cellcolor[HTML]{FDE4AF}4.6 &
  \cellcolor[HTML]{FDE0AF}4.3 &
  \cellcolor[HTML]{FBCCB1}3.0 &
  \cellcolor[HTML]{FAEFA6}5.7 &
  \cellcolor[HTML]{FCF0A9}5.6 &
  \cellcolor[HTML]{FDE6AF}4.7 \\
Sand/gravel &
  \cellcolor[HTML]{FEEDAE}5.2 &
  \cellcolor[HTML]{FCD8B0}3.8 &
  \cellcolor[HTML]{FDE6AF}4.7 &
  \cellcolor[HTML]{EAE991}6.6 &
  \cellcolor[HTML]{FCDDB0}4.1 &
  \cellcolor[HTML]{E8E88E}6.7 \\
Grass surface &
  \cellcolor[HTML]{DCE47E}7.4 &
  \cellcolor[HTML]{F5ED9F}6.0 &
  \cellcolor[HTML]{DFE582}7.2 &
  \cellcolor[HTML]{D5E174}7.8 &
  \cellcolor[HTML]{E1E685}7.1 &
  \cellcolor[HTML]{C3DA5C}8.8 \\
Brick/cobblestone &
  \cellcolor[HTML]{D7E277}7.7 &
  \cellcolor[HTML]{E7E88C}6.8 &
  \cellcolor[HTML]{E7E88C}6.8 &
  \cellcolor[HTML]{BFD857}9.0 &
  \cellcolor[HTML]{D1DF6F}8.0 &
  \cellcolor[HTML]{D3E072}7.9 \\
Manhole covers &
  \cellcolor[HTML]{D5E174}7.8 &
  \cellcolor[HTML]{DFE582}7.2 &
  \cellcolor[HTML]{DAE37B}7.5 &
  \cellcolor[HTML]{CADD66}8.4 &
  \cellcolor[HTML]{C7DB61}8.6 &
  \cellcolor[HTML]{E1E685}7.1 \\ 
  % \bottomrule
\end{tabular}%
}
    \caption{The rank-order results from Part 3 using the Kemeny-Young rank aggregation method. The barriers are sorted by the average rank position across groups from the most difficult (top) to the least (bottom). Table cells are colored in pink (most difficult) and green (least difficult) scale.}
    \label{tab:survey-part3-ranking-results-table}
\end{table}

\textit{Missing curb ramps} were consistently ranked as the most challenging barrier except by walking cane users, who ranked it sixth and, instead, ranked \textit{steeply sloped sidewalks} first. As one manual wheelchair user said: \sayit{high curbs without ramps} is the most difficult barrier as it \sayit{forces me to seek for alternative routes}. 
Interestingly, all five groups rated \textit{grassy sidewalks}, \textit{brick/cobblestone surfaces}, and \textit{manhole/utility panel covers} as least difficult, although there was a slight variation in the exact order across groups. 
While \textit{steep sidewalks} posed significant problems to cane users (Rank 1) and manual wheelchairs (Rank 2), they were less challenging to the powered aids: motorized wheelchairs (Rank 4) and mobility scooters (Rank 5), as well as the walkers (Rank 6). 
It is not just the uphills that are impediments---requiring significant strength and endurance to overcome, but also the downhill slopes. 
As one rollator user commented: \sayit{the sharp downhill declines require a lot of braking and care.} The mobility scooter rankings and motorized wheelchair rankings are most similar but differ slightly in their ranking of \textit{path narrowness} (Rank 4 \textit{vs.} 3), \textit{broken surfaces} (Rank 3 \textit{vs.} 5), and \textit{sidewalk steepness} (Rank 5 \textit{vs.} 4).

\begin{figure}[t]
    \centering
    \includegraphics[width=1\linewidth]{figures/figure-bump-chart.png}
    \caption{The rank-order results from Part 3 presented in a bump-chart. All groups except \textit{walking cane} rank missing curb ramps as most challenging. All five groups ranked grass surface, brick/cobblestone, and manhole covers (utility panels) as the least challenging barrier (though order changed slightly across groups).}
    \Description{This is an infographic image. It shows the rank order results from Survey Part 3 using the Kemeny-Young rank aggregation method. The barriers listed in the bump chart are sorted by the average rank position across groups from the most difficult (top) to the least (bottom). All groups except walking cane rank missing curb ramps as most challenging. All five groups ranked grass surface, brick/cobblestone, and manhole covers (utility panels) as the least challenging barrier (though the order changed slightly across groups).}
    \label{fig:survey-part3-ranking-results-chart}
\end{figure}





