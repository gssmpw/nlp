\section{Method}

To study how people who use mobility aids perceive sidewalk barriers, we created an image-based online survey. 
Participants were shown a curated set of images that contain sidewalk barriers and asked to respond accordingly based on their lived experience and their mobility aid usage. 
We aim to address three overarching research questions (RQs):
\begin{itemize}
    \item RQ1: How do people with different mobility aids perceive mobility barriers?
    \item RQ2: What are the key similarities and differences between mobility aid groups?
    \item RQ3: What types of barriers are the most severe and why?
\end{itemize}

Below, we describe our study method, including the iterative survey design and development, participant recruitment, and our data and analysis. 

\subsection{Study Method}
We begin by outlining our five primary mobility aid groups, followed by an overview of our sidewalk image dataset and the survey.

\subsubsection{User Groups}
Many different types of sensory~\cite{giudice_blind_2008,raina_relationship_2004}, cognitive~\cite{pillette_systematic_2023,buchman_cognitive_2011}, and physical disabilities~\cite{groessl_physical_2019,garg_associations_2016} can impact mobility. 
We specifically focus on people with ambulatory disabilities that require a mobility aid, including \textit{walking canes}, \textit{walkers}, \textit{mobility scooters}, \textit{manual wheelchairs}, and \textit{motorized wheelchairs}.
These categories were informed by the \textit{National Household Travel Survey} (NHTS)~\cite{us_department_of_transporation_national_2022}, \textit{National Survey on Health and Disability} (NSHD)~\cite{the_university_of_kansas_national_2018}, and \textit{Canadian Survey on Disability}~\cite{government_of_canada_canadian_2022}, as well as insights from our multi-disciplinary research team. 
We included the mobility devices that appeared consistently across all three surveys~\cite{us_department_of_transporation_national_2022, the_university_of_kansas_national_2018, government_of_canada_canadian_2022}, except for \textit{crutches}, which we excluded as a main category due to their temporary nature~\cite{manocha_injuries_2021}.
We also excluded several other aids as main categories: \textit{white canes}~\cite{us_department_of_transporation_national_2022, the_university_of_kansas_national_2018}, as they are typically used by blind and low-vision individuals and our survey relied on visual examination of images; \textit{artificial limbs} or \textit{prosthetics}~\cite{the_university_of_kansas_national_2018, government_of_canada_canadian_2022}, as they are highly customized to individuals and provide fewer insights when studying mobility aid users as a group; and \textit{service animals}~\cite{us_department_of_transporation_national_2022, the_university_of_kansas_national_2018}, as they are not \textit{device aids}. 
\autoref{tab:survey-groups} provides a comprehensive list of aids mentioned in the three surveys, accompanied by a rationale for each aid that was excluded from our study as a user group.
To acknowledge the diversity of mobility devices, we include an \textit{other} category where users can specify alternative mobility aids not covered by the main categories.

\begin{figure*}
    \centering
    \includegraphics[width=0.75\linewidth]{figures/figure-part3-ranking.png}
    \caption{In Part 3, the survey asked participants to rank order nine types of sidewalk barriers, including: \textit{missing curb ramps}, \textit{narrow sidewalks}, \textit{brick/cobblestone surfaces}, \textit{uneven sidewalk panels}, \textit{steep slopes}, \textit{broken surfaces/cracks}, \textit{grass surfaces}, \textit{sand/gravel surfaces}, and \textit{manholes on sidewalks}. To elicit the most accurate responses, participants could click on the image icon to see an example image for each barrier.}
    \Description{This figure shows a screenshot from the online survey. In Part 3, the participants were asked to rank order nine types of sidewalk barriers, including: missing curb ramps, narrow sidewalks, brick/cobblestone surfaces, uneven sidewalk panels, steep slopes, broken surface/cracks, grass surfaces, sand/gravel surfaces and manholes on sidewalks. To ground responses, participants could click on the image icon to see an example image for each  barrier.}
    \label{fig:survey-part3-ranking}
\end{figure*}

\subsubsection{Sidewalk Image Dataset}
Our survey uses images as the primary stimulus: we show participants example sidewalk images and ask them to respond given their lived experience and specific mobility aid usage. 
The research team selected these images from \textit{Project Sidewalk}~\cite{saha_project_2019}---an open-source web tool where users virtually find, label, and rate sidewalk conditions through interactive streetscape imagery.
Project Sidewalk is deployed in 21 cities across eight countries, amassing over one million labeled sidewalk images. 
As the built environment can differ by country and geographic context (\textit{e.g.,} rural \textit{vs.} urban), our initial focus is on studying North American infrastructure. Thus, our images derive primarily from North American cities: Seattle, WA; Oradell, NJ; Chicago, IL; Columbus, OH; and Mexico City, Mexico.

To select and curate our image dataset, we used Project Sidewalk's \textit{Image Gallery} tool~\cite{duan_sidewalk_2021}, which provides an interactive gallery of all labeled sidewalk images filterable based on seven high-level sidewalk feature and barrier categories (\textit{e.g.,} curb ramps, surface problems, obstacles), ~40 tag categories (\textit{e.g.,} uplifts, cracks, cobblestone), and a five-point severity scale. Through an iterative process of selection and discussion across three research members, we finalized a dataset of 52 images covering four major label types and nine tag categories---see \autoref{fig:image-dataset-grid} and ~\autoref{tab:label-categories}. To  pinpoint key issues, we consolidated the five-point severity scale into a three-point scale: high, medium, and low. For each tag category, we selected two images per severity level, as shown in \autoref{tab:label-categories}.

Our overarching goal was to curate a dataset that showcased a variety of common sidewalk accessibility problems of varying severities. To allow others to build on our research, our dataset is available on Github\footnote{\href{https://github.com/makeabilitylab/accessibility-for-whom}{https://github.com/makeabilitylab/accessibility-for-whom}}.

\begin{figure*}
    \centering
    \includegraphics[width=1\linewidth]{figures/figure-demographics.png}
    \caption{Participant demographics (\textit{N=}144) broken down by mobility aid, including age, gender, and travel frequency. \textit{n.b.} is non-binary; \textit{1-3/m} is 1-3 travels per month or less,  \textit{1-3/w} is 1-3 travels per week, \textit{4-6/w} is 4-6 travels per week, \textit{7+/w} is 7 travels per week or more.}
    \Description{This is a data visualization of bar plots showing participant demographics broken down by mobility aid, including age, gender, and travel frequency.}
    \label{fig:demographics}
\end{figure*}

\begin{figure*}
    \centering
    \includegraphics[width=1\linewidth]{figures/figure-image-selection-results.png}
    \caption{Passability assessment results for all 52 images and categories of \textit{obstacle}, \textit{surface problem}, \textit{curb ramp} and \textit{missing curb ramp}. Lines on top of bar charts represent results of analysis of variance based on mixed multinomial logistic regression, *$p<.05$, **$p< .01$, ***$p<.001$.}
    \Description{This is a data visualization of bar plots of passibility assessment results for 52 images all together and categories of obstacle, surface problem, curb ramp and missing curb ramp. Lines on top of bar charts represent results of analysis of variance based on mixed multinomial logistic regression.}
    \label{fig:image-selection-results}
\end{figure*}

\subsubsection{Survey Design}
Our survey had three parts: (1) study overview and background information, (2) image-based sidewalk passability rating and pair-wise comparisons, and (3) a ranking of sidewalk factors. 
The full survey is available as a PDF in supplementary material and online at \href{https://sidewalk-survey.github.io/}{https://sidewalk-survey.github.io/}.

\textbf{Part 1: Background information.}
The survey began with a study description and informed consent. 
The opening page stated the study goal was \sayit{to understand how people using different mobility devices perceive barriers in urban environments} and explained the \$10 USD remuneration, as well as that participants could save their responses and return to the survey later. 
The survey then collected basic demographic and mobility aid information. 
If respondents indicated that they used multiple mobility aids, we asked which \sayit{they use more frequently when going outside your home}. 
We then asked an open-form question about \sayit{What are the most difficult sidewalk barriers you encounter [using that mobility aid]?}. 

\textbf{Part 2.1: Image-based passability ratings.}
Part 2 included two sub-parts: (2.1) image-based passability ratings and (2.2) pair-wise comparisons. 
In Part 2.1, participants were shown images of sidewalk barriers and asked to judge their passability. 
Specifically, for each image, we asked, \sayit{When using your [mobility aid], do you feel confident passing this?} (\autoref{fig:survey-part2-instructions}); 
participants could select \sayit{Yes,} \sayit{No,} or \sayit{Unsure}. 
For each image, we added a salient red dot highlighting the target of interest and instructed participants to focus on the red dot when responding. 
To help guide participants and gather consistent responses, Part 2.1 began with an interactive tutorial showing an example image with the prompt: \sayit{Imagine yourself encountering these situations in real life. Would you be able to pass by the barrier?} (\autoref{fig:survey-part2-instructions}a).
The interactive tutorial then provided a definition of passability (\autoref{fig:survey-part2-instructions}b). 

After presenting instructions, the survey showed participants individual sidewalk images and, for each, voted on passability. 
Images were grouped into the nine distinct sets based on sidewalk feature or barrier type, including \textit{curb ramps}, \textit{surface problems}, and \textit{obstacles}---\autoref{tab:label-categories} and \autoref{fig:image-dataset-grid}. 
To mitigate ordering effects, we randomized both the sequence of the nine image sets as well as the order of images within each set. After a single image set was completed, the results were used to compute dynamic pairwise comparisons, and the participant entered Part 2.2.


\textbf{Part 2.2: Pairwise comparisons}
For Part 2.2, participants were shown the same images from Part 2.1 but asked to compare them with the question: \sayit{When using your [mobility aid], which do you feel more confident passing?}---see \autoref{fig:survey-part2b-pairwise}. 
Participants could select the \sayit{left image,} \sayit{right image,} or \sayit{the same}. 
These visual comparison studies are becoming increasingly common in urban science to evaluate perceptions of safety~\cite{salesses_collaborative_2013}, bikeability~\cite{evans-cowley_streetseen_2014}, beauty~\cite{goodspeed_research_2017}, and more. 
Comparing all 52 images to one another, however, would require 1,326 pairwise comparisons---an intractable number. 
Thus, we developed a different strategy informed by~\cite{harker_incomplete_1987}: first, images were compared only within their image set from Part 2.1. 
Second, images marked \sayit{Yes} were placed into one comparison set, while those marked \sayit{No} were placed in another; images marked \sayit{Unsure} were placed in both. 
This grouping strategy allowed for a more nuanced analysis of \sayit{Unsure} responses and reduced the total number of comparisons. 
In total, each participant could make between 6 and 15 comparisons depending on their Part 2.1 responses. 
Once they completed pairwise comparisons of a given image set, participants began another set, starting again in a new Part 2.1 set. 
This process was repeated until all nine image sets were completed.

\textbf{Part 3: Ranking of sidewalk barriers.} 
Finally, we asked participants to rank-order nine types of sidewalk barriers drawn from the literature~\cite{kasemsuppakorn_personalised_2009,kasemsuppakorn_understanding_2015,saha_project_2019, wheeler_personalized_2020} as well as from ADA guidelines~\cite{us_department_of_justice_2010_2010}, including \textit{uneven sidewalk panels}, \textit{narrow sidewalks}, \textit{missing curb ramps}, and \textit{sand/gravel surfaces} (\autoref{fig:survey-part3-ranking}). 
To aid comprehension, each barrier was accompanied by an example image (viewable by clicking on the image icon), and we randomized the initial rank-order options. 

\textbf{Survey completion.}
The survey ended with a thank-you and contact address. Participants who reported using multiple mobility aids in Part 1 were given the option to re-take Part 2 for their other selected aid(s). These participants could complete the additional survey immediately, defer, or decline.

\begin{figure*}
    \centering
    \includegraphics[width=1\linewidth]{figures/figure-surface-problem-example.jpg}
    \caption{Passibility assessment results for category \textit{`cracks/broken surfaces + height differences'} on three severity levels of (A) low, (B) mid, and (C) high. While the assessment for both low and high severity are similar across mobility aids, we see a divide in the mid severity category, with walker and mobility scooters more likely to perceive them as impassible.}
    \label{fig:surface-problem-example}
    \Description{This figure is a data visualization of bar plots of passibility assessment results of category ‘cracks. Broken surfaces + height differences; on three severity levels of low, mid, and high. Images A,B,C show an example of each of the severity levels. While the assessment for both low severity and high severity are similar across mobility aids, we see a divide in the mid severity, with walker and mobility scooters more likely to perceive them as impassable.}
\end{figure*}

\subsection{Iterative Survey Development}
We designed the web survey in \textit{Figma} and implemented it in \textit{ReactJS v18.2}~\footnote{https://react.dev} (frontend) and \textit{Firebase v10.13.1}~\footnote{https://firebase.google.com/} (backend) hosted on \textit{GitHub Pages}. See our Github repo for details~\footnote{\href{https://github.com/makeabilitylab/accessibility-for-whom}{https://github.com/makeabilitylab/accessibility-for-whom}}. To design the survey, we used a human-centered, iterative process starting with with five rounds of internal testing amongst the research team and then four external pilots with mobility aid users. For the latter, we conducted a think-aloud session via Zoom and participants sharing their screens. Each session lasted for 60-90 minutes and participants were compensated \$30 for their time. 

Overall, pilot participants responded positively to the study topic, questions, and survey format. For example, pilot participant 3 (a manual wheelchair user) stated, \sayit{It’s very simple, this medium really helps because it’s also like a quick question, and quick decision[s] you have to make in real time. So you're triggering this similar emotional response that would be triggered from a real-life scenario.} Based on pilot study findings, we: 
(1) improved ranking question design with example images; 
(2) refined image sets to avoid overly similar or unclear images; 
(3) clarified terminology (\textit{e.g.,} "manholes" instead of "utility panels"); 
(4) implemented UI improvements, including increased font sizes for explainer texts and exit confirmation alerts.

\subsection{Survey Advertising and Recruitment}
Participants were recruited through disability organizations, social media, and word of mouth. 
Study advertisements linked to a screener, which asked about demographics, mobility aid use, and vision loss.
The screener questions and response options were designed based on prior surveys ~\cite{us_department_of_transporation_national_2022, the_university_of_kansas_national_2018, government_of_canada_canadian_2022, guensler_sidewalk_2017}.
See supplementary material for the complete list of screener questions.
We filtered for adults (18+) who use a mobility aid (\textit{walking cane}, \textit{walker}, \textit{mobility scooter}, \textit{manual wheelchair}, and \textit{motorized wheelchair}). 
Because our survey relied on a visual examination of images, we also excluded users of screen readers. 
The survey was posted for about two months in the summer of 2024. 

Similar to other recent online surveys~\cite{lawlor_suspicious_2021,griffin_ensuring_2022}, we experienced problems with fraudulent sign-ups. To mitigate this, we filtered screening responses based on IP address (no duplicates) as well as the respondent's qualitative descriptions of a prompt image (\textit{N=}5,239). 
In addition, we filtered out a smaller number of actual survey responses (\textit{N=}68) based on a combination of improbable completion times, IP and email address duplication, small screen sizes (making it difficult to see images), and whether the open-form responses seemed to be AI-generated (\textit{e.g.,} nonsensical responses).

\begin{figure*}
    \centering
    \includegraphics[width=1\linewidth]{figures/figure-curb-ramps-example.jpg}
    \caption{Passibility assessment results for category curb ramps (A \& B) and missing curb ramps (C). From the bar charts, we can see poorly designed or badly maintained curb ramps seem to be restrictive for people using mobility scooters. For missing curb ramps, wheeled mobility users perceive them as more challenging than walking cane and walker users.}
    \label{fig:curb-ramps-example}
    \Description{This figure is a data visualization of bar plots of passibility assessment results  for category curb ramps (image A \& B) and missing curb ramps (image C). From the bar charts we learned that poorly designed or badly maintained curb ramps come across as restrictive for people using mobility scooters. As for missing curb ramps, wheeled mobility users perceive them more challenging than walker and walking cane users}
\end{figure*}

\subsection{Data and Analysis}
We describe our analysis approach for each part of the survey. For the open-form questions, we employed a qualitative open coding method~\cite{charmaz_constructing_2006}, where one researcher developed a set of deductive themes based on the sidewalk barrier categories, then coded the responses accordingly. 
For survey Part 2.1, the passability assessment, we calculated the counts and the percentages of \say{Yes,} \say{No,} and \say{Unsure} votes for each mobility aid group. 
To investigate the differences in perceived passibility between different mobility groups, we conducted an analysis of variance based on mixed multinomial logistic regression, implemented using the multinomial-Poisson transformation~\cite{baker_multinomial-poisson_1994, guimaraes_understanding_2004, chen_note_2001}.  \textit{Post hoc} pairwise comparisons were conducted using the multinomial-Poisson transformation~\cite{baker_multinomial-poisson_1994} and corrected with Holm’s sequential Bonferroni procedure~\cite{holm_simple_1979}.

For Part 2.2, the pairwise comparison, we used Q score~\cite{salesses_collaborative_2013}, which is commonly employed in urban science literature for street-scene comparisons ~\cite{zhang_measuring_2018, goodspeed_research_2017, salesses_collaborative_2013, evans-cowley_streetseen_2014}. 
While some prior work uses a basic Win Ratio statistic~\cite{goodspeed_research_2017}, Q Score can accommodate tie scenarios, which, in our case, included pairwise results where the participant selected \say{Unsure.} 
Q score enhances Win Ratio of a certain image by incorporating the average Win Ratio of images it was preferred over, while subtracting the average loss ratios of images that were chosen over it~\cite{salesses_collaborative_2013}. 
Q score ranges from 0 to 10, with 10 indicating the most passable image and 0  the least (for that image set). 
To analyze our Q score data, we ranked the Q scores within each image subcategories and mobility group, and then conducted mixed-effects ordinal logistic regression on ranks~\cite{hedeker_random-effects_1994}.

Finally, we used Kemeny-Young rank aggregation~\cite{young_condorcets_1988, young_optimal_1995, kemeny_mathematics_1959} to analyze the rank order question in Part 3. Kemeny-Young is a prominent rank aggregation method in social choice theory~\cite{hamm_computing_2021}; it is based on the Kendall's Tau distance between rankings and outputs a consensus ranking that minimizes the sum of distances to the input rankings. 