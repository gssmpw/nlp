\section{Discussion}
Through our application of personalized accessibility maps and routing applications, we showed how data and insights from our survey findings can help inform the development of more accurate navigation and analytical tools. 
We now situate our findings in related work, highlight how this survey contributes to personalized routing and accessibility mapping for mobility disability groups as well as present directions for future research.

\subsection{Online Image Survey Method}
In this study, we conducted a large-scale image survey (\textit{N=}190) to gather perceptions of sidewalk barriers from different mobility aid user groups. 
This approach helped us to collect insights on the differences between mobility aid user groups as well as shared challenges.
Previous research exploring the relationship between mobility aids and physical environment have mainly employed methods including in-person interviews~\cite{rosenberg_outdoor_2013}, GPS tracking~\cite{prescott_exploration_2021, prescott_factors_2020,rosenberg_outdoor_2013}, and online questionnaires~\cite{carlson_wheelchair_2002}. While interviews and tracking studies typically yield rich detailed information, they are limited to a small sample size. Online text based questionnaires often achieve larger sample sizes but at a cost of depth and nuance. Our image survey method struck a balance between sample size and detail. We collected a large sample within a relatively short time frame, enabling us to gather valuable insights and synthesize patterns across user groups.

Despite advantages, our approach has some limitations. Although street view images help situate and ground a participant's response---as one pilot participant said ``\textit{You're triggering a similar response to a real-life scenario''}, they cannot fully replicate the experience of evaluating a sidewalk \textit{in situ}. The lack of physical interaction with the environment limits the assessment of certain factors. For instance, one of our pilot participants noted that determining whether they could navigate past an obstacle like a trash can varies depending on \sayit{whether the trash can is light enough so I can push it away.} Using our findings as a backdrop, future work should conduct follow-up interviews and in-person evaluations. Such approaches would complement the quantitative data with richer qualitative insights, allowing researchers to better understand the patterns observed in quantitative data as well as the reasoning behind mobility aids users’ assessment.

\subsection{Personalized Accessibility Maps}
Our approach to infuse accessibility maps and routing algorithms with personalized information contributes to the field of accessible urban navigation and analytics. 
Based on our findings, we implemented two accessibility-oriented mapping prototypes, which demonstrate how our data can be used in urban accessibility analytics and personalized routing algorithms. While our current implementation serves as a proof of concept, future research could explore using our findings with more advanced modeling methods such as fuzzy logic~\cite{kasemsuppakorn_personalised_2009, gharebaghi_user-specific_2021, hashemi_collaborative_2017} and AHP~\cite{kasemsuppakorn_personalised_2009,kasemsuppakorn_understanding_2015, hashemi_collaborative_2017}. 

For our current map applications, we used a single set of open-source sidewalk data from Project Sidewalk. However, we acknowledge that other important factors are not included, such as sidewalk topography, width, stairs, crossing conditions, paving material, lighting conditions, weather, and pedestrian traffic~\cite{rosenberg_outdoor_2013,kasemsuppakorn_personalised_2009,darko_adaptive_2022,hashemi_collaborative_2017,sobek_u-access_2006,bigonnesse_role_2018}. 
Future work should build upon our foundation by incorporating more crowdsourced and government official datasets.

While mobility aids play a crucial role in determining accessibility needs, we must recognize that individuals using the same type of mobility aid may have diverse preferences. As one of our pilot participants stated, \sayit{your wheelchair has to be shaped and fitted to your body similar to how you need shoes specifically for your feet.} This insight underscores the need for personalization beyond broad mobility aid categories. Other factors including age~\cite{rosenberg_outdoor_2013}, disability type~\cite{prescott_factors_2020}, body strength~\cite{prescott_factors_2020}, and route familiarity~\cite{kasemsuppakorn_understanding_2015} should be explored in the future. Our attempt in creating personalized maps is not to provide a one-size-fits-all solution for generalized mobility aid groups, but rather to leverage the power of defaults~\cite{nielsen_power_2005} and offer users an improved baseline from which they can easily customize based on their individual needs.

\subsection{Limitations and Future Work}
Due to the visual nature of our survey—images were the primary stimuli—we specifically excluded people who are blind or have low vision\footnote{That said, the custom online survey was made fully screen reader accessible; see \href{https://sidewalk-survey.github.io/}{https://sidewalk-survey.github.io/} for the images and alt text.}. However, as noted previously, many different disabilities can impact mobility, including sensory, physical, and cognitive. Prior research has explored the incorporation of visually impaired or blind individuals into route generation~\cite{volkel_routecheckr_2008}, recognizing shared barriers and the prevalence of multiple disabilities among users. Building upon this foundation, future work should expand the participant pool to include a broader range of disabilities, thereby providing a more comprehensive understanding of diverse accessibility needs.

While we demonstrated two basic scenario applications, our survey findings and personalized mapping approach have potential for broader implementation. One promising direction is in developing barrier removal strategies for policymakers~\cite{eisenberg_barrier-removal_2022}. Current government plans often rely on simple metrics, such as population density or proximity to public buildings~\cite{seattle_department_of_transportation_seattle_2021}. Our methodology could enhance these efforts by identifying sidewalk barriers whose removal would yield the greatest overall benefit to the largest percentage of mobility aid users in the form of connected, safe, accessible routes.