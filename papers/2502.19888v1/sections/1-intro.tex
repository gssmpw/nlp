\section{Introduction}

\begin{figure*}
    \centering
    \includegraphics[width=1\linewidth]{figures/figure-image-grid.jpg}
    \caption{Our final image dataset consists of 52 images across nine categories, including \textit{fire hydrants + poles}, \textit{overgrown vegetation}, \textit{parked bikes/cars/scooters}, \textit{cracks/height differences}, \textit{bricks/cobblestone + utility panels }, \textit{sand/gravel + grass}, \textit{narrow}, \textit{curb ramp}, \textit{missing curb ramp},  Above, we show two sample images from each category.}
    \Description{
    This figure shows our final image dataset composed of 52 images across nine categories, including poles/fire hydrants, vegetation, parked bikes/cars, cracks/broken surfaces, bricks/cobblestone, sand/gravel/grass, narrow, curb ramp, missing curb ramp, In this figure, we show two sample images from each of the nine categories.
    }
    \label{fig:image-dataset-grid}
\end{figure*}

In 2022, over 18 million U.S. adults reported having a mobility-related disability, with nearly half (49.3\%) using an assistive aid such as a cane, crutches, walker, scooter or wheelchair~\cite{firestine_travel_2024}. Mobility aid users confront an array of environmental barriers in their everyday travel, such as missing curb ramps, uneven sidewalks, and major obstacles on the sidewalk like impassable street furniture or overgrown vegetation~\cite{meyers_barriers_2002, ding_design_2007,rosenberg_outdoor_2013}. These challenges can be mitigated by offering directions that avoid barriers and guide mobility aid users to destinations safely, accurately, and efficiently. However, current navigation systems (\textit{e.g.,} Google Maps) and commercial analytical services (\textit{e.g.,} Walk Score~\cite{walk_score_walk_2007}) fail to account for the unique requirements and preferences of people with mobility disabilities. Several research projects have developed routing and mapping systems that incorporate mobility disabilities, such as \textit{MAGUS}~\cite{matthews_modelling_2003}, \textit{U-Access}~\cite{sobek_u-access_2006}, \textit{AccessScore}~\cite{li_interactively_2018}, and \textit{AccessMap}~\cite{bolten_accessmap_2019}.
Though promising, prior work often overlooks the heterogeneity among users of different devices and focuses predominantly on wheelchair  users~\cite{kasemsuppakorn_personalised_2009,saha_project_2019,kasemsuppakorn_understanding_2015,matthews_modelling_2003}. However, a larger percentage of users gain mobility from canes, crutches, or walkers~\cite{firestine_travel_2024}. This contrast underscores the need to incorporate a wider range of mobility aid users into our mapping tools and to better characterize the unique challenges each group faces ~\cite{shoemaker_development_2010}. For example, a missing curb ramp at an intersection may pose a significant barrier for wheelchair and mobility scooter users but is less challenging for those using walking canes or walkers. 

To examine perceptions of sidewalk barriers across disability groups and to inform the design of future personalized, disability-aware maps, we developed a large-scale online image survey for five mobility groups: walking cane, walker, mobility scooter, manual wheelchair, and motorized wheelchair users. The survey featured a curated set of 52 sidewalk barrier images collected through an online crowdsourcing platform called Project Sidewalk~\cite{saha_project_2019}, the images include nine barrier categories: fire hydrant \& pole, tree \& vegetation, parked bikes/scooter/cars, height difference \& sidewalk cracks, manholes \& brick/cobblestone, grass \& sand/gravel, narrow and (missing) curb ramps (\autoref{fig:image-dataset-grid}). We asked respondents to evaluate their confidence in passing the scenarios from the images while using their respective mobility aid(s). The survey used a combination of rating, ranking, and adaptive pairwise comparison as well as open-ended text questions.


Our findings (\textit{N=}190) suggest that walking cane users were more likely to perceive sidewalk issues as passable, while mobility scooter users were more likely to perceive sidewalk issues as challenges. Although each group faces unique barriers, high-severity obstacles, surface problems, and missing curb ramps impose significant restrictions for all users. Interestingly, while the groups generally agree on the passability of low and high severity issues, their perceptions diverge for mid-severity issues. Notably, wheeled mobility users show higher sensitivity to missing curb ramps compared to walking cane and walker users. Despite variations in assessing individual sidewalk issues, users across all mobility aid types demonstrate consistent judgments when ranking images within each category from most to least passable.

To demonstrate the value of this data to HCI, accessibility, and urban planning researchers, we created two example applications. First, we synthesized user preferences from our findings, to create \textit{interactive accessibility rating maps} based on each user group’s perceived passability. These maps reveal similar yet distinct patterns across different mobility device groups, providing nuanced views of city-wide accessibility levels. Second, we created a \textit{disability-aware routing prototype} based on OSMnx~\cite{boeing_osmnx_2017} to generate personalized, optimal paths for each mobility group. These applications showcase the potential of our data to inform those with disabilities about residential and social choices, provide personalized route planning strategies, and develop analytical tools that identify obstacles and assess the impact of their removal.

\begin{table*}[t]
\centering
\renewcommand{\arraystretch}{1.25}
\resizebox{\linewidth}{!}{%
\begin{tabular}{l|lll|llll|l|l}
 &
  \multicolumn{3}{c|}{\textbf{Obstacle}} &
  \multicolumn{4}{c|}{\textbf{Surface Problems}} &
  \textbf{Curb Ramps} &
  \textbf{Missing Curb Ramps} \\ 
  \hline
\rowcolor[HTML]{F3F3F3}
\textbf{Subcategories} &
  \begin{tabular}[c]{@{}l@{}}Fire hydrant \\ + pole\end{tabular} &
  \begin{tabular}[c]{@{}l@{}}Overgrown \\ vegetation\end{tabular} &
  \begin{tabular}[c]{@{}l@{}}Parked cars, \\ bikes, scooters\end{tabular} &
  \begin{tabular}[c]{@{}l@{}}Cracks \\ + height \\ differences\end{tabular} &
  \begin{tabular}[c]{@{}l@{}}Brick/\\ cobblestone, \\ utility panels\end{tabular} &
  \begin{tabular}[c]{@{}l@{}}Sand/gravel \\ + grass\end{tabular} &
  Narrow &
  Curb Ramps &
  Missing Curb Ramps \\ 
  % \hline
\textbf{Severities} &
  \begin{tabular}[c]{@{}l@{}}2 low, \\ 2 mid, \\ 2 high\end{tabular} &
  \begin{tabular}[c]{@{}l@{}}2 low, \\ 2 mid, \\ 2 high\end{tabular} &
  \begin{tabular}[c]{@{}l@{}}2 low, \\ 2 mid, \\ 2 high\end{tabular} &
  \begin{tabular}[c]{@{}l@{}}2 low, \\ 2 mid, \\ 2 high\end{tabular} &
  \begin{tabular}[c]{@{}l@{}}2 low, \\ 2 mid, \\ 2 high\end{tabular} &
  \begin{tabular}[c]{@{}l@{}}2 low, \\ 2 mid, \\ 2 high\end{tabular} &
  \begin{tabular}[c]{@{}l@{}}2 low, \\ 2 mid, \\ 2 high\end{tabular} &
  \begin{tabular}[c]{@{}l@{}}2 low, \\ 2 mid, \\ 2 high\end{tabular} &
  \begin{tabular}[c]{@{}l@{}}2 low, \\ 2 mid, \\ 2 high\end{tabular} \\
\end{tabular}%
}
\vspace{1em}
\caption{We organized the final image dataset hierarchically across four top-level categories: \textit{obstacles}, \textit{surface problems}, \textit{curb ramps}, and \textit{missing curb ramps} as well as nine subcategories. For each subcategory, we selected two low, mid, and high severity images (as drawn from Project Sidewalk ratings). One exception was the \textit{narrow} subcategory for surface problems, which had two low and mid subcategories only.}
\label{tab:label-categories}
\end{table*}

In summary, our contributions are threefold: 
(1) we present \textit{results from a large-scale survey of people with diverse mobility aids}, providing insights into how specific mobility aids shape people’s perceptions of the built environment;
(2) we demonstrate \textit{how to apply these findings} to generate accessibility rating maps and enhance personalized routing algorithms;
and (3) we contribute an open-source dataset and our analysis code\footnote{\href{https://github.com/makeabilitylab/accessibility-for-whom}{https://github.com/makeabilitylab/accessibility-for-whom}}, enabling researchers and developers to leverage this work to further promote accessibility solutions for mobility aid users. Our work advances HCI/accessibility research and urban planning by complementing existing sidewalk datasets with diverse perspectives from multiple mobility aid users. Our dataset and findings enable the development of more accurate, tailored routing algorithms for people with different mobility needs, while providing urban planners and policymakers with crucial data to prioritize and target accessibility improvements and renovations.

