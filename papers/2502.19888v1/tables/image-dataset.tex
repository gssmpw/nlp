\begin{table*}[t]
\centering
\renewcommand{\arraystretch}{1.25}
\resizebox{\linewidth}{!}{%
\begin{tabular}{l|lll|llll|l|l}
 &
  \multicolumn{3}{c|}{\textbf{Obstacle}} &
  \multicolumn{4}{c|}{\textbf{Surface Problems}} &
  \textbf{Curb Ramps} &
  \textbf{Missing Curb Ramps} \\ 
  \hline
\rowcolor[HTML]{F3F3F3}
\textbf{Subcategories} &
  \begin{tabular}[c]{@{}l@{}}Fire hydrant \\ + pole\end{tabular} &
  \begin{tabular}[c]{@{}l@{}}Overgrown \\ vegetation\end{tabular} &
  \begin{tabular}[c]{@{}l@{}}Parked cars, \\ bikes, scooters\end{tabular} &
  \begin{tabular}[c]{@{}l@{}}Cracks \\ + height \\ differences\end{tabular} &
  \begin{tabular}[c]{@{}l@{}}Brick/\\ cobblestone, \\ utility panels\end{tabular} &
  \begin{tabular}[c]{@{}l@{}}Sand/gravel \\ + grass\end{tabular} &
  Narrow &
  Curb Ramps &
  Missing Curb Ramps \\ 
  % \hline
\textbf{Severities} &
  \begin{tabular}[c]{@{}l@{}}2 low, \\ 2 mid, \\ 2 high\end{tabular} &
  \begin{tabular}[c]{@{}l@{}}2 low, \\ 2 mid, \\ 2 high\end{tabular} &
  \begin{tabular}[c]{@{}l@{}}2 low, \\ 2 mid, \\ 2 high\end{tabular} &
  \begin{tabular}[c]{@{}l@{}}2 low, \\ 2 mid, \\ 2 high\end{tabular} &
  \begin{tabular}[c]{@{}l@{}}2 low, \\ 2 mid, \\ 2 high\end{tabular} &
  \begin{tabular}[c]{@{}l@{}}2 low, \\ 2 mid, \\ 2 high\end{tabular} &
  \begin{tabular}[c]{@{}l@{}}2 low, \\ 2 mid, \\ 2 high\end{tabular} &
  \begin{tabular}[c]{@{}l@{}}2 low, \\ 2 mid, \\ 2 high\end{tabular} &
  \begin{tabular}[c]{@{}l@{}}2 low, \\ 2 mid, \\ 2 high\end{tabular} \\
\end{tabular}%
}
\vspace{1em}
\caption{We organized the final image dataset hierarchically across four top-level categories: \textit{obstacles}, \textit{surface problems}, \textit{curb ramps}, and \textit{missing curb ramps} as well as nine subcategories. For each subcategory, we selected two low, mid, and high severity images (as drawn from Project Sidewalk ratings). One exception was the \textit{narrow} subcategory for surface problems, which had two low and mid subcategories only.}
\label{tab:label-categories}
\end{table*}