%%
%% This is file `sample-sigconf-authordraft.tex',
%% generated with the docstrip utility.
%%
%% The original source files were:
%%
%% samples.dtx  (with options: `all,proceedings,bibtex,authordraft')
%% 
%% IMPORTANT NOTICE:
%% 
%% For the copyright see the source file.
%% 
%% Any modified versions of this file must be renamed
%% with new filenames distinct from sample-sigconf-authordraft.tex.
%% 
%% For distribution of the original source see the terms
%% for copying and modification in the file samples.dtx.
%% 
%% This generated file may be distributed as long as the
%% original source files, as listed above, are part of the
%% same distribution. (The sources need not necessarily be
%% in the same archive or directory.)
%%
%%
%% Commands for TeXCount
%TC:macro \cite [option:text,text]
%TC:macro \citep [option:text,text]
%TC:macro \citet [option:text,text]
%TC:envir table 0 1
%TC:envir table* 0 1
%TC:envir tabular [ignore] word
%TC:envir displaymath 0 word
%TC:envir math 0 word
%TC:envir comment 0 0
%%
%%
%% The first command in your LaTeX source must be the \documentclass
%% command.
%%
%% For submission and review of your manuscript please change the
%% command to \documentclass[manuscript, screen, review]{acmart}.
%%
%% When submitting camera ready or to TAPS, please change the command
%% to \documentclass[sigconf]{acmart} or whichever template is required
%% for your publication.
%%
%%
% \documentclass[sigconf,authordraft]{acmart}
% \documentclass[manuscript,review,anonymous]{acmart}
% \documentclass[sigconf]{acmart}
\documentclass[sigconf,authorversion,nonacm]{acmart}

\ifpdf
\DeclareGraphicsExtensions{.eps,.pdf,.png,.jpg}
\else
\DeclareGraphicsExtensions{.eps}
\fi

% Prevent itemized lists from running into the left margin inside theorems and proofs
\setlist[enumerate]{leftmargin=.5in}
\setlist[itemize]{leftmargin=.5in}

% Add a serial/Oxford comma by default.
\newcommand{\creflastconjunction}{, and~}

% Setting for algorithms
\Crefname{ALC@unique}{Line}{Lines}

% Tikz libraries
\usetikzlibrary{positioning}
\usetikzlibrary{calc}
\usetikzlibrary{shapes.geometric}
\usetikzlibrary{angles}
\usetikzlibrary{decorations.pathreplacing}
\usetikzlibrary{calligraphy}


%%
%% \BibTeX command to typeset BibTeX logo in the docs
\AtBeginDocument{%
  \providecommand\BibTeX{{%
    Bib\TeX}}}

%% Rights management information.  This information is sent to you
%% when you complete the rights form.  These commands have SAMPLE
%% values in them; it is your responsibility as an author to replace
%% the commands and values with those provided to you when you
%% complete the rights form.
% \setcopyright{acmlicensed}
% \copyrightyear{2018}
% \acmYear{2018}
% \acmDOI{XXXXXXX.XXXXXXX}

%% These commands are for a PROCEEDINGS abstract or paper.
% \acmConference[Conference acronym 'XX]{Make sure to enter the correct
%   conference title from your rights confirmation emai}{June 03--05,
%   2018}{Woodstock, NY}
%%
%%  Uncomment \acmBooktitle if the title of the proceedings is different
%%  from ``Proceedings of ...''!
%%
%%\acmBooktitle{Woodstock '18: ACM Symposium on Neural Gaze Detection,
%%  June 03--05, 2018, Woodstock, NY}
% \acmISBN{978-1-4503-XXXX-X/18/06}

\copyrightyear{2025}
\acmYear{2025}
\setcopyright{cc}
\setcctype{by}
\acmConference[CHI '25]{CHI Conference on Human Factors in Computing Systems}{April 26-May 1, 2025}{Yokohama, Japan}
\acmBooktitle{CHI Conference on Human Factors in Computing Systems (CHI '25), April 26-May 1, 2025, Yokohama, Japan}\acmDOI{10.1145/3706598.3713421}
\acmISBN{979-8-4007-1394-1/25/04}

%%
%% Submission ID.
%% Use this when submitting an article to a sponsored event. You'll
%% receive a unique submission ID from the organizers
%% of the event, and this ID should be used as the parameter to this command.
%%\acmSubmissionID{123-A56-BU3}

%%
%% For managing citations, it is recommended to use bibliography
%% files in BibTeX format.
%%
%% You can then either use BibTeX with the ACM-Reference-Format style,
%% or BibLaTeX with the acmnumeric or acmauthoryear sytles, that include
%% support for advanced citation of software artefact from the
%% biblatex-software package, also separately available on CTAN.
%%
%% Look at the sample-*-biblatex.tex files for templates showcasing
%% the biblatex styles.
%%

%%
%% The majority of ACM publications use numbered citations and
%% references.  The command \citestyle{authoryear} switches to the
%% "author year" style.
%%
%% If you are preparing content for an event
%% sponsored by ACM SIGGRAPH, you must use the "author year" style of
%% citations and references.
%% Uncommenting
%% the next command will enable that style.
%%\citestyle{acmauthoryear}


%%
%% end of the preamble, start of the body of the document source.
\begin{document}

%%
%% The "title" command has an optional parameter,
%% allowing the author to define a "short title" to be used in page headers.
\title[Accessibility for Whom? ]{Accessibility for Whom? \\ Perceptions of Sidewalk Barriers Across Disability Groups and Implications for Designing Personalized Maps}

%%
%% The "author" command and its associated commands are used to define
%% the authors and their affiliations.
%% Of note is the shared affiliation of the first two authors, and the
%% "authornote" and "authornotemark" commands
%% used to denote shared contribution to the research.

\author{Chu Li}
\orcid{0009-0003-7612-6224}
\affiliation{%
    \institution{Allen School of Computer Science}
  \institution{University of Washington, USA}
  \country{}
}
\email{chuchuli@cs.washington.edu}

\author{Rock Yuren Pang}
\orcid{0000-0001-8613-498X}
\affiliation{%
    \institution{Allen School of Computer Science}
  \institution{University of Washington, USA}
  \country{}
}
\email{ypang2@cs.washington.edu}

\author{Delphine Labbé}
\orcid{0000-0002-3730-4310}
\affiliation{%
    \institution{Disability and Human Development}
  \institution{University of Illinois at Chicago, USA}
  % \city{Chicago}
  % \state{Illinois}
  \country{}
}
\email{dlabbe@uic.edu}

\author{Yochai Eisenberg}
\orcid{0000-0001-5598-771X}
\affiliation{%
    \institution{Disability and Human Development}
  \institution{University of Illinois at Chicago, USA}
  % \city{Chicago}
  % \state{Illinois}
  \country{}
}
\email{yeisen2@uic.edu}

\author{Maryam Hosseini}
\orcid{0000-0002-4088-810X}
\affiliation{%
    \institution{City and Regional Planning}
  \institution{UC Berkeley, USA}
  % \city{Chicago}
  % \state{Illinois}
  \country{}
}
\email{maryamh@berkeley.edu}

\author{Jon E. Froehlich}
\orcid{0000-0001-8291-3353}
\affiliation{%
    \institution{Allen School of Computer Science}
  \institution{University of Washington, USA}
  % \city{Seattle}
  % \state{Washington}
  \country{}
}
\email{jonf@cs.washington.edu}

%%
%% By default, the full list of authors will be used in the page
%% headers. Often, this list is too long, and will overlap
%% other information printed in the page headers. This command allows
%% the author to define a more concise list
%% of authors' names for this purpose.
\renewcommand{\shortauthors}{Li et al.}

%%
%% The abstract is a short summary of the work to be presented in the
%% article.
\begin{abstract}
Large language model (LLM)-based agents have shown promise in tackling complex tasks by interacting dynamically with the environment. 
Existing work primarily focuses on behavior cloning from expert demonstrations and preference learning through exploratory trajectory sampling. However, these methods often struggle in long-horizon tasks, where suboptimal actions accumulate step by step, causing agents to deviate from correct task trajectories.
To address this, we highlight the importance of \textit{timely calibration} and the need to automatically construct calibration trajectories for training agents. We propose \textbf{S}tep-Level \textbf{T}raj\textbf{e}ctory \textbf{Ca}libration (\textbf{\model}), a novel framework for LLM agent learning. 
Specifically, \model identifies suboptimal actions through a step-level reward comparison during exploration. It constructs calibrated trajectories using LLM-driven reflection, enabling agents to learn from improved decision-making processes. These calibrated trajectories, together with successful trajectory data, are utilized for reinforced training.
Extensive experiments demonstrate that \model significantly outperforms existing methods. Further analysis highlights that step-level calibration enables agents to complete tasks with greater robustness. 
Our code and data are available at \url{https://github.com/WangHanLinHenry/STeCa}.
\end{abstract}

%%
%% The code below is generated by the tool at http://dl.acm.org/ccs.cfm.
%% Please copy and paste the code instead of the example below.
%%
\begin{CCSXML}
<ccs2012>
   <concept>
       <concept_id>10003120.10011738.10011776</concept_id>
       <concept_desc>Human-centered computing~Accessibility systems and tools</concept_desc>
       <concept_significance>500</concept_significance>
       </concept>
   <concept>
       <concept_id>10003120.10003121.10003129</concept_id>
       <concept_desc>Human-centered computing~Interactive systems and tools</concept_desc>
       <concept_significance>500</concept_significance>
       </concept>
   <concept>
       <concept_id>10002951.10003260.10003282.10003296</concept_id>
       <concept_desc>Information systems~Crowdsourcing</concept_desc>
       <concept_significance>500</concept_significance>
       </concept>
 </ccs2012>
\end{CCSXML}

\ccsdesc[500]{Human-centered computing~Accessibility systems and tools}
\ccsdesc[500]{Human-centered computing~Interactive systems and tools}
\ccsdesc[500]{Information systems~Crowdsourcing}

%%
%% Keywords. The author(s) should pick words that accurately describe
%% the work being presented. Separate the keywords with commas.
\keywords{accessibility, online image survey, mapping tools, urban planning}
%% A "teaser" image appears between the author and affiliation
%% information and the body of the document, and typically spans the
%% page.
\begin{teaserfigure}
  \includegraphics[width=\textwidth]{figures/figure-teaser.png}
  \caption{We present a large-scale online image survey gathering perceptions of sidewalk barriers from five mobility groups: users of walking canes, walkers, mobility scooters, manual wheelchairs, and motorized wheelchairs. Findings were used to generate user profiles that informed the design of personalized accessibility maps and routing tools.}
  \Description{This figure shows our pipeline for large-scale online image survey gathering perceptions of sidewalk barriers from five mobility groups: users of walking canes, walkers, mobility scooters, manual wheelchairs, and motorized wheelchairs. Findings were used to generate user profiles that informed the design of personalized accessibility maps and routing tools.}
  \label{fig:teaser}
\end{teaserfigure}

% \received{20 February 2007}
% \received[revised]{12 March 2009}
% \received[accepted]{5 June 2009}

%%
%% This command processes the author and affiliation and title
%% information and builds the first part of the formatted document.
\maketitle

%!TEX root = gcn.tex
\section{Introduction}
Graphs, representing structural data and topology, are widely used across various domains, such as social networks and merchandising transactions.
Graph convolutional networks (GCN)~\cite{iclr/KipfW17} have significantly enhanced model training on these interconnected nodes.
However, these graphs often contain sensitive information that should not be leaked to untrusted parties.
For example, companies may analyze sensitive demographic and behavioral data about users for applications ranging from targeted advertising to personalized medicine.
Given the data-centric nature and analytical power of GCN training, addressing these privacy concerns is imperative.

Secure multi-party computation (MPC)~\cite{crypto/ChaumDG87,crypto/ChenC06,eurocrypt/CiampiRSW22} is a critical tool for privacy-preserving machine learning, enabling mutually distrustful parties to collaboratively train models with privacy protection over inputs and (intermediate) computations.
While research advances (\eg,~\cite{ccs/RatheeRKCGRS20,uss/NgC21,sp21/TanKTW,uss/WatsonWP22,icml/Keller022,ccs/ABY318,folkerts2023redsec}) support secure training on convolutional neural networks (CNNs) efficiently, private GCN training with MPC over graphs remains challenging.

Graph convolutional layers in GCNs involve multiplications with a (normalized) adjacency matrix containing $\numedge$ non-zero values in a $\numnode \times \numnode$ matrix for a graph with $\numnode$ nodes and $\numedge$ edges.
The graphs are typically sparse but large.
One could use the standard Beaver-triple-based protocol to securely perform these sparse matrix multiplications by treating graph convolution as ordinary dense matrix multiplication.
However, this approach incurs $O(\numnode^2)$ communication and memory costs due to computations on irrelevant nodes.
%
Integrating existing cryptographic advances, the initial effort of SecGNN~\cite{tsc/WangZJ23,nips/RanXLWQW23} requires heavy communication or computational overhead.
Recently, CoGNN~\cite{ccs/ZouLSLXX24} optimizes the overhead in terms of  horizontal data partitioning, proposing a semi-honest secure framework.
Research for secure GCN over vertical data  remains nascent.

Current MPC studies, for GCN or not, have primarily targeted settings where participants own different data samples, \ie, horizontally partitioned data~\cite{ccs/ZouLSLXX24}.
MPC specialized for scenarios where parties hold different types of features~\cite{tkde/LiuKZPHYOZY24,icml/CastigliaZ0KBP23,nips/Wang0ZLWL23} is rare.
This paper studies $2$-party secure GCN training for these vertical partition cases, where one party holds private graph topology (\eg, edges) while the other owns private node features.
For instance, LinkedIn holds private social relationships between users, while banks own users' private bank statements.
Such real-world graph structures underpin the relevance of our focus.
To our knowledge, no prior work tackles secure GCN training in this context, which is crucial for cross-silo collaboration.


To realize secure GCN over vertically split data, we tailor MPC protocols for sparse graph convolution, which fundamentally involves sparse (adjacency) matrix multiplication.
Recent studies have begun exploring MPC protocols for sparse matrix multiplication (SMM).
ROOM~\cite{ccs/SchoppmannG0P19}, a seminal work on SMM, requires foreknowledge of sparsity types: whether the input matrices are row-sparse or column-sparse.
Unfortunately, GCN typically trains on graphs with arbitrary sparsity, where nodes have varying degrees and no specific sparsity constraints.
Moreover, the adjacency matrix in GCN often contains a self-loop operation represented by adding the identity matrix, which is neither row- nor column-sparse.
Araki~\etal~\cite{ccs/Araki0OPRT21} avoid this limitation in their scalable, secure graph analysis work, yet it does not cover vertical partition.

% and related primitives
To bridge this gap, we propose a secure sparse matrix multiplication protocol, \osmm, achieving \emph{accurate, efficient, and secure GCN training over vertical data} for the first time.

\subsection{New Techniques for Sparse Matrices}
The cost of evaluating a GCN layer is dominated by SMM in the form of $\adjmat\feamat$, where $\adjmat$ is a sparse adjacency matrix of a (directed) graph $\graph$ and $\feamat$ is a dense matrix of node features.
For unrelated nodes, which often constitute a substantial portion, the element-wise products $0\cdot x$ are always zero.
Our efficient MPC design 
avoids unnecessary secure computation over unrelated nodes by focusing on computing non-zero results while concealing the sparse topology.
We achieve this~by:
1) decomposing the sparse matrix $\adjmat$ into a product of matrices (\S\ref{sec::sgc}), including permutation and binary diagonal matrices, that can \emph{faithfully} represent the original graph topology;
2) devising specialized protocols (\S\ref{sec::smm_protocol}) for efficiently multiplying the structured matrices while hiding sparsity topology.


 
\subsubsection{Sparse Matrix Decomposition}
We decompose adjacency matrix $\adjmat$ of $\graph$ into two bipartite graphs: one represented by sparse matrix $\adjout$, linking the out-degree nodes to edges, the other 
by sparse matrix $\adjin$,
linking edges to in-degree nodes.

%\ie, we decompose $\adjmat$ into $\adjout \adjin$, where $\adjout$ and $\adjin$ are sparse matrices representing these connections.
%linking out-degree nodes to edges and edges to in-degree nodes of $\graph$, respectively.

We then permute the columns of $\adjout$ and the rows of $\adjin$ so that the permuted matrices $\adjout'$ and $\adjin'$ have non-zero positions with \emph{monotonically non-decreasing} row and column indices.
A permutation $\sigma$ is used to preserve the edge topology, leading to an initial decomposition of $\adjmat = \adjout'\sigma \adjin'$.
This is further refined into a sequence of \emph{linear transformations}, 
which can be efficiently computed by our MPC protocols for 
\emph{oblivious permutation}
%($\Pi_{\ssp}$) 
and \emph{oblivious selection-multiplication}.
% ($\Pi_\SM$)
\iffalse
Our approach leverages bipartite graph representation and the monotonicity of non-zero positions to decompose a general sparse matrix into linear transformations, enhancing the efficiency of our MPC protocols.
\fi
Our decomposition approach is not limited to GCNs but also general~SMM 
by 
%simply 
treating them 
as adjacency matrices.
%of a graph.
%Since any sparse matrix can be viewed 

%allowing the same technique to be applied.

 
\subsubsection{New Protocols for Linear Transformations}
\emph{Oblivious permutation} (OP) is a two-party protocol taking a private permutation $\sigma$ and a private vector $\xvec$ from the two parties, respectively, and generating a secret share $\l\sigma \xvec\r$ between them.
Our OP protocol employs correlated randomnesses generated in an input-independent offline phase to mask $\sigma$ and $\xvec$ for secure computations on intermediate results, requiring only $1$ round in the online phase (\cf, $\ge 2$ in previous works~\cite{ccs/AsharovHIKNPTT22, ccs/Araki0OPRT21}).

Another crucial two-party protocol in our work is \emph{oblivious selection-multiplication} (OSM).
It takes a private bit~$s$ from a party and secret share $\l x\r$ of an arithmetic number~$x$ owned by the two parties as input and generates secret share $\l sx\r$.
%between them.
%Like our OP protocol, o
Our $1$-round OSM protocol also uses pre-computed randomnesses to mask $s$ and $x$.
%for secure computations.
Compared to the Beaver-triple-based~\cite{crypto/Beaver91a} and oblivious-transfer (OT)-based approaches~\cite{pkc/Tzeng02}, our protocol saves ${\sim}50\%$ of online communication while having the same offline communication and round complexities.

By decomposing the sparse matrix into linear transformations and applying our specialized protocols, our \osmm protocol
%($\prosmm$) 
reduces the complexity of evaluating $\numnode \times \numnode$ sparse matrices with $\numedge$ non-zero values from $O(\numnode^2)$ to $O(\numedge)$.

%(\S\ref{sec::secgcn})
\subsection{\cgnn: Secure GCN made Efficient}
Supported by our new sparsity techniques, we build \cgnn, 
a two-party computation (2PC) framework for GCN inference and training over vertical
%ly split
data.
Our contributions include:

1) We are the first to explore sparsity over vertically split, secret-shared data in MPC, enabling decompositions of sparse matrices with arbitrary sparsity and isolating computations that can be performed in plaintext without sacrificing privacy.

2) We propose two efficient $2$PC primitives for OP and OSM, both optimally single-round.
Combined with our sparse matrix decomposition approach, our \osmm protocol ($\prosmm$) achieves constant-round communication costs of $O(\numedge)$, reducing memory requirements and avoiding out-of-memory errors for large matrices.
In practice, it saves $99\%+$ communication
%(Table~\ref{table:comm_smm}) 
and reduces ${\sim}72\%$ memory usage over large $(5000\times5000)$ matrices compared with using Beaver triples.
%(Table~\ref{table:mem_smm_sparse}) ${\sim}16\%$-

3) We build an end-to-end secure GCN framework for inference and training over vertically split data, maintaining accuracy on par with plaintext computations.
We will open-source our evaluation code for research and deployment.

To evaluate the performance of $\cgnn$, we conducted extensive experiments over three standard graph datasets (Cora~\cite{aim/SenNBGGE08}, Citeseer~\cite{dl/GilesBL98}, and Pubmed~\cite{ijcnlp/DernoncourtL17}),
reporting communication, memory usage, accuracy, and running time under varying network conditions, along with an ablation study with or without \osmm.
Below, we highlight our key achievements.

\textit{Communication (\S\ref{sec::comm_compare_gcn}).}
$\cgnn$ saves communication by $50$-$80\%$.
(\cf,~CoGNN~\cite{ccs/KotiKPG24}, OblivGNN~\cite{uss/XuL0AYY24}).

\textit{Memory usage (\S\ref{sec::smmmemory}).}
\cgnn alleviates out-of-memory problems of using %the standard 
Beaver-triples~\cite{crypto/Beaver91a} for large datasets.

\textit{Accuracy (\S\ref{sec::acc_compare_gcn}).}
$\cgnn$ achieves inference and training accuracy comparable to plaintext counterparts.
%training accuracy $\{76\%$, $65.1\%$, $75.2\%\}$ comparable to $\{75.7\%$, $65.4\%$, $74.5\%\}$ in plaintext.

{\textit{Computational efficiency (\S\ref{sec::time_net}).}} 
%If the network is worse in bandwidth and better in latency, $\cgnn$ shows more benefits.
$\cgnn$ is faster by $6$-$45\%$ in inference and $28$-$95\%$ in training across various networks and excels in narrow-bandwidth and low-latency~ones.

{\textit{Impact of \osmm (\S\ref{sec:ablation}).}}
Our \osmm protocol shows a $10$-$42\times$ speed-up for $5000\times 5000$ matrices and saves $10$-2$1\%$ memory for ``small'' datasets and up to $90\%$+ for larger ones.



\section{Related Work}
Our work draws on and contributes to research in mobility aids and the built environment, online image-based survey for urban assessment, personalized routing applications and accessibility maps.

\subsection{Mobility Aids and the Built Environment}
People who use mobility aids (\textit{e.g.,} canes, walkers, mobility scooters, manual wheelchairs and motorized wheelchairs) face significant challenges navigating their communities.
Studies have repeatedly found that sidewalk conditions can significantly impede mobility among these users~\cite{bigonnesse_role_2018,fomiatti_experience_2014,f_bromley_city_2007,rosenberg_outdoor_2013, harris_physical_2015,korotchenko_power_2014}. 
In a review of the physical environment's role in mobility, \citet{bigonnesse_role_2018} summarized factors affecting mobility aid users, including uneven or narrow sidewalks (\textit{e.g.,}~\cite{fomiatti_experience_2014,f_bromley_city_2007}), rough pavements (\textit{e.g.,}~\cite{fomiatti_experience_2014,f_bromley_city_2007}), absent or poorly designed curb ramps (\textit{e.g.,}~\cite{rosenberg_outdoor_2013, f_bromley_city_2007, korotchenko_power_2014}), lack of crosswalks (\textit{e.g.,}~\cite{harris_physical_2015}), and various temporary obstacles (\textit{e.g.,}~\cite{harris_physical_2015}).

Though most research on mobility disability and the built environment has focused on wheelchair users~\cite{bigonnesse_role_2018}, mobility challenges are not experienced uniformly across different user populations~\cite{prescott_factors_2020, bigonnesse_role_2018}. 
For example, crutch users could overcome a specific physical barrier (such as two stairs down to a street), whereas motorized wheelchair users could not (without a ramp)~\cite{bigonnesse_role_2018}. 
Such variability demonstrates how person-environment interaction can differ based on mobility aids and environmental factors~\cite{sakakibara_rasch_2018,smith_review_2016}.
Further, mobility aids such as canes, crutches, or walkers are more commonly used than wheelchairs in the U.S.~\cite{taylor_americans_2014, firestine_travel_2024}: in 2022, approximately 4.7 million adults used a cane, crutches, or a walker, compared to 1.7 million who used a wheelchair~\cite{firestine_travel_2024}.
This underscores the importance of considering a diverse range of mobility aid users in urban accessibility research.
For example, \citet{prescott_factors_2020} explored the daily path areas of users of manual wheelchairs, motorized wheelchairs, scooters, walkers, canes, and crutches and found that the type of mobility device had a strong association with users' daily path area size.
Our study aims to further advance knowledge of how different mobility aid users perceive sidewalk barriers, with a more inclusive understanding of urban accessibility.

\begin{figure*}
    \centering
    \includegraphics[width=1\linewidth]{figures/figure-tutorial.png}
    \caption{Survey Part 2.1 showed all 52 images and asked participants to rate their passability based on their lived experience and use of their mobility aid. Above is the interactive tutorial we showed at the beginning of this part.}
    \Description{This figure shows a screenshot from the online survey. In survey part 2.1, participants were presented with 52 images and were asked to rate their passibility based on their lived experience and use of their mobility aid. The screenshot shows the interactive tutorial shown before this section.}
    \label{fig:survey-part2-instructions}
\end{figure*}

\subsection{Online Image-Based Survey for Urban Assessment}
Sidewalk barriers hinder individuals with mobility impairments not just by preventing particular travel paths but also by reducing confidence in self-navigating and decreasing one's willingness to travel to areas that might be physically challenging or unsafe~\cite{vasudevan_exploration_2016,clarke_mobility_2008}.
Prior work in this area traditionally uses three main study methods: in-person interviews (\textit{e.g}.~\cite{rosenberg_outdoor_2013,castrodale_mobilizing_2018}), GPS-based activity studies (\textit{e.g.,}~\cite{prescott_exploration_2021, prescott_factors_2020,rosenberg_outdoor_2013}), and online-questionnaires (\textit{e.g.,}~\cite{carlson_wheelchair_2002}). 
In-person interviews, while providing detailed and nuanced information, are limited by small sample sizes~\cite{rosenberg_outdoor_2013}. GPS-based activity studies involve tracking mobility aids user activity over a period of time, offering insights into movement patterns and activity space; however, these studies are constrained by geographical location~\cite{prescott_exploration_2021}. In contrast, online questionnaires can reach much larger populations and cover broader geographical regions, but they often yield high-level information that lacks the depth and nuance of the other approaches~\cite{carlson_wheelchair_2002}.
Our study aims to strike a balance between these approaches, capturing nuanced perspectives of mobility aid users about the built environment while maintaining a sufficiently large enough sample size for robust statistical analysis. 
Building on~\citet{bigonnesse_role_2018}'s work, we explore not only the types of factors considered to be barriers, but the \textit{intensity} of these barriers and their differential impacts.

Visual assessment of environmental features has long been employed by researchers across diverse fields, including human well-being~\cite{humpel_environmental_2002}, ecosystem sustainability~\cite{gobster_shared_2007}, and public policy~\cite{dobbie_public_2013}. 
These studies examine the relationship between images and the reactions they provoke in respondents or compare differences in reactions between groups.
Over the past decade, online visual preference surveys have gained popularity (\textit{e.g.,}~\cite{evans-cowley_streetseen_2014, salesses_collaborative_2013, goodspeed_research_2017}), where respondents are asked to make pairwise comparisons between randomly selected images.
Using this approach has two advantages: it adheres to the law of comparative judgment~\cite{thurstone_law_2017} by allowing respondents to make direct comparisons, and it prevents inter-rater inconsistency possible with scale ratings~\cite{goodspeed_research_2017}.
Additionally, online surveys generally offer advantages of increased sample sizes, reduced costs, and greater flexibility~\cite{wherrett_issues_1999}.
For people with disabilities, online surveys can be particularly beneficial. They help reach hidden or difficult-to-access populations~\cite{cook_challenges_2007,wright_researching_2005} and are believed to encourage more honest answers to sensitive questions~\cite{eckhardt_research_2007} by providing a higher level of anonymity and confidentiality~\cite{cook_challenges_2007, wright_researching_2005}.

\begin{figure*}
    \centering
    \includegraphics[width=1\linewidth]{figures/figure-comaprison-screenshot.png}
    \caption{In survey Part 2.2, participants were asked to perform a series of pairwise comparisons based on their 2.1 responses.}
    \Description{This figure shows a screenshot from the online survey. In Survey Part 2.2, participants were asked to perform a series of pairwise comparisons based on their 2.1 responses.}
    \label{fig:survey-part2b-pairwise}
\end{figure*}

\subsection{Personalized Routing Applications and Accessibility Maps}
Navigation challenges faced by mobility aid users can be mitigated through the provision of routes and directions that guide them to destinations safely, accurately, and efficiently~\cite{kasemsuppakorn_understanding_2015}. However, current commercial routing applications (\textit{e.g.}, \textit{Google Maps}) do not provide sufficient guidance for mobility aid users.
To address this gap, significant research has focused on routing systems for this population over the past two decades~\cite{barczyszyn_collaborative_2018, karimanzira_application_2006, matthews_modelling_2003, kasemsuppakorn_understanding_2015, volkel_routecheckr_2008, holone_people_2008, wheeler_personalized_2020, gharebaghi_user-specific_2021, ding_design_2007}.
One early, well-known prototype system is \textit{MAGUS}~\cite{matthews_modelling_2003}, which computes optimal routes for wheelchair users based on shortest distance, minimum barriers, fewest slopes, and limits on road crossings and challenging surfaces.
\textit{U-Access}~\cite{sobek_u-access_2006} provides the shortest route for people with three accessibility levels: unaided mobility, aided mobility (using crutch, cane, or walker), and wheelchair users.
However, U-Access only considers distance and ignores other
important factors for mobility aid users~\cite{barczyszyn_collaborative_2018}.
A series of projects by Kasemsuppakorn \textit{et al}.~\cite{kasemsuppakorn_personalised_2009, kasemsuppakorn_understanding_2015} attempted to create personalized routes for wheelchair users using fuzzy logic and \textit{Analytic Hierarchy Process} (AHP).

While influential, many personalized routing prototypes face limited adoption due to a scarcity of accessibility data for the built environment. 
Geo-crowdsourcing~\cite{karimi_personalized_2014}, a.k.a. volunteered geographic information (VGI)~\cite{goodchild_citizens_2007}, has emerged as an effective solution~\cite{karimi_personalized_2014, wheeler_personalized_2020}.
In this approach, users annotate maps with specific criteria or share personal experiences of locations, typically using web applications based on Google Maps or \textit{OpenStreetMap} (OSM)~\cite{karimi_personalized_2014}.
Examples include \textit{Wheelmap}~\cite{mobasheri_wheelmap_2017}, \textit{CAP4Access}~\cite{cap4access_cap4access_2014}, \textit{AXS Map}~\cite{axs_map_axs_2012}, and \textit{Project Sidewalk}~\cite{saha_project_2019}.
Recent research demonstrated the potential of using crowdsourced geodata for personalized routing~\cite{goldberg_interactive_2016, bolten_accessmap_2019,menkens_easywheel_2011, neis_measuring_2015}.
For example, \textit{EasyWheel}~\cite{menkens_easywheel_2011}, a mobile social navigation system based on OSM, provides wheelchair users with optimized routing, accessibility information for points of interest, and a social community for reporting barriers. 
\textit{AccessMap}~\cite{bolten_accessmap_2019} offers routing information tailored to users of canes, manual wheelchairs, or powered wheelchairs, calculating routes based on OSM data that includes slope, curbs, stairs and landmarks. 
Our work builds on the above by gathering perceptions of sidewalk obstacles from different mobility aid users to create generalizable profiles based on mobility aid type. We envision that these profiles can provide starting points in tools like Google Maps for personalized routing but can be further customized by the end user to specify additional needs (\textit{e.g.}, ability to navigate hills, \textit{etc.})

Beyond routing applications, our study data can contribute to modeling and visualizing higher-level abstractions of accessibility. 
Similar to \textit{AccessScore}~\cite{li_interactively_2018}, data from our survey can provide personalizable and interactive visual analytics of city-wide accessibility. By identifying both differences between mobility groups and common barriers within groups, we can develop analytical tools to prioritize barriers and assess the impact of their mitigation or removal, potentially benefiting the broadest range of mobility group users. Incorporating perceptions of passibility into urban planning processes provides a new dimension for urban planners' toolkits, which are often narrowly focused on compliance with ADA standards.




\vspace{-5pt}
\section{Method}
\label{sec:method}
\section{Overview}

\revision{In this section, we first explain the foundational concept of Hausdorff distance-based penetration depth algorithms, which are essential for understanding our method (Sec.~\ref{sec:preliminary}).
We then provide a brief overview of our proposed RT-based penetration depth algorithm (Sec.~\ref{subsec:algo_overview}).}



\section{Preliminaries }
\label{sec:Preliminaries}

% Before we introduce our method, we first overview the important basics of 3D dynamic human modeling with Gaussian splatting. Then, we discuss the diffusion-based 3d generation techniques, and how they can be applied to human modeling.
% \ZY{I stopp here. TBC.}
% \subsection{Dynamic human modeling with Gaussian splatting}
\subsection{3D Gaussian Splatting}
3D Gaussian splatting~\cite{kerbl3Dgaussians} is an explicit scene representation that allows high-quality real-time rendering. The given scene is represented by a set of static 3D Gaussians, which are parameterized as follows: Gaussian center $x\in {\mathbb{R}^3}$, color $c\in {\mathbb{R}^3}$, opacity $\alpha\in {\mathbb{R}}$, spatial rotation in the form of quaternion $q\in {\mathbb{R}^4}$, and scaling factor $s\in {\mathbb{R}^3}$. Given these properties, the rendering process is represented as:
\begin{equation}
  I = Splatting(x, c, s, \alpha, q, r),
  \label{eq:splattingGA}
\end{equation}
where $I$ is the rendered image, $r$ is a set of query rays crossing the scene, and $Splatting(\cdot)$ is a differentiable rendering process. We refer readers to Kerbl et al.'s paper~\cite{kerbl3Dgaussians} for the details of Gaussian splatting. 



% \ZY{I would suggest move this part to the method part.}
% GaissianAvatar is a dynamic human generation model based on Gaussian splitting. Given a sequence of RGB images, this method utilizes fitted SMPLs and sampled points on its surface to obtain a pose-dependent feature map by a pose encoder. The pose-dependent features and a geometry feature are fed in a Gaussian decoder, which is employed to establish a functional mapping from the underlying geometry of the human form to diverse attributes of 3D Gaussians on the canonical surfaces. The parameter prediction process is articulated as follows:
% \begin{equation}
%   (\Delta x,c,s)=G_{\theta}(S+P),
%   \label{eq:gaussiandecoder}
% \end{equation}
%  where $G_{\theta}$ represents the Gaussian decoder, and $(S+P)$ is the multiplication of geometry feature S and pose feature P. Instead of optimizing all attributes of Gaussian, this decoder predicts 3D positional offset $\Delta{x} \in {\mathbb{R}^3}$, color $c\in\mathbb{R}^3$, and 3D scaling factor $ s\in\mathbb{R}^3$. To enhance geometry reconstruction accuracy, the opacity $\alpha$ and 3D rotation $q$ are set to fixed values of $1$ and $(1,0,0,0)$ respectively.
 
%  To render the canonical avatar in observation space, we seamlessly combine the Linear Blend Skinning function with the Gaussian Splatting~\cite{kerbl3Dgaussians} rendering process: 
% \begin{equation}
%   I_{\theta}=Splatting(x_o,Q,d),
%   \label{eq:splatting}
% \end{equation}
% \begin{equation}
%   x_o = T_{lbs}(x_c,p,w),
%   \label{eq:LBS}
% \end{equation}
% where $I_{\theta}$ represents the final rendered image, and the canonical Gaussian position $x_c$ is the sum of the initial position $x$ and the predicted offset $\Delta x$. The LBS function $T_{lbs}$ applies the SMPL skeleton pose $p$ and blending weights $w$ to deform $x_c$ into observation space as $x_o$. $Q$ denotes the remaining attributes of the Gaussians. With the rendering process, they can now reposition these canonical 3D Gaussians into the observation space.



\subsection{Score Distillation Sampling}
Score Distillation Sampling (SDS)~\cite{poole2022dreamfusion} builds a bridge between diffusion models and 3D representations. In SDS, the noised input is denoised in one time-step, and the difference between added noise and predicted noise is considered SDS loss, expressed as:

% \begin{equation}
%   \mathcal{L}_{SDS}(I_{\Phi}) \triangleq E_{t,\epsilon}[w(t)(\epsilon_{\phi}(z_t,y,t)-\epsilon)\frac{\partial I_{\Phi}}{\partial\Phi}],
%   \label{eq:SDSObserv}
% \end{equation}
\begin{equation}
    \mathcal{L}_{\text{SDS}}(I_{\Phi}) \triangleq \mathbb{E}_{t,\epsilon} \left[ w(t) \left( \epsilon_{\phi}(z_t, y, t) - \epsilon \right) \frac{\partial I_{\Phi}}{\partial \Phi} \right],
  \label{eq:SDSObservGA}
\end{equation}
where the input $I_{\Phi}$ represents a rendered image from a 3D representation, such as 3D Gaussians, with optimizable parameters $\Phi$. $\epsilon_{\phi}$ corresponds to the predicted noise of diffusion networks, which is produced by incorporating the noise image $z_t$ as input and conditioning it with a text or image $y$ at timestep $t$. The noise image $z_t$ is derived by introducing noise $\epsilon$ into $I_{\Phi}$ at timestep $t$. The loss is weighted by the diffusion scheduler $w(t)$. 
% \vspace{-3mm}

\subsection{Overview of the RTPD Algorithm}\label{subsec:algo_overview}
Fig.~\ref{fig:Overview} presents an overview of our RTPD algorithm.
It is grounded in the Hausdorff distance-based penetration depth calculation method (Sec.~\ref{sec:preliminary}).
%, similar to that of Tang et al.~\shortcite{SIG09HIST}.
The process consists of two primary phases: penetration surface extraction and Hausdorff distance calculation.
We leverage the RTX platform's capabilities to accelerate both of these steps.

\begin{figure*}[t]
    \centering
    \includegraphics[width=0.8\textwidth]{Image/overview.pdf}
    \caption{The overview of RT-based penetration depth calculation algorithm overview}
    \label{fig:Overview}
\end{figure*}

The penetration surface extraction phase focuses on identifying the overlapped region between two objects.
\revision{The penetration surface is defined as a set of polygons from one object, where at least one of its vertices lies within the other object. 
Note that in our work, we focus on triangles rather than general polygons, as they are processed most efficiently on the RTX platform.}
To facilitate this extraction, we introduce a ray-tracing-based \revision{Point-in-Polyhedron} test (RT-PIP), significantly accelerated through the use of RT cores (Sec.~\ref{sec:RT-PIP}).
This test capitalizes on the ray-surface intersection capabilities of the RTX platform.
%
Initially, a Geometry Acceleration Structure (GAS) is generated for each object, as required by the RTX platform.
The RT-PIP module takes the GAS of one object (e.g., $GAS_{A}$) and the point set of the other object (e.g., $P_{B}$).
It outputs a set of points (e.g., $P_{\partial B}$) representing the penetration region, indicating their location inside the opposing object.
Subsequently, a penetration surface (e.g., $\partial B$) is constructed using this point set (e.g., $P_{\partial B}$) (Sec.~\ref{subsec:surfaceGen}).
%
The generated penetration surfaces (e.g., $\partial A$ and $\partial B$) are then forwarded to the next step. 

The Hausdorff distance calculation phase utilizes the ray-surface intersection test of the RTX platform (Sec.~\ref{sec:RT-Hausdorff}) to compute the Hausdorff distance between two objects.
We introduce a novel Ray-Tracing-based Hausdorff DISTance algorithm, RT-HDIST.
It begins by generating GAS for the two penetration surfaces, $P_{\partial A}$ and $P_{\partial B}$, derived from the preceding step.
RT-HDIST processes the GAS of a penetration surface (e.g., $GAS_{\partial A}$) alongside the point set of the other penetration surface (e.g., $P_{\partial B}$) to compute the penetration depth between them.
The algorithm operates bidirectionally, considering both directions ($\partial A \to \partial B$ and $\partial B \to \partial A$).
The final penetration depth between the two objects, A and B, is determined by selecting the larger value from these two directional computations.

%In the Hausdorff distance calculation step, we compute the Hausdorff distance between given two objects using a ray-surface-intersection test. (Sec.~\ref{sec:RT-Hausdorff}) Initially, we construct the GAS for both $\partial A$ and $\partial B$ to utilize the RT-core effectively. The RT-based Hausdorff distance algorithms then determine the Hausdorff distance by processing the GAS of one object (e.g. $GAS_{\partial A}$) and set of the vertices of the other (e.g. $P_{\partial B}$). Following the Hausdorff distance definition (Eq.~\ref{equation:hausdorff_definition}), we compute the Hausdorff distance to both directions ($\partial A \to \partial B$) and ($\partial B \to \partial A$). As a result, the bigger one is the final Hausdorff distance, and also it is the penetration depth between input object $A$ and $B$.


%the proposed RT-based penetration depth calculation pipeline.
%Our proposed methods adopt Tang's Hausdorff-based penetration depth methods~\cite{SIG09HIST}. The pipeline is divided into the penetration surface extraction step and the Hausdorff distance calculation between the penetration surface steps. However, since Tang's approach is not suitable for the RT platform in detail, we modified and applied it with appropriate methods.

%The penetration surface extraction step is extracting overlapped surfaces on other objects. To utilize the RT core, we use the ray-intersection-based PIP(Point-In-Polygon) algorithms instead of collision detection between two objects which Tang et al.~\cite{SIG09HIST} used. (Sec.~\ref{sec:RT-PIP})
%RT core-based PIP test uses a ray-surface intersection test. For purpose this, we generate the GAS(Geometry Acceleration Structure) for each object. RT core-based PIP test takes the GAS of one object (e.g. $GAS_{A}$) and a set of vertex of another one (e.g. $P_{B}$). Then this computes the penetrated vertex set of another one (e.g. $P_{\partial B}$). To calculate the Hausdorff distance, these vertex sets change to objects constructed by penetrated surface (e.g. $\partial B$). Finally, the two generated overlapped surface objects $\partial A$ and $\partial B$ are used in the Hausdorff distance calculation step.

Our goal is to increase the robustness of T2I models, particularly with rare or unseen concepts, which they struggle to generate. To do so, we investigate a retrieval-augmented generation approach, through which we dynamically select images that can provide the model with missing visual cues. Importantly, we focus on models that were not trained for RAG, and show that existing image conditioning tools can be leveraged to support RAG post-hoc.
As depicted in \cref{fig:overview}, given a text prompt and a T2I generative model, we start by generating an image with the given prompt. Then, we query a VLM with the image, and ask it to decide if the image matches the prompt. If it does not, we aim to retrieve images representing the concepts that are missing from the image, and provide them as additional context to the model to guide it toward better alignment with the prompt.
In the following sections, we describe our method by answering key questions:
(1) How do we know which images to retrieve? 
(2) How can we retrieve the required images? 
and (3) How can we use the retrieved images for unknown concept generation?
By answering these questions, we achieve our goal of generating new concepts that the model struggles to generate on its own.

\vspace{-3pt}
\subsection{Which images to retrieve?}
The amount of images we can pass to a model is limited, hence we need to decide which images to pass as references to guide the generation of a base model. As T2I models are already capable of generating many concepts successfully, an efficient strategy would be passing only concepts they struggle to generate as references, and not all the concepts in a prompt.
To find the challenging concepts,
we utilize a VLM and apply a step-by-step method, as depicted in the bottom part of \cref{fig:overview}. First, we generate an initial image with a T2I model. Then, we provide the VLM with the initial prompt and image, and ask it if they match. If not, we ask the VLM to identify missing concepts and
focus on content and style, since these are easy to convey through visual cues.
As demonstrated in \cref{tab:ablations}, empirical experiments show that image retrieval from detailed image captions yields better results than retrieval from brief, generic concept descriptions.
Therefore, after identifying the missing concepts, we ask the VLM to suggest detailed image captions for images that describe each of the concepts. 

\vspace{-4pt}
\subsubsection{Error Handling}
\label{subsec:err_hand}

The VLM may sometimes fail to identify the missing concepts in an image, and will respond that it is ``unable to respond''. In these rare cases, we allow up to 3 query repetitions, while increasing the query temperature in each repetition. Increasing the temperature allows for more diverse responses by encouraging the model to sample less probable words.
In most cases, using our suggested step-by-step method yields better results than retrieving images directly from the given prompt (see 
\cref{subsec:ablations}).
However, if the VLM still fails to identify the missing concepts after multiple attempts, we fall back to retrieving images directly from the prompt, as it usually means the VLM does not know what is the meaning of the prompt.

The used prompts can be found in \cref{app:prompts}.
Next, we turn to retrieve images based on the acquired image captions.

\vspace{-3pt}
\subsection{How to retrieve the required images?}

Given $n$ image captions, our goal is to retrieve the images that are most similar to these captions from a dataset. 
To retrieve images matching a given image caption, we compare the caption to all the images in the dataset using a text-image similarity metric and retrieve the top $k$ most similar images.
Text-to-image retrieval is an active research field~\cite{radford2021learning, zhai2023sigmoid, ray2024cola, vendrowinquire}, where no single method is perfect.
Retrieval is especially hard when the dataset does not contain an exact match to the query \cite{biswas2024efficient} or when the task is fine-grained retrieval, that depends on subtle details~\cite{wei2022fine}.
Hence, a common retrieval workflow is to first retrieve image candidates using pre-computed embeddings, and then re-rank the retrieved candidates using a different, often more expensive but accurate, method \cite{vendrowinquire}.
Following this workflow, we experimented with cosine similarity over different embeddings, and with multiple re-ranking methods of reference candidates.
Although re-ranking sometimes yields better results compared to simply using cosine similarity between CLIP~\cite{radford2021learning} embeddings, the difference was not significant in most of our experiments. Therefore, for simplicity, we use cosine similarity between CLIP embeddings as our similarity metric (see \cref{tab:sim_metrics}, \cref{subsec:ablations} for more details about our experiments with different similarity metrics).

\vspace{-3pt}
\subsection{How to use the retrieved images?}
Putting it all together, after retrieving relevant images, all that is left to do is to use them as context so they are beneficial for the model.
We experimented with two types of models; models that are trained to receive images as input in addition to text and have ICL capabilities (e.g., OmniGen~\cite{xiao2024omnigen}), and T2I models augmented with an image encoder in post-training (e.g., SDXL~\cite{podellsdxl} with IP-adapter~\cite{ye2023ip}).
As the first model type has ICL capabilities, we can supply the retrieved images as examples that it can learn from, by adjusting the original prompt.
Although the second model type lacks true ICL capabilities, it offers image-based control functionalities, which we can leverage for applying RAG over it with our method.
Hence, for both model types, we augment the input prompt to contain a reference of the retrieved images as examples.
Formally, given a prompt $p$, $n$ concepts, and $k$ compatible images for each concept, we use the following template to create a new prompt:
``According to these examples of 
$\mathord{<}c_1\mathord{>:<}img_{1,1}\mathord{>}, ... , \mathord{<}img_{1,k}\mathord{>}, ... , \mathord{<}c_n\mathord{>:<}img_{n,1}\mathord{>}, ... , $
$\mathord{<}img_{n,k}\mathord{>}$,
generate $\mathord{<}p\mathord{>}$'', 
where $c_i$ for $i\in{[1,n]}$ is a compatible image caption of the image $\mathord{<}img_{i,j}\mathord{>},  j\in{[1,k]}$. 

This prompt allows models to learn missing concepts from the images, guiding them to generate the required result. 

\textbf{Personalized Generation}: 
For models that support multiple input images, we can apply our method for personalized generation as well, to generate rare concept combinations with personal concepts. In this case, we use one image for personal content, and 1+ other reference images for missing concepts. For example, given an image of a specific cat, we can generate diverse images of it, ranging from a mug featuring the cat to a lego of it or atypical situations like the cat writing code or teaching a classroom of dogs (\cref{fig:personalization}).
\vspace{-2pt}
\begin{figure}[htp]
  \centering
   \includegraphics[width=\linewidth]{Assets/personalization.pdf}
   \caption{\textbf{Personalized generation example.}
   \emph{ImageRAG} can work in parallel with personalization methods and enhance their capabilities. For example, although OmniGen can generate images of a subject based on an image, it struggles to generate some concepts. Using references retrieved by our method, it can generate the required result.
}
   \label{fig:personalization}\vspace{-10pt}
\end{figure}
\section{Findings}

The inductive analysis across different robotic artists revealed recurrent factors that contribute to artistic creativity in robotic artwork. Here we present four such facets---\textit{Embodiment and Materiality}, \textit{Malfunction}, \textit{Audience's Reaction and Reception}, and \textit{Process of Creation and Exhibition}. Robotic art is unique in each of them. We argue that these factors are salient in the real-world practices of robotic art---uses of robots in artistic or creation activities. By investigating the practice of robotic art, our study contributes empirically to understanding how computing machines are creatively used for artistic and non-pragmatic purposes. Building upon prior works on artistic input to HCI ~\cite{kang2022electronicists}, we advance the discourse by exploring how artistic practices, values, attitudes, and ways of thinking can serve as resources for HCI practitioners studying or designing for creative activities with machines.

\subsection{Embodiment and Materiality}
\label{f:emb}
Embodiment and materiality are key factors in artistic creativity, shaping the design of robotic artworks. As embodied forms, robots interact with physical space, materials, and humans, matching with human cognition through bodily perception~\cite{davis2012embodied}. Their embodiment encompasses physical appearance, movement, and human interaction, aspects crucial for HCI researchers designing robots to engage with their environment~\cite{marshall2013introduction}. For most of our artists (N=7), understanding robots' material and embodied nature deeply influences their creative process, shaping their thinking and inspiring new ideas. While embodiment imposes physical limitations, it also enhances artistic expression, fostering new styles and aesthetics.

\paragraph{Expressivity From Embodiment}
The embodied property of robots produces an important expressivity and artistic style in robotic art that is challenging to replicate without physical embodiment. For example, David compared drawing by physical robots with drawing in computer programs, concluding that the former is more expressive in an artistic sense because the action of drawing by robots is embodied in the physical world rather than being ``simulated'' in computer programs: ``\textit{I use embodiment (embodied action of drawing by robots)... the drawings work because they do real gestures, it (the drawing) is not simulated. So the drawing has this dynamic feel to it because it is really the movements and the gestures and things... there is a certain speed that it (the embodiment) gives this expressivity to the drawing}.'' The embodied drawing by robots adheres to the physical properties of the material and environmental factors (e.g., pencil, paper, table, robotic arm's degree of freedom, humidity, lighting of the scene), making the drawing process complex, and at times, random and uncontrollable. This complexity introduces more possibilities for artistic expression.

The degree of artistic expressivity depends on which specific materials enable the embodiment of drawing by robots. Interestingly, David claimed that industrial robots, though can draw with high precision, produce less expressive drawings than his self-built robots whose robotic arm's movement is not that precise but more dynamic, flexible, and turbulent:

\begin{quote}
    I don't use industrial robots, because industrial robots are pen plotters. They do exactly what you ask. But they (non-industrial robots) are flexible and... not that precise... when it's drawn, it (the drawing by non-industrial robots) has more expressivity because of the embodiment. The embodiment is very important. It's only because I use those types of arms (self-built robotic arms). It would be far less important if I was using industrial robots.
\end{quote}

He also mentioned explicitly that precise drawing is not artistic: ``\textit{But anyway, that (precise drawing robot) is the technology. And it works very nicely, but I couldn't find it artistic. I was actually disappointed when I got it to work.}'' Similarly, Sophie noted that plotting/printing robots create different drawings than painting robots do: ``\textit{I wanted it (the artwork) to be painted and I didn't want it for the visuality of it or the behavior of it. I didn't want it to be plotted or printed, [it] feels different [and] has a different existence.}''

Although both industrial robots and self-built robots draw in embodied ways, the results can appear either precise or dynamic, depending on how the robots are built and programmed---in other words, how the artists configure the material aspects of robots to realize the embodiment. In practice, our robotic artists need to think about ways of utilizing embodiment and properties of robots and all other involved materials to be artistically expressive, to be creative.

\paragraph{Inspiration From Embodiment}
We found that the embodied nature of robotic art often becomes a source of inspiration for new artistic ideas. Linda, an artist-engineer who conducts scholarly research at the intersection of robotics and dance, reflected on how interacting with embodied robots makes her think about the differences between human and robotic bodies:

\begin{quote}
    I've never felt more human. You just feel, you notice, oh, I can fall here and I can get right back up, but it (the robot) falls and it can't get right back up. Or how soft am I? How wet? Like, (patted her face) there's so much water content and squish when I lay on the floor. And it [the robot] doesn't have that... That generates new ideas and helps me be creative.
\end{quote}

She also explained how robotic bodies allow her to examine human movements: ``\textit{The robot is doing something that I can't do on my own body---pure right (her arm was moving toward her right), and [then]... [I] can look at my messy right [movement] next to its [robot's] pure right... that's creative, that's energizing to me to see and play with movement profiles with such a pure tool for decomposing the elements of it, making me notice them}.'' She also shared an anecdote that building a special robot with high degrees of freedom inspired her to explore the differences between human and robotic bodies, enabling her to see new things for her art projects.

\paragraph{Creativity From Embodiment}
Our artists emphasized the embodied nature of creativity and intelligence in general based on their artistic practices, asserting that creativity is inherently embodied rather than disembodied, symbolic, spiritual, and something only happens in the human head. For our robotic artists, creativity is built upon understanding embodied entities in the environment rather than abstract concepts in the mind. Samuel used ChatGPT as an example to argue how the disembodied way of communication between humans and machines limits creative interactions:

\begin{quote}
  I think most of the creativity is coming from non-verbal information flow. So when we are discussing with ChatGPT only through text... the creativity that we can experience is so limited because we do have to sit in front of ChatGPT and we cannot move around or ChatGPT is not going to move around. So our conversation is... very limited... that missing embodiment... is also missing creativity in the conversation with ChatGPT.
\end{quote}

The design of ChatGPT aligns with the mainstream approach to disembodied chatbots running as computer programs. In these designs, symbolic content (e.g., text, images, videos, audio) serves as the communicative medium, but bodily interaction is minimized (i.e., users primarily sit and type). While creativity is arguably rooted in embodied interaction with other material bodies, current interactive agents (e.g., Copilot and Midjourney) designed to support creative work remain largely symbolic and disembodied. Limiting human-machine communication to symbolic channels may lose the benefits of embodiment in acquiring creativity.

According to our artists, one reason for the lack of attention to the embodied dimension of creativity is the historical dichotomy between mind and body, which categorizes creativity as something in the mind:

\begin{quote}
     It (the idea that creativity is disembodied) was very much driven by a view that you can split the body and the mind, and intelligence is happening in a symbolic way, mainly in the brain... [This] led to a large focus on software applications and delayed focus on robotic hardware improvements. And still today, you can see the split of hardware and software... [F]or a lot of organic entities, the integration of bodily capacities with their environment could be seen as more intelligent than the representational capacities... [A]s an artist, I am trained to work with bodies interacting with environments or with other bodies, also this fluent transition from bodily action to semantic questioning. (Daniel)
\end{quote}

This dichotomy, which may have formulated the engineering of computing systems, is rarely compatible with the artist's view that intelligence and creativity can be more richly manifested through bodily interaction and relationships.

Embodiment has been an essential prerequisite of creativity for some artists since their creative production requires understanding embodied entities. To summarize this subsection, embodiment is an important source of creativity for robotic artists. Practically, it yields new artistic expressions and aesthetics whose complexity is difficult to replicate by computer programs. The embodied form of robots, in turn,  inspires creative ideas for artworks. These ideas can arise from understanding the entities embodied in the physical world, whether robots, humans, or other bodies in the environment. The symbolic and disembodied modes of interaction between human creators and machines in creative activities can be complemented and strengthened by embodied interaction.

\subsection{Malfunction: ``Ghost in the Machine''}
Robots, encompassing both mechanical and digital devices, are inherently susceptible to malfunction, with physical robots being more prone to errors, glitches, and noise than virtual agents. These malfunctions are widespread in robotics. In robotic art, such errors hold unique significance, influencing the interpretation and value of the art. Unlike engineers, who aim to fix errors, robotic artists often embrace malfunctions as part of their creative process (N=7).

\paragraph{Embracing Errors and Uncertainty}
Evelyn views machine errors not as obstacles, but as opportunities for unique artistic expression. She embraces the imperfections that arise from machine errors, seeing them as a way to humanize the machine and its output:

\begin{quote}
     I embrace these errors. For me, it is the way to show that using the machine in a way that's very counter-intuitive... celebrating that error instead of trying to perfect it, or slowing down the machine instead of trying to create commodities as fast as we can... what's interesting with the machine [is] to actually turn it upside down and think that the machine is a bit like a human child, and everything it does actually slow, it's imperfect, it's full of mistakes.
\end{quote}

Evelyn’s approach challenges the conventional expectation of machines as flawless and efficient executors. By slowing down the machine and celebrating its errors, she imbues the machine with a human-like quality of imperfection. This perspective turns the machine into something capable of growth and learning, much like a human child. The errors, therefore, potentially become a source of uniqueness and individuality in the artwork, adding depth and complexity to the artistic expression. This ``counterintuitive'' way of viewing error resonates with Alex who contrasts this view with the engineering tendency that strives to be neat, rational, and organized through monitoring and fixing errors: ``\textit{[S]ometimes, it's (error/glitch is) like a source of treasure. Like you find something that you could work on, you find something that people don't really use... But when we are tinkering, we sometimes reach this point of, `ah, okay, now this is visible.'... I think sometimes even just those things (errors/glitches) could be a work of [art]}.''

Alex and others see robotic malfunctions as opportunities to imagine alternative approaches and values. When robotic systems' behaviors deviate from their programs, they often refuse to ``fix'' the unexpected behaviors, instead, they allow the unexpected to unfold as serendipitous events that can inspire new design features. Preserving malfunctions allows the artists to think about the artistic potential of something derailing from the initial plan and make informed adjustments accordingly. These values would not be examined, integrated, or utilized to contribute to creativity if the immediate response to malfunctions was negation and subjecting the malfunctions as inferior to the planned behaviors. As Sophie noted, artistic practices are inherently unpredictable and shaped by the contingencies of the creation process.

Many of our artists described how they perceive, evaluate, and appreciate the unexpectedness of robotic art, revealing new artistic ideas that would not have emerged otherwise: \textit{``[I]nstead of an ink particle, you had a hole in the form of that part... I was like, `Oh, we'll see that the material is saturated, I will not push it (brush).' But the robot doesn't have this understanding and pushes it. And I thought, `Oh, it's actually a good outcome. It's actually both conceptually and aesthetically very pleasing to me'}'' (Sophie).

In this case, the robot performed an action that a human artist normally would not perform---pushing the brush on the canvas. The robot made an unusual decision and breakthrough in expression, called by many artists as ``surprise.'' Once the artist recognizes its artistic value, it may be further explored and developed. Linda shared a similar anecdote where an unexpected jitter from the way the motor pulls a string gives a ``texture'' to the robotic movement, which she sees as creative.

\paragraph{Incorporating Malfunctions as Intended Design}
Our artists deliberately incorporate errors into their artworks. It demonstrates how valuing malfunctions and the unexpected can directly contribute to the work's artistic creativity. Linda articulated the idea that humans are capable not only of learning from mistakes but also of intentionally leveraging these errors to their advantage, echoing insights from our other artists: \textit{``Glitches are 100\% part of the creative and artistic process... It's undeniable that we recover better from mistakes [than computers do], but I think it's more than that. We actually can incorporate mistakes and make them part of an intended design.''}

By making malfunctions part of the intended design, the artists engage with and utilize them to enhance artistic expression or similar ends. Choosing not to fix these issues offers the artists alternatives to designing and realizing their robotic art. For example, David recounted an anecdote about a bug---a flaw in a computer program's software or hardware---that unexpectedly made a line drawing ``beautiful.'' Rather than fixing the bug, he decided to make it an optional feature, allowing him to switch it on or off:

\begin{quote}
    Generally I don't take care of them (glitches). So there are those glitches that give this unpredictable because I like to have drawings that are not predictable... I fix it (bug) and then I have the possibility of using or not using the bug... I'm always surprised by the output... it (bug) creates a surprise for the spectator who is looking at the robot drawing... I just left it (bug) and it's still there. Sometimes I switch it (bug) on, sometimes I switch it off.
\end{quote}

If David had fixed the bug, without retaining it in the program, he would not have possessed such a feature of expression. This shows how differently robotic artists handle technical malfunctions than typical engineers or roboticists. Malfunctions should be avoided in engineering but may yield creative outcomes for robotic art. This is not to claim that the creative value is innate within malfunctions. As our findings have shown, malfunctions are raw materials that can be deliberately utilized by the artists to achieve creativity. When malfunctions are not desirable in art, they may primarily be engineering challenges, as the following examples illustrate.

\paragraph{Avoiding Malfunctions}
The fragility of robots is a widely shared concern among our robotic artists (N=8). Regardless of their origin---self-built, modified, or off-the-shelf---all robots are susceptible to breakdowns in real-world environments, particularly during extended exhibitions without proper maintenance by artists or qualified personnel. For exhibitions, malfunctions are generally unacceptable, and robots ought to \textit{perform flawlessly} when showcasing to the audience. To address malfunctions, the artists came up with different strategies, such as having backup materials for replacement and assembling the robots on-site at the exhibition. One strategy is reducing the complexity of the robotic system, simplifying it to minimize the risk of failure or loss of control. Their approach involves designing robots that resist internal breakdowns and withstand external environmental factors, such as moisture and gravity. Linda explained, ``\textit{If I do build them (robots), I try to keep them simple and I try to make something that will withstand its environment... Sometimes that might be outdoors next to the ocean for six days}.'' David further emphasized that the concern for fragility leads to the need for simplicity in robotic design:

\begin{quote}
    [I]f you're used to do programs that are disembodied, that are only on the computer, you can do very complex things. But as soon as you work with robots, you have to simplify everything... They exist in the same physical world [as us]. Dynamic, the speed, the time, the weight of thing are the same for us. So there are all those limits, which [requires] you to simplify a lot of the programming.
\end{quote}

Mitigating malfunctions and recognizing their artistic potential are not mutually exclusive. Designs that address engineering malfunctions can also yield artistic qualities. As illustrated in the findings, utilizing and mitigating malfunctions occur at different phases of artistic practice. In the production phase within the studios, artists often regard malfunctions not as impediments but as sources of inspiration. By celebrating serendipitous errors and the unexpected, they deliberately integrate these elements into their robotic creations, pushing material and expressive boundaries. In this phase, the primary interaction happens at the individual level---between the artist and the robot(s). In contrast, within exhibition spaces like museums, malfunctions conflict with the expectation that the robots should function flawlessly, risking being disqualified from display. Here, the interaction shifts to a social context, where artists must negotiate with curators and audiences on how to present the robot. This transition from studio to exhibition thus signifies an important change in, context, practice, and actors involved. Hence, next, we highlight the significance of audience reaction and reception that shape creative outcomes.

\subsection{Audience's Reaction and Reception}

\begin{quote}
    I suppose [that] every project I do is a collaboration between me, the machine, and the interactant to some extent. --- Robert
\end{quote}

The artistic and creative value in robotic artwork is determined not just by the work itself but often by the audience’s reactions and interpretations. Our artists (N=7) mentioned that they observe or think about audience reaction, and often incorporate them into subsequent iterations of their work. Alex, for instance, is motivated in the first place by observing how people react to robots, drawing inspiration from their perceptions.

\paragraph{Audience Reaction Shapes Robotic Design}
One of the most direct ways audiences influence the practice of robotic art is through the artists, even when it is unintentional. For instance, after observing that some audience members interact with his robots by squeezing two springs on the robot together---causing a short circuit---Robert decided to revise the material design of the robots to prevent such accidents: ``\textit{I knew darn well that the children were going to squeeze the springs together. So I was very excited to find that even if they did that, I put a kind of a self-healing fuse, polycrystalline that will heal itself... it was an important component of the design}.''

Robert’s response highlights the importance of audience reaction, which he observes and integrates into his robot designs. While in this case, the reaction led to the resolution of a technical issue rather than adding an artistic element,  Alex's experience illustrates how audience interaction can inspire new aesthetics in his work. Alex described how he adapted the environment around his robots based on the audience’s tendency to project personalities onto them:

\begin{quote}
    People project something like animals or themselves or something [on the robots]. And then I got inspiration from that. Then I made a little brighter setup with some objects, a little bit like forest kind of setup. And then people try to imagine more stories. And then I also put some effect to [make the setting] looks like night or daylight or morning. Then people really see [the robots] differently.
\end{quote}

These examples demonstrate how the audience's explicit and implicit feedback (action, projection, and imagination) influences artists’ decisions in designing robots. Audiences are not passive recipients of the artists’ creations; rather, they become part of a collective creative process, leaving their mark on the final work.

\paragraph{Audience Reaction Shapes Robotic Performance}
Linda described how she designed a robotic component for ``\textit{onstage performer[s] as well as audience members to come and interact with [the] robot in a creative way},'' emphasizing the importance of creating a space for audience interaction. Robert further suggested that these interactions during the exhibition possess performative features, which he views as an artwork: ``\textit{I would consider the final product (the drawing by his autonomous robots) as the art. And I would also consider the [audience's] experience of watching them (the robots) paint also as a kind of performative artwork}.'' Robert views robots not as static objects but as responsive entities capable of meaningful interactions with both their environment and the audience. He views robots as possessing ``emergent agency'':

\begin{quote}
    I think that's an agency I would call emergent agency, which is to say that the system software, the physical structure itself in relation to the viewer, interactant creates a kind of emergent behavior where the robot is, and it's designed to some extent to react or respond either with sound or motion in some way to the viewer. And by doing so, it then allows the viewer to see that response, which then reprograms the viewer's response to that. So there's almost a kind of feedback loop that I find happens a lot with robotic art.
\end{quote}

Daniel mentioned a similar idea in the context of live dance performance. The performance benefits from incorporating ``real-time learning interactive systems'' because that makes the performance not solely predefined but ``\textit{[emerged] in the moment of interaction which was not there before [the performance].}'' Without the audience serving as the stimulus, interactive robots in exhibitions would not be perceived as they were. In other words, robots react to the audience, which casts changes in the audience's perception, then robots sense the changes and react again, forming a continuous feedback loop or improvisation between the robots and the audience.

\paragraph{Open Interpretations Make Robotic Art}
Artwork that remains open, undetermined, complex, and vague often invites diverse interpretations~\cite{eco1989open}. The same applies to interaction design where systems may not have a single user interpretation~\cite{sengers2006staying}. Samuel built three humanoid robots with different levels of functionality. The third robot, though technologically more advanced, received less curiosity from the audience than the first, more rudimentary robot:

\begin{quote}
    [For the third-gen robot],... people immediately understand what he (the robot) is doing. So people just leave after five minutes. But [for] the first one (first-gen robot), people tend to spend like 20 [or] 30 minutes because people don't understand what he's doing. But now it [the third-gen robot] is interpretable, so I understand that... giving him too much meaning is dangerous, [when] work[ing] on an art stuff, because people get tired... people are used to those things (technological functions), which [have] tons of meaning [about] what the machine is doing.
\end{quote}

Here, incorporating technical functionalities into the robot assigns clear objectives easily grasped by the audience, making the perceived meanings more rigid and restricting the scope for diversified interpretation.

Beyond the individual level, the way of interpretation is also socially shaped. Samuel made the point that the perception of creativity is also partly a social product because ``\textit{creativity is depending on what kind of society we are in and what kind of people we are interacting with}.'' Mark and Robert extended that the perception of robotic art is culturally conditioned, varying across different societies and generations. They mentioned how the animist cultural tendency of some East Asian societies potentially makes people more willing to accept and interested in robots and non-human entities (e.g., plants and animals) behaving as if intelligent and agentic. The way that the social context of interpretation and perception determines artistic values reiterates our claim that the audience's reception of robotic artwork is one of the key aspects of robotic art practice. It suggests that in achieving certain artistic goals by robotic art, considering the audience's background and ``horizon of expectations''~\cite{jauss1982toward}---the socially and historically conditioned structure by which a person comprehends, interprets, and appraises any text based on cultural codes and lived experiences---may be constructive in refining the work's idea.

\subsection{Process of Creation and Exhibition}

Many of the artists we interviewed (N=6) emphasized, or alluded to, the artistic value in the \textit{process} of making robotic art. Specifically, two types of processes are discussed here---the process of \textit{creation} and the process of \textit{exhibition}, reflecting two salient temporal stages of robotic art practice. We do not, by any means, suggest that process is unique to robotic art; apparently, other forms of art also attend to processes of their art practice. Our intention has been to situate the analysis of process in the emerging, particular context of robotic art and to reveal how process leads to a new understanding of robot's uses and roles in real-world scenarios.

Sophie builds robotic systems capable of physically painting on canvas. She uses these robots to explore the painting process itself rather than to focus on the final product—what she referred to as images instead of paintings. Her case exemplifies that the act of making can become the focal point of artistic interest. In her view, paintings as artifacts are space-and-time bound ``material-based work'' that requires ``interactive practice'' and ``decision making,'' whereas the resulting images are ``merely digital representation[s]'' of this process. The difference between images generated by computer programs and paintings created through human touch underscores her rationale for utilizing robots: to bring the tactile, material process of painting to the forefront.

\begin{quote}
   [I]n the end, if I'm trying to crop everything (all my ideas) together, then it (the commonality) is to make the temporality of the decision making process of painting more visible and present. So I'm not really interested in how the image looks. And we experience an object that actually has a temporal element, how it's been created with layers, with tons of decision making... because I am interested in painting as a process and less [as] a product, I'm trying to use the process of making a painting to reflect a lot of our human creativity, our relationship to machines, questions of agencies, and so on.
\end{quote}

She has been building robotic systems that have ``adaptive behavior[s]'' during the painting process, where the systems are designed to ``\textit{analyze a stroke [on the canvas] and then create a successive one}.'' This design ensures that robots' actions are not exclusively dictated by the pre-programmed instructions but also influenced by the constantly changing ``state of the world,'' which includes factors such as the evolving canvas, environmental conditions, and the interaction between the robot and its surroundings. Consequently, a painting is not just a visual product but represents a series of actions with a temporal dimension.

Another important process for robotic art practice is exhibition. In the exhibition space, robotic artworks often take the forms of performances or improvisations, actively interacting and potentially shaping their environment in real time. 
For example, Alex's robots paint spontaneous color patterns on canvas during the exhibition, transforming the event into a performative art experience that aligns with his intention of foregrounding the painting process. The dynamic nature of live drawing at the exhibition---``making a show live''---has been central to Alex's artistic approach.
Moreover, new qualities of robotic artwork not only emerge by interacting with other entities, such as viewers or environmental factors, but also through the artwork itself as it develops over time. Daniel recounted an instance where a crack in his robotic installation continued to expand, gradually altering the artwork throughout the exhibition:

\begin{quote}
    I used [a] dome as a costume of the robot, and it (the robot) was an interactive real-time installation. The foam [on the dome] got a crack, and I decided to keep it cracking throughout the exhibition for one week. The crack in the costume was tearing down and it created a different artistic situation I could not have planned. It was so strong that it changed the whole work... I want to be sensible to those moments and see them as part of the process... I don't see that (situation) as, `okay, that is now destroying my artwork.' No, it is evolving or creating a new one within.
\end{quote}

This case illustrates how robotic artwork is not fixed but remains malleable even during the exhibition stage; temporal changes within the artwork can introduce new artistic qualities that evolve the work beyond its original design. Highlighting the artwork's temporality here allows for elucidating how the current state of the created artifact and creativity come to be. The practice of robotic art thus extends beyond the creation stage, encompassing the exhibition period. While in many cases the creation process is well planned, and temporal changes during the exhibition are typically unforeseen, both processes reveal that robotic art is in a state of ongoing creation across time. By paying attention to these processes, we unravel the temporal dimension that contributes to the creative values in robotic art.

In this Findings section, we have highlighted four aspects of robotic art practice that contribute to the artistic quality of the work or to achieving some artistic goals. The analysis reveals how various actors—artists, robots, audiences, and environments---are involved in the practice, influencing one another. These interactive patterns explain how creativity in robotic art is distributed within and emerges from the relations of actors. This idea echos with Daniel's reflection, as he noted that he sees robotic artwork as \textit{``a product of a situation of a creative potential that is part of the environment, all the entities involved as well as me,''} emphasizing the distributed and emergent nature of creativity in robotic art.

\section{Applications}
\label{section:applications}
We now demonstrate how our survey findings can be used to create accessibility-oriented analytical maps and personalized routing algorithms. We first synthesize our  findings into user preferences before describing our two prototypes.

\subsection{User Preferences}
While ~\autoref{section:findings} was largely organized around barrier types, here we summarize findings by mobility aid. 
Our intent is to provide a more holistic synthesis across different survey parts and demonstrate how this data can be used to create more personalized, disability-infused mapping applications.

\begin{figure}[b]
    \centering
    \includegraphics[width=1\linewidth]{figures/figure-most-impassable-2column.png}
    \caption{Examples of the least passable images across mobility groups.}
    \Description{This figure shows an array of six examples of the least passable images for each mobility group.}
    \label{fig:least-passable}
\end{figure}

\textbf{Walking canes.}
Walking cane users generally showed more confidence in maneuvering through or around sidewalk barriers compared to other groups. However, they still perceive high severity obstacles and high severity surface problems to be challenging (37\% and 44\% passable votes, respectively). 
The top two most difficult sidewalk barriers for walking cane users were overgrown vegetation on an already narrow sidewalk and branches obstructing the walkway (\autoref{fig:least-passable}A and B), with only 19\% and 23\% of users, respectively, indicating they could confidently pass.

\begin{figure*}
  \centering
  \includegraphics[width=\linewidth]{figures/access-score-maps.png}
  \caption{AccessScore maps visualizing sidewalk quality in Seattle for two groups: walking cane and mobility scooter (red is least accessible; green is most). Top two shows AccessScore by neighborhood; bottom two shows AccessScore by sidewalk segment. From the comparisons between walking cane users and mobility scooter users, we can see while downtown area may be equally accessible for both user groups, other areas are less accessible for mobility scooter users. }
  \Description{This figure shows AccessScore maps visualizing sidewalk quality in Seattle. Top two shows AccessScore by neighborhood; bottom two shows AccessScore by sidewalk segment. From the comparisons between walking cane users and mobility scooter users, we can see while downtown area may be equally accessible for both user groups, other areas are less accessible for mobility scooter users.}
  \label{fig:fig:access-maps}
\end{figure*}

\textbf{Walkers.}
Walker users were particularly sensitive to narrow sidewalks, including sidewalks narrowed by obstacles such as vegetation (40\% of walker users voted passable), parked cars, scooters, and bikes (32\%), as well as inherently narrow sidewalk surfaces (32\%). 
People who use walkers also struggle with cracks and uneven sidewalks, with more than 45\% of the votes indicating they are difficult to pass. 
The most challenging barriers for walker users were a parked bike in the middle of the sidewalk and branches obstructing the walkway (\autoref{fig:least-passable}C and B), with only 9\% and 10\% of users, respectively, indicating they could pass these obstacles.

\textbf{Mobility scooters.}
Mobility scooter users marked the most images as impassable (24 of 52 images). 
Examining users' passability confidence across severity levels revealed that these users were more likely to find images in both mid- and high-severity levels impassable, with only a 55\% passable ratio. 
This is lower compared to all other mobility aid users: walking cane (74\% ), walker (58\%), manual wheelchair (68\%), and motorized wheelchair (59\%).
Mobility scooter users were also particularly sensitive to poorly designed curb ramps, with a low passibility rate for curb ramps of 49\%. 
The top three most difficult sidewalk barriers for mobility scooter users were overgrown vegetation on a narrow sidewalk (\autoref{fig:least-passable}A), a broken sidewalk surface with mud (\autoref{fig:least-passable}D), and an uplifted sidewalk panel due to tree roots (\autoref{fig:least-passable}E), each with only 14\% of users indicating they could pass these barriers.

\textbf{Manual wheelchairs.}
Manual wheelchair users found high severity obstacles (18\% passable), surface problems (29\% passable), and all missing curb ramps (24\% passable) to be particularly challenging. 
Their top two most difficult sidewalk barriers were overgrown vegetation on a narrow sidewalk (\autoref{fig:least-passable}A) and a pole in the middle of the sidewalk with slope (\autoref{fig:least-passable}F), with only 11\% of users indicating they could pass these obstacles for both barriers.

\textbf{Motorized wheelchairs.}
Motorized wheelchair users showed similar patterns to manual wheelchair users but were even more sensitive to missing curb ramps (20\%  passable). This echoed an insight from one of our pilot participants: \sayit{If I am on a manual wheelchair and I see a missing curb ramp, I can do a wheelie to get on top of it, but it might not be possible when using a motorized wheelchair.} The top two most difficult sidewalk barriers for motorized wheelchair users were overgrown vegetation on a narrow sidewalk (\autoref{fig:least-passable}A) and a parked bike in the middle of the sidewalk (\autoref{fig:least-passable}C), with only 6\% of users indicating they could pass these obstacles for each barrier.

\begin{figure*}
  \centering
  \includegraphics[width=\linewidth]{figures/figure-routing-application-new.png}
  \caption{Routing application using OSMnx to generate routes between A \& B based on our survey data. Yellow route shows the absolute shortest path; teal shows the route for walking cane, this route favours fewer sidewalk barriers regardless of category; purple shows the route for motorized wheelchair, this route avoids missing curb ramps at all costs. When hovering over the labels, users can see what the sidewalk issues look like in streetview.}
  \Description{This figure shows routing application using OSMnx to generate routes between A \& B based on user preferences. Yellow route shows the absolute shortest path; teal shows the route for walking cane, this route favors fewer sidewalk barriers regardless of category; purple shows the route for motorized wheelchair, this route avoids missing curb ramps at all costs.}
  \label{fig:routing}
\end{figure*}

\subsection{Accessibility Map}

High-quality sidewalks play a vital role in the urban environment by encouraging physical activity~\cite{lopez_obesity_2006}, facilitating connectivity~\cite{randall_evaluating_2001}, increasing safety~\cite{abou-senna_investigating_2022}, and enhancing the sense of community~\cite{demerath_social_2003, bise_sidewalks_2018}. 
Current commercial tools like Walk Score~\cite{walk_score_walk_2007} take into account the use of sidewalks in gaining access to important amenities, and have been widely used by people to make informed decisions about where to live and which transportation modes to use. 
However, these tools often fail to capture the nuances of sidewalk accessibility for people with varying levels of mobility. 
The same sidewalk infrastructure can present drastically different levels of quality and usability for mobility aid users.

To address this problem, we prototyped an urban analytic tool that showcases sidewalk quality based on different mobility aid groups using data from our survey.
We used Project Sidewalk open label dataset (curb ramps, missing curb ramps, obstacles, and surface problems) from Seattle\footnote{\href{https://seattle.projectsidewalk.org/api}{https://seattle.projectsidewalk.org/api}} and mapped the labels onto sidewalk geometry gathered from the \textit{Seattle Open Data Portal}\footnote{\href{https://data-seattlecitygis.opendata.arcgis.com/datasets/SeattleCityGIS::sidewalks-1/about}{https://data-seattlecitygis.opendata.arcgis.com/datasets/SeattleCityGIS::sidewalks-1/about}}. 
We extended previous methods of using Project Sidewalk~\cite{li_interactively_2018, hara_scalable_2014, li_pilot_2022} labels to calculate \textit{AccessScore} by incorporating our survey findings.
The confidence that a sidewalk barrier type is not passable ($C_{label}$) was determined using the percentage of \sayit{No} and \sayit{Unsure} votes from ~\autoref{fig:image-selection-results}. For example, $C_{SurfaceProblem}$ for walking cane users is $0.54$, thus we weighted surface problems by $0.54$ when calculating their \textit{AccessScore}.
We generated sidewalk accessibility maps at both segment and neighborhood scales, with scores ranging from 0 (least accessible) to 1 (most accessible). 

\autoref{fig:fig:access-maps} compares the results for walking cane and mobility scooter users. 
The results show that, while downtown Seattle may be accessible for both groups, mobility scooter users face more challenges in other geographic areas. 
Such visualizations act like a Walk Score~\cite{walk_score_walk_2007} for mobility aid users, they can \textit{help people in choosing suitable living locations} and \textit{guide officials in prioritizing accessibility improvements}.

\subsection{Personalized Routing}
Existing navigation tools (\textit{e.g.,} Google Maps, Apple Maps) fail to address the needs of people with mobility disabilities. This section demonstrates how "one-size-fits-all" applications are insufficient for people with different mobility aids and how our survey data enables more accurate personalized routing. 

To develop a routing prototype, we first created a topologically connected routable network for our study area using the sidewalk network from OSM.
We then integrated Project Sidewalk labels by mapping obstacles and surface problems onto sidewalk segments, and (missing curb ramps) were mapped onto the crossing segments. 
To incorporate user profiles, we again used the confidence score that a sidewalk barrier type is not passable ($C_{label}$). 
Then, for each segment in the sidewalk network, we calculated the weighted distance for each segment as the segment length plus $C_{label}$ multiplied by the number of labels and 10\% of the segment length~\cite{tannert_disabled_2018}. 
Using OSMnx~\cite{boeing_osmnx_2017}, we next calculated the shortest distance between two intersection points (30th Avenue and East Columbia Street; 38th Avenue and East Union Street in Seattle) based on these weighted distances.

~\autoref{fig:routing} shows the shortest paths using absolute length and weighted length for walking cane users and motorized wheelchair users, with Project Sidewalk labels overlaid on the map.
The results demonstrate that users are given different optimal paths based on their specific needs and preferences. 
Walking cane users are routed along a path with some missing curb ramps but almost free of surface problems and obstacles, while motorized wheelchair users are given a longer path that avoids all areas with missing curb ramps. 
The results powerfully demonstrate how leveraging crowdsourced accessibility data and user preferences can yield more \textit{accurate and personalized routing algorithms} for mobility aid users.






\section{Discussion}
Through our application of personalized accessibility maps and routing applications, we showed how data and insights from our survey findings can help inform the development of more accurate navigation and analytical tools. 
We now situate our findings in related work, highlight how this survey contributes to personalized routing and accessibility mapping for mobility disability groups as well as present directions for future research.

\subsection{Online Image Survey Method}
In this study, we conducted a large-scale image survey (\textit{N=}190) to gather perceptions of sidewalk barriers from different mobility aid user groups. 
This approach helped us to collect insights on the differences between mobility aid user groups as well as shared challenges.
Previous research exploring the relationship between mobility aids and physical environment have mainly employed methods including in-person interviews~\cite{rosenberg_outdoor_2013}, GPS tracking~\cite{prescott_exploration_2021, prescott_factors_2020,rosenberg_outdoor_2013}, and online questionnaires~\cite{carlson_wheelchair_2002}. While interviews and tracking studies typically yield rich detailed information, they are limited to a small sample size. Online text based questionnaires often achieve larger sample sizes but at a cost of depth and nuance. Our image survey method struck a balance between sample size and detail. We collected a large sample within a relatively short time frame, enabling us to gather valuable insights and synthesize patterns across user groups.

Despite advantages, our approach has some limitations. Although street view images help situate and ground a participant's response---as one pilot participant said ``\textit{You're triggering a similar response to a real-life scenario''}, they cannot fully replicate the experience of evaluating a sidewalk \textit{in situ}. The lack of physical interaction with the environment limits the assessment of certain factors. For instance, one of our pilot participants noted that determining whether they could navigate past an obstacle like a trash can varies depending on \sayit{whether the trash can is light enough so I can push it away.} Using our findings as a backdrop, future work should conduct follow-up interviews and in-person evaluations. Such approaches would complement the quantitative data with richer qualitative insights, allowing researchers to better understand the patterns observed in quantitative data as well as the reasoning behind mobility aids users’ assessment.

\subsection{Personalized Accessibility Maps}
Our approach to infuse accessibility maps and routing algorithms with personalized information contributes to the field of accessible urban navigation and analytics. 
Based on our findings, we implemented two accessibility-oriented mapping prototypes, which demonstrate how our data can be used in urban accessibility analytics and personalized routing algorithms. While our current implementation serves as a proof of concept, future research could explore using our findings with more advanced modeling methods such as fuzzy logic~\cite{kasemsuppakorn_personalised_2009, gharebaghi_user-specific_2021, hashemi_collaborative_2017} and AHP~\cite{kasemsuppakorn_personalised_2009,kasemsuppakorn_understanding_2015, hashemi_collaborative_2017}. 

For our current map applications, we used a single set of open-source sidewalk data from Project Sidewalk. However, we acknowledge that other important factors are not included, such as sidewalk topography, width, stairs, crossing conditions, paving material, lighting conditions, weather, and pedestrian traffic~\cite{rosenberg_outdoor_2013,kasemsuppakorn_personalised_2009,darko_adaptive_2022,hashemi_collaborative_2017,sobek_u-access_2006,bigonnesse_role_2018}. 
Future work should build upon our foundation by incorporating more crowdsourced and government official datasets.

While mobility aids play a crucial role in determining accessibility needs, we must recognize that individuals using the same type of mobility aid may have diverse preferences. As one of our pilot participants stated, \sayit{your wheelchair has to be shaped and fitted to your body similar to how you need shoes specifically for your feet.} This insight underscores the need for personalization beyond broad mobility aid categories. Other factors including age~\cite{rosenberg_outdoor_2013}, disability type~\cite{prescott_factors_2020}, body strength~\cite{prescott_factors_2020}, and route familiarity~\cite{kasemsuppakorn_understanding_2015} should be explored in the future. Our attempt in creating personalized maps is not to provide a one-size-fits-all solution for generalized mobility aid groups, but rather to leverage the power of defaults~\cite{nielsen_power_2005} and offer users an improved baseline from which they can easily customize based on their individual needs.

\subsection{Limitations and Future Work}
Due to the visual nature of our survey—images were the primary stimuli—we specifically excluded people who are blind or have low vision\footnote{That said, the custom online survey was made fully screen reader accessible; see \href{https://sidewalk-survey.github.io/}{https://sidewalk-survey.github.io/} for the images and alt text.}. However, as noted previously, many different disabilities can impact mobility, including sensory, physical, and cognitive. Prior research has explored the incorporation of visually impaired or blind individuals into route generation~\cite{volkel_routecheckr_2008}, recognizing shared barriers and the prevalence of multiple disabilities among users. Building upon this foundation, future work should expand the participant pool to include a broader range of disabilities, thereby providing a more comprehensive understanding of diverse accessibility needs.

While we demonstrated two basic scenario applications, our survey findings and personalized mapping approach have potential for broader implementation. One promising direction is in developing barrier removal strategies for policymakers~\cite{eisenberg_barrier-removal_2022}. Current government plans often rely on simple metrics, such as population density or proximity to public buildings~\cite{seattle_department_of_transportation_seattle_2021}. Our methodology could enhance these efforts by identifying sidewalk barriers whose removal would yield the greatest overall benefit to the largest percentage of mobility aid users in the form of connected, safe, accessible routes.
\section{Conclusion}

In this paper, we introduce STeCa, a novel agent learning framework designed to enhance the performance of LLM agents in long-horizon tasks. 
STeCa identifies deviated actions through step-level reward comparisons and constructs calibration trajectories via reflection. 
These trajectories serve as critical data for reinforced training. Extensive experiments demonstrate that STeCa significantly outperforms baseline methods, with additional analyses underscoring its robust calibration capabilities.
\begin{acks}
We thank all participants who took part in this study, without whom this project would not have been possible. We also thank Jacob O. Wobbrock for his help with data analysis, Michael Saugstad, Zhihan Zhang, Minchu Kulkarni, Jerry Cao for their help in survey/visualization development, as well as academic writing advisor Sandy Kaplan, UW CREATE community enagagement manager Kathleen Quin Voss, and the Allen School Computer Science Laboratory Group. This work was supported by NSF SCC-IRG \#2125087.
\end{acks}



%%
%% The next two lines define the bibliography style to be used, and
%% the bibliography file.
\bibliographystyle{ACM-Reference-Format}
\bibliography{references}

\newpage
\subsection{Lloyd-Max Algorithm}
\label{subsec:Lloyd-Max}
For a given quantization bitwidth $B$ and an operand $\bm{X}$, the Lloyd-Max algorithm finds $2^B$ quantization levels $\{\hat{x}_i\}_{i=1}^{2^B}$ such that quantizing $\bm{X}$ by rounding each scalar in $\bm{X}$ to the nearest quantization level minimizes the quantization MSE. 

The algorithm starts with an initial guess of quantization levels and then iteratively computes quantization thresholds $\{\tau_i\}_{i=1}^{2^B-1}$ and updates quantization levels $\{\hat{x}_i\}_{i=1}^{2^B}$. Specifically, at iteration $n$, thresholds are set to the midpoints of the previous iteration's levels:
\begin{align*}
    \tau_i^{(n)}=\frac{\hat{x}_i^{(n-1)}+\hat{x}_{i+1}^{(n-1)}}2 \text{ for } i=1\ldots 2^B-1
\end{align*}
Subsequently, the quantization levels are re-computed as conditional means of the data regions defined by the new thresholds:
\begin{align*}
    \hat{x}_i^{(n)}=\mathbb{E}\left[ \bm{X} \big| \bm{X}\in [\tau_{i-1}^{(n)},\tau_i^{(n)}] \right] \text{ for } i=1\ldots 2^B
\end{align*}
where to satisfy boundary conditions we have $\tau_0=-\infty$ and $\tau_{2^B}=\infty$. The algorithm iterates the above steps until convergence.

Figure \ref{fig:lm_quant} compares the quantization levels of a $7$-bit floating point (E3M3) quantizer (left) to a $7$-bit Lloyd-Max quantizer (right) when quantizing a layer of weights from the GPT3-126M model at a per-tensor granularity. As shown, the Lloyd-Max quantizer achieves substantially lower quantization MSE. Further, Table \ref{tab:FP7_vs_LM7} shows the superior perplexity achieved by Lloyd-Max quantizers for bitwidths of $7$, $6$ and $5$. The difference between the quantizers is clear at 5 bits, where per-tensor FP quantization incurs a drastic and unacceptable increase in perplexity, while Lloyd-Max quantization incurs a much smaller increase. Nevertheless, we note that even the optimal Lloyd-Max quantizer incurs a notable ($\sim 1.5$) increase in perplexity due to the coarse granularity of quantization. 

\begin{figure}[h]
  \centering
  \includegraphics[width=0.7\linewidth]{sections/figures/LM7_FP7.pdf}
  \caption{\small Quantization levels and the corresponding quantization MSE of Floating Point (left) vs Lloyd-Max (right) Quantizers for a layer of weights in the GPT3-126M model.}
  \label{fig:lm_quant}
\end{figure}

\begin{table}[h]\scriptsize
\begin{center}
\caption{\label{tab:FP7_vs_LM7} \small Comparing perplexity (lower is better) achieved by floating point quantizers and Lloyd-Max quantizers on a GPT3-126M model for the Wikitext-103 dataset.}
\begin{tabular}{c|cc|c}
\hline
 \multirow{2}{*}{\textbf{Bitwidth}} & \multicolumn{2}{|c|}{\textbf{Floating-Point Quantizer}} & \textbf{Lloyd-Max Quantizer} \\
 & Best Format & Wikitext-103 Perplexity & Wikitext-103 Perplexity \\
\hline
7 & E3M3 & 18.32 & 18.27 \\
6 & E3M2 & 19.07 & 18.51 \\
5 & E4M0 & 43.89 & 19.71 \\
\hline
\end{tabular}
\end{center}
\end{table}

\subsection{Proof of Local Optimality of LO-BCQ}
\label{subsec:lobcq_opt_proof}
For a given block $\bm{b}_j$, the quantization MSE during LO-BCQ can be empirically evaluated as $\frac{1}{L_b}\lVert \bm{b}_j- \bm{\hat{b}}_j\rVert^2_2$ where $\bm{\hat{b}}_j$ is computed from equation (\ref{eq:clustered_quantization_definition}) as $C_{f(\bm{b}_j)}(\bm{b}_j)$. Further, for a given block cluster $\mathcal{B}_i$, we compute the quantization MSE as $\frac{1}{|\mathcal{B}_{i}|}\sum_{\bm{b} \in \mathcal{B}_{i}} \frac{1}{L_b}\lVert \bm{b}- C_i^{(n)}(\bm{b})\rVert^2_2$. Therefore, at the end of iteration $n$, we evaluate the overall quantization MSE $J^{(n)}$ for a given operand $\bm{X}$ composed of $N_c$ block clusters as:
\begin{align*}
    \label{eq:mse_iter_n}
    J^{(n)} = \frac{1}{N_c} \sum_{i=1}^{N_c} \frac{1}{|\mathcal{B}_{i}^{(n)}|}\sum_{\bm{v} \in \mathcal{B}_{i}^{(n)}} \frac{1}{L_b}\lVert \bm{b}- B_i^{(n)}(\bm{b})\rVert^2_2
\end{align*}

At the end of iteration $n$, the codebooks are updated from $\mathcal{C}^{(n-1)}$ to $\mathcal{C}^{(n)}$. However, the mapping of a given vector $\bm{b}_j$ to quantizers $\mathcal{C}^{(n)}$ remains as  $f^{(n)}(\bm{b}_j)$. At the next iteration, during the vector clustering step, $f^{(n+1)}(\bm{b}_j)$ finds new mapping of $\bm{b}_j$ to updated codebooks $\mathcal{C}^{(n)}$ such that the quantization MSE over the candidate codebooks is minimized. Therefore, we obtain the following result for $\bm{b}_j$:
\begin{align*}
\frac{1}{L_b}\lVert \bm{b}_j - C_{f^{(n+1)}(\bm{b}_j)}^{(n)}(\bm{b}_j)\rVert^2_2 \le \frac{1}{L_b}\lVert \bm{b}_j - C_{f^{(n)}(\bm{b}_j)}^{(n)}(\bm{b}_j)\rVert^2_2
\end{align*}

That is, quantizing $\bm{b}_j$ at the end of the block clustering step of iteration $n+1$ results in lower quantization MSE compared to quantizing at the end of iteration $n$. Since this is true for all $\bm{b} \in \bm{X}$, we assert the following:
\begin{equation}
\begin{split}
\label{eq:mse_ineq_1}
    \tilde{J}^{(n+1)} &= \frac{1}{N_c} \sum_{i=1}^{N_c} \frac{1}{|\mathcal{B}_{i}^{(n+1)}|}\sum_{\bm{b} \in \mathcal{B}_{i}^{(n+1)}} \frac{1}{L_b}\lVert \bm{b} - C_i^{(n)}(b)\rVert^2_2 \le J^{(n)}
\end{split}
\end{equation}
where $\tilde{J}^{(n+1)}$ is the the quantization MSE after the vector clustering step at iteration $n+1$.

Next, during the codebook update step (\ref{eq:quantizers_update}) at iteration $n+1$, the per-cluster codebooks $\mathcal{C}^{(n)}$ are updated to $\mathcal{C}^{(n+1)}$ by invoking the Lloyd-Max algorithm \citep{Lloyd}. We know that for any given value distribution, the Lloyd-Max algorithm minimizes the quantization MSE. Therefore, for a given vector cluster $\mathcal{B}_i$ we obtain the following result:

\begin{equation}
    \frac{1}{|\mathcal{B}_{i}^{(n+1)}|}\sum_{\bm{b} \in \mathcal{B}_{i}^{(n+1)}} \frac{1}{L_b}\lVert \bm{b}- C_i^{(n+1)}(\bm{b})\rVert^2_2 \le \frac{1}{|\mathcal{B}_{i}^{(n+1)}|}\sum_{\bm{b} \in \mathcal{B}_{i}^{(n+1)}} \frac{1}{L_b}\lVert \bm{b}- C_i^{(n)}(\bm{b})\rVert^2_2
\end{equation}

The above equation states that quantizing the given block cluster $\mathcal{B}_i$ after updating the associated codebook from $C_i^{(n)}$ to $C_i^{(n+1)}$ results in lower quantization MSE. Since this is true for all the block clusters, we derive the following result: 
\begin{equation}
\begin{split}
\label{eq:mse_ineq_2}
     J^{(n+1)} &= \frac{1}{N_c} \sum_{i=1}^{N_c} \frac{1}{|\mathcal{B}_{i}^{(n+1)}|}\sum_{\bm{b} \in \mathcal{B}_{i}^{(n+1)}} \frac{1}{L_b}\lVert \bm{b}- C_i^{(n+1)}(\bm{b})\rVert^2_2  \le \tilde{J}^{(n+1)}   
\end{split}
\end{equation}

Following (\ref{eq:mse_ineq_1}) and (\ref{eq:mse_ineq_2}), we find that the quantization MSE is non-increasing for each iteration, that is, $J^{(1)} \ge J^{(2)} \ge J^{(3)} \ge \ldots \ge J^{(M)}$ where $M$ is the maximum number of iterations. 
%Therefore, we can say that if the algorithm converges, then it must be that it has converged to a local minimum. 
\hfill $\blacksquare$


\begin{figure}
    \begin{center}
    \includegraphics[width=0.5\textwidth]{sections//figures/mse_vs_iter.pdf}
    \end{center}
    \caption{\small NMSE vs iterations during LO-BCQ compared to other block quantization proposals}
    \label{fig:nmse_vs_iter}
\end{figure}

Figure \ref{fig:nmse_vs_iter} shows the empirical convergence of LO-BCQ across several block lengths and number of codebooks. Also, the MSE achieved by LO-BCQ is compared to baselines such as MXFP and VSQ. As shown, LO-BCQ converges to a lower MSE than the baselines. Further, we achieve better convergence for larger number of codebooks ($N_c$) and for a smaller block length ($L_b$), both of which increase the bitwidth of BCQ (see Eq \ref{eq:bitwidth_bcq}).


\subsection{Additional Accuracy Results}
%Table \ref{tab:lobcq_config} lists the various LOBCQ configurations and their corresponding bitwidths.
\begin{table}
\setlength{\tabcolsep}{4.75pt}
\begin{center}
\caption{\label{tab:lobcq_config} Various LO-BCQ configurations and their bitwidths.}
\begin{tabular}{|c||c|c|c|c||c|c||c|} 
\hline
 & \multicolumn{4}{|c||}{$L_b=8$} & \multicolumn{2}{|c||}{$L_b=4$} & $L_b=2$ \\
 \hline
 \backslashbox{$L_A$\kern-1em}{\kern-1em$N_c$} & 2 & 4 & 8 & 16 & 2 & 4 & 2 \\
 \hline
 64 & 4.25 & 4.375 & 4.5 & 4.625 & 4.375 & 4.625 & 4.625\\
 \hline
 32 & 4.375 & 4.5 & 4.625& 4.75 & 4.5 & 4.75 & 4.75 \\
 \hline
 16 & 4.625 & 4.75& 4.875 & 5 & 4.75 & 5 & 5 \\
 \hline
\end{tabular}
\end{center}
\end{table}

%\subsection{Perplexity achieved by various LO-BCQ configurations on Wikitext-103 dataset}

\begin{table} \centering
\begin{tabular}{|c||c|c|c|c||c|c||c|} 
\hline
 $L_b \rightarrow$& \multicolumn{4}{c||}{8} & \multicolumn{2}{c||}{4} & 2\\
 \hline
 \backslashbox{$L_A$\kern-1em}{\kern-1em$N_c$} & 2 & 4 & 8 & 16 & 2 & 4 & 2  \\
 %$N_c \rightarrow$ & 2 & 4 & 8 & 16 & 2 & 4 & 2 \\
 \hline
 \hline
 \multicolumn{8}{c}{GPT3-1.3B (FP32 PPL = 9.98)} \\ 
 \hline
 \hline
 64 & 10.40 & 10.23 & 10.17 & 10.15 &  10.28 & 10.18 & 10.19 \\
 \hline
 32 & 10.25 & 10.20 & 10.15 & 10.12 &  10.23 & 10.17 & 10.17 \\
 \hline
 16 & 10.22 & 10.16 & 10.10 & 10.09 &  10.21 & 10.14 & 10.16 \\
 \hline
  \hline
 \multicolumn{8}{c}{GPT3-8B (FP32 PPL = 7.38)} \\ 
 \hline
 \hline
 64 & 7.61 & 7.52 & 7.48 &  7.47 &  7.55 &  7.49 & 7.50 \\
 \hline
 32 & 7.52 & 7.50 & 7.46 &  7.45 &  7.52 &  7.48 & 7.48  \\
 \hline
 16 & 7.51 & 7.48 & 7.44 &  7.44 &  7.51 &  7.49 & 7.47  \\
 \hline
\end{tabular}
\caption{\label{tab:ppl_gpt3_abalation} Wikitext-103 perplexity across GPT3-1.3B and 8B models.}
\end{table}

\begin{table} \centering
\begin{tabular}{|c||c|c|c|c||} 
\hline
 $L_b \rightarrow$& \multicolumn{4}{c||}{8}\\
 \hline
 \backslashbox{$L_A$\kern-1em}{\kern-1em$N_c$} & 2 & 4 & 8 & 16 \\
 %$N_c \rightarrow$ & 2 & 4 & 8 & 16 & 2 & 4 & 2 \\
 \hline
 \hline
 \multicolumn{5}{|c|}{Llama2-7B (FP32 PPL = 5.06)} \\ 
 \hline
 \hline
 64 & 5.31 & 5.26 & 5.19 & 5.18  \\
 \hline
 32 & 5.23 & 5.25 & 5.18 & 5.15  \\
 \hline
 16 & 5.23 & 5.19 & 5.16 & 5.14  \\
 \hline
 \multicolumn{5}{|c|}{Nemotron4-15B (FP32 PPL = 5.87)} \\ 
 \hline
 \hline
 64  & 6.3 & 6.20 & 6.13 & 6.08  \\
 \hline
 32  & 6.24 & 6.12 & 6.07 & 6.03  \\
 \hline
 16  & 6.12 & 6.14 & 6.04 & 6.02  \\
 \hline
 \multicolumn{5}{|c|}{Nemotron4-340B (FP32 PPL = 3.48)} \\ 
 \hline
 \hline
 64 & 3.67 & 3.62 & 3.60 & 3.59 \\
 \hline
 32 & 3.63 & 3.61 & 3.59 & 3.56 \\
 \hline
 16 & 3.61 & 3.58 & 3.57 & 3.55 \\
 \hline
\end{tabular}
\caption{\label{tab:ppl_llama7B_nemo15B} Wikitext-103 perplexity compared to FP32 baseline in Llama2-7B and Nemotron4-15B, 340B models}
\end{table}

%\subsection{Perplexity achieved by various LO-BCQ configurations on MMLU dataset}


\begin{table} \centering
\begin{tabular}{|c||c|c|c|c||c|c|c|c|} 
\hline
 $L_b \rightarrow$& \multicolumn{4}{c||}{8} & \multicolumn{4}{c||}{8}\\
 \hline
 \backslashbox{$L_A$\kern-1em}{\kern-1em$N_c$} & 2 & 4 & 8 & 16 & 2 & 4 & 8 & 16  \\
 %$N_c \rightarrow$ & 2 & 4 & 8 & 16 & 2 & 4 & 2 \\
 \hline
 \hline
 \multicolumn{5}{|c|}{Llama2-7B (FP32 Accuracy = 45.8\%)} & \multicolumn{4}{|c|}{Llama2-70B (FP32 Accuracy = 69.12\%)} \\ 
 \hline
 \hline
 64 & 43.9 & 43.4 & 43.9 & 44.9 & 68.07 & 68.27 & 68.17 & 68.75 \\
 \hline
 32 & 44.5 & 43.8 & 44.9 & 44.5 & 68.37 & 68.51 & 68.35 & 68.27  \\
 \hline
 16 & 43.9 & 42.7 & 44.9 & 45 & 68.12 & 68.77 & 68.31 & 68.59  \\
 \hline
 \hline
 \multicolumn{5}{|c|}{GPT3-22B (FP32 Accuracy = 38.75\%)} & \multicolumn{4}{|c|}{Nemotron4-15B (FP32 Accuracy = 64.3\%)} \\ 
 \hline
 \hline
 64 & 36.71 & 38.85 & 38.13 & 38.92 & 63.17 & 62.36 & 63.72 & 64.09 \\
 \hline
 32 & 37.95 & 38.69 & 39.45 & 38.34 & 64.05 & 62.30 & 63.8 & 64.33  \\
 \hline
 16 & 38.88 & 38.80 & 38.31 & 38.92 & 63.22 & 63.51 & 63.93 & 64.43  \\
 \hline
\end{tabular}
\caption{\label{tab:mmlu_abalation} Accuracy on MMLU dataset across GPT3-22B, Llama2-7B, 70B and Nemotron4-15B models.}
\end{table}


%\subsection{Perplexity achieved by various LO-BCQ configurations on LM evaluation harness}

\begin{table} \centering
\begin{tabular}{|c||c|c|c|c||c|c|c|c|} 
\hline
 $L_b \rightarrow$& \multicolumn{4}{c||}{8} & \multicolumn{4}{c||}{8}\\
 \hline
 \backslashbox{$L_A$\kern-1em}{\kern-1em$N_c$} & 2 & 4 & 8 & 16 & 2 & 4 & 8 & 16  \\
 %$N_c \rightarrow$ & 2 & 4 & 8 & 16 & 2 & 4 & 2 \\
 \hline
 \hline
 \multicolumn{5}{|c|}{Race (FP32 Accuracy = 37.51\%)} & \multicolumn{4}{|c|}{Boolq (FP32 Accuracy = 64.62\%)} \\ 
 \hline
 \hline
 64 & 36.94 & 37.13 & 36.27 & 37.13 & 63.73 & 62.26 & 63.49 & 63.36 \\
 \hline
 32 & 37.03 & 36.36 & 36.08 & 37.03 & 62.54 & 63.51 & 63.49 & 63.55  \\
 \hline
 16 & 37.03 & 37.03 & 36.46 & 37.03 & 61.1 & 63.79 & 63.58 & 63.33  \\
 \hline
 \hline
 \multicolumn{5}{|c|}{Winogrande (FP32 Accuracy = 58.01\%)} & \multicolumn{4}{|c|}{Piqa (FP32 Accuracy = 74.21\%)} \\ 
 \hline
 \hline
 64 & 58.17 & 57.22 & 57.85 & 58.33 & 73.01 & 73.07 & 73.07 & 72.80 \\
 \hline
 32 & 59.12 & 58.09 & 57.85 & 58.41 & 73.01 & 73.94 & 72.74 & 73.18  \\
 \hline
 16 & 57.93 & 58.88 & 57.93 & 58.56 & 73.94 & 72.80 & 73.01 & 73.94  \\
 \hline
\end{tabular}
\caption{\label{tab:mmlu_abalation} Accuracy on LM evaluation harness tasks on GPT3-1.3B model.}
\end{table}

\begin{table} \centering
\begin{tabular}{|c||c|c|c|c||c|c|c|c|} 
\hline
 $L_b \rightarrow$& \multicolumn{4}{c||}{8} & \multicolumn{4}{c||}{8}\\
 \hline
 \backslashbox{$L_A$\kern-1em}{\kern-1em$N_c$} & 2 & 4 & 8 & 16 & 2 & 4 & 8 & 16  \\
 %$N_c \rightarrow$ & 2 & 4 & 8 & 16 & 2 & 4 & 2 \\
 \hline
 \hline
 \multicolumn{5}{|c|}{Race (FP32 Accuracy = 41.34\%)} & \multicolumn{4}{|c|}{Boolq (FP32 Accuracy = 68.32\%)} \\ 
 \hline
 \hline
 64 & 40.48 & 40.10 & 39.43 & 39.90 & 69.20 & 68.41 & 69.45 & 68.56 \\
 \hline
 32 & 39.52 & 39.52 & 40.77 & 39.62 & 68.32 & 67.43 & 68.17 & 69.30  \\
 \hline
 16 & 39.81 & 39.71 & 39.90 & 40.38 & 68.10 & 66.33 & 69.51 & 69.42  \\
 \hline
 \hline
 \multicolumn{5}{|c|}{Winogrande (FP32 Accuracy = 67.88\%)} & \multicolumn{4}{|c|}{Piqa (FP32 Accuracy = 78.78\%)} \\ 
 \hline
 \hline
 64 & 66.85 & 66.61 & 67.72 & 67.88 & 77.31 & 77.42 & 77.75 & 77.64 \\
 \hline
 32 & 67.25 & 67.72 & 67.72 & 67.00 & 77.31 & 77.04 & 77.80 & 77.37  \\
 \hline
 16 & 68.11 & 68.90 & 67.88 & 67.48 & 77.37 & 78.13 & 78.13 & 77.69  \\
 \hline
\end{tabular}
\caption{\label{tab:mmlu_abalation} Accuracy on LM evaluation harness tasks on GPT3-8B model.}
\end{table}

\begin{table} \centering
\begin{tabular}{|c||c|c|c|c||c|c|c|c|} 
\hline
 $L_b \rightarrow$& \multicolumn{4}{c||}{8} & \multicolumn{4}{c||}{8}\\
 \hline
 \backslashbox{$L_A$\kern-1em}{\kern-1em$N_c$} & 2 & 4 & 8 & 16 & 2 & 4 & 8 & 16  \\
 %$N_c \rightarrow$ & 2 & 4 & 8 & 16 & 2 & 4 & 2 \\
 \hline
 \hline
 \multicolumn{5}{|c|}{Race (FP32 Accuracy = 40.67\%)} & \multicolumn{4}{|c|}{Boolq (FP32 Accuracy = 76.54\%)} \\ 
 \hline
 \hline
 64 & 40.48 & 40.10 & 39.43 & 39.90 & 75.41 & 75.11 & 77.09 & 75.66 \\
 \hline
 32 & 39.52 & 39.52 & 40.77 & 39.62 & 76.02 & 76.02 & 75.96 & 75.35  \\
 \hline
 16 & 39.81 & 39.71 & 39.90 & 40.38 & 75.05 & 73.82 & 75.72 & 76.09  \\
 \hline
 \hline
 \multicolumn{5}{|c|}{Winogrande (FP32 Accuracy = 70.64\%)} & \multicolumn{4}{|c|}{Piqa (FP32 Accuracy = 79.16\%)} \\ 
 \hline
 \hline
 64 & 69.14 & 70.17 & 70.17 & 70.56 & 78.24 & 79.00 & 78.62 & 78.73 \\
 \hline
 32 & 70.96 & 69.69 & 71.27 & 69.30 & 78.56 & 79.49 & 79.16 & 78.89  \\
 \hline
 16 & 71.03 & 69.53 & 69.69 & 70.40 & 78.13 & 79.16 & 79.00 & 79.00  \\
 \hline
\end{tabular}
\caption{\label{tab:mmlu_abalation} Accuracy on LM evaluation harness tasks on GPT3-22B model.}
\end{table}

\begin{table} \centering
\begin{tabular}{|c||c|c|c|c||c|c|c|c|} 
\hline
 $L_b \rightarrow$& \multicolumn{4}{c||}{8} & \multicolumn{4}{c||}{8}\\
 \hline
 \backslashbox{$L_A$\kern-1em}{\kern-1em$N_c$} & 2 & 4 & 8 & 16 & 2 & 4 & 8 & 16  \\
 %$N_c \rightarrow$ & 2 & 4 & 8 & 16 & 2 & 4 & 2 \\
 \hline
 \hline
 \multicolumn{5}{|c|}{Race (FP32 Accuracy = 44.4\%)} & \multicolumn{4}{|c|}{Boolq (FP32 Accuracy = 79.29\%)} \\ 
 \hline
 \hline
 64 & 42.49 & 42.51 & 42.58 & 43.45 & 77.58 & 77.37 & 77.43 & 78.1 \\
 \hline
 32 & 43.35 & 42.49 & 43.64 & 43.73 & 77.86 & 75.32 & 77.28 & 77.86  \\
 \hline
 16 & 44.21 & 44.21 & 43.64 & 42.97 & 78.65 & 77 & 76.94 & 77.98  \\
 \hline
 \hline
 \multicolumn{5}{|c|}{Winogrande (FP32 Accuracy = 69.38\%)} & \multicolumn{4}{|c|}{Piqa (FP32 Accuracy = 78.07\%)} \\ 
 \hline
 \hline
 64 & 68.9 & 68.43 & 69.77 & 68.19 & 77.09 & 76.82 & 77.09 & 77.86 \\
 \hline
 32 & 69.38 & 68.51 & 68.82 & 68.90 & 78.07 & 76.71 & 78.07 & 77.86  \\
 \hline
 16 & 69.53 & 67.09 & 69.38 & 68.90 & 77.37 & 77.8 & 77.91 & 77.69  \\
 \hline
\end{tabular}
\caption{\label{tab:mmlu_abalation} Accuracy on LM evaluation harness tasks on Llama2-7B model.}
\end{table}

\begin{table} \centering
\begin{tabular}{|c||c|c|c|c||c|c|c|c|} 
\hline
 $L_b \rightarrow$& \multicolumn{4}{c||}{8} & \multicolumn{4}{c||}{8}\\
 \hline
 \backslashbox{$L_A$\kern-1em}{\kern-1em$N_c$} & 2 & 4 & 8 & 16 & 2 & 4 & 8 & 16  \\
 %$N_c \rightarrow$ & 2 & 4 & 8 & 16 & 2 & 4 & 2 \\
 \hline
 \hline
 \multicolumn{5}{|c|}{Race (FP32 Accuracy = 48.8\%)} & \multicolumn{4}{|c|}{Boolq (FP32 Accuracy = 85.23\%)} \\ 
 \hline
 \hline
 64 & 49.00 & 49.00 & 49.28 & 48.71 & 82.82 & 84.28 & 84.03 & 84.25 \\
 \hline
 32 & 49.57 & 48.52 & 48.33 & 49.28 & 83.85 & 84.46 & 84.31 & 84.93  \\
 \hline
 16 & 49.85 & 49.09 & 49.28 & 48.99 & 85.11 & 84.46 & 84.61 & 83.94  \\
 \hline
 \hline
 \multicolumn{5}{|c|}{Winogrande (FP32 Accuracy = 79.95\%)} & \multicolumn{4}{|c|}{Piqa (FP32 Accuracy = 81.56\%)} \\ 
 \hline
 \hline
 64 & 78.77 & 78.45 & 78.37 & 79.16 & 81.45 & 80.69 & 81.45 & 81.5 \\
 \hline
 32 & 78.45 & 79.01 & 78.69 & 80.66 & 81.56 & 80.58 & 81.18 & 81.34  \\
 \hline
 16 & 79.95 & 79.56 & 79.79 & 79.72 & 81.28 & 81.66 & 81.28 & 80.96  \\
 \hline
\end{tabular}
\caption{\label{tab:mmlu_abalation} Accuracy on LM evaluation harness tasks on Llama2-70B model.}
\end{table}

%\section{MSE Studies}
%\textcolor{red}{TODO}


\subsection{Number Formats and Quantization Method}
\label{subsec:numFormats_quantMethod}
\subsubsection{Integer Format}
An $n$-bit signed integer (INT) is typically represented with a 2s-complement format \citep{yao2022zeroquant,xiao2023smoothquant,dai2021vsq}, where the most significant bit denotes the sign.

\subsubsection{Floating Point Format}
An $n$-bit signed floating point (FP) number $x$ comprises of a 1-bit sign ($x_{\mathrm{sign}}$), $B_m$-bit mantissa ($x_{\mathrm{mant}}$) and $B_e$-bit exponent ($x_{\mathrm{exp}}$) such that $B_m+B_e=n-1$. The associated constant exponent bias ($E_{\mathrm{bias}}$) is computed as $(2^{{B_e}-1}-1)$. We denote this format as $E_{B_e}M_{B_m}$.  

\subsubsection{Quantization Scheme}
\label{subsec:quant_method}
A quantization scheme dictates how a given unquantized tensor is converted to its quantized representation. We consider FP formats for the purpose of illustration. Given an unquantized tensor $\bm{X}$ and an FP format $E_{B_e}M_{B_m}$, we first, we compute the quantization scale factor $s_X$ that maps the maximum absolute value of $\bm{X}$ to the maximum quantization level of the $E_{B_e}M_{B_m}$ format as follows:
\begin{align}
\label{eq:sf}
    s_X = \frac{\mathrm{max}(|\bm{X}|)}{\mathrm{max}(E_{B_e}M_{B_m})}
\end{align}
In the above equation, $|\cdot|$ denotes the absolute value function.

Next, we scale $\bm{X}$ by $s_X$ and quantize it to $\hat{\bm{X}}$ by rounding it to the nearest quantization level of $E_{B_e}M_{B_m}$ as:

\begin{align}
\label{eq:tensor_quant}
    \hat{\bm{X}} = \text{round-to-nearest}\left(\frac{\bm{X}}{s_X}, E_{B_e}M_{B_m}\right)
\end{align}

We perform dynamic max-scaled quantization \citep{wu2020integer}, where the scale factor $s$ for activations is dynamically computed during runtime.

\subsection{Vector Scaled Quantization}
\begin{wrapfigure}{r}{0.35\linewidth}
  \centering
  \includegraphics[width=\linewidth]{sections/figures/vsquant.jpg}
  \caption{\small Vectorwise decomposition for per-vector scaled quantization (VSQ \citep{dai2021vsq}).}
  \label{fig:vsquant}
\end{wrapfigure}
During VSQ \citep{dai2021vsq}, the operand tensors are decomposed into 1D vectors in a hardware friendly manner as shown in Figure \ref{fig:vsquant}. Since the decomposed tensors are used as operands in matrix multiplications during inference, it is beneficial to perform this decomposition along the reduction dimension of the multiplication. The vectorwise quantization is performed similar to tensorwise quantization described in Equations \ref{eq:sf} and \ref{eq:tensor_quant}, where a scale factor $s_v$ is required for each vector $\bm{v}$ that maps the maximum absolute value of that vector to the maximum quantization level. While smaller vector lengths can lead to larger accuracy gains, the associated memory and computational overheads due to the per-vector scale factors increases. To alleviate these overheads, VSQ \citep{dai2021vsq} proposed a second level quantization of the per-vector scale factors to unsigned integers, while MX \citep{rouhani2023shared} quantizes them to integer powers of 2 (denoted as $2^{INT}$).

\subsubsection{MX Format}
The MX format proposed in \citep{rouhani2023microscaling} introduces the concept of sub-block shifting. For every two scalar elements of $b$-bits each, there is a shared exponent bit. The value of this exponent bit is determined through an empirical analysis that targets minimizing quantization MSE. We note that the FP format $E_{1}M_{b}$ is strictly better than MX from an accuracy perspective since it allocates a dedicated exponent bit to each scalar as opposed to sharing it across two scalars. Therefore, we conservatively bound the accuracy of a $b+2$-bit signed MX format with that of a $E_{1}M_{b}$ format in our comparisons. For instance, we use E1M2 format as a proxy for MX4.

\begin{figure}
    \centering
    \includegraphics[width=1\linewidth]{sections//figures/BlockFormats.pdf}
    \caption{\small Comparing LO-BCQ to MX format.}
    \label{fig:block_formats}
\end{figure}

Figure \ref{fig:block_formats} compares our $4$-bit LO-BCQ block format to MX \citep{rouhani2023microscaling}. As shown, both LO-BCQ and MX decompose a given operand tensor into block arrays and each block array into blocks. Similar to MX, we find that per-block quantization ($L_b < L_A$) leads to better accuracy due to increased flexibility. While MX achieves this through per-block $1$-bit micro-scales, we associate a dedicated codebook to each block through a per-block codebook selector. Further, MX quantizes the per-block array scale-factor to E8M0 format without per-tensor scaling. In contrast during LO-BCQ, we find that per-tensor scaling combined with quantization of per-block array scale-factor to E4M3 format results in superior inference accuracy across models. 


\end{document}
\endinput
%%
%% End of file `sample-sigconf-authordraft.tex'.
