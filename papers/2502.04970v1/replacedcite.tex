\section{Related Work}
Deep learning survival models often extend the Cox regression framework by using NNs to parameterize the log-risk function, optimizing a Cox-based loss (negative log-partial likelihood of the Cox model) ____. Examples include \emph{DeepSurv} ____, which employs feedforward NNs to capture non-linear feature-hazard relationships while adhering to the Proportional Hazards (PH) assumption. It states that the hazard ratio between any two individuals remains constant over time, meaning the effect of features on the hazard function is multiplicative and does not vary with time. \emph{CoxTime} ____ incorporates time-dependent feature effects by including time as an additional feature. Another category of survival NNs adopts discrete-time methods, treating time as discrete to leverage classification techniques. The most prominent model in this category is \emph{DeepHit} ____, which directly models the joint distribution of survival times and event probabilities without assumptions about the stochastic process, allowing dynamic feature-risk relationships. Additional approaches include methods based on piecewise-exponential models, ordinary differential equations, and ranking techniques. For a comprehensive review, see ____. To date, the interpretability of survival NNs has primarily been addressed through model-agnostic, post-hoc methods. Prominent local XAI techniques in this domain include SurvLIME and SurvSHAP(t). For a comprehensive review of interpretability methods in survival analysis, we refer to ____. So far, application-focused studies have employed existing model-specific XAI methods, such as simple gradients, to analyze deep learning survival models — primarily to identify important nodes or generate saliency maps for singular images ____. However, to the best of our knowledge, no study has explicitly extended these methods to general time-dependent explainability approaches for survival NNs.