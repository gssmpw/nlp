%%%%%%%% ICML 2024 EXAMPLE LATEX SUBMISSION FILE %%%%%%%%%%%%%%%%%

\documentclass{article}

% Recommended, but optional, packages for figures and better typesetting:
\usepackage{microtype}
\usepackage{graphicx}
\usepackage{subfigure}
\usepackage{booktabs} % for professional tables
\usepackage{wrapfig}

% hyperref makes hyperlinks in the resulting PDF.
% If your build breaks (sometimes temporarily if a hyperlink spans a page)
% please comment out the following usepackage line and replace
% \usepackage{icml2024} with \usepackage[nohyperref]{icml2025} above.
\usepackage{hyperref}


% Attempt to make hyperref and algorithmic work together better:
\newcommand{\theHalgorithm}{\arabic{algorithm}}

% Use the following line for the initial blind version submitted for review:
% \usepackage{icml2025}

% If accepted, instead use the following line for the camera-ready submission:
\usepackage[accepted]{icml2025}

% For theorems and such
\usepackage{amsmath}
\usepackage{amssymb}
\usepackage{mathtools}
\usepackage{amsthm}
\usepackage{bm} % added by NK

% if you use cleveref..
\usepackage[capitalize,noabbrev]{cleveref}

%%%%%%%%%%%%%%%%%%%%%%%%%%%%%%%%
% THEOREMS
%%%%%%%%%%%%%%%%%%%%%%%%%%%%%%%%
\theoremstyle{plain}
\newtheorem{theorem}{Theorem}[section]
\newtheorem{proposition}[theorem]{Proposition}
\newtheorem{lemma}[theorem]{Lemma}
\newtheorem{corollary}[theorem]{Corollary}
\theoremstyle{definition}
\newtheorem{definition}[theorem]{Definition}
\newtheorem{assumption}[theorem]{Assumption}
\theoremstyle{remark}
\newtheorem{remark}[theorem]{Remark}

% Todonotes is useful during development; simply uncomment the next line
%    and comment out the line below the next line to turn off comments
%\usepackage[disable,textsize=tiny]{todonotes}
\usepackage[textsize=tiny]{todonotes}


% The \icmltitle you define below is probably too long as a header.
% Therefore, a short form for the running title is supplied here:
\icmltitlerunning{Gradient-based Explanations for Deep Learning Survival Models}

\begin{document}

\twocolumn[
\icmltitle{Gradient-based Explanations for Deep Learning Survival Models}

% It is OKAY to include author information, even for blind
% submissions: the style file will automatically remove it for you
% unless you've provided the [accepted] option to the icml2025
% package.

% List of affiliations: The first argument should be a (short)
% identifier you will use later to specify author affiliations
% Academic affiliations should list Department, University, City, Region, Country
% Industry affiliations should list Company, City, Region, Country

% You can specify symbols, otherwise they are numbered in order.
% Ideally, you should not use this facility. Affiliations will be numbered
% in order of appearance and this is the preferred way.
\icmlsetsymbol{equal}{*}

\begin{icmlauthorlist}
\icmlauthor{Sophie Hanna Langbein}{equal,bips,bre}
\icmlauthor{Niklas Koenen}{equal,bips,bre}
\icmlauthor{Marvin N. Wright}{bips,bre,cop}
\end{icmlauthorlist}

\icmlaffiliation{bips}{Leibniz Institute for Prevention Research and
Epidemiology – BIPS, Bremen, Germany}
\icmlaffiliation{bre}{Faculty of Mathematics and Computer Science,
University of Bremen, Germany}
\icmlaffiliation{cop}{Department of Public Health, University of
Copenhagen, Denmark}
%\icmlaffiliation{bips}{Department of XXX, University of YYY, Location, Country}
%\icmlaffiliation{bre}{Company Name, Location, Country}
%\icmlaffiliation{cop}{School of ZZZ, Institute of WWW, Location, Country}

\icmlcorrespondingauthor{Marvin N. Wright}{wright@leibniz-bips.de}

% You may provide any keywords that you
% find helpful for describing your paper; these are used to populate
% the "keywords" metadata in the PDF but will not be shown in the document
\icmlkeywords{Deep Learning, Survival Analysis, Explainabile Artificial Intelligence, Interpretable Machine Learning, XAI, IML, Feature Attribution}

\vskip 0.3in
]

% this must go after the closing bracket ] following \twocolumn[ ...

% This command actually creates the footnote in the first column
% listing the affiliations and the copyright notice.
% The command takes one argument, which is text to display at the start of the footnote.
% The \icmlEqualContribution command is standard text for equal contribution.
% Remove it (just {}) if you do not need this facility.

%\printAffiliationsAndNotice{}  % leave blank if no need to mention equal contribution
\printAffiliationsAndNotice{\icmlEqualContribution} % otherwise use the standard text.

\begin{abstract}
Deep learning survival models often outperform classical methods in time-to-event predictions, particularly in personalized medicine, but their "black box" nature hinders broader adoption. We propose a framework for gradient-based explanation methods tailored to survival neural networks, extending their use beyond regression and classification. We analyze the implications of their theoretical assumptions for time-dependent explanations in the survival setting and propose effective visualizations incorporating the temporal dimension. Experiments on synthetic data show that gradient-based methods capture the magnitude and direction of local and global feature effects, including time dependencies. We introduce GradSHAP(t), a gradient-based counterpart to SurvSHAP(t), which outperforms SurvSHAP(t) and SurvLIME in a computational speed vs. accuracy trade-off. Finally, we apply these methods to medical data with multi-modal inputs, revealing relevant tabular features and visual patterns, as well as their temporal dynamics.
\end{abstract}

\section{Introduction}

As medical databases expand to include patients' detailed medical history and genetic information, healthcare is shifting from population-based models and traditional statistical approaches targeting the "average" patient to more complex personalized medicine, which tailors diagnoses to individual patient characteristics. Survival analysis is fundamental to medical data analysis, modeling time-to-event outcomes while accounting for censoring, enabling personalized risk predictions, and assessing treatment effects to advance clinical research and evidence-based medicine. Deep learning methods hold significant potential for advancing survival analysis by framing pathogenic identification as a data-driven problem, uncovering correlations between patient profiles and disease phenotypes, and seamlessly learning from unstructured or high-dimensional data such as images, text, or omics, thereby revealing hidden and complex patterns undetectable by classical approaches \cite{zhang_personalized_2019}. While machine learning models show great promise for survival analysis, their inherent opacity raises legitimate concerns, as in fields like life sciences, interpretability is crucial to support sensitive decision-making, mitigate biases, promote equity, and ensure compliance with regulatory standards \cite{rahman2022fair,vellido2020importance}.
%
\begin{figure}[t]
    \centering
    \includegraphics[width=1\columnwidth]{figures/intro_overview.pdf} 
    \vskip 0.1in
    \caption{
        Overview of our workflow for generating time-dependent post-hoc explanations using gradient-based methods by the example of overall brain cancer survival prediction. The approach utilizes a survival deep learning model with multi-modal input data, providing insights into the temporal dynamics of feature effects through tailored visualizations for different feature types.
    }
    \label{fig:intro_overview}
    \vskip -0.1in
\end{figure}
%
In recent years, several post-hoc eXplainable AI (XAI) methods for population-wide (\emph{global}) and individual (\emph{local}) insights into the decision-making process of machine learning survival models have been proposed \cite{langbein2024}. For personalized medicine, the local model-agnostic approaches SurvLIME \cite{kovalev2020survlime}, an extension of LIME \cite{ribeiro2016should}, and SurvSHAP(t) \cite{krzyzinski2023survshap}, a generalization of SHapley Additive exPlanations (SHAP) \cite{lundberg2017unified}, have prevailed. However, no methods specifically targeted at or practical to survival deep learning models have been introduced.

In this paper, we introduce a formal framework generalizing gradient-based explanation methods to survival neural networks (NNs), extending their applicability to functional form outcomes beyond traditional regression and classification. Our contributions include the systematic assessment of the theoretical assumptions underlying a representative set of gradient-based explainability methods (e.g., Saliency, IntegratedGradient, GradSHAP and Gradient$\times$Input) and implications of the extension to time-dependent explanations for functional outcomes in the survival setting. Secondly, we propose effective visualizations that incorporate the temporal dimension and highlight the underlying differences of the XAI methods. Additionally, we introduce GradSHAP(t), a gradient-based, model-specific counterpart to SurvSHAP(t). A quantitative comparison confirms that the gradient-based approach outperforms the sample-based version and SurvLIME, offering a better balance between computational speed and local accuracy. By presenting a medical data example with multi-modal inputs, including histopathological images, we demonstrate that the methods can effectively identify prediction-relevant tabular features and their temporal effects, as well as recognize relevant visual patterns and their evolution over time. Our workflow of generating explanations for survival predictions, with overall brain cancer survival as an example is shown in Fig.~\ref{fig:intro_overview}. Finally, we provide an accessible implementation\footnote{\url{https://anonymous.4open.science/r/Survival_XAI-8A8B/README.md}} of gradient-based explanations in R utilizing \texttt{torch}, which is compatible with PyTorch models trained in \texttt{pycox} \cite{kvamme_coxtime_2019}. 

\section{Background}
\subsection{Survival Data}\label{sec:survival_data}
We are considering survival analysis as a supervised prediction task for the time-to-event distribution of a given dataset $\mathcal{D}$. We limit ourselves to the standard right-censored setting, in which the data consist of $n$ triplets, i.e., $\mathcal{D} =\left\{ (\bm{x}^{(i)},y^{(i)},\delta^{(i)})\right\}_{i=1}^{n}$. The first component, $\bm{x}^{(i)} = (x^{(i)}_1, x^{(i)}_2, \dots, x^{(i)}_p) \in \mathcal{X}$, represents the $p$-dimensional vector of predictive features for \emph{individual} $i$, which may include data types commonly used in classical supervised learning, such as images or tabular data. The observed right-censored time $y^{(i)}$ is defined as the minimum of the event time $t^{(i)} \in \mathbb{R}^{+}_{0}$ and the censoring time $c^{(i)} \in \mathbb{R}^{+}_{0}$, i.e., $y^{(i)} = \min(t^{(i)},c^{(i)})$. The binary event indicator $\delta^{(i)} \in \{0,1\}$ takes the value $0$ if the observation is censored ($t^{(i)} > c^{(i)}$) and $1$ if the event occurs ($t^{(i)} < c^{(i)}$). For clarity, we will omit the superscript for an instance $i$ in the remaining text when it is not necessary.

\subsection{Survival Distribution Representations}\label{sec:survival_distribution_representations}

Key quantities to be modeled and predicted in survival analysis are distributional representations of the random variable $T$. We generalize them as $f:\mathcal{X} \times\mathcal{T} \rightarrow \mathbb{R}$, a set of functions that map value combinations from the feature space $\mathcal{X}$ and the time space $\mathcal{T}$ to a one-dimensional outcome. The most popular representations are survival $S$, hazard $h$, and the cumulative hazard function $H$. 

\begin{definition}[Survival Function] 
The \emph{survival function} $S:\mathcal{X} \times \mathcal{T} \rightarrow [0,1]$ describes the probability of the time-to-event (survival time) being greater than or equal to a specific time point $t \geq 0$ conditional on the observed features $\bm{x} \in \mathcal{X}$ 
%
\begin{align}
    S(t|\bm{x}) \coloneq \mathbb{P}(T \geq t | \mathbf{x}) = 1 - \mathbb{P}(T \leq t | \bm{x}) \; \text{.}
\end{align}
%
\end{definition}

\begin{definition}[Hazard Function] 
The \emph{hazard} or \emph{risk function} $h:\mathcal{X} \times \mathcal{T} \rightarrow \mathbb{R}^{+}_{0}$ describes the instantaneous risk of occurrence of the event of interest in an infinitesimally small time interval $[t,t + \Delta t]$ for continuous $t \in \mathbb{R}^{+}_{0}$, given that it has not yet occurred before time $t$ and conditional on the observed features $\bm{x} \in \mathcal{X}$ 
%
\begin{align}
    h(t|\bm{x}) &\coloneq \lim_{\Delta t \rightarrow 0} \frac{\mathbb{P}(t \leq T \leq t + \Delta t | T \geq t, \bm{x})}{\Delta t} \\ &\hphantom{:}= -\frac{\text{d}}{\text{d}t} \ln S(t|\bm{x}) \; \text{.}
\end{align}
%
\end{definition}

\begin{definition}[Cumulative Hazard Function (CHF)] 
The \emph{cumulative hazard function}  $H:\mathcal{X} \times \mathcal{T} \rightarrow \mathbb{R}^{+}_{0}$ describes the accumulated risk of experiencing the event of interest up to a specific time $t \in \mathbb{R}^{+}_{0}$ conditional on the observed features $\bm{x} \in \mathcal{X}$ 
%
\begin{align}
    H(t|\bm{x}) \coloneq \int_{0}^{t} h(u| \bm{x}) du = -\log (S(t|\bm{x}))\; \text{.}
\end{align}
%
\end{definition}

\subsection{Related Work}
Deep learning survival models often extend the Cox regression framework by using NNs to parameterize the log-risk function, optimizing a Cox-based loss (negative log-partial likelihood of the Cox model) \cite{cox1972}. Examples include \emph{DeepSurv} \cite{katzman_deepsurv_2018}, which employs feedforward NNs to capture non-linear feature-hazard relationships while adhering to the Proportional Hazards (PH) assumption. It states that the hazard ratio between any two individuals remains constant over time, meaning the effect of features on the hazard function is multiplicative and does not vary with time. \emph{CoxTime} \cite{kvamme_coxtime_2019} incorporates time-dependent feature effects by including time as an additional feature. Another category of survival NNs adopts discrete-time methods, treating time as discrete to leverage classification techniques. The most prominent model in this category is \emph{DeepHit} \cite{lee_deephit_2018}, which directly models the joint distribution of survival times and event probabilities without assumptions about the stochastic process, allowing dynamic feature-risk relationships. Additional approaches include methods based on piecewise-exponential models, ordinary differential equations, and ranking techniques. For a comprehensive review, see \citet{wiegrebe2024deep}. To date, the interpretability of survival NNs has primarily been addressed through model-agnostic, post-hoc methods. Prominent local XAI techniques in this domain include SurvLIME and SurvSHAP(t). For a comprehensive review of interpretability methods in survival analysis, we refer to \citet{langbein2024}. So far, application-focused studies have employed existing model-specific XAI methods, such as simple gradients, to analyze deep learning survival models — primarily to identify important nodes or generate saliency maps for singular images \cite{mobadersany2018predicting,hao2019page,cho2023}. However, to the best of our knowledge, no study has explicitly extended these methods to general time-dependent explainability approaches for survival NNs. 


\section{Taxonomy of Deep Survival Models}
For the application of gradient-based feature attribution techniques, we categorize survival NNs based on two criteria: (1) supported \textbf{input feature modalities} and (2) their \textbf{prediction outcomes}.

\paragraph{Prediction outcome.} In survival NNs, the prediction outcome refers to the network's output, $f: \mathcal{X} \times \mathcal{T} \rightarrow \mathcal{Y}$. These outcomes correspond to the distributional representations discussed in Sec.~\ref{sec:survival_distribution_representations}. Differentiating based on prediction outcome is critical as it constitutes the quantities decomposed during attribution and capturing the effects of changes in input features.

\paragraph{Input feature modalities.} Differentiation by feature modality is essential, as it affects how attribution values are visualized. The input modalities a network can process depend on its architecture; while many survival DL models use feedforward NNs, other architectures such as convolutional networks, recurrent NNs, generative adversarial networks, and autoencoders are also employed \cite{wiegrebe2024deep}. Below, we discuss the relevant feature modalities and their coverage in this work.
%
(1) \textit{Time Dependence}: Time dependence can be incorporated in two ways: 1) time-varying effects of time-constant features (TVE) or 2) time-varying features (TVF). The PH assumption simplifies the feature effect on the hazard scale to a one-dimensional setting $f: \mathcal{X} \rightarrow \mathcal{Y}$, similar to standard regression or classification. However, attribution values are generally not time-constant, even in PH models, when evaluated on survival or cumulative hazard scales, necessitating time-dependent feature attribution computation and visualization. TVF in time-to-event prediction are analogous to longitudinal input data and require specialized architectures, such as recurrent NNs, along with tailored interpretability methods \cite{ferreira2021predictive}.(2) \textit{Data Modalities}: Mixed tabular data is one of the most common input types for survival analysis and many survival deep learning methods have been (first) developed for it \citep{katzman_deepsurv_2018,kvamme_coxtime_2019,lee_deephit_2018}. High-dimensional (multi-)omics data are another popular input type for survival NNs, with many specialized NNs developed for this purpose \cite{ching2018cox,hao2018cox,hao2019page}. Conceptually, feature attribution techniques apply similarly to high- and low-dimensional inputs, as feature contributions are considered individually. Some omics-specific NNs assign biological meaning to network nodes, enabling feature attributions to quantify the impact of biologically significant quantities. Another key data modality for survival NNs is (often medical) image data, typically processed using convolutional NN architectures \cite{zhu2016deep,mobadersany2018predicting,tang2019capsurv}. Since many gradient-based feature attribution methods were originally designed for image data, their adaptation to survival models is straightforward for a single outcome time point, where traditional saliency maps \cite{simonyan_deep_2014} can be generated. Saliency across multiple time points can be represented using different maps, colors or visual markers in a single map. Other input formats, such as text data, are less frequently used for time-to-event predictions and are not covered in this work. In our experiments we consider DeepSurv, CoxTime and DeepHit as a representational set of models. 

\section{Gradient-based Survival Explanations}\label{sec:methodology_grad}

Gradient-based feature attribution methods are a set of local model-specific XAI techniques that assign relevance scores to input features based on their contribution to the NN's prediction. These methods efficiently leverage the automatic differentiation capabilities inherent in modern deep learning libraries \cite{chollet2015keras,pytorch} to quantify how each input feature influences the model's output \cite{ancona_gradient_based_2019,koenen2023interpreting}. 
In their original formulation, these methods are defined for scalar outputs resulting in an attribution value $R_j$ for each feature $j$. However, in the survival context, the outcome is represented as a prediction vector for different time points, which complicates the computation and adds an additional dimension to the explanations. Instead of being applied to a single $f(\bm{x}) \in \mathbb{R}$, a survival XAI method is applied at each discretized time point $f(t_0 \mid \bm{x}), \ldots, f(t_T \mid \bm{x})$, resulting in an attribution value $R_j^{t_k}$ for each feature $j$ and time point $t_k$, thus, an ensemble of explanations including their temporal interplay. In the following, we extend the most common gradient-based feature attribution methods to survival networks and thereby propose their time-dependent counterparts Grad(t), SG(t), G$\times$I(t), IntGrad(t) and GradSHAP(t); see Fig.~\ref{fig:fa_summary} for an overview of the proposed methods.  


\begin{figure*}[ht]
\vskip 0.2in
\begin{center}
\resizebox{0.9\linewidth}{!}{
    \section{Overview}

\revision{In this section, we first explain the foundational concept of Hausdorff distance-based penetration depth algorithms, which are essential for understanding our method (Sec.~\ref{sec:preliminary}).
We then provide a brief overview of our proposed RT-based penetration depth algorithm (Sec.~\ref{subsec:algo_overview}).}



\section{Preliminaries }
\label{sec:Preliminaries}

% Before we introduce our method, we first overview the important basics of 3D dynamic human modeling with Gaussian splatting. Then, we discuss the diffusion-based 3d generation techniques, and how they can be applied to human modeling.
% \ZY{I stopp here. TBC.}
% \subsection{Dynamic human modeling with Gaussian splatting}
\subsection{3D Gaussian Splatting}
3D Gaussian splatting~\cite{kerbl3Dgaussians} is an explicit scene representation that allows high-quality real-time rendering. The given scene is represented by a set of static 3D Gaussians, which are parameterized as follows: Gaussian center $x\in {\mathbb{R}^3}$, color $c\in {\mathbb{R}^3}$, opacity $\alpha\in {\mathbb{R}}$, spatial rotation in the form of quaternion $q\in {\mathbb{R}^4}$, and scaling factor $s\in {\mathbb{R}^3}$. Given these properties, the rendering process is represented as:
\begin{equation}
  I = Splatting(x, c, s, \alpha, q, r),
  \label{eq:splattingGA}
\end{equation}
where $I$ is the rendered image, $r$ is a set of query rays crossing the scene, and $Splatting(\cdot)$ is a differentiable rendering process. We refer readers to Kerbl et al.'s paper~\cite{kerbl3Dgaussians} for the details of Gaussian splatting. 



% \ZY{I would suggest move this part to the method part.}
% GaissianAvatar is a dynamic human generation model based on Gaussian splitting. Given a sequence of RGB images, this method utilizes fitted SMPLs and sampled points on its surface to obtain a pose-dependent feature map by a pose encoder. The pose-dependent features and a geometry feature are fed in a Gaussian decoder, which is employed to establish a functional mapping from the underlying geometry of the human form to diverse attributes of 3D Gaussians on the canonical surfaces. The parameter prediction process is articulated as follows:
% \begin{equation}
%   (\Delta x,c,s)=G_{\theta}(S+P),
%   \label{eq:gaussiandecoder}
% \end{equation}
%  where $G_{\theta}$ represents the Gaussian decoder, and $(S+P)$ is the multiplication of geometry feature S and pose feature P. Instead of optimizing all attributes of Gaussian, this decoder predicts 3D positional offset $\Delta{x} \in {\mathbb{R}^3}$, color $c\in\mathbb{R}^3$, and 3D scaling factor $ s\in\mathbb{R}^3$. To enhance geometry reconstruction accuracy, the opacity $\alpha$ and 3D rotation $q$ are set to fixed values of $1$ and $(1,0,0,0)$ respectively.
 
%  To render the canonical avatar in observation space, we seamlessly combine the Linear Blend Skinning function with the Gaussian Splatting~\cite{kerbl3Dgaussians} rendering process: 
% \begin{equation}
%   I_{\theta}=Splatting(x_o,Q,d),
%   \label{eq:splatting}
% \end{equation}
% \begin{equation}
%   x_o = T_{lbs}(x_c,p,w),
%   \label{eq:LBS}
% \end{equation}
% where $I_{\theta}$ represents the final rendered image, and the canonical Gaussian position $x_c$ is the sum of the initial position $x$ and the predicted offset $\Delta x$. The LBS function $T_{lbs}$ applies the SMPL skeleton pose $p$ and blending weights $w$ to deform $x_c$ into observation space as $x_o$. $Q$ denotes the remaining attributes of the Gaussians. With the rendering process, they can now reposition these canonical 3D Gaussians into the observation space.



\subsection{Score Distillation Sampling}
Score Distillation Sampling (SDS)~\cite{poole2022dreamfusion} builds a bridge between diffusion models and 3D representations. In SDS, the noised input is denoised in one time-step, and the difference between added noise and predicted noise is considered SDS loss, expressed as:

% \begin{equation}
%   \mathcal{L}_{SDS}(I_{\Phi}) \triangleq E_{t,\epsilon}[w(t)(\epsilon_{\phi}(z_t,y,t)-\epsilon)\frac{\partial I_{\Phi}}{\partial\Phi}],
%   \label{eq:SDSObserv}
% \end{equation}
\begin{equation}
    \mathcal{L}_{\text{SDS}}(I_{\Phi}) \triangleq \mathbb{E}_{t,\epsilon} \left[ w(t) \left( \epsilon_{\phi}(z_t, y, t) - \epsilon \right) \frac{\partial I_{\Phi}}{\partial \Phi} \right],
  \label{eq:SDSObservGA}
\end{equation}
where the input $I_{\Phi}$ represents a rendered image from a 3D representation, such as 3D Gaussians, with optimizable parameters $\Phi$. $\epsilon_{\phi}$ corresponds to the predicted noise of diffusion networks, which is produced by incorporating the noise image $z_t$ as input and conditioning it with a text or image $y$ at timestep $t$. The noise image $z_t$ is derived by introducing noise $\epsilon$ into $I_{\Phi}$ at timestep $t$. The loss is weighted by the diffusion scheduler $w(t)$. 
% \vspace{-3mm}

\subsection{Overview of the RTPD Algorithm}\label{subsec:algo_overview}
Fig.~\ref{fig:Overview} presents an overview of our RTPD algorithm.
It is grounded in the Hausdorff distance-based penetration depth calculation method (Sec.~\ref{sec:preliminary}).
%, similar to that of Tang et al.~\shortcite{SIG09HIST}.
The process consists of two primary phases: penetration surface extraction and Hausdorff distance calculation.
We leverage the RTX platform's capabilities to accelerate both of these steps.

\begin{figure*}[t]
    \centering
    \includegraphics[width=0.8\textwidth]{Image/overview.pdf}
    \caption{The overview of RT-based penetration depth calculation algorithm overview}
    \label{fig:Overview}
\end{figure*}

The penetration surface extraction phase focuses on identifying the overlapped region between two objects.
\revision{The penetration surface is defined as a set of polygons from one object, where at least one of its vertices lies within the other object. 
Note that in our work, we focus on triangles rather than general polygons, as they are processed most efficiently on the RTX platform.}
To facilitate this extraction, we introduce a ray-tracing-based \revision{Point-in-Polyhedron} test (RT-PIP), significantly accelerated through the use of RT cores (Sec.~\ref{sec:RT-PIP}).
This test capitalizes on the ray-surface intersection capabilities of the RTX platform.
%
Initially, a Geometry Acceleration Structure (GAS) is generated for each object, as required by the RTX platform.
The RT-PIP module takes the GAS of one object (e.g., $GAS_{A}$) and the point set of the other object (e.g., $P_{B}$).
It outputs a set of points (e.g., $P_{\partial B}$) representing the penetration region, indicating their location inside the opposing object.
Subsequently, a penetration surface (e.g., $\partial B$) is constructed using this point set (e.g., $P_{\partial B}$) (Sec.~\ref{subsec:surfaceGen}).
%
The generated penetration surfaces (e.g., $\partial A$ and $\partial B$) are then forwarded to the next step. 

The Hausdorff distance calculation phase utilizes the ray-surface intersection test of the RTX platform (Sec.~\ref{sec:RT-Hausdorff}) to compute the Hausdorff distance between two objects.
We introduce a novel Ray-Tracing-based Hausdorff DISTance algorithm, RT-HDIST.
It begins by generating GAS for the two penetration surfaces, $P_{\partial A}$ and $P_{\partial B}$, derived from the preceding step.
RT-HDIST processes the GAS of a penetration surface (e.g., $GAS_{\partial A}$) alongside the point set of the other penetration surface (e.g., $P_{\partial B}$) to compute the penetration depth between them.
The algorithm operates bidirectionally, considering both directions ($\partial A \to \partial B$ and $\partial B \to \partial A$).
The final penetration depth between the two objects, A and B, is determined by selecting the larger value from these two directional computations.

%In the Hausdorff distance calculation step, we compute the Hausdorff distance between given two objects using a ray-surface-intersection test. (Sec.~\ref{sec:RT-Hausdorff}) Initially, we construct the GAS for both $\partial A$ and $\partial B$ to utilize the RT-core effectively. The RT-based Hausdorff distance algorithms then determine the Hausdorff distance by processing the GAS of one object (e.g. $GAS_{\partial A}$) and set of the vertices of the other (e.g. $P_{\partial B}$). Following the Hausdorff distance definition (Eq.~\ref{equation:hausdorff_definition}), we compute the Hausdorff distance to both directions ($\partial A \to \partial B$) and ($\partial B \to \partial A$). As a result, the bigger one is the final Hausdorff distance, and also it is the penetration depth between input object $A$ and $B$.


%the proposed RT-based penetration depth calculation pipeline.
%Our proposed methods adopt Tang's Hausdorff-based penetration depth methods~\cite{SIG09HIST}. The pipeline is divided into the penetration surface extraction step and the Hausdorff distance calculation between the penetration surface steps. However, since Tang's approach is not suitable for the RT platform in detail, we modified and applied it with appropriate methods.

%The penetration surface extraction step is extracting overlapped surfaces on other objects. To utilize the RT core, we use the ray-intersection-based PIP(Point-In-Polygon) algorithms instead of collision detection between two objects which Tang et al.~\cite{SIG09HIST} used. (Sec.~\ref{sec:RT-PIP})
%RT core-based PIP test uses a ray-surface intersection test. For purpose this, we generate the GAS(Geometry Acceleration Structure) for each object. RT core-based PIP test takes the GAS of one object (e.g. $GAS_{A}$) and a set of vertex of another one (e.g. $P_{B}$). Then this computes the penetrated vertex set of another one (e.g. $P_{\partial B}$). To calculate the Hausdorff distance, these vertex sets change to objects constructed by penetrated surface (e.g. $\partial B$). Finally, the two generated overlapped surface objects $\partial A$ and $\partial B$ are used in the Hausdorff distance calculation step.
}
\caption{Mathematical representations of gradient-based feature attribution methods adapted to survival NNs. Each block corresponds to a different underlying objective. For example, in the case of feature-wise relevances $R_{j}^t$ obtained from G$\times$I(t), the goal is to achieve a sum that equals $f(t|\bm{x})$, i.e., $\sum_{j=1}^p R_{j}^t = f(t|\bm{x})$.}
\label{fig:fa_summary}
\end{center}
\vskip -0.2in
\end{figure*}

\subsection{Output-Sensitivity Methods}

Although they do not represent attributions in the classical sense, output-sensitive methods are often categorized as feature attribution as well. As pointed out in \citet{koenen2024toward}, methods in this category do not produce actual attributions but rather local importance tendencies.

\paragraph{Grad(t).} The Gradient method, also known as Vanilla Gradient or Saliency Maps in the image domain, developed by \citet{simonyan_deep_2014}, is one of the earliest and most intuitive attribution methods in deep learning. For Grad(t) we compute relevance scores as the (absolute) partial derivatives of the target outcome $f(t|\bm{x})$ with respect to the corresponding input feature $\bm{x}_j$ at a particular time point $t$, as shown in Fig.~\ref{fig:fa_summary}. This captures how sensitive the prediction is to changes in each feature at $t$ and, over all time points, how the relevance evolves over time. 

\paragraph{SG(t).} \citet{smilkov_smoothgrad_2017} propose smoothed gradients (SmoothGrad) as an extension to reduce noise in the raw gradients. Equivalently, in SG(t), the expected values of the gradients are computed over random Gaussian perturbations of the input, $\varepsilon \sim \mathcal{N}(0,\sigma^2)$ (see Fig.~\ref{fig:fa_summary}). 
In practice, the expected value of the gradients is estimated as an average over $K$ samples, so that the accuracy of the estimation can be improved by larger values of $K$. The Gaussian standard deviation $\sigma$ controls the sharpness of the explanation and is mostly indirectly specified through a noise level $\sigma = \frac{\sigma}{x_{\max} - x_{\min}}$, determining the proportion of the total range of the input domain that is covered by the standard deviation $\sigma$. Note, that in the survival setting, we can also perturb inputs over different time points (e.g., for CoxTime) to capture the temporal dynamics of feature effects.

\subsection{Attribution-based Methods}

Feature attribution methods typically aim to approximate a decomposition of a model's prediction-based quantity into feature-wise additive contributions \cite{ancona_gradient_based_2019, shrikumar_not_2017, sundararajan_axiomatic_2017}. This quantity depends on the method and is often the model's output or a baseline-adjusted version. This property is commonly referred to as \emph{local accuracy}. 

\paragraph{G$\times$I(t).} A simple and computationally efficient approach for decomposing a prediction is to multiply the gradient by the corresponding feature value, as proposed by \citet{shrikumar_not_2017} known as Gradient$\times$Input. Our survival adoption is denoted as G$\times$I(t) in Fig.~\ref{fig:fa_summary}. Mathematically, this method is based on a first-order Taylor expansion at the implicitly set reference value zero, effectively providing a linear approximation of the model's output \cite{ancona_gradient_based_2019, montavon_explaining_2017}. However, since survival functions are inherently highly nonlinear, this approach often leads to limitations in the accuracy of the decomposition.

\paragraph{IntGrad(t).} The Integrated Gradients (IntGrad) method by \citet{sundararajan_axiomatic_2017} attributes the contribution of each feature by comparing a model’s prediction for a given input $\bm{x}$ to that of a baseline instance $\bm{\tilde{x}}$. This method satisfies several desirable properties, including completeness, sensitivity, linearity, and implementation invariance, making it a widely accepted approach for feature attribution. IntGrad computes contributions by integrating gradients along a path from the baseline to the input, typically following a straight line. In practice, this integral is approximated by averaging gradients at discrete intervals. The generalized IntGrad(t) results in a decomposition of the targeted curve of $\bm{x}$ minus the baseline curve of $\bm{\tilde{x}}$, e.g., for the survival curve, $\hat{S}(t\mid \bm{x}) - \hat{S}(t\mid \bm{\tilde{x}})$ for each time point $t$. Typical baseline values are zeros or the feature mean representing the "average patient". However, especially with non-linear or multi-modal distributions, careful selection of the reference value is required. It can be shown that for a nonnegatively homogeneous model and $\tilde{\bm{x}} = \bm{0}$, IG is equivalent to Grad$\times$Input \cite{hesse_fast_2021}.

\paragraph{GradSHAP(t).} In many cases, it is most meaningful for the baseline value $\tilde{\bm{x}}$ to conceptually reflect the complete absence of the features. However, for tabular data, choosing an adequate baseline can be challenging since a zero value does not necessarily coincide with feature absence. In survival analysis, zero as a baseline can misrepresent missingness because it often carries specific clinical meaning (e.g., a zero lab value might imply a specific medical condition), leading to biased explanations if not carefully chosen. Our time-dependent extension, GradSHAP(t), of the GradSHAP method \cite{lundberg2017unified, erion_improving_2021} addresses this by taking the expectation of IntGrad(t) explanations evaluated at randomly drawn reference values from $\mathcal{D}$ instead of using a single potentially off-manifold baseline value (see Fig.~\ref{fig:fa_summary}). In survival analysis, using the expectation over a reference distribution can provide a generalized baseline, reflecting the "mean patient's survival time" and offering more stable and less biased comparisons. 
In practice, the expectation is estimated by Monte Carlo integration using the sample average of randomly drawn baseline values from $\mathcal{D}$ and parameterized points $\alpha \sim \mathcal{U}(0,1)$ for the integration path. This results in a decomposition of $f(t\mid \bm{x}) - \mathbb{E}_{\bm{\tilde{x}}}[f(t\mid \bm{\tilde{x}})]$ and describes a gradient-based approximation of SHAP values at selected time points.

\section{Experiments}

Unlike for the evaluation of ML models, there is no well-established framework for XAI evaluation due to the absence of a definitive ground truth for explanations and the reliance on an imprecise black-box model trained with imperfect data \cite{liu2021synthetic,vilone2021notions,antoniadi2021current}. To ensure a comprehensive evaluation strategy, we conduct experiments on simulated and real data to answer the following research questions: (1) Can gradient-based methods correctly identify time-(in)dependent effects in different survival NNs? (2) How does GradSHAP(t) compare to SurvSHAP(t) and SurvLIME in terms of local accuracy, computational speed and global feature rankings? (3) How can gradient-based methods be leveraged to explain survival predictions based on multi-modal input data?

\subsection{Experiments on Simulated Data}

\subsubsection{Time-independent Effects}\label{sec:time_id_effects}
\paragraph{Setup.} In the first experiment, we generate synthetic data to demonstrate that gradient-based explanations can accurately capture time-independent feature effects, provided the models correctly identify them. The data consist of $N = 10,000$ observations simulated from a standard Cox PH model. The baseline hazard function is monotonically increasing and modeled using a Weibull distribution with a shape parameter $\gamma$ of 2.5. The features include one "harmful" feature $x_1$ with a log hazard ratio of 1.7, one "protective" feature $x_2$ with a log hazard ratio of -2.4, and one feature with no effect on the hazard $x_3$. The maximum follow-up period is set to $t=7$. More details on the simulation setting, training process, fitted models and selected observations can be found in Appendix~\ref{app:time-independent-effects}. 

\paragraph{Results.} These first experiments aim to show how different gradient-based explanation methods identify the effects of time-independent features. For this purpose, we split the data into training ($9,500$ observations) and test set ($500$ observations) and fit a DeepSurv \cite{katzman_deepsurv_2018}, a CoxTime \cite{kvamme_coxtime_2019}, and a DeepHit \cite{lee_deephit_2018} model to the training set. 

Grad(t) (Fig.~\ref{fig:grad_plot_tid}) and SG(t) (Fig.~\ref{fig:smoothgrad_plot_tid}) are output-sensitive methods; as such they indicate the models' sensitivity to feature changes rather than appropriately capturing local effects on the survival prediction. Therefore, the global ground-truth effects are accurately reconstructed, with $x_1$ having a negative, $x_2$ a stronger positive effect on survival, and $x_3$ no substantial effect on survival over time across all models for both of the randomly chosen instances. In contrast, the attribution curves in  Figures~\ref{fig:gradin_plot_tid}, \ref{fig:smoothgradin_plot_tid} - \ref{fig:gshap_plot_tid} capture feature-wise local effects. G$\times$I(t) uses simple scaling of the sensitivity to account for the magnitude of the feature’s contribution to the prediction. Thus, despite a negative global ground truth effect of $x_2$, since $x_2 < 0$ for the 13th observation, its attribution curve only takes positive values as highlighted in Fig.~\ref{fig:grad_gradin_tid}. By multiplying the gradient by the input value, the method implicitly assumes that the relationship between the input and the model’s output is locally linear near zero, which can produce misleading interpretations of feature contributions due to the inherent nonlinearity of survival prediction curves. The cubic shape of the survival prediction curves leads to approximately parabolic relevance curves when considering time-independent feature effects, primarily because the survival probability tends to exhibit fewer changes at the extreme ends of time. This behavior is particularly pronounced in Cox-based NNs due to the shape assumptions inherent in their design. Since the survival curves in the DeepSurv model are constrained to be proportional, the resulting relevance curves also exhibit approximate proportionality. Consequently, in time-independent scenarios, CoxTime and DeepSurv yield similar results, indicating that CoxTime identifies the time-independence of the features.    

\begin{figure}[ht]
    \centering
    \includegraphics[width=0.9\columnwidth]{figures/time-independent/sim_tid_grad_gradin_plot.pdf} 
    \vskip 0.1in
    \caption{
        Grad(t) (top) and G$\times$I(t) (bottom) relevance curves for selected observations using the DeepSurv model trained on the time-independent simulation dataset. The relevance values for each feature are represented by different colors (y-axis) and are plotted across time (x-axis), highlighting the temporal dynamics of feature contributions.  
    }
    \label{fig:grad_gradin_tid}
    \vskip 0.1in
\end{figure}


\subsubsection{Time-dependent Effects}

\paragraph{Setup.} In this experiment, we generate synthetic data to demonstrate that gradient-based explanations can accurately capture variables with time-dependent effects, provided the models correctly identify them. The dataset consists of $N=10,000$ observations simulated from a Weibull model analogous to Sec.~\ref{sec:time_id_effects} with $\gamma =1.5$. The features include one time-dependent feature $x_1$ with a "harmful" effect for $t < 2$ and a "protective" effect for $t > 2$, as well as two time-independent features: one with a "harmful" effect ($x_2$) and one with a stronger "protective" effect ($x_3$). Additionally, there is one feature with no effect on the hazard ($x_4$). The maximum follow-up period is set to $t = 7$. Further details on the simulation settings, training process and fitted models are provided in Appendix~\ref{app:time-dependent-effects}. 

\paragraph{Results.} These experiments are designed to demonstrate how the different gradient-based explanation methods capture the effects of time-dependent features. The relevance curves derived from output-sensitive methods (Figures~\ref{fig:grad_plot_td}, \ref{fig:smoothgrad_plot_td}) effectively reveal the time-dependent effect of $x_1$ on the survival predictions, by indicating a positive effect at earlier times and a negative effect later on. This time-dependent effect is accurately captured by CoxTime and DeepHit (as illustrated in Fig.~\ref{fig:grad_deephit_td}), but not by DeepSurv, which is inherently constrained by the PH assumption and thus unable to model time-dependence. These results underscore the ability of gradient-based methods to uncover such differences between models, focusing on explaining model behavior rather than data, and are thus valuable for assessing whether time-dependent variables are correctly modeled. 

\begin{figure}[ht]
    \centering
    \includegraphics[width=0.9\columnwidth]{figures/time-dependent/main_grad_deephit_td.pdf} 
    \vskip 0.1in
    \caption{Grad(t) relevance curves for the selected observations and the DeepHit model trained on the time-dependent simulation dataset.}
    \label{fig:grad_deephit_td}
    \vskip 0.1in
\end{figure}

In addition to time-dependence in feature effects, difference-to-reference methods (i.e., IntGrad(t) and GradSHAP(t)) provide insights into the relative scale, direction, and magnitude of feature effects by comparing predictions to a meaningful reference, as displayed in Figures~\ref{fig:intgrad0_plot_comp_td}, \ref{fig:intgradmean_plot_comp_td} and \ref{fig:gshap_plot_comp_td}. \emph{Contribution plots} (Figures~\ref{fig:intgrad0_plot_contr_td}, \ref{fig:intgradmean_plot_contr_td} and \ref{fig:gshap_plot_td_contr}) effectively visualize the normalized absolute contribution of each feature to the difference between reference and (survival) prediction over time, as shown for the GradSHAP(t) method and the CoxTime model for the two selected observation in Fig.~\ref{fig:gshap_coxtime_td}. Complementarily, \emph{force plots} (Figures~\ref{fig:intgrad0_plot_force_td}, \ref{fig:intgradmean_plot_force_td} and \ref{fig:gshap_plot_force_td}) emphasize the relative contribution and direction of each feature at a set of representative survival times, likewise exemplarily highlighted in  Fig.~\ref{fig:gshap_coxtime_td}. For example, the opposite effects of low vs. high values of $x_1$ are effectively captured in the plots. In observation 79, a low $x_1$ positively influences survival at later time points ($t>2$) compared to the overall average survival in the dataset, resulting in its largest contributions occurring at these times. Conversely, in observation 428, a high $x_1$ induces substantial contributions at earlier time points ($t<2$), but negatively impacts survival at later times, reflecting its early event as a consequence of the high $x_1$ and the strong negative effect of $x_3$. The average normalized absolute contribution, displayed on the right side of the contribution plots, offers a time-independent measure of feature importance, confirming the dominance of $x_3$ for the survival prediction of instance 428. Additionally, the visualizations suggest that CoxTime partially attributes the time-varying effect of $x_1$ to the other features, as the model, being non-parametric and lacking explicit knowledge of the time-dependent functional form, struggles to precisely disentangle and localize this effect.

\begin{figure}[ht]
    \centering
    \includegraphics[width=0.9\columnwidth]{figures/time-dependent/main_gshap_coxtime_td.pdf} 
    \vskip 0.1in
    \caption{GradSHAP(t) relevance curves, with corresponding survival prediction curves, reference curve and their difference (top), contribution plots (middle) and force plots (bottom) for the selected observations and the CoxTime model trained on the time-dependent simulation dataset.}
    \label{fig:gshap_coxtime_td}
    \vskip 0.1in
\end{figure}

\subsection{GradSHAP(t) vs SurvSHAP(t)}

One of the most established XAI methods are SHAP values, which -- rooted in game theory -- offer intuitive interpretations by measuring the "gain" of each feature to a prediction \cite{chen2023algorithms}. In the survival context, the only existing estimation approach is the model-agnostic SurvSHAP(t) method. This sample-based strategy becomes computationally inefficient for high-dimensional feature spaces or deep NNs. Our proposed extension, GradSHAP(t), provides a model-specific counterpart, leveraging gradients for more efficient and scalable attribution. In the following, we compare both methods in terms of accuracy, runtime, and their ability to correctly estimate global feature rankings, particularly in comparison to SurvLIME. Further details of the simulations and comparable results for the other not-shown model classes can be found in Appendix~\ref{app:survgradshap}.

% Local Accuracy
\paragraph{Local Accuracy.}
To evaluate the accuracy of the method, we use the time-dependent adoption of the local accuracy measure proposed in \citet{krzyzinski2023survshap}. This metric measures the decomposition error and normalizes it against the mean prediction at each time point. The simulation setup follows the same structure as described in Sec.~\ref{sec:time_id_effects}, with $p = 30$ features and $100$ test instances. The features have a uniformly increasing effect strength from 0 to 1 on the log-hazard, with alternating signs.
The results for a DeepSurv model are shown in Fig.~\ref{fig:local_acc}. In our simulations, only a subsample of the $100$ test data instances is used for each calculation, and the number of interval samples (indicated in parentheses) for estimating the integral in GradSHAP(t) is varied. Our GradSHAP(t) explanations provide highly accurate approximations, even with a limited number of integration samples. Upon user demand, accuracy can be further improved by increasing the number of integration samples, albeit at the cost of a longer computation time. 

\begin{figure}
    \centering
    \includegraphics[width=0.98\linewidth]{figures/gradshap/gradshapt_fig.pdf}
    \caption{Runtime (left) and local accuracy (right) comparison of SurvSHAP(t) and GradSHAP(t) with varying numbers of integration samples (5, 25, 50) showing the trade-off between accuracy and efficiency.}
    \label{fig:local_acc}
\end{figure}

 % Runtime
 \paragraph{Runtime.}
A key advantages of GradSHAP(t) is its superior scalability in higher dimensions, making it especially valuable for deep NNs. Fig.~\ref{fig:local_acc} illustrates the runtime of SurvSHAP(t) and GradSHAP(t) (with 5, 25, and 30 integration samples) as a function of the input dimension $p$ for DeepHit and DeepSurv. The plot clearly demonstrates that the gradient-based method is significantly faster and maintains good scalability even for larger $p$. However, it also highlights the trade-off between accuracy and runtime: increasing the number of integration points results in longer computation times. This is mainly due to the computation of the gradients of a temporary instance for each sample and each integration point, leading to a computational complexity of $\mathcal{O}(n \cdot n_\text{samples} \cdot n_\text{int})$.


\paragraph{Global Importance Ranking.} 

Beyond patient-wise local effects, it is important to assess whether features consistently rank as influential on a global level. Stable importance rankings indicate that the model captures robust patterns rather than instance-specific artifacts. For this simulation, we use $p = 5$ features, each having an evenly increasing effect on the log-hazard function with alternating signs. Additionally, we include the importance ranking of SurvLIME weights as a competing global XAI method. SurvLIME is an extension of the LIME framework adapted for survival models, fitting local surrogates. As shown in Fig.~\ref{fig:global}, the feature importance rankings of SurvSHAP(t) and GradSHAP(t) are nearly identical and agree with the data-generating process. The discrepancy to SurvLIME is consistent with observations reported in previous studies \cite{krzyzinski2023survshap}.

\begin{figure}
    \centering
    \includegraphics[width=0.98\linewidth]{figures/gradshap/gradshapt_ranking_fig.pdf}
    \caption{Global importance rankings for GradSHAP(t), SurvSHAP(t), and SurvLIME across $300$ test samples. GradSHAP(t) and SurvSHAP(t) show consistent rankings aligned with the data-generating process.}
    \label{fig:global}
\end{figure}


\subsection{Example on Real Multi-modal Medical Data} 

Despite the critical role of time-to-event prediction in medical decision-making, deep learning remains underutilized in this domain, even with its success in other medical applications. Our primary motivation for developing gradient-based explanation methods for survival deep learning is to help researchers harness the unique abilities of deep learning to extract and integrate complex features from high-dimensional, unstructured data while ensuring interpretability, transparency in individual-level decision-making, and facilitating new domain knowledge discovery. 

In order to show a use case, we apply the methods to a CNN-based extension of a DeepHit model trained on a real-world multi-modal medical dataset predicting overall survival in diffuse gliomas \cite{mobadersany2018predicting}. We use four tabular features selected based on previous results (age, sex, absence or presence of IDH mutation and 1p/19q codeletion) and histologic images of the regions of interest of whole-slide image tissue sections from formalin-fixed, paraffin-embedded specimens from The Cancer Genome Atlas (TCGA) Lower-Grade Glioma (LGG) and Glioblastoma (GBM) projects. The molecular features are known to be helpful for predicting survival in gliomas and other brain tumors. Based on the WHO's histologic classification of gliomas \cite{who_classification}, the isocitrate dehydrogenase mutation (IDH) involves alterations in the IDH1 or IDH2 genes and are associated with a more favorable prognosis. The 1p/19q codeletion refers to the simultaneous loss of parts of chromosomes 1 and 19. The absence of a 1p/19q codeletion is associated with more invasive and treatment-resistant gliomas, leading to a worse survival prognosis. For the image data of resized shape 226x226, we use a standard ResNet34 architecture \cite{he_resnet}. The high-level representations extracted from the ResNet are flattened to 256 features and fused with the tabular data. This combined representation is then passed into a final dense network, which serves as the base model for a multi-modal DeepHit architecture. This multi-modal model is trained on a total of $1,239$ training samples and evaluated on $266$ test samples, achieving a C-index of $0.713$ and an integrated Brier score of $0.092$. We use the standard DeepHit loss function with an $\alpha$-value of $0.5$ to balance the rank loss and log-likelihood loss equally.

Fig.~\ref{fig:realdata} shows the force plot of GradSHAP(t) explanations for a 43-year-old male patient with an IDH mutation but no 1p/19q codeletion. The black solid line represents the difference between the patient's survival prediction and the dataset-wide average. Initially, the line remains above the x-axis, suggesting a survival advantage in the early months - likely due to the protective effect of the IDH mutation, which contributes positively at all time points. However, the curve eventually shifts below the x-axis, indicating a survival disadvantage over time, potentially driven by the absence of the 1p/19q codeletion and the diminishing effect of the IDH mutation at later time points. The aggregated effect of the image modality consistently indicates a negative impact on survival probability. This is further evidenced in the image explanations (bottom), predominantly highlighting cells with a negative contribution.


\begin{figure}
    \centering
    \includegraphics[width=0.95\linewidth]{figures/real-data/real_data_grad_shap_force.pdf}
    \includegraphics[width=0.12\linewidth]{figures/real-data/gshap_2.pdf}%
    \includegraphics[width=0.86\linewidth]{figures/real-data/real_data_grad_shap_img_exp.pdf}
    \caption{GradSHAP(t) explanation for a multi-modal DeepHit model, showing temporal contributions of tabular genetic and clinical features (top). Below, corresponding image patches over time are visualized, with the original histology image shown at the bottom left.}
    \label{fig:realdata}
\end{figure}

\section{Conclusion}
In this work, we introduce a set of novel model-specific local XAI method for survival deep learning. The methods are the first to provide time-dependent explanations for multi-modal data, including images, in the survival setting, but is likewise easily generalizable to any functional outputs. Our work equips practitioners with a toolkit to derive meaningful insights from fitted survival NNs while accounting for underlying model assumptions grounded in survival analysis theory and the interpretability offered by different gradient-based techniques. This promotes transparency, accountability, and fairness in sensitive applications such as clinical decision-making, the development of targeted therapies, medical interventions, and other healthcare contexts. However, it is important to note that these explanations do not imply causal relationships, as the models lack knowledge of the true causal structure of the data-generating process. Future work will extend these methods beyond right-censored survival to include competing risks, multi-state models, and recurrent events. We also aim to develop targeted XAI methods for deep learning that can detect both individual feature effects and interactions.


\section*{Acknowledgements}
The results shown here are in part based upon data generated by the TCGA Research Network: \hyperlink{https://www.cancer.gov/tcga}{https://www.cancer.gov/tcga}. MNW was supported by the German Research Foundation (DFG), Grant Numbers: 437611051, 459360854.

\section*{Impact Statement}
This paper introduces a framework for interpreting individual predictions of survival analysis deep learning models, which has has the potential to enable more transparent and actionable predictions of time-to-event outcomes in critical fields such as healthcare, insurance, and personalized medicine. Therefore, this work addresses ethical concerns regarding the opacity of deep learning models, ensuring that predictions can be understood and trusted by stakeholders. This transparency is crucial for mitigating biases, fostering equitable decision-making, and ensuring compliance with regulatory standards. It needs to be stressed, that interpretable machine learning methods are useful to discover knowledge, to debug or
justify, as well as control and improve models and their predictions, but not to draw causal conclusions from the data. While there are risks of misuse and misunderstanding, we believe the net positive impact substantially outweighs the risks. Future societal consequences include enhanced patient care through personalized treatment plans, improved risk assessment in critical industries, and broader public trust in AI-driven systems.


\clearpage
\bibliography{reference,ICML24/example_paper}
\bibliographystyle{icml2025}

\newpage
\appendix
\onecolumn

\renewcommand{\thefigure}{A.\arabic{figure}}
\setcounter{figure}{0}
\renewcommand{\thetable}{A.\arabic{table}}
\setcounter{table}{0}

\section{Experiments on Simulated Data}
\subsection{Time-independent Effects}\label{app:time-independent-effects}
For a chosen observation, the hazard function from which the data are generated is of the form:
%
\begin{align}
    h(t| \bm{x}) = \lambda\, \gamma\, t^{\gamma - 1} \exp(1.7 x_1 - 2.4 x_2)
\end{align}
%
with $\lambda = 0.1$ and $\gamma = 2.5$ and $x_1, x_2, x_3 \sim \mathcal{N}(0,1)$. To generate the event times $t^{(i)}$ for instance $i$, the method of \cite{bender2005generating} is applied and the \texttt{simsurv} package \citep{brillemann2020simsurv} is used. Observations are artificially censored for $t^{(i)} \geq 7$.    

The data is split into training ($9,500$ observations) and test set ($500$ observations) and DeepSurv, CoxTime and DeepHit models with two dense layers with 32 nodes are fit to the training data without tuning, using $500$ epochs, early stopping, a batch size of $1,024$ and a dropout probability of $0.1$ applied to all layers. For any other hyperparameters, including the activations the default values set in the \texttt{pycox} \cite{kvamme_coxtime_2019} Python package are used, which are based on the default values suggested by \cite{katzman_deepsurv_2018, kvamme_coxtime_2019, lee_deephit_2018}. More details are provided in our code supplement. 

\begin{wraptable}{l}{0.35\columnwidth}
\caption{Performance metrics for different survival models fitted on time-independent simulation data (C-index: higher is better, $0.5$ indicates random prediction); IBS: lower is better). }
\label{tab:perform_tid}
\vskip 0.15in
\begin{center}
\begin{small}
\begin{sc}
\begin{tabular}{lcc}
\toprule
Model      & C-index ($\uparrow$)       & IBS ($\downarrow$)       \\
\midrule
CoxTime    & 0.807        & 0.1    \\
DeepSurv   & 0.809        & 0.099    \\
DeepHit    & 0.809        & 0.142    \\
\bottomrule
\end{tabular}
\end{sc}
\end{small}
\end{center}
\vskip -0.1in
\end{wraptable}

The models’ performance expressed in the Brier score is shown in Fig.~\ref{fig:brier_score_tid} and Table~\ref{tab:perform_tid} shows the Concordance index (C-index) and the Integrated Brier Score (IBS) as aggregated performance measures. DeepSurv slightly outperforms CoxTime and DeepHit in the Brier Score, likely because of its inherent assumption of proportional hazards, which the data simulated from a Cox model with time-independent effects are subject to. Even though CoxTime performs similarly to DeepSurv in Brier Score, this does not necessarily imply conformity to the PH assumption; it has to be assessed, for instance using gradient-based explanations. DeepHit has a higher C-index and IBS, suggesting poorly calibrated probabilistic predictions compared to DeepSurv and CoxTime, which is a consequence of the emphasis on C-index maximization in the loss function.

\begin{wrapfigure}{r}{0.4\columnwidth} 
\centering
\includegraphics[width=0.4\columnwidth]{figures/time-independent/sim_tid_brier_plot.pdf} 
\vskip 0.1in
\caption{Performance of DeepSurv, CoxTime, and DeepHit models over time measured by Brier score (lower is better; Brier score of 0.25 indicates prediction at random).}
\label{fig:brier_score_tid}
\vskip -0.1in
\end{wrapfigure}

Two observations are randomly chosen to illustrate the gradient-based explanation methods for survival deep learning models delineated in Sec.~\ref{sec:methodology_grad}. Their respective feature values and observed survival times are denoted in Table~\ref{tab:obs_tid}. The survival curves predicted by the selected survival NN models are shown in Fig.~\ref{fig:surv_plot_tid}. 

\begin{wraptable}{l}{0.45\columnwidth}
\caption{Feature values, observed survival time and event status of two randomly chosen observations (ID 13 and ID 387) from the test set of the simulated dataset with time-independent feature effects}
\label{tab:obs_tid}
\vskip 0.15in
\begin{center}
\begin{small}
\begin{sc}
\begin{tabular}{lccccc}
\toprule
ID & Time & Status   & x1       & x2       & x3       \\
\midrule
13           & 2.665 & 1        & -0.435 & 0.1162 & -0.081 \\
387          & 0.958 & 1        & 2.455 & 0.2462 & -0.043 \\
\bottomrule
\end{tabular}
\end{sc}
\end{small}
\end{center}
\vskip -0.1in
\end{wraptable}

The relevance values for each feature are represented by different colors are plotted across time in Figures~\ref{fig:grad_plot_tid}-\ref{fig:gshap_plot_tid} for the selected observations and models, highlighting the temporal dynamics of feature contributions.

\begin{figure}[ht]
    \centering
    \begin{minipage}[t]{0.48\columnwidth}
        \centering
        \includegraphics[width=\columnwidth]{figures/time-independent/sim_tid_grad_plot.pdf}
        \vskip 0.1in
        \caption{
            Grad(t) relevance curves for the selected observations and models trained on the time-independent simulation dataset. The relevance values for each feature are represented by different colors (y-axis) and are plotted across time (x-axis).}
        \label{fig:grad_plot_tid}
    \end{minipage}%
    \hfill
    \begin{minipage}[t]{0.48\columnwidth}
        \centering
        \includegraphics[width=\columnwidth]{figures/time-independent/sim_tid_gradin_plot.pdf}
        \vskip 0.1in
        \caption{
            G$\times$I(t) relevance curves for the selected observations and models trained on the time-independent simulation dataset. The relevance values for each feature are represented by different colors (y-axis) and are plotted across time (x-axis).}
        \label{fig:gradin_plot_tid}
    \end{minipage}
\end{figure}

\begin{figure}[ht]
    \centering
    \begin{minipage}[t]{0.48\columnwidth}
        \centering
        \includegraphics[width=\columnwidth]{figures/time-independent/sim_tid_smoothgrad_plot.pdf}
        \vskip 0.1in
        \caption{
            SG(t) relevance curves for the selected observations and models trained on the time-independent simulation dataset. The relevance values for each feature are represented by different colors (y-axis) and are plotted across time (x-axis).}
        \label{fig:smoothgrad_plot_tid}
    \end{minipage}%
    \hfill
    \begin{minipage}[t]{0.48\columnwidth}
        \centering
        \includegraphics[width=\columnwidth]{figures/time-independent/sim_tid_smoothgradin_plot.pdf}
        \vskip 0.1in
        \caption{
            SG$\times$I(t) relevance curves for the selected observations and models trained on the time-independent simulation dataset. The relevance values for each feature are represented by different colors (y-axis) and are plotted across time (x-axis).}
        \label{fig:smoothgradin_plot_tid}
    \end{minipage}
\end{figure}

\begin{figure}[ht]
    \centering
    \begin{minipage}[t]{0.48\columnwidth}
        \centering
        \includegraphics[width=\columnwidth]{figures/time-independent/sim_tid_intgrad0_plot.pdf}
        \vskip 0.1in
        \caption{
            IntGrad(t) relevance curves for the selected observations and models trained on the time-independent simulation dataset. The reference value is the null observations (all feature values set to zero). The relevance values for each feature are represented by different colors (y-axis) and are plotted across time (x-axis).}
        \label{fig:intgrad0_plot_tid}
    \end{minipage}%
    \hfill
    \begin{minipage}[t]{0.48\columnwidth}
        \centering
        \includegraphics[width=\columnwidth]{figures/time-independent/sim_tid_intgradmean_plot.pdf}
        \vskip 0.1in
        \caption{
            IntGrad(t) relevance curves for the selected observations and models trained on the time-independent simulation dataset. The reference value is the mean observation (feature values set to the average over all observations). The relevance values for each feature are represented by different colors (y-axis) and are plotted across time (x-axis).}
        \label{fig:intgradmean_plot_tid}
    \end{minipage}
\end{figure}

\begin{figure}[ht]
    \centering
    \begin{minipage}[t]{0.48\columnwidth}
        \centering
        \includegraphics[width=\columnwidth]{figures/time-independent/sim_tid_gshap_plot.pdf}
        \vskip 0.1in
        \caption{
            GradSHAP(t) relevance curves for the selected observations and models trained on the time-independent simulation dataset. The relevance values for each feature are represented by different colors (y-axis) and are plotted across time (x-axis).}
        \label{fig:gshap_plot_tid}
    \end{minipage}%
    \hfill
    \begin{minipage}[t]{0.42\columnwidth}
        \centering
        \includegraphics[width=\columnwidth]{figures/time-independent/sim_tid_surv_plot.pdf}
        \vskip 0.1in
        \caption{
            Predicted survival curves of CoxTime, DeepHit and DeepSurv model for randomly chosen observations (blue: ID 13, orange: ID 387). The models unanimously predict a higher probability of survival at any given time point for ID 13.}
        \label{fig:surv_plot_tid}
    \end{minipage}
\end{figure}


\clearpage
\subsection{Time-dependent Effects}\label{app:time-dependent-effects}
For a chosen observation, the hazard function from which the data are generated is of the form:
%
\begin{align}
    h(t\mid \bm{x}) = \lambda\, \gamma\, t^{\gamma - 1} \exp(-3x_1 + 1.7x_2 - 2.4x_3+ 6x_1\log(t))
\end{align}
%
with $\lambda = 0.1$ and $\gamma = 1.5$ and $x_1 \sim \mathcal{U}(0,1)$, $x_2, x_3 \sim \mathcal{N}(0,1)$, $x_4 \sim \mathcal{U}(-1,1)$. To generate the event times $t^{(i)}$ for instance $i$, the method of \cite{bender2005generating} is applied and the \texttt{simsurv} package \citep{brillemann2020simsurv} is used. Observations are artificially censored for $t^{(i)} \geq 7$. The Kaplan-Meier survival curves grouped by low and high values of feature $x_1$ demonstrate the time-dependent nature of its effect. Individuals with high values of $x_1$ initially show a higher average probability of survival at earlier time points ($t<2$), but their survival probability declines more rapidly over time, leading to a lower predicted probability of survival at later time points ($t>2$). A real-world example of such an effect could be cancer medication that provides strong early benefits by effectively slowing tumor progression or reducing symptoms. However, over time, the medication's efficacy might diminish due to drug resistance or cumulative side effects, resulting in worse long-term outcomes for patients compared to those on lower dosage treatments.  

\begin{wraptable}{l}{0.4\columnwidth}
\caption{Performance metrics for different survival models fitted on time-dependent simulation data (C-index: higher is better, $0.5$ indicates random prediction; IBS: lower is better). }
\label{tab:perform_td}
\vskip 0.15in
\begin{center}
\begin{small}
\begin{sc}
\begin{tabular}{lcc}
\toprule
Model      & C-index ($\uparrow$)       & IBS ($\downarrow$)       \\
\midrule
CoxTime    & 0.85        & 0.058    \\
DeepSurv   & 0.86        & 0.06    \\
DeepHit    & 0.805        & 0.095    \\
\bottomrule
\end{tabular}
\end{sc}
\end{small}
\end{center}
\vskip -0.1in
\end{wraptable}

The data is split into training ($9,500$ observations) and test set ($500$ observations) and DeepSurv, CoxTime and DeepHit models with two dense layers of 32 hidden nodes are fit to the training data without tuning, using $500$ epochs, early stopping, a batch size of $1,024$ and a dropout probability of $0.1$ applied to all layers. For any other hyperparameters, including the activations the default values set in the \texttt{pycox} \cite{kvamme_coxtime_2019} Python package are used, which are based on the default values suggested by \cite{katzman_deepsurv_2018, kvamme_coxtime_2019, lee_deephit_2018}. More details are provided in our code supplement. 

The models’ performance expressed in the Brier score is shown in Fig.~\ref{fig:brier_score_td} and Table~\ref{tab:perform_td} shows the Concordance index (C-index) and the Integrated Brier Score (IBS) as aggregated performance measures. CoxTime slightly outperforms DeepSurv and DeepHit in the Brier Score, likely because it is able to appropriately capture the violation of the assumption of proportional hazards in the time-dependent effects simulation. However, a superior performance does not necessarily imply that feature effects are captured correctly; this has to be assessed, for instance, using gradient-based explanations. DeepHit has a lower C-index and higher IBS, suggesting poorly calibrated probabilistic predictions as well as poor discriminatory power compared to DeepSurv and CoxTime, perhaps as a result of the Cox-based nature of the simulation. 

\begin{figure}[ht]
    \centering
    \begin{minipage}[t]{0.45\columnwidth}
        \centering
        \includegraphics[width=\columnwidth]{figures/time-dependent/sim_td_brier_plot.pdf}
        \vskip 0.1in
        \caption{
            Performance of DeepSurv, CoxTime, and DeepHit models over time measured by Brier score (lower is better; Brier score of 0.25 indicates prediction at random)}
        \label{fig:brier_score_td}
    \end{minipage}%
    \hfill
    \begin{minipage}[t]{0.45\columnwidth}
        \centering
        \includegraphics[width=\columnwidth]{figures/time-dependent/sim_td_km_plot.pdf}
        \vskip 0.1in
        \caption{
            Kaplan-Meier survival curves comparing individuals based on high (orange) and low (blue) values of feature $x_1$. Each curve represents the estimated survival rate for the corresponding group.}
        \label{fig:km_plot_td}
    \end{minipage}
\end{figure}


\begin{wraptable}{l}{0.5\columnwidth}
\caption{Feature values, observed survival time and event status of two randomly chosen observations (ID 79 and ID 428) from the test set of the simulated dataset with time-dependent feature effects.}
\label{tab:obs_td}
\vskip 0.15in
\begin{center}
\begin{small}
\begin{sc}
\begin{tabular}{lcccccc}
\toprule
ID & Time & Status   & x1       & x2       & x3   & x4    \\
\midrule
79           & 4.306 & 1        & 0.194 & -0.307 & -0.148	& 0.637 \\
428          & 1.417	 & 1        & 0.753 & -0.378 & -1.15 & -0.642	 \\
\bottomrule
\end{tabular}
\end{sc}
\end{small}
\end{center}
\vskip -0.1in
\end{wraptable}

Two observations are randomly chosen to illustrate the gradient-based explanation methods for survival deep learning models delineated in Sec.~\ref{sec:methodology_grad}. Their respective feature values and observed survival times are denoted in Table~\ref{tab:obs_td}. The survival curves predicted by the selected survival NN models are shown in Fig.~\ref{fig:surv_plot_td}. 

We propose several approaches to effectively visualize relevance values while incorporating the temporal dimension for difference-to-reference methods, beyond merely plotting the computed relevance values for individual features over time. In difference-to-reference methods, relevance values explain the deviation between the prediction (i.e., the predicted survival curve) for a selected observation and a chosen reference curve (e.g., the predicted survival curve for an observation where all feature values are set to zero, or where feature values are set to their respective means). To enhance clarity, we can plot the prediction (\texttt{pred}), the reference prediction (\texttt{pred\_ref}), and their difference (\texttt{pred\_diff}) alongside the relevance curves, as demonstrated in Figures~\ref{fig:intgrad0_plot_comp_td}, \ref{fig:intgradmean_plot_comp_td}, and \ref{fig:gshap_plot_comp_td}. Contribution plots (Figures~\ref{fig:intgrad0_plot_contr_td}, \ref{fig:intgradmean_plot_contr_td}, \ref{fig:gshap_plot_td_contr}) visualize the absolute, normalized feature-wise contributions to the prediction to reference difference, with the contributions represented as shaded areas colored by feature. These plots highlight how each feature relevances the prediction-to-reference difference. The absolute normalized contributions are calculated as

\begin{align}
    R^{t, \text{norm}}_{j} = \frac{|R^{t}_{j}|}{\sum_k |R^{t}_{k}|},
\end{align}

and are then plotted in a stacked form, maintaining the feature order by using the cumulative sum across the features. This provides insight into the magnitude of each feature's percentage contribution to the prediction-to-reference difference at each point in time. Furthermore, the absolute normalized contributions can be averaged over time to derive a time-independent local feature importance measure, which highlights the overall impact of each feature across the entire time period:

\begin{align}
    R^{\text{norm}}_{j} = \frac{1}{|t|} \sum_t \left ( \frac{|R^{t}_{j}|}{\sum_k |R^{t}_{k}|} \right ),
\end{align}

which is plotted on the right hand side of Figures~\ref{fig:intgrad0_plot_contr_td}, \ref{fig:intgradmean_plot_contr_td}, \ref{fig:gshap_plot_td_contr}. The force plots in Figures~\ref{fig:intgrad0_plot_force_td}, \ref{fig:intgradmean_plot_force_td}, and \ref{fig:gshap_plot_force_td} are the time-dependent equivalent of the well-established SHAP force plots, illustrating how individual feature contributions combine to explain the prediction-to-reference difference. These plots provide a detailed breakdown of the magnitude and direction of the factors influencing the prediction-to-reference difference. A representative set of equidistant observed survival time points (e.g., 10 points) is selected, and the contribution of each feature is visualized using stacked bar plots. The direction of each feature's effect is emphasized by upturned arrows for positive contributions and downturned arrows for negative contributions. Different colors represent the respective features, and the magnitude of each feature's contribution is displayed as a label within its corresponding bar. The overall prediction-to-reference difference is shown as a black line for reference.  

\begin{figure}[ht]
    \centering
    \begin{minipage}[t]{0.48\columnwidth}
        \centering
        \includegraphics[width=\columnwidth]{figures/time-dependent/sim_td_grad_plot.pdf}
        \vskip 0.1in
        \caption{
            Grad(t) relevance curves for the selected observations and models trained on the time-dependent simulation dataset. The relevance values for each feature are represented by different colors (y-axis) and are plotted across time (x-axis).}
        \label{fig:grad_plot_td}
    \end{minipage}%
    \hfill
    \begin{minipage}[t]{0.48\columnwidth}
        \centering
        \includegraphics[width=\columnwidth]{figures/time-dependent/sim_td_gradin_plot.pdf}
        \vskip 0.1in
        \caption{
            G$\times$I(t) relevance curves for the selected observations and models trained on the time-dependent simulation dataset. The relevance values for each feature are represented by different colors (y-axis) and are plotted across time (x-axis).}
        \label{fig:gradin_plot_td}
    \end{minipage}
\end{figure}

\begin{figure}[ht]
    \centering
    \begin{minipage}[t]{0.48\columnwidth}
        \centering
        \includegraphics[width=\columnwidth]{figures/time-dependent/sim_td_smoothgrad_plot.pdf}
        \vskip 0.1in
        \caption{
            SG(t) relevance curves for the selected observations and models trained on the time-dependent simulation dataset. The relevance values for each feature are represented by different colors (y-axis) and are plotted across time (x-axis).}
        \label{fig:smoothgrad_plot_td}
    \end{minipage}%
    \hfill
    \begin{minipage}[t]{0.48\columnwidth}
        \centering
        \includegraphics[width=\columnwidth]{figures/time-dependent/sim_td_smoothgradin_plot.pdf}
        \vskip 0.1in
        \caption{
            SG$\times$I(t) relevance curves for the selected observations and models trained on the time-dependent simulation dataset. The relevance values for each feature are represented by different colors (y-axis) and are plotted across time (x-axis).}
        \label{fig:smoothgradin_plot_td}
    \end{minipage}
\end{figure}

\begin{figure}[ht]
    \centering
    \begin{minipage}[t]{0.48\columnwidth}
        \centering
        \includegraphics[width=\columnwidth]{figures/time-dependent/sim_td_intgrad0_plot.pdf}
        \vskip 0.1in
        \caption{
            IntGrad(t) relevance curves for the selected observations and models trained on the time-dependent simulation dataset. The reference value is the null observations (all feature values set to zero). The relevance values for each feature are represented by different colors (y-axis) and are plotted across time (x-axis).}
        \label{fig:intgrad0_plot_td}
    \end{minipage}%
    \hfill
    \begin{minipage}[t]{0.48\columnwidth}
        \centering
        \includegraphics[width=\columnwidth]{figures/time-dependent/sim_td_intgradmean_plot.pdf}
        \vskip 0.1in
        \caption{
            IntGrad(t) relevance curves for the selected observations and models trained on the time-dependent simulation dataset. The reference value is the mean observation (feature values set to the average over all observations). The relevance values for each feature are represented by different colors (y-axis) and are plotted across time (x-axis).}
        \label{fig:intgradmean_plot_td}
    \end{minipage}
\end{figure}

\begin{figure}[ht]
    \centering
    \begin{minipage}[t]{0.48\columnwidth}
        \centering
        \includegraphics[width=\columnwidth]{figures/time-dependent/sim_td_gshap_plot.pdf}
        \vskip 0.1in
        \caption{
            GradSHAP(t) relevance curves for the selected observations and models trained on the time-dependent simulation dataset. The reference value is the mean observation (feature values set to the average over all observations). The relevance values for each feature are represented by different colors (y-axis) and are plotted across time (x-axis).}
        \label{fig:gshap_plot_td}
    \end{minipage}%
    \hfill
    \begin{minipage}[t]{0.42\columnwidth}
        \centering
        \includegraphics[width=\columnwidth]{figures/time-dependent/sim_td_surv_plot.pdf}
        \vskip 0.1in
        \caption{
            Predicted survival curves of CoxTime, DeepHit and DeepSurv model for randomly chosen observations (blue: ID 79, orange: ID 428). The models unanimously predict a higher probability of survival at any given time point for ID 428.}
        \label{fig:surv_plot_td}
    \end{minipage}
\end{figure}

\begin{figure}[ht]
    \centering
    \begin{minipage}[t]{0.48\columnwidth}
        \centering
        \includegraphics[width=\columnwidth]{figures/time-dependent/sim_td_intgrad0_plot_comp.pdf}
        \vskip 0.1in
        \caption{
            IntGrad(t) relevance curves (yellow, turquoise, blue, purple) and predicted survival curves for the selected observations (ref), predicted survival curve for the reference observation (pred\_ref) and their difference (pred\_diff) for models trained on the time-dependent simulation dataset. The reference value is the null observations (all feature values set to zero).}
        \label{fig:intgrad0_plot_comp_td}
    \end{minipage}%
    \hfill
    \begin{minipage}[t]{0.48\columnwidth}
        \centering
        \includegraphics[width=\columnwidth]{figures/time-dependent/sim_td_intgrad0_plot_contr.pdf}
        \vskip 0.1in
        \caption{
            IntGrad(t) contribution plots for the selected observations and models trained on the time-dependent simulation dataset. The reference value is the mean observation (all feature values set to zero).}
        \label{fig:intgrad0_plot_contr_td}
    \end{minipage}
\end{figure}

\begin{figure}[ht]
    \centering
    \begin{minipage}[t]{0.48\columnwidth}
        \centering
        \includegraphics[width=\columnwidth]{figures/time-dependent/sim_td_intgrad0_plot_force.pdf}
        \vskip 0.1in
        \caption{
            IntGrad(t) force plots for the selected observations and models trained on the time-dependent simulation dataset. The reference value is the mean observation (all feature values set to zero).}
        \label{fig:intgrad0_plot_force_td}
    \end{minipage}%
    \hfill
    \begin{minipage}[t]{0.48\columnwidth}
        \centering
        \includegraphics[width=\columnwidth]{figures/time-dependent/sim_td_intgradmean_plot_comp.pdf}
        \vskip 0.1in
        \caption{
            IntGrad(t) relevance curves (yellow, turquoises, blue, purple) and predicted survival curves for the selected observations (ref), predicted survival curve for the reference observation (pred\_ref) and their difference (pred\_diff) for models trained on the time-dependent simulation dataset. The reference value is the null observations (feature values set to the average over all observations).}
        \label{fig:intgradmean_plot_comp_td}
    \end{minipage}
\end{figure}

\begin{figure}[ht]
    \centering
    \begin{minipage}[t]{0.48\columnwidth}
        \centering
        \includegraphics[width=\columnwidth]{figures/time-dependent/sim_td_intgradmean_plot_contr.pdf}
        \vskip 0.1in
        \caption{
            IntGrad(t) contribution plots for the selected observations and models trained on the time-dependent simulation dataset. The reference value is the mean observation (feature values set to the average over all observations).}
        \label{fig:intgradmean_plot_contr_td}
    \end{minipage}%
    \hfill
    \begin{minipage}[t]{0.48\columnwidth}
        \centering
        \includegraphics[width=\columnwidth]{figures/time-dependent/sim_td_intgradmean_plot_force.pdf}
        \vskip 0.1in
        \caption{
            IntGrad(t) force plots for the selected observations and models trained on the time-dependent simulation dataset. The reference value is the mean observation (feature values set to the average over all observations).}
        \label{fig:intgradmean_plot_force_td}
    \end{minipage}
\end{figure}

\begin{figure}[ht]
    \centering
    \begin{minipage}[t]{0.48\columnwidth}
        \centering
        \includegraphics[width=\columnwidth]{figures/time-dependent/sim_td_gshap_plot_comp.pdf}
        \vskip 0.1in
        \caption{
            GradSHAP(t) relevance curves (yellow, turquoises, blue, purple) and predicted survival curves for the selected observations (ref), predicted survival curve for the reference observation (pred\_ref) and their difference (pred\_diff) for models trained on the time-dependent simulation dataset. The reference value is the mean observation (feature values set to the average over all observations).}
        \label{fig:gshap_plot_comp_td}
    \end{minipage}%
    \hfill
    \begin{minipage}[t]{0.48\columnwidth}
        \centering
        \includegraphics[width=\columnwidth]{figures/time-dependent/sim_td_gshap_plot_contr.pdf}
        \vskip 0.1in
        \caption{
            GradSHAP(t) contribution plots for the selected observations and models trained on the time-dependent simulation dataset. The reference value is the mean observation (feature values set to the average over all observations).}
        \label{fig:gshap_plot_td_contr}
    \end{minipage}
\end{figure}

\begin{figure}[ht]
    \centering
    \begin{minipage}[t]{0.85\columnwidth}
        \centering
        \includegraphics[width=\columnwidth]{figures/time-dependent/sim_td_gshap_plot_force.pdf}
        \vskip 0.1in
        \caption{
            GradSHAP(t) force plots for the selected observations and models trained on the time-dependent simulation dataset. The reference value is the mean observation (feature values set to the average over all observations).}
        \label{fig:gshap_plot_force_td}
    \end{minipage}%
    %\hfill
    %\begin{minipage}[t]{0.48\columnwidth}
    %    \centering
    %    \includegraphics[width=\columnwidth]{}
    %    \vskip 0.1in
    %    \caption{}
    %    \label{fig:intgradmean_plot_contr_td}
    %\end{minipage}
\end{figure}


\clearpage

\subsection{GradSHAP(t) vs SurvSHAP(t)}\label{app:survgradshap}

\subsubsection{Local Accuracy}

The concept of "local accuracy" originates from Shapley values and refers to the property that individual feature contributions sum up to the difference between the prediction and the average (i.e., marginal) prediction \cite{lundberg2017unified}:  
\begin{align}
    \sum_{j=1}^p \phi_j = f(\bm{x}) - \mathbb{E}_{\bm{\tilde{x}}}[f(\bm{\tilde{x}})].
\end{align}
However, in the survival context, this decomposition must be considered for each time point, where the decomposition quantity dynamically changes over time since survival functions are monotonically decreasing. \citeauthor{krzyzinski2023survshap} propose a time-dependent variation of this measure to account for these dynamics:  
\begin{align}
    \sqrt{\frac{\mathbb{E}_{\bm{x}}\left[\left(f(t \mid \bm{x}) - \mathbb{E}_{\bm{\tilde{x}}}[f(t \mid \bm{\tilde{x}})] - \sum_{j=1}^p R_j(t \mid \bm{x})\right)^2 \right]}{\mathbb{E}_{\bm{x}}\left[f(t \mid \bm{x})\right]}}.
\end{align}
This formulation gives greater weight to discrepancies at time points where the decomposition target becomes negligibly small, thus addressing situations where standard local accuracy would be less informative.  

In our simulation study, we follow a similar setup as described for the time-independent effects (see Sec.~\ref{app:time-independent-effects}). We use $p = 20$ features with coefficients linearly increasing from 0 to 1 and alternating signs, i.e., $\beta_j = (-1)^j \frac{j - 1}{19}$, and artificially censor the event time at $10$. The data is split into a training set ($1,000$ observations) and a test set ($100$ observations). We train DeepSurv, CoxTime, and DeepHit models with two dense layers and 32 hidden nodes on the training data, using $30\%$ validation data and early stopping for up to $500$ epochs. For GradSHAP(t), we vary the number of integration samples (5, 25, 50). Additionally, for both XAI methods, we apply the methods on all $100$ test instances. The results for all model classes are displayed in Fig.~\ref{fig:app_localacc}. Further details can be found in the code supplement on GitHub. 

\begin{figure}[h]
    \centering
    \includegraphics[width=0.9\linewidth]{figures/gradshap/gradshapt_localacc.pdf}
    \caption{Local accuracy (y-axis), measured as the normalized standard deviation of the difference between the black-box model output and the explanation (lower is better), plotted over time (x-axis) for SurvSHAP(t) (purple) and GradSHAP(t) with varying numbers of integration samples (blue = 5, turquoise = 25, yellow = 50). GradSHAP(t) achieves local accuracy comparable to SurvSHAP(t) while significantly reducing runtime.}
    \label{fig:app_localacc}
\end{figure}


\subsubsection{Runtime}

To evaluate the computational efficiency of the proposed feature attribution methods, we compare SurvSHAP(t) and GradSHAP(t) with varying numbers of integration samples (5, 25, 50) across CoxTime, DeepHit, and DeepSurv models. The survival data is generated analog to the previous section but with varying $p$ and split the data into a training ($1,000$ observations) and a test set ($100$ observations). We train DeepSurv, CoxTime, and DeepHit models with two dense layers and 32 hidden nodes on the training data, using $30\%$ validation split and early stopping for up to $500$ epochs. To obtain stable time measurements, we take the median runtime of 20 repetitions for a single trained model and repeat this five times. Additionally, we apply the XAI methods on all available test instances. To ensure a fair comparison of runtime performance, we do not employ any parallelization beyond controlling the number of threads. Specifically, we limit the number of \texttt{torch} threads and inter-op threads to 10 each. Further details can be found in the code supplement on GitHub. 

Fig.~\ref{fig:app_runtime} illustrates the runtime (y-axis) as a function of the number of features (x-axis). The results reveal notable differences in computational demands between the methods. SurvSHAP(t) consistently exhibits higher runtime across all feature set sizes and model classes, with the computational cost rapidly increasing as the number of features grows. This can be attributed to its reliance on sampling multiple subsets of features for Shapley value estimation. In contrast, GradSHAP(t) demonstrates significantly improved efficiency, particularly when using fewer integration samples. As the number of features increases, GradSHAP(t) maintains computational efficiency, outperforming SurvSHAP(t) in all configurations.

Notably, increasing the number of integration samples from 5 to 50 for GradSHAP(t) results in higher runtimes but remains computationally more efficient than SurvSHAP(t) even for large feature sets. These findings underscore the scalability and efficiency advantages of gradient-based attribution methods for survival models when handling high-dimensional feature spaces.

\begin{figure}[h]
    \centering
    \includegraphics[width=0.9\linewidth]{figures/gradshap/gradshapt_runtime.pdf}
    \caption{Runtime (y-axis) for generating attributions using SurvSHAP(t) (purple) and GradSHAP(t) with varying numbers of integration samples (blue = 5, turquoise = 25, yellow = 50), measured on simulated datasets with an increasing number of features (x-axis). GradSHAP(t) significantly improves computational efficiency as the number of features grows, outperforming SurvSHAP(t) in all settings. For smaller feature sets, GradSHAP(t) with low integration samples also achieves faster runtimes than SurvSHAP(t). The results are averaged over multiple runs (20 per model).}
    \label{fig:app_runtime}
\end{figure}


\subsubsection{Global Feature Rankings}

To assess the ability of the attribution methods to capture global feature importance, we evaluate GradSHAP(t), SurvSHAP(t), and SurvLIME on a simulated dataset with five features of predefined importance $(x_1 > x_2 > x_3 > x_4 > x_5)$. The survival data is generated analogously to the runtime analysis with $p = 5$, $2,000$ train and $300$ test samples, and models (CoxTime, DeepHit, DeepSurv) are trained as described in the previous section. Further details can be found in the code supplement on GitHub. 

Fig.~\ref{fig:app_rankings} presents the ranking distribution of features across $300$ predictions for each model. Rankings are derived from the relative importance scores assigned to each feature. Both GradSHAP(t) and SurvSHAP(t) demonstrate robust performance by consistently maintaining the predefined importance hierarchy across all models. This consistency highlights their ability to capture the underlying feature relationships. In contrast, SurvLIME shows a more uniform distribution of rankings, leading to a reduced capacity to differentiate features of varying importance. Overall, the results emphasize that gradient-based methods like GradSHAP(t) and sample-based methods like SurvSHAP(t) are well-suited for identifying global feature importance in survival models, whereas SurvLIME may lack the precision required for nuanced analyses.

\begin{figure}[h]
    \centering
    \includegraphics[width=0.9\linewidth]{figures/gradshap/gradshapt_ranking.pdf}
    \caption{Comparison of local and global importance rankings for 300 predictions from the CoxTime model (top), DeepHit model (middle) and  DeepSurv model (middle) on a simulated dataset including five features of increasing importance ($x_1>x_2>x_3>x_4>x_5$). Colors arranged in a gradient from blue to yellow reflect the global ranking of features within each model. GradSHAP(t) and SurvSHAP(t) perform similarly, consistently retaining the majority of observations for each consecutive feature (with the exception of $x_3$ in at the 3rd spot). This demonstrates superior performance compared to SurvLIME, which produces more uniformly distributed rankings.}
    \label{fig:app_rankings}
\end{figure}



\end{document}


% This document was modified from the file originally made available by
% Pat Langley and Andrea Danyluk for ICML-2K. This version was created
% by Iain Murray in 2018, and modified by Alexandre Bouchard in
% 2019 and 2021 and by Csaba Szepesvari, Gang Niu and Sivan Sabato in 2022.
% Modified again in 2023 and 2024 by Sivan Sabato and Jonathan Scarlett.
% Previous contributors include Dan Roy, Lise Getoor and Tobias
% Scheffer, which was slightly modified from the 2010 version by
% Thorsten Joachims & Johannes Fuernkranz, slightly modified from the
% 2009 version by Kiri Wagstaff and Sam Roweis's 2008 version, which is
% slightly modified from Prasad Tadepalli's 2007 version which is a
% lightly changed version of the previous year's version by Andrew
% Moore, which was in turn edited from those of Kristian Kersting and
% Codrina Lauth. Alex Smola contributed to the algorithmic style files.
