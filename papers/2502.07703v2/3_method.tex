% \begin{figure*}[t!]
%     \centering
%     \includegraphics[width=0.8\textwidth]
%     {figure/pipeline.pdf}
%     \caption{
%     Temp pipeline.
%     }
%     \label{fig:overview}
%     \vspace{-3mm}
% \end{figure*}

\section{Preliminary}
\label{sec:method}

\subsection{Notation}
\label{subsec: Notation}
In this paper, $x_k^j$ denotes the $j_{th}$ iteration of Kalman filter update for $k_{th}$ state, and $(), \widehat{()}, \bar{()}$ denote ground truth, propagated, and optimal state. Global frame ${}^G()$ is set as initial \ac{IMU} frame, while ${}^A()$ denote sensor frames. The state description and definition of $\boxplus/\boxminus$ are as follows:

\vspace{-4mm}\small
\begin{eqnarray}
\label{eq:state}
	\mathcal{M} &\triangleq& SE_{2}(3) \times \mathbb{R}^{6}, \:   
	    \textbf{x} \: \triangleq \: [\mathcal{X}^T, \: \textbf{b}_{\omega}^T, \: \textbf{b}_a^T ]^T \in\mathcal{M} \nonumber  \\
	\textbf{u} &\triangleq& \left[\omega_m^T \quad  a_m^T \right]^T,\, \: \textbf{n} \: \triangleq \: \left[n_\omega^T, \: n_a^T, \: n_{b\omega}^T, \: n_{ba}^T  \right]^T \\
    \label{eq:SE2(3) define}
    \mathcal{X}  &\triangleq& \begin{bmatrix}
    {}^G\textbf{R}_{I} & {}^G\textbf{v}_{I} & {}^G\textbf{p}_{I} \\
     0 & 1 & 0 \\
     0 & 0 & 1 \\
    \end{bmatrix}\in SE_2(3) \nonumber  \\
    % \label{eq:SE2(3) inverse define}
    % \mathcal{X}^{-1}  &\triangleq& \begin{bmatrix}
    % {}^G\textbf{R}_{I}^T & -{}^G\textbf{R}_{I}^T{}^G\textbf{v}_{I} & -{}^G\textbf{R}_{I}^T{}^G\textbf{p}_{I} \\
    %  0 & 1 & 0 \\
    %  0 & 0 & 1 \\
    % \end{bmatrix} \nonumber  \\
     \delta \widehat{\textbf{x}}_{k}&\triangleq&\textbf{x}_{k} \boxminus \widehat{\textbf{x}}_{k} = \left [ \delta \widehat{\theta}_{k}, \delta \widehat{\textbf{v}}_{k},\delta \widehat{\textbf{p}}_{k},\delta \widehat{\textbf{b}}_{\omega_{k}},\delta \widehat{\textbf{b}}_{a_k} \right ]  \nonumber\\
     &=& \left [\mathrm{Log}( \mathcal{X}_k\widehat{\mathcal{X}}_k^{-1}), {\textbf{b}}_{\omega_{k}}-\widehat{\textbf{b}}_{\omega_{k}}, {\textbf{b}}_{a_k}- \widehat{\textbf{b}}_{a_k} \right ] \in \mathbb{R}^{15}
     \nonumber \\
     \textbf{x} \boxplus \delta \widehat{\textbf{x}} &=& \begin{bmatrix}\mathcal {X} \\ \mathbf {b} \end{bmatrix} \boxplus \begin{bmatrix} {\xi } \\ \widehat{\mathbf {b}} \end{bmatrix} \triangleq \begin{bmatrix}\rm Exp({\xi }) \cdot \mathcal {X} \\ \mathbf {b} + \widehat{\mathbf {b}} \end{bmatrix} \nonumber
     \vspace{-8mm}
\end{eqnarray}
\normalsize
, where $\mathbf {b}, \widehat{\mathbf {b}} \in \mathbb{R}^{6}$ and ${\xi} \in \mathbb{R}^{9}$. In \eqref{eq:state}, ${}^G\textbf{R}_{I}$, ${}^G\textbf{v}_{I}$, and ${}^G\textbf{p}_{I}$ denote the rotation, velocity, and position of the \ac{IMU} in the global frame, while $\textbf{b}$ represents the IMU bias. The angular velocity and linear acceleration from the \ac{IMU} are denoted by $\omega_m$ and $a_m$, with $\textbf{n}$ representing the Gaussian white noise associated with these measurements and biases.

\subsection{State Propagation}
\label{subsec: State Propagation}
For state prediction within the time interval $[k, k+1)$, we discretize the continuous-time \ac{IMU} kinematic model using a zero-order holder with an \ac{IMU} sampling period $\Delta t$ \cite{FAST-LIO}. The resulting discretized motion model \( \textbf{f} \), is formulated as:

\vspace{-5mm}\small
\begin{align}
\label{eq:propagate}
	\textbf{x}_{i+1} &= \textbf{x}_{i} \boxplus \left(\Delta t \, \textbf{f}\left(\textbf{x}_i, \textbf{u}_i, \textbf{n}_i\right)\right), \\
	\textbf{f}\left(\textbf{x}_i, \textbf{u}_i, \textbf{n}_i\right) &= \begin{bmatrix} 
    	\omega_{m_i} - b_{\omega_i} - n_{\omega_i} \\
    	{}^G\textbf{R}_{I_i} \left(a_{m_i} - b_{a_i} - n_{a_i}\right) + {}^G\textbf{g}_i \\
        {{}^G\textbf{v}}_{I_i} + \frac{1}{2}\left({}^G\textbf{R}_{I_i} \left(a_{m_i} - b_{a_i} - n_{a_i}\right) + {}^G\textbf{g}_i\right) \Delta t \nonumber \\
    	n_{b\omega_i} \\
    	n_{ba_i} 
    \end{bmatrix}
\end{align}
\normalsize
State prediction is conducted using the discrete \ac{IMU} kinematic model from \eqref{eq:propagate} by setting the noise term $\textbf{n}=0$:

\vspace{-3mm}\small
\begin{eqnarray}
\label{eq:state_predict}
\widehat{\textbf{x}}_{\tau+1} = \widehat{\textbf{x}}_{\tau} \boxplus \left(\Delta t \textbf{f}\left(\widehat{\textbf{x}}_\tau, \textbf{u}_\tau, 0\right)\right)\:;\;\;t_k \leq t_\tau <t_{k+1} 
\vspace{-1mm}
\end{eqnarray}
\normalsize



\subsection{Error State Prediction}
\label{subsec:Error State Prediction}
To ensure the system depends on the error rather than the state, we define the error using a right-invariant error \cite{hartley2020contact}. The error-state transition matrix is derived by combining the error state \eqref{eq:state} with the \ac{IMU} motion model \eqref{eq:propagate}. The error-state $\delta \widehat{\textbf{x}}_{\tau+1}$ and its covariance $\widehat{\textbf{P}}_{\tau+1}$ are then computed as:

\vspace{-5mm}\small
\begin{align}
\label{eq:nominal state prop}
  \delta \widehat{\textbf{x}}_{\tau+1} =& \: {\textbf{x}}_{\tau+1} - \widehat{\textbf{x}}_{\tau+1}
  \simeq \textbf{F}_{\delta \widehat{\textbf{x}}_{\tau}}\delta \widehat{\textbf{x}}_{\tau} + \textbf{F}_{\textbf{n}_\tau} \textbf{n}_\tau \\ 
  % \delta \widehat{\theta}_{\tau+1} =& \mathrm{Log}\left( {}^G\textbf{R}_{I,{\tau+1}} {}^G\widehat{\textbf{R}}_{I,\tau+1}^T\right)\\
  % , \nonumber \\
  % =& \mathrm{Log} \Big( {}^G\textbf{R}_{I,{\tau}} \, \mathrm{Exp}\left(\Delta t( \omega_{m_\tau} - b_{\omega_\tau} - n_{\omega_\tau}) \right) \nonumber \\
  % &\quad \quad \mathrm{Exp}\left(\Delta t( \omega_{m_\tau} - \widehat{b}_{\omega_\tau}) \right)^\top \, {}^G\widehat{\textbf{R}}_{I,{\tau}}^\top \Big) \\
  % \delta \widehat{\textbf{v}}_{\tau+1} =& {}^G\textbf{v}_{I,\tau+1} - {}^G\textbf{R}_{I,{\tau+1}} {}^G\widehat{\textbf{R}}_{I,\tau+1}^T {}^G\widehat{\textbf{v}}_{I,\tau+1} \\
  % \delta \widehat{\textbf{p}}_{\tau+1} =& {}^G\textbf{p}_{I,\tau+1} - {}^G\textbf{R}_{I,{\tau+1}} {}^G\widehat{\textbf{R}}_{I,\tau+1}^T {}^G\widehat{\textbf{p}}_{I,\tau+1}
  \label{eq:transition matrix}
  \widehat{\textbf{P}}_{\tau+1} =& \: \textbf{F}_{\delta \widehat{\textbf{x}}_{\tau}}\widehat{\textbf{P}}_{\tau}\textbf{F}^T_{\delta \widehat{\textbf{x}}_{\tau}} + \textbf{F}_{\textbf{n}_\tau}\textbf{Q}_{\tau}\textbf{F}^T_{{\textbf{n}}_\tau} \nonumber
\end{align}
\normalsize

% \vspace{-3mm}\small
% \begin{align}
% \label{eq:nominal state prop}
%   \delta \widehat{\textbf{x}}_{\tau+1} =& (\textbf{x}_{\tau} \boxplus \left(\Delta t \textbf{f}\left(\textbf{x}_\tau, \textbf{u}_\tau, \textbf{n}_\tau\right)\right))  \boxminus(\widehat{\textbf{x}}_{\tau} \boxplus \left(\Delta t \textbf{f}\left(\widehat{\textbf{x}}_\tau, \textbf{u}_\tau, 0\right)\right)) \nonumber \\
%   \simeq& \textbf{F}_{\delta \widehat{\textbf{x}}_{\tau}}\delta \widehat{\textbf{x}}_{\tau} + \textbf{F}_{\textbf{n}_\tau} \textbf{n}_\tau \\
%   % \delta \widehat{\theta}_{\tau+1} =& \mathrm{Log}\left( {}^G\textbf{R}_{I,{\tau+1}} {}^G\widehat{\textbf{R}}_{I,\tau+1}^T\right)\\
%   % , \nonumber \\
%   % =& \mathrm{Log} \Big( {}^G\textbf{R}_{I,{\tau}} \, \mathrm{Exp}\left(\Delta t( \omega_{m_\tau} - b_{\omega_\tau} - n_{\omega_\tau}) \right) \nonumber \\
%   % &\quad \quad \mathrm{Exp}\left(\Delta t( \omega_{m_\tau} - \widehat{b}_{\omega_\tau}) \right)^\top \, {}^G\widehat{\textbf{R}}_{I,{\tau}}^\top \Big) \\
%   % \delta \widehat{\textbf{v}}_{\tau+1} =& {}^G\textbf{v}_{I,\tau+1} - {}^G\textbf{R}_{I,{\tau+1}} {}^G\widehat{\textbf{R}}_{I,\tau+1}^T {}^G\widehat{\textbf{v}}_{I,\tau+1} \\
%   % \delta \widehat{\textbf{p}}_{\tau+1} =& {}^G\textbf{p}_{I,\tau+1} - {}^G\textbf{R}_{I,{\tau+1}} {}^G\widehat{\textbf{R}}_{I,\tau+1}^T {}^G\widehat{\textbf{p}}_{I,\tau+1}
%   \label{eq:transition matrix}
%   \widehat{\textbf{P}}_{\tau+1} =& \textbf{F}_{\delta \widehat{\textbf{x}}_{\tau}}\widehat{\textbf{P}}_{\tau}\textbf{F}^T_{\delta \widehat{\textbf{x}}_{\tau}} + \textbf{F}_{\textbf{n}_\tau}\textbf{Q}_{\tau}\textbf{F}^T_{{\textbf{n}}_\tau}
% \end{align}
% \normalsize
The initial value of $\delta \widehat{\textbf{x}}_{i}$ is set to zero. $\textbf{F}_{\delta \widehat{\textbf{x}}_{\tau}}$ and $\textbf{F}_{\textbf{n}_\tau}$ denote the Jacobian matrix of $\delta \widehat{\textbf{x}}_{\tau+1}$ with respect to $\delta \widehat{\textbf{x}}_{\tau}$ evaluated under the condition that $\delta \widehat{\textbf{x}}_{\tau}=0$ and $\textbf{n}_\tau=0$.
% \begin{eqnarray}\small
%   \label{eq:transition matrix}
%     \widehat{\textbf{P}}_{\tau+1} &=& \textbf{F}_{\delta \widehat{\textbf{x}}_{\tau}}\widehat{\textbf{P}}_{\tau}\textbf{F}^T_{\delta \widehat{\textbf{x}}_{\tau}} + \textbf{F}_{\textbf{n}_\tau}\textbf{Q}_{\tau}\textbf{F}^T_{{\textbf{n}}_\tau}
% \end{eqnarray}

\subsection{Iterated State Update}
\label{subsec:Iterated State Update}
% \subsubsection{Prior Distribution}
The state $\widehat{\textbf{x}}_{k+1}$ and covariance $\widehat{\textbf{P}}_{k+1}$ are used as the prior distribution of $\textbf{x}_{k+1}$. With the measurement at time index $k+1$ serving as the observation, the posterior distribution of the state can be determined \cite{he2021kalman}. Using the prior distribution, the error state propagation can be performed as follows:

\vspace{-3mm}\small
\begin{eqnarray}
\begin{aligned}
  \label{eq:prior distribution}
    \delta\widehat{\textbf{x}}_{k+1}
    % &=\textbf{x}_{k+1}\boxminus\widehat{\textbf{x}}_{k+1}\sim\mathcal{N}(0,\widehat{P}_{k+1}) \\
    % &=(\widehat{\textbf{x}}_{k+1}^j\boxplus\delta\textbf{x}_j)\boxminus\widehat{\textbf{x}}_{k+1}
    &=(\widehat{\textbf{x}}_{k+1}^j\boxminus\widehat{\textbf{x}}_{k+1})+\textbf{J}_{k+1}\delta\textbf{x}_j \\
    \delta\textbf{x}_j&\sim\mathcal{N}(-{\textbf{J}_{k+1}^{-1}}(\widehat{\textbf{x}}_{k+1}^j\boxplus\widehat{\textbf{x}}_{k+1}),{\textbf{J}_{k+1}^{-1}}\widehat{P}_{k+1}{\textbf{J}_{k+1}^{-\top}})
\end{aligned}
\end{eqnarray}
\normalsize
$\textbf{J}_{k+1}$ is the Jacobian of $\delta\widehat{\textbf{x}}_{k+1}$ w.r.t $\delta\textbf{x}_j$ at 0, and detailed derivation is in \cite{shi2023invariant}. After project $\widehat{\textbf{x}}_{k+1}^j$ to the same space of $\delta\textbf{x}_j$, we can get an optimal $\delta\textbf{x}_j$ combining \eqref{eq:prior distribution} and measurement model  $z_{k+1}^M = \textbf{h}_{M^j}(\textbf{x}_k,\textbf{v}_j^M)$ :

\vspace{-3mm}\small
\begin{eqnarray}
\label{eq:MAP}
  \min\limits_{\delta\textbf{x}_j} ( \lVert \textbf{x}_{k+1} \boxminus \widehat{\textbf{x}}_{k+1} \rVert ^2_{\textbf{P}} + \sum_{M}\sum^{|M|}_{j=1} \lVert \textbf{r}_{M^j} +\textbf{H}_{M^j}\delta \widehat{\textbf{x}}_k \rVert ^{2}_{{R_j^M}} )
\end{eqnarray}
\normalsize
where $\textbf{P}={\textbf{J}_{k+1}^{-1}}\widehat{\textbf{P}}_{k+1}{\textbf{J}_{k+1}^{-\top}}$, $\textbf{H}_{M^j}$ is the Jacobian matrix of ${\textbf{h}}_{M^j}$ w.r.t $\delta \widehat{\textbf{x}}_{k}$ and measurement model which is used in this paper is described in \secref{subsec:first_update} and \secref{subsec:second_update}. The $\delta x_j$ that minimizes equation \eqref{eq:MAP} can be obtained through the Kalman filter update process using $\mathbf{H}=\left[\mathbf{H}_{M^1},\cdots \right], \mathbf{R}=diag\left[{R}_1^M,\cdots\right]$. If the state converges, the optimal state $\bar{\textbf{x}}_{k+1}$ and covariance $\bar{\mathbf{P}}_{k+1}$ are updated and details are in \cite{he2021kalman}.
% \subsubsection{First Update}
% In the first update, the point-to-plane residual and pointwise velocity residual are used for the update process. By combining equations \eqref{eq:prior distribution}, \eqref{eq:LiDAR Measurement Model}, and \eqref{eq:radar measurement model} the following maximum a posteriori can be yielded:

% \vspace{-3mm}\small
% \begin{align}
% \label{eq:MAP}
%   \min\limits_{\delta\textbf{x}_j} ( \lVert \textbf{x}_{k+1} \boxminus \widehat{\textbf{x}}_{k+1} \rVert ^2_{\textbf{P}} + \sum_{M = L, R}\sum^{|M|}_{j=1} \lVert \textbf{r}_{M^j} +\textbf{H}_{M^j}\delta \widehat{\textbf{x}}_k \rVert ^{2}_{{R_j^M}} )
% \end{align}
% \normalsize
% where $\textbf{P}={\textbf{J}_{k+1}^{-1}}\widehat{\textbf{P}}_{k+1}{\textbf{J}_{k+1}^{-\top}}$. The $\delta x_j$ that minimizes equation \eqref{eq:MAP} can be obtained through the Kalman filter update process using $\mathbf{H}=\left[\mathbf{H}_{L^1},\cdots,\mathbf{H}_{R^1},\cdots\right], \mathbf{R}=diag\left[{R}_1^L,\cdots,{R}_1^R,\cdots\right]$.
% \vspace{-3mm}\small
% \begin{eqnarray}
% \begin{aligned}
%   \label{eq:Kalman filter update}
%     \mathbf{H}&=\left[\mathbf{H}_{L^1},\cdots,\mathbf{H}_{R^1},\cdots\right]\\
%     \mathbf{R}&=diag\left[{R}_1^L,\cdots,{R}_1^R,\cdots\right]\\
%     \mathbf{K}&=\left(\mathbf{H}^\top \mathbf{R}^{-1}\mathbf{H}+\mathbf{P}^{-1}\right)^{-1}\mathbf{H}^\top \mathbf{R}^{-1} \\
%     \Tilde{\textbf{z}}_{k+1}^j&=\left[r_{L^1},\cdots,r_{R^1},\cdots\right] \\
%     \delta \textbf{x}_j &=-\textbf{K}\Tilde{\textbf{z}}_{k+1}^j-(\textbf{I}-\textbf{K}\textbf{H}){\textbf{J}_{k+1}^{-1}}(\widehat{\textbf{x}}_{k+1}^j\boxminus\widehat{\textbf{x}}_{k+1}) \\
%     \widehat{\textbf{x}}_{k+1}^{j+1} &= \widehat{\textbf{x}}_{k+1}^j\boxplus\textbf{x}_{j}
% \end{aligned}   
% \end{eqnarray}
% \normalsize
% If the state converges, the first optimal state $\bar{\textbf{x}}_{k+1}^1$ and covariance $\bar{\mathbf{P}}_{k+1}^1$ are updated and details can be found in \cite{he2021kalman}.

% \vspace{-3mm}\small
% \begin{eqnarray}
% \begin{aligned}
%   \label{eq:state update}
%     \bar{\textbf{x}}_{k+1}&=\widehat{\textbf{x}}_{k+1}^{\kappa+1} \\
%     \widehat{\mathbf{P}}_{k+1}^\kappa &= (\textbf{I}-\textbf{K}{\textbf{H})\textbf{J}_{k+1}^{-1}}\widehat{\mathbf{P}}_{k+1}{\textbf{H})\textbf{J}_{k+1}^{-\top}} \\    \bar{\mathbf{P}}_{k+1}&=\textbf{L}_{k+1}\widehat{\mathbf{P}}_{k+1}^\kappa\textbf{L}_{k+1}^\top
% \end{aligned} 
% \end{eqnarray}
% \normalsize
% $\textbf{L}_{k+1}$ is a Jacobian of $\delta \widehat{\textbf{x}}_{k+1}$ w.r.t. $\delta \textbf{x}_\kappa=\delta \textbf{x}_\kappa^0$. The details are in \cite{he2021kalman}.

% \subsubsection{Second Update}
% As mentioned in \secref{sec:gravity}, the gravity residual is used to correct the drift in the roll and pitch directions. This update exploits the optimal state $\bar{\textbf{x}}_{k+1}^1$ obtained from the first update. Since $\dim(r_g) < \dim(\mathcal{M})$, the Kalman gain is computed differently as $\textbf{K}=\mathbf{P}\textbf{H}^\top(\textbf{H}\mathbf{P}\textbf{H}^\top+\textbf{R})^{-1}$. However, the process is similar to that of the first update.
% \subsection{Mapping(Temp)}
% \label{subsec:Mapping}
% save point to IKD tree using static radar point + LiDAR

\section{Gravity-enhanced Radar-LiDAR Fusion}
\subsection{Radar-LiDAR Scan Synchronization}
\label{subsec:Scan_Synchronization}
Since 4D radar operates at a higher frequency than \ac{LiDAR}, we utilized sweep reconstruction \cite{SR-LIVO} to synchronize the end timestamp of \ac{LiDAR} and radar, using the radar's timestamp $r_k$. A sweep $S_{k+1}$ is constructed using \ac{LiDAR} points and \ac{IMU} measurements recorded between $r_k$ and $r_{k+1}$, along with radar captured at $r_{k+1}$. To address possible missing segments in the radar-based sweep reconstruction, we filled in missing \ac{FOV} sections of $S_{k+1}$ with data from the previous sweep $S_k$ to generate a complete \ac{LiDAR} scan. 
Once the sweep $S_{k+1}$ is constructed, state propagation is performed using \ac{IMU} measurements as \eqref{eq:state_predict}. \ac{LiDAR} points in $S_{k+1}$ are subsequently corrected with motion compensation to be transformed into the $\left ( \cdot \right )^{r_{k+1}}$ frame.

\subsection{Radar-guided LiDAR Sweep Preprocessing}
\label{subsec:Scan Filtering}
Before the sweep $S_{k+1}$ is utilized for the update module, a preprocessing step is conducted to remove dynamic points from the sweep based on radar Doppler measurements.
\subsubsection{Radar Point Cloud Pre-processing}
The 4D radar provides both 3D points and Doppler velocities, enabling it to compute 3D ego velocity in the $\left ( \cdot  \right )^{r_k}$ frame through the 3-Point RANSAC-LSQ method \cite{ekf-radar}. With this velocity, radar can filter out moving objects within a single frame without needing sequential data. 
This filtering segments the radar point cloud into static points, ${}^RP_s$, and dynamic points, ${}^RP_d$, which include both moving objects and noise.

\subsubsection{Dynamic Removal in LiDAR}
% Dynamic points filtered from the radar point cloud are used to remove dynamic points from the LiDAR point cloud, ${}^LP$. Compared to the LiDAR point cloud, the radar point cloud has a sparse density and exhibits uncertainty in the z-direction, making it difficult to establish accurate correspondences between points.
Dynamic points ${}^RP_d$ are exploited to remove dynamic points from the \ac{LiDAR} point cloud, ${}^LP$. However, due to the higher uncertainty in the z-direction and sparser nature of the radar point cloud used in this work, compared to the \ac{LiDAR} point cloud, establishing precise point correspondences becomes challenging.

To overcome this, we projected both the \ac{LiDAR} and radar dynamic point clouds onto the $xy$ plane and compared the two in 2D space. Using the pointwise uncertainty matrix $\Sigma_{r_{i}}$ \cite{4DRadarSLAM}, we computed the Mahalanobis distance between each \ac{LiDAR} point and the radar dynamic point as follows:

\vspace{-3mm} \small
\begin{eqnarray}
\begin{aligned}
    \label{eq:lidar_dynamic}
    d=(P_{XY}({}^Rp_i-{}^R\textbf{T}_L{}^Lp_j))^\top\bar{\Sigma}_{r_i}(P_{XY}({}^Rp_i-{}^R\textbf{T}_L{}^Lp_j)),
    % \\ \nonumber
    % P_{XY}=
    % \begin{bmatrix}
    %     1 & 0 & 0 \\
    %     0 & 1 & 0 \\
    %     0 & 0 & 0
    % \end{bmatrix}, \bar{\Sigma}_{r_i} = P_{XY}\Sigma_{r_i}, {}^RP_i\in{}^RP_d, {}^LP_j\in{}^LP.
\end{aligned}
\end{eqnarray}
\normalsize
where $P_{XY}$ is the projection matrix on $xy$ plane, $\bar{\Sigma}_{r_i} = P_{XY}\Sigma_{r_i}, {}^Rp_i\in{}^RP_d, {}^Lp_j\in{}^LP$. \ac{LiDAR} points within a threshold $d<\epsilon$ are considered dynamic and filtered out.

\subsection{Pointwise Residuals for First Stage Update}
\label{subsec:first_update}
In the first stage update, radar and \ac{LiDAR} measurements are integrated to update the state using pointwise residuals.

\subsubsection{Point-to-Plane Residual}
The residual is calculated using the dynamic-removed \ac{LiDAR} point cloud, ${}^LP_{k+1}$, obtained from \secref{subsec:Scan Filtering}. Each point ${}^Lp_j$ in ${}^LP_{k+1}$ is first transformed into the global frame. Then, five neighboring points are selected for each point using an ikd-tree \cite{cai2021ikd}, and the residual for each point is computed as follows:

\vspace{-4mm}\small
\begin{eqnarray}
\begin{aligned}
  \label{eq:point_to_plane residual}
    \textbf{h}_{L^j}(\textbf{x}_k,\textbf{v}_j^L)=&\\{}^Gu_j^\top({}^G\textbf{R}&_{I_k}({}^I\textbf{R}_L({}^Lp_j+\textbf{v}_j^L)+{}^Ip_L)+{}^Gp_{I_k}-{}^Gq_j),
\end{aligned}
\end{eqnarray}
\normalsize
where ${}^I\textbf{R}_L$, ${}^Ip_L$ are \ac{LiDAR}-\ac{IMU} extrinsic parameter, ${}^Gq_j$ is the centroid of five points, ${}^Gu_j$ is their normal vector and $\textbf{v}_j^L$ is a measurement noise of \ac{LiDAR}. Based on the residuals in \eqref{eq:point_to_plane residual}, the Jacobian matrix of ${\textbf{h}}_{L^j}$ w.r.t $\delta \widehat{\textbf{x}}_{k}$, $\textbf{H}_{L^j}$ is computed according to the definition of the invariant error: 

\vspace{-4mm}\small
\begin{eqnarray}
  \label{eq:LiDAR Measurement Model}
    % \textbf{h}_{L^j}(\textbf{x}_k,\textbf{v}_j^L)&=\textbf{h}_{L^j}(\widehat{\textbf{x}}_{k}+\delta\widehat{\textbf{x}}_k,\textbf{v}_j^L) \\
    % &=\textbf{h}_{L^j}(\widehat{\textbf{x}}_k,0)+\textbf{H}_{L^j}\delta\widehat{\textbf{x}}_k+n_{L,j} \\
    \textbf{H}_{L^j}=%\left.\frac{\partial\textbf{h}_{L^j}(\textbf{x}_k,\textbf{v}_j^L)}{\partial\delta\widehat{\textbf{x}}_k}\right\vert_{\delta\textbf{x}_k=0} \nonumber\\
    {}^Gu_j^\top\left[-\lfloor{}^Gp_{j}\rfloor_\times \quad \textbf{0}\quad\textbf{I}\quad\textbf{0}\quad\textbf{0}\right] \nonumber\\
    \textbf{r}_{L_j}+\textbf{H}_{L^j}\delta\widehat{\textbf{x}}_k\sim\mathcal{N}(0,R_j^L)
\end{eqnarray}
\normalsize
where ${}^Gp_{j}$ is \ac{LiDAR} point in global frame and we set $R_j^L$ as the same value.

\begin{figure}[!t]
    \centering
    \includegraphics[width=1\linewidth]{figure/double_integ.pdf}
    \vspace{-6mm}
    \caption{Gravity estimation using the velocity-ignorant (yellow) and velocity-aware method (ours, blue) on different platforms.}

    \label{fig:double integration}
    \vspace{-8mm}
\end{figure}

\subsubsection{Pointwise Velocity Residual}
The static radar points ${}^Rp_j$ are exploited as measurements to compute the velocity residual. The pointwise velocity residual is as follows:

\vspace{-4mm}\small
\begin{align}
    \textbf{h}_{R^j}(\textbf{x}_k,\textbf{v}_j^R)
    &= \textbf{u}({}^Rp_j)^\top {}^G\widehat{\textbf{v}}_R-{}^R\textbf{v}_{m,j} \\
    % &=\frac{{}^Rp_j^\top}{||{}^Rp_j||}{}^G\widehat{\textbf{v}}_R-{}^R\textbf{v}_{m,j} \\
    &=\textbf{u}({}^Rp_j)^\top{}^I\textbf{R}_R^\top\left({}^G\textbf{R}_{I_k}{}^G\textbf{v}_I+\lfloor \omega_I \rfloor_{\times}{}^Ip_R\right)-{}^R\textbf{v}_{m,j} \nonumber
  \label{eq:pointwise velocity residual}
\end{align}
\normalsize
where ${}^I\textbf{R}_R$ and ${}^I\textbf{p}_R$ are the radar-\ac{IMU} extrinsic parameters, and ${}^R\textbf{v}_{m,j}$ is the Doppler measurement at point ${}^R\textbf{p}_j$. The function $\textbf{u}(\cdot)$ represents the unit vector.

Since static radar points may still contain outliers, each static point's uncertainty is assessed using the modified z-score $\mathrm{M_j}$, which is based on deviation between ${}^R\textbf{v}_{m,j}$ and the estimated ego velocity from \secref{subsec:Scan Filtering}. Utilizing these residuals and uncertainties, the Jacobian of the pointwise velocity residual can be computed as follows:

\vspace{-4mm}\small
\begin{eqnarray}
\begin{aligned}
  \label{eq:radar measurement model}
    % \textbf{h}_{R^j}(\textbf{x}_k,\textbf{v}_j^R)&=\textbf{h}_{R^j}(\widehat{\textbf{x}}_{k}+\delta\widehat{\textbf{x}}_k,\textbf{v}_j^R) \nonumber\\
    % &=\textbf{h}_{R^j}(\widehat{\textbf{x}}_k,0)+\textbf{H}_{R^j}\delta\widehat{\textbf{x}}_k+n_{R,j}\\
    \textbf{H}_{R^j}&=%\left.\frac{\partial\textbf{h}_{R^j}(\textbf{x}_k,\textbf{v}_j^R)}{\partial\delta\widehat{\textbf{x}}_k}\right\vert_{\delta\textbf{x}_k=0} \nonumber\\
    \textbf{u}({}^Rp_j)^\top {}^I\textbf{R}_R^\top\left[\textbf{0}\quad{}^G\widehat{R}_{I_k}^\top\quad\textbf{0}\quad\lfloor{}^Ip_{R}\rfloor_\times\quad\textbf{0}\right] \nonumber\\
    \textbf{r}_{R^j}&+\textbf{H}_{R^j}\delta\widehat{\textbf{x}}_k=-n_{R,j}\sim\mathcal{N}(0,R_j^R) \nonumber
\end{aligned}
\end{eqnarray} \vspace{-3mm}
\begin{equation}
       R_j^R = 
        \begin{cases}
            R_v+\alpha\mathrm{M_j} & \text{if $\left|\mathrm{M_j}\right|>3.5 $} \\
            R_v & \text{otherwise}
        \end{cases}
\end{equation}
\normalsize
By incorporating two residuals, the state is updated according to \eqref{eq:MAP}, where $M$ includes $L$ and $R$. The optimal state $\bar{\textbf{x}}_{k+1}^1$ and covariance $\bar{\mathbf{P}}_{k+1}^1$ from the first stage update are then utilized in the second stage along with the gravity residual.

% \subsubsection{Iterated Update}
% By combining equations \eqref{eq:prior distribution}, \eqref{eq:LiDAR Measurement Model} and \eqref{eq:radar measurement model} the following maximum a posteriori can be yielded:
% \small
% \begin{align}
% \label{eq:MAP_first}
%   \min\limits_{\delta\textbf{x}_j} ( \lVert \textbf{x}_{k+1} \boxminus \widehat{\textbf{x}}_{k+1} \rVert ^2_{\textbf{P}}+\sum_{M = L, R}\sum^{|M|}_{j=1} \lVert \textbf{r}_{M^j} +\textbf{H}_{M^j}\delta \widehat{\textbf{x}}_k \rVert ^{2}_{{R_j^M}} )
% \end{align}
% \normalsize 
% Kalman filter update process iteratively continues until it converges to the optimal state $\bar{\textbf{x}}_{k+1}^1$ using $\mathbf{H}=\left[\mathbf{H}_{L^1},\cdots,\mathbf{H}_{R^1},\cdots\right]$, $\mathbf{R}=\operatorname{diag}\left[{R}_1^L,\cdots,{R}_1^R,\cdots\right]$.


\subsection{Gravity Residual for Second Stage Update}
\label{subsec:second_update}
As will be shown in this paper, incorporating radar velocity measurements into \ac{LIO} effectively enhances the state estimation. Furthermore, the velocity-aware approach enables more precise local gravity estimation using the optimal state of first update. By utilizing the \ac{IMU} preintegration with estimated biases, a velocity constraint $\beta_{k+1}^{k}$ for the time interval can be derived as follows \cite{qin2018vins}: 

\vspace{-4mm}\small
\begin{eqnarray}
  \label{eq:IMU preintegration}
    \beta_{k+1}^k=\int_{t\in[r_k,r_{k+1})}{}^{I_k}\textbf{R}_{I_t}(\widehat{a}_t-\widehat{b_a}_t)dt~,
\end{eqnarray}
\normalsize
where subscript ${I_t}$ denotes the \ac{IMU} frame at index $t$. Using $\beta_{k+1}^{k}$, we can establish the relationship between \ac{IMU} acceleration measurement and estimated velocity which was updated with radar measurements:

\vspace{-4mm}\small
\begin{eqnarray}
  \label{eq:gravity relation}
    \beta_{k+1}^k={}^{I_k}\textbf{R}_{I_{k+1}}{}^{I_{k+1}}\textbf{v}_{I_{k+1}}+{}^{I_k}\textbf{g}\Delta t-{}^{I_k}\textbf{v}_{I_k}
\end{eqnarray}
\normalsize

When solving for gravity $\textbf{g}$ and expressing it in terms of the state, it can be written as follows:

\vspace{-3mm}\small
\begin{align}
  \label{eq:gravity term}
   {}^{I_k}\widehat{\textbf{g}}&=\frac{\beta^{k}_{k+1}-{}^G\bar{\textbf{R}}_{I_k}^\top{}^G\widehat{\textbf{R}}_{I_k+1}{}^G\widehat{\textbf{R}}_{I_k+1}^\top{}^G\widehat{\textbf{v}}_{I_{k+1}}+{}^G\bar{\textbf{R}}_{I_k}^\top{}^G\bar{\textbf{v}}_{I_k}}{\Delta t} \nonumber \\
   {}^G\widehat{\textbf{g}}&={}^G\bar{\textbf{R}}_{I_k}{}^{I_k}\widehat{\textbf{g}}
   =\frac{{}^G\bar{\textbf{R}}_{I_k}\beta^k_{k+1}-{}^G\widehat{\textbf{v}}_{I_{k+1}}+{}^G\bar{\textbf{v}}_{I_k}}{\Delta t}
\end{align}
\normalsize

World frame gravity ${}^G\textbf{g}$ is calculated through static initialization using \ac{IMU} acceleration. By calculating the residual between ${}^G\textbf{g}$ and the predicted gravity ${}^G\widehat{\textbf{g}}_r$ on the $\mathcal{S}^2$ manifold from ${}^G\widehat{\textbf{g}} \in \mathbb{R}^3$, compensating the deviation of roll and pitch directions is possible. Gravity residual is defined using cosine similarity, and its Jacobian is represented as follows:

\vspace{-4mm}\small
\begin{align}
  \label{eq:gravity measurement model}
    \textbf{h}_g(\textbf{x}_k,\textbf{v}_k^g)&=1-{}^G\textbf{g}^\top{}^G\widehat{\textbf{g}}_r(\textbf{x}_k,\textbf{v}_k^g) 
    % &=\textbf{h}_g(\widehat{\textbf{x}}_{k}+\delta\widehat{\textbf{x}}_k,\textbf{v}_g) \nonumber\\
    =\textbf{h}_g(\widehat{\textbf{x}}_k,0)+\textbf{H}_{g}\delta\widehat{\textbf{x}}_k+n_{g} \nonumber \\
    \textbf{H}_{g}&=%\left.\frac{\partial\textbf{h}_{g}(\textbf{x}_k,\textbf{v}_g)}{\partial\delta\widehat{\textbf{x}}_k}\right\vert_{\delta\textbf{x}_k=0} \nonumber\\
    \frac{-1}{||{}^G\widehat{\textbf{g}}_r||\Delta t}\left[ (\lfloor{}^G\textbf{g}\rfloor_\times{}^G\widehat{\textbf{v}}_{I_{k+1}})^\top\quad -{}^G\textbf{g}^\top\quad\textbf{0}\quad\textbf{0}\quad\textbf{0}\right] \nonumber\\
    \textbf{r}_{g}&+\textbf{H}_{g}\delta\widehat{\textbf{x}}_k=-n_{g}\sim\mathcal{N}(0,R_g)
\end{align}
\normalsize

Since $\dim(r_g) < \dim(\mathcal{M})$, the Kalman gain is computed differently as $\textbf{K}=\mathbf{P}\textbf{H}^\top(\textbf{H}\mathbf{P}\textbf{H}^\top+\textbf{R})^{-1}$. Apart from this difference, the process remains similar to the first update.
