%% gravity 썻을때 안썼을때 map 비교
\begin{table}[t]
\scriptsize
\caption{Effect of Velocity Measurement on Gravity Estimation Evaluated on Car Sequences from the NTU4DRadLM and Snail-Radar Datasets.}
\label{tab:velocity measurement}
\vspace{-1mm}
\centering
\resizebox{0.48\textwidth}{!}{%
\begin{tabular}{c|cc|cc|cc} \toprule
         & \multicolumn{2}{c|}{\textit{loop2}} & \multicolumn{2}{c|}{\textit{if}} & \multicolumn{2}{c}{\textit{sl}} \\\cmidrule[0.4pt](r{0.125em}){1-1} \cmidrule[0.4pt](r{0.125em}){2-7}
        & mean        & std         & mean           & std            & mean           & std            \\\cmidrule[0.4pt](r{0.125em}){1-1}\cmidrule[0.4pt](r{0.125em}){2-3} \cmidrule[0.4pt](r{0.125em}){4-5} \cmidrule[0.4pt](r{0.125em}){6-7}
\texttt{w/ vel}     & \textbf{0.035}     & \textbf{0.053}     & \textbf{0.036}        & \textbf{0.030}        & \textbf{0.164}         & \textbf{0.123}         \\
\texttt{w/o vel} & 0.037     & 0.055     & 0.043        & 0.031        & 0.167         & 0.123        \\
\bottomrule
\end{tabular}%
}
% \tiny{ }

% \footnotesize
% \raggedright
% Best performance in \textbf{bold}.  
\vspace{-10mm}
\end{table}


\section{experiment}
\label{sec:experiment}

\subsection{Datasets and Evaluation Metric}
% Our algorithm was evaluated using the NTU4DRadLM dataset \cite{zhang2023ntu4dradlm}, the Snail-Radar dataset  \cite{huai2024snail} and $\textit{Fog-Filled hallway}$ sequence of the DR-LRIO \cite{DR-LRIO}. All these datasets provide 4D radar, \ac{LiDAR}, and \ac{IMU} data across various sequences. The NTU4DRadLM dataset includes a solid-state LiDAR, while the Snail-Radar and $\textit{Fog-Filled hallway}$ dataset employs a spinning LiDAR, resulting in differences in \ac{FOV}. 


% The NTU4DRadLM dataset consists of a total of six sequences. The $\textit{loop*}$ sequences involve a car platform driving in a semi-structured environment at 25-30 km/h. The $\textit{cp}$ and $\textit{nyl}$ sequences are recorded using a handcart platform moving at 1 m/s. $\textit{loop1}$ and $\textit{garden}$ sequences are not used for evaluation due to the absence of some raw data.

% The NTU4DRadLM dataset includes six sequences: $\textit{loop*}$ sequences feature a car platform driving at 25-30 km/h in a semi-structured environment, while $\textit{cp}$ and $\textit{nyl}$ are recorded with a handcart moving at 1 m/s. $\textit{loop1}$ and $\textit{garden}$ sequences are excluded from evaluation due to missing raw data.

% The Snail-Radar dataset is collected using handheld, e-bike, and SUV, capturing data under adverse weather conditions like heavy rain. Compared to NTU4DRadLM, it features more structured and dynamic environments. 

% The $\textit{Fog-Filled hallway}$ sequence was collected using a drone in a geometrically uninformative environment. Due to the absence of a ground truth trajectory, this dataset was used exclusively for gravity estimation experiments.

% For the gravity estimation experiments, the difference from the ground truth was calculated using the $\mathbb{S}^2 \boxminus$ operation \cite{he2021kalman}. The global gravity ground truth was obtained through stationary initialization.
% We compared it with three state-of-the-art LIO algorithms: FAST-LIO2 \cite{FAST-LIO}, Point-LIO \cite{POINT-LIO}, and DLIO \cite{DLIO}. The comparison was conducted by calculating the \ac{RMSE} of the absolute trajectory error (ATE) using the evo \cite{grupp2017evo}. Translation error is measured in meters, and angle error in degrees.

We evaluated GaRLIO on three datasets: NTU4DRadLM \cite{zhang2023ntu4dradlm}, which uses a solid-state LiDAR, and Snail-Radar \cite{huai2024snail} and $\textit{Fog-Filled hallway}$ \cite{DR-LRIO}, both employing spinning LiDARs.
% We evaluated GaRLIO on three datasets: NTU4DRadLM \cite{zhang2023ntu4dradlm}, Snail-Radar \cite{huai2024snail}, and $\textit{Fog-Filled hallway}$ \cite{DR-LRIO}. 
% All datasets provide 4D radar, LiDAR, and IMU, with 
% NTU4DRadLM uses a solid-state LiDAR and the others employing spinning LiDARs.
% , resulting in different \ac{FOV}.
% The NTU4DRadLM dataset consists of six sequences: the $\textit{loop*}$ sequences involve a car platform moving at 25-30 km/h in semi-structured environments, while $\textit{cp}$ and $\textit{nyl}$ are recorded with a handcart at 1 m/s. We excluded $\textit{loop1}$ and $\textit{garden}$ from evaluation due to missing raw data. Snail-Radar was collected using handheld devices, e-bikes, and SUVs under adverse weather conditions like heavy rain, presenting more structured and dynamic environments compared to NTU4DRadLM. Finally, the $\textit{Fog-Filled hallway}$ sequence, gathered with a drone in a geometrically uninformative setting, was exclusively used for gravity estimation experiments due to the absence of a ground truth trajectory.
NTU4DRadLM, collected using handcart and car in semi-structured environments, includes $\textit{loop*}$ sequences with significant elevation changes. Snail-Radar, collected with handheld devices, e-bikes, and SUVs in adverse weather like heavy rain, presents high dynamics in structured environments. $\textit{Fog-Filled hallway}$, gathered by drone in a geometrically uninformative area, was used only for gravity evaluation due to the lack of ground truth.
% We excluded $\textit{loop1}$ and $\textit{garden}$ from evaluation due to missing raw data.


We compared GaRLIO with the three state-of-the-art LIOs: FAST-LIO2 \cite{FAST-LIO}, Point-LIO \cite{POINT-LIO}, and DLIO \cite{DLIO}. Although it would be ideal to compare our method with other radar-LiDAR-Inertial odometry approaches such as DR-LRIO \cite{DR-LRIO}, the unavailability of their code has precluded such evaluations. Consequently, we selected our comparison targets from established LIO algorithms.

We computed the \ac{RMSE} of the \ac{ATE} using evo \cite{grupp2017evo}, with translation errors in meters and angular errors in degrees. Results are \textbf{bold} for best, and \underline{underlined} for second best. Legends for \figref{fig:iaef_traj} and \figref{fig:sl_sequence} are the same as in \figref{fig:loop3}.

% Please add the following required packages to your document preamble:
% \usepackage{multirow}
\begin{table}[t]
\scriptsize
\caption{NTU4DRadLM Dataset Evaluation}
\label{tab:NTU4DRadLM}
\vspace{-1mm}
\centering
\resizebox{0.45\textwidth}{!}{%
\begin{tabular}{cccccc} \toprule
           \multicolumn{1}{l}{}        &            & Fast LIO2 & Point LIO & DLIO   & GaRLIO
\\ \midrule
\multirow{2}{*}{\textit{cp}}         & \multicolumn{1}{|c|}{ATE$_t$} &  \textbf{0.103}    & 0.162    & 0.174 & \ul{0.107} \\
                            & \multicolumn{1}{|c|}{ATE$_r$}   &  \textbf{0.574}    & \ul{0.706}    & 1.697  & 0.750
\\ \midrule

\multirow{2}{*}{\textit{nyl}}        & \multicolumn{1}{|c|}{ATE$_t$} & 1.259     &  \textbf{0.507}    & 1.214  & \ul{0.973} \\
                            & \multicolumn{1}{|c|}{ATE$_r$}   & \ul{0.996}    &  \textbf{0.722}    & 1.688  & 1.454 \\
\midrule

\multirow{2}{*}{\textit{loop3}}      & \multicolumn{1}{|c|}{ATE$_t$} & 11.94     & 9.548     & \ul{8.332}  &  \textbf{4.271}  \\
                            & \multicolumn{1}{|c|}{ATE$_r$}   & \ul{1.714}     & 2.082     & 2.101  &  \textbf{1.496}  \\
\midrule

\multirow{2}{*}{\textit{loop2}}      & \multicolumn{1}{|c|}{ATE$_t$} & 15.81     & 20.27     & \ul{10.38}  &  \textbf{3.976}  \\
                            & \multicolumn{1}{|c|}{ATE$_r$}   & \ul{2.176}     & 4.188     & 2.525  &  \textbf{1.043}  \\ 
\bottomrule
\end{tabular}%
}
\vspace{-4mm}
\end{table}
\begin{figure}[!t]
    \centering
    \includegraphics[trim=0cm 0.2cm 0cm 0cm, clip, width=0.97\columnwidth]{figure/plot_xy_loop3.pdf}
    \caption{Estimated $z$ (top) and $xy$ (bottom) trajectory for the $\textit{loop3}$ sequence. $\textbf{Top}$ : From 120 to 350 seconds, GaRLIO accurately predicts elevation during the downhill descent, outperforming other methods. $\textbf{Bottom}$ : GaRLIO results closely match the ground truth. The black dot represents the start point. }
    \label{fig:loop3}
    \vspace{-7mm}
\end{figure}

\subsection{Evaluation on Velocity-aware Gravity Estimation}
\label{subsec:gravity test}
% We first validated the performance of our velocity-aware approach in gravity estimation.
We performed a comparative analysis between our velocity-aware gravity estimation method and velocity-ignorant approach that utilizes double integration. The evaluation was performed across different platforms, including \ac{UGV} $(\textit{loop2}$), and drone ($\textit{Fog-Filled hallway}$). For gravity evaluation, we measured the deviation from the ground truth using the $\boxminus$ operation on $\mathcal{S}^2$ \cite{he2021kalman}. The global gravity ground truth is obtained through stationary initialization. As illustrated in \figref{fig:double integration}, our method exhibits robust gravity estimation results, maintaining minimal deviation. The disparity between velocity-ignorant and velocity-aware methods is more significant on the \ac{UGV} platform. 
\ac{UGV} introduces significant bias and noise in \ac{IMU} measurements arise from contact friction. These errors are exacerbated due to double integration, thus resulting in inaccuracy. In conclusion, our velocity-aware approach is more feasible for predicting gravity.

The effect of leveraging radar measurements on gravity estimation is depicted in \tabref{tab:velocity measurement}. Exploiting velocity updates with radar measurement consistently enhances the accuracy of estimated gravity, thus highlighting the effectiveness of radar measurement in gravity estimation.

\subsection{NTU4DRadLM Dataset}
\label{subsec:ntu}

The evaluation results for the NTU4DRadLM dataset are shown in \tabref{tab:NTU4DRadLM}. In the \textit{loop2} and \textit{loop3} sequences, where data were collected using a car, our method outperformed other LIO methods in both translation and rotation. \figref{fig:loop3} shows the qualitative results in \textit{loop3}. While other baselines show degraded performance due to steep uphill and downhill segments, GaRLIO achieves accurate estimation results in such challenging scenarios. This exhibits the robustness of our velocity-aware gravity estimation, even with significant elevation changes.
However, in $\textit{nyl}$, Point-LIO achieved the best performance.
In \textit{nyl}, the data was collected using a handcart, which causes significant vibrations.
Point-LIO effectively handles these noisy measurements by estimating acceleration and angular velocity as part of the state.
GaRLIO is not specifically designed for such conditions, although it achieves comparable performance.

% 
% \begin{figure}[!t]
%     \centering
%     \includegraphics[width=1\linewidth]{figure/xyplot.pdf}
%     \caption{Estimated trajectory for the $\textit{loop2}$ sequence. Black dot represent start point and the endpoint is highlighted in detail.
%  }
%     \label{fig:loop2(xy)}
%     \vspace{-6mm}
% \end{figure}

% Please add the following required packages to your document preamble:
% \usepackage{multirow}
\begin{table}[t]
\scriptsize
\caption{Snail Radar Dataset Evaluation}
\label{tab:Snail-Radar}
\vspace{-1mm}
\centering
\resizebox{0.45\textwidth}{!}{%
\begin{tabular}{cccccc} \toprule
           \multicolumn{1}{l}{}        &            & Fast LIO2 & Point LIO & DLIO   & GaRLIO \\ \midrule
\multirow{2}{*}{ \textit{sl}} & \multicolumn{1}{|c|}{ATE$_t$} & 14.34     & 10.98     & \ul{5.054}  &  \textbf{1.059}  \\
                            & \multicolumn{1}{|c|}{ATE$_r$}   & 3.318     & 3.321     & \ul{2.718}  &  \textbf{1.170}  \\ \midrule
\multirow{2}{*}{\textit{if}} & \multicolumn{1}{|c|}{ATE$_t$} & 5.747     & 9.018     &  \textbf{0.308} & \ul{0.450} \\
                            & \multicolumn{1}{|c|}{ATE$_r$}   & 1.479     & 3.291     & \ul{0.730} &  \textbf{0.591} \\ \midrule
\multirow{2}{*}{\textit{iaf}} & \multicolumn{1}{|c|}{ATE$_t$} & 24.99     & 73.19     & \ul{6.192}  &  \textbf{4.672}  \\
                            & \multicolumn{1}{|c|}{ATE$_r$}   & 2.826     & 9.542     & \ul{1.560}  &  \textbf{1.539}  \\ \midrule
\multirow{2}{*}{\textit{iaef}} & \multicolumn{1}{|c|}{ATE$_t$} & 30.26     & 90.81     & \ul{15.86}  &  \textbf{6.450}  \\
                            & \multicolumn{1}{|c|}{ATE$_r$}   & 3.317     & 9.004     & \ul{1.436}  &  \textbf{1.397}  \\
\bottomrule
\end{tabular}%
}
\vspace{-3mm}
\end{table}
\begin{figure}[!t]
    \centering
    \includegraphics[trim=0cm 0.4cm 0cm 0cm, clip,width=0.97\columnwidth]{figure/xyplot_zoom_snail.pdf}
    \caption{Estimated trajectory for the $\textit{iaef}$ sequence and GaRLIO demonstrated superior performance over such a long sequence. 
 }
    \label{fig:iaef_traj}
    \vspace{-7mm}
\end{figure}




\subsection{Snail-Radar dataset}
\label{subsec:snail}

% For the Snail Radar dataset, sequences collected using the E-Bike ($\textit{sl}$) and SUV ($\textit{if, iaf, iaef}$) were used for evaluation.

% Notably, in the $\textit{iaef}$ sequence, which includes a tunnel, the use of velocity residuals enables robust state estimation.
As shown in \tabref{tab:Snail-Radar} and \figref{fig:iaef_traj}, our method outperforms other LIO algorithms throughout the experiments. 
Specifically, in the $\textit{sl}$ sequence, other baselines exhibit significant errors in the $z$-axis due to the vigorous dynamic movements of the e-bike as shown in \figref{fig:sl_sequence}. Additionally, the Snail-Radar dataset contains various dynamic objects, such as vehicles and pedestrians, which affected the state predictions of FAST-LIO2 and Point-LIO, leading to degraded RMSE performance.
Conversely, GaRLIO effectively mitigates drift in the roll and pitch directions through the velocity-aware gravity residuals, resulting in robust elevation estimation. Furthermore, GaRLIO successfully removes dynamic objects within the LiDAR by leveraging radar, as shown \figref{fig:dynamic}, resulting in more reliable performance compared to other methods.
DLIO shows second-best performance across most sequences attributed to Geometric Observer and a continuous-time method. However, the absence of a dynamic handling mechanism and an update module based on velocity information reveals limitations along the experiments.

\begin{figure}[!t]
    \centering
    \includegraphics[trim=0cm 0.3cm 0cm 0cm, clip, width=1\columnwidth]{figure/plot_snail.pdf} 
    \caption{$z$, pitch, and roll error in $\textit{sl}$. The RMSE values for $z$, pitch, and roll are above each plot. Our method shows the lowest error along the baselines.
 }
    \label{fig:sl_sequence}
    \vspace{-3mm}
\end{figure}
\begin{figure}[]
    \centering
    \includegraphics[width=0.95\columnwidth]{figure/dynamic_remove.pdf}
    \caption{Result of Dynamic removal module in Snail-Radar. The red points in the right LiDAR point cloud represent dynamic points that were removed using radar-based dynamic point.}
    \label{fig:dynamic}
    \vspace{-7mm}
\end{figure}


\subsection{Ablation Study}
\label{subsec:ablation}

To evaluate the effect of each residual and dynamic removal module, we conducted experiments using various platforms, including a handcart (\textit{cp}, \textit{nyl}), a car (\textit{loop2}, \textit{loop3}), and an e-bike (\textit{sl}). We created five variants by disabling two of the residuals ($\texttt{g\&v}$) and Dynamic removal module ($\texttt{d}$): $\texttt{w/o g\&v\&d}$, $\texttt{w/o g\&v}$, $\texttt{w/o v}$, $\texttt{w/o g}$, $\texttt{FULL}$, evaluated in terms of \ac{ATE}.
% The results of the ablation study are presented in \tabref{tab:Ablation}.
% We observed that, except for the $\textit{cp}$ sequence, the variant that included both residuals had a lower RMSE compared to the others.

\subsubsection{Effect of Velocity Residual}
As depicted in \tabref{tab:Ablation}, incorporating velocity residuals significantly enhances performance. This result corroborates the discussion in \secref{subsec:gravity test}, demonstrating that incorporating velocity measurements facilitates more precise gravity estimation, thereby improving odometry estimation accuracy. 
Furthermore, the \textit{loop3} and \textit{loop2} sequences,
which were collected using a car with higher speeds,
demonstrate substantial improvements in state estimation through the incorporation of velocity residuals compared to \textit{cp}, \textit{nyl}, and \textit{sl}, which are the slower sequences. This improvement is attributed to the inherent limitations of existing LIO methods that depend on kinematic models for velocity estimation, which are less accurate than our method at higher speeds.

\begin{figure}[]
    \centering
    \includegraphics[width=0.95\columnwidth]{figure/gravity.pdf}
    \caption{(a) Mapping result of the $\textit{if}$ sequence generated using GaRLIO. Our method shows accurate alignment when revisiting previously traversed areas. At the revisited starting point: (b) With gravity residuals, ground regions are accurately aligned; (c) Without gravity residuals, vertical drift becomes apparent.}
    \label{fig:gravity}
    \vspace{-2mm}
\end{figure}

\section{Discussion}
\subsection{Limitation of rotation prediction task}
\begin{figure}[t!]
    \centering
    \begin{subfigure}{0.1\textwidth}
        \includegraphics[width=\linewidth]{figure/ship_sample.png}
        \caption{"Ship"}
        \label{fig:ship_sample}
    \end{subfigure}
    \vfill
    \begin{subfigure}{0.22\textwidth}
        \includegraphics[width=\linewidth]{figure/all_classes_main_likelihood_ship.pdf}
        \caption{Likelihood of all 10 classes over steps}
        \label{fig:all_classes_main_likelihood_hard_ship}
    \end{subfigure}
    \begin{subfigure}{0.22\textwidth}
        \includegraphics[width=\linewidth]{figure/all_classes_aux_likelihood_ship.pdf}
        \caption{Likelihood of all 4 rotation classes over steps}
        \label{fig:all_class_likelihood_aux_hard_ship}
    \end{subfigure}
    \caption{Likelihood across iteration of a level 5 Gaussian Noise sample in class "Ship"}
    \label{fig:likelihood_ship}
\end{figure}
Figure \ref{fig:likelihood_ship} illustrates the likelihood of different classes across iterations, using a sample (Figure \ref{fig:ship_sample}) from the "Ship" class with a rotation angle of 0 degrees. The sample's features, which include many edge features, allow the model to predict rotation effectively and maintain stability over iterations (Figure \ref{fig:all_class_likelihood_aux_hard_ship}). This stability and accuracy in rotation prediction enable the recurrent model to better estimate the optimal iteration for the main task.

However, predicting the rotation angle becomes more challenging for samples with fewer edge features or isotropic characteristics, such as the sample from the "Cat" class in Figure \ref{fig:cat_sample}. Figure \ref{fig:all_classes_main_likelihood_hard_cat}, \ref{fig:all_class_likelihood_aux_hard_cat} shows that while the likelihood of the "Cat" class continues to increase over iterations and remains significantly higher than other classes, the likelihood of the ground truth rotation (0 degrees) in the self-supervised task is very low . Instead, the model tends to predict the class corresponding to a 270-degree rotation. This behavior negatively impacts the estimation of the optimal iteration for the main task based on the self-supervised task.

We acknowledge this as a limitation of using the rotation prediction task as a self-supervised task.

\subsection{Converge to a fixed point does not ensure to mitigate "overthinking"}

\cite{bansal2022endtoend} highlights the relationship between changes in feature maps across iterations and the issue of "overthinking".
Specifically, they showed that for recurrent models where the norm $\|h_t - h_{t-1}\|$ converges to $0$, it can be assumed that the feature map has converged to a fixed point, and thus the prediction results of iterations beyond this fixed point remain unchanged, effectively addressing "overthinking." 

However, we observe that this assumption is not entirely accurate. Figure \ref{fig:visualize norm} illustrates that the norm $\|h_t - h_{t-1}\|$ of Conv-GRU converges to $0$ after more than $20$ iterations. Nevertheless, Figure \ref{fig:visualize loss} reveals that both the classification loss and self-supervised loss of the model exhibit divergence, corresponding to Conv-GRU encountering "overthinking" on corruption test sets, as shown in Figure \ref{fig:cnn_gru_gn_alpha_0.0}. 

In contrast, Conv-LiGRU demonstrates greater stability. Specifically, not only does the norm $\|h_t - h_{t-1}\|$ converge to $0$ (Figure \ref{fig:visualize norm}), but both the main loss and auxiliary loss also converge smoothly to $0$ (Figure \ref{fig:visualize loss}). Additionally, Figure \ref{fig:cnn_ligru_alpha_0.0} shows that Conv-LiGRU significantly mitigates the "overthinking" phenomenon compared to Conv-GRU. 

In conclusion, we assert that the convergence of $\|h_t - h_{t-1}\|$ alone does not guarantee that the model is free from "overthinking."

\begin{figure}[t!]
    \centering
    \begin{subfigure}{0.1\textwidth}
        \includegraphics[width=\linewidth]{figure/cat_sample.png}
        \caption{"Cat"}
        \label{fig:cat_sample}
    \end{subfigure}
    \vfill
    \begin{subfigure}{0.22\textwidth}
        \includegraphics[width=\linewidth]{figure/all_classes_main_likelihood_cat.pdf}
        \caption{Likelihood of all 10 classes over steps}
        \label{fig:all_classes_main_likelihood_hard_cat}
    \end{subfigure}
    \begin{subfigure}{0.22\textwidth}
        \includegraphics[width=\linewidth]{figure/all_classes_aux_likelihood_cat.pdf}
        \caption{Likelihood of all 4 rotation classes over steps}
        \label{fig:all_class_likelihood_aux_hard_cat}
    \end{subfigure}
    \caption{Likelihood across iteration of a level 5 Gaussian Noise sample in class "Cat"}
    \label{fig:likelihood_cat}
\end{figure}

\begin{figure*}[htbp]
    \centering
    \begin{subfigure}{0.4\textwidth}
        \includegraphics[width=\linewidth]{figure/visualize_norm.pdf}
        \caption{The $\|h_t - h_{t-1}\|$ across iterations}
        \label{fig:visualize norm}
    \end{subfigure}
    \hfill
    \begin{subfigure}{0.55\textwidth}
        \includegraphics[width=\linewidth]{figure/visualize_loss.pdf}
        \caption{The loss value across iterations}
        \label{fig:visualize loss}
    \end{subfigure}
    \caption{\textbf{Left:} The change in norm of feature maps of Conv-GRU and Conv-LiGRU. \textbf{Right:} Loss value across iterations of Conv-GRU and Conv-LiGRU}
    \label{fig:layer_norm}
\end{figure*}
% \begin{algorithm}[H]
% \caption{Incremental Progress Training Algorithm}
% \label{alg:incremental_progress}
% \begin{algorithmic}[1]
% \Require Parameter vector $\theta$, integer $m$, weight $\alpha$
% \For{batch\_idx = 1, 2, \dots}
%     \State Choose $n \sim U\{0, m-1\}$ and $k \sim U\{1, m-n\}$
%     \State Compute $\phi_n$ with $n$ iterations without tracking gradients
%     \State Compute $\hat{y}_{\text{prog}}$ with additional $k$ iterations
%     \State Compute $\hat{y}_m$ with a new forward pass of $m$ iterations
%     \State Compute $\mathcal{L}_{\text{max\_iters}}$ with $\hat{y}_m$
%     \State Compute $\mathcal{L}_{\text{progressive}}$ with $\hat{y}_{\text{prog}}$
%     \State Compute $\mathcal{L} = (1 - \alpha) \cdot \mathcal{L}_{\text{max\_iters}} + \alpha \cdot \mathcal{L}_{\text{progressive}}$
%     \State Compute $\nabla_\theta \mathcal{L}$ and update $\theta$
% \EndFor
% \end{algorithmic}
% \end{algorithm}

% \begin{figure*}[htbp]
%     \centering
%     \begin{subfigure}{0.48\textwidth}
%         \includegraphics[width=\linewidth]{figure/cnn_gru_gn_alpha_0.1.png}
%         \caption{Caption for Graph 2}
%         \label{fig:cnn_gru_gn_alpha_0.1}
%     \end{subfigure}
%     \hfill
%     \begin{subfigure}{0.48\textwidth}
%         \includegraphics[width=\linewidth]{figure/cnn_ligru_alpha_0.1.png}
%         \caption{Caption for Graph 2}
%         \label{fig:cnn_ligru_alpha_0.1}
%     \end{subfigure}
%     \caption{A horizontal arrangement of three graphs.}
%     \label{fig:alpha0.1}
% \end{figure*}


% To mitigate the problem of overthinking in recurrent models, \cite{bansal2022endtoend} proposed progressive loss (\ref{alg:incremental_progress}). We trained Conv-GRU and Conv-LiGRU using \ref{alg:incremental_progress} with $\alpha = 0.1$. 

% Figures \ref{fig:cnn_gru_gn_alpha_0.1} and \ref{fig:cnn_ligru_alpha_0.1} indicate that progressive loss does not limit overthinking in the Conv-GRU and Conv-LiGRU models and also reduces the accuracy stability between iterations. However, we figure out that the Progressive Loss can buff the accuracy of RNNs (Table 3). 

\subsubsection{Effect of Gravity Residual}
When using gravity residuals without the addition of velocity measurements, a decline in performance was observed in $\textit{loop3}$ and $\textit{loop2}$, as indicated by the comparison between $\texttt{w/o g\&v}$ and $\texttt{w/o v}$. 
As shown in \tabref{tab:velocity measurement}, GaRLIO relies on the accuracy of the velocity for precise gravity estimation, which likely leads to performance degradation in the absence of velocity measurements. 
Accurate gravity estimation through velocity updates significantly improves odometry performance, as demonstrated by the comparison between $\texttt{w/o g}$ and $\texttt{FULL}$. These enhancements are particularly noticeable in elevation estimation, as seen in \figref{fig:gravity}.

\subsubsection{Effect of Dynamic Removal Module}
The impact of the dynamic removal module can also be observed in \tabref{tab:Ablation}.
Removing dynamic objects using radar point clouds improves performance in most sequences, with particularly notable enhancements in those containing numerous dynamic objects, such as \textit{loop3}, \textit{loop2}, and \textit{sl}.
These results demonstrate that the proposed algorithm is robust in environments with a high density of dynamic points.
