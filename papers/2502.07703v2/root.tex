

%%%%%%%%%%%%%%%%%%%%%%%%%%%%%%%%%%%%%%%%%%%%%%%%%%%%%%%%%%%%%%%%%%%%%%%%%%%%%%%%
%2345678901234567890123456789012345678901234567890123456789012345678901234567890
%        1         2         3         4         5         6         7         8

\documentclass[letterpaper, 10 pt, conference]{ieeeconf}  % Comment this line out if you need a4paper
% \usepackage{hyperref}
%\documentclass[a4paper, 10pt, conference]{ieeeconf}      % Use this line for a4 paper

\IEEEoverridecommandlockouts                              % This command is only needed if
                                                          % you want to use the \thanks command

\overrideIEEEmargins                                      % Needed to meet printer requirements.
% \usepackage{hyperref}
% this is to use \citeauthor comment (not compatible with cite package)
% natbib hack
% http://newsgroups.derkeiler.com/Archive/Comp/comp.text.tex/2006-02/msg00834.html
\makeatletter
\let\NAT@parse\undefined
\makeatother
\usepackage[numbers,sort&compress]{natbib}

% See the \addtolength command later in the file to balance the column lengths
% on the last page of the document
%\makeatletter
%\let\NAT@parse\undefined
%\makeatother

% The following packages can be found on http:\\www.ctan.org
\usepackage[pdftex]{graphicx}
\usepackage{amsmath}
\usepackage{amssymb}  % assumes amsmath package installed
\usepackage{subfigure}
\usepackage{subcaption}
\usepackage{multirow}
\usepackage{array,booktabs}
\usepackage{diagbox}
\usepackage{balance}
\usepackage{xspace}
\usepackage{algorithm}
\usepackage{algpseudocode}
%\usepackage{kotex} %한글
%\usepackage{enumitem}
%\usepackage{slashbox}
\usepackage{adjustbox}
\usepackage{romannum}
%\captionsetup[subfigure]{labelformat=empty}

\usepackage[final]{hyperref}
\hypersetup{
 colorlinks=true,
 linkcolor=blue,
 filecolor=magenta,
 urlcolor=blue,
 citecolor=black
}
\usepackage{cleveref}
\crefname{figure}{Fig.}{Figs.}
\Crefname{figure}{Fig.}{Figs.}
\usepackage[font=small]{caption}
\usepackage{rpm_SIunits}
\usepackage{rpm_acronyms}
\usepackage{rpm_math}
\usepackage{rpm_misc}


% for comments
\usepackage{soul,color}
\usepackage{lipsum}


% To highlight the revised MS
\usepackage{xcolor}
\newcommand{\gr}[1]{{\textcolor{green}{#1}}}
\newcommand{\bl}[1]{{\textcolor{blue}{#1}}}
\newcommand{\B}[1]{{\textbf{#1}}}
\newtheorem{theorem}{Theorem}

\DeclareMathOperator*{\argmin}{argmin}
\DeclareMathOperator{\Tr}{Tr}

\usepackage{tikz} % \checkmark
\usepackage{subcaption}
\usepackage{caption}
\usepackage[font=scriptsize]{caption}
\title{\LARGE \bf GaRLIO: Gravity enhanced Radar-LiDAR-Inertial Odometry
}     

\author{Chiyun Noh${}^{1}$, Wooseong Yang${}^{1}$, Minwoo Jung${}^{1}$, Sangwoo Jung${}^{1}$ and Ayoung Kim${}^{1*}$
\thanks{$^\dagger$This work was supported by the National Research Foundation of Korea(NRF) grant funded by the Korea government(MSIT) (No. RS-2023-00241758), and in part by the Robotics and AI (RAI) Institute.}
\thanks{$^{1}$C. Noh, W. Yang, M. Jung, S. Jung and A. Kim are with the Dept. of Mechanical Engineering, SNU, Seoul, S. Korea {\tt\small [gch06208, yellowish, moonshot, dan0130, ayoungk]@snu.ac.kr}}%
}
\begin{document}

%\onecolumn
\maketitle
\thispagestyle{empty}
\pagestyle{empty}


\begin{abstract}

% Recent works to jointly reconstruct 3D human and object from a single RGB image, are mostly model-based, that fail to capture the fine details of the clothed human body and object surface. In this paper, we introduce ReCHOR, a novel, model-free, first-method to produce realistic clothed human-object reconstructions from a monocular view. This is extremely challenging due to human-object occlusions, diverse interactions and depth ambiguity, as it needs to infer both 3D spatial awareness and high resolution details. Our core idea is based on estimating neural implicit representations for human and object respectively by an attention-based neural implicit model that attends to pixel-aligned features from both the global human-object image for spatial awareness and  the local separate view of human and object images for high quality details. Additionally, the network is conditioned on semantic features from an initial estimated human-object pose prior and a generative diffusion model that inpaints occluded regions, thus enabling the retrieval of details from them.
% We also propose a synthetic dataset with rendered scenes of diverse, inter-occluded 3D human and object scans, to train our network. We evaluate our method on the synthetic and real world BEHAVE dataset. Our experiments show that our method outperforms the SOTA in achieving realistic clothed human-object reconstructions.
Recent approaches to jointly reconstruct 3D humans and objects from a single RGB image represent 3D shapes with template-based or coarse models, which fail to capture details of loose clothing on human bodies. In this paper, we introduce a novel implicit approach for jointly reconstructing realistic 3D clothed humans and objects from a monocular view. For the first time, we model both the human and the object with an implicit representation, allowing to capture more realistic details such as clothing. This task is extremely challenging due to human-object occlusions and the lack of 3D information in 2D images, often leading to poor detail reconstruction and depth ambiguity. To address these problems, we propose a novel attention-based neural implicit model that leverages image pixel alignment from both the input human-object image for a global understanding of the human-object scene and from local separate views of the human and object images to improve realism with, for example, clothing details. Additionally, the network is conditioned on semantic features derived from an estimated human-object pose prior, which provides 3D spatial information about the shared space of humans and objects. To handle human occlusion caused by objects, we use a generative diffusion model that inpaints the occluded regions, recovering otherwise lost details. For training and evaluation, we introduce a synthetic dataset featuring rendered scenes of inter-occluded 3D human scans and diverse objects. Extensive evaluation on both synthetic and real-world datasets demonstrates the superior quality of the proposed human-object reconstructions over competitive methods.
\end{abstract}
\section{Introduction}\label{sec:intro}

In computational finance, Monte Carlo simulations are used extensively to estimate the expected value of financial payoffs based on the solution of stochastic differential equations (SDEs) which model the evolution of stock prices, interest rates, exchange rates and other quantities \cite{glasserman04}.  Monte Carlo methods are very general and flexible, but for high accuracy it requires generating a large number of costly SDE path approximations, which has motivated research into a number of variance reduction or, equivalently, cost reduction techniques. One such method is
Multilevel Monte Carlo (MLMC), which was proposed in \cite{GILES2008} and was adapted for various applications that are summarised in \cite{Giles_overview17} and successfully combined with other methods such as quasi-Monte Carlo methods. The main idea of MLMC is to approximate the payoff using different time stepping resolutions when numerically solving the underlying SDE and to generate an optimal number of samples on each level, such that the overall computational cost is minimised subject to the desired bound on the variance. %, such that the total computational cost is minimised. 
The computational savings come from the fact that most samples are computed on the coarser levels and hence are less expensive while only a few samples from the finest levels are required \cite{GILES2008}.


Among the directions in which the computational cost 
of MLMC methods could further be reduced, an important avenue is the use of lower precision calculations, especially for the first Monte Carlo levels where the targeted accuracy is relatively low. 
 An overview of the research on mixed precision for the standard Monte Carlo (MC) framework is provided in \cite{ChowMixedPrecisionStandardMC} but only a few references study the potential of low precision computation in the MLMC framework \cite{Rounding_error_oliver}. To the best of our knowledge, the only MLMC framework with customised precision in the literature is \cite{brugger2014mixed}, but they use a uniform precision for all operations on each Monte Carlo level instead of optimising 
 the precision of each intermediary variable to reduce as much as possible the cost of path generation.
 
An important motivation for an MLMC framework with variable precision would be performing the low precision computations on reconfigurable hardware devices such as Field Programmable Gate Arrays (FPGAs). FPGAs contain customizable logic blocks and connectors that make it easy to adapt the digital circuit architecture for a specific application, leading to a highly parallel and optimised implementation. Therefore they are successfully exploited in applications that require high speed and have high computational workload, such as signal processing \cite{woods2008fpga}, and real time applications like high frequency trading \cite{HFT1,HFT2}. That is why a number of previous works in hardware architecture design implemented the MLMC algorithm to price financial options using FPGAs as accelerators, which resulted in improved speed and power efficiency compared to full CPU architectures \cite{Schryver2013AMM}. The paper \cite{lindsey2016domain} also proposed 
a Domain Specific Language to automate the configuration of FPGAs for this specific application. However, only \cite{brugger2014mixed} proposed a heuristic to reduce the precision in calculations.

In addition, all aforementioned works considered that the random number generation (RNG) is performed in single or double precision. Yet in most cases an important portion of the workload in the overall MLMC simulation comes from the RNG and in \cite{brugger2014mixed} this limited the total computational savings.
To reduce the cost of MLMC simulations in particular those based on the Geometric Brownian Motion (GBM), \cite{approximateICDF_Oliver, NestedOliver} have proposed to use approximate random numbers that are generated by applying an approximation of the inverse CDF to uniform random numbers. In \cite{NestedOliver}, the authors proposed a way to integrate these lower precision random variables into a \textit{nested} MLMC framework and completed a numerical analysis to bound the resulting error at each MC level by a product of the time step and the error in the random number approximation. The same authors show in \cite{approximateICDF_Oliver} that using approximate random variables reduces the cost of path generation by a factor 7.


In this paper we propose a nested MLMC framework that combines the use of approximate random normal variables and lower precision calculations to reduce the computational cost of MLMC even further than \cite{brugger2014mixed,NestedOliver}. We illustrate the efficiency of our framework in Matlab, after making several assumptions on the cost of operations and size of the errors that we carefully justify. We focus on the case of GBM and use the approximate RNG methods presented in \cite{approximateICDF_Oliver} as well as a new slightly modified method that combines CDF inversion and the central limit theorem. To choose the precision of the variables in the low precision path generation, we introduce a novel method to optimise the bit-widths. This optimisation is performed before the main path generation loop is executed and is based on a linear model of the payoff error  
due to rounding when computing in low precision. The error model relies on algorithmic differentiation in a similar manner to \cite{unifying-bwoptim,bitwidth-AD,ADAPT}. The bit-width optimisation procedure can be performed off-line, so this stage can be excluded from the on-line time complexity of our framework. The user specified desired accuracy is then enforced by calculating on-line the number of samples that need to be generated.

In terms of hardware design, we suggest implementing the low precision path generation on FPGAs and the full-precision ones on a CPU or GPU. 
The FPGA offers enough flexibility to define a separate bit-width for every variable in the low precision path generation, and can be reconfigured periodically to update the bit-widths when the market parameters have changed considerably. 


The paper is organized as follows : \Cref{sec:MLMC} introduces MLMC and nested MLMC to make clear the estimator that is implemented in our framework. Then in \Cref{sec:RNG} we detail the methods that could be used to obtain approximate random normally distributed numbers very cheaply for the low precision path generation. In \Cref{sec:error_model} and \Cref{sec:costModel} we propose an error model and a cost model (resp.) that we then use to formulate the optimisation problem that is solved to obtain the optimal bit-widths of fixed point variables in \Cref{sec:optimisation}. Finally we summarise our results and future directions in \Cref{sec:conclusion}.



\section{Related Work}
\label{sec:related work}
% In this section, we review the existing literature on point cloud denoising and unsupervised image denoising.
%-------------------------------------------------------------------------
\subsection{Point cloud denoising}

    \subsubsection{Traditional methods}
Traditional approaches to point cloud denoising include statistical methods \cite{computingpointset2003,definingpointset2004,wlop2009HH}, filtering techniques\cite{pointsetsurfaces2001,Robustmoving2005, zaman2017density}, and optimization-based methods \cite{l1sparse2010,clop2014PR,digne2017bilateral,multi-projection2018duan,hu2020featuregraph} . These techniques often rely on handcrafted features and heuristics to distinguish signal from noise. For example, statistical methods may use distribution models to identify and remove outliers. Filtering methods, such as mean or median filters, operate under the assumption that noise is statistically different from the signal. Optimization-based methods formulate denoising as an energy minimization problem, where regularization terms constrain the solution to ensure certain smoothness cirterion or adherence to prior knowledge.

%-------------------------------------------------------------------------
    \subsubsection{Supervised learning based methods}
In recent years, several deep learning-based methods \cite{rakotosaona2020PCN,hermosilla2019TotalDenoising,luo2020DMR,luo_score-based_2021} have been proposed for point cloud denoising. NPD \cite{NPD2019} is the first learning-based point cloud denoising method that directly processes noisy data without requiring noise characteristics or neighboring point definitions. This approach combines local and global information by projecting noisy points onto estimated reference planes, effectively removing noise and enhancing robustness against variations in noise intensity and curvature. PointCleanNet\cite{rakotosaona2020PCN} first removes outlier points and then combines them with residual connectivity to predict the inverse displacement \cite{Guerrero2017PCPNetLL}, and iteratively shifts noisy points to remove noise. Pistilli \etal proposed GPDNet \cite{gpdnet2020}, which is a graph convolutional network to improve denoising robustness at high noise levels. Luo \etal also proposed  DMRDenoise \cite{luo2020DMR}, which filter
points by first downsampling the noisy inputs and reconstructing the local subsurface to perform point upsampling. However, this resampling method is difficult to maintain a good local shape. ScoreDenoise \cite{luo_score-based_2021} is proposed to tackle the aforementioned issues by iteratively updating the point position in implicit gradient fields learned by neural networks. For inference, they follows an iterative procedure with a decaying step size, which stabilizes point movement and prevents over-correction, allowing points to converge gradually toward the underlying geometry. The authors of \cite{de_Silva_Edirimuni_2023_CVPR} proposed IterativePFN - an iterative method that use a novel loss function that utilizes an adaptive ground truth target at each iteration to capture the relationship between intermediate filtering results during training. Zheng \etal proposed a end-to-end network for joint normal filtering-based point cloud denoising \cite{10173632}. They introduce an auxiliary normal filtering task to enhance the network's ability to remove noise while preserving geometric features more accurately.

Supervised methods can achieve impressive results, but are limited by the availability and quality of the training data, as they typically require paired noisy and clean point clouds to train the neural network. The absence of clean data in real-world scenario pose a significant challenge on applicability of these algorithms.

%-------------------------------------------------------------------------
    \subsubsection{Unsupervised learning methods}
Unsupervised learning-based methods for point cloud denoising do not require ground-truth clean data. Instead, these methods leverage the inherent structure or distribution of the point cloud to guide the denoising process. Unsupervised methods show promise in scenarios where clean data is absent or hard to obtain.

Hermosilla \etal first introduced Total Denoising (TotalDn) \cite{hermosilla2019TotalDenoising} as an unsupervised learning approach for point cloud denoising, relying solely on noisy data without requiring clean ground truth. TotalDn approximates the underlying surfaces by regressing points from the distribution of unstructured total noise, utilizing a spatial prior term to refine the estimation of geometry. 

In DMRDenoise \cite{luo2020DMR}, an unsupervised version is proposed which utilizes a loss function that identify local neighborhoods using a probabilistic Gaussian mask on the k-nearest neighbors, which selectively retains points likely to represent the underlying surface. By leveraging an Earth Mover's Distance (EMD) assignment, it achieves a one-to-one correspondence between input and predicted points, aligning them to reduce noise within local neighborhoods.
This approach enhances robustness to noise and adapts well to varied surface geometries. However, the probabilistic masking and EMD calculation add computational complexity, which can slow down inference in dense or noisy point clouds. 

ScoreDenoise \cite{luo_score-based_2021} also introduced an unsupervised version that employs ensemble score function and an adaptive neighborhood-covering loss for model training.  
Score-u is probably the most relevant work to our method. However, the defined score in \cite{luo_score-based_2021} is only an displacement-alike version of the score function and there is no explicit formula relating the estimated score to the final denoising result. Also, the iterative process is computationally expensive, and can suffer from fluctuation, leading to perturbed and unstable solution.

Most recently, Noise4Denoise \cite{noise4Wang2024} method is proposed that use an additional doubly-noisy point cloud to learn noise displacement by comparing the two noise levels. This approach effectively leverages synthetic noise for training, allowing the model to estimate residuals without relying on clean data.
However, in practical applications, noise parameters are often unknown and variable, making it challenging to replicate the exact conditions assumed during training. This reliance on predefined noise characteristics can limit the model's applicability to real-world scenarios where noise distributions may differ significantly from synthetic settings. 
%-------------------------------------------------------------------------
\subsection{Unsupervised image denoising}
Recently unsupervised image denoising has made significant progress. Non-Bayesian methods include PURE \cite{luisier2010image}, SURE \cite{SURE2018} \textit{etc.}, which are based on various unbiased risk estimator under certain noise distribution. Other methods explore self-similarity in natural images \cite{xu2015patch, doi:10.1137/23M1614456} or exploits the statistical properties of noise to achieve denoising effect \cite{gravel2004method}.  

Noise2Noise \cite{2018Noise2NoiseLI} is a pioneering method that does not require clean data, but it requires multiple noisy versions of the same image for training. To address this limitation, methods such as Noise2Void \cite{2018Noise2VoidL}, Noise2Self \cite{2019Noise2SelfBD}, \textit{etc.}, have been developed, which use only a single noisy image. These methods are particularly important for practical applications where obtaining clean images or multiple noisy realizations of the same image is difficult or impossible. Neighbor2Neighbor \cite{huang2021neighbor2neighbor} proposed a two-step method with a a random neighbor sub-sampler that generates training  pairs and a denosing network. Kim \etal proposed Noise2Score\cite{kim_noise2score_2021}, a novel Bayesian framework for self-supervised image denoising without clean data. The core of Noise2Score is the usage of Tweedie's formula, which provides an explicit representation of the denoised image through a score function. Combined with score function estimation, Noise2Score can be applied to image denoising with any exponential family noise. Kim \etal also proposed the Noise Distribution Adaptive Self-Supervised Image Denoising method \cite{kim_noise_2022}, which further extends the application of Noise2Score by combining the Tweedie distribution with score matching. This enables adaptive handling of various noise distributions and dynamically adjusts the denoising process by estimating noise parameters. On the other hand, Xie \etal \cite{scoreXie2024} broadened the denoising scope of Noise2Score by allowing it to handle complex noise models, such as multiplicative and structurally correlated noise, demonstrating broad applicability to diverse noise models.

These development of unsupervised image denoising method motivate us to explore similar ideas in 3D point cloud denoising.






\section{Methodology}
\paragraph{Preliminaries.}
We primarily focus on the homologous model merging, in which $\boldsymbol{\theta}_i$ all come from the same base model $\boldsymbol{\theta}_{\rm{base}}$. Given $K$ tasks $\{T_1,T_2,\cdots,T_K\}$ and $K$ corresponding fine-tuned models with parameters $\{\boldsymbol{\theta}_1,\boldsymbol{\theta}_2,\cdots,\boldsymbol{\theta}_K\}$, model merging aims to combine $K$ fine-tuned models into one single model simultaneously performing on $\{T_1,T_2,\cdots,T_K\}$ without post-training~\cite{method_p1_1,method_p1_2}.
Task vector~\cite{ilharco2023editing,yang2024adamerging} is a key element in merging method which could enhances the base model‘s ability or enable the model to handle other tasks. Specifically, for task $T_i$, the task vector $\boldsymbol\tau_i\in \mathbb{R}^D$ is defined as the vector obtained by subtracting the SFT weights $\boldsymbol{\theta}_i$ from the base model weight
$\boldsymbol{\theta}_{\rm{base}}$, \emph{i.e.}, $\boldsymbol\tau_i=\boldsymbol{\theta}_i-\boldsymbol{\theta}_{\rm{base}}$. The merged model could be denoted as $\boldsymbol{\theta}_m=\boldsymbol{\theta}_{\rm{base}}+\sum_i \lambda_i\boldsymbol{\tau}_i$, which $\lambda_i$ is the scaling factor measuring the importance of task vector. For clarification, we also denote the neuron set in $\boldsymbol{\theta}_i$ as $\mathcal{N}_i$, the neuron set in $\boldsymbol{\tau}_i$ as $\mathcal{T}_i$.



\begin{algorithm}[!ht]
    \caption{LED-Merging}
    \label{alg1}
    \begin{algorithmic}[1]
        \REQUIRE  base model $\boldsymbol{\theta}_{\rm{base}}$, SFT models $\{\boldsymbol{\theta}_{i}\mid i\in [K]\}$, mask ratios \{$r_{i} \mid i\in [K]\}$, scaling factors $\{\lambda_i\mid i\in[K]\}$, location datasets $\{\mathcal{X}_{i}\mid i\in[K]\}$
        \ENSURE merged parameter $\boldsymbol{\theta}_{m}$
        \STATE $\mathcal{M}\leftarrow\phi$
        \STATE $\boldsymbol{\theta}_{m}\leftarrow \boldsymbol{\theta}_{\rm{base}}$
        \FOR{$i\in [K]$}
        \STATE $I(\boldsymbol{\theta}_i)=\mathbb{E}_{x\sim \mathcal{X}_i}|\boldsymbol{\theta}_{i}\odot \nabla_{\boldsymbol{\theta}_i}\mathcal{L}(x)|$
        \STATE $I(\boldsymbol{\theta}_{\rm{base}})=\mathbb{E}_{x\sim \mathcal{X}_i}|\boldsymbol{\theta}_{\rm{base}}\odot \nabla_{\boldsymbol{\theta}_{\rm{base}}}\mathcal{L}(x)|$
        
        \STATE calculate $\mathcal{T}^{r_i}_{i}$ following Equation \ref{vote}
        \STATE  $\mathcal{M}\leftarrow \mathcal{M}\cup\{\mathcal{T}^{r_i}_i\}$
       
        
   
        
        
        \ENDFOR  
        \FOR{$i\in [K]$}
        
        \STATE calculate $\text{Disjoint}(\mathcal{T}_i^{r_i})$ use Equation~\ref{disjoint_safety}
        \STATE $\boldsymbol{m}_i \leftarrow \boldsymbol{0}$
        \FOR{$d\in \mathcal{T}_i^{r_i}$}
        \STATE $\boldsymbol{m}_{i,d}=1$
        \ENDFOR
        \STATE $\boldsymbol{\theta}_{m}\leftarrow \boldsymbol{\theta}_{m}+\lambda_i \boldsymbol{\tau}_i\odot \boldsymbol{m}_{i}$
        \ENDFOR
    \end{algorithmic}
\end{algorithm}
    %\vspace{-5pt}
\begin{figure*}[h!]
    \centering
    \includegraphics[width=\linewidth]{figs/pipeline_v2.pdf}
    \vspace{-40mm}
    \caption{Overview of our two-stage training pipeline {\ours}.}
    \label{fig:pipeline}
\end{figure*}


\paragraph{LED-Merging: Location, Election, and Disjoint Merging}
To address the neuron misidentification and interference issues in existing model merging methods, we propose LED-Merging (Location, Election, and Disjoint Merging). Specifically, previous studies \cite{modelstock, ilharco2023editing, tiesmerging} fail to accurately identify safety-related neurons in task vectors with a single magnitude score, namely \textit{neuron misidentification}. Meanwhile, there exists an interference between safety-related and utility-related task vector neurons during the merging process, namely \textit{neuron interference}. To address neuron misidentification, we first locate important neurons both in the base and fine-tuned models and then elect neurons from the task vector considering these two scores together. Subsequently, to mitigate the interference, we introduce a disjoint step, isolating these important neurons so that they influence different base neurons. The whole process is illustrated in Figure~\ref{fig:method}. 




In the location and election step, we consider the importance score from base and fine-tuned models simultaneously to locate task-specific neurons. In this way, it is more accurate than relying on the magnitude score alone because task-specific neurons with high importance score in the fine-tuned model may not necessarily score high in the base model, and vice versa.

{\textbf{Location}}.  We first calculate importance scores for each neuron in a base/fine-tuned model. Given a location dataset $\mathcal{X}_i=\{(x,y)_k\}$, where $x$ is the question and $y$ is the answer, we calculate the importance scores for the weight $\boldsymbol{\theta}_i\in\mathbb{R}^D$ in any  layer as follows~\cite{snip,spareseGPT,sun2024a}:
\begin{equation}
    I(\boldsymbol{\theta}_i)=\mathbb{E}_{x\sim \mathcal{X}_i}[\boldsymbol{\theta}_i\odot \nabla _{\boldsymbol{\theta}_i}\mathcal{L}(x)],
    \label{location}
\end{equation}
which $\mathcal{L}(x)=-\log p(y\mid x)$ is the conditional negative log-likelihood loss. We choose the SNIP score~\cite{snip} because it balances computational efficiency and performance~\cite{cq}. Please refer to Sec.~\ref{sec:ablation} for the comparison between different location methods. After computing importance scores, we choose top-$r_i$ neurons as the important neuron subset $\mathcal{N}_{i}^{r_i}$ from $I(\boldsymbol{\theta}_i)$.
 
 % After computing locating scores, we select the neurons scoring both high in base and fine-tuned models as important neurons in task vectors. Then in the disjoint step,  with preventing  polysemantic neurons  from receiving gradient updates towards different directions,
 % we use set difference to isolate the safety   and utility-related neurons  and construct corresponding masks for merging process,

{\textbf{Election}}. A natural question is how to select important neurons in the task vector $\boldsymbol{\tau}_i$ based on $I(\boldsymbol{\theta}_{\rm{base}})$ and $I(\boldsymbol{\theta}_{i})$. The important neurons in the base model may be different from neurons in the fine-tuned model. Therefore, we introduce the following election strategy to select neurons with high scores in both base and fine-tuned models:
\begin{equation}
    \mathcal{T}_i^{r_i}=\mathcal{N}_i^{r_i}\cap \mathcal{N}_{\rm{base}}^{r_i}.
    \label{vote}
\end{equation}
\emph{Remark}. We compare different choosing methods, including scoring low or high in base or fine-tuned model in Section~\ref{sec:ablation} and find that Equation \ref{vote} achieves the best performance.





{\textbf{Disjoint}}. As important neurons from different task vectors may conflict with each other at the same position, we use the set difference to disjoint the neurons from others to prevent interference:
\begin{equation}
    \text{Disjoint}(\mathcal{T}^{r_i}_{i})=\mathcal{T}^{r_i}_{i}-\mathop{\cup}\limits_{{J}\subsetneqq [K],|J|\geq 2}\mathop{\cap}\limits_{j\in {J}}\mathcal{T}^{r_j}_{j}.
    \label{disjoint_safety}
\end{equation}

Next, we construct a mask $\boldsymbol{m}_i\in\mathbb{R}^D$ to implement disjoint in the merging process. Specifically, this mask $\boldsymbol{m}_i$ is used to select neurons from $\mathcal{T}_i$. The mask ratio is $r_i$, where $r\in(0,1]$. The mask $\boldsymbol{m}_i$ can be derived from:
\begin{equation}
    \boldsymbol{m}_{i,d}=\begin{aligned} &\left\{ \begin{array}{ll} 1, & \text{if } d\in \text{Disjoint}(\mathcal{T}_{i}^{r_i}), \\ 0, & \text{otherwise}. \end{array} \right. \end{aligned}
    \label{mask_safety}
\end{equation}


% \subsection{Merging Models with Masks}
{\textbf{Merging}}. The final
merged task vector $\boldsymbol{\tau}_m$ is as follows:
\begin{equation}
    \boldsymbol{\tau}_m= \sum_i \lambda_i\boldsymbol{\tau}_{i}\odot\boldsymbol{m}_i.
    \label{merged_task_vector}
\end{equation}
We summarize the workflow in Algorithm \ref{alg1}.



\section{Experiments}
\label{sec:experiment}

Experiments are carried out on NVIDIA RTX4090 GPUs using PyTorch 2.2.0 \cite{paszke2019pytorch} and the rotation detection tool kits: MMRotate 1.0.0 \cite{zhou2022mmrotate}. All the experiments follow the same hyper-parameters (learning rate, batch size, optimizer, etc.).

Average precision (AP) is adopted as the primary metric. All the models are configured upon ResNet50 \cite{he2016deep} and trained with AdamW \cite{loshchilov2018decoupled}.
\textbf{1) Learning rate.} Initialized at 5e-5, warm-up for 500 iterations, and divided by ten at each decay step. 
\textbf{2) Epochs.} 72 for HRSC; 12 for the others.
\textbf{3) Augmentation.} Random rotation/flip for HRSC; random flip for the others.
\textbf{4) Image size.} Split into 1,024 $\times$ 1,024 with an overlap of 200 for DOTA/FAIR1M/STAR; scaled to 800 $\times$ 800 for others.
\textbf{5) Multi-scale.} All experiments evaluated without multi-scale technique \cite{zhou2022mmrotate}. 
\textbf{6) Datasets.} Six remote sensing and one retail scene datasets, covering all datasets used by the main counterparts \cite{yu2024point2rbox, luo2024pointobb, cao2023p2rbox}:

\begin{table*}[!tb]
\fontsize{8.5pt}{10pt}\selectfont
\setlength{\tabcolsep}{0.65mm}
\setlength{\aboverulesep}{0.4ex}
\setlength{\belowrulesep}{0.4ex}
\setlength{\abovecaptionskip}{1.5mm}
\centering
\begin{tabular}{l|c|c|c|c|c|c|c|c|c|c}
\toprule
{\textbf{Methods}} & {*} & {\textbf{\,DOTA-v1.0\,}} & {\textbf{\,DOTA-v1.5\,}} & {\textbf{\,DOTA-v2.0\,}} & {\textbf{~~DIOR~~}} & {\textbf{~~HRSC~~}} & {\textbf{\,FAIR1M\,}} & {\textbf{~~STAR~~}} & {\textbf{\,SKU110K\,}} & {\textbf{~~RSAR~~}} \\
\hline
\rowcolor{gray!20} \multicolumn{11}{l}{$\blacktriangledown$ \textit{RBox-supervised OOD}} \\ \hline
RetinaNet (2017) \cite{lin2017focal} & \checkmark & 68.69 & 60.57        & 47.00 & 54.96 & 84.49   & 37.67   & 21.80 & 78.50 & 57.67  \\
GWD (2021) \cite{yang2021rethinking} & \checkmark & 71.66 & 63.27        & 48.87 & 57.60 & 86.67   & 39.11   & 25.30 & 79.16 & 57.80 \\
FCOS (2019) \cite{tian2019fcos} & \checkmark & 72.44 & 64.53        & 51.77    &  59.83  & 88.99  & 41.25   & \textbf{28.10} & 80.09 & \textbf{66.66} \\
S$^2$A-Net (2022) \cite{han2022align} & \checkmark & \textbf{75.81} & \textbf{66.53} & \textbf{52.39} & \textbf{61.41} & \textbf{90.10} & \textbf{42.44}   & 27.30 & \textbf{80.36} & 66.47 \\
\hline
\rowcolor{gray!20} \multicolumn{11}{l}{$\blacktriangledown$ \textit{HBox-supervised OOD}} \\ \hline
Sun et al. (2021) \cite{sun2021oriented} & $\times$ & 38.60 & - & - & - & - & - & - & - & - \\
KCR (2023) \cite{zhu2023knowledge} & \checkmark & - & - & - & - &  79.10  & -  & - & - & -  \\
H2RBox (2023) \cite{yang2023h2rbox} & \checkmark & 70.05 & 61.70        & 48.68    & 57.80 &  7.03  & 35.94  & 17.20 & 57.15 & 49.92    \\
H2RBox-v2 (2023) \cite{yu2023h2rboxv2} & \checkmark & 72.31 & 64.76 & 50.33 & 57.64 & \textbf{89.66} & \textbf{42.27} & \textbf{27.30} & \textbf{70.70} & \textbf{65.16} \\
AFWS (2024) \cite{lu2024afws} & \checkmark & \textbf{72.55} & \textbf{65.92} & \textbf{51.73} & \textbf{59.07} & - & 41.80 & - & - & - \\
\hline
\rowcolor{gray!20} \multicolumn{11}{l}{$\blacktriangledown$ \textit{Point-supervised OOD}} \\ \hline
P2RBox (2024) \cite{cao2023p2rbox}$^\dagger$ & $\times$ & \underline{59.04} & -        & - & - & -   & -  & -  & - & -  \\
PointSAM (2024) \cite{liu2024pointsam}$^\dagger$ & $\times$ & - & - & - & \textbf{46.20} & -   & -  & -  & - & - \\
PointOBB (2024) \cite{luo2024pointobb} & $\times$ & 30.08 & 10.66        & 5.53     &  37.31  & -   & 11.19 & 9.19  & - & 13.80    \\
Point2RBox+SK (2024) \cite{yu2024point2rbox}$^\dagger$ & \checkmark & 40.27 & 30.51        & 23.43    & 27.34 & 79.40   & 20.03 & 7.86  & 3.41 & 27.81    \\
PointOBB-v2 (2025) \cite{ren2024pointobbv2} & $\times$ & 41.68 & 30.59        & 20.64    &  39.56  & -   & 13.36 & 9.00  & 56.63 & 18.99   \\
PointOBB-v3 (2025) \cite{zhang2025pointobbv3} & $\checkmark$ & 41.20 & 31.25 & 22.82 & 37.60 & - & 11.42  & 11.31 & - & 15.84 \\
PointOBB-v3 (2025) \cite{zhang2025pointobbv3} & $\times$ & 49.24 & 33.79 & 23.52 & 40.18 & - & 18.35 & \underline{12.85} & - & 22.60 \\
\rowcolor{gray!20} Point2RBox-v2 (ours) & \checkmark & 51.00 & \underline{39.45} & \underline{27.11} & 34.70 & \underline{82.67} & \underline{25.72} & 7.80 & \underline{64.00} & \underline{28.60}
 \\
\rowcolor{gray!20} Point2RBox-v2 (ours) & $\times$ & \textbf{62.61} & \textbf{54.06}        & \textbf{38.79}   & \underline{44.45}  & \textbf{86.15}   & \textbf{34.71}  & \textbf{14.20} & \textbf{65.64} & \textbf{30.90}    \\
\bottomrule
\specialrule{0pt}{2pt}{0pt}
\multicolumn{11}{l}{$^*$Comparison tracks: \checkmark = End-to-end training and testing; $\times$ = Generating pseudo labels to train the FCOS detector (two-stage training).} \\
\multicolumn{11}{l}{$^\dagger$Using additional priors. P2RBox/PointSAM: Pre-trained SAM model; Point2RBox+SK: One-shot sketches for each class.} \\
\bottomrule
\end{tabular}
\caption{Accuracy (AP$_{50}$) comparisons on the DOTA-v1.0/1.5/2.0, DIOR, HRSC, FAIR1M, STAR, SKU110K, and RSAR datasets.}
\label{tab:exp_other}
\vspace{-4pt}
\end{table*}

\begin{itemize}
    \item \textbf{DOTA \cite{xia2018dota}.} DOTA-v1.0 has 2,806 aerial images annotated with 15 categories, while DOTA-v1.5/2.0 are the extended versions with more small objects and categories.
    
    \item \textbf{DIOR \cite{cheng2022anchor}.} It is an aerial image dataset re-annotated with RBoxes based on its original HBox version \cite{li2020object}, with a high variation in object size and high intra‐class diversity. 

    \item \textbf{HRSC \cite{liu2017hrsc}.} It contains ship instances on the sea and inshore. The train/val/test set includes 436/181/444 images.

    \item \textbf{FAIR1M \cite{sun2022fair1m}.} It has more than 1 million instances and more than 40,000 images for fine-grained object recognition in remote sensing imagery, annotated with 37 categories. The results are evaluated on FAIR1M-1.0.

    \item \textbf{STAR \cite{li2024star}.} It is extensive for scene graph generation, covering more than 210,000 objects with diverse spatial resolutions, classified into 48 fine-grained categories and precisely annotated with oriented bounding boxes. 

    \item \textbf{SKU110K \cite{pan2020dynamic}.} It focuses on the detection of densely packed retail scenes with 110,712 objects in 11,762 images. The density reaches 86 instances per image. 

    \item \textbf{RSAR \cite{zhang2025rsar}.} It is a remote sensing dataset based on Synthetic Aperture Radar (SAR) imagery with 6 categories.

\end{itemize}

\begin{table*}[!tb]
\fontsize{8.5pt}{10pt}\selectfont
\setlength{\tabcolsep}{2.08mm}
\setlength{\aboverulesep}{0.4ex}
\setlength{\belowrulesep}{0.4ex}
\setlength{\abovecaptionskip}{1.5mm}
\hspace{1pt}
\begin{minipage}[t]{0.315\linewidth}
\centering
\begin{tabular}{c|cc|cc}
\toprule
\multirow{2}{*}{$w_\text{O}$} & \multicolumn{2}{c|}{\textbf{DOTA}} & \multicolumn{2}{c}{\textbf{HRSC}} \\
                  & {E2E} & {FCOS} & {E2E} & {FCOS} \\ \midrule
3  & 48.76 & 61.62 & 81.85 & 84.36 \\
5  & 49.81 & 62.44 & 82.46 & 85.76 \\
\rowcolor{gray!20} 10 & \textbf{51.00} & \textbf{62.61} & \textbf{82.67} & \textbf{86.15} \\
30 & 45.88 & 57.83 & 81.56 & 85.61 \\
\bottomrule
\end{tabular}
\caption{Ablation with the weight of $\mathcal{L}_\text{O}$.}
\label{tab:abl_lo}
\end{minipage}
\quad
\begin{minipage}[t]{0.315\linewidth}
\centering
\begin{tabular}{c|cc|cc}
\toprule
\multirow{2}{*}{$w_\text{W}$} & \multicolumn{2}{c|}{\textbf{DOTA}} & \multicolumn{2}{c}{\textbf{HRSC}} \\
                  & {E2E} & {FCOS} & {E2E} & {FCOS} \\ \midrule
3  & 50.85 & 56.78 & 78.42 & 83.49 \\
\rowcolor{gray!20} 5  & \textbf{51.00} & \textbf{62.61} & \textbf{82.67} & \textbf{86.15} \\
10 & 49.15 & 60.54 & 30.37 & 35.13 \\
30 & 42.84 & 52.53 & 23.89 & 25.91 \\
\bottomrule
\end{tabular}
\caption{Ablation with the weight of $\mathcal{L}_\text{W}$.}
\label{tab:abl_lw}
\end{minipage}
\quad
\begin{minipage}[t]{0.315\linewidth}
\setlength{\tabcolsep}{2.04mm}
\centering
\begin{tabular}{c|cc|cc}
\toprule
\multirow{2}{*}{$w_\text{E}$} & \multicolumn{2}{c|}{\textbf{DOTA}} & \multicolumn{2}{c}{\textbf{HRSC}} \\
                  & {E2E} & {FCOS} & {E2E} & {FCOS} \\ \midrule
0.1 & 48.75 & 57.62 & 34.71 & 39.45 \\
\rowcolor{gray!20} 0.3 & 51.00 & 62.61 & \textbf{82.67} & \textbf{86.15} \\
0.5 & \textbf{51.36} & \textbf{62.63} & 76.85 & 85.22 \\
1.0 & 49.05 & 60.63 & 56.59 & 59.59 \\
\bottomrule
\end{tabular}
\caption{Ablation with the weight of $\mathcal{L}_\text{E}$.}
\label{tab:abl_le}
\end{minipage}
\vspace{-4pt}
\end{table*}

\begin{table*}[!tb]
\fontsize{8.5pt}{10pt}\selectfont
\setlength{\tabcolsep}{2.04mm}
\setlength{\aboverulesep}{0.4ex}
\setlength{\belowrulesep}{0.4ex}
\setlength{\abovecaptionskip}{1.5mm}
\hspace{1pt}
\begin{minipage}[t]{0.315\linewidth}
\centering
\begin{tabular}{c|cc|cc}
\toprule
\multirow{2}{*}{$w_\text{ss}$} & \multicolumn{2}{c|}{\textbf{DOTA}} & \multicolumn{2}{c}{\textbf{HRSC}} \\
                  & {E2E} & {FCOS} & {E2E} & {FCOS} \\ \midrule
0.1 & 49.28 & 59.66 & 73.66 & 78.92 \\
\rowcolor{gray!20} 1.0 & \textbf{51.00} & \textbf{62.61} & \textbf{82.67} & \textbf{86.15} \\
3.0 & 49.15 & 59.20 & 1.30  & 1.65 \\
\bottomrule
\end{tabular}
\caption{Ablation with the weight of $\mathcal{L}_\text{ss}$.}
\label{tab:abl_lss}
\end{minipage}
\quad
\begin{minipage}[t]{0.647\linewidth}
\setlength{\tabcolsep}{3.05mm}
\centering
\begin{tabular}{c|c|c||c|c|c}
\toprule
{R / F / S} & {\textbf{DOTA}} & {\textbf{HRSC}} & {R / F / S} & {\textbf{DOTA}} & {\textbf{HRSC}} \\
 \midrule
90\% / 10\% / 0\% & 60.42 & 85.46 & 80\% / 20\% / 0\%  & 59.46 & 84.73 \\
75\% / 0\% / 25\% & 60.79 & 86.22 & 60\% / 15\% / 25\% & 62.38 & 84.21 \\
\cellcolor{gray!20}68\% / 7\% / 25\% & \cellcolor{gray!20}\textbf{62.61} & \cellcolor{gray!20}\textbf{86.15} & 38\% / 37\% / 25\% & 45.87 & 8.56  \\
45\% / 5\% / 50\% & 60.55 & 85.34 & 40\% / 10\% / 50\% & 60.49 & 10.74 \\
\bottomrule
\end{tabular}
\caption{Ablation with the proportion of augmented views in self-supervision.}
\label{tab:abl_pro}
\end{minipage}
\vspace{-10pt}
\end{table*}

\subsection{Main Results on DOTA-v1.0}
\label{sec:experiment-main}

Table \ref{tab:exp_dota} compares Point2RBox-v2 with the state-of-the-art methods, which can be categorized into two tracks: 

\textbf{1) End-to-end training.} These methods apply the trained weakly-supervised detector directly to the test set. Without relying on priors, our approach demonstrates an improvement of 16.93\% (51.00\% vs. 34.07\%) compared to Point2RBox. Even when compared to Point2RBox+SK, which incorporates additional data-side priors (i.e. one-shot examples for each class), our method still outperforms it by 10.73\% (51.00\% vs. 40.27\%).

\textbf{2) Two-stage training.} These methods generate RBox labels on train/val sets, with which the FCOS detector is trained. In this two-stage mode, Point2RBox-v2 achieves an accuracy of 62.61\%, considerably surpassing PointOBB series. Remarkably, it even outperforms the SAM-powered method P2RBox by 3.57\% (62.61\% vs. 59.04\%).

\textbf{Class-wise analysis.} The FCOS detector trained with labels generated by Point2RBox-v2 achieves accuracy nearly equivalent to RBox-supervised FCOS across six high-density categories: SH (86.9\% vs. 87.1\%), SV (79.6\% vs. 79.8\%), LV (76.3\% vs. 79.8\%), PL (88.0\% vs. 89.1\%), ST (82.9\% vs. 84.6\%), and TC (89.1\% vs. 90.4\%). Interestingly, these six high-density categories account for 88\% of DOTA instances. By annotating these categories with points and generating RBoxes using Point2RBox-v2 while labeling the other sparse categories with RBoxes, we can significantly reduce annotation labor without sacrificing much accuracy, highlighting the valuable role our method can play.

\begin{figure*}[t!]
\setlength{\abovecaptionskip}{1.2mm}
\centering
\includegraphics[width=0.96\linewidth]{figs/case.pdf}
\caption{Qualitative analysis on failed cases and overlap cases.}
\label{fig:case}
\vspace{-6pt}
\end{figure*}

\subsection{Results on More Datasets}

The results are displayed in Table \ref{tab:exp_other}.
On more challenging DOTA-v1.5/2.0, Point2RBox-v2 presents a similar trend, 23.47\%/18.15\% higher than PointOBB-v2 in the pseudo-generation track. 
On the ship detection dataset HRSC, the gap between Point2RBox-v2 and RBox-supervised FCOS is only 2.84\% (86.15\% vs. 88.99\%).
DIOR is relatively sparse, leading to less improvement with our methods---lower than PointSAM (44.45\% vs. 46.20\%) but still higher than methods that do not use SAM. 
Our method also provides competitive performance on fine-grained datasets FAIR1M and STAR. 
In addition to remote sensing scenarios, we carry out experiments on SKU110K for densely packed retail scenes. Existing point-supervised methods struggle in this case, whereas Point2RBox-v2 achieves performance on par with HBox-supervised H2RBox (65.64\% vs. 57.15\%).

\begin{table}[!tb]
\fontsize{8.5pt}{10pt}\selectfont
\setlength{\tabcolsep}{1.78mm}
\setlength{\aboverulesep}{0.4ex}
\setlength{\belowrulesep}{0.4ex}
\setlength{\abovecaptionskip}{1.5mm}
\centering
\begin{tabular}{ccccc|cc|cc}
\toprule
\multicolumn{5}{c|}{\textbf{Modules}} & \multicolumn{2}{c|}{\textbf{DOTA}} & \multicolumn{2}{c}{\textbf{HRSC}} \\
$\mathcal{L}_\text{O}$ & $\mathcal{L}_\text{W}$ & $\mathcal{L}_\text{ss}$ & $\mathcal{L}_\text{E}$ & \textit{CP} & {E2E} & {FCOS} & {E2E} & {FCOS} \\ \midrule
\checkmark & & & & & 0.00 & 0.00 & 0.00 & 0.00 \\
\checkmark & \checkmark & & & & 41.54 & 52.98 & 17.96 & 19.64 \\
\checkmark & \checkmark & \checkmark & & & 46.64 & 54.26 & 18.10 & 22.13 \\
\checkmark & \checkmark & \checkmark & \checkmark & & 49.55 & 61.88 & 78.79 & 83.79 \\
& \checkmark & \checkmark & \checkmark & \checkmark & 48.58 & 59.56 & 20.35 & 24.76 \\
\checkmark & & \checkmark & \checkmark & \checkmark & 38.94 & 48.44 & 11.64 & 14.93 \\
\checkmark & \checkmark & \checkmark & & \checkmark & 47.08 & 55.05 & 19.58 & 21.78 \\
\rowcolor{gray!20} \checkmark & \checkmark & \checkmark & \checkmark & \checkmark & \textbf{51.00} & \textbf{62.61} & \textbf{82.67} & \textbf{86.15} \\
\bottomrule
\end{tabular}
\caption{Ablation with incremental addition of modules.}
\label{tab:abl_mod}
\vspace{-4pt}
\end{table}

\begin{table}[!tb]
\fontsize{8.5pt}{10pt}\selectfont
\setlength{\tabcolsep}{2.85mm}
\setlength{\aboverulesep}{0.4ex}
\setlength{\belowrulesep}{0.4ex}
\setlength{\abovecaptionskip}{1.5mm}
\centering
\begin{tabular}{c|c|c||c|c|c}
\toprule
16 & \cellcolor{gray!20}$K\!=\!24$ & 32 & 1.2 & \cellcolor{gray!20}$\beta\!=\!1.6$ & 2.0 \\ \midrule
50.87 & \cellcolor{gray!20}\textbf{51.00} & 48.08 & 48.14 & \cellcolor{gray!20}51.00 & \textbf{51.33} \\
\bottomrule
\end{tabular}
\caption{Ablation with $K$ and $\beta$ in edge loss on DOTA (E2E).}
\label{tab:abl_edgeparam}
\vspace{-4pt}
\end{table}

\begin{table}[!tb]
\fontsize{8.5pt}{10pt}\selectfont
\setlength{\tabcolsep}{1.75mm}
\setlength{\aboverulesep}{0.4ex}
\setlength{\belowrulesep}{0.4ex}
\setlength{\abovecaptionskip}{1.5mm}
\centering
\begin{tabular}{c|cc|cc|cc}
\toprule
\multirow{2}{*}{$\sigma$} & \multicolumn{2}{c|}{Point2RBox} & \multicolumn{2}{c|}{PointOBB-v2} & \multicolumn{2}{c}{Point2RBox-v2} \\
 & {\textbf{DOTA}} & {\textbf{HRSC}} & {\textbf{DOTA}} & {\textbf{HRSC}} & {\textbf{DOTA}} & {\textbf{HRSC}} \\ \midrule
0\%  & 40.27 & 79.40 & 44.85 & - & 62.61 & 86.15 \\
10\% & 39.60 & 78.81 & 42.30 & - & 61.58 & 85.76 \\
30\% & 38.42 & 78.28 & 38.46 & - & 60.31 & 85.71 \\
\bottomrule
\end{tabular}
\caption{Ablation with the inaccuracy in point annotations.}
\label{tab:abl_noise}
\vspace{-10pt}
\end{table}

\subsection{Ablation Studies}
\label{sec:experiment-ablation}

Tables \ref{tab:abl_lo}-\ref{tab:abl_noise} display the ablation studies on DOTA-v1.0 and HRSC. ``E2E'' denotes end-to-end training; ``FCOS'' denotes two-stage training (i.e. generating pseudo labels to train FCOS). The final values adopted are highlighted in gray.

\textbf{Weight of each loss.} Tables \ref{tab:abl_lo}-\ref{tab:abl_le} determine the weights of the proposed losses. Based on these experiments, the weights $(w_\text{O},w_\text{W},w_\text{E},w_\text{ss})$ are set to $(10, 5, 0.3, 1)$.

\textbf{Proportion of augmented views.} Table \ref{tab:abl_pro} studies the proportion between rotation, flip, and scale. The results are reported with two-stage training (FCOS). Based on the results, the proportion is set to 68\%, 7\%, and 25\%.

\textbf{Incremental addition of modules.} Table \ref{tab:abl_mod} demonstrates the constraints from Gaussian and Voronoi achieve an accuracy of 52.98\% on DOTA. Adding consistency loss and edge loss further boosts it to 54.26\% and 61.88\%, respectively, whereas the improvement from copy-paste is 0.73\%. We also demonstrate the impact of omitting each core loss.

\textbf{Edge loss parameters.} We set $K=24$ and $\beta=1.6$ as they are observed to discern the correct edges during code development. Table \ref{tab:abl_edgeparam} provides a more precise ablation.

\textbf{Annotation inaccuracy.} We offset the annotated points by a noise from the uniform distribution $\left[-\sigma H, +\sigma H \right ]$, where $H$ is the height of objects. Table \ref{tab:abl_noise} shows that the AP$_{50}$ of Point2RBox-v2 decreases by less than 3\% when noise is added to point annotations, demonstrating the robustness of the proposed learning mechanisms.

\subsection{More Discussions}
\label{sec:experiment-discussions}

The qualitative analysis on the failed/overlap cases is shown in Fig. \ref{fig:case}. \textbf{1) Failed cases.} Although our method performs well overall, it struggles with certain categories that are sparse and not constrained by other objects. \textbf{2) Overlap cases.} 
Minimizing overlap as a soft constraint during training does not entirely eliminate overlap. Once trained, the model remains robust to some overlap during inference.

\section{Conclusion}

%In this paper, w
We propose a new PEFT method called DiffoRA, which enables efficient and adaptive LLM fine-tuning based on LoRA. 
Instead of adjusting every interior rank, 
%of the decomposition matrices 
%of all modules, 
we argue that adopting LoRA module-wisely is sufficient. 
To achieve this, we construct a DAM to select the modules that are most suitable and essential to fine-tune. We theoretically analyze how the DAM impacts the convergence rate and generalization capability.
%of the pre-trained model. 
Furthermore, we adopt continuous relaxation and discretization to establish DAM.
%for each task. 
To alleviate the issue of discretization discrepancy, we utilize the weight-sharing strategy for optimization. 
%We fully implement our method and t
The experimental results demonstrate that our DiffoRA works consistently better than the baselines across all benchmarks. 

\newpage
% \newpage

%\section*{ACKNOWLEDGMENT}
%\balance
%\small
\bibliographystyle{IEEEtranN} %citeauthor
\bibliography{string-short,references}


\end{document}
