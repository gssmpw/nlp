\section{Conclusion}
\label{sec:conclusion}

In this paper, we introduce GaRLIO, a gravity-enhanced radar-LiDAR-inertial odometry that provides a novel gravity estimation method that utilizes radar Doppler measurements.
Differing from the velocity-ignorant approaches, our method ensures robust gravity estimation along the various platforms.
Furthermore, the fusion of radar with LIO facilitates the dynamic removal within LiDAR point clouds.
We validated its performance in public datasets, demonstrating its robustness even in challenging scenarios such as downhill and dynamic object-rich conditions.
Remarkably, our approach represents superior improvements in mitigating vertical drift. GaRLIO, the first method to combine radar and gravity,  is anticipated to establish new research directions for advancing robust SLAM systems based on \ac{UGV}.


% overcoming the challenges in vertical drift arising in traditional LIO approaches.

% We presented GaRLIO, a gravity-enhanced Radar-\ac{LiDAR}-Inertial Odometry system designed to mitigate vertical drift in \ac{LIO} frameworks. By integrating radar Doppler measurements, GaRLIO leverages velocity information to improve gravity estimation, effectively addressing inaccuracies from double integration of \ac{IMU} data. This velocity-aware approach also enables dynamic object removal from \ac{LiDAR} point clouds, enhancing robustness in diverse environments.

% Our evaluations on the NTU4DRadLM, Snail-Radar, and Fog-Filled hallway datasets demonstrate that GaRLIO consistently outperforms state-of-the-art \ac{LIO} methods in both translation and rotation accuracy. Ablation studies highlight the critical roles of velocity and gravity residuals in achieving superior odometry performance, especially in high-speed and vibration-prone scenarios. Additionally, GaRLIO shows resilience in challenging conditions, such as steep elevation changes and adverse weather, where traditional \ac{LIO} systems often experience significant vertical drift.

% GaRLIO represents a significant advancement in \ac{LIO} technology, offering more reliable and precise pose estimation for mobile robotics and autonomous systems. Future work will explore further sensor integrations and optimize computational efficiency to enhance real-time performance and applicability in more complex environments.

% This paper presents GaRLIO, the first gravity-enhanced Radar-\ac{LiDAR}-Inertial Odometry system designed to mitigate vertical drift in \ac{LIO} frameworks. GaRLIO integrates radar Doppler measurements to leverage velocity information, improving gravity estimation and addressing inaccuracies from double integration of \ac{IMU} data. Additionally, our velocity-aware approach enables dynamic object removal from \ac{LiDAR} point clouds, enhancing robustness in diverse environments.

% Evaluations on public datasets—NTU4DRadLM, Snail-Radar, and Fog-Filled hallway—demonstrate that GaRLIO outperforms existing state-of-the-art \ac{LIO} methods in both translation and rotation accuracy. Ablation studies validate the effectiveness of velocity and gravity residuals in enhancing odometry performance, particularly in high-speed and vibration-prone scenarios. GaRLIO also shows resilience under challenging conditions such as steep elevation changes and adverse weather, where traditional \ac{LIO} systems often experience significant vertical drift.

% As the first gravity-enhanced \ac{LIO} framework, GaRLIO opens new avenues for future research, including further sensor integrations and optimization for real-time applications in more complex and dynamic environments.