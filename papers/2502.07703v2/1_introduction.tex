\section{Introduction}
\label{sec:intro}

Range-based sensor \ac{SLAM} is widely used for precise pose estimation of mobile robots in unknown environments. Along with many of range sensors, \ac{LiDAR} has drawn attention for its intuitive spatial information, especially being integrated with \ac{IMU} to exhibit high precision and robustness \cite{FAST-LIO, LINS, POINT-LIO}. However, due to the sparse vertical resolution and remaining uncertainties in \ac{LiDAR} measurements \cite{lidar_bias}, \ac{LIO} is still susceptible to vertical drift. 

In this context, exploiting gravity shows promising performance in mitigating vertical drift from recent works \cite{kubelka2022gravity, D-LIOM, nemiroff2023joint,wildcat,agha2021nebula}. Since gravity maintains size and direction constant across diverse conditions, precise local gravity estimation allows for an accurate rotation calculation between the global and body frame. To be more specific, accurate gravity estimation may lead to reliable observation of roll and pitch, reducing the vertical position error. Most existing methods measure gravity vector relying on the relationship between \ac{IMU} acceleration measurements and estimated pose \cite{D-LIOM, nemiroff2023joint,wildcat} or only when the robot motion is stationary \cite{agha2021nebula}. However, establishing the relationship between pose and acceleration generally results in amplified errors due to the double integrated bias and noise of the \ac{IMU} acceleration \cite{forster2016manifold}.

% A potential solution to the double integration problem is leveraging the relationship between velocity and acceleration.
% To address this, we observe that incorporating velocity measurements can significantly enhance the accuracy of gravity estimation. Nonetheless, since \ac{LIO} lacks direct velocity observations in the sensor measurement level, the accuracy of velocity estimation is inherently dependent on the accuracy of estimated pose \cite{yuan2023liwo}. Several studies explore the fusion of radar with traditional \ac{LIO} to effectively enhance state estimation performance by providing velocity measurements \cite{DR-LRIO}.
% Although incorporating velocity information partially mitigates vertical drift, there are inherent limitations in fully resolving the issue due to the characteristic radar having greater elevation uncertainty than azimuth uncertainty \cite{instruments2020imaging}. 
Since traditional \ac{LIO} lacks direct velocity observations at the raw measurement level, they rely on pose estimation and \ac{IMU} measurements to estimate gravity. Yet, we observe that incorporating velocity measurements can significantly enhance the accuracy of gravity estimation. While one study has explored the fusion of radar with \ac{LIO} to improve state estimation performance by providing velocity measurements \cite{DR-LRIO}, no existing work has integrated velocity measurements specifically for the purpose of gravity estimation.

%FIGURE
\begin{figure}[!t]
    \centering
    \includegraphics[width=\linewidth]{figure/figure1_revised.pdf}
    \caption{\textbf{Top}: Trajectories of GaRLIO and other methods with ground truth (white) overlaid on the \ac{TLS} map. \textbf{Bottom}: Elevation plot along path length. Our method (blue) reported only $\unit{1.21}{m}$ vertical mean error over $\unit{2.045}{km}$ path length.}
    \label{fig:pipeline}
    \vspace{-7mm}
\end{figure}

\begin{figure*}[t!]
    \centering
    \includegraphics[width=0.8\textwidth]
    {figure/pipeline.pdf}
    \caption{
    GaRLIO is divided into four primary modules. Each module contributes to achieving the optimal state by removing \ac{LiDAR} dynamic points and calculating both pointwise velocity and velocity-aware gravity residuals.
    }
    \label{fig:overview}
    \vspace{-8mm}
\end{figure*}

In this paper, we propose GaRLIO, a gravity-enhanced Radar-\ac{LiDAR}-Inertial Odometry that introduces a novel idea of exploiting radar for gravity estimation. GaRLIO integrates a new gravity estimation technique that leverages radar Doppler measurements to address vertical drift and inaccuracy. Integrating radar with \ac{LIO} improves velocity estimation accuracy, allowing for more precise local gravity estimation, which reduces vertical drift. Furthermore, this integration facilitates the dynamic object filtering in \ac{LiDAR} point cloud, ensuring performance in diverse environments. GaRLIO is evaluated on public datasets with various challenging scenarios, including downhill conditions where vertical drift prevalently occurs in \ac{LIO}, exhibiting surpassing performance compared with \ac{SOTA} methods. % 얘도 dataset citation 추가
The pipeline of our method is as shown in \figref{fig:overview}. Our contributions are as follows:
\begin{itemize}
    \item We propose GaRLIO that utilizes gravity to address the vertical drift of \ac{LIO}. GaRLIO introduces a novel velocity-aware gravity estimation that leverages radar Doppler measurements to reduce vertical drift and inaccuracy common in velocity-ignorant approaches. To our knowledge, this is the first method using radar velocity to estimate gravity.
    \item GaRLIO fuses radar with \ac{LIO} and demonstrated enhanced state estimation performance by exploiting pointwise velocity residuals based on radar measurements. Additionally, our fusion approach effectively removes dynamic points within \ac{LiDAR} point clouds. 
    \item GaRLIO is evaluated on public datasets with challenging scenarios, including downhill, where the introduced residuals demonstrated performance improvement. We open-source GaRLIO to encourage further development in radar-\ac{LiDAR}-\ac{IMU} fusion.
\end{itemize}



% \section{Introduction}
% \label{sec:intro}
% %%%%%%%%%%%%%%%%%%%%%%%%%%%%%%%%%%%%%%%%%%%%%%%%%%%%%%%%%%%%%%%%%%%%%%
% % LiDAR is widely utilized for state estimation due to its ability to directly perceive three-dimensional environments and provide rich perceptual information. However, LiDAR odometry is significantly challenged by motion distortion and the degradation of perceptual performance under extreme environmental conditions. To improve odometry performance, numerous studies have integrated LiDAR with proprioceptive sensors, such as Inertial Measurement Units (IMUs) \cite{FAST-LIO,  LINS, POINT-LIO}. \ac{LIO} systems achieve high precision and robustness in GPS-denied environments; however, they are susceptible to upward drift along the z-axis over long-term operation \cite{Versatile_Ground}.

% %LiDAR, which directly perceives surrounding 3D environments by providing rich perceptual information, is widely leveraged for its high precision and robustness in GPS-denied environments when integrated with proprioceptive sensors such as \ac{IMU} \cite{FAST-LIO, LINS, POINT-LIO}, known as \ac{LIO} system.
% % \ac{LiDAR} is widely utilized for state estimation due to its ability to directly perceive three-dimensional environments and provide rich perceptual information. To compensate for motion distortion in \ac{LiDAR} scan, \ac{LO} is often combined with \ac{IMU} to assist with motion compensation and provide prior guesses, known as \ac{LIO} system \cite{FAST-LIO, LINS, POINT-LIO}. Despite the improved accuracy of \ac{LIO} system with the inclusion of an \ac{IMU} sensor, vertical drift along the z-axis remains a challenge \cite{Versatile_Ground}. 


% % range based slam에 lidar-inertial 묶어도 될거같긴한데 어떨거같음? 아래 두문장을 하나로 합쳐서 처음부터 LIO로 가는거
% % 아래 문장 라이다 카메라 imu 퓨전에서 긁어온 문장인데 걍 이런식으로도 쓸수도 있는거같음
% % A wide range of sensors can be applied for accurate 6-DoF motion estimation, among which LiDAR, IMU, and camera might be the most popular and widely used.

% % Range-based \ac{SLAM} is a highly researched field aimed at achieving robust pose estimation. 
% % Range-based \ac{SLAM} is a widely used approach for precise pose estimation of mobile robots operating in unknown environments. Over the decades, \ac{LiDAR} integrated with \ac{IMU} has been recognized for its capability to provide rich and intuitive spatial information, solidifying its role as the primary sensor set in range-based \ac{SLAM} \cite{FAST-LIO, LINS, POINT-LIO}. Although \ac{LIO} systems exhibit high precision and robustness, they are susceptible to vertical drift along the z-axis over extended periods due to uncertainties in LiDAR measurements \cite{lidar_bias}.


% %In addition, several studies have explored the fusion of radar, LiDAR, and IMU to compensate for LiDAR's degradation in perceptual performance under extreme environmental conditions by utilizing radar's measurements. For instance, \cite{DR-LRIO} demonstrated that incorporating Doppler measurements from radar into traditional \ac{LIO} can improve state estimation in certain environments. Despite these advancements, vertical drift along the z-axis remains a challenge in SLAM due to uncertainties in radar and LiDAR measurements in the z-axis \cite{instruments2020imaging, lidar_bias}.
% % 라이다 칭찬하는데 이러한 한계가 있고, fusion 이 제안되었지만 vertical drift는 아직 안풀렸다.

% % 지난 시간동안 라이다는 풍부하고 직관적인 공간적 정보를 제공해주는 성능을 인정받아 slam에서 primary sensor로 발전되어왔다.
% % 문제가 1. degeneracy / 2. vertical drift -> gravity ?
% % degeneracy,  motion compensation를 보완하기위해 imu와의 fusion이 보편적으로 제시되어 왔으며
% % 최근들어 레이더의 doppler와 융합하여 degeneracy를 해결하려는 노력이 제안되었다.
% % 하지만 이러한 발전에도 불구하고 vertical drift는 still remains the challenge

% % graity가 이러한 문제 해결에 효과적임은 다양한 연구들을 통해 증명하였다.
% % (one approach보다 gravity가 효과적이라는 말이 강조되면 좋을거같음)
% % One approach to reducing vertical drift is by applying a gravity constraint \cite{D-LIOM, nemiroff2023joint,wildcat,agha2021nebula}.
% % % 뭔가 아래 문장은 related에 들어가야할거같은말이라 차라리 gravity를 어떻게 estimate하고 있는지나 어떻게 factor로서 쓰고있는지 간단하게 써주고 다음 문장에서 however 이런 방법들은 이런저런 문제가 있다 로 넘어가면 좋지 않을까
% % The study in \cite{kubelka2022gravity} demonstrated that utilizing gravity for \ac{LIO} can enhance performance by reducing vertical drift through gravity-aligned 4DOF state estimation.
% % % 특히 velocity-ignorant이면 왜 안좋은지가 들어가면 좋을거같음
% % However, most of these methods measure gravity only when the motion is stationary or use velocity-ignorant approaches that rely on the relationship between acceleration and pose.
% % Additionally, since these methods are based on window-based estimation of gravity, real-time online estimation is not feasible.


% % In this context, gravity plays a crucial role in mitigating vertical drift \cite{kubelka2022gravity, D-LIOM, nemiroff2023joint,wildcat,agha2021nebula}. Since gravity maintains a constant value and direction across diverse conditions, precise estimation of local gravity allows for the calculation of rotation between the global and the body frame. Most of existing methods measure gravity only when the motion is stationary \cite{agha2021nebula} or rely on the relationship between \ac{IMU} acceleration measurements and estimated pose \cite{D-LIOM, nemiroff2023joint,wildcat}. However, establishing the relationship between pose and acceleration requires double integration of the acceleration data, a process in which errors due to inaccurate bias can be significantly amplified. 
% Range sensor \ac{SLAM} is widely used for precise pose estimation of mobile robots operating in unknown environments. 
% Along with many range sensors, \ac{LiDAR} has drawn attention for its capability to provide intuitive spatial information, especially being integrated with \ac{IMU} to exhibit high precision and robustness \cite{FAST-LIO, LINS, POINT-LIO}. 
% However, due to the sparse vertical resolution and remaining uncertainties in \ac{LiDAR} measurements \cite{lidar_bias}, \ac{LIO} is still susceptible to vertical drift with extended periods. 

% In this context, exploiting gravity showed promising performance in mitigating vertical drift from recent works \cite{kubelka2022gravity, D-LIOM, nemiroff2023joint,wildcat,agha2021nebula}. 
% Since gravity maintains size and direction constant across diverse conditions, precise local gravity estimation allows for an accurate rotation calculation between the global and body frame. 
% To be more specific, accurate gravity estimation may lead to reliable observation of roll and pitch, reducing the vertical position error. % 일단 추가해뒀습니다 ㅋㅋㅋ 윗 문장이랑 합치거나 좀 다듬어서 쓰면 될듯???
% Most existing methods measure gravity vector relying on the relationship between \ac{IMU} acceleration measurements and estimated pose \cite{D-LIOM, nemiroff2023joint,wildcat} or only when the robot motion is stationary \cite{agha2021nebula}. 
% However, establishing the relationship between pose and acceleration generally results in amplified errors due to the double time integrated bias and noise of the \ac{IMU} acceleration \cite{forster2016manifold}.

% A potential solution to the double integration problem is leveraging the relationship between velocity and acceleration. 
% Nonetheless, since \ac{LIO} lacks direct velocity observations in the sensor measurement level, the accuracy of velocity estimation is inherently dependent on the accuracy of estimated pose \cite{yuan2023liwo}.
% Several studies explored the fusion of radar with traditional \ac{LIO} to effectively enhance state estimation performance by providing velocity measurements. 
% Although incorporating velocity information can partially mitigate vertical drift, there are inherent limitations in fully resolving the issue due to the characteristic radar having greater elevation uncertainty than azimuth uncertainty \cite{instruments2020imaging}. 

% % Incorporating radar not only provides additional velocity measurements but also enabling dynamic removal in LiDAR point cloud using radar Doppler measurements, thereby enabling more robust state estimation.


% %FIGURE
% \begin{figure}[!t]
%     \centering
%     \includegraphics[width=\linewidth]{figure/figure1_revised.pdf}
%     \caption{(a) Results showing trajectories overlaid on the reference \ac{TLS} map.  (b) Our method (blue) demonstrates reduced vertical drift compared to the other approaches.}
%     \label{fig:pipeline}
%     \vspace{-7mm}
% \end{figure}

% \begin{figure*}[t!]
%     \centering
%     \includegraphics[width=0.8\textwidth]
%     {figure/pipeline.pdf}
%     \caption{
%     The proposed method is divided into four primary modules. Each module contributes to achieving the optimal state through the removal of LiDAR dynamic points, and the calculation of pointwise velocity and velocity-dependent gravity residuals.
%     }
%     \label{fig:overview}
%     \vspace{-6mm}
% \end{figure*}
% %%0911
% % Incorporating new velocity measurements not only facilitates robust state estimation but also provides crucial information that enhances the accuracy and reliability of gravity estimation. 
% % Unlike other method, in this paper, we propose GaRLIO, a gravity-enhanced Radar-LiDAR-Inertial odometry method that utilize a novel idea of using the radar for gravity estimation. GaRLIO incorporates a novel gravity estimation method using radar Doppler measurements, which effectively mitigates vertical drift and reduces inaccuracies common in velocity-ignorant approaches due to double integration. 

% In this paper, we propose GaRLIO, a gravity-enhanced Radar-\ac{LiDAR}-Inertial Odometry approach that introduces a novel idea of using the radar for gravity estimation. 
% GaRLIO integrates a new gravity estimation technique that leverages radar Doppler measurements to effectively mitigate both vertical drift and address inaccuracies typically associated with velocity-ignorant approaches caused by double integration.
% Integrating radar with \ac{LIO} improves velocity estimation accuracy, allowing for more precise local gravity estimation, which reduces vertical drift on odometry. 
% Furthermore, this integration facilitates the filtering of dynamic objects from the \ac{LiDAR} point cloud, ensuring reliable performance in diverse environments. 
% GaRLIO is evaluated with public datasets with various challenging scenarios, including downhill conditions where vertical drift is particularly prevalent, exhibiting outperforming performance compared with current \ac{SOTA} methods. % 얘도 citation 추가

% % Unlike existing methods, this paper proposes GaRLIO, a gravity-enhanced Radar-\ac{LiDAR}-Inertial Odometry approach that introduces a novel idea of using the radar for gravity estimation. 
% % GaRLIO integrates a new gravity estimation technique that leverages radar Doppler measurements to effectively mitigate vertical drift and address inaccuracies typically associated with velocity-ignorant approaches caused by double integration.
% % Integrating radar into the system improves accuracy of velocity estimation, allowing for more precise local gravity estimation. 
% % This integration not only enhances gravity estimation accuracy but also facilitates the filtering of dynamic objects from the \ac{LiDAR} point cloud, ensuring reliable performance across diverse environments. 
% % We evaluated GaRLIO's performance in various challenging scenarios, including downhill conditions, where vertical drift is particularly prevalent.

% % Recently, the advent of high-resolution 4D radar has led to an increase in research on radar-based odometry \cite{4DRadarSLAM, 4diriom, drio}. Unlike \ac{LiDAR}, radar, with its longer wavelengths, can penetrate environments with smoke or fog and provides doppler measurements. This capability enables radar odometry to maintain robust accuracy even in \ac{LiDAR}-degenerate or geometrically uninformative environments. Additionally, to improve performance in perceptually-degraded conditions, several studies have explored the fusion of radar, LiDAR, and IMU to leverage complementary information. For example, \cite{DR-LRIO} demonstrated that incorporating doppler measurements into traditional \ac{LIO} can aid in state estimation. Although incorporating velocity information can partially mitigate vertical drift, there are inherent limitations in fully resolving the issue due to the radar sensor's characteristic of having greater elevation uncertainty compared to azimuth uncertainty \cite{instruments2020imaging}.
% Our contributions are as follows:
% \begin{itemize}
%     \item We propose GaRLIO that utilizes local gravity to address the vertical drift of \ac{LIO}. 
%     GaRLIO introduces a novel approach estimating gravity by leveraging radar Doppler measurements, successfully addressing vertical drift and inaccuracies common in velocity-ignorant approaches from double integration. 
%     To our knowledge, this is the first approach using radar to estimate gravity.
%     \item GaRLIO fuses radar with \ac{LIO} and demonstrated enhanced state estimation performance by exploiting pointwise velocity residuals based on radar measurements. 
%     Additionally, our fusion approach effectively removes dynamic points within \ac{LiDAR} point clouds. 
%     \item GaRLIO is evaluated on various public datasets with challenging scenarios, including downhill, where the introduced residuals demonstrated performance improvement. 
%     We open-source GaRLIO to encourage further development in radar-\ac{LiDAR}-\ac{IMU} fusion.
% \end{itemize}

