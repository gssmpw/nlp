\subsection{Training Process}
\label{subsec:pretrain_train_process}


To enhance computational efficiency in pre-training LLMs, we employ three key techniques: staged pre-training from scratch, depth up-scaling, and pruning and distillation.
In Section \ref{subsec:pretrain_staged_pretrain}, we first train 26.8B and 8B models using a staged pre-training approach, which serves as the foundation for obtaining LLMs at various scales.
In Section \ref{subsec:pretrain_dus}, we describe the process to obtaining \textit{Kanana Flag 32.5B} and \textit{Kanana Essence 9.8B} models by depth up-scaling from 26.8B and 8B models, respectively.
In Section \ref{subsec:pretrain_pd}, we derive \textit{Kanana Nano 2.1B} model through pruning and distillation from the 8B model, reducing training costs while achieving superior performance compared to training a model from scratch.


\subsubsection{Staged Pre-training from Scratch}
\label{subsec:pretrain_staged_pretrain}

\begin{figure}[h]
    \centering
    \subfloat[Stage 1 data]{
        \includegraphics[width=0.45\textwidth]{figures/pretraining/data-stage1.pdf}
        \label{fig:pre_stage1_data}
    }
    \hfill
    \subfloat[Stage 2 data]{
        \includegraphics[width=0.45\textwidth]{figures/pretraining/data-stage2.pdf}
        \label{fig:pre_stage2_data}
    }
    \caption{Kanana's staged pre-training data mixture.}
    \label{fig:pre_data_stats}
\end{figure}


\begin{table}[h]
\centering
\resizebox{\columnwidth}{!}{
\begin{tabular}{lc|cccccc|c}
\toprule
    \multirow{2}{*}{Models} & \multirow{2}{*}{Stage} & \textbf{MMLU} & \textbf{KMMLU} & \textbf{HAE-RAE} & \textbf{HumanEval} & \textbf{MBPP} & \textbf{GSM8K} & \multirow{2}{*}{\textbf{Avg}} \\
    & & \textit{5-shot} & \textit{5-shot} & \textit{5-shot} & \textit{0-shot} & \textit{3-shot} & \textit{5-shot} & \\
\midrule
\multirow{2}{*}{26.8B} & Stage 1 & 73.38 & 54.26 & 84.97 & 32.32 & 47.20 & 57.77 & 58.32 \\
& Stage 2 & 74.27 & 59.04 & 88.45 & 51.22 & 61.60 & 67.48 & 67.01 \\
\midrule
\multirow{2}{*}{8B} & Stage 1 & 63.48 & 45.51 & 77.27 & 23.78 & 35.80 & 35.03 & 46.81 \\
& Stage 2 & 64.22 & 48.30 & 83.41 & 40.24 & 51.40 & 57.09 & 57.44 \\
\bottomrule
\end{tabular}
}
\caption{
Performance of from-scratch Kanana models at the end of each training stage.
} \label{table:from-scratch}
\end{table}

To maximize performance under fixed compute budget, we adopt the staged pre-training strategy \citep{minicpm, gemma, opencoder, yi-lightning, granite} with two stages.
Staged pre-training divides the pre-training process into multiple stages, starting with training LLMs on a large amount of moderate-quality data in the initial stages, and gradually increasing the proportion of high quality data in the subsequent stages. 

We begin by training 8B from scratch using the diverse 2.7 trillion in stage 1 as shown in Figure \autoref{fig:pre_stage1_data}.
In stage 2, we further train the model using 300 billion tokens shown in Figure \ref{fig:pre_stage2_data}. 
Specifically, we set aside high quality data for each category using the available high quality classifiers.
Then, we perform lightweight annealing experiments to select candidate datasets to search for the data mixture following \citet{llama3}.
Then, the optimal data mixture is selected through ablation study.
The final model of stage 2 results in a 2.79 point increase in KMMLU and a 10.63 point increase in average performance, demonstrating the effectiveness and efficiency of staged pre-training. 
We apply the same data mixture that was used during the training of 8B to 26.8B model. 
Direct application of the recipe consistently yields remarkable performance and stable training as shown in \autoref{table:from-scratch}, demonstrating the scalability of our recipe. See Appendix \ref{appendix:pretrain-details} for our pre-training configurations.


\subsubsection{Depth Up-scaling}
\label{subsec:pretrain_dus}

To further enhance the model performance within limited resources after pre-training, we adopt the depth up-scaling (DUS) which increases model capacity by stacking additional layers \citep{kim2023solar}.
We apply DUS to expand Kanana 8B into Kanana Essence 9.8B and Kanana 26.8B into Kanana Flag 32.5B.
After the up-scaling process, each model variant is further trained on the same data mixtures used in pre-training, with 100 billion tokens dedicated to stage 1 and another 100 billion to stage 2.
Results of the up-scaling strategy demonstrates that the additional layers consistently contribute to performance enhancements as summarized in \autoref{tab:performance}.
\begin{table}[ht]
    \centering
    \resizebox{\columnwidth}{!}{%
    \begin{tabular}{l|cccccc|c}
    \toprule
    \multirow{2}{*}{\textbf{Models}} & \textbf{MMLU} & \textbf{KMMLU} & \textbf{HAE-RAE} & \textbf{HumanEval} & \textbf{MBPP} & \textbf{GSM8K} & \multirow{2}{*}{\textbf{Avg}} \\
    & \textit{5-shot} & \textit{5-shot} & \textit{5-shot} & \textit{0-shot} & \textit{3-shot} & \textit{5-shot} & \\
    \midrule
    26.8B + DUS (32.5B) & 77.68 & 62.10 & 90.47 & 51.22 & 63.40 & 70.05 & 69.15 \\
    26.8B & 74.27 & 59.04 & 88.45 & 51.22 & 61.60 & 67.48 & 67.01 \\
    \midrule
    8B + DUS (9.8B)     & 67.61 & 50.67 & 84.98 & 40.24 & 53.60 & 63.61 & 60.10 \\
    8B & 64.22 & 48.30 & 83.41 & 40.24 & 51.40 & 57.09 & 57.44 \\
    \bottomrule
    \end{tabular}%
    }
    \caption{Performance comparison of Kanana models before and after depth up-scaling.}
    \label{tab:performance}
\end{table}

\autoref{tab:performance} illustrates the performance improvements achieved through depth up-scaling.
Kanana Essence 9.8B consistently outperforms its non-upscaled version, Kanana 8B with the average score rising from 57.52 to 60.12. 
This improvement is evident in MMLU, KMMLU, HAE-RAE, MBPP, and GSM8K, except for HumanEval. 
Similarly, Kanana Flag 32.5B achieves average score of 69.15, notably surpassing the non-upscaled Kanana 26.8B model. 
These results emphasize the effectiveness of depth up-scaling in improving various benchmark scores.



Notably, our strategy saves 11.06\% of total computational cost compared to the training of 9.8B and 32.5B LLMs from scratch.
This strategy of increasing model capacity through depth up-scaling only occupies about 6.67\% of the total computing resources across the entire training procedure. 
In combination with pre-training, depth up-scaling offers a strategic approach to significantly enhance model performance without introducing heavy computational demands of building new models from scratch.


\subsubsection{Pruning and Distillation}
\label{subsec:pretrain_pd}


In opposition to efficiently up-scaling the model size, knowledge distillation is an effective method to efficiently down-scale the model size \citep{hinton2015knowledge-distillation, gunter2024apple, llama3.2}.
Leveraging the 8B model from Section \ref{subsec:pretrain_staged_pretrain}, we efficiently produce smaller models by improving the pruning and distillation of Minitron \citep{muralidharan2024compact, sreenivas2024llm}.
This process allows us to produce models with better performance at one-tenth of the data size compared to training from scratch, as shown in \autoref{tab:pd-vs-fs}.
We further show that iteratively extending the process beyond two iterations remains effective, preserving 87-99\% of KMMLU score at only 50\% of the model size, as shown in \autoref{tab:pd-iterative}.
Our models achieve competitive performance to recent open-source models \citep{allal2025smollm2smolgoesbig, llama3, gemma2024gemma2, qwen25techreport}, as presented in \autoref{tab:pd-base-all}.


\begin{table}[ht]
    \centering
    \resizebox{\columnwidth}{!}{%
    \begin{tabular}{l|c|cccccc|c}
    \toprule
    \multirow{2}{*}{\textbf{Models}} & \multirow{2}{*}{\textbf{\makecell{Training \\ Tokens}}} & \textbf{MMLU} & \textbf{KMMLU} & \textbf{HAERAE} & \textbf{HumanEval} & \textbf{MBPP} & \textbf{GSM8K} & \multirow{2}{*}{\textbf{Avg}} \\
    & & \textit{5-shot} & \textit{5-shot} & \textit{5-shot} & \textit{0-shot} & \textit{3-shot} & \textit{5-shot} & \\
    \midrule
    2.1B PD & 0.3T & 54.83 & 44.80 & 77.09 & 31.10 & 46.20 & 46.32 & 50.06 \\
    2.1B & 3T & 50.66 & 36.61 & 68.74 & 24.45 & 41.60 & 36.69 & 43.13 \\
    \bottomrule
    \end{tabular}%
    }
    \caption{
    Token consumption and performance of pruning \& distillation (PD) from preceding models and training from scratch. We use the same 2.1B architecture.
    }
    \label{tab:pd-vs-fs}
\end{table}


\begin{table}[ht]
    \centering
    \begin{tabular}{l|cccccc|c}
    \toprule
    \multirow{2}{*}{\textbf{Models}} & \textbf{MMLU} & \textbf{KMMLU} & \textbf{HAERAE} & \textbf{HumanEval} & \textbf{MBPP} & \textbf{GSM8K} & \multirow{2}{*}{\textbf{Avg}} \\
    & \textit{5-shot} & \textit{5-shot} & \textit{5-shot} & \textit{0-shot} & \textit{3-shot} & \textit{5-shot} \\
    \midrule
    \tikzmark{t} 8B$^\dag$ & 64.22 & 48.30 & 83.41 & 40.24 & 51.40 & 57.09 & 57.44 \\
    \tikzmark{a} 4.5B & 59.74 & 48.09 & 82.58 & 34.76 & 48.60 & 57.01 & 55.13 \\
    \tikzmark{b} 2.1B & 54.83 & 44.80 & 77.09 & 31.10 & 46.20 & 46.32 & 50.06 \\
    \tikzmark{c} 1.3B & 53.55 & 39.91 & 72.59 & 28.05 & 39.60 & 36.01 & 44.95 \\
    \tikzmark{d} 635M & 46.28 & 34.60 & 62.69 & 23.17 & 31.40 & 19.26 & 36.23 \\
    \tikzmark{e} 385M & 41.16 & 31.70 & 47.94 & 18.90 & 24.00 & 10.83 & 29.08 \\
    \tikzmark{f} 192M & 26.11 & 30.16 & 19.71 & 12.80 & 12.40 & 2.43  & 17.27 \\
    \bottomrule
    \end{tabular}%
    \caption{Performance through iterative compression beyond two iterations. Each model is pruned from the preceding model. $^\dag$ Each model is distilled using the 8B model as the teacher.}
    \label{tab:pd-iterative}
    \begin{tikzpicture}[overlay, remember picture, shorten >=1pt, shorten <=1pt, transform canvas={yshift=.25\baselineskip}]
        \draw [-stealth] ({pic cs:t}) [bend right=50] to ({pic cs:a});
        \draw [-stealth] ({pic cs:a}) [bend right=50] to ({pic cs:b});
        \draw [-stealth] ({pic cs:b}) [bend right=50] to ({pic cs:c});
        \draw [-stealth] ({pic cs:c}) [bend right=50] to ({pic cs:d});
        \draw [-stealth] ({pic cs:d}) [bend right=50] to ({pic cs:e});
        \draw [-stealth] ({pic cs:e}) [bend right=50] to ({pic cs:f});
    \end{tikzpicture}
\end{table}


In order to improve the pruning and distillation process, we refine Minitron's width importance scoring while preserving its simplicity and efficiency.
Its scoring process begins by measuring the importance of embedding channels, feed-forward neurons, and attention heads using activations from a small calibration dataset.
Next, we show that summing layer-wise scores plays a crucial role in performance, whereas the prior work performed ablations along batch and sequence axes.
Moreover, for Grouped-Query Attention (GQA) \citep{ainslie2023gqa}, we improve performance by ensuring query-key-value alignment. Specifically, we remove an equal number of query heads within each group, as shown in \autoref{fig:pd-gqa}.
Additionally, since Kanana employs SwiGLU \citep{swiglu}, we choose between averaging gate and up states or using intermediate states, whereas the original formulation relies on pre-activation values.
All ablation results for importance scoring are in \autoref{tab:pd-score-detail}.


We further enhance the pruning strategies with a focus on intermediate model structures.
Consistent with the findings from Minitron, we observe that excessive single-step compression leads to significant degradation.
Although maintaining attention heads is generally beneficial, our experiments reveal that pruning them for smaller models is effective when done earlier at larger scales as presented in \autoref{tab:pd-attn-heads}.
Additionally, we find that input and output embeddings can be tied by averaging without causing noticeable degradation, which we apply when pruning from 4.5B to 2.1B as shown in \autoref{tab:pd-tie}.


Lastly, we observe that the composition of distillation data directly influences the performance, while pruning data is less important.
For models larger than 2B, we use high-quality 300 billion tokens of stage 2 described in Section \ref{subsec:pretrain_staged_pretrain}.
However, for smaller models, increasing the proportion of general-domain English data increases both English performance and other benchmark scores, as shown in \autoref{tab:pd-distill-data}.


In conclusion, our comprehensive pre-training process, which includes staged pre-training, depth up-scaling, and iterative pruning and distillation, offers a compute-efficient strategy for developing high-performing language models. 
This combined approach not only enhances performance across diverse benchmarks, but also ensures computational efficiency, demonstrating the effectiveness of our strategy in producing a robust family of models spanning the range from 2.1B to 32.5B. See Appendix \ref{appendix:pd-details} for our pruning and distillation configurations.