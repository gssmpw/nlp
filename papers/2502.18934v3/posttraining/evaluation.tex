\subsection{Performance}
\label{subsec:post_train_performance}




\begin{table}[h]
\centering
\resizebox{0.95\columnwidth}{!}{
\begin{tabular}{l|cccc|c}
\toprule
\multirow{2}{4em}{\textbf{Models}} & \multicolumn{4}{c|}{\textit{Chat}} & \textit{Instruction Following} \\
      & \textbf{MT-Bench} & \textbf{LogicKor} & \textbf{KoMT-Bench} & \textbf{WildBench} & \textbf{IFEval} \\
\midrule
\rowcolor{yellow} Kanana Flag 32.5B & 8.356 & \textbf{9.524} & \textbf{8.058} & 54.14 & \textbf{0.856} \\
Qwen2.5 32B & 8.331 & 8.988 & 7.847 & 51.13 & 0.822 \\
Gemma 2 27B & 8.088 & 8.869 & 7.373 & 46.46 & 0.817 \\
EXAONE-3.5-32B & \textbf{8.375} & 9.202 & 7.907 & \textbf{54.30} & 0.845 \\
Aya Expanse 32B & 7.788 & 8.941 & 7.626 & 48.36 & 0.735 \\
\midrule
\rowcolor{yellow} Kanana Essence 9.8B & 7.769 & 8.964 & 7.706 & 47.27 & 0.799 \\
Llama 3.1 8B & 7.500 & 6.512 & 5.336 & 33.20 & 0.772 \\
Qwen2.5 7B & 7.625 & 7.952 & 6.808 & 41.31 & 0.760 \\
Gemma 2 9B & 7.633 & 8.643 & 7.029 & 40.92 & 0.750 \\
EXAONE-3.5-7.8B & \textbf{8.213} & \textbf{9.357} & \textbf{8.013} & \textbf{50.98} & \textbf{0.826} \\
Aya Expanse 8B & 7.131 & 8.357 & 7.006 & 38.50 & 0.645\\
\midrule
\rowcolor{yellow} Kanana Nano 2.1B &  6.400 & 7.964 & 5.857 & 25.41 & 0.720 \\
Llama 3.2 3B & 7.050 & 4.452 & 3.967 & 21.91 & 0.767 \\
Qwen2.5 3B & 6.969 & 6.488 & 5.274 & 25.76 & 0.355 \\
Gemma 2 2B & 7.225 & 5.917 & 4.835 & 28.71 & 0.428 \\
EXAONE-3.5-2.4B & \textbf{7.919} & \textbf{8.941} & \textbf{7.223} & \textbf{41.68} & \textbf{0.790} \\
\midrule\midrule
Llama 3.1 70B & 8.275 & 8.250 & 6.970 & 46.50 & 0.875 \\
Qwen2.5 72B & 8.619 & 9.214 & 8.281 & 55.25 & 0.861 \\
\bottomrule
\end{tabular}
}
\caption{
Performance of Kanana and previous instruction-tuned models in general chat and instruction following benchmarks.
Across all \textit{Chat} benchmarks, we use \texttt{gpt-4o-2024-08-06} as a judge model.
The best scores are denoted in \textbf{bold}.
70B sized models have been included for reference purposes.
}\label{table:chat-eval-2}
\end{table}


% Benchmark Table
\begin{table}[h]
\centering
\resizebox{\columnwidth}{!}{
\begin{tabular}{l|ccc|cc|cc}
\toprule
\multirow{2}{4em}{\textbf{Models}} & \multicolumn{3}{c|}{\textit{General}} & \multicolumn{2}{c|}{\textit{Coding}} & \multicolumn{2}{c}{\textit{Mathematics}} \\
      & \textbf{MMLU} & \textbf{KMMLU} & \textbf{HAE-RAE} &  \textbf{HumanEval+} & \textbf{MBPP+} & \textbf{GSM8K} & \textbf{MATH} \\
\midrule
\rowcolor{yellow} Kanana Flag 32.5B & 81.08 & \textbf{64.19} & \textbf{68.18} & 77.44 & 69.84 & 90.83 & 57.82 \\
Qwen2.5 32B & \textbf{84.40} & 59.37 & 48.30 & \textbf{82.32} & \textbf{71.96} & \textbf{95.30} & \textbf{81.90} \\
Gemma 2 27B & 78.01 & 49.98 & 46.02 & 70.12 & 70.90 & 91.05 & 53.80 \\
EXAONE-3.5-32B & 78.30 & 55.44 & 52.27 & 78.66 & 70.90 & 93.56 & 76.80 \\
Aya Expanse 32B & 74.49 & 42.35 & 51.14 & 64.63 & 65.61 & 75.06 & 42.82 \\
\midrule
\rowcolor{yellow} Kanana Essence 9.8B & 70.64 & 50.76 & \textbf{47.16} & 72.56 & 69.05 & 84.91 & 42.24 \\
Llama 3.1 8B & 71.18 & 39.24 & 40.91 & 60.98 & 57.67 & 82.71 & 49.86 \\
Qwen2.5 7B & \textbf{77.23} & 46.87 & 37.50 & 73.78 & \textbf{70.63} & \textbf{91.58} & \textbf{75.22} \\
Gemma 2 9B & 73.47 & 44.47 & 39.77 & 59.76 & 64.55 & 87.72 & 48.10 \\
EXAONE-3.5-7.8B & 72.62 & \textbf{52.09} & 46.02 & \textbf{79.27} & 66.67 & 89.99 & 73.50 \\
Aya Expanse 8B & 61.23 & 35.78 & 39.20 & 42.68 & 56.88 & 78.85 & 30.80 \\
\midrule
\rowcolor{yellow} Kanana Nano 2.1B & 52.48 & \textbf{38.51} & \textbf{33.52} & 63.41 & 62.43 & 72.32 & 29.26 \\
Llama 3.2 3B & 56.09 & 3.07 & 17.05 & 56.71 & 50.26 & 66.57 & 38.18 \\
Qwen2.5 3B & \textbf{69.18} & 38.33 & 32.39 & 67.68 & \textbf{64.02} & \textbf{84.00} & \textbf{65.72} \\
Gemma 2 2B & 57.69 & 6.99 & 7.95 & 35.37 & 45.24 & 49.81 & 21.68 \\
EXAONE-3.5-2.4B & 63.19 & 14.27 & 14.20 & \textbf{70.73} & 59.79 & 83.78 & 64.04 \\
\midrule\midrule
Llama 3.1 70B & 83.48 & 39.08 & 53.41 & 75.61 & 66.40 & 91.66 & 63.98 \\
Qwen2.5 72B & 87.14 & 65.78 & 60.80 & 81.10 & 75.66 & 95.45 & 82.60 \\
% HyperCLOVA X\footnotemark[1] & 66.50 & 49.09 & 59.65 & - & 62.17 & 63.46 & -  \\
\bottomrule
\end{tabular}
}
\caption{
Performance of Kanana post-trained models on a set of standard benchmarks. 
All benchmarks under General category are measured using 0-shot CoT with respective chat-template of each model.
The best scores are denoted in \textbf{bold}.
70B sized models have been included for reference purposes.
}\label{table:chat-eval-1}
\end{table}


We evaluate our instruction-tuned models across various tasks: chat, instruction following, general knowledge, coding, and mathematics and compare their performance to previous instruction-tuned models.
For general chat ability, we use MT-Bench \citep{zheng2023judging}, LogicKor \citep{park2024logickor}, KoMT-Bench \citep{KoMT-Bench}, and WildBench \citep{lin2024wildbench}.
To test instruction following ability, we use IFEval\footnote{We report the average of Prompt-level strict-accuracy and Instruct-level strict-accuracy.}\citep{zhou2023instructionfollowingevaluationlargelanguage}.
For general knowledge tasks, we use MMLU \citep{hendryckstest2021}, KMMLU \citep{son2024kmmlu}, and HAE-RAE\footnote{We report general knowledge category scores in this section.} \citep{son-etal-2024-hae}, with zero-shot chain-of-thought (CoT) \citep{wei2022chainofthought} setting along with the chat template.
Employing zero-shot CoT with the chat template, rather than multi-shot prompts, allows us to evaluate the inherent capabilities of the instruction model, without residual traces from the pre-trained model.
For coding ability, we use HumanEval+ \citep{evalplus} and MBPP+ \citep{evalplus}.
For Mathematical ability, we use GSM8K \citep{cobbe2021gsm8k} and MATH \citep{hendrycksmath2021}.
See Appendix \ref{subsec:evaluation-prompts-for-post-trained-models} for detailed prompts of benchmarks.

\autoref{table:chat-eval-2} and \autoref{table:chat-eval-1} show that our models excel similar sized models on Korean tasks.
The 32.5B model achieves the highest performance in Korean chat tasks (LogicKor, KoMT-Bench) and Korean knowledge tasks (KMMLU, HAE-RAE). 
The 9.8B and 2.1B models rank second in Korean chat tasks and either best or second-best in Korean knowledge tasks.
Additionally, our models exhibit competitive performance across other tasks except in math.

