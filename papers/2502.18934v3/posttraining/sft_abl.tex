\begin{table}[t]
\centering
\resizebox{\columnwidth}{!}{
\begin{tabular}{cccc|cccccc}
\toprule
\multicolumn{4}{c|}{\textbf{Datasets used}} & \multicolumn{5}{c}{\textbf{Normalized Scores}} \\
General & Instruction Following & Code & Math & MT-Bench & IFEval & HumanEval+ & MBPP+ & GSM8K & MATH \\
\midrule
\cmark & \cmark & \cmark & \cmark & 1.00 & 1.00 & 1.00 & 1.00 & 1.00 & 1.00 \\
\midrule
\cmark & \textcolor{red}{\xmark} & \cmark & \cmark & 0.98 & \textcolor{red}{0.72} & 1.06 & 0.99 & 1.03 & 1.07 \\
\cmark & \cmark & \textcolor{red}{\xmark} & \cmark & 0.99 & 1.00 & \textcolor{red}{0.66} & \textcolor{red}{0.72} & 1.01 & 1.05 \\
\cmark & \cmark & \cmark & \textcolor{red}{\xmark} & 0.98 & 1.00 & 1.04 & 1.00 & \textcolor{red}{0.60} & \textcolor{red}{0.59} \\
\bottomrule
\end{tabular}
}
\caption{
Domain mixture ablation for SFT dataset.
All scores are normalized by the score of the SFT model when datasets of all domains have been included in the training set.
We see that removing a specific domain from the training dataset exclusively deteriorates the performance of the respective domain by a significant amount.
}\label{table:sft_abl}
\end{table}