\section{Evaluation Details}

\subsection{Evaluation prompts for post-trained models} \label{subsec:evaluation-prompts-for-post-trained-models}

We employ 0-shot CoT prompts for a number of evaluations. 
See \autoref{fig:evaluation-prompts-general} for MMLU, KMMLU and HAE-RAE.
For math-related tasks we employ 0-shot prompt.
We refer the readers to \autoref{fig:evaluation-prompts-math} for the prompts corresponding to GSM8K and MATH.

\begin{figure}
    \centering
    \subfloat[MMLU prompt]{
        \begin{tcolorbox}[colframe=black!80!white, colback=black!2!white, boxrule=0.5mm, width=\textwidth, arc=2mm, auto outer arc, title=MMLU prompt (0-shot CoT), fonttitle=\color{white}\bfseries]
            \setstretch{1.2}
            The following are multiple choice questions about \texttt{\{mmlu\_subject\}}. Summarize your reasoning concisely, then conclude with "Therefore, the answer is: X" where X is one of A, B, C, or D.\\
            
            Question: \texttt{\{question\}}\\
            A. \texttt{\{choice\_A\}}\\
            B. \texttt{\{choice\_B\}}\\
            C. \texttt{\{choice\_C\}}\\
            D. \texttt{\{choice\_D\}}\\
        \end{tcolorbox}
        \label{fig:mmlu-zero-shot-cot-prompt}
    }
    
    \subfloat[KMMLU prompt]{
        \begin{tcolorbox}[colframe=black!80!white, colback=black!2!white, boxrule=0.5mm, width=\textwidth, arc=2mm, auto outer arc, title=KMMLU prompt (0-shot CoT), fonttitle=\color{white}\bfseries]
            \setstretch{1.2}
            \begin{CJK}{UTF8}{mj}
            다음은 \texttt{\{kmmlu\_subject\}}에 관한 객관식 문제입니다. 당신의 추론 과정을 간결하게 요약한 후, "따라서, 정답은: X"라고 결론지으십시오. 여기서 X는 A, B, C, D 중 하나입니다.\\
            
            질문: \texttt{\{question\}}\\
            A. \texttt{\{choice\_A\}}\\
            B. \texttt{\{choice\_B\}}\\
            C. \texttt{\{choice\_C\}}\\
            D. \texttt{\{choice\_D\}}\\
            \end{CJK}
        \end{tcolorbox}
        \label{fig:kmmlu-zero-shot-cot-prompt}
    }
    
    \subfloat[HAE-RAE prompt]{
        \begin{tcolorbox}[colframe=black!80!white, colback=black!2!white, boxrule=0.5mm, width=\textwidth, arc=2mm, auto outer arc, title=HAE-RAE (0-shot CoT), fonttitle=\color{white}\bfseries]
        \setstretch{1.2}
    \begin{CJK}{UTF8}{mj}
    다음은 객관식 문제입니다. 당신의 추론 과정을 간결하게 요약한 후, "따라서, 정답은: X"라고 결론지으십시오. 여기서 X는 A, B, C, D, E 중 하나입니다.\\
    \{query\}
    \end{CJK}
        \end{tcolorbox}
        \label{fig:haerae-zero-shot-cot-prompt}
    }
    
    \caption{
    Evaluation prompts for MMLU, KMMLU, and HAE-RAE.
    The prompts are used to evaluate instruction-tuned models.
    }
    \label{fig:evaluation-prompts-general}
\end{figure}

\begin{figure}
    \centering
    \subfloat[GSM8K prompt]{
    \begin{tcolorbox}[colframe=black!80!white, colback=black!2!white, boxrule=0.5mm, width=\textwidth, arc=2mm, auto outer arc, title=GSM8K (0-shot), fonttitle=\color{white}\bfseries]
    \setstretch{1.2}
Put your final answer within \textbackslash boxed\{\}.\\\\
\{question\}
    \end{tcolorbox}
        \label{fig:gsm8k-zero-shot-prompt}
    }
    
    \subfloat[MATH prompt]{
    \begin{tcolorbox}[colframe=black!80!white, colback=black!2!white, boxrule=0.5mm, width=\textwidth, arc=2mm, auto outer arc, title=MATH (0-shot), fonttitle=\color{white}\bfseries]
    \setstretch{1.2}
Put your final answer within \textbackslash boxed\{\}.\\\\
\{problem\}
    \end{tcolorbox}
        \label{fig:math-zero-shot-prompt}
    }
    
    \caption{
    Evaluation prompts for GSM8K and MATH.
    The prompts are used to evaluate instruction-tuned models.
    }
    \label{fig:evaluation-prompts-math}
\end{figure}


\section{Qualititive Results} \label{sec:qualititive_results}

In this section, we include qualitative experimental results.
We mainly focus on changes made during post-training.
See \autoref{tab:completions_during_post_train_1} and \autoref{tab:completions_during_post_train_2} for representative examples.

\begin{CJK}{UTF8}{mj}

\begin{longtable}{lp{0.8\textwidth}}
\toprule
\textbf{Prompt} & \textbf{1 더하기 1은 귀요미야! 1 더하기 1은?} \\
\midrule
\endfirsthead

\multicolumn{2}{c}{\tablename\ \thetable\ -- \textit{Continued from previous page}} \\
\toprule
\textbf{Prompt} & \textbf{1 더하기 1은 귀요미야! 1 더하기 1은?} \\
\midrule
\endhead

\midrule
\multicolumn{2}{r}{\textit{Continued on next page}} \\
\endfoot

\bottomrule
\caption{Prompt and completions of each post-training stage.}
\label{tab:completions_during_post_train_1} \\
\endlastfoot

SFT & 1 더하기 1은 2입니다. \\\\

DPO & 1 더하기 1은 2입니다. 귀요미라는 말은 상황에 따라 다르게 해석될 수 있지만, 일반적으로는 '귀여운 것'을 의미합니다. 예를 들어, 아이들이 수학 문제를 재미있게 풀기 위해 "1 더하기 1은 귀요미" 같은 구호를 사용하기도 합니다. 하지만 수학적으로는 1 + 1 = 2입니다. \\\\
Online DPO & 1 더하기 1은 2입니다. 귀요미라는 표현은 수학적인 계산보다는 감정적인 표현으로 보입니다. 혹시 다른 질문이나 추가적인 설명이 필요하시면 알려주세요! \\

\end{longtable}

\begin{longtable}{lp{0.8\textwidth}}
\toprule
\textbf{Prompt} & \textbf{
Convert these dates to YYYY/MM/DD format:\newline
12/31/2021\newline
31-12-21\newline
01/02/2022\newline
02-01-22\newline
} \\
\midrule
\endfirsthead

\multicolumn{2}{c}{\tablename\ \thetable\ -- \textit{Continued from previous page}} \\
\toprule
\textbf{Prompt} & \textbf{
Convert these dates to YYYY/MM/DD format:\newline
12/31/2021\newline
31-12-21\newline
01/02/2022\newline
02-01-22\newline
} \\
\midrule
\endhead

\midrule
\multicolumn{2}{r}{\textit{Continued on next page}} \\
\endfoot

\bottomrule
\caption{Prompt and completions of each post-training stage.}
\label{tab:completions_during_post_train_2} \\
\endlastfoot

SFT & Here is a Python solution for the problem:

\begin{lstlisting}[language=Python, basicstyle=\scriptsize]
def convert_date(date):
    if '-' in date:
        date_parts = date.split('-')
        return f"{date_parts[2]}/{date_parts[1]}/{date_parts[0]}"
    else:
        date_parts = date.split('/')
        return f"{date_parts[2]}/{date_parts[0]}/{date_parts[1]}"
 
dates = ["12/31/2021", "31-12-21", "01/02/2022", "02-01-22"]
for date in dates:
    print(convert_date(date))
\end{lstlisting}

When you run this code, it will print: \newline\newline
2021/12/31\newline
2021/12/31\newline
2022/02/01\newline
2022/01/02\newline
This code works by checking if the input date contains a '-' character.
If it does, it assumes the date is in the format DD-MM-YYYY and converts it to YYYY/MM/DD.
If the input date does not contain a '-' character, it assumes the date is in the format MM/DD/YYYY and converts it to YYYY/MM/DD.

\\\\

DPO & Here is a Python solution for the problem:

\begin{lstlisting}[language=Python, basicstyle=\scriptsize]
def convert_date(date):
    if '/' in date:
        month, day, year = date.split('/')
        return f"{year}/{month.zfill(2)}/{day.zfill(2)}"
    elif '-' in date:
        day, month, year = date.split('-')
        return f"{year}/{month.zfill(2)}/{day.zfill(2)}"
    else:
        raise ValueError("Invalid date format")
 
dates = ["12/31/2021", "31-12-21", "01/02/2022", "02-01-22"]
 
for date in dates:
    print(convert_date(date))
\end{lstlisting}
This script will convert the dates to the desired format. The zfill(2) function is used to add leading zeros if the month or day is a single digit.\newline\newline
 
When you run this script, it will output:\newline
\newline
\newline
2021/12/31\newline
2021/12/31\newline
2022/01/02\newline
2022/01/02\newline
Note that the dates "31-12-21" and "02-01-22" are interpreted as "31 December 2021" and "2 January 2022" respectively, because the year is given as "21" and "22". If these dates are meant to represent "12-31-2021" and "1-2-2022" (i.e., December 31, 2021 and January 2, 2022), then the script will need to be modified to handle this interpretation.
\\\\
Online DPO & 
Sure, here are the dates converted to the YYYY/MM/DD format:\newline\newline
12/31/2021 $\rightarrow$ 2021/12/31\newline
31-12-21 $\rightarrow$ 2021/12/31\newline
01/02/2022 $\rightarrow$ 2022/01/02\newline
02-01-22 $\rightarrow$ 2022/01/02\newline
Note that the dates "31-12-21" and "02-01-22" are ambiguous and could potentially mean the opposite year, but based on common sense and the context provided, the most likely interpretation is the one given above (i.e., 2021 and 2022 respectively). If the exact year is unknown, it's crucial to have additional context.
\\

\end{longtable}

\end{CJK}

