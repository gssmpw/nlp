\subsection{Suboptimal extraction of open-source datasets}\label{appendix:suboptimal-arxiv-wiki}

\begin{figure}[h]
\centering
\begin{tcolorbox}[colframe=black!80!white, colback=black!2!white, boxrule=0.5mm, width=\textwidth, arc=2mm, auto outer arc, title=Example of suboptimal extraction from arXiv, fonttitle=\color{white}\bfseries]
\begin{lstlisting}[
  basicstyle=\ttfamily,
  breaklines=true,
  breakatwhitespace=false,
  columns=fullflexible,
  keepspaces=true,
  breakautoindent=false,
  breakindent=0ex
]
(...)
In this work, we use sine and cosine functions of different frequencies:

\begin{align*}
    PE_{(pos,2i)} = sin(pos / 10000^{2i/d_{\text{model}}}) \\
    PE_{(pos,2i+1)} = cos(pos / 10000^{2i/d_{\text{model}}})
\end{align*}

where $pos$ is the position and $i$ is the dimension.  That is, each dimension of the positional encoding corresponds to a sinusoid.  The wavelengths form a geometric progression from $2\pi$ to $10000 \cdot 2\pi$.  We chose this function because we hypothesized it would allow the model to easily learn to attend by relative positions, since for any fixed offset $k$, $PE_{pos+k}$ can be represented as a linear function of $PE_{pos}$.

We also experimented with using learned positional embeddings \citep{JonasFaceNet2017} instead, and found that the two versions produced nearly identical results (see Table~\ref{tab:variations} row (E)).  We chose the sinusoidal version because it may allow the model to extrapolate to sequence lengths longer than the ones encountered during training.

\section{Introduction}

\documentclass[../main.tex]{subfiles}
\graphicspath{{../images/}}
\makeatletter
\def\input@path{{../images/}}
\makeatother
\begin{document}
\section{Introduction}
\begin{figure}
\centering
\begin{tikzpicture}
\node[inner sep=0pt] (ws) at (0, 0) {
\includegraphics[height=.4\textwidth, trim={10cm 0 10cm 0},clip]{world_space.png}};
\node[inner sep=0pt] (cs) at (6,0) {\includegraphics[height=.4\textwidth, trim={10cm 1cm 10cm 4cm},clip]{conf_space.png}};
\end{tikzpicture}
\vspace{-5pt}
\label{fig:pbrm_intro}
\caption{\textbf{Left}: Shows world space obstacles as grey spheres. Robots start and goal configuration is colored red and green, respectively. Configurations along the computed path are colored transparent blue. \textbf{Right:} Mapped world space scenario to configuration space. Obstacle region is the grey mesh. Red spheres are collision-free regions computed by the neural SCDF. The optimized shortest path in the convex corridor is the blue curve.}
\vspace{-25pt}
\end{figure}
Motion planning is the problem of finding a collision-free trajectory that connects a given start and goal configuration. The planning takes place in the configuration space of the robot. For single body robots, like mobile robots or drones, the configuration space and the world space are usually the same. This simplifies the planning, since explicit obstacle representations are available which enables geometrical tools like separating hyperplanes, smallest distance to obstacles etc., to be used when designing motion planning algorithms. For multi-body robots like manipulators, the situation is completely different. The world space obstacles are usually mapped to non-convex regions, and to make the problem even harder, the mapping is usually not known. Forming explicit representations of the obstacle region in the configuration space is usually too expensive or intractable. Despite all of this, sampling based planners are used with great success, which mainly is due to their use of implicit representations of the obstacle region. The basic idea is to construct a graph in the configuration space that covers and connects the collision-free region. From this graph, a path can be extracted that connects a given start and goal configuration. The approach is computationally expensive, since the graph is constructed with the smallest geometrical building block available, points, which represents a collision-check. Furthermore, the extracted paths from the graph are non-smooth and jagged due to the stochastic nature of the approach. This adds an additional post-processing step to the process, where the paths are shortcutted and smoothened, before the path can be used for tracking. Clearly a lot of time is invested to form this graph and produce smooth paths. Thus, if the obstacles start to move, then all of this work is done in no use, since all points that make up this graph need to be re-verified, which is simply too time consuming to be done in real time.
\\\\
In this work, we want to address the existing drawbacks of the sampling based planners. Our main contribution is an improved motion planner where each vertex in the graph covers a collision-free region in the form of a sphere instead of a point and where the edges are formed with neighboring intersecting spheres. This representation has the advantage of instead of returning piecewise linear paths, returning a sequence of overlapping spheres, i.e. a convex corridor, that connects a given start and goal configuration, illustrated in Figure \ref{fig:pbrm_intro}. This convex corridor allows us to use convex optimization to produce smooth trajectories, instead of computationally expensive post-processing methods. The representation further allows us to estimate the coverage of the collision-free space, which gives us awareness and feedback in the offline roadmap construction phase. Finally, our representation is simple to adapt to moving obstacles, simply requery for the new radii and recheck for intersections. 
\\\\
The spherical collision-free regions are formed using a signed distance function (SDF), which is a function that returns the smallest distance from an arbitrary point to the boundary of an obstacle. As the name implies, the distance is signed, thus if the point is inside the obstacle it is negative otherwise positive. If the distance is positive, a sphere with radius equal to the distance is guaranteed to cover a collision-free region. Using an SDF in motion planning is not new, but what is novel about our approach is that we express the distance in the configuration space instead of the world space and by doing so allows us to form these convex collision-free regions. We refer to the resulting SDF as a signed configuration distance function (SCDF). Computing an SCDF analytically is non-trivial, our approach is therefore to parameterize the SCDF with a deep neural network and learn the mapping by supervised learning. Our resulting neural SCDF can compute distances for different parameter values of obstacle shapes and we also show how multiple distances can be combined, thus making our approach flexible.
\section{Related work}
Motion planning algorithms can roughly be divided into three families, grid-based, sampling based and optimization based methods. Grid-based methods (GBM) discretize the planning space from which a graph is then compiled. A standard search method is A$^\star$ \citep{a_star}, which is classified as an \textit{informed} search method, since it employs a heuristic function to speed up the search. A$^\star$ guarantees to return an optimal path at the level of discretization used. GBMs usually discretize the planning space by a regular lattice and this limits the GBMs to problems with low dimensionality due to the curse of dimensionality. Thus, GBMs are usually limited to single-body robots where the degrees of freedom (DOF) are low. To overcome the inherent scaling problem with the GBMs, stochastic methods are usually used for multi-body robots. These methods are termed as sampling-based methods (SBM) and core members within this family are the rapidly-exploring random trees (RRT) \citep{rrt} and the probabilistic roadmap (PRM) \citep{prm}. RRT grows a tree from the start configuration and explores the collision-free region in a rapid way until it is able to connect to the goal region. RRT is usually improved by bi-directional planning \citep{rrt_connect}, i.e. an additional tree is grown from the goal configuration and the trees are tested for connection after any tree has been expanded. RRT is a single-query method, thus it searches for a path from scratch each time it is queried. Contrary to this, PRM is a multi-query method, which solves for multiple queries without starting from scratch. PRM does this by creating a roadmap (graph) that covers the collision-free space as an offline step. The graph is then used to solve for multiple queries. PRMs are used in cases where the environment does not change since the extra offline step is too computationally costly and needs to be re-done if the environment is changed. In our work, we address this inherent issue by using a different roadmap representation. Our vertices in the graph cover a collision-free region in the form of spheres and we form the edges by checking for intersecting spheres. If something in the environment changes, we recompute the spheres radii and recheck the intersections, without relying on collision detection. We use a trained neural network to compute the sphere radius, therefore querying for the radius can be done fast, hence our representation enables the PRM for dynamic environments.
\\\\
In the recent decades, optimization based methods (OBM) \citep{chomp, schulman, itomp, stomp} have been introduced as an alternative to SBM for multi-body robots. Like the SBM, the OBMs scale well to higher dimensional problems and produce smoother motion. It is common to use a SDF in the optimization since it is a smooth function, thus enabling gradient-based methods. However, the standard way of expressing the SDF is in world space. The distance therefore needs to be mapped to the configuration space by the forward kinematics. This mapping makes the optimization problem a non-linear program (NLP), which is computationally expensive to solve. Recently, a different approach has been proposed. In \cite{mp_gcs} motion planning is formulated as a convex optimization problem by using the graph of convex sets framework \citep{gcs}. The underlying idea is to decompose the collision-free space into intersecting convex sets from which a convex optimization problem is formulated. In cases where an explicit representation of the obstacles in the configuration space exists, like for single-body robots, creating collision-free convex regions can be done fast \citep{iris}. For multi-body robots, this is non-trivial. Existing work does this successfully \citep{iris_nlp, iris_c} by an optimization based approach, but the methods are still too time consuming to be used in the presence of moving obstacles. Our approach is instead to use deep learning to learn an SDF expressed in the configuration space. With this, we can query for shortest distances to the collision boundary, which allows us to expand spherical regions which are collision-free. Our approach is fast and therefore enables our suggested roadmap planner to be used in dynamic environments.
\\\\
Recent research has focused on learning collision detection \citep{fk_kernel_distance, diffco, graphdistnet} by predicting the signed distance between the robot links and the surrounding obstacles in the world space. The learned SDF is used in trajectory optimization but since the distance is expressed in the world space, the problem becomes an NLP and therefore takes a long time to solve. We take a novel approach and suggest to instead express the signed distance in the configuration space. This allows us to improve the PRM at the same time as it enables convex optimization for trajectory optimization, which runs faster and is more reliable than NLP solvers. In \cite{cspf} a learned signed distance function in the configuration space is proposed similar to our approach. However, their approach is restricted to point cloud representations, while we propose to represent the obstacles as parameterized geometric shapes, e.g. spheres. Furthermore, we also show how to use our learned SCDF to improve an existing roadmap planner.
\section{Problem formulation}
A robot is located in the world space, $\W \subset \R^3 $. The unique location of the robot is given by its configuration $\q \in \C$, where $\C$ is the configuration space. The set of points covered by the robots bodies at a certain configuration is expressed as $\B(\q) \subset \W$. The robot is surrounded by $\NrObst$ obstacles $\O = \bigcup_{i=1}^{\NrObst} \O_i$, where  $\O_i \subset \W$. The representation of the obstacle in the configuration space is the set $\C\O_i = \{\q \in \C \: |\: \B(\q) \cap \O_i \neq \emptyset \}$. The obstacle space is formed as $\Co = \bigcup_{i=1}^{\NrObst} \C \O_i$. The complement is referred to as the free space, $\Cf = \C \setminus \Co$. The path planning problem is a tuple, ($\Cf$, $\qStart$, $\qGoal$), where we want to connect a query pair, consisting of a start, $\qStart$, and goal configuration, $\qGoal$, with a geometric path, $\q(s): [0, 1] \mapsto \Cf$, such that $\q(0)=\qStart$ and $\q(1)=\qGoal$, or report correctly when such a path does not exist.
\end{document}


\section{Background}

\section{Basic Background: Supervised Learning and the PAC Model}
\label{sec:background}

At this point almost everyone has heard of machine learning (ML). Anyone likely to stumble upon this article will have also heard of its most influential special case, supervised learning, and those theoretically inclined will also be familiar with the PAC model. Nonetheless, I will set the stage by  recapping the basics.

\subsection{Basics of Supervised Learning}%Let's set the stage in any case

\emph{Supervised Learning} is the task of ``coming up'' with a function $f: \X \to \Y$ to ``explain'' or ``fit'' a sequence of input/output examples   $(x_1,y_1), \ldots, (x_n,y_n)$, with $x_i \in \X$ and $y_i \in \Y$.  Here $\X$ is a \emph{data domain} consisting of \emph{datapoints} $x \in \X$, $\Y$ is a \emph{label set} consisting of \emph{labels} $y \in \Y$, and the sequence $(x_1,y_1),\ldots,(x_n,y_n)$ is the \emph{training data} consisting of \emph{labeled examples (a.k.a. samples)}~$(x_i,y_i)$.  I~will refer to the chosen function $f$ as a \emph{predictor}, and to $n$ as the \emph{sample size}. A \emph{learning algorithm} takes as input training data, and outputs (some representation of) a predictor $f \in \Y^\X$.\footnote{Note that this describes the usual \emph{batch}, a.k.a.~\emph{offline}, setting of supervised learning. I do not discuss other paradigms such as online or active learning in this article.} 



Success in supervised learning is defined as \emph{generalization} to  future examples: For a typical \emph{test example}  $(x_{\tst},y_{\tst})$, the predicted label $y'_{\tst}=f(x_{\tst})$ should ``equal'' $y_{\tst}$, perhaps approximately. We usually assume the test example is drawn from the same  ``source'' as the training data  --- commonly, i.i.d.~from the same distribution. The quality of the prediction is quantified by $\ell(y'_{\tst},y_{\tst})$, where $\ell:~\Y~\times~\Y \to \RR_{\geq 0}$ is a \emph{loss function} chosen as part of the problem definition. Common loss functions include the 0-1 loss $\ell_{0-1}(y',y) = [y' \neq y]$ for \emph{classification} problems,\footnote{The notation $[P]$ denotes $1$ when predicate $P$ is true, and denotes $0$ when $P$ is false.} as well as the absolute loss $|y'-y|$ or squared loss $(y'-y)^2$ for \emph{regression problems} featuring $\Y  \sse \RR$.

Nontrivial generalization properties are typically only possible if one assumes something about the data.\footnote{The need for such an assumption is formalized by the  \emph{no free lunch theorems} of supervised learning \cite{wolpert_connection_1992,wolpert_lack_1996,schaffer_conservation_1994}.} The Bayesian approach to  machine learning, common in many applications, assumes some parametric form for the distribution generating the data, and postulates a prior on the parameters. This is not the approach I will take in this article. Instead, I will focus on the frequentist --- and some would say ``worst-case'' or ``adversarial'' ---  approach that is common in the computational learning theory community, embodied by the PAC model. Here we assume that the (training and test) data can be explained, perhaps approximately, by a function in some ``simple enough to learn'' class of functions $\H \sse \Y^\X$, often called the \emph{hypotheses}. Equivalently, we  seek a predictor which explains the unseen data roughly  as well as the best hypothesis $h^* \in \H$, whether or not we assume that $h^*$ itself provides a perfect explanation.



 \paragraph{Common Algorithmic Templates.} Perhaps the best known general-purpose supervised learning algorithm is \emph{empirical risk minimization (ERM)}, which chooses as its predictor a hypothesis $f \in \H$ minimizing $\frac{1}{n} \sum_{i=1}^n \ell(f(x_i),y_i)$ --- a quantity called the \emph{training error}, \emph{empirical error}, or \emph{empirical risk} of $f$. %\footnote{When multiple hypotheses minimize the empirical risk, we assume ERM breaks ties arbitrarily.}
A common template for generalizing ERM involves adding a \emph{regularization term} $\psi(f)$ to the  objective function, typically chosen to measure some notion of ``hypothesis complexity.'' An algorithm instantiating this template is known as a \emph{structural risk minimizer (SRM)}, and chooses as its predictor the hypothesis $f \in \H$ minimizing the \emph{structural risk} $\frac{1}{n} \sum_{i=1}^n \ell(f(x_i),y_i) + \psi(f)$. Other well-known algorithms, such as gradient descent and its variations,  can frequently be interpreted as approximate implementations of ERM or SRM.


\paragraph{Proper vs Improper Learning.} A learning algorithm is said to be \emph{proper} if its predictor $f$ is always chosen from the hypothesis class, i.e., $f \in \H$, otherwise it is said to be \emph{improper}. ERM  is an example of a proper learning algorithm, as are SRM algorithms of the form described above.  In the \emph{proper regime} of learning, algorithms are required to be proper. This article will be concerned with the more flexible \emph{improper regime} (a.k.a \emph{representation-independent learning}), where no such constraint is placed on the learner. In other words, all we care about is predictive power at test time, rather than any insights derived from the functional form or representation of the predictor~itself.


\subsection{The PAC Model}
A standard mathematical setup for evaluation of supervised learning algorithms, at least in the theoretical computer science community, is Valiant's \emph{Probably Approximately Correct (PAC) model} of learning (see e.g.~\cite{kearns_introduction_1994,mohri_foundations_2018}). Here, we assume there is an unknown distribution $\D$ on $\X \times \Y$ from which training and test data are  drawn.  Specifically, the labeled datapoints of the training set  $(x_1,y_1), \ldots, (x_n,y_n)$, as well as the test data  $(x_\tst,y_\tst)$, are i.i.d.~from $\D$. Often it is assumed that $\D$ lies in some class of distributions of interest. The \emph{true expected loss}, or simply \emph{loss}, of a predictor $f: \X \to \Y$ is the expected loss it incurs on draws from $\D$, written $L_\D(f) = \Ex_{(x,y) \sim \D} \ell(f(x),y)$.


There are two main ``settings'' in PAC learning. The  \emph{realizable setting} only requires that the data be perfectly explained by some hypothesis in $\H$. More generally, the \emph{agnostic setting} makes no assumption relating the data to the hypotheses, but shifts the goalposts as necessary to allow nontrivial guarantees: the expected loss at test time is evaluated only ``relative'' to that of the best hypothesis $h^* \in \H$. There are other settings which make more nuanced assumptions, such as $\D$ being of a particular parametric form or its support living in some (unknown) lower-dimensional space, etc. I will mostly discuss the realizable and agnostic settings in this article, those being the simplest and most studied from a theoretical perspective. %TODO:We will briefly discuss other settings in Section ??

The PAC model demands high probability guarantees of learners, in the worst case over distributions of interest. Consider first the realizable setting, where $\D$ is such that $\min_{h \in \H} L_{\D}(h) = 0$. A PAC learner has \emph{error} $\epsilon=\epsilon(n)$ and \emph{confidence} $\delta=\delta(n)$ if, when training data consists of $n$ i.i.d~samples from a realizable distribution $\D$, it produces a predictor $f$  satisfying $L_\D(f) \leq \epsilon$ with probability at least $1-\delta$. In the agnostic setting, where $\D$ can be arbitrary, we require $L_\D(f) - \min_{h \in \H} L_\D(h) \leq \epsilon$ with probability $1-\delta$.

In both the realizable and agnostic settings, we look for PAC learners with small $\epsilon$ and $\delta$ as a function of the sample size $n$. An equivalent perspective looks at the sample complexity $m(\epsilon,\delta)$, which is the minimum sample size which guarantees error  at most $\epsilon$ with probability at least $1-\delta$. We say a problem is \emph{PAC learnable} if its PAC sample complexity is finite whenever $\epsilon,\delta > 0$.

For most PAC learning problems, learnability and sample complexity are characterized in terms of a  ``dimension'' of the hypothesis class. Most prominently this is the \emph{VC dimension} for binary classification, the \emph{fat shattering dimension} for agnostic regression, and the \emph{DS dimension} for multiclass classification (see \cite{anthony_neural_1999,daniely_optimal_2014,brukhim_characterization_2022}). Treatment of these is beyond the scope of this article. The unfamiliar reader need not worry, however,  as dimensions will feature only tangentially in our~discussion.




%\paragraph{Learning settings: Realizable, Agnostic, etc.} In learning theory, evaluating a supervised learning algorithm requires specifying a data model and an objective. We will leave the details of the data model flexible for now, to allow for both the PAC model and the adversarial transductive model. Nonetheless we will describe two variations, which we call ``settings'', which cut across different models. The  \emph{realizable setting}  requires only that the data be perfectly explained by some hypothesis $h \in \H$ --- i.e., there exists a hypothesis which is guaranteed to suffer a loss of $0$ on training and test data. The performance of the learning algorithm is its expected loss at test time for some ``worst case'' realizable instance. More generally, the \emph{agnostic setting} makes no assumption relating the data to the hypotheses, but shifts the goalposts as necessary to allow nontrivial guarantees: the expected loss at test time is evaluated only ``relative'' to that of the best hypothesis $h^* \in \H$, again for some ``worst case'' instance. There are other settings which make more nuanced assumptions about the data, such as it is drawn from a distribution of a particular parametric form, or that it lives in some (unknown) lower-dimensional space, etc. We will mostly discuss the realizable and agnostic settings, those being the simplest and most studied from a theoretical perspective.




%%% Local Variables:
%%% mode: latex
%%% TeX-master: "learning_matching"
%%% End:


\section{Model Architecture}

\section{Model Architectures}
\label{sect:architecture}

The principal architecture introduced in this work is the \nomdef{Long Short-Term Memory AE with Soft Actor-Critic}{\lstmaesac}. This model leverages a RAE, specifically an LSTM-based AE as previously defined in \autoref{sect:rae}, to handle path history feature extraction. The LSTMAE is coupled with the \nomdef{Soft Actor-Critic}{SAC} algorithm, which has demonstrated robust performance in continuous action spaces with large observation spaces. The \lstmaesac architecture is designed to efficiently process temporal information from the agent's path history, potentially reducing the need for an extremely large network as observed in previous work \citep{ewers_deep_2025}.

\begin{table*}[htb!]
    \centering
    \caption{Definitions of architectures}
    \label{tbl:method:definitions_of_architectures}
    \begin{tabular}{@{}l|lL{3cm}L{2.5cm}|L{2.5cm}@{}}
        \toprule
        Architecture title & RL Algorithm & Path Observation Augmentation & Inner-model Feature Extraction & Observation Space                                \\ \midrule
        \lstmaesac         & SAC          & LSTM-AE                       & None                           & $\left( z_\mathrm{path}, s_\mathrm{PDM} \right)$ \\ \midrule
        \lstmaeppo         & PPO          & LSTM-AE                       & None                           & $\left( z_\mathrm{path}, s_\mathrm{PDM} \right)$ \\
        \lstmppo           & RPPO         & None                          & None                           & $\left( s_\mathrm{pos}, s_\mathrm{PDM}\right)$   \\
        \fssaclstm         & SAC          & Frame Stacking                & LSTM                           & $\left( s_\mathrm{path},s_\mathrm{PDM}\right)$   \\
        \fssacfcn          & SAC          & Frame Stacking                & Fully Connected Network        & $\left( s_\mathrm{path},s_\mathrm{PDM}\right)$   \\
        \fssacconvtwod     & SAC          & Frame Stacking                & 2D Convolution                 & $\left( s_\mathrm{path},s_\mathrm{PDM}\right)$   \\
        \fsppolstm         & PPO          & Frame Stacking                & LSTM                           & $\left( s_\mathrm{path},s_\mathrm{PDM}\right)$   \\
        \fsppofcn          & PPO          & Frame Stacking                & Fully Connected Network        & $\left( s_\mathrm{path},s_\mathrm{PDM}\right)$   \\
        \fsppoconvtwod     & PPO          & Frame Stacking                & 2D Convolution                 & $\left( s_\mathrm{path},s_\mathrm{PDM}\right)$   \\ \bottomrule
    \end{tabular}
\end{table*}

To comprehensively evaluate the efficacy of \lstmaesac, a suite of comparative architectures were developed. These vary in their DRL algorithms, path observation augmentation techniques, and inner-model feature extraction methods. The key variants are defined in \autoref{tbl:method:definitions_of_architectures} and were designed to systematically explore the impact of different components:
\begin{itemize}
    \item Efficacy of path observation augmentation,
    \item Differences in DRL algorithm,
    \item Impact on inner-model feature extraction in lue of path observation augmentation.
\end{itemize}

\begin{figure*}[htbp]
    \centering

    \tikzset{
        block/.style = {draw, fill=white, rectangle, minimum height=3em},
        input/.style = {fill=none, rectangle},
        output/.style= {fill=none, rectangle},
    }
    \subfloat[Policy for the LSTMAE variation. Note that the LSTM encoder module is frozen and its parameters do not get updated during training. \label{fig:method:lstmae_policy}]{
        \begin{tikzpicture}[]
            {[start chain]
                \node [input, on chain] (spath) {$s_\mathrm{pos}$};
                \node [block, ml/encoder, on chain, dashed] (lstm_e) {LSTM AE};
                \node [sum, on chain] (concatenate)  {};
                \node [block, on chain] (policy) {FCN};
                \node [output, on chain] (action) {$a$};
            }
            \node [input, above of=spath, densely dotted] (h)  {$h_{t-1}$};
            \draw [->, ml/weights/arrow] (h) -| (lstm_e);
            \node [input, below of=spath] (spdm)  {$s_\mathrm{PDM}$};
            \draw [->, ml/weights/arrow] (spdm) -| (concatenate);
            \path (concatenate) -- node[above] {$s$}(policy);
            \path (lstm_e) -- node[above] {$z$}(concatenate);
        \end{tikzpicture}
    }
    \hfill
    \subfloat[RPPO policy where $h_{t-1}$ is the LSTM hidden state from the previous time-step.\label{fig:method:rppo_policy}]{
        \begin{tikzpicture}[]
            {[start chain]
                \node [input, on chain] (spath) {$s_\mathrm{pos}$};
                \node [sum, on chain] (concatenate)  {};
                \node [block, on chain] (policy_lstm) {LSTM};
                \node [block, on chain] (policy_fcn) {FCN};
                \node [output, on chain] (action) {$a$};
            }
            \node [input, above of=spath] (h)  {$h_{t-1}$};
            \node [input, below of=spath] (spdm)  {$s_\mathrm{PDM}$};
            \draw [->, ml/weights/arrow] (spdm) -| (concatenate);
            \draw [->, ml/weights/arrow, densely dotted] (h) -| (policy_lstm);
            \path (concatenate) -- node[above] {$s$}(policy_lstm);
        \end{tikzpicture}
    }
    \hfill
    \subfloat[FCN inner-model feature extraction.\label{fig:method:fcn_policy}]{
        \begin{tikzpicture}[]
            {[start chain]
                \node [input, on chain] (spath) {$s_\mathrm{path}$};
                \node [ml/encoder, on chain] (path_fe) {FCN};
                \node [sum, on chain] (concatenate)  {};
                \node [block, on chain] (policy) {FCN};
                \node [output, on chain] (action) {$a$};
            }
            \node [input, below of=spath] (spdm)  {$s_\mathrm{PDM}$};
            \draw [->, ml/weights/arrow] (spdm) -| (concatenate);
            \path (concatenate) -- node[above] {$s$}(policy);
        \end{tikzpicture}
    }
    \vfill
    \subfloat[LSTM inner-model feature extraction.\label{fig:method:lstm_policy}]{
        \begin{tikzpicture}[]
            {[start chain]
                \node [input, on chain] (spath) {$s_\mathrm{path}$};
                \node [ml/encoder, on chain] (path_fe_lstm) {LSTM};
                \node [ml/encoder, on chain] (path_fe_fcn) {FCN};
                \node [sum, on chain] (concatenate)  {};
                \node [block, on chain] (policy) {FCN};
                \node [output, on chain] (action) {$a$};
            }
            \node [input, below of=spath] (spdm)  {$s_\mathrm{PDM}$};
            \node[rectangle, inner sep=0.2cm, draw=none, fit=(path_fe_lstm)] (path_fe_lstm_border) {};
            \draw [->, ml/weights/arrow, densely dotted]
            (path_fe_lstm.east)++(0,0.15)
            -- ($(path_fe_lstm_border.east)+(0,0.15)$)
            -- (path_fe_lstm_border.north east)
            -- node[above] {$h$} (path_fe_lstm_border.north west)
            -- ($(path_fe_lstm_border.west)+(0,0.15)$)
            -- ($(path_fe_lstm.west)+(0,0.15)$);

            \path (concatenate) --node[above] {$s$} (policy);
            \draw [->, ml/weights/arrow] (spdm) -| (concatenate);
        \end{tikzpicture}
    }
    \hfill
    \subfloat[2D CNN inner-model feature extraction.\label{fig:method:cnn_policy}]{
        \begin{tikzpicture}[]
            {[start chain]
                \node [input, on chain] (spath) {$s_\mathrm{path}$};
                \node [ml/encoder, on chain] (path_fe_cnn) {2D CNN};
                \node [ml/encoder, on chain] (path_fe_fcn) {FCN};
                \node [sum, on chain] (concatenate)  {};
                \node [block, on chain] (policy) {FCN};
                \node [output, on chain] (action) {$a$};
            }
            \node [input, below of=spath] (spdm)  {$s_\mathrm{PDM}$};

            \path (concatenate) --node[above] {$s$} (policy);
            \draw [->, ml/weights/arrow] (spdm) -| (concatenate);
        \end{tikzpicture}
    }
    \caption{The five proposed policy architectures for use with either PPO, RPPO, or SAC. Figure \protect\subref{fig:method:rppo_policy} is only used with RPPO.}
    \label{<label>}
\end{figure*}

% \begin{figure}[htbp]
%     \centering
%     \subfloat[
%         2D convolution
%         \label{fig:buffered_linestring_with_overlap}
%     ]{
%         \includegraphics[height=6cm]{path_feature_extractor_conv2d.pdf}
%     }
%     \hfill
%     \subfloat[
%         LSTM
%         \label{fig:buffered_linestring_triangulated}
%     ]{
%         \includegraphics[height=6cm]{path_feature_extractor_lstm.pdf}
%     }
%     \hfill
%     \subfloat[
%         Fully cconnected network
%         \label{fig:buffered_linestring_triangulated}
%     ]{
%         \includegraphics[height=6cm]{path_feature_extractor_fcn.pdf}
%     }
%     \caption{Visualizations of concepts related to the buffered polygon representation of the seen area.}
%     \label{fig:buffered_linestring}
% \end{figure}

\section{Why Self-Attention}
\input{why_self_attention}

\section{Training}

% \section{With what \emph{data}?}\label{sec:data}
\section{How do we \emph{train models}?}\label{sec:model_training}

% Which scales? How? 
%  - [DONE] Too small may not hold
%  - [DONE] number of models can affect confidence intervals
%  - [DONE] counting size also matters data (tokens vs bits) and flop counts kaplan vs hoffman CITE; non-embedding vs embedding 
%  - DITCH? how you scale architecture matters; for eg - clark shows scaling N and E is needed to separate; enc and decoder scaling can be different for architectures; in the 
%  - [DONE] 6ND is a common approximation - does not hold for very long context (these days 128k to 1M)


In order to fit a scaling law, one needs to train a range of models across multiple orders of magnitude in model size and/or dataset size. Researchers must first decide the range and distribution of $N$ and $D$ values for their training runs, in order to achieve stable convergence to a solution with high confidence, while limiting the total compute budget of all experiments. Many papers did not specify the number of data points used to fit each scaling law; those that did range from 4 to several hundred, but most used fewer than 50 data points. The specific $N$ and $D$ values also skew the optimization process towards a certain range of $N/D$ ratios, which may be too narrow to include the true optimum. Some approaches, such as using IsoFLOPs \citep{hoffmann2022training}, additionally dictate rules for choosing $N$ and $D$ values. Moreover, using a minimum $N$ or $D$ value may result in outlier values that may need to be dropped \citep{porian2024resolving,shin2023scaling,henighan2020scaling}. We investigate this choice in Section \S\ref{sec:repl-model_training}

The definition of $N$, $D$, or compute cost $C$ can affect the results of a scaling study. For example, if a study studies variation in tokenizers, a definition of training data size based on character count may be more appropriate than one based on token count \citep{tao2024scaling}. The inclusion or exclusion of embedding layer compute and parameters, may also skew the results of a study - a major factor in the different in optimal ratios determined by \cite{kaplan2020scaling} and \cite{hoffmann2022training} has been attributed to not factoring embedding FLOPs into the final compute cost \citep{pearce2024reconcilingkaplanchinchillascaling, porian2024resolving}. Given the increase in extremely long context models (128k-1M) \cite{reid2024gemini}, the commonly used training FLOPs approximation $C = 6 ND$ (see Appendix \ref{app:full-details}) may not hold for such models, given the additional cost proportional to the context length and model dimension - \citet{bi2024deepseek} introduce a new terms non-embedding FLOPs/token to account for this.

% \ml{we've gotta decide where this discussion should go }

% \luke{I agree we could probably cut some of the next few paragraphs for space if needed. The last paragraph ends well though I think.}
% \srk{ DITCH? how you scale architecture matters; for eg - clark shows scaling N and E is needed to separate; enc and decoder scaling can be different for architectures}

 % - counting size also matters data (tokens vs bits) and flop counts kaplan vs hoffman CITE; non-embedding vs embedding 

% The goal of scaling laws is generally to extrapolate findings to larger compute budgets. It is unclear which $N$ and $D$ values should be included in the data in order to predict loss at a larger scale. 


% \srk{discuss outliers being dropped and therefore, need to be sure about the scale of training}



% The scaling law form identifies the relevant variables (e.g., $N$ and $D$). However, there remain many decisions affecting the way in which data is selected to for scaling law optimization.

%  - [DONE] knowing data composition matters because knowing data quality can change exponent across different studies ofc CITE


% Moreover, hyperparams can matter
%  - [DONE] For example, learning rate schedule can changes results CITE
%  - [DONE] batch size can change
%  - [DONE] optimal hparams change with scale so determining those matters
%  - [DONE] embedding size has been shown to matter - part of scaling law 



Scaling law fit depends on the performance of each individual checkpoint, which is highly dependent on factors such as training data source, architecture and hyperparameter choice. \citet{bansal2022data} and \citet{goyal2024scaling}, for instance, discuss the effect of data quality and composition on power law exponents and constants. Repeating data has also been found to yield different scaling patterns in large language models \citep{muennighoff2024scaling,goyal2024scaling}. 

Researchers have also studied the effect of architecture choice on scaling - \citet{hestness2017deep} find that architectural improvements only shift the irreducible loss, while \citet{poli2024mechanistic} suggest that these improvements may be more significant. The way in which a model is scaled can also affect results. Within the same architecture family, \citet{clark2022unified} show that increasing the number of experts in a routed language model has diminishing returns beyond a point, while \citet{ghorbani2021scaling} find that scaling the encoder and decoder have different effects on model performance. Scaling embedding size can also drastically change scaling trends \citep{tao2024scaling}.

The optimal hyperparameters to train a model changes with scale. Changing batch size, for example, can change model performance \cite{mccandlish2018empirical, kaplan2020scaling}. Optimal learning rate is another hyperparameter shown to change with scale, though techniques such as those proposed in Tensor Programs series of papers \citep{yang2022tensor} can keep this factor constant with simple changes to initialization. More specifically, changing the learning rate schedule from a cosine decay to a constant learning rate with a cooldown (or even changing the learning rate hyperparameters) has been found to greatly affect the results of scaling laws studies \citep{hu2024minicpm,porian2024resolving,hagele2024scaling, hoffmann2022training}.

% \ml{add paragraph about max model params}

% \srk{talk abt }

% - LR \citep{hu2024minicpm}.
% - batch size

% Some factors, such as training data distribution, do not have a single optimal value and may simply be held constant for all experiments, but they change the absolute value of all power law parameters, e.g. it is not meaningful to compare power laws fit to experiments on different training data distributions. Other factors, such as batch size, may have optimal values depending on other hyperparameters, data budget, architecture, etc. It is sometimes possible to fix these factors in relation to others (e.g., scale batch size with the parameter count). There are other factors for which there is no smooth interpolation between points. For example, the width and depth of a transformer are limited to integer values, and further limited by the convention of using widths equal to small multiples of powers of 2.

% \srk{we should concretely discuss learning rate schedules and cite miniCPM, the LR schedule rate paper}
One common motivation for fitting a scaling laws is extrapolation to higher compute budgets. However, there is no consensus on the orders of magnitude up that one can project a scaling law and still find it accurate, nor on the breadth of compute budgets that should be covered by the data. We find that the range of model size $N$ and dataset size $D$ greatly varies, with the maximum value of $N$ in each paper ranging from 10M parameters to around 7B and that of $D$ being as large as 400B tokens. 
% \textcolor{blue}{
For most papers we survey, the scales are relatively modest: 13 of 51 papers train models beyond 2B parameters; most only train models smaller than 1B parameters.
% }
% \srk{idk what this refers to: Some papers show projected results at as much as 6 or 7 orders of magnitude greater than the compute budgets they experimented with. } 
It has been shown, with some controversy \cite{schaeffer2023emergent}, that scaling to significantly larger scales can result in new abilities that did not appear in smaller models \citep{wei2022emergent}. Forecasting limits to extrapolation and the appearance of new abilities at new scales is an open question.

% 
% - while SOTA models go till 100s of B; 
% - scaling laws often only go until 1b
% - it remains unclear what the corrcet scale is

% \srk{mention emergent abilities; and counterpt}

% \srk{Training data [DONE] and downstream metrics chosen may completely change exponent?}

\section{Results} \label{sec:results}
\begin{table}[ht!]
\centering
\caption{\textbf{Super Resolution Performance Results.} Our proposed WGAN EEG Spatial Upsampling method significantly outperforms a baseline of Bicubic Interpolation commonly used in EEG upsampling pipelines.}
\label{tab:results}
\resizebox{0.8\linewidth}{!}{%
\begin{tabular}{@{}cccccc@{}}
\toprule
\multirow{2}{*}{\textbf{Dataset}} & \multirow{2}{*}{\textbf{Scale}} & \multicolumn{2}{c}{\textbf{Bicubic}} & \multicolumn{2}{c}{\textbf{WGAN}} \\ \cmidrule(l){3-6} 
                      &   & \textbf{MSE} & \textbf{MAE} & \textbf{MSE}    & \textbf{MAE}   \\
\toprule
\multirow{2}{*}{Val}  & 2 & 3.71E7       & 3.89E3       & \textbf{2.01E3} & \textbf{24.38} \\
                      & 4 & 7.23E7       & 6.42E3       & \textbf{8.53E3} & \textbf{63.83} \\
\midrule
\multirow{2}{*}{Test} & 2 & 3.75E7       & 3.91E3       & \textbf{2.06E3} & \textbf{24.66} \\
                      & 4 & 7.30E7       & 6.45E3       & \textbf{8.68E3} & \textbf{64.39} \\
\bottomrule
\end{tabular}%
}
\end{table}

\section{Conclusion}
In this work, we presented the Transformer, the first sequence transduction model based entirely on attention, replacing the recurrent layers most commonly used in encoder-decoder architectures with multi-headed self-attention.
(...)
\end{lstlisting}
\end{tcolorbox}
\caption{
    Example of suboptimal extraction from arXiv subset of \citet{redpajama}. 
    The original content is from \citet{Vaswani+2017}.
}
\end{figure}

\begin{figure}[h]
\centering
\subfloat[Open-source]{
\begin{tcolorbox}[colframe=black!80!white, colback=black!2!white, boxrule=0.5mm, width=\textwidth, arc=2mm, auto outer arc, title=Example of suboptimal extraction from Wikipedia, fonttitle=\color{white}\bfseries]
\begin{CJK}{UTF8}{mj}
발생원인\\
\\
회전 좌표계\\
\\
회전좌표계
좌표계 x, y, z와 좌표계 x', y', z'을 보자 두 좌표계의 원점은 같다. 각각의 경우에 대해 벡터 .은 두 좌표계에서 다음과 같이 표시된다.\\
\\
. (x, y, z 좌표)\\
\\
. (x', y', z' 좌표계)\\
\\
벡터의 내적을 이용해 x, y, z를 (.), (.), (.)으로 표현할 수 있다. 내적의 방법은 다음과 같다.\\
\\
.\\
\\
.\\
\\
.
\\
으로 표현되는 것을 확인할 수 있다.
\end{CJK}
\end{tcolorbox}
}

\subfloat[Improved]{
\begin{tcolorbox}[colframe=black!80!white, colback=black!2!white, boxrule=0.5mm, width=\textwidth, arc=2mm, auto outer arc, title=Example of improved extraction from Wikipedia, fonttitle=\color{white}\bfseries]
\begin{CJK}{UTF8}{mj}
\#\# 발생원인\\
\\
\#\#\# 회전 좌표계\\
\\
\#\#\#\# 회전좌표계\\
좌표계 x, y, z와 좌표계 x', y', z'을 보자 두 좌표계의 원점은 같다. 각각의 경우에 대해 벡터 $\mathbf{r}$.은 두 좌표계에서 다음과 같이 표시된다.\\
\\
$\mathbf{r} = x\hat x  + y\hat y  + z\hat z  $. (x, y, z 좌표)\\
\\
$\mathbf{r} = x'\hat x'  + y'\hat y'  + z'\hat z'  $. (x', y', z' 좌표계)\\
\\
벡터의 내적을 이용해 x, y, z를 ($\ x', \hat x', \hat x $.), ($\ y', \hat y', \hat y $.), ($\ z', \hat z', \hat z $.)으로 표현할 수 있다. 내적의 방법은 다음과 같다.\\
\\
$  \mathbf{r}{\hat x} =x = (x'\hat x'  + y'\hat y'  + z'\hat z')(\hat x) = x'(\hat x' \hat x)  + y'(\hat y' \hat x)  + z'(\hat z' \hat x) $.\\
\\
$  \mathbf{r} {\hat y} =y = (x'\hat x'  + y'\hat y'  + z'\hat z')(\hat y) = x'(\hat x' \hat y)  + y'(\hat y' \hat y)  + z'(\hat z' \hat y) $.\\
\\
$  \mathbf{r} {\hat z} =z = (x'\hat x'  + y'\hat y'  + z'\hat z')(\hat z) = x'(\hat x' \hat z)  + y'(\hat y' \hat z)  + z'(\hat z' \hat z) $.\\
\\
으로 표현되는 것을 확인할 수 있다.
\end{CJK}
\end{tcolorbox}
}

\caption{
    Example of suboptimal and our improved extraction from open-source Wikipedia dataset (\url{https://huggingface.co/datasets/wikimedia/wikipedia}).
    The original content is from the Korean Wikipedia article on the Coriolis effect.
}
\end{figure}