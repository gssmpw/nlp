\section{Related Work}
\label{SOTASection}

Detecting traffic anomalies is a critical challenge in modern traffic management and urban planning, where understanding the spatial and temporal evolution of traffic is essential for effective decision-making \cite{Patil2022}. Machine learning approaches, particularly neural networks and generative models, have been widely applied to this problem \cite{Sakurada2015, Wang2019, norlander2019latent, astrom2024improved}. However, existing methods often fail to fully capture the complexity of urban traffic systems, which are influenced by both temporal dynamics and spatial topology \cite{Zhang2022}.

Long Short-Term Memory (LSTM) networks \cite{Hochreiter1997} have been extensively used for traffic forecasting due to their ability to model long-term temporal dependencies. Prior research on the Gothenburg dataset primarily employed LSTM networks to model temporal dynamics from individual cameras \cite{Sopasakis2019}. While effective in capturing temporal patterns, these methods overlook the crucial spatial dependencies between traffic nodes, which are essential for accurately predicting traffic flow \cite{Yu2018, Li2018}.

Graph Neural Networks (GNNs) provide a natural extension to traffic forecasting by explicitly modeling spatial dependencies between traffic nodes. GNNs have shown great potential in capturing spatial correlations and the influence of nearby traffic conditions \cite{Wu2020, Zhang2022}. For instance, Diffusion Convolutional Recurrent Neural Networks (DCRNN) effectively model spatiotemporal traffic data by integrating GNNs with recurrent neural networks \cite{Li2018, Yu2018}. However, many GNN-based methods inadequately incorporate temporal dynamics, which are essential for understanding traffic evolution and detecting anomalies arising due to spatiotemporal interactions \cite{Zhou2020}.

To address these limitations, recent research has focused on integrating GNNs with temporal models within a generative framework. The Spatiotemporal Generative Adversarial Network (STGAN) framework exemplifies this approach by jointly modeling spatial and temporal dependencies in traffic data \cite{Deng2022}. STGAN employs an adversarial architecture \cite{Wang2021} where a generator synthesizes plausible traffic sequences and a discriminator distinguishes them from real data, enhancing predictive accuracy and providing a robust mechanism for anomaly detection \cite{Goodfellow2014, Zhang2022}.

Our research builds upon these works by adopting the STGAN framework and applying it to real-time traffic data from Gothenburg. Integrating GNNs with LSTMs within a generative adversarial framework, allows to more effectively capture the complex spatiotemporal dynamics of urban traffic. This approach may improve anomaly detection accuracy by addressing some limitations of previous methods that either overlooked spatial dependencies or did not fully model temporal dynamics.