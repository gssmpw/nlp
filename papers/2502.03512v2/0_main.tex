% This must be in the first 5 lines to tell arXiv to use pdfLaTeX, which is strongly recommended.
\pdfoutput=1
% In particular, the hyperref package requires pdfLaTeX in order to break URLs across lines.

\RequirePackage[svgnames]{xcolor}

\documentclass[11pt]{article}

% Change "review" to "final" to generate the final (sometimes called camera-ready) version.
% Change to "preprint" to generate a non-anonymous version with page numbers.
\usepackage[final]{acl}

% Standard package includes
\usepackage{times}
\usepackage{latexsym}

\usepackage[draft,textsize=footnotesize,textwidth=15mm]{todonotes}
%\usepackage[usenames,dvipsnames]{color}


% For proper rendering and hyphenation of words containing Latin characters (including in bib files)
%\usepackage[T1]{fontenc}
% For Vietnamese characters
% \usepackage[T5]{fontenc}
% See https://www.latex-project.org/help/documentation/encguide.pdf for other character sets

% This assumes your files are encoded as UTF8
\usepackage[utf8]{inputenc}
\usepackage{inconsolata}
\usepackage{algorithm}
\usepackage{algpseudocode}
\usepackage{amsmath,amssymb}
\usepackage{soul}
% \usepackage[margin=2cm]{geometry}
\usepackage{tikz}

% Style definition
\tikzset{rndblock/.style={rounded corners,rectangle,draw,scale=0.8,outer sep=0pt}}

% Command Definition
% 1 optional to customize the aspect, 2 mandatory: text to be framed
\newcommand{\tframed}[2][]{\tikz[baseline=(h.base)]\node[rndblock,#1] (h) {#2};}

\usetikzlibrary{shapes.geometric}
\usepackage{framed}
\usepackage{enumitem}
\newlist{RQ}{enumerate}{1}
\setlist[RQ]{label=\textbf{RQ\,\arabic*},ref={RQ\,\arabic*}}
\usepackage{comment}
\usepackage{natbib}
\usepackage{multibib}
\makeatletter
\usepackage{booktabs}
\usepackage[inkscapeformat=png]{svg}
\usepackage{graphicx}
\usepackage{caption}
\usepackage{subcaption}
\usepackage{tabularx}
\usepackage{soul}
\usepackage{float}
\usepackage{enumitem}
\usepackage{pifont}
\usepackage{arydshln}
\usepackage{lipsum}
%%%%%%%%%%%%%%%%%%%%%%%%%%%%%%%%%%%%%%%%%%%%%%%%%%%%%%%%%%%%
\usepackage[T1]{fontenc}
\usepackage{pifont}
\usepackage{amsmath}
\usepackage{soul}
\usepackage[utf8]{inputenc}
\usepackage{inconsolata}
\usepackage{tikz}
\usepackage{arydshln}
\usepackage{lipsum}
\usepackage[normalem]{ulem}
\usepackage{wrapfig,graphicx,lipsum}% http://ctan.org/pkg/{wrapfig,graphicx,lipsum}
% \usepackage{graphicx}
%\usepackage[table,xcdraw]{xcolor}
\usepackage{colortbl} 
%\usepackage[dvipsnames]{xcolor}

%%%%%%%%%%%%%%%%%%%%%%%%%%%%%%%%%%%%
\usepackage{uncial}

\usepackage{soul}
\usepackage{graphicx}
\usepackage{booktabs}
\usepackage{multirow}
\usepackage{colortbl}
%\usepackage{xcolor}
%\usepackage[table]{xcolor}
\usepackage{afterpage}  % load the afterpage package
\usepackage{tabularx} % Add to your preamble
\usepackage{multicol} % For structured multiline text
\usepackage{array}    % Enhanced column types
\usepackage{rotating}
\usepackage{tabularx}
\usepackage{booktabs}
\usepackage{amsmath}
\usepackage{amssymb}
\usepackage{array}



% \usepackage[draft,textsize=footnotesize,textwidth=15mm]{todonotes}
% \newcommand\ac[1]{\todo[author=AC,color=blue!40]{#1}}
% \newcommand\acil[1]{\todo[author=AC,color=blue!40,inline]{#1}}
% \newcommand\acb[1]{\textcolor{blue}{#1}}





\usepackage[most,many]{tcolorbox}

% \usepackage{tcolorbox}
% \tcbuselibrary{skins}

\newtcolorbox{defin}{colback=Teal!5!White,enhanced,title=DPO - Kernels (at-a-glance),
	attach boxed title to top left={xshift=0mm},boxrule=0pt,after skip=1cm,before skip=1cm,right skip=0cm,breakable,fonttitle=\bfseries,toprule=0pt,bottomrule=0pt,rightrule=0pt,leftrule=3pt,arc=0mm,skin=enhancedlast jigsaw,sharp corners,colframe=Teal!55!black,colbacktitle=Teal!55!black,boxed title style={
		frame code={ 
			\fill[Teal!25!black](frame.south west)--(frame.north west)--(frame.north east)--([xshift=3mm]frame.east)--(frame.south east)--cycle;
			\draw[line width=1mm,Teal!25!black]([xshift=2mm]frame.north east)--([xshift=5mm]frame.east)--([xshift=2mm]frame.south east);
			\draw[line width=1mm,Teal!25!black]([xshift=5mm]frame.north east)--([xshift=8mm]frame.east)--([xshift=5mm]frame.south east);
			\fill[Teal!25!black](frame.south west)--+(4mm,-2mm)--+(4mm,2mm)--cycle;
		}
	}
}

% \setlist{leftmargin=1mm}
\usetikzlibrary{shapes.geometric, arrows}
\usetikzlibrary{decorations.markings}

\usepackage{fancybox}

\usepackage{hyperref}
 \definecolor{darkblue}{rgb}{0, 0, 0.5}
  \hypersetup{colorlinks=true, citecolor=darkblue, linkcolor=darkblue, urlcolor=darkblue}

\definecolor{vgreen}{HTML}{60A917}
\definecolor{vred}{HTML}{CE3A29}

\usepackage{xstring}
\usepackage{longtable}
\usepackage{supertabular}



\usepackage{lipsum}
\usepackage{tikz}
\usetikzlibrary{trees,shapes}

\usepackage{epigraph}
\definecolor{hidden-draw}{RGB}{20,68,106}
\definecolor{hidden-pink}{RGB}{255,245,247}
\definecolor{paired-light-yellow}{HTML}{FFFF88}
\definecolor{paired-light-blue}{HTML}{CCE5FF}
\definecolor{paired-light-orange}{HTML}{FFCC99}
\definecolor{paired-dark-yellow}{HTML}{FFF2CC}
\definecolor{paired-light-pink}{HTML}{FFCCCC}
\definecolor{paired-cyan}{HTML}{D5E8D4}
\definecolor{paired-gray}{HTML}{eeeeee}
\definecolor{paired-green}{HTML}{cdeb8b}
\definecolor{paired-blue}{HTML}{dae8fc}
\definecolor{paired-dark-cyan}{HTML}{a2e6eb}
\definecolor{paired-dark-pink}{HTML}{e7b2d2}
\definecolor{paired-purple}{HTML}{9999ff}
\definecolor{paired-pink}{HTML}{cc99ff}
\definecolor{paired-orange}{HTML}{ffcc99}


\definecolor{a1}{RGB}{241,233,191}
\definecolor{a2}{RGB}{255,241,218}

\definecolor{a3}{RGB}{255,239,213}
\definecolor{a4}{RGB}{250,235,215}
\definecolor{a5}{RGB}{255,239,219}
\definecolor{a6}{RGB}{255,246,225}
\definecolor{a7}{RGB}{246,227,201}
\definecolor{a8}{RGB}{254,235,226}
\definecolor{a9}{RGB}{247,220,111}
\definecolor{a10}{RGB}{199,211,189}
\definecolor{a11}{RGB}{209,196,233}
\definecolor{a12}{RGB}{214,234,248}
\definecolor{a13}{RGB}{232,245,233}
\definecolor{a14}{RGB}{237,248,177}
\definecolor{a15}{RGB}{255,228,225}
\definecolor{a16}{RGB}{255,228,181}
\definecolor{a17}{RGB}{255,222,173}
\definecolor{a18}{RGB}{255,218,185}
\definecolor{a19}{RGB}{255,203,164}
\definecolor{a20}{RGB}{247,202,201}

\definecolor{a21}{RGB}{241,254,255}
\definecolor{a22}{RGB}{230,252,252}
\definecolor{a23}{RGB}{179,236,255}
\definecolor{a24}{RGB}{174,226,249}
\definecolor{a25}{RGB}{208,234,246}
\definecolor{a26}{RGB}{189,226,219}
\definecolor{a27}{RGB}{177,204,201}

\definecolor{a28}{RGB}{216,195,216}
\definecolor{a29}{RGB}{195,155,211}
\definecolor{a30}{RGB}{208,152,223}
\definecolor{a31}{RGB}{255,183,209}
\definecolor{a32}{RGB}{255,167,209}
\definecolor{a33}{RGB}{254,235,167}
\definecolor{a34}{RGB}{255,222,137}
\definecolor{a35}{RGB}{254,180,154}
\definecolor{a36}{RGB}{247,148,161}
\definecolor{a37}{RGB}{239,154,154}
\definecolor{a38}{RGB}{255,130,171}
\definecolor{a39}{RGB}{255,105,180}
\definecolor{a40}{RGB}{251,142,172}


\usepackage[edges]{forest}
\usepackage{lipsum}
\usepackage{tikz}
\usetikzlibrary{trees,shapes}
\usepackage{forest}
\usepackage{graphicx}
\usepackage{booktabs}
\usepackage{longtable}
\usepackage[export]{adjustbox} % Add this in the preamble
\usepackage{listings} % Add this in the preamble

% In the document
\lstset{
    basicstyle=\ttfamily\small,
    breaklines=true,
    frame=single,
    xleftmargin=0.05\columnwidth,
    xrightmargin=0.05\columnidth
}

\usepackage{amsmath} % For math formatting
%\usepackage{tcolorbox} % For colored boxes
%\tcbuselibrary{listingsutf8} % Optional for better customization

% Define a simple tcolorbox style
\tcbset{
  mybox/.style={
    colback=white,
    colframe=black,
    boxrule=0.5mm,
    width=\textwidth,
    left=2mm,
    right=2mm,
    bottom=2mm,
    top=2mm,
    sharp corners,
  }
}


\usepackage{tabularray}

\DefTblrTemplate{firsthead,middlehead,lasthead}{default}{
}
\DefTblrTemplate{firstfoot}{default}{
  \UseTblrTemplate{contfoot}{default}
  \UseTblrTemplate{caption}{default}
}
\DefTblrTemplate{middlefoot}{default}{
  \UseTblrTemplate{contfoot}{default}
  \UseTblrTemplate{capcont}{default}
}
\DefTblrTemplate{lastfoot}{default}{
  \UseTblrTemplate{note}{default}
  \UseTblrTemplate{remark}{default}
  \UseTblrTemplate{capcont}{default}
}


\newcolumntype{P}[1]{>{\centering\arraybackslash}p{#1}}
% Multi-line left-aligned text with manual line breaks.
% The base line is in centre.
\newcommand*{\mline}[1]{%
\begingroup
    \renewcommand*{\arraystretch}{1.1}%
   \begin{tabular}[c]{@{}>{\raggedright\arraybackslash}p{2cm}@{}}#1\end{tabular}%
  \endgroup
}

\usepackage{color}
\tcbuselibrary{skins}

\usepackage[export]{adjustbox} % for the valign option

\usepackage{setspace}
%\usepackage[capitalise]{cleveref}
\usepackage[capitalise,nameinlink]{cleveref}

%\crefname{chapter}{chap.}{chap.}
\crefname{section}{Sec.}{Sec.}

\usepackage{microtype}
\usepackage{hyperref}
\usepackage{graphicx}
\usepackage{comment}
\usepackage{amssymb}
\usepackage{algorithm}
\usepackage{amsmath}
\usepackage{algpseudocode}
\usepackage{colortbl}
\usepackage[export]{adjustbox} % for the valign option
\usepackage{enumitem}
\setlist{leftmargin=1mm}
\usepackage{pifont}
\usepackage{booktabs}
\usepackage{multirow}
\usepackage{subcaption}
\usepackage{resizegather}
\usepackage{breqn}
\usepackage[capitalise]{cleveref}
\usepackage{graphicx}
\usepackage{tikz}
\usetikzlibrary{shapes.geometric, arrows}
\usetikzlibrary{decorations.markings}
\usepackage{soul}
\usepackage{wrapfig,graphicx,lipsum}% http://ctan.org/pkg/{wrapfig,graphicx,lipsum}
\usepackage{extsizes}
\usepackage{cuted}
\usepackage{flushend}
\usepackage{float}
\usepackage{changepage,threeparttable}
\usepackage{setspace}
\usepackage{caption}
\usepackage{booktabs}
\usepackage{dblfloatfix} 
\usepackage{fixltx2e}
\usepackage[normalem]{ulem}

\usepackage{longtable}
\usepackage{amsmath}
\usepackage{array}
% \usepackage[margin=1in]{geometry} % Adjust page margins if necessary
\usepackage[most]{tcolorbox} 
%\usepackage{listings}
\setlength{\belowdisplayskip}{2pt} % Reduce spacing below equations
\setlength{\abovedisplayskip}{2pt} % Reduce spacing above equations
\setlength{\abovedisplayshortskip}{1pt} % Reduce spacing in inline equations
\setlength{\belowdisplayshortskip}{1pt}
%\documentclass[table,xcdraw]{article}
\usepackage{longtable} % For long tables that span multiple pages
\usepackage{graphicx} % For including images
\usepackage{array} % For defining custom column widths
\usepackage{caption} % For customizing captions
%usepackage{xcolor} % For text colors (optional, if needed)


% Optional: If using rotation or alignment enhancements
\usepackage{rotating} % For rotating text or images
\usepackage{booktabs} % For better table formatting (optional)
\usepackage{adjustbox} % To adjust image sizes dynamically







\DeclareRobustCommand{\hlpink}[1]{{\sethlcolor{pink}\hl{#1}}}
\DeclareRobustCommand{\hlgreen}[1]{{\sethlcolor{green}\hl{#1}}}

\usepackage{environ}

\newlength{\myl}
\expandafter\let\expandafter\origequation\csname equation*\endcsname
\expandafter\let\expandafter\endorigequation\csname endequation*\endcsname
\long\def\[#1\]{\begin{equation*}#1\end{equation*}}
\RenewEnviron{equation*}{
  \settowidth{\myl}{$\displaystyle\BODY$} % calculate width and save as \myl
  \origequation
    \ifdim\myl>\linewidth
      \resizebox{\linewidth}{!}{$\displaystyle\BODY$}% \myl > \linewidth
    \else
      \BODY % \myl <= \linewidth
    \fi
  \endorigequation
}


\makeatletter
\newcommand{\DrawLine}{%
  \begin{tikzpicture}
  \path[use as bounding box] (0,0) -- (\linewidth,0);
  \draw[color=blue!75!black,dashed,dash phase=.5pt]
        (0-\kvtcb@leftlower-\kvtcb@boxsep,0)--
        (\linewidth+\kvtcb@rightlower+\kvtcb@boxsep,0);
  \end{tikzpicture}%
  }
\makeatother

%resize/scale equations
\newcommand*{\Scale}[2][4]{\scalebox{#1}{$#2$}}%
\newcommand*{\Resize}[2]{\resizebox{#1}{!}{$#2$}}%


%--Vipula---
% \newcommand\vr[1]{\todo[author=VR,color=green!20]{#1}}
% \newcommand\vril[1]{\todo[author=VR,color=green!20,inline,caption={}]{#1}}


% %--Aman---
% \newcommand\ac[1]{\todo[author=AC,color=blue!40]{#1}}
% \newcommand\acil[1]{\todo[author=AC,color=blue!40,inline]{#1}}
% \newcommand\acb[1]{\textcolor{blue}{#1}}
% \newcommand\act[1]{\textcolor{blue}{#1}}


% %--Amitava---
% \newcommand\ad[1]{\todo[author=AD,color=purple!40]{#1}}
% \newcommand\adil[1]{\todo[author=AD,color=purple!40,inline]{#1}}
% \newcommand\adt[1]{\textcolor{purple}{#1}}


\usepackage{euscript}[mathcal]
% \usepackage[margin=1in]{geometry}

\newcommand*{\affaddr}[1]{#1}
\newcommand*{\affmark}[1][*]{\textsuperscript{#1}}
\newcommand*{\email}[1]{\texttt{#1}}
%\newcommand*{\affmark}[1][*]{\textsuperscript{#1}}
\author{
  Amitava Das\affmark[1], \bf Yaswanth Narsupalli\affmark[1], Gurpreet Singh\affmark[1],  \bf Vinija Jain\affmark[2]\thanks{\,\,\,Work done outside of role at Meta.}, 
  \bf Vasu Sharma\affmark[2]\footnotemark[1], \\\bf Suranjana Trivedy\affmark[1], 
  \bf Aman Chadha\affmark[3]\thanks{\,\,\,Work done outside of role at Amazon.}, Amit Sheth\affmark[1] \\
  \affaddr{\affmark[1]Artificial Intelligence Institute, University of South Carolina, USA,}\\
  \affaddr{\affmark[2]Meta AI, USA,}
  \affaddr{\affmark[3]Amazon AI, USA}
}



\title{\centering\includegraphics[width=\textwidth]{img/banner.pdf} }
%{\fontfamily{cmss}\selectfont DPO - Kernels}: A Semantically-Aware, Kernel-Enhanced, and Divergence-Rich Paradigm for Direct Preference Optimization}




% Author information can be set in various styles:
% For several authors from the same institution:
% \author{Author 1 \and ... \and Author n \\
%         Address line \\ ... \\ Address line}
% if the names do not fit well on one line use
%         Author 1 \\ {\bf Author 2} \\ ... \\ {\bf Author n} \\
% For authors from different institutions:
% \author{Author 1 \\ Address line \\  ... \\ Address line
%         \And  ... \And
%         Author n \\ Address line \\ ... \\ Address line}
% To start a seperate ``row'' of authors use \AND, as in
% \author{Author 1 \\ Address line \\  ... \\ Address line
%         \AND
%         Author 2 \\ Address line \\ ... \\ Address line \And
%         Author 3 \\ Address line \\ ... \\ Address line}

\forestset{
  my leaf/.style={
    fill=#1,
    draw=none
  }
}

\setlength{\topmargin}{-1.25cm}
% \setlength{\bottom}{-1.25cm}
\setlength{\textheight}{24.25cm}

\begin{document}
\maketitle
\begin{abstract}
Precise alignment in Text-to-Image (T2I) systems is crucial to ensure that generated visuals not only accurately encapsulate user intents but also conform to stringent ethical and aesthetic benchmarks. Incidents like the Google Gemini fiasco, where misaligned outputs triggered significant public backlash, underscore the critical need for robust alignment mechanisms. In contrast, Large Language Models (LLMs) have achieved notable success in alignment. Building on these advancements, researchers are eager to apply similar alignment techniques, such as Direct Preference Optimization (DPO), to T2I systems to enhance image generation fidelity and reliability.

% We introduce \textbf{YinYangAlign}, a comprehensive benchmark designed to evaluate the alignment of T2I systems across six pairs of contradictory objectives. 
We present \textbf{YinYangAlign}, an advanced benchmarking framework that systematically quantifies the alignment fidelity of T2I systems, addressing six fundamental and inherently contradictory design objectives. Each pair represents fundamental tensions in image generation, such as balancing adherence to user prompts with creative modifications or maintaining diversity alongside visual coherence. YinYangAlign includes detailed axiom datasets featuring human prompts, aligned (chosen) responses, misaligned (rejected) AI-generated outputs, and explanations of the underlying contradictions.

In addition to presenting this benchmark, we introduce \textbf{Contradictory Alignment Optimization (CAO)}, a novel extension of DPO. The CAO framework incorporates a per-axiom loss design to explicitly model and address competing objectives. Then it optimizes these objectives using multi-objective optimization techniques, including \textit{synergy-driven global preferences}, \textit{axiom-specific regularization}, and the novel \textit{synergy Jacobian} for effectively balancing contradictory goals. By utilizing tools such as the \textit{Sinkhorn-regularized Wasserstein Distance}, CAO achieves both stability and scalability while setting new performance benchmarks across all six contradictory alignment objectives.


%In addition to introducing this benchmark, we present results from applying DPO to YinYangAlign. Furthermore, we propose a novel loss function for DPO, termed \textbf{Hybrid Loss}, which demonstrates improved performance in balancing conflicting objectives compared to traditional DPO. Nonetheless, our evaluation reveals that \textbf{YinYangAlign} sets a new, stringent benchmark that current techniques, including Hybrid Loss, must strive to meet. This underscores YinYangAlign’s role in pushing the boundaries of T2I alignment and highlights the ongoing need for sophisticated alignment mechanisms. 
\end{abstract}



\section{Introduction}
Backdoor attacks pose a concealed yet profound security risk to machine learning (ML) models, for which the adversaries can inject a stealth backdoor into the model during training, enabling them to illicitly control the model's output upon encountering predefined inputs. These attacks can even occur without the knowledge of developers or end-users, thereby undermining the trust in ML systems. As ML becomes more deeply embedded in critical sectors like finance, healthcare, and autonomous driving \citep{he2016deep, liu2020computing, tournier2019mrtrix3, adjabi2020past}, the potential damage from backdoor attacks grows, underscoring the emergency for developing robust defense mechanisms against backdoor attacks.

To address the threat of backdoor attacks, researchers have developed a variety of strategies \cite{liu2018fine,wu2021adversarial,wang2019neural,zeng2022adversarial,zhu2023neural,Zhu_2023_ICCV, wei2024shared,wei2024d3}, aimed at purifying backdoors within victim models. These methods are designed to integrate with current deployment workflows seamlessly and have demonstrated significant success in mitigating the effects of backdoor triggers \cite{wubackdoorbench, wu2023defenses, wu2024backdoorbench,dunnett2024countering}.  However, most state-of-the-art (SOTA) backdoor purification methods operate under the assumption that a small clean dataset, often referred to as \textbf{auxiliary dataset}, is available for purification. Such an assumption poses practical challenges, especially in scenarios where data is scarce. To tackle this challenge, efforts have been made to reduce the size of the required auxiliary dataset~\cite{chai2022oneshot,li2023reconstructive, Zhu_2023_ICCV} and even explore dataset-free purification techniques~\cite{zheng2022data,hong2023revisiting,lin2024fusing}. Although these approaches offer some improvements, recent evaluations \cite{dunnett2024countering, wu2024backdoorbench} continue to highlight the importance of sufficient auxiliary data for achieving robust defenses against backdoor attacks.

While significant progress has been made in reducing the size of auxiliary datasets, an equally critical yet underexplored question remains: \emph{how does the nature of the auxiliary dataset affect purification effectiveness?} In  real-world  applications, auxiliary datasets can vary widely, encompassing in-distribution data, synthetic data, or external data from different sources. Understanding how each type of auxiliary dataset influences the purification effectiveness is vital for selecting or constructing the most suitable auxiliary dataset and the corresponding technique. For instance, when multiple datasets are available, understanding how different datasets contribute to purification can guide defenders in selecting or crafting the most appropriate dataset. Conversely, when only limited auxiliary data is accessible, knowing which purification technique works best under those constraints is critical. Therefore, there is an urgent need for a thorough investigation into the impact of auxiliary datasets on purification effectiveness to guide defenders in  enhancing the security of ML systems. 

In this paper, we systematically investigate the critical role of auxiliary datasets in backdoor purification, aiming to bridge the gap between idealized and practical purification scenarios.  Specifically, we first construct a diverse set of auxiliary datasets to emulate real-world conditions, as summarized in Table~\ref{overall}. These datasets include in-distribution data, synthetic data, and external data from other sources. Through an evaluation of SOTA backdoor purification methods across these datasets, we uncover several critical insights: \textbf{1)} In-distribution datasets, particularly those carefully filtered from the original training data of the victim model, effectively preserve the model’s utility for its intended tasks but may fall short in eliminating backdoors. \textbf{2)} Incorporating OOD datasets can help the model forget backdoors but also bring the risk of forgetting critical learned knowledge, significantly degrading its overall performance. Building on these findings, we propose Guided Input Calibration (GIC), a novel technique that enhances backdoor purification by adaptively transforming auxiliary data to better align with the victim model’s learned representations. By leveraging the victim model itself to guide this transformation, GIC optimizes the purification process, striking a balance between preserving model utility and mitigating backdoor threats. Extensive experiments demonstrate that GIC significantly improves the effectiveness of backdoor purification across diverse auxiliary datasets, providing a practical and robust defense solution.

Our main contributions are threefold:
\textbf{1) Impact analysis of auxiliary datasets:} We take the \textbf{first step}  in systematically investigating how different types of auxiliary datasets influence backdoor purification effectiveness. Our findings provide novel insights and serve as a foundation for future research on optimizing dataset selection and construction for enhanced backdoor defense.
%
\textbf{2) Compilation and evaluation of diverse auxiliary datasets:}  We have compiled and rigorously evaluated a diverse set of auxiliary datasets using SOTA purification methods, making our datasets and code publicly available to facilitate and support future research on practical backdoor defense strategies.
%
\textbf{3) Introduction of GIC:} We introduce GIC, the \textbf{first} dedicated solution designed to align auxiliary datasets with the model’s learned representations, significantly enhancing backdoor mitigation across various dataset types. Our approach sets a new benchmark for practical and effective backdoor defense.







% Please add the following required packages to your document preamble:
% \usepackage{booktabs}
% \usepackage[normalem]{ulem}
% \useunder{\uline}{\ul}{}
% \usepackage{longtable}
% Note: It may be necessary to compile the document several times to get a multi-page table to line up properly




\begin{figure*}[ht!]
    \centering
    \includegraphics[width=\textwidth]{img/stochastic.pdf}
    \caption{Illustrative example of aligning T2I models with Faithfulness to Prompt vs. Artistic Freedom. The chosen outputs adhere closely to the prompt, depicting a highly detailed and accurate portrait of Albert Einstein in a realistic oil painting style, while the rejected outputs deviate significantly, introducing surreal or unrelated elements. This highlights the importance of balancing prompt adherence with artistic flexibility in alignment optimization.}
    \label{fig:stochastic_generation}
    \vspace{-2mm}
\end{figure*}



\section{YinYangAlign: Six Contradictory Alignment Objectives}

Current research and benchmarking in T2I alignment primarily focus on isolated objectives \cite{guo2022survey}, such as fidelity to prompts \cite{ramesh2021zero}, aesthetic quality \cite{rombach2022high}, or bias mitigation \cite{zhao2023mitigating}, often treating these goals independently. However, there is a clear gap in benchmarks that evaluate how T2I systems balance multiple, often contradictory objectives. 
%This is a critical limitation, as real-world applications demand systems capable of achieving diverse outputs while maintaining emotional neutrality and relevance to user prompts. 
The lack of multi-objective benchmarks restricts the ability to holistically assess and improve T2I alignment, ultimately affecting their reliability and effectiveness in practical scenarios.


\noindent
\textbf{Selection of Six Contradictory Objectives}: YinYangAlign introduces six carefully selected pairs of contradictory objectives that capture the fundamental tensions in T2I image generation. These pairs are chosen for their relevance and significance in real-world applications. \cref{fig:alignment_axioms} introduces the core trade-offs central to the YinYangAlign framework, each representing a critical conflict that T2I systems must navigate to balance user expectations and ethical considerations. The trade-offs include: \textit{Faithfulness to Prompt vs. Artistic Freedom}, which involves adhering to user instructions while minimizing creative deviations; \textit{Emotional Impact vs. Neutrality}, requiring a balance between evoking emotions and maintaining objective representation; and \textit{Visual Realism vs. Artistic Freedom}, focusing on achieving photorealistic outputs without compromising artistic liberties. Additionally, \textit{Originality vs. Referentiality} addresses the challenge of fostering stylistic innovation while avoiding reliance on established artistic styles to ensure uniqueness. \textit{Verifiability vs. Artistic Freedom} emphasizes balancing factual accuracy with creative liberties to minimize misinformation. Finally, \textit{Cultural Sensitivity vs. Artistic Freedom} underscores the need to respect cultural representations while ensuring that creative freedoms do not lead to misrepresentation or insensitivity. \cref{tab:yinyang_axioms} provides illustrative examples of these alignment axioms.

YinYangAlign serves as a holistic benchmark for evaluating alignment performance, ensuring that T2I models are not only accurate and reliable but also adaptable, ethical, and capable of meeting complex user demands and societal expectations.





\begin{comment}
\begin{figure*}[ht!]
    \centering
    \resizebox{1.0\textwidth}{!}{
       \includegraphics[width=\linewidth]{img/yinyang-1.pdf}    
    }
    \caption{Architecture of PECCAVI-Image.}
    \label{fig:strength_var}
\end{figure*}

\begin{figure*}[ht!]
    \centering
    \resizebox{1.0\textwidth}{!}{
       \includegraphics[width=\linewidth]{img/yinyang-2.pdf}    
    }
    \caption{Architecture of PECCAVI-Image.}
    \label{fig:strength_var}
\end{figure*}
\end{comment}

\begin{comment}
\subsection{Faithfulness to Prompt vs. Creative Enhancement}

This axiom highlights the tension between adhering to user instructions and introducing creative elements to enhance the generated images.

\textbf{Objective Pair Overview:}
\begin{itemize}
    \item \textbf{Faithfulness to Prompt:} Ensures the generated image accurately reflects the user's textual description, maintaining fidelity to all specified details.
    \item \textbf{Creative Enhancement:} Introduces imaginative elements to enhance the aesthetic or interpretative depth, which may sometimes deviate from the original instructions.
\end{itemize}


\textbf{Core Conflict:}  
Balancing fidelity to user prompts with creative enhancements is a nuanced challenge. While faithfulness guarantees precision, creative additions risk altering the intent to enrich the output. \cref{tab:axiom_1} illustrates this tradeoff with examples of chosen and rejected AI-generated responses.
\end{comment}































\section{Data Sources}\label{sec:data}
In this section, we introduce our training data, including unlabeled light curves for pretraining and labeled samples for the downstream classification task. 

\subsection{Unlabeled data - MACHO}
The MACHO project \citep{1993Natur.365..621A} aimed to detect Massive Compact Halo Objects (MACHO) to find evidence of dark matter in the Milky Way halo by searching for gravitational microlensing events. Light curves were collected from 1992 to 1999, producing light curves of more than a thousand observations \citep{1999PASP..111.1539A} in bands B and R.
The observed sky was subdivided into 403 fields. Each field was constructed by observing a region of the sky or tile. The resulting data is available in a public repository\footnote{\url{https://macho.nci.org.au/macho_photometry}} which contains millions of light curves in bands B and R. 

We selected a subset of fields 1, 101, 102, 103, and 104 containing \num{1454792} light curves for training. Similarly, we select field 10 for testing, with a total of \num{74594} light curves. MACHO observed in both bands simultaneously, therefore having two magnitudes associated with each MJD. Since we are looking to improve on Astromer 2, we maintain the single band input.  The light curves from this dataset that exhibited Gaussian noise characteristics were removed based on the criteria: $|\text{Kurtosis}| > 10$, $|\text{Skewness}| > 1$, and $\text{Std} > 0.1$. Additionally, we excluded observations with negative uncertainties (indicative of faulty measurements) or uncertainties greater than one (to maintain photometric quality). Outliers were also removed by discarding the 1st and 99th percentiles for each light curve. This additional filtering does not affect the total number of samples but reduces the number of observations when the criteria were applied.

\begin{figure}
    \centering
    \includegraphics[scale=.88]{figures/data/magnitude_datasets.pdf}
    \caption{Magnitude distributions for the MACHO, Alcock, and ATLAS datasets. The plotted magnitudes reflect their original values as reported in the datasets; however, they are normalized during training, eliminating the differences in their mean positions.
    The Alcock catalog exhibits multimodality. In contrast, the ATLAS magnitudes show significant more variation, as they originate from a different survey.}
    \label{fig:macho-alcock-magn}
\end{figure}

\subsection{Labeled data}
To ensure a fair comparison with Astromer 1, we used the same sample selection from the MACHO \citep[hereafter referred to as Alcock; ][]{Alcock2001Variable} and the  Asteroid Terrestrial-impact Last Alert System \citep[hereafter referred to as ATLAS; ][]{heinze2018first} labeled catalogs. The former has a similar magnitude distribution, whereas the latter differs, as shown in Fig. \ref{fig:macho-alcock-magn}.

\subsubsection{Alcock}
For labeled data, we use the catalog of variable stars from \citet{Alcock2001Variable}, which contains labels for a subset of the MACHO light curves originating from 30 fields from the Large Magellanic Cloud. This labeled data will be used to train and evaluate the performance of the different embeddings on the classification task. 

The selected data comprises \num{20894} light curves, which are categorized into six classes: Cepheid variables pulsating in the fundamental (Cep\_0) and first overtone (Cep\_1), Eclipsing Binaries (EC), Long Period Variables (LPV), RR Lyrae ab and c (RRab and RRc, respectively). Table \ref{tab:alcock} summarizes the number of samples per class. We note that the catalog used is an updated version, as described in \cite{astromer}.

\begin{table}
\caption{Alcock catalog distribution.}              
\label{tab:alcock}  
\centering 
% \begin{tabular}{c c c} 
\begin{tabular}{l l r} 
\hline\hline         
Tag & Class Name & \# of sources \\ \hline
 Cep\_0 & Cepheid type I &\num{1182} \\
 Cep\_1 &Cepheid type II & \num{683} \\
 EC &Eclipsing binary & \num{6824} \\
 LPV &Long period variable &  \num{3046} \\
 RRab &RR Lyrae type ab  &  \num{7397} \\
 RRc &RR Lyrae type c &  \num{1762} \\
 Total & & \textbf{\num{20894}} \\
\hline                            
\end{tabular}
\end{table}

Figure \ref{fig:macho-alcock-magn} compares the magnitude distributions between the Alcock and MACHO datasets. The former exhibits a bimodal distribution, which aligns with the fact that it represents a subset of the light curves from MACHO fields, while the latter encompasses light curves from only five fields. 

Similarly, we compare the distribution of time differences between consecutive observations ($\Delta t$). Figure \ref{fig:macho-alcock-mjd} shows similar distributions, with comparable ranges and means of three and four days for MACHO and Alcock, respectively.
\begin{figure}
    % \centering
    \includegraphics[scale=0.7]{figures/data/mjd_datasets.pdf}
    \caption{Distributions of consecutive observation time differences ($\Delta t$) for the Alcock, MACHO, and ATLAS datasets. The boxplots illustrate the variability in observation cadences across the datasets. The Alcock and MACHO datasets show relatively consistent sampling with narrower distributions, while the ATLAS dataset exhibits a broader range of $\Delta t$, reflecting more diverse observation intervals. The y-axis is shown on a logarithmic scale to highlight differences across several orders of magnitude }
    \label{fig:macho-alcock-mjd}
\end{figure}

\subsection{ATLAS}
The Asteroid Terrestrial-impact Last Alert System \citep[ATLAS; ][]{Tonry2018} is a survey developed by the University of Hawaii and funded by NASA. Operating since 2015, ATLAS has a global network telescopes, primarily focused on detecting asteroids and comets that could potentially threaten Earth. Observing in $c$ (blue), $o$ (orange), and $t$ (red) filters.

The variable star dataset used in this work was presented by \citet{heinze2018first} and includes 4.7 million candidate variable objects, included in the labeled and unclassified objects, as well as a dubious class. According to their estimates, this class is predominantly composed of $90\%$ instrumental noise and only $10\%$ genuine variable stars.

We analyze \num{141376} light curves from the ATLAS dataset, as detailed in Table \ref{tab:ATLAS}. These observations, measured in the $o$ passband, have a median cadence of $\sim$15 minutes, which is significantly shorter than the typical cadence in the MACHO dataset. This substantial difference poses a challenge for the model, as it must adapt to such a distinct temporal distribution. 

\begin{table}[h!]
\caption{ATLAS catalog distribution.}              
\label{tab:ATLAS}
\centering 
\begin{tabular}{l l r} 
\hline\hline         
Tag & Class Name & \# of sources \\
\hline
CB & Close Binaries &  \num{80218} \\
DB & Detached Binary &  \num{28767} \\
Mira & Mira &  \num{7370} \\
Pulse &RR Lyrae, $\delta$-Scuti, Cepheids &  \num{25021} \\
Total & & \textbf{\num{141376}}\\
\hline                            
\end{tabular}
\end{table}

As done in \citet{astromer} and to standardize the labels with other datasets, we combine detached eclipsing binaries identified by full or half periods into the close binaries (CB) category and similarly merge detached binaries (DB). However, objects with labels derived from Fourier analysis are excluded, as these classifications do not directly align with astrophysical categories.

\subsection{MACHO vs ATLAS}\label{sec:machovsatlas}
Figures \ref{fig:macho-alcock-magn} and \ref{fig:macho-alcock-mjd} illustrate the distributional differences between the unlabeled MACHO dataset and the labeled subsets discussed earlier. While the magnitudes show a notable shift between MACHO and ATLAS, our training strategy normalizes the light curves to a zero mean. As a result, the relationships between observations take precedence over the raw magnitude values. Consequently, we do not expect a substantial performance drop when transitioning between datasets. However, for $\Delta t$, the smaller values of $\Delta t$ present a significant challenge, as the model must extrapolate and account for fast variations to capture short-time information effectively. We evidence this in our first results from Astromer 2, where the F1 score on the ATLAS dataset was lower compared to MACHO when having fewer labels for classification. 





\section{Contradictory Alignment Optimization (CAO)}

The \textbf{YinYangAlign} framework, models the challenge of balancing \emph{inherently contradictory} objectives. For example, prioritizing \emph{Faithfulness to Prompt} can limit \emph{Artistic Freedom}, while emphasizing \emph{Emotional Impact} may erode \emph{Neutrality}. To address these tensions, we introduce \textbf{Contradictory Alignment Optimization (CAO)}, which employs a \emph{per-axiom} loss design to explicitly model competing goals. CAO employs a dynamic weighting mechanism to prioritize sub-objectives within each axiom, facilitating granular control over trade-offs and enabling adaptive optimization across diverse alignment paradigms. Additionally, CAO integrates \emph{Pareto optimality} principles with the \emph{Bradley-Terry} preference framework, introducing a novel \emph{global synergy} mechanism that unifies all contradictory objectives into a cohesive optimization strategy. This unique combination of multi-objective synergy defines the core innovation of CAO, distinguishing it from existing T2I alignment methods.

%In \textbf{YinYangAlign} setup, T2I models must reconcile \emph{intrinsically contradictory} aims. For instance, \emph{Faithfulness to Prompt} can restrict \emph{Artistic Freedom}, while prioritizing \emph{Emotional Impact} may erode \emph{Neutrality}. These tensions motivate our DPO-\textbf{Contradictory Alignment Optimization (CAO)} design, which introduces a \emph{per-axiom} loss design to explicitly model competing goals. By assigning specific weights to each sub-objective within an axiom, CAO allows for flexible tradeoffs tailored to individual alignment requirements. Importantly, CAO merges \emph{Pareto optimality} concepts with the \emph{Bradley-Terry} preference framework, yielding a novel \emph{global synergy} mechanism that unifies all contradictory objectives under one coherent optimization strategy. This union of multi-objective synergy and directed preference optimization is the core novelty of CAO, setting it apart from prior T2I alignment approaches.



\begin{figure*}[ht!]
    \centering
    \includegraphics[width=\textwidth]{img/axiom_pairs_visualization.png}
    \caption{
        Visualization of error loss surface tension for six axiom pairs in YinYang alignment. Each pair highlights the inherent trade-offs between \emph{competing objectives} using a 3D surface plot (left) and a 2D contour plot (right). \textcolor{blue}{Blue regions} represent synergy (low tension), \textcolor{red}{red regions} indicate conflict (high tension), while \textcolor{green}{Green markers} highlight "sweet spots" where the tension is minimal. The first axiom pair, \textit{Faithfulness to Prompt vs. Artistic Freedom}, shows sweet spots centered around moderate values, suggesting balanced trade-offs. For \textit{Emotional Impact vs. Neutrality}, sweet spots are sparse, reflecting the difficulty in balancing emotional engagement and neutrality. The axiom pair \textit{Visual Realism vs. Artistic Freedom} shows distributed sweet spots, indicating achievable trade-offs between realism and creative freedom. In \textit{Originality vs. Referentiality}, sweet spots are concentrated, emphasizing the challenge of balancing uniqueness and references. The pair \textit{Verifiability vs. Artistic Freedom} has central sweet spots, suggesting harmony between factual accuracy and creative expression. Lastly, \textit{Cultural Sensitivity vs. Artistic Freedom} shows fewer sweet spots, reflecting the complexity of respecting cultural norms while granting artistic liberties. This visualization underscores the inherent trade-offs in T2I systems and identifies potential areas of optimization for aligning competing objectives.
    }
    \label{fig:axiom_pairs_tension}
    \vspace{-2mm}
\end{figure*}




\subsection{Axiom-Wise Loss Expansion and Synergy}

\paragraph{Local Axiom-Wise Loss}: Below, we illustrate how each axiom’s loss is defined, before showing how these losses connect into a global synergy framework. For each axiom \(a\), CAO defines a loss function \(f_a(I)\) that blends two competing sub-objectives, \(\mathcal{L}_p(I)\) and \(\mathcal{L}_q(I)\), via a mixing parameter \(\alpha_a\):
\[
f_a(I)
=
\alpha_a \,\mathcal{L}_p(I)
+
\bigl(1 - \alpha_a\bigr)\,\mathcal{L}_q(I).
\]
For example, \(\mathcal{L}_p(I)\) might emphasize \emph{faithfulness to prompt}, while \(\mathcal{L}_q(I)\) favors \emph{artistic freedom}, or any other pair of conflicting objectives. Varying \(\alpha_a\) adjusts the per-axiom balance according to domain or policy needs.


\begin{tcolorbox}[colframe=black,colback=white,boxrule=0.5mm,width=\columnwidth,sharp corners]
\scriptsize
\begin{itemize}[left=-4pt,itemsep=0pt,topsep=0pt,parsep=0pt]
    \item \textbf{Faithfulness to Prompt vs.\ Artistic Freedom}
    \vspace{-3mm}
    \[
    f_{\text{faith\_artistic}}(I) 
    = \alpha_1 \cdot \mathcal{L}_{\text{faith}}
    + (1 - \alpha_1) \cdot \mathcal{L}_{\text{artistic}}
    \]
    \vspace{-6mm}

    \item \textbf{Emotional Impact vs.\ Neutrality}
    \vspace{-3mm}
    \[
    f_{\text{emotion\_neutrality}}(I) 
    = \alpha_2 \cdot \mathcal{L}_{\text{emotion}}
    + (1 - \alpha_2) \cdot \mathcal{L}_{\text{neutrality}}
    \]
    \vspace{-6mm}

    \item \textbf{Visual Realism vs.\ Artistic Freedom}
    \vspace{-3mm}
    \[
    f_{\text{visual\_style}}(I) 
    = \alpha_3 \cdot \mathcal{L}_{\text{realism}}
    + (1 - \alpha_3) \cdot \mathcal{L}_{\text{artistic}}
    \]
    \vspace{-6mm}

    \item \textbf{Originality vs.\ Referentiality}
    \vspace{-3mm}
    \[
    f_{\text{originality\_referentiality}}(I) 
    = \alpha_4 \cdot \mathcal{L}_{\text{originality}}
    + (1 - \alpha_4) \cdot \mathcal{L}_{\text{referentiality}}
    \]
    \vspace{-6mm}

    \item \textbf{Verifiability vs.\ Artistic Freedom}
    \vspace{-3mm}
    \[
    f_{\text{verifiability\_creative}}(I) 
    = \alpha_5 \cdot \mathcal{L}_{\text{verifiability}}
    + (1 - \alpha_5) \cdot \mathcal{L}_{\text{artistic}}
    \]
    \vspace{-6mm}

    \item \textbf{Cultural Sensitivity vs.\ Artistic Freedom}
    \vspace{-3mm}
    \[
    f_{\text{cultural\_artistic}}(I) 
    = \alpha_6 \cdot \mathcal{L}_{\text{cultural}}
    + (1 - \alpha_6) \cdot \mathcal{L}_{\text{artistic}}
    \]
\end{itemize}
\end{tcolorbox}

The resulting loss surfaces and their corresponding \emph{sweet spots}, where competing objectives are in harmony, are visualized in \cref{fig:axiom_pairs_tension}.


\paragraph{Multi-Objective Aggregator and Pareto Frontiers:} Although \(f_a(I)\) provides \emph{local} control over each axiom \(a\), reconciling multiple axioms at once requires a \emph{global} view. We thus define a \textbf{multi-objective synergy function}:
\[
\mathcal{S}(I)
=
\sum_{a=1}^A
\omega_a \, f_a(I),
\]
where the \(\{\omega_a\}\) are global coefficients reflecting the relative priority of each axiom. By varying these synergy weights, we trace out a Pareto frontier~\cite{miettinen1999nonlinear, yang2021towards, lin2023pareto} in the T2I objective space, clarifying how small concessions in one axiom can yield major gains in another.




\smallskip
\noindent
\textbf{Interpretation and Importance.}\quad
In \emph{multi-objective optimization}, the \emph{Pareto frontier} is the set of all solutions where improving any one objective strictly worsens at least one other~\cite{deb2001multiobjective, zhou2022pareto}. By tuning \(\{\omega_a\}\), we systematically explore these tradeoffs, finding, for example, that a slight drop in \emph{visual realism} could allow for notably higher \emph{stylistic freedom}. Such multi-objective approaches have been central in \emph{multi-task learning}~\cite{ma2020quadratic, navon2022multi, yu2020gradient} and \emph{modular/decomposed learning}~\cite{liebenwein2021provable, lin2022pareto}, ensuring transparent control over each tension point (e.g., verifiability vs.\ creativity) and easy adaptation to new constraints. cf  \cref{sec:appendix:dpo-cao}. 


\subsection{Connecting Synergy to Pairwise Preference}

To fully implement both \emph{local} axiom-wise guidance and \emph{global} synergy-based tradeoffs, we integrate the synergy function into the DPO framework. Concretely, each \(f_a(I)\) enters a Bradley-Terry style preference:
\[
P_{ij}^a
=
\frac{
\exp\!\bigl(f_a(I_i)\bigr)
}{
\exp\!\bigl(f_a(I_i)\bigr) + \exp\!\bigl(f_a(I_j)\bigr)
},
\]
ensuring local interpretability for each axiom. Meanwhile, a \emph{combined preference} over \(\mathcal{S}(I)\) expresses the global tradeoff:
\[
P_{ij}^{\mathcal{S}}
=
\frac{\exp\!\bigl(\mathcal{S}(I_i)\bigr)}
     {\exp\!\bigl(\mathcal{S}(I_i)\bigr) + \exp\!\bigl(\mathcal{S}(I_j)\bigr)}.
\]
A hyperparameter \(\lambda\) then balances how much this \textbf{global synergy} affects the final optimization vs.\ how much weight is given to \textbf{local} per-axiom preferences:
\[
\mathcal{L}_{\text{CAO}}
=
- \sum_{a=1}^{A}
\sum_{(i,j)}
\log\!\bigl(P_{ij}^a\bigr)
+
\lambda
\sum_{(i,j)}
\Bigl[
-\,\log\!\bigl(P_{ij}^{\mathcal{S}}\bigr)
\Bigr].
\]

\subsection{Unified CAO Loss}

We can consolidate the local and global preferences into a single loss function. One straightforward approach is:
\[
\mathcal{L}_{\text{CAO}}
=
\underbrace{
- \sum_{a=1}^{6}
  \sum_{(i,j)}
  \log\!\bigl(P_{ij}^a\bigr)
}_{\mathcal{L}_{\text{local}}}
+
\lambda
\underbrace{
\left[
  - \sum_{(i,j)}
  \log\!\bigl(P_{ij}^{\mathcal{S}}\bigr)
\right]
}_{\mathcal{L}_{\text{synergy}}}.
\]

\paragraph{Local Terms (\(\mathcal{L}_{\text{local}}\)).}  
Each axiom \(a\) retains interpretability and ensures the model handles \emph{faithfulness vs.\ artistry}, \emph{emotional impact vs.\ neutrality}, and so on, at a granular level.

\paragraph{Global Term (\(\mathcal{L}_{\text{synergy}}\)).}  
This enforces coordinated tradeoffs by encouraging consistency with the aggregator \(\mathcal{S}(I)\). A larger \(\lambda\) implies stronger synergy constraints and places more emphasis on global equilibrium across axioms, while a smaller \(\lambda\) prioritizes local alignment objectives.

\begin{figure*}[ht!]
\centering
\begin{tcolorbox}[
  enhanced,
  colback=white,
  colframe=black,
  boxrule=1pt,
  borderline={0.6pt}{2pt}{black},
  sharp corners,
  width=\textwidth
]

\begin{minipage}{\textwidth}
\scriptsize

\subsection*{(A) Local Axiom Preferences}
\begin{equation*}
\mathcal{L}_{\text{local}}
~\;=\;
- \Bigl[
   \sum_{(i,j)} \log\!\bigl(P_{ij}^{\text{faith_artistic}}\bigr)
 + \sum_{(i,j)} \log\!\bigl(P_{ij}^{\text{emotion_neutrality}}\bigr)
 + \sum_{(i,j)} \log\!\bigl(P_{ij}^{\text{visual_style}}\bigr)
 + \sum_{(i,j)} \log\!\bigl(P_{ij}^{\text{originality_referentiality}}\bigr)
 + \sum_{(i,j)} \log\!\bigl(P_{ij}^{\text{verifiability_creative}}\bigr)
 + \sum_{(i,j)} \log\!\bigl(P_{ij}^{\text{cultural_artistic}}\bigr)
\Bigr].
\end{equation*}

\vspace{-2mm}
\noindent
Here, each term is a negative log-likelihood over 
\(
   P_{ij}^{a}
   =
   \frac{\exp\!\bigl(f_{a}(I_i)\bigr)}
        {\exp\!\bigl(f_{a}(I_i)\bigr)+\exp\!\bigl(f_{a}(I_j)\bigr)}
\)
for axiom \(a\).

\vspace{-3mm}
\subsection*{(B) Global Synergy Preference}
\begin{equation*}
\mathcal{L}_{\text{synergy}}
~\;=\;
\sum_{(i,j)}
  \log\!\Bigl(
    \frac{
      \exp\!\Bigl(\omega_{1}f_{\text{faithArtistic}}(I_i)
      + \ldots + \omega_{6}f_{\text{culturalArtistic}}(I_i)\Bigr)
    }{
      \exp\!\Bigl(\omega_{1}f_{\text{faithArtistic}}(I_i)
      + \ldots + \omega_{6}f_{\text{culturalArtistic}}(I_i)\Bigr)
      +
      \exp\!\Bigl(\omega_{1}f_{\text{faithArtistic}}(I_j)
      + \ldots + \omega_{6}f_{\text{culturalArtistic}}(I_j)\Bigr)
    }
  \Bigr).
\end{equation*}

\vspace{-2mm}
\noindent
This term encodes the preference for 
\(
   \mathcal{S}(I) 
   = 
   \sum_{a=1}^{6} \omega_a\, f_a(I)
\).

\vspace{-2.5mm}
\subsection*{(C) Axiom-Specific Regularizers}
\begin{equation*}
\sum_{a=1}^{6} \tau_{a}\,\mathcal{R}_a
~=\;
\tau_{1}\,\frac{\displaystyle
   \int_{\mathcal{X}}\!\!\int_{\mathcal{X}}
   \|x-y\|\,
   P_{\text{faith}}(x)\,
   Q_{\text{artistic}}(y)
   \,\mathrm{d}x\,\mathrm{d}y
}{
   \displaystyle
   \int_{\mathcal{X}}P_{\text{faith}}(x)\,\mathrm{d}x
   \;\times\;
   \int_{\mathcal{X}}Q_{\text{artistic}}(y)\,\mathrm{d}y
}
~+~
\ldots
~+~
\tau_{6}\,\frac{\displaystyle
   \int_{\mathcal{X}}\!\!\int_{\mathcal{X}}
   \|x-y\|\,
   P_{\text{cultural}}(x)\,
   Q_{\text{artistic}}(y)
   \,\mathrm{d}x\,\mathrm{d}y
}{
   \displaystyle
   \int_{\mathcal{X}}P_{\text{cultural}}(x)\,\mathrm{d}x
   \;\times\;
   \int_{\mathcal{X}}Q_{\text{artistic}}(y)\,\mathrm{d}y
}.
\end{equation*}

\end{minipage}
\end{tcolorbox}

\vspace{-2mm}
\includegraphics[width=\textwidth]{img/ablation_error_loss.png}

%\captionsetup{justification=justified, singlelinecheck=false}
\captionsetup{justification=justified, singlelinecheck=false}
\caption{A modular breakdown of the CAO loss. 
\textbf{(A)} Local per-axiom preferences, 
\textbf{(B)} global synergy preference, 
\textbf{(C)} axiom-specific regularizers. 
Three error loss surfaces from the ablation study demonstrate the progressive impact of incorporating components of the YinYang alignment objective. 
The first plot, with only the \textit{Local Axiom Preferences}, shows an unstable gradient landscape. 
Adding in the second plot smooths the loss surface significantly. 
Finally, introducing additional \textit{Regularization Terms} in the third plot further stabilizes and smooths the surface, making optimization more efficient and robust.}






% \caption{A modular breakdown of the CAO loss. 
% \textbf{(A)} Local per-axiom preferences,
% \textbf{(B)} global synergy preference,
% \textbf{(C)} axiom-specific regularizers.
% Three error loss surfaces from the ablation study demonstrate
% the progressive impact of incorporating components of the YinYang
% alignment objective. The first plot, with only the
% \textit{Local Axiom Preferences} \(- \sum_{a=1}^{A} \sum_{(i,j)} \log\bigl(P_{ij}^a\bigr)\),
% shows an unstable gradient landscape. Adding \textit{Global Synergy Preference}
% \(-\lambda \sum_{(i,j)} \log\bigl(P_{ij}^{\mathcal{S}}\bigr)\) in the second plot
% smooths the loss surface significantly. Finally, introducing additional
% \textit{Regularization Terms} \(\sum_{a=1}^{A} \tau_a \mathcal{R}_a\)
% in the third plot further stabilizes and smooths the surface,
% making optimization more efficient and robust.}
\label{fig:dpo-cao-expanded}
\end{figure*}

% \begin{figure*}[ht!]
% \centering
% \begin{tcolorbox}[
%   enhanced,
%   colback=white,        % background color
%   colframe=black,       % color of the primary frame
%   boxrule=1pt,        % thickness of the primary frame
%   borderline={0.6pt}{2pt}{black}, % second line: thickness=0.6pt, distance=2pt, color=black
%   sharp corners,        % no rounded corners
%   width=\textwidth      % occupy full width in figure*
% ]


% \begin{minipage}{\textwidth}
% \scriptsize
% %-----------------------------------
% % (A) Local Axiom Preferences
% %-----------------------------------
% \subsection*{(A) Local Axiom Preferences}
% % \vspace{-0.5em}
% \begin{equation*}
% \mathcal{L}_{\text{local}}
% ~\;=\;
% - \Bigl[
%    \sum_{(i,j)} \log\!\bigl(P_{ij}^{\text{faith_artistic}}\bigr)
%  + \sum_{(i,j)} \log\!\bigl(P_{ij}^{\text{emotion_neutrality}}\bigr)
%  + \sum_{(i,j)} \log\!\bigl(P_{ij}^{\text{visual_style}}\bigr)
%  + \sum_{(i,j)} \log\!\bigl(P_{ij}^{\text{originality_referentiality}}\bigr)
%  + \sum_{(i,j)} \log\!\bigl(P_{ij}^{\text{verifiability_creative}}\bigr)
%  + \sum_{(i,j)} \log\!\bigl(P_{ij}^{\text{cultural_artistic}}\bigr)
% \Bigr].
% \end{equation*}

% \vspace{-2mm}
% \noindent
% Here, each term is a negative log-likelihood over the Bradley-Terry preference 
% \(\displaystyle P_{ij}^{a} = \frac{\exp\!\bigl(f_{a}(I_i)\bigr)}
%                                   {\exp\!\bigl(f_{a}(I_i)\bigr)+\exp\!\bigl(f_{a}(I_j)\bigr)}\)
% for axiom \(a\).

% \vspace{-3mm}
% %-----------------------------------
% % (B) Global Synergy Preference
% %-----------------------------------
% \subsection*{(B) Global Synergy Preference}
% \vspace{-0.5em}
% \begin{equation*}
% \mathcal{L}_{\text{synergy}}
% ~\;=\;
% \sum_{(i,j)}
%   \log\!\Bigl(
%     \frac{
%       \exp\!\Bigl(\omega_{1}f_{\text{faithArtistic}}(I_i)
%       + \ldots + \omega_{6}f_{\text{culturalArtistic}}(I_i)
%       \Bigr)
%     }{
%       \exp\!\Bigl(\omega_{1}f_{\text{faithArtistic}}(I_i)
%       + \ldots + \omega_{6}f_{\text{culturalArtistic}}(I_i)
%       \Bigr)
%       ~+~
%       \exp\!\Bigl(\omega_{1}f_{\text{faithArtistic}}(I_j)
%       + \ldots + \omega_{6}f_{\text{culturalArtistic}}(I_j)
%       \Bigr)
%     }
%   \Bigr).
% \end{equation*}

% \vspace{-2mm}
% \noindent
% This term encodes the preference for a \emph{global aggregator} 
% \(\displaystyle \mathcal{S}(I) = \sum_{a=1}^{6} \omega_a\, f_a(I)\),
% where each \(\omega_a\) is a weight signifying axiom \(a\)’s priority.

% \vspace{-2.5mm}
% %-----------------------------------
% % (C) Axiom-Specific Regularizers
% %-----------------------------------
% \subsection*{(C) Axiom-Specific Regularizers}
% \vspace{-0.25em}
% \begin{equation*}
% \sum_{a=1}^{6} \tau_{a}\,\mathcal{R}_a
% ~=\;
% \tau_{1}\,\frac{\displaystyle
%    \int_{\mathcal{X}}\!\!\int_{\mathcal{X}}
%    \|x-y\|\,
%    P_{\text{faith}}(x)\,
%    Q_{\text{artistic}}(y)
%    \,\mathrm{d}x\,\mathrm{d}y
% }{
%    \displaystyle
%    \int_{\mathcal{X}}P_{\text{faith}}(x)\,\mathrm{d}x
%    \;\times\;
%    \int_{\mathcal{X}}Q_{\text{artistic}}(y)\,\mathrm{d}y
% }
% ~+~
% \tau_{2}\,\frac{\displaystyle
%    \int_{\mathcal{X}}\!\!\int_{\mathcal{X}}
%    \|x-y\|\,
%    P_{\text{emotion}}(x)\,
%    Q_{\text{neutrality}}(y)
%    \,\mathrm{d}x\,\mathrm{d}y
% }{
%    \displaystyle
%    \int_{\mathcal{X}}P_{\text{emotion}}(x)\,\mathrm{d}x
%    \;\times\;
%    \int_{\mathcal{X}}Q_{\text{neutrality}}(y)\,\mathrm{d}y
% }
% ~+~
% \ldots
% ~+~
% \tau_{6}\,\frac{\displaystyle
%    \int_{\mathcal{X}}\!\!\int_{\mathcal{X}}
%    \|x-y\|\,
%    P_{\text{cultural}}(x)\,
%    Q_{\text{artistic}}(y)
%    \,\mathrm{d}x\,\mathrm{d}y
% }{
%    \displaystyle
%    \int_{\mathcal{X}}P_{\text{cultural}}(x)\,\mathrm{d}x
%    \;\times\;
%    \int_{\mathcal{X}}Q_{\text{artistic}}(y)\,\mathrm{d}y
% }.
% \end{equation*}

% \vspace{-2mm}
% \noindent
% Here, the first axiom’s regularizer is fully expanded with a Wasserstein-type cost, while subsequent axioms use a shorter notation \(W(\cdot,\cdot)\) for brevity.

% \end{minipage}
% \end{tcolorbox}

% \vspace{-2mm}
% \includegraphics[width=\textwidth]{img/ablation_error_loss.png}
% %\captionsetup{width=\textwidth}
% \caption{A modular breakdown of the CAO loss. 
% \textbf{(A)} Local per-axiom preferences,
% \textbf{(B)} global synergy preference,
% \textbf{(C)} axiom-specific regularizers.
% Three error loss surfaces from the ablation study demonstrate
% the progressive impact of incorporating components of the YinYang
% alignment objective. The first plot, with only the
% \textit{Local Axiom Preferences} \(- \sum_{a=1}^{A} \sum_{(i,j)} \log\bigl(P_{ij}^a\bigr)\),
% shows an unstable gradient landscape. Adding \textit{Global Synergy Preference}
% \(-\lambda \sum_{(i,j)} \log\bigl(P_{ij}^{\mathcal{S}}\bigr)\) in the second plot
% smooths the loss surface significantly. Finally, introducing additional
% \textit{Regularization Terms} \(\sum_{a=1}^{A} \tau_a \mathcal{R}_a\)
% in the third plot further stabilizes and smooths the surface,
% making optimization more efficient and robust.}
% \label{fig:dpo-cao-expanded}
% \end{figure*}



\paragraph{Why Keep Both Local \emph{and} Global?}
\begin{itemize}[left=5pt]
\item 
\emph{Local Preferences \((P_{ij}^a)\)} show how the model balances each contradictory pair (e.g., “\emph{Did we favor faithfulness over artistry?}”), preserving interpretability at the axiom level.
\item 
\emph{Global Preference \((P_{ij}^{\mathcal{S}})\)} ensures the T2I model, \emph{as a whole}, follows the overarching synergy profile, capturing \emph{all} tensions in unison.
\end{itemize}
Hence, \(\lambda\) “\emph{dials in}” how much to respect the overall synergy aggregator vs.\ each per-axiom preference.

\subsection{Axiom-Specific Regularization in CAO}

To stabilize the optimization and prevent overfitting to any single objective, CAO also provides a regularization term for each axiom:
\[
\mathcal{L}_{\text{CAO}}
=
\sum_{a=1}^6
\Bigl[
  f_a(I) 
  + 
  \tau_a \,\mathcal{R}_a
\Bigr],
\]
where \(\tau_a\) scales the influence of the regularizer \(\mathcal{R}_a\). While KL-divergence is a common choice, it can be unstable in high-dimensional T2I scenarios; \textbf{Wasserstein Distance}~\cite{arjovsky2017wasserstein} or \emph{Sinkhorn regularization}~\cite{cuturi2013sinkhorn} typically offer more robust optimization. cf \cref{sec:appendix_wasserstein_Sinkhorn} for the rationale behind Wasserstein Distance and Sinkhorn Regularization.








\subsection{Putting It All Together: Final CAO Formulation}

Bringing together the synergy function, local Bradley-Terry preferences, and axiom-specific regularization leads to the final CAO objective:
\[
\mathcal{L}_{\text{CAO}}
=
\underbrace{
- \sum_{a=1}^{A} 
  \sum_{(i,j)}
  \log\!\bigl(P_{ij}^a\bigr)
}_{\text{Local Axiom Preferences}}
-
\lambda
\underbrace{
\sum_{(i,j)}
\log\!\bigl(P_{ij}^{\mathcal{S}}\bigr)
}_{\text{Global Synergy Preference}}
+
\sum_{a=1}^{A}
\tau_a\,\mathcal{R}_a.
\]




\paragraph{Role of the Synergy Jacobian $(\mathbf{J}_{\mathcal{S}}$)}: The Synergy Jacobian \(\mathbf{J}_{\mathcal{S}}\) is a vital component in managing \emph{gradient interactions} across multiple axioms during training. While the regularization parameter \(\lambda\) balances local and global objectives, \(\mathbf{J}_{\mathcal{S}}\) quantifies how updates to model parameters for one axiom impact the alignment of others. Mathematically, \(\mathbf{J}_{\mathcal{S}}\) is defined as:
\[
\mathbf{J}_{\mathcal{S}} = \frac{\partial \mathcal{S}(I)}{\partial \theta},
\]
where \(\mathcal{S}(I)\) represents the synergy aggregator that measures overall alignment, \(I\) denotes the input, and \(\theta\) are the model parameters. This Jacobian provides a structured view of the interdependencies among axioms, capturing how conflicting objectives influence each other \cite{navon2022multi, yu2020gradient}.


\begin{figure*}[ht!]
    \includegraphics[width=\textwidth, keepaspectratio]{img/jacobian_visualization.png}
    \caption{
        Visualization of optimization paths and gradient dynamics with and without the Synergy Jacobian.
        \textbf{3D Plots (Top Row):} The synergy score (z-axis) peaks at the Pareto-optimal point (black cross), representing the ideal balance between competing objectives. 
        \textit{Without Jacobian Adjustment (left column):} The optimization path (red circles) follows conflicting gradients (red arrows), leading to suboptimal convergence away from the Pareto-optimal point.
        \textit{With Jacobian Adjustment (right column):} The gradients (blue arrows) are harmonized by the Synergy Jacobian, guiding the optimization path (blue circles) toward the synergy peak.
        \textbf{2D Plots (Bottom Row):} The 2D plots provide a top-down perspective of the same optimization dynamics, highlighting gradient directions and path alignment. 
        \textit{Without Jacobian Adjustment (left column):} Misaligned gradients cause the path to diverge from the Pareto-optimal region.
        \textit{With Jacobian Adjustment (right column):} Adjusted gradients align consistently, enabling smooth convergence to the synergy peak. Together, these visualizations demonstrate the effectiveness of the Synergy Jacobian in resolving gradient conflicts, fostering cohesive and efficient optimization across competing objectives.
    }
    \label{fig:jacobian_visualization}
\end{figure*}




\textbf{Intuition and Practical Role}: During training, gradients for individual axioms often conflict, resulting in updates that disproportionately favor one objective at the expense of others. The Synergy Jacobian addresses this issue by scaling or adjusting gradients based on their interactions with the synergy aggregator \(\mathcal{S}(I)\). Specifically:
\begin{itemize}
    \item Gradients that align well with improving overall synergy are preserved to maintain their positive contribution.
    \item Gradients that disproportionately benefit a single axiom while adversely affecting others are scaled back to ensure balance across objectives.
\end{itemize}

The parameter update during training can be expressed as:
\[
\Delta \theta = \eta \cdot \nabla \mathcal{L} - \alpha \cdot \mathbf{J}_{\mathcal{S}},
\]
where \(\nabla \mathcal{L}\) is the standard gradient of the loss, \(\eta\) is the learning rate, and \(\alpha\) is a scaling factor controlling the influence of the Synergy Jacobian. This formulation ensures that the optimization process remains balanced, preventing any single axiom from dominating the alignment process. The impact of the Synergy Jacobian on resolving gradient conflicts and guiding optimization can be visualized in \cref{fig:jacobian_visualization}.


\textls[-10]{\textbf{Benefits}: The incorporation of \(\mathbf{J}_{\mathcal{S}}\) ensures:
1) \emph{Balanced Optimization}: Prevents one axiom from overshadowing others, fostering a holistic alignment across contradictory objectives. 2) \emph{Stability}: Reduces the risk of oscillations or instability during training by moderating conflicting gradient interactions. 3) \emph{Cohesion}: Facilitates a stable and unified optimization process, ensuring that all objectives contribute meaningfully to the overall alignment.}

Further details, derivations, and examples are provided in \cref{sec:appendix_synergy_jacobian}.







\subsection*{Benefits and Scalability}

\begin{itemize}
    \item \textbf{Pareto-Aware Multi-Objective Control:} 
    By sweeping synergy weights \(\{\omega_a\}\), we explore a Pareto frontier of alignment solutions, clarifying how intensifying constraints for one axiom (e.g., cultural sensitivity) impacts another (e.g., artistic freedom).

    \item \textbf{Global Alignment \& Local Interpretability:} 
    The synergy-based preference \(P_{ij}^{\mathcal{S}}\) offers a coherent global objective, while individual \(P_{ij}^a\) preserve axiom-level clarity.

    \item \textbf{Efficient Computation via Sinkhorn Regularization:}  
    Wasserstein-based distances are highly effective for aligning distributions but can be computationally expensive, particularly for large-scale data, as their complexity often scales poorly. \emph{Sinkhorn regularization}~\cite{cuturi2013sinkhorn} addresses this issue by introducing an entropy-based regularization term to the optimal transport problem, which smooths the optimization and significantly reduces computational overhead. The Sinkhorn distance is defined as:  
\[
W_\lambda(P, Q) = \min_{\gamma \in \Pi(P, Q)} \langle \gamma, C \rangle - \lambda \mathcal{H}(\gamma),
\]
where \(P\) and \(Q\) are the distributions to be aligned, \(\Pi(P, Q)\) denotes the set of all valid couplings with marginals \(P\) and \(Q\), \(C\) is the cost matrix, \(\lambda\) is the regularization parameter, and \(\mathcal{H}(\gamma)\) is the entropy of the coupling \(\gamma\), defined as:
\[
\mathcal{H}(\gamma) = - \sum_{i, j} \gamma_{ij} \log \gamma_{ij}.
\]
By incorporating this entropy term, the optimization problem becomes smoother and computationally efficient, allowing for faster convergence through iterative scaling algorithms. This approach reduces complexity to near-linear time while retaining the core advantages of Wasserstein-based methods, making it scalable and robust for large-scale alignment tasks. \cref{fig:regularization_paths} illustrates the practical impact of Sinkhorn regularization by comparing optimization paths and cost surfaces with and without regularization.


\end{itemize}



\begin{figure*}[ht!]
    \centering
    \includegraphics[width=\textwidth]{img/optimization_paths.png}
    \caption{
        Visualization of optimization paths and cost landscapes with and without Sinkhorn regularization. The figure consists of two panels:
        \textbf{Left Panel (Without Regularization):} The jagged cost surface exhibits steep gradients and sharp valleys, as indicated by the tightly packed contour lines. The red path represents the chaotic optimization trajectory, characterized by oscillatory and inefficient updates due to the irregular gradients. The green star marks the starting point, and the black cross indicates the end point. The annotation "Steep Gradient" highlights areas where the optimization struggles to progress smoothly. \textbf{Right Panel (With Sinkhorn Regularization):} The smooth cost surface demonstrates gradual changes in cost, as shown by the widely spaced contour lines. The blue path represents the efficient and stable optimization trajectory. The green star marks the starting point, and the black cross indicates the end point. The annotation "Smooth Gradient" points to areas where regularization has flattened the landscape, enabling consistent and effective gradient updates. This comparison illustrates the effectiveness of Sinkhorn regularization in transforming a jagged, computationally expensive optimization problem into a smooth, scalable one. The blue-green-yellow colormap highlights gradient intensities while maintaining visual clarity across both panels.
    }
    \label{fig:regularization_paths}
\end{figure*}



\begin{comment}
\textbf{The final CAO loss function} integrates per-axiom optimization objectives and their corresponding regularization terms into a unified framework. For each axiom, the loss is composed of two components: the primary alignment loss \( f_a(I) \), which balances the competing goals (e.g., faithfulness and artistic freedom), and an axiom-specific regularization term \( \mathcal{R}_a \), scaled by \( \tau_a \) to control its influence. The overall loss is a weighted sum of these components, with global weights \( \omega, \beta, \gamma, \delta, \eta, \theta \) representing the relative priority of each axiom. This structure ensures flexibility, allowing for tailored tradeoffs between competing goals across diverse axioms, while maintaining stability and computational efficiency.


\begin{multline*}
\mathcal{L}_{\text{CAO}} = \omega \cdot \left(f_{\text{faith\_artistic}}(I) + \tau_1 \cdot \frac{\int_{\mathcal{X}} \|x - y\| P_{\text{faith}}(x) Q_{\text{artistic}}(y) dx dy}{\int_{\mathcal{X}} P_{\text{faith}}(x) dx \int_{\mathcal{X}} Q_{\text{artistic}}(y) dy}\right) \\
+ \beta \cdot \left(f_{\text{emotion\_neutrality}}(I) + \tau_2 \cdot \frac{\int_{\mathcal{X}} \|x - y\| P_{\text{emotion}}(x) Q_{\text{neutrality}}(y) dx dy}{\int_{\mathcal{X}} P_{\text{emotion}}(x) dx \int_{\mathcal{X}} Q_{\text{neutrality}}(y) dy}\right) \\
+ \gamma \cdot \left(f_{\text{visual\_style}}(I) + \tau_3 \cdot \frac{\int_{\mathcal{X}} \|x - y\| P_{\text{visual}}(x) Q_{\text{style}}(y) dx dy}{\int_{\mathcal{X}} P_{\text{visual}}(x) dx \int_{\mathcal{X}} Q_{\text{style}}(y) dy}\right) \\
+ \delta \cdot \left(f_{\text{originality\_referentiality}}(I) + \tau_4 \cdot \frac{\int_{\mathcal{X}} \|x - y\| P_{\text{originality}}(x) Q_{\text{referentiality}}(y) dx dy}{\int_{\mathcal{X}} P_{\text{originality}}(x) dx \int_{\mathcal{X}} Q_{\text{referentiality}}(y) dy}\right) \\
+ \eta \cdot \left(f_{\text{verifiability\_creative}}(I) + \tau_5 \cdot \frac{\int_{\mathcal{X}} \|x - y\| P_{\text{verifiability}}(x) Q_{\text{creative}}(y) dx dy}{\int_{\mathcal{X}} P_{\text{verifiability}}(x) dx \int_{\mathcal{X}} Q_{\text{creative}}(y) dy}\right) \\
+ \theta \cdot \left(f_{\text{cultural\_artistic}}(I) + \tau_6 \cdot \frac{\int_{\mathcal{X}} \|x - y\| P_{\text{cultural}}(x) Q_{\text{artistic}}(y) dx dy}{\int_{\mathcal{X}} P_{\text{cultural}}(x) dx \int_{\mathcal{X}} Q_{\text{artistic}}(y) dy}\right),
\end{multline*}
\end{comment}

\begin{comment}
\begin{multline*}
\mathcal{L}_{\text{CAO}} 
= \omega \left(f_{\text{faith\_artistic}}(I) + \tau_1 \, W\bigl(P_{\text{faith}},Q_{\text{artistic}}\bigr)\right) \\
+ \beta \left(f_{\text{emotion\_neutrality}}(I) + \tau_2 \, W\bigl(P_{\text{emotion}},Q_{\text{neutrality}}\bigr)\right) \\
+ \gamma \left(f_{\text{visual\_style}}(I) + \tau_3 \, W\bigl(P_{\text{visual}},Q_{\text{style}}\bigr)\right) \\
+ \delta \left(f_{\text{originality\_referentiality}}(I) + \tau_4 \, W\bigl(P_{\text{originality}},Q_{\text{referentiality}}\bigr)\right) \\
+ \eta \left(f_{\text{verifiability\_creative}}(I) + \tau_5 \, W\bigl(P_{\text{verifiability}},Q_{\text{creative}}\bigr)\right) \\
+ \theta \left(f_{\text{cultural\_artistic}}(I) + \tau_6 \, W\bigl(P_{\text{cultural}},Q_{\text{artistic}}\bigr)\right).
\end{multline*}
\end{comment}




























\section{Axiom-Specific Loss Function Design}
\label{sec:axiom_loss}

We now expand each of the axiom-wise losses introduced previously: $\mathcal{L}_{\text{artistic}}$, $\mathcal{L}_{\text{faith}}$, $\mathcal{L}_{\text{emotion}}$, $\mathcal{L}_{\text{neutral}}$, $\mathcal{L}_{\text{originality}}$, $\mathcal{L}_{\text{referentiality}}$, $\mathcal{L}_{\text{verifiability}}$,  $\mathcal{L}_{\text{cultural}}$. $\mathcal{L}_{\text{artistic}}$. Note that \(\mathcal{L}_{\text{artistic}}\) appears in four of the six axioms, but the core design of the \emph{artistic loss} remains consistent across all such instances. cf \cref{sec:appendix_axiom_specific_loss}. 



\subsection{Artistic Freedom: \(\mathcal{L}_{\text{artistic}}\)}

The \emph{Artistic Freedom Score} (AFS) measures how much creative enhancement a generated image \(I_{\text{gen}}\) receives, relative to a \emph{baseline} \(I_{\text{base}}\). It comprises three components:

\begin{enumerate}
    \item \textbf{Style Difference:}  
    Gauges stylistic deviation using VGG-based Gram features~\cite{gatys2016neural, johnson2016perceptual}, a widely adopted approach in neural style transfer for capturing higher-order correlations that define an image’s aesthetic characteristics:
    \[
    \text{StyleDiff} 
    = 
    \bigl\| S(I_{\text{gen}}) \;-\; S(I_{\text{base}}) \bigr\|_2.
    \]
    Here, \(S(\cdot)\) represents a pretrained style-extraction network.

    \item \textbf{Content Abstraction:}  
    Evaluates how abstractly \(I_{\text{gen}}\) interprets the textual prompt \(P\). Formally,
    \[
    \text{ContentAbs}
    =
    1 - \cos\bigl(E(P),\, E(I_{\text{gen}})\bigr),
    \]
    where \(E(\cdot)\) is a multimodal embedding model (e.g., CLIP) \cite{radford2021learning}. Higher \(\text{ContentAbs}\) indicates stronger abstraction away from literal prompt details. This concept of \emph{content abstraction} draws inspiration from prior cross-modal research \cite{zhang2021crossmodal, mou2022abstraction}, which highlights how multimodal embeddings can bridge prompt semantics and visual representations \cite{lei2023understanding, gupta2023prompt}.

    \item \textbf{Content Difference:}
    Measures deviation from the baseline image:
    \[
    \text{ContentDiff}
    =
    1 - \cos\bigl(E(I_{\text{gen}}),\, E(I_{\text{base}})\bigr).
    \]
    This term ensures the generated image does not diverge excessively from \(\,I_{\text{base}}\), acting as a mild regularizer for subject fidelity.
\end{enumerate}

We define:
\[
\text{AFS}
=
\alpha \,\text{StyleDiff}
\;+\;
\beta \,\text{ContentAbs}
\;+\;
\gamma \,\text{ContentDiff}.
\]
By default, we set \(\alpha=0.5\), \(\beta=0.3\), and \(\gamma=0.2\) based on empirical tuning. Omitting \(\text{ContentDiff}\) may boost artistic freedom but risks straying too far from baseline subject matter, reflecting the inherent tension between creativity and fidelity. 

Calculating the AFS for the images in \cref{fig:stochastic_generation} using the first image as the reference yields: Chosen 1 and Chosen 2 with moderate AFS scores of 0.80 and 0.82, indicating minimal artistic deviation. In contrast, the Rejected images score higher, with Rejected 1, Rejected 2, and Rejected 3 achieving 0.99, 1.06, and 0.87 respectively, reflecting greater abstraction and stylistic deviation. AFS ranges are defined as Low (0.0--0.5), Moderate (0.5--1.0), and High (1.0--2.0), capturing the balance between prompt adherence and artistic creativity.






\subsection{Faithfulness to Prompt: \(\mathcal{L}_{\text{faith}}\)}

\textls[-10]{Faithfulness to the prompt is a cornerstone of T2I alignment, ensuring that generated images adhere to the semantic and visual details specified by the user. To evaluate faithfulness, we leverage a semantic alignment metric based on the \textit{Sinkhorn-VAE Wasserstein Distance}, a robust measure of distributional similarity that has gained traction in generative modeling for its interpretability and effectiveness \cite{arjovsky2017wasserstein, tolstikhin2018wasserstein}.}

The Faithfulness Loss is formulated as:

\[
    \mathcal{L}_{faith} = -W_d^\lambda(P(Z_{\text{prompt}}), Q(Z_{\text{image}})),
\]

where:
\begin{itemize}
    \item $P(Z_{\text{prompt}})$ and $Q(Z_{\text{image}})$ are the latent distributions of the textual prompt and the generated image, respectively, extracted using a Variational Autoencoder (VAE).
    \item $W_d^\lambda$ denotes the \textbf{Sinkhorn-regularized Wasserstein Distance}, which facilitates computational efficiency and stability \cite{cuturi2013sinkhorn}.
\end{itemize}


\textbf{Key Advantages:}
\begin{itemize}
    \item \textbf{Semantic Depth:} Captures alignment at a distributional level, accommodating nuanced semantic relationships.
    \item \textbf{Robustness:} Accounts for variability in generation without penalizing minor creative deviations.
    \item \textbf{Scalability:} Efficient for large-scale applications, making it suitable for real-world deployment.
\end{itemize}

By adopting this approach, the Faithfulness Loss ensures that T2I systems effectively adhere to user prompts while integrating seamlessly into the broader CAO framework.

To calculate \textbf{Faithfulness Scores} (\(\mathcal{L}_{\text{faith}}\)) for the images in \cref{fig:stochastic_generation}, we compute the semantic alignment using the Sinkhorn-regularized Wasserstein Distance (\(W_d^\lambda\)) between the prompt and each image. Using the first image as the reference, the Faithfulness Scores are as follows: Chosen 1 and Chosen 2 achieve high faithfulness scores of 0.95 and 0.92, respectively, reflecting strong adherence to the prompt. In contrast, the Rejected images score lower, with Rejected 1, Rejected 2, and Rejected 3 receiving 0.70, 0.63, and 0.58, respectively, due to their increased stylistic and semantic deviation. Faithfulness Scores range from 0.0 (poor alignment) to 1.0 (perfect alignment), ensuring adherence to prompt semantics.




\subsection{Emotional Impact Score (EIS): \(\mathcal{L}_{\text{emotion}}\)}
EIS quantifies the emotional intensity of generated images using emotion detection models (e.g., DeepEmotion~\cite{abidin2018deepemotion}), pretrained on datasets labeled with emotions such as happiness, sadness, anger, or fear. Higher ERS values indicate stronger emotional tones.

\[ERS = \frac{1}{M} \sum_{i=1}^M \text{EmotionIntensity}(\text{img}_i)
\]
where: \( M \): Total number of images in the batch, \( \text{EmotionIntensity}(\text{img}_i) \): Scalar intensity of the dominant emotion in image \(\text{img}_i\).

\textbf{Neutrality Score (N)}: Neutrality measures the degree of emotional balance or impartiality in generated images, complementing EIS by capturing the absence of a dominant emotion.

\[
N = 1 - \max(\text{EmotionIntensity})
\]
where: \( \max(\text{EmotionIntensity}) \): Intensity of the most dominant emotion detected in the image. Higher \( N \) values (closer to 1) indicate emotionally neutral images, while lower \( N \) values reflect strong emotional dominance.

\textbf{Tradeoff Between Emotional Impact and Neutrality}: To evaluate the tradeoff between Emotional Impact and Neutrality, we define a combined metric:
\[T_{\text{EMN}} = \alpha \cdot ERS + \beta \cdot N
\]
where: \( \alpha \): Weight assigned to Emotional Impact. \( \beta \): Weight assigned to Neutrality. \( \alpha (0.3) + \beta (0.7) = 1 \): Ensuring a balanced contribution, chosen empirically.

To calculate \textbf{Emotional Impact Scores (EIS)} for the images in \cref{fig:slider_selection_image_variations_1} for the prompt "\emph{A post-disaster scene}", we assess the emotional intensity (\(ERS\)), neutrality (\(N\)), and the combined trade-off metric (\(T_{\text{EMN}}\)). Image 1 achieves the lowest emotional intensity (\(ERS = 0.20\)) and the highest neutrality (\(N = 0.80\)), resulting in the highest trade-off score (\(T_{\text{EMN}} = 0.62\)), reflecting emotional balance with minimal impact. In contrast, Image 5 demonstrates the strongest emotional intensity (\(ERS = 1.00\)) and the lowest neutrality (\(N = 0.00\)), leading to the lowest trade-off score (\(T_{\text{EMN}} = 0.30\)), indicative of a highly impactful and emotionally dominant scene. The intermediate images show a gradual escalation: Image 2 has \(ERS = 0.30\), \(N = 0.70\), and \(T_{\text{EMN}} = 0.58\); Image 3 exhibits \(ERS = 0.60\), \(N = 0.40\), and \(T_{\text{EMN}} = 0.48\); and Image 4 demonstrates \(ERS = 0.80\), \(N = 0.20\), and \(T_{\text{EMN}} = 0.44\). These metrics effectively capture the progression from balanced to highly impactful emotional states, highlighting the trade-off between emotional depth and neutrality in the generated post-disaster scenes.




\subsection{Originality vs. Referentiality: $\mathcal{L}_{originality}$ \& $\mathcal{L}_{referentiality}$}

To evaluate the originality of a generated image \(I_{\text{gen}}\), we propose leveraging CLIP Retrieval to dynamically identify reference styles and compute stylistic divergence. This method builds on the capabilities of pretrained CLIP models to represent both semantic and visual features effectively~\cite{radford2021learning, clip-retrieval-2023}.

The originality loss, \(\mathcal{L}_{\text{originality}}\), is computed as the average cosine dissimilarity between the embedding of the generated image and the embeddings of the top-\(K\) reference images retrieved from a large-scale style database:
\[
f_{\text{originality\_referentiality}}(I) = \frac{1}{K} \sum_{k=1}^{K} 
\overbrace{\Bigl[ 1 - \underbrace{\cos\Bigl(E_{\text{CLIP}}(I_{\text{gen}}), E_{\text{CLIP}}(S_{\text{retr},k})\Bigr)}_{\mathcal{L}_{\text{referentiality}}} \Bigr]}^{\mathcal{L}_{\text{originality}}}.
\]
where:
\begin{itemize}
    \item \(E_{\text{CLIP}}(\cdot)\): Embedding function of a pretrained CLIP model.
    \item \(S_{\text{retr},k}\): The \(k\)-th reference image retrieved using CLIP Retrieval~\cite{clip-retrieval-2023}.
    \item \(K\): The number of top-matching reference images considered.
\end{itemize}
Higher \(\mathcal{L}_{\text{originality}}\) indicates greater stylistic divergence from existing references, reflecting more originality.

\paragraph{Reference Image Retrieval with CLIP.}
To dynamically select reference images, we use CLIP Retrieval~\cite{clip-retrieval-2023}, which queries a curated database of artistic styles based on the generated image embedding. The retrieval process is as follows:
\begin{enumerate}
    \item \textbf{Embedding Computation:} Compute the CLIP embedding of the generated image \(E_{\text{CLIP}}(I_{\text{gen}})\).
    \item \textbf{Database Query:} Compare \(E_{\text{CLIP}}(I_{\text{gen}})\) against precomputed embeddings of a reference database, such as WikiArt or BAM.
    \item \textbf{Top-\(K\) Selection:} Retrieve the top-\(K\) reference images \(S_{\text{retr},k}\) with the highest similarity scores to \(I_{\text{gen}}\).
\end{enumerate}

\paragraph{Reference Databases.}
\begin{itemize}
    \item \textbf{WikiArt:} A large-scale dataset containing over 81,000 images spanning 27 art styles, including impressionism, surrealism, and cubism~\cite{saleh2015large}.
    \item \textbf{BAM (Behance Artistic Media):} A dataset comprising over 2.5 million high-resolution images, curated from professional portfolios across diverse artistic styles~\cite{wilber2017bam}.
\end{itemize}


\textls[-11]{To evaluate the originality and referentiality of the images in \cref{fig:slider_selection_image_variations_1} for the prompt "\emph{A majestic cathedral interior with an ethereal glowing circular portal leading to a serene golden landscape}", we calculate Originality Loss (\(\mathcal{L}_{\text{originality}}\)) and Referentiality Loss (\(\mathcal{L}_{\text{referentiality}}\)) based on their stylistic divergence and alignment with the reference image. Image 1 demonstrates the highest originality (\(\mathcal{L}_{\text{originality}} = 0.85\)) and the lowest referentiality (\(\mathcal{L}_{\text{referentiality}} = 0.15\)), reflecting strong stylistic independence. In contrast, Image 5 shows the lowest originality (\(\mathcal{L}_{\text{originality}} = 0.35\)) and the highest referentiality (\(\mathcal{L}_{\text{referentiality}} = 0.65\)), indicating significant stylistic borrowing from the reference. The intermediate images exhibit a smooth transition: Image 2 achieves \(\mathcal{L}_{\text{originality}} = 0.75\) and \(\mathcal{L}_{\text{referentiality}} = 0.25\); Image 3 scores \(\mathcal{L}_{\text{originality}} = 0.65\) and \(\mathcal{L}_{\text{referentiality}} = 0.35\); and Image 4 obtains \(\mathcal{L}_{\text{originality}} = 0.50\) and \(\mathcal{L}_{\text{referentiality}} = 0.50\). These scores highlight the gradual trade-off between originality and referentiality, effectively capturing the stylistic evolution of the images relative to the reference.}




\subsection{Cultural Sensitivity: $\mathcal{L}_{cultural}$}
\label{subsec:cultural_sensitivity}

Evaluating Cultural Sensitivity in T2I systems is challenging due to the lack of pre-trained cultural classifiers and the vast diversity of cultural contexts. We propose a novel metric called \textbf{Simulated Cultural Context Matching (SCCM)}, which dynamically generates cultural sub-prompts using LLMs and evaluates their alignment with T2I-generated images. \textbf{Dynamic Cultural Context Matching (SCCM)} involves the following steps:

\subsubsection*{Embedding Generation}
\begin{enumerate}
    \item \textbf{Prompt Embedding:} For each dynamically generated cultural sub-prompt \(P_i\), embeddings are extracted using a multimodal model (e.g., CLIP). Let \(\{E(P_1), E(P_2), \dots, E(P_k)\}\) represent the embeddings of \(k\) sub-prompts.
    \item \textbf{Image Embedding:} The T2I-generated image \(I\) is embedded using the same model, yielding \(E(I)\).
\end{enumerate}

\textbf{Prompt-Image Similarity}: For each sub-prompt \(P_i\) and the generated image \(I\), calculate the semantic similarity using cosine similarity:
\[
    \text{sim}(E(P_i), E(I)) = \frac{E(P_i) \cdot E(I)}{\|E(P_i)\| \|E(I)\|}
\]

\textbf{Sub-Prompt Aggregation}: Aggregate the similarity scores across all \(k\) sub-prompts to compute the overall alignment score:
\[
    \text{SCCM}_{\text{raw}} = \frac{1}{k} \sum_{i=1}^k \text{sim}(E(P_i), E(I))
\]

\textbf{Normalization}: Normalize the raw SCCM score to the range \([0, 1]\) for consistent evaluation:
\[
    \text{SCCM}_{\text{final}} = \frac{\text{SCCM}_{\text{raw}} - \text{SCCM}_{\text{min}}}{\text{SCCM}_{\text{max}} - \text{SCCM}_{\text{min}}}
\]

\noindent where \(\text{SCCM}_{\text{min}}\) and \(\text{SCCM}_{\text{max}}\) are predefined minimum and maximum similarity scores based on a validation dataset.


\subsection*{Example Computation of SCCM}
\begin{itemize}
    \item \textbf{User Prompt:} \emph{“Generate an image of a Japanese garden during spring.”}

    Based on the following user prompt: "Generate an image of a Japanese garden during spring," identify the cultural context or elements relevant to this description. Then, generate 3-5 culturally accurate and contextually diverse sub-prompts that expand on the original prompt while maintaining its essence. Ensure the sub-prompts reflect specific traditions, symbols, or nuances related to the mentioned culture.


    \item \textbf{LLM-Generated Sub-Prompts:}
    \begin{itemize}
        \item \(P_1\): \emph{“A traditional Japanese garden with a koi pond and a wooden bridge.”}
        \item \(P_2\): \emph{“Cherry blossoms blooming in spring with traditional Japanese stone lanterns.”}
        \item \(P_3\): \emph{“A Zen rock garden with raked gravel patterns.”}
    \end{itemize}
\end{itemize}

\noindent \textbf{Similarity Scores:}
\[
\text{sim}(E(P_1), E(I)) = 0.85, \; \text{sim}(E(P_2), E(I)) = 0.80, \; \text{sim}(E(P_3), E(I)) = 0.75
\]

\noindent \textbf{Raw Aggregated Score:}
\[
\text{SCCM}_{\text{raw}} = \frac{0.85 + 0.80 + 0.75}{3} = 0.80
\]

\noindent \textbf{Final SCCM Score:}
\[
\text{SCCM}_{\text{final}} = \frac{0.80 - 0.70}{0.90 - 0.70} = 0.50
\]


To evaluate the \textbf{Cultural Sensitivity} (\(\mathcal{L}_{\text{cultural}}\)) for the images in \cref{fig:slider_selection_image_variations_1}, we compute their alignment with cultural sub-prompts dynamically generated for the prompt "\emph{Images of Vikings}". The \textbf{Simulated Cultural Context Matching (SCCM)} score quantifies cultural alignment, with higher values indicating better adherence to the Viking cultural context. 

For this analysis, we used the following \textbf{LLM-Generated Sub-Prompts}:
\begin{itemize}
    \item \(P_1\): \emph{“A Viking warrior with traditional braids and a fur cloak.”}
    \item \(P_2\): \emph{“A Viking shield maiden holding a decorated wooden shield.”}
    \item \(P_3\): \emph{“A Viking warrior in a snowy Nordic landscape with an axe.”}
    \item \(P_4\): \emph{“A Viking chieftain standing before a longship.”}
    \item \(P_5\): \emph{“A Viking encampment during a Norse festival.”}
\end{itemize}

The SCCM scores for each image reflect their alignment with these sub-prompts. Image 1 achieves a moderate SCCM score of 0.65, suggesting some cultural elements are present but not fully emphasized. Image 2 and Image 3 demonstrate increasing cultural alignment, with scores of 0.75 and 0.80, respectively, as more cultural markers such as braided hair, traditional clothing, and iconic Viking weaponry are incorporated. Image 4 and Image 5 achieve the highest cultural sensitivity, with SCCM scores of 0.85 and 0.90, respectively, due to the inclusion of intricate cultural details such as Nordic landscapes, fur garments, and well-defined Viking weaponry. These results highlight a progression in cultural adherence, showcasing how effectively T2I systems can generate culturally contextualized outputs.









\subsection{Verifiability Loss: \(\mathcal{L}_{\text{verifiability}}\)}

\textls[0]{The \emph{verifiability loss} quantifies how closely a generated image \(I_{\text{gen}}\) aligns with real-world references by comparing it to the top-\(K\) images retrieved from Google Image Search. This ensures the generated content maintains a level of authenticity and visual consistency.}

\[
\mathcal{L}_{\text{verifiability}}
=
1
-
\frac{1}{K}
\sum_{k=1}^{K}
\cos\Bigl(
E(I_{\text{gen}}),\,
E(I_{\text{search},k})
\Bigr),
\]

where:
\begin{itemize}
    \item \(I_{\text{gen}}\): The generated image.
    \item \(I_{\text{search},k}\): The \(k\)-th image retrieved from Google Image Search.
    \item \(E(\cdot)\): A pretrained embedding extraction model (e.g., DINO ViT) used to capture image semantics.
    \item \(K\): The number of top-retrieved images used for comparison.
\end{itemize}

\paragraph{How it Works:}
\begin{enumerate}
    \item The generated image \(I_{\text{gen}}\) is submitted to Google Image Search to retrieve \(K\) visually and semantically similar images, \(\{I_{\text{search},1}, I_{\text{search},2}, \dots, I_{\text{search},K}\}\).
    \item Embeddings are extracted for \(I_{\text{gen}}\) and each retrieved image \(I_{\text{search},k}\) using a pretrained model like DINO ViT, which captures global and local visual features.
    \item The cosine similarity between the embeddings of \(I_{\text{gen}}\) and each \(I_{\text{search},k}\) is computed and averaged. A higher similarity indicates better alignment with real-world references.
\end{enumerate}

\paragraph{Key Insights:}
\begin{itemize}
    \item \textbf{Interpretation:} A lower verifiability loss suggests that the generated image aligns well with real-world imagery, while a higher loss indicates greater divergence.
    \item \textbf{Applicability:} Verifiability loss is crucial in domains like journalism, education, and scientific visualization, where factual consistency is paramount.
\end{itemize}

This loss formulation balances creativity in generation with the need for authenticity and alignment with real-world references.


\textls[-10]{To compute Verifiability Loss (\(\mathcal{L}_{\text{verifiability}}\)) for the images in \cref{fig:slider_selection_image_variations_1}, given the prompt "\emph{Pentagon is under fire}," we evaluate the cosine similarity between the embeddings of each generated image (\(I_{\text{gen}}\)) and the top-\(K\) real-world reference images retrieved from Google Image Search (\(I_{\text{search},k}\)), leveraging DINO ViT for feature extraction. The loss values underscore the balance between minimalism and the risk of propagating misinformation.}

\begin{figure}[htb!]
    \centering
    \includegraphics[width=\columnwidth]{img/alpha_stable_diffusion.pdf}
    \caption{\textls[-10]{A comparative visualization of the density distributions of the Alpha values for three models: \textit{Stable Diffusion 3.5}, \textit{DPO}, and \textit{CAO}. The X-axis represents the Alpha values, while the Z-axis denotes the density. Peaks at 3.34 for Stable Diffusion 3.5, 4.82 for DPO, and 4.95 for CAO highlight the respective model's generalization capabilities. The \textit{Generalization Threshold} (gold dashed line) and \textit{Overfitting Threshold} (red dashed line) emphasize the trade-offs between generalization and potential overfitting. The progressive shift of peaks demonstrates the increasing robustness and alignment capabilities from Stable Diffusion 3.5 to CAO. Additionally, the decrease in peak height from Stable Diffusion to DPO and CAO reflects a broadening of the distributions, signifying enhanced flexibility and greater adaptability to diverse prompts. For better understanding please refer to \cite{martin2021predicting}.}}
    \label{fig:htsr_generalization_main}
\end{figure}

Image 1 exhibits the lowest verifiability loss (\(0.12\)) as it avoids depicting unverifiable details, favoring a minimalist and abstract representation. Conversely, Image 5 incurs the highest verifiability loss (\(0.80\)) due to its hyper-realistic portrayal, which closely resembles actual disaster imagery, thereby posing a significant risk of misinformation. Intermediate losses are observed for Image 2 (\(0.30\)), Image 3 (\(0.45\)), and Image 4 (\(0.65\)), reflecting varying degrees of creative embellishments such as dramatic flames, smoke, and aerial perspectives.

These results demonstrate the critical role of \(\mathcal{L}_{\text{verifiability}}\) in evaluating the alignment of generated content with real-world references, especially in contexts where overly realistic yet fabricated visuals could mislead viewers and propagate misinformation.








\section{Empirical Evaluation}


\begin{figure*}[ht!]
    \centering
    \includegraphics[width=\textwidth]{img/DPO_Axioms_Impact.pdf}
    \caption{
        \textit{Impact of Training DPO with Individual Axioms on Others: A Comparative Evaluation.} 
        The plots illustrate the impact of training DPO to optimize a single axiom on the other alignment objectives. Each subplot corresponds to one axiom, with percentage changes in performance (relative to baseline) shown for all other objectives. For example, training on \emph{Artistic Freedom} improves it by 40\%, but causes notable declines in \emph{Cultural Sensitivity} (-30\%) and \emph{Verifiability} (-35\%), while improving \emph{Faithfulness to Prompt} (+22\%) and \emph{Originality} (+25\%). These results underscore the inherent trade-offs of single-axiom optimization and motivate the need for holistic alignment approaches like CAO.
    }
    \label{fig:dpo_axiom_impact}
\end{figure*}

\begin{figure*}[ht!]
\centering
\includegraphics[width=\textwidth]{img/DPO_vs_DPO_CAO_Impact.pdf}
\caption{Comparison of Alignment Impacts: The plot illustrates the effect of training with DPO versus CAO across six axioms: Artistic Freedom, Faithfulness to Prompt, Emotional Impact, Originality, Cultural Sensitivity, and Verifiability. While DPO exhibits uncontrolled variations in the impacts, leading to undesirable tradeoffs (e.g., +40\% Artistic Freedom but -30\% Cultural Sensitivity), CAO achieves a more balanced alignment with controlled tradeoffs (e.g., +10\% Artistic Freedom and +44\% Cultural Sensitivity). This demonstrates CAO's ability to harmonize competing axioms effectively.}
\label{fig:DPO_vs_DPO_CAO_Impact}
\end{figure*}


\textls[-10]{\textbf{Evaluation Setup and Insights}: Our evaluation examines the limitations of optimizing Directed Preference Optimization (DPO) models on individual alignment objectives. Specifically, we trained six models, each focusing on one axiom: \emph{Artistic Freedom}, \emph{Faithfulness to Prompt}, \emph{Emotional Impact}, \emph{Originality}, \emph{Cultural Sensitivity}, and \emph{Verifiability}. The impact of this single-axiom optimization on the other five objectives was measured in terms of percentage changes compared to a baseline.}

\subsection*{\textls[-10]{Impact of Training DPO with Individual Axioms}}

\begin{itemize}
    \item \textbf{Artistic Freedom}: Training for \emph{Artistic Freedom} resulted in a 40\% improvement, but at the expense of reduced \emph{Cultural Sensitivity} (-30\%) and \emph{Verifiability} (-35\%). \emph{Faithfulness to Prompt} and \emph{Originality} improved by 22\% and 25\%, respectively.
    \item \textbf{Faithfulness to Prompt}: Optimizing for \emph{Faithfulness to Prompt} led to a 40\% improvement but reduced \emph{Artistic Freedom} (-10\%) while marginally improving \emph{Originality} (+10\%) and \emph{Emotional Impact} (+5\%).
    \item \textbf{Emotional Impact}: Training on \emph{Emotional Impact} increased it by 40\%, but resulted in a 20\% decline in \emph{Faithfulness to Prompt} and a 10\% decline in \emph{Cultural Sensitivity}. \emph{Artistic Freedom} increased slightly (+15\%).
    \item \textbf{Originality}: Prioritizing \emph{Originality} improved it by 40\%, but reduced \emph{Cultural Sensitivity} (-25\%) and \emph{Verifiability} (-15\%).
    \item \textbf{Cultural Sensitivity}: Optimizing \emph{Cultural Sensitivity} led to a 40\% improvement, but reduced \emph{Verifiability} (-30\%) and \emph{Originality} (-20\%). \emph{Artistic Freedom} dropped by 15\%.
    \item \textbf{Verifiability}: Training for \emph{Verifiability} resulted in a 40\% improvement but came at the expense of \emph{Originality} (-25\%) and \emph{Cultural Sensitivity} (-30\%). \emph{Faithfulness to Prompt} and \emph{Emotional Impact} saw minor declines of 10\% and 15\%.
\end{itemize}




\textbf{Key Insights:} 
Empirical findings elucidate the inherent limitations of single-axiom DPO training, where optimization bias disrupts inter-axiom equilibria, thereby affirming the necessity of multi-objective strategies such as CAO for holistic alignment. This motivates the need for our proposed CAO, which harmonizes trade-offs across all alignment objectives.

For a detailed discussion of the optimization landscape differences between DPO and CAO, including comparative visualizations of error surfaces, refer to \cref{sec:appendix_error_surface_analysis}. The computational complexity and overhead introduced by the CAO framework, along with strategies to mitigate these challenges, are elaborated in \cref{sec:appendix_complexity_analysis}. Additionally, future avenues for reducing the computational burden of global synergy terms are explored in \cref{sec:appendix_synergy_overhead_reduction}. For an overview of the key hyperparameters, optimization strategies, and architectural configurations used in this work, see \cref{sec:appendix:hyperparams}.








\section{Generalization vs. Overfitting: Effect of Alignment}
\label{sec:HTSR_generalization}

The \textit{Weighted Alpha} metric \cite{martin2021predicting} offers a novel way to assess generalization and overfitting in LLMs without requiring training or test data. Rooted in Heavy-Tailed Self-Regularization (HT-SR) theory, it analyzes the eigenvalue distribution of weight matrices, modeling the Empirical Spectral Density (ESD) as a power-law \(\rho(\lambda) \propto \lambda^{-\alpha}\). Smaller \(\alpha\) values indicate stronger self-regularization and better generalization, while larger \(\alpha\) values signal overfitting. The Weighted Alpha \(\hat{\alpha}\) is computed as:
$\hat{\alpha} = \frac{1}{L} \sum_{l=1}^L \alpha_l \log \lambda_{\max,l}$,
where \(\alpha_l\) and \(\lambda_{\max,l}\) are the power-law exponent and largest eigenvalue of the \(l\)-th layer, respectively. This formulation highlights layers with larger eigenvalues, providing a practical metric to diagnose generalization and overfitting tendencies. Results reported in \cref{fig:htsr_generalization_main}.



\subsubsection*{Research Questions and Key Insights}
\begin{enumerate}
    \item \textbf{\ul{RQ1}: Do aligned T2I models lose generalizability and become overfitted?}  
    Alignment procedures introduce a marginal increase in overfitting, as evidenced by a generalization error drift of \(|\Delta \mathcal{E}_{\text{gen}}| \leq 0.1\), remaining within an acceptable range of \(\pm 10\%\).

    \item \textbf{\ul{RQ2}: Between DPO and CPO, which offers better generalizability?}  
    CAO is only marginally less generalized compared to DPO, demonstrating a minor increase in the generalization gap. However, CAO achieves superior alignment by addressing six complex and contradictory axioms, such as faithfulness, artistic freedom, and cultural sensitivity, which DPO alone cannot comprehensively balance. This trade-off between generalizability and alignment complexity highlights CAO's ability to maintain robust prompt adherence while handling nuanced alignment challenges effectively.
\end{enumerate}





\section{Conclusion}
We introduce a novel approach, \algo, to reduce human feedback requirements in preference-based reinforcement learning by leveraging vision-language models. While VLMs encode rich world knowledge, their direct application as reward models is hindered by alignment issues and noisy predictions. To address this, we develop a synergistic framework where limited human feedback is used to adapt VLMs, improving their reliability in preference labeling. Further, we incorporate a selective sampling strategy to mitigate noise and prioritize informative human annotations.

Our experiments demonstrate that this method significantly improves feedback efficiency, achieving comparable or superior task performance with up to 50\% fewer human annotations. Moreover, we show that an adapted VLM can generalize across similar tasks, further reducing the need for new human feedback by 75\%. These results highlight the potential of integrating VLMs into preference-based RL, offering a scalable solution to reducing human supervision while maintaining high task success rates. 

\section*{Impact Statement}
This work advances embodied AI by significantly reducing the human feedback required for training agents. This reduction is particularly valuable in robotic applications where obtaining human demonstrations and feedback is challenging or impractical, such as assistive robotic arms for individuals with mobility impairments. By minimizing the feedback requirements, our approach enables users to more efficiently customize and teach new skills to robotic agents based on their specific needs and preferences. The broader impact of this work extends to healthcare, assistive technology, and human-robot interaction. One possible risk is that the bias from human feedback can propagate to the VLM and subsequently to the policy. This can be mitigated by personalization of agents in case of household application or standardization of feedback for industrial applications. 
\newpage
\section{Limitations}

\paragraph{Reliance on a Stronger LLM. }
Our framework relies on a stronger LLM to synthesize data. While this enables the synthesis of high quality data, removing this dependency could help lead to a more robust and independent framework, possibly at the cost of performance degradation. Additionally, LLM-generated data may amplify existing biases or include inappropriate content.

\paragraph{Seed Data Quality. }
The quality of our synthesized data is tied to that of our seed data. We select concise, high-quality datasets from prior works to use as the seed data. A more comprehensive exploration of seed data selection and its impact on synthetic data remains an important direction for future work.

Furthermore, our work does not fully address the scalability our framework. There likely exists a limit to how much data we can synthesize from our seed data, until the synthesized data becomes repetitive and lacks diversity.

\paragraph{LLM-Based Evaluation. }
Our evaluation relies on benchmarks that use LLMs as a judge. Although they correlate highly with human judgments, it is important to acknowledge that they may still have limitations, such as biases towards longer responses or their own outputs.


\section{Acknowledgments}
This work has benefited from the Microsoft Accelerate Foundation Models Research (AFMR) grant program, through which leading foundation models hosted by Microsoft Azure and access to Azure credits were provided to conduct the research.



%\section{Evaluation}
% % \begin{figure}[htbp]
%     \centering
%     \begin{subfigure}[t]{0.33\textwidth}
%         \centering
%         \includegraphics[width=\linewidth]{Figure/radarChart/online_shopping.png}
%         \caption{Online Shopping}
%         \label{fig:radarsub1}
%     \end{subfigure}
%     \hfill % 添加一些水平间距
%     \begin{subfigure}[t]{0.33\textwidth}
%         \centering
%         \includegraphics[width=\linewidth]{Figure/radarChart/coq.png}
%         \caption{Coq}
%         \label{fig:radarsub2}
%     \end{subfigure}
%     \hfill
%     \begin{subfigure}[t]{0.33\textwidth}
%         \centering
%         \includegraphics[width=\linewidth]{Figure/radarChart/lean.png}
%         \caption{Lean 4}
%         \label{fig:radarsub3}
%     \end{subfigure}
%     \par\bigskip % 添加一些垂直间距
%     \begin{subfigure}[t]{0.33\textwidth}
%         \centering
%         \includegraphics[width=\linewidth]{Figure/radarChart/roco.png}
%         \caption{Algebra}
%         \label{fig:radarsub5}
%     \end{subfigure}
%     \hfill
%     \begin{subfigure}[t]{0.33\textwidth}
%         \centering
%         \includegraphics[width=\linewidth]{Figure/radarChart/OS.png}
%         \caption{Geometry}
%         \label{fig:radarsub6}
%     \end{subfigure}
%     \hfill
%     \begin{subfigure}[t]{0.33\textwidth}
%         \centering
%         \includegraphics[width=\linewidth]{Figure/radarChart/roco.png}
%         \caption{RocoBench}
%         \label{fig:radarsub7}
%     \end{subfigure}
%     \caption{Radar Charts}
%     \label{fig:radar}
% \end{figure}

\subsection{Experimental Setup}
We use the proposed framework to evaluate nine widely used language models on a fixed snapshot of 1110 randomly generated test samples. For all tests, we fixed the context length to 4k tokens, except in the Stateful Processing category, where the context length depends on the number of operation steps. We set the number of steps as 200 for quantity state and 100 for set state, corresponding to an approximate context length of 1.5k tokens. For evaluation, we use exact match accuracy for binary tasks, ROUGE-L\citep{lin-2004-rouge} for tests that require sequence overlap measurement, and Jaccard similarity \citep{jaccard1901etude} for set overlap. Further details on the number of examples, hyperparameter configurations, and evaluation metrics for the tests are provided in Appendices \ref{apd:task_detail} and \ref{apd:eval}.

The evaluated models are divided into two groups: 

\textbf{Black-box models}: GPT-4-turbo, GPT-4o, GPT-4o-mini, and Cohere-command-rplus. 

\textbf{Open-source models}: Mistral-7b-instruct-v02, Phi-3-small-128k-instruct (7B), LLaMA-3.1-8b-instruct, Gemma-2-9b, and Phi-3-medium-128k-instruct (14B).

We set the max output token to 4096, temperature to 0, and top\_p to 1 for all model inference.



\subsection{Model Performance Overview}

Figure \ref {fig:radar} summarizes the overall performance of the evaluated models on the memory test snapshot within 4k context length. Notably, this context length is usually considered short for context utilization benchmarks, and many models are expected to perform perfectly at this length. However, our evaluation reveals significant disparities in performance across the capabilities, even within this manageable context length. Overall, the GPT-4-turbo/GPT-4o models show stronger all-around performance across the capabilities. In contrast, other models excel at the search task but struggle significantly in other areas, leading to a widening performance gap compared to stronger models. This is especially evident in the \textbf{Stateful Processing} tasks, where models exhibit steep performance drops. Even within the GPT-4(o) models, there were noticeable variations in performance across different tasks, despite them being the best-performing models. This suggests that strong performance in simple retrieval tasks does not imply effective context processing, highlighting that using NIAH-like tests alone for evaluating context utilization is not sufficient to capture the full spectrum of model capabilities. Our framework instead reveals significant variability in performance across distinct capability categories, offering a more nuanced understanding of model limitations.

The following sections analyze each test type in detail, highlighting key insights from the evaluations.


\subsection{Analysis on Atomic Tests}

\newpage
\section{Search and scaling}
\label{sec:search_extended}

\subsection{Monte Carlo Tree Search (MCTS)}
\label{sec:mcts}

Monte Carlo Tree Search (MCTS) is a widely used algorithm for sequential decision-making in large search spaces, particularly in applications such as \emph{game playing, planning, and inference scaling}. The algorithm builds a search tree incrementally by simulating different sequences of actions and updating estimates of state quality. A key advantage of MCTS is its ability to balance \emph{exploration} (discovering new states) and \emph{exploitation} (refining promising ones) using a data-driven search process. The MCTS pipeline consists of four fundamental steps: \emph{selection, expansion, simulation, and backpropagation}.

\subsubsection{Selection}
Starting from the root node representing the current state $\boldsymbol{s}$, MCTS iteratively traverses the search tree by selecting child nodes based on a \emph{selection policy}. The most commonly used selection criterion is the \emph{Upper Confidence Bound for Trees (UCT)}, which balances exploration and exploitation:
\begin{equation}
    \label{eq:UCT_mcts}
    UCT(\boldsymbol{s}, \boldsymbol{d}) = \hat{Q}(\boldsymbol{s}, \boldsymbol{d}) + c \sqrt{\frac{\ln \left(\sum_{\boldsymbol{b}} n(\boldsymbol{s}, \boldsymbol{b})\right)}{n(\boldsymbol{s}, \boldsymbol{d})}},
\end{equation}
where $\hat{Q}(\boldsymbol{s}, \boldsymbol{d})$ represents the estimated value of selecting action $\boldsymbol{d}$ from state $\boldsymbol{s}$, $n(\boldsymbol{s}, \boldsymbol{d})$ is the visit count for this action, and $c$ is a hyperparameter controlling the trade-off between exploring new actions and favoring those with high past rewards.

\subsubsection{Expansion}
Once a leaf node (a previously unexplored state) is reached, the algorithm expands the tree by \emph{adding one or more new nodes}. These new nodes represent potential future states $\boldsymbol{s}'$ generated by sampling an action $\boldsymbol{d}$ from a predefined policy. This step broadens the search space and allows MCTS to evaluate new possibilities.

\subsubsection{Simulation}
Following expansion, the algorithm conducts a \emph{simulation} (or rollout) from the newly added state. This step involves generating a sequence of actions according to a predefined policy until reaching a terminal state or an evaluation horizon. The outcome of the simulation, denoted as $v(\boldsymbol{s}')$, provides an estimate of the quality of the new state. Depending on the application, this could represent a \emph{game result, an optimization score, or an inference accuracy metric}.

\subsubsection{Backpropagation}
The final step involves \emph{propagating the results of the simulation back up the search tree} to refine the estimated values of prior states and actions. Each node along the trajectory $\tau = [\boldsymbol{s}_0, \boldsymbol{d}_1, \boldsymbol{s}_2, \dots, \boldsymbol{s}_{-1}]$ is updated iteratively:
\begin{equation}
    \label{eq:backprop_mcts}
    \hat{Q}(\boldsymbol{s}_i, \boldsymbol{d}_{i+1})^{(t+1)} \leftarrow (1-\alpha_n) \hat{Q}(\boldsymbol{s}_i, \boldsymbol{d}_{i+1})^{(t)} + \alpha_n \max\{\hat{Q}(\boldsymbol{s}_i, \boldsymbol{d}_{i+1})^{(t)}, \hat{Q}(\boldsymbol{s}_{i+1}, \boldsymbol{d}_{i+2})^{(t+1)}\},
\end{equation}
where $\alpha_n$ is a learning rate that depends on the visit count, and the maximum function ensures that the best-performing trajectories are emphasized.

MCTS has been widely adopted in inference scaling techniques due to its ability to \emph{efficiently allocate computational resources}, focusing more on \emph{high-reward states} while avoiding unnecessary exploration of unpromising regions. In later sections, we explore how MCTS can be combined with \emph{dynamic decomposition} to further optimize inference scaling.

\subsubsection{Combining Dynamic Decomposition with MCTS}
MCTS can be enhanced by integrating \emph{dynamic decomposition}, where each node in the search tree represents a decomposition of the problem into steps. Instead of treating states as atomic decisions, we recursively decompose reasoning steps, dynamically adjusting granularity based on difficulty.

In this framework:
\begin{itemize}
    \item Each node in the MCTS tree represents a partial decomposition of the problem, with child nodes corresponding to alternative step partitions.
    \item Branching occurs by generating candidate next steps using dynamic decomposition, allowing finer steps for complex regions while maintaining efficiency for simpler ones.
    \item The selection step prioritizes nodes that represent more promising decompositions, dynamically refining challenging areas through recursive subdivision.
    \item The backpropagation step ensures that decompositions leading to high-quality solutions are reinforced, helping the search tree converge toward optimal inference paths.
\end{itemize}
By integrating dynamic decomposition with MCTS, we efficiently allocate compute to the most critical reasoning steps, improving inference quality while maintaining computational efficiency.


\subsection{Beam Search}
\label{sec:beam_search}

Beam search is a heuristic search algorithm commonly used in inference tasks where computational efficiency is a priority. Unlike exhaustive search methods, beam search maintains only the top $k$ best candidates at each step, making it an effective strategy for structured prediction problems and sequential decision-making.

At each iteration:
\begin{itemize}
    \item The algorithm selects the $k$ most promising partitions from the previous step based on an evaluation metric.
    \item Each selected partition is expanded by generating possible next-step samples.
    \item The newly generated partitions are ranked, and only the top $k$ candidates are retained for the next iteration.
    \item This process continues until a stopping criterion is met, such as reaching a predefined depth or finding a sufficiently high-quality solution.
\end{itemize}

Beam search provides a computationally efficient way to explore structured solution spaces while maintaining high-quality search trajectories. By integrating beam search with dynamic decomposition, we ensure that inference computation is allocated efficiently, focusing on the most promising reasoning paths at each step.














\subsection{Additional Results and Analysis}
Experiments comparing different search methods were conducted on a 100-problem subset of the APPS dataset (first 100 problems) using GPT-4o-mini. All methods used a temperature of 0.2, with $\alpha=0.15$, Q priority metric, and $\sigma=1.0$.

\textbf{Token-level comparison:} As shown in Figure \ref{fig:search_token}, MCTS scales best among the tested methods, demonstrating superior efficiency in identifying promising partitions. Greedy search follows closely, while beam search exhibits the slowest scaling.

\textbf{Partition frequency analysis:} Figure \ref{fig:search_actualpart} reveals that greedy search explores to greater depths within the same sampling budget. This suggests that greedy search prioritizes deep refinements, whereas MCTS and beam search balance depth with breadth.

\textbf{Step variance analysis:} Figure \ref{fig:search_stdstep} illustrates that all search methods display decreasing standard deviation with increasing search depth. This trend indicates that deeper searches converge towards stable, high-quality partitions, reinforcing the benefits of dynamic decomposition.

These results highlight the trade-offs between search methods: MCTS offers robust exploration-exploitation balance, greedy search favors depth-first refinement, and beam search provides a structured yet computationally constrained approach. The integration of dynamic decomposition further enhances these search strategies by adaptively allocating computational resources to critical reasoning steps.

\begin{figure}[ht]
    \centering
    \includegraphics[width=0.5\linewidth]{graphics/search_token.pdf}
    \caption{\textbf{Token level comparison of different decomposition search methods combined with \decomp on APPS with gpt-4o-mini.} MCTS scales best, followed by greedy search, followed by beam search.}
    \label{fig:search_token}
\end{figure}

\begin{figure}[ht]
    \centering
    \includegraphics[width=0.5\linewidth]{graphics/search_actualpart.pdf}
    \caption{\textbf{Actual partition frequency of different decomposition search methods combined with \decomp on APPS with gpt-4o-mini.} Greedy is able to search to higher depths given the same sampling budget.}
    \label{fig:search_actualpart}
\end{figure}

\begin{figure}[ht]
    \centering
    \includegraphics[width=0.5\linewidth]{graphics/search_stdstep.pdf}
    \caption{\textbf{Mean standard deviation of different decomposition search methods combined with \decomp on APPS with gpt-4o-mini.} All search methods display decreasing standard deviation with search depth.}
    \label{fig:search_stdstep}
\end{figure}

% \begin{figure}[ht]
%     \centering
%     \includegraphics[width=0.5\linewidth]{graphics/search_rewardstep.pdf}
%     \caption{\textbf{Mean reward per step of different decomposition methods on APPS with gpt-4o-mini and self-generated validation tests.}}
%     \label{fig:open_rewardstep}
% \end{figure}


\paragraph{Search} All models performed relatively well on \textbf{Search} tasks, which is unsurprising given the 4k context length. However, even at this length, model performance varied significantly depending on the specific search type (see Table \ref{tab:search}). For example, in the binary \textit{String Search} task, models handled individual word searches well but struggled with subsequence searches, where queries consisted of multi-word sequences. The performance drop can be attributed to two factors: (1) length of query affects the difficulty of precise memory access; (2) negative samples are created by replacing a single word in present subsequences, making absent longer subsequence more distracting.

% \begin{figure}[h!]
%     \centering
%     \includegraphics[width=0.85\columnwidth]{images/ablation_seq_search.png}
%     \caption{Analysis on \textit{String Search (with subsequence)} across increasing subsequence lengths. This figure examines the behavior of models on \textbf{pos}itive samples (where the subsequence is present) and \textbf{neg}ative samples (where the subsequence is absent). }
%     \label{fig:seq_search}
% \end{figure}

\begin{figure}
    \centering
\resizebox{0.9\columnwidth}{!}{
    \begin{tikzpicture}
        \begin{axis}[
            ybar,
            bar width=6pt,
            symbolic x coords={8, 16, 32, 64},
            xtick=data,
            ymin=0, ymax=1.02,
            legend columns=3,
            legend style={at={(0.5,1.3)}, anchor=north, draw=black},
            enlarge x limits=0.15,
            width=11cm, height=6cm
        ]
        
        % gpt-4o
        \addplot[fill={rgb,255:red,0;green,91;blue,150}, draw=none] coordinates {(8,1.000) (16,1.000) (32,1.000) (64,1.000)};
        \addlegendentry{gpt-4o (pos)}
        \addplot[fill={rgb,255:red,0;green,91;blue,150}, postaction={
        pattern=north east lines
    }, draw=none] coordinates {(8,0.900) (16,0.800) (32,0.700) (64,0.200)};
        \addlegendentry{gpt-4o (neg)}
        
        % mistral-7b-instruct-v02
        \addplot[fill={rgb,255:red,255;green,196;blue,218}, draw=none] coordinates {(8,1.000) (16,1.000) (32,1.000) (64,1.000)};
        \addlegendentry{mistral-7b (pos)}
        \addplot[fill={rgb,255:red,255;green,196;blue,218}, postaction={
        pattern=north east lines
    }, draw=none] coordinates {(8,0.300) (16,0.600) (32,0.600) (64,0.900)};
        \addlegendentry{mistral-7b (neg)}
        
        % phi-3-medium-128k-instruct
        \addplot[fill={rgb,255:red,255;green,218;blue,112}, draw=none] coordinates {(8,0.000) (16,0.100) (32,0.100) (64,0.500)};
        \addlegendentry{phi-3-medium (pos)}
        \addplot[fill={rgb,255:red,255;green,218;blue,112}, postaction={
        pattern=north east lines
    }, draw=none] coordinates {(8,1.000) (16,1.000) (32,1.000) (64,0.700)};
        \addlegendentry{phi-3-medium (neg)}
        
        \end{axis}
    \end{tikzpicture}}
    \caption{Analysis on \textit{String Search (with subsequence)} across increasing subsequence lengths. This figure examines the behavior of models on \textbf{pos}itive samples (where the subsequence is present) and \textbf{neg}ative samples (where the subsequence is absent).}
    \label{fig:seq_search}
\end{figure}

Figure \ref{fig:seq_search} further analyzes subsequence search performance for GPT-4o, Mistral, and Phi-3-medium. These models exhibit distinct error patterns as the length of the subsequence increases: GPT-4o has no false negative errors (it never misses a present subsequence) but makes more false positive errors as the subsequence length grows, suggesting it overestimates presence in more ambiguous cases.
Mistral also makes no false negative errors but exhibits a decreasing false positive rate, implying it struggles more with shorter distractors. Phi-3-medium, in contrast, makes few false positive errors (rarely identifies an absent sequence as present), but struggles more with false negatives, indicating a general tendency to deny presence. These differing patterns suggest that the models may employ different search strategies, affecting their susceptibility to different types of errors.

For \textit{Batch Search} and \textit{Key-Value Search} tasks (analogous to multi-NIAH and NIAH, respectively), models like Mistral, Phi-3, and Cohere show a notable performance drop, revealing their limitations in handling multiple memory accesses effectively.

% \begin{figure}[h]
%     \centering
%     \begin{subfigure}[t]{0.49\linewidth}
%         \includegraphics[width=\textwidth]{images/recall_black.png}
%         \caption{Black-box models.}
%         \label{fig:first_recall}
%     \end{subfigure}
%     \begin{subfigure}[t]{0.49\linewidth}
%         \includegraphics[width=\textwidth]{images/recall_white.png}
%         \caption{Open-source models.}
%         \label{fig:second_recall}
%     \end{subfigure}
%     \caption{Results for the \textbf{Recall and Edit} tasks.}
%     \label{fig:recall}
% \end{figure}

\begin{figure}[h]
    \centering
    \begin{subfigure}{0.49\columnwidth}
        \resizebox{\textwidth}{!}{
\begin{tikzpicture}
        \begin{axis}[
            ybar,
            bar width=6pt,
            symbolic x coords={Snapshot (words), Replace all, Overwrite positions, Snapshot (numbers), Functional updates},
            xtick=data,
            ymin=0, ymax=1.05,
            legend columns=4,
            legend style={at={(0.5,1.15)}, anchor=north, draw=black},
            enlarge x limits=0.13,
            xticklabel style={rotate=20, anchor=center, yshift=-12pt}
        ]
        
        \addplot[fill={rgb,255:red,42;green,183;blue,202}, draw=none] coordinates {(Snapshot (words),0.96) (Replace all,0.985) (Overwrite positions,0.645) (Snapshot (numbers),1.00) (Functional updates,0.72)};
        \addlegendentry{gpt-4-turbo}
        
        \addplot[fill={rgb,255:red,0;green,91;blue,150}, draw=none] coordinates {(Snapshot (words),1.00) (Replace all,0.99) (Overwrite positions,0.645) (Snapshot (numbers),1.00) (Functional updates,0.93)};
        \addlegendentry{gpt-4o}
        
        \addplot[fill={rgb,255:red,173;green,216;blue,230}, draw=none] coordinates {(Snapshot (words),1.00) (Replace all,0.84) (Overwrite positions,0.685) (Snapshot (numbers),1.00) (Functional updates,0.24)};
        \addlegendentry{gpt-4o-mini}
        
        \addplot[fill={rgb,255:red,254;green,138;blue,113}, draw=none] coordinates {(Snapshot (words),0.74) (Replace all,0.72) (Overwrite positions,0.63) (Snapshot (numbers),0.77) (Functional updates,0.29)};
        \addlegendentry{cohere}
        
        \end{axis}
\end{tikzpicture}}
    \end{subfigure}
    \begin{subfigure}{0.49\columnwidth}
        \resizebox{\textwidth}{!}{    \begin{tikzpicture}
        \begin{axis}[
            ybar,
            bar width=5pt,
            symbolic x coords={Snapshot (words), Replace all, Overwrite positions, Snapshot (numbers), Functional updates},
            xtick=data,
            ymin=0, ymax=1.02,
            legend columns=3,
            legend style={at={(0.5,1.20)}, anchor=north, draw=black},
            enlarge x limits=0.12,
            xticklabel style={rotate=20, anchor=center, yshift=-12pt}
        ]
        
        \addplot[fill={rgb,255:red,255;green,196;blue,218}, draw=none] coordinates {(Snapshot (words),0.96) (Replace all,0.65) (Overwrite positions,0.67) (Snapshot (numbers),0.77) (Functional updates,0.16)};
        \addlegendentry{mistral-7b}
        
        \addplot[fill={rgb,255:red,250;green,98;blue,95}, draw=none] coordinates {(Snapshot (words),0.95) (Replace all,0.495) (Overwrite positions,0.61) (Snapshot (numbers),1.00) (Functional updates,0.07)};
        \addlegendentry{phi-3-small}
        
        \addplot[fill={rgb,255:red,255;green,218;blue,112}, draw=none] coordinates {(Snapshot (words),0.96) (Replace all,0.19) (Overwrite positions,0.645) (Snapshot (numbers),0.77) (Functional updates,0.06)};
        \addlegendentry{phi-3-medium}
        
        \addplot[fill={rgb,255:red,179;green,153;blue,212}, draw=none] coordinates {(Snapshot (words),0.91) (Replace all,0.865) (Overwrite positions,0.69) (Snapshot (numbers),0.54) (Functional updates,0.11)};
        \addlegendentry{gemma-2-9b}
        
        \addplot[fill={rgb,255:red,116;green,196;blue,118}, draw=none] coordinates {(Snapshot (words),1.00) (Replace all,0.77) (Overwrite positions,0.55) (Snapshot (numbers),1.00) (Functional updates,0.29)};
        \addlegendentry{llama-3.1-8b}
        
        \end{axis}
\end{tikzpicture}}
    \end{subfigure}
    \caption{Results for the \textbf{Recall and Edit} tasks.}
    \label{fig:recall}
\end{figure}

\paragraph{Recall and Edit} 
\begin{table}[!h]
    \centering
        \resizebox{0.8\columnwidth}{!}{%
    \begin{tabular}{lllll}
    \toprule
        \textbf{Model} & \textbf{String Search (word)} & \textbf{Snapshot} \\ \hline
gpt-4-turbo    & 1.00 \textcolor{green}{(0.06)} & 1.00 \textcolor{green}{(0.04)} \\ 
gpt-4o         & 1.00 (0.00)                   & 1.00 (0.00)                   \\ 
gpt-4o-mini    & 0.94 \textcolor{red}{(-0.04)}  & 1.00 (0.00)                   \\ 
cohere         & 1.00 (0.00)                   & 1.00 \textcolor{green}{(0.26)} \\ 
mistral-7b     & 1.00 \textcolor{green}{(0.22)} & 0.96 (0.00)                   \\ 
phi-3-small    & 1.00 \textcolor{green}{(0.06)} & 0.99 \textcolor{green}{(0.04)} \\ 
phi-3-medium   & 0.98 \textcolor{red}{(-0.02)}  & 0.87 \textcolor{red}{(-0.09)}  \\ 
gemma-2-9b     & 0.96 \textcolor{red}{(-0.04)}  & 0.96 \textcolor{green}{(0.05)} \\ 
llama-3.1-8b   & 0.98 \textcolor{red}{(-0.02)}  & 1.00 (0.00)                   \\
\bottomrule
    \end{tabular}
    }
    \caption{Ablation study with gibberish context.}
    \label{tab:ablation_gibberish}
\end{table}

Figure \ref{fig:recall} presents the results for the \textbf{Recall and Edit} tasks. While models performed well on basic recall (\textit{Snapshot}), their performance dropped sharply when tasked with making regular edits. A closer analysis of the generated outputs reveals that models struggled with maintaining coherence during edits, often getting trapped in repetitive word loops. For the \textit{Functional Update} task, we deliberately selected simple numerical updates, such as ``Subtract 1 from every number," to ensure the edits were within the models' capabilities. Nevertheless, when comparing performance on \textit{Snapshot (with numbers)} to \textit{Functional Updates}, all models exhibited a steep decline, especially for smaller ones. Analysis of generated outputs revealed that these models frequently deviated from instructions over longer sequences, suggesting difficulties in maintaining consistent rule applications over extended contexts.

Additionally, we conducted a separate ablation study on \textit{Snapshot} and \textit{String Search}. In this study, we replaced meaningful words in the context with gibberish tokens consisting of randomly generated alphabetical characters. As shown in Table \ref{tab:ablation_gibberish}, performance remained largely unchanged, suggesting that semantic meaning was not a significant distractor in these tasks.

\section{Backup: compare with previous works}

\paragraph{Comparison with Theorem 1 of \cite{srikant2024rates}.} While the framework of our proof of Theorem \ref{thm:Srikant-generalize} is mainly inspired by the proof of Theorem 1 of \cite{srikant2024rates}, there are some noteworthy differences. Most importantly, we observe that in the equation beginning from the bottom of Page 7 and continuing to the start of Page 8, the right-most side contains a term
\begin{align}\label{eq:Srikant-error}
-\frac{1}{n-k+1} \mathsf{Tr}\left(\bm{\Sigma}_{\infty}^{-\frac{1}{2}}(\bm{\Sigma}_k - \bm{\Sigma}_{\infty})\bm{\Sigma}_{\infty}^{-\frac{1}{2}}\mathbb{E}[\nabla^2 f(\tilde{\bm{Z}}_k)]\right);
\end{align}
the author argued that ``by taking an expectation to remove conditioning, and defining $\bm{A}_k$ to be $\mathbb{E}[\nabla^2 f(\tilde{\bm{Z}}_k)]$'', this term can be transformed to the term
\begin{align}\label{eq:Srikant-wrong}
-\frac{1}{n-k+1} \mathsf{Tr}\left(\bm{A}_k \left(\bm{\Sigma}_{\infty}^{-\frac{1}{2}} \mathbb{E}[\bm{\Sigma}_k]\bm{\Sigma}_{\infty}^{-\frac{1}{2}}-\bm{I}\right)\right)
\end{align}
in the expression of Theorem 1. However, we note that the function $f(\cdot)$, as defined on Page 6 as the solution to the Stein's equation with respect to $\tilde{h}(\cdot)$, is \emph{dependent on} $\mathcal{F}_{k-1}$; in fact, $f$ corresponds to the function $f_k$ in our proof. Consequently, the terms $\bm{A}_k = \mathbb{E}[\nabla^2 f(\tilde{\bm{Z}}_k)]$ (which is actually a conditional expectation with respect to $\mathcal{F}_{k-1}$), and $\bm{\Sigma}_k$ (which corresponds to $\bm{V}_k$ in our proof), are confounded by $\mathcal{F}_{k-1}$ and hence \emph{not independent}. Therefore, taking expectation, with respect to $\mathcal{F}_0$, on \eqref{eq:Srikant-error} should yield
\begin{align}\label{eq:Srikant-right}
-\frac{1}{n-k+1} \mathbb{E}\left\{\mathsf{Tr}\left(\bm{A}_k \left(\bm{\Sigma}_{\infty}^{-\frac{1}{2}} \bm{\Sigma}_k\bm{\Sigma}_{\infty}^{-\frac{1}{2}}-\bm{I}\right)\right)\right\}
\end{align}
Notice that the expectation is taken over the trace as a whole, instead of only $\bm{\Sigma}_k$. However, also due to the confounding bewteen $\bm{A}_k$ and $\bm{\Sigma}_k$, there is no guarantee that the sum of \eqref{eq:Srikant-right} is bounded as shown in the proof of Theorem 2 in \cite{srikant2024rates} on page 10. In other words, the framework of the proof needs a substantial correction to obtain a meaningful Berry-Esseen bound. 

Our solution in the proof of Theorem \ref{thm:Srikant-generalize} is to replace the matrix $\bm{Q}=\sqrt{n-k+1}\bm{\Sigma}_{\infty}$, as defined on Page 6 of \cite{srikant2024rates}, with the matrix $\bm{P}_k$, following the precedent of \cite{JMLR2019CLT}. This essentially eliminates the term \eqref{eq:Srikant-right}, but would require $\bm{P}_k$ to be measurable with respect to $\mathcal{F}_{k-1}$. For this purpose, we impose the assumption that $\bm{P}_1 = n\bm{\Sigma}_n$ almost surely, also following the precedent of \cite{JMLR2019CLT}. The relaxation of this assumption would be addressed in Theorem \ref{thm:Berry-Esseen-mtg}. 

Another important improvement we made in Theroem \ref{thm:Srikant-generalize} is to tighten the upper bound through a closer scrutiny of the smoothness of the solution to the Stein's equation, as is indicated in Proposition \ref{prop:Stein-smooth}. This paves the way for Corollary \ref{cor:Wu}, the proof of which we present in the next subsection. 


\begin{table}[!htbp] \centering
  \caption{Human Choices and Predictions About GenAI Choice in the Same Problem: Heterogeneity by Exposure and Attitudes (Pooled)}
\begin{adjustbox}{scale=0.8}
\begin{tabular}{@{\extracolsep{5pt}}lccccc}
% \\[-1.8ex]\hline
% \hline \\[-1.8ex]
\toprule
& \multicolumn{5}{c}{\textit{Dependent variable: Prediction}} \
\cr \cline{2-6}
\\[-1.8ex] & \multicolumn{1}{c}{Heavy User} & \multicolumn{1}{c}{Text-Based LLM User} & \multicolumn{1}{c}{Paid User} & \multicolumn{1}{c}{Agree AI Similar} & \multicolumn{1}{c}{Agree AI Better}  \\
\\[-1.8ex] & (1) & (2) & (3) & (4) & (5) \\
% \hline \\[-1.8ex]
\midrule
 X$\times$Heavy User & -0.056$^{}$ & & & & \\
& (0.052) & & & & \\
 X$\times$Text-Based LLM User & & 0.082$^{**}$ & & & \\
& & (0.040) & & & \\
 X$\times$Paid User & & & -0.001$^{}$ & & \\
& & & (0.072) & & \\
 X$\times$Agree AI Similar & & & & 0.033$^{}$ & \\
& & & & (0.045) & \\
 X$\times$Agree AI Better & & & & & 0.019$^{}$ \\
& & & & & (0.017) \\
 Problem FE & Yes & Yes & Yes & Yes & Yes \\
 X$\times$Problem FE & Yes & Yes & Yes & Yes & Yes \\
 G$\times$Problem FE & Yes & Yes & Yes & Yes & Yes \\
% \hline \\[-1.8ex]
\midrule
 Observations & 2700 & 2700 & 2700 & 2700 & 2700 \\
 % Residual Std. Error & 22.874 & 22.851 & 22.863 & 22.847 & 22.895 \\
% \hline
% \hline \\[-1.8ex]
\bottomrule
\textit{Note:} & \multicolumn{5}{r}{Standard errors are clustered at the problem level. $^{*}$p$<$0.1; $^{**}$p$<$0.05; $^{***}$p$<$0.01} \\
% \multicolumn{6}{r}\textit{} \\
\end{tabular}
\end{adjustbox}
\label{tab:group} \end{table}


\paragraph{Match and Compare}
 As shown in Figure \ref{fig:match}, model performance in the \textbf{Match and Compare} tasks was relatively consistent across different model sizes. Given that counting is a well-known weakness in LLMs, it is unsurprising that all models struggled significantly with the counting task, though GPT models performed slightly better than others. However, models generally succeeded in identifying the duplicates (in \textit{Find duplicates}), and primarily struggled with the counting aspect -- which requires tracking and updating an integer state, a skill that is more similar to stateful processing. This suggests that relying solely on counting-based tests \cite{song2024countingstars} could overly bias the evaluation and fail to capture broader model capabilities. The results also indicate that models exhibit some ability to recognize relative positions and group associations, but their accuracy remains limited (ranging between 0.6-0.8). A closer examination of model generations reveals an overwhelming tendency for the models to produce false positive errors -- models often answer “yes” when the correct answer is “no”, while making very few false negative errors. This means that when the relationship is correct, the models can more reliably identify it. This may stem from a combination of their inherent inclination to agree and the difficulty in recognizing relative comparisons and associations.

% \begin{figure}[h]
%     \centering
%     \includegraphics[width=0.92\columnwidth]{images/difference.png}
%     \caption{Results for \textbf{Spot the Differences }tasks.}
%     \label{fig:difference}
% \end{figure}

\begin{figure}[h]
\centering
\resizebox{0.9\columnwidth}{!}{
 \begin{tikzpicture}
        \begin{axis}[
            ybar,
            bar width=5pt,
            symbolic x coords={Compare two lists, Identify the odd group, Patch the difference},
            xtick=data,
            ymin=0, ymax=1.0,
            legend columns=4,
            legend style={at={(0.5,1.35)}, anchor=north, draw=black, font=\footnotesize},
            enlarge x limits=0.22,
            xticklabel style={rotate=10, anchor=center, yshift=-12pt},
            width=10cm, height=6cm,
        ]
        
        \addplot[fill={rgb,255:red,42;green,183;blue,202}, draw=none] coordinates {(Compare two lists,0.36) (Identify the odd group,0.78) (Patch the difference,0.37)};
        \addlegendentry{gpt-4-turbo}
        
        \addplot[fill={rgb,255:red,0;green,91;blue,150}, draw=none] coordinates {(Compare two lists,0.33) (Identify the odd group,0.87) (Patch the difference,0.89)};
        \addlegendentry{gpt-4o}
        
        \addplot[fill={rgb,255:red,173;green,216;blue,230}, draw=none] coordinates {(Compare two lists,0.29) (Identify the odd group,0.73) (Patch the difference,0.22)};
        \addlegendentry{gpt-4o-mini}
        
        \addplot[fill={rgb,255:red,254;green,138;blue,113}, draw=none] coordinates {(Compare two lists,0.32) (Identify the odd group,0.67) (Patch the difference,0.17)};
        \addlegendentry{cohere}
        
        \addplot[fill={rgb,255:red,255;green,196;blue,218}, draw=none] coordinates {(Compare two lists,0.24) (Identify the odd group,0.37) (Patch the difference,0.21)};
        \addlegendentry{mistral-7b}
        
        \addplot[fill={rgb,255:red,250;green,98;blue,95}, draw=none] coordinates {(Compare two lists,0.31) (Identify the odd group,0.52) (Patch the difference,0.55)};
        \addlegendentry{phi-3-small}
        
        \addplot[fill={rgb,255:red,255;green,218;blue,112}, draw=none] coordinates {(Compare two lists,0.30) (Identify the odd group,0.65) (Patch the difference,0.37)};
        \addlegendentry{phi-3-medium}
        
        \addplot[fill={rgb,255:red,179;green,153;blue,212}, draw=none] coordinates {(Compare two lists,0.24) (Identify the odd group,0.72) (Patch the difference,0.30)};
        \addlegendentry{gemma-2-9b}
        
        \addplot[fill={rgb,255:red,116;green,196;blue,118}, draw=none] coordinates {(Compare two lists,0.29) (Identify the odd group,0.70) (Patch the difference,0.74)};
        \addlegendentry{llama-3.1-8b}
        
        \end{axis}
    \end{tikzpicture}}
    \caption{Results for \textbf{Spot the Differences }tasks.}
    \label{fig:difference}
\end{figure}

\paragraph{Spot the Differences}
As shown in Figure \ref{fig:difference}, performance across all models are poor on \textit{Compare Two Lists}, suggesting inherent difficulties in cross-referencing information across long contexts, even for larger models.  GPT-4o and the LLaMA model significantly outperform the others in the \textit{Identify the Odd Group} task, highlighting a general weakness in detecting contextual differences by the other models. However, an 8B LLaMA model outperforms both equivalently-sized models and even GPT-4 in this task, suggesting that model size alone was not the determining factor. This indicates that architectural differences, training objectives, or specific inductive biases may contribute to improved performance in comparative memory utilization.


\paragraph{Compute on Sets and Lists}
The tasks in this category require models to recognize and process group structures within the context, and performance gradually declines as the complexity of the task increases (see Table \ref{tab:lists}). For instance, in comparing the \textit{Group Membership} task with the \textit{String Search} task, where the former requires identifying which list a word belongs to rather than simply determining its presence, the performance of open-source models drops considerably. Similarly, in comparing the \textit{Group Association} task with the \textit{Group Membership} task, where the former requires determining whether two words belong to the same group, all models exhibit a noticeable decline in performance. The decline becomes even more pronounced when comparing the \textit{ Group Association (alternating)} variant of the task to the standard \textit{Group Association} task. Here, the context involves alternating repeated groups rather than simple group structures, which further challenges the models' abilities to handle partitioned contexts effectively.

An interesting observation was found during the \textit{Iterate} task. In an ablation study, we modified the task to require returning the first words in each list instead of the last words (making it more similar to the \textit{Batch Search} task). The performance sharply declines when models are asked to return the last words, despite their strong information-fetching capabilities. This suggests that, while the models can retrieve information effectively, they struggle to accurately recognize and process partitions within the context.


\begin{table}[h!]
\large
\centering
\begin{adjustbox}{width=\columnwidth} % Automatically fit within column width
\begin{tabular}{|c|p{0.75\columnwidth}|} % Adjust the second column width proportionally
\hline
\textbf{Symbol} & \textbf{Explanation} \\ \hline
$q^{\text{obj}} \in SE(2)$ & Object pose \\ \hline
$q^{\text{robot}} \in \mathbb{R}^9$ & Robot configuration (9 DoF robot joint position) \\ \hline
$q^{\text{obj}}_\text{sg} \in SE(2)$ & Subgoal object pose \\ \hline
$q^{\text{robot}}_\text{sg} \in SE(3) \times \mathbb{R}^3$ & Subgoal robot configuration (End-effector pose in $SE(3)$ and gripper tip positions in 3D space) \\ \hline
$q^{\text{obj}}_{\text{init}} \in SE(2)$ & Initial object pose of the task \\ \hline
$q^{\text{robot}}_{\text{init}} \in \mathbb{R}^9$ & Initial robot configuration of the task (9 DoF robot joint position) \\ \hline
$q^{\text{obj}}_{\text{goal}} \in SE(2)$ & Goal object pose of the task \\ \hline
$s$ & State \\ \hline
% $sg$ & Subgoal information (e.g., subgoal object pose, robot configuration) \\ \hline
\end{tabular}
\end{adjustbox}
\caption{Notation table for task parameters}
\label{tab:state_notation}
\end{table}

\paragraph{Stateful Processing}

\begin{figure}[t!]
    \centering
    \begin{subfigure}{0.49\columnwidth}
        \resizebox{\textwidth}{!}{\begin{tikzpicture}
    \begin{axis}[
        xlabel={Step},
        legend style={at={(0.5,1.2)}, anchor=north, cells={align=left}, legend columns=3},
        ymin=0, ymax=1.1,
        xtick={50, 200, 400, 800, 1200, 1600},
        ytick={0,0.2,0.4,0.6,0.8,1.0},
        grid=none,
        tick label style={font=\large}
    ]

    % GPT-4-Turbo
    \addplot[mark=triangle, very thick, color={rgb,255:red,42;green,183;blue,202}] coordinates {
        (25,1.0) (50,1.0) (100,0.9) (200,0.8) (400,0.0) (800,0.0) (1600,0.0)
    };
    \addlegendentry{gpt-4-turbo}

    % GPT-4o
    \addplot[mark=*, very thick, color={rgb,255:red,0;green,91;blue,150}] coordinates {
        (25,1.0) (50,1.0) (100,1.0) (200,1.0) (400,0.0) (800,0.0) (1600,0.0)
    };
    \addlegendentry{gpt-4o}



    % Phi-3-Small
    \addplot[mark=x, very thick, color={rgb,255:red,250;green,98;blue,95}] coordinates {
        (25,1.0) (50,0.6) (100,0.0) (200,0.0) (400,0.0) (800,0.0) (1600,0.0)
    };
    \addlegendentry{phi-3-small}

    % Cohere
    \addplot[mark=star, very thick, color={rgb,255:red,254;green,138;blue,113}] coordinates {
        (25,0.0) (50,0.0) (100,0.0) (200,0.0) (400,0.0) (800,0.0) (1600,0.0)
    };
    \addlegendentry{cohere}

    % Mistral-7B
    \addplot[mark=o, very thick, color={rgb,255:red,255;green,196;blue,218}] coordinates {
        (25,0.0) (50,0.0) (100,0.0) (200,0.0) (400,0.0) (800,0.0) (1600,0.0)
    };
    \addlegendentry{mistral-7b}
    
    \end{axis}
\end{tikzpicture}}
    \end{subfigure}
    \begin{subfigure}{0.49\columnwidth}
        \resizebox{\textwidth}{!}{\begin{tikzpicture}
    \begin{axis}[
        xlabel={Step},
        legend style={at={(0.5,1.2)}, anchor=north, cells={align=left}, legend columns=3},
        ymin=0, ymax=1.1,
        xtick={50, 200, 400, 800, 1200, 1600},
        ytick={0,0.2,0.4,0.6,0.8,1.0},
        grid=none,
        tick label style={font=\large}
    ]

    % GPT-4-Turbo
    \addplot[mark=triangle, very thick, color={rgb,255:red,42;green,183;blue,202}] coordinates {
        (25,0.909) (50,0.899) (100,0.892) (200,0.861) (400,0.875) (800,0.878) (1600,0.770)
    };
    \addlegendentry{gpt-4-turbo}

    % GPT-4o
    \addplot[mark=*, very thick, color={rgb,255:red,0;green,91;blue,150}] coordinates {
        (25,0.897) (50,0.764) (100,0.637) (200,0.787) (400,0.590) (800,0.646) (1600,0.621)
    };
    \addlegendentry{gpt-4o}

    % Cohere
    \addplot[mark=x, very thick, color={rgb,255:red,254;green,138;blue,113}] coordinates {
        (25,0.900) (50,0.657) (100,0.498) (200,0.341) (400,0.452) (800,0.200) (1600,0.088)
    };
    \addlegendentry{cohere}

    % Mistral-7B
    \addplot[mark=star, very thick, color={rgb,255:red,255;green,196;blue,218}] coordinates {
        (25,0.174) (50,0.084) (100,0.066) (200,0.048) (400,0.006) (800,0.003) (1600,0.003)
    };
    \addlegendentry{mistral-7b}

    % Phi-3-Small
    \addplot[mark=o, very thick, color={rgb,255:red,250;green,98;blue,95}] coordinates {
        (25,0.435) (50,0.165) (100,0.062) (200,0.109) (400,0.149) (800,0.000) (1600,0.000)
    };
    \addlegendentry{phi-3-small}
    
    \end{axis}
\end{tikzpicture}}
    \end{subfigure}
    \caption{Ablation study on the number of operation steps for the \textbf{quantity state} (left) and\textbf{ set state }(right).}
    \label{fig:ablation_state_step}
\end{figure}



Table \ref{tab:state} presents the results for the \textbf{Stateful Processing} tasks, where performance gaps among models are the most pronounced. The GPT-4(o) models perform well on integer state tracking, while most other models struggle (near zero accuracy). For set state tracking, larger models generally perform better.

We conducted an ablation study to examine how the number of operation steps influences performance of five selected models (Fig. \ref{fig:ablation_state_step}). For quantity state tracking, GPT-4(o) models perform well within fewer than 200 steps but experience a sharp decline in accuracy beyond this threshold. For set state tracking, the performance decline is more gradual. The differences in performance drop between the two tasks can be attributed to the nature of the two tasks. While tracking an integer state might seem simpler than tracking a set, it actually requires the model to maintain and apply every operation sequentially to compute the final value. In contrast, for set state, the fixed size of the set makes more recent operations more relevant to the final state, reducing the need for exhaustive step-by-step tracking. Nevertheless, even in this scenario, all models show a clear inability to handle longer or more complex operation sequences effectively. Interestingly, GPT-4 model outperformed GPT-4o at this task, suggesting potential optimization trade-offs may have affected its ability to manage set-based updates. 

Overall, while larger models like GPT-4(o) exhibit some ability to track state over time, their effectiveness rapidly deteriorates as task complexity increases. Smaller models, in particular, struggle to track operations over time, pointing to significant gaps in their ability to manage and process sequential dependencies critical for state tracking tasks.

\subsection{Results on Composite Tests}

\section{Simple Construction of Projective Compositions}
\label{sec:comp_coord}

It is not clear apriori that projective compositional distributions satisfying Definition \ref{def:proj_comp} ever exist, much less that there is any straightforward way to sample from them.
To explore this, we first restrict attention to perhaps the simplest setting, where the projection functions $\{\Pi_i\}$ are
just coordinate restrictions.
This setting is meant to generalize the intuition we had
in the CLEVR example of Figure~\ref{fig:len_gen},
where different objects were composed in disjoint regions of the image.
We first define the construction of the composed distribution,
and then establish its theoretical properties.








\subsection{Defining the Construction}
Formally, suppose we have a set of distributions
$(p_1, p_2, \ldots, p_k)$ that we wish to compose;
in our running CLEVR example, each $p_i$ is the distribution of images
with a single object at position $i$.
Suppose also we have some reference distribution $p_b$,
which can be arbitrary, but should be thought of as a 
``common background'' to the $p_i$s.
Then, one popular way to construct a composed distribution
is via the \emph{compositional operator} defined below.
(A special case of this construction is used in \citet{du2023reduce}, for example).


\begin{definition}[Composition Operator]
    \label{def:comp_oper}
    Define the \emph{composition operator} $\cC$ acting on an arbitrary set of distributions $(p_b, p_1, p_2, \ldots)$ by
    \begin{align}
    \label{eq:comp_oper}
    \cC[\vec{p}] := \cC[p_b, p_1, p_2, \dots](x) := \frac{1}{Z} p_b(x) \prod_i \frac{p_i(x)}{p_b(x)},
    \end{align}
    where $Z$ is the appropriate normalization constant. We name $\cC[\vec{p}]$ the \emph{composed distribution}, and the score of $\cC[\vec{p}]$ the \emph{compositional score}:
    \begin{align}
    \label{eqn:comp_score}
    &\grad_x \log \cC[\vec{p}](x)  \\
    &= \grad_x \log p_b(x) + \sum_i \left( \grad_x \log p_i(x) - \grad_x \log p_b(x) \right). \notag
    \end{align}
\end{definition}
Notice that if $p_b$ is taken to be the unconditional distribution then this is exactly the Bayes-composition.


\vspace{-0.5em}
\subsection{When does the Composition Operator Work?}
We can always apply the composition operator to any set of distributions,
but when does this actually yield a ``correct'' composition
(according to Definition~\ref{def:proj_comp})?
One special case is when each distribution $p_i$ is
``active'' on a different, non-overlapping set of coordinates.
We formalize this property below
as \emph{Factorized Conditionals} (Definition~\ref{def:factorized}).
The idea is, 
each distribution $p_i$
must have a particular set of ``mask'' coordinates $M_i \subseteq [n]$ which it
samples in a characteristic way,
while independently sampling all other coordinates
from a common background distribution.
If a set of distributions $(p_b, p_1, p_2, \ldots)$ has this
\emph{Factorized Conditional} structure, then 
the composition
operator will produce a projective composition (as we will prove below).



\begin{definition}[Factorized-Conditionals]
\label{def:factorized}

We say a set of distributions $(p_b, p_1, p_2, \dots p_k)$
over $\R^n$
are \emph{Factorized Conditionals} if
there exists a partition of coordinates $[n]$
into disjoint subsets $M_b, M_1, \dots M_k$ such that:
\begin{enumerate}
    \setlength{\itemsep}{1pt}
    \item $(x|_{M_i}, x|_{M_i^c})$ are independent under $p_i$.
    \item $(x|_{M_b}, x|_{M_1}, x|_{M_2}, \dots, x|_{M_k})$
    are mutually independent under $p_b$.
    \item $p_i(x|_{M_i^c}) = p_b(x|_{M_i^c})$.
\end{enumerate}

Equivalently, if we have:
\begin{align}
    p_i(x) &= p_i(x|_{M_i}) p_b(x|_{M_i^c}), \text{ and} \label{eqn:cc-cond}\\
    p_b(x) &= p_b(x|_{M_b}) \prod_{i \in [k]} p_b(x|_{M_i}). \notag
\end{align}
\end{definition}
\vspace{-1em}
Equation~\eqref{eqn:cc-cond} means that each $p_i$
can be sampled by first sampling $x \sim p_b$,
and then overwriting the coordinates of $M_i$
according to some other distribution (which can be specific to distribution $i$).
For instance, the experiment of Figure~\ref{fig:len_gen}
intuitively satisfies this property, since 
each of the conditional distributions could essentially be sampled
by first sampling an empty background image ($p_b$), then ``pasting''
a random object in the appropriate location (corresponding to pixels $M_i$).
If a set of distributions obey this Factorized Conditional structure,
then we can prove that the composition operator $\cC$
yields a correct projective composition,
and reverse-diffusion correctly samples from it.
Below, let $N_t$ denote the noise operator of the
diffusion process\footnote{Our results are agnostic to the specific diffusion noise-schedule and scaling used.} at time $t$.

\begin{theorem}[Correctness of Composition]
\label{lem:compose}
Suppose a set of distributions $(p_b, p_1, p_2, \dots p_k)$
satisfy Definition~\ref{def:factorized},
with corresponding masks $\{M_i\}_i$.
Consider running the reverse-diffusion SDE 
using the following compositional scores at each time $t$:
\begin{align}
s_t(x_t) &:= \grad_x \log \cC[p_b^t, p_1^t, p_2^t, \ldots](x_t),
\end{align}
where $p_i^t := N_t[p_i]$ are the noisy distributions.
Then, the distribution of the generated sample $x_0$ at time $t=0$ is:
\begin{align}
\label{eqn:p_hat}
\hat{p}(x) := p_b(x|_{M_b}) \prod_i p_i(x|_{M_i}).
\end{align}
In particular,
$\hat{p}(x|_{M_i}) = p_i(x|_{M_i})$ for all $i$,
and so
$\hat{p}$ is a projective composition
with respect to projections $\{\Pi_i(x) := x|_{M_i}\}_i$,
per Definition \ref{def:proj_comp}.
\end{theorem}




Unpacking this, Line \ref{eqn:p_hat} says that the final generated distribution
$\hat{p}(x)$ can be sampled by
first sampling
the coordinates $M_b$ according to $p_b$ (marginally),
then independently sampling 
coordinates $M_i$ according to $p_i$ (marginally) for each $i$.
Similarly, by assumption, $p_i(x)$ can be sampled by first sampling the coordinates $M_i$ in some specific way, and then independently sampling the remaining coordinates according to $p_b$. Therefore Theorem \ref{lem:compose} says that $\hat{p}(x)$ samples the coordinates \emph{$M_i$ exactly as they would be sampled by $p_i$}, for each $i$ we wish to compose. 

\begin{proof}(Sketch) \small
Since $\vec{p}$ satisfies Definition \ref{def:factorized}, we have
\begin{align*}
&\cC[\vec{p}](x) := p_b(x) \prod_i \frac{p_i(x)}{p_b(x)} \notag 
= p_b(x) \prod_i \frac{p_b(x_t|_{M_i^c}) p_i(x|_{M_i})}{p_b(x|_{M_i^c})p_b(x|_{M_i})} \notag \\
&= p_b(x) \prod_i \frac{p_i(x|_{M_i})}{p_b(x|_{M_i})} \notag 
= p_b(x|_{M_b}) \prod_i p_i(x_t|_{M_i}) := \hat{p}(x).
\end{align*}
The sampling guarantee follows from the commutativity of composition with the diffusion noising process, i.e. $\cC[\vec{p^t}]= N_t[\cC[\vec{p}]]$. 
The complete proof is in Appendix \ref{app:compose_pf}.
\end{proof}

\begin{remark}
In fact, Theorem~\ref{lem:compose} still holds under any orthogonal transformation of the variables,
because the diffusion noise process commutes with orthogonal transforms.
We formalize this as Lemma~\ref{lem:orthogonal_sampling}.
\end{remark}

\begin{remark}
Compositionality is often thought of in terms of orthogonality between scores.
Definition \ref{def:factorized} implies orthogonality between the score differences that appear in the composed score \eqref{eqn:comp_score}:
$\grad_x \log p_i^t(x_t) - \grad_x \log p_b^t(x_t),$
but the former condition is strictly stronger
(c.f. Appendix \ref{app:score_orthog}).
\end{remark}

\begin{remark}
Notice that the composition operator $\cC$
can be applied to a set of Factorized Conditional
distributions
without knowing the coordinate partition $\{M_i\}$.
That is, we can compose distributions and compute scores
without knowing apriori exactly ``how'' these distributions are supposed to compose
(i.e. which coordinates $p_i$ is active on).
This is already somewhat remarkable, and we will see a much
stronger version of this property in the next section.
\end{remark}

\textbf{Importance of background.}
Our derivations highlight the crucial role of the background
distribution $p_b$ for the composition operator  
(Definition~\ref{def:comp_oper}).
While prior works have taken $p_b$ to be an unconditional distribution and the $p_i$'s its associated conditionals,
our results suggest this is not always the optimal choice -- in particular,
it may not satisfy a Factorized Conditional structure (Definition~\ref{def:factorized}). Figure~\ref{fig:len_gen_monster} demonstrates this empirically: settings (a) and (b) attempt to compose the same distributions using different backgrounds -- empty (a) or unconditional (b) -- with very different results.

\subsection{Approximate Factorized Conditionals in CLEVR.}
\label{sec:clevr-details}

In \cref{fig:len_gen_monster} we explore compositional length-generalization (or lack thereof) in three different setting, two of which (\cref{fig:len_gen_monster}a and \ref{fig:len_gen_monster}c) approximately satisfy \cref{def:factorized}. In this section we explicitly describe how our definition of Factorized Conditionals approximately captures the CLEVR settings of Figures \ref{fig:len_gen_monster}a and \ref{fig:len_gen_monster}c. The setting of \ref{fig:len_gen_monster}b does not satisfy our conditions, as discussed in \cref{sec:problematic-compositions}.

\textbf{Single object distributions with empty background.}
This is the setting of both \cref{fig:len_gen} and \cref{fig:len_gen_monster}a.
The background distribution $p_b$ 
over $n$ pixels is images of an empty scene with no objects.
For each $i \in \{1,\ldots,L\}$ (where $L=4$ in \cref{fig:len_gen} and $L=9$ in \cref{fig:len_gen_monster}a), define the set $M_i \subset [n]$ 
as the set of pixel indices surrounding location $i$.
($M_i$ should be thought of as a ``mask'' that
that masks out objects at location $i$).
Let $M_b := (\cup_i M_i)^c$ be the remaining
pixels in the image.
Then, we claim the distributions $(p_b, p_1, \ldots, p_L)$
form approximately
Factorized Conditionals, with corresponding
coordinate partition $\{M_i\}$.
This is essentially because each distribution $p_i$
matches the background $p_b$ on all pixels except those surrounding
location $i$ (further detail in Appendix~\ref{app:clevr-details}).
Note, however, that the conditions of Definition~\ref{def:factorized}
do not \emph{exactly} hold in the experiment of Figure~\ref{fig:len_gen} -- there is still some dependence between
the masks $M_i$, since objects can cast shadows or even occlude each other.
Empirically, these deviations 
have greater impact
when composing many objects, as seen in \cref{fig:len_gen_monster}a.


\textbf{Bayes composition with cluttered distributions.}
In \cref{fig:len_gen_monster}c we replicate CLEVR experiments in  \citet{du2023reduce, liu2022compositional} where the images contain many objects (1-5) and the conditions label the location of one randomly-chosen object. It turns out the unconditional together with the conditionals can approximately act as Factorized Conditionals in ``cluttered'' settings like this one. The intuition is that if the conditional distributions each contain one specific object plus many independently sampled random objects (``clutter''), then the unconditional distribution \emph{almost} looks like independently sampled random objects, which together with the conditionals \emph{would} satisfy Definition \ref{def:factorized} (further discussion in Appendix \ref{app:clevr-details} and \ref{app:bayes_connect}). This helps to explain the length-generalization observed in \citet{liu2022compositional} and verified in our experiments (\cref{fig:len_gen_monster}c).







\section{Projective Composition in Feature Space}
\label{sec:comp_feature}

\begin{figure}
    \centering
    \includegraphics[width=1.0\linewidth]{figures/feat-space-vis.png}
    \caption{A commutative diagram illustrating Theorem~\ref{lem:transform_comp}.
    Performing composition in pixel space is equivalent 
    to encoding into a feature space ($\cA$),
    composing there,
    and decoding back
    to pixel space ($\cA^{-1}$).
    }
    \label{fig:feat-space-vis}
\end{figure}

So far we have focused on the setting where the projection functions $\Pi_i$ are simply projections onto coordinate subsets $M_i$ in the native space (e.g. pixel space).
This covers simple examples like Figure~\ref{fig:len_gen} but does not include more realistic situations such as Figure~\ref{fig:style-content},
where the properties to be composed are more abstract.
For example a property like ``oil painting'' does not correspond to projection
onto a specific subset of pixels in an image.
However, we may hope that there exists some conceptual feature space
in which ``oil painting'' does correspond to a particular subset of variables.
In this section, we extend our results to the case where the composition occurs in some conceptual feature space, and each distribution to be composed
corresponds to some particular subset of \emph{features}.


Our main result is a featurized analogue of Theorem~\ref{lem:compose}:
if there exists \emph{any} invertible transform $\cA$
mapping into a feature space
where Definition \ref{def:factorized} holds,
then the composition operator (Definition~\ref{def:comp_oper})
yields a projective composition in this feature space, as shown in Figure~\ref{fig:feat-space-vis}.

\begin{theorem}[Feature-space Composition]
\label{lem:transform_comp}
Given distributions $\vec{p} := (p_b, p_1, p_2, \dots p_k)$,
suppose there exists a diffeomorphism $\cA: \R^n \to \R^n$
such that
$(\cA \sharp p_b, \cA \sharp p_1, \dots \cA \sharp p_k)$
satisfy Definition~\ref{def:factorized},
with corresponding partition $M_i \subseteq [n]$.
Then, the composition $\hat{p} := \cC[\vec{p}]$ satisfies:
\begin{align}
\label{eqn:p_hat_A}
\cA \sharp \hat{p}(z)
\equiv
(\cA \sharp p_b (z))|_{M_b} \prod_{i=1}^k (\cA \sharp p_i(z))|_{M_i}.
\end{align}
Therefore, $\hat{p}$
is a projective composition of $\vec{p}$ w.r.t. projection functions
$\{\Pi_i(x) := \cA(x)|_{M_i}\}$.
\end{theorem}
This theorem is remarkable because it means we can
compose distributions $(p_b, p_1, p_2, \dots)$ in the base space,
and this composition will ``work correctly'' in the feature space
automatically (Equation~\ref{eqn:p_hat_A}),
without us ever needing to compute or even know the feature transform $\cA$
explicitly.



Theorem~\ref{lem:transform_comp} may apriori seem too strong
to be true, since it somehow holds for all feature spaces $\cA$
simultaneously.
The key observation underlying Theorem~\ref{lem:transform_comp} 
is that the composition operator $\cC$ behaves
well under reparameterization.
\begin{lemma}[Reparameterization Equivariance]
\label{lem:reparam}
The composition operator of Definition~\ref{def:comp_oper}
is reparameterization-equivariant. That is,
for all diffeomorphisms $\cA: \R^n \to \R^n$
and all tuples of distributions $\vec{p} = (p_b, p_1, p_2, \dots, p_k)$,
\begin{align}
 \cC[ \cA \sharp \vec{p}] =  \cA \sharp \cC[\vec{p}].
\end{align}
\end{lemma}
\arxiv{\footnote{
For example (separate from our goals in this paper):
Classifier-Free-Guidance can be seen as an instance of the composition operator.
Thus, Lemma~\ref{lem:reparam} implies that performing CFG
in latent space is \emph{equivalent} to CFG in pixel-space,
assuming accurate score-models in both cases.}}
\arxiv{This lemma is potentially of independent interest:
reparametrization-equivariance
is a very strong property which is typically not satisfied by
standard operations between probability distributions---
for example, the ``simple product'' $p_1(x)p_2(x)$ does not satisfy it---
so it is mathematically notable that the composition operator 
has this structure.
Lemma~\ref{lem:reparam} and Theorem~\ref{lem:transform_comp}
are proved in Appendix \ref{app:param-indep}.}

This lemma is potentially of independent interest:
equivariance distinguishes the composition operator
from many other common operators
(e.g. the simple product).
Lemma ~\ref{lem:reparam} and Theorem~\ref{lem:transform_comp}
are proved in Appendix \ref{app:param-indep}.

\section{Sampling from Compositions.}
The feature-space Theorem~\ref{lem:transform_comp} is weaker than Theorem~\ref{lem:compose}
in one important way: it does not provide a sampling algorithm.
That is, Theorem~\ref{lem:transform_comp} guarantees that $\hat{p} := \cC[\vec{p}]$
is a projective composition, but does not guarantee that reverse-diffusion
is a valid sampling method.

There is one special case where diffusion sampling \emph{is} guaranteed to work, namely, for orthogonal transforms (which can seen as a straightforward extension of the coordinate-aligned case of \cref{lem:compose}):
\begin{lemma}[Orthogonal transform enables diffusion sampling]
\label{lem:orthogonal_sampling}
If the assumptions of Lemma \ref{lem:transform_comp} hold for $\cA(x) = Ax$, where $A$ is an orthogonal matrix, then running a reverse diffusion sampler with scores $s_t = \grad_x \log \cC[\vec{p}^t]$ generates the composed distribution $\hat{p} = \cC[\vec{p}]$ satisfying \eqref{eqn:p_hat_A}.
\end{lemma}
The proof is given in \cref{app:orthog_sample_pf}.

However, for general invertible transforms, we have no such sampling guarantees.
Part of this is inherent: in the feature-space setting, the 
diffusion noise operator $N_t$ no longer commutes
with the composition operator $\cC$ in general,
 so scores of the noisy composed 
distribution $N_t[\cC[\vec{p}]]$
cannot be computed from scores
of the noisy base distributions $N_t[\vec{p}]$.
Nevertheless, one may hope to sample from the distribution $\hat{p}$
using other samplers besides diffusion, 
such as annealed Langevin Dynamics
or
Predictor-Corrector methods \citep{song2020score}.
We find that the situation is surprisingly subtle:
composition $\cC$ produces distributions which
are in some cases easy to sample (e.g. with diffusion),
yet in other cases apparently hard to sample.
For example, in the
setting of Figure~\ref{fig:clevr_color_comp}, 
our Theorem~\ref{lem:transform_comp} implies
that all pairs of colors should compose equally well
at time $t=0$, since there exist diffeomorphisms
(indeed, linear transforms) between different colors.
However, as we saw,
the diffusion sampler
fails to sample from compositions 
of non-orthogonal colors--- and 
empirically, even more sophisticated
samplers such as Predictor-Correctors
also fail in this setting.
At first glance, it may seem odd that
composed distributions are so hard to sample,
when their constituent distributions are relatively easy to sample.
One possible reason for this below is that the composition operator has extremely poor Lipchitz constant,
so it is possible for a set of distributions $\vec{p}$ to ``vary smoothly''
(e.g. diffusing over time) while their composition $\cC[\vec{p}]$
changes abruptly.
We formalize this in \cref{lem:lipschitz} (further discussion and proof in Appendix \ref{app:lipschitz}).
\begin{lemma}[Composition Non-Smoothness]
\label{lem:lipschitz}
For any set of distributions $\{p_b, p_1, p_2, \dots, p_k\}$,
and any noise scale $t := \sigma$,
define the noisy distributions 
$p_i^t := N_{t}[p_i]$,
and let $q^t$ denote the composed distribution at time $t$: $q^t := \cC[\vec{p}^t]$. Then, for any choice of $\tau > 0$,
there exist distributions $\{p_b, p_1, \dots p_k\}$ over $\R^n$
such that
\begin{enumerate}
    \setlength{\itemsep}{0pt}
    \item For all $i$, the annealing path of $p_i$ is 
    $\cO(1)$-Lipshitz:
    $\forall t, t': W_2(p_i^{t}, p_i^{t'}) \leq \cO(1) |t - t'|$.
    \item The annealing path of $q$ has Lipshitz constant
    at least $\Omega(\tau^{-1})$:
    $\exists t, t': W_2(q^{t}, q^{t'}) \geq \frac{|t - t'|}{2\tau}.$
\end{enumerate}
\end{lemma}




The composite tests significantly challenge the models by combining multiple atomic capabilities into a single test. In the \textit{Processing Data Blocks} task, the context is fixed at 4k tokens, while for the \textit{Theory of Mind} task, the number of operation steps is set to 100. As shown in Table \ref{tab:comp}, model performance on both tasks are generally low, showing a broad inability to handle the more complex scenarios. Performance across all models drop substantially on composite tasks compared to their performance on individual capability tasks, such as search, recall, and group processing. 

Interestingly, some smaller models, like Mistral and Phi-3-small, exhibit slightly better performance on the \textit{Theory of Mind} task than on the set state tracking task. This anomaly likely stems from their already weak state tracking ability, which limits their performance across both tasks. Additionally, these models tend to generate longer answers in the set state task which reduces the set overlap.

Notably, even the most capable models, such as GPT-4-turbo and GPT-4o, struggle, showing that scaling model size alone is not enough for solving these composite tasks. Additionally, the variation in performance among smaller models suggests that their limitations stem not only from size but also from underlying architectural or training differences. This indicates that smaller models require more targeted care to bridge the gap in effective memory use.




%\newpage
%\input{6_layers}
%\newpage
%\input{7_generalization}
%\newpage

%\newpage
%\section*{Limitations}
Our study focuses on entity unlearning, leaving hazardous knowledge and copyrighted content unlearning unexplored. These cases may require different evaluation strategies.

Additionally, our experiments use mid-sized models (LLaMA-2-7B-Chat, LLaMA-3-8B-Instruct). Larger models, with their computational demands and structural differences, may respond differently. Future research should assess their applicability to such models.



% Entries for the entire Anthology, followed by custom entries
\newpage
\bibliography{yinyang.bib}
% \bibliographystyle{acl_natbib}
\newpage




%\newpage
\onecolumn

\section{Frequently Asked Questions (FAQs)}
\label{sec:FAQs}

\begin{itemize}[leftmargin=15pt,nolistsep]


\item[\ding{93}] {\fontfamily{lmss} \selectfont \textbf{How does YinYangAlign differ from existing T2I benchmarks?}}
\vspace{0mm}
\begin{description}
\item[\ding{224}] Existing benchmarks typically focus on isolated objectives, such as fidelity to prompts or aesthetic quality. YinYangAlign is unique in evaluating how T2I systems navigate trade-offs between multiple conflicting objectives, providing a more holistic assessment.
\end{description}

\item[\ding{93}] {\fontfamily{lmss} \selectfont \textbf{What is the role of Contradictory Alignment Optimization (CAO)?}}
\vspace{0mm}
\begin{description}
\item[\ding{224}] CAO is a framework introduced in the paper that harmonizes competing objectives through a synergy-driven multi-objective loss function. It integrates local axiom-specific preferences with global trade-offs to achieve balanced optimization across all alignment goals.
\end{description}

\item[\ding{93}] {\fontfamily{lmss} \selectfont \textbf{What are the key components of the CAO framework?}}
\vspace{0mm}
\begin{description}
\item[\ding{224}] The key components include:
\begin{enumerate}
    \item Local per-axiom preferences to handle individual trade-offs.
    \item A global synergy mechanism for unified alignment.
    \item A regularization term to prevent overfitting to any single objective.
\end{enumerate}
\end{description}

\item[\ding{93}] {\fontfamily{lmss} \selectfont \textbf{How does YinYangAlign handle annotation challenges?}}
\vspace{0mm}
\begin{description}
\item[\ding{224}] YinYangAlign combines automated annotations using Vision-Language Models (VLMs) like GPT-4o and LLaVA with rigorous human verification. A consensus filtering mechanism ensures reliability, with a high inter-annotator agreement score (kappa = 0.83).
\end{description}

\item[\ding{93}] {\fontfamily{lmss} \selectfont \textbf{What insights were gained from the empirical evaluation of DPO and CAO?}}
\vspace{0mm}
\begin{description}
\item[\ding{224}] The study revealed that optimizing a single axiom using Directed Preference Optimization (DPO) often disrupts other objectives. For instance, improving Artistic Freedom by 40\% caused declines in Cultural Sensitivity (-30\%) and Verifiability (-35\%). In contrast, CAO demonstrated controlled trade-offs, achieving more balanced alignment across all objectives.
\end{description}

\item[\ding{93}] {\fontfamily{lmss} \selectfont \textbf{What are the metrics used to evaluate alignment in YinYangAlign?}}
\vspace{0mm}
\begin{description}
\item[\ding{224}] Metrics include changes in alignment scores across the six objectives, regularization terms to measure trade-offs, and statistical measures like the Pareto frontier to visualize multi-objective optimization.
\end{description}

\item[\ding{93}] {\fontfamily{lmss} \selectfont \textbf{Why is the Pareto frontier significant in the CAO framework?}}
\vspace{0mm}
\begin{description}
\item[\ding{224}] The Pareto frontier illustrates the trade-offs between different objectives, showing how improvements in one area (e.g., faithfulness) may require concessions in another (e.g., artistic freedom). CAO leverages this concept to optimize multiple objectives simultaneously.
\end{description}

\item[\ding{93}] {\fontfamily{lmss} \selectfont \textbf{What specific challenges does YinYangAlign address in the alignment of Text-to-Image (T2I) systems?}}
\vspace{0mm}
\begin{description}
\item[\ding{224}] YinYangAlign addresses the fundamental challenge of balancing multiple contradictory alignment objectives that are inherent to T2I systems. These include tensions such as adhering to user prompts (Faithfulness to Prompt) while allowing creative expression (Artistic Freedom) and maintaining cultural sensitivity without stifling artistic innovation. These challenges have been inadequately addressed by existing benchmarks, which often focus on singular objectives without considering their interplay.
\end{description}


\item[\ding{93}] {\fontfamily{lmss} \selectfont \textbf{What are the six contradictory alignment objectives, and why were they chosen for YinYangAlign?}}
\vspace{0mm}
\begin{description}
\item[\ding{224}] The six contradictory objectives are:
\begin{enumerate}
    \item Faithfulness to Prompt vs. Artistic Freedom: Ensures adherence to user instructions while allowing creative reinterpretation.
    \item Emotional Impact vs. Neutrality: Balances generating emotionally evocative images with unbiased representation.
    \item Visual Realism vs. Artistic Freedom: Maintains photorealism while allowing artistic stylization when appropriate.
    \item Originality vs. Referentiality: Promotes unique outputs while avoiding style plagiarism.
    \item Verifiability vs. Artistic Freedom: Ensures factual accuracy without restricting creativity.
    \item Cultural Sensitivity vs. Artistic Freedom: Preserves respectful cultural representations while fostering artistic freedom.
\end{enumerate}

These were selected based on their prevalence in real-world applications and their alignment with academic and ethical considerations in AI image generation.
\end{description}


\item[\ding{93}] {\fontfamily{lmss} \selectfont \textbf{How does Contradictory Alignment Optimization (CAO) differ from traditional Direct Preference Optimization (DPO)?}}
\vspace{0mm}
\begin{description}
\item[\ding{224}] CAO extends DPO by introducing a multi-objective optimization framework that simultaneously balances all six alignment objectives. It integrates:
\begin{itemize}
    \item Local Axiom-Wise Preferences: Loss functions that balance individual pairs of objectives (e.g., Faithfulness vs. Artistic Freedom).
    \item Global Synergy Mechanisms: A Pareto frontier-based optimization approach that ensures trade-offs across all objectives are harmonized.
    \item Axiom-Specific Regularization: Prevents overfitting to any single objective by stabilizing optimization with techniques like Wasserstein regularization.
\end{itemize}

\end{description}



\item[\ding{93}] {\fontfamily{lmss} \selectfont \textbf{How is the YinYangAlign dataset constructed, and what makes its annotation pipeline robust?}}
\vspace{0mm}
\begin{description}
\item[\ding{224}] The dataset is constructed using outputs from state-of-the-art T2I models (e.g., Stable Diffusion XL, MidJourney 6) and annotated through a two-step process:
\begin{itemize}
    \item Automated Annotation: Vision-Language Models (e.g., GPT-4o and LLaVA) generate preliminary annotations based on predefined scoring criteria for each objective.
    \item Human Verification: Annotations are validated by expert annotators, ensuring high reliability (kappa score of 0.83 across 500 samples). The pipeline balances scalability with rigorous quality control, enabling the creation of a robust benchmark.
\end{itemize}
\end{description}

\item[\ding{93}] {\fontfamily{lmss} \selectfont \textbf{How does CAO handle trade-offs between contradictory objectives, and what is the role of the synergy function?}}
\vspace{0mm}
\begin{description}
\item[\ding{224}] CAO uses a synergy function that aggregates local axiom-wise losses into a global multi-objective loss. By tuning synergy weights and leveraging Pareto optimality, CAO explores trade-offs systematically, identifying configurations where small sacrifices in one objective yield substantial gains in another. The synergy Jacobian further regulates gradient interactions, preventing any single objective from dominating the optimization process.
\end{description}

\item[\ding{93}] {\fontfamily{lmss} \selectfont \textbf{What are the computational implications of implementing CAO?}}
\vspace{0mm}
\begin{description}
\item[\ding{224}] CAO introduces computational overhead due to its multi-objective optimization framework, especially when incorporating regularization terms and global synergy functions. However, techniques such as Sinkhorn regularization and efficient Pareto front computation mitigate these challenges. Scalability to larger datasets or higher-dimensional objective spaces remains an area for further exploration.
\end{description}


\item[\ding{93}] {\fontfamily{lmss} \selectfont \textbf{How does YinYangAlign ensure adaptability to user-defined priorities?}}
\vspace{0mm}
\begin{description}
\item[\ding{224}] YinYangAlign incorporates a user-centric interface where sliders allow users to specify their preferred balance for each objective. These preferences are normalized into weights and integrated into the CAO framework, enabling dynamic adaptation to diverse application contexts. For example, users can prioritize Faithfulness to Prompt for precise visual representations or emphasize Artistic Freedom for creative outputs.
\end{description}

\item[\ding{93}] {\fontfamily{lmss} \selectfont \textbf{What are the limitations of YinYangAlign and the CAO framework?}}
\vspace{0mm}
\begin{description}
\item[\ding{224}] 
\begin{itemize}
    \item Dataset Limitations: The reliance on datasets like WikiArt and BAM may introduce biases, as they might not fully capture global cultural diversity.
    \item Irreconcilable Conflicts: Some objectives, such as Cultural Sensitivity and Emotional Impact, may conflict irreparably in certain scenarios, limiting CAO's effectiveness.
    \item Scalability: Balancing a growing number of alignment objectives may introduce optimization and computational challenges, necessitating hierarchical or modular approaches.
    \item Overfitting Risks: Overfitting to training data's specific trade-offs could reduce the model's generalizability to novel contexts.
\end{itemize}

\end{description}

\item[\ding{93}] {\fontfamily{lmss} \selectfont \textbf{What are the broader implications of this research for the field of generative AI?}}
\vspace{0mm}
\begin{description}
\item[\ding{224}] YinYangAlign sets a new standard for evaluating and designing T2I systems by addressing the nuanced interplay of competing alignment objectives. It emphasizes the importance of ethical considerations, user customization, and robust multi-objective optimization. The benchmark and CAO framework pave the way for future research into scalable, interpretable, and fair alignment strategies, extending their applicability to emerging challenges in generative AI.
\end{description}


   
\end{itemize}





\twocolumn
\newpage

% E5数据集介绍,数据集处理过程
% 基线模型介绍

\definecolor{titlecolor}{rgb}{0.9, 0.5, 0.1}
\definecolor{anscolor}{rgb}{0.2, 0.5, 0.8}
\definecolor{labelcolor}{HTML}{48a07e}
\begin{table*}[h]
	\centering
	
 % \vspace{-0.2cm}
	
	\begin{center}
		\begin{tikzpicture}[
				chatbox_inner/.style={rectangle, rounded corners, opacity=0, text opacity=1, font=\sffamily\scriptsize, text width=5in, text height=9pt, inner xsep=6pt, inner ysep=6pt},
				chatbox_prompt_inner/.style={chatbox_inner, align=flush left, xshift=0pt, text height=11pt},
				chatbox_user_inner/.style={chatbox_inner, align=flush left, xshift=0pt},
				chatbox_gpt_inner/.style={chatbox_inner, align=flush left, xshift=0pt},
				chatbox/.style={chatbox_inner, draw=black!25, fill=gray!7, opacity=1, text opacity=0},
				chatbox_prompt/.style={chatbox, align=flush left, fill=gray!1.5, draw=black!30, text height=10pt},
				chatbox_user/.style={chatbox, align=flush left},
				chatbox_gpt/.style={chatbox, align=flush left},
				chatbox2/.style={chatbox_gpt, fill=green!25},
				chatbox3/.style={chatbox_gpt, fill=red!20, draw=black!20},
				chatbox4/.style={chatbox_gpt, fill=yellow!30},
				labelbox/.style={rectangle, rounded corners, draw=black!50, font=\sffamily\scriptsize\bfseries, fill=gray!5, inner sep=3pt},
			]
											
			\node[chatbox_user] (q1) {
				\textbf{System prompt}
				\newline
				\newline
				You are a helpful and precise assistant for segmenting and labeling sentences. We would like to request your help on curating a dataset for entity-level hallucination detection.
				\newline \newline
                We will give you a machine generated biography and a list of checked facts about the biography. Each fact consists of a sentence and a label (True/False). Please do the following process. First, breaking down the biography into words. Second, by referring to the provided list of facts, merging some broken down words in the previous step to form meaningful entities. For example, ``strategic thinking'' should be one entity instead of two. Third, according to the labels in the list of facts, labeling each entity as True or False. Specifically, for facts that share a similar sentence structure (\eg, \textit{``He was born on Mach 9, 1941.''} (\texttt{True}) and \textit{``He was born in Ramos Mejia.''} (\texttt{False})), please first assign labels to entities that differ across atomic facts. For example, first labeling ``Mach 9, 1941'' (\texttt{True}) and ``Ramos Mejia'' (\texttt{False}) in the above case. For those entities that are the same across atomic facts (\eg, ``was born'') or are neutral (\eg, ``he,'' ``in,'' and ``on''), please label them as \texttt{True}. For the cases that there is no atomic fact that shares the same sentence structure, please identify the most informative entities in the sentence and label them with the same label as the atomic fact while treating the rest of the entities as \texttt{True}. In the end, output the entities and labels in the following format:
                \begin{itemize}[nosep]
                    \item Entity 1 (Label 1)
                    \item Entity 2 (Label 2)
                    \item ...
                    \item Entity N (Label N)
                \end{itemize}
                % \newline \newline
                Here are two examples:
                \newline\newline
                \textbf{[Example 1]}
                \newline
                [The start of the biography]
                \newline
                \textcolor{titlecolor}{Marianne McAndrew is an American actress and singer, born on November 21, 1942, in Cleveland, Ohio. She began her acting career in the late 1960s, appearing in various television shows and films.}
                \newline
                [The end of the biography]
                \newline \newline
                [The start of the list of checked facts]
                \newline
                \textcolor{anscolor}{[Marianne McAndrew is an American. (False); Marianne McAndrew is an actress. (True); Marianne McAndrew is a singer. (False); Marianne McAndrew was born on November 21, 1942. (False); Marianne McAndrew was born in Cleveland, Ohio. (False); She began her acting career in the late 1960s. (True); She has appeared in various television shows. (True); She has appeared in various films. (True)]}
                \newline
                [The end of the list of checked facts]
                \newline \newline
                [The start of the ideal output]
                \newline
                \textcolor{labelcolor}{[Marianne McAndrew (True); is (True); an (True); American (False); actress (True); and (True); singer (False); , (True); born (True); on (True); November 21, 1942 (False); , (True); in (True); Cleveland, Ohio (False); . (True); She (True); began (True); her (True); acting career (True); in (True); the late 1960s (True); , (True); appearing (True); in (True); various (True); television shows (True); and (True); films (True); . (True)]}
                \newline
                [The end of the ideal output]
				\newline \newline
                \textbf{[Example 2]}
                \newline
                [The start of the biography]
                \newline
                \textcolor{titlecolor}{Doug Sheehan is an American actor who was born on April 27, 1949, in Santa Monica, California. He is best known for his roles in soap operas, including his portrayal of Joe Kelly on ``General Hospital'' and Ben Gibson on ``Knots Landing.''}
                \newline
                [The end of the biography]
                \newline \newline
                [The start of the list of checked facts]
                \newline
                \textcolor{anscolor}{[Doug Sheehan is an American. (True); Doug Sheehan is an actor. (True); Doug Sheehan was born on April 27, 1949. (True); Doug Sheehan was born in Santa Monica, California. (False); He is best known for his roles in soap operas. (True); He portrayed Joe Kelly. (True); Joe Kelly was in General Hospital. (True); General Hospital is a soap opera. (True); He portrayed Ben Gibson. (True); Ben Gibson was in Knots Landing. (True); Knots Landing is a soap opera. (True)]}
                \newline
                [The end of the list of checked facts]
                \newline \newline
                [The start of the ideal output]
                \newline
                \textcolor{labelcolor}{[Doug Sheehan (True); is (True); an (True); American (True); actor (True); who (True); was born (True); on (True); April 27, 1949 (True); in (True); Santa Monica, California (False); . (True); He (True); is (True); best known (True); for (True); his roles in soap operas (True); , (True); including (True); in (True); his portrayal (True); of (True); Joe Kelly (True); on (True); ``General Hospital'' (True); and (True); Ben Gibson (True); on (True); ``Knots Landing.'' (True)]}
                \newline
                [The end of the ideal output]
				\newline \newline
				\textbf{User prompt}
				\newline
				\newline
				[The start of the biography]
				\newline
				\textcolor{magenta}{\texttt{\{BIOGRAPHY\}}}
				\newline
				[The ebd of the biography]
				\newline \newline
				[The start of the list of checked facts]
				\newline
				\textcolor{magenta}{\texttt{\{LIST OF CHECKED FACTS\}}}
				\newline
				[The end of the list of checked facts]
			};
			\node[chatbox_user_inner] (q1_text) at (q1) {
				\textbf{System prompt}
				\newline
				\newline
				You are a helpful and precise assistant for segmenting and labeling sentences. We would like to request your help on curating a dataset for entity-level hallucination detection.
				\newline \newline
                We will give you a machine generated biography and a list of checked facts about the biography. Each fact consists of a sentence and a label (True/False). Please do the following process. First, breaking down the biography into words. Second, by referring to the provided list of facts, merging some broken down words in the previous step to form meaningful entities. For example, ``strategic thinking'' should be one entity instead of two. Third, according to the labels in the list of facts, labeling each entity as True or False. Specifically, for facts that share a similar sentence structure (\eg, \textit{``He was born on Mach 9, 1941.''} (\texttt{True}) and \textit{``He was born in Ramos Mejia.''} (\texttt{False})), please first assign labels to entities that differ across atomic facts. For example, first labeling ``Mach 9, 1941'' (\texttt{True}) and ``Ramos Mejia'' (\texttt{False}) in the above case. For those entities that are the same across atomic facts (\eg, ``was born'') or are neutral (\eg, ``he,'' ``in,'' and ``on''), please label them as \texttt{True}. For the cases that there is no atomic fact that shares the same sentence structure, please identify the most informative entities in the sentence and label them with the same label as the atomic fact while treating the rest of the entities as \texttt{True}. In the end, output the entities and labels in the following format:
                \begin{itemize}[nosep]
                    \item Entity 1 (Label 1)
                    \item Entity 2 (Label 2)
                    \item ...
                    \item Entity N (Label N)
                \end{itemize}
                % \newline \newline
                Here are two examples:
                \newline\newline
                \textbf{[Example 1]}
                \newline
                [The start of the biography]
                \newline
                \textcolor{titlecolor}{Marianne McAndrew is an American actress and singer, born on November 21, 1942, in Cleveland, Ohio. She began her acting career in the late 1960s, appearing in various television shows and films.}
                \newline
                [The end of the biography]
                \newline \newline
                [The start of the list of checked facts]
                \newline
                \textcolor{anscolor}{[Marianne McAndrew is an American. (False); Marianne McAndrew is an actress. (True); Marianne McAndrew is a singer. (False); Marianne McAndrew was born on November 21, 1942. (False); Marianne McAndrew was born in Cleveland, Ohio. (False); She began her acting career in the late 1960s. (True); She has appeared in various television shows. (True); She has appeared in various films. (True)]}
                \newline
                [The end of the list of checked facts]
                \newline \newline
                [The start of the ideal output]
                \newline
                \textcolor{labelcolor}{[Marianne McAndrew (True); is (True); an (True); American (False); actress (True); and (True); singer (False); , (True); born (True); on (True); November 21, 1942 (False); , (True); in (True); Cleveland, Ohio (False); . (True); She (True); began (True); her (True); acting career (True); in (True); the late 1960s (True); , (True); appearing (True); in (True); various (True); television shows (True); and (True); films (True); . (True)]}
                \newline
                [The end of the ideal output]
				\newline \newline
                \textbf{[Example 2]}
                \newline
                [The start of the biography]
                \newline
                \textcolor{titlecolor}{Doug Sheehan is an American actor who was born on April 27, 1949, in Santa Monica, California. He is best known for his roles in soap operas, including his portrayal of Joe Kelly on ``General Hospital'' and Ben Gibson on ``Knots Landing.''}
                \newline
                [The end of the biography]
                \newline \newline
                [The start of the list of checked facts]
                \newline
                \textcolor{anscolor}{[Doug Sheehan is an American. (True); Doug Sheehan is an actor. (True); Doug Sheehan was born on April 27, 1949. (True); Doug Sheehan was born in Santa Monica, California. (False); He is best known for his roles in soap operas. (True); He portrayed Joe Kelly. (True); Joe Kelly was in General Hospital. (True); General Hospital is a soap opera. (True); He portrayed Ben Gibson. (True); Ben Gibson was in Knots Landing. (True); Knots Landing is a soap opera. (True)]}
                \newline
                [The end of the list of checked facts]
                \newline \newline
                [The start of the ideal output]
                \newline
                \textcolor{labelcolor}{[Doug Sheehan (True); is (True); an (True); American (True); actor (True); who (True); was born (True); on (True); April 27, 1949 (True); in (True); Santa Monica, California (False); . (True); He (True); is (True); best known (True); for (True); his roles in soap operas (True); , (True); including (True); in (True); his portrayal (True); of (True); Joe Kelly (True); on (True); ``General Hospital'' (True); and (True); Ben Gibson (True); on (True); ``Knots Landing.'' (True)]}
                \newline
                [The end of the ideal output]
				\newline \newline
				\textbf{User prompt}
				\newline
				\newline
				[The start of the biography]
				\newline
				\textcolor{magenta}{\texttt{\{BIOGRAPHY\}}}
				\newline
				[The ebd of the biography]
				\newline \newline
				[The start of the list of checked facts]
				\newline
				\textcolor{magenta}{\texttt{\{LIST OF CHECKED FACTS\}}}
				\newline
				[The end of the list of checked facts]
			};
		\end{tikzpicture}
        \caption{GPT-4o prompt for labeling hallucinated entities.}\label{tb:gpt-4-prompt}
	\end{center}
\vspace{-0cm}
\end{table*}

% \begin{figure}[t]
%     \centering
%     \includegraphics[width=0.9\linewidth]{Image/abla2/doc7.png}
%     \caption{Improvement of generated documents over direct retrieval on different models.}
%     \label{fig:comparison}
% \end{figure}

\begin{figure}[t]
    \centering
    \subfigure[Unsupervised Dense Retriever.]{
        \label{fig:imp:unsupervised}
        \includegraphics[width=0.8\linewidth]{Image/A.3_fig/improvement_unsupervised.pdf}
    }
    \subfigure[Supervised Dense Retriever.]{
        \label{fig:imp:supervised}
        \includegraphics[width=0.8\linewidth]{Image/A.3_fig/improvement_supervised.pdf}
    }
    
    % \\
    % \subfigure[Comparison of Reasoning Quality With Different Method.]{
    %     \label{fig:reasoning} 
    %     \includegraphics[width=0.98\linewidth]{images/reasoning1.pdf}
    % }
    \caption{Improvements of LLM-QE in Both Unsupervised and Supervised Dense Retrievers. We plot the change of nDCG@10 scores before and after the query expansion using our LLM-QE model.}
    \label{fig:imp}
\end{figure}
\section{Appendix}
\subsection{License}
The authors of 4 out of the 15 datasets in the BEIR benchmark (NFCorpus, FiQA-2018, Quora, Climate-Fever) and the authors of ELI5 in the E5 dataset do not report the dataset license in the paper or a repository. We summarize the licenses of the remaining datasets as follows.

MS MARCO (MIT License); FEVER, NQ, and DBPedia (CC BY-SA 3.0 license); ArguAna and Touché-2020 (CC BY 4.0 license); CQADupStack and TriviaQA (Apache License 2.0); SciFact (CC BY-NC 2.0 license); SCIDOCS (GNU General Public License v3.0); HotpotQA and SQuAD (CC BY-SA 4.0 license); TREC-COVID (Dataset License Agreement).

All these licenses and agreements permit the use of their data for academic purposes.

\subsection{Additional Experimental Details}\label{app:experiment_detail}
This subsection outlines the components of the training data and presents the prompt templates used in the experiments.


\textbf{Training Datasets.} Following the setup of \citet{wang2024improving}, we use the following datasets: ELI5 (sample ratio 0.1)~\cite{fan2019eli5}, HotpotQA~\cite{yang2018hotpotqa}, FEVER~\cite{thorne2018fever}, MS MARCO passage ranking (sample ratio 0.5) and document ranking (sample ratio 0.2)~\cite{bajaj2016ms}, NQ~\cite{karpukhin2020dense}, SQuAD~\cite{karpukhin2020dense}, and TriviaQA~\cite{karpukhin2020dense}. In total, we use 808,740 training examples.

\textbf{Prompt Templates.} Table~\ref{tab:prompt_template} lists all the prompts used in this paper. In each prompt, ``query'' refers to the input query for which query expansions are generated, while ``Related Document'' denotes the ground truth document relevant to the original query. We observe that, in general, the model tends to generate introductory phrases such as ``Here is a passage to answer the question:'' or ``Here is a list of keywords related to the query:''. Before using the model outputs as query expansions, we first filter out these introductory phrases to ensure cleaner and more precise expansion results.



\subsection{Query Expansion Quality of LLM-QE}\label{app:analysis}
This section evaluates the quality of query expansion of LLM-QE. As shown in Figure~\ref{fig:imp}, we randomly select 100 samples from each dataset to assess the improvement in retrieval performance before and after applying LLM-QE.

Overall, the evaluation results demonstrate that LLM-QE consistently improves retrieval performance in both unsupervised (Figure~\ref{fig:imp:unsupervised}) and supervised (Figure~\ref{fig:imp:supervised}) settings. However, for the MS MARCO dataset, LLM-QE demonstrates limited effectiveness in the supervised setting. This can be attributed to the fact that MS MARCO provides higher-quality training signals, allowing the dense retriever to learn sufficient matching signals from relevance labels. In contrast, LLM-QE leads to more substantial performance improvements on the NQ and HotpotQA datasets. This indicates that LLM-QE provides essential matching signals for dense retrievers, particularly in retrieval scenarios where high-quality training signals are scarce.


\subsection{Case Study}\label{app:case_study}
\begin{figure}[htb]
\small
\begin{tcolorbox}[left=3pt,right=3pt,top=3pt,bottom=3pt,title=\textbf{Conversation History:}]
[human]: Craft an intriguing opening paragraph for a fictional short story. The story should involve a character who wakes up one morning to find that they can time travel.

...(Human-Bot Dialogue Turns)... \textcolor{blue}{(Topic: Time-Travel Fiction)}

[human]: Please describe the concept of machine learning. Could you elaborate on the differences between supervised, unsupervised, and reinforcement learning? Provide real-world examples of each.

...(Human-Bot Dialogue Turns)... \textcolor{blue}{(Topic: Machine learning Concepts and Types)}


[human]: Discuss antitrust laws and their impact on market competition. Compare the antitrust laws in US and China along with some case studies

...(Human-Bot Dialogue Turns)... \textcolor{blue}{(Topic: Antitrust Laws and Market Competition)}

[human]: The vertices of a triangle are at points (0, 0), (-1, 1), and (3, 3). What is the area of the triangle?

...(Human-Bot Dialogue Turns)... \textcolor{blue}{(Topic: Area and Circumcircle of Triangle)}

...(Human-Bot Dialogue Interaction on Other Topics)... 

\end{tcolorbox}
\begin{tcolorbox}[left=3pt,right=3pt,top=3pt,bottom=3pt,title=\textbf{Question:}]
What title reflects the thematic essence of the Doctor's experience with time travel as described in the story?
\end{tcolorbox}
\begin{tcolorbox}[left=3pt,right=3pt,top=3pt,bottom=3pt,title=\textbf{Ground-Truth Answer:}]
Awakening to Eternity: The Doctor's New Dawn
\end{tcolorbox}
\begin{tcolorbox}[left=3pt,right=3pt,top=3pt,bottom=3pt,title=\textbf{Retrieved Memory (Turn-Level):}]
[human]: Describe the process and write out the balanced chemical equation for the reaction that occurs when solid calcium carbonate reacts with hydrochloric acid to form aqueous calcium chloride, carbon dioxide, and water. What type of reaction is this, and what observations might indicate that the reaction is taking place?

[bot]: When solid calcium carbonate ($\text{CaCO}_{3}$) reacts with hydrochloric acid (HCl), the products formed are aqueous calcium chloride ($\text{CaCl}_{2}$), carbon dioxide ($\text{CO}_{2}$), and water ($\text{H}_{2}\text{O}$). The balanced chemical equation for this reaction is: ... \textcolor{blue}{(Topic: Calcium Carbonate and Hydrochloric Acid Reaction)}

\textcolor{brown}{
[human]: Write a title for your story.}

\textcolor{brown}{
[bot]: "Awakening to Eternity: The Doctor's New Dawn".
} \textcolor{blue}{(Topic: Time-Travel Fiction)}

[human]: Analyze the following customer reviews from different sources for three different smartphones - the latest iPhone, Samsung Galaxy, and Google Pixel - and provide an overall rating for each phone on a scale of 1 to 10. Consider the following complex and contradictory reviews: - TechRadar's review of the latest iPhone: The new iPhone is a stunning triumph of engineering that sets a new bar for smartphone performance and camera quality. However, the incremental design and high price mean it lacks the 'wow' factor of previous iPhones. Still, its power and intelligence are unrivaled. - CNET's review of the latest Samsung Galaxy: ... Return the answer as a JSON object with the overall ratings for each phone out of 10, to one decimal place.

[bot]: ```json { "iPhone": 8.5, "Samsung Galaxy": 8.0, "Google Pixel": 7.5 } ```

\textcolor{blue}{(Topic: Smartphone Review Ratings Analysis)}
\end{tcolorbox}
\begin{tcolorbox}[left=3pt,right=3pt,top=3pt,bottom=3pt,title=\textbf{Answer (Turn-Level Retrieval):}]
As an AI language model, I do not have access to the story you mentioned earlier. Please provide me with the story or the necessary details so that I can assist you better.

\textbf{\textcolor{red}{(The next part of this figure~\ref{fig: case_study_segment_vs_turn} will be continued on the next page.)}}
\end{tcolorbox}
\end{figure}

\begin{figure}[htb]
% \ContinuedFloat
\small
\begin{tcolorbox}[left=3pt,right=3pt,top=3pt,bottom=3pt,title=\textbf{Retrieved Memory (Ours):}]
[human]: Craft an intriguing opening paragraph for a fictional short story. The story should involve a character who wakes up one morning to find that they can time travel.

[bot]: In the hushed pre-dawn hours, when the world still slumbers under the cloak of night, Alex found themselves caught in the fragile boundary between sleep and wakefulness. ...

[human]: Summarize the story with three bullet points using only nouns and adjectives, without verbs.

[bot]: - Alex, pre-dawn awakening, time-travel discovery - Disorientation, vertigo, reality's malleability - Sunlight, new day, astonishing truth

[human]: Is there any famous fictional character that can time travel? If so, can you rewrite your story with him or her?

[bot]: - \textcolor{brown}{Doctor Who, a renowned time-traveling character from British television, is a perfect fit for this scenario.} In the silent embrace of early morning, where shadows merge with the fading night, the Doctor found themselves in the thin veil separating dreams from consciousness. It was here, amidst the soft whispers of time, that an unusual sensation enveloped them—a sense of time's fabric slipping through their fingers like grains of sand. As dawn's light broke through the darkness, painting the room in hues of gold and amber, the Doctor's eyes opened to what appeared to be an ordinary day. \textcolor{brown}{However, upon standing, a wave of dizziness overwhelmed them, the surroundings blurred, and in that moment of confusion, the Doctor realized a profound truth: they had not only awakened to a new dawn but had also stumbled upon the incredible ability to navigate the currents of time at will.}

\textcolor{brown}{[human]: Write a title for your story.}

\textcolor{brown}{[bot]: "Awakening to Eternity: The Doctor's New Dawn"} \textcolor{blue}{(Topic: Time-Travel Fiction)}
\end{tcolorbox}
\begin{tcolorbox}[left=3pt,right=3pt,top=3pt,bottom=3pt,title=\textbf{Answer (Ours):}]
The title "Awakening to Eternity: The Doctor's New Dawn" reflects the thematic essence of the Doctor's experience with time travel as described in the story. It captures the idea of a new beginning, a sense of eternity, and the awakening of a new power that will change the course of the Doctor's life forever.
\end{tcolorbox}
\caption{Case study comparing our \sysname\ with the \textit{turn-level} memory baseline. When constructing the memory bank at the \textit{turn-level}, the agent fails to provide a satisfactory response due to fragmentation: the retrieved dialogue turns can not capture a complete human-agent interaction related to the query, missing critical information highlighted in \textcolor{brown}{brown}. The dialogue topic is emphasized in \textcolor{blue}{blue}. Some context is omitted (replaced by ...) due to length constraints.}
\label{fig: case_study_segment_vs_turn}
\end{figure}

\newpage

\begin{figure}[htbp]
\small
\begin{tcolorbox}[left=3pt,right=3pt,top=3pt,bottom=3pt,title=\textbf{Conversation History:}]
[human]: Explain what's base rate fallacy and list five specific examples of ...

... (Human-Bot Dialogue Turns) ... \textcolor{blue}{(Topic: Base Rate Fallacy In Politics)}


[human]: How many integers are in the solution of the inequality $|x + 5| < 10$?

... (Human-Bot Dialogue Turns) ... \textcolor{blue}{(Topic: Number of Integers in Inequalities)}

[human]: How have the Alps and Rhine River influenced settlement and agriculture in Western Europe?

... (Human-Bot Dialogue Turns) ... \textcolor{blue}{(Topic: Alps and Rhine River Impact)}

[human]: In a survey conducted at a local high school, preferences for a new school color were measured: 58\% of students liked the color blue, 45\% preferred green, and 22\% liked both colors. If we randomly pick a student from the school, what's the probability that they would like neither blue nor green?

... (Human-Bot Dialogue Turns) ... \textcolor{blue}{(Topic: Probability of Liking Colors in Survey)}

[human]: Imagine you are participating in a race with a group of people. If you have just overtaken the second person, what's your current position? Where is the person you just overtook?

... (Human-Bot Dialogue Turns) ... \textcolor{blue}{(Topic: Overtaking Positions in Race)}
\end{tcolorbox}
\begin{tcolorbox}[left=3pt,right=3pt,top=3pt,bottom=3pt,title=\textbf{Question:}]
[human]: In the inequality question, how many positive integers are in the solution of the second inequality?
\end{tcolorbox}
\begin{tcolorbox}[left=3pt,right=3pt,top=3pt,bottom=3pt,title=\textbf{Ground-Truth Answer:}]
In the second inequality, $(|x + 10| < 5)$, the solution range is $(-15 < x < -5)$. Within this range, there are no positive integers, as all values of $(x)$ that satisfy the inequality are negative.
\end{tcolorbox}
\begin{tcolorbox}[left=3pt,right=3pt,top=3pt,bottom=3pt,title=\textbf{Retrieved Memory (Session-Level):}]
[human]: Explain what's base rate fallacy and list five specific examples of ...

... (Human-Bot Dialogue Turns) ... \textcolor{blue}{(Topic: Base Rate Fallacy In Politics)}

[human]: How many integers are in the solution of the inequality $|x + 5| < 10$?

[bot]: To solve the inequality $|x + 5| < 10$, ... 

\textcolor{brown}{
[human]: What about $|x + 10| < 5$?
}

\textcolor{brown}{
[bot]: To solve the inequality $|x + 10| < 5$, we ...} \textcolor{gray}{(``overwhelmed'' by other irrelevant, redundant content)}

... (Human-Bot Dialogue Turns) ...  \textcolor{blue}{(Topic: Number of Integers in Inequalities)}

% \textcolor{brown}{
% [human]: What about $|x + 10| < 5$
% }

% \textcolor{brown}{
% [bot]: To solve the inequality $|x + 10| < 5$, ...
% } \textcolor{blue}{(Topic: Number of Integers in Inequalities)}

[human]: How have the Alps and Rhine River influenced settlement and agriculture in Western Europe?

... (Human-Bot Dialogue Turns) ... \textcolor{blue}{(Topic: Alps and Rhine River Impact)}

[human]: In a survey conducted at a local high school, ...

... (Human-Bot Dialogue Turns) ... \textcolor{blue}{(Topic: Probability of Liking Colors in Survey)}
\end{tcolorbox}
\begin{tcolorbox}[left=3pt,right=3pt,top=3pt,bottom=3pt,title=\textbf{Answer (Session-Level Retrieval):}]
I apologize for the confusion earlier. To answer your question, the second inequality was not provided in our previous conversation. If you could provide the inequality, I would be happy to help you find the solution and the number of positive integers in the solution set.

\textbf{\textcolor{red}{(The next part of this figure~\ref{fig: case_study_segment_vs_session} will be continued on the next page.)}}
\end{tcolorbox}
\end{figure}
\begin{figure}[htbp]
% \ContinuedFloat
\small
\begin{tcolorbox}[left=3pt,right=3pt,top=3pt,bottom=3pt,title=\textbf{Retrieved Memory (Ours):}]
[human]: How many integers are in the solution of the inequality $|x + 5| < 10$?

[bot]: To solve the inequality $|x + 5| < 10$, we break it into two cases based on the definition of absolute value: 
1. $x + 5 < 10$ 
2. $-(x + 5) < 10$, which simplifies to $x + 5 > -10$ 
Solving each case: 
1. $x < 5$ 
2. $x > -15$ 
Combining these, we get $-15 < x < 5$. This inequality means that $x$ can be any number greater than $-15$ and less than $5$. To find how many integers satisfy this condition, we list them: $-14, -13, -12, -11, -10, -9, -8, -7, -6, -5, -4, -3, -2, -1, 0, 1, 2, 3, 4$. There are 19 integers that satisfy the inequality $|x + 5| < 10$.

\textcolor{brown}{
[human]: What about $|x + 10| < 5$?
}

\textcolor{brown}{
[bot]: To solve the inequality $|x + 10| < 5$, we again consider two cases: 
1. $x + 10 < 5$ 
2. $-(x + 10) < 5$, which simplifies to $x + 10 > -5$ 
Solving each case: 
1. $x < -5$ 
2. $x > -15$ 
Combining these, we get $-15 < x < -5$. This inequality means that $x$ can be any number greater than $-15$ and less than $-5$. To find how many integers satisfy this condition, we list them: $-14, -13, -12, -11, -10, -9, -8, -7, -6$. There are 9 integers that satisfy the inequality $|x + 10| < 5$.
}

[human]: What about $|x + 10| < 10$?

[bot]: To solve the inequality $|x + 10| < 10$, we consider two cases: 
1. $x + 10 < 10$ 
2. $-(x + 10) < 10$, which simplifies to $x + 10 > -10$ 
Solving each case: 
1. $x < 0$ 
2. $x > -20$ 
Combining these, we get $-20 < x < 0$. This inequality means that $x$ can be any number greater than $-20$ and less than $0$. To find how many integers satisfy this condition, we list them: $-19, -18, -17, -16, -15, -14, -13, -12, -11, -10, -9, -8, -7, -6, -5, -4, -3, -2, -1$. There are 19 integers that satisfy the inequality $|x + 10| < 10$. \textcolor{blue}{(Topic: Number of Integers in Inequalities)}
\end{tcolorbox}
\begin{tcolorbox}[left=3pt,right=3pt,top=3pt,bottom=3pt,title=\textbf{Answer (Ours):}]
To find the number of positive integers that satisfy the inequality $(|x + 10| < 5)$, we need to look for positive integers in the solution set $(-15 < x < -5)$. Since there are no positive integers in this interval, the answer is 0.

\end{tcolorbox}
\caption{Case study comparing our \sysname\ with the \textit{session-level} memory baseline. When constructing the memory bank at the \textit{session-level}, the agent fails to provide a satisfactory response due to the inclusion of excessive irrelevant information within a session. This distracts the generator from focusing on the truly useful information highlighted in \textcolor{brown}{brown}. The dialogue topic is emphasized in \textcolor{blue}{blue}. Some context is omitted (replaced by ...) due to length constraints.}
\label{fig: case_study_segment_vs_session}
\end{figure}



\begin{figure}[htb]
\small
\begin{tcolorbox}[left=3pt,right=3pt,top=3pt,bottom=3pt,title=\textbf{Conversation History:}]
[human]: Photosynthesis is a vital process for life on Earth. Could you outline the two main stages of photosynthesis, including where they take place within the chloroplast, and the primary inputs and outputs for each stage? ... (Human-Bot Dialogue Turns)... \textcolor{blue}{(Topic: Photosynthetic Energy Production)}

[human]: Please assume the role of an English translator, tasked with correcting and enhancing spelling and language. Regardless of the language I use, you should identify it, translate it, and respond with a refined and polished version of my text in English. 

... (Human-Bot Dialogue Turns)...  \textcolor{blue}{(Topic: Language Translation and Enhancement)}

[human]: Suggest five award-winning documentary films with brief background descriptions for aspiring filmmakers to study.

\textcolor{brown}{[bot]: ...
5. \"An Inconvenient Truth\" (2006) - Directed by Davis Guggenheim and featuring former United States Vice President Al Gore, this documentary aims to educate the public about global warming. It won two Academy Awards, including Best Documentary Feature. The film is notable for its straightforward yet impactful presentation of scientific data, making complex information accessible and engaging, a valuable lesson for filmmakers looking to tackle environmental or scientific subjects.}

... (Human-Bot Dialogue Turns)... 
\textcolor{blue}{(Topic: Documentary Films Recommendation)}

[human]: Given the following records of stock prices, extract the highest and lowest closing prices for each month in the year 2022. Return the results as a CSV string, with one line allocated for each month. Date,Open,High,Low,Close,Volume ... ... (Human-Bot Dialogue Turns)...  \textcolor{blue}{(Topic: Stock Prices Analysis)}

[human]: The city of Vega intends to build a bridge that will span the Vegona River, covering a distance of 1.8 kilometers. The proposed location falls within a seismically active area that has experienced several high-magnitude earthquakes. Given these circumstances, what would be the best approach to constructing the bridge? ... (Human-Bot Dialogue Turns)...  \textcolor{blue}{(Topic: Seismic Bridge Construction)}
\end{tcolorbox}
\begin{tcolorbox}[left=3pt,right=3pt,top=3pt,bottom=3pt,title=\textbf{Question:}]
Which documentary won two Academy Awards and what lesson does it offer to filmmakers?
\end{tcolorbox}
\begin{tcolorbox}[left=3pt,right=3pt,top=3pt,bottom=3pt,title=\textbf{Ground-Truth Answer:}]
``An Inconvenient Truth" (2006) won two Academy Awards. It offers the lesson of making complex information accessible and engaging, a valuable lesson for filmmakers looking to tackle environmental or scientific subjects.
\end{tcolorbox}
\begin{tcolorbox}[left=3pt,right=3pt,top=3pt,bottom=3pt,title=\textbf{Memory generated by RecurSum~\citep{wang2023recursively}:}]
The dialogue provided detailed information on various topics, including the stages of photosynthesis, energy production estimates through photosynthesis for different organisms, translations and enhancements of phrases in different languages, recommendations for documentary films, a pitch for a film about overcoming adversity, analysis of stock prices, and considerations for constructing a bridge in a seismically active area. Here's a summary of the key points: ...

\textbf{Documentary Film Recommendations}: Suggested five award-winning documentary films for aspiring filmmakers to study, with brief background descriptions for each. The response included:

-  "March of the Penguins" (2005) - Emperor penguins' journey in Antarctica; noted for its cinematography and portrayal of the penguins' life cycle.

-  "Citizenfour" (2014) - Edward Snowden and NSA spying scandal; known for its real-time exploration of Snowden's revelations.

-  "20 Feet from Stardom" (2013) - Stories of backup singers behind musical legends; celebrates unsung heroes in the music industry.

-  "Man on Wire" (2008) - Philippe Petit's 1974 high-wire walk between the Twin Towers; combines personal footage, interviews, and re-enactments.

-  \textcolor{brown}{"An Inconvenient Truth" (2006) - Al Gore's documentary on global warming; known for its presentation of scientific data.}
\end{tcolorbox}
\begin{tcolorbox}[left=3pt,right=3pt,top=3pt,bottom=3pt,title=\textbf{Answer (RSum):}]
I'm sorry, but based on the updated context you provided, I don't see any previous discussion related to the question you asked. However, to answer your question, the documentary that won two Academy Awards is "Free Solo," which won for Best Documentary Feature and Best Achievement in Film Editing in 2019. 
% The lesson it offers to filmmakers is the importance of taking risks and pushing boundaries in order to create something truly unique and impactful.
....
The film follows climber Alex Honnold as he attempts to free solo climb El Capitan in Yosemite National Park, and the filmmakers had to take risks themselves in order to capture the footage and tell the story.

\textbf{\textcolor{red}{(The next part of this figure~\ref{fig: case_study_segment_vs_rsum} will be continued on the next page.)}}
\end{tcolorbox}
\end{figure}
\begin{figure}[htbp]
% \ContinuedFloat
\small
\begin{tcolorbox}[left=3pt,right=3pt,top=3pt,bottom=3pt,title=\textbf{Retrieved Memory (Ours):}]
[human]: Suggest five award-winning documentary films with brief background descriptions for aspiring filmmakers to study.

[bot]: 1. **"March of the Penguins" (2005)** - Directed by Luc Jacquet, this French feature-length nature documentary delves into the annual journey of Emperor penguins across the Antarctic. It won the Academy Award for Best Documentary Feature. The film is celebrated for its stunning cinematography and its detailed, heartfelt portrayal of the penguins' life cycle, making it a must-study for filmmakers interested in wildlife documentaries. 2. **"Citizenfour" (2014)** ... 5. \textcolor{brown}{**"An Inconvenient Truth" (2006)** - Directed by Davis Guggenheim and featuring former United States Vice President Al Gore, this documentary aims to educate the public about global warming. It won two Academy Awards, including Best Documentary Feature.} The film is notable for its straightforward yet impactful presentation of scientific data, making complex information accessible and engaging, \textcolor{brown}{a valuable lesson for filmmakers looking to tackle environmental or scientific subjects.}

[human]: With the spirit in the first film, craft a succinct and persuasive pitch for a film about overcoming adversity.

[bot]: Drawing inspiration from the resilience and majestic journey of Emperor penguins in "March of the Penguins," our documentary, "Unyielding Spirits: The Human Odyssey," will explore the profound stories of individuals around the globe who have faced and overcome staggering adversities. Through breathtaking cinematography and intimate storytelling, we ...
\textcolor{blue}{(Topic: Documentary Films Recommendation)}
\end{tcolorbox}
\begin{tcolorbox}[left=3pt,right=3pt,top=3pt,bottom=3pt,title=\textbf{Answer (Ours):}]
``An Inconvenient Truth" won two Academy Awards, including Best Documentary Feature. The film offers a valuable lesson to filmmakers looking to tackle environmental or scientific subjects, as it presents complex information in a straightforward yet impactful way, making it accessible and engaging to a wider audience.


\end{tcolorbox}
\caption{Case study comparing our \sysname\ with RecurSum~\citep{wang2023recursively}. When using RecurSum, the agent fails to provide a satisfactory answer due to information loss during the summarization process. Summarization omits essential details needed to answer the user's new request, as highlighted in \textcolor{brown}{brown}. The dialogue topic is emphasized in \textcolor{blue}{blue}. Some context is omitted (replaced by ...) due to length constraints.}
\label{fig: case_study_segment_vs_rsum}
\end{figure}




\begin{figure}[htb]
\small
\begin{tcolorbox}[left=3pt,right=3pt,top=3pt,bottom=3pt,title=\textbf{Conversation History:}]
[human]: Photosynthesis is a vital process for life on Earth. Could you outline the two main stages of photosynthesis, including where they take place within the chloroplast, and the primary inputs and outputs for each stage? ... (Human-Bot Dialogue Turns)... \textcolor{blue}{(Topic: Photosynthetic Energy Production)}

[human]: Please assume the role of an English translator, tasked with correcting and enhancing spelling and language. Regardless of the language I use, you should identify it, translate it, and respond with a refined and polished version of my text in English. 

... (Human-Bot Dialogue Turns)...  \textcolor{blue}{(Topic: Language Translation and Enhancement)}

[human]: Suggest five award-winning documentary films with brief background descriptions for aspiring filmmakers to study.

\textcolor{brown}{[bot]: ...
5. \"An Inconvenient Truth\" (2006) - Directed by Davis Guggenheim and featuring former United States Vice President Al Gore, this documentary aims to educate the public about global warming. It won two Academy Awards, including Best Documentary Feature. The film is notable for its straightforward yet impactful presentation of scientific data, making complex information accessible and engaging, a valuable lesson for filmmakers looking to tackle environmental or scientific subjects.}

... (Human-Bot Dialogue Turns)... 
\textcolor{blue}{(Topic: Documentary Films Recommendation)}

[human]: Given the following records of stock prices, extract the highest and lowest closing prices for each month in the year 2022. Return the results as a CSV string, with one line allocated for each month. Date,Open,High,Low,Close,Volume ... ... (Human-Bot Dialogue Turns)...  \textcolor{blue}{(Topic: Stock Prices Analysis)}

[human]: The city of Vega intends to build a bridge that will span the Vegona River, covering a distance of 1.8 kilometers. The proposed location falls within a seismically active area that has experienced several high-magnitude earthquakes. Given these circumstances, what would be the best approach to constructing the bridge? ... (Human-Bot Dialogue Turns)...  \textcolor{blue}{(Topic: Seismic Bridge Construction)}
\end{tcolorbox}
\begin{tcolorbox}[left=3pt,right=3pt,top=3pt,bottom=3pt,title=\textbf{Question:}]
Which documentary won two Academy Awards and what lesson does it offer to filmmakers?
\end{tcolorbox}
\begin{tcolorbox}[left=3pt,right=3pt,top=3pt,bottom=3pt,title=\textbf{Ground-Truth Answer:}]
"An Inconvenient Truth" (2006) won two Academy Awards. It offers the lesson of making complex information accessible and engaging, a valuable lesson for filmmakers looking to tackle environmental or scientific subjects.
\end{tcolorbox}
\begin{tcolorbox}[left=3pt,right=3pt,top=3pt,bottom=3pt,title=\textbf{Memory generated by ConditionMem~\citep{yuan2023evolving}:}]
- \textcolor{brown}{The user is seeking recommendations for award-winning documentary films, along with brief background descriptions, for educational purposes.} This request likely stems from a desire to learn from successful examples in the documentary filmmaking field, possibly to improve their own filmmaking skills or to gain inspiration for their projects. 

- \textcolor{brown}{User is interested in award-winning documentary films for educational purposes, aiming to study their backgrounds and filmmaking techniques.}

- The user's request for crafting a succinct and persuasive pitch for a film about overcoming adversity is inspired by the spirit of "March of the Penguins," a documentary that beautifully captures the Emperor penguins' annual journey across the Antarctic, highlighting themes of endurance and survival against the odds. The user seeks to emulate the emotional depth and cinematic quality that made "March of the Penguins" compelling, aiming to create a pitch that resonates with similar themes of resilience and triumph over challenges. 

- User is interested in creating content that focuses on themes of overcoming adversity, inspired by the emotional depth and cinematic excellence of documentaries like "March of the Penguins." They value storytelling that captures the essence of endurance and survival against the odds.
% : The user's request to write a budget plan for the video production directly follows the task of creating a video script aimed at training new employees of a car wash business in Japan, focusing on business etiquette norms.
% The user is likely seeking to understand the financial requirements to bring the previously crafted script to life, indicating a move from the planning and content development phase to the practical aspects of production. 

... (Other Irrelevant Memory Records) ...
\end{tcolorbox}
\begin{tcolorbox}[left=3pt,right=3pt,top=3pt,bottom=3pt,title=\textbf{Answer (Conditional Memory):}]
The documentary that won two Academy Awards is "Free Solo," which follows climber Alex Honnold as he attempts to free solo climb El Capitan in Yosemite National Park. The film offers a lesson to filmmakers on the importance of capturing authentic and compelling human stories, as well as the technical challenges of filming in extreme environments.

\textbf{\textcolor{red}{(The next part of this figure~\ref{fig: case_study_segment_vs_condmem} will be continued on the next page.)}}
\end{tcolorbox}
\end{figure}
\begin{figure}[htbp]
% \ContinuedFloat
\small
\begin{tcolorbox}[left=3pt,right=3pt,top=3pt,bottom=3pt,title=\textbf{Retrieved Memory (Ours):}]
[human]: Suggest five award-winning documentary films with brief background descriptions for aspiring filmmakers to study.

[bot]: 1. **"March of the Penguins" (2005)** - Directed by Luc Jacquet, this French feature-length nature documentary delves into the annual journey of Emperor penguins across the Antarctic. It won the Academy Award for Best Documentary Feature. The film is celebrated for its stunning cinematography and its detailed, heartfelt portrayal of the penguins' life cycle, making it a must-study for filmmakers interested in wildlife documentaries. 2. **"Citizenfour" (2014)** ... 5. \textcolor{brown}{**"An Inconvenient Truth" (2006)** - Directed by Davis Guggenheim and featuring former United States Vice President Al Gore, this documentary aims to educate the public about global warming. It won two Academy Awards, including Best Documentary Feature.} The film is notable for its straightforward yet impactful presentation of scientific data, making complex information accessible and engaging, a valuable lesson for filmmakers looking to tackle environmental or scientific subjects.

[human]: With the spirit in the first film, craft a succinct and persuasive pitch for a film about overcoming adversity.

[bot]: Drawing inspiration from the resilience and majestic journey of Emperor penguins in "March of the Penguins," our documentary, "Unyielding Spirits: The Human Odyssey," will explore the profound stories of individuals around the globe who have faced and overcome staggering adversities. Through breathtaking cinematography and intimate storytelling, we ...
\textcolor{blue}{(Topic: Documentary Films Recommendation)}
\end{tcolorbox}
\begin{tcolorbox}[left=3pt,right=3pt,top=3pt,bottom=3pt,title=\textbf{Answer (Ours):}]
``An Inconvenient Truth" won two Academy Awards, including Best Documentary Feature. The film offers a valuable lesson to filmmakers looking to tackle environmental or scientific subjects, as it presents complex information in a straightforward yet impactful way, making it accessible and engaging to a wider audience.
\end{tcolorbox}
\caption{Case study comparing our \sysname\ with ConditionMem~\citep{yuan2023evolving}. When using ConditionMem, the agent fails to provide a satisfactory answer due to (1) information loss during the summarization process and (2) the incorrect discarding of turns that are actually useful, as highlighted in \textcolor{brown}{brown}. The dialogue topic is emphasized in \textcolor{blue}{blue}. Some context is omitted (replaced by ...) due to length constraints.}
\label{fig: case_study_segment_vs_condmem}
\end{figure}


To further demonstrate the effectiveness of LLM-QE, we conduct a case study by randomly sampling a query from the evaluation dataset. We then compare retrieval performance using the raw queries, expanded queries by vanilla LLM, and expanded queries by LLM-QE.

As shown in Table~\ref{tab:case_study}, query expansion significantly improves retrieval performance compared to using the raw query. Both vanilla LLM and LLM-QE generate expansions that include key phrases, such as ``temperature'', ``humidity'', and ``coronavirus'', which provide crucial signals for document matching. However, vanilla LLM produces inconsistent results, including conflicting claims about temperature ranges and virus survival conditions. In contrast, LLM-QE generates expansions that are more semantically aligned with the golden passage, such as ``the virus may thrive in cooler and more humid environments, which can facilitate its transmission''. This further demonstrates the effectiveness of LLM-QE in improving query expansion by aligning with the ranking preferences of both LLMs and retrievers.



%\newpage
%\input{10_faq}
%\newpage
%\section{Appendix}

\begin{figure}
    \centering
    \includegraphics[width=0.9\linewidth]{figures/02_traning_taxonomies.png}
    \caption{Training Taxonomies}
    \label{fig:traning_taxonomies}
\end{figure}

\begin{figure}
    \centering
    \includegraphics[width=1.0\linewidth]{figures/03_preference_tuning_taxonomies.png}
    \caption{Preference Tuning Taxonomies}
    \label{fig:preference_tuning_taxonogy}
\end{figure}

% \begin{table*}[ht]
%     \centering
%     \begin{tabular}{|>{\raggedright\arraybackslash}m{4cm}|>{\raggedright\arraybackslash}m{3cm}|>{\raggedright\arraybackslash}m{7cm}|}
\hline
\textbf{Name} & \textbf{Notation} & \textbf{Description} \\
\hline
Input Sequence & $x$ & Input sequence that is passed to the model. \\
Output Sequence & $y$ & Expected label or output of the model. \\
\hline
Dispreferred Response & $y_l$ & Negative samples for reward model training. \\
Preferred Response & $y_w$ & Positive samples for reward model training. \\
\hline
Optimal Policy Model & $\pi^*$ & Optimal policy model. \\
Policy Model & $\pi_\theta$ & Generative model that takes the input prompt and returns a sequence of output or probability distribution. \\
Reference Policy Model & $\pi_{\text{ref}}$ & Generative model that is used as a reference to ensure the policy model is not deviated significantly. \\
\hline
Preference Dataset & $\mathcal{D}_{\text{pref}}$ & Dataset with a set of preferred and dispreferred responses to train a reward model. \\
SFT Dataset & $\mathcal{D}_{\text{sft}}$ & Dataset with a set of input and label for supervised fine-tuning. \\
\hline
Loss Function & $\mathcal{L}$ & Loss function. \\
Regularization Hyper-parameters & $\alpha, \beta_{\text{reg}}$ & Regularization Hyper-parameters for preference tuning. \\
Reward & $r$ & Reward score. \\
Target Reward Margin & $\gamma$ & The margin separating the winning and losing responses. \\
Variance & $\beta_i$ & Variance (or noise schedule) used in diffusion models. \\
\hline
\end{tabular}
%     \caption{Caption}
%     \label{tab:notation}
% \end{table*}

\end{document}