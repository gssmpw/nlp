% This must be in the first 5 lines to tell arXiv to use pdfLaTeX, which is strongly recommended.
\pdfoutput=1
% In particular, the hyperref package requires pdfLaTeX in order to break URLs across lines.

\RequirePackage[svgnames]{xcolor}

\documentclass[11pt]{article}

% Change "review" to "final" to generate the final (sometimes called camera-ready) version.
% Change to "preprint" to generate a non-anonymous version with page numbers.
\usepackage[final]{acl}

% Standard package includes
\usepackage{times}
\usepackage{latexsym}

\usepackage[draft,textsize=footnotesize,textwidth=15mm]{todonotes}
%\usepackage[usenames,dvipsnames]{color}


% For proper rendering and hyphenation of words containing Latin characters (including in bib files)
%\usepackage[T1]{fontenc}
% For Vietnamese characters
% \usepackage[T5]{fontenc}
% See https://www.latex-project.org/help/documentation/encguide.pdf for other character sets

% This assumes your files are encoded as UTF8
\usepackage[utf8]{inputenc}
\usepackage{inconsolata}
\usepackage{algorithm}
\usepackage{algpseudocode}
\usepackage{amsmath,amssymb}
\usepackage{soul}
% \usepackage[margin=2cm]{geometry}
\usepackage{tikz}

% Style definition
\tikzset{rndblock/.style={rounded corners,rectangle,draw,scale=0.8,outer sep=0pt}}

% Command Definition
% 1 optional to customize the aspect, 2 mandatory: text to be framed
\newcommand{\tframed}[2][]{\tikz[baseline=(h.base)]\node[rndblock,#1] (h) {#2};}

\usetikzlibrary{shapes.geometric}
\usepackage{framed}
\usepackage{enumitem}
\newlist{RQ}{enumerate}{1}
\setlist[RQ]{label=\textbf{RQ\,\arabic*},ref={RQ\,\arabic*}}
\usepackage{comment}
\usepackage{natbib}
\usepackage{multibib}
\makeatletter
\usepackage{booktabs}
\usepackage[inkscapeformat=png]{svg}
\usepackage{graphicx}
\usepackage{caption}
\usepackage{subcaption}
\usepackage{tabularx}
\usepackage{soul}
\usepackage{float}
\usepackage{enumitem}
\usepackage{pifont}
\usepackage{arydshln}
\usepackage{lipsum}
%%%%%%%%%%%%%%%%%%%%%%%%%%%%%%%%%%%%%%%%%%%%%%%%%%%%%%%%%%%%
\usepackage[T1]{fontenc}
\usepackage{pifont}
\usepackage{amsmath}
\usepackage{soul}
\usepackage[utf8]{inputenc}
\usepackage{inconsolata}
\usepackage{tikz}
\usepackage{arydshln}
\usepackage{lipsum}
\usepackage[normalem]{ulem}
\usepackage{wrapfig,graphicx,lipsum}% http://ctan.org/pkg/{wrapfig,graphicx,lipsum}
% \usepackage{graphicx}
%\usepackage[table,xcdraw]{xcolor}
\usepackage{colortbl} 
%\usepackage[dvipsnames]{xcolor}

%%%%%%%%%%%%%%%%%%%%%%%%%%%%%%%%%%%%
\usepackage{uncial}

\usepackage{soul}
\usepackage{graphicx}
\usepackage{booktabs}
\usepackage{multirow}
\usepackage{colortbl}
%\usepackage{xcolor}
%\usepackage[table]{xcolor}
\usepackage{afterpage}  % load the afterpage package
\usepackage{tabularx} % Add to your preamble
\usepackage{multicol} % For structured multiline text
\usepackage{array}    % Enhanced column types
\usepackage{rotating}
\usepackage{tabularx}
\usepackage{booktabs}
\usepackage{amsmath}
\usepackage{amssymb}
\usepackage{array}



% \usepackage[draft,textsize=footnotesize,textwidth=15mm]{todonotes}
% \newcommand\ac[1]{\todo[author=AC,color=blue!40]{#1}}
% \newcommand\acil[1]{\todo[author=AC,color=blue!40,inline]{#1}}
% \newcommand\acb[1]{\textcolor{blue}{#1}}





\usepackage[most,many]{tcolorbox}

% \usepackage{tcolorbox}
% \tcbuselibrary{skins}

\newtcolorbox{defin}{colback=Teal!5!White,enhanced,title=DPO - Kernels (at-a-glance),
	attach boxed title to top left={xshift=0mm},boxrule=0pt,after skip=1cm,before skip=1cm,right skip=0cm,breakable,fonttitle=\bfseries,toprule=0pt,bottomrule=0pt,rightrule=0pt,leftrule=3pt,arc=0mm,skin=enhancedlast jigsaw,sharp corners,colframe=Teal!55!black,colbacktitle=Teal!55!black,boxed title style={
		frame code={ 
			\fill[Teal!25!black](frame.south west)--(frame.north west)--(frame.north east)--([xshift=3mm]frame.east)--(frame.south east)--cycle;
			\draw[line width=1mm,Teal!25!black]([xshift=2mm]frame.north east)--([xshift=5mm]frame.east)--([xshift=2mm]frame.south east);
			\draw[line width=1mm,Teal!25!black]([xshift=5mm]frame.north east)--([xshift=8mm]frame.east)--([xshift=5mm]frame.south east);
			\fill[Teal!25!black](frame.south west)--+(4mm,-2mm)--+(4mm,2mm)--cycle;
		}
	}
}

% \setlist{leftmargin=1mm}
\usetikzlibrary{shapes.geometric, arrows}
\usetikzlibrary{decorations.markings}

\usepackage{fancybox}

\usepackage{hyperref}
 \definecolor{darkblue}{rgb}{0, 0, 0.5}
  \hypersetup{colorlinks=true, citecolor=darkblue, linkcolor=darkblue, urlcolor=darkblue}

\definecolor{vgreen}{HTML}{60A917}
\definecolor{vred}{HTML}{CE3A29}

\usepackage{xstring}
\usepackage{longtable}
\usepackage{supertabular}



\usepackage{lipsum}
\usepackage{tikz}
\usetikzlibrary{trees,shapes}

\usepackage{epigraph}
\definecolor{hidden-draw}{RGB}{20,68,106}
\definecolor{hidden-pink}{RGB}{255,245,247}
\definecolor{paired-light-yellow}{HTML}{FFFF88}
\definecolor{paired-light-blue}{HTML}{CCE5FF}
\definecolor{paired-light-orange}{HTML}{FFCC99}
\definecolor{paired-dark-yellow}{HTML}{FFF2CC}
\definecolor{paired-light-pink}{HTML}{FFCCCC}
\definecolor{paired-cyan}{HTML}{D5E8D4}
\definecolor{paired-gray}{HTML}{eeeeee}
\definecolor{paired-green}{HTML}{cdeb8b}
\definecolor{paired-blue}{HTML}{dae8fc}
\definecolor{paired-dark-cyan}{HTML}{a2e6eb}
\definecolor{paired-dark-pink}{HTML}{e7b2d2}
\definecolor{paired-purple}{HTML}{9999ff}
\definecolor{paired-pink}{HTML}{cc99ff}
\definecolor{paired-orange}{HTML}{ffcc99}


\definecolor{a1}{RGB}{241,233,191}
\definecolor{a2}{RGB}{255,241,218}

\definecolor{a3}{RGB}{255,239,213}
\definecolor{a4}{RGB}{250,235,215}
\definecolor{a5}{RGB}{255,239,219}
\definecolor{a6}{RGB}{255,246,225}
\definecolor{a7}{RGB}{246,227,201}
\definecolor{a8}{RGB}{254,235,226}
\definecolor{a9}{RGB}{247,220,111}
\definecolor{a10}{RGB}{199,211,189}
\definecolor{a11}{RGB}{209,196,233}
\definecolor{a12}{RGB}{214,234,248}
\definecolor{a13}{RGB}{232,245,233}
\definecolor{a14}{RGB}{237,248,177}
\definecolor{a15}{RGB}{255,228,225}
\definecolor{a16}{RGB}{255,228,181}
\definecolor{a17}{RGB}{255,222,173}
\definecolor{a18}{RGB}{255,218,185}
\definecolor{a19}{RGB}{255,203,164}
\definecolor{a20}{RGB}{247,202,201}

\definecolor{a21}{RGB}{241,254,255}
\definecolor{a22}{RGB}{230,252,252}
\definecolor{a23}{RGB}{179,236,255}
\definecolor{a24}{RGB}{174,226,249}
\definecolor{a25}{RGB}{208,234,246}
\definecolor{a26}{RGB}{189,226,219}
\definecolor{a27}{RGB}{177,204,201}

\definecolor{a28}{RGB}{216,195,216}
\definecolor{a29}{RGB}{195,155,211}
\definecolor{a30}{RGB}{208,152,223}
\definecolor{a31}{RGB}{255,183,209}
\definecolor{a32}{RGB}{255,167,209}
\definecolor{a33}{RGB}{254,235,167}
\definecolor{a34}{RGB}{255,222,137}
\definecolor{a35}{RGB}{254,180,154}
\definecolor{a36}{RGB}{247,148,161}
\definecolor{a37}{RGB}{239,154,154}
\definecolor{a38}{RGB}{255,130,171}
\definecolor{a39}{RGB}{255,105,180}
\definecolor{a40}{RGB}{251,142,172}


\usepackage[edges]{forest}
\usepackage{lipsum}
\usepackage{tikz}
\usetikzlibrary{trees,shapes}
\usepackage{forest}
\usepackage{graphicx}
\usepackage{booktabs}
\usepackage{longtable}
\usepackage[export]{adjustbox} % Add this in the preamble
\usepackage{listings} % Add this in the preamble

% In the document
\lstset{
    basicstyle=\ttfamily\small,
    breaklines=true,
    frame=single,
    xleftmargin=0.05\columnwidth,
    xrightmargin=0.05\columnidth
}

\usepackage{amsmath} % For math formatting
%\usepackage{tcolorbox} % For colored boxes
%\tcbuselibrary{listingsutf8} % Optional for better customization

% Define a simple tcolorbox style
\tcbset{
  mybox/.style={
    colback=white,
    colframe=black,
    boxrule=0.5mm,
    width=\textwidth,
    left=2mm,
    right=2mm,
    bottom=2mm,
    top=2mm,
    sharp corners,
  }
}


\usepackage{tabularray}

\DefTblrTemplate{firsthead,middlehead,lasthead}{default}{
}
\DefTblrTemplate{firstfoot}{default}{
  \UseTblrTemplate{contfoot}{default}
  \UseTblrTemplate{caption}{default}
}
\DefTblrTemplate{middlefoot}{default}{
  \UseTblrTemplate{contfoot}{default}
  \UseTblrTemplate{capcont}{default}
}
\DefTblrTemplate{lastfoot}{default}{
  \UseTblrTemplate{note}{default}
  \UseTblrTemplate{remark}{default}
  \UseTblrTemplate{capcont}{default}
}


\newcolumntype{P}[1]{>{\centering\arraybackslash}p{#1}}
% Multi-line left-aligned text with manual line breaks.
% The base line is in centre.
\newcommand*{\mline}[1]{%
\begingroup
    \renewcommand*{\arraystretch}{1.1}%
   \begin{tabular}[c]{@{}>{\raggedright\arraybackslash}p{2cm}@{}}#1\end{tabular}%
  \endgroup
}

\usepackage{color}
\tcbuselibrary{skins}

\usepackage[export]{adjustbox} % for the valign option

\usepackage{setspace}
%\usepackage[capitalise]{cleveref}
\usepackage[capitalise,nameinlink]{cleveref}

%\crefname{chapter}{chap.}{chap.}
\crefname{section}{Sec.}{Sec.}

\usepackage{microtype}
\usepackage{hyperref}
\usepackage{graphicx}
\usepackage{comment}
\usepackage{amssymb}
\usepackage{algorithm}
\usepackage{amsmath}
\usepackage{algpseudocode}
\usepackage{colortbl}
\usepackage[export]{adjustbox} % for the valign option
\usepackage{enumitem}
\setlist{leftmargin=1mm}
\usepackage{pifont}
\usepackage{booktabs}
\usepackage{multirow}
\usepackage{subcaption}
\usepackage{resizegather}
\usepackage{breqn}
\usepackage[capitalise]{cleveref}
\usepackage{graphicx}
\usepackage{tikz}
\usetikzlibrary{shapes.geometric, arrows}
\usetikzlibrary{decorations.markings}
\usepackage{soul}
\usepackage{wrapfig,graphicx,lipsum}% http://ctan.org/pkg/{wrapfig,graphicx,lipsum}
\usepackage{extsizes}
\usepackage{cuted}
\usepackage{flushend}
\usepackage{float}
\usepackage{changepage,threeparttable}
\usepackage{setspace}
\usepackage{caption}
\usepackage{booktabs}
\usepackage{dblfloatfix} 
\usepackage{fixltx2e}
\usepackage[normalem]{ulem}

\usepackage{longtable}
\usepackage{amsmath}
\usepackage{array}
% \usepackage[margin=1in]{geometry} % Adjust page margins if necessary
\usepackage[most]{tcolorbox} 
%\usepackage{listings}
\setlength{\belowdisplayskip}{2pt} % Reduce spacing below equations
\setlength{\abovedisplayskip}{2pt} % Reduce spacing above equations
\setlength{\abovedisplayshortskip}{1pt} % Reduce spacing in inline equations
\setlength{\belowdisplayshortskip}{1pt}
%\documentclass[table,xcdraw]{article}
\usepackage{longtable} % For long tables that span multiple pages
\usepackage{graphicx} % For including images
\usepackage{array} % For defining custom column widths
\usepackage{caption} % For customizing captions
%usepackage{xcolor} % For text colors (optional, if needed)


% Optional: If using rotation or alignment enhancements
\usepackage{rotating} % For rotating text or images
\usepackage{booktabs} % For better table formatting (optional)
\usepackage{adjustbox} % To adjust image sizes dynamically







\DeclareRobustCommand{\hlpink}[1]{{\sethlcolor{pink}\hl{#1}}}
\DeclareRobustCommand{\hlgreen}[1]{{\sethlcolor{green}\hl{#1}}}

\usepackage{environ}

\newlength{\myl}
\expandafter\let\expandafter\origequation\csname equation*\endcsname
\expandafter\let\expandafter\endorigequation\csname endequation*\endcsname
\long\def\[#1\]{\begin{equation*}#1\end{equation*}}
\RenewEnviron{equation*}{
  \settowidth{\myl}{$\displaystyle\BODY$} % calculate width and save as \myl
  \origequation
    \ifdim\myl>\linewidth
      \resizebox{\linewidth}{!}{$\displaystyle\BODY$}% \myl > \linewidth
    \else
      \BODY % \myl <= \linewidth
    \fi
  \endorigequation
}


\makeatletter
\newcommand{\DrawLine}{%
  \begin{tikzpicture}
  \path[use as bounding box] (0,0) -- (\linewidth,0);
  \draw[color=blue!75!black,dashed,dash phase=.5pt]
        (0-\kvtcb@leftlower-\kvtcb@boxsep,0)--
        (\linewidth+\kvtcb@rightlower+\kvtcb@boxsep,0);
  \end{tikzpicture}%
  }
\makeatother

%resize/scale equations
\newcommand*{\Scale}[2][4]{\scalebox{#1}{$#2$}}%
\newcommand*{\Resize}[2]{\resizebox{#1}{!}{$#2$}}%


%--Vipula---
% \newcommand\vr[1]{\todo[author=VR,color=green!20]{#1}}
% \newcommand\vril[1]{\todo[author=VR,color=green!20,inline,caption={}]{#1}}


% %--Aman---
% \newcommand\ac[1]{\todo[author=AC,color=blue!40]{#1}}
% \newcommand\acil[1]{\todo[author=AC,color=blue!40,inline]{#1}}
% \newcommand\acb[1]{\textcolor{blue}{#1}}
% \newcommand\act[1]{\textcolor{blue}{#1}}


% %--Amitava---
% \newcommand\ad[1]{\todo[author=AD,color=purple!40]{#1}}
% \newcommand\adil[1]{\todo[author=AD,color=purple!40,inline]{#1}}
% \newcommand\adt[1]{\textcolor{purple}{#1}}


\usepackage{euscript}[mathcal]
% \usepackage[margin=1in]{geometry}

\newcommand*{\affaddr}[1]{#1}
\newcommand*{\affmark}[1][*]{\textsuperscript{#1}}
\newcommand*{\email}[1]{\texttt{#1}}
%\newcommand*{\affmark}[1][*]{\textsuperscript{#1}}
\author{
  Amitava Das\affmark[1], \bf Yaswanth Narsupalli\affmark[1], Gurpreet Singh\affmark[1],  \bf Vinija Jain\affmark[2]\thanks{\,\,\,Work done outside of role at Meta.}, 
  \bf Vasu Sharma\affmark[2]\footnotemark[1], \\\bf Suranjana Trivedy\affmark[1], 
  \bf Aman Chadha\affmark[3]\thanks{\,\,\,Work done outside of role at Amazon.}, Amit Sheth\affmark[1] \\
  \affaddr{\affmark[1]Artificial Intelligence Institute, University of South Carolina, USA,}\\
  \affaddr{\affmark[2]Meta AI, USA,}
  \affaddr{\affmark[3]Amazon AI, USA}
}



\title{\centering\includegraphics[width=\textwidth]{img/banner.pdf} }
%{\fontfamily{cmss}\selectfont DPO - Kernels}: A Semantically-Aware, Kernel-Enhanced, and Divergence-Rich Paradigm for Direct Preference Optimization}




% Author information can be set in various styles:
% For several authors from the same institution:
% \author{Author 1 \and ... \and Author n \\
%         Address line \\ ... \\ Address line}
% if the names do not fit well on one line use
%         Author 1 \\ {\bf Author 2} \\ ... \\ {\bf Author n} \\
% For authors from different institutions:
% \author{Author 1 \\ Address line \\  ... \\ Address line
%         \And  ... \And
%         Author n \\ Address line \\ ... \\ Address line}
% To start a seperate ``row'' of authors use \AND, as in
% \author{Author 1 \\ Address line \\  ... \\ Address line
%         \AND
%         Author 2 \\ Address line \\ ... \\ Address line \And
%         Author 3 \\ Address line \\ ... \\ Address line}

\forestset{
  my leaf/.style={
    fill=#1,
    draw=none
  }
}

\setlength{\topmargin}{-1.25cm}
% \setlength{\bottom}{-1.25cm}
\setlength{\textheight}{24.25cm}

\begin{document}
\maketitle
\begin{abstract}
Precise alignment in Text-to-Image (T2I) systems is crucial to ensure that generated visuals not only accurately encapsulate user intents but also conform to stringent ethical and aesthetic benchmarks. Incidents like the Google Gemini fiasco, where misaligned outputs triggered significant public backlash, underscore the critical need for robust alignment mechanisms. In contrast, Large Language Models (LLMs) have achieved notable success in alignment. Building on these advancements, researchers are eager to apply similar alignment techniques, such as Direct Preference Optimization (DPO), to T2I systems to enhance image generation fidelity and reliability.

% We introduce \textbf{YinYangAlign}, a comprehensive benchmark designed to evaluate the alignment of T2I systems across six pairs of contradictory objectives. 
We present \textbf{YinYangAlign}, an advanced benchmarking framework that systematically quantifies the alignment fidelity of T2I systems, addressing six fundamental and inherently contradictory design objectives. Each pair represents fundamental tensions in image generation, such as balancing adherence to user prompts with creative modifications or maintaining diversity alongside visual coherence. YinYangAlign includes detailed axiom datasets featuring human prompts, aligned (chosen) responses, misaligned (rejected) AI-generated outputs, and explanations of the underlying contradictions.

In addition to presenting this benchmark, we introduce \textbf{Contradictory Alignment Optimization (CAO)}, a novel extension of DPO. The CAO framework incorporates a per-axiom loss design to explicitly model and address competing objectives. Then it optimizes these objectives using multi-objective optimization techniques, including \textit{synergy-driven global preferences}, \textit{axiom-specific regularization}, and the novel \textit{synergy Jacobian} for effectively balancing contradictory goals. By utilizing tools such as the \textit{Sinkhorn-regularized Wasserstein Distance}, CAO achieves both stability and scalability while setting new performance benchmarks across all six contradictory alignment objectives.


%In addition to introducing this benchmark, we present results from applying DPO to YinYangAlign. Furthermore, we propose a novel loss function for DPO, termed \textbf{Hybrid Loss}, which demonstrates improved performance in balancing conflicting objectives compared to traditional DPO. Nonetheless, our evaluation reveals that \textbf{YinYangAlign} sets a new, stringent benchmark that current techniques, including Hybrid Loss, must strive to meet. This underscores YinYangAlign’s role in pushing the boundaries of T2I alignment and highlights the ongoing need for sophisticated alignment mechanisms. 
\end{abstract}



\section{Introduction}
\label{sec:intro}
% Image editing methods in diffusion models depend on user-defined control directions - users can unlock their creativity using these methods by specifying the desired manipulation through prompts~\cite{gandikota2023concept}, reference images~\cite{ruiz2022dreambooth, kumari2022customdiffusion, gal2022image, chen2024trainingfreeregionalpromptingdiffusion}, or attribute vectors~\cite{parmar2023zero,hertz2022prompt}. In this work, we ask a fundamentally different question: \emph{Can we automatically discover the underlying visual structure of a concept within diffusion model's knowledge?} %Rather than requiring user-specified controls, we aim to decompose the model's internal knowledge into meaningful directions.

% This question touches on a fundamental limitation in how we interact with diffusion models. Current control methods ~\cite{zhang2023addingconditionalcontroltexttoimage, gandikota2023concept, ye2023ipadaptertextcompatibleimage,ye2023ipadaptertextcompatibleimage, hertz2024stylealignedimagegeneration, li2023photomaker, shi2024instantbooth, chen2024trainingfreeregionalpromptingdiffusion} require users to specify their desired manipulations in advance, limiting interactive creativity. This contrasts with natural human artistic workflows, where creators dynamically explore creative ideas while jointly refining them toward meaningful artistic outcomes~\cite{hoffmann2016modeling}. This synergy between specification and exploration is not new to generative models. Early GAN architectures naturally developed disentangled latent spaces that enabled continuous\cite{harkonen2020ganspace,radford2015unsupervised, wu2021stylespace, shen2020interfacegan}, compositional control over generated images. Users could explore these spaces to discover interesting variations that would be difficult to describe in words~\cite{wu2021stylespace}, then combine them to achieve their creative goals~\cite{grabe2022towards}. 


% While diffusion models have largely superseded GANs in conditional image synthesis~\cite{dhariwal2021diffusion},  their underlying structure remains less understood. Diffusion models achieve remarkable diversity through high-dimensional latents, unlike GANs' compact latent spaces.  With a single prompt, diffusion models can generate radically different variations through different random initializations of input noise. We ask - Is it possible to discover interpretable structure within this vast space of variations?

Text-to-image diffusion models are capable of generating remarkable visual variations from a single prompt through different random initializations. However, this vast creative potential remains largely opaque to users---while we can generate diverse images, we lack understanding of the underlying structure of these variations. This presents a fundamental challenge: how can we discover and expose the latent visual capabilities encoded within these models?

\let\thefootnote\relax \footnote{$^{*}$Correspondence to \texttt{gandikota.ro@northeastern.edu}}

The challenge touches on a key limitation in how we interact with diffusion models today. Current control methods require users to explicitly specify their desired edits in advance through prompts~\cite{gandikota2023concept}, reference images~\cite{zhang2023addingconditionalcontroltexttoimage, chen2024trainingfreeregionalpromptingdiffusion, ruiz2022dreambooth,kumari2022customdiffusion, Ryu_lora, hu2021lora}, or attribute vectors~\cite{ye2023ipadaptertextcompatibleimage, hertz2024stylealignedimagegeneration, li2023photomaker, shi2024instantbooth,parmar2023zero,hertz2022prompt}. That contrasts sharply with natural human creative workflows, where artists dynamically explore creative ideas and jointly refine them toward meaningful artistic outcomes~\cite{hoffmann2016modeling}. The need for pre-specified controls creates a barrier between users and the full creative potential of these models.

Interestingly, earlier generative models like GANs~\cite{gans,karras2019style,brock2018large} naturally developed more interpretable internal structures. Their compact latent spaces often exhibited emergent disentanglement~\cite{harkonen2020ganspace,radford2015unsupervised, wu2021stylespace, shen2020interfacegan}, enabling continuous and compositional control over generated images. Users could explore these spaces to discover interesting variations that would be difficult to describe in words~\cite{wu2021stylespace}, then combine them to achieve their creative goals~\cite{grabe2022towards}.

Diffusion models have largely superseded GANs in conditional image synthesis~\cite{dhariwal2021diffusion}, achieving greater diversity through much higher-dimensional latents. And yet an understanding of the underlying structure of these larger latent spaces has remained elusive. In this work, we ask a fundamental question: \emph{Can we automatically discover the visual structure within a diffusion model's knowledge of a concept?} Rather than requiring user-specified controls, we aim to decompose the model's internal representations into expressive directions that users can explore and combine.

To address these needs, we present \textbf{SliderSpace}, a framework that brings systematic explorability to diffusion models. Given just a text prompt, SliderSpace discovers a canonical set of meaningful, diverse, and controllable directions within the model's knowledge of that concept. Each direction is implemented as a low-rank adapter~\cite{hu2021lora} that can be scaled and composed with others, allowing users to explore and smoothly combine different aspects of variation, as shown in Figure~\ref{fig:intro}.

We ground SliderSpace discovery in three key requirements for meaningful decomposition of a diffusion model's visual manifold: 
\begin{enumerate}
    \item \textbf{Unsupervised Discovery:} The decomposition process should emerge from the intrinsic structure of the model's learned representation, rather than being guided by predefined attributes. This ensures we capture the true topology of the model's knowledge space rather than projecting our assumptions onto it.
    
    \item \textbf{Semantic Orthogonality:} Each discovered control must represent a distinct semantic direction. This is enforced in a semantic feature space, like CLIP, where every slider has an orthogonal effect in embeddings. This prevents discovering multiple controls that create similar semantic effects, making the system more efficient and easier.
    
    \item \textbf{Distribution Consistency:} Directions must induce consistent transformations across both random seeds and prompt variations. 
\end{enumerate}

These requirements naturally lead to our proposed framework, which we formalize in Section~\ref{sec:method}. As we show in our experiments, SliderSpace is architecture-agnostic, working with both conventional U-Net based models like Stable Diffusion~\cite{rombach2022high, rombach2022sd20, podell2023sdxl, turbo, dmd} and recent transformer-based architectures like Flux~\cite{flux}.

We demonstrate the expressiveness of SliderSpace through three applications: First, we show how SliderSpace can decompose high-level concepts into diverse and expressive components, revealing the natural axes of variation in the model's understanding. Second, we explore artistic style variation, where SliderSpace discovers directions that match or exceed the diversity of manually curated artist lists while being judged more useful by human evaluators. Finally, we show how SliderSpace can help reverse the mode collapse commonly observed in distilled diffusion models, restoring diversity while maintaining generation speed.

Beyond providing practical creative control, SliderSpace opens new avenues for understanding and utilizing the latent capabilities of diffusion models. By mapping these models' visual potential into intuitive, composable directions, we take a step toward making their creative possibilities more accessible and interpretable to users.

% Image editing methods in diffusion models unlock the creativity of users. In this work we ask an alternate question: \emph{Can we organize and expose what of the diffusion model is already capable of?}.
% Existing methods for controlling image generation typically require users to manually specify edit directions for desired changes. This process is time-consuming, requires technical expertise, and limits the spontaneity of the creative process. For instance, if a user wants to adjust the smile of a generated person, they must explicitly request this edit, often through imprecise prompt engineering or model fine-tuning. This approach of predefined controls or manual specifications restricts users from fully exploring the latent capabilities of the model. There may be interesting stylistic variations or attributes that the model can generate, but users have no easy way to discover or utilize these.

% Natural visual disentanglement was an emergent property in the latent space of Generative Adversarial Models (GANs) \cite{harkonen2020ganspace,radford2015unsupervised, wu2021stylespace, shen2020interfacegan}. In particular, it has been observed that StyleGAN~\cite{karras2019style} stylespace neurons offer detailed control over many meaningful aspects of images that would be difficult to describe in words~\cite{wu2021stylespace}. However, diffusion models do not share such a compact latent space~\cite{park2023unsupervised}; and efforts to uncover such a space in the semantic embeddings of the text conditioning have met with limited success \nik{Nick - is there a specific citation you were thinking about?}.

% In this work we introduce \textbf{SliderSpace}, which takes a step towards uncovering an analogous low dimensional representation of diffusion models' visual breadth; in essence treating the diffusion model as many generators sharing parameters, where a particular generator is defined by a specific prompt. For a given prompt we sample many random seeds (and optionally prompt expansions using an LLM), generate the corresponding images, and apply an off the shelf feature extractor (in this work CLIP, but our method can be applied to any differentiable feature extractor). We use PCA to analyze these features, and for each of the leading $k$ principal components we train a LoRA \cite{} which causes the diffusion model to produces images which increase the feature magnitude along that component when passed back through the same feature extractor. This leads to a 'Slider' for each principal component, because each LoRA can be scaled and applied to the original diffusion model, continuously varying those visual features in the generated results (as measured, in our case, by CLIP).

% There are many other works that enhance the controllability of diffusion models. One common approach is enabling users to add spatial constraints to a generation either manually, or via a reference image \cite{zhang2023addingconditionalcontroltexttoimage, chen2024trainingfreeregionalpromptingdiffusion}, a second is leveraging more abstract embeddings (e.g. identity, style) extracted from a reference image \cite{ye2023ipadaptertextcompatibleimage, hertz2024stylealignedimagegeneration, li2023photomaker, shi2024instantbooth}, a third is finetuning a foundation model to better generate a concept important to the user \cite{ruiz2022dreambooth, kumari2022customdiffusion, Ryu_lora, hu2021lora}, and a fourth (most relevant to this work) is finding low-rank adaptors of the model based on a prompt or small training set which can be scaled to provide continous control over one aspect of generated image (e.g. night vs day, basic vs luxury, etc.) \cite{gandikota2023concept}. SliderSpace is complementary to all of these methods and offers something distinct. All of the other methods we are aware require the user (and / or model designer) to know in advance what type of control they want. In contrast SliderSpace assists users in discovering and controlling hidden capabilities present in the diffusion model's distribution of possible generations.

%We propose that truly intuitive creative control in a text-to-image model should meet three key criteria: \emph{discoverability}, \emph{intuitiveness}, and \emph{specificity}. The model should reveal controllable attributes that may not be immediately obvious, offer controls that are easy to understand and manipulate, and ensure each control affects a distinct attribute of the generated image.

% We demonstrate the utility and power of SliderSpace using three applications built on top of SDXL-DMD \cite{dmd}, because its fast generation speed lends itself well to the continuous control offered by SliderSpace.

% First, we study concept decomposition (Section \ref{sec:concept_exp}), where we learn sliders for a specific concept (e.g. 'monster', 'waterfall', 'car'). Through quantitative metrics of diversity and text alignment we demonstrate that the learned sliders dramatically boost the diversity of generations when randomly applied without harming text alignment; we also ask humans to qualitatively judge these results in a user study where they find the SliderSpace results to be more 'Diverse', 'Useful', and 'Creative' than our baselines.

% Second, we attempt to compare the automatic discoveries of SliderSpace to a large scale manual study of artistic styles (Section \ref{sec:art_exp}), open-sourced by ParrotZone \cite{parrotzone}. In this study SDXL was prompted with over 4300 artist names,  and based on visual inspection the cases of successful stylistic mimicry recorded. Quantitatively SliderSpace more closely matches the distribution of artistic variation discovered by ParrotZone than other baselines, and in our user studies was judged to be significantly more 'Diverse' and 'Useful' than the baselines. To our surprise humans even judged SliderSpace results to be slightly more 'Diverse' than the results generated by the manually discovered artist names of \cite{parrotzone}.

% Third, we attempt to use SliderSpace to reverse the mode collapse commonly observed in distilled few-step diffusion models relative to the original teacher model (Section \ref{sec:diverse_exp}). We quantitatively demonstrate that applying SliderSpace to SDXL-DMD leads to more closely matching the distribution of images by the original teacher, SDXL.

%Through extensive experiments on various state-of-the-art text-to-image models, we demonstrate that SliderSpace significantly enhances user control and creative expression in AI-assisted image generation tasks. Our method enables a range of applications, including concept decomposition and control, diversity improvement in generated images, customization dissection and edits, and the exploration of artistic styles inherent in the model.

% SliderSpace goes beyond providing a practical tool for enhanced creative control. By mapping the visual potential of diffusion models it can open new avenues for generative creativity and deepens our understanding of each model's hidden potential.




% Please add the following required packages to your document preamble:
% \usepackage{booktabs}
% \usepackage[normalem]{ulem}
% \useunder{\uline}{\ul}{}
% \usepackage{longtable}
% Note: It may be necessary to compile the document several times to get a multi-page table to line up properly




\begin{figure*}[ht!]
    \centering
    \includegraphics[width=\textwidth]{img/stochastic.pdf}
    \caption{Illustrative example of aligning T2I models with Faithfulness to Prompt vs. Artistic Freedom. The chosen outputs adhere closely to the prompt, depicting a highly detailed and accurate portrait of Albert Einstein in a realistic oil painting style, while the rejected outputs deviate significantly, introducing surreal or unrelated elements. This highlights the importance of balancing prompt adherence with artistic flexibility in alignment optimization.}
    \label{fig:stochastic_generation}
    \vspace{-2mm}
\end{figure*}



\section{YinYangAlign: Six Contradictory Alignment Objectives}

Current research and benchmarking in T2I alignment primarily focus on isolated objectives \cite{guo2022survey}, such as fidelity to prompts \cite{ramesh2021zero}, aesthetic quality \cite{rombach2022high}, or bias mitigation \cite{zhao2023mitigating}, often treating these goals independently. However, there is a clear gap in benchmarks that evaluate how T2I systems balance multiple, often contradictory objectives. 
%This is a critical limitation, as real-world applications demand systems capable of achieving diverse outputs while maintaining emotional neutrality and relevance to user prompts. 
The lack of multi-objective benchmarks restricts the ability to holistically assess and improve T2I alignment, ultimately affecting their reliability and effectiveness in practical scenarios.


\noindent
\textbf{Selection of Six Contradictory Objectives}: YinYangAlign introduces six carefully selected pairs of contradictory objectives that capture the fundamental tensions in T2I image generation. These pairs are chosen for their relevance and significance in real-world applications. \cref{fig:alignment_axioms} introduces the core trade-offs central to the YinYangAlign framework, each representing a critical conflict that T2I systems must navigate to balance user expectations and ethical considerations. The trade-offs include: \textit{Faithfulness to Prompt vs. Artistic Freedom}, which involves adhering to user instructions while minimizing creative deviations; \textit{Emotional Impact vs. Neutrality}, requiring a balance between evoking emotions and maintaining objective representation; and \textit{Visual Realism vs. Artistic Freedom}, focusing on achieving photorealistic outputs without compromising artistic liberties. Additionally, \textit{Originality vs. Referentiality} addresses the challenge of fostering stylistic innovation while avoiding reliance on established artistic styles to ensure uniqueness. \textit{Verifiability vs. Artistic Freedom} emphasizes balancing factual accuracy with creative liberties to minimize misinformation. Finally, \textit{Cultural Sensitivity vs. Artistic Freedom} underscores the need to respect cultural representations while ensuring that creative freedoms do not lead to misrepresentation or insensitivity. \cref{tab:yinyang_axioms} provides illustrative examples of these alignment axioms.

YinYangAlign serves as a holistic benchmark for evaluating alignment performance, ensuring that T2I models are not only accurate and reliable but also adaptable, ethical, and capable of meeting complex user demands and societal expectations.





\begin{comment}
\begin{figure*}[ht!]
    \centering
    \resizebox{1.0\textwidth}{!}{
       \includegraphics[width=\linewidth]{img/yinyang-1.pdf}    
    }
    \caption{Architecture of PECCAVI-Image.}
    \label{fig:strength_var}
\end{figure*}

\begin{figure*}[ht!]
    \centering
    \resizebox{1.0\textwidth}{!}{
       \includegraphics[width=\linewidth]{img/yinyang-2.pdf}    
    }
    \caption{Architecture of PECCAVI-Image.}
    \label{fig:strength_var}
\end{figure*}
\end{comment}

\begin{comment}
\subsection{Faithfulness to Prompt vs. Creative Enhancement}

This axiom highlights the tension between adhering to user instructions and introducing creative elements to enhance the generated images.

\textbf{Objective Pair Overview:}
\begin{itemize}
    \item \textbf{Faithfulness to Prompt:} Ensures the generated image accurately reflects the user's textual description, maintaining fidelity to all specified details.
    \item \textbf{Creative Enhancement:} Introduces imaginative elements to enhance the aesthetic or interpretative depth, which may sometimes deviate from the original instructions.
\end{itemize}


\textbf{Core Conflict:}  
Balancing fidelity to user prompts with creative enhancements is a nuanced challenge. While faithfulness guarantees precision, creative additions risk altering the intent to enrich the output. \cref{tab:axiom_1} illustrates this tradeoff with examples of chosen and rejected AI-generated responses.
\end{comment}































\section{Data} 
\label{sec:data}

We now describe the data used for training and deploying our AI-based approach. 
Gathering this data required a significant digitization process, extracting and processing 5.2 million pages of deeds stretching back to the 1850s (\S~\ref{sec:digit}). We then supplemented this data with historical deed records available online from around the country (\S~\ref{sec:otherdata}), which has the coincidental benefit of enabling us to assess the model's robustness across jurisdictions.  Finally, we manually annotate 3,801 deeds to build a training dataset and held-out evaluation dataset for our AI pipeline (\S~\ref{sec:annotation}).


\subsection{Digitization, Collection, and Sharing of Real Property Deeds}
\label{sec:digit}
The Santa Clara County Clerk-Recorder's Office has an extensive archive of over 24 million real property deeds. Of these records, approximately 18 million -- issued since 1980 -- are stored digitally, while the remaining 6 million deeds -- created before 1980 -- were originally preserved on physical microfiche sheets. More than a decade prior to our work, the County had engaged a vendor to scan these records into a proprietary system known as Digital Reel; however, as we discuss in Appendix \ref{appendix_ocr}, the quality of these scans was poor and required significant post-processing.

Our partnership around exploring the use of AI began in October 2022. One of the notable barriers to transparency around deed records in California lies in a statutory mandate to charge fees for any copies of recorded documents.\footnote{Cal.\ Gov.\ Code \S~27366 provides: ``The fee for any copy of any other record or paper on file in the office of the recorder, when the copy is made by the recorder, shall be set by the board of supervisors in an amount necessary to recover the direct and indirect costs of providing the product or service or the cost of enforcing any regulation for which the fee or charge is levied.'' This provision has been subject to extensive litigation. See, e.g., California Public Records Research, Inc.\ v.\ County of Stanislaus, 246 Cal.\ App.\ 4th 1432 (2016);  California Public Records Research v.\ County of Yolo, 4 Cal.\ App.\ 5th 150 (2016).}  In other words, despite their status as public records, deed documents are available only on an individual fee basis. Given the massive scale of the review task, purchasing deed records would, of course, have been prohibitively expensive.\footnote{At a cost of \$4 for the first page and \$2 for each subsequent page, purchasing the 5.2M pages (with the average deed running 2.5 pages) might have cost over \$13 million.} Through our partnership, we developed unique a data sharing agreement, enabling the Stanford team  to process deed data, with the County retaining ownership of the records.

We began our work on samples of 20,000 pages of property deeds filed between 1900 and 1940, manually exported from the County's Digital Reel system. This 20,000-page sample enabled us to rapidly develop and refine our automated detection pipeline. 

After this piloting phase, the County extracted the full collection of pre-1980 scans in February 2024. This represents roughly 5.2 million pages of real property documents from 1865 to 1980. We focus our analysis on documents from 1902 to 1980 for two reasons. First, deeds filed prior to 1902 were handwritten rather than typed, and we found no available OCR tools to be effective at transcribing these documents.\footnote{We did explore developing a bespoke computer vision or multimodal text-vision system.} Second, records after 1980 contain protected fields like Social Security information, so we avoided ingesting sensitive data and potentially training our model on it, which may have raised privacy and legal concerns. As we note above and consistent with our results in Section~\ref{sec:evolution}, 1902 to 1980 likely covers the vast majority of racial covenants in the County; the first racial covenant we find was filed in 1907 and the last in 1974.

\subsection{Data Augmentation}
\label{sec:otherdata}
Both within California and across the nation, historical property deeds vary significantly in format, phrasing, and, when digitized, OCR quality. In order to build a system that is robust to these variations, we supplemented the Santa Clara County dataset with property deeds from around the nation, both with and without racial covenants.

Since property records in California counties are not freely accessible, we expanded our search to other counties in the United States. Using \href{https://govos.com/products/public-record-access/channel/}{GovOS Cloud Search}, we identified seven counties whose ``Official Records Search’’ platforms allowed users to freely search and download real property deeds, although downloads were limited to fifty records per batch.\footnote{\url{https://kofilehelp.zendesk.com/hc/en-us/sections/4416665864343-Cloud-Search-Active-Sites}.} These platforms enabled searches by metadata and keyword terms. To gather a seed dataset of deeds with a high probability of containing racial covenants, we conducted manual searches for terms typically associated with such covenants, such as ``No person of,'' ``Caucasian,'' ``Negro,'' and other relevant racial terms. This method provided us with more than 10,000 property deeds from seven counties: Bexar County, Texas; Cuyahoga County, Ohio; Denton County, Texas; Franklin County, Ohio; Hidalgo County, Texas; and Lawrence County, Pennsylvania. This approach not only helped us collect relevant data but also allowed us to assess the generalizability of our model across different jurisdictions. As we discuss below, we specifically investigate the limitations of keyword-based approaches, and find that context-aware language models boost performance substantially. 




\subsection{Annotation} \label{sec:annotation}

We labeled our data collection by identifying quotes that contain racial covenants on each page. This annotation occurred over three stages: initial training data generation, model prediction review, and rich annotation.

In our initial round of annotation, we selected a sample of 3,000 pages in our collection based on keywords that almost certainly indicate the presence of a racial covenant in the deed text. These include terms like ``Negro,'' ``Mongolian,'' and ``Asiatic.'' We partnered with data annotation company CloudFactory to help us identify and label racial covenants in these pages.\footnote{During our collaboration with CloudFactory for data annotation, we carefully prepared comprehensive documentation to guide the annotators through the task. Given the potentially sensitive nature of the material—historical property deeds containing racially restrictive covenants, as well as accounts that could be considered offensive or harmful to some readers—we issued a clear advisory to approach the content with care. We emphasized that the annotators could stop the task at any point if they felt uncomfortable. In addition, we consulted with CloudFactory’s management to ensure that appropriate counseling and support resources would be available to their team, should any annotator feel the need for assistance or support. Our priority was to handle this material with the utmost sensitivity, while ensuring the well-being of those involved in the annotation process.}

After training models and generating predictions, we reviewed their performance. For all positive predictions, we labeled whether they were true positives or false positives. These ensured that we verified the small number of positive examples as well as hard negative examples. We additionally sampled and verified negative predictions to ensure some balance in the data. These new annotations were incorporated into the training set of future models.

Recognizing the need to easily validate model predictions and locate racial covenants on a page, we built a web application to assist with rich annotation. This made it easy to precisely select a bounding box on the image of the deed book page and compute a text span for the annotation process, while simultaneously allowing us to visualize predictions for verification.

All combined, including both Santa Clara County documents and documents from across the country, we collected 3,801 annotations of deed pages, of which 2,987 (78.6\%) contained a racially restrictive covenant. Notably, this annotation requires human review, but at a much smaller scale than reviewing all records.  






 









\section{Contradictory Alignment Optimization (CAO)}

The \textbf{YinYangAlign} framework, models the challenge of balancing \emph{inherently contradictory} objectives. For example, prioritizing \emph{Faithfulness to Prompt} can limit \emph{Artistic Freedom}, while emphasizing \emph{Emotional Impact} may erode \emph{Neutrality}. To address these tensions, we introduce \textbf{Contradictory Alignment Optimization (CAO)}, which employs a \emph{per-axiom} loss design to explicitly model competing goals. CAO employs a dynamic weighting mechanism to prioritize sub-objectives within each axiom, facilitating granular control over trade-offs and enabling adaptive optimization across diverse alignment paradigms. Additionally, CAO integrates \emph{Pareto optimality} principles with the \emph{Bradley-Terry} preference framework, introducing a novel \emph{global synergy} mechanism that unifies all contradictory objectives into a cohesive optimization strategy. This unique combination of multi-objective synergy defines the core innovation of CAO, distinguishing it from existing T2I alignment methods.

%In \textbf{YinYangAlign} setup, T2I models must reconcile \emph{intrinsically contradictory} aims. For instance, \emph{Faithfulness to Prompt} can restrict \emph{Artistic Freedom}, while prioritizing \emph{Emotional Impact} may erode \emph{Neutrality}. These tensions motivate our DPO-\textbf{Contradictory Alignment Optimization (CAO)} design, which introduces a \emph{per-axiom} loss design to explicitly model competing goals. By assigning specific weights to each sub-objective within an axiom, CAO allows for flexible tradeoffs tailored to individual alignment requirements. Importantly, CAO merges \emph{Pareto optimality} concepts with the \emph{Bradley-Terry} preference framework, yielding a novel \emph{global synergy} mechanism that unifies all contradictory objectives under one coherent optimization strategy. This union of multi-objective synergy and directed preference optimization is the core novelty of CAO, setting it apart from prior T2I alignment approaches.



\begin{figure*}[ht!]
    \centering
    \includegraphics[width=\textwidth]{img/axiom_pairs_visualization.png}
    \caption{
        Visualization of error loss surface tension for six axiom pairs in YinYang alignment. Each pair highlights the inherent trade-offs between \emph{competing objectives} using a 3D surface plot (left) and a 2D contour plot (right). \textcolor{blue}{Blue regions} represent synergy (low tension), \textcolor{red}{red regions} indicate conflict (high tension), while \textcolor{green}{Green markers} highlight "sweet spots" where the tension is minimal. The first axiom pair, \textit{Faithfulness to Prompt vs. Artistic Freedom}, shows sweet spots centered around moderate values, suggesting balanced trade-offs. For \textit{Emotional Impact vs. Neutrality}, sweet spots are sparse, reflecting the difficulty in balancing emotional engagement and neutrality. The axiom pair \textit{Visual Realism vs. Artistic Freedom} shows distributed sweet spots, indicating achievable trade-offs between realism and creative freedom. In \textit{Originality vs. Referentiality}, sweet spots are concentrated, emphasizing the challenge of balancing uniqueness and references. The pair \textit{Verifiability vs. Artistic Freedom} has central sweet spots, suggesting harmony between factual accuracy and creative expression. Lastly, \textit{Cultural Sensitivity vs. Artistic Freedom} shows fewer sweet spots, reflecting the complexity of respecting cultural norms while granting artistic liberties. This visualization underscores the inherent trade-offs in T2I systems and identifies potential areas of optimization for aligning competing objectives.
    }
    \label{fig:axiom_pairs_tension}
    \vspace{-2mm}
\end{figure*}




\subsection{Axiom-Wise Loss Expansion and Synergy}

\paragraph{Local Axiom-Wise Loss}: Below, we illustrate how each axiom’s loss is defined, before showing how these losses connect into a global synergy framework. For each axiom \(a\), CAO defines a loss function \(f_a(I)\) that blends two competing sub-objectives, \(\mathcal{L}_p(I)\) and \(\mathcal{L}_q(I)\), via a mixing parameter \(\alpha_a\):
\[
f_a(I)
=
\alpha_a \,\mathcal{L}_p(I)
+
\bigl(1 - \alpha_a\bigr)\,\mathcal{L}_q(I).
\]
For example, \(\mathcal{L}_p(I)\) might emphasize \emph{faithfulness to prompt}, while \(\mathcal{L}_q(I)\) favors \emph{artistic freedom}, or any other pair of conflicting objectives. Varying \(\alpha_a\) adjusts the per-axiom balance according to domain or policy needs.


\begin{tcolorbox}[colframe=black,colback=white,boxrule=0.5mm,width=\columnwidth,sharp corners]
\scriptsize
\begin{itemize}[left=-4pt,itemsep=0pt,topsep=0pt,parsep=0pt]
    \item \textbf{Faithfulness to Prompt vs.\ Artistic Freedom}
    \vspace{-3mm}
    \[
    f_{\text{faith\_artistic}}(I) 
    = \alpha_1 \cdot \mathcal{L}_{\text{faith}}
    + (1 - \alpha_1) \cdot \mathcal{L}_{\text{artistic}}
    \]
    \vspace{-6mm}

    \item \textbf{Emotional Impact vs.\ Neutrality}
    \vspace{-3mm}
    \[
    f_{\text{emotion\_neutrality}}(I) 
    = \alpha_2 \cdot \mathcal{L}_{\text{emotion}}
    + (1 - \alpha_2) \cdot \mathcal{L}_{\text{neutrality}}
    \]
    \vspace{-6mm}

    \item \textbf{Visual Realism vs.\ Artistic Freedom}
    \vspace{-3mm}
    \[
    f_{\text{visual\_style}}(I) 
    = \alpha_3 \cdot \mathcal{L}_{\text{realism}}
    + (1 - \alpha_3) \cdot \mathcal{L}_{\text{artistic}}
    \]
    \vspace{-6mm}

    \item \textbf{Originality vs.\ Referentiality}
    \vspace{-3mm}
    \[
    f_{\text{originality\_referentiality}}(I) 
    = \alpha_4 \cdot \mathcal{L}_{\text{originality}}
    + (1 - \alpha_4) \cdot \mathcal{L}_{\text{referentiality}}
    \]
    \vspace{-6mm}

    \item \textbf{Verifiability vs.\ Artistic Freedom}
    \vspace{-3mm}
    \[
    f_{\text{verifiability\_creative}}(I) 
    = \alpha_5 \cdot \mathcal{L}_{\text{verifiability}}
    + (1 - \alpha_5) \cdot \mathcal{L}_{\text{artistic}}
    \]
    \vspace{-6mm}

    \item \textbf{Cultural Sensitivity vs.\ Artistic Freedom}
    \vspace{-3mm}
    \[
    f_{\text{cultural\_artistic}}(I) 
    = \alpha_6 \cdot \mathcal{L}_{\text{cultural}}
    + (1 - \alpha_6) \cdot \mathcal{L}_{\text{artistic}}
    \]
\end{itemize}
\end{tcolorbox}

The resulting loss surfaces and their corresponding \emph{sweet spots}, where competing objectives are in harmony, are visualized in \cref{fig:axiom_pairs_tension}.


\paragraph{Multi-Objective Aggregator and Pareto Frontiers:} Although \(f_a(I)\) provides \emph{local} control over each axiom \(a\), reconciling multiple axioms at once requires a \emph{global} view. We thus define a \textbf{multi-objective synergy function}:
\[
\mathcal{S}(I)
=
\sum_{a=1}^A
\omega_a \, f_a(I),
\]
where the \(\{\omega_a\}\) are global coefficients reflecting the relative priority of each axiom. By varying these synergy weights, we trace out a Pareto frontier~\cite{miettinen1999nonlinear, yang2021towards, lin2023pareto} in the T2I objective space, clarifying how small concessions in one axiom can yield major gains in another.




\smallskip
\noindent
\textbf{Interpretation and Importance.}\quad
In \emph{multi-objective optimization}, the \emph{Pareto frontier} is the set of all solutions where improving any one objective strictly worsens at least one other~\cite{deb2001multiobjective, zhou2022pareto}. By tuning \(\{\omega_a\}\), we systematically explore these tradeoffs, finding, for example, that a slight drop in \emph{visual realism} could allow for notably higher \emph{stylistic freedom}. Such multi-objective approaches have been central in \emph{multi-task learning}~\cite{ma2020quadratic, navon2022multi, yu2020gradient} and \emph{modular/decomposed learning}~\cite{liebenwein2021provable, lin2022pareto}, ensuring transparent control over each tension point (e.g., verifiability vs.\ creativity) and easy adaptation to new constraints. cf  \cref{sec:appendix:dpo-cao}. 


\subsection{Connecting Synergy to Pairwise Preference}

To fully implement both \emph{local} axiom-wise guidance and \emph{global} synergy-based tradeoffs, we integrate the synergy function into the DPO framework. Concretely, each \(f_a(I)\) enters a Bradley-Terry style preference:
\[
P_{ij}^a
=
\frac{
\exp\!\bigl(f_a(I_i)\bigr)
}{
\exp\!\bigl(f_a(I_i)\bigr) + \exp\!\bigl(f_a(I_j)\bigr)
},
\]
ensuring local interpretability for each axiom. Meanwhile, a \emph{combined preference} over \(\mathcal{S}(I)\) expresses the global tradeoff:
\[
P_{ij}^{\mathcal{S}}
=
\frac{\exp\!\bigl(\mathcal{S}(I_i)\bigr)}
     {\exp\!\bigl(\mathcal{S}(I_i)\bigr) + \exp\!\bigl(\mathcal{S}(I_j)\bigr)}.
\]
A hyperparameter \(\lambda\) then balances how much this \textbf{global synergy} affects the final optimization vs.\ how much weight is given to \textbf{local} per-axiom preferences:
\[
\mathcal{L}_{\text{CAO}}
=
- \sum_{a=1}^{A}
\sum_{(i,j)}
\log\!\bigl(P_{ij}^a\bigr)
+
\lambda
\sum_{(i,j)}
\Bigl[
-\,\log\!\bigl(P_{ij}^{\mathcal{S}}\bigr)
\Bigr].
\]

\subsection{Unified CAO Loss}

We can consolidate the local and global preferences into a single loss function. One straightforward approach is:
\[
\mathcal{L}_{\text{CAO}}
=
\underbrace{
- \sum_{a=1}^{6}
  \sum_{(i,j)}
  \log\!\bigl(P_{ij}^a\bigr)
}_{\mathcal{L}_{\text{local}}}
+
\lambda
\underbrace{
\left[
  - \sum_{(i,j)}
  \log\!\bigl(P_{ij}^{\mathcal{S}}\bigr)
\right]
}_{\mathcal{L}_{\text{synergy}}}.
\]

\paragraph{Local Terms (\(\mathcal{L}_{\text{local}}\)).}  
Each axiom \(a\) retains interpretability and ensures the model handles \emph{faithfulness vs.\ artistry}, \emph{emotional impact vs.\ neutrality}, and so on, at a granular level.

\paragraph{Global Term (\(\mathcal{L}_{\text{synergy}}\)).}  
This enforces coordinated tradeoffs by encouraging consistency with the aggregator \(\mathcal{S}(I)\). A larger \(\lambda\) implies stronger synergy constraints and places more emphasis on global equilibrium across axioms, while a smaller \(\lambda\) prioritizes local alignment objectives.

\begin{figure*}[ht!]
\centering
\begin{tcolorbox}[
  enhanced,
  colback=white,
  colframe=black,
  boxrule=1pt,
  borderline={0.6pt}{2pt}{black},
  sharp corners,
  width=\textwidth
]

\begin{minipage}{\textwidth}
\scriptsize

\subsection*{(A) Local Axiom Preferences}
\begin{equation*}
\mathcal{L}_{\text{local}}
~\;=\;
- \Bigl[
   \sum_{(i,j)} \log\!\bigl(P_{ij}^{\text{faith_artistic}}\bigr)
 + \sum_{(i,j)} \log\!\bigl(P_{ij}^{\text{emotion_neutrality}}\bigr)
 + \sum_{(i,j)} \log\!\bigl(P_{ij}^{\text{visual_style}}\bigr)
 + \sum_{(i,j)} \log\!\bigl(P_{ij}^{\text{originality_referentiality}}\bigr)
 + \sum_{(i,j)} \log\!\bigl(P_{ij}^{\text{verifiability_creative}}\bigr)
 + \sum_{(i,j)} \log\!\bigl(P_{ij}^{\text{cultural_artistic}}\bigr)
\Bigr].
\end{equation*}

\vspace{-2mm}
\noindent
Here, each term is a negative log-likelihood over 
\(
   P_{ij}^{a}
   =
   \frac{\exp\!\bigl(f_{a}(I_i)\bigr)}
        {\exp\!\bigl(f_{a}(I_i)\bigr)+\exp\!\bigl(f_{a}(I_j)\bigr)}
\)
for axiom \(a\).

\vspace{-3mm}
\subsection*{(B) Global Synergy Preference}
\begin{equation*}
\mathcal{L}_{\text{synergy}}
~\;=\;
\sum_{(i,j)}
  \log\!\Bigl(
    \frac{
      \exp\!\Bigl(\omega_{1}f_{\text{faithArtistic}}(I_i)
      + \ldots + \omega_{6}f_{\text{culturalArtistic}}(I_i)\Bigr)
    }{
      \exp\!\Bigl(\omega_{1}f_{\text{faithArtistic}}(I_i)
      + \ldots + \omega_{6}f_{\text{culturalArtistic}}(I_i)\Bigr)
      +
      \exp\!\Bigl(\omega_{1}f_{\text{faithArtistic}}(I_j)
      + \ldots + \omega_{6}f_{\text{culturalArtistic}}(I_j)\Bigr)
    }
  \Bigr).
\end{equation*}

\vspace{-2mm}
\noindent
This term encodes the preference for 
\(
   \mathcal{S}(I) 
   = 
   \sum_{a=1}^{6} \omega_a\, f_a(I)
\).

\vspace{-2.5mm}
\subsection*{(C) Axiom-Specific Regularizers}
\begin{equation*}
\sum_{a=1}^{6} \tau_{a}\,\mathcal{R}_a
~=\;
\tau_{1}\,\frac{\displaystyle
   \int_{\mathcal{X}}\!\!\int_{\mathcal{X}}
   \|x-y\|\,
   P_{\text{faith}}(x)\,
   Q_{\text{artistic}}(y)
   \,\mathrm{d}x\,\mathrm{d}y
}{
   \displaystyle
   \int_{\mathcal{X}}P_{\text{faith}}(x)\,\mathrm{d}x
   \;\times\;
   \int_{\mathcal{X}}Q_{\text{artistic}}(y)\,\mathrm{d}y
}
~+~
\ldots
~+~
\tau_{6}\,\frac{\displaystyle
   \int_{\mathcal{X}}\!\!\int_{\mathcal{X}}
   \|x-y\|\,
   P_{\text{cultural}}(x)\,
   Q_{\text{artistic}}(y)
   \,\mathrm{d}x\,\mathrm{d}y
}{
   \displaystyle
   \int_{\mathcal{X}}P_{\text{cultural}}(x)\,\mathrm{d}x
   \;\times\;
   \int_{\mathcal{X}}Q_{\text{artistic}}(y)\,\mathrm{d}y
}.
\end{equation*}

\end{minipage}
\end{tcolorbox}

\vspace{-2mm}
\includegraphics[width=\textwidth]{img/ablation_error_loss.png}

%\captionsetup{justification=justified, singlelinecheck=false}
\captionsetup{justification=justified, singlelinecheck=false}
\caption{A modular breakdown of the CAO loss. 
\textbf{(A)} Local per-axiom preferences, 
\textbf{(B)} global synergy preference, 
\textbf{(C)} axiom-specific regularizers. 
Three error loss surfaces from the ablation study demonstrate the progressive impact of incorporating components of the YinYang alignment objective. 
The first plot, with only the \textit{Local Axiom Preferences}, shows an unstable gradient landscape. 
Adding in the second plot smooths the loss surface significantly. 
Finally, introducing additional \textit{Regularization Terms} in the third plot further stabilizes and smooths the surface, making optimization more efficient and robust.}






% \caption{A modular breakdown of the CAO loss. 
% \textbf{(A)} Local per-axiom preferences,
% \textbf{(B)} global synergy preference,
% \textbf{(C)} axiom-specific regularizers.
% Three error loss surfaces from the ablation study demonstrate
% the progressive impact of incorporating components of the YinYang
% alignment objective. The first plot, with only the
% \textit{Local Axiom Preferences} \(- \sum_{a=1}^{A} \sum_{(i,j)} \log\bigl(P_{ij}^a\bigr)\),
% shows an unstable gradient landscape. Adding \textit{Global Synergy Preference}
% \(-\lambda \sum_{(i,j)} \log\bigl(P_{ij}^{\mathcal{S}}\bigr)\) in the second plot
% smooths the loss surface significantly. Finally, introducing additional
% \textit{Regularization Terms} \(\sum_{a=1}^{A} \tau_a \mathcal{R}_a\)
% in the third plot further stabilizes and smooths the surface,
% making optimization more efficient and robust.}
\label{fig:dpo-cao-expanded}
\end{figure*}

% \begin{figure*}[ht!]
% \centering
% \begin{tcolorbox}[
%   enhanced,
%   colback=white,        % background color
%   colframe=black,       % color of the primary frame
%   boxrule=1pt,        % thickness of the primary frame
%   borderline={0.6pt}{2pt}{black}, % second line: thickness=0.6pt, distance=2pt, color=black
%   sharp corners,        % no rounded corners
%   width=\textwidth      % occupy full width in figure*
% ]


% \begin{minipage}{\textwidth}
% \scriptsize
% %-----------------------------------
% % (A) Local Axiom Preferences
% %-----------------------------------
% \subsection*{(A) Local Axiom Preferences}
% % \vspace{-0.5em}
% \begin{equation*}
% \mathcal{L}_{\text{local}}
% ~\;=\;
% - \Bigl[
%    \sum_{(i,j)} \log\!\bigl(P_{ij}^{\text{faith_artistic}}\bigr)
%  + \sum_{(i,j)} \log\!\bigl(P_{ij}^{\text{emotion_neutrality}}\bigr)
%  + \sum_{(i,j)} \log\!\bigl(P_{ij}^{\text{visual_style}}\bigr)
%  + \sum_{(i,j)} \log\!\bigl(P_{ij}^{\text{originality_referentiality}}\bigr)
%  + \sum_{(i,j)} \log\!\bigl(P_{ij}^{\text{verifiability_creative}}\bigr)
%  + \sum_{(i,j)} \log\!\bigl(P_{ij}^{\text{cultural_artistic}}\bigr)
% \Bigr].
% \end{equation*}

% \vspace{-2mm}
% \noindent
% Here, each term is a negative log-likelihood over the Bradley-Terry preference 
% \(\displaystyle P_{ij}^{a} = \frac{\exp\!\bigl(f_{a}(I_i)\bigr)}
%                                   {\exp\!\bigl(f_{a}(I_i)\bigr)+\exp\!\bigl(f_{a}(I_j)\bigr)}\)
% for axiom \(a\).

% \vspace{-3mm}
% %-----------------------------------
% % (B) Global Synergy Preference
% %-----------------------------------
% \subsection*{(B) Global Synergy Preference}
% \vspace{-0.5em}
% \begin{equation*}
% \mathcal{L}_{\text{synergy}}
% ~\;=\;
% \sum_{(i,j)}
%   \log\!\Bigl(
%     \frac{
%       \exp\!\Bigl(\omega_{1}f_{\text{faithArtistic}}(I_i)
%       + \ldots + \omega_{6}f_{\text{culturalArtistic}}(I_i)
%       \Bigr)
%     }{
%       \exp\!\Bigl(\omega_{1}f_{\text{faithArtistic}}(I_i)
%       + \ldots + \omega_{6}f_{\text{culturalArtistic}}(I_i)
%       \Bigr)
%       ~+~
%       \exp\!\Bigl(\omega_{1}f_{\text{faithArtistic}}(I_j)
%       + \ldots + \omega_{6}f_{\text{culturalArtistic}}(I_j)
%       \Bigr)
%     }
%   \Bigr).
% \end{equation*}

% \vspace{-2mm}
% \noindent
% This term encodes the preference for a \emph{global aggregator} 
% \(\displaystyle \mathcal{S}(I) = \sum_{a=1}^{6} \omega_a\, f_a(I)\),
% where each \(\omega_a\) is a weight signifying axiom \(a\)’s priority.

% \vspace{-2.5mm}
% %-----------------------------------
% % (C) Axiom-Specific Regularizers
% %-----------------------------------
% \subsection*{(C) Axiom-Specific Regularizers}
% \vspace{-0.25em}
% \begin{equation*}
% \sum_{a=1}^{6} \tau_{a}\,\mathcal{R}_a
% ~=\;
% \tau_{1}\,\frac{\displaystyle
%    \int_{\mathcal{X}}\!\!\int_{\mathcal{X}}
%    \|x-y\|\,
%    P_{\text{faith}}(x)\,
%    Q_{\text{artistic}}(y)
%    \,\mathrm{d}x\,\mathrm{d}y
% }{
%    \displaystyle
%    \int_{\mathcal{X}}P_{\text{faith}}(x)\,\mathrm{d}x
%    \;\times\;
%    \int_{\mathcal{X}}Q_{\text{artistic}}(y)\,\mathrm{d}y
% }
% ~+~
% \tau_{2}\,\frac{\displaystyle
%    \int_{\mathcal{X}}\!\!\int_{\mathcal{X}}
%    \|x-y\|\,
%    P_{\text{emotion}}(x)\,
%    Q_{\text{neutrality}}(y)
%    \,\mathrm{d}x\,\mathrm{d}y
% }{
%    \displaystyle
%    \int_{\mathcal{X}}P_{\text{emotion}}(x)\,\mathrm{d}x
%    \;\times\;
%    \int_{\mathcal{X}}Q_{\text{neutrality}}(y)\,\mathrm{d}y
% }
% ~+~
% \ldots
% ~+~
% \tau_{6}\,\frac{\displaystyle
%    \int_{\mathcal{X}}\!\!\int_{\mathcal{X}}
%    \|x-y\|\,
%    P_{\text{cultural}}(x)\,
%    Q_{\text{artistic}}(y)
%    \,\mathrm{d}x\,\mathrm{d}y
% }{
%    \displaystyle
%    \int_{\mathcal{X}}P_{\text{cultural}}(x)\,\mathrm{d}x
%    \;\times\;
%    \int_{\mathcal{X}}Q_{\text{artistic}}(y)\,\mathrm{d}y
% }.
% \end{equation*}

% \vspace{-2mm}
% \noindent
% Here, the first axiom’s regularizer is fully expanded with a Wasserstein-type cost, while subsequent axioms use a shorter notation \(W(\cdot,\cdot)\) for brevity.

% \end{minipage}
% \end{tcolorbox}

% \vspace{-2mm}
% \includegraphics[width=\textwidth]{img/ablation_error_loss.png}
% %\captionsetup{width=\textwidth}
% \caption{A modular breakdown of the CAO loss. 
% \textbf{(A)} Local per-axiom preferences,
% \textbf{(B)} global synergy preference,
% \textbf{(C)} axiom-specific regularizers.
% Three error loss surfaces from the ablation study demonstrate
% the progressive impact of incorporating components of the YinYang
% alignment objective. The first plot, with only the
% \textit{Local Axiom Preferences} \(- \sum_{a=1}^{A} \sum_{(i,j)} \log\bigl(P_{ij}^a\bigr)\),
% shows an unstable gradient landscape. Adding \textit{Global Synergy Preference}
% \(-\lambda \sum_{(i,j)} \log\bigl(P_{ij}^{\mathcal{S}}\bigr)\) in the second plot
% smooths the loss surface significantly. Finally, introducing additional
% \textit{Regularization Terms} \(\sum_{a=1}^{A} \tau_a \mathcal{R}_a\)
% in the third plot further stabilizes and smooths the surface,
% making optimization more efficient and robust.}
% \label{fig:dpo-cao-expanded}
% \end{figure*}



\paragraph{Why Keep Both Local \emph{and} Global?}
\begin{itemize}[left=5pt]
\item 
\emph{Local Preferences \((P_{ij}^a)\)} show how the model balances each contradictory pair (e.g., “\emph{Did we favor faithfulness over artistry?}”), preserving interpretability at the axiom level.
\item 
\emph{Global Preference \((P_{ij}^{\mathcal{S}})\)} ensures the T2I model, \emph{as a whole}, follows the overarching synergy profile, capturing \emph{all} tensions in unison.
\end{itemize}
Hence, \(\lambda\) “\emph{dials in}” how much to respect the overall synergy aggregator vs.\ each per-axiom preference.

\subsection{Axiom-Specific Regularization in CAO}

To stabilize the optimization and prevent overfitting to any single objective, CAO also provides a regularization term for each axiom:
\[
\mathcal{L}_{\text{CAO}}
=
\sum_{a=1}^6
\Bigl[
  f_a(I) 
  + 
  \tau_a \,\mathcal{R}_a
\Bigr],
\]
where \(\tau_a\) scales the influence of the regularizer \(\mathcal{R}_a\). While KL-divergence is a common choice, it can be unstable in high-dimensional T2I scenarios; \textbf{Wasserstein Distance}~\cite{arjovsky2017wasserstein} or \emph{Sinkhorn regularization}~\cite{cuturi2013sinkhorn} typically offer more robust optimization. cf \cref{sec:appendix_wasserstein_Sinkhorn} for the rationale behind Wasserstein Distance and Sinkhorn Regularization.








\subsection{Putting It All Together: Final CAO Formulation}

Bringing together the synergy function, local Bradley-Terry preferences, and axiom-specific regularization leads to the final CAO objective:
\[
\mathcal{L}_{\text{CAO}}
=
\underbrace{
- \sum_{a=1}^{A} 
  \sum_{(i,j)}
  \log\!\bigl(P_{ij}^a\bigr)
}_{\text{Local Axiom Preferences}}
-
\lambda
\underbrace{
\sum_{(i,j)}
\log\!\bigl(P_{ij}^{\mathcal{S}}\bigr)
}_{\text{Global Synergy Preference}}
+
\sum_{a=1}^{A}
\tau_a\,\mathcal{R}_a.
\]




\paragraph{Role of the Synergy Jacobian $(\mathbf{J}_{\mathcal{S}}$)}: The Synergy Jacobian \(\mathbf{J}_{\mathcal{S}}\) is a vital component in managing \emph{gradient interactions} across multiple axioms during training. While the regularization parameter \(\lambda\) balances local and global objectives, \(\mathbf{J}_{\mathcal{S}}\) quantifies how updates to model parameters for one axiom impact the alignment of others. Mathematically, \(\mathbf{J}_{\mathcal{S}}\) is defined as:
\[
\mathbf{J}_{\mathcal{S}} = \frac{\partial \mathcal{S}(I)}{\partial \theta},
\]
where \(\mathcal{S}(I)\) represents the synergy aggregator that measures overall alignment, \(I\) denotes the input, and \(\theta\) are the model parameters. This Jacobian provides a structured view of the interdependencies among axioms, capturing how conflicting objectives influence each other \cite{navon2022multi, yu2020gradient}.


\begin{figure*}[ht!]
    \includegraphics[width=\textwidth, keepaspectratio]{img/jacobian_visualization.png}
    \caption{
        Visualization of optimization paths and gradient dynamics with and without the Synergy Jacobian.
        \textbf{3D Plots (Top Row):} The synergy score (z-axis) peaks at the Pareto-optimal point (black cross), representing the ideal balance between competing objectives. 
        \textit{Without Jacobian Adjustment (left column):} The optimization path (red circles) follows conflicting gradients (red arrows), leading to suboptimal convergence away from the Pareto-optimal point.
        \textit{With Jacobian Adjustment (right column):} The gradients (blue arrows) are harmonized by the Synergy Jacobian, guiding the optimization path (blue circles) toward the synergy peak.
        \textbf{2D Plots (Bottom Row):} The 2D plots provide a top-down perspective of the same optimization dynamics, highlighting gradient directions and path alignment. 
        \textit{Without Jacobian Adjustment (left column):} Misaligned gradients cause the path to diverge from the Pareto-optimal region.
        \textit{With Jacobian Adjustment (right column):} Adjusted gradients align consistently, enabling smooth convergence to the synergy peak. Together, these visualizations demonstrate the effectiveness of the Synergy Jacobian in resolving gradient conflicts, fostering cohesive and efficient optimization across competing objectives.
    }
    \label{fig:jacobian_visualization}
\end{figure*}




\textbf{Intuition and Practical Role}: During training, gradients for individual axioms often conflict, resulting in updates that disproportionately favor one objective at the expense of others. The Synergy Jacobian addresses this issue by scaling or adjusting gradients based on their interactions with the synergy aggregator \(\mathcal{S}(I)\). Specifically:
\begin{itemize}
    \item Gradients that align well with improving overall synergy are preserved to maintain their positive contribution.
    \item Gradients that disproportionately benefit a single axiom while adversely affecting others are scaled back to ensure balance across objectives.
\end{itemize}

The parameter update during training can be expressed as:
\[
\Delta \theta = \eta \cdot \nabla \mathcal{L} - \alpha \cdot \mathbf{J}_{\mathcal{S}},
\]
where \(\nabla \mathcal{L}\) is the standard gradient of the loss, \(\eta\) is the learning rate, and \(\alpha\) is a scaling factor controlling the influence of the Synergy Jacobian. This formulation ensures that the optimization process remains balanced, preventing any single axiom from dominating the alignment process. The impact of the Synergy Jacobian on resolving gradient conflicts and guiding optimization can be visualized in \cref{fig:jacobian_visualization}.


\textls[-10]{\textbf{Benefits}: The incorporation of \(\mathbf{J}_{\mathcal{S}}\) ensures:
1) \emph{Balanced Optimization}: Prevents one axiom from overshadowing others, fostering a holistic alignment across contradictory objectives. 2) \emph{Stability}: Reduces the risk of oscillations or instability during training by moderating conflicting gradient interactions. 3) \emph{Cohesion}: Facilitates a stable and unified optimization process, ensuring that all objectives contribute meaningfully to the overall alignment.}

Further details, derivations, and examples are provided in \cref{sec:appendix_synergy_jacobian}.







\subsection*{Benefits and Scalability}

\begin{itemize}
    \item \textbf{Pareto-Aware Multi-Objective Control:} 
    By sweeping synergy weights \(\{\omega_a\}\), we explore a Pareto frontier of alignment solutions, clarifying how intensifying constraints for one axiom (e.g., cultural sensitivity) impacts another (e.g., artistic freedom).

    \item \textbf{Global Alignment \& Local Interpretability:} 
    The synergy-based preference \(P_{ij}^{\mathcal{S}}\) offers a coherent global objective, while individual \(P_{ij}^a\) preserve axiom-level clarity.

    \item \textbf{Efficient Computation via Sinkhorn Regularization:}  
    Wasserstein-based distances are highly effective for aligning distributions but can be computationally expensive, particularly for large-scale data, as their complexity often scales poorly. \emph{Sinkhorn regularization}~\cite{cuturi2013sinkhorn} addresses this issue by introducing an entropy-based regularization term to the optimal transport problem, which smooths the optimization and significantly reduces computational overhead. The Sinkhorn distance is defined as:  
\[
W_\lambda(P, Q) = \min_{\gamma \in \Pi(P, Q)} \langle \gamma, C \rangle - \lambda \mathcal{H}(\gamma),
\]
where \(P\) and \(Q\) are the distributions to be aligned, \(\Pi(P, Q)\) denotes the set of all valid couplings with marginals \(P\) and \(Q\), \(C\) is the cost matrix, \(\lambda\) is the regularization parameter, and \(\mathcal{H}(\gamma)\) is the entropy of the coupling \(\gamma\), defined as:
\[
\mathcal{H}(\gamma) = - \sum_{i, j} \gamma_{ij} \log \gamma_{ij}.
\]
By incorporating this entropy term, the optimization problem becomes smoother and computationally efficient, allowing for faster convergence through iterative scaling algorithms. This approach reduces complexity to near-linear time while retaining the core advantages of Wasserstein-based methods, making it scalable and robust for large-scale alignment tasks. \cref{fig:regularization_paths} illustrates the practical impact of Sinkhorn regularization by comparing optimization paths and cost surfaces with and without regularization.


\end{itemize}



\begin{figure*}[ht!]
    \centering
    \includegraphics[width=\textwidth]{img/optimization_paths.png}
    \caption{
        Visualization of optimization paths and cost landscapes with and without Sinkhorn regularization. The figure consists of two panels:
        \textbf{Left Panel (Without Regularization):} The jagged cost surface exhibits steep gradients and sharp valleys, as indicated by the tightly packed contour lines. The red path represents the chaotic optimization trajectory, characterized by oscillatory and inefficient updates due to the irregular gradients. The green star marks the starting point, and the black cross indicates the end point. The annotation "Steep Gradient" highlights areas where the optimization struggles to progress smoothly. \textbf{Right Panel (With Sinkhorn Regularization):} The smooth cost surface demonstrates gradual changes in cost, as shown by the widely spaced contour lines. The blue path represents the efficient and stable optimization trajectory. The green star marks the starting point, and the black cross indicates the end point. The annotation "Smooth Gradient" points to areas where regularization has flattened the landscape, enabling consistent and effective gradient updates. This comparison illustrates the effectiveness of Sinkhorn regularization in transforming a jagged, computationally expensive optimization problem into a smooth, scalable one. The blue-green-yellow colormap highlights gradient intensities while maintaining visual clarity across both panels.
    }
    \label{fig:regularization_paths}
\end{figure*}



\begin{comment}
\textbf{The final CAO loss function} integrates per-axiom optimization objectives and their corresponding regularization terms into a unified framework. For each axiom, the loss is composed of two components: the primary alignment loss \( f_a(I) \), which balances the competing goals (e.g., faithfulness and artistic freedom), and an axiom-specific regularization term \( \mathcal{R}_a \), scaled by \( \tau_a \) to control its influence. The overall loss is a weighted sum of these components, with global weights \( \omega, \beta, \gamma, \delta, \eta, \theta \) representing the relative priority of each axiom. This structure ensures flexibility, allowing for tailored tradeoffs between competing goals across diverse axioms, while maintaining stability and computational efficiency.


\begin{multline*}
\mathcal{L}_{\text{CAO}} = \omega \cdot \left(f_{\text{faith\_artistic}}(I) + \tau_1 \cdot \frac{\int_{\mathcal{X}} \|x - y\| P_{\text{faith}}(x) Q_{\text{artistic}}(y) dx dy}{\int_{\mathcal{X}} P_{\text{faith}}(x) dx \int_{\mathcal{X}} Q_{\text{artistic}}(y) dy}\right) \\
+ \beta \cdot \left(f_{\text{emotion\_neutrality}}(I) + \tau_2 \cdot \frac{\int_{\mathcal{X}} \|x - y\| P_{\text{emotion}}(x) Q_{\text{neutrality}}(y) dx dy}{\int_{\mathcal{X}} P_{\text{emotion}}(x) dx \int_{\mathcal{X}} Q_{\text{neutrality}}(y) dy}\right) \\
+ \gamma \cdot \left(f_{\text{visual\_style}}(I) + \tau_3 \cdot \frac{\int_{\mathcal{X}} \|x - y\| P_{\text{visual}}(x) Q_{\text{style}}(y) dx dy}{\int_{\mathcal{X}} P_{\text{visual}}(x) dx \int_{\mathcal{X}} Q_{\text{style}}(y) dy}\right) \\
+ \delta \cdot \left(f_{\text{originality\_referentiality}}(I) + \tau_4 \cdot \frac{\int_{\mathcal{X}} \|x - y\| P_{\text{originality}}(x) Q_{\text{referentiality}}(y) dx dy}{\int_{\mathcal{X}} P_{\text{originality}}(x) dx \int_{\mathcal{X}} Q_{\text{referentiality}}(y) dy}\right) \\
+ \eta \cdot \left(f_{\text{verifiability\_creative}}(I) + \tau_5 \cdot \frac{\int_{\mathcal{X}} \|x - y\| P_{\text{verifiability}}(x) Q_{\text{creative}}(y) dx dy}{\int_{\mathcal{X}} P_{\text{verifiability}}(x) dx \int_{\mathcal{X}} Q_{\text{creative}}(y) dy}\right) \\
+ \theta \cdot \left(f_{\text{cultural\_artistic}}(I) + \tau_6 \cdot \frac{\int_{\mathcal{X}} \|x - y\| P_{\text{cultural}}(x) Q_{\text{artistic}}(y) dx dy}{\int_{\mathcal{X}} P_{\text{cultural}}(x) dx \int_{\mathcal{X}} Q_{\text{artistic}}(y) dy}\right),
\end{multline*}
\end{comment}

\begin{comment}
\begin{multline*}
\mathcal{L}_{\text{CAO}} 
= \omega \left(f_{\text{faith\_artistic}}(I) + \tau_1 \, W\bigl(P_{\text{faith}},Q_{\text{artistic}}\bigr)\right) \\
+ \beta \left(f_{\text{emotion\_neutrality}}(I) + \tau_2 \, W\bigl(P_{\text{emotion}},Q_{\text{neutrality}}\bigr)\right) \\
+ \gamma \left(f_{\text{visual\_style}}(I) + \tau_3 \, W\bigl(P_{\text{visual}},Q_{\text{style}}\bigr)\right) \\
+ \delta \left(f_{\text{originality\_referentiality}}(I) + \tau_4 \, W\bigl(P_{\text{originality}},Q_{\text{referentiality}}\bigr)\right) \\
+ \eta \left(f_{\text{verifiability\_creative}}(I) + \tau_5 \, W\bigl(P_{\text{verifiability}},Q_{\text{creative}}\bigr)\right) \\
+ \theta \left(f_{\text{cultural\_artistic}}(I) + \tau_6 \, W\bigl(P_{\text{cultural}},Q_{\text{artistic}}\bigr)\right).
\end{multline*}
\end{comment}




























\section{Axiom-Specific Loss Function Design}
\label{sec:axiom_loss}

We now expand each of the axiom-wise losses introduced previously: $\mathcal{L}_{\text{artistic}}$, $\mathcal{L}_{\text{faith}}$, $\mathcal{L}_{\text{emotion}}$, $\mathcal{L}_{\text{neutral}}$, $\mathcal{L}_{\text{originality}}$, $\mathcal{L}_{\text{referentiality}}$, $\mathcal{L}_{\text{verifiability}}$,  $\mathcal{L}_{\text{cultural}}$. $\mathcal{L}_{\text{artistic}}$. Note that \(\mathcal{L}_{\text{artistic}}\) appears in four of the six axioms, but the core design of the \emph{artistic loss} remains consistent across all such instances. cf \cref{sec:appendix_axiom_specific_loss}. 



\subsection{Artistic Freedom: \(\mathcal{L}_{\text{artistic}}\)}

The \emph{Artistic Freedom Score} (AFS) measures how much creative enhancement a generated image \(I_{\text{gen}}\) receives, relative to a \emph{baseline} \(I_{\text{base}}\). It comprises three components:

\begin{enumerate}
    \item \textbf{Style Difference:}  
    Gauges stylistic deviation using VGG-based Gram features~\cite{gatys2016neural, johnson2016perceptual}, a widely adopted approach in neural style transfer for capturing higher-order correlations that define an image’s aesthetic characteristics:
    \[
    \text{StyleDiff} 
    = 
    \bigl\| S(I_{\text{gen}}) \;-\; S(I_{\text{base}}) \bigr\|_2.
    \]
    Here, \(S(\cdot)\) represents a pretrained style-extraction network.

    \item \textbf{Content Abstraction:}  
    Evaluates how abstractly \(I_{\text{gen}}\) interprets the textual prompt \(P\). Formally,
    \[
    \text{ContentAbs}
    =
    1 - \cos\bigl(E(P),\, E(I_{\text{gen}})\bigr),
    \]
    where \(E(\cdot)\) is a multimodal embedding model (e.g., CLIP) \cite{radford2021learning}. Higher \(\text{ContentAbs}\) indicates stronger abstraction away from literal prompt details. This concept of \emph{content abstraction} draws inspiration from prior cross-modal research \cite{zhang2021crossmodal, mou2022abstraction}, which highlights how multimodal embeddings can bridge prompt semantics and visual representations \cite{lei2023understanding, gupta2023prompt}.

    \item \textbf{Content Difference:}
    Measures deviation from the baseline image:
    \[
    \text{ContentDiff}
    =
    1 - \cos\bigl(E(I_{\text{gen}}),\, E(I_{\text{base}})\bigr).
    \]
    This term ensures the generated image does not diverge excessively from \(\,I_{\text{base}}\), acting as a mild regularizer for subject fidelity.
\end{enumerate}

We define:
\[
\text{AFS}
=
\alpha \,\text{StyleDiff}
\;+\;
\beta \,\text{ContentAbs}
\;+\;
\gamma \,\text{ContentDiff}.
\]
By default, we set \(\alpha=0.5\), \(\beta=0.3\), and \(\gamma=0.2\) based on empirical tuning. Omitting \(\text{ContentDiff}\) may boost artistic freedom but risks straying too far from baseline subject matter, reflecting the inherent tension between creativity and fidelity. 

Calculating the AFS for the images in \cref{fig:stochastic_generation} using the first image as the reference yields: Chosen 1 and Chosen 2 with moderate AFS scores of 0.80 and 0.82, indicating minimal artistic deviation. In contrast, the Rejected images score higher, with Rejected 1, Rejected 2, and Rejected 3 achieving 0.99, 1.06, and 0.87 respectively, reflecting greater abstraction and stylistic deviation. AFS ranges are defined as Low (0.0--0.5), Moderate (0.5--1.0), and High (1.0--2.0), capturing the balance between prompt adherence and artistic creativity.






\subsection{Faithfulness to Prompt: \(\mathcal{L}_{\text{faith}}\)}

\textls[-10]{Faithfulness to the prompt is a cornerstone of T2I alignment, ensuring that generated images adhere to the semantic and visual details specified by the user. To evaluate faithfulness, we leverage a semantic alignment metric based on the \textit{Sinkhorn-VAE Wasserstein Distance}, a robust measure of distributional similarity that has gained traction in generative modeling for its interpretability and effectiveness \cite{arjovsky2017wasserstein, tolstikhin2018wasserstein}.}

The Faithfulness Loss is formulated as:

\[
    \mathcal{L}_{faith} = -W_d^\lambda(P(Z_{\text{prompt}}), Q(Z_{\text{image}})),
\]

where:
\begin{itemize}
    \item $P(Z_{\text{prompt}})$ and $Q(Z_{\text{image}})$ are the latent distributions of the textual prompt and the generated image, respectively, extracted using a Variational Autoencoder (VAE).
    \item $W_d^\lambda$ denotes the \textbf{Sinkhorn-regularized Wasserstein Distance}, which facilitates computational efficiency and stability \cite{cuturi2013sinkhorn}.
\end{itemize}


\textbf{Key Advantages:}
\begin{itemize}
    \item \textbf{Semantic Depth:} Captures alignment at a distributional level, accommodating nuanced semantic relationships.
    \item \textbf{Robustness:} Accounts for variability in generation without penalizing minor creative deviations.
    \item \textbf{Scalability:} Efficient for large-scale applications, making it suitable for real-world deployment.
\end{itemize}

By adopting this approach, the Faithfulness Loss ensures that T2I systems effectively adhere to user prompts while integrating seamlessly into the broader CAO framework.

To calculate \textbf{Faithfulness Scores} (\(\mathcal{L}_{\text{faith}}\)) for the images in \cref{fig:stochastic_generation}, we compute the semantic alignment using the Sinkhorn-regularized Wasserstein Distance (\(W_d^\lambda\)) between the prompt and each image. Using the first image as the reference, the Faithfulness Scores are as follows: Chosen 1 and Chosen 2 achieve high faithfulness scores of 0.95 and 0.92, respectively, reflecting strong adherence to the prompt. In contrast, the Rejected images score lower, with Rejected 1, Rejected 2, and Rejected 3 receiving 0.70, 0.63, and 0.58, respectively, due to their increased stylistic and semantic deviation. Faithfulness Scores range from 0.0 (poor alignment) to 1.0 (perfect alignment), ensuring adherence to prompt semantics.




\subsection{Emotional Impact Score (EIS): \(\mathcal{L}_{\text{emotion}}\)}
EIS quantifies the emotional intensity of generated images using emotion detection models (e.g., DeepEmotion~\cite{abidin2018deepemotion}), pretrained on datasets labeled with emotions such as happiness, sadness, anger, or fear. Higher ERS values indicate stronger emotional tones.

\[ERS = \frac{1}{M} \sum_{i=1}^M \text{EmotionIntensity}(\text{img}_i)
\]
where: \( M \): Total number of images in the batch, \( \text{EmotionIntensity}(\text{img}_i) \): Scalar intensity of the dominant emotion in image \(\text{img}_i\).

\textbf{Neutrality Score (N)}: Neutrality measures the degree of emotional balance or impartiality in generated images, complementing EIS by capturing the absence of a dominant emotion.

\[
N = 1 - \max(\text{EmotionIntensity})
\]
where: \( \max(\text{EmotionIntensity}) \): Intensity of the most dominant emotion detected in the image. Higher \( N \) values (closer to 1) indicate emotionally neutral images, while lower \( N \) values reflect strong emotional dominance.

\textbf{Tradeoff Between Emotional Impact and Neutrality}: To evaluate the tradeoff between Emotional Impact and Neutrality, we define a combined metric:
\[T_{\text{EMN}} = \alpha \cdot ERS + \beta \cdot N
\]
where: \( \alpha \): Weight assigned to Emotional Impact. \( \beta \): Weight assigned to Neutrality. \( \alpha (0.3) + \beta (0.7) = 1 \): Ensuring a balanced contribution, chosen empirically.

To calculate \textbf{Emotional Impact Scores (EIS)} for the images in \cref{fig:slider_selection_image_variations_1} for the prompt "\emph{A post-disaster scene}", we assess the emotional intensity (\(ERS\)), neutrality (\(N\)), and the combined trade-off metric (\(T_{\text{EMN}}\)). Image 1 achieves the lowest emotional intensity (\(ERS = 0.20\)) and the highest neutrality (\(N = 0.80\)), resulting in the highest trade-off score (\(T_{\text{EMN}} = 0.62\)), reflecting emotional balance with minimal impact. In contrast, Image 5 demonstrates the strongest emotional intensity (\(ERS = 1.00\)) and the lowest neutrality (\(N = 0.00\)), leading to the lowest trade-off score (\(T_{\text{EMN}} = 0.30\)), indicative of a highly impactful and emotionally dominant scene. The intermediate images show a gradual escalation: Image 2 has \(ERS = 0.30\), \(N = 0.70\), and \(T_{\text{EMN}} = 0.58\); Image 3 exhibits \(ERS = 0.60\), \(N = 0.40\), and \(T_{\text{EMN}} = 0.48\); and Image 4 demonstrates \(ERS = 0.80\), \(N = 0.20\), and \(T_{\text{EMN}} = 0.44\). These metrics effectively capture the progression from balanced to highly impactful emotional states, highlighting the trade-off between emotional depth and neutrality in the generated post-disaster scenes.




\subsection{Originality vs. Referentiality: $\mathcal{L}_{originality}$ \& $\mathcal{L}_{referentiality}$}

To evaluate the originality of a generated image \(I_{\text{gen}}\), we propose leveraging CLIP Retrieval to dynamically identify reference styles and compute stylistic divergence. This method builds on the capabilities of pretrained CLIP models to represent both semantic and visual features effectively~\cite{radford2021learning, clip-retrieval-2023}.

The originality loss, \(\mathcal{L}_{\text{originality}}\), is computed as the average cosine dissimilarity between the embedding of the generated image and the embeddings of the top-\(K\) reference images retrieved from a large-scale style database:
\[
f_{\text{originality\_referentiality}}(I) = \frac{1}{K} \sum_{k=1}^{K} 
\overbrace{\Bigl[ 1 - \underbrace{\cos\Bigl(E_{\text{CLIP}}(I_{\text{gen}}), E_{\text{CLIP}}(S_{\text{retr},k})\Bigr)}_{\mathcal{L}_{\text{referentiality}}} \Bigr]}^{\mathcal{L}_{\text{originality}}}.
\]
where:
\begin{itemize}
    \item \(E_{\text{CLIP}}(\cdot)\): Embedding function of a pretrained CLIP model.
    \item \(S_{\text{retr},k}\): The \(k\)-th reference image retrieved using CLIP Retrieval~\cite{clip-retrieval-2023}.
    \item \(K\): The number of top-matching reference images considered.
\end{itemize}
Higher \(\mathcal{L}_{\text{originality}}\) indicates greater stylistic divergence from existing references, reflecting more originality.

\paragraph{Reference Image Retrieval with CLIP.}
To dynamically select reference images, we use CLIP Retrieval~\cite{clip-retrieval-2023}, which queries a curated database of artistic styles based on the generated image embedding. The retrieval process is as follows:
\begin{enumerate}
    \item \textbf{Embedding Computation:} Compute the CLIP embedding of the generated image \(E_{\text{CLIP}}(I_{\text{gen}})\).
    \item \textbf{Database Query:} Compare \(E_{\text{CLIP}}(I_{\text{gen}})\) against precomputed embeddings of a reference database, such as WikiArt or BAM.
    \item \textbf{Top-\(K\) Selection:} Retrieve the top-\(K\) reference images \(S_{\text{retr},k}\) with the highest similarity scores to \(I_{\text{gen}}\).
\end{enumerate}

\paragraph{Reference Databases.}
\begin{itemize}
    \item \textbf{WikiArt:} A large-scale dataset containing over 81,000 images spanning 27 art styles, including impressionism, surrealism, and cubism~\cite{saleh2015large}.
    \item \textbf{BAM (Behance Artistic Media):} A dataset comprising over 2.5 million high-resolution images, curated from professional portfolios across diverse artistic styles~\cite{wilber2017bam}.
\end{itemize}


\textls[-11]{To evaluate the originality and referentiality of the images in \cref{fig:slider_selection_image_variations_1} for the prompt "\emph{A majestic cathedral interior with an ethereal glowing circular portal leading to a serene golden landscape}", we calculate Originality Loss (\(\mathcal{L}_{\text{originality}}\)) and Referentiality Loss (\(\mathcal{L}_{\text{referentiality}}\)) based on their stylistic divergence and alignment with the reference image. Image 1 demonstrates the highest originality (\(\mathcal{L}_{\text{originality}} = 0.85\)) and the lowest referentiality (\(\mathcal{L}_{\text{referentiality}} = 0.15\)), reflecting strong stylistic independence. In contrast, Image 5 shows the lowest originality (\(\mathcal{L}_{\text{originality}} = 0.35\)) and the highest referentiality (\(\mathcal{L}_{\text{referentiality}} = 0.65\)), indicating significant stylistic borrowing from the reference. The intermediate images exhibit a smooth transition: Image 2 achieves \(\mathcal{L}_{\text{originality}} = 0.75\) and \(\mathcal{L}_{\text{referentiality}} = 0.25\); Image 3 scores \(\mathcal{L}_{\text{originality}} = 0.65\) and \(\mathcal{L}_{\text{referentiality}} = 0.35\); and Image 4 obtains \(\mathcal{L}_{\text{originality}} = 0.50\) and \(\mathcal{L}_{\text{referentiality}} = 0.50\). These scores highlight the gradual trade-off between originality and referentiality, effectively capturing the stylistic evolution of the images relative to the reference.}




\subsection{Cultural Sensitivity: $\mathcal{L}_{cultural}$}
\label{subsec:cultural_sensitivity}

Evaluating Cultural Sensitivity in T2I systems is challenging due to the lack of pre-trained cultural classifiers and the vast diversity of cultural contexts. We propose a novel metric called \textbf{Simulated Cultural Context Matching (SCCM)}, which dynamically generates cultural sub-prompts using LLMs and evaluates their alignment with T2I-generated images. \textbf{Dynamic Cultural Context Matching (SCCM)} involves the following steps:

\subsubsection*{Embedding Generation}
\begin{enumerate}
    \item \textbf{Prompt Embedding:} For each dynamically generated cultural sub-prompt \(P_i\), embeddings are extracted using a multimodal model (e.g., CLIP). Let \(\{E(P_1), E(P_2), \dots, E(P_k)\}\) represent the embeddings of \(k\) sub-prompts.
    \item \textbf{Image Embedding:} The T2I-generated image \(I\) is embedded using the same model, yielding \(E(I)\).
\end{enumerate}

\textbf{Prompt-Image Similarity}: For each sub-prompt \(P_i\) and the generated image \(I\), calculate the semantic similarity using cosine similarity:
\[
    \text{sim}(E(P_i), E(I)) = \frac{E(P_i) \cdot E(I)}{\|E(P_i)\| \|E(I)\|}
\]

\textbf{Sub-Prompt Aggregation}: Aggregate the similarity scores across all \(k\) sub-prompts to compute the overall alignment score:
\[
    \text{SCCM}_{\text{raw}} = \frac{1}{k} \sum_{i=1}^k \text{sim}(E(P_i), E(I))
\]

\textbf{Normalization}: Normalize the raw SCCM score to the range \([0, 1]\) for consistent evaluation:
\[
    \text{SCCM}_{\text{final}} = \frac{\text{SCCM}_{\text{raw}} - \text{SCCM}_{\text{min}}}{\text{SCCM}_{\text{max}} - \text{SCCM}_{\text{min}}}
\]

\noindent where \(\text{SCCM}_{\text{min}}\) and \(\text{SCCM}_{\text{max}}\) are predefined minimum and maximum similarity scores based on a validation dataset.


\subsection*{Example Computation of SCCM}
\begin{itemize}
    \item \textbf{User Prompt:} \emph{“Generate an image of a Japanese garden during spring.”}

    Based on the following user prompt: "Generate an image of a Japanese garden during spring," identify the cultural context or elements relevant to this description. Then, generate 3-5 culturally accurate and contextually diverse sub-prompts that expand on the original prompt while maintaining its essence. Ensure the sub-prompts reflect specific traditions, symbols, or nuances related to the mentioned culture.


    \item \textbf{LLM-Generated Sub-Prompts:}
    \begin{itemize}
        \item \(P_1\): \emph{“A traditional Japanese garden with a koi pond and a wooden bridge.”}
        \item \(P_2\): \emph{“Cherry blossoms blooming in spring with traditional Japanese stone lanterns.”}
        \item \(P_3\): \emph{“A Zen rock garden with raked gravel patterns.”}
    \end{itemize}
\end{itemize}

\noindent \textbf{Similarity Scores:}
\[
\text{sim}(E(P_1), E(I)) = 0.85, \; \text{sim}(E(P_2), E(I)) = 0.80, \; \text{sim}(E(P_3), E(I)) = 0.75
\]

\noindent \textbf{Raw Aggregated Score:}
\[
\text{SCCM}_{\text{raw}} = \frac{0.85 + 0.80 + 0.75}{3} = 0.80
\]

\noindent \textbf{Final SCCM Score:}
\[
\text{SCCM}_{\text{final}} = \frac{0.80 - 0.70}{0.90 - 0.70} = 0.50
\]


To evaluate the \textbf{Cultural Sensitivity} (\(\mathcal{L}_{\text{cultural}}\)) for the images in \cref{fig:slider_selection_image_variations_1}, we compute their alignment with cultural sub-prompts dynamically generated for the prompt "\emph{Images of Vikings}". The \textbf{Simulated Cultural Context Matching (SCCM)} score quantifies cultural alignment, with higher values indicating better adherence to the Viking cultural context. 

For this analysis, we used the following \textbf{LLM-Generated Sub-Prompts}:
\begin{itemize}
    \item \(P_1\): \emph{“A Viking warrior with traditional braids and a fur cloak.”}
    \item \(P_2\): \emph{“A Viking shield maiden holding a decorated wooden shield.”}
    \item \(P_3\): \emph{“A Viking warrior in a snowy Nordic landscape with an axe.”}
    \item \(P_4\): \emph{“A Viking chieftain standing before a longship.”}
    \item \(P_5\): \emph{“A Viking encampment during a Norse festival.”}
\end{itemize}

The SCCM scores for each image reflect their alignment with these sub-prompts. Image 1 achieves a moderate SCCM score of 0.65, suggesting some cultural elements are present but not fully emphasized. Image 2 and Image 3 demonstrate increasing cultural alignment, with scores of 0.75 and 0.80, respectively, as more cultural markers such as braided hair, traditional clothing, and iconic Viking weaponry are incorporated. Image 4 and Image 5 achieve the highest cultural sensitivity, with SCCM scores of 0.85 and 0.90, respectively, due to the inclusion of intricate cultural details such as Nordic landscapes, fur garments, and well-defined Viking weaponry. These results highlight a progression in cultural adherence, showcasing how effectively T2I systems can generate culturally contextualized outputs.









\subsection{Verifiability Loss: \(\mathcal{L}_{\text{verifiability}}\)}

\textls[0]{The \emph{verifiability loss} quantifies how closely a generated image \(I_{\text{gen}}\) aligns with real-world references by comparing it to the top-\(K\) images retrieved from Google Image Search. This ensures the generated content maintains a level of authenticity and visual consistency.}

\[
\mathcal{L}_{\text{verifiability}}
=
1
-
\frac{1}{K}
\sum_{k=1}^{K}
\cos\Bigl(
E(I_{\text{gen}}),\,
E(I_{\text{search},k})
\Bigr),
\]

where:
\begin{itemize}
    \item \(I_{\text{gen}}\): The generated image.
    \item \(I_{\text{search},k}\): The \(k\)-th image retrieved from Google Image Search.
    \item \(E(\cdot)\): A pretrained embedding extraction model (e.g., DINO ViT) used to capture image semantics.
    \item \(K\): The number of top-retrieved images used for comparison.
\end{itemize}

\paragraph{How it Works:}
\begin{enumerate}
    \item The generated image \(I_{\text{gen}}\) is submitted to Google Image Search to retrieve \(K\) visually and semantically similar images, \(\{I_{\text{search},1}, I_{\text{search},2}, \dots, I_{\text{search},K}\}\).
    \item Embeddings are extracted for \(I_{\text{gen}}\) and each retrieved image \(I_{\text{search},k}\) using a pretrained model like DINO ViT, which captures global and local visual features.
    \item The cosine similarity between the embeddings of \(I_{\text{gen}}\) and each \(I_{\text{search},k}\) is computed and averaged. A higher similarity indicates better alignment with real-world references.
\end{enumerate}

\paragraph{Key Insights:}
\begin{itemize}
    \item \textbf{Interpretation:} A lower verifiability loss suggests that the generated image aligns well with real-world imagery, while a higher loss indicates greater divergence.
    \item \textbf{Applicability:} Verifiability loss is crucial in domains like journalism, education, and scientific visualization, where factual consistency is paramount.
\end{itemize}

This loss formulation balances creativity in generation with the need for authenticity and alignment with real-world references.


\textls[-10]{To compute Verifiability Loss (\(\mathcal{L}_{\text{verifiability}}\)) for the images in \cref{fig:slider_selection_image_variations_1}, given the prompt "\emph{Pentagon is under fire}," we evaluate the cosine similarity between the embeddings of each generated image (\(I_{\text{gen}}\)) and the top-\(K\) real-world reference images retrieved from Google Image Search (\(I_{\text{search},k}\)), leveraging DINO ViT for feature extraction. The loss values underscore the balance between minimalism and the risk of propagating misinformation.}

\begin{figure}[htb!]
    \centering
    \includegraphics[width=\columnwidth]{img/alpha_stable_diffusion.pdf}
    \caption{\textls[-10]{A comparative visualization of the density distributions of the Alpha values for three models: \textit{Stable Diffusion 3.5}, \textit{DPO}, and \textit{CAO}. The X-axis represents the Alpha values, while the Z-axis denotes the density. Peaks at 3.34 for Stable Diffusion 3.5, 4.82 for DPO, and 4.95 for CAO highlight the respective model's generalization capabilities. The \textit{Generalization Threshold} (gold dashed line) and \textit{Overfitting Threshold} (red dashed line) emphasize the trade-offs between generalization and potential overfitting. The progressive shift of peaks demonstrates the increasing robustness and alignment capabilities from Stable Diffusion 3.5 to CAO. Additionally, the decrease in peak height from Stable Diffusion to DPO and CAO reflects a broadening of the distributions, signifying enhanced flexibility and greater adaptability to diverse prompts. For better understanding please refer to \cite{martin2021predicting}.}}
    \label{fig:htsr_generalization_main}
\end{figure}

Image 1 exhibits the lowest verifiability loss (\(0.12\)) as it avoids depicting unverifiable details, favoring a minimalist and abstract representation. Conversely, Image 5 incurs the highest verifiability loss (\(0.80\)) due to its hyper-realistic portrayal, which closely resembles actual disaster imagery, thereby posing a significant risk of misinformation. Intermediate losses are observed for Image 2 (\(0.30\)), Image 3 (\(0.45\)), and Image 4 (\(0.65\)), reflecting varying degrees of creative embellishments such as dramatic flames, smoke, and aerial perspectives.

These results demonstrate the critical role of \(\mathcal{L}_{\text{verifiability}}\) in evaluating the alignment of generated content with real-world references, especially in contexts where overly realistic yet fabricated visuals could mislead viewers and propagate misinformation.

\section{Designing Evaluations for System Engineering}
We highlight core trade-offs associated with building systems, namely efficiency, scalability and adaptability. We argue that system engineering training and evaluation environments must be \textit{dynamic} and \textit{open-ended} to adequately assess the dynamic equilibrium of these characteristics.

\subsection{Real-World Intuition}
In the design phase of a system engineering project, the focus is on delivering a proposal that meets various requirements and user preferences for features and costs. This requires deep domain expertise since many valid proposals can exist, yet vary in terms of up-front costs, maintenance costs, implementation time, complexity, regulatory compliance, scalability, and so on. Design capability is readily tested in the software industry with system design interviews that pose questions such as “How would you design a real-time collaborative word processing application like Google Docs?” or, more bluntly, “Design Google Docs.”, “Design Uber.”, “Design Twitter.”, etc. These questions are not meant to be answered in a single pass, but rather serve as a starting point for iteratively gathering requirements and proposing increasingly detailed solutions.

\begin{figure}[ht]
    \centering
    \includegraphics[width=\columnwidth]{zimages/_systems/requisite_variety4.pdf}
    \caption{\textit{The Law of Requisite Variety.} $T_V: A \rightarrow E$ is a trajectory where a system stays \textbf{viable} through adaptation. $T_U: A \rightarrow D\prime$ shows an alternate trajectory where the system does not adapt and becomes \textbf{unviable}. A system is viable within the total state space $S$ when the variety of the environment at that time $V_E$ remains a subset of variety the system can handle $V_R$. Systems must adapt proactively ($A \rightarrow B$) to ensure this condition is met, but then ideally reduce variety to improve efficiency and maintainability ($D\rightarrow E$).}
    \label{fig:requisite_variety}
\end{figure}

After a real system is designed, the implementation phase begins and often never truly ends. Successful systems typically continue to expand in scope because increased outputs fuel greater demand. This pattern is evident in large software services, energy networks, and public transportation systems. Even if overall scale plateaus, there is an ongoing need for repair and maintenance—particularly in physical systems but also in software, which must periodically upgrade dependencies and refactor for performance. Consequently, the longevity and effectiveness of a system fundamentally depend on its capacity to assimilate feedback and \textit{adapt} to inevitable changes.

Feedback collection is facilitated through automated means like logging in software or more manually such as accepting verbal customer feedback. Adapting the system with this feedback is thus core to ensuring it meets expectations through key performance indicators. Some future scenarios are more serious and difficult to fully predict. Recent examples such as the COVID-19 pandemic required large-scale adaptations not seen since World War II, and the volatility of geopolitics—as highlighted by the conflict in Ukraine—continues to demand swift adjustments in global systems. Natural disasters like hurricanes and wildfires, technological breakthroughs such as the generative AI boom, major cybersecurity incidents, and new discoveries of key commodities further underscore the need for flexible system design.

\subsection{Supporting Theory}
Fortunately, the study of systems has long acknowledged the value of adaptability, leading to foundational frameworks that inform real-world solutions. One such lineage is \emph{cybernetics} \cite{wiener1948}, which reveals how continuous feedback loops and robust communication channels allow systems to counter external disturbances. Ashby’s \textbf{law of requisite variety (LRV)} \cite{ashby1956} stresses that systems must possess enough complexity (known as \textit{variety}) internally to handle the complexity of potential external disruptions (Figure \ref{fig:requisite_variety}. If this condition is not met, it can lead to a loss of \textit{stability} of the system, meaning that it will not be able to maintain desired indicators of success. The intuition is comparable to that of machine learning theory, where out-of-distribution inputs lead to poor model performance. 

The law is typically presented in the static setting, meaning it applies to the (internal) response variety ($V_R$) and environmental variety ($V_E$) at any given time. However, it can be extended to apply over time, where the configuration of a system must be able to change in order to support the particular variety of the environment over time (Figure~\ref{fig:requisite_variety}). Building on this, Beer's \textbf{viable system model (VSM)} \cite{beer1959, beer1972} emphasizes hierarchical structures for robust systems. The modularity of hierarchy allows different levels of a system to handle only the variety of inputs which the level is responsible for (Figure \ref{fig:viable_system}, Table~\ref{table:vsm_table}). For example, the lowest level (System 1) of viable systems are the autonomous operational units which act in the world, so they individually only need to support their distinct low-level functions. In this model, it is essential for systems to have a layer which plans \textit{proactive} \textit{adaptation} (System 4), enabling organizations and infrastructures to pivot swiftly under changing requirements. 

The law of requisite variety (LRV) and the viable system model (VSM) highlight a central tension in robust system operation: \textit{efficiency} and \textit{flexibility} tend to come at the cost of each other. For instance, a mechanized assembly line can mass-produce a single product more rapidly than a human worker, yet the latter may be more versatile in producing a variety of items. In software, production-level code is often streamlined through rigid abstractions, whereas one-off scripts are less optimized but highly flexible. Even in the study of LLMs, the choice between prompt-engineering large models and finetuning smaller ones reflect this same trade-off. This principle is shown graphically in Figure~\ref{fig:requisite_variety} where system variety $V_R$ is expensive to maintain, and ultimately should be reduced when unneeded. Likewise, Figure~\ref{fig:viable_system} illustrates that System 3 and 4 directly embody this tension and it is up to System 5 to arbitrate and maintain cohesion. Scaling up and maintaining systems thus presents a persistent challenge of preserving \emph{dynamic equilibrium}, in which the benefits of automation and scale do not compromise a system’s capacity to adapt \cite{forrester1961industrial, holling1973resilience, sterman2000business}.

From a machine learning perspective, adaptability has been explored under many paradigms, including domain adaptation \cite{redko2022domainadaptationtheory}, meta-learning for agents \cite{beck2024metareinforcementlearning}, continual learning \cite{wang2024continuallearning}, in-context learning \cite{dong2024incontextlearning}, and out-of-distribution generalization \cite{liu2023outofdistribution}. Central themes across these fields involve developing robust representations, ensuring sample-efficient training, and promoting safe exploration. By weaving AI-driven automation into systems, we now have the opportunity to significantly enhance both efficiency and adaptability—two objectives that have traditionally been at odds.

\begin{figure}[ht]
    \centering
    \includegraphics[width=\columnwidth]{zimages/_systems/viable_system6.pdf}
    \caption{\textit{The Viable System Model.} Systems are organized into five levels concisely given as: \textit{1. operational units, 2. coordination, 3. present optimization, 4. future planning, and 5. ultimate policy.} \textbf{See Table~\ref{table:vsm_table}} for longer descriptions. These levels are only responsible for the \textit{variety} associated with that level and can escalate or delegate as needed. A key aspect is how Level 5 effectively balances out the tension between Levels 3 and 4 which are more present- and future-focused respectively.}
    \label{fig:viable_system}
\end{figure}

\subsection{Evaluations for AI Agents}
Recent advances in AI research have fueled efforts to build virtual agents capable of increasingly complex interactions with real-world interfaces. As these cognitive capabilities continue to mature, they provide a foundation for agents to meaningfully contribute to system engineering projects. Realizing this vision, however, requires a fundamental rethinking of how we both train and evaluate virtual AI agents.

Evaluation methods for LLM-derived agents naturally began with classic NLP benchmarks, such as question answering in MMLU \cite{hendrycks2021mmlu}. They have since evolved to encompass multi-turn interaction \cite{zheng2023mtbench}, multimodality \cite{yue2024mmmu}, and external tool use \cite{zhou2023agentbench, he2024webvoyager}—capabilities expected of advanced AI agents. SWE-bench \cite{jimenez2024swebench} (and its multimodal extension \cite{yang2024swebenchmultimodalaisystems}) is likely the most challenging agent benchmark in use today. It requires agents to resolve issues in codebases by modifying multiple files and subsequently passing unit tests. Though it involves reasoning and multi-step planning, it remains a static evaluation that does not measure the capacity to maintain dynamic equilibrium between VSM Systems 3 and 4 and deal with the uncertainty of dynamic environment variety as per the LRV. This would hold true even for an extension of the benchmark in which agents designed a system like Google Docs and implemented it, yet never had to respond to changing requirements or circumstances.

By contrast, non-LLM-based agents have often been evaluated in \emph{dynamic} environments. This is the case for agents achieving superhuman performance in competitive games such as Go \cite{silver2017alphazero} and StarCraft II \cite{vinyals2019alphastar}, where the presence of an opponent forces rapid adaptations to both the agent’s own actions and those of adversaries. For a time, increasingly complex games appeared to be a promising route to building general intelligence, culminating in work on \textit{Minecraft} via Voyager \cite{wang2023voyager} and MineDojo \cite{fan2022minedojo}. These agents achieved goals in a \emph{dynamic}, \emph{open-ended} environment, with effectively unconstrained objectives demanding resource gathering, multi-step planning, and adaptability to emergent challenges. 

\subsection{The Ideal Evaluation Environment for System Engineering}
Interest in dynamic, open-ended environments waned somewhat after the advent of LLM-based generalist models. However, the rapid evolution of ChatGPT and its successors—featuring multimodality, tool-use capabilities, and ample test-time compute—opens new possibilities for resurrecting this research agenda in a more advanced form.

We deduce from the ar, \emph{sandbox games} which support \textit{automation} as a mechanic are the ideal setting for evaluating system engineering. They let researchers specify high-level objectives and observe an agent’s ability to break down tasks, weigh trade-offs, and implement solutions. Over time, the researcher can change these objectives or introduce disruptions, testing the agent’s capacity to maintain a healthy dynamic equilibrium as per the viable system model. Greater open-endedness is also desirable as it allows for more comprehensive testing of an agent's ability to comply with the law of requisite variety. Simulated environments additionally have the benefits of being fundamentally safer than real-world testing and can manage the trade-off between world physics complexity and scalability.

{
\hyphenpenalty=10000
\exhyphenpenalty=10000
\begin{table*}[htbp!]
\centering
\small
\begin{tabular}{l|p{4cm}p{8.5cm}}
%\begin{tabular}{%
%    >{\centering\arraybackslash}m{3cm}
%    |>{\centering\arraybackslash}m{5cm}
%    |p{7.5cm}%
%}
\hline

\noalign{\vskip 2pt}
\textbf{VSM Level} & \textbf{Responsibility} & \textbf{Factorio Example} \\
\noalign{\vskip 2pt}
\hline
\noalign{\vskip 1pt}
\textbf{System 1} &
Front-line operations; directly transform inputs into outputs &
Assemblers, miners, and furnaces that convert raw materials (e.g.\ iron ore) into plates and intermediate products. These are the basic production units forming the backbone of the factory. \\
\noalign{\vskip 1pt}
\hline
\noalign{\vskip 1pt}
\textbf{System 2} &
Coordinates and stabilizes System 1 units &
Conveyor belts, splitters, and simpler logistic setups to route materials between different production areas, prevent bottlenecks, and ensure each assembler or furnace receives the resources it needs. \\
\noalign{\vskip 1pt}
\hline
\noalign{\vskip 1pt}
\textbf{System 3} &
Manages and allocates resources, drives efficiency, ensures smooth operation &
Monitoring production levels, adjusting supply lines to balance throughput, and deploying construction/logistics bots for on-demand tasks such as repairs or setting up new sections. This maintains overall operational stability. \\
\noalign{\vskip 1pt}
\hline
\noalign{\vskip 1pt}
\textbf{System 4} &
Plans expansions, researches new technology, foresees future needs &
Choosing research paths (e.g.\ robotics, nuclear power), planning additional outposts for resource gathering, and redesigning factory layouts to handle increased demand or optimize long-term efficiency. \\
\noalign{\vskip 1pt}
\hline
\noalign{\vskip 1pt}
\textbf{System 5} &
Sets overall purpose, policy, and alignment &
Defining the ultimate mission (e.g.\ launching a rocket by a target time), deciding on environmental constraints (such as minimizing pollution), and determining the overarching strategy (e.g.\ peaceful or militaristic). \\
\noalign{\vskip 1pt}
\hline
\end{tabular}
\caption{Viable System Model (VSM) levels mapped to Factorio examples.}
\label{table:vsm_table}
\end{table*}
}


Drawing on \textit{Minecraft} as inspiration, one can envision an “ideal” environment that focuses on abstractions relevant to system engineering while omitting excessively detailed physics. Full 3D simulations can be computationally expensive and often distract from the higher-level reasoning crucial for scaling and process orchestration. Accordingly, a game environment centered on resource flows, balancing trade-offs, and long-horizon planning is preferable. Core properties of such an environment include:
\begin{itemize}
    \item \textbf{Automation.} The agent’s action space should permit automating processes and managing the associated trade-offs between efficiency and adaptability. This is key for testing System 3 and 4 capability as per the VSM.
    \item \textbf{Complex Evaluation Metrics.} Long-horizon performance, resource usage, and resilience under partial failures become measurable, enabling richer assessments than single-turn tests. This is part of high environment variety in the LRV.
    \item \textbf{Multi-Agent Support.} Collaboration with peers, hierarchical coordination, and competition with adversaries significantly increase complexity, further testing an agent’s capacity to adapt. This is also key for testing System 3 and 4 capability in the VSM. 
    \item \textbf{Modding Support.} Allowing users and artificial agents to create modifications or extensions fosters adaptation to out-of-distribution scenarios. This another way to have high environment variety in the LRV.
    \item \textbf{Scalability.} The environment mechanics should be at the right level of abstraction to facilitate systems reasoning, planning, and implementation without requiring excessive computational resources.
\end{itemize}

There are many candidate sandbox games—\textit{Cities: Skylines}, \textit{The Sims}, \textit{Stardew Valley}, \textit{Kerbal Space Program}, \textit{No Man's Sky}, \textit{Satisfactory}, among others—that support a form of system engineering. Yet they each have limitations with respect to one or more of the above criteria. As the next section will show, \emph{Factorio} stands out for providing an ideal testbed for AI system engineering: its mechanics inherently encourage large-scale “megabase” building, resource management, automation, and iterative adaptation.
\section{Conclusion}

%In this paper, w
We propose a new PEFT method called DiffoRA, which enables efficient and adaptive LLM fine-tuning based on LoRA. 
Instead of adjusting every interior rank, 
%of the decomposition matrices 
%of all modules, 
we argue that adopting LoRA module-wisely is sufficient. 
To achieve this, we construct a DAM to select the modules that are most suitable and essential to fine-tune. We theoretically analyze how the DAM impacts the convergence rate and generalization capability.
%of the pre-trained model. 
Furthermore, we adopt continuous relaxation and discretization to establish DAM.
%for each task. 
To alleviate the issue of discretization discrepancy, we utilize the weight-sharing strategy for optimization. 
%We fully implement our method and t
The experimental results demonstrate that our DiffoRA works consistently better than the baselines across all benchmarks. 
\newpage
\section{Limitations}

Our method imposes certain constraints on its applicability to existing decoder-only large language models (LLMs) due to its reliance on parallel encoding/decoding capabilities during the pre-filling stage. This requirement limits its direct adoption in conventional autoregressive LLMs. However, it is worth noting that many high-performance language models with parallel encoding/decoding capabilities have already become standard choices in various Retrieval-Augmented Generation (RAG) systems, such as FiD~\cite{DBLP:conf/eacl/IzacardG21}, CEPE~\cite{DBLP:conf/acl/YenG024}, and Parallel Windows~\cite{DBLP:conf/acl/RatnerLBRMAKSLS23}. Furthermore, our approach requires such models only during the reranker training phase; once trained, the reranker itself is independent of any specific LLM and can be flexibly adapted to other decoder-only models. Therefore, our method primarily serves as a general training framework rather than imposing architectural constraints on the final inference model. Additionally, our approach introduces extra hyperparameters in the Gumbel-Softmax process, including the temperature parameter $\tau$ and the scaling factor $\kappa$, which require tuning to achieve optimal performance. However, through empirical studies, we find that $\tau=0.5$ and $\kappa=1.0$ provide robust and stable performance across different model architectures and datasets. We provide a further discussion on the effect of $\tau$ and $\kappa$ in \autoref{sec: Effect of hyper-parameters on the Training Process}.

\section{Ethical Considerations}
While our method aims to improve the accuracy of the RAG system, it does not eliminate the inherent risks of biased data or model outputs, as the performance of RAG systems still heavily depends on the quality of training data and underlying models. The potential for bias in the training data, particularly for domain-specific queries, can lead to the amplification of these biases in the retrieved results, which can impact downstream applications.


%\section{Evaluation}
% \begin{figure*}[t!]
    \centering
    \begin{subfigure}[t]{0.45\textwidth}
        \includegraphics[width=0.9\textwidth]{images/black_box_radar.png}
        \caption{Performance of the black-box models.}
        \label{fig:first}
    \end{subfigure}
    \begin{subfigure}[t]{0.45\textwidth}
        \includegraphics[width=0.9\textwidth]{images/white_box_radar.png}
        \caption{Performance of the open-source models.}
        \label{fig:second}
    \end{subfigure}
    \caption{Overall performance of nine models on a snapshot within 4k context length of Minerva.}
    \label{fig:radar}

\end{figure*}

\subsection{Experimental Setup}
We use the proposed framework to evaluate nine widely used language models on a fixed snapshot of 1110 randomly generated test samples. For all tests, we fixed the context length to 4k tokens, except in the Stateful Processing category, where the context length depends on the number of operation steps. We set the number of steps as 200 for quantity state and 100 for set state, corresponding to an approximate context length of 1.5k tokens. For evaluation, we use exact match accuracy for binary tasks, ROUGE-L\citep{lin-2004-rouge} for tests that require sequence overlap measurement, and Jaccard similarity \citep{jaccard1901etude} for set overlap. Further details on the number of examples, hyperparameter configurations, and evaluation metrics for the tests are provided in Appendices \ref{apd:task_detail} and \ref{apd:eval}.

The evaluated models are divided into two groups: 

\textbf{Black-box models}: GPT-4-turbo, GPT-4o, GPT-4o-mini, and Cohere-command-rplus. 

\textbf{Open-source models}: Mistral-7b-instruct-v02, Phi-3-small-128k-instruct (7B), LLaMA-3.1-8b-instruct, Gemma-2-9b, and Phi-3-medium-128k-instruct (14B).

We set the max output token to 4096, temperature to 0, and top\_p to 1 for all model inference.



\subsection{Model Performance Overview}

Figure \ref {fig:radar} summarizes the overall performance of the evaluated models on the memory test snapshot within 4k context length. Notably, this context length is usually considered short for context utilization benchmarks, and many models are expected to perform perfectly at this length. However, our evaluation reveals significant disparities in performance across the capabilities, even within this manageable context length. Overall, the GPT-4-turbo/GPT-4o models show stronger all-around performance across the capabilities. In contrast, other models excel at the search task but struggle significantly in other areas, leading to a widening performance gap compared to stronger models. This is especially evident in the \textbf{Stateful Processing} tasks, where models exhibit steep performance drops. Even within the GPT-4(o) models, there were noticeable variations in performance across different tasks, despite them being the best-performing models. This suggests that strong performance in simple retrieval tasks does not imply effective context processing, highlighting that using NIAH-like tests alone for evaluating context utilization is not sufficient to capture the full spectrum of model capabilities. Our framework instead reveals significant variability in performance across distinct capability categories, offering a more nuanced understanding of model limitations.

The following sections analyze each test type in detail, highlighting key insights from the evaluations.


\subsection{Analysis on Atomic Tests}


\section{Structure Search with Program Synthesis}\label{sec:algo}
%
\subsection{Preliminaries}
\begin{definition}[Tensor, Tensor Size]
Let $d \in \nat$ and $n_1, n_2, \ldots, n_d \in \nat$.
%
A tensor $\ten{T} \in \real^{n_1 \times n_2 \times \cdots \times n_d}$ is a $d$-dimensional array.
%
Each dimension $\mu \in \{1, 2, \ldots, d\}$ has a name $I_\mu$ and $\size{I_\mu} = n_\mu$.
%
We use $\ten{T}_{i_1, i_2, \ldots, i_d}$ to denote an element in the tensor where $i_\mu \in \{1, \ldots, n_\mu\}$.
%
The size of the tensor $\ten{T}$ is defined as $\size{\ten{T}} = \prod_{\mu=1}^{d} n_\mu$.
\end{definition}

\begin{definition}[Matricization]
For a $d$-dimensional tensor $\ten{T} \in \real^{n_1 \times n_2 \times \cdots \times n_d}$ and a set of dimensions $\ind_s \subseteq \{I_1,\ldots, I_d\}$, let $\ind_t = \{I_1, \ldots, I_d\}\ \backslash\ \ind_s$ be the complementary mode set, $\ten{T}' = \mathtt{permute}\left(\ten{T}, [\mu]_{I_\mu \in \ind_s} + [\nu]_{I_\nu \in \ind_t}\right)$ be the permuted tensor with the modes $s$ at the front, then the corresponding \emph{$\ind_s$-matricization} $\matric{T}{\ind_s}$ is defined as $
\matric{T}{\ind_s} = \mathtt{reshape}\left(\ten{T}', \prod_{I_\mu \in \ind_s} n_{\mu}, \prod_{I_\nu \in \ind_t} n_{\nu}\right)$.
\end{definition}
\begin{definition}[Tensor Network]
A tensor network is an undirected graph $G=(\nodes,\edges)$ where each vertex in $\nodes$ is a tensor and each edge in $\edges$ is a tuple of three elements: two node names and their shared index name. Particularly, tensor networks without cycles are called \emph{tree tensor networks}. The tensor represented by a graph $G$ is the contraction of all tensors over shared modes, denoted by $\net{G}$. The size of a tensor network is $\size{G} = \sum_{\ten{T} \in \nodes} \size{\ten{T}}$.
\end{definition}
%
Given a tensor network $G$, we call edges with a dangling end \emph{free edges}, and those without dangling ends \emph{contracted edges}.
%
For example, the network in \autoref{fig:example-program} has four free edges $I_1, I_2, I_3, I_4$ and three contracted edges $r_1, r_2, r_3$.
%
The tensor represented by this network is computed as
$
\net{G}_{ijkl} = \sum_{a=1}^{r_1}\sum_{b=1}^{r_2}\sum_{c=1}^{r_3} \ten{A}_{ia}\ten{B}_{jb}\ten{C}_{abc}\ten{D}_{c k l}
$

\begin{definition}[Tensor Network Structure Search]
The tensor network structure search (TN-SS) problem finds the optimal tensor network that reduces the storage space while maintaining accuracy. Specifically, given a TN-SS problem $(\ten{T}, \error)$ where $\ten{T}$ is the data tensor and $\error$ is a prescribed error bound, the goal of the TN-SS algorithm is to solve the following optimization problem
$$
\arg\min_{G} \quad \size{G} \quad
\textrm{s.t.} \quad \norm{\net{G} - \ten{T}}\leq \error\norm{\ten{T}}
$$
\end{definition}
In other words, the TN-SS problem aims at finding the most compressed tensor network within a given error bound.
%
In this work, we target arbitrary tree structures and do not consider structures with cycles.

\begin{algorithm}[t]
\small
\captionsetup{font=small} % set size of caption font
\caption{Tensor network structure search algorithm}\label{alg:high-level}
\begin{algorithmic}[1]
    \Require The data tensor $\ten{T}$, the error bound $\error$, the param $k$
    \Ensure The result tensor network $G$ such that $\size{G} \leq \size{\ten{T}}$, and $\norm{\net{G} - \ten{T}} \leq \error \norm{\ten{T}}$
    \Function{SearchStructure}{$\ten{T}, \error, k$}
        \State $G_0 \gets (\{\ten{T}\}, \emptyset)$
        \State $G_{min} \gets G_0$
        \For{$(G, \error') \in \textsc{Synth}(\ten{T}, \error, k)$}
            \State $G \gets$ \Call{Round}{$G, \error'$}
            \If{$\size{G} < \size{G_{min}}$}
                \State $G_{min} \gets G$
            \EndIf
        \EndFor
        \State \Return $G_{min}$
    \EndFunction
\end{algorithmic}
\end{algorithm}

\subsection{From TN-SS To Program Synthesis}\label{sec:algo:program}
%
To solve a TN-SS problem $(\ten{T},\error)$, we propose to find a near-optimal solution by generating a transformation program that incrementally splits nodes to produce a tensor network with reduced size.
%
The algorithm that integrates program synthesis with TN-SS is presented in~\cref{alg:high-level}.
%
The process begins by constructing an initial graph $G_0$ containing a single node $\ten{T}$.
%
Then, the algorithm enters a loop, where it repeatedly calls the function \textsc{Synth} (Line 4).
%
This function uses a program synthesizer to enumerate and execute transformation programs, generating various tensor networks $G$.
%
At the end of each iteration, the algorithm calls the procedure \textsc{Round} for further compression, and updates the current minimum structure $G_{min}$ if the newly generated structure has a lower cost.
%
Finally, the algorithm returns the structure with the minimum cost.
%
In the remaining section, we introduce the design of the transformation language and the formal reduction of TN-SS to program synthesis.

\begin{figure}[t]
    \centering
    \small
    \vspace{-1em}
    \begin{alignat*}{2}
        \text{Programs or sketches}\quad & P,\sketch && := \emptyprog \mid \expr \mid \seq{P}{\expr} \\
        \text{Expressions}\quad & \expr &&:= \osplit(\ind, r) \\
        \text{Index sets}\quad & \ind  &&:= \{I_\mu\} \mid \ind \cup \{I_\mu\} \\
        \text{Ranks or holes}\quad & r     &&:=\ \hole \mid 1 \mid 2 \mid \cdots \\
        \text{Dimensions}\quad & \mu   &&:= 1, 2, 3, \ldots
    \end{alignat*}
    \vspace{-2em}
    \begin{align*}
    \sem{E}(\bot) = \bot \quad&\quad \sem{\emptyprog}(G, \error) = (G, \error) \\
    \sem{\osplit(\indices, r)}(G, \error) &= \textsc{ExecOSplit}(G, \error, \ind, r)\\
    \sem{\seq{P}{\expr}}(G,\error) &= \sem{\expr}(\sem{P}(G,\error))
    \end{align*}
    \vspace{-2em}
    \caption{Syntax and semantics for the DSL.}\label{fig:dsl}
    % \vspace{-1em}
\end{figure}

\paragraph{Syntax and Semantics of the Language}
%
As illustrated in~\cref{fig:dsl}, a program $P$ can either be empty, denoted by $\emptyprog$, or a sequence of expressions $\seq{P}{E}$.
%
Each expression represents a split operation that takes a set of indices $\ind$ and a rank $r$ as inputs.
%
The index set consists of indices from different modes $I_\mu$.
%
The rank $r$ can be either an integer or a hole $\hole$, which can be filled later.
%
If a program consists of ranks as holes, it is referred to as a \emph{program sketch}.
%
A sketch $\sketch$ can be completed with a rank assignment $\many{\hole_s \mapsto r_s}$ (we use $\many{X}$ to denote a set of similar elements), which is a mapping from holes to integers.
%
A rank assignment can \emph{complete} a sketch by filling holes with corresponding integers, resulting in a complete program, represented as $\sketch[\many{\subst{\hole_s}{r_s}}]$.
%
The execution of a complete program $\sem{P}$ takes a tensor network $G$ and an error bound $\error$ as inputs, and outputs a new tensor network $G'$ and the remaining error bound $\error' \leq \error$.
%
If an execution fails, it skips the remaining expressions and return $\bot$.
%
Multiple expressions in a program are executed sequentially.
%
For each split operation, it is executed using the method \textsc{ExecOSplit}, which produces an updated tensor network while ensuring the result stays within the specified error bound.
%
For example, \cref{fig:example-program} (left) shows a program of three splits.
%
Its execution produces the network structure depicted on the right.
%
The steps of how this result is reached is detailed in \cref{fig:output-directed-example}, and further explained in \cref{sec:algo:split}.

\paragraph{Reduction to Program Synthesis}
%
Using the language defined above, we reduce the TN-SS problem $(\ten{T}, \error)$ to an optimal program synthesis problem:
\begin{align*}
\arg\min_{P}~\size{G} \quad
\textrm{s.t.} \quad \sem{P}(G_0, \error) = (G, \error^{*}) \land \error \leq \error^{*}
\end{align*}
where $G_0 = (\{\ten{T}\}, \emptyset)$ is the initial tensor network containing the data tensor and $\error^{*}$ is the remaining error.
%
Its solution is a program, execution of which produces a tensor network $G$ of the smallest size within the error bound $\error$.

\cref{alg:synth} outlines our high-level program synthesis algorithm.
%
The core idea is to enumerate and rank sketches, and then complete sketches by filling in their holes.
%
The algorithm starts from an empty sketch, and incrementally appends new expressions to construct all possible sketches (Line 3-7).
%
The details of sketch construction and the implementation of the function \validexpr\ are described in \cref{sec:algo:split}.
%
Then, the algorithm fills holes in each sketch and creates complete programs that generate various network structures.
%
As the last step, the algorithm extracts the top-$k$ network structures according to their approximated costs (Line 12).
%
Both the sketch completion algorithm and the cost computation are elaborated in \cref{sec:algo:fillhole}.

\begin{algorithm}[t]
\small
\captionsetup{font=small} % set size of caption font
\caption{The high-level program synthesis algorithm}\label{alg:synth}
\begin{algorithmic}[1]
    % \Require The data tensor $\ten{T}$, the error bound $\error$, the param $k$
    % \Ensure Networks $Gs$ such that $\norm{\net{G}-\ten{T}} \leq \error \norm{\ten{T}}$
    \Function{Synth}{$\ten{T}, \error, k$}
        \State $\sketch s, Gs \gets \{\emptyprog\}, \emptyset$
        \For{$\sketch \in Ss$} %\Comment{Compute all sketch structures}
            \For{$\expr \in \validexpr(\ten{T}, \sketch)$}
                \For{$P \in \fillholes(\seq{\sketch}{\expr}, \error)$}
                    \State $(G, \error') \gets \sem{P}(G_0, \error)$
                    \State $Gs \gets Gs \cup \{(G, \error')\}$
                \EndFor
                \State $Ss \gets Ss \cup \{\seq{\sketch}{\expr}\}$
            \EndFor
        \EndFor
        \State \Return{\Call{TopK}{$Gs, k$}}
    \EndFunction
\end{algorithmic}
\end{algorithm}

\subsection{Programs with Output-Directed Splits}\label{sec:algo:split}
%
\paragraph{Output-Directed Splits}
%
An output-directed split expression $\osplit(\ind, r)$ takes two arguments: a partition block $\ind$ of free indices from the data tensor and a target rank $r$.
%
The semantic of this expression is as follows: given a tensor network $G$ and an error bound $\error$, the expression transforms $G$ by splitting a node into two.
%
The resulting edge connecting the two new nodes forms a free index partition that includes the block $\ind$.
%
\cref{fig:output-directed-example} provides a step-by-step walk-through of the execution of output-directed splits from the program shown in \cref{fig:example-program}.
%
In step \textcircled{\scriptsize 1}, the split expression aims to create a partition where one block is $\{I_1\}$.
%
Since the current network consists of a single node $\ten{T}$, it is picked and split into $\ten{T}_1$ and $\ten{T}_2$.
%
A new edge labelled $r_1$ is created between them, dividing the free indices into two disjoint sets $\{I_1\}$ and $\{I_2, I_3, I_4\}$ and thereby satisfying the split requirement.
%
In step \textcircled{\scriptsize 2}, the goal is to form a new partition block $\{I_1, I_2\}$.
%
The execution procedure discovers that splitting $\ten{T}_2$ and isolating $I_2$ and $r_1$ fulfills the requirement.
%
Finally, step \textcircled{\scriptsize 3} expects to further split a previously created partition block.
%
Splitting the node $\ten{T}_3$ at the index $I_2$ meets the expectation.

The key idea of executing output-directed splits is to convert $\osplit(\ind, r)$ into equivalent, natural node splits $\isplit(\ten{T}, \ind_s, r)$, which takes the node name as arguments and we call them \emph{input-directed splits}.
%
Execution of an input-directed split is built on truncated singular value decomposition (SVD) with minor modifications.
%
Due to the space limitation, we leave the details of output- and input-directed splits to \cref{sec:appendix:alg}.
%
Note that the execution of output-directed splits may fail if the combination  is invalid.
%
For instance, $\osplit(\{I_1, I_2\}, r)$ followed by $\osplit(\{I_1, I_3\}, r)$ is invalid as we cannot put the index $I_1$ in two partition blocks if one is not a subset of the other.
%
%
During program synthesis, the function \validexpr\ in~\cref{alg:synth} takes charge of filtering out invalid combinations to avoid failures.
%

\begin{figure}[t]
    \centering
    \resizebox{0.85\linewidth}{!}{
    \begin{tikzpicture}
\Vertex[x=1,y=1,label=\ten{T},Math,size=.5,color=splitcolor]{G}
\Vertex[x=0,y=1,IdAsLabel,style={color=white},Math,size=.5]{I_1}
\Vertex[x=0.5,y=1.75,IdAsLabel,style={color=white},Math,size=.5]{I_2}
\Vertex[x=1.5,y=1.75,IdAsLabel,style={color=white},Math,size=.5]{I_3}
\Vertex[x=2,y=1,IdAsLabel,style={color=white},Math,size=.5]{I_4}
\Edge(I_1)(G)
\Edge(I_2)(G)
\Edge(I_3)(G)
\Edge(I_4)(G)

\node[circle, draw, minimum size=0.1cm, inner sep=1pt, align=center] (step) at (2.7,1.25) {\tiny 1};
\draw[->] (2.5, 1) node {} -- node[anchor=south]{\scriptsize $\quad\osplit(\{I_1\}, r_1)$} (5, 1) node {};

\Vertex[x=5.5,y=1,label=\ten{T}_1,Math,size=.5,color=splitcolor]{G_1}
\Vertex[x=6.5,y=1,label=\ten{T}_2,Math,size=.5,color=splitcolor]{G_2}
\Vertex[x=5.5,y=2,label=I_1,style={color=white},Math,size=.5]{I1_1}
\Vertex[x=6.5,y=2,label=I_2,style={color=white},Math,size=.5]{I2_1}
\Vertex[x=7.25,y=1.75,label=I_3,style={color=white},Math,size=.5]{I3_1}
\Vertex[x=7.5,y=1,label=I_4,style={color=white},Math,size=.5]{I4_1}
\Edge(I1_1)(G_1)
\Edge(I2_1)(G_2)
\Edge(I3_1)(G_2)
\Edge(I4_1)(G_2)
\Edge[color=splitcolor,lw=2,label=r_1,Math,position=above](G_1)(G_2)
% \draw[dashed,thick,orange] (5,2) node {} -- (5,0) node {};
% \draw[dashed,thick] (4.05,0.5) rectangle (6,1.5);


% \draw[->] (7, 1) node {} -- node[anchor=south]{\scriptsize $\opsplit(\{I_1, I_2\})$} (9, 1) node {};

% \Vertex[x=9.5,y=1,label=G_1,Math]{G1_2}
% \Vertex[x=10.5,y=1,label=G_3,Math]{G3}
% \Vertex[x=11.5,y=1,label=G_4,Math]{G4}
% \Vertex[x=9.5,y=2,label=I_1,style={color=white},Math]{I1_2}
% \Vertex[x=10.5,y=2,label=I_2,style={color=white},Math]{I2_2}
% \Vertex[x=11.5,y=2,label=I_3,style={color=white},Math]{I3_2}
% \Vertex[x=11.5,y=0,label=I_4,style={color=white},Math]{I4_2}
% \Edge(I1_2)(G1_2)
% \Edge(I2_2)(G3)
% \Edge(I3_2)(G4)
% \Edge(I4_2)(G4)
% \Edge[label=r_1,Math,position=above](G1_2)(G3)
% \Edge[color=orange,lw=2,label=r_2,Math,position=above](G3)(G4)
% % \draw[dashed,thick,orange] (11,2) node {} -- (11,0) node {};

% \draw[->] (0, -2) node {} -- node[anchor=south]{\scriptsize $\opsplit(\{I_2\})$} (1.5, -2) node {};

% \Vertex[x=3,y=-1.5,label=G_1,Math]{G1_3}
% \Vertex[x=3,y=-2.5,label=G_5,Math]{G5}
% \Vertex[x=4,y=-2,label=G_6,Math]{G6}
% \Vertex[x=5,y=-2,label=G_4,Math]{G4_3}
% \Vertex[x=2,y=-1,label=I_1,style={color=white},Math]{I1_3}
% \Vertex[x=2,y=-3,label=I_2,style={color=white},Math]{I2_3}
% \Vertex[x=5,y=-1,label=I_3,style={color=white},Math]{I3_3}
% \Vertex[x=6,y=-2,label=I_4,style={color=white},Math]{I4_3}
% \Edge(I1_3)(G1_3)
% \Edge(I2_3)(G5)
% \Edge(I3_3)(G4_3)
% \Edge(I4_3)(G4_3)
% \Edge[label=r_1,Math,position=above right](G1_3)(G6)
% \Edge[color=orange,lw=2,label=r_3,Math,position=below right](G5)(G6)
% \Edge[label=r_2,Math,position=above](G6)(G4_3)
% % \draw[dashed,thick,orange] (3,-2) node {} -- (4,-3) node {};

% \draw[->] (6.5, -2) node {} -- node[anchor=south]{\scriptsize $\opsplit(\{I_4\})$} (8, -2) node {};

% \Vertex[x=9.5,y=-1.5,label=G_1,Math]{G1_4}
% \Vertex[x=9.5,y=-2.5,label=G_5,Math]{G5_4}
% \Vertex[x=10.5,y=-2,label=G_6,Math]{G6_4}
% \Vertex[x=11.5,y=-2,label=G_7,Math]{G7}
% \Vertex[x=12.5,y=-2,label=G_8,Math]{G8}
% \Vertex[x=8.5,y=-1,label=I_1,style={color=white},Math]{I1_4}
% \Vertex[x=8.5,y=-3,label=I_2,style={color=white},Math]{I2_4}
% \Vertex[x=11.5,y=-1,label=I_3,style={color=white},Math]{I3_4}
% \Vertex[x=12.5,y=-1,label=I_4,style={color=white},Math]{I4_4}
% \Edge(I1_4)(G1_4)
% \Edge(I2_4)(G5_4)
% \Edge(I3_4)(G7)
% \Edge(I4_4)(G8)
% \Edge[label=r_1,Math,position=above right](G1_4)(G6_4)
% \Edge[label=r_3,Math,position=below right](G5_4)(G6_4)
% \Edge[label=r_2,Math,position=above](G6_4)(G7)
% \Edge[color=orange,lw=2,label=r_4,Math,position=above](G7)(G8)
% \draw[dashed,thick,orange] (12,-1) node {} -- (12,-3) node {};


\end{tikzpicture}
    }
    \resizebox{0.95\linewidth}{!}{
    \begin{tikzpicture}
\Vertex[x=4.5,y=1,label=\ten{T}_1,Math,size=.5]{G_1}
\Vertex[x=5.5,y=1,label=\ten{T}_2,Math,size=.5,color=splitcolor]{G_2}
\Vertex[x=4.5,y=2,label=I_1,style={color=white},Math,size=.5]{I1_1}
\Vertex[x=5.5,y=2,label=I_2,style={color=white},Math,size=.5]{I2_1}
\Vertex[x=6.25,y=1.75,label=I_3,style={color=white},Math,size=.5]{I3_1}
\Vertex[x=6.5,y=1,label=I_4,style={color=white},Math,size=.5]{I4_1}
\Edge(I1_1)(G_1)
\Edge(I2_1)(G_2)
\Edge(I3_1)(G_2)
\Edge(I4_1)(G_2)
\Edge[label=r_1,Math,position=above](G_1)(G_2)
% \draw[dashed,thick,orange] (5,2) node {} -- (5,0) node {};
% \draw[dashed,thick] (4.05,0.5) rectangle (6,1.5);

\node[circle, draw, minimum size=0.1cm, inner sep=1pt, align=center] (step) at (7, 1.25) {\tiny 2};
\draw[->] (7, 1) node {} -- node[anchor=south]{\scriptsize $\quad\osplit(\{I_1, I_2\}, r_2)$} (9.5, 1) node {};

\Vertex[x=10,y=1,label=\ten{T}_1,Math,size=.5]{G1_2}
\Vertex[x=11,y=1,label=\ten{T}_3,Math,size=.5,color=splitcolor]{G3}
\Vertex[x=12,y=1,label=\ten{T}_4,Math,size=.5,color=splitcolor]{G4}
\Vertex[x=10,y=2,label=I_1,style={color=white},Math,size=.5]{I1_2}
\Vertex[x=11,y=2,label=I_2,style={color=white},Math,size=.5]{I2_2}
\Vertex[x=12,y=2,label=I_3,style={color=white},Math,size=.5]{I3_2}
\Vertex[x=13,y=1,label=I_4,style={color=white},Math,size=.5]{I4_2}
\Edge(I1_2)(G1_2)
\Edge(I2_2)(G3)
\Edge(I3_2)(G4)
\Edge(I4_2)(G4)
\Edge[label=r_1,Math,position=above](G1_2)(G3)
\Edge[color=splitcolor,lw=2,label=r_2,Math,position=above](G3)(G4)
\end{tikzpicture}
    }
    \resizebox{\linewidth}{!}{
    \begin{tikzpicture}
\Vertex[x=0,y=1,label=\ten{T}_1,Math,size=.5]{G1_2}
\Vertex[x=1,y=1,label=\ten{T}_3,Math,size=.5,color=splitcolor]{G3}
\Vertex[x=2,y=1,label=\ten{T}_4,Math,size=.5]{G4}
\Vertex[x=0,y=2,label=I_1,style={color=white},Math,size=.5]{I1_2}
\Vertex[x=1,y=2,label=I_2,style={color=white},Math,size=.5]{I2_2}
\Vertex[x=2,y=2,label=I_3,style={color=white},Math,size=.5]{I3_2}
\Vertex[x=3,y=1,label=I_4,style={color=white},Math,size=.5]{I4_2}
\Edge(I1_2)(G1_2)
\Edge(I2_2)(G3)
\Edge(I3_2)(G4)
\Edge(I4_2)(G4)
\Edge[label=r_1,Math,position=above](G1_2)(G3)
\Edge[label=r_2,Math,position=above](G3)(G4)
% \draw[dashed,thick,orange] (11,2) node {} -- (11,0) node {};

\node[circle, draw, minimum size=0.1cm, inner sep=1pt, align=center] (step) at (3.25,1.25) {\tiny 3};
\draw[->] (3.25, 1) node {} -- node[anchor=south]{\scriptsize $\quad\osplit(\{I_2\}, r_3)$} (5.25, 1) node {};

\Vertex[x=6.5,y=1.5,label=\ten{T}_1,Math,size=.5]{G1_3}
\Vertex[x=6.5,y=0.5,label=\ten{T}_5,Math,size=.5,color=splitcolor]{G5}
\Vertex[x=7.5,y=1,label=\ten{T}_6,Math,size=.5,color=splitcolor]{G6}
\Vertex[x=8.5,y=1,label=\ten{T}_4,Math,size=.5]{G4_3}
\Vertex[x=5.5,y=1.5,label=I_1,style={color=white},Math,size=.5]{I1_3}
\Vertex[x=5.5,y=0.5,label=I_2,style={color=white},Math,size=.5]{I2_3}
\Vertex[x=8,y=2,label=I_3,style={color=white},Math,size=.5]{I3_3}
\Vertex[x=9,y=2,label=I_4,style={color=white},Math,size=.5]{I4_3}
\Edge(I1_3)(G1_3)
\Edge(I2_3)(G5)
\Edge(I3_3)(G4_3)
\Edge(I4_3)(G4_3)
\Edge[label=r_1,Math,position=above right](G1_3)(G6)
\Edge[color=splitcolor,lw=2,label=r_3,Math,position=below right](G5)(G6)
\Edge[label=r_2,Math,position=above](G6)(G4_3)
\end{tikzpicture}
    }
    % \begin{tikzpicture}
\Vertex[x=2.75,y=-1.5,label=\ten{T}_1,Math,size=.5]{G1_3}
\Vertex[x=2.75,y=-2.5,label=\ten{T}_5,Math,size=.5]{G5}
\Vertex[x=3.5,y=-2,label=\ten{T}_6,Math,size=.5]{G6}
\Vertex[x=4.25,y=-2,label=\ten{T}_4,Math,size=.5,color=splitcolor]{G4_3}
\Vertex[x=2,y=-1,label=I_1,style={color=white},Math,size=.5]{I1_3}
\Vertex[x=2,y=-3,label=I_2,style={color=white},Math,size=.5]{I2_3}
\Vertex[x=4.25,y=-1,label=I_3,style={color=white},Math,size=.5]{I3_3}
\Vertex[x=4.25,y=-3,label=I_4,style={color=white},Math,size=.5]{I4_3}
\Edge(I1_3)(G1_3)
\Edge(I2_3)(G5)
\Edge(I3_3)(G4_3)
\Edge(I4_3)(G4_3)
\Edge[label=r_1,Math,position=above right](G1_3)(G6)
\Edge[label=r_3,Math,position=below right](G5)(G6)
\Edge[label=r_2,Math,position=above](G6)(G4_3)
% \draw[dashed,thick,orange] (3,-2) node {} -- (4,-3) node {};

\node[circle, draw, minimum size=0.1cm, inner sep=1pt, align=center] (step) at (4.65, -1.75) {\tiny 4};
\draw[->] (4.75, -2) node {} -- node[anchor=south]{\scriptsize $\opsplit(\{I_4\}, r_4)$} (6.75, -2) node {};

\Vertex[x=7.5,y=-1.5,label=\ten{T}_1,Math,size=.5]{G1_4}
\Vertex[x=7.5,y=-2.5,label=\ten{T}_5,Math,size=.5]{G5_4}
\Vertex[x=8.25,y=-2,label=\ten{T}_6,Math,size=.5]{G6_4}
\Vertex[x=9,y=-2,label=\ten{T}_7,Math,size=.5,color=splitcolor]{G7}
\Vertex[x=9.75,y=-2,label=\ten{T}_8,Math,size=.5,color=splitcolor]{G8}
\Vertex[x=6.75,y=-1,label=I_1,style={color=white},Math,size=.5]{I1_4}
\Vertex[x=6.75,y=-3,label=I_2,style={color=white},Math,size=.5]{I2_4}
\Vertex[x=9,y=-1,label=I_3,style={color=white},Math,size=.5]{I3_4}
\Vertex[x=9.75,y=-1,label=I_4,style={color=white},Math,size=.5]{I4_4}
\Edge(I1_4)(G1_4)
\Edge(I2_4)(G5_4)
\Edge(I3_4)(G7)
\Edge(I4_4)(G8)
\Edge[label=r_1,Math,position=above right](G1_4)(G6_4)
\Edge[label=r_3,Math,position=below right](G5_4)(G6_4)
\Edge[label=r_2,Math,position=above](G6_4)(G7)
\Edge[color=splitcolor,lw=2,label=r_4,Math,position=above](G7)(G8)
\end{tikzpicture}
    \caption{Execution of output-directed splits from \cref{fig:example-program}. Nodes involved in split operations are highlighted in purple.}
    \label{fig:output-directed-example}
\end{figure}

\begin{figure}[t]
    \centering
    \resizebox{\linewidth}{!}{
    \begin{tikzpicture}
\Vertex[x=5,y=1,label=\ten{T}_1,Math,size=.5,color=splitcolor]{G1}
\Vertex[x=6,y=1,label=\ten{T}_2,Math,size=.5]{G2}
\Vertex[x=5,y=2,style={color=white},label=I_1,Math,size=.5]{I1_1}
\Vertex[x=6,y=2,style={color=white},label=I_2,Math,size=.5]{I2_1}
\Vertex[x=6.5,y=1.75,style={color=white},label=I_3,Math,size=.5]{I3_1}
\Vertex[x=7,y=1,style={color=white},label=I_4,Math,size=.5]{I4_1}
\Edge(I1_1)(G1)
\Edge(I2_1)(G2)
\Edge(I3_1)(G2)
\Edge(I4_1)(G2)
\Edge[Math,label=r_1,position=above](G1)(G2)

% \node[circle, draw, minimum size=0.1cm, inner sep=1pt, align=center] (step) at (6.85,1.25) {\tiny 2};
\draw[->] (7.25, 1) node {} -- node[anchor=south]{\scriptsize $\isplit(\ten{T}_1, \{I_1\},r_2)$} (10, 1) node {};

\Vertex[x=10.5,y=1,label=\ten{T}_3,Math,size=.5,color=splitcolor]{G3_2}
\Vertex[x=11.5,y=1,label=\ten{T}_4,Math,size=.5,color=splitcolor]{G4_2}
\Vertex[x=12.5,y=1,label=\ten{T}_2,Math,size=.5]{G2_2}
\Vertex[x=10.5,y=2,style={color=white},label=I_1,Math,size=.5]{I1_1}
\Vertex[x=12.5,y=2,style={color=white},label=I_2,Math,size=.5]{I2_1}
\Vertex[x=13,y=1.75,style={color=white},label=I_3,Math,size=.5]{I3_1}
\Vertex[x=13.5,y=1,style={color=white},label=I_4,Math,size=.5]{I4_1}
\Edge(I1_1)(G3_2)
\Edge[color=splitcolor,Math,label=r_2,position=above](G3_2)(G4_2)
\Edge(I2_1)(G2_2)
\Edge(I3_1)(G2_2)
\Edge(I4_1)(G2_2)
\Edge[Math,label=r_1,position=above](G4_2)(G2_2)
\end{tikzpicture}
    }
    \caption{The problem of suboptimality in input-directed splits: the resulting structure of this operation is suboptimal.}
    \label{fig:suboptimal}
\end{figure}

\begin{figure*}[t]
    \centering
    \begin{minipage}{.45\linewidth}
    \centering
    \resizebox{.9\linewidth}{!}{
    \begin{tikzpicture}
\Vertex[x=1,y=1,label=\ten{T},Math,size=.5,color=splitcolor]{G}
\Vertex[x=0,y=1,label=I_1,style={color=white},Math,size=.5]{I1_0}
\Vertex[x=0.5,y=1.75,label=I_2,style={color=white},Math,size=.5]{I2_0}
\Vertex[x=1.5,y=1.75,label=I_3,style={color=white},Math,size=.5]{I3_0}
\Vertex[x=2,y=1,label=I_4,style={color=white},Math,size=.5]{I4_0}
\Edge(I1_0)(G)
\Edge(I2_0)(G)
\Edge(I3_0)(G)
\Edge(I4_0)(G)

\node[circle, draw, minimum size=0.1cm, inner sep=1pt, align=center] (step) at (2.5,1.25) {\tiny 1.1};
\draw[->] (2.5, 1) node {} -- node[anchor=south]{\scriptsize $\quad\isplit(\ten{T}, \{I_1\},r_1)$} (5, 1) node {};

\Vertex[x=5.5,y=1,label=\ten{T}_1,Math,size=.5,color=splitcolor]{G1}
\Vertex[x=6.5,y=1,label=\ten{T}_2,Math,size=.5,color=splitcolor]{G2}
\Vertex[x=5.5,y=2,label=I_1,style={color=white},Math,size=.5]{I1_1}
\Vertex[x=6.5,y=2,label=I_2,style={color=white},Math,size=.5]{I2_1}
\Vertex[x=7.25,y=1.75,label=I_3,style={color=white},Math,size=.5]{I3_1}
\Vertex[x=7.5,y=1,label=I_4,style={color=white},Math,size=.5]{I4_1}
\Edge(I1_1)(G1)
\Edge(I2_1)(G2)
\Edge(I3_1)(G2)
\Edge(I4_1)(G2)
\Edge[color=splitcolor,Math,label=r_1,position=above](G1)(G2)

% \draw[->] (0, -1.5) node {} -- node[anchor=south]{\scriptsize $\opsplit(\mathcal{A}_1, I_1,r_2)$} (2, -1.5) node {};

% \Vertex[x=2.5,y=-1.5,label=\mathcal{A}_3,Math]{G3_2}
% \Vertex[x=3.5,y=-1.5,label=\mathcal{A}_4,Math]{G4_2}
% \Vertex[x=4.5,y=-1.5,label=\mathcal{A}_2,Math]{G2_2}
% \Vertex[x=2.5,y=-0.5,style={color=white},label=I_1,Math]{I1_1}
% \Vertex[x=5.5,y=-1.5,style={color=white},label=I_2,Math]{I2_1}
% \Vertex[x=4.5,y=-0.5,style={color=white},label=I_3,Math]{I3_1}
% \Vertex[x=4.5,y=-2.5,style={color=white},label=I_4,Math]{I4_1}
% \Edge(I1_1)(G3_2)
% \Edge[Math,label=r_2,position=above](G3_2)(G4_2)
% \Edge(I2_1)(G2_2)
% \Edge(I3_1)(G2_2)
% \Edge(I4_1)(G2_2)
% \Edge[Math,label=r_1,position=above](G4_2)(G2_2)
\end{tikzpicture}
    }
    \resizebox{\linewidth}{!}{
    \input{figs/input_split_step12}
    }
    \end{minipage}
    \hfill
    \vline
    \hfill
    \begin{minipage}{.45\linewidth}
    \centering
    \resizebox{.9\linewidth}{!}{
    \begin{tikzpicture}
\Vertex[x=1,y=1,label=\ten{T},Math,size=.5,color=splitcolor]{G}
\Vertex[x=0,y=1,label=I_1,style={color=white},Math,size=.5]{I1_0}
\Vertex[x=0.5,y=1.75,label=I_2,style={color=white},Math,size=.5]{I2_0}
\Vertex[x=1.5,y=1.75,label=I_3,style={color=white},Math,size=.5]{I3_0}
\Vertex[x=2,y=1,label=I_4,style={color=white},Math,size=.5]{I4_0}
\Edge(I1_0)(G)
\Edge(I2_0)(G)
\Edge(I3_0)(G)
\Edge(I4_0)(G)

\node[circle, draw, minimum size=0.1cm, inner sep=1pt, align=center] (step) at (2.5,1.25) {\tiny 2.1};
\draw[->] (2.5, 1) node {} -- node[anchor=south]{\scriptsize $\quad\isplit(\ten{T}, \{I_1, I_2\},r_2)$} (5.5, 1) node {};

\Vertex[x=6,y=1,label=\ten{T}_1,Math,size=.5,color=splitcolor]{G1}
\Vertex[x=7,y=1,label=\ten{T}_2,Math,size=.5,color=splitcolor]{G2}
\Vertex[x=5.5,y=2,style={color=white},label=I_1,Math,size=.5]{I1_1}
\Vertex[x=6.5,y=2,style={color=white},Math,size=.5]{I2_1}
\Vertex[x=6.5,y=2,style={color=white},Math,size=.5]{I3_1}
\Vertex[x=7.5,y=2,style={color=white},label=I_4,Math,size=.5]{I4_1}
\node[] (step) at (6.25,2) {\tiny $I_2$};
\node[] (step) at (6.75,2) {\tiny $I_3$};
\Edge(I1_1)(G1)
\Edge(I2_1)(G1)
\Edge(I3_1)(G2)
\Edge(I4_1)(G2)
\Edge[color=splitcolor,lw=2,Math,label=r_2,position=above](G1)(G2)
\end{tikzpicture}
    }
    \resizebox{\linewidth}{!}{
    \begin{tikzpicture}
\Vertex[x=0,y=1,label=\ten{T}_1,Math,size=.5,color=splitcolor]{G1}
\Vertex[x=1,y=1,label=\ten{T}_2,Math,size=.5]{G2}
\Vertex[x=-0.5,y=2,style={color=white},label=I_1,Math,size=.5]{I1_1}
\Vertex[x=0.5,y=2,style={color=white},Math,size=.5]{I2_1}
\Vertex[x=0.5,y=2,style={color=white},Math,size=.5]{I3_1}
\Vertex[x=1.5,y=2,style={color=white},label=I_4,Math,size=.5]{I4_1}
\node[] (step) at (0.25,2) {\tiny $I_2$};
\node[] (step) at (0.75,2) {\tiny $I_3$};
\Edge(I1_1)(G1)
\Edge(I2_1)(G1)
\Edge(I3_1)(G2)
\Edge(I4_1)(G2)
\Edge[Math,label=r_2,position=above](G1)(G2)


\node[circle, draw, minimum size=0.1cm, inner sep=1pt, align=center] (step) at (1.9,1.25) {\tiny 2.2};
\draw[->] (2, 1) node {} -- node[anchor=south]{\scriptsize $\quad\isplit(\ten{T}_1, \{I_1\},r_1)$} (4.5, 1) node {};

\Vertex[x=5,y=1,label=\ten{T}_3,Math,size=.5,color=splitcolor]{G3}
\Vertex[x=6,y=1,label=\ten{T}_4,Math,size=.5,color=splitcolor]{G4}
\Vertex[x=7,y=1,label=\ten{T}_2,Math,size=.5]{G2}
\Vertex[x=5,y=2,style={color=white},label=I_1,Math,size=.5]{I1_1}
\Vertex[x=6,y=2,style={color=white},label=I_2,Math,size=.5]{I2_1}
\Vertex[x=6.5,y=2,style={color=white},label=I_3,Math,size=.5]{I3_1}
\Vertex[x=7.5,y=2,style={color=white},label=I_4,Math,size=.5]{I4_1}
\Edge(I1_1)(G3)
\Edge(I2_1)(G4)
\Edge(I3_1)(G2)
\Edge(I4_1)(G2)
\Edge[color=splitcolor,lw=2,Math,label=r_1,position=above](G3)(G4)
\Edge[Math,label=r_2,position=above](G2)(G4)
\end{tikzpicture}
    }
    \end{minipage}
    \caption{The problem of redundancy in input-directed splits: two different sequences of input-directed splits result in the same network structure, and reasoning of their equivalence is complicated.}
    \label{fig:input-splits}
\end{figure*}


\paragraph{Output- versus Input-Directed Splits}
%
Output-directed splits have advantages over input-directed splits in several aspects.
%
First, output-directed splits create a more succinct search space while maintaining the same level of expressiveness.
%
They exclude many suboptimal structures produced by input-directed splits, as shown in~\cref{fig:suboptimal}.
%
This structure consists of two edges $r_1$ and $r_2$ corresponding to the same partition of free indices, which makes the structure unlikely to be an optimal one because the middle node $\ten{T}_4$ is redundant and can be merged with $\ten{T}_3$ or $\ten{T}_2$ to get a more compact structure.
%
%
Second, input-directed splits result in an infinite search space because one can keep splitting the new node introduced by a previous split operation and get a fresh topology.
%
This is problematic and a boundary limit is needed to make the search algorithm tractable.
%
Third, different orders of input-directed splits may produce the same structure, and it is challenging to reason about their equivalence without execution.
%
For example, \cref{fig:input-splits} (left) and (right) are two programs of input-directed splits that lead to the same structure.
%
It is difficult to tell these two programs are equivalent without sophisticated analysis.
%
However, the equivalence of programs with output-directed splits is simple because programs of the same set of splits are equivalent.

\begin{theorem}[Completeness of Output-Directed Splits]\label{thm:completeness}
If $G$ is the optimal tree tensor network for a tensor $\ten{T}$ and an error bound $\error$,
then there exists a program with output-directed splits $P$ that $\sem{P}(G_0, \error) = G$ where $G_0 = (\{\ten{T}\},\emptyset)$.
\end{theorem}

\subsection{Synthesis of Transformation Programs}\label{sec:algo:fillhole}
%
As mentioned in~\cref{sec:algo:program}, our synthesis algorithm consists of two components: sketch generation and sketch completion.
%
In the phase of sketch generation, the algorithm exhaustively enumerates all semantically correct sketches.
%
The second component, sketch completion, is particularly interesting in the context of tensor networks.
%
Given a sketch $\sketch$, an initial tensor network $G_0$, and an error bound $\error$, the goal of sketch completion is to find a rank assignment $\many{\hole_i \mapsto r_i}$  such that the program execution succeeds after the completion, \ie $\sem{\sketch[\many{\hole_i \mapsto r_i}]}(G_0, \error) \not= \bot$.
%
Through sketch completion, we hope to discover which sketches are more promising to produce tensor network structures with high compression ratio.
%
Achieving this requires calculating the optimal rank assignment for each sketch, but trying all possible rank assignments is computationally prohibitive.
%
An alternative approach is to compress each sketch by distributing errors equally between splits, as done in TT or HT.
%
This approach sacrifices the compression quality for faster speed, but it also suffers from performance issues when dealing with large tensors or small error bounds as we show in \cref{sec:eval:ablation}.
%
To address this challenge, we propose to encode the sketch completion task as integer programming constraints.
%
By leveraging the efficiency of constraint solvers, we can quickly estimate the optimal rank assignment without the need for costly program executions.

\paragraph{Truncation Algorithm}
%
Before detailing the encoding for sketch completion, we illustrate how to find valid rank assignments for a hole.
%
Given a data tensor $\ten{T}$ and an error bound $\error$, completions of the hole in $\osplit(\ind_s, \hole)$ are computed by truncated SVD.
%
Specifically, after converting $\osplit(\ind_s, \hole)$ into an input-directed split, we apply SVD to compute singular values $\Sigma$ for the splitting node, where $\Sigma = \many{\sigma_i}$ and $\sigma_1 \leq \sigma_2 \leq \cdots \leq \sigma_{\size{\Sigma}}$.
%
A valid assignment $\hole \mapsto (\size{\Sigma} - r)$ is constructed if $\sum_{i=1}^{r} \sigma_{i}^2 \leq (\error \norm{\ten{T}})^2$.
%
Truncated SVD achieves a low-rank approximation of the original tensor by discarding the $r$ smallest singular values.
%
This connection between discarded singular values and the error bound forms the foundation of our sketch completion method.

\paragraph{Sketch Completion Using Integer Programming}
%
Given a data tensor $\ten{T}$, an error bound $\error$, and a sketch $\sketch = \many{\osplit(\ind_s, \hole_s)}$, we formulate the problem of finding the optimal rank assignment within the error bound as an integer programming problem.
%
Each hole in the sketch is assigned an integer variable $r_s$.
%
The network cost and the error bound requirements are encoded as follows:
\begin{equation}
\begin{aligned}
    \min_{r_1, r_2, \ldots, r_s} &\size{\sem{\sketch[\many{\subst{\hole_s}{(\size{\Sigma_s} - r_s)}}]}(G_0, \error)} \\
    \text{s.t.} & \quad \sum_{i=1}^{s}\sum_{j=1}^{r_i} \sigma_{ij}^2 \leq \left(\error\norm{\ten{T}}\right)^2
\end{aligned}
\label{eqn:ip}
\end{equation}
Here, $\sigma_{ij}$ is the $j$th smallest singular values at the $i$th split, $G_0$ is the initial tensor network containing the input data tensor.
%
The constraint accumulates the truncation errors, \ie the sum of truncated singular values, from all split operations, guaranteeing that the total error remains within the specified error bound.
%
The objective minimizes the symbolic representation of the network cost and asks the solver to find an optimal solution.
%
However, this encoding is not directly feasible because the singular values for a split operation cannot be determined until all preceding ones are completed.
%
Earlier truncation affect the singular values of subsequent split operations.
%
To overcome this challenge, we utilize the following property:
\begin{theorem}[Upper Bound of Singular Values]\label{thm:rank-bound}
Let $\ten{T} \in \real^{n_1 \times \cdots \times n_d}$ be a $d$-dimensional tensor with free indices $I_1, I_2, \ldots, I_d$, and $G=(\{\ten{T}\},\emptyset)$ be the graph with a single tensor.
%
If a complete program $P = \osplit(\ind_1, r_1); \ldots; \osplit(\ind_n, r_n)$ such that $\sem{P}(G, \error) = (G_n, \error_n)$,
then for every $1 \leq i, s \leq n$, we have $\sigma_j(\net{G_{i-1}}_{\langle\ind_s\rangle}) \leq \sigma_j(\matric{T}{\ind_s})$ where $\sigma_j(A)$ is the $j$th smallest singular value of the matrix $A$.
\end{theorem}

In other words, for each index partition $\ind_s$ of the data tensor, the singular values at this partition in the original tensor provide an upper bound for the singular values of any truncated tensor at the same partition.
%
This theorem allows us to approximate singular values for all splits, which enables us to calculate an upper-bound cost through \cref{eqn:ip}.

\begin{theorem}[Upper Bound of Costs]\label{thm:cost-bound}
Given a data tensor $\ten{T}$ and an error bound $\error$, if \cref{eqn:ip} with upper bounded singular values yields $r_1, \ldots, r_n$ for a sketch program $\osplit(\indices_1, \hole_1); \ldots; \osplit(\indices_n, \hole_n)$, then $\many{\hole_i \mapsto (\Sigma_i - r_i)}$ is a valid completion.
\end{theorem}

This rank assignment is not only used to select the promising sketches, but also serves as the initial completion of the holes, which is followed by a rounding operation that further compresses the network structure~\cite{ht}.

\paragraph{Search} All models performed relatively well on \textbf{Search} tasks, which is unsurprising given the 4k context length. However, even at this length, model performance varied significantly depending on the specific search type (see Table \ref{tab:search}). For example, in the binary \textit{String Search} task, models handled individual word searches well but struggled with subsequence searches, where queries consisted of multi-word sequences. The performance drop can be attributed to two factors: (1) length of query affects the difficulty of precise memory access; (2) negative samples are created by replacing a single word in present subsequences, making absent longer subsequence more distracting.

% \begin{figure}[h!]
%     \centering
%     \includegraphics[width=0.85\columnwidth]{images/ablation_seq_search.png}
%     \caption{Analysis on \textit{String Search (with subsequence)} across increasing subsequence lengths. This figure examines the behavior of models on \textbf{pos}itive samples (where the subsequence is present) and \textbf{neg}ative samples (where the subsequence is absent). }
%     \label{fig:seq_search}
% \end{figure}

\begin{figure}
    \centering
\resizebox{0.9\columnwidth}{!}{
    \begin{tikzpicture}
        \begin{axis}[
            ybar,
            bar width=6pt,
            symbolic x coords={8, 16, 32, 64},
            xtick=data,
            ymin=0, ymax=1.02,
            legend columns=3,
            legend style={at={(0.5,1.3)}, anchor=north, draw=black},
            enlarge x limits=0.15,
            width=11cm, height=6cm
        ]
        
        % gpt-4o
        \addplot[fill={rgb,255:red,0;green,91;blue,150}, draw=none] coordinates {(8,1.000) (16,1.000) (32,1.000) (64,1.000)};
        \addlegendentry{gpt-4o (pos)}
        \addplot[fill={rgb,255:red,0;green,91;blue,150}, postaction={
        pattern=north east lines
    }, draw=none] coordinates {(8,0.900) (16,0.800) (32,0.700) (64,0.200)};
        \addlegendentry{gpt-4o (neg)}
        
        % mistral-7b-instruct-v02
        \addplot[fill={rgb,255:red,255;green,196;blue,218}, draw=none] coordinates {(8,1.000) (16,1.000) (32,1.000) (64,1.000)};
        \addlegendentry{mistral-7b (pos)}
        \addplot[fill={rgb,255:red,255;green,196;blue,218}, postaction={
        pattern=north east lines
    }, draw=none] coordinates {(8,0.300) (16,0.600) (32,0.600) (64,0.900)};
        \addlegendentry{mistral-7b (neg)}
        
        % phi-3-medium-128k-instruct
        \addplot[fill={rgb,255:red,255;green,218;blue,112}, draw=none] coordinates {(8,0.000) (16,0.100) (32,0.100) (64,0.500)};
        \addlegendentry{phi-3-medium (pos)}
        \addplot[fill={rgb,255:red,255;green,218;blue,112}, postaction={
        pattern=north east lines
    }, draw=none] coordinates {(8,1.000) (16,1.000) (32,1.000) (64,0.700)};
        \addlegendentry{phi-3-medium (neg)}
        
        \end{axis}
    \end{tikzpicture}}
    \caption{Analysis on \textit{String Search (with subsequence)} across increasing subsequence lengths. This figure examines the behavior of models on \textbf{pos}itive samples (where the subsequence is present) and \textbf{neg}ative samples (where the subsequence is absent).}
    \label{fig:seq_search}
\end{figure}

Figure \ref{fig:seq_search} further analyzes subsequence search performance for GPT-4o, Mistral, and Phi-3-medium. These models exhibit distinct error patterns as the length of the subsequence increases: GPT-4o has no false negative errors (it never misses a present subsequence) but makes more false positive errors as the subsequence length grows, suggesting it overestimates presence in more ambiguous cases.
Mistral also makes no false negative errors but exhibits a decreasing false positive rate, implying it struggles more with shorter distractors. Phi-3-medium, in contrast, makes few false positive errors (rarely identifies an absent sequence as present), but struggles more with false negatives, indicating a general tendency to deny presence. These differing patterns suggest that the models may employ different search strategies, affecting their susceptibility to different types of errors.

For \textit{Batch Search} and \textit{Key-Value Search} tasks (analogous to multi-NIAH and NIAH, respectively), models like Mistral, Phi-3, and Cohere show a notable performance drop, revealing their limitations in handling multiple memory accesses effectively.

% \begin{figure}[h]
%     \centering
%     \begin{subfigure}[t]{0.49\linewidth}
%         \includegraphics[width=\textwidth]{images/recall_black.png}
%         \caption{Black-box models.}
%         \label{fig:first_recall}
%     \end{subfigure}
%     \begin{subfigure}[t]{0.49\linewidth}
%         \includegraphics[width=\textwidth]{images/recall_white.png}
%         \caption{Open-source models.}
%         \label{fig:second_recall}
%     \end{subfigure}
%     \caption{Results for the \textbf{Recall and Edit} tasks.}
%     \label{fig:recall}
% \end{figure}

\begin{figure}[h]
    \centering
    \begin{subfigure}{0.49\columnwidth}
        \resizebox{\textwidth}{!}{
\begin{tikzpicture}
        \begin{axis}[
            ybar,
            bar width=6pt,
            symbolic x coords={Snapshot (words), Replace all, Overwrite positions, Snapshot (numbers), Functional updates},
            xtick=data,
            ymin=0, ymax=1.05,
            legend columns=4,
            legend style={at={(0.5,1.15)}, anchor=north, draw=black},
            enlarge x limits=0.13,
            xticklabel style={rotate=20, anchor=center, yshift=-12pt}
        ]
        
        \addplot[fill={rgb,255:red,42;green,183;blue,202}, draw=none] coordinates {(Snapshot (words),0.96) (Replace all,0.985) (Overwrite positions,0.645) (Snapshot (numbers),1.00) (Functional updates,0.72)};
        \addlegendentry{gpt-4-turbo}
        
        \addplot[fill={rgb,255:red,0;green,91;blue,150}, draw=none] coordinates {(Snapshot (words),1.00) (Replace all,0.99) (Overwrite positions,0.645) (Snapshot (numbers),1.00) (Functional updates,0.93)};
        \addlegendentry{gpt-4o}
        
        \addplot[fill={rgb,255:red,173;green,216;blue,230}, draw=none] coordinates {(Snapshot (words),1.00) (Replace all,0.84) (Overwrite positions,0.685) (Snapshot (numbers),1.00) (Functional updates,0.24)};
        \addlegendentry{gpt-4o-mini}
        
        \addplot[fill={rgb,255:red,254;green,138;blue,113}, draw=none] coordinates {(Snapshot (words),0.74) (Replace all,0.72) (Overwrite positions,0.63) (Snapshot (numbers),0.77) (Functional updates,0.29)};
        \addlegendentry{cohere}
        
        \end{axis}
\end{tikzpicture}}
    \end{subfigure}
    \begin{subfigure}{0.49\columnwidth}
        \resizebox{\textwidth}{!}{    \begin{tikzpicture}
        \begin{axis}[
            ybar,
            bar width=5pt,
            symbolic x coords={Snapshot (words), Replace all, Overwrite positions, Snapshot (numbers), Functional updates},
            xtick=data,
            ymin=0, ymax=1.02,
            legend columns=3,
            legend style={at={(0.5,1.20)}, anchor=north, draw=black},
            enlarge x limits=0.12,
            xticklabel style={rotate=20, anchor=center, yshift=-12pt}
        ]
        
        \addplot[fill={rgb,255:red,255;green,196;blue,218}, draw=none] coordinates {(Snapshot (words),0.96) (Replace all,0.65) (Overwrite positions,0.67) (Snapshot (numbers),0.77) (Functional updates,0.16)};
        \addlegendentry{mistral-7b}
        
        \addplot[fill={rgb,255:red,250;green,98;blue,95}, draw=none] coordinates {(Snapshot (words),0.95) (Replace all,0.495) (Overwrite positions,0.61) (Snapshot (numbers),1.00) (Functional updates,0.07)};
        \addlegendentry{phi-3-small}
        
        \addplot[fill={rgb,255:red,255;green,218;blue,112}, draw=none] coordinates {(Snapshot (words),0.96) (Replace all,0.19) (Overwrite positions,0.645) (Snapshot (numbers),0.77) (Functional updates,0.06)};
        \addlegendentry{phi-3-medium}
        
        \addplot[fill={rgb,255:red,179;green,153;blue,212}, draw=none] coordinates {(Snapshot (words),0.91) (Replace all,0.865) (Overwrite positions,0.69) (Snapshot (numbers),0.54) (Functional updates,0.11)};
        \addlegendentry{gemma-2-9b}
        
        \addplot[fill={rgb,255:red,116;green,196;blue,118}, draw=none] coordinates {(Snapshot (words),1.00) (Replace all,0.77) (Overwrite positions,0.55) (Snapshot (numbers),1.00) (Functional updates,0.29)};
        \addlegendentry{llama-3.1-8b}
        
        \end{axis}
\end{tikzpicture}}
    \end{subfigure}
    \caption{Results for the \textbf{Recall and Edit} tasks.}
    \label{fig:recall}
\end{figure}

\paragraph{Recall and Edit} 
\begin{table}[!h]
    \centering
        \resizebox{0.8\columnwidth}{!}{%
    \begin{tabular}{lllll}
    \toprule
        \textbf{Model} & \textbf{String Search (word)} & \textbf{Snapshot} \\ \hline
gpt-4-turbo    & 1.00 \textcolor{green}{(0.06)} & 1.00 \textcolor{green}{(0.04)} \\ 
gpt-4o         & 1.00 (0.00)                   & 1.00 (0.00)                   \\ 
gpt-4o-mini    & 0.94 \textcolor{red}{(-0.04)}  & 1.00 (0.00)                   \\ 
cohere         & 1.00 (0.00)                   & 1.00 \textcolor{green}{(0.26)} \\ 
mistral-7b     & 1.00 \textcolor{green}{(0.22)} & 0.96 (0.00)                   \\ 
phi-3-small    & 1.00 \textcolor{green}{(0.06)} & 0.99 \textcolor{green}{(0.04)} \\ 
phi-3-medium   & 0.98 \textcolor{red}{(-0.02)}  & 0.87 \textcolor{red}{(-0.09)}  \\ 
gemma-2-9b     & 0.96 \textcolor{red}{(-0.04)}  & 0.96 \textcolor{green}{(0.05)} \\ 
llama-3.1-8b   & 0.98 \textcolor{red}{(-0.02)}  & 1.00 (0.00)                   \\
\bottomrule
    \end{tabular}
    }
    \caption{Ablation study with gibberish context.}
    \label{tab:ablation_gibberish}
\end{table}

Figure \ref{fig:recall} presents the results for the \textbf{Recall and Edit} tasks. While models performed well on basic recall (\textit{Snapshot}), their performance dropped sharply when tasked with making regular edits. A closer analysis of the generated outputs reveals that models struggled with maintaining coherence during edits, often getting trapped in repetitive word loops. For the \textit{Functional Update} task, we deliberately selected simple numerical updates, such as ``Subtract 1 from every number," to ensure the edits were within the models' capabilities. Nevertheless, when comparing performance on \textit{Snapshot (with numbers)} to \textit{Functional Updates}, all models exhibited a steep decline, especially for smaller ones. Analysis of generated outputs revealed that these models frequently deviated from instructions over longer sequences, suggesting difficulties in maintaining consistent rule applications over extended contexts.

Additionally, we conducted a separate ablation study on \textit{Snapshot} and \textit{String Search}. In this study, we replaced meaningful words in the context with gibberish tokens consisting of randomly generated alphabetical characters. As shown in Table \ref{tab:ablation_gibberish}, performance remained largely unchanged, suggesting that semantic meaning was not a significant distractor in these tasks.

\begin{table}
\centering
% \footnotesize
\resizebox{\linewidth}{!}{
\setlength{\tabcolsep}{5pt}
\begin{tabular}[t]{l|ccc}
\toprule
 \makecell[c]{\textbf{Method}} & \makecell[c]{\textbf{Self}\\\textbf{Reflection}} & \makecell[c]{\textbf{Memory}} & \makecell[c]{\textbf{Length}\\\textbf{Generalization}} \\
\midrule
Revision~\cite{DBLP:journals/corr/abs-2408-03314} & \redcross & \greencheck & \redcross \\
Self-Refine~\cite{DBLP:conf/nips/MadaanTGHGW0DPY23} & \greencheck & \greencheck & \redcross \\
Best-of-N~\cite{DBLP:journals/corr/abs-2407-21787} & \redcross & \redcross & \greencheck \\
Beam Search~\cite{ow1988filtered} & \redcross & \redcross & \greencheck \\
Guided Beam Search~\cite{DBLP:conf/nips/XieKZZKHX23} & \greencheck & \redcross & \greencheck \\
\midrule
\textbf{FTTT (ours)} & \greencheck & \greencheck & \greencheck \\
\bottomrule
\end{tabular}
}
% \vspace{-5pt}
\caption{Comparing the advantages and drawbacks of FTTT and related works.}
\label{tab:compare}
% \vspace{-0.5cm}
\end{table}

\begin{table}[!htbp] \centering
  \caption{Human Choices and Predictions About GenAI Choice in the Same Problem: Heterogeneity by Exposure and Attitudes (Pooled)}
\begin{adjustbox}{scale=0.8}
\begin{tabular}{@{\extracolsep{5pt}}lccccc}
% \\[-1.8ex]\hline
% \hline \\[-1.8ex]
\toprule
& \multicolumn{5}{c}{\textit{Dependent variable: Prediction}} \
\cr \cline{2-6}
\\[-1.8ex] & \multicolumn{1}{c}{Heavy User} & \multicolumn{1}{c}{Text-Based LLM User} & \multicolumn{1}{c}{Paid User} & \multicolumn{1}{c}{Agree AI Similar} & \multicolumn{1}{c}{Agree AI Better}  \\
\\[-1.8ex] & (1) & (2) & (3) & (4) & (5) \\
% \hline \\[-1.8ex]
\midrule
 X$\times$Heavy User & -0.056$^{}$ & & & & \\
& (0.052) & & & & \\
 X$\times$Text-Based LLM User & & 0.082$^{**}$ & & & \\
& & (0.040) & & & \\
 X$\times$Paid User & & & -0.001$^{}$ & & \\
& & & (0.072) & & \\
 X$\times$Agree AI Similar & & & & 0.033$^{}$ & \\
& & & & (0.045) & \\
 X$\times$Agree AI Better & & & & & 0.019$^{}$ \\
& & & & & (0.017) \\
 Problem FE & Yes & Yes & Yes & Yes & Yes \\
 X$\times$Problem FE & Yes & Yes & Yes & Yes & Yes \\
 G$\times$Problem FE & Yes & Yes & Yes & Yes & Yes \\
% \hline \\[-1.8ex]
\midrule
 Observations & 2700 & 2700 & 2700 & 2700 & 2700 \\
 % Residual Std. Error & 22.874 & 22.851 & 22.863 & 22.847 & 22.895 \\
% \hline
% \hline \\[-1.8ex]
\bottomrule
\textit{Note:} & \multicolumn{5}{r}{Standard errors are clustered at the problem level. $^{*}$p$<$0.1; $^{**}$p$<$0.05; $^{***}$p$<$0.01} \\
% \multicolumn{6}{r}\textit{} \\
\end{tabular}
\end{adjustbox}
\label{tab:group} \end{table}


\paragraph{Match and Compare}
 As shown in Figure \ref{fig:match}, model performance in the \textbf{Match and Compare} tasks was relatively consistent across different model sizes. Given that counting is a well-known weakness in LLMs, it is unsurprising that all models struggled significantly with the counting task, though GPT models performed slightly better than others. However, models generally succeeded in identifying the duplicates (in \textit{Find duplicates}), and primarily struggled with the counting aspect -- which requires tracking and updating an integer state, a skill that is more similar to stateful processing. This suggests that relying solely on counting-based tests \cite{song2024countingstars} could overly bias the evaluation and fail to capture broader model capabilities. The results also indicate that models exhibit some ability to recognize relative positions and group associations, but their accuracy remains limited (ranging between 0.6-0.8). A closer examination of model generations reveals an overwhelming tendency for the models to produce false positive errors -- models often answer “yes” when the correct answer is “no”, while making very few false negative errors. This means that when the relationship is correct, the models can more reliably identify it. This may stem from a combination of their inherent inclination to agree and the difficulty in recognizing relative comparisons and associations.

% \begin{figure}[h]
%     \centering
%     \includegraphics[width=0.92\columnwidth]{images/difference.png}
%     \caption{Results for \textbf{Spot the Differences }tasks.}
%     \label{fig:difference}
% \end{figure}

\begin{figure}[h]
\centering
\resizebox{0.9\columnwidth}{!}{
 \begin{tikzpicture}
        \begin{axis}[
            ybar,
            bar width=5pt,
            symbolic x coords={Compare two lists, Identify the odd group, Patch the difference},
            xtick=data,
            ymin=0, ymax=1.0,
            legend columns=4,
            legend style={at={(0.5,1.35)}, anchor=north, draw=black, font=\footnotesize},
            enlarge x limits=0.22,
            xticklabel style={rotate=10, anchor=center, yshift=-12pt},
            width=10cm, height=6cm,
        ]
        
        \addplot[fill={rgb,255:red,42;green,183;blue,202}, draw=none] coordinates {(Compare two lists,0.36) (Identify the odd group,0.78) (Patch the difference,0.37)};
        \addlegendentry{gpt-4-turbo}
        
        \addplot[fill={rgb,255:red,0;green,91;blue,150}, draw=none] coordinates {(Compare two lists,0.33) (Identify the odd group,0.87) (Patch the difference,0.89)};
        \addlegendentry{gpt-4o}
        
        \addplot[fill={rgb,255:red,173;green,216;blue,230}, draw=none] coordinates {(Compare two lists,0.29) (Identify the odd group,0.73) (Patch the difference,0.22)};
        \addlegendentry{gpt-4o-mini}
        
        \addplot[fill={rgb,255:red,254;green,138;blue,113}, draw=none] coordinates {(Compare two lists,0.32) (Identify the odd group,0.67) (Patch the difference,0.17)};
        \addlegendentry{cohere}
        
        \addplot[fill={rgb,255:red,255;green,196;blue,218}, draw=none] coordinates {(Compare two lists,0.24) (Identify the odd group,0.37) (Patch the difference,0.21)};
        \addlegendentry{mistral-7b}
        
        \addplot[fill={rgb,255:red,250;green,98;blue,95}, draw=none] coordinates {(Compare two lists,0.31) (Identify the odd group,0.52) (Patch the difference,0.55)};
        \addlegendentry{phi-3-small}
        
        \addplot[fill={rgb,255:red,255;green,218;blue,112}, draw=none] coordinates {(Compare two lists,0.30) (Identify the odd group,0.65) (Patch the difference,0.37)};
        \addlegendentry{phi-3-medium}
        
        \addplot[fill={rgb,255:red,179;green,153;blue,212}, draw=none] coordinates {(Compare two lists,0.24) (Identify the odd group,0.72) (Patch the difference,0.30)};
        \addlegendentry{gemma-2-9b}
        
        \addplot[fill={rgb,255:red,116;green,196;blue,118}, draw=none] coordinates {(Compare two lists,0.29) (Identify the odd group,0.70) (Patch the difference,0.74)};
        \addlegendentry{llama-3.1-8b}
        
        \end{axis}
    \end{tikzpicture}}
    \caption{Results for \textbf{Spot the Differences }tasks.}
    \label{fig:difference}
\end{figure}

\paragraph{Spot the Differences}
As shown in Figure \ref{fig:difference}, performance across all models are poor on \textit{Compare Two Lists}, suggesting inherent difficulties in cross-referencing information across long contexts, even for larger models.  GPT-4o and the LLaMA model significantly outperform the others in the \textit{Identify the Odd Group} task, highlighting a general weakness in detecting contextual differences by the other models. However, an 8B LLaMA model outperforms both equivalently-sized models and even GPT-4 in this task, suggesting that model size alone was not the determining factor. This indicates that architectural differences, training objectives, or specific inductive biases may contribute to improved performance in comparative memory utilization.


\paragraph{Compute on Sets and Lists}
The tasks in this category require models to recognize and process group structures within the context, and performance gradually declines as the complexity of the task increases (see Table \ref{tab:lists}). For instance, in comparing the \textit{Group Membership} task with the \textit{String Search} task, where the former requires identifying which list a word belongs to rather than simply determining its presence, the performance of open-source models drops considerably. Similarly, in comparing the \textit{Group Association} task with the \textit{Group Membership} task, where the former requires determining whether two words belong to the same group, all models exhibit a noticeable decline in performance. The decline becomes even more pronounced when comparing the \textit{ Group Association (alternating)} variant of the task to the standard \textit{Group Association} task. Here, the context involves alternating repeated groups rather than simple group structures, which further challenges the models' abilities to handle partitioned contexts effectively.

An interesting observation was found during the \textit{Iterate} task. In an ablation study, we modified the task to require returning the first words in each list instead of the last words (making it more similar to the \textit{Batch Search} task). The performance sharply declines when models are asked to return the last words, despite their strong information-fetching capabilities. This suggests that, while the models can retrieve information effectively, they struggle to accurately recognize and process partitions within the context.



\begin{table}[t!]
\centering
    \resizebox{0.7\columnwidth}{!}{%

\begin{tabular}{lcc}
\toprule
\textbf{Model} & \textbf{Quantity state} & \textbf{Set state} \\
\midrule
gpt-4-turbo & 0.8 & \textbf{0.80} \\
gpt-4o & \textbf{1.0} & 0.65 \\
gpt-4o-mini & 0.7 & 0.24 \\
cohere & 0.0 & 0.58 \\
mistral-7b & 0.0 & 0.08 \\
phi-3-small & 0.0 & 0.13 \\
phi-3-medium & 0.0 & 0.11 \\
gemma-2-9b & 0.0 & 0.24 \\
llama-3.1-8b & 0.0 & 0.13 \\
\bottomrule
\end{tabular}
}
\caption{Results for \textbf{Stateful Processing} tasks.}
\label{tab:state}
\end{table}
\paragraph{Stateful Processing}

\begin{figure}[t!]
    \centering
    \begin{subfigure}{0.49\columnwidth}
        \resizebox{\textwidth}{!}{\begin{tikzpicture}
    \begin{axis}[
        xlabel={Step},
        legend style={at={(0.5,1.2)}, anchor=north, cells={align=left}, legend columns=3},
        ymin=0, ymax=1.1,
        xtick={50, 200, 400, 800, 1200, 1600},
        ytick={0,0.2,0.4,0.6,0.8,1.0},
        grid=none,
        tick label style={font=\large}
    ]

    % GPT-4-Turbo
    \addplot[mark=triangle, very thick, color={rgb,255:red,42;green,183;blue,202}] coordinates {
        (25,1.0) (50,1.0) (100,0.9) (200,0.8) (400,0.0) (800,0.0) (1600,0.0)
    };
    \addlegendentry{gpt-4-turbo}

    % GPT-4o
    \addplot[mark=*, very thick, color={rgb,255:red,0;green,91;blue,150}] coordinates {
        (25,1.0) (50,1.0) (100,1.0) (200,1.0) (400,0.0) (800,0.0) (1600,0.0)
    };
    \addlegendentry{gpt-4o}



    % Phi-3-Small
    \addplot[mark=x, very thick, color={rgb,255:red,250;green,98;blue,95}] coordinates {
        (25,1.0) (50,0.6) (100,0.0) (200,0.0) (400,0.0) (800,0.0) (1600,0.0)
    };
    \addlegendentry{phi-3-small}

    % Cohere
    \addplot[mark=star, very thick, color={rgb,255:red,254;green,138;blue,113}] coordinates {
        (25,0.0) (50,0.0) (100,0.0) (200,0.0) (400,0.0) (800,0.0) (1600,0.0)
    };
    \addlegendentry{cohere}

    % Mistral-7B
    \addplot[mark=o, very thick, color={rgb,255:red,255;green,196;blue,218}] coordinates {
        (25,0.0) (50,0.0) (100,0.0) (200,0.0) (400,0.0) (800,0.0) (1600,0.0)
    };
    \addlegendentry{mistral-7b}
    
    \end{axis}
\end{tikzpicture}}
    \end{subfigure}
    \begin{subfigure}{0.49\columnwidth}
        \resizebox{\textwidth}{!}{\begin{tikzpicture}
    \begin{axis}[
        xlabel={Step},
        legend style={at={(0.5,1.2)}, anchor=north, cells={align=left}, legend columns=3},
        ymin=0, ymax=1.1,
        xtick={50, 200, 400, 800, 1200, 1600},
        ytick={0,0.2,0.4,0.6,0.8,1.0},
        grid=none,
        tick label style={font=\large}
    ]

    % GPT-4-Turbo
    \addplot[mark=triangle, very thick, color={rgb,255:red,42;green,183;blue,202}] coordinates {
        (25,0.909) (50,0.899) (100,0.892) (200,0.861) (400,0.875) (800,0.878) (1600,0.770)
    };
    \addlegendentry{gpt-4-turbo}

    % GPT-4o
    \addplot[mark=*, very thick, color={rgb,255:red,0;green,91;blue,150}] coordinates {
        (25,0.897) (50,0.764) (100,0.637) (200,0.787) (400,0.590) (800,0.646) (1600,0.621)
    };
    \addlegendentry{gpt-4o}

    % Cohere
    \addplot[mark=x, very thick, color={rgb,255:red,254;green,138;blue,113}] coordinates {
        (25,0.900) (50,0.657) (100,0.498) (200,0.341) (400,0.452) (800,0.200) (1600,0.088)
    };
    \addlegendentry{cohere}

    % Mistral-7B
    \addplot[mark=star, very thick, color={rgb,255:red,255;green,196;blue,218}] coordinates {
        (25,0.174) (50,0.084) (100,0.066) (200,0.048) (400,0.006) (800,0.003) (1600,0.003)
    };
    \addlegendentry{mistral-7b}

    % Phi-3-Small
    \addplot[mark=o, very thick, color={rgb,255:red,250;green,98;blue,95}] coordinates {
        (25,0.435) (50,0.165) (100,0.062) (200,0.109) (400,0.149) (800,0.000) (1600,0.000)
    };
    \addlegendentry{phi-3-small}
    
    \end{axis}
\end{tikzpicture}}
    \end{subfigure}
    \caption{Ablation study on the number of operation steps for the \textbf{quantity state} (left) and\textbf{ set state }(right).}
    \label{fig:ablation_state_step}
\end{figure}



Table \ref{tab:state} presents the results for the \textbf{Stateful Processing} tasks, where performance gaps among models are the most pronounced. The GPT-4(o) models perform well on integer state tracking, while most other models struggle (near zero accuracy). For set state tracking, larger models generally perform better.

We conducted an ablation study to examine how the number of operation steps influences performance of five selected models (Fig. \ref{fig:ablation_state_step}). For quantity state tracking, GPT-4(o) models perform well within fewer than 200 steps but experience a sharp decline in accuracy beyond this threshold. For set state tracking, the performance decline is more gradual. The differences in performance drop between the two tasks can be attributed to the nature of the two tasks. While tracking an integer state might seem simpler than tracking a set, it actually requires the model to maintain and apply every operation sequentially to compute the final value. In contrast, for set state, the fixed size of the set makes more recent operations more relevant to the final state, reducing the need for exhaustive step-by-step tracking. Nevertheless, even in this scenario, all models show a clear inability to handle longer or more complex operation sequences effectively. Interestingly, GPT-4 model outperformed GPT-4o at this task, suggesting potential optimization trade-offs may have affected its ability to manage set-based updates. 

Overall, while larger models like GPT-4(o) exhibit some ability to track state over time, their effectiveness rapidly deteriorates as task complexity increases. Smaller models, in particular, struggle to track operations over time, pointing to significant gaps in their ability to manage and process sequential dependencies critical for state tracking tasks.

\subsection{Results on Composite Tests}

\section{Simple Construction of Projective Compositions}
\label{sec:comp_coord}

It is not clear apriori that projective compositional distributions satisfying Definition \ref{def:proj_comp} ever exist, much less that there is any straightforward way to sample from them.
To explore this, we first restrict attention to perhaps the simplest setting, where the projection functions $\{\Pi_i\}$ are
just coordinate restrictions.
This setting is meant to generalize the intuition we had
in the CLEVR example of Figure~\ref{fig:len_gen},
where different objects were composed in disjoint regions of the image.
We first define the construction of the composed distribution,
and then establish its theoretical properties.








\subsection{Defining the Construction}
Formally, suppose we have a set of distributions
$(p_1, p_2, \ldots, p_k)$ that we wish to compose;
in our running CLEVR example, each $p_i$ is the distribution of images
with a single object at position $i$.
Suppose also we have some reference distribution $p_b$,
which can be arbitrary, but should be thought of as a 
``common background'' to the $p_i$s.
Then, one popular way to construct a composed distribution
is via the \emph{compositional operator} defined below.
(A special case of this construction is used in \citet{du2023reduce}, for example).


\begin{definition}[Composition Operator]
    \label{def:comp_oper}
    Define the \emph{composition operator} $\cC$ acting on an arbitrary set of distributions $(p_b, p_1, p_2, \ldots)$ by
    \begin{align}
    \label{eq:comp_oper}
    \cC[\vec{p}] := \cC[p_b, p_1, p_2, \dots](x) := \frac{1}{Z} p_b(x) \prod_i \frac{p_i(x)}{p_b(x)},
    \end{align}
    where $Z$ is the appropriate normalization constant. We name $\cC[\vec{p}]$ the \emph{composed distribution}, and the score of $\cC[\vec{p}]$ the \emph{compositional score}:
    \begin{align}
    \label{eqn:comp_score}
    &\grad_x \log \cC[\vec{p}](x)  \\
    &= \grad_x \log p_b(x) + \sum_i \left( \grad_x \log p_i(x) - \grad_x \log p_b(x) \right). \notag
    \end{align}
\end{definition}
Notice that if $p_b$ is taken to be the unconditional distribution then this is exactly the Bayes-composition.


\vspace{-0.5em}
\subsection{When does the Composition Operator Work?}
We can always apply the composition operator to any set of distributions,
but when does this actually yield a ``correct'' composition
(according to Definition~\ref{def:proj_comp})?
One special case is when each distribution $p_i$ is
``active'' on a different, non-overlapping set of coordinates.
We formalize this property below
as \emph{Factorized Conditionals} (Definition~\ref{def:factorized}).
The idea is, 
each distribution $p_i$
must have a particular set of ``mask'' coordinates $M_i \subseteq [n]$ which it
samples in a characteristic way,
while independently sampling all other coordinates
from a common background distribution.
If a set of distributions $(p_b, p_1, p_2, \ldots)$ has this
\emph{Factorized Conditional} structure, then 
the composition
operator will produce a projective composition (as we will prove below).



\begin{definition}[Factorized-Conditionals]
\label{def:factorized}

We say a set of distributions $(p_b, p_1, p_2, \dots p_k)$
over $\R^n$
are \emph{Factorized Conditionals} if
there exists a partition of coordinates $[n]$
into disjoint subsets $M_b, M_1, \dots M_k$ such that:
\begin{enumerate}
    \setlength{\itemsep}{1pt}
    \item $(x|_{M_i}, x|_{M_i^c})$ are independent under $p_i$.
    \item $(x|_{M_b}, x|_{M_1}, x|_{M_2}, \dots, x|_{M_k})$
    are mutually independent under $p_b$.
    \item $p_i(x|_{M_i^c}) = p_b(x|_{M_i^c})$.
\end{enumerate}

Equivalently, if we have:
\begin{align}
    p_i(x) &= p_i(x|_{M_i}) p_b(x|_{M_i^c}), \text{ and} \label{eqn:cc-cond}\\
    p_b(x) &= p_b(x|_{M_b}) \prod_{i \in [k]} p_b(x|_{M_i}). \notag
\end{align}
\end{definition}
\vspace{-1em}
Equation~\eqref{eqn:cc-cond} means that each $p_i$
can be sampled by first sampling $x \sim p_b$,
and then overwriting the coordinates of $M_i$
according to some other distribution (which can be specific to distribution $i$).
For instance, the experiment of Figure~\ref{fig:len_gen}
intuitively satisfies this property, since 
each of the conditional distributions could essentially be sampled
by first sampling an empty background image ($p_b$), then ``pasting''
a random object in the appropriate location (corresponding to pixels $M_i$).
If a set of distributions obey this Factorized Conditional structure,
then we can prove that the composition operator $\cC$
yields a correct projective composition,
and reverse-diffusion correctly samples from it.
Below, let $N_t$ denote the noise operator of the
diffusion process\footnote{Our results are agnostic to the specific diffusion noise-schedule and scaling used.} at time $t$.

\begin{theorem}[Correctness of Composition]
\label{lem:compose}
Suppose a set of distributions $(p_b, p_1, p_2, \dots p_k)$
satisfy Definition~\ref{def:factorized},
with corresponding masks $\{M_i\}_i$.
Consider running the reverse-diffusion SDE 
using the following compositional scores at each time $t$:
\begin{align}
s_t(x_t) &:= \grad_x \log \cC[p_b^t, p_1^t, p_2^t, \ldots](x_t),
\end{align}
where $p_i^t := N_t[p_i]$ are the noisy distributions.
Then, the distribution of the generated sample $x_0$ at time $t=0$ is:
\begin{align}
\label{eqn:p_hat}
\hat{p}(x) := p_b(x|_{M_b}) \prod_i p_i(x|_{M_i}).
\end{align}
In particular,
$\hat{p}(x|_{M_i}) = p_i(x|_{M_i})$ for all $i$,
and so
$\hat{p}$ is a projective composition
with respect to projections $\{\Pi_i(x) := x|_{M_i}\}_i$,
per Definition \ref{def:proj_comp}.
\end{theorem}




Unpacking this, Line \ref{eqn:p_hat} says that the final generated distribution
$\hat{p}(x)$ can be sampled by
first sampling
the coordinates $M_b$ according to $p_b$ (marginally),
then independently sampling 
coordinates $M_i$ according to $p_i$ (marginally) for each $i$.
Similarly, by assumption, $p_i(x)$ can be sampled by first sampling the coordinates $M_i$ in some specific way, and then independently sampling the remaining coordinates according to $p_b$. Therefore Theorem \ref{lem:compose} says that $\hat{p}(x)$ samples the coordinates \emph{$M_i$ exactly as they would be sampled by $p_i$}, for each $i$ we wish to compose. 

\begin{proof}(Sketch) \small
Since $\vec{p}$ satisfies Definition \ref{def:factorized}, we have
\begin{align*}
&\cC[\vec{p}](x) := p_b(x) \prod_i \frac{p_i(x)}{p_b(x)} \notag 
= p_b(x) \prod_i \frac{p_b(x_t|_{M_i^c}) p_i(x|_{M_i})}{p_b(x|_{M_i^c})p_b(x|_{M_i})} \notag \\
&= p_b(x) \prod_i \frac{p_i(x|_{M_i})}{p_b(x|_{M_i})} \notag 
= p_b(x|_{M_b}) \prod_i p_i(x_t|_{M_i}) := \hat{p}(x).
\end{align*}
The sampling guarantee follows from the commutativity of composition with the diffusion noising process, i.e. $\cC[\vec{p^t}]= N_t[\cC[\vec{p}]]$. 
The complete proof is in Appendix \ref{app:compose_pf}.
\end{proof}

\begin{remark}
In fact, Theorem~\ref{lem:compose} still holds under any orthogonal transformation of the variables,
because the diffusion noise process commutes with orthogonal transforms.
We formalize this as Lemma~\ref{lem:orthogonal_sampling}.
\end{remark}

\begin{remark}
Compositionality is often thought of in terms of orthogonality between scores.
Definition \ref{def:factorized} implies orthogonality between the score differences that appear in the composed score \eqref{eqn:comp_score}:
$\grad_x \log p_i^t(x_t) - \grad_x \log p_b^t(x_t),$
but the former condition is strictly stronger
(c.f. Appendix \ref{app:score_orthog}).
\end{remark}

\begin{remark}
Notice that the composition operator $\cC$
can be applied to a set of Factorized Conditional
distributions
without knowing the coordinate partition $\{M_i\}$.
That is, we can compose distributions and compute scores
without knowing apriori exactly ``how'' these distributions are supposed to compose
(i.e. which coordinates $p_i$ is active on).
This is already somewhat remarkable, and we will see a much
stronger version of this property in the next section.
\end{remark}

\textbf{Importance of background.}
Our derivations highlight the crucial role of the background
distribution $p_b$ for the composition operator  
(Definition~\ref{def:comp_oper}).
While prior works have taken $p_b$ to be an unconditional distribution and the $p_i$'s its associated conditionals,
our results suggest this is not always the optimal choice -- in particular,
it may not satisfy a Factorized Conditional structure (Definition~\ref{def:factorized}). Figure~\ref{fig:len_gen_monster} demonstrates this empirically: settings (a) and (b) attempt to compose the same distributions using different backgrounds -- empty (a) or unconditional (b) -- with very different results.

\subsection{Approximate Factorized Conditionals in CLEVR.}
\label{sec:clevr-details}

In \cref{fig:len_gen_monster} we explore compositional length-generalization (or lack thereof) in three different setting, two of which (\cref{fig:len_gen_monster}a and \ref{fig:len_gen_monster}c) approximately satisfy \cref{def:factorized}. In this section we explicitly describe how our definition of Factorized Conditionals approximately captures the CLEVR settings of Figures \ref{fig:len_gen_monster}a and \ref{fig:len_gen_monster}c. The setting of \ref{fig:len_gen_monster}b does not satisfy our conditions, as discussed in \cref{sec:problematic-compositions}.

\textbf{Single object distributions with empty background.}
This is the setting of both \cref{fig:len_gen} and \cref{fig:len_gen_monster}a.
The background distribution $p_b$ 
over $n$ pixels is images of an empty scene with no objects.
For each $i \in \{1,\ldots,L\}$ (where $L=4$ in \cref{fig:len_gen} and $L=9$ in \cref{fig:len_gen_monster}a), define the set $M_i \subset [n]$ 
as the set of pixel indices surrounding location $i$.
($M_i$ should be thought of as a ``mask'' that
that masks out objects at location $i$).
Let $M_b := (\cup_i M_i)^c$ be the remaining
pixels in the image.
Then, we claim the distributions $(p_b, p_1, \ldots, p_L)$
form approximately
Factorized Conditionals, with corresponding
coordinate partition $\{M_i\}$.
This is essentially because each distribution $p_i$
matches the background $p_b$ on all pixels except those surrounding
location $i$ (further detail in Appendix~\ref{app:clevr-details}).
Note, however, that the conditions of Definition~\ref{def:factorized}
do not \emph{exactly} hold in the experiment of Figure~\ref{fig:len_gen} -- there is still some dependence between
the masks $M_i$, since objects can cast shadows or even occlude each other.
Empirically, these deviations 
have greater impact
when composing many objects, as seen in \cref{fig:len_gen_monster}a.


\textbf{Bayes composition with cluttered distributions.}
In \cref{fig:len_gen_monster}c we replicate CLEVR experiments in  \citet{du2023reduce, liu2022compositional} where the images contain many objects (1-5) and the conditions label the location of one randomly-chosen object. It turns out the unconditional together with the conditionals can approximately act as Factorized Conditionals in ``cluttered'' settings like this one. The intuition is that if the conditional distributions each contain one specific object plus many independently sampled random objects (``clutter''), then the unconditional distribution \emph{almost} looks like independently sampled random objects, which together with the conditionals \emph{would} satisfy Definition \ref{def:factorized} (further discussion in Appendix \ref{app:clevr-details} and \ref{app:bayes_connect}). This helps to explain the length-generalization observed in \citet{liu2022compositional} and verified in our experiments (\cref{fig:len_gen_monster}c).







\section{Projective Composition in Feature Space}
\label{sec:comp_feature}

\begin{figure}
    \centering
    \includegraphics[width=1.0\linewidth]{figures/feat-space-vis.png}
    \caption{A commutative diagram illustrating Theorem~\ref{lem:transform_comp}.
    Performing composition in pixel space is equivalent 
    to encoding into a feature space ($\cA$),
    composing there,
    and decoding back
    to pixel space ($\cA^{-1}$).
    }
    \label{fig:feat-space-vis}
\end{figure}

So far we have focused on the setting where the projection functions $\Pi_i$ are simply projections onto coordinate subsets $M_i$ in the native space (e.g. pixel space).
This covers simple examples like Figure~\ref{fig:len_gen} but does not include more realistic situations such as Figure~\ref{fig:style-content},
where the properties to be composed are more abstract.
For example a property like ``oil painting'' does not correspond to projection
onto a specific subset of pixels in an image.
However, we may hope that there exists some conceptual feature space
in which ``oil painting'' does correspond to a particular subset of variables.
In this section, we extend our results to the case where the composition occurs in some conceptual feature space, and each distribution to be composed
corresponds to some particular subset of \emph{features}.


Our main result is a featurized analogue of Theorem~\ref{lem:compose}:
if there exists \emph{any} invertible transform $\cA$
mapping into a feature space
where Definition \ref{def:factorized} holds,
then the composition operator (Definition~\ref{def:comp_oper})
yields a projective composition in this feature space, as shown in Figure~\ref{fig:feat-space-vis}.

\begin{theorem}[Feature-space Composition]
\label{lem:transform_comp}
Given distributions $\vec{p} := (p_b, p_1, p_2, \dots p_k)$,
suppose there exists a diffeomorphism $\cA: \R^n \to \R^n$
such that
$(\cA \sharp p_b, \cA \sharp p_1, \dots \cA \sharp p_k)$
satisfy Definition~\ref{def:factorized},
with corresponding partition $M_i \subseteq [n]$.
Then, the composition $\hat{p} := \cC[\vec{p}]$ satisfies:
\begin{align}
\label{eqn:p_hat_A}
\cA \sharp \hat{p}(z)
\equiv
(\cA \sharp p_b (z))|_{M_b} \prod_{i=1}^k (\cA \sharp p_i(z))|_{M_i}.
\end{align}
Therefore, $\hat{p}$
is a projective composition of $\vec{p}$ w.r.t. projection functions
$\{\Pi_i(x) := \cA(x)|_{M_i}\}$.
\end{theorem}
This theorem is remarkable because it means we can
compose distributions $(p_b, p_1, p_2, \dots)$ in the base space,
and this composition will ``work correctly'' in the feature space
automatically (Equation~\ref{eqn:p_hat_A}),
without us ever needing to compute or even know the feature transform $\cA$
explicitly.



Theorem~\ref{lem:transform_comp} may apriori seem too strong
to be true, since it somehow holds for all feature spaces $\cA$
simultaneously.
The key observation underlying Theorem~\ref{lem:transform_comp} 
is that the composition operator $\cC$ behaves
well under reparameterization.
\begin{lemma}[Reparameterization Equivariance]
\label{lem:reparam}
The composition operator of Definition~\ref{def:comp_oper}
is reparameterization-equivariant. That is,
for all diffeomorphisms $\cA: \R^n \to \R^n$
and all tuples of distributions $\vec{p} = (p_b, p_1, p_2, \dots, p_k)$,
\begin{align}
 \cC[ \cA \sharp \vec{p}] =  \cA \sharp \cC[\vec{p}].
\end{align}
\end{lemma}
\arxiv{\footnote{
For example (separate from our goals in this paper):
Classifier-Free-Guidance can be seen as an instance of the composition operator.
Thus, Lemma~\ref{lem:reparam} implies that performing CFG
in latent space is \emph{equivalent} to CFG in pixel-space,
assuming accurate score-models in both cases.}}
\arxiv{This lemma is potentially of independent interest:
reparametrization-equivariance
is a very strong property which is typically not satisfied by
standard operations between probability distributions---
for example, the ``simple product'' $p_1(x)p_2(x)$ does not satisfy it---
so it is mathematically notable that the composition operator 
has this structure.
Lemma~\ref{lem:reparam} and Theorem~\ref{lem:transform_comp}
are proved in Appendix \ref{app:param-indep}.}

This lemma is potentially of independent interest:
equivariance distinguishes the composition operator
from many other common operators
(e.g. the simple product).
Lemma ~\ref{lem:reparam} and Theorem~\ref{lem:transform_comp}
are proved in Appendix \ref{app:param-indep}.

\section{Sampling from Compositions.}
The feature-space Theorem~\ref{lem:transform_comp} is weaker than Theorem~\ref{lem:compose}
in one important way: it does not provide a sampling algorithm.
That is, Theorem~\ref{lem:transform_comp} guarantees that $\hat{p} := \cC[\vec{p}]$
is a projective composition, but does not guarantee that reverse-diffusion
is a valid sampling method.

There is one special case where diffusion sampling \emph{is} guaranteed to work, namely, for orthogonal transforms (which can seen as a straightforward extension of the coordinate-aligned case of \cref{lem:compose}):
\begin{lemma}[Orthogonal transform enables diffusion sampling]
\label{lem:orthogonal_sampling}
If the assumptions of Lemma \ref{lem:transform_comp} hold for $\cA(x) = Ax$, where $A$ is an orthogonal matrix, then running a reverse diffusion sampler with scores $s_t = \grad_x \log \cC[\vec{p}^t]$ generates the composed distribution $\hat{p} = \cC[\vec{p}]$ satisfying \eqref{eqn:p_hat_A}.
\end{lemma}
The proof is given in \cref{app:orthog_sample_pf}.

However, for general invertible transforms, we have no such sampling guarantees.
Part of this is inherent: in the feature-space setting, the 
diffusion noise operator $N_t$ no longer commutes
with the composition operator $\cC$ in general,
 so scores of the noisy composed 
distribution $N_t[\cC[\vec{p}]]$
cannot be computed from scores
of the noisy base distributions $N_t[\vec{p}]$.
Nevertheless, one may hope to sample from the distribution $\hat{p}$
using other samplers besides diffusion, 
such as annealed Langevin Dynamics
or
Predictor-Corrector methods \citep{song2020score}.
We find that the situation is surprisingly subtle:
composition $\cC$ produces distributions which
are in some cases easy to sample (e.g. with diffusion),
yet in other cases apparently hard to sample.
For example, in the
setting of Figure~\ref{fig:clevr_color_comp}, 
our Theorem~\ref{lem:transform_comp} implies
that all pairs of colors should compose equally well
at time $t=0$, since there exist diffeomorphisms
(indeed, linear transforms) between different colors.
However, as we saw,
the diffusion sampler
fails to sample from compositions 
of non-orthogonal colors--- and 
empirically, even more sophisticated
samplers such as Predictor-Correctors
also fail in this setting.
At first glance, it may seem odd that
composed distributions are so hard to sample,
when their constituent distributions are relatively easy to sample.
One possible reason for this below is that the composition operator has extremely poor Lipchitz constant,
so it is possible for a set of distributions $\vec{p}$ to ``vary smoothly''
(e.g. diffusing over time) while their composition $\cC[\vec{p}]$
changes abruptly.
We formalize this in \cref{lem:lipschitz} (further discussion and proof in Appendix \ref{app:lipschitz}).
\begin{lemma}[Composition Non-Smoothness]
\label{lem:lipschitz}
For any set of distributions $\{p_b, p_1, p_2, \dots, p_k\}$,
and any noise scale $t := \sigma$,
define the noisy distributions 
$p_i^t := N_{t}[p_i]$,
and let $q^t$ denote the composed distribution at time $t$: $q^t := \cC[\vec{p}^t]$. Then, for any choice of $\tau > 0$,
there exist distributions $\{p_b, p_1, \dots p_k\}$ over $\R^n$
such that
\begin{enumerate}
    \setlength{\itemsep}{0pt}
    \item For all $i$, the annealing path of $p_i$ is 
    $\cO(1)$-Lipshitz:
    $\forall t, t': W_2(p_i^{t}, p_i^{t'}) \leq \cO(1) |t - t'|$.
    \item The annealing path of $q$ has Lipshitz constant
    at least $\Omega(\tau^{-1})$:
    $\exists t, t': W_2(q^{t}, q^{t'}) \geq \frac{|t - t'|}{2\tau}.$
\end{enumerate}
\end{lemma}




The composite tests significantly challenge the models by combining multiple atomic capabilities into a single test. In the \textit{Processing Data Blocks} task, the context is fixed at 4k tokens, while for the \textit{Theory of Mind} task, the number of operation steps is set to 100. As shown in Table \ref{tab:comp}, model performance on both tasks are generally low, showing a broad inability to handle the more complex scenarios. Performance across all models drop substantially on composite tasks compared to their performance on individual capability tasks, such as search, recall, and group processing. 

Interestingly, some smaller models, like Mistral and Phi-3-small, exhibit slightly better performance on the \textit{Theory of Mind} task than on the set state tracking task. This anomaly likely stems from their already weak state tracking ability, which limits their performance across both tasks. Additionally, these models tend to generate longer answers in the set state task which reduces the set overlap.

Notably, even the most capable models, such as GPT-4-turbo and GPT-4o, struggle, showing that scaling model size alone is not enough for solving these composite tasks. Additionally, the variation in performance among smaller models suggests that their limitations stem not only from size but also from underlying architectural or training differences. This indicates that smaller models require more targeted care to bridge the gap in effective memory use.




%\newpage
%\input{6_layers}
%\newpage
%\input{7_generalization}
%\newpage

%\newpage
%\section{Limitations}

Although our method is generally applicable to all common LLM architectures, as they share the same language modeling head and embeddings, only dense decoders were used in our experiments. 
In addition, only models with up to $N=2.6\B$ parameters have been tested.
The cosine decay learning rate schedule was applied throughout all experiments (App.~\ref{app:hyperparameters}). Alternatives such as an infinite learning rate schedule are not incorporated in our study.
Furthermore, as mentioned at the end of Sec.~\ref{sec:results}, we have not explicitly verified that the slight residual shift of the mean embedding, which is observed even for Coupled Adam, is caused by weight tying.
Finally, we have used a straightforward implementation of Coupled Adam, closely following Algorithm~\ref{alg:algorithm_adam}. More sophisticated implementations might lead to increased efficiency and further improvements; we leave it for future work to investigate this.


% Entries for the entire Anthology, followed by custom entries
\newpage
\bibliography{yinyang.bib}
% \bibliographystyle{acl_natbib}
\newpage




%\newpage
\onecolumn

\section{Frequently Asked Questions (FAQs)}
\label{sec:FAQs}

\begin{itemize}[leftmargin=15pt,nolistsep]


\item[\ding{93}] {\fontfamily{lmss} \selectfont \textbf{How does YinYangAlign differ from existing T2I benchmarks?}}
\vspace{0mm}
\begin{description}
\item[\ding{224}] Existing benchmarks typically focus on isolated objectives, such as fidelity to prompts or aesthetic quality. YinYangAlign is unique in evaluating how T2I systems navigate trade-offs between multiple conflicting objectives, providing a more holistic assessment.
\end{description}

\item[\ding{93}] {\fontfamily{lmss} \selectfont \textbf{What is the role of Contradictory Alignment Optimization (CAO)?}}
\vspace{0mm}
\begin{description}
\item[\ding{224}] CAO is a framework introduced in the paper that harmonizes competing objectives through a synergy-driven multi-objective loss function. It integrates local axiom-specific preferences with global trade-offs to achieve balanced optimization across all alignment goals.
\end{description}

\item[\ding{93}] {\fontfamily{lmss} \selectfont \textbf{What are the key components of the CAO framework?}}
\vspace{0mm}
\begin{description}
\item[\ding{224}] The key components include:
\begin{enumerate}
    \item Local per-axiom preferences to handle individual trade-offs.
    \item A global synergy mechanism for unified alignment.
    \item A regularization term to prevent overfitting to any single objective.
\end{enumerate}
\end{description}

\item[\ding{93}] {\fontfamily{lmss} \selectfont \textbf{How does YinYangAlign handle annotation challenges?}}
\vspace{0mm}
\begin{description}
\item[\ding{224}] YinYangAlign combines automated annotations using Vision-Language Models (VLMs) like GPT-4o and LLaVA with rigorous human verification. A consensus filtering mechanism ensures reliability, with a high inter-annotator agreement score (kappa = 0.83).
\end{description}

\item[\ding{93}] {\fontfamily{lmss} \selectfont \textbf{What insights were gained from the empirical evaluation of DPO and CAO?}}
\vspace{0mm}
\begin{description}
\item[\ding{224}] The study revealed that optimizing a single axiom using Directed Preference Optimization (DPO) often disrupts other objectives. For instance, improving Artistic Freedom by 40\% caused declines in Cultural Sensitivity (-30\%) and Verifiability (-35\%). In contrast, CAO demonstrated controlled trade-offs, achieving more balanced alignment across all objectives.
\end{description}

\item[\ding{93}] {\fontfamily{lmss} \selectfont \textbf{What are the metrics used to evaluate alignment in YinYangAlign?}}
\vspace{0mm}
\begin{description}
\item[\ding{224}] Metrics include changes in alignment scores across the six objectives, regularization terms to measure trade-offs, and statistical measures like the Pareto frontier to visualize multi-objective optimization.
\end{description}

\item[\ding{93}] {\fontfamily{lmss} \selectfont \textbf{Why is the Pareto frontier significant in the CAO framework?}}
\vspace{0mm}
\begin{description}
\item[\ding{224}] The Pareto frontier illustrates the trade-offs between different objectives, showing how improvements in one area (e.g., faithfulness) may require concessions in another (e.g., artistic freedom). CAO leverages this concept to optimize multiple objectives simultaneously.
\end{description}

\item[\ding{93}] {\fontfamily{lmss} \selectfont \textbf{What specific challenges does YinYangAlign address in the alignment of Text-to-Image (T2I) systems?}}
\vspace{0mm}
\begin{description}
\item[\ding{224}] YinYangAlign addresses the fundamental challenge of balancing multiple contradictory alignment objectives that are inherent to T2I systems. These include tensions such as adhering to user prompts (Faithfulness to Prompt) while allowing creative expression (Artistic Freedom) and maintaining cultural sensitivity without stifling artistic innovation. These challenges have been inadequately addressed by existing benchmarks, which often focus on singular objectives without considering their interplay.
\end{description}


\item[\ding{93}] {\fontfamily{lmss} \selectfont \textbf{What are the six contradictory alignment objectives, and why were they chosen for YinYangAlign?}}
\vspace{0mm}
\begin{description}
\item[\ding{224}] The six contradictory objectives are:
\begin{enumerate}
    \item Faithfulness to Prompt vs. Artistic Freedom: Ensures adherence to user instructions while allowing creative reinterpretation.
    \item Emotional Impact vs. Neutrality: Balances generating emotionally evocative images with unbiased representation.
    \item Visual Realism vs. Artistic Freedom: Maintains photorealism while allowing artistic stylization when appropriate.
    \item Originality vs. Referentiality: Promotes unique outputs while avoiding style plagiarism.
    \item Verifiability vs. Artistic Freedom: Ensures factual accuracy without restricting creativity.
    \item Cultural Sensitivity vs. Artistic Freedom: Preserves respectful cultural representations while fostering artistic freedom.
\end{enumerate}

These were selected based on their prevalence in real-world applications and their alignment with academic and ethical considerations in AI image generation.
\end{description}


\item[\ding{93}] {\fontfamily{lmss} \selectfont \textbf{How does Contradictory Alignment Optimization (CAO) differ from traditional Direct Preference Optimization (DPO)?}}
\vspace{0mm}
\begin{description}
\item[\ding{224}] CAO extends DPO by introducing a multi-objective optimization framework that simultaneously balances all six alignment objectives. It integrates:
\begin{itemize}
    \item Local Axiom-Wise Preferences: Loss functions that balance individual pairs of objectives (e.g., Faithfulness vs. Artistic Freedom).
    \item Global Synergy Mechanisms: A Pareto frontier-based optimization approach that ensures trade-offs across all objectives are harmonized.
    \item Axiom-Specific Regularization: Prevents overfitting to any single objective by stabilizing optimization with techniques like Wasserstein regularization.
\end{itemize}

\end{description}



\item[\ding{93}] {\fontfamily{lmss} \selectfont \textbf{How is the YinYangAlign dataset constructed, and what makes its annotation pipeline robust?}}
\vspace{0mm}
\begin{description}
\item[\ding{224}] The dataset is constructed using outputs from state-of-the-art T2I models (e.g., Stable Diffusion XL, MidJourney 6) and annotated through a two-step process:
\begin{itemize}
    \item Automated Annotation: Vision-Language Models (e.g., GPT-4o and LLaVA) generate preliminary annotations based on predefined scoring criteria for each objective.
    \item Human Verification: Annotations are validated by expert annotators, ensuring high reliability (kappa score of 0.83 across 500 samples). The pipeline balances scalability with rigorous quality control, enabling the creation of a robust benchmark.
\end{itemize}
\end{description}

\item[\ding{93}] {\fontfamily{lmss} \selectfont \textbf{How does CAO handle trade-offs between contradictory objectives, and what is the role of the synergy function?}}
\vspace{0mm}
\begin{description}
\item[\ding{224}] CAO uses a synergy function that aggregates local axiom-wise losses into a global multi-objective loss. By tuning synergy weights and leveraging Pareto optimality, CAO explores trade-offs systematically, identifying configurations where small sacrifices in one objective yield substantial gains in another. The synergy Jacobian further regulates gradient interactions, preventing any single objective from dominating the optimization process.
\end{description}

\item[\ding{93}] {\fontfamily{lmss} \selectfont \textbf{What are the computational implications of implementing CAO?}}
\vspace{0mm}
\begin{description}
\item[\ding{224}] CAO introduces computational overhead due to its multi-objective optimization framework, especially when incorporating regularization terms and global synergy functions. However, techniques such as Sinkhorn regularization and efficient Pareto front computation mitigate these challenges. Scalability to larger datasets or higher-dimensional objective spaces remains an area for further exploration.
\end{description}


\item[\ding{93}] {\fontfamily{lmss} \selectfont \textbf{How does YinYangAlign ensure adaptability to user-defined priorities?}}
\vspace{0mm}
\begin{description}
\item[\ding{224}] YinYangAlign incorporates a user-centric interface where sliders allow users to specify their preferred balance for each objective. These preferences are normalized into weights and integrated into the CAO framework, enabling dynamic adaptation to diverse application contexts. For example, users can prioritize Faithfulness to Prompt for precise visual representations or emphasize Artistic Freedom for creative outputs.
\end{description}

\item[\ding{93}] {\fontfamily{lmss} \selectfont \textbf{What are the limitations of YinYangAlign and the CAO framework?}}
\vspace{0mm}
\begin{description}
\item[\ding{224}] 
\begin{itemize}
    \item Dataset Limitations: The reliance on datasets like WikiArt and BAM may introduce biases, as they might not fully capture global cultural diversity.
    \item Irreconcilable Conflicts: Some objectives, such as Cultural Sensitivity and Emotional Impact, may conflict irreparably in certain scenarios, limiting CAO's effectiveness.
    \item Scalability: Balancing a growing number of alignment objectives may introduce optimization and computational challenges, necessitating hierarchical or modular approaches.
    \item Overfitting Risks: Overfitting to training data's specific trade-offs could reduce the model's generalizability to novel contexts.
\end{itemize}

\end{description}

\item[\ding{93}] {\fontfamily{lmss} \selectfont \textbf{What are the broader implications of this research for the field of generative AI?}}
\vspace{0mm}
\begin{description}
\item[\ding{224}] YinYangAlign sets a new standard for evaluating and designing T2I systems by addressing the nuanced interplay of competing alignment objectives. It emphasizes the importance of ethical considerations, user customization, and robust multi-objective optimization. The benchmark and CAO framework pave the way for future research into scalable, interpretable, and fair alignment strategies, extending their applicability to emerging challenges in generative AI.
\end{description}


   
\end{itemize}





\twocolumn
\newpage

\appendix

\section{Appendix: Prompt}
\label{sec:appendix}
``Here is a sketch of an image. 
$\{input\_color\_mask\}$, while the rest of the white space is the background. 
I need you to infer details of the image based on the given sketch.
The details should include the possible background likely to be present with the $\{input\_color\_mask\}$, the attribute of each object (like wearing, texture, color etc.), the state (including action, posture, etc.) of each object, the direction of each object and the relationships between objects.

You should first analyze the mask carefully, considering the size, location, and relative position of each object mask. Ensure that specific actions are analyzed based on the mask, and infer each aspect with a reasoning process before providing the final output.
The final output format should be: $\{format\_example\}$, and you should refer to the example: $\{few\_shot\}$. You are going to complete the "" in each item, you need to complete them in multiple short phrases based on your above reasoning.

The state and relationship should be as detailed as possible while ensuring they align with the mask, formatted as: objectA action/spatial relation objectB, with both objectA and objectB included.
You should properly refer to some examples of attributes of object $\{attributes\}$ and relationships $\{relationships\}$.
Do not include words like `or', `possibly' in your final output, there should no ambiguity in your output.
Make sure all aspects of given mask is filled.''

%\newpage
%\input{10_faq}
%\newpage
%\section*{Appendix}
The appendix contains supplementary data, which, while not part of the analysis, may provide additional details that can help in replicating the study. 

\subsection*{Prompt Engineering}
\label{sec: prompt engineering}
Tables~\ref{tab: prompt engineering} and \ref{tab: prompt constraint rationale} detail the prompts used in GPT and AI generative models, along with the specific constraints applied for each modality. According to the ChatGPT API specifications, the ``system role'' defines the model's assumed role in the session, while ``user input'' provides the prompt guiding the model's response.

\subsection*{Demographics Profile}
We collated participants' topic knowledge and topic interest (TK-TI) scores, their self-reported self-reflection and insight scale (SRIS) scores, VARK scores and external exposures and interactions with the different modalities in both reflective nudge (see Tables~\ref{tab: st1-demo} and \ref{tab: st2-demo}), to account for any potential fixed effects associated with these four covariates. 

In study 1, it's worth noting that all four covariates showed no statistically significant impact on the rankings of the modalities. Furthermore, we found no meaningful correlations between any of the four covariates and their interactions with any of the nudges.

In study 2, we report the covariates that are significant in section~\ref{sec: quantitative}.

\subsection*{Summary of Qualitative Feedback for Study 1}
Qualitative feedback for study 1 for both reflective nudges across the four modalities are summarized in Figures~\ref{fig: butterfly chart text} to \ref{fig: butterfly chart audio}. Feedback is classified using Kahneman's dual system thinking model~\cite{kahneman2002maps} to illustrate that the same modalities can produce varying or even contradictory results depending on the type of reflective nudge applied.

\newpage

\begin{table*}[!htbp]
\caption{The prompts utilized to generate the modalities along with the corresponding constraints}
\label{tab: prompt engineering}
\scalebox{0.6}{
\begin{tabular}{|l|lll|l|}
\hline
\textbf{Reflective Nudges} &
  \multicolumn{1}{l|}{\textbf{Prompt Template for Text Modality}} &
  \multicolumn{1}{l|}{\textbf{Prompt Template for Image Modality}} &
  \textbf{Prompt Template for Video Modality} &
  \textbf{Prompt Template for Audio Modality} \\ \hline
\textit{\begin{tabular}[c]{@{}l@{}}For both reflective \\ nudges\end{tabular}} &
  \multicolumn{3}{l|}{\begin{tabular}[c]{@{}l@{}}\textbf{system role:} You are a helpful assistant focusing on supporting users' self-reflection on a given topic.\\      \\ \textbf{User input:} Topic: {[}topic{]}.\end{tabular}} &
  \multirow{3}{*}{\begin{tabular}[c]{@{}l@{}}Audio prompts are not required for this \\ process, as the text generated in the text \\ modality is directly input into the text-to-\\ speech AI tool. The tool automatically \\ converts the provided text into audio, \\ generating the narration based on the \\ content supplied. We manually \\ selected a gender-appropriate voice that \\ matches the gender as specified in the \\ text modality.\end{tabular}} \\ \cline{1-4}
\begin{tabular}[c]{@{}l@{}}Direct Reflective \\ Nudge (Persona)\end{tabular} &
  \multicolumn{1}{l|}{\begin{tabular}[c]{@{}l@{}}For the above topic, create ten distinct personas \\ representing different perspectives on the topic. \\ Provide the name, age and occupation for each \\ persona. \\      \\ Here is the format of the results: \\ 1. {[}Name1{]}, {[}Age1{]}, {[}Occupation1{]}, {[}Perspective1{]}\\ 2. {[}Name2{]}, {[}Age2{]}, {[}Occupation2{]}, {[}Perspective2{]}\\      ...\\      \\ Requirements:\\ Create five male personas and five female personas;\\ For each perspective, be concise, giving at most three \\ sentences;\\ No duplicates;\\ Ten distinct versions only\end{tabular}} &
  \multicolumn{1}{l|}{\begin{tabular}[c]{@{}l@{}}\textit{For each persona generated in the text} \\ \textit{modality, we use the following prompt to} \\ \textit{generate the corresponding image via AI:}\\      \\ Persona: {[}persona1{]}     \\ Create a photorealistic image with realistic \\ textures and lighting of the persona. The\\ background should be related to the \\ occupation of the persona. \\      \\ Requirements: \\ Age, gender and occupation should follow \\ the persona. The style is photorealistic and \\ not cartoonish. The image has no wordings.\end{tabular}} &
  \begin{tabular}[c]{@{}l@{}}\textit{For each persona generated in the text} \\ \textit{modality, we input the entire script into} \\ \textit{the AI.} \textit{We then use the following prompt} \\ \textit{to generate the corresponding video:}\\      \\ Persona: {[}persona1{]}     \\ Create a photorealistic video featuring the \\ persona. The background should be realistic \\ and aligned with the persona’s occupation \\ and worldview, with natural lighting and no \\ cartoonish elements. The voice narration \\ should match the persona’s gender and follow \\ the script exactly as provided, with no \\ additional text. The video should be visually \\ compelling and immersive, showcasing \\ a background relevant to the persona’s \\ occupation or environment. \\      \\ Requirements: \\ Resolution: 1080p\\ Audience: Relevant to the persona’s occupation\\ Style: Photorealistic, professional\\ Platform: YouTube Shorts or Instagram Reels\\ Script: Follow the provided persona text exactly \\ without any modifications.\end{tabular} &
   \\ \cline{1-4}
\begin{tabular}[c]{@{}l@{}}Indirect Reflective \\ Nudge (Storytelling)\end{tabular} &
  \multicolumn{1}{l|}{\begin{tabular}[c]{@{}l@{}}Following the ten personas created earlier, generate a \\ story for each of the persona on the topic matter. \\      \\ Here is the format of the results:\\ Story1:\\ Story2:\\ …\\      \\ Requirements:\\ Create five stories with a positive tone and \\ five stories with a negative tone;\\ No duplicates or similar story line;\\ Ten distinct versions only\\      \\ * Depending on the length of the story generated, \\ we prompt GPT to either lengthen (Extend Story1) \\ or condense (Make Story1 concise) the narrative.\end{tabular}} &
  \multicolumn{1}{l|}{\begin{tabular}[c]{@{}l@{}}\textit{For each story generated in the text modality,} \\ \textit{we prompt AI to create images that visually} \\ \textit{represent key moments in the narrative. We} \\ \textit{generate one image at a time, which are then} \\ \textit{combined to form a cohesive visual }\\ \textit{representation of the entire story.}\\      \\ Section of the story: {[}story section1{]}.     \\ Create a photorealistic image with realistic \\ textures and lighting of the section of the story. \\ The background should be related to the \\ occupation of the character in the story. \\ \\      \\ Requirements: \\ Age, gender and occupation should follow \\ the character in the story.\\ The style is photorealistic and not cartoonish. \\ The image has no wordings. \\ \textasciicircum The style should be consistent with the \\ previous image.\\      \\ \textasciicircum Only for subsequent images generated \\ by the AI to maintain visual coherence.\end{tabular}} &
  \begin{tabular}[c]{@{}l@{}}\textit{For each story generated in the text modality, }\\ \textit{we input the entire script into the AI.} \\ \textit{Pauses were manually added to ensure natural }\\ \textit{pacing and alignment with the story's speech }\\ \textit{patterns. We then use the following prompt to }\\ \textit{generate the corresponding video:}\\      \\ Story: {[}story1{]}     \\ Create a photorealistic video featuring the \\ story. The background should be realistic \\ and aligned with the character's occupation and \\ worldview, with natural lighting and no cartoonish \\ elements. The voice narration should match the \\ character's gender and follow the script exactly as \\ provided, with no additional text. The video should \\ be visually compelling and immersive, showcasing \\ a background relevant to the character's \\ occupation or environment. \\      \\ Requirements: \\ Resolution: 1080p\\ Audience: Relevant to the character's occupation\\ Style: Photorealistic, professional\\ Platform: YouTube Shorts or Instagram Reels\\ Script: Follow the provided text exactly without any \\ modifications.\end{tabular} &
   \\ \hline
\end{tabular}}
\Description{This table details the prompts used to generate the variants for each of the modalities, along with the corresponding constraints applied when communicating with the ChatGPT API and AI generative models. These prompts adhere to the template established by White et al., which entails defining a task, incorporating constraints, and setting clear expectations for the generated output.}
\end{table*}

\newpage

\begin{table*}[!htbp]
\caption{Rationales of the constraints set out for GPT and the AI Generative tools}
\label{tab: prompt constraint rationale}
\centering
\begin{tblr}{
  width = \textwidth,
  colspec = {Q[352]Q[588]},
  row{1} = {c},
  hlines,}
\textbf{Constraints} & \textbf{Rationale} \\
{(Direct Reflective Nudge: Persona) Create five male personas and five female personas. \\~} & {\labelitemi\hspace{\dimexpr\labelsep+0.5\tabcolsep}Scholars have found that large language models such as GPT-3 produce gender stereotypes and biases~\cite{brown2020language,lucy2021gender, huang2019reducing, nozza2021honest, johnson2022ghost} \\\labelitemi\hspace{\dimexpr\labelsep+0.5\tabcolsep} Hence, this constraint was set out to mitigate any potential gender imbalances as the discussion topic is a contentious one. Moreover, GPT was specifically instructed to assign gender-neutral names to the personas. This approach prevents the association of particular names with male or female identities, fostering a more equitable and unbiased representation. \\\labelitemi\hspace{\dimexpr\labelsep+0.5\tabcolsep} Ensures GPT does not provide viewpoints that gives preferential treatment of one gender over the other.} \\
(Direct Reflective Nudge: Persona) For each perspective, be concise, giving at most three sentences. & {\labelitemi\hspace{\dimexpr\labelsep+0.5\tabcolsep}This constraint was introduced after observing GPT's tendency to generate long perspectives during testing. \\\labelitemi\hspace{\dimexpr\labelsep+0.5\tabcolsep} Ensures GPT deliver concise perspectives.} \\
(Indirect Reflective Nudge: Storytelling) Of the ten stories, create five with a positive tone and five with a negative tone. & \labelitemi\hspace{\dimexpr\labelsep+0.5\tabcolsep} Ensures that the reflector captures a broad spectrum of impacts, including both positive and negative aspects to avert one-sidedness. \\
(Indirect Reflective Nudge: Storytelling) Extend or make concise. & {\labelitemi\hspace{\dimexpr\labelsep+0.5\tabcolsep} Ensures a well-rounded representation of stories --- three long stories, three short stories, and four stories of medium length. \\\labelitemi\hspace{\dimexpr\labelsep+0.5\tabcolsep} This strategy allows us to capture the full spectrum of narrative possibilities and thereby provides a more thorough analysis of the reflector's effectiveness.} \\
(Image Modality) Image has no wordings. & {\labelitemi\hspace{\dimexpr\labelsep+0.5\tabcolsep} AI-generated images often produce words that are not human-readable, as the algorithms focus on replicating the visual shapes of letters and numbers rather than rendering actual text. This issue is similar to the challenges AI faces in accurately generating human hands, which frequently results in distorted representations due to the complexity of shapes involved~\cite{keyes2023hands}.} \\
(Video Modality) Platform: YouTube Shorts or Instagram Reels. & {\labelitemi\hspace{\dimexpr\labelsep+0.5\tabcolsep} Ensures that the generated video aligns with the format requirements of popular media platforms such as TikTok, Instagram Reels, and YouTube Shorts. By adhering to these guidelines, the video remains concise, typically between 90-120 seconds, in line with the standard duration of videos commonly found on these platforms.} \\
(Video Modality) Script: Follow the provided text exactly without any modifications. & {\labelitemi\hspace{\dimexpr\labelsep+0.5\tabcolsep} During testing, AI video generation platforms often augment scripts by adding extra narrative elements to enhance immersion and provide more context. To maintain consistency across all modalities, we explicitly constrain the AI to avoid any content alterations, ensuring that the script remains unchanged throughout the different formats.} \\
Here is the format of the results & \labelitemi\hspace{\dimexpr\labelsep+0.5\tabcolsep} Ensures that GPT provides results in a consistent format.      
\end{tblr}
\Description{This table provides a breakdown of the constraints applied to GPT and AI generative models during the generation of the variants for each modality, along with the underlying rationales for each constraint. The table has two columns with column header: constraints and rationale. In the constraints column, we list the specific constraints implemented for each modality and reflective nudge. Meanwhile, the rationale column elucidates the motivations behind the establishment of these constraints.}
\end{table*}

\newpage

\begin{table*}[!htbp]
\caption{Demographic Profiles of Participants in Study 1}
\label{tab: st1-demo}
\scalebox{0.74}{
\begin{tabular}{|ll|c|c|}
\hline
\multicolumn{1}{|l|}{\textbf{Category}} &
  \textbf{Dimensions} &
  \textbf{Direct Reflective Nudge (Persona)} &
  \textbf{Indirect Reflective Nudge (Storytelling)} \\ \hline
\multicolumn{2}{|l|}{Total   Number of Participants} &
  10 &
  10 \\ \hline
\multicolumn{1}{|l|}{\multirow{2}{*}{Gender}} &
  Total Number of Males &
  4 &
  4 \\ \cline{2-4} 
\multicolumn{1}{|l|}{} &
  Total Number of Females &
  6 &
  6 \\ \hline
\multicolumn{1}{|l|}{Age} &
  Average Age &
  23.9 &
  24.1 \\ \hline
\multicolumn{1}{|l|}{\multirow{2}{*}{Ethnicity}} &
  Asian or Pacific Islander &
  10 &
  9 \\ \cline{2-4} 
\multicolumn{1}{|l|}{} &
  Hispanic or Latino &
  0 &
  1 \\ \hline
\multicolumn{1}{|l|}{\multirow{2}{*}{Education}} &
  Bachelor's Degree &
  8 &
  8 \\ \cline{2-4} 
\multicolumn{1}{|l|}{} &
  Post-Graudate Degree &
  2 &
  2 \\ \hline
\multicolumn{1}{|l|}{\multirow{2}{*}{Language}} &
  English is first language &
  7 &
  8 \\ \cline{2-4} 
\multicolumn{1}{|l|}{} &
  English is NOT first language &
  3 &
  2 \\ \hline
\multicolumn{2}{|l|}{TITK Score ($M \pm S.D.$)} &
  $20.10 \pm 5.07$ &
  $18.70 \pm 3.92$ \\ \hline
\multicolumn{2}{|l|}{SRIS Score ($M \pm S.D.$)} &
  $89.00 \pm 11.07$ &
  $86.80 \pm 10.05$ \\ \hline
\multicolumn{1}{|l|}{\multirow{4}{*}{VARK Model (Internal -   Inherent Reflecting Styles)}} &
  Visual (V) Score ($M \pm S.D.$) &
  $8.70 \pm 2.71$ &
  $9.10 \pm 3.31$ \\ \cline{2-4} 
\multicolumn{1}{|l|}{} &
  Audio (A) Score ($M \pm S.D.$) &
  $7.70 \pm 3.30$ &
  $5.10 \pm 2.18$ \\ \cline{2-4} 
\multicolumn{1}{|l|}{} &
  Read/Write (R) Score ($M \pm S.D.$) &
  $6.80 \pm 3.71$ &
  $7.00 \pm 3.46$ \\ \cline{2-4} 
\multicolumn{1}{|l|}{} &
  Kinesthetic (K) Score ($M \pm S.D.$) &
  $10.90 \pm 2.64$ &
  $9.70 \pm 2.26$ \\ \hline
\multicolumn{1}{|l|}{\multirow{4}{*}{Daily Usage (External -  Influences and Interactions)}} &
  \begin{tabular}[c]{@{}l@{}}Frequency of Exposure to Text:\\ Blog Posts, Online Articles, News \\ ($M \pm S.D.$)\end{tabular} &
  $4.20 \pm 0.79$ &
  $3.30 \pm 1.25$ \\ \cline{2-4} 
\multicolumn{1}{|l|}{} &
  \begin{tabular}[c]{@{}l@{}}Frequency of Exposure to Images: \\ Instagram Posts, Facebook Posts, Pinterest\\ ($M \pm S.D.$)\end{tabular} &
  $4.00 \pm 0.94$ &
  $3.40 \pm 1.43$ \\ \cline{2-4} 
\multicolumn{1}{|l|}{} &
  \begin{tabular}[c]{@{}l@{}}Frequency of Exposure to Video:\\ Instagram Reels, Youtube Shorts, TikTok\\ ($M \pm S.D.$)\end{tabular} &
  $4.60 \pm 0.97$ &
  $4.10 \pm 1.45$ \\ \cline{2-4} 
\multicolumn{1}{|l|}{} &
  \begin{tabular}[c]{@{}l@{}}Frequency of Exposure to Audio: \\ Podcasts ($M \pm S.D.$)\end{tabular} &
  $2.20 \pm 1.23$ &
  $2.50 \pm 1.18$ \\ \hline
\multicolumn{2}{|l|}{Average Completion Time in Minutes} &
  45.6 &
  48.0 \\ \hline
\end{tabular}}
\Description{This table provides a comprehensive overview of the demographic characteristics of the participants in Study 1. Demographic profile captures Participants, Gender, Age, Education, TK-TI Scores, SRIS Scores, VARK scores and Daily Interactions with the modalities.}
\end{table*}


\begin{table*}[!htbp]
\caption{Demographic Profiles of Participants in Study 2}
\label{tab: st2-demo}
\scalebox{0.5}{
\begin{tabular}{|ll|c|c|c|c|c|c|c|c|}
\hline
\multicolumn{1}{|l|}{\textbf{Category}} &
  \textbf{Dimensions} &
  \textbf{Persona (Text)} &
  \textbf{Persona (Image)} &
  \textbf{Persona (Video)} &
  \textbf{Persona (Audio)} &
  \textbf{Storytelling (Text)} &
  \textbf{Storytelling (Image)} &
  \textbf{Storytelling (Video)} &
  \textbf{Storytelling (Audio)} \\ \hline
\multicolumn{2}{|l|}{Total Number of Participants} &
  25 &
  25 &
  25 &
  25 &
  25 &
  25 &
  25 &
  25 \\ \hline
\multicolumn{1}{|l|}{\multirow{3}{*}{Gender}} &
  Total Number of Males &
  14 &
  11 &
  11 &
  12 &
  13 &
  11 &
  15 &
  18 \\ \cline{2-10} 
\multicolumn{1}{|l|}{} &
  Total Number of Females &
  10 &
  14 &
  14 &
  13 &
  12 &
  14 &
  10 &
  7 \\ \cline{2-10} 
\multicolumn{1}{|l|}{} &
  Prefer Not to Say &
  1 &
  0 &
  0 &
  0 &
  0 &
  0 &
  0 &
  0 \\ \hline
\multicolumn{1}{|l|}{Age} &
  Average Age &
  30.72 &
  34.28 &
  32.48 &
  35.84 &
  33.68 &
  33.44 &
  37.32 &
  36.52 \\ \hline
\multicolumn{1}{|l|}{\multirow{7}{*}{Ethnicity}} &
  Asian or Pacific Islander &
  3 &
  2 &
  0 &
  2 &
  0 &
  0 &
  3 &
  3 \\ \cline{2-10} 
\multicolumn{1}{|l|}{} &
  Black or African American &
  0 &
  0 &
  1 &
  1 &
  1 &
  0 &
  0 &
  0 \\ \cline{2-10} 
\multicolumn{1}{|l|}{} &
  Hispanic or Latino &
  0 &
  1 &
  1 &
  0 &
  0 &
  0 &
  0 &
  1 \\ \cline{2-10} 
\multicolumn{1}{|l|}{} &
  Multiracial or Biracial &
  0 &
  0 &
  0 &
  0 &
  0 &
  0 &
  0 &
  0 \\ \cline{2-10} 
\multicolumn{1}{|l|}{} &
  Native American or Alaska Native &
  3 &
  0 &
  1 &
  1 &
  0 &
  1 &
  0 &
  0 \\ \cline{2-10} 
\multicolumn{1}{|l|}{} &
  White or Caucasian &
  18 &
  22 &
  22 &
  21 &
  24 &
  24 &
  22 &
  21 \\ \cline{2-10} 
\multicolumn{1}{|l|}{} &
  Other &
  1 &
  0 &
  0 &
  0 &
  0 &
  0 &
  0 &
  0 \\ \hline
\multicolumn{1}{|l|}{\multirow{5}{*}{Education}} &
  No Diploma or less &
  0 &
  0 &
  0 &
  0 &
  0 &
  0 &
  0 &
  0 \\ \cline{2-10} 
\multicolumn{1}{|l|}{} &
  High School, Diploma or the equivalent &
  3 &
  1 &
  2 &
  2 &
  0 &
  1 &
  0 &
  2 \\ \cline{2-10} 
\multicolumn{1}{|l|}{} &
  Associate's Degree &
  1 &
  1 &
  1 &
  2 &
  1 &
  1 &
  0 &
  3 \\ \cline{2-10} 
\multicolumn{1}{|l|}{} &
  Bachelor's Degree &
  15 &
  23 &
  19 &
  20 &
  23 &
  18 &
  20 &
  18 \\ \cline{2-10} 
\multicolumn{1}{|l|}{} &
  Post-graduate Degree &
  6 &
  0 &
  3 &
  1 &
  1 &
  5 &
  5 &
  2 \\ \hline
\multicolumn{1}{|l|}{\multirow{2}{*}{Language}} &
  English is first language &
  23 &
  25 &
  24 &
  25 &
  25 &
  25 &
  24 &
  25 \\ \cline{2-10} 
\multicolumn{1}{|l|}{} &
  English is NOT first language &
  2 &
  0 &
  1 &
  0 &
  0 &
  0 &
  1 &
  0 \\ \hline
\multicolumn{2}{|l|}{TITK Score ($M \pm S.D.$)} &
  $21.04 \pm 4.89$ &
  $23.16 \pm 3.50$ &
  $22.28 \pm 5.26$ &
  $22.28 \pm 6.78$ &
  $23.20 \pm 4.37$ &
  $23.20 \pm 5.98$ &
  $27.08 \pm 5.99$ &
  $26.12 \pm 5.90$ \\ \hline
\multicolumn{2}{|l|}{SRIS Score ($M \pm S.D.$)} &
  $81.32 \pm 11.91$ &
  $84.84 \pm 9.48$ &
  $84.64 \pm 13.73$ &
  $85.64 \pm 12.68$ &
  $84.68 \pm 18.45$ &
  $86.24 \pm 13.82$ &
  $83.20 \pm 12.43$ &
  $82.76 \pm 10.33$ \\ \hline
\multicolumn{1}{|l|}{\multirow{4}{*}{VARK Model (Internal -   Inherent Reflecting Styles)}} &
  Visual (V) Score ($M \pm S.D.$) &
  $7.52 \pm 3.11$ &
  $8.40 \pm 3.72$ &
  $7.92 \pm 3.98$ &
  $6.92 \pm 3.59$ &
  $7.68 \pm 3.87$ &
  $5.40 \pm 2.47$ &
  $6.12 \pm 2.71$ &
  $5.88 \pm 3.23$ \\ \cline{2-10} 
\multicolumn{1}{|l|}{} &
  Audio (A) Score ($M \pm S.D.$) &
  $8.96 \pm 3.85$ &
  $7.36 \pm 2.84$ &
  $8.48 \pm 3.63$ &
  $7.36 \pm 4.19$ &
  $8.28 \pm 3.68$ &
  $7.88 \pm 3.10$ &
  $8.64 \pm 2.96$ &
  $6.80 \pm 2.33$ \\ \cline{2-10} 
\multicolumn{1}{|l|}{} &
  Read/Write (R) Score ($M \pm S.D.$) &
  $7.96 \pm 3.51$ &
  $7.68 \pm 3.20$ &
  $8.16 \pm 2.78$ &
  $7.36 \pm 3.77$ &
  $7.48 \pm 3.79$ &
  $6.92 \pm 2.41$ &
  $6.32 \pm 2.76$ &
  $6.12 \pm 3.49$ \\ \cline{2-10} 
\multicolumn{1}{|l|}{} &
  Kinesthetic (K) Score ($M \pm S.D.$) &
  $8.44 \pm 3.50$ &
  $9.00 \pm 3.35$ &
  $8.08 \pm 2.71$ &
  $7.72 \pm 3.66$ &
  $8.52 \pm 3.20$ &
  $6.84 \pm 3.10$ &
  $8.32 \pm 2.41$ &
  $7.20 \pm 2.72$ \\ \hline
\multicolumn{1}{|l|}{\multirow{4}{*}{Daily Usage (External -   Influences and Interactions)}} &
  \begin{tabular}[c]{@{}l@{}}Frequency of Exposure to Text: \\ Blog Posts, Online Articles, News\\ ($M \pm S.D.$)\end{tabular} &
  $3.72 \pm 1.10$ &
  $3.80 \pm 0.65$ &
  $3.84 \pm 0.99$ &
  $3.92 \pm 0.91$ &
  $3.76 \pm 0.72$ &
  $3.72 \pm 0.98$ &
  $4.20 \pm 0.71$ &
  $4.24 \pm 0.88$ \\ \cline{2-10} 
\multicolumn{1}{|l|}{} &
  \begin{tabular}[c]{@{}l@{}}Frequency of Exposure to Images: \\ Instagram Posts, Facebook   Posts, Pinterest\\ ($M \pm S.D.$)\end{tabular} &
  $4.28 \pm 0.74$ &
  $4.24 \pm 0.93$ &
  $4.24 \pm 0.88$ &
  $4.36 \pm 1.11$ &
  $4.04 \pm 0.93$ &
  $4.04 \pm 1.06$ &
  $4.68 \pm 0.48$ &
  $4.32 \pm 0.99$ \\ \cline{2-10} 
\multicolumn{1}{|l|}{} &
  \begin{tabular}[c]{@{}l@{}}Frequency of Exposure to Video:\\ Instagram Reels, Youtube Shorts, TikTok\\ ($M \pm S.D.$)\end{tabular} &
  $4.20 \pm 0.91$ &
  $4.56 \pm 0.87$ &
  $4.08 \pm 0.95$ &
  $4.44 \pm 0.96$ &
  $4.40 \pm 0.76$ &
  $4.16 \pm 0.99$ &
  $4.64 \pm 0.70$ &
  $4.44 \pm 0.92$ \\ \cline{2-10} 
\multicolumn{1}{|l|}{} &
  \begin{tabular}[c]{@{}l@{}}Frequency of Exposure to Audio: \\ Podcasts ($M \pm S.D.$)\end{tabular} &
  $3.76 \pm 1.16$ &
  $3.52 \pm 1.08$ &
  $3.60 \pm 1.15$ &
  $3.68 \pm 1.07$ &
  $3.56 \pm 1.00$ &
  $3.56 \pm 1.04$ &
  $3.92 \pm 1.00$ &
  $3.88 \pm 1.10$ \\ \hline
\multicolumn{2}{|l|}{Average Completion Time in Minutes} &
  25.66 &
  47.22 &
  25.50 &
  32.18 &
  28.31 &
  32.88 &
  24.34 &
  24.96 \\ \hline
\end{tabular}}
\Description{This table provides a comprehensive overview of the demographic characteristics of the participants in Study 2. Demographic profile captures Participants, Gender, Age, Education, TK-TI Scores, SRIS Scores, VARK scores and Daily Interactions with the modalities.}
\end{table*}

\clearpage
\newpage

\begin{figure*}[!htbp]
  \centering
  \includegraphics[width=\linewidth]{figures/Butterfly_Text.png}
  \caption{Butterfly Chart for the Text Modality across Both Nudge Types}
  \label{fig: butterfly chart text}
  \Description{Butterfly chart showing how the text modality fare for each of the categories under Kahneman's dual-system thinking model.}
\end{figure*}

\begin{figure*}[!htbp]
  \centering
  \includegraphics[width=\linewidth]{figures/Butterfly_Image.png}
  \caption{Butterfly Chart for the Image Modality across Both Nudge Types}
  \label{fig: butterfly chart image}
  \Description{Butterfly chart showing how the image modality fare for each of the categories under Kahneman's dual-system thinking model.}
\end{figure*}

\begin{figure*}[!htbp]
  \centering
  \includegraphics[width=\linewidth]{figures/Butterfly_Video.png}
  \caption{Butterfly Chart for the Video Modality across Both Nudge Types}
  \label{fig: butterfly chart video}
  \Description{Butterfly chart showing how the video modality fare for each of the categories under Kahneman's dual-system thinking model.}
\end{figure*}

\begin{figure*}[!htbp]
  \centering
  \includegraphics[width=\linewidth]{figures/Butterfly_Audio.png}
  \caption{Butterfly Chart for the Audio Modality across Both Nudge Types}
  \label{fig: butterfly chart audio}
  \Description{Butterfly chart showing how the audio modality fare for each of the categories under Kahneman's dual-system thinking model.}
\end{figure*}

\end{document}