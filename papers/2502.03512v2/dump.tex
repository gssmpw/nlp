----------------------------------------
Originality is measured as divergence from common styles or references using CLIP-based style embeddings:
\begin{equation}
    \mathcal{L}_{originality} = \text{sim}(E(I_{gen}), E(S_{\text{ref}}))
\end{equation}
where $E(S_{\text{ref}})$ represents embeddings of popular style categories (e.g., "impressionism," "surrealism").

\subsection{Verifiability Loss ($\mathcal{L}_{verifiability}$)}
Verifiability evaluates alignment with factual accuracy, using pretrained LLMs and multimodal models:
\begin{equation}
    \mathcal{L}_{verifiability} = 1 - \text{sim}(E(P_{\text{fact}}), E(I_{gen}))
\end{equation}
where $P_{\text{fact}}$ is the factually enriched prompt embedding.

\subsection{Referentiality Score (RS)}
\label{subsec:referentiality_score}

\textbf{Overview:} The Referentiality Score (RS) quantifies the alignment of the generated image with known artistic styles or references using a pretrained CLIP model. This automated approach avoids the need for training a classifier by leveraging textual descriptions of styles.

\subsubsection{CLIP-Based Style Similarity}
To compute RS, we compare the generated image embedding with textual embeddings of style descriptions (e.g., "impressionist painting", "surrealist art", "photorealistic image") using the CLIP model.

\textbf{Steps:}
\begin{enumerate}
    \item Extract the embedding for the generated image:
    \[
    E(I_{\text{gen}})
    \]
    where $I_{\text{gen}}$ is the generated image.
    
    \item Define a set of textual descriptions of artistic styles:
    \[
    T = \{ "impressionist painting", "surrealist art", "photorealistic image", \dots \}
    \]
    Compute embeddings for each textual description:
    \[
    E(T_i) \quad \forall \, T_i \in T
    \]
    
    \item Calculate the cosine similarity between the image embedding and each textual embedding:
    \[
    \text{Similarity}(I_{\text{gen}}, T_i) = \frac{E(I_{\text{gen}}) \cdot E(T_i)}{\|E(I_{\text{gen}})\| \|E(T_i)\|}
    \]
    
    \item Aggregate the similarities to compute RS. For example, use the maximum similarity across all textual descriptions:
    \[
    RS = \max_{i} \text{Similarity}(I_{\text{gen}}, T_i)
    \]
\end{enumerate}

\textbf{Advantages:}
\begin{itemize}
    \item Fully automated and does not require additional training.
    \item Easily extendable by adding new textual descriptions of styles.
    \item Leverages the representational strength of CLIP embeddings for style and semantic alignment.
\end{itemize}

\textbf{Example:}
\begin{itemize}
    \item \textbf{Prompt:} "Generate a painting in the style of Van Gogh."
    \item \textbf{Generated Image:} An image with swirling brushstrokes and vivid, contrasting colors.
    \item \textbf{Textual Descriptions:} { "impressionist painting," "surrealist art," "photorealistic image," "abstract geometric design," "digital art with futuristic themes," "classical oil painting," "modern pop art," "black-and-white sketch," "comic book style illustration," and "cubist artwork."
    \item \textbf{Computed RS:}
    \[
    RS = \max \{ 0.65, 0.32, 0.85 \} = 0.85
    \]
\end{itemize}

\textbf{Citation:} This method is based on the CLIP model introduced by Radford et al. (2021) \cite{radford2021learning}.


\subsection{Cultural Sensitivity Loss ($\mathcal{L}_{cultural}$)}
Cultural Sensitivity is assessed via Simulated Cultural Context Matching (SCCM):
\begin{equation}
    \mathcal{L}_{cultural} = 1 - \frac{1}{N} \sum_{i=1}^N \text{sim}(E(P_i), E(I_{gen}))
\end{equation}
where $P_i$ are culturally contextual prompts generated dynamically by LLMs.

\subsection{Artistic Freedom Loss ($\mathcal{L}_{artistic}$)}
Artistic Freedom is measured by comparing creative deviations from a baseline image:
\begin{equation}
    \mathcal{L}_{artistic} = \alpha \cdot \text{StyleDiff} + \beta \cdot \text{ContentDiff}
\end{equation}
where $\text{StyleDiff}$ and $\text{ContentDiff}$ are computed using style and content embeddings, respectively, as detailed in Section~\ref{sec:style_transfer}.

\subsection{Hyperparameter Tuning}
The weights $\alpha, \beta, \gamma, \delta, \eta, \theta, \lambda$ are optimized to balance the tradeoff between conflicting objectives. Grid search or Bayesian optimization can be employed to determine optimal values for specific tasks.



\subsection{Faithfulness to Prompt vs. Creative Enhancement}
We propose a hybrid approach to evaluate \textbf{Faithfulness to Prompt} and \textbf{Creative Enhancement} using two automated metrics: the \textbf{Semantic Alignment Score (SAS)} based on Sinkhorn-VAE Wasserstein Distance and the \textbf{Creative Enhancement Score (CES)} based on Style Transfer Deviation (STD). These metrics are combined into a single \textbf{Tradeoff Score} to quantify the balance.

\subsubsection{Semantic Alignment Score (SAS)}
The SAS quantifies the semantic alignment between the user prompt and the generated image using the \textbf{Sinkhorn-VAE Wasserstein Distance}:

\begin{equation}
SAS_{\text{Sinkhorn-VAE}} = -W_d^\lambda(P(Z_{\text{prompt}}), Q(Z_{\text{image}}))
\end{equation}

\noindent where:
\begin{itemize}
    \item $P(Z_{\text{prompt}})$ and $Q(Z_{\text{image}})$ are latent distributions of the prompt and image embeddings.
    \item $W_d^\lambda$ represents the Sinkhorn-regularized Wasserstein Distance \cite{peyreetal2020, wasserstein_autoencoders}.
\end{itemize}

Higher $SAS_{\text{Sinkhorn-VAE}}$ values indicate better adherence to the prompt.

\subsubsection{Creative Enhancement Score (CES)}
The CES quantifies creative enhancements by comparing the generated image ($I_{\text{gen}}$) with:
\begin{itemize}
    \item A baseline image generated with minimal creative enhancement ($I_{\text{base}}$).
    \item The textual prompt to ensure high-level fidelity.
\end{itemize}

The CES consists of two components:
\begin{enumerate}
    \item \textbf{Style Difference:} Measures visual style differences between $I_{\text{gen}}$ and $I_{\text{base}}$ using style embeddings:
    \begin{equation}
    \text{StyleDiff} = \| S(I_{\text{gen}}) - S(I_{\text{base}}) \|_2
    \end{equation}
    where $S(\cdot)$ represents a pretrained network (e.g., VGG) that extracts style embeddings via Gram Matrices \cite{gatys2016neural}.
    
    \item \textbf{Content Difference:} Measures divergence in visual content by comparing embeddings of $I_{\text{gen}}$ and $I_{\text{base}}$ using cosine similarity:
    \begin{equation}
    \text{ContentDiff} = 1 - \cos(E(I_{\text{gen}}), E(I_{\text{base}}))
    \end{equation}
    where $E(\cdot)$ represents a feature extraction model (e.g., CLIP) for content embeddings \cite{radford2021clip}.
\end{enumerate}

The CES is computed as:
\begin{equation}
CES = \alpha \cdot \text{StyleDiff} + \beta \cdot \text{ContentDiff}
\end{equation}
where $\alpha$ and $\beta$ control the weight of style versus content.

\subsubsection{Faithfulness and Creativity Tradeoff}
The tradeoff between Faithfulness and Creativity is represented as:
\begin{equation}
\text{Tradeoff} = \gamma \cdot SAS + (1 - \gamma) \cdot CES
\end{equation}
where $\gamma \in [0,1]$ determines the priority given to faithfulness versus creativity.

\subsection{Example Application}
\begin{itemize}
    \item \textbf{Prompt:} “Illustrate a serene garden with a bench under a cherry blossom tree.”
    \begin{itemize}
        \item \textbf{Chosen Response:} A serene garden scene with sunlight filtering through cherry blossoms and butterflies fluttering around. 
        \begin{itemize}
            \item $SAS_{\text{Sinkhorn-VAE}} = -0.12$, $CES = 0.05$
        \end{itemize}
        \item \textbf{Rejected Response:} A glowing neon cherry blossom tree with a levitating bench.
        \begin{itemize}
            \item $SAS_{\text{Sinkhorn-VAE}} = -0.50$, $CES = 0.45$
        \end{itemize}
    \end{itemize}
    \item \textbf{Tradeoff Score:} $0.7 \cdot (-0.12) + 0.3 \cdot 0.05 = -0.066$ (Chosen response is better aligned).
\end{itemize}

\subsection{Advantages}
\begin{itemize}
    \item \textbf{Scalability:} Fully automated metrics enable large-scale evaluation.
    \item \textbf{Semantic-Artistic Balance:} Captures both faithfulness and creativity in a unified framework.
    \item \textbf{Robustness:} Accounts for style and content variability.
\end{itemize}


\begin{figure}[h!]
    \centering
    \includegraphics[width=\columnwidth]{img/Heatmap_for_SAS_and_CES_Tradeoff.png}
    \caption{}
    \label{fig:}
\end{figure}

\subsection{Diversity vs. Coherence}
\label{subsec:diversity_coherence}

\textbf{Overview:} Diversity and Coherence are fundamental yet often conflicting objectives in the evaluation of Text-to-Image (T2I) systems. Diversity fosters creative exploration by encouraging varied outputs, while Coherence ensures logical consistency and adherence to the user’s prompt. A well-aligned T2I system must balance these objectives, generating outputs that are both diverse and semantically faithful. To this end, we define two metrics: Batch Diversity (BD) and Content Matching (CM).

\textbf{Batch Definition:} A batch is defined as \(N=10\) images generated for the same prompt. This approach captures the inherent stochasticity of T2I systems, reflecting how users often request multiple outputs for creative exploration. This batch size strikes a balance between computational feasibility and adequately representing system variability.

\subsubsection{Batch Diversity (BD)}
Batch Diversity quantifies the variation among images generated from a single prompt. For a batch of images \(\{I_1, I_2, \dots, I_N\}\) with embeddings \(\{E(I_1), E(I_2), \dots, E(I_N)\}\) extracted using a pretrained feature extractor (e.g., CLIP \cite{radford2021learning}), BD is defined as:

\begin{equation}
    BD = \frac{1}{N(N-1)} \sum_{i=1}^{N} \sum_{j=i+1}^{N} \text{d}(E(I_i), E(I_j)),
\end{equation}

where \(\text{d}(\cdot, \cdot)\) represents a pairwise distance metric such as cosine distance or Wasserstein distance \cite{arjovsky2017wasserstein}. BD captures the average pairwise feature distance, measuring the model’s ability to generate visually distinct outputs.

This metric builds on the success of Wasserstein-based evaluations in generative modeling \cite{arjovsky2017wasserstein} and leverages the representational strength of CLIP embeddings \cite{radford2021learning} to assess diversity effectively.

\subsubsection{Content Matching (CM)}
Content Matching evaluates the semantic alignment of generated images with the provided prompt. For a prompt \(P\) and generated images \(\{I_1, I_2, \dots, I_N\}\), embeddings for the prompt \(E(P)\) and images \(E(I_i)\) are extracted using a multimodal model (e.g., CLIP). CM is computed as:

\begin{equation}
    CM = \frac{1}{N} \sum_{i=1}^{N} \text{sim}(E(P), E(I_i)),
\end{equation}

where \(\text{sim}(\cdot, \cdot)\) is a similarity measure such as cosine similarity or CLIPScore \cite{srinivas2021clipscore}. CM ensures that generated images align semantically with the textual prompt, emphasizing coherence across the batch.

This metric aligns with recent work on multimodal alignment \cite{srinivas2021clipscore}, highlighting the use of CLIP embeddings for coherence evaluation. It also echoes studies on maintaining logical consistency in T2I systems \cite{lee2021multi}.

\subsubsection{Tradeoff Between Diversity and Coherence}
The inherent tradeoff between BD and CM reflects the challenge of achieving both creative variability and semantic fidelity. High BD indicates diverse outputs, while high CM ensures that these outputs remain consistent with the prompt. An ideal T2I system must balance these objectives, achieving high scores for both metrics.

By capturing the stochastic variability of T2I systems and evaluating the outputs against clearly defined criteria, BD and CM provide a comprehensive framework for assessing Diversity vs. Coherence in generative models.

\begin{figure}[h!]
    \centering
    \includegraphics[width=\columnwidth]{img/BD_CM.png}
    \caption{}
    \label{fig:}
\end{figure}




\subsection{Artistic Freedom: Artistic Novelty Score (ANS)}

Artistic Freedom quantifies the creative liberties taken by a Text-to-Image (T2I) system in generating outputs. We propose the \textbf{Artistic Novelty Score (ANS)}, which evaluates the degree of innovation in the generated image while maintaining a connection to the original prompt. This metric combines \textit{Style Deviation}, \textit{Content Abstraction}, and \textit{Aesthetic Quality}.

\subsubsection{Style Deviation (StyleDev)}
Style Deviation measures how much the generated image deviates in style compared to a "literal" baseline image generated with minimal creative enhancement:

\begin{equation}
\text{StyleDev} = \| S(I_{\text{gen}}) - S(I_{\text{base}}) \|_2,
\end{equation}

where \(S(\cdot)\) represents style embeddings extracted using a pretrained network, such as the Gram Matrix of VGG features \cite{gatys2016neural}.

\subsubsection{Content Abstraction (ContentAbs)}
Content Abstraction evaluates how abstract the generated image is compared to the textual prompt:

\begin{equation}
\text{ContentAbs} = 1 - \cos(E(P), E(I_{\text{gen}})),
\end{equation}

where \(E(\cdot)\) represents feature embeddings extracted using a multimodal model, such as CLIP \cite{radford2021learning}, and \(\cos(\cdot, \cdot)\) is the cosine similarity function. Higher values of \(\text{ContentAbs}\) indicate more abstraction.

\subsubsection{Aesthetic Quality (AestheticQ)}
Aesthetic Quality assesses the visual appeal of the generated image using a pretrained aesthetic predictor (e.g., LAION aesthetic predictor \cite{schuhmann2022laion}):

\begin{equation}
\text{AestheticQ} \in [0, 1],
\end{equation}

where higher values denote greater aesthetic appeal.

\subsubsection{Artistic Novelty Score (ANS)}
The overall Artistic Novelty Score combines the above components as follows:

\begin{equation}
\text{ANS} = \alpha \cdot \text{StyleDev} + \beta \cdot \text{ContentAbs} + \gamma \cdot \text{AestheticQ},
\end{equation}

where \(\alpha, \beta, \gamma \in [0,1]\) control the relative importance of style, content, and aesthetics, respectively. These weights can be adjusted based on task-specific requirements.

\subsubsection{Example Application}
\textbf{Prompt:} ``Create a surreal forest with floating lanterns.''

\begin{itemize}
    \item \textbf{Baseline Image:} A literal forest scene with lanterns hanging from trees.
    \item \textbf{Generated Image:} A glowing forest where lanterns float mid-air, trees are neon-colored, and the ground reflects like water.
\end{itemize}

Calculated metrics for this example:
\begin{itemize}
    \item \(\text{StyleDev} = 0.78\)
    \item \(\text{ContentAbs} = 0.45\)
    \item \(\text{AestheticQ} = 0.9\)
\end{itemize}

\noindent Final ANS:
\begin{equation}
\text{ANS} = 0.4 \cdot 0.78 + 0.4 \cdot 0.45 + 0.2 \cdot 0.9 = 0.64.
\end{equation}

\subsubsection{Advantages}
\begin{itemize}
    \item \textbf{Quantifiable Creativity:} Provides a structured way to measure artistic freedom.
    \item \textbf{Unified Framework:} Balances style, abstraction, and aesthetics.
    \item \textbf{Scalable:} Fully automated and suitable for large-scale evaluation.
\end{itemize}

\subsubsection{References}
\begin{itemize}
    \item Gatys et al. \cite{gatys2016neural}: Introduced the use of Gram Matrices for style representation.
    \item Radford et al. \cite{radford2021learning}: Proposed CLIP for multimodal alignment, which forms the basis for \(\text{ContentAbs}\).
    \item Schuhmann et al. \cite{schuhmann2022laion}: Presented the LAION dataset and aesthetic predictors.
\end{itemize}


\begin{figure}[h!]
    \centering
    \includegraphics[width=\columnwidth]{img/output (38).png}
    \caption{}
    \label{fig:}
\end{figure}


\subsection{Emotional Impact vs. Neutrality}
\label{subsec:emotional_impact_neutrality}

\textbf{Overview:} Emotional Impact and Neutrality are inherently conflicting objectives in Text-to-Image (T2I) systems. Emotional Impact ensures that the generated image evokes a strong and intended emotional response, while Neutrality prioritizes unbiased and emotion-neutral representations. These objectives are crucial in applications like educational content, unbiased storytelling, and mental health, where emotional tone significantly influences usability.

\subsubsection{Emotion Recognition Score (ERS)}
\textbf{Description:} ERS quantifies the emotional intensity of generated images using emotion detection models (e.g., DeepEmotion~\cite{abidin2018deepemotion}), pretrained on datasets labeled with emotions such as happiness, sadness, anger, or fear. Higher ERS values indicate stronger emotional tones.

\textbf{Equation:}
\begin{equation}
ERS = \frac{1}{M} \sum_{i=1}^M \text{EmotionIntensity}(\text{img}_i)
\end{equation}
where:
\begin{itemize}
    \item \( M \): Total number of images in the batch.
    \item \( \text{EmotionIntensity}(\text{img}_i) \): Scalar intensity of the dominant emotion in image \(\text{img}_i\).
\end{itemize}

\subsubsection{Neutrality Score (N)}
\textbf{Description:} Neutrality measures the degree of emotional balance or impartiality in generated images, complementing ERS by capturing the absence of a dominant emotion.

\textbf{Equation:}
\begin{equation}
N = 1 - \max(\text{EmotionIntensity})
\end{equation}
where:
\begin{itemize}
    \item \( \max(\text{EmotionIntensity}) \): Intensity of the most dominant emotion detected in the image.
\end{itemize}

Higher \( N \) values (closer to 1) indicate emotionally neutral images, while lower \( N \) values reflect strong emotional dominance.

\subsubsection{Tradeoff Between Emotional Impact and Neutrality}
To evaluate the tradeoff between Emotional Impact and Neutrality, we define a combined metric:
\begin{equation}
T_{\text{EMN}} = \alpha \cdot ERS + \beta \cdot N
\end{equation}
where:
\begin{itemize}
    \item \( \alpha \): Weight assigned to Emotional Impact.
    \item \( \beta \): Weight assigned to Neutrality.
    \item \( \alpha + \beta = 1 \): Ensuring a balanced contribution.
\end{itemize}

\subsection{Visual Realism vs. Stylized Aesthetics}
\label{subsec:visual_realism_aesthetics}

\textbf{Objective Overview:} 
The evaluation of Visual Realism versus Stylized Aesthetics (SA) captures the balance between generating photorealistic visuals and producing artistically stylized images. To measure Visual Realism, we propose a fully automated \textbf{Photorealism Score (PRS)} that leverages CLIP embeddings.

\subsubsection{Photorealism Score (PRS)}
The Photorealism Score quantifies how realistic a generated image appears by measuring its alignment with the semantic concept of realism.

\paragraph{Key Components:}
\begin{itemize}
    \item \textbf{Text Embedding for "Realistic Image"}:
    The textual embedding for "realistic image" (\(E("realistic~image")\)) is obtained using the text encoder of the CLIP model \cite{radford2021learning}. \emph{Why "Realistic Image" Works}: When you provide the text "realistic image" to the CLIP text encoder, it generates an embedding based on its learned representation of the word "realistic" in conjunction with the concept of an "image." Although CLIP has not been explicitly trained to specialize in photorealism, its vast pretraining data ensures it has seen sufficient examples of what constitutes "realistic" visuals.
    \[
    E("realistic~image") = \text{CLIPTextEncoder}("realistic~image")
    \]

    \item \textbf{Image Embedding for Generated Image}:
    The visual embedding for the generated image (\(E(I_{\text{gen}})\)) is obtained using the image encoder of the CLIP model.

    \item \textbf{Cosine Similarity}:
    PRS is calculated as the cosine similarity between \(E(I_{\text{gen}})\) and \(E("realistic~image")\):
    \[
    PRS = \cos(E(I_{\text{gen}}), E("realistic~image"))
    \]
    Higher values of \(PRS\) indicate a closer alignment with the concept of realism, reflecting a more photorealistic output.
\end{itemize}

\paragraph{Advantages of PRS:}
\begin{enumerate}
    \item \textbf{Automation:} Fully automated and does not require additional training or labeling.
    \item \textbf{Scalability:} Efficient for large-scale evaluations of T2I systems.
    \item \textbf{Adaptability:} Can be extended to domain-specific realism (e.g., "realistic landscape" or "realistic portrait").
\end{enumerate}

\subsubsection{Stylized Aesthetic Score (SAS)}
To complement PRS, a \textbf{Stylized Aesthetic Score (SAS)} evaluates the artistic stylization of generated images. SAS measures the alignment of the generated image with textual descriptors of artistic styles (e.g., "impressionist painting," "surrealist art").

\paragraph{Key Components:}
\begin{itemize}
    \item \textbf{Text Embedding for Style Descriptor}:
    The textual embedding for a specific artistic style (\(E(\text{style})\)) is obtained using the CLIP text encoder.

    \item \textbf{Cosine Similarity}:
    SAS is computed as:
    \[
    SAS = \cos(E(I_{\text{gen}}), E(\text{style}))
    \]
    Higher SAS values indicate stronger alignment with the specified style.
\end{itemize}

\subsubsection{Tradeoff Score for Visual Realism and Stylized Aesthetics}
The tradeoff between Visual Realism (\(PRS\)) and Stylized Aesthetics (\(SAS\)) is represented as:
\[
\text{Tradeoff} = \gamma \cdot PRS + (1 - \gamma) \cdot SAS
\]
where \(\gamma \in [0,1]\) controls the emphasis on realism versus artistic stylization.

\subsubsection{Example Application}
\begin{itemize}
    \item \textbf{Prompt:} "A city street during sunset."
    \begin{itemize}
        \item \textbf{Realistic Output:} A photorealistic depiction of a city street with warm, natural lighting.
        \begin{itemize}
            \item \(PRS = 0.85\), \(SAS = 0.20\)
        \end{itemize}
        \item \textbf{Stylized Output:} A painterly impression of a city street with exaggerated colors and textures.
        \begin{itemize}
            \item \(PRS = 0.45\), \(SAS = 0.90\)
        \end{itemize}
    \end{itemize}
    \item \textbf{Tradeoff Score:} \(0.7 \cdot 0.85 + 0.3 \cdot 0.20 = 0.655\) (Realistic output preferred for this weighting).
\end{itemize}

\paragraph{Advantages:}
\begin{itemize}
    \item Provides a unified framework to evaluate realism and artistic stylization.
    \item Fully automated using CLIP, avoiding the need for additional training.
    \item Adaptable for different realism or style preferences based on \(\gamma\).
\end{itemize}


\subsection{Diversity vs. Coherence}

\textbf{Objective Pair Overview:}
\begin{itemize}
    \item \textbf{Diversity:} Generating a wide range of unique and varied images in response to similar or identical prompts, ensuring that each output is distinct in style, composition, or elements.
    \item \textbf{Coherence:} Maintaining logical and visual consistency within each image, ensuring that all elements are harmoniously integrated and the overall composition is aesthetically pleasing and meaningful.
\end{itemize}

\textbf{Core Conflict:}
Enhancing diversity often involves introducing varied styles, elements, or interpretations, which can sometimes compromise the coherence and overall harmony of the image. Conversely, striving for high coherence might limit the variability and uniqueness of generated outputs.

\textbf{Significance in T2I Systems:}
\begin{itemize}
    \item \textbf{Creative Exploration:} High diversity allows users to explore multiple interpretations of a prompt, fostering creativity and innovation.
    \item \textbf{Quality Assurance:} Coherent images ensure that the generated content is not only diverse but also of high quality, with well-integrated elements that make sense collectively.
    \item \textbf{User Control:} Users may seek either diverse options to choose from or highly coherent images for specific purposes. Balancing these needs is crucial for versatile T2I applications.
\end{itemize}

\textbf{Implications for Alignment:}
YinYangAlign assesses T2I models on their ability to produce diverse outputs without sacrificing the internal coherence of each image. This balance ensures that models can offer a rich variety of options while maintaining the quality and integrity of each generated image.

\textbf{Example Scenario:}
\begin{itemize}
    \item \textbf{Prompt:} "A futuristic city skyline at night."
    \item \textbf{Diverse Output:} Multiple images showcasing different architectural styles, color schemes, and atmospheric effects, each offering a unique vision of a futuristic city.
    \item \textbf{Coherent Output:} Each image maintains a logical structure where architectural elements, lighting, and atmospheric effects are harmoniously integrated, ensuring that the cityscape appears realistic and visually consistent.
\end{itemize}

\subsection{Cultural Sensitivity vs. Artistic Freedom}

\textbf{Objective Pair Overview:}
\begin{itemize}
    \item \textbf{Cultural Sensitivity:} Respecting and accurately representing diverse cultural elements, avoiding stereotypes, misrepresentations, or appropriation, and ensuring that generated images are inclusive and respectful of different cultural contexts.
    \item \textbf{Artistic Freedom:} Allowing for unrestricted creative expression, enabling the introduction of novel and imaginative elements that may push the boundaries of conventional representations.
\end{itemize}

\textbf{Core Conflict:}
Promoting cultural sensitivity necessitates careful consideration of cultural elements, which may limit the extent of creative liberties to prevent misrepresentation or offense. Conversely, prioritizing artistic freedom might lead to the inadvertent inclusion of culturally insensitive or inappropriate elements.

\textbf{Significance in T2I Systems:}
\begin{itemize}
    \item \textbf{Ethical Responsibility:} Ensuring that AI-generated images do not perpetuate cultural biases or stereotypes is crucial for ethical AI deployment.
    \item \textbf{Global Applicability:} T2I systems are used by a diverse, global user base. Cultural sensitivity ensures that the generated content is appropriate and respectful across different cultural contexts.
    \item \textbf{Creative Integrity:} While sensitivity is paramount, maintaining artistic freedom is essential for innovation and the creation of compelling, original content.
\end{itemize}

\textbf{Implications for Alignment:}
YinYangAlign evaluates T2I models on their ability to balance cultural sensitivity with artistic freedom, ensuring that creative expression does not come at the expense of respect and accuracy in representing diverse cultures.

\textbf{Example Scenario:}
\begin{itemize}
    \item \textbf{Prompt:} "A traditional Japanese tea ceremony in a modern setting."
    \item \textbf{Culturally Sensitive Output:} An image that accurately depicts the elements of a Japanese tea ceremony, respecting traditional attire, setting, and practices, while subtly integrating modern elements without misrepresentation.
    \item \textbf{Artistically Enhanced Output:} The same scene with imaginative modern design elements, innovative color schemes, or abstract interpretations that enhance the visual appeal while maintaining cultural accuracy.
\end{itemize}

\subsection{Emotional Impact vs. Neutrality}

\textbf{Objective Pair Overview:}
\begin{itemize}
    \item \textbf{Emotional Impact:} Creating images that evoke specific emotions, such as joy, sadness, fear, or excitement, enhancing the viewer's emotional engagement and response.
    \item \textbf{Neutrality:} Maintaining an unbiased, objective representation that does not intentionally evoke strong emotions, ensuring that the image remains impartial and universally interpretable.
\end{itemize}

\textbf{Core Conflict:}
Enhancing emotional impact often involves deliberate artistic choices that can influence the viewer's feelings, which may conflict with the goal of maintaining neutrality. Conversely, striving for neutrality might result in less engaging or impactful images.

\textbf{Significance in T2I Systems:}
\begin{itemize}
    \item \textbf{User Intent Alignment:} Different applications may require varying levels of emotional engagement. For instance, marketing materials might aim for high emotional impact, while informational content may prioritize neutrality.
    \item \textbf{Versatility:} T2I models must cater to diverse user needs, balancing the ability to evoke emotions with the need for impartiality in different contexts.
    \item \textbf{Ethical Considerations:} Manipulating emotional responses can have ethical implications, especially in sensitive applications like mental health or news media.
\end{itemize}

\textbf{Implications for Alignment:}
YinYangAlign assesses T2I models on their capacity to generate images that can either evoke specific emotions or maintain neutrality based on the context, ensuring that models can adapt to varying user requirements while adhering to ethical standards.

\textbf{Example Scenario:}
\begin{itemize}
    \item \textbf{Prompt:} "A peaceful morning in a quiet village."
    \item \textbf{Emotionally Impactful Output:} An image with warm lighting, soft colors, and serene elements that evoke feelings of tranquility and happiness.
    \item \textbf{Neutral Output:} An image that accurately represents a quiet village morning without deliberate emotional cues, allowing viewers to interpret the scene without specific emotional guidance.
\end{itemize}

\subsection{Alignment with User Intent vs. AI Autonomy}

\textbf{Objective Pair Overview:}
\begin{itemize}
    \item \textbf{Alignment with User Intent:} Ensuring that the generated image precisely matches the user's explicit instructions and expectations, maintaining control over the output based on user directives.
    \item \textbf{AI Autonomy:} Allowing the AI system to exercise its own judgment and creativity to enhance or alter the outcome, potentially introducing elements that the user did not explicitly specify but may improve the overall quality or appeal of the image.
\end{itemize}

\textbf{Core Conflict:}
Strict alignment with user intent ensures that the output adheres closely to user specifications, but it may limit the AI's ability to enhance or innovate. Granting AI autonomy allows for creative improvements but risks deviating from the user's original intent.

\textbf{Significance in T2I Systems:}
\begin{itemize}
    \item \textbf{User Control vs. Innovation:} Users may desire different levels of control over the output. Some prefer exact adherence to their prompts, while others appreciate the AI's creative input.
    \item \textbf{Trust and Reliability:} Maintaining alignment with user intent builds trust in the system's reliability, ensuring that users can depend on the AI to produce expected results.
    \item \textbf{Creative Potential:} Allowing AI autonomy can lead to more creative and unexpected outcomes, enhancing the value of T2I systems in creative industries.
\end{itemize}

\textbf{Implications for Alignment:}
YinYangAlign evaluates T2I models on their ability to balance user-directed alignment with the AI's autonomous creative enhancements, ensuring that models can deliver both precise and innovative outputs as required.

\textbf{Example Scenario:}
\begin{itemize}
    \item \textbf{Prompt:} "A vintage car parked under a tree in autumn."
    \item \textbf{User-Aligned Output:} An image that exactly matches the prompt, featuring a specific vintage car model, a particular type of tree, and autumnal colors.
    \item \textbf{AI-Autonomous Output:} The same scene with additional creative elements, such as falling leaves creating dynamic motion, a reflection in the car's surface, or a subtle background ambiance that enhances the overall aesthetic without altering the core elements of the prompt.
\end{itemize}

\subsection{Originality vs. Referentiality}

\textbf{Objective Pair Overview:}
\begin{itemize}
    \item \textbf{Originality:} Creating unique and novel images that introduce fresh ideas, styles, or compositions, fostering innovation and distinguishing the output from existing works.
    \item \textbf{Referentiality:} Incorporating recognizable elements, styles, or references to existing artworks, cultures, or iconic imagery to meet user expectations or provide contextual relevance.
\end{itemize}

\textbf{Core Conflict:}
Fostering originality encourages the creation of new and unique content, which may diverge from familiar references, potentially making the image less immediately recognizable or relatable. Emphasizing referentiality ensures that the image aligns with known standards or expectations but may limit the scope for innovation.

\textbf{Significance in T2I Systems:}
\begin{itemize}
    \item \textbf{Creative Innovation:} Originality is essential for pushing the boundaries of artistic expression and providing users with novel and inspiring content.
    \item \textbf{User Familiarity:} Referentiality ensures that users can relate to or recognize elements within the generated images, enhancing usability and relevance, especially in commercial applications.
    \item \textbf{Market Differentiation:} Balancing originality and referentiality allows T2I systems to cater to diverse user needs, from seeking unique artistic creations to requiring specific, recognizable imagery.
\end{itemize}

\textbf{Implications for Alignment:}
YinYangAlign assesses T2I models on their ability to generate images that balance originality with referentiality, ensuring that models can produce both innovative and contextually relevant outputs as per user requirements.

\textbf{Example Scenario:}
\begin{itemize}
    \item \textbf{Prompt:} "A mythical creature in a fantasy landscape."
    \item \textbf{Original Output:} An entirely new and imaginative creature with unique features and an abstract fantasy landscape that has not been seen before.
    \item \textbf{Referential Output:} A mythical creature inspired by existing folklore or popular fantasy genres, integrated into a landscape that incorporates recognizable elements from established fantasy settings.
\end{itemize}

\subsection{Conclusion}

Each of the six contradictory alignment objectives in **YinYangAlign** addresses fundamental challenges in the development and evaluation of Text-to-Image AI systems. By systematically assessing these conflicting goals, YinYangAlign provides a comprehensive framework that ensures T2I models are not only accurate and reliable but also versatile, creative, and ethically responsible. This balanced evaluation fosters the advancement of AI-driven image generation technologies that can adeptly navigate the complex interplay of user expectations, creative innovation, and societal standards.





\section{Explanation of the Conflicting Objective Pairs}

\subsection{1. Faithfulness to Prompt vs. Creative Enhancement}
\begin{itemize}
    \item \textbf{Faithfulness to Prompt:} Accurately representing the details and instructions provided in the user’s text prompt.
    \item \textbf{Creative Enhancement:} Incorporating creative elements that enhance the image beyond the literal prompt.
    \item \textbf{Core Conflict:} Balancing adherence to the user’s instructions with the introduction of creative improvements that may alter the original intent.
\end{itemize}

\subsection{2. Diversity vs. Coherence}
\begin{itemize}
    \item \textbf{Diversity:} Generating a wide variety of images to cover different aspects of the prompt.
    \item \textbf{Coherence:} Ensuring that each image is logically consistent and unified in its representation.
    \item \textbf{Core Conflict:} Balancing the generation of diverse outputs with the maintenance of logical and visual coherence within each image.
\end{itemize}

\subsection{3. Cultural Sensitivity vs. Artistic Freedom}
\begin{itemize}
    \item \textbf{Cultural Sensitivity:} Respectfully representing diverse cultures and avoiding stereotypes or misrepresentations.
    \item \textbf{Artistic Freedom:} Allowing unrestricted creative expression without cultural constraints.
    \item \textbf{Core Conflict:} Ensuring that creative expressions do not exclude or misrepresent any group, thereby maintaining diversity while promoting artistic innovation.
\end{itemize}

\subsection{4. Emotional Impact vs. Neutrality}
\begin{itemize}
    \item \textbf{Emotional Impact:} Creating images that evoke strong emotions or convey specific moods.
    \item \textbf{Neutrality:} Maintaining an unbiased and impartial representation without intended emotional influence.
    \item \textbf{Core Conflict:} Balancing the desire to evoke emotions with the need to remain neutral and objective in representation.
\end{itemize}

\subsection{5. Alignment with User Intent vs. AI Autonomy}
\begin{itemize}
    \item \textbf{Alignment with User Intent:} Ensuring the generated image closely matches what the user envisioned based on their prompt.
    \item \textbf{AI Autonomy:} Allowing the AI system to exercise its own judgment and interpretation in image generation.
    \item \textbf{Core Conflict:} Balancing strict adherence to user instructions with the AI’s ability to make autonomous creative decisions that may enhance or alter the outcome.
\end{itemize}

\subsection{6. Originality vs. Referentiality}
\begin{itemize}
    \item \textbf{Originality:} Creating unique and novel images that do not closely resemble existing works.
    \item \textbf{Referentiality:} Incorporating recognizable elements or styles from existing sources or references.
    \item \textbf{Core Conflict:} Balancing the creation of original content with the use of familiar references to meet user expectations or contextual requirements.
\end{itemize}

\section{Key Takeaways}

\begin{enumerate}
    \item \textbf{Balancing Act:} Each pair exemplifies the inherent tension between two important objectives in AI alignment.
    \item \textbf{Context Matters:} The chosen and rejected responses highlight how context influences the prioritization of objectives.
    \item \textbf{Impact on Users:} The contradictions often affect user trust, perception, and the overall user experience.
    \item \textbf{Ethical Implications:} Prioritizing one objective over another can lead to ethical dilemmas and unintended consequences.
\end{enumerate}

\section{Next Steps for Developing the Benchmark}

\begin{enumerate}
    \item \textbf{Develop Detailed Tables:} For each of the six conflicting objective pairs, create detailed tables including:
        \begin{itemize}
            \item \textbf{Example ID (A \& B)}
            \item \textbf{Human Prompt}
            \item \textbf{Chosen (Aligned) Response}
            \item \textbf{Rejected (Misaligned) Response}
            \item \textbf{Contradiction Explanation}
        \end{itemize}
    \item \textbf{Define Evaluation Metrics:} Establish specific metrics to quantitatively and qualitatively assess how well AI systems balance each pair. Consider metrics such as:
        \begin{itemize}
            \item \textbf{User Satisfaction Ratings}
            \item \textbf{Factual Accuracy Scores}
            \item \textbf{Ethical Compliance Checks}
            \item \textbf{Aesthetic Quality Assessments}
        \end{itemize}
    \item \textbf{Compile Datasets:} Gather or generate a diverse set of prompts and corresponding AI-generated responses for each pair to build a comprehensive benchmark dataset.
    \item \textbf{Implement Testing Procedures:} Develop standardized testing protocols to evaluate AI systems against the defined conflicting objective pairs, ensuring consistent and objective assessments.
    \item \textbf{Engage Stakeholders:} Collaborate with AI ethicists, designers, developers, and end-users to validate and refine the benchmark, ensuring it addresses real-world challenges and needs.
    \item \textbf{Continuous Refinement:} Regularly update the benchmark to incorporate new insights, emerging trends, and feedback from evaluations to maintain its relevance and effectiveness.
\end{enumerate}




\setlength{\LTpre}{0pt}
\setlength{\LTpost}{0pt}

\renewcommand{\arraystretch}{1.5} % Adjust row height for better readability

\onecolumn
\begin{longtable}{|p{3cm}|p{4cm}|p{6cm}|}
\caption{Evaluation Metrics for YinYangAlign's Six Contradictory Alignment Objectives with Equations and Explanations.} \\
\hline
\textbf{Axiom} & \textbf{Evaluation Metric} & \textbf{Description, Justification, and Equation} \\ 
\hline
\endfirsthead
\multicolumn{3}{c}{\textit{(Continued from previous page)}} \\
\hline
\textbf{Axiom} & \textbf{Evaluation Metric} & \textbf{Description, Justification, and Equation} \\ 
\hline
\endhead
\hline \multicolumn{3}{c}{\textit{(Continued on next page)}} \\
\endfoot
\hline
\endlastfoot

Faithfulness to Prompt vs. Creative Enhancement & 
Semantic Alignment Score (SAS) & 
\textbf{Description:} Measures semantic similarity between the textual prompt and the generated image's description using multimodal models like CLIP or BLIP. Faithfulness is evaluated using cosine similarity, while creative enhancement is assessed via human judgment. \newline
\textbf{Equation:} 
\[
SAS = \text{cos}(E_{\text{prompt}}, E_{\text{image}})
\]
where \(E_{\text{prompt}}\) and \(E_{\text{image}}\) are the embeddings of the prompt and image respectively. High scores indicate stronger semantic alignment. \\

Diversity vs. Coherence & 
Batch Diversity (BD) and Coherence Metrics (CM) & 
\textbf{Description:} Diversity is measured by the average pairwise cosine distance of embeddings in a batch, while coherence assesses the consistency of elements within an image. \newline
\textbf{Diversity Equation:}
\[
BD = \frac{1}{N(N-1)} \sum_{i=1}^N \sum_{j=i+1}^N \text{cos}(E_{\text{img}_i}, E_{\text{img}_j})
\]
\textbf{Coherence Equation:}
\[
CM = \frac{1}{M} \sum_{i=1}^M \text{ConsistencyScore}(\text{img}_i)
\]
where \(N\) is the batch size, \(M\) is the number of images, and \(\text{ConsistencyScore}\) evaluates logical element placement within each image. \\

Cultural Sensitivity vs. Artistic Freedom & 
Cultural Sensitivity Classifier & 
\textbf{Description:} Sensitivity is measured using fine-tuned classifiers (e.g., ViT or CLIP) trained on culturally annotated datasets. Artistic freedom is quantified by introducing controlled deviations from norms. \newline
\textbf{Equation:}
\[
CS = \frac{1}{M} \sum_{i=1}^M \text{SensitivityScore}(\text{img}_i)
\]
where \(\text{SensitivityScore}\) assesses appropriateness of cultural elements. Freedom is indirectly measured by the variance in visual styles across outputs. \\

Emotional Impact vs. Neutrality & 
Emotion Recognition Score (ERS) & 
\textbf{Description:} ERS evaluates the emotional tone in generated images using emotion detection models (e.g., DeepEmotion). Neutrality is the complement of emotion intensity. \newline
\textbf{Equation:}
\[
ERS = \frac{1}{M} \sum_{i=1}^M \text{EmotionIntensity}(\text{img}_i)
\]
Neutrality can be represented as:
\[
N = 1 - \text{max}(\text{EmotionIntensity})
\]
where higher \(ERS\) scores indicate stronger emotions, and \(N\) values closer to 1 indicate neutrality. \\

Alignment with User Intent vs. AI Autonomy & 
Intent Satisfaction Score (ISS) & 
\textbf{Description:} Measures alignment by computing similarity between the embeddings of the user's prompt and the generated image caption. \newline
\textbf{Equation:}
\[
ISS = \text{cos}(E_{\text{prompt}}, E_{\text{caption}})
\]
where \(E_{\text{caption}}\) is the embedding of the generated caption. Autonomy is evaluated as the degree of deviation from the prompt, measured by cosine distance:
\[
\text{Autonomy} = 1 - \text{cos}(E_{\text{prompt}}, E_{\text{caption}})
\] \\

Originality vs. Referentiality & 
Originality and Referentiality Index (ORI, RI) & 
\textbf{Description:} Originality is measured by comparing images against training data or benchmarks, while referentiality evaluates similarity to known references. \newline
\textbf{Originality Equation:}
\[
ORI = 1 - \text{sim}(E_{\text{img}}, E_{\text{training}})
\]
\textbf{Referentiality Equation:}
\[
RI = \text{sim}(E_{\text{img}}, E_{\text{reference}})
\]
where \(\text{sim}\) denotes cosine similarity, and embeddings are derived from the generated image and training/reference datasets. \\

\end{longtable}


\twocolumn



\pagebreak