



% Please add the following required packages to your document preamble:
% \usepackage{booktabs}
% \usepackage[normalem]{ulem}
% \useunder{\uline}{\ul}{}
% \usepackage{longtable}
% Note: It may be necessary to compile the document several times to get a multi-page table to line up properly




\begin{figure*}[ht!]
    \centering
    \includegraphics[width=\textwidth]{img/stochastic.pdf}
    \caption{Illustrative example of aligning T2I models with Faithfulness to Prompt vs. Artistic Freedom. The chosen outputs adhere closely to the prompt, depicting a highly detailed and accurate portrait of Albert Einstein in a realistic oil painting style, while the rejected outputs deviate significantly, introducing surreal or unrelated elements. This highlights the importance of balancing prompt adherence with artistic flexibility in alignment optimization.}
    \label{fig:stochastic_generation}
    \vspace{-2mm}
\end{figure*}



\section{YinYangAlign: Six Contradictory Alignment Objectives}

Current research and benchmarking in T2I alignment primarily focus on isolated objectives \cite{guo2022survey}, such as fidelity to prompts \cite{ramesh2021zero}, aesthetic quality \cite{rombach2022high}, or bias mitigation \cite{zhao2023mitigating}, often treating these goals independently. However, there is a clear gap in benchmarks that evaluate how T2I systems balance multiple, often contradictory objectives. 
%This is a critical limitation, as real-world applications demand systems capable of achieving diverse outputs while maintaining emotional neutrality and relevance to user prompts. 
The lack of multi-objective benchmarks restricts the ability to holistically assess and improve T2I alignment, ultimately affecting their reliability and effectiveness in practical scenarios.


\noindent
\textbf{Selection of Six Contradictory Objectives}: YinYangAlign introduces six carefully selected pairs of contradictory objectives that capture the fundamental tensions in T2I image generation. These pairs are chosen for their relevance and significance in real-world applications. \cref{fig:alignment_axioms} introduces the core trade-offs central to the YinYangAlign framework, each representing a critical conflict that T2I systems must navigate to balance user expectations and ethical considerations. The trade-offs include: \textit{Faithfulness to Prompt vs. Artistic Freedom}, which involves adhering to user instructions while minimizing creative deviations; \textit{Emotional Impact vs. Neutrality}, requiring a balance between evoking emotions and maintaining objective representation; and \textit{Visual Realism vs. Artistic Freedom}, focusing on achieving photorealistic outputs without compromising artistic liberties. Additionally, \textit{Originality vs. Referentiality} addresses the challenge of fostering stylistic innovation while avoiding reliance on established artistic styles to ensure uniqueness. \textit{Verifiability vs. Artistic Freedom} emphasizes balancing factual accuracy with creative liberties to minimize misinformation. Finally, \textit{Cultural Sensitivity vs. Artistic Freedom} underscores the need to respect cultural representations while ensuring that creative freedoms do not lead to misrepresentation or insensitivity. \cref{tab:yinyang_axioms} provides illustrative examples of these alignment axioms.

YinYangAlign serves as a holistic benchmark for evaluating alignment performance, ensuring that T2I models are not only accurate and reliable but also adaptable, ethical, and capable of meeting complex user demands and societal expectations.





\begin{comment}
\begin{figure*}[ht!]
    \centering
    \resizebox{1.0\textwidth}{!}{
       \includegraphics[width=\linewidth]{img/yinyang-1.pdf}    
    }
    \caption{Architecture of PECCAVI-Image.}
    \label{fig:strength_var}
\end{figure*}

\begin{figure*}[ht!]
    \centering
    \resizebox{1.0\textwidth}{!}{
       \includegraphics[width=\linewidth]{img/yinyang-2.pdf}    
    }
    \caption{Architecture of PECCAVI-Image.}
    \label{fig:strength_var}
\end{figure*}
\end{comment}

\begin{comment}
\subsection{Faithfulness to Prompt vs. Creative Enhancement}

This axiom highlights the tension between adhering to user instructions and introducing creative elements to enhance the generated images.

\textbf{Objective Pair Overview:}
\begin{itemize}
    \item \textbf{Faithfulness to Prompt:} Ensures the generated image accurately reflects the user's textual description, maintaining fidelity to all specified details.
    \item \textbf{Creative Enhancement:} Introduces imaginative elements to enhance the aesthetic or interpretative depth, which may sometimes deviate from the original instructions.
\end{itemize}


\textbf{Core Conflict:}  
Balancing fidelity to user prompts with creative enhancements is a nuanced challenge. While faithfulness guarantees precision, creative additions risk altering the intent to enrich the output. \cref{tab:axiom_1} illustrates this tradeoff with examples of chosen and rejected AI-generated responses.
\end{comment}






























