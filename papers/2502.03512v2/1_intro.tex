
\begin{figure}[t!]
    \centering
    \resizebox{\columnwidth}{!}{%
      \begin{forest}
        forked edges,
        for tree={
          grow=east,
          reversed=true,
          anchor=base west,
          parent anchor=east,
          child anchor=west,
          base=center,
          font=\normalsize,
          rectangle,
          draw=none,            % Border removed for all main nodes
          rounded corners,
          align=center,
          text centered,
          edge+={darkgray, line width=1pt},
          s sep=3pt,
          inner xsep=1pt,
          inner ysep=3pt,
          line width=0.8pt,
          ver/.style={rotate=90, child anchor=north, parent anchor=south, anchor=center},
        },
        where level=1{text width=10em, font=\LARGE}{},
        where level=2{text width=34em, font=\LARGE}{},
        where level=3{text width=40em, font=\LARGE}{},
        [ % Root node with an image
          {\includegraphics[width=0.15\linewidth]{img/yinyang_text.png}}, 
          for tree={fill=a35}
          % First branch
          [ 
            {\textbf{\Large Faithfulness} \\ \textbf{\Large to Prompt} \\ vs. \\ \textbf{\Large Artistic Freedom}}, 
            for tree={fill=paired-light-blue},
            [
              \textbf{Core Conflict:} \text{Adhering to user instructions}\\ 
              \text{while minimizing creative deviations that}\\
              \text{could alter the original intent.}, 
              my leaf={paired-light-blue}
            ]
          ]
          % Second branch
          [ 
            {\textbf{\Large Emotional} \\ \textbf{\Large Impact} \\ vs. \\ \textbf{\Large Neutrality}}, 
            for tree={fill=a26},
            [
              \textbf{Core Conflict:} \text{Striking a balance between}\\
              \text{evoking specific emotions and maintaining}\\
              \text{an unbiased, objective representation.},
              my leaf={a26}
            ]
          ]
          % Third branch
          [ 
            {\textbf{\Large Visual} \\ \textbf{\Large Realism} \\ vs. \\ \textbf{\Large Artistic Freedom}}, 
            for tree={fill=paired-light-pink},
            [
              \textbf{Core Conflict:} \text{Create photorealistic images}\\
              \text{mimicking real-world visuals, incorporating}\\
              \text{artistic styles like impressionism or surrealism}\\
              \text{only when requested.},
              my leaf={paired-light-pink}
            ]
          ]
          % Fourth branch
          [ 
            {\textbf{\Large Originality} \\ vs. \\ \textbf{\Large Referentiality}}, 
            for tree={fill=paired-light-orange},
            [
              \textbf{Core Conflict:} \text{Maintain stylistic originality}\\
              \text{while avoiding over-reliance on established}\\
              \text{artistic styles that risks style plagiarism.},
              my leaf={paired-light-orange}
            ]
          ]
          % Fifth branch
          [ 
            {\textbf{\Large Verifiability} \\ vs. \\ \textbf{\Large Artistic Freedom}}, 
            for tree={fill=paired-light-yellow},
            [
              \textbf{Core Conflict:} \text{Balancing factual accuracy}\\
              \text{against creative freedom to minimize}\\
              \text{misinformation.},
              my leaf={paired-light-yellow}
            ]
          ]
          % Sixth branch
          [ 
            {\textbf{\Large Cultural} \\ \textbf{\Large Sensitivity} \\ vs. \\ \textbf{\Large Artistic Freedom}}, 
            for tree={fill=a31},
            [
              \textbf{Core Conflict:} \text{Respecting cultural representations}\\
              \text{while ensuring creative liberties do not}\\
              \text{lead to insensitivity or misrepresentation.},
              my leaf={a31}
            ]
          ]
        ]
      \end{forest}
    }
    \caption{The figure illustrates six core trade-offs (e.g., Faithfulness vs. Freedom, Emotional Impact vs. Neutrality), highlighting key conflicts and their implications.}
    \label{fig:alignment_axioms}
    \vspace{-3mm}
\end{figure}


% \begin{figure}[t!]
%     \centering
%     \resizebox{\columnwidth}{!}{%
%       \begin{forest}
%         forked edges,
%         for tree={
%           grow=east,
%           reversed=true,
%           anchor=base west,
%           parent anchor=east,
%           child anchor=west,
%           base=center,
%           font=\normalsize,
%           rectangle,
%           draw=none,            % Border removed for all main nodes
%           rounded corners,
%           align=center,
%           text centered,
%           edge+={darkgray, line width=1pt},
%           s sep=3pt,
%           inner xsep=1pt,
%           inner ysep=3pt,
%           line width=0.8pt,
%           ver/.style={rotate=90, child anchor=north, parent anchor=south, anchor=center},
%         },
%         where level=1{text width=10em, font=\LARGE}{},
%         where level=2{text width=34em, font=\LARGE}{},
%         where level=3{text width=40em, font=\LARGE}{},
%         [ % Root node with an image
%           {\includegraphics[width=0.15\linewidth]{img/yinyang_text.png}}, 
%           for tree={fill=a35}
%           % First branch
%           [ 
%             {\textbf{\Large Faithfulness} \\ \textbf{\Large to Prompt} \\ vs. \\ \textbf{\Large Artistic Freedom}}, 
%             for tree={fill=paired-light-blue},
%             [
%               \textbf{Core Conflict:} \text{Adhering to user instructions}\\ 
%               \text{while minimizing creative deviations that}\\
%               \text{could alter the original intent.}, 
%               leaf, 
%               for tree={
%                 fill=paired-light-blue,
%                 draw=none           % Border removed for this leaf node
%               }
%             ]
%           ]
%           % Second branch
%           [ 
%             {\textbf{\Large Emotional} \\ \textbf{\Large Impact} \\ vs. \\ \textbf{\Large Neutrality}}, 
%             for tree={fill=a26},
%             [
%               \textbf{Core Conflict:} \text{Striking a balance between}\\
%               \text{evoking specific emotions and maintaining}\\
%               \text{an unbiased, objective representation.},
%               leaf, 
%               for tree={
%                 fill=a26,
%                 draw=none           % Border removed for this leaf node
%               }
%             ]
%           ]
%           % Third branch
%           [ 
%             {\textbf{\Large Visual} \\ \textbf{\Large Realism} \\ vs. \\ \textbf{\Large Artistic Freedom}}, 
%             for tree={fill=paired-light-pink},
%             [
%               \textbf{Core Conflict:} \text{Create photorealistic images}\\
%               \text{mimicking real-world visuals, incorporating}\\
%               \text{artistic styles like impressionism or surrealism}\\
%               \text{only when requested.},
%               leaf, 
%               for tree={
%                 fill=paired-light-pink,
%                 draw=none           % Border removed for this leaf node
%               }
%             ]
%           ]
%           % Fourth branch
%           [ 
%             {\textbf{\Large Originality} \\ vs. \\ \textbf{\Large Referentiality}}, 
%             for tree={fill=paired-light-orange},
%             [
%               \textbf{Core Conflict:} \text{Maintain stylistic originality}\\
%               \text{while avoiding over-reliance on established}\\
%               \text{artistic styles that risks style plagiarism.},
%               leaf, 
%               for tree={
%                 fill=paired-light-orange,
%                 draw=none           % Border removed for this leaf node
%               }
%             ]
%           ]
%           % Fifth branch
%           [ 
%             {\textbf{\Large Verifiability} \\ vs. \\ \textbf{\Large Artistic Freedom}}, 
%             for tree={fill=paired-light-yellow},
%             [
%               \textbf{Core Conflict:} \text{Balancing factual accuracy}\\
%               \text{against creative freedom to minimize}\\
%               \text{misinformation.},
%               leaf, 
%               for tree={
%                 fill=paired-light-yellow,
%                 draw=none           % Border removed for this leaf node
%               }
%             ]
%           ]
%           % Sixth branch
%           [ 
%             {\textbf{\Large Cultural} \\ \textbf{\Large Sensitivity} \\ vs. \\ \textbf{\Large Artistic Freedom}}, 
%             for tree={fill=a31},
%             [
%               \textbf{Core Conflict:} \text{Respecting cultural representations}\\
%               \text{while ensuring creative liberties do not}\\
%               \text{lead to insensitivity or misrepresentation.},
%               leaf, 
%               for tree={
%                 fill=a31,
%                 draw=none           % Border removed for this leaf node
%               }
%             ]
%           ]
%         ]
%       \end{forest}
%     }
%     \caption{The figure illustrates six core trade-offs (e.g., Faithfulness vs. Freedom, Emotional Impact vs. Neutrality), highlighting key conflicts and their implications.}
%     \label{fig:alignment_axioms}
%     \vspace{-3mm}
% \end{figure}


% \begin{figure}[t!]
%     \centering
%     \resizebox{\columnwidth}{!}{
%         \begin{forest}
%             forked edges,
%             for tree={
%                 grow=east,
%                 reversed=true,
%                 anchor=base west,
%                 parent anchor=east,
%                 child anchor=west,
%                 base=center,
%                 font=\normalsize,
%                 rectangle,
%                 draw=hidden-draw,
%                 rounded corners,
%                 align=center,
%                 text centered,
%                 % minimum width=10em,
%                 % text width=15em,
%                 edge+={darkgray, line width=1pt},
%                 s sep=3pt,
%                 inner xsep=1pt,
%                 inner ysep=3pt,
%                 line width=0.8pt,
%                 ver/.style={rotate=90, child anchor=north, parent anchor=south, anchor=center},
%             },
%             where level=1{text width=10em,font=\LARGE,}{},
%             where level=2{text width=30em,font=\LARGE,}{},
%             where level=3{text width=40em,font=\LARGE,}{},
%             [\node{\includegraphics[width=0.15\linewidth]{img/yinyang_text.png}}, for tree={fill=a35}
%                 % [\textbf{Faithfulness to Prompt} \\ vs. \\ \textbf{Artistic Freedom}, for tree={fill=paired-light-blue}
%                 %     [\textbf{Core Conflict:} \text{Adhering to user instructions}\\ \text{while minimizing creative deviations that}\\ \text{could alter the original intent.}, leaf, for tree={fill=paired-light-blue}]
%                 % ]
% %                 [\normalsize\textbf{Faithfulness} \\ \normalsize\textbf{to Prompt} \\ \normalsize \textbf{vs.} \\ \normalsize\textbf{Artistic Freedom}, for tree={fill=paired-light-blue}
% %     [\textbf{Core Conflict:} \text{Adhering to user instructions}\\ \text{while minimizing creative deviations that}\\ \text{could alter the original intent.}, leaf, for tree={fill=paired-light-blue}]
% % ]
% [\vspace{-1ex}\textbf{\Large Faithfulness} \\  \textbf{\Large to Prompt} \\ % Larger font
%  vs. \\ % Larger font
%  \textbf{\Large Artistic Freedom}, for tree={fill=paired-light-blue}
%     [\textbf{Core Conflict:} \text{Adhering to user instructions}\\ \text{while minimizing creative deviations that}\\ \text{could alter the original intent.}, leaf, for tree={fill=paired-light-blue}]
% ]



%                 [\vspace{-1ex}\textbf{\Large Emotional} \\ \textbf{\Large Impact} \\ vs. \\ \textbf{\Large Neutrality}, for tree={fill=a26}
%                     [\textbf{Core Conflict:} \text{Striking a balance between}\\ \text{evoking specific emotions and maintaining an}\\ \text{unbiased, objective representation.}, leaf, for tree={fill=a26}]
%                 ]
%                 [\vspace{-1ex}\textbf{\Large Visual} \\ \textbf{\Large Realism} \\ vs. \\ \textbf{\Large Artistic Freedom}, for tree={fill=paired-light-pink}
%                     [\hspace{-0.7em}\textbf{Core Conflict:} \text{Create photorealistic images} \\ \hspace{-0.7em}\text {mimicking real-world visuals, incorporating}
%                     \\ \text{ artistic styles like impressionism, surrealism,}\\ \text{ or comic-like aesthetics only when requested.}, leaf, for tree={fill=paired-light-pink}]
%                 ]
%                 [\textbf{\Large Originality} \\ vs. \\ \textbf{\Large Referentiality}, for tree={fill=paired-light-orange}
%                     [\textbf{Core Conflict:} \text{Maintain stylistic originality}\\ \text{while avoiding dependence on established} \\ \text{artistic styles to prevent potential style}\\ \text{plagiarism.}, leaf, for tree={fill=paired-light-orange}]
%                 ]
%                 [\textbf{\Large Verifiability} \\ vs. \\  \textbf{\Large Artistic Freedom}, for tree={fill=paired-light-yellow}
%                     [\textbf{Core Conflict:} \text{Balancing factual accuracy  }\\ \text{against creative freedom to minimize} \\ \text{ misinformation.}, leaf, for tree={fill=paired-light-yellow}]
%                 ]
%                 [\vspace{-1ex}\textbf{\Large Cultural} \\ \textbf{\Large Sensitivity} \\ vs. \\ \textbf{\Large Artistic Freedom}, for tree={fill=a31}
%                     [\textbf{Core Conflict:} \text{Respecting cultural }\\ \text{ representations by ensuring creative liberties }\\ \text{do not lead to insensitivity} \\ \text{or misrepresentation.}, leaf, for tree={fill=a31}]
%                 ]
%             ]
%         \end{forest}
%     }
%     \caption{The figure illustrates six core trade-offs (\textit{e.g., Faithfulness to Prompt vs. Artistic Freedom, Emotional Impact vs. Neutrality}), highlighting key conflicts and their practical implications.}
%     \label{fig:alignment_axioms}
%     \vspace{-3mm}
% \end{figure}











\begin{figure*}[ht!]
    \centering
    \begin{subfigure}[t]{0.48\textwidth}
        \centering
        \includegraphics[width=\linewidth]{img/yinyang-1.pdf}
        \label{fig:peccavi_image_1}
    \end{subfigure}%
    \hfill
    \begin{subfigure}[t]{0.48\textwidth}
        \centering
        \includegraphics[width=\linewidth]{img/yinyang-2.pdf}
        \label{fig:peccavi_image_2}
    \end{subfigure}
    \vspace{-4mm}
    \caption{Illustrative examples of all six contradictory alignment axioms, with each row highlighting specific trade-offs between competing objectives (e.g., \textit{Faithfulness to Prompt vs. Artistic Freedom}, \textit{Emotional Impact vs. Neutrality}). Chosen and rejected outputs demonstrate the inherent tensions during text-to-image generation, underscoring the need for a multi-objective optimization framework. Examples of \textit{Originality vs. Referentiality} are inspired by recent \href{https://www.wired.com/story/ai-art-copyright-matthew-allen/}{copyright disputes reviewed by the U.S. Copyright Office}. The \textit{Verifiability vs. Artistic Freedom} case reflects incidents like the dissemination of a fake Pentagon explosion image by ‘verified’ Twitter accounts, causing confusion \href{https://www.cnn.com/2023/05/22/tech/twitter-fake-image-pentagon-explosion/index.html}{(CNN report)}. To mitigate misinformation caused harm, the system should avoid unverifiable content or produce subdued visuals when necessary. Lastly, the \href{https://www.theguardian.com/technology/2024/feb/22/google-pauses-ai-generated-images-of-people-after-ethnicity-criticism}{Google Gemini fiasco} underscores the need for Cultural Sensitivity in T2I systems, inspiring our \textit{Cultural Sensitivity vs. Artistic Freedom} example. cf \cref{fig:slider_selection} depicts controls and \cref{fig:slider_selection_image_variations_1} and \cref{fig:slider_selection_image_variations_2}
    \label{fig:peccavi_image_comparison} resultant genrations with varied control on generations.}
    \label{tab:yinyang_axioms} 
    \vspace{-2mm}
\end{figure*}


% Add the footnote without a number or marker
%{\renewcommand{\thefootnote}{}% Remove marker
%\footnotetext{* YinYangAlign is protected under a U.S. patent. As per the Patent Act, research or adoption of this technology for non-profit purposes is permitted, while unauthorized use for commercial purposes is strictly prohibited.}%
%}


\section{Why and How T2I Models Must Be Aligned?}

The alignment of T2I models is essential to ensure that generated visuals faithfully represent user intentions while adhering to ethical and aesthetic standards. This necessity is underscored by projections from EUROPOL, which estimate that by the end of 2026, approximately 90\% of web content will be generated by AI \cite{europol2023report}. The widespread use of AI-generated content underscores the need for robust alignment mechanisms to prevent misleading, biased, or unethical visuals. The recent announcement by social media plarforms \cite{meta2025speech} to remove all third-party fact-checking tools and adopt a more laissez-faire approach has sparked concerns about a potential misinformation apocalypse. This shift not only amplifies the risk of unchecked falsehoods spreading across the platform but also places greater responsibility on AI systems to manage and mitigate the flow of misleading content.

Alignment has been a vibrant area of research in Large Language Models (LLMs), with substantial progress achieved. Techniques like Reinforcement Learning from Human Feedback (RLHF) \cite{christiano2017deep} and Direct Preference Optimization (DPO) \cite{ouyang2022training} have been instrumental in enabling LLMs to generate responses that are both ethically sound and contextually appropriate. Moreover, several benchmarks \cite{bai2022traininghelpfulharmlessassistant,wang2023helpsteermultiattributehelpfulnessdataset,zheng2023judgingllmasajudgemtbenchchatbot,chiang2024chatbotarenaopenplatform,dubois2024alpacafarmsimulationframeworkmethods,lightman2023letsverifystepstep,cui2024ultrafeedbackboostinglanguagemodels,starling2023,lv2023supervised,daniele2023amplifyinstruct,daniele2023amplify,guo2022survey} have been developed to comprehensively evaluate alignment dimensions such as accuracy, safety, reasoning, and instruction-following. 

In contrast, alignment research for multimodal systems, especially T2I models, remains nascent with limited studies \cite{bansal2024safree,wallace2023diffusion,lee2023aligningtexttoimagemodelsusing,yarom2023readimprovingtextimagealignment}. The field lacks large-scale benchmarks and diverse alignment axioms, hindering holistic evaluation and optimization of T2I systems.

%Text-to-Image (T2I) AI lacks a robust benchmark for evaluating alignment across multiple objectives. Current evaluations are narrowly focused on aspects like prompt fidelity or aesthetic quality, neglecting the trade-offs between conflicting goals \cite{guo2022survey}. For example, increasing creative freedom may reduce image accuracy, while prioritizing diversity can undermine visual coherence. This absence of holistic evaluation methods hinders progress in optimizing T2I systems to balance these often contradictory objectives effectively.








%To address this gap, we introduce \textbf{YinYangAlign}, a comprehensive benchmark specifically designed to evaluate the alignment of T2I systems across six pairs of contradictory objectives. Each pair encapsulates fundamental tensions in image generation, such as maintaining adherence to user prompts while allowing for creative modifications, or balancing diversity with visual coherence. YinYangAlign provides detailed axiom datasets that include human prompts alongside aligned (chosen) and misaligned (rejected) AI-generated responses, accompanied by explanations of the underlying contradictions. This multifaceted approach enables a more rigorous and nuanced assessment of T2I models, facilitating the identification of strengths and areas for improvement in alignment strategies.

%In addition to introducing this new benchmark, we report the results of applying \textbf{Direct Preference Optimization (DPO)} to YinYangAlign. Furthermore, we propose a novel loss function for DPO, termed \textbf{Hybrid Loss}, which demonstrates improved performance in balancing conflicting objectives compared to traditional DPO. Despite these advancements, our evaluation reveals that \textbf{YinYangAlign} sets a new, stringent benchmark that current techniques, including our proposed Hybrid Loss, must strive to meet. This underscores the benchmark's role in pushing the boundaries of T2I alignment and highlights the ongoing need for sophisticated alignment mechanisms. Through YinYangAlign, we establish a rigorous standard for the community, driving future research aimed at developing more balanced and ethically aligned AI-driven creative tools.






