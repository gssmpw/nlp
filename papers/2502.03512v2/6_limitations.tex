

\clearpage
\newpage

\section{Discussion and Limitations} \label{sec:limitaions}

The development of \textbf{YinYangAlign} introduces a novel paradigm for balancing contradictory axioms in Text-to-Image (T2I) systems, offering both theoretical contributions and practical implications. However, as with any sophisticated framework, its deployment and efficacy raise important points of discussion and reveal inherent limitations. This section critically examines the strengths and potential areas for improvement in \textbf{YinYangAlign}, situating it within the broader landscape of T2I alignment research.

We begin by reflecting on the broader implications of our methodology, including its adaptability to diverse tasks and its capacity to integrate user preferences dynamically. We then address the limitations that stem from reliance on predefined axioms, the scalability of the framework across domains, and the challenges associated with data diversity and representation. These reflections aim to provide a balanced perspective, guiding future refinements and encouraging dialogue within the research community to advance T2I alignment technologies further.


\begin{figure}[!htb]
    \centering
    \includegraphics[width=\columnwidth]{img/scales.png}
    \caption{This interface allows users to dynamically set their preferences for balancing contradictory axioms in Text-to-Image (T2I) generation. Each slider represents a specific trade-off, such as \emph{Faithfulness to Prompt vs. Artistic Freedom}, enabling fine-grained control over the alignment objectives. The left and right labels denote opposing axiom components, with the slider position reflecting the user's preferred weight distribution. These inputs are translated into weights for the Contradictory Alignment Optimization (CAO) framework, guiding the system toward generating outputs tailored to user-defined priorities.
    }
    \label{fig:slider_selection}
\end{figure}


\begin{figure*}[htb!]
    \centering
    \includegraphics[width=\textwidth]{img/yinyang_details-1.pdf}
    \includegraphics[width=\textwidth]{img/yinyang_details-2.pdf}
\caption{A Comprehensive Visual Depiction of Trade-offs Between Alignment Axioms Across Prompts and Visual Styles.  
Each row represents a specific textual prompt, showcasing variations in alignment across different axioms.  
\textbf{(Row 1:)} \textit{Illustrate a peaceful garden with a bench under a cherry blossom tree.}  
This row explores the trade-off between \textbf{Faithfulness to Prompt} and \textbf{Artistic Freedom}, transitioning from highly creative interpretations (left) to more realistic depictions (right).  
\textbf{(Row 2:)} \textit{A post-disaster scene.}  
This row examines the balance between \textbf{Emotional Impact} and \textbf{Neutrality}, ranging from emotionally intense scenes (right) to neutral and documentary-style visuals (left).  
\textbf{(Row 3:)} \textit{A portrait of Albert Einstein, showcasing his iconic wild hair and mustache, dressed in a classic suit.}  
Here, the interplay between \textbf{Visual Realism} and \textbf{Artistic Freedom} is illustrated, with images evolving from abstract and stylized (left) to photorealistic (right).  
\textbf{(Row 4:)} \textit{A majestic cathedral interior with an ethereal glowing circular portal leading to a serene golden landscape.}  
This row highlights the trade-off between \textbf{Originality} and \textbf{Referentiality}, transitioning from imaginative, fantastical architecture (left) to "Théâtre d'Opéra Spatial" style grounded representations (right).  
Adjustable parameters and metrics are shown for each prompt, underscoring how alignment affects the model's ability to balance creativity and fidelity.}    \label{fig:slider_selection_image_variations_1}
\end{figure*}

\begin{figure*}[htb!]
    \centering
    \includegraphics[width=\textwidth]{img/yinyang_details-3.pdf}
    \caption{A Comprehensive Visual Depiction of Alignment Trade-offs for \textit{"Pentagon is under fire"} and \textit{"Images of Vikings"} Across Alignment Axioms.  
\textbf{(Row 1:)} \textit{Pentagon is under fire.}  
This row demonstrates the trade-off between \textbf{Verifiability} and \textbf{Artistic Freedom}. The rightmost image depicts a verifiable and realistic representation of the Pentagon under fire, emphasizing factual accuracy. Progressing to the left, the images increasingly prioritize artistic freedom, featuring surreal fire patterns, dramatic lighting, and exaggerated destruction, illustrating the tension between verifiability and creativity.  
\textbf{(Row 2:)} \textit{Images of Vikings.}  
This row examines the balance between \textbf{Cultural Sensitivity} and \textbf{Artistic Freedom}. The leftmost image highlights cultural diversity and sensitivity, showcasing gender-balanced and ethnically varied Vikings, including Asian, African, and Mexican influences. Moving towards the right, artistic freedom faded, leading to depictions of Nordic-centric, rugged warriors with reduced diversity. This evolution highlights how cultural sensitivity diminishes as artistic freedom decreases.  
\textbf{Adjustable Parameters:} Alignment parameters, such as \textbf{Faithfulness}, \textbf{Artistic Freedom}, \textbf{Verifiability}, and \textbf{Cultural Sensitivity}, are depicted through sliders for each prompt. These settings demonstrate the trade-offs influencing the alignment results, enabling an evaluation of the model's ability to balance competing objectives.}
    \label{fig:slider_selection_image_variations_2}
    \vspace{-3mm}
\end{figure*}


\subsection{Mapping User Preferences to Multi-Objective Optimization Weights}

YinYangAlign introduces a flexible and user-centric framework (cf. \cref{fig:slider_selection} for controls and \cref{fig:slider_selection_image_variations_1} and \cref{fig:slider_selection_image_variations_2} for the effect of varied controls on the output) for aligning text-to-image (T2I) models with potentially contradictory axioms. A core strength of this framework lies in its adaptability: given sufficient annotated data, end-users/developer can specify their desired balance between competing objectives, such as \emph{Faithfulness to Prompt} versus \emph{Artistic Freedom} or \emph{Cultural Sensitivity} versus \emph{Creative Expression}. This customization is facilitated by the Contradictory Alignment Optimization (CAO) mechanism, which translates user-defined preferences into weights for multi-objective optimization.


By leveraging the sliders, users directly influence the blending of contradictory axioms, enabling a tailored optimization process that reflects individual or application-specific requirements. For instance, a use case focused on creative content generation may prioritize \emph{Artistic Freedom}, while another requiring factual accuracy and cultural sensitivity may emphasize \emph{Verifiability} and \emph{Cultural Sensitivity}. The CAO framework dynamically adapts to these preferences, ensuring that the optimization process aligns with user-defined priorities.

This section details how user-selected scales, representing preferences for contradictory axioms, are normalized and integrated into the multi-objective optimization process. The mathematical foundation of this mapping ensures clarity, reproducibility, and seamless adaptability for various use cases. Below, we describe the key steps involved in translating user preferences into actionable weights for CAO's optimization pipeline.


\subsection*{1. Normalize Slider Values}
Each slider value \(v_i\) is normalized to compute the weight \(\alpha_i\) for the \(i\)-th axiom. The normalization ensures the weights sum to 1:
\[
\alpha_i = \frac{v_i}{\sum_{j=1}^N v_j}, \quad \text{for } i = 1, \dots, N,
\]

\vspace{-3mm}
where:
\begin{itemize}[leftmargin=10pt, itemsep=0pt, topsep=3pt]
    \item \(v_i\): Value of the \(i\)-th slider (e.g., \(v_1 = 67\) for Faithfulness to Prompt).
    \item \(N\): Total number of axioms (e.g., \(N = 6\)).
\end{itemize}

\clearpage

\subsection*{2. Define Multi-Objective Loss Function}
Using the computed weights \(\alpha_i\), the multi-objective loss function is defined as:
\[
\mathcal{L}_{\text{multi}} = \sum_{i=1}^N \alpha_i \cdot \mathcal{L}_i,
\]
where:
\begin{itemize}[leftmargin=10pt, itemsep=0pt, topsep=3pt]
    \item \(\mathcal{L}_i\): Loss function corresponding to the \(i\)-th axiom (e.g., \(\mathcal{L}_{\text{faith}}\), \(\mathcal{L}_{\text{emotion}}\)).
    \item \(\alpha_i\): Weight derived from the slider value \(v_i\).
\end{itemize}

\subsection*{3. Example Calculation}
Given the following slider values: \text{Faithfulness to Prompt: } $67$, 
\text{Emotional Impact: } $55$, 
\text{Visual Realism: } $75$, 
\text{Originality: } $66$, 
\text{Verifiability: } $72$, 
\text{Cultural Sensitivity: } $63$.
The total slider value is:
\[
\sum_{i=1}^N v_i = 67 + 55 + 75 + 66 + 72 + 63 = 398.
\]
The normalized weights are:
\[
\alpha_1 = \frac{67}{398}, \quad \alpha_2 = \frac{55}{398}, \quad \alpha_3 = \frac{75}{398}, \quad \alpha_4 = \frac{66}{398}, \quad \alpha_5 = \frac{72}{398}, \quad \alpha_6 = \frac{63}{398}.
\]

\subsection*{4. Final Multi-Objective Loss Function}
The resulting multi-objective loss is:

\begin{multline*}
\mathcal{L}_{\text{multi}} = \alpha_1 \cdot \mathcal{L}_{\text{faith}}
+ \alpha_2 \cdot \mathcal{L}_{\text{emotion}}
+ \alpha_3 \cdot \mathcal{L}_{\text{realism}} \\
+ \alpha_4 \cdot \mathcal{L}_{\text{originality}}
+ \alpha_5 \cdot \mathcal{L}_{\text{verifiability}}
+ \alpha_6 \cdot \mathcal{L}_{\text{cultural}},
\end{multline*}

where \(\alpha_1, \alpha_2, \ldots, \alpha_6\) are the normalized weights derived from the user-selected slider values.

\subsection*{Advantages}
\begin{itemize}[leftmargin=10pt, itemsep=0pt, topsep=3pt]
    \item \textbf{Flexibility:} The weights are dynamically adjustable based on user preferences.
    \item \textbf{Interpretability:} Slider positions directly correspond to the weight of each objective.
    \item \textbf{Adaptive Optimization:} The weights can guide optimization algorithms to achieve a user-preferred balance among competing objectives.
\end{itemize}



\subsection{Limitations}
While \textbf{YinYangAlign} provides a robust framework for evaluating alignment in Text-to-Image (T2I) systems, it has certain limitations that warrant further exploration:
\begin{itemize}
    \item \textbf{Dataset Diversity:} The evaluation uses reference datasets like WikiArt and BAM, which are widely used benchmarks in artistic style and media research \cite{saleh2015large, wilber2017bam}. While these datasets are extensive, containing diverse styles and high-resolution media, they may not fully capture the breadth of cultural or stylistic nuances present globally. This limitation introduces potential biases in alignment evaluation, particularly for underrepresented styles or cultural contexts, a concern echoed in prior work on dataset fairness and representativeness in machine learning \cite{gebru2018datasheets, dodge2021documenting}. Future efforts could focus on expanding these datasets to include a broader range of cultural expressions, ensuring more equitable and robust alignment evaluations.
    \item \textbf{Annotation Bottlenecks:} Despite leveraging Vision-Language Models (VLMs) and human verification for annotations, the process is time-intensive. Scaling YinYangAlign to larger datasets or additional alignment axes might require more automated yet reliable annotation methods.
    \item \textbf{Assumption of Trade-off Synergies:} The Contradictory Alignment Optimization (CAO) framework presumes that all alignment objectives can be synergized through weighted trade-offs. However, certain objectives, such as Cultural Sensitivity and Emotional Impact, may present irreconcilable conflicts in specific contexts. For example, an emotionally impactful image might unintentionally invoke cultural insensitivity, particularly in cross-cultural scenarios. Similar challenges in handling competing objectives have been discussed in multi-objective optimization literature, such as Pareto efficiency in high-dimensional spaces \cite{lin2023pareto, miettinen1999nonlinear, navon2022multi}. These inherent tensions could lead to suboptimal outcomes for tasks requiring careful navigation of such conflicts. We encourage further research to identify cases where trade-offs fail and propose adaptive mechanisms to address irreconcilable objectives while maintaining alignment robustness.
    \item \textbf{CAO with numerous contradictory axioms:} While CAO effectively balances contradictory objectives, its scalability with an increasing number of axioms remains uncertain. The weighted aggregation of per-axiom preferences may introduce computational and optimization challenges, such as diminishing returns or unintended conflicts. Similar concerns are raised in hierarchical multi-task optimization \cite{ma2020quadratic, liebenwein2021provable}, where clustering objectives into modular sub-problems has shown promise. We urge the community to further experiment with and explore the scalability of synergy mechanisms in multi-axiom settings. Addressing these challenges forms a core agenda for future extensions of this work, with a focus on exploring hierarchical or modular synergy mechanisms that cluster related axioms into hierarchical levels, thereby reducing computational overhead while ensuring robustness and effectiveness in diverse alignment scenarios.
    \item \textbf{Risk of Overfitting to Training Trade-offs:} While the CAO framework effectively balances contradictory objectives, it risks overfitting to the specific trade-offs and preferences defined in the training data. This overfitting could limit the model's generalizability across diverse prompts or domains, potentially reducing its adaptability to novel or unseen scenarios. Future work could explore techniques such as domain adaptation or prompt diversity augmentation to mitigate this limitation.
\end{itemize}

\subsection{Ethical Considerations \& Benifits}

The development of the \textbf{YinYangAlign} framework presents significant ethical considerations, given the model's potential to influence societal norms, cultural representations, and artistic expressions. Below, we revisit these aspects with a grounded perspective:


\begin{itemize}
    \item \textbf{Bias Mitigation:} By introducing alignment axes such as Cultural Sensitivity vs.\ Artistic Freedom, \textbf{YinYangAlign} explicitly incorporates mechanisms to detect and mitigate cultural insensitivity or stereotyping in generated content. This is particularly important for creating inclusive and respectful outputs.
    \item \textbf{Social Manipulation Risks:} The inclusion of objectives like Emotional Impact and Faithfulness to Prompt makes the framework powerful for persuasive content generation. However, this capability introduces significant risks of misuse, particularly in generating emotionally manipulative or misleading content for political campaigns or advertising~\cite{hwang2020deepfake, zihao2022disinformation}. Such uses could amplify societal polarization, manipulate public opinion, or exploit consumer vulnerabilities. Mitigating these risks necessitates embedding transparency and accountability mechanisms into the generation pipeline, such as digital watermarks~\cite{ferreira2021watermarking} and provenance tracking systems~\cite{agarwal2019blockchain}, to ensure traceability and authenticity. These measures, when integrated effectively, can safeguard against unethical deployment while maintaining the technical utility of the framework.
    \item \textbf{Environmental Impact:} Training and deploying models like \textbf{YinYangAlign} demand considerable computational resources, contributing to carbon emissions. Studies have shown that large-scale model training can have a substantial carbon footprint~\cite{strubell2019energy, patterson2021carbon}. Ethical deployment requires addressing this environmental footprint by optimizing computational efficiency and exploring carbon-offsetting measures~\cite{anthony2020carbontracker}. 
    \item \textbf{Call to Action for the Research Community:} We urge the research community to adopt a proactive role in auditing and improving alignment frameworks like \textbf{YinYangAlign}. Collaborations with \emph{ethicists, social scientists, and legal experts} are essential to navigate the nuanced challenges posed by such technologies. Transparency in the model's design and decision-making processes, coupled with ongoing community engagement, will be critical to its responsible development and use.
\end{itemize}
