


\section{Conclusion}

In this work, we introduced \textbf{YinYangAlign}, a novel benchmark for evaluating Text-to-Image (T2I) systems across six contradictory alignment objectives, each representing fundamental trade-offs in AI image generation. The study demonstrated that optimizing for a single alignment axiom, such as \emph{Artistic Freedom} or \emph{Faithfulness to Prompt}, often disrupts the balance of other alignment objectives, leading to significant performance declines in areas like \emph{Cultural Sensitivity} and \emph{Verifiability}. This emphasizes the critical need for holistic optimization strategies.

\textls[-10]{To navigate these alignment tensions, we propose \textbf{Contradictory Alignment Optimization (CAO)}, a transformative extension of Direct Preference Optimization (DPO). The CAO framework introduces multi-objective optimization mechanisms, including \textit{synergy-driven global preferences}, \textit{axiom-specific regularization}, and the innovative \textit{synergy Jacobian} to balance competing objectives effectively. By leveraging tools such as \textit{Sinkhorn-regularized Wasserstein Distance}, CAO achieves stability and scalability while delivering state-of-the-art performance across all six objectives.}

Empirical results validate the robustness of the proposed framework, showcasing not only its ability to align T2I outputs with diverse user intents but also its adaptability across multiple datasets and alignment goals. Moreover, the \textbf{YinYangAlign} dataset and benchmark provide a critical resource for advancing future research in generative AI alignment, emphasizing fairness, creativity, and cultural sensitivity.

This work establishes a foundation for the principled design and evaluation of alignment strategies, paving the way for scalable, interpretable, and ethically sound T2I systems. Future work will explore adaptive mechanisms for dynamic weight tuning and extend the framework to emerging alignment challenges, further cementing YinYangAlign and CAO as cornerstones in the field of generative AI.



%\textbf{YinYangAlign} not only sets a new standard for T2I alignment evaluation but also reveals the limitations of existing approaches. By rigorously benchmarking alignment models and presenting a comprehensive optimization strategy, this work lays a strong foundation for the development of future T2I systems capable of balancing creativity, fidelity, and ethical considerations.
