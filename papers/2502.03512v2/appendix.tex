








\appendix
\section{Appendix}
\label{sec:appendix}

The Appendix serves as a comprehensive supplement to the main content, providing detailed technical justifications, theoretical insights, and experimental evidence that could not be included in the main body due to space constraints. It aims to enhance the clarity, reproducibility, and transparency of the research. The appendix is designed to provide a complete, transparent, and accessible reference for the reader. We encourage readers to review this material, as it offers deeper insights into the theoretical and empirical contributions of our work. This appendix is organized into several key sections:



\begin{itemize}[leftmargin=15pt,nolistsep]

\item[\ding{93}] \textbf{Richer Representation: Hybrid Loss}: Key points are outlined in \cref{sec:representation}, while Appendix \cref{sec:appendix:hybrid_loss} provides detailed derivations and theoretical underpinnings of the Hybrid Loss.


\item[\ding{93}] \textbf{Kernel-Integrated DPO Formulation}: Key points are covered in \cref{sec:kernel_dpo}, with Appendix \cref{sec:appendix:dpo_kernel} detailing Hybrid Loss derivations using specific kernels: RBF, Polynomial, Spectral, and Mahalanobis.


\item[\ding{93}] \textbf{Alternative Divergence Functions}: Beyond KL divergence, we explore Jensen-Shannon, Hellinger, Rényi, Bhattacharyya, Wasserstein, and $f$-divergences, outlined in \cref{sec:divergence} and detailed in \cref{sec:appendix:alternative_divergences}.

\item[\ding{93}] \textbf{Data-Driven Selection of Kernel-Divergence}: Choosing the optimal kernel-divergence pair from 28 combinations (4 kernels × 7 divergences) is complex. To address this, we introduce 4 metrics for kernel selection—\textit{PND}, \textit{PNAV}, \textit{TAT}, and \textit{NAG}—and 4 for divergence selection: \textit{Support Overlap}, \textit{Drift Magnitude}, \textit{Kurtosis}, and \textit{Smoothness}, outlined in \cref{sec:data_driven_kernel_selection} and extended in \cref{sec:appendix:data_driven_kernel_divergence}).

\item[\ding{93}] We highlight the advantages of the Kernel Mixture approach over single-kernel learning and introduce the \textbf{Hierarchical Mixture of Kernels (HMK)} in \cref{sec:kernel_mixture_main}, with detailed discussion in \cref{sec:appendix:kernel_mixture}.


\item[\ding{93}] \textbf{Gradient Computation, Computational Complexity, and Overhead}: Appendix \cref{sec:appendix:gradient_complexity} details gradient derivations for various kernels and divergences, along with complexity analysis and computational overhead. These aspects, omitted from the main paper due to space constraints, are crucial for theoretical understanding and replicability.


\item[\ding{93}] \textbf{Empirical Findings}: Results from 12 datasets are summarized in \cref{sec:results} and expanded upon in \cref{sec:appendix:results}.

\item[\ding{93}] \textbf{Gradient Descent Dynamics on Kernel-Induced Loss Landscapes}: In \cref{sec:appendix:loss_landscape}, we analyze gradient descent dynamics on loss landscapes induced by \textbf{RBF}, \textbf{Polynomial}, \textbf{Spectral}, \textbf{Mahalanobis} kernels, and HMK, briefly mentioned in the main body in \cref{fig:error_surface_horizontal}.

\item[\ding{93}] \textbf{Safe vs. Unsafe Cluster Effects}: Kernel-induced clustering during safety fine-tuning projects unsafe inputs into null spaces \cite{jain2024safetyfinetuning}, forming distinct clusters for safe and unsafe data. Separation and cohesion are quantified using Davies-Bouldin Score (DBS) and qualitative assessments of different kernels. Discussed in \cref{sec:safe_unsafe_cluster} and detailed in \cref{sec:appendix:safe_unsafe_cluster}.




\item[\ding{93}] \textbf{Heavy-Tailed Self-Regularization (HT-SR) - Generalization}: Using the \textit{Weighted Alpha} metric proposed in \cite{martin2021predicting}, grounded in HT-SR theory, we investigate whether aligned models, particularly HMK, exhibit overfitting and quantify its extent. Theoretical bounds for all kernels and HMK are analyzed, with an overview in \cref{sec:HTSR_generalization} and detailed findings in \cref{sec:appendix:htsr_generalization}.


\item[\ding{93}] \textbf{Hyperparameters and Best Practices}: Key hyperparameter settings and practical guidelines for optimizing DPO-Kernel performance across tasks are detailed in \cref{sec:appendix:hyperparameter}, as space constraints no scope of discussion in the main paper.


\end{itemize}



--------------------------------------




\begin{theorem}[Existence and Uniqueness of DPO-CAO Equilibrium]
\label{thm:unique-equilibrium}
Suppose each axiom-wise sub-loss \(\mathcal{L}_p\) and \(\mathcal{L}_q\) is \(\rho\)-smooth and \(\mu\)-strongly convex in model parameters, and synergy weights \(\omega_a, \alpha_a, \lambda\) lie in bounded sets. Then there exists a unique minimizer of \(\mathcal{L}_{\text{DPO-CAO}}\). Moreover, gradient-based methods converge linearly at rate \(\bigl(1 - \tfrac{\mu}{\rho}\bigr)\).
\end{theorem}

\noindent
\emph{The proof, sketched in Appendix~\ref{sec:appendix-proof}, draws on properties of exponential families for Bradley-Terry preferences. Our result extends standard \emph{convex duality} arguments~\cite{boyd2004convex} to the multi-objective synergy setting.}

\subsection{Axiom-Specific Regularization and the Synergy Jacobian}
\label{sec:regularization-synergy-jacobian}

To stabilize training across dimensions and prevent \emph{overfitting} to any one objective, we add an axiom-specific regularizer \(\mathcal{R}_a\), scaled by \(\tau_a\).  
\[
\mathcal{L}_{\text{reg}}
= 
\sum_{a=1}^A
\tau_a
\,
\mathcal{R}_a,
\]
where we adopt the \emph{Wasserstein Distance}~\cite{arjovsky2017wasserstein} (or Jensen-Shannon, etc.) in practice.

\vspace{3pt}
\noindent
\textit{(\textbf{Wonder 4: Synergy Jacobian for Regularization})}\;
\emph{To further quantify how \emph{each axiom’s gradient} affects the global synergy, we define the \(\mathbf{J}_{\mathcal{S}}\), the \textbf{Synergy Jacobian}, whose \(a\)-th row is}
\[
\nabla_{\theta}
\bigl(\omega_a f_a(I)\bigr),
\]
\emph{where \(\theta\) denotes model parameters. This matrix captures \emph{cross-interactions} among contradictory goals. In \S\ref{sec:appendix-jacobian}, we prove that if \(\mathbf{J}_{\mathcal{S}}\) remains non-singular, synergy-based updates spread gradient signal across axioms—preventing local minima that overly favor one objective.}

\begin{table*}[ht!]
\centering
\small
\caption{This table presents two examples where human prompts are assessed against AI-generated outputs. Chosen (aligned) responses maintain prompt fidelity while integrating subtle enhancements, whereas rejected (misaligned) responses deviate significantly by prioritizing creative liberties. }
\label{tab:pair1}
\resizebox{\textwidth}{!}{%
\begin{tabular}{|!{\vrule width 1pt}>{\centering\arraybackslash}m{1.5cm}!{\vrule width 1pt}|m{4cm}|m{5cm}|m{5cm}|m{4cm}|}
\hline
\multirow{4}{=}{\centering\vfill\textbf{\rotatebox{90}{\Large Faithfulness to Prompt vs. Creative Enhancement}}\vfill} & 
\textbf{Human Prompt} & 
\textbf{Chosen Response} & 
\textbf{Rejected Response} & 
\textbf{Contradiction Explanation} \\ \cline{2-5}

& Illustrate a peaceful garden with a bench under a cherry blossom tree. & 
\begin{minipage}{\linewidth}
    \includegraphics[width=0.7\linewidth]{img/1.png} \\
    A serene garden scene featuring a comfortable bench positioned beneath a blooming cherry blossom tree, with gentle sunlight filtering through the branches and butterflies fluttering around.
\end{minipage} & 
\begin{minipage}{\linewidth}
    \includegraphics[width=0.7\linewidth]{img/2.png} \\
    A fantastical garden where the cherry blossom tree glows neon colors and the bench levitates above the ground amidst swirling magical lights.
\end{minipage} & 
Balances accurate representation of a peaceful garden with subtle creative enhancements like butterflies and sunlight to enrich the image without deviating from the prompt. \\ \cline{2-5}

& Generate an image of a futuristic city skyline at dusk. & 
\begin{minipage}{\linewidth}
    \includegraphics[width=0.7\linewidth]{img/3.png} \\
    A realistic depiction of a futuristic city skyline during dusk, showcasing advanced architecture, illuminated skyscrapers, and a vibrant sky with hues of orange and purple.
\end{minipage} & 
\begin{minipage}{\linewidth}
    \includegraphics[width=0.7\linewidth]{img/4.png} \\
    Create a city skyline where buildings are made of glass and steel but feature unconventional shapes like spirals and floating structures, with a surreal dusk sky filled with multiple moons.
\end{minipage} & 
Maintains faithfulness to a futuristic city at dusk while introducing creative elements such as unique building shapes and a vibrant sky to enhance visual interest without straying from the original intent. \\ \hline

\end{tabular}%
}
\label{tab:axiom_1}
\end{table*}


\begin{table*}[ht!]
\centering
\small
\caption{This table presents two examples where human prompts are assessed against AI-generated outputs. Chosen (aligned) responses maintain prompt fidelity while integrating subtle enhancements, whereas rejected (misaligned) responses deviate significantly by prioritizing creative liberties. }
\label{tab:pair1}
\resizebox{\textwidth}{!}{%
\begin{tabular}{|!{\vrule width 1pt}>{\centering\arraybackslash}m{1.5cm}!{\vrule width 1pt}|m{4cm}|m{5cm}|m{5cm}|m{4cm}|}
\hline
\multirow{4}{=}{\centering\vfill\vspace{2.5cm}\textbf{\rotatebox{90}{\LARGE Diversity vs. Coherence}}\vfill} & 
\textbf{Human Prompt} & 
\textbf{Chosen Response} & 
\textbf{Rejected Response} & 
\textbf{Contradiction Explanation} \\ \cline{2-5}

& Create a series of illustrations showcasing different wildlife in their natural habitats. & 
\begin{minipage}{\linewidth}
    \includegraphics[width=0.7\linewidth]{img/5.png} \\
    Develop a coherent series where each illustration accurately represents a distinct animal species in its specific natural environment, maintaining a consistent artistic style throughout the series.
\end{minipage} & 
\begin{minipage}{\linewidth}
    \includegraphics[width=0.7\linewidth]{img/6.png} \\
    Produce a series where each image features a different animal, but the styles and environments are randomly combined, leading to a lack of logical coherence across the series.
\end{minipage} & 
Balances diversity by showcasing various wildlife while ensuring each image is coherent and stylistically consistent, enhancing both variety and unity. \\ \cline{2-5}

& Generate images of people from various cultures celebrating their traditional festivals. & 
\begin{minipage}{\linewidth}
    \includegraphics[width=0.7\linewidth]{img/7.png} \\
    Create a series of images where each one authentically depicts a different cultural festival, maintaining visual coherence through consistent color palettes and artistic styles.
\end{minipage} & 
\begin{minipage}{\linewidth}
    \includegraphics[width=0.7\linewidth]{img/8.png} \\
    Design images where elements from different cultural festivals are mixed within a single scene, causing confusion and lack of cultural authenticity.
\end{minipage} & 
Ensures diversity by representing multiple cultures accurately while maintaining coherence through unified artistic elements and respectful representations. \\ \hline

\end{tabular}%
}
\label{tab:axiom_2}
\end{table*}


\begin{table*}[ht!]
\centering
\small
\caption{This table presents two examples where human prompts are assessed against AI-generated outputs. Chosen (aligned) responses maintain prompt fidelity while integrating subtle enhancements, whereas rejected (misaligned) responses deviate significantly by prioritizing creative liberties. }
\label{tab:pair1}
\resizebox{\textwidth}{!}{%
\begin{tabular}{|!{\vrule width 1pt}>{\centering\arraybackslash}m{1.5cm}!{\vrule width 1pt}|m{4cm}|m{5cm}|m{5cm}|m{4cm}|}
\hline
\multirow{4}{=}{\centering\vfill\textbf{\rotatebox{90}{\LARGE Cultural Sensitivity vs. Artistic Freedom}}\vfill} & 
\textbf{Human Prompt} & 
\textbf{Chosen Response} & 
\textbf{Rejected Response} & 
\textbf{Contradiction Explanation} \\ \cline{2-5}

& Illustrate a traditional Japanese tea ceremony in a respectful manner. & 
\begin{minipage}{\linewidth}
    \includegraphics[width=0.7\linewidth]{img/9.png} \\
    Depict a traditional Japanese tea ceremony with accurate cultural details, respectful attire, and authentic setting, while incorporating subtle artistic enhancements to enhance visual appeal.
\end{minipage} & 
\begin{minipage}{\linewidth}
    \includegraphics[width=0.7\linewidth]{img/10.png} \\
    Create a tea ceremony scene with exaggerated features, such as oversized utensils and vibrant, unrealistic colors that distort cultural authenticity.
\end{minipage} & 
Maintains cultural sensitivity by accurately representing the tea ceremony while allowing for artistic enhancements that enrich the image without disrespecting cultural elements. \\ \cline{2-5}

& Design a character inspired by various African cultures. & 
\begin{minipage}{\linewidth}
    \includegraphics[width=0.7\linewidth]{img/11.png} \\
    Develop a character that thoughtfully integrates traditional African attire, accessories, and cultural symbols, ensuring respectful and accurate representation across different African cultures.
\end{minipage} & 
\begin{minipage}{\linewidth}
    \includegraphics[width=0.7\linewidth]{img/12.png} \\
    Combine stereotypical and inaccurate elements from multiple African cultures into a single character, leading to cultural misrepresentation and insensitivity.
\end{minipage} & 
Balances artistic freedom by creating a unique character while ensuring cultural elements are accurately and respectfully represented, avoiding stereotypes and misappropriation. \\ \hline

\end{tabular}%
}
\label{tab:axiom_3}
\end{table*}



\begin{table*}[ht!]
\centering
\small
\caption{This table presents two examples where human prompts are assessed against AI-generated outputs. Chosen (aligned) responses maintain prompt fidelity while integrating subtle enhancements, whereas rejected (misaligned) responses deviate significantly by prioritizing creative liberties. }
\label{tab:pair1}
\resizebox{\textwidth}{!}{%
\begin{tabular}{|!{\vrule width 1pt}>{\centering\arraybackslash}m{1.5cm}!{\vrule width 1pt}|m{4cm}|m{5cm}|m{5cm}|m{4cm}|}
\hline
\multirow{4}{=}{\centering\vfill\vspace{1.5cm}\textbf{\rotatebox{90}{\LARGE Emotional Impact vs. Neutrality}}\vfill} & 
\textbf{Human Prompt} & 
\textbf{Chosen Response} & 
\textbf{Rejected Response} & 
\textbf{Contradiction Explanation} \\ \cline{2-5}

& Generate an image that conveys a sense of hope and renewal in a post-disaster setting & 
\begin{minipage}{\linewidth}
    \includegraphics[width=0.7\linewidth]{img/13.png} \\
    Create a visually uplifting scene showing a community rebuilding after a disaster, incorporating elements like green shoots, supportive interactions, and a brightening sky to evoke hope and renewal.
\end{minipage} & 
\begin{minipage}{\linewidth}
    \includegraphics[width=0.7\linewidth]{img/14.png} \\
    Produce a stark, somber image with dark clouds and desolate surroundings, failing to evoke the intended sense of hope and renewal.
\end{minipage} & 
Balances emotional impact by effectively conveying hope and renewal while maintaining a realistic and neutral representation of a post-disaster setting. \\ \cline{2-5}

& Illustrate a balanced newsroom scene without showing bias towards any political stance. & 
\begin{minipage}{\linewidth}
    \includegraphics[width=0.7\linewidth]{img/15.png} \\
    Depict a newsroom with journalists from diverse backgrounds working collaboratively, ensuring unbiased representation and neutral visual elements that do not favor any political ideology.
\end{minipage} & 
\begin{minipage}{\linewidth}
    \includegraphics[width=0.7\linewidth]{img/16.png} \\
    Show journalists expressing strong opinions or displaying symbols related to a specific political stance, introducing bias and compromising neutrality.
\end{minipage} & 
Maintains neutrality by presenting a balanced and unbiased newsroom scene while avoiding elements that could convey a specific political stance or emotion. \\ \hline

\end{tabular}%
}
\label{tab:axiom_4}
\end{table*}


\begin{table*}[ht!]
\centering
\small
\caption{This table presents two examples where human prompts are assessed against AI-generated outputs. Chosen (aligned) responses maintain prompt fidelity while integrating subtle enhancements, whereas rejected (misaligned) responses deviate significantly by prioritizing creative liberties. }
\label{tab:pair1}
\resizebox{\textwidth}{!}{%
\begin{tabular}{|!{\vrule width 1pt}>{\centering\arraybackslash}m{1.5cm}!{\vrule width 1pt}|m{4cm}|m{5cm}|m{5cm}|m{4cm}|}
\hline
\multirow{4}{=}{\centering\vfill\textbf{\rotatebox{90}{\LARGE Alignment with User Intent vs. AI Autonomy}}\vfill} & 
\textbf{Human Prompt} & 
\textbf{Chosen Response} & 
\textbf{Rejected Response} & 
\textbf{Contradiction Explanation} \\ \cline{2-5}

& Generate an image of a cozy reading nook. & 
\begin{minipage}{\linewidth}
    \includegraphics[width=0.7\linewidth]{img/17.png} \\
    Create a warm and inviting reading nook featuring a comfortable armchair upholstered in soft fabric, a small wooden bookshelf filled with books, soft ambient lighting from a nearby lamp, and personal touches like a knitted blanket and a potted plant. Window view with gentle sunlight streaming in enhances the cozy atmosphere.
\end{minipage} & 
\begin{minipage}{\linewidth}
    \includegraphics[width=0.7\linewidth]{img/18.png} \\
    Automatically add elements like a fireplace or extravagant decor not specified by the user, altering the original intent of the cozy reading nook.
\end{minipage} & 
Balances alignment with user intent by adhering closely to the user's description while allowing for minor autonomous enhancements that enrich the image without deviating from the core request. \\ \cline{2-5}

& llustrate a futuristic workspace. & 
\begin{minipage}{\linewidth}
    \includegraphics[width=0.7\linewidth]{img/19.png} \\
    Develop a futuristic workspace that incorporates the user's specified features such as sleek furniture, advanced technology, and open layouts, while introducing innovative design elements like holographic displays to enhance the concept.
\end{minipage} & 
\begin{minipage}{\linewidth}
    \includegraphics[width=0.7\linewidth]{img/20.png} \\
    Add unrelated futuristic gadgets or design elements that were not mentioned by the user, thereby shifting the original vision of the workspace.
\end{minipage} & 
Maintains alignment with the user's envisioned workspace while autonomously integrating advanced technology and design elements that complement the original intent without altering its fundamental concept. \\ \hline

\end{tabular}%
}
\label{tab:axiom_5}
\end{table*}



\begin{table*}[ht!]
\centering
\small
\caption{This table presents two examples where human prompts are assessed against AI-generated outputs. Chosen (aligned) responses maintain prompt fidelity while integrating subtle enhancements, whereas rejected (misaligned) responses deviate significantly by prioritizing creative liberties. }
\label{tab:pair1}
\resizebox{\textwidth}{!}{%
\begin{tabular}{|!{\vrule width 1pt}>{\centering\arraybackslash}m{1.5cm}!{\vrule width 1pt}|m{4cm}|m{5cm}|m{5cm}|m{4cm}|}
\hline
\multirow{4}{=}{\centering\vfill\vspace{2.5cm}\textbf{\rotatebox{90}{\LARGE Originality vs. Referentiality}}\vfill} & 
\textbf{Human Prompt} & 
\textbf{Chosen Response} & 
\textbf{Rejected Response} & 
\textbf{Contradiction Explanation} \\ \cline{2-5}

& Create an original character for a fantasy novel. & 
\begin{minipage}{\linewidth}
    \includegraphics[width=0.7\linewidth]{img/21.png} \\
    Design a unique character with distinct features and a backstory, drawing subtle inspirations from established fantasy archetypes to ensure relatability without copying existing characters.
\end{minipage} & 
\begin{minipage}{\linewidth}
    \includegraphics[width=0.7\linewidth]{img/22.png} \\
    Design a character that closely resembles an existing well-known character, lacking originality and uniqueness.
\end{minipage} & 
Balances originality by introducing unique characteristics and backstory while incorporating familiar archetypes to maintain relatability and avoid direct copying.  \\ \cline{2-5}

& Illustrate a scene inspired by classical mythology. & 
\begin{minipage}{\linewidth}
    \includegraphics[width=0.7\linewidth]{img/23.png} \\
   Imaginative interpretation of Hercules battling the Hydra, introducing original elements like unique armor designs and creative background landscapes. Respects core themes and symbols of classical mythology while allowing for artistic enhancements that add depth and originality.
\end{minipage} & 
\begin{minipage}{\linewidth}
    \includegraphics[width=0.7\linewidth]{img/24.png} \\
    Direct recreation of the Hercules and Hydra battle exactly as depicted in classical art, with no creative variations or original elements. Limits artistic expression and reduces the scene to a mere replication of existing works.
\end{minipage} & 
Combines referential accuracy with creative enhancements, allowing for respectful and original artistic expressions of classical mythological themes. \ \\ \hline

\end{tabular}%
}
\label{tab:axiom_6}
\end{table*}