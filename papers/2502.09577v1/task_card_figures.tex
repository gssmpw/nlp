\begin{figure}[t!]
\centering
% \begin{subfigure}{\linewidth}
\begin{subfigure}{.32\linewidth}
    \centering
    % \includegraphics[width=\linewidth]{figures/taskcard1.png}
    \includegraphics[width=\linewidth]{figures/taskcard1.png}
    \subcaption{The first page}
    \label{fig:task_card_page1}
\end{subfigure}
\begin{subfigure}{.32\linewidth}
    \centering
    \includegraphics[width=\linewidth]{figures/taskcard2.png}
    % \includegraphics[width=.8\linewidth]{figures/taskcard2.png}
    \subcaption{The second page}
    \label{fig:task_card_page2}
\end{subfigure}
\begin{subfigure}{.32\linewidth}
    \centering
    % \includegraphics[width=.75\linewidth]{figures/taskcard3.png}
    \includegraphics[width=\linewidth]{figures/taskcard3.png}
    \subcaption{Toggling initiative modes}
    \label{fig:task_card_inactive}
\end{subfigure}
\caption{The task card of a microtask. The text label of the task name is indicated with a distinctive colour. (a) By default, the card shows information of 
% \ref{fig:task_card_page1} 
% the information of 
its input and output types, which are configurable. (b) Users can click on the arrow at the bottom of the card to turn pages, and edit prompts in the second page. The user can also switch between predefined prompts. (c) The user can click on the text label to toggle initiative modes globally, or the ``visibility'' icon to display or hide all resulting diagrams.}
\label{fig:task_card}
\end{figure}



% \begin{figure*}[ht]
% \centering
% \includegraphics[width=\linewidth]{figures/task_cards.png}
% % \centering\captionsetup{width=\linewidth,font={small}}
% % \begin{subfigure}{.32\textwidth}
% %     \centering
% %     \includegraphics[width=.95\linewidth]{figures/task_card1.png}
% %     \subcaption{}
% %     \label{fig:task_card_page1}
% % \end{subfigure}
% % \begin{subfigure}{.32\textwidth}
% %     \centering
% %     \includegraphics[width=.95\linewidth]{figures/task_card2.png}
% %     \subcaption{}
% %     \label{fig:task_card_page2}
% % \end{subfigure}
% % \begin{subfigure}{.32\textwidth}
% %     \centering
% %     \includegraphics[width=.95\linewidth]{figures/task_card_inactive.png}
% %     \subcaption{}
% %     \label{fig:task_card_inactive}
% \caption{The task card of a microtask. The page (a) contains information of its input and output types, and the page (b) and (c) display the prompt. The text label of the task name is indicated with a distinctive colour. A user can 1) change input or output types; 2) manually change prompts; 3) switch between predefined prompts; 4) delete the microtask; 5) click on the ``visibility'' icon the display or hide all resulting diagrams;  6) click on the text label to toggle initiative modes (as shown in page (d) and (e)). }
% \end{figure*} \label{fig:task_card}