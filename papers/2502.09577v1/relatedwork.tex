\section{Related Work}
In this section, we briefly review related work on human-AI co-creativity, writing support, and recent large large language models augmented writing tools.

\subsection{From Human-human Co-creativity to Human-AI Co-creativity}
Supporting human co-creativity scenarios such as group brainstorming have been extensively explored in CSCW. Early systems generally used visual or cross-modality stimuli to facilitate ideation in face-to-face brainstorming sessions. For example, Wang et al. proposed IdeaExpander, which was deployed in a group chat to process group conversations, and retrieve related pictures to stimulate ideation~\cite{wang2010idea}. It later proved to produce significantly more ideas than no picture support in a follow-up study~\cite{wang2011diversity}. Shi et al. proposed IdeaWall that continuously presents semantic and visual stimuli by extracting key information from verbal discussions~\cite{shi2017ideawall}. Although no significant enhancement of diversity or quantity of ideas was reported, they found that pictorial support can significantly stimulate group dynamics, which were measured as duration and quantity of lulls during conversations.

Another line of research aims to leverage group dynamics to make brainstorming more productive. Wang et al., for example, probed the cultural differences as a source of conceptual diversity~\cite{wang2011diversity}. Their study found that intercultural brainstorming pairs can cover significantly broader concepts than intracultural pairs given proper visual stimuli. Chan et al. proposed a groupware that augmented crowd innovation with real-time expert guidance~\cite{chan2016improving}. They found that experienced expert facilitators increased both the quantity and creativity of crowd workers' ideas compared to the group without facilitators. Shih et al.~\cite{shih2009groupmind} designed GroupMind that enabled multiple users to contribute in parallel to a digital mind-mapping whiteboard. In a controlled lab study, they found that the parallel-input workflow can produce significantly more unique ideas than serial inputs.

With the advance of AI, especially generative AI models, how they might shape creative activities and influence creative output has become a question of research interest. Wan et al. studied the human-AI co-creativity process during the prewriting stage~\cite{wan2024felt}. They identified three key stages of collaboration with large language models: Ideation, Illumination, and Implementation. Shaer et al. examined the potential of large language models to support collaborative ideation in two phases of creativity: the divergent phase as an idea generator, and the convergent phase as an idea evaluator~\cite{shaer2024ai}. Their findings suggest that GPT-4 can be used to provide unique and expanded perspectives, and evaluate ideas with similar qualities compared to human experts.

In this paper, we contributed a novel workflow to support human-AI co-creativity during prewriting. Inspired by prior work on human-human and human-AI co-creativity workflow, we enabled parallel collaboration with multiple heterogeneous microtasks processed by LLMs, which in our user study proved to afford more user agency, control, and creativity.

\subsection{Writing Support and Prewriting}
Writing support is a long-standing research topic in HCI. For decades a number of systems were designed to support different aspects of writing in varying phases, such as spelling and grammar checking \cite{leacock2010automated}, real-time summarisation \cite{dang2022beyond}, computer-generated feedback \cite{villalon2008glosser}, visualisation of topic flow \cite{o2009visualizing,o2011visualizing}, etc. For example, Dang et al. \cite{dang2022beyond} proposed a text editor that provides auto-updating summaries of paragraphs as margin annotations to help users plan, structure, and reflect on their writing. Iqbal et al. \cite{iqbal2018multitasking} proposed a micro-productivity tool that decomposed the complex task of document editing into smaller and simpler ``microtasks'' to enable users to edit Word documents via mobile devices. These work usually aimed to improve writing efficiency through various computer support.

There is a line of research that particularly focused on collaborative writing, in which the idea of ``microtask'' was constantly used. For example, to facilitate group collaboration, Teevan et al. \cite{teevan2016supporting} decomposed a writing task into three types of microtasks: idea generation, idea labeling and organisation, and writing paragraphs based on given ideas. Each microtask can be completed with limited awareness of the progress of others. Bernstein et al. \cite{bernstein2010soylent} proposed a tool that splits the task of writing and editing into stages and invites crowd workers to make suggestions, shorten, and proofread text.

Few writing tools cover the prewriting process, which often values both creativity and efficiency. Sadauskas et al. \cite{sadauskas2015mining} designed a dedicated system to support prewriting. Their system helps high school students prepare meaningful writing topics by mining their existing content on their social media. Lu et al. \cite{lu2018inkplanner} proposed a pen-and-touch-based tool to support prewriting via intelligent visual diagramming. Their system supports a set of diagram-based features such as the structuring of diagrams, automatic outline generation, etc.

Several creativity support tools (CSTs) were also developed to assist in generating writing ideas, although not explicitly framed as prewriting tools. For instance, Huang et al. \cite{huang2020heteroglossia} designed an add-on for Google Docs that enables writers to gather story ideas from the crowd using a role-play strategy. To support creative writing, Gero \cite{gero2019metaphoria} et al. designed a data-driven interface for writers to brainstorm metaphorical connections to an input word that are coherent with the context.
% \revision{
% To facilitate brainstorming in a human collaboration setting, Shih et al.~\cite{shih2009groupmind} designed a real-time interactive mind mapping tool. In controlled lab experiments, they found that parallel input afforded by the tool can produce significantly more unique ideas.
% }

Our system contributes to HCI with an LLM-enhanced prewriting tool, which borrows the idea of ``microtask'' to support human-AI collaborative diagramming. The system aims to support pre-writing, and features a microtask management workflow to scaffold the collaborative diagramming process.
Many common features of prewriting and writing support (e.g., real-time summary \cite{dang2022beyond}, idea generation \cite{bernstein2010soylent}, argument elaboration \cite{uto2015academic}, etc.) can be achieved using our predefined microtasks, and users can also define their own microtasks to extend the system's capability for various purposes.

\subsection{Large Language Models to Support Writing}
Over the past few years, research in Artificial Intelligence (AI) has led to the development of increasingly powerful large language models (LLMs), including OpenAI's GPT-series \cite{radford2018improving} (Generative Pre-trained Transformer) such as ChatGPT and GPT-4 \cite{openai2023gpt4}. These LLMs demonstrate astonishing capabilities in handling a variety of complex Natural Language Processing (NLP) tasks.

With the development of LLMs, there has been an increasing interest in design goals beyond just efficiency, which has historically been the focus of writing assistance research \cite{jakesch2023co}. As such, there have been various design efforts to explore the potential of LLMs in supporting writing interface and interaction, including story writing \cite{singh2022hide}, text revision \cite{cui2020justcorrect, zhang2019type}, supporting inspiration \cite{inspiration1, lee2022coauthor, singh2022hide}, language abilities \cite{buschek2021impact}, or creative writing \cite{clark2018creative}. For example, Chung et al. proposed TaleBrush \cite{chung2022talebrush,chung2022talebrush_EA}, a system that enables users to co-create stories with LLMs via sketch interaction. Users can sketch how the protagonist's fortune should change chronologically in a story, and TaleBrush will generate story sentences and visualize the resulting fortune in the generation over the original sketch. Clark et al. \cite{clark2018creative} investigated the user experience of writing with language models by conducting case studies involving 46 participants given a slogan writing and a story writing task. Their findings reveal that users find unexpectedness in generations helpful and generally prefer a ``machine-in-the-loop'' mode where the language model plays a supporting role as a proactive writing partner.


% From the Devlin, et al. proposed BERT \cite{bert}, to OpenAI proposed GPT ((Generative Pre-trained Transformer)) \cite{radford2018improving}, and now GPT-4 \cite{openai2023gpt4},
% research in Artificial Intelligence (AI) has significantly scaled up the size of language models, demonstrating astonishing capabilities of handling a range of complex Natural Language Processing (NLP) tasks. With these large language models (LLMs) development, some design efforts have already been made to leverage these LLMs to support writing. For example, Chung et al. \cite{chung2022talebrush,chung2022talebrush_EA} proposed TaleBrush that enables users to co-create stories with LLMs via sketch interaction. Users can sketch how the protagonist's fortune should change chronologically in a story, and TaleBrush will generate story sentences and visualise the resulting fortune in the generation over the original sketch. Clark et al. \cite{clark2018creative} investigated user experience of writing with language models by conducting case studies involving 46 participant given a slogan writing and a story writing task. Their findings reveal that people find unexpectedness in generations helpful, and generally preferred a ``machine-in-the-loop'' mode where the language model plays a supporting role like an active writing partner.

It is worth noting that, randomness and unexpectedness in LLM generations can often be intrusive or cause breakdowns during human-AI collaborative writing \cite{clark2018creative}. Therefore, along this line of research, some literature was particularly dedicated to making LLM generations more controllable and explainable. For example, Wu et al. \cite{wu2022ai,wu2022promptchainer} proposed the idea of chaining LLM operations together. They break up complex writing tasks into smaller steps where the output of one or more steps can be used as input for the next.

Despite the potential benefits of using LLMs to support prewriting, there is still a paucity of research on effectively integrating LLMs into the prewriting process.
Recently, Jiang et al.~\cite{jiang2023graphologue} proposed a novel LLM augmented diagramming interface to decompose complex information in prewriting tasks.
Following their paradigm, we aim to transform the existing turn-taking workflow into a more efficient and creative alternative. \revision{It is similar to other diagrammatic interface such as ChainForge~\cite{arawjo2024chainforge,arawjo2023chainforge} or Sensecape~\cite{suh2023sensecape} that present multiple results simultaneously, but we allow AI agents for distinct purposes to proactively contribute visual diagrams in parallel.
To this end we use the notion of ``microtasking'' in human collaboration, which proves to be effective in our study.}