\newtcbox{\BboxS}{on line,
  colframe=brainstorm,colback=brainstorm,
  boxrule=0.5pt,arc=1pt,boxsep=0pt,left=2pt,right=2pt,top=2pt,bottom=2pt}
\newtcbox{\SboxS}{on line,
  colframe=summarise,colback=summarise,
  boxrule=0.5pt,arc=1pt,boxsep=0pt,left=2pt,right=2pt,top=2pt,bottom=2pt}
\newtcbox{\EboxS}{on line,
  colframe=elaborate,colback=elaborate,
  boxrule=0.5pt,arc=1pt,boxsep=0pt,left=2pt,right=2pt,top=2pt,bottom=2pt}
\newtcbox{\DboxS}{on line,
  colframe=draft,colback=draft,
  boxrule=0.5pt,arc=1pt,boxsep=0pt,left=2pt,right=2pt,top=2pt,bottom=2pt}
\newtcbox{\FboxS}{on line,
  colframe=freewrite,colback=freewrite,
  boxrule=0.5pt,arc=1pt,boxsep=0pt,left=2pt,right=2pt,top=2pt,bottom=2pt}
\newtcbox{\AboxS}{on line,
  colframe=associate,colback=associate,
  boxrule=0.5pt,arc=1pt,boxsep=0pt,left=2pt,right=2pt,top=2pt,bottom=2pt}

\newtcbox{\CboxS}{on line,
colframe=custom,colback=custom,
boxrule=0.5pt,arc=1pt,boxsep=0pt,left=2pt,right=2pt,top=2pt,bottom=2pt}

\newtcbox{\CCboxS}{on line,
colframe=custom2,colback=custom2,
boxrule=0.5pt,arc=1pt,boxsep=0pt,left=2pt,right=2pt,top=2pt,bottom=2pt}
  
\section{Use Case Scenario}
In this section, we illustrate the workflow of \textit{Polymind} via a use case scenario. Suppose that Bob, an HCI researcher, would like to use fictional narratives to promote his research on social media. He therefore chooses to use \textit{Polymind} to plan his fiction writing.

\subsection*{Task Management}
Bob starts with the key insights of his paper, that parallel inputs from AI agents such as LLMs can significantly increase writers' creativity. His primary goals of using \textit{Polymind} are brainstorming narrative lines, and working out a rough outline. Therefore he keeps three microtasks in the proactive mode to quickly expand his ideas: \BboxS{\textcolor{white}{Brainstorm}}, \EboxS{\textcolor{white}{Elaborate}}, and \AboxS{\textcolor{white}{Associate}}; and switched all other microtasks to the reactive mode so that they will not be intrusive.

\subsection*{Collaborative Brainstorming}
To get some initial ideas, Bob uses \textit{Polymind} to perform mind mapping, which focuses on tracking spontaneous and free-form ideas, and their associations.
He first creates some keywords and concepts on the canvas: such as ``parallel collaboration'', ``creative writing'', etc. The three proactive microtasks, \BboxS{\textcolor{white}{Brainstorm}}, \EboxS{\textcolor{white}{Elaborate}}, \AboxS{\textcolor{white}{Associate}} shortly returns some related keywords, example scenarios, and associations between these diagrams.

Bob accepts three diagrams: ``synchronous tasks'', ``mutual goals'', and ``flash fiction''. These remind him of a story where a former fiction writer revolutionizes the fiction writing industry by leveraging multiple robots in an assembly line to mass produce flash fictions. Each robot is configured to handle a distinct microtask on the assembly line, working synchronously towards a common goal.

\subsection*{Task Delegation}
At this point, Bob feels he has obtained some concrete ideas, and would like to organize them via concept mapping.
This method aims to outline structures and relationships between concepts. He also wants to ask the LLM how to make a story more engaging. Therefore, he delegates a new microtask \CboxS{\textcolor{white}{Improve}}, which returns suggestions for improvements.
He leaves it in the \textit{reactive} mode so that it can provide suggestions based on the ever-changing diagram upon request.

\subsection*{Idea Clarification}
Bob starts concept mapping by specifying elements of the story. He creates some diagram nodes, such as ``Character: Fiction Writer \& Robots'', ``Event: Revolutionizing the industry through assembly lines of flash fictions'', etc. He then creates a section over these diagrams to group them together, and uses the microtask \DboxS{\textcolor{white}{Draft}} to request an outline, and clicks the \FboxS{\textcolor{white}{Freewrite}} microtask several times to continue writing to see different endings.

Bob feels that these results are somewhat bland, and therefore asks the microtask \CboxS{\textcolor{white}{Improve}} to suggest improvements based on these diagrams. The results suggest adding to the beginning the conflict between the fiction writer and his former boss that fired him for lacking creativity. It reminds Bob that he could depict the former boss as a firm advocate of turn-taking conversational robots, and unveil the superiority of a parallel collaboration through the main character's adventure.

% \subsubsection*{Task Management}
% Bob starts with the core concept of using LLMs to support prewriting, and he hopes to develop ideas about system design and potential contributions. His primary goals when using \textit{Polymind} are generating new ideas and organizing existing thoughts.
% %He hopes to track his thought process while using the AI to expand on his ideas.
% Therefore, he chooses to keep three microtasks: \BboxS{\textcolor{white}{Brainstorm}}, \EboxS{\textcolor{white}{Elaborate}}, and \AboxS{\textcolor{white}{Associate}} in the proactive mode to augment divergent thinking, and leave the other three microtasks, \FboxS{\textcolor{white}{Freewrite}}, \SboxS{\textcolor{white}{Summarise}}, and \DboxS{\textcolor{white}{Draft}}, in the reactive mode.
% %Therefore, he  running in the proactive mode so that they can constantly contribute to diagramming, and , generating diagram elements only upon requests.

% \subsubsection*{Collaborative Brainstorming}
% To get some initial ideas, Bob uses \textit{Polymind} to perform mind mapping, which focuses on tracking spontaneous and free-form ideas, and their associations.
% He first creates a keyword on the canvas: ``prewriting''. The \BboxS{\textcolor{white}{Brainstorm}} microtask shortly returns several relevant keyword suggestions. Bob accepts two of them, ``Freewriting'' and ``Clustering''.
% %, which he believes are common but under-explored  prewriting strategies in HCI research. 

% Bob realizes that there might be design opportunities to combine these two strategies, while the \AboxS{\textcolor{white}{Associate}} microtask starts to operate on these newly added nodes.
% Bob expands results and sees three suggestions connecting ``Freewriting'' with ``Clustering'': ``Unstructured writing'', ``Visual organization'', and ``Idea generation''.
% Bob accepts ``Unstructured writing'' and ``Visual organization'', which reminds him of visual summaries of semantic information on a freewriting interface. 
% He then creates a concept ``Hierarchical clustering of freewriting'' to ``Hierarchical freewriting'' and connects it to ``Unstructured writing'' and ``Visual organization''.

% \subsubsection*{Task Delegation}
% At this point, Bob feels he has obtained some concrete ideas, and would like to organize them by concept mapping.
% This method aims to outline structures and relationships between concepts. He also wants to leverage the LLM to evaluate the feasibility of his research ideas. Therefore, he delegates a new microtask \CboxS{\textcolor{white}{Evaluate}}, which returns potential weaknesses for individual ideas. 
% %He leaves it in the reactive mode so that it can provide suggestions based on the ever-changing diagram upon requests.

% \subsubsection*{Idea Clarification}
% Bob starts concept mapping by creating a diagram node ``semantic visualization'', and connects it with ``visual organization'', as he thinks of the feature of semantic visual summary. The \EboxS{\textcolor{white}{Elaborate}} microtask returns three examples of ``semantic visualization'', ``word clouds'', ``tree maps'', and ``network diagrams''. Bob accepts them all as he believes multiple visualization techniques can be used to address different levels of details. He also creates a keyword ``three LoD'' (level of details) and connects it to the three examples.
% Bob moves to evaluate the novelty and significance of his ideas. He creates a sticky note on the canvas ``Contributions of a freewriting interface to HCI that supports visual organization of semantic information'', and asks the \FboxS{\textcolor{white}{Freewrite}} microtask to continue. The microtask returns a paragraph of text suggesting several potential strengths of such systems.
% %including facilitating group collaboration and communication, enhancing reflection of ideas, allowing multiple users to build upon or contribute to one user’s ideas, etc.
% Finally, Bob creates a section over these nodes named ``freewriting interface that supports semantic visualization'', and invokes the \CboxS{\textcolor{white}{Evaluate}} microtask.
% The result reminds him that such a system could overwhelm users cognitively.

\begin{figure}[ht]
% \centering\captionsetup{width=\linewidth,font={small}}
\includegraphics[width=.85\linewidth]{figures/UI_cropped.png}
\caption{The interface of \textit{Polymind} comprises: A. a diagramming canvas B. a toolbar C. a task board}
\label{fig:UI}
\end{figure}