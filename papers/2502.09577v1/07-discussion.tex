\section{Discussion}
In this section, we discuss insights of our design and study, and reflect on potential future work.
\subsection{Revisiting AI Initiative}
Previous literature suggests that AI can be frustrating and cause breakdowns and distrust if it fails to understand users' intentions while taking the initiative \cite{buschek2018researchime,clark2018creative,oh2018lead}. Our evaluation further reveals that the preferences of initiative modes are most likely task-dependent.
%To address this, a perfect inference of user intentions and quality of generations in various complex tasks would be required, or contrarily, a comprehensive understanding of an AI's limitations.
Considering these challenges, we suggest that a generative AI should be more transparent about its task-specific limitations \cite{smith2022real}, and dynamically adjust its level of initiative accordingly. A generative AI could proactively provide suggestions on tasks it is good at, but it should play mostly a supporting role on tasks where it might be unreliable.

Furthermore, we believe that initiative modes of future generative AI design should go beyond a simple on/off switch. Task-specific capabilities of AI can vary greatly across a continuous spectrum; therefore, the control of AI initiative should also be at least partly continuous. For example, while an AI is taking the initiative and proactively providing suggestions, it can signal uncertainty and make suggestions more subtle and non-intrusive when the user switches to a task it is not familiar with or good at, instead of being completely ``turned off''.
We might also consider designing features to support users in managing initiative levels of AI instead of fully automating it.

Additionally, while supporting users through microtasking, or with multiple functions or roles, the design of AI initiative should also take into account the frequency of use (as suggested by S6), the intrusiveness of potential generations (as suggested by S8), and users' possible processing order of microtasks (as suggested by S10), based on the nature of specific tasks.
Both our formative study and system evaluation also suggest that users are more likely to let AI take control in divergent thinking tasks such as brainstorming. Therefore, it might be reasonable to apply different initiative modes for divergent and convergent thinking scenarios: AI-dominant mode for divergent thinking, and user-dominant mode for convergent thinking. The corresponding microtask management features should also be adapted in accordance with the human-AI dynamics.

\subsection{Designing Awareness Features} \label{designing-awareness-features}
Our study unveils the challenge of balancing awareness features of AI (or multiple AI functions or roles) and the overall non-intrusiveness of the collaboration workflow. This is especially challenging on a diagramming canvas because an operation might occur on an arbitrary node, thus calling for users' attention from arbitrary locations. In general, our participants perceived our awareness features to be non-intrusive, but it is worth pointing out that this perception might be biased due to participants finding microtask suggestions to be overall helpful and relevant. Our findings also suggest that users might not always be satisfied with a minimal preview or notification as awareness information:
some found it could be hard to notice, but enlarging it might be distracting.

To this end, we believe a future study is needed to investigate the specific design requirements of awareness features for generative AI results on other diagram-based (or even canvas-based in general) interface beyond prewriting (e.g., sketch-based diagramming \cite{zeleznik2008lineogrammer,kara2004hierarchical}, visual programming \cite{li2020supporting}, etc.). It is also worth studying how attention and interruption should be balanced on such interfaces~\cite{gluck2007matching}. Furthermore, we believe awareness features should go beyond animations of discrete levels of attention-drawing strength. It is reasonable to assume that suggestions with higher quality deserve more attention.
\revision{Our user study suggests that perceptions of result quality can have a significant impact on perceived mental demand. For creative tasks such as prewriting, ``quality'' implies both originality and relevance. How to take into account these contextual factors in awareness design poses yet another significant question.}

\subsection{Diagramming and Prewriting}
Our evaluation shows that a microtasking workflow on a diagramming canvas can efficiently provide creative writing ideas in an organised and non-interrupting manner. This aligns with previous results about parallel input workflows in human collaboration scenarios, which found them to be more creative than serial interactions~\cite{shih2009groupmind,hymes1992unblocking}. Microtasking complements Jiang et al.~\cite{jiang2023graphologue}'s diagrammatic workflow by facilitating parallel collaboration to enhance creativity. The implementation of this workflow, including awareness features, initiative modes, etc., could set a paradigm for future endeavor to incorporate LLMs into diagramming interfaces for prewriting.
%However, it is worth noting that such workflow might as well be distracting or interrupting in certain cases, compared to serial, or serial-like interactions should there be one devised in the future.

Furthermore, the choices of predefined microtasks in \textit{Polymind}'s workflow represented a range of prewriting needs, which covered both convergent and divergent modes of thinking in a creativity process. Findings of our study suggests that, \textit{Polymind} could not only provide more ideas, but also effectively organise or synthesize existing information on the canvas for further iterations. Users were able to quickly generate a first draft by sectioning over a set of diagrams, or request new ideas based on a summary of existing diagrams. This finding might inspire future design to transform prewriting workflows, including but not limited to visual diagramming. Although prewriting by nature is iterative not linear~\cite{rohman1965pre}, most previous prewriting tools divides the prewriting process into distinct linear phases: Lu et al.'s Inkplanner roughly has three phases: diagramming, structuring, and outlining~\cite{lu2018inkplanner}; Sadauskas's design covers three phases: content aggregation, content re-experience, and narrative development~\cite{sadauskas2015mining}. As a result, these tools might not accommodate intermediate results well, especially those generated in iterations while collaborating with LLMs. Additionally, as suggested by previous studies of diagrammatic prewriting~\cite{davies2011concept,lorenz2009using}, efficient information synthesis is key to fruitful results. Future design can further consider to support organising and digesting information of complex diagram information in an iterative prewriting workflow.

\subsection{Generalizability and Future Work}
Although the design of \textit{Polymind} was motivated by parallel thinking strategies for creativity and prewriting, we expect that the microtasking workflow proposed can generalise to other diagramming interface, or other scenarios beyond prewriting or writing to enable parallel collaboration with generative AI models. \revision{As our evaluation suggests, the microtasking workflow affords multiple small, distinct, and manageable interactions at the same time for enhanced efficiency and creativity, but requires significant efforts of configuring, and compromises in-depth serial interactions.}
To apply such workflows to other human-AI collaboration tasks, the task management features with respect to our \textit{Goal 2} need to be adapted to the new context, which entails the redesign of input, output, and prompts specification, and features to support progress tracking, feedback and awareness.

For a writing task, for example, we anticipate that a microtasking interface will enhance the overall writing efficiency, quality, and creativity. Future work can adapt NLP models to accommodate writing tasks at differnt structural levels, such as sentence-level proofreading, paragraph-level summary, document-level ideation, etc. Task management features can be designed based on traditional group writing scenarios (e.g., Google Docs) \cite{birnholtz2012tracking,birnholtz2013write}. Users can delegate their own customized microtasks, or configure task specifications, including input or output types, initiative modes, prompts, etc.

For other scenarios such as drawing or design prototyping, we need to take into account the affordances of AI models across different modalities, as well as the nature of the task or interface.
For instance, on a drawing canvas, we can use multiple AI models for different purposes, such as sketch completion \cite{shi2020emog}, colorization \cite{yan2022flatmagic}, style transfer \cite{wu2023styleme}, etc. Each AI model might have unique input and output types, so the progress visualization, initiative modes, and awareness features, such as previews or notifications, should be properly redesigned on the drawing canvas.

\subsection{Limitations}
Despite promising results of our user study, our system has several limitations that should be taken into account.
First, the predefined microtasks we utilized were primarily based on existing literature on writing and creativity and may not cover all possible real-life prewriting scenarios, although users can define their own microtasks as needed. Second, our user evaluation was conducted as a controlled session of around 1 hour, which may not fully reflect the challenges and benefits of using our system in actual prewriting settings. %We did observe some instances of breakdown during the evaluation due to users' unfamiliarity with our system.
\revision{Third, due to time constraints and the complexity of our evaluation task, the structure of the resulting diagrams was simple. 
While our participants did not report cognitive overload, future work could investigate users' cognitive load in more involved prewriting scenarios that would produce more complex diagrams.
To deploy our prototype in real-life scenarios, we should also consider using auto-layout algorithms to help semi-automatically arrange diagrams on the canvas.
}
