% \begin{figure}[ht]
% \centering
% \begin{subfigure}{.45\linewidth}
%     \centering
%     \includegraphics[width=.85\linewidth]{figures/keyword.png}
%     \label{fig:keyword}
% \end{subfigure}
% \begin{subfigure}{.45\linewidth}
%     \centering
%     \includegraphics[width=.85\linewidth]{figures/concept.png}
%     \label{fig:concept}
% \end{subfigure}
% \begin{subfigure}{.45\linewidth}
%     \centering
%     \includegraphics[width=.9\linewidth]{figures/sticky_note.png}
%     \label{fig:sticky_note}
% \end{subfigure}
% \begin{subfigure}{.45\linewidth}
%     \centering
%     \includegraphics[width=\linewidth]{figures/section.png}
%     \label{fig:section}
% \end{subfigure}
% \caption{Three basic diagrams (or nodes) in \textit{Polymind}, and the section, which stores multiple diagrams and their connections.}
% \end{figure} \label{fig:diagrams}

\begin{figure}[ht]
\centering
\includegraphics[width=\linewidth]{figures/UIComponents.png}
\caption{\textit{Polymind} supports three basic diagrams (or nodes), and allows users to draw a section over diagrams. The task board interface supports microtask management, and maps microtask input and output to different diagram types on the canvas. The results of microtasks are displayed as notifications and previews on the task header before being expanded and accepted.}
\label{fig:diagrams}
\end{figure}