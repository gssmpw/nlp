%%
%% This is file `sample-manuscript.tex',
%% generated with the docstrip utility.
%%
%% The original source files were:
%%
%% samples.dtx  (with options: `manuscript')
%% 
%% IMPORTANT NOTICE:
%% 
%% For the copyright see the source file.
%% 
%% Any modified versions of this file must be renamed
%% with new filenames distinct from sample-manuscript.tex.
%% 
%% For distribution of the original source see the terms
%% for copying and modification in the file samples.dtx.
%% 
%% This generated file may be distributed as long as the
%% original source files, as listed above, are part of the
%% same distribution. (The sources need not necessarily be
%% in the same archive or directory.)
%%
%%
%% Commands for TeXCount
%TC:macro \cite [option:text,text]
%TC:macro \citep [option:text,text]
%TC:macro \citet [option:text,text]
%TC:envir table 0 1
%TC:envir table* 0 1
%TC:envir tabular [ignore] word
%TC:envir displaymath 0 word
%TC:envir math 0 word
%TC:envir comment 0 0
%%
%%
%% The first command in your LaTeX source must be the \documentclass command.
% \documentclass[manuscript,review,anonymous]{acmart}


\documentclass[acmsmall]{acmart}

% \documentclass[acmsmall,review,anonymous]{acmart}
% \documentclass[sigconf]{acmart}

% \usepackage[bookmarks=false]{hyperref}

% \usepackage{hyperref}
% \usepackage{hyperxmp}
\usepackage{svg}
\usepackage{makecell}
\usepackage{graphicx}
\usepackage{subcaption}
\usepackage{enumitem}

\usepackage[frozencache,cachedir=minted-cache]{minted} 

\usepackage{xcolor}
\usepackage{multirow}
\usepackage[most]{tcolorbox}

\definecolor{brainstorm}{HTML}{FFB347}
\definecolor{summarise}{HTML}{966FD6}
\definecolor{elaborate}{HTML}{71C562}
\definecolor{draft}{HTML}{FB6B89}
\definecolor{freewrite}{HTML}{B39EB5}
\definecolor{associate}{HTML}{D64A4A}
\definecolor{custom}{HTML}{B7AFA3}
\definecolor{custom2}{HTML}{E8D0A9}

\definecolor{keyword}{HTML}{CED8DF}
% \definecolor{keyword}{HTML}{E9EDF3}
\definecolor{concept}{HTML}{FFB5B7}
\definecolor{sticky_note}{HTML}{748B97}
\definecolor{section}{HTML}{0099FF}
%% Custom command for revised content
%% Change 'red' to 'black' for the clean version
\newcommand{\revision}[1]{{\color{black}#1}}

\newcommand{\RNum}[1]{\uppercase\expandafter{\romannumeral #1\relax}}
%%
%% \BibTeX command to typeset BibTeX logo in the docs
\AtBeginDocument{%
  \providecommand\BibTeX{{%
    Bib\TeX}}}

%% Rights management information.  This information is sent to you
%% when you complete the rights form.  These commands have SAMPLE
%% values in them; it is your responsibility as an author to replace
%% the commands and values with those provided to you when you
%% complete the rights form.
\setcopyright{acmcopyright}
\copyrightyear{2024}
\acmYear{2024}
% \acmDOI{XXXXXXX.XXXXXXX}

%% These commands are for a PROCEEDINGS abstract or paper.
% \acmConference[Conference acronym '24]{}{June 03--05, 2024}{Woodstock, NY}
%\acmPrice{15.00}
%\acmISBN{978-1-4503-XXXX-X/18/06}

% \acmConference[CHI '24]{Proceedings of the 2024
% CHI Conference on Human Factors in Computing Systems}{May 11--16, 2024}{Hawai'i, USA}

% \acmBooktitle{Proceedings of the 2024 CHI Conference on Human Factors in
% Computing Systems (CHI '24), May 11--16, 2024, Hawai'i, USA} 
% \acmPrice{15.00}


%%
%% Submission ID.
%% Use this when submitting an article to a sponsored event. You'll
%% receive a unique submission ID from the organizers
%% of the event, and this ID should be used as the parameter to this command.
%%\acmSubmissionID{123-A56-BU3}

%%
%% For managing citations, it is recommended to use bibliography
%% files in BibTeX format.
%%
%% You can then either use BibTeX with the ACM-Reference-Format style,
%% or BibLaTeX with the acmnumeric or acmauthoryear sytles, that include
%% support for advanced citation of software artefact from the
%% biblatex-software package, also separately available on CTAN.
%%
%% Look at the sample-*-biblatex.tex files for templates showcasing
%% the biblatex styles.
%%

%%
%% The majority of ACM publications use numbered citations and
%% references.  The command \citestyle{authoryear} switches to the
%% "author year" style.
%%
%% If you are preparing content for an event
%% sponsored by ACM SIGGRAPH, you must use the "author year" style of
%% citations and references.
%% Uncommenting
%% the next command will enable that style.
%%\citestyle{acmauthoryear}

%%
%% end of the preamble, start of the body of the document source.
\begin{document}

%%
%% The "title" command has an optional parameter,
%% allowing the author to define a "short title" to be used in page headers.
\title[Polymind]{Polymind: Parallel Visual Diagramming with Large Language Models to Support Prewriting Through Microtasks}

%%
%% The "author" command and its associated commands are used to define
%% the authors and their affiliations.
%% Of note is the shared affiliation of the first two authors, and the
%% "authornote" and "authornotemark" commands
%% used to denote shared contribution to the research.
\author{Qian Wan}
%\authornote{Both authors contributed equally to this research.}
\orcid{0000-0002-4250-8780}
%\author{G.K.M. Tobin}
%\authornotemark[1]
%\email{webmaster@marysville-ohio.com}
\affiliation{%
  \institution{City University of Hong Kong}
  \streetaddress{83 Tat Chee Avenue}
  \city{Hong Kong}
  %\state{Ohio}
  \country{Hong Kong}
  %\postcode{43017-6221}
}
\email{qianwan3-c@my.cityu.edu.hk}

\author{Jiannan Li}
\affiliation{%
  \institution{Singapore Management University}
  % \streetaddress{83 Tat Chee Avenue}
  \city{Singapore}
  \country{Singapore}
  }
\email{jiannanli@smu.edu.sg}

\author{Huanchen Wang}
\orcid{0000-0001-9339-1941}
\affiliation{
  \department{Department of Computer Science}
  \institution{City University of Hong Kong}
  \city{Hong Kong}
  \country{China}
}

\affiliation{
  \department{Department of Computer Science and Engineering}
  \institution{Southern University of Science and Technology}
  \city{Shenzhen}
  \state{Guangdong}
  \country{China}
}
\email{hc.wang@my.cityu.edu.hk}

\author{Zhicong Lu}
\affiliation{%
  \institution{George Mason University}
  % \streetaddress{83 Tat Chee Avenue}
  \city{Fairfax}
  % \state{Virginia}
  \country{USA}
  %\country{Iceland}
}
\email{zhiconlu@cityu.edu.hk}


%%
%% By default, the full list of authors will be used in the page
%% headers. Often, this list is too long, and will overlap
%% other information printed in the page headers. This command allows
%% the author to define a more concise list
%% of authors' names for this purpose.
\renewcommand{\shortauthors}{Wan et al.}

%%
%% The abstract is a short summary of the work to be presented in the
%% article.

\begin{abstract}
Retrieval-Augmented Generation (RAG) is often used with Large Language Models (LLMs) to infuse domain knowledge or user-specific information. In RAG, given a user query, a retriever extracts chunks of relevant text from a knowledge base. These chunks are sent to an LLM as part of the input prompt. Typically, any given chunk is repeatedly retrieved across user questions. However, currently, for every question, attention-layers in LLMs fully compute the key values (KVs) repeatedly for the input chunks, as state-of-the-art methods cannot reuse KV-caches when chunks appear at arbitrary locations with arbitrary contexts. Naive reuse leads to output quality degradation.  This leads to potentially redundant computations on expensive GPUs and increases latency. In this work, we propose \sys, a system for managing and reusing precomputed KVs corresponding to the text chunks (we call \textit{chunk-caches}) in RAG-based systems. We present how to identify \hl{\textit{chunk-caches} that are reusable}, how to efficiently perform a small fraction of recomputation to \textit{fix} the cache to maintain output quality, and how to efficiently store and evict \textit{chunk-caches} in the hardware for maximizing reuse while masking any overheads. With real production workloads as well as synthetic datasets, we show that \sys reduces redundant computation by \textbf{51\%} over SOTA prefix-caching and \textbf{75\%} over full recomputation.
\hl{Additionally, with continuous batching on a real production workload, we get a \textbf{1.6$\times$} speedup in throughput and a \textbf{2$\times$} reduction in end-to-end response latency over prefix-caching while maintaining quality, for both the \llama-3-8B and \llama-3-70B models. 
}
\end{abstract}






%%
%% The code below is generated by the tool at http://dl.acm.org/ccs.cfm.
%% Please copy and paste the code instead of the example below.
%%
\begin{CCSXML}
<ccs2012>
   <concept>
       <concept_id>10003120.10003121.10003129</concept_id>
       <concept_desc>Human-centered computing~Interactive systems and tools</concept_desc>
       <concept_significance>500</concept_significance>
       </concept>
 </ccs2012>
\end{CCSXML}

\ccsdesc[500]{Human-centered computing~Interactive systems and tools}

%%
%% Keywords. The author(s) should pick words that accurately describe
%% the work being presented. Separate the keywords with commas.
\keywords{Prewriting, Diagramming, Creativity Support, Microtasking, Human-AI Collaboration}

%%
%% This command processes the author and affiliation and title
%% information and builds the first part of the formatted document.

\begin{teaserfigure}
    \centering
    \includegraphics[width=\linewidth]{figures/teaser_new_v6.png}
    \caption{The microtasking workflow of \textit{Polymind}: A) A user can delegate new microtasks, and configure microtasks by specifying input \& output types, prompts, initiative modes, etc. B) An active microtask notifies users when results are ready and provide previews on the canvas. C) Once expanded, resulting diagrams are displayed using a hollow shape in contrast to user created diagrams, and distinctive border colours indicating which microtasks they are from}
    \label{fig:teaser}
\end{teaserfigure}

\maketitle


The increasing reliance on LLMs for multimodal tasks across far-reaching sectors such as healthcare, finance, and manufacturing underscores the need to assess the accuracy and reliability of the information they generate. Vision-Language Models (VLM) have achieved state-of-the-art (SoTA) performance on Visual Question-Answering (VQA) benchmarks, and these models often utilize Retrieval-Augmented Generation (RAG) to maintain factual accuracy and relevance in a dynamic information environment. However, this has led to uncertainty in the information the LLM bases its answer on, as it may choose between parametric memory and retrieved sources. When models rely on memorized information instead of dynamically retrieving information, they may inadvertently propagate outdated or incorrect information, causing serious legal and ethical risks and undermining trust and reliability in AI systems \citep{huang2023survey}.
% The ability to strike a balance between generalization and specialization in AI systems is therefore crucial for ensuring the safe, reliable use of these technologies in real-world applications.

Despite these concerns, the way that Vision-Language models (VLMs) memorize and retrieve information, particularly in complex multimodal tasks, remains under-explored. Current research often focuses on either the general capabilities of large language models (LLMs) or the specialized retrieval mechanisms in retrieval augmented generation systems (RAG) \citep{incontext_rag,chen_murag_2022,liu_universal_2023}. Particularly in the context of multimodal retrieval and multihop reasoning, few studies analyze the tradeoff between finetuning for specialized tasks and zero-shot prompting for general-purpose vision-language capabilities. A lack of consensus on how to approach this tradeoff motivates the development of measures to quantify reliance on parametric memory, as well as metrics for quantifying the potential performance impact of extending LLMs with RAG systems.

To address this gap, we investigate how multimodal QA models balance accuracy with memorization on the WebQA benchmark. We compare finetuned multimodal systems against zero-shot VLMs, analyzing how retrieval performance influences QA accuracy. In particular, we focus on cases where retrieval fails, allowing us to measure reliance on parametric memory through two proposed metrics---the \ppr (\PPR) which quantifies how much model accuracy is influenced by retrieval quality, contrasting performance in best-case versus worst-case retrieval scenarios, and the \ucr (\UCR) which measures how often correct QA responses are generated when the retriever fails, providing a proxy for memorization.

To enable this analysis, we make several methodological contributions. For the finetuned QA models, we investigate Vision-Transformer (ViT) architectures, which allow for multihop reasoning over multiple sources. To investigate the impact of retrieval performance on trained LMs, we propose a variable-input Fusion-in-Decoder (FiD) model \cite{tanaka_slidevqa_2023, nlvr2}, building upon the VoLTA architecture \citep{pramanick_volta_2023}. For the zero-shot case, we build upon previous research on In-Context Retrieval \citep{incontext_rag} by demonstrating that LLMs such as GPT-4o are capable of performing the final ranking step of the retrieval process. In doing so, we find that GPT-4o, a general-purpose LLM, achieves SoTA performance on the WebQA task, outperforming existing finetuned RAG models by a significant margin (7\% higher accuracy). 

Crucially, our results reveal that while retrieval-augmented models reduce memorization, the training paradigm plays an important role. Finetuned models exhibit higher reliance on parametric memory, whereas zero-shot RAG approaches have lower memorization scores at the cost of accuracy. This suggests that while retrieval modules may mitigate the risks associated with outdated or incorrect information, SoTA performance requires that they be coupled with specialized QA models. Our memorization measures contribute to the development of transparent and reliable AI systems, particularly in applications where the sourcing of up-to-date, factual information is critical.



% We investigate the impact of question complexity on the ability of these models to integrate multiple data sources—such as images, text, and external retrievers—and produce coherent and accurate answers. We also explore whether in-context retrieval can be a viable alternative to traditional retrieval-augmented systems, offering a more streamlined approach to multimodal QA.

% To achieve this, we first compare zero-shot prompting multimodal LLMs with finetuned multimodal systems. We evaluate both types of models on the WebQA benchmark, a dataset designed for complex question answering that requires reasoning across both image and text sources. For the finetuned models, we use a Fusion-in-Decoder (FiD) architecture, which allows for multihop reasoning over multiple sources. Additionally, we introduce the concept of In-Context Retrieval Language Modeling (RLM), where the LLM itself performs retrieval tasks without the need for external retrievers. This method builds upon existing research in in-context learning  and aims to explore the viability of LLMs retrieving relevant sources and generating accurate answers directly from their context window.

% In order to investigate source utilization in finetuned multimodal models and LLMs, three lines of inquiry are established; 
% \begin{itemize}
%     \item Study 1: retrieval vs QA performance on webQA (motivating example, does QA answer correctly even with incorrect sources?)
%     \item Study 2: performance on adversarial examples where parametric knowledge would be incorrect by design
%     \item Study 3: improving performance on adversarial examples by fine-tuning (i.e model robustness)
% \end{itemize}

% Note, there is one weakness in this plan which is tying in the work we've already done. 
% If we added something from adversarial generation to the retrieval experiment (like a combination of study 1 + 3) it would be complete. So for instance we could try fine-tuning the retriever with adversarial examples (and not just the QA model)

% \begin{figure}
%     \centering
%     \includegraphics[width=0.95\linewidth]{figures/segmentation/webqa_segment_infill.png}
%     \caption{Example of the segmentation substitution pipeline from the WebQA task.}
%     % d5c76d760dba11ecb1e81171463288e9
%     \label{fig:seg_sub_pipeline}
% \end{figure}



% Retrieval augmented generation (RAG) with zero-shot prompting and fine-tuning Large Language Models (LLMs) have become the go-to methods for tasks relying on information retrieval and text generation. In many cases the LLMs parametric memory can sufficiently generalize to answer questions without being provided with retrieval mechanisms for out-of-domain knowledge. However, LLMs often hallucinate and provide wrong information in certain scenarios. This problem is amplified even further on open-domain Question Answering (QA) tasks involving multiple modalities. Grounded text generation using retrieved sources \citep{lewis2021retrievalaugmented} has been extensively studied for text-to-text QA tasks, but its application in multimodal settings has not been studied as much.


% Multimodal reasoning and question answering have gained prominence in recent research endeavors, with an increasing emphasis on handling various forms of data, particularly text and images. In this study, we address a specific gap in the existing literature by focusing on the development of a versatile multihop model capable of accommodating varying numbers of input images.

% Our motivation for this research lies in the growing complexity of answering questions using information on the web, where the challenge of navigating the open-domain setting is further complicated by the presence of multiple modalities and sometimes requires reasoning over multiple sources. WebQA is an ideal dataset on which to compare performance of finetuned RAG systems against general purpose LLMs; it is multimodal, with correct answers requiring reasoning over image and text sources. It is multihop, requiring a complex reasoning process over multiple sources. Finally, WebQA questions from different categories can be broken down into subdomains to analyze performance over domains of varying cardinality.

% Motivated by the real-world challenges of building retrieval and question answering (QA) systems, we design and finetune a closed domain, multimodal, multihop QA model, that is capable of reasoning over a varying number of sources taken as input from an external retriever module. This research contributes to the relatively underexplored domain of multihop reasoning across various input sources and modalities. Our goal is to explore the challenges posed by these scenarios and develop strategies that enable QA models to retrieve relevant information, conduct logical or numerical reasoning across diverse modalities, and generate coherent responses in natural language. To our knowledge, this is the first application of the Fusion-in-Decoder (FiD) architecture \cite{tanaka_slidevqa_2023, nlvr2} that is shown to work with a variable number of inputs, enabling multi-hop reasoning over sources.

% In-Context Learning refers to the ability of LLMs to perform any task by simply providing examples in the input prompt \citep{dong2022survey,min2022rethinking}. Inspired by this research, we propose a method to use the LLM itself as a multimodal retriever, potentially eschewing the requirement of a distinct retrieval module, thereby allowing the design of simpler retrieval-augmented QA systems. We dub this method In-Context Retrieval Language Modeling (RLM). To the best of the authors knowledge, In-Content RLM is disparate from other retrieval augmented approaches which utilize external retrieval modules \citep{incontext_rag,chen_murag_2022,liu_universal_2023}. Despite being a natural extension of In-Context learning, In-Context RLM has not yet been studied empirically.

% To expand on our contribution of In-Context Retrieval, this stems from the well-researched in-context learning of LLMs. In-context learning is the ability of a model to perform any task given a sufficient context window \citep{dong2022survey,min2022rethinking}. Such tasks could include retrieval and ranking, but typically, the go-to solution for tasks requiring retrieval has been RAG. To the best of the authors knowledge, In-Context Retrieval is distinct from In-Context Retrieval Augmented Language Modelling (RALM), and despite being a natural extension of In-Context learning, In-Context Retrieval has not yet been shown empirically.

% Finally, we explore the tradeoff between using zero-shot prompting LLMs and the fine-tuning approach. While we find that, overall, GPT-4o obtains SoTA performance on the WebQA task, outperforming the accuracy of existing finetuned RAG approaches by 7\%, finetuned approaches still perform better on more restricted subdomains\footnote{``In-Context RLM" @ \url{https://eval.ai/web/challenges/challenge-page/1255/leaderboard/3168}}. Finally, we validate that GPT-4o is relying on retrieval abilities to solve the task; we find that GPT-4o is capable of retrieving relevant sources in the presence of distractors and furthermore, when GPT-4o fails to retrieve correct sources, it answers incorrectly 75\% of the time, meaning that it is not relying on parametric memory for this task.

% \paragraph{Contributions}
% Based on our experimentation and analysis on the WebQA benchmark, we make the following contributions:
% \begin{itemize}
%     \item Propose a new architecture for multimodal multihop QA that takes variable number of input sources inspired by the Fusion-in-Decoder method.
%     \item Comparison of general purpose LLMs vs specialized models on the WebQA benchmark.
%     \item Observation of In-Context Multimodal Retrieval abilities of GPT-4o and that it does not rely on parametric memory for multimodal QA.
%     \item Analysis of relationship between retrieval and QA task performance.
%     \item Analysis of task and query complexity on the performance of retrieval and QA tasks.
% \end{itemize}
















% Throughout this paper, we will present our methodology, experiments, and findings, emphasizing our approach to multihop reasoning over varying numbers of input images. We believe that our work contributes to a deeper understanding of multimodal reasoning and has the potential to enhance the capabilities of question-answering systems in the intricate, multimodal landscape of web-based information.
\section{Rethinking Sparse Attention Methods}
\label{sec:critique}

Modern sparse attention methods have made significant strides in reducing the theoretical computational complexity of transformer models. However, most approaches predominantly apply sparsity during inference while retaining a pretrained Full Attention backbone, potentially introducing architectural bias that limits their ability to fully exploit sparse attention's advantages. Before introducing our native sparse architecture, we systematically analyze these limitations through two critical lenses.


\begin{figure*}[t] 
\centering 
\includegraphics[width=1\textwidth]{figures/fig2.pdf} 
\caption{Overview of \method{}'s architecture. Left: The framework processes input sequences through three parallel attention branches: For a given query, preceding keys and values are processed into compressed attention for coarse-grained patterns, selected attention for important token blocks, and sliding attention for local context. Right: Visualization of different attention patterns produced by each branch. Green areas indicate regions where attention scores need to be computed, while white areas represent regions that can be skipped.}
\label{fig:framework}
\end{figure*}


\subsection{The Illusion of Efficient Inference}

Despite achieving sparsity in attention computation, many methods fail to achieve corresponding reductions in inference latency, primarily due to two challenges:

\textbf{Phase-Restricted Sparsity.}
Methods such as H2O \citep{h2o} apply sparsity during autoregressive decoding while requiring computationally intensive pre-processing (e.g. attention map calculation, index building) during prefilling. In contrast, approaches like MInference \citep{minference} focus solely on prefilling sparsity. 
These methods fail to achieve acceleration across all inference stages, as at least one phase remains computational costs comparable to Full Attention.
The phase specialization reduces the speedup ability of these methods in prefilling-dominated workloads like book summarization and code completion, or decoding-dominated workloads like long chain-of-thought~\citep{cot} reasoning.

\textbf{Incompatibility with Advanced Attention Architecture.}
Some sparse attention methods fail to adapt to modern decoding efficient architectures like Mulitiple-Query Attention~(MQA) \citep{mqa} and Grouped-Query Attention~(GQA) \citep{gqa}, which significantly reduced the memory access bottleneck during decoding by sharing KV across multiple query heads. For instance, in approaches like Quest \citep{quest}, each attention head independently selects its KV-cache subset. Although it demonstrates consistent computation sparsity and memory access sparsity in Multi-Head Attention (MHA) models, it presents a different scenario in models based on architectures like GQA, where the memory access volume of KV-cache corresponds to the union of selections from all query heads within the same GQA group. This architectural characteristic means that while these methods can reduce computation operations, the required KV-cache memory access remains relatively high.
This limitation forces a critical choice: while some sparse attention methods reduce computation, their scattered memory access pattern conflicts with efficient memory access design from advanced architectures.

These limitations arise because many existing sparse attention methods focus on KV-cache reduction or theoretical computation reduction, but struggle to achieve significant latency reduction in advanced frameworks or backends.
This motivates us to develop algorithms that combine both advanced architectural and hardware-efficient implementation to fully leverage sparsity for improving model efficiency.


\subsection{The Myth of Trainable Sparsity}
Our pursuit of native trainable sparse attention is motivated by two key insights from analyzing inference-only approaches:
(1) \textbf{\textit{Performance Degradation}}: Applying sparsity post-hoc forces models to deviate from their pretrained optimization trajectory. As demonstrated by \citet{magicpig}, top 20\% attention can only cover 70\% of the total attention scores, rendering structures like retrieval heads in pretrained models vulnerable to pruning during inference.
(2)~\textbf{\textit{Training Efficiency Demands}}: 
Efficient handling of  long-sequence training is crucial for modern LLM development. This includes both pretraining on longer documents to enhance model capacity, and subsequent adaptation phases such as long-context fine-tuning and reinforcement learning. However, existing sparse attention methods primarily target inference, leaving the computational challenges in training largely unaddressed. This limitation hinders the development of more capable long-context models through efficient training. Additionally, efforts to adapt existing sparse attention for training also expose challenges:



\textbf{Non-Trainable Components.} Discrete operations in methods like ClusterKV~\citep{clusterkv} 
(includes k-means clustering) and MagicPIG~\citep{magicpig} (includes SimHash-based selecting) create discontinuities in the computational graph. These non-trainable components prevent gradient flow through the token selection process, limiting the model's ability to learn optimal sparse patterns. 

\textbf{Inefficient Back-propagation.} Some theoretically trainable sparse attention methods suffer from practical training inefficiencies. Token-granular selection strategy used in approaches like HashAttention~\citep{desai2024hashattention} leads to the need to load a large number of individual tokens from the KV cache during attention computation. 
This non-contiguous memory access prevents efficient adaptation of fast attention techniques like FlashAttention, which rely on contiguous memory access and blockwise computation to achieve high throughput.
As a result, implementations are forced to fall back to low hardware utilization, significantly degrading training efficiency.



\subsection{Native Sparsity as an Imperative}

These limitations in inference efficiency and training viability motivate our fundamental redesign of sparse attention mechanisms.
We propose \method{}, a natively sparse attention framework that addresses both computational efficiency and training requirements.
In the following sections, we detail the algorithmic design and operator implementation of \method{}.

\section{Formative Study}
To understand how large language models could support diagramming-based prewriting, we conducted a formative study involving 10 participants with daily writing habits ranging from news articles to fiction writing. We used snowball sampling to recruit students in writing- or creativity-related majors (e.g., creative media, design, English literature etc.). All participants were ethnically Chinese, and English was their second language.

\subsection{Protocol}
After participants consented to the study, we asked them to develop a science fiction or thriller story plot using an LLM together with the traditional tools (a pen and a piece of paper) for diagramming or illustrating their ideas.
Participants accessed the LLM using the GPT-3 playground interface.
They were asked to think aloud during the prewriting process.
After completing the tasks, we asked participants to reflect on their experiences and strategies.

The whole process was audio-taped, screen-recorded, and later transcribed for analysis. Two coders performed thematic analysis \cite{corbin2014basics} of the transcripts with reference to the screen recording to extract collaboration strategies, patterns, and breakdowns. The coders later held a discussion to reach a consensus on the themes.

\subsection{Findings}
We report the findings of our formative study in two themes: human-LLM collaboration workflow and common challenges encountered. We found our participants already anthropomorphized the LLM as a collaborator, and delegated to it distinct tasks for both divergent and convergent thinking.
% We first summarise how users leveraged the LLM to support prewriting, and clarify the functions and expected initiative of the LLM during the process. Then we report the two most significant challenges, progress tracking and communication breakdown.
%% Ideally say something related to our later design choice here

\subsubsection{Human-LLM Collaboration Workflow: Tasks and Initiative} \label{the_usage_of_LLMs_for_prewriting}
We observed that participants already implemented the parallel thinking strategy consciously or unconsciously.
They expected the LLM to perform multiple but distinctive functions, including generating additional ideas, elaborating on concepts, organizing fleeting thoughts, and enriching existing writing with details.
These functions were typically seen in creativity processes, covering both convergent and divergent thinking phases.
% From the first to the third stage, ideas become increasingly concrete, articulated, and formalised.
% This human-LLM collaboration workflow aligns with previous conceptualisations of the creativity process in both Psychology \cite{wallas1926art} and Human-Computer Interaction (HCI) \cite{frich2019mapping,shneidenmab2003supporting}, because prewriting is inherently a stage of creativity, or a stage of discovery as described by Rohman \cite{rohman1965pre}.

We found that users almost always preferred to take the initiative during the whole collaboration process, unless they ran out of ideas. They would use LLMs mostly to enrich their ideas with details, such as bridging a logical gap in a plot or providing a nuanced portrait of a scene.
Only when users had no initial ideas or hit writer's block, would they let the LLM take the initiative to generate ideas. P3 described the ideal role of LLM as similar to a ``\textit{second mind}'' that ``\textit{processes all the context in parallel}'' and ``\textit{provides ideas when requested}'', while he could still take control of the general prewriting process.

Meanwhile, we also noticed that users were generally tolerant and did not mind following the ideas in LLM output, which were often initially vague or confusing. Some users (P3-4, P8-9) were found to spend a long time iteratively refining their own prompts to improve the LLM's output.

\subsubsection{Challenges: Progress Tracking \& Communication Breakdown} \label{progress_tracking}
\paragraph{\underline{Progress Tracking}}
Many of our participants (P2-4, P6-7) particularly emphasized that tracking collaboration progress, and maintaining the ever-changing collaboration history while prewriting with the LLM was challenging. Because prewriting is iterative, and LLM generations can be random, our participants frequently needed to expand on specific points within a lengthy piece of writing in a new context, or re-generate content from a previous version. On such occasions, some participants (e.g., P2, P7, P10) mentioned they would need examples, suggestions, or templates as a reference to polish their prompts, and requested features to maintain the history of collaboration with LLM. 
%Progress tracking was considered by P7 to be even more challenging in semi-structured strategies such as diagrams, because different types of generations at each request can be easily conflated with users' content.
\paragraph{\underline{Communication Breakdown}}
The uncertainty of prompt-based communication can often cause LLMs to generate unsatisfactory or even nonsensical results during prewriting. P2 and P4 reflected that, to effectively communicate with an LLM, proper and sufficient context should be articulated via prompts, which can be difficult in complex writing tasks. On these occasions, instead of rewriting their prompts, most participants (e.g., P1-4, P6, P9-10) chose to provide more context and ask LLMs to polish previous output. For example, P4 found the ending of an LLM-generated science fiction lacks originality. Instead of deleting the result and requesting a new one, she asked the LLM to avoid using banal superhero endings and explore existential questions, using an imperative sentence as if giving feedback to a human collaborator.

\subsection{Summary}
Our formative study reveals the creative nature of human-LLM collaboration during prewriting, where LLMs could offer additional perspectives and handle a range of complex ideation tasks. Although participants would like to take control most of the time, it is particularly beneficial that LLMs can run in parallel with users' diagramming activities and provide assistance when requested. During the collaboration, participants often found progress tracking and communication with the LLM using a conversational interface challenging. They requested features to support managing collaboration progress, and favoured an incremental feedback process to facilitate iteration.

\section{Design Goals}
Our primary design goal is to integrate LLMs into a diagramming-based interface to facilitate the application of the parallel thinking strategy in prewriting. We first introduce how we derive the core idea of ``microtasking'' to operationalize parallel thinking, and then report our three design goals to support a microtasking workflow.

\subsubsection*{\textbf{Microtasking for parallel thinking}}
%We are deeply inspired by P3's account of LLMs' collaboration role a ``second mind'', which resonates with the idea of ``parallel thinking'' proposed in the book ``\textit{Six Thinking Hats}'' \cite{de2017six,bono1985six}. 
The notion of ``parallel thinking'' separates the human thinking process into distinct functions and roles. 
%Similarly, LLMs can process a variety of complex but distinct tasks that run in parallel with the user's own thinking process.
In reality, parallel thinking can often be practiced through group collaboration, where each of the distinct roles can be played by different individuals or groups in parallel. We seek to simulate group collaboration to break down prewriting workflows into smaller, manageable, and independent tasks that support both divergent and convergent thinking processes.
%so that each task handled by LLMs can contribute in parallel to efficiently provide productive and creative results.

To this end, we adopt the concept of ``microtasks'', commonly used in crowd sourcing \cite{latoza2014microtask,chen2017retool}, collaborative writing \cite{iqbal2018multitasking,birnholtz2013write,teevan2016supporting}, and group brainstorming \cite{chilton2019visiblends,teevan2016supporting}. Previous studies suggest that a microtasking workflow simplifies complex tasks \cite{cheng2015break,kokkalis2013taskgenies}, facilitates recovery from interruptions, and leads to higher quality of results \cite{cheng2015break}. Similarly, in a human-LLM collaboration scenario, we expect small manageable microtasks that require little context of one another to run concurrently can make complex prewriting tasks easier to coordinate, and save the need to iteratively articulate complex context in prompts.
Specifically, we derive the following three design goals to implement this idea. 

\newtcbox{\BboxL}{on line,
  colframe=brainstorm,colback=brainstorm,
  boxrule=0.5pt,arc=1.5pt,boxsep=0pt,left=3pt,right=3pt,top=3pt,bottom=3pt}
\newtcbox{\SboxL}{on line,
  colframe=summarise,colback=summarise,
  boxrule=0.5pt,arc=1.5pt,boxsep=0pt,left=3pt,right=3pt,top=3pt,bottom=3pt}
\newtcbox{\EboxL}{on line,
  colframe=elaborate,colback=elaborate,
  boxrule=0.5pt,arc=1.5pt,boxsep=0pt,left=3pt,right=3pt,top=3pt,bottom=3pt}
\newtcbox{\DboxL}{on line,
  colframe=draft,colback=draft,
  boxrule=0.5pt,arc=1.5pt,boxsep=0pt,left=3pt,right=3pt,top=3pt,bottom=3pt}
\newtcbox{\FboxL}{on line,
  colframe=freewrite,colback=freewrite,
  boxrule=0.5pt,arc=1.5pt,boxsep=0pt,left=3pt,right=3pt,top=3pt,bottom=3pt}
\newtcbox{\AboxL}{on line,
  colframe=associate,colback=associate,
  boxrule=0.5pt,arc=1.5pt,boxsep=0pt,left=3pt,right=3pt,top=3pt,bottom=3pt}

\renewcommand{\arraystretch}{0.75}
% \begin{table*}[htb]
%     \centering
%     \resizebox{\linewidth}{!}{\begin{tabular}{lcccl}
%         \toprule
%         \textbf{Microtask} & \textbf{Input Type} & \textbf{Output Type} & \textbf{Creativity Phase} & \textbf{Prompt} \\
%         \midrule
%         \renewcommand{\arraystretch}{2.5}
%         \BboxL{\textcolor{white}{\textbf{Brainstorm}}}~\cite{huang2020heteroglossia,gero2019stylistic,wang2010idea,teevan2016supporting,lu2018inkplanner,wang2022interpretable} & keyword & keyword & \makecell[c]{divergent \\ divergent}
%         & \makecell[l]{\underline{Find Related:} Brainstorm keywords related to [placeholder]. \\ \underline{Find Synonym:} Find synonyms for [placeholder].} \\
%         & & & \\
%         \SboxL{\textcolor{white}{\textbf{{Summarise}}}}~\cite{sadauskas2015mining,dang2022beyond} & sticky note & sticky note & \makecell[c]{convergent \\ convergent}
%         & \makecell[l]{\underline{TLDR:} Provide a TLDR version of the following:\textbackslash n[placeholder] \\ \underline{Top 3 keywords:} Summarise top 3 keywords of the following:\textbackslash n[placeholder]} \\
%         & & & \\
%         \EboxL{\textcolor{white}{\textbf{Elaborate}}}~\cite{sadauskas2015mining,uto2015academic,jeon2021fashionq} & concept & concept & \makecell[c]{divergent \\ convergent} & \makecell[l]{\underline{Provide Examples:} What are examples of [placeholder]. \\ \underline{Clarification:} Provide a simple explanation of [placeholder].} \\
%         & & & \\
%         \DboxL{\textcolor{white}{\textbf{Draft}}}~\cite{lu2018inkplanner,chung2022talebrush} & section & sticky note & \makecell[c]{convergent \\ convergent} & \makecell[l]{\underline{Abstract:} [placeholder]\textbackslash n\textbackslash nWrite an abstract of the above outline. \\ \underline{Overview:} [placeholder].\textbackslash n\textbackslash nWrite an overview of the above outline.} \\
%         & & & \\
%         \FboxL{\textcolor{white}{\textbf{Freewrite}}}~\cite{baroudy2008procedural,lu2018inkplanner} & sticky note & sticky note & divergent & \makecell[l]{\underline{Co-creation:} [placeholder].\textbackslash n Continue to write.} \\
%         & & & \\
%         \AboxL{\textcolor{white}{\textbf{Associate}}}~\cite{gero2019metaphoria,chilton2019visiblends,wang2021popblends,faste2012untapped,hope2022scaling,wang2022interpretable} & nodes & keyword & divergent &  \makecell[l]{\underline{Find Relationship:} Clarify the relationship between [placeholder] and [placeholder] in simple words.} \\
%         \bottomrule
%     \end{tabular}}
%     \caption{Six default microtasks of \textit{Polymind}}
%     \label{table:microtasks} 
% \end{table*}

\renewcommand{\arraystretch}{0.75}
\begin{table*}[htb]
    \centering
    \resizebox{\linewidth}{!}{\begin{tabular}{lccl}
        \toprule
        \textbf{Microtask} & \textbf{Input Type} & \textbf{Output Type} &
        \textbf{Prompt} \\
        \midrule
        \renewcommand{\arraystretch}{2.5}
        \BboxL{\textcolor{white}{\textbf{Brainstorm}}}~\cite{huang2020heteroglossia,gero2019stylistic,wang2010idea,teevan2016supporting,lu2018inkplanner,wang2022interpretable} & keyword & keyword & %\makecell[c]{divergent \\ divergent} &
        \makecell[l]{\underline{Find Related:} Brainstorm keywords related to [placeholder]. \\ \underline{Find Synonym:} Find synonyms for [placeholder].} \\
        & & & \\
        \SboxL{\textcolor{white}{\textbf{{Summarise}}}}~\cite{sadauskas2015mining,dang2022beyond} & sticky note & sticky note & \makecell[l]{\underline{TLDR:} Provide a TLDR version of the following:\textbackslash n[placeholder] \\ \underline{Top 3 keywords:} Summarise top 3 keywords of the following:\textbackslash n[placeholder]} \\
        & & & \\
        \EboxL{\textcolor{white}{\textbf{Elaborate}}}~\cite{sadauskas2015mining,uto2015academic,jeon2021fashionq} & concept & concept & \makecell[l]{\underline{Provide Examples:} What are examples of [placeholder]. \\ \underline{Clarification:} Provide a simple explanation of [placeholder].} \\
        & & & \\
        \DboxL{\textcolor{white}{\textbf{Draft}}}~\cite{lu2018inkplanner,chung2022talebrush} & section & sticky note & \makecell[l]{\underline{Abstract:} [placeholder]\textbackslash n\textbackslash nWrite an abstract of the above outline. \\ \underline{Overview:} [placeholder].\textbackslash n\textbackslash nWrite an overview of the above outline.} \\
        & & & \\
        \FboxL{\textcolor{white}{\textbf{Freewrite}}}~\cite{baroudy2008procedural,lu2018inkplanner} & sticky note & sticky note & \makecell[l]{\underline{Co-creation:} [placeholder].\textbackslash n Continue to write.} \\
        & & & \\
        \AboxL{\textcolor{white}{\textbf{Associate}}}~\cite{gero2019metaphoria,chilton2019visiblends,wang2021popblends,faste2012untapped,hope2022scaling,wang2022interpretable} & nodes & keyword &  \makecell[l]{\underline{Find Relationship:} Clarify the relationship between [placeholder] and [placeholder] in simple words.} \\
        \bottomrule
    \end{tabular}}
    \caption{Six default microtasks of \textit{Polymind}}
    \label{table:microtasks} 
\end{table*}

\subsubsection*{\textbf{Goal 1: Scaffold visual-diagramming-based prewriting with microtasks}}
We aim to scaffold the prewriting process with a diagramming tool that supports common strategies such as concept mapping, mind mapping, outlining, etc. To enable natural collaboration with an LLM while diagramming, we are inspired by the concept of ``macros'' \cite{kurlander1992history} and seek to address the uncertainty of collaboration goals by defining default microtasks, and allowing users to customise their requirements or rapidly delegate their own microtasks.

Our formative study inspired us to draw upon existing literature on creativity support tools (CSTs) beyond prewriting itself to define default microtasks, as the collaboration workflow appeared to be a typical creativity process. We conducted a survey of both CST and writing tool literature by searching two academic databases, ACM Digital Library and Google Scholar, using three keywords: ``writing'', ``prewriting'', and ``creativity support''. We reviewed the top 100 entries for each keyword.
%We then performed a thematic analysis of system design papers among top 100 entries of each search result (200 entries in total), recording their key features and functions. Afterwards, we grouped these features or functions into different categories to derive our pre-defined microtasks.
Based on this survey, we identified 6 microtasks: ``Brainstorm'', ``Elaborate'', ``Summarise'', ``Draft'', ``Freewrite'', and ``Associate'', as summarised in \autoref{table:microtasks}. Of these microtasks, ``Brainstorm'' and ``Associate'' are typical divergent thinking tasks seen in existing CSTs (e.g., conceptual blending of \cite{wang2021popblends,chilton2019visiblends}, attribute detection of \cite{jeon2021fashionq}, group ideation of \cite{teevan2016supporting,wang2010idea}, etc.)
``Draft'' and ``Freewrite'' are common features in prewriting tools \cite{lu2018inkplanner,sadauskas2015mining}. ``Elaborate'' and ``Summarise'' are commonly used in the literature of writing support~\cite{uto2015academic,dang2022beyond}, as convergent thinking tasks to help articulate or organise existing ideas.

\subsubsection*{\textbf{Goal 2: Facilitate task management}}
Task management is crucial in human collaboration. In our scenario, a user acts as a leader who determines the goals and progress of the collaboration. Therefore, she should be granted sufficient control to manage microtasks. To this end, we further derive three sub-goals from the literature on human collaboration and our formative study.
\paragraph{\textbf{Goal 2.1: Provide awareness}}
The awareness information about other collaborators while using groupware \cite{gutwin2002descriptive} is vital in tasks such as collaborative writing \cite{birnholtz2013write} and collaborative learning \cite{fransen2011mediating}. On a prewriting interface (e.g., a diagramming canvas), where elements can be loosely organised and often scattered around, it might be hard to notice other collaborators' operations without proper design support. Therefore we seek to provide awareness features so that users can easily track the status of each microtask, and the results returned by each microtask.
As informed by ~\cite{gluck2007matching}, we aim to design different levels of awareness features that match the utility of different interruption types.

\paragraph{\textbf{Goal 2.2: Support progress tracking}}
Progress tracking is an essential aspect of many collaboration tasks, especially collaborative writing \cite{birnholtz2013write,birnholtz2012tracking}. Our formative study suggests that it is also a concern while collaborating with an LLM. We aim to help users manage the results of each microtask in a less demanding way, so that they do not clutter users' diagrams but can be merged into them once accepted.
\paragraph{\textbf{Goal 2.3: Facilitate human feedback}}
Feedback is essential for improvements \cite{dow2011shepherding,haug2021feeasy,huang2018feedback}. In our formative study, users generally preferred feedback-like communication upon unsatisfactory generations. Therefore we aim to facilitate user feedback to enhance LLM-generated content. In our system, the ``feedback'' is provided to an LLM, which implies that it should convert user requirements into actionable prompts.

\subsubsection*{\textbf{Goal 3: Apply mixed initiative}}
Although users preferred to maintain control most of the time in our formative study, they still wanted the LLM to take the initiative when running out of ideas, a common hurdle during prewriting. Sometimes, they even expected the LLM to further clarify or improve its output. Therefore, we aim to apply the principle of ``mixed initiative'' \cite{horvitz1999principles}, and allow a microtasking LLM to infer the focus of attention of the user to determine the timing of suggestions. To keep users in control and minimize the cost of inference errors, we aim to enable users to manage the initiative of each individual microtask. 
\newtcbox{\BboxS}{on line,
  colframe=brainstorm,colback=brainstorm,
  boxrule=0.5pt,arc=1pt,boxsep=0pt,left=2pt,right=2pt,top=2pt,bottom=2pt}
\newtcbox{\SboxS}{on line,
  colframe=summarise,colback=summarise,
  boxrule=0.5pt,arc=1pt,boxsep=0pt,left=2pt,right=2pt,top=2pt,bottom=2pt}
\newtcbox{\EboxS}{on line,
  colframe=elaborate,colback=elaborate,
  boxrule=0.5pt,arc=1pt,boxsep=0pt,left=2pt,right=2pt,top=2pt,bottom=2pt}
\newtcbox{\DboxS}{on line,
  colframe=draft,colback=draft,
  boxrule=0.5pt,arc=1pt,boxsep=0pt,left=2pt,right=2pt,top=2pt,bottom=2pt}
\newtcbox{\FboxS}{on line,
  colframe=freewrite,colback=freewrite,
  boxrule=0.5pt,arc=1pt,boxsep=0pt,left=2pt,right=2pt,top=2pt,bottom=2pt}
\newtcbox{\AboxS}{on line,
  colframe=associate,colback=associate,
  boxrule=0.5pt,arc=1pt,boxsep=0pt,left=2pt,right=2pt,top=2pt,bottom=2pt}

\newtcbox{\CboxS}{on line,
colframe=custom,colback=custom,
boxrule=0.5pt,arc=1pt,boxsep=0pt,left=2pt,right=2pt,top=2pt,bottom=2pt}

\newtcbox{\CCboxS}{on line,
colframe=custom2,colback=custom2,
boxrule=0.5pt,arc=1pt,boxsep=0pt,left=2pt,right=2pt,top=2pt,bottom=2pt}
  
\section{Use Case Scenario}
In this section, we illustrate the workflow of \textit{Polymind} via a use case scenario. Suppose that Bob, an HCI researcher, would like to use fictional narratives to promote his research on social media. He therefore chooses to use \textit{Polymind} to plan his fiction writing.

\subsection*{Task Management}
Bob starts with the key insights of his paper, that parallel inputs from AI agents such as LLMs can significantly increase writers' creativity. His primary goals of using \textit{Polymind} are brainstorming narrative lines, and working out a rough outline. Therefore he keeps three microtasks in the proactive mode to quickly expand his ideas: \BboxS{\textcolor{white}{Brainstorm}}, \EboxS{\textcolor{white}{Elaborate}}, and \AboxS{\textcolor{white}{Associate}}; and switched all other microtasks to the reactive mode so that they will not be intrusive.

\subsection*{Collaborative Brainstorming}
To get some initial ideas, Bob uses \textit{Polymind} to perform mind mapping, which focuses on tracking spontaneous and free-form ideas, and their associations.
He first creates some keywords and concepts on the canvas: such as ``parallel collaboration'', ``creative writing'', etc. The three proactive microtasks, \BboxS{\textcolor{white}{Brainstorm}}, \EboxS{\textcolor{white}{Elaborate}}, \AboxS{\textcolor{white}{Associate}} shortly returns some related keywords, example scenarios, and associations between these diagrams.

Bob accepts three diagrams: ``synchronous tasks'', ``mutual goals'', and ``flash fiction''. These remind him of a story where a former fiction writer revolutionizes the fiction writing industry by leveraging multiple robots in an assembly line to mass produce flash fictions. Each robot is configured to handle a distinct microtask on the assembly line, working synchronously towards a common goal.

\subsection*{Task Delegation}
At this point, Bob feels he has obtained some concrete ideas, and would like to organize them via concept mapping.
This method aims to outline structures and relationships between concepts. He also wants to ask the LLM how to make a story more engaging. Therefore, he delegates a new microtask \CboxS{\textcolor{white}{Improve}}, which returns suggestions for improvements.
He leaves it in the \textit{reactive} mode so that it can provide suggestions based on the ever-changing diagram upon request.

\subsection*{Idea Clarification}
Bob starts concept mapping by specifying elements of the story. He creates some diagram nodes, such as ``Character: Fiction Writer \& Robots'', ``Event: Revolutionizing the industry through assembly lines of flash fictions'', etc. He then creates a section over these diagrams to group them together, and uses the microtask \DboxS{\textcolor{white}{Draft}} to request an outline, and clicks the \FboxS{\textcolor{white}{Freewrite}} microtask several times to continue writing to see different endings.

Bob feels that these results are somewhat bland, and therefore asks the microtask \CboxS{\textcolor{white}{Improve}} to suggest improvements based on these diagrams. The results suggest adding to the beginning the conflict between the fiction writer and his former boss that fired him for lacking creativity. It reminds Bob that he could depict the former boss as a firm advocate of turn-taking conversational robots, and unveil the superiority of a parallel collaboration through the main character's adventure.

% \subsubsection*{Task Management}
% Bob starts with the core concept of using LLMs to support prewriting, and he hopes to develop ideas about system design and potential contributions. His primary goals when using \textit{Polymind} are generating new ideas and organizing existing thoughts.
% %He hopes to track his thought process while using the AI to expand on his ideas.
% Therefore, he chooses to keep three microtasks: \BboxS{\textcolor{white}{Brainstorm}}, \EboxS{\textcolor{white}{Elaborate}}, and \AboxS{\textcolor{white}{Associate}} in the proactive mode to augment divergent thinking, and leave the other three microtasks, \FboxS{\textcolor{white}{Freewrite}}, \SboxS{\textcolor{white}{Summarise}}, and \DboxS{\textcolor{white}{Draft}}, in the reactive mode.
% %Therefore, he  running in the proactive mode so that they can constantly contribute to diagramming, and , generating diagram elements only upon requests.

% \subsubsection*{Collaborative Brainstorming}
% To get some initial ideas, Bob uses \textit{Polymind} to perform mind mapping, which focuses on tracking spontaneous and free-form ideas, and their associations.
% He first creates a keyword on the canvas: ``prewriting''. The \BboxS{\textcolor{white}{Brainstorm}} microtask shortly returns several relevant keyword suggestions. Bob accepts two of them, ``Freewriting'' and ``Clustering''.
% %, which he believes are common but under-explored  prewriting strategies in HCI research. 

% Bob realizes that there might be design opportunities to combine these two strategies, while the \AboxS{\textcolor{white}{Associate}} microtask starts to operate on these newly added nodes.
% Bob expands results and sees three suggestions connecting ``Freewriting'' with ``Clustering'': ``Unstructured writing'', ``Visual organization'', and ``Idea generation''.
% Bob accepts ``Unstructured writing'' and ``Visual organization'', which reminds him of visual summaries of semantic information on a freewriting interface. 
% He then creates a concept ``Hierarchical clustering of freewriting'' to ``Hierarchical freewriting'' and connects it to ``Unstructured writing'' and ``Visual organization''.

% \subsubsection*{Task Delegation}
% At this point, Bob feels he has obtained some concrete ideas, and would like to organize them by concept mapping.
% This method aims to outline structures and relationships between concepts. He also wants to leverage the LLM to evaluate the feasibility of his research ideas. Therefore, he delegates a new microtask \CboxS{\textcolor{white}{Evaluate}}, which returns potential weaknesses for individual ideas. 
% %He leaves it in the reactive mode so that it can provide suggestions based on the ever-changing diagram upon requests.

% \subsubsection*{Idea Clarification}
% Bob starts concept mapping by creating a diagram node ``semantic visualization'', and connects it with ``visual organization'', as he thinks of the feature of semantic visual summary. The \EboxS{\textcolor{white}{Elaborate}} microtask returns three examples of ``semantic visualization'', ``word clouds'', ``tree maps'', and ``network diagrams''. Bob accepts them all as he believes multiple visualization techniques can be used to address different levels of details. He also creates a keyword ``three LoD'' (level of details) and connects it to the three examples.
% Bob moves to evaluate the novelty and significance of his ideas. He creates a sticky note on the canvas ``Contributions of a freewriting interface to HCI that supports visual organization of semantic information'', and asks the \FboxS{\textcolor{white}{Freewrite}} microtask to continue. The microtask returns a paragraph of text suggesting several potential strengths of such systems.
% %including facilitating group collaboration and communication, enhancing reflection of ideas, allowing multiple users to build upon or contribute to one user’s ideas, etc.
% Finally, Bob creates a section over these nodes named ``freewriting interface that supports semantic visualization'', and invokes the \CboxS{\textcolor{white}{Evaluate}} microtask.
% The result reminds him that such a system could overwhelm users cognitively.

\begin{figure}[ht]
% \centering\captionsetup{width=\linewidth,font={small}}
\includegraphics[width=.85\linewidth]{figures/UI_cropped.png}
\caption{The interface of \textit{Polymind} comprises: A. a diagramming canvas B. a toolbar C. a task board}
\label{fig:UI}
\end{figure}
\section{Designing Polymind}
\revision{
\textit{Polymind}'s interface provides a range of diagramming features commonly used in prewriting strategies, and a ``task board'' overlaid on the canvas for microtask management, as shown in ~\autoref{fig:teaser} and ~\autoref{fig:UI}. We map the input and output of each microtask to diagram types on the canvas. To facilitate collaboration and task management, we introduce the notion of ``task header'', and ``task cards''. The former displays notifications and previews of microtask results on a specific diagram, while the latter displays specifications of a microtask on the ``task board''.
To apply mixed initiative, we define two initiative modes: proactive and reactive. We also design a set of workflows to provide awareness information.
In this section, we first provide an overview of the \textit{Polymind} interface, and then elaborate on each of our key features.
}

% \begin{figure}[ht]
% \centering
% \begin{subfigure}{.45\linewidth}
%     \centering
%     \includegraphics[width=.85\linewidth]{figures/keyword.png}
%     \label{fig:keyword}
% \end{subfigure}
% \begin{subfigure}{.45\linewidth}
%     \centering
%     \includegraphics[width=.85\linewidth]{figures/concept.png}
%     \label{fig:concept}
% \end{subfigure}
% \begin{subfigure}{.45\linewidth}
%     \centering
%     \includegraphics[width=.9\linewidth]{figures/sticky_note.png}
%     \label{fig:sticky_note}
% \end{subfigure}
% \begin{subfigure}{.45\linewidth}
%     \centering
%     \includegraphics[width=\linewidth]{figures/section.png}
%     \label{fig:section}
% \end{subfigure}
% \caption{Three basic diagrams (or nodes) in \textit{Polymind}, and the section, which stores multiple diagrams and their connections.}
% \end{figure} \label{fig:diagrams}

\begin{figure}[ht]
\centering
\includegraphics[width=\linewidth]{figures/UIComponents.png}
\caption{\textit{Polymind} supports three basic diagrams (or nodes), and allows users to draw a section over diagrams. The task board interface supports microtask management, and maps microtask input and output to different diagram types on the canvas. The results of microtasks are displayed as notifications and previews on the task header before being expanded and accepted.}
\label{fig:diagrams}
\end{figure}

\subsection{Main Interface}
The interface of \textit{Polymind} comprises a diagramming canvas, a task board, and a toolbar (\autoref{fig:UI}).
\revision{The diagramming canvas includes all diagrams and their topological connections. We defined three basic diagrams: \textbf{\textcolor{keyword}{\textit{keyword}}}, \textbf{\textcolor{concept}{\textit{concept}}}, \textbf{\textcolor{sticky_note}{\textit{sticky note}}}, each with a unique shape affording varying text lengths (keywords the shortest, and sticky notes for the longest pieces of text). Users can connect diagrams through directed or undirected edges, or draw a \textbf{\textcolor{section}{\textit{section}}} over diagrams (see ~\autoref{fig:diagrams}).

The task board displays all microtasks and their specifications in distinct ``task cards'', each including input and output types, prompts, etc. The results of each microtask are displayed as notifications and previews on the ``task header'' over that processed diagram before they are expanded and accepted (see ~\autoref{fig:task_status}).}

\subsubsection{The Diagramming Canvas}
The diagramming canvas supports most common diagrams. Over each of these diagrams, a task header is attached to display the status of microtasks. All resulting diagrams of microtasks are displayed using a hollow shape in contrast to user created diagrams.

\paragraph{\underline{Diagrams}}
To scaffold the diagramming process for \textit{Goal 1}, we define three primitive diagrams that are commonly used in diagramming or prewriting tools (e.g., Inkplanner \cite{lu2018inkplanner}, Figma \footnote{https://www.figma.com/}, etc.): \textbf{\textcolor{keyword}{\textit{keyword}}} (displayed as text labels), \textbf{\textcolor{concept}{\textit{concept}}} (represented by an ellipse), and \textbf{\textcolor{sticky_note}{\textit{sticky note}}} (see \autoref{fig:diagrams}). We use sizes, shapes and the placeholder text upon creation (``Add Keyword'', ``Add Concept'', ``Add text'') to guide users to type text of different lengths and of different functions into different diagrams (e.g, brief words in \textbf{\textcolor{keyword}{\textit{keyword}}}, short phrases in \textbf{\textcolor{concept}{\textit{concept}}}, long paragraphs in \textbf{\textcolor{sticky_note}{\textit{sticky note}}}), so that we can properly define microtasks handling various types of input to support all three stages of LLM usage (see \ref{the_usage_of_LLMs_for_prewriting}).

These three primitive diagrams can be selected, moved, resized or scaled as per users' needs. Additionally, users can establish two types of connection between these diagrams: directed (arrow) or undirected (line). We also allow users to create a \textbf{\textcolor{section}{\textit{section}}} (\autoref{fig:diagrams}) among these diagrams, and assign titles to these sections. On such an interface, users can perform most diagram-based prewriting strategies such as mind mapping, concept mapping, argument mapping, etc.

It is important to note that the diagramming canvas maintains a graph-like (usually tree-like) structure, and we later refer to those primitive diagrams as ``nodes'' at times for simplicity.

\paragraph{\underline{The Task Header}}
To support \textit{Goal 2} (especially \textit{Goal 2.1}) and \textit{Goal 3}, we design a task header that is attached to each diagram (i.e., \textbf{\textcolor{keyword}{\textit{keyword}}}, \textbf{\textcolor{concept}{\textit{concept}}}, \textbf{\textcolor{sticky_note}{\textit{sticky note}}}, and \textbf{\textcolor{section}{\textit{section}}}) on the diagramming canvas to display the status of each microtask on the diagram element (\autoref{fig:task_status}). On the task header, the name of each microtask is displayed as small text labels. By default, all microtasks are activated (see \ref{task_initiative}) and filled with distinctive colours. Each task header also has a preview panel (\autoref{fig:notifications_and_previews}) that displays key points of unread microtask results. It will only pop up when users hover over the task header, and there are unread microtask results.

\begin{figure}[h]
\begin{minipage}{.5\textwidth}
    \captionsetup{width=\linewidth}
    \vspace{0pt}
    \includegraphics[width=\linewidth]{figures/feedback_v2.png}
    \caption{(a) When the user hovers over the three icons over the resulting diagram, suggestions to improve the generation will pop up. The user can click on these suggestions to request a regeneration. (b) If a user clicks on the question mark icon, the system will return an explanation of the generated result.}
    \label{fig:feedback}
\end{minipage}
\begin{minipage}{.45\textwidth}
    \captionsetup{width=.9\linewidth}
    \vspace{0pt}
    \includegraphics[width=\linewidth]{figures/preview_summary2.png}
    \caption{Once the mouse hovers over key points of a microtask on the preview panel for 1.5 seconds, the system will present a summary of the generated results using a news ticker effect.}
    \label{fig:preview_summary}
\end{minipage}
\end{figure}



\paragraph{\underline{Results of Microtasks}} To support \textit{Goal 2.2} we display all resulting diagram nodes using a hollow shape (filled in white), including \textbf{\textcolor{keyword}{\textit{keyword}}}, \textbf{\textcolor{concept}{\textit{concept}}}, and \textbf{\textcolor{sticky_note}{\textit{sticky note}}}. The border colour indicates the specific microtask each resulting node belongs to. A user can choose to discard or accept a resulting node; upon acceptance it will be displayed normally, the same as user-created diagram nodes.

Apart from the two icons for accepting or discarding the diagram node, to support \textit{Goal 2.3}, we design a ``question mark'' icon (see \autoref{fig:feedback}) that users can click on to request an explanation for this specific generation. For each resulting node, we also provide three heuristic suggestions for users to request a regenerated node, ``Be creative'', ``Be more specific'', and ``Be brief''. These suggestions will pop up if users hover over those three icons (see \autoref{fig:feedback}).

\subsubsection{The Task Board}
The task board of \textit{Polymind} enables users to configure and delegate microtasks, to support our \textit{Goal 2}. The task board borrows the design of Trello \footnote{https://trello.com/} board, which consists of a list of ``task cards''. Users can click on the ``add'' icon to delegate a new microtask, or toggle the ``visibility'' switch on the upper-right corner to hide all task headers on the diagramming canvas.

\paragraph{\underline{Task Card}}
A task card contains specifications of a microtask, including the task name, input type, output type, initiative mode (global), visibility (global), and prompts to communicate with the LLM (\autoref{fig:task_card}). The name of the microtask is always displayed in the upper-left corner, each indicated by a distinctive colour (same as resulting diagrams or labels on task headers).

Users can browse through task specifications by clicking on the ``page turn'' icon at the bottom; delete the microtask by clicking on the ``delete'' icon; toggle visibility of all resulting diagrams on the canvas by clicking on the ``visibility'' icon; toggle the initiative modes (see \ref{task_initiative}) by clicking on the task name (the text label). They can also change the specifications of each microtask (see \ref{task_configuration}).

\begin{figure}[t!]
\centering
% \begin{subfigure}{\linewidth}
\begin{subfigure}{.32\linewidth}
    \centering
    % \includegraphics[width=\linewidth]{figures/taskcard1.png}
    \includegraphics[width=\linewidth]{figures/taskcard1.png}
    \subcaption{The first page}
    \label{fig:task_card_page1}
\end{subfigure}
\begin{subfigure}{.32\linewidth}
    \centering
    \includegraphics[width=\linewidth]{figures/taskcard2.png}
    % \includegraphics[width=.8\linewidth]{figures/taskcard2.png}
    \subcaption{The second page}
    \label{fig:task_card_page2}
\end{subfigure}
\begin{subfigure}{.32\linewidth}
    \centering
    % \includegraphics[width=.75\linewidth]{figures/taskcard3.png}
    \includegraphics[width=\linewidth]{figures/taskcard3.png}
    \subcaption{Toggling initiative modes}
    \label{fig:task_card_inactive}
\end{subfigure}
\caption{The task card of a microtask. The text label of the task name is indicated with a distinctive colour. (a) By default, the card shows information of 
% \ref{fig:task_card_page1} 
% the information of 
its input and output types, which are configurable. (b) Users can click on the arrow at the bottom of the card to turn pages, and edit prompts in the second page. The user can also switch between predefined prompts. (c) The user can click on the text label to toggle initiative modes globally, or the ``visibility'' icon to display or hide all resulting diagrams.}
\label{fig:task_card}
\end{figure}



% \begin{figure*}[ht]
% \centering
% \includegraphics[width=\linewidth]{figures/task_cards.png}
% % \centering\captionsetup{width=\linewidth,font={small}}
% % \begin{subfigure}{.32\textwidth}
% %     \centering
% %     \includegraphics[width=.95\linewidth]{figures/task_card1.png}
% %     \subcaption{}
% %     \label{fig:task_card_page1}
% % \end{subfigure}
% % \begin{subfigure}{.32\textwidth}
% %     \centering
% %     \includegraphics[width=.95\linewidth]{figures/task_card2.png}
% %     \subcaption{}
% %     \label{fig:task_card_page2}
% % \end{subfigure}
% % \begin{subfigure}{.32\textwidth}
% %     \centering
% %     \includegraphics[width=.95\linewidth]{figures/task_card_inactive.png}
% %     \subcaption{}
% %     \label{fig:task_card_inactive}
% \caption{The task card of a microtask. The page (a) contains information of its input and output types, and the page (b) and (c) display the prompt. The text label of the task name is indicated with a distinctive colour. A user can 1) change input or output types; 2) manually change prompts; 3) switch between predefined prompts; 4) delete the microtask; 5) click on the ``visibility'' icon the display or hide all resulting diagrams;  6) click on the text label to toggle initiative modes (as shown in page (d) and (e)). }
% \end{figure*} \label{fig:task_card}

\paragraph{\underline{Default Microtasks}} There are 6 predefined microtasks in \textit{Polymind}, which are \BboxS{\textcolor{white}{Brainstorm}}, \EboxS{\textcolor{white}{Elaborate}}, \SboxS{\textcolor{white}{Summarise}}, \DboxS{\textcolor{white}{Draft}}, \FboxS{\textcolor{white}{Freewrite}}, and \AboxS{\textcolor{white}{Associate}}, as specified in our \textit{Goal 1}. Each microtask has a default input and output type, and prompt templates to communicate with the LLM. In practice, when operating on an input node, each microtask will replace the ``[placeholder]'' with the text in that node to prompt the LLM. For details, we refer our readers to \autoref{table:microtasks}. These defaults can all be changed later as per users' needs.

For the input type of ``nodes'', the microtask will, given a node on the canvas, sample another nearby node to perform an operation, and generated diagrams will be linked to both nodes. For the ``section'' input, the microtask will calculate the outline of all nodes (similar to \cite{lu2018inkplanner}) within the section by performing a depth-first search (DFS) of all non-leaf nodes. For example, for a section with a standalone keyword ``creativity'', and a root node ``writing'' connecting to another two keywords, ``drafting'', ``proofreading'', the resulting DFS sequence will be:
\begin{quote}
Writing

 -\ \ \ Drafting
 
 -\ \ \ Proofreading
 
 Creativity
\end{quote}

\subsubsection{The Toolbar}
The toolbar comprises four icons, a pile of sticky notes, and an ellipse representing a concept (see \autoref{fig:teaser}). Users can drag a sticky note or a concept (the ellipse) from the toolbar to the diagramming canvas, or click on the ``text'' icon and then click on the canvas again to add a keyword.
The two leftmost icons are used for connecting diagram nodes using arrows (directed) and lines (undirected). Users can click on the icon and select anchor points of a node to connect with another. There is also an icon for sectioning where users can click on it, and draw a rectangle over diagrams as a section.

\subsection{Processing a Proactive Microtask}
To achieve our \textit{Goal 3}, we introduce two initiative modes: proactive and reactive (\autoref{fig:task_status}). In the reactive mode, the microtask will function similarly to a button, and will not execute until a user clicks on its label on a task header. Those reactive microtasks will be displayed using a gray colour on the task header. By default, all microtasks are activated and run automatically in parallel with user activities.
\revision{
In practice, we sample every \textit{x} ($x=5$ in practice) seconds a diagram node for each input type: \textbf{\textcolor{keyword}{\textit{keyword}}}, \textbf{\textcolor{concept}{\textit{concept}}}, \textbf{\textcolor{sticky_note}{\textit{sticky note}}}, and \textbf{\textcolor{section}{\textit{section}}}.
Each microtask will operate on the sampled diagram corresponding to its input type, if it does not have unchecked notifications and is not displaying results on that diagram.
}
Results of these microtasks will be displayed as notifications and previews before users manually expand all resulting diagrams.

\subsubsection{Inferring User Attention}
In an earlier pilot study of our system, we found that users might lose track of generation results if a microtask was constantly operating on diagram nodes far away from users' focus of attention. Therefore, for our \textit{Goal 2.1}, we want each proactive microtask to infer the user's attention and mainly operate on diagrams of a user's focus. We make the assumption that diagram nodes that can be most easily selected are nodes of users' focus. In other words, the wider the node, or the closer to the mouse cursor, the more likely the node is the focus of users' attention, and operating on it will more likely make users aware. In practice, every \textit{x} seconds, for each node among each of the 5 types of input, we compute an index of difficulty ($ID$) according to Fitts' law \cite{fitts1954information}:
\begin{equation*}
    ID = \log_2 \frac{2D}{W}
\end{equation*}
where $D$ denotes the distance between the current mouse position and the node centre, and $W$ denotes the width of the node.

However, if we choose the node with the lowest $ID$ each time, we will end up always choosing the same node for operation if a user barely moves the mouse within a period of time. Therefore, in practice, we sample a node from a uniform distribution where the probability of a node $i$ being sampled as an input type $T$ is computed as:
\begin{equation*}
p(i, T) = \frac{ID_i}{\sum_{j \in T} ID_j}
\end{equation*}
$T$ includes \textbf{\textcolor{keyword}{\textit{keyword}}}, \textbf{\textcolor{concept}{\textit{concept}}}, \textbf{\textcolor{sticky_note}{\textit{sticky note}}}, and \textbf{\textcolor{section}{\textit{section}}}.
\revision{For microtasks that operate on two \textbf{\textit{nodes}} (e.g., the default \AboxS{\textcolor{white}{Associate}}), we first sample a primitive diagram, and then randomly sample a nearby diagram for prompting.}
%and \textbf{\textit{nodes}}.

\subsubsection{Notifications and Previews of Microtask Results}
We use two levels of attention draw features: notifications and previews of microtask results to support \textit{Goal 2.1}, so that results ready for display will not intrude users' diagramming process.

\begin{figure*}[ht]
% \centering\captionsetup{width=\linewidth,font={small}}
\includegraphics[width=\textwidth]{figures/notifications_previews4.png}
\caption{The processing of a proactive microtask. Once results are obtained from the LLM, it will first ``draw a curtain'' over the task header to notify users, displaying key points of the generation. Users can click on the ``expand'' icon on the ``curtain'' for a quick view. If a user does not expand these result, it will be marked as unread notifications. Once users hovers over the task header, a preview panel will pop up, showing a preview of all unread results.}
\label{fig:notifications_and_previews}
\end{figure*}

\paragraph{\underline{Notifications}}
To notify the user once results of a proactive microtask are ready, it will first ``draw a curtain'' (see \autoref{fig:notifications_and_previews}) over the task header of the node being operated. The curtain will be filled with a distinctive colour indicating which microtask those results are from, and meanwhile, display key points summarising the generated results. Users can click on the ``expand'' icon on the right to display results, and then click on the ``collapse'' icon to hide them. If the user does not perform any operations, the curtain will collapse, and those results will be marked as ``unread'' with a small red circle on the upper-left corner as a sign of notification. (the red circle in \autoref{fig:notifications_and_previews}).

\paragraph{\underline{Previews}}
Once there are unread results, the preview panel of a task header will store a preview of these results. When a user hovers over the task header, the preview panel (should there be any results unread) will pop up, displaying key points of results of each microtask, using labels filled with distinctive colours (see \autoref{fig:notifications_and_previews}). Once a user hovers over a specific label for 1.5 seconds, the key point will turn into a brief summary of the microtask results, displayed using effects resembling a news ticker (see \autoref{fig:preview_summary})

% \begin{figure}[ht]
\centering
\includegraphics[width=.5\linewidth]{figures/preview_summary2.png}
\caption{Once the mouse hovers over key points of a microtask on the preview panel for 1.5 seconds, the system will present a summary of the generated results using a news ticker effect.}
\label{fig:preview_summary}
\end{figure}

\subsection{Managing Microtasks}
To support our \textit{Goal 2} and \textit{Goal 3}, we design a set of features to support configuring existing microtasks, and delegating new microtasks.

\begin{figure*}[ht]
\centering
\includegraphics[width=\textwidth]{figures/microtask_status_v6.png}
% \centering\captionsetup{width=0.64\linewidth}
% \begin{subfigure}{0.32\textwidth}
%     \centering
%     \includegraphics[width=\linewidth]{figures/status1.png}
%     \subcaption{normal mode}
%     \label{fig:status1}
% \end{subfigure}
% \begin{subfigure}{0.32\textwidth}
%     \centering
%     \includegraphics[width=\linewidth]{figures/status2.png}
%     \subcaption{notification}
%     \label{fig:status2}
% \end{subfigure}
% \begin{subfigure}{0.32\textwidth}
%     \centering
%     \includegraphics[width=\linewidth]{figures/status3.png}
%     \subcaption{display mode}
%     \label{fig:status3}
% \end{subfigure}
% \begin{subfigure}{0.45\textwidth}
%     \centering
%     \includegraphics[width=.85\linewidth]{figures/status4.png}
%     \subcaption{inactive mode}
%     \label{fig:status4}
% \end{subfigure}
% \begin{subfigure}{0.45\textwidth}
%     \centering
%     \includegraphics[width=.85\linewidth]{figures/status5.png}
%     \subcaption{displaying results in the inactive mode (when clicked)}
%     \label{fig:status5}
% \end{subfigure}
\caption{Different status of a microtask on a particular node.}
\label{fig:task_status}
\end{figure*}
\subsubsection{Toggling Visibility}
For the purpose of \textit{Goal 2.2}, users can toggle the visibility of microtask results both globally and locally. By clicking on the ``visibility'' icon on each task card \autoref{fig:task_card_page1}, users can display all corresponding generations on the canvas. Users can also click on the task name label on each task header \autoref{fig:task_status} to display generations corresponding to the specific node locally.

For a reactive microtask, generations are only requested once users click on the label on a task header. On such occasions, the label turns black once generations are ready, and users can click on the label again to hide all generations. For a proactive microtask, besides clicking on the ``expand'' icon on the ``curtain'' (\autoref{fig:notifications_and_previews}), users can also click on the label on task headers to see all resulting diagrams of a microtask. In this case, the microtask on this node is in ``display'' mode, and the text on the label is underlined. Should there be unread results, there would also be an ``expand all'' icon that users can click on to show the results of all microtasks. Should there be any microtask in ``display'' mode, there would also be a ``close all'' icon for users to hide the results of all microtasks.

\subsubsection{Toggling Initiative Modes} \label{task_initiative}
The initiative modes (proactive vs. reactive) can also be toggled both globally and locally for our \textit{Goal2} and \textit{Goal 3}. Users can either click on the task name label on the ``task card'' to deactivate it globally (\autoref{fig:task_card_inactive}), or control-click labels on each task header to deactivate the microtask for a specific node (\autoref{fig:task_status}). Once a microtask is globally deactivated, the corresponding label will turn gray and function solely as a button on all task headers, provided that it is not in ``display'' mode and has no notifications. Task headers of newly created diagrams will also have this microtask turned off.

\subsubsection{Task Configuration and Delegation}
To support our \textit{Goal 2}, we allow users to specify microtask requirements, including input and output types, and prompts.
We also enable users to rapidly create and delegate a new microtask to the LLM.

\paragraph{\underline{Configuring Task Specifications}} \label{task_configuration}
Users can change the input and output types of a microtask by clicking on the left and right arrow triangle icon (\autoref{fig:task_card}a). We currently support four input types (\textbf{\textcolor{keyword}{\textit{keyword}}}, \textbf{\textcolor{concept}{\textit{concept}}}, \textbf{\textcolor{sticky_note}{\textit{sticky note}}}, \textbf{\textcolor{section}{\textit{section}}}) and three output types (\textbf{\textcolor{keyword}{\textit{keyword}}}, \textbf{\textcolor{concept}{\textit{concept}}}, \textbf{\textcolor{sticky_note}{\textit{sticky note}}}). Users can also change prompts by double-clicking the text box of the prompt (\autoref{fig:task_card}b). For some microtasks, they can also switch between predefined prompt examples. Each prompt should have a ``[placeholder]'' to be filled with the text of a node (\autoref{fig:task_card}b).

\paragraph{\underline{Delegating a New Microtask}}
For our \textit{Goal 2.3}, a user can click the ``add'' icon on the task board to create a new microtask (\autoref{fig:teaser}B). After clicking, the task board will spawn a new card for users to specify the task name and the example prompt. Users can later change input and output type, or toggle visibility or initiative on a newly added task card after clicking ``confirm''. When a user clicks to specify a task name, the system will prompt the LLM to suggest a task name. Once a task name is specified, the system will also request an example prompt for users from the LLM. \autoref{fig:add_new_task} illustrates the whole process of adding a new microtask.

% \begin{figure*}[ht]
% \begin{subfigure}{0.45\textwidth}
%     \centering
%     \includegraphics[width=.85\linewidth]{figures/new_task1.png}
%     \subcaption{}
%     \label{fig:new_task1}
% \end{subfigure}
% \begin{subfigure}{0.45\textwidth}
%     \centering
%     \includegraphics[width=.85\linewidth]{figures/new_task2.png}
%     \subcaption{}
%     \label{fig:new_task2}
% \end{subfigure}
% \begin{subfigure}{0.45\textwidth}
%     \centering
%     \includegraphics[width=.85\linewidth]{figures/new_task3.png}
%     \subcaption{}
%     \label{fig:new_task3}
% \end{subfigure}
% \begin{subfigure}{0.45\textwidth}
%     \centering
%     \includegraphics[width=.85\linewidth]{figures/new_task4.png}
%     \subcaption{}
%     \label{fig:new_task4}
% \end{subfigure}
% \caption{}
% \end{figure*} \label{fig:add_new_task}

\begin{figure*}[ht]
% \centering\captionsetup{width=\linewidth,font={small}}
\includegraphics[width=\textwidth]{figures/add_new_task_v3.png}
\caption{The workflow of delegating a new microtask. (a) The user first clicks on the ``add'' icon and a new card will appear. (b) After user inputs a task name, the system requests an example prompt from the LLM. (c) User clicks ``confirm'' to add the microtask.}
\label{fig:add_new_task}
\end{figure*}

\subsection{Implementation}
\textit{Polymind} was implemented using Javascript, React \footnote{https://react.dev/} and react-konva \footnote{https://konvajs.org/docs/react/Intro.html}. The backend LLM was ChatGPT (GPT-4), and the temperature was set to $0.7$.
We set $x=5$ based on our pilot study, i.e., every 5 seconds, we sample a diagram node for each of the 5 input types to be processed by proactive microtasks. If \textit{x} were too small, users would be overwhelmed with notifications of new results; if too large, there would be noticeable latency, and users will often end up waiting for results. Once the user confirms an update of a diagram node's text content, a re-sampling of the corresponding input type will be executed immediately.

To reduce the computational cost, we quit prompting ChatGPT with the sampled diagram node if a proactive microtask has unchecked results (i.e., notifications) or expanded diagrams on it.
To make generations less random, in each prompt we added constraints on its output.
\revision{For \textbf{\textcolor{keyword}{\textit{keyword}}}, we request 3 generations, each no more than 3 words; for \textbf{\textcolor{concept}{\textit{concept}}} we request 3 generations no more than 5 words; for \textbf{\textcolor{sticky_note}{\textit{sticky note}}} we request 1 generation no more than 150 words.}
Whenever ChatGPT returns results, we preserve the previous dialogue in each resulting diagram node. Once users request an explanation or a regeneration of the node, we further prompt the ChatGPT based on the saved dialogue (see \ref{appendix_feedback}). When the system requests a microtask name, we use names of predefined microtask as a few-shot template to prompt the ChatGPT. When the system requests a prompt for a new microtask, we use name-prompt pairs of predefined microtasks as a few-shot template. We refer our readers to \ref{appendix_newtask} for more details.
\section{Evaluation}
\label{evaluation}

\subsection{Simulated Evaluation of Multimodal Reference}
\label{sec:simulatedEvaluation}
To quantitatively evaluate how multimodal reference can help our human-in-the-loop optimization, we designed a simulation test to compare two linear subspace initialization methods: using multimodal reference and random fonts.
\subsubsection{Procedure}
We illustrate the procedure of the simulation test in \autoref{fig:multimodalInputIniitalizationEvaluation}. 
Given a base font character (\eg~``A''), the goal of the simulation test is to resemble the target font character (\autoref{fig:multimodalInputIniitalizationEvaluation}(d)) by exploring the style latent space through optimization.
Specifically, we simulate user selections using the following process. 
At each iteration, our method selects a point in the slider's search subspace with the minimum perceptual metric (we use \textit{DreamSim}~\cite{fu2023dreamsim}) against the target font character.
Then, the selected point is used to request Bayesian optimization to recommend the next linear subspace.
We iterate this process to observe the convergence of the optimization progress using both initialization methods.

For initializing using multimodal reference, we test \textit{text input} and \textit{font file input} in this experiment.
For the text input, we create a descriptive text that characterizes the target font and use it to initialize the search subspace.
For font file input, we manually select a font from candidate fonts that closely resembles the target font and use it to initialize the search subspace.
Finally, for the baseline method, we choose a font randomly from our font database and use it to construct the initial search subspace.

\begin{figure}[ht]
    \centering
    \includegraphics[width=\linewidth]{figures_pdf/synthetic_eval_v7.pdf}
    \caption{
    \textbf{Evaluation of linear subspace initialization methods.}
    We compared two initialization methods for exploration with Bayesian optimization.
    (a) One method uses input text or a similar font file for initialization, while (b) the other initialize method uses a randomly sampled font from a font database.
    After initialization, both methods follow the same automatic exploration process (c), where the optimal point on the single linear subspace is repeatedly identified and submitted to the system.
    In each iteration, we measure the distance between the generated character and the target font character to identify the optimal point, as shown in (d).
    Note that we use the bitmap format of the character for distance calculation, without vectorizing it.
    }
    \label{fig:multimodalInputIniitalizationEvaluation}
\end{figure}
We conducted this experiment using the character ``A'' for $10$ different target fonts, randomly selected from our font database.
We collected $12$ kinds of fonts from which we chose a similar font to each target font for the font file input.
For each target font, we choose the most similar font out of the $12$ candidate fonts.
The $12$ candidate fonts consist of two popular font families, \textit{Roboto} and \textit{NotoSerif}, and each font family has six variations: \textit{Light}, \textit{Light Italic}, \textit{Regular}, \textit{Regular Italic}, \textit{Bold}, and \textit{Bold Italic}.
This selection simulated a scenario where users start with popular fonts and design new fonts based on one of these similar candidates.
For each target font, the optimization process includes $10$ iterations of Bayesian optimization.

\subsubsection{Results}
In \autoref{fig:multimodalInputIniitalizationEvaluationResult}, we show the mean and standard deviation of the distances between the optimized results and all target fonts.
We can observe that the optimization processes with text and font file references converge to a lower \textit{DreamSim} distance to the target font character compared to those initialized with a randomly selected font.
These results indicate that using multimodal references for initializing the human-in-the-loop optimization leads to more effective exploration than random initialization.


\begin{figure}[ht]
    \centering
    \includegraphics[width=0.5\textwidth]{figures_pdf/multimodalInitialixationEvaluationResult_v4.pdf}
    \caption{
    \textbf{Convergence comparison between two initialization methods.}
    The figure illustrates how the \textit{DreamSim} distance between the designed font character and the target font character converges during exploration with human-in-the-loop optimization.
    The optimization processes initialized by text font references (orange) obtain better results compared to processes initialized by random font (blue).
    }
    \label{fig:multimodalInputIniitalizationEvaluationResult}
\end{figure}


\subsection{User Study}
\label{sec:user-study}
To evaluate the effectiveness of our proposed system, we conducted a user study in which participants were asked to design fonts using both a baseline system and our system. 
The goals of this study were threefold: 
\begin{itemize}
    \item to assess the overall effectiveness of our system, including the integration of Bayesian optimization, multimodal reference, history interface, and style propagation.
    \item to compare the fonts designed by participants both qualitatively and quantitatively against those created using the baseline system.
    \item to gather qualitative feedback on the user experience with our system.
\end{itemize}

\subsubsection{Comparison Systems}
For the user study, we added a special feature called \textsc{Font Palette} to our proposed system.
By clicking the \textsc{Font Palette} button, users can view a visualization of the \num{12} popular fonts described in \autoref{sec:simulatedEvaluation} and select one to input as their preference, simplifying the process of inputting a font file.
Additionally, we removed the \textsc{Upload Image} and \textsc{Upload Font} buttons from the UI in \autoref{fig:UI} for simplicity.
As a result, users can now easily input text and font files using the \textsc{Text} and \textsc{Font Palette} buttons, respectively.

To assess the effectiveness of the multimodal reference and style propagation features in our system, we created a baseline system that includes only a single slider, as illustrated in \autoref{fig:baselineSystemUI}.
In this baseline system, users can explore the font style latent space solely by adjusting the slider, guided by the Bayesian optimization process. 
Unlike our proposed system, the baseline’s one-dimensional search space is initialized by connecting a fixed point with a randomly initialized point.
The fixed point corresponds to the style of the \textit{IPAex Gothic} font, as described in \autoref{sec:simulatedEvaluation}. 
If users encounter difficulties during exploration, they can reset their preference history in the Bayesian optimization process and restart from a newly randomized search subspace.
Additionally, this baseline method lacks a style propagation function, requiring users to design each character individually.

\begin{figure}[ht]
    \centering
    \includegraphics[width=0.5\textwidth]{figures_pdf/baseline_ui_v6.pdf}
    \caption{
    \textbf{User interface of the baseline system.}
    In the (a) character design area, users use a slider to explore the one-dimensional subspace within the font style latent space recommended by Bayesian optimization.
    By clicking the \textsc{Reset} button, users can reset their preference history in the Bayesian optimization process, randomly reinitializing the search subspace.
    The users can check the characters that they have already designed are displayed in the (b) character collection area.
}
\label{fig:baselineSystemUI}
\end{figure}

\subsubsection{Procedure}
We recruited ten people for the user study.
Each participant was presented with a target font and asked to design three characters, ``A'', ``B'', and ``C'' that closely match the target font using both \systemName and the baseline system.
For this user study, we prepared two target fonts, Font 1 and Font 2.
Each font design session continued until one of the following conditions was met: (1) the participant was satisfied with the quality of the characters they designed, (2) they felt that further improvement was difficult, or (3) the $7$-minute time limit was reached.
The user study followed this sequence: (Tutorial of \systemName $\rightarrow$ Font 1 with \systemName $\rightarrow$ Font 2 with \systemName $\rightarrow$ Tutorial of baseline $\rightarrow$ Font 1 with baseline $\rightarrow$ Font 2 with baseline $\rightarrow$ Survey).
The order of using \systemName and the baseline system was randomized for each participant.
After the font design sessions, participants were asked to complete a questionnaire that validated our system.
The entire user study took approximately $60$ minutes, with each tutorial lasting $10$ minutes, each font design session $7$ minutes, and the survey $10$ minutes.



\subsubsection{Results and Discussion}
We compared the designed fonts using our system and the baseline system both quantitatively and qualitatively.
For the quantitative evaluation, we calculated the distance between the target font characters and the designed characters in the \textit{DreamSim} latent space.
As shown in \autoref{tab:userStudyResult}(a), the characters designed with our system closely resembled the target font characters compared to those designed with the baseline system.
Additionally, we measured the style consistency between all characters designed by each participant by calculating the mean distance between the characters ``A'', ``B'', and ``C'' in the \textit{DreamSim} latent space.
As shown in \autoref{tab:userStudyResult}(b), the distance is smaller when using our system to the baseline system, indicating our system enables more style-consistent character design.
In \autoref{fig:userStudyResult}, we showed the characters designed by all participants (P1--P10).
For Font 1, the ``A'' characters designed by P2, P3, P4, P6, and P8 using our system closely matched the slanted style of the target ``A,'' while they failed to design the slanted style using the baseline system, indicating that participants effectively captured the italic feature through multimodal reference.
On the other hand, characters designed by P1, P2, P3, P4, P5, P8, and P10 using the baseline system showed inconsistencies in style within the same font (\eg~variations in size, height, and weight).
In contrast, characters designed with our system exhibited greater consistency, suggesting that style propagation helped create more cohesive designs.

\begin{figure}[ht]
    \centering
    \includegraphics[width=\linewidth]{figures_pdf/fontDesignResult_v6.pdf}
    \caption{
    \textbf{Characters designed by user study participants.}
    Our system enables users to design characters that are more similar to the target font characters and maintain higher consistency between each other.
    In the case of Font 1, participants successfully designed all characters with the slant style using our system, while some participants failed to create the slant style for ``A'' using the baseline system.
    For Font 2, all participants designed characters with consistent styles using our system, whereas the styles of characters designed using the baseline system were inconsistent.
}
\label{fig:userStudyResult}
\end{figure}


\aptLtoX[graphic=no,type=html]{\begin{table}[ht]
\centering
\begin{tabular}{lll}
\multicolumn{3}{c}{\bf (a) Target font similarity $\downarrow$~~~~~}\\
\hline
                & Font 1          & Font 2          \\ \hline
Baseline & 0.1680          & 0.1416          \\
\systemName  & \bestcell{0.1591} & \bestcell{0.1355} \\ \hline
\end{tabular}
\begin{tabular}{lll}
\multicolumn{3}{c}{~~~~~\bf (b) Designed character consistency $\downarrow$}\\
\hline
                & Font 1          & Font 2          \\ \hline
Baseline & 0.3303          & 0.2893          \\
\systemName     & \bestcell{0.2983} & \bestcell{0.2793} \\ \hline
\end{tabular}
\caption{
(a) 
We calculated the distance between the characters designed by the participants and the target font characters.
Each value represents the mean distance across the $12$ characters (``A'', ``B'', ``C'' designed by the four participants).
The characters designed using our system are closer to the ground truth compared to those with the baseline system. 
(b)
We measured the character consistency between the characters ``A'', ``B'', and ``C'' designed by each participant.
Each value represents the mean distance across the three characters designed by each participant.
The distance among the three characters designed using our system is smaller than that with the baseline system, which indicates our system enables more style-consistent character design. 
($\downarrow$ denotes the lower values are better and we highlight the \besthint{best} result for each target font.)
}
\label{tab:userStudyResult}
\end{table}}{\begin{table}[ht]
\centering
\subfloat[Target font similarity $\downarrow$]{
\begin{tabular}{lll}
\toprule
                & Font 1          & Font 2          \\ \midrule
Baseline & 0.1680          & 0.1416          \\
\systemName  & \bestcell{0.1591} & \bestcell{0.1355} \\ \bottomrule
\end{tabular}
}
\subfloat[Designed character consistency $\downarrow$]{
\begin{tabular}{lll}
\toprule
                & Font 1          & Font 2          \\ \midrule
Baseline & 0.3303          & 0.2893          \\
\systemName     & \bestcell{0.2983} & \bestcell{0.2793} \\ \bottomrule
\end{tabular}
}
\caption{
(a) 
We calculated the distance between the characters designed by the participants and the target font characters.
Each value represents the mean distance across the $12$ characters (``A'', ``B'', ``C'' designed by the four participants).
The characters designed using our system are closer to the ground truth compared to those with the baseline system. 
(b)
We measured the character consistency between the characters ``A'', ``B'', and ``C'' designed by each participant.
Each value represents the mean distance across the three characters designed by each participant.
The distance among the three characters designed using our system is smaller than that with the baseline system, which indicates our system enables more style-consistent character design. 
($\downarrow$ denotes the lower values are better and we highlight the \besthint{best} result for each target font.)
}
\label{tab:userStudyResult}
\end{table}}

Next, we evaluated participant feedback to validate the effectiveness of our system.
We asked questions about the functions in our system, including slider operation, multimodal reference, style propagation, and history interface.
When we asked the question \textit{``Were you satisfied with the designed characters?''}, seven out of the ten participants answered yes, while P2 and P10 commented neutral, and P9 expressed no.
P9 noted that he observed distortions in the generated characters and felt the system was not good at generating straight lines.
In response to the question \textit{``Do you think you were able to design fonts easily with the system?''}, all ten participants answered yes, demonstrating the system's effectiveness in enabling non-expert users to design fonts with ease.

In response to the question \textit{``Do you think you were able to effectively use the slider operation for font design?''}, nine participants answered yes.
P4, who answered no, expressed dissatisfaction, stating that while the combination of slider manipulation and multimodal reference was effective, using only the slider and repeatedly clicking the \textsc{Update} button sometimes resulted in a linear subspace that excluded the desired character style.
P4 emphasized the importance of using multimodal reference at the right moments to avoid unsatisfactory suggestions and stated that relying solely on the slider was not effective.
P4 also highlighted that the history interface was useful for reverting to a previous point, leading to the escape of an undesirable search subspace suggested by the system.
P4's feedback reflects the findings suggested in Chan~\etal~\cite{Chan2022}, which indicate that designers working with BO may experience a loss of agency.
In contrast, our method provides users with a way to contribute concrete ideas that guide the BO process, thereby helping them regain a sense of agency.

In response to the question, ``\textit{Do you think you were able to effectively use text input?}'', eight of ten participants answered yes.
P1, P2, P4, P6, P7, P9, and P10 found text input helpful for making broad changes, such as adjusting weight or slant, but not for fine-tuning details or specifying complicated characteristics.
Additionally, P1, P6, and P7 mentioned that understanding typographical terms like ``bold'' and ``italic'' was necessary.
This feedback indicates that while text input is useful for exploring rough font styles, it has limitations in designing font details and requires some typographic knowledge.

Regarding the question, \textit{``Do you think you were able to effectively use the similar fonts provided by font palette?''}, eight of ten answered yes.
P4 and P7 commented that the font palette is particularly helpful when it is difficult to describe the desired font style in texts.
P8, P9, and P10 stated that initializing the search subspace using the font palette function allowed them to begin the design task more smoothly compared to the baseline system.
However, P5 expressed dissatisfaction, stating that the style of the character generated did not perfectly align with the font they selected from the font palette.
This discrepancy, caused by the encoding-decoding process of the font generative model, could lead to confusion among users.
To address this issue, it is important to communicate to users that the generated characters may not always perfectly match the multimodal reference.
Additionally, we anticipate that newer font generative models could help mitigate this discrepancy.
It is worth emphasizing that our proposed system is compatible with any font generative model, provided an efficient font style latent space can be established within it.
On the other hand, P2 explained that he did not use the font palette because he preferred to describe the target font style using text input.
This feedback suggests that using similar font files and text input complement each other.

When asked, \textit{``Do you think you were able to effectively use the \textsc{Update all} button?''}, nine participants responded positively, with eight participants noting that it was more convenient than designing each character individually.
P10, who answered no, expressed dissatisfaction, commenting that it would be more convenient if users could toggle between adjusting either all characters at once or individually. 
In particular, he felt that having a feature to switch to individual adjustments is crucial during the fine-tuning stage.
We focus on the simplicity of the UI in this user study and this individual adjustments function is effective especially when designing many characters like all Roman characters.

In response to the question, ``\textit{Do you feel that you could design characters with a sense of agency using our system?}'', posed only to P5--P10, all six participants responded affirmatively.
P7 and P10 noted that in the baseline system, the line search space was initialized randomly, making the process feel highly dependent on luck. 
In contrast, they appreciated that our proposed system allowed them to control the initialization by specifying their preferences through multimodal references.
This insight aligns with the findings in \autoref{sec:simulatedEvaluation}, which show the initialization using multimodal references leads to better results compared to random initialization.
Additionally, P9 commented that he felt he could convey his intentions to the system by inputting texts.
These insights indicate that our system, leveraging multimodal references, provides users with a greater sense of agency compared to the baseline system.

Seven participants highlighted the usefulness of the history interface during the design process.
P5 remarked that the feature was particularly effective, as there were times when he felt a previous font was better.
In such cases, the history interface allowed him to revisit and continue from that point, saving effort.
He also noted the inconvenience of the baseline system lacking this feature.
P6 commented that comparing the current font displayed on the slider with previously created fonts helped him determine which one aligned more closely with his intended design.
He also said that in the baseline system, he found it challenging to reset after creating a satisfactory font.
In contrast, our system's history interface made him feel more confident about updating or resetting, as it allowed him to aim for even better results without hesitation.
This exemplifies that the history interface is useful not only for storing the designed characters and enabling the users to go back to a past point but also for making them advance the design process as boldly as they want.
It also indicates that the history interface reduces stress and increases freedom and creativity in the design task.

Overall, the feedback suggests that the proposed functions in our system effectively support font design for different participants based on their design preferences and familiarity with typography.



\section{Discussion}
\subsection{Giving Human-like Skills to AI} 
This study showed that for one form of humor - Gen-Z style Instagram image captioning humor - our AI-written humor was funnier than GPT's native humorsense, and as funny at the top 5 highest rated Instagram captions. We attribute this to a variety of features we added to the system. First, the visual detail extraction was able to find aspects of the image to poke fun at that were often sharper than GPT's joke target and more similar to the Instagram captions' joke target. Second, the narrative extrapolation step allowed the system to broaden its base of relatable joke targets - moving the focus away from making fun of the literal objects in the image, but using them as metaphors for relatable joke targets like relationship disasters, teamwork breakdowns, and the burden of student loans. This opened more creativity possibilities for joke targets. Lastly, we used an LLM-as-judge to rank the outputs accord to Gen-Z humor taste, thus giving the system some notion of the audience. These skills - detail observation, finding analogous and relatable social situations, and modeling the audience through fine tuning - are all considered somewhat ``human.'' Skills like reasoning and chaining are considered more typical of machines. But this shows that machines might be able to approach these more human skills with the right architecture and training.  


% In this paper, we showed that a model of GPT that is enhanced to have 3 human-like skills used in humor 
% \color{red}
% (observation, sense of story, and in-group knowledge) 
% \color{black}
% outperforms standard GPT-4o. Many other researchers have devised prompting techniques and architectures for improving LLM's reasoning capabilities such as reflection, chain-of-thought, and prompt chaining. However, fewer papers have explored how social skills can enhance LLMs communication abilities. Systems like Generative Agents ~\cite{joon_agents}
% and Character.AI \cite{characterAI} do this to great success. In this paper, we gave VLMs a few simple ``skills'' that were relevant to humor generation and showed that overall, it improved AI's ability to write humor. The focus of this paper was the human evaluation to see whether AI could get closer to parity with most upvoted human captions. However, if even a simple system like this can improve humor, perhaps more sophisticated systems could do better. 

There are many ways to improve the skills in HumorSkills. Building and testing a better Gen Z humor ranking would probably improve the filtering of bad captions. More fine-tuning could improve the breadth of Gen Z slang and references. More narratives and conflicts would expand its vocabulary of relatable situations. Finding ways to automatically collect narratives and conflicts to be applied would accelerate this process. Adding new skills would also be future research. Theories of humor abound. With recent advances in LLM's ability to do long chains of logical reasoning in DeepSeek and GPT-4o, it would be interesting to have AI try to analyze the humor and extract it's own theories or techniques for humor.

One of biggest shortcomings of the captions is that some of them are not logical enough to make sense, but are also not illogical enough to be absurd. These sound like mistakes. As future work, one could test whether an AI-based reflection step could think through the logic of a joke and decide whether it actually made sense or not. 
% The current rating system seems to let some of these by. 

% Other papers have tried fine-tuning b

Further testing or ablation studies could help shed light on which skills are most helpful. However, humor ratings have high variance among raters, and the data required to get statistical significance is often quite high. There may not be an effect of each skill individually - they might only work together. 



\subsection{Implications of Machine with Human-like Social Skills}
Human-like social skills - like humor - are often used for human bonding. If AI can write humor as well as the best people, the AI has the potential to both disingenuously create human bonding ~\cite{diresta2024spammersscammersleverageaigenerated,naaman_opinion} and to augment human's ability to bond~\cite{socialglue}. Either way, this has the potential to change the nature of human trust and communication.
In many ways, this is already happening in other domains. 
ChatGPT and Gmail SmartCompose~\cite{smartcompose} can already rewrite emails to sound more polite and we really are.
AI sales and scams can trick people into giving money to what they think are friend or loved ones in need~\cite{ai_scams}. 
AI has successfully been integrated into Gen Z dating apps that suggest messages to send to potential dates based on both a dating profile (for the opening line) or message history (for continued conversation)~\cite{majic2024rizz}. Many apps attempt this, but the quality of the suggested text sets them apart - the apps that generate more human-like texts have millions of active paying users~\cite{majic2024rizz}.
To some, this potential for disingenuousness is horrifying. Although disingenuous portrayals of oneself for dating purposes far precede the invention of generative AI, there is a possibility that AI will amplify this ability. 


As AI for social, cultural, and personally relevant communication improves, we may need a way to discern genuine from disingenuous communication. There are high-tech ways of doing this, such as making a video of oneself (until AI can do that). There are also low-tech ways of doing this, like talking in person. It would be highly ironic if the advancement of AI drives people to abandon technology, because it could not be trusted to be genuine. 

% For people who adopt these products, a common reason is that the bar for texting banter is so high for Gen Z, that help is appreciated, even when the sentiment is genuine. 

% Even at work, polite communication is socially demanded, and with ever-increasing amounts of communication, the emotional labor of even typing simple pleasantries is tiring. Early generative AI applications like Gmail Smart Compose~\cite{smartcompose} were noted for lowering the burden of writing a polite introductory line in emails. Although these lines are perfunctory and don't necessarily need to be genuine, it makes a difference to readers whether they are there or not. Social effort matters.



% \textit{Although hopefully AI will not force civilization back into a barter economy that necessatiate personal interaction to establish trust.} (LYDIA: TOO MUCH?)


% emotional labor. 

% Also with AI friends like Persona aI?



\section{Limitations}
This study targeted only one form of humor for only one audience:  Gen Z humor Instagram captions. This type of humor tends toward absurdities, which can be easier to generate than something that needs to be logically sound. Being illogically surprising is probably easier than being logical and surprising. Future work would have to test whether similar techniques work on other humor tastes. Some of our techniques, like fine-tuning, would likely work generically for all humor types, but other skills might need to be tried.

The caption humor is difficult, but it is more well-defined than other forms of humor. Caption humor only requires a punchline for a given image (the setup). Other forms of humor like standup comedy and popular humor magazines require generating both the setup and the punchline. A future direction is to explore what additional "skills" are needed to generate jokes with both setup and punchline. 

The humor generated here is for a public audience, but most humor made spontaneously is made for friends, and often users insider knowledge about the friends, their background, and their shared experiences. LLMs would likely struggle to make in-joke humor without a source of inside information to train on.

In our baseline captions, we crafted a simple prompt for GPT-4o to write humor. It is possible that with a better prompt or multiple generations, one could generate similar results. However, that is effectively what the system performs. It might be possible that there are prompts that don't employ any humor skills that can also generate jokes funnier than baseline GPT. Future work should test more prompts - both with and without skills to see if there are approaches other than skills that can enhance LLM humor generation.

% ensuring better generations and more consistent 







\input{08-limitations}
\section{Conclusion}



This paper introduces \sysname, an AI-assisted system designed to enhance the process of visual blend ideation by leveraging metaphors. 
%Our system utilizes large language models and commonsense knowledge bases to explore objects and their associated attributes, forming metaphorical connections with abstract concepts.
Our system utilizes LLMs and commonsense knowledge bases to explore objects and their associated attributes, forming metaphorical connections with abstract concepts. 
It offers the capability to automatically generate blending proposals based on user selections, facilitating rapid creative realization for verification through the T2I model.
To evaluate the system, we conducted a user study involving 24 participants who had AI experience. The findings demonstrate that \sysname\ has the potential to enhance the creativity of the generated ideation results and enable the expression of abstract concepts more metaphorically.
Additionally, this research offers insights into user preferences regarding visual blend design and potential future approaches for supporting design with generative AI.



\bibliographystyle{ACM-Reference-Format}
\bibliography{references}

\clearpage
\appendix
\onecolumn
\clearpage
\begin{appendices}

\section{Production Fault Trace}
\label{appendix:production-fault-trace}
The production fault trace was collected from an 8-GPU node pretrain cluster with 2880 GPUs over a period of 160 days. The trace includes details such as fault start time, fault end time, and the ID of the faulty node. \figref{fig:simulation:trace:timetrace} and \figref{fig:simulation:trace:cdf} provide a macro-level overview of the production fault trace. On average, the ratio of faulty 8-GPU nodes at any given time is $3.83\%$, with a p99 value of $7.22\%$.

\begin{figure}[h!t]
    \centering
    \begin{subfigure}[b]{0.23\textwidth}
        \centering
        \includegraphics[width=\textwidth]{figs/evaluation/fault_server_ratio.pdf}
        \caption{Fault Node Ratio Trace.}
        \label{fig:simulation:trace:timetrace}
    \end{subfigure}
    \hspace{2pt}
    \begin{subfigure}[b]{0.23\textwidth}
        \centering
        \includegraphics[width=\textwidth]{figs/evaluation/fault_server_cdf.pdf}
        \caption{Cumulative Distribution.}
        \label{fig:simulation:trace:cdf}
    \end{subfigure}
    \vspace{-2ex}
    \caption{Fault node trace in the production AI DC.}
    \label{fig:simulation:trace}
\end{figure}

Since most of failure events are GPU faults, we normalized the trace of 8-GPU nodes to generate 4-GPU nodes trace. Assuming that the fault rates of GPUs are i.i.d. with a fault probability of $p$ for each GPU, and considering that a node is deemed faulty if any GPU within it fails, the fault rate of an 8-GPU node is calculated as:  

\vspace{-1em}
$$
P_{fault}(8\text{-GPU}) = 1 - (1-p)^8 = 3.83\%.
$$  

From this, we derive $p = 0.49\%$. The fault rate for a 4-GPU node is then:  
$$
P_{fault}(4\text{-GPU}) = 1 - (1-p)^4 = 1.93\%.
$$  

The fault event of 4-GPU node is generate with Bayesian Equation, as:


\begin{align*}\label{eq:convert-trace}
& P_{fault}( \text{4-GPU} \mid  \text{8-GPU})\\ 
    &=\frac{P_{fault}(\text{8-GPU} \mid \text{4-GPU}) P_{fault}(\text{4-GPU})}{P_{fault}(\text{8-GPU})} \\ 
    & =  \frac{1 \times 1.93\%}{3.83\%} = 50.39\% \\
\end{align*}

Thus, whenever a fault occurs in an 8-GPU node in the original trace, each of the two corresponding 4-GPU nodes at the same location has a $50.39\%$ probability of fault. This method is used to convert the traces.

As node faults are i.i.d., the simulator linearly maps the fault trace to different network architectures.

\section{GPT-MoE Architecture}
\label{appendix:gpt-moe}
This model is a mixture-of-experts (MoE) model with the following configuration:

\para{Model Configuration:}
\begin{itemize}
    \item \textbf{Number of Layers:} 192
    \item \textbf{Inner Layer Dimension:} 49152
    \item \textbf{Embedding Dimension:} 12288
    \item \textbf{Hidden Dimension:} 12288
    \item \textbf{Vocabulary Size:} 64000
    \item \textbf{Number of Attention Heads:} 128
    \item \textbf{Maximum Sequence Length:} 2048
    \item \textbf{Number of Experts:} 8
    \item \textbf{MoE Layer Ratio:} 0.5
    \item \textbf{Top-K Experts:} 2
\end{itemize}

\para{Runtime Configuration:}
\begin{itemize}
    \item \textbf{Virtual Pipeline Parallelism:} 3
    \item \textbf{Micro Batch Size:} 1
    \item \textbf{Global Batch Size:} 1536
    \item \textbf{Max Sequence Length:} 2048
\end{itemize}




\section{Theoretical analysis of wasted GPU ratio for \sys}
\label{appendix:ft-anay}

The count of backup lines as $2K - 2$ will significantly influence the fault tolerance of \sys. We use the expectation of waste ratio caused by GPU failure and fragmentation problem to evaluate this design, the result is shown in \tabref{table:design:1.5ratio}.

For one single working server in the middle of line, the count of breakpoints $B$ on its two sides has the expectation as:

\vspace{-1em}
\begin{equation*}
E_B(\eta = 1,middle) = 2(P_s^K + P_s^{2K})
\end{equation*}

Where $P_s$ is the fail probability of GPU server, and $\eta$ is count of servers. The expectation of breakpoints count is:

Once the distance between one server and the tail of line is $\alpha < K$, it will connect to all servers between itself and the last one, so there will be no breakpoints on this side, and the expectation of breakpoints count is less than servers in the middle of line.
Then, for any server in the line topology:

\vspace{-1em}
$$
E_B(\eta = 1) \leq E_B(\eta = 1,middle) 
$$

When the distance between two servers is $\beta \geq K$, the breakpoints among them can be calculated as independent.
Once the distance $\beta < K$, as all servers in this range are connected to these two servers, there will be no breakpoints between them. So, the expectation is less than two independent servers. Then,



\vspace{-1em}
\begin{align*}
E_B(\eta =& 2) < E_B(\eta = 2, \beta \geq K) =  2E(\eta = 1)   \\ 
 E_B(\eta =& N_s) \leq N_s E_B(\eta = 1) 
\end{align*}

For a LLM job which require a ring communication size (TP .etc) as $N_t$, \sys   will cut the whole line topology into several sub lines with the length of $N_t/R$.
Once \sys is cutting a new sub line from the remaining servers in the line, 
all $N_t$ GPU will be wasted when one break point exist in the middle of this sub line required, shown in \fig{fig:subline-waste}. 
Then the expectation for waste GPU caused by one single break point is:

\vspace{-1em}
$$
E_W(B=1) = N_t R\cdot (1 - (N_t/R)^{-1} ) = R(N_t -R)
$$

\begin{figure}[h!t]
    \centering
    \includegraphics[width=0.8\linewidth]{figs/design/intra-topo/break-topo.drawio.pdf}
    \caption{Break point can cause server waste compare to ideal situation.}
    \vspace{-1em}
    \label{fig:subline-waste}
\end{figure}

As the influence between two break points only reduce the expectation of wasted GPUs, we can have this for $X$ break points:

\vspace{-1em}
\begin{equation*}
E_W(B = X) \leq XE_W(B=1) = XR(N_t-R)
\end{equation*}

So the expectation of wasted GPU for a servers cluster with $N_s$ GPU servers is:

\vspace{-1em}
\begin{align*}
E_W(\eta = N_s) &\leq \sum P(B=X ,\eta = N_s) \cdot X\cdot  E_W(B=1)\\
&= E_B(\eta = N_s)\cdot E_W(B=1)\\
&\leq  \lim_{P_s\rightarrow 0}2N_s\cdot R \cdot (N_t-R)P_s^K
\end{align*}



The final expectation of GPUs waste ratio is \eqref{eq:design:ratio}:

\begin{equation}
E_{WR}(\eta = N_s) = \frac{E_W(\eta = N_s)}{N_g} \leq 2(N_t-R)(P_s)^K
\label{eq:design:ratio}
\end{equation}

In our trace for a 160 days long pre-train job on 10K-GPU, the p99 failure rate for 8-card machines is 7\%. If a TP32 jobs is running on \sys, we can get the upper bond for waste ratio expectation for various configuration in \tabref{table:design:1.5ratio}.

\begin{table}[h!t]
\centering
\begin{tabular}{cccc}
    \toprule
        & $K=2$&$K=3$&$K=4$\\
    \midrule
     R=4& $7.35\%$ & $0.26\%$ & $9.00\times 10^{-4}$ \\
     R=8& $27.4\%$ & $1.92\%$ & $0.13\%$ \\
     \bottomrule
\end{tabular}
\caption{Upper bond for waste ratio expectation of GPU, where GPU failure rate is 0.875\% and X is 32}
\vspace{-2em}
\label{table:design:1.5ratio}
\end{table}

As shown in the table, for 4 GPU server ($R=4$) 3 bundles ($K = 3$) design, the additional waste of GPU is less than 0.26\%, while the waste ratio for $R=8,K=4$ is less than 0.13\%. This is sufficient for production clusters. 

\section{Orchestration For Fat-Tree}
\label{appendix:orch-algo}
In this section, we introduce the orchestration algorithm under Fat-Tree DCN in detail.

\para{Notations}
\label{appendix:orch-algo:notation}
To ensure rigorous mathematical reasoning, we introduce the following notations:

\begin{itemize}
    \item {
        $n$: number of nodes in the data-center.
    }
    \item {
        $K$: \docs{} bundle (see \S\ref{section:design:topology}).
    }
    \item {
        $S_{all}$: ordered set, represents all nodes numbered from 1 according to their physical connection order in DCN fabric. $|S_{all}|=n$.
    }
    \item {
        $S$: ordered subset, represents nodes, $\forall u \in S, u \in S_{all}$. Adjacent elements in $S$ are also adjacent from the perspective of the \SYS{} topology. 
    }
    \item{
        $E$: The set of edges across $S$, should be equal to $\{ (S_i, S_j) \mid 1 \leq i < j \leq n, j - i \leq K \} $, representing the connections between nodes, including both primary and backup links, and $O(|E|) = O(K|S|)$.
    }
    \item {
        $InfHBD=<S,E>$: the topology of \SYS{} as an undirected graph.
    }
    \item {
        $F$: faulty nodes.
    }
    \item {
        $HealthyHBD=<H,HE>$: healthy node subgraph where the set of healthy nodes $H = S - F$ and the edge set $HE = \{ (u, v) \mid u \in H \text{ and } v \in H \text{ and } (u, v) \in E \}$.
    }
    \item{
        $t$: TP size, number of GPUs in one TP Group.
    }
    \item{
        $r$: GPU ranks per node.
    }
    \item{
        $m=t/r$: number of nodes in a TP group.
    }
    % \item{
    %     $k$: number of rails in rail-optimized network.
    % }
    \item{
        $s$: job scale, number of GPUs required for the job.
    }
    \item{
        $d$: Aggregation-Switches Domain size. Number of nodes under coverage of one group of Aggregation-Switches.
    }
    \item{
        $n_{constrains}$: number of applied constraints in binary-search-based orchestration algorithm.
    }
    \item{
        $p$: number of nodes under each ToR.
    }
    \item{
        $l$: shortest sub-line length under fat-tree orchestration.
    }
    \item{
        $n_{maxsubline}=\lfloor \frac{nd}{p} \rfloor$: max number of sub-lines.
    }
    \item{
        $G_{deploy}=<S_{deploy},E_{deploy}>$: deployed topology. After applying the deployment strategy, the topology from the perspective of \SYS{} is described as follows: $S_{\text{deploy}}$ is an ordered set where adjacent elements correspond to adjacent nodes in \SYS{}, and $E_{\text{deploy}}$ represents the connections between nodes.
    }
    
\end{itemize}


% For the \SYS{} the orchestration algorithm in ideal conditions is relatively straightforward. The detailed steps of the algorithm are outlined in \algref{alg:orchestration-ideal}.

% Assume that the \SYS{}(with \docs{} direction $K$) is represented as an undirected graph $ \text{InfHBD} = \langle S, E \rangle $, where the ordered set of nodes $ S $ represents nodes. Adjacent elements in $S$ are also adjacent from the perspective of the \SYS{} topology. The set of edges $E$ should be equal to $\{ (S_i, S_j) \mid 1 \leq i < j \leq n, j - i \leq K \} $, representing the connections between nodes, including both primary and backup links, and $O(|E|) = O(K|S|)$. The set of faulty nodes is denoted as $ F \subseteq S $.

% The algorithm proceeds as follows:

% \begin{enumerate}
%     \item {\textbf{Extract the Healthy Node Subgraph:} First, extract the subgraph $\text{HealthyHBD} = \langle H, HE \rangle$ where the set of healthy nodes $H = S - F$ and the edge set $HE = \{ (u, v) \mid u \in H \text{ and } v \in H \text{ and } (u, v) \in E \}$. See \algref{alg:orchestration-ideal}.
%     }
%     \item {\textbf{Identify Connected Components:} Next, identify all connected components in the graph $\text{HealthyHBD}$. Faulty nodes may cause disconnections in the \SYS{} fabric, splitting the original cluster into multiple sub-HBDs. These sub-HBDs are the connected components, and TP Groups cannot span across these disconnected sub-HBDs. We use a simple Depth-First Search (DFS) algorithm here. See \algref{alg:dfs}.}
%     \item {\textbf{Generate Placement Scheme:} Given the excellent physical properties of the \SYS{}, TP Groups can be arranged sequentially within each connected component to generate placement scheme maximizing GPU utilization. See \algref{alg:orchestration-ideal}.
%     }
% \end{enumerate}

% Since each of the three steps involves traversing the entire graph's edges and nodes only once, 
The orchestration algorithm (\algref{alg:orchestration-ideal}) without considering DCN has the overall time complexity $3\cdot O(|H| + |HE|) = O(|S| + |E|) = O((K+1)|S|) = O(|S|)$.

% \begin{algorithm}[!h]
% \small
% \caption{Connected-Component-DFS}
% \label{alg:dfs}
% \SetAlgoNlRelativeSize{-1}
% \SetAlgoNlRelativeSize{1}
%  \KwIn{ $node$, $HealthyHBD$, $visited$}
%  \KwOut{ $component$}

%  Initialize $stack = [node]$ \;
%  Initialize $component = []$\;

% \While{ stack is not empty}
% {
%      $current = stack.pop()$\;
%     \If{$current$ not in $visited$}
%     {
%          Add $current$ to $visited$\;
%          Add $current$ to $component$\;
%         \For{ each neighbor in $HealthyHBD.neighbors(current)$}
%         {
%              $stack.push(neighbor)$\;
%         }
%     }
% }
        
% \KwRet{$component$}
% \end{algorithm}

\begin{algorithm}[!h]
\small
\caption{Orchestration-DCN-Free}
\label{alg:orchestration-ideal}
\SetAlgoNlRelativeSize{-1}
\SetAlgoNlRelativeSize{1}
\KwIn{$\text{InfHBD}=\langle S, E \rangle$, $F$, $m$}
\KwOut{ Placement scheme maximizing GPU utilization}

 Initialize $H = S - F$\;
 Initialize $HE = \{ (u, v) \mid u \in H \text{ and } v \in H \text{ and } (u, v) \in E \}$\;
 Create subgraph $HealthyHBD = \langle H, HE \rangle$\;
 Initialize $component\_list = []$\;
 Initialize $visited = \{\}$\;
 Initialize $placement\_scheme= \{\}$\;

\For{ each node $s$ in $H$}
{
    \uIf{ $s$ not in $visited$}
    {
         $component = Connected-Component-DFS(s, HealthyHBD, visited)$\;
         Add $component.sortedinHBD()$ to $component\_list$\;
    }
}
\For{ each $component$ in $component\_list$}
{
    \While{ $component.size()\geq m$}
    {
         Add $component.pop(m)$ to $placement\_scheme$\;
    }
}
        
 \KwRet{$placement\_scheme$}
 \end{algorithm}
 
% \subsection{Algorithms under Rail-Optimized Network}
% \label{appendix:orch-algo:rail-optimized}

% This subsection provides a detailed description of the orchestration algorithm for Rail-Optimized network.  

% The rail-optimized network topology is specifically designed for highly regular machine learning workload traffic patterns, making it a commonly used and effective architecture. As illustrated in \fig{fig:rail-topo}, Rail Switch $i$ connects to GPU $i$ in node, dividing the network into multiple rails. Let $r$ denote the GPU ranks per node, and $k$ the number of rails. In traditional rail-optimized networks, $k = r$, and a typical training strategy involves running TP $r$ within the single-node HBD, while DP operates between HBDs. Since in DP, GPUs only communicate with GPUs of the same rank in different TP groups, in other words, DP traffic is confined to the rail itself. Therefore, the Rail-Optimized topology perfectly meets this requirement.

% % \begin{figure}[!h]
% %     \centering
% %     \includegraphics[width=\linewidth]{figs/design/Orchestration/rail-optimized.drawio.pdf}
% %     \caption{Rail-Optimized Network: GPU ranks per node $r=4$, Number of rails $k=8$, Aggregation-Switches Domain size $d$, Number of Aggregation-Switches Domain $nd$, Node IDs from 1 to $nd\cdot d$. }
% %     \label{fig:rail-topo}
% % \end{figure}

% \para{Orchestration Constraints. }To minimize the cross-rail traffic which can lead to congestion and latency, the rail-optimized network introduces two key constraints for orchestration algorithms:


% \begin{itemize}
%     \item {
%         \textbf{Aggregation-Switches Domain Coverage Constraint. }
%         The coverage domian of a group of Aggregation-Switches is limited, meaning that TP groups spanning across Aggregation-Switches domains would result in cross-rail traffic, which should be avoided as much as possible.
%     }
%     \item {
%         \textbf{Node Rail State Constraint. }When$ k = r$, this constraint does not apply, as there is no cross-rail traffic.However, as HBDs extend beyond single nodes and the need for larger DP scales due to the expansion of LLM scale, scenarios with $k = p \cdot r$ may arise. This results in $p$ different node states within the data center, with each state occupying $r$ rails, and inter-state communication leads to cross-rail traffic. The specific form of this constraint depends on the deployment strategy.
%     }
% \end{itemize}

% \para{Deployment Strategy. }If the \SYS{} connections continue to follow the physical layout of nodes on the DCN Fabric, avoiding cross-rail traffic would require each TP Group to have an equal number of nodes from each state, making the algorithm to maximize GPU Utilization NP-Complete (see Appendix.\ref{appendix:np-hard-orchestration}). However, by altering the physical connection sequence of \SYS{}, this NP-Complete problem can be reduced to polynomial time. As shown in \fig{fig:parallel-line}, nodes of each state are arranged into $p$ parallel sub-lines, which are then connected end-to-end to form a single line. By restricting DP to operate within sub-lines, all DP traffic remains within the rails, effectively reducing the $k = p * r$ scenario to $k = r$. 

% % \begin{figure}[!h]
% %     \centering
% %     \includegraphics[width=\linewidth]{figs/design/Orchestration/parallel-line.drawio.pdf}
% %     \caption{The deployment strategy example with $p=4$ and Aggregation-Switches Domain size $K=8$. Node IDs from 1 to n are arranged according to their connection order in the DCN Fabric.}
% %     \label{fig:parallel-line}
% % \end{figure}

% \para{The binary search-based Orchestration algorithm.} Based on the above-mentioned constraints and the deployment strategy, we developed an orchestration algorithm that maximizes the number of constraints satisfied while meeting the job scale requirements. This is achieved using a binary search approach with the number of satisfied constraints as the variable. Both types of constraints essentially involve splitting the Line into sub-lines. Therefore, controlling the number of constraints translates to managing the number of sub-lines: fewer sub-lines mean longer sub-lines, leading to higher GPU Utilization. Since the Ideal orchestration algorithm with complexity $O(n)$ can be applied within sub-lines.

% \algref{alg:orchestration-fat-tree} is the main binary-search-based orchestration algorithm. It begins by generating the topology from the perspective of \SYS{} based on the hardware deployment strategy (\algref{alg:deployment-strategy}). Using the number of satisfied constraints as a variable, the algorithm performs a binary search to identify the placement scheme that maximizes the number of satisfied constraints while meeting the job scale requirements.  

% \algref{alg:placement-rail-optimized} calculates the placement scheme for a given number of constraints. It divides the topology into multiple ideal sub-lines and applies the ideal-case orchestration algorithm (\algref{alg:orchestration-ideal}) to each sub-line.  

% Since the time complexity of \algref{alg:orchestration-ideal} is $O(|S|)$, the complexity of \algref{alg:placement-rail-optimized} is 

% \begin{align*}
% &\sum_{i=1}^{n_{constraints}} O(|S_{subline}|) \\
% &= O(\sum_{i=1}^{n_{constraints}} |S_{subline}|) \\
% &= O(|S_{all}|) = O(n)
% \end{align*}

% Thus, the overall time complexity of \algref{alg:orchestration-rail-optimized} is $O(n \log n)$.

\begin{algorithm}[!h]
\small
\caption{Deployment-Strategy}
\label{alg:deployment-strategy}
\SetAlgoNlRelativeSize{-1}
\SetAlgoNlRelativeSize{1}
 \KwIn{Node ordered set $S$, \docs{} direction $K$, parallel factor $p$}
 \KwOut{Deployment topology $G_{deploy}=<S_{deploy},E_{deploy}>$}
 Initialize ordered set $S_{deploy}=[]$\;
 Initialize $l=\lfloor \frac{|S|}{p}\rfloor$\;
\For{$i$ in $0...p-1$}
{
    \For{$j$ in $0...l-1$}{
         Add $i+j\cdot p$ to $S_{deploy}$\;}
}
 Create $E_{deploy}=\{(S_{deploy}^i,S_{deploy}^j)|1\leq i\le j\leq |S_{deploy}|, j-i\leq K \}$\;
 \KwRet{$G_{deploy}=<S_{deploy},E_{deploy}>$}
\end{algorithm}


% \begin{algorithm}[!h]
% \small
% \caption{Placement-Rail-Optimized}
% \label{alg:placement-rail-optimized}
% \SetAlgoNlRelativeSize{-1}
% \SetAlgoNlRelativeSize{1}
%  \KwIn{Deployment topology $G_{deploy}=<S_{deploy},E_{deploy}>$, Number of applied constraints $n_{constraints}$, Faulty node $F$, Sub-line length $l$, Number of node in one TP group $m$}
%  \KwOut{Placement scheme}
%  Initialize $placement\_scheme=\{\}$\;
% \For{$i$ in $1..n_{constraints}$}
% {
%      $S_{subline}=S_{deploy}.pop(l)$\;
%      $E_{subline}=\{(u,v)\mid u\in S_{subline} \text{ and } v\in S_{subline} \text{ and } (u,v)\in E_{subline}\}$\;
%      $F_{subline}=F\cap S_{subline}$\;
%      $placement\_scheme=placement\_scheme\cup \text{Orchestration-Ideal}(<S_{subline},E_{subline}>, F_{subline}, m)$\;
% }
%  $E_{res}=\{(u,v)\mid u \in S_{deploy} \text{ and } v \in S_{deploy} \text{ and } (u,v) \in E_{deploy}\}$\;
%  $F_{res}=F\cap S_{deploy}$\;
%  $placement\_scheme=placement\_scheme\cup \text{Orchestration-Ideal}(<S_{deploy},E_{res}>, F_{res},m)$\;
%  \KwRet{$placement\_scheme$}
% \end{algorithm}


% \begin{algorithm}[!h]
% \small
% \caption{Orchestration-Rail-Optimized}
% \label{alg:orchestration-rail-optimized}
% \SetAlgoNlRelativeSize{-1}
% \SetAlgoNlRelativeSize{1}
%  \KwIn{Node ordered set $S$ (from 1 to n in DCN Fabric), GPU ranks per node $r$, Number of rails $k$, Faulty set $F$, TP size $t$, Job scale $s$ (number of GPUs required for the job), Aggregation-Switches Domain size $d$, \docs{} directions $K$.}
%  \KwOut{Placement scheme that satisfies job scale and minimizes cross-rail traffic.}
%  Initialize $p=k/r$, $m=t/r$, $n=|S|$, $l=\lfloor \frac{d}{p}\rfloor$\;
%  Create graph $G_{deploy}=<S_{deploy},E_{deploy}>=\text{Deployment-Strategy}(S,K,p)$\;
%  Initialize $high=\lfloor\frac{nd}{p}\rfloor$\;
%  Initialize $low=0$\;
%  Initialize $placement\_scheme=\{\}$\;
% \While{ $low \leq$ high}
% {
%      $mid=\lfloor \frac{low+high}{2} \rfloor$\;
%      $placement\_scheme=\text{Placement-Rail-Optimized}(G_{deploy},mid,F,l,m)$\;
%     \eIf {$|placement\_scheme|\cdot m\cdot r\ge s$}
%     {
%          $low=mid+1$\;
%     }
%     {
%          $high=mid-1$\;
%     }
% }
    
% \eIf{$|placement\_scheme|\cdot m\cdot r\ge s$}
% {
%   \KwRet {$placement\_scheme$}
% }
% {
%     \KwRet {None}
% }
% \end{algorithm}
  

Fat-Tree topology is another common data center topology. A typical training strategy for this topology aims to maximize the bandwidth utilization under ToR (Top of Rack) Switches. Using Meta's two-stage clos topology\cite{sigcomm2024meta} as a reference, it can be observed that there is an attempt to run CP under ToR.

\para{Deployment Strategy:} Assuming there are $p$ nodes under each ToR, nodes with the same index under each ToR are deployed along the same parallel sub-line, and the $p$ sub-lines are connected end-to-end, as shown in \fig{fig:fat-tree-topo}. The training strategy involves running CP $p$ across the sub-lines and running TP within them.

\para{Orchestration Constraints. }To maximize the utilization of ToR bandwidth and minimize cross-ToR traffic, the fat-tree topology introduces two constraints:

\begin{packeditemize}
    \item {
        \textbf{Aggregation-Switches Domain Constraint: }The coverage domian of a group of Aggregation Switches is limited, meaning that TP groups spanning across Aggregation Switches domains would result in cross-rail traffic, which should be avoided as much as possible.
    }
    \item {
        \textbf{TP Group Alignment Constraint: } A CP Group consists of TP Groups across parallel sub-lines. To keep CP traffic within the ToR, the TP Groups must be aligned. If a node fails under one ToR, all nodes under that ToR are considered failed, expanding the failure radius by a factor of $p$. 
    }
\end{packeditemize}

\para{Binary-Search-Based Orchestration Algorithm.} Based on the constraints and deployment strategy, we develop a binary search orchestration algorithm (see \algref{alg:orchestration-fat-tree}) that adjusts the number of satisfied constraints. The binary search first relaxes the TP Group alignment constraints within the Aggregation-Switches Domain and then relaxes the TP Group crossing constraints between Aggregation-Switch domains (see \algref{alg:placement-fat-tree}). This process is monotonic.


% \begin{figure}[!h]
%     \centering
%     \includegraphics[width=\linewidth]{figs/design/Orchestration/meta-topo.drawio.pdf}
%     \caption{Orchestration example for Fat-Tree Topology under single Aggregation-Switches Domain with $p=2$. Green indicates active node, red indicates faulty node and yellow indicates idle nodes}
%     \label{fig:meta-topo}
% \end{figure}


The time complexity of \algref{alg:orchestration-ideal} is $O(|S|)$, and the complexity of \algref{alg:placement-fat-tree} is 

$$\sum_{i=1}^{n_{subline}} O(|S_{subline}|) = O(\sum_{i=1}^{n_{subline}} |S_{subline}|) = O(|S_{all}|) = O(n)$$  

Thus, the overall time complexity of \algref{alg:orchestration-fat-tree} is $O(n \log n)$.

\begin{algorithm}[!h]
\small
\caption{Placement-Fat-Tree}
\label{alg:placement-fat-tree}
\SetAlgoNlRelativeSize{-1}
\SetAlgoNlRelativeSize{1}
 \KwIn{$G_{deploy}=<S_{deploy},E_{deploy}>$, $n_{constraints}$, $F$, $l$, $m$, $n_{maxsubline}$, $d$, $p$}
 \KwOut{Placement scheme}
 Initialize $placement\_scheme=\{\}$\;
 Initialize $n_{align}=max(0,n_{constraints}-n_{maxsubline})$, $n_{subline}=min(n_{maxsubline},n_{constraints})$\;
 
\For{$i$ in $0..n_{align}-1$}
{
    \For{$j$ in $1..d$}
    {
        $sid=i*d+j$\;
        \If{$sid \in F$}
        {
            $F\cup \{\lfloor \frac{sid-1}{p}\rfloor\cdot p+1..(\lfloor \frac{sid-1}{p}\rfloor+1)\cdot p \}$\;
        }
    }
}
\For{$i$ in $1..n_{subline}$}
{
     $S_{subline}=S_{deploy}.pop(l)$\;
     $E_{subline}=\{(u,v)\mid u\in S_{subline} \text{ and } v\in S_{subline} \text{ and } (u,v)\in E_{subline}\}$\;
     $F_{subline}=F\cap S_{subline}$\;
     $placement\_scheme=placement\_scheme\cup \text{Orchestration-Ideal}(<S_{subline},E_{subline}>, F_{subline}, m)$\;
}
 $E_{res}=\{(u,v)\mid u \in S_{deploy} \text{ and } v \in S_{deploy} \text{ and } (u,v) \in E_{deploy}\}$\;
 $F_{res}=F\cap S_{deploy}$\;
 $placement\_scheme=placement\_scheme\cup \text{Orchestration-Ideal}(<S_{deploy},E_{res}>, F_{res},m)$\;
 \KwRet{$placement\_scheme$}
\end{algorithm}

\begin{algorithm}[!h]
\small
\caption{Orchestration-Fat-Tree}
\label{alg:orchestration-fat-tree}
\SetAlgoNlRelativeSize{-1}
\SetAlgoNlRelativeSize{1}
 \KwIn{$S$, $r$, $p$, $F$, $t$, $s$, $d$, $K$.}
 \KwOut{Placement scheme that satisfies job scale and minimizes cross-rail traffic.}
 Initialize $m=t/r$, $n=|S|$, $l=\lfloor\frac{d}{p}\rfloor$\, $n_{domain}=\lfloor\frac{n}{d}\rfloor$, $n_{maxsubline}=\lfloor\frac{nd}{p}\rfloor$\;
 Create graph $G_{deploy}=<S_{deploy},E_{deploy}>=\text{Deployment-Strategy}(S,K,p)$\;
 Initialize $high=n_{domain}+n_{maxsubline}$\;
 Initialize $low=0$\;
 Initialize $placement\_scheme=\{\}$\;
\While{ $low \leq$ high}
{
     $mid=\lfloor \frac{low+high}{2} \rfloor$\;
     $placement\_scheme=\text{Placement-Fat-Tree}(G_{deploy},mid,F,l,m,n_{maxsubline},d,p)$\;
    \eIf {$|placement\_scheme|\cdot m\cdot r\ge s$}
    {
         $low=mid+1$\;
    }
    {
         $high=mid-1$\;
    }
}
    
\eIf{$|placement\_scheme|\cdot m\cdot r\ge s$}
{
    \KwRet {$placement\_scheme$}
}
{
    \KwRet {None}
}
\end{algorithm}





\section{Additional Simulation Results for Fault Resilience}
\label{appendix:wasted-GPUs-ratio}
This section presents additional simulation results related to \S\ref{sec:simulation:fault}. \figref{fig:simulation:wasted-trace} shows the variation of the GPU waste ratio over time under the production fault trace. \figref{fig:simulation:waste-cdf:gr4:supple} presents the CDF data for the GPU waste ratio. \figref{fig:simulation:model:wasted-gr4} illustrates the waste GPU ratio for different HBD architectures under various node failure rates, including the results for TP-8 to TP-64. \figref{fig:simulation:breakdown-duration-supple} shows the proportion of job-fault waiting time relative to total time for different job scales. All the aforementioned experiments include results for TP-8, TP-16, TP-32, and TP-64 configurations.








\begin{figure*}[h!t]
    \centering
    \begin{subfigure}[b]{0.23\linewidth}
        \centering
        \includegraphics[width=\linewidth]{figs/evaluation/fault_trace_based/frag_trace_tp8_gr4.pdf}
        \caption{TP-8.}
        \label{fig:simulation:wasted-trace:tp8-4gpu}
    \end{subfigure}
    \hspace{2pt}
    \begin{subfigure}[b]{0.23\linewidth}
        \centering
        \includegraphics[width=\linewidth]{figs/evaluation/fault_trace_based/frag_trace_tp16_gr4.pdf}
        \caption{TP-16.}
        \label{fig:simulation:wasted-trace:tp16-4gpu}
    \end{subfigure}
    \hspace{2pt}
    \begin{subfigure}[b]{0.23\linewidth}
        \centering
        \includegraphics[width=\linewidth]{figs/evaluation/fault_trace_based/frag_trace_tp32_gr4.pdf}
        \caption{TP-32.}
        \label{fig:simulation:wasted-trace:tp32-4gpu}
    \end{subfigure}
    \hspace{2pt}
    \begin{subfigure}[b]{0.23\linewidth}
        \centering
        \includegraphics[width=\linewidth]{figs/evaluation/fault_trace_based/frag_trace_tp64_gr4.pdf}
        \caption{TP-64.}
        \label{fig:simulation:wasted-trace:tp64-4gpu}
    \end{subfigure}

    \vspace{-1ex}
    \caption{GPU waste ratio over production fault trace, 4 GPU node.}
    \label{fig:simulation:wasted-trace}
\end{figure*}


\begin{figure*}[h!t]
    \centering
    \begin{subfigure}[b]{0.23\linewidth}
        \centering
        \includegraphics[width=\linewidth]{figs/evaluation/fault_trace_based/cdf_trace_waste_tp8_gr4.pdf}
        \caption{TP-8.}
        \label{fig:simulation:waste-cdf:tp8-gr4}
    \end{subfigure}
    \hspace{2pt}
    \begin{subfigure}[b]{0.23\linewidth}
        \centering
        \includegraphics[width=\linewidth]{figs/evaluation/fault_trace_based/cdf_trace_waste_tp16_gr4.pdf}
        \caption{TP-16.}
        \label{fig:simulation:waste-cdf:tp16-gr4}
    \end{subfigure}
    \hspace{2pt}
    \begin{subfigure}[b]{0.23\linewidth}
        \centering
        \includegraphics[width=\linewidth]{figs/evaluation/fault_trace_based/cdf_trace_waste_tp32_gr4.pdf}
        \caption{TP-32.}
        \label{fig:simulation:waste-cdf:tp32-gr4}
    \end{subfigure}
    \hspace{2pt}
    \begin{subfigure}[b]{0.23\linewidth}
        \centering
        \includegraphics[width=\linewidth]{figs/evaluation/fault_trace_based/cdf_trace_waste_tp64_gr4.pdf}
        \caption{TP-64.}
        \label{fig:simulation:waste-cdf:tp64-gr4}
    \end{subfigure}
    \vspace{-1ex}
    \caption{CDF of GPU waste ratio over production fault trace, 4 GPU node.}
    \label{fig:simulation:waste-cdf:gr4:supple}
\end{figure*}


\begin{figure*}[h!t]
    \centering
    \begin{subfigure}[b]{0.23\linewidth}
        \centering
        \includegraphics[width=\linewidth]{figs/evaluation/fault_model_based/frag_ratio_tp8_gr4.pdf}
        \caption{TP-8.}
        \label{fig:simulation:model:wasted:tp8}
    \end{subfigure}
    \hspace{2pt}
    \begin{subfigure}[b]{0.23\linewidth}
        \centering
        \includegraphics[width=\linewidth]{figs/evaluation/fault_model_based/frag_ratio_tp16_gr4.pdf}
        \caption{TP-16.}
        \label{fig:simulation:model:wasted:tp16}
    \end{subfigure}
    \hspace{2pt}
    \begin{subfigure}[b]{0.23\linewidth}
        \centering
        \includegraphics[width=\linewidth]{figs/evaluation/fault_model_based/frag_ratio_tp32_gr4.pdf}
        \caption{TP-32.}
        \label{fig:simulation:model:wasted:tp32}
    \end{subfigure}
    \hspace{2pt}
    \begin{subfigure}[b]{0.23\linewidth}
        \centering
        \includegraphics[width=\linewidth]{figs/evaluation/fault_model_based/frag_ratio_tp64_gr4.pdf}
        \caption{TP-64.}
        \label{fig:simulation:model:wasted:tp64}
    \end{subfigure}
    \vspace{-1ex}
    \caption{GPU wastes ratio with different GPU fault ratio, 4-GPU node.}
    \label{fig:simulation:model:wasted-gr4}
\end{figure*}



\begin{figure*}[h!t]
    \centering
    \begin{subfigure}[b]{0.23\linewidth}
        \centering
        \includegraphics[width=\linewidth]{figs/evaluation/fault_trace_based/breakdown_ratio_tp8_gr4.pdf}
        \caption{TP-8.}
        \label{fig:simulation:breakdown-duration:tp8-4gpu}
    \end{subfigure}
    \hspace{2pt}
    \begin{subfigure}[b]{0.23\linewidth}
        \centering
        \includegraphics[width=\linewidth]{figs/evaluation/fault_trace_based/breakdown_ratio_tp16_gr4.pdf}
        \caption{TP-16.}
        \label{fig:simulation:breakdown-duration:tp16-4gpu}
    \end{subfigure}
    \hspace{2pt}
    \begin{subfigure}[b]{0.23\linewidth}
        \centering
        \includegraphics[width=\linewidth]{figs/evaluation/fault_trace_based/breakdown_ratio_tp32_gr4.pdf}
        \caption{TP-32.}
        \label{fig:simulation:breakdown-duration:tp32-4gpu}
    \end{subfigure}
    \hspace{2pt}
    \begin{subfigure}[b]{0.23\linewidth}
        \centering
        \includegraphics[width=\linewidth]{figs/evaluation/fault_trace_based/breakdown_ratio_tp64_gr4.pdf}
        \caption{TP-64.}
        \label{fig:simulation:breakdown-duration:tp64-4gpu}
    \end{subfigure}
    \vspace{-1ex}
    \caption{Job fault-waiting duration with different levels of job-scale, 4 GPU node}
    \label{fig:simulation:breakdown-duration-supple}
\end{figure*}





\vspace{-12em}
\section{Detailed Cost and power consumption Analysis}
\label{appendix:cost}
In this section, \tabref{tab:eval:components} provides a detailed description of the quantity, cost, bandwidth, and power consumption of the interconnect components in various network architectures, including Google TPUv4~\cite{isca2023tpu}, NVIDIA GB200 NVL series~\cite{nvl72}, Alibaba HPN\cite{sigcomm2024hpn}, and \sys{}.


\begin{table*}[h!t] \small
    \centering
    \begin{tabular}{lllll}
    \toprule
    
    \textbf{Component} & \textbf{Quantity} & \textbf{Unit Cost (\$)}  & \textbf{Unit Bandwidth (GBps)} & \textbf{Unit Power (W)} \\

    \midrule
    \multicolumn{5}{c}{\textbf{Google TPUv4\cite{isca2023tpu} with 4096 GPU, bandwidth 300GBps/GPU}} \\
    
    \midrule
    OCS\cite{sigcomm2023lightwave} & 48 & 80000 & 6400 & 108 \\
    DAC Cable\cite{400G_DAC} & 5120 & 63.60 & 50 & 0.1 \\
    Optical Module\cite{400G_OPTICAL_MODULE} & 6144 & 360 & 50 & 12  \\
    Fiber\cite{FIBER}& 6144 & 6.80 & 50 & 0 \\
    
    \midrule
    \multicolumn{5}{c}{\textbf{NVIDIA GB200 NVL-36\cite{SEMIANALYSIS_GB200} with 36 GPU, bandwidth 900GBps/GPU}}\\
    \midrule
    NVLink Switch\cite{SEMIANALYSIS_Power} & 9 & 28000 & 3600 & 275 \\
    DAC Cable\cite{200G_DAC} & 2592 & 35.60 & 25 & 0.1 \\
    
    \midrule
    \multicolumn{5}{c}{\textbf{NVIDIA GB200 NVL-72\cite{nvl72}\cite{SEMIANALYSIS_GB200} with 72 GPU, bandwidth 900GBps/GPU}}\\
    \midrule
    NVLink Switch\cite{SEMIANALYSIS_Power} & 18 & 28000 & 3600 & 275 \\
    DAC Cable\cite{200G_DAC} & 5184 & 35.60 & 25 & 0.1 \\
    \midrule
    \multicolumn{5}{c}{\textbf{NVIDIA GB200 NVL-36x2\cite{SEMIANALYSIS_GB200} with 72 GPU, bandwidth 900GBps/GPU}}\\
    \midrule
    NVLink Switch\cite{SEMIANALYSIS_Power} & 36 & 28000 & 3600 &  275\\
    DAC Cable\cite{200G_DAC} & 6480 & 35.60 & 25 & 0.1 \\
    ACC Cable\cite{SEMIANALYSIS_Power} & 162 & 320 & 200 & 2.5 \\

    \midrule
    \multicolumn{5}{c}{\textbf{NVIDIA GB200 NVL-576\cite{SEMIANALYSIS_GB200} with 576 GPU, bandwidth 900GBps/GPU}}\\
    \midrule
    NVLink Switch\cite{SEMIANALYSIS_Power} & 432 & 28000 & 3600 & 275 \\
    DAC Cable\cite{200G_DAC} & 41472 & 35.60 & 25 & 0.1 \\
    Optical Module\cite{OSFPXD} & 4608 & 850 & 200 & 25 \\
    Fiber\cite{FIBER} & 4608 & 6.80 & 200 & 0 \\

    \midrule
    \multicolumn{5}{c}{\textbf{Alibaba HPN\cite{sigcomm2024hpn} with 16320 GPU, bandwidth 50GBps/GPU}}\\
    \midrule
    EPS\cite{51.2T_EPS} & 360 & 14960 & 6400 & 3145 \\
    DAC Cable\cite{200G_DAC} & 32640 & 35.60 & 25 & 0.1\\
    Optical Module\cite{400G_OPTICAL_MODULE} & 28800 & 360 & 50 & 12 \\
    Fiber\cite{FIBER} & 14400 & 6.80 & 50 & 0 \\

    \midrule
    \multicolumn{5}{c}{\textbf{\SYS{}($K=2$)  with 4 GPU, bandwidth 800GBps/GPU}}\\
    \midrule
    DAC Cable\cite{1.6T_DAC}& 4 & 199.60 & 200 & 0.1\\
    dOCS Module & 16 & 600 & 100 & 12 \\
    Fiber\cite{FIBER} & 16 & 6.80 & 100 & 0 \\

    \midrule
    \multicolumn{5}{c}{\textbf{\SYS{}($K=3$)  with 4 GPU, bandwidth 800GBps/GPU}}\\
    \midrule
    DAC Cable\cite{1.6T_DAC} & 2 & 199.60 & 200 & 0.1\\
    dOCS Module & 24 & 600 & 100 & 12 \\
    Fiber\cite{FIBER} & 24 & 6.80 & 100 & 0 \\
    \bottomrule
    \end{tabular}
    \caption{Interconnect cost and power consumption of components used in different network architectures.}
    \label{tab:eval:components}
\end{table*}


\end{appendices}





\end{document}
\endinput
%%
%% End of file `sample-manuscript.tex'.
