%%
%% This is file `sample-manuscript.tex',
%% generated with the docstrip utility.
%%
%% The original source files were:
%%
%% samples.dtx  (with options: `manuscript')
%% 
%% IMPORTANT NOTICE:
%% 
%% For the copyright see the source file.
%% 
%% Any modified versions of this file must be renamed
%% with new filenames distinct from sample-manuscript.tex.
%% 
%% For distribution of the original source see the terms
%% for copying and modification in the file samples.dtx.
%% 
%% This generated file may be distributed as long as the
%% original source files, as listed above, are part of the
%% same distribution. (The sources need not necessarily be
%% in the same archive or directory.)
%%
%%
%% Commands for TeXCount
%TC:macro \cite [option:text,text]
%TC:macro \citep [option:text,text]
%TC:macro \citet [option:text,text]
%TC:envir table 0 1
%TC:envir table* 0 1
%TC:envir tabular [ignore] word
%TC:envir displaymath 0 word
%TC:envir math 0 word
%TC:envir comment 0 0
%%
%%
%% The first command in your LaTeX source must be the \documentclass command.
% \documentclass[manuscript,review,anonymous]{acmart}


\documentclass[acmsmall]{acmart}

% \documentclass[acmsmall,review,anonymous]{acmart}
% \documentclass[sigconf]{acmart}

% \usepackage[bookmarks=false]{hyperref}

% \usepackage{hyperref}
% \usepackage{hyperxmp}
\usepackage{svg}
\usepackage{makecell}
\usepackage{graphicx}
\usepackage{subcaption}
\usepackage{enumitem}

\usepackage[frozencache,cachedir=minted-cache]{minted} 

\usepackage{xcolor}
\usepackage{multirow}
\usepackage[most]{tcolorbox}

\definecolor{brainstorm}{HTML}{FFB347}
\definecolor{summarise}{HTML}{966FD6}
\definecolor{elaborate}{HTML}{71C562}
\definecolor{draft}{HTML}{FB6B89}
\definecolor{freewrite}{HTML}{B39EB5}
\definecolor{associate}{HTML}{D64A4A}
\definecolor{custom}{HTML}{B7AFA3}
\definecolor{custom2}{HTML}{E8D0A9}

\definecolor{keyword}{HTML}{CED8DF}
% \definecolor{keyword}{HTML}{E9EDF3}
\definecolor{concept}{HTML}{FFB5B7}
\definecolor{sticky_note}{HTML}{748B97}
\definecolor{section}{HTML}{0099FF}
%% Custom command for revised content
%% Change 'red' to 'black' for the clean version
\newcommand{\revision}[1]{{\color{black}#1}}

\newcommand{\RNum}[1]{\uppercase\expandafter{\romannumeral #1\relax}}
%%
%% \BibTeX command to typeset BibTeX logo in the docs
\AtBeginDocument{%
  \providecommand\BibTeX{{%
    Bib\TeX}}}

%% Rights management information.  This information is sent to you
%% when you complete the rights form.  These commands have SAMPLE
%% values in them; it is your responsibility as an author to replace
%% the commands and values with those provided to you when you
%% complete the rights form.
\setcopyright{acmcopyright}
\copyrightyear{2024}
\acmYear{2024}
% \acmDOI{XXXXXXX.XXXXXXX}

%% These commands are for a PROCEEDINGS abstract or paper.
% \acmConference[Conference acronym '24]{}{June 03--05, 2024}{Woodstock, NY}
%\acmPrice{15.00}
%\acmISBN{978-1-4503-XXXX-X/18/06}

% \acmConference[CHI '24]{Proceedings of the 2024
% CHI Conference on Human Factors in Computing Systems}{May 11--16, 2024}{Hawai'i, USA}

% \acmBooktitle{Proceedings of the 2024 CHI Conference on Human Factors in
% Computing Systems (CHI '24), May 11--16, 2024, Hawai'i, USA} 
% \acmPrice{15.00}


%%
%% Submission ID.
%% Use this when submitting an article to a sponsored event. You'll
%% receive a unique submission ID from the organizers
%% of the event, and this ID should be used as the parameter to this command.
%%\acmSubmissionID{123-A56-BU3}

%%
%% For managing citations, it is recommended to use bibliography
%% files in BibTeX format.
%%
%% You can then either use BibTeX with the ACM-Reference-Format style,
%% or BibLaTeX with the acmnumeric or acmauthoryear sytles, that include
%% support for advanced citation of software artefact from the
%% biblatex-software package, also separately available on CTAN.
%%
%% Look at the sample-*-biblatex.tex files for templates showcasing
%% the biblatex styles.
%%

%%
%% The majority of ACM publications use numbered citations and
%% references.  The command \citestyle{authoryear} switches to the
%% "author year" style.
%%
%% If you are preparing content for an event
%% sponsored by ACM SIGGRAPH, you must use the "author year" style of
%% citations and references.
%% Uncommenting
%% the next command will enable that style.
%%\citestyle{acmauthoryear}

%%
%% end of the preamble, start of the body of the document source.
\begin{document}

%%
%% The "title" command has an optional parameter,
%% allowing the author to define a "short title" to be used in page headers.
\title[Polymind]{Polymind: Parallel Visual Diagramming with Large Language Models to Support Prewriting Through Microtasks}

%%
%% The "author" command and its associated commands are used to define
%% the authors and their affiliations.
%% Of note is the shared affiliation of the first two authors, and the
%% "authornote" and "authornotemark" commands
%% used to denote shared contribution to the research.
\author{Qian Wan}
%\authornote{Both authors contributed equally to this research.}
\orcid{0000-0002-4250-8780}
%\author{G.K.M. Tobin}
%\authornotemark[1]
%\email{webmaster@marysville-ohio.com}
\affiliation{%
  \institution{City University of Hong Kong}
  \streetaddress{83 Tat Chee Avenue}
  \city{Hong Kong}
  %\state{Ohio}
  \country{Hong Kong}
  %\postcode{43017-6221}
}
\email{qianwan3-c@my.cityu.edu.hk}

\author{Jiannan Li}
\affiliation{%
  \institution{Singapore Management University}
  % \streetaddress{83 Tat Chee Avenue}
  \city{Singapore}
  \country{Singapore}
  }
\email{jiannanli@smu.edu.sg}

\author{Huanchen Wang}
\orcid{0000-0001-9339-1941}
\affiliation{
  \department{Department of Computer Science}
  \institution{City University of Hong Kong}
  \city{Hong Kong}
  \country{China}
}

\affiliation{
  \department{Department of Computer Science and Engineering}
  \institution{Southern University of Science and Technology}
  \city{Shenzhen}
  \state{Guangdong}
  \country{China}
}
\email{hc.wang@my.cityu.edu.hk}

\author{Zhicong Lu}
\affiliation{%
  \institution{George Mason University}
  % \streetaddress{83 Tat Chee Avenue}
  \city{Fairfax}
  % \state{Virginia}
  \country{USA}
  %\country{Iceland}
}
\email{zhiconlu@cityu.edu.hk}


%%
%% By default, the full list of authors will be used in the page
%% headers. Often, this list is too long, and will overlap
%% other information printed in the page headers. This command allows
%% the author to define a more concise list
%% of authors' names for this purpose.
\renewcommand{\shortauthors}{Wan et al.}

%%
%% The abstract is a short summary of the work to be presented in the
%% article.

\begin{abstract}  
Test time scaling is currently one of the most active research areas that shows promise after training time scaling has reached its limits.
Deep-thinking (DT) models are a class of recurrent models that can perform easy-to-hard generalization by assigning more compute to harder test samples.
However, due to their inability to determine the complexity of a test sample, DT models have to use a large amount of computation for both easy and hard test samples.
Excessive test time computation is wasteful and can cause the ``overthinking'' problem where more test time computation leads to worse results.
In this paper, we introduce a test time training method for determining the optimal amount of computation needed for each sample during test time.
We also propose Conv-LiGRU, a novel recurrent architecture for efficient and robust visual reasoning. 
Extensive experiments demonstrate that Conv-LiGRU is more stable than DT, effectively mitigates the ``overthinking'' phenomenon, and achieves superior accuracy.
\end{abstract}  

%%
%% The code below is generated by the tool at http://dl.acm.org/ccs.cfm.
%% Please copy and paste the code instead of the example below.
%%
\begin{CCSXML}
<ccs2012>
   <concept>
       <concept_id>10003120.10003121.10003129</concept_id>
       <concept_desc>Human-centered computing~Interactive systems and tools</concept_desc>
       <concept_significance>500</concept_significance>
       </concept>
 </ccs2012>
\end{CCSXML}

\ccsdesc[500]{Human-centered computing~Interactive systems and tools}

%%
%% Keywords. The author(s) should pick words that accurately describe
%% the work being presented. Separate the keywords with commas.
\keywords{Prewriting, Diagramming, Creativity Support, Microtasking, Human-AI Collaboration}

%%
%% This command processes the author and affiliation and title
%% information and builds the first part of the formatted document.

\begin{teaserfigure}
    \centering
    \includegraphics[width=\linewidth]{figures/teaser_new_v6.png}
    \caption{The microtasking workflow of \textit{Polymind}: A) A user can delegate new microtasks, and configure microtasks by specifying input \& output types, prompts, initiative modes, etc. B) An active microtask notifies users when results are ready and provide previews on the canvas. C) Once expanded, resulting diagrams are displayed using a hollow shape in contrast to user created diagrams, and distinctive border colours indicating which microtasks they are from}
    \label{fig:teaser}
\end{teaserfigure}

\maketitle


% humans are sensitive to the way information is presented.

% introduce framing as the way we address framing. say something about political views and how information is represented.

% in this paper we explore if models show similar sensitivity.

% why is it important/interesting.



% thought - it would be interesting to test it on real world data, but it would be hard to test humans because they come already biased about real world stuff, so we tested artificial.


% LLMs have recently been shown to mimic cognitive biases, typically associated with human behavior~\citep{ malberg2024comprehensive, itzhak-etal-2024-instructed}. This resemblance has significant implications for how we perceive these models and what we can expect from them in real-world interactions and decisionmaking~\citep{eigner2024determinants, echterhoff-etal-2024-cognitive}.

The \textit{framing effect} is a well-known cognitive phenomenon, where different presentations of the same underlying facts affect human perception towards them~\citep{tversky1981framing}.
For example, presenting an economic policy as only creating 50,000 new jobs, versus also reporting that it would cost 2B USD, can dramatically shift public opinion~\cite{sniderman2004structure}. 
%%%%%%%% 图1:  %%%%%%%%%%%%%%%%
\begin{figure}[t]
    \centering
    \includegraphics[width=\columnwidth]{Figs/01.pdf}
    \caption{Performance comparison (Top-1 Acc (\%)) under various open-vocabulary evaluation settings where the video learners except for CLIP are tuned on Kinetics-400~\cite{k400} with frozen text encoders. The satisfying in-context generalizability on UCF101~\cite{UCF101} (a) can be severely affected by static bias when evaluating on out-of-context SCUBA-UCF101~\cite{li2023mitigating} (b) by replacing the video background with other images.}
    \label{fig:teaser}
\end{figure}


Previous research has shown that LLMs exhibit various cognitive biases, including the framing effect~\cite{lore2024strategic,shaikh2024cbeval,malberg2024comprehensive,echterhoff-etal-2024-cognitive}. However, these either rely on synthetic datasets or evaluate LLMs on different data from what humans were tested on. In addition, comparisons between models and humans typically treat human performance as a baseline rather than comparing patterns in human behavior. 
% \gabis{looks good! what do we mean by ``most studies'' or ``rarely'' can we remove those? or we want to say that we don't know of previous work doing both at the same time?}\gili{yeah the main point is that some work has done each separated, but not all of it together. how about now?}

In this work, we evaluate LLMs on real-world data. Rather than measuring model performance in terms of accuracy, we analyze how closely their responses align with human annotations. Furthermore, while previous studies have examined the effect of framing on decision making, we extend this analysis to sentiment analysis, as sentiment perception plays a key explanatory role in decision-making \cite{lerner2015emotion}. 
%Based on this, we argue that examining sentiment shifts in response to reframing can provide deeper insights into the framing effect. \gabis{I don't understand this last claim. Maybe remove and just say we extend to sentiment analysis?}

% Understanding how LLMs respond to framing is crucial, as they are increasingly integrated into real-world applications~\citep{gan2024application, hurlin2024fairness}.
% In some applications, e.g., in virtual companions, framing can be harnessed to produce human-like behavior leading to better engagement.
% In contrast, in other applications, such as financial or legal advice, mitigating the effect of framing can lead to less biased decisions.
% In both cases, a better understanding of the framing effect on LLMs can help develop strategies to mitigate its negative impacts,
% while utilizing its positive aspects. \gabis{$\leftarrow$ reading this again, maybe this isn't the right place for this paragraph. Consider putting in the conclusion? I think that after we said that people have worked on it, we don't necessarily need this here and will shorten the long intro}


% If framing can influence their outputs, this could have significant societal effects,
% from spreading biases in automated decision-making~\citep{ghasemaghaei2024understanding} to reducing public trust in AI-generated content~\citep{afroogh2024trust}. 
% However, framing is not inherently negative -- understanding how it affects LLM outputs can offer valuable insights into both human and machine cognition.
% By systematically investigating the framing effect,


%It is therefore crucial to systematically investigate the framing effect, to better understand and mitigate its impact. \gabis{This paragraph is important - I think that right now it's saying that we don't want models to be influenced by framing (since we want to mitigate its impact, right?) When we talked I think we had a more nuanced position?}




To better understand the framing effect in LLMs in comparison to human behavior,
we introduce the \name{} dataset (Section~\ref{sec:data}), comprising 1,000 statements, constructed through a three-step process, as shown in Figure~\ref{fig:fig1}.
First, we collect a set of real-world statements that express a clear negative or positive sentiment (e.g., ``I won the highest prize'').
%as exemplified in Figure~\ref{fig:fig1} -- ``I won the highest prize'' positive base statement. (2) next,
Second, we \emph{reframe} the text by adding a prefix or suffix with an opposite sentiment (e.g., ``I won the highest prize, \emph{although I lost all my friends on the way}'').
Finally, we collect human annotations by asking different participants
if they consider the reframed statement to be overall positive or negative.
% \gabist{This allows us to quantify the extent of \textit{sentiment shifts}, which is defined as labeling the sentiment aligning with the opposite framing, rather then the base sentiment -- e.g., voting ``negative'' for the statement ``I won the highest prize, although I lost all my friends on the way'', as it aligns with the opposite framing sentiment.}
We choose to annotate Amazon reviews, where sentiment is more robust, compared to e.g., the news domain which introduces confounding variables such as prior political leaning~\cite{druckman2004political}.


%While the implications of framing on sensitive and controversial topics like politics or economics are highly relevant to real-world applications, testing these subjects in a controlled setting is challenging. Such topics can introduce confounding variables, as annotators might rely on their personal beliefs or emotions rather than focusing solely on the framing, particularly when the content is emotionally charged~\cite{druckman2004political}. To balance real-world relevance with experimental reliability, we chose to focus on statements derived from Amazon reviews. These are naturally occurring, sentiment-rich texts that are less likely to trigger strong preexisting biases or emotional reactions. For instance, a review like ``The book was engaging'' can be framed negatively without invoking specific cultural or political associations. 

 In Section~\ref{sec:results}, we evaluate eight state-of-the-art LLMs
 % including \gpt{}~\cite{openai2024gpt4osystemcard}, \llama{}~\cite{dubey2024llama}, \mistral{}~\cite{jiang2023mistral}, \mixtral{}~\cite{mistral2023mixtral}, and \gemma{}~\cite{team2024gemma}, 
on the \name{} dataset and compare them against human annotations. We find  that LLMs are influenced by framing, somewhat similar to human behavior. All models show a \emph{strong} correlation ($r>0.57$) with human behavior.
%All models show a correlation with human responses of more than $0.55$ in Pearson's $r$ \gabis{@Gili check how people report this?}.
Moreover, we find that both humans and LLMs are more influenced by positive reframing rather than negative reframing. We also find that larger models tend to be more correlated with human behavior. Interestingly, \gpt{} shows the lowest correlation with human behavior. This raises questions about how architectural or training differences might influence susceptibility to framing. 
%\gabis{this last finding about \gpt{} stands in opposition to the start of the statement, right? Even though it's probably one of the largest models, it doesn't correlate with humans? If so, better to state this explicitly}

This work contributes to understanding the parallels between LLM and human cognition, offering insights into how cognitive mechanisms such as the framing effect emerge in LLMs.\footnote{\name{} data available at \url{https://huggingface.co/datasets/gililior/WildFrame}\\Code: ~\url{https://github.com/SLAB-NLP/WildFrame-Eval}}

%\gabist{It also raises fundamental philosophical and practical questions -- should LLMs aim to emulate human-like behavior, even when such behavior is susceptible to harmful cognitive biases? or should they strive to deviate from human tendencies to avoid reproducing these pitfalls?}\gabis{$\leftarrow$ also following Itay's comment, maybe this is better in the dicsussion, since we don't address these questions in the paper.} %\gabis{This last statement brings the nuance back, so I think it contradicts the previous parapgraph where we talked about ``mitigating'' the effect of framing. Also, I think it would be nice to discuss this a bit more in depth, maybe in the discussion section.}






\section{Rethinking Sparse Attention Methods}
\label{sec:critique}

Modern sparse attention methods have made significant strides in reducing the theoretical computational complexity of transformer models. However, most approaches predominantly apply sparsity during inference while retaining a pretrained Full Attention backbone, potentially introducing architectural bias that limits their ability to fully exploit sparse attention's advantages. Before introducing our native sparse architecture, we systematically analyze these limitations through two critical lenses.


\begin{figure*}[t] 
\centering 
\includegraphics[width=1\textwidth]{figures/fig2.pdf} 
\caption{Overview of \method{}'s architecture. Left: The framework processes input sequences through three parallel attention branches: For a given query, preceding keys and values are processed into compressed attention for coarse-grained patterns, selected attention for important token blocks, and sliding attention for local context. Right: Visualization of different attention patterns produced by each branch. Green areas indicate regions where attention scores need to be computed, while white areas represent regions that can be skipped.}
\label{fig:framework}
\end{figure*}


\subsection{The Illusion of Efficient Inference}

Despite achieving sparsity in attention computation, many methods fail to achieve corresponding reductions in inference latency, primarily due to two challenges:

\textbf{Phase-Restricted Sparsity.}
Methods such as H2O \citep{h2o} apply sparsity during autoregressive decoding while requiring computationally intensive pre-processing (e.g. attention map calculation, index building) during prefilling. In contrast, approaches like MInference \citep{minference} focus solely on prefilling sparsity. 
These methods fail to achieve acceleration across all inference stages, as at least one phase remains computational costs comparable to Full Attention.
The phase specialization reduces the speedup ability of these methods in prefilling-dominated workloads like book summarization and code completion, or decoding-dominated workloads like long chain-of-thought~\citep{cot} reasoning.

\textbf{Incompatibility with Advanced Attention Architecture.}
Some sparse attention methods fail to adapt to modern decoding efficient architectures like Mulitiple-Query Attention~(MQA) \citep{mqa} and Grouped-Query Attention~(GQA) \citep{gqa}, which significantly reduced the memory access bottleneck during decoding by sharing KV across multiple query heads. For instance, in approaches like Quest \citep{quest}, each attention head independently selects its KV-cache subset. Although it demonstrates consistent computation sparsity and memory access sparsity in Multi-Head Attention (MHA) models, it presents a different scenario in models based on architectures like GQA, where the memory access volume of KV-cache corresponds to the union of selections from all query heads within the same GQA group. This architectural characteristic means that while these methods can reduce computation operations, the required KV-cache memory access remains relatively high.
This limitation forces a critical choice: while some sparse attention methods reduce computation, their scattered memory access pattern conflicts with efficient memory access design from advanced architectures.

These limitations arise because many existing sparse attention methods focus on KV-cache reduction or theoretical computation reduction, but struggle to achieve significant latency reduction in advanced frameworks or backends.
This motivates us to develop algorithms that combine both advanced architectural and hardware-efficient implementation to fully leverage sparsity for improving model efficiency.


\subsection{The Myth of Trainable Sparsity}
Our pursuit of native trainable sparse attention is motivated by two key insights from analyzing inference-only approaches:
(1) \textbf{\textit{Performance Degradation}}: Applying sparsity post-hoc forces models to deviate from their pretrained optimization trajectory. As demonstrated by \citet{magicpig}, top 20\% attention can only cover 70\% of the total attention scores, rendering structures like retrieval heads in pretrained models vulnerable to pruning during inference.
(2)~\textbf{\textit{Training Efficiency Demands}}: 
Efficient handling of  long-sequence training is crucial for modern LLM development. This includes both pretraining on longer documents to enhance model capacity, and subsequent adaptation phases such as long-context fine-tuning and reinforcement learning. However, existing sparse attention methods primarily target inference, leaving the computational challenges in training largely unaddressed. This limitation hinders the development of more capable long-context models through efficient training. Additionally, efforts to adapt existing sparse attention for training also expose challenges:



\textbf{Non-Trainable Components.} Discrete operations in methods like ClusterKV~\citep{clusterkv} 
(includes k-means clustering) and MagicPIG~\citep{magicpig} (includes SimHash-based selecting) create discontinuities in the computational graph. These non-trainable components prevent gradient flow through the token selection process, limiting the model's ability to learn optimal sparse patterns. 

\textbf{Inefficient Back-propagation.} Some theoretically trainable sparse attention methods suffer from practical training inefficiencies. Token-granular selection strategy used in approaches like HashAttention~\citep{desai2024hashattention} leads to the need to load a large number of individual tokens from the KV cache during attention computation. 
This non-contiguous memory access prevents efficient adaptation of fast attention techniques like FlashAttention, which rely on contiguous memory access and blockwise computation to achieve high throughput.
As a result, implementations are forced to fall back to low hardware utilization, significantly degrading training efficiency.



\subsection{Native Sparsity as an Imperative}

These limitations in inference efficiency and training viability motivate our fundamental redesign of sparse attention mechanisms.
We propose \method{}, a natively sparse attention framework that addresses both computational efficiency and training requirements.
In the following sections, we detail the algorithmic design and operator implementation of \method{}.

\section{Formative Study}
To understand how large language models could support diagramming-based prewriting, we conducted a formative study involving 10 participants with daily writing habits ranging from news articles to fiction writing. We used snowball sampling to recruit students in writing- or creativity-related majors (e.g., creative media, design, English literature etc.). All participants were ethnically Chinese, and English was their second language.

\subsection{Protocol}
After participants consented to the study, we asked them to develop a science fiction or thriller story plot using an LLM together with the traditional tools (a pen and a piece of paper) for diagramming or illustrating their ideas.
Participants accessed the LLM using the GPT-3 playground interface.
They were asked to think aloud during the prewriting process.
After completing the tasks, we asked participants to reflect on their experiences and strategies.

The whole process was audio-taped, screen-recorded, and later transcribed for analysis. Two coders performed thematic analysis \cite{corbin2014basics} of the transcripts with reference to the screen recording to extract collaboration strategies, patterns, and breakdowns. The coders later held a discussion to reach a consensus on the themes.

\subsection{Findings}
We report the findings of our formative study in two themes: human-LLM collaboration workflow and common challenges encountered. We found our participants already anthropomorphized the LLM as a collaborator, and delegated to it distinct tasks for both divergent and convergent thinking.
% We first summarise how users leveraged the LLM to support prewriting, and clarify the functions and expected initiative of the LLM during the process. Then we report the two most significant challenges, progress tracking and communication breakdown.
%% Ideally say something related to our later design choice here

\subsubsection{Human-LLM Collaboration Workflow: Tasks and Initiative} \label{the_usage_of_LLMs_for_prewriting}
We observed that participants already implemented the parallel thinking strategy consciously or unconsciously.
They expected the LLM to perform multiple but distinctive functions, including generating additional ideas, elaborating on concepts, organizing fleeting thoughts, and enriching existing writing with details.
These functions were typically seen in creativity processes, covering both convergent and divergent thinking phases.
% From the first to the third stage, ideas become increasingly concrete, articulated, and formalised.
% This human-LLM collaboration workflow aligns with previous conceptualisations of the creativity process in both Psychology \cite{wallas1926art} and Human-Computer Interaction (HCI) \cite{frich2019mapping,shneidenmab2003supporting}, because prewriting is inherently a stage of creativity, or a stage of discovery as described by Rohman \cite{rohman1965pre}.

We found that users almost always preferred to take the initiative during the whole collaboration process, unless they ran out of ideas. They would use LLMs mostly to enrich their ideas with details, such as bridging a logical gap in a plot or providing a nuanced portrait of a scene.
Only when users had no initial ideas or hit writer's block, would they let the LLM take the initiative to generate ideas. P3 described the ideal role of LLM as similar to a ``\textit{second mind}'' that ``\textit{processes all the context in parallel}'' and ``\textit{provides ideas when requested}'', while he could still take control of the general prewriting process.

Meanwhile, we also noticed that users were generally tolerant and did not mind following the ideas in LLM output, which were often initially vague or confusing. Some users (P3-4, P8-9) were found to spend a long time iteratively refining their own prompts to improve the LLM's output.

\subsubsection{Challenges: Progress Tracking \& Communication Breakdown} \label{progress_tracking}
\paragraph{\underline{Progress Tracking}}
Many of our participants (P2-4, P6-7) particularly emphasized that tracking collaboration progress, and maintaining the ever-changing collaboration history while prewriting with the LLM was challenging. Because prewriting is iterative, and LLM generations can be random, our participants frequently needed to expand on specific points within a lengthy piece of writing in a new context, or re-generate content from a previous version. On such occasions, some participants (e.g., P2, P7, P10) mentioned they would need examples, suggestions, or templates as a reference to polish their prompts, and requested features to maintain the history of collaboration with LLM. 
%Progress tracking was considered by P7 to be even more challenging in semi-structured strategies such as diagrams, because different types of generations at each request can be easily conflated with users' content.
\paragraph{\underline{Communication Breakdown}}
The uncertainty of prompt-based communication can often cause LLMs to generate unsatisfactory or even nonsensical results during prewriting. P2 and P4 reflected that, to effectively communicate with an LLM, proper and sufficient context should be articulated via prompts, which can be difficult in complex writing tasks. On these occasions, instead of rewriting their prompts, most participants (e.g., P1-4, P6, P9-10) chose to provide more context and ask LLMs to polish previous output. For example, P4 found the ending of an LLM-generated science fiction lacks originality. Instead of deleting the result and requesting a new one, she asked the LLM to avoid using banal superhero endings and explore existential questions, using an imperative sentence as if giving feedback to a human collaborator.

\subsection{Summary}
Our formative study reveals the creative nature of human-LLM collaboration during prewriting, where LLMs could offer additional perspectives and handle a range of complex ideation tasks. Although participants would like to take control most of the time, it is particularly beneficial that LLMs can run in parallel with users' diagramming activities and provide assistance when requested. During the collaboration, participants often found progress tracking and communication with the LLM using a conversational interface challenging. They requested features to support managing collaboration progress, and favoured an incremental feedback process to facilitate iteration.

\section{Design Goals}
Our primary design goal is to integrate LLMs into a diagramming-based interface to facilitate the application of the parallel thinking strategy in prewriting. We first introduce how we derive the core idea of ``microtasking'' to operationalize parallel thinking, and then report our three design goals to support a microtasking workflow.

\subsubsection*{\textbf{Microtasking for parallel thinking}}
%We are deeply inspired by P3's account of LLMs' collaboration role a ``second mind'', which resonates with the idea of ``parallel thinking'' proposed in the book ``\textit{Six Thinking Hats}'' \cite{de2017six,bono1985six}. 
The notion of ``parallel thinking'' separates the human thinking process into distinct functions and roles. 
%Similarly, LLMs can process a variety of complex but distinct tasks that run in parallel with the user's own thinking process.
In reality, parallel thinking can often be practiced through group collaboration, where each of the distinct roles can be played by different individuals or groups in parallel. We seek to simulate group collaboration to break down prewriting workflows into smaller, manageable, and independent tasks that support both divergent and convergent thinking processes.
%so that each task handled by LLMs can contribute in parallel to efficiently provide productive and creative results.

To this end, we adopt the concept of ``microtasks'', commonly used in crowd sourcing \cite{latoza2014microtask,chen2017retool}, collaborative writing \cite{iqbal2018multitasking,birnholtz2013write,teevan2016supporting}, and group brainstorming \cite{chilton2019visiblends,teevan2016supporting}. Previous studies suggest that a microtasking workflow simplifies complex tasks \cite{cheng2015break,kokkalis2013taskgenies}, facilitates recovery from interruptions, and leads to higher quality of results \cite{cheng2015break}. Similarly, in a human-LLM collaboration scenario, we expect small manageable microtasks that require little context of one another to run concurrently can make complex prewriting tasks easier to coordinate, and save the need to iteratively articulate complex context in prompts.
Specifically, we derive the following three design goals to implement this idea. 

\newtcbox{\BboxL}{on line,
  colframe=brainstorm,colback=brainstorm,
  boxrule=0.5pt,arc=1.5pt,boxsep=0pt,left=3pt,right=3pt,top=3pt,bottom=3pt}
\newtcbox{\SboxL}{on line,
  colframe=summarise,colback=summarise,
  boxrule=0.5pt,arc=1.5pt,boxsep=0pt,left=3pt,right=3pt,top=3pt,bottom=3pt}
\newtcbox{\EboxL}{on line,
  colframe=elaborate,colback=elaborate,
  boxrule=0.5pt,arc=1.5pt,boxsep=0pt,left=3pt,right=3pt,top=3pt,bottom=3pt}
\newtcbox{\DboxL}{on line,
  colframe=draft,colback=draft,
  boxrule=0.5pt,arc=1.5pt,boxsep=0pt,left=3pt,right=3pt,top=3pt,bottom=3pt}
\newtcbox{\FboxL}{on line,
  colframe=freewrite,colback=freewrite,
  boxrule=0.5pt,arc=1.5pt,boxsep=0pt,left=3pt,right=3pt,top=3pt,bottom=3pt}
\newtcbox{\AboxL}{on line,
  colframe=associate,colback=associate,
  boxrule=0.5pt,arc=1.5pt,boxsep=0pt,left=3pt,right=3pt,top=3pt,bottom=3pt}

\renewcommand{\arraystretch}{0.75}
% \begin{table*}[htb]
%     \centering
%     \resizebox{\linewidth}{!}{\begin{tabular}{lcccl}
%         \toprule
%         \textbf{Microtask} & \textbf{Input Type} & \textbf{Output Type} & \textbf{Creativity Phase} & \textbf{Prompt} \\
%         \midrule
%         \renewcommand{\arraystretch}{2.5}
%         \BboxL{\textcolor{white}{\textbf{Brainstorm}}}~\cite{huang2020heteroglossia,gero2019stylistic,wang2010idea,teevan2016supporting,lu2018inkplanner,wang2022interpretable} & keyword & keyword & \makecell[c]{divergent \\ divergent}
%         & \makecell[l]{\underline{Find Related:} Brainstorm keywords related to [placeholder]. \\ \underline{Find Synonym:} Find synonyms for [placeholder].} \\
%         & & & \\
%         \SboxL{\textcolor{white}{\textbf{{Summarise}}}}~\cite{sadauskas2015mining,dang2022beyond} & sticky note & sticky note & \makecell[c]{convergent \\ convergent}
%         & \makecell[l]{\underline{TLDR:} Provide a TLDR version of the following:\textbackslash n[placeholder] \\ \underline{Top 3 keywords:} Summarise top 3 keywords of the following:\textbackslash n[placeholder]} \\
%         & & & \\
%         \EboxL{\textcolor{white}{\textbf{Elaborate}}}~\cite{sadauskas2015mining,uto2015academic,jeon2021fashionq} & concept & concept & \makecell[c]{divergent \\ convergent} & \makecell[l]{\underline{Provide Examples:} What are examples of [placeholder]. \\ \underline{Clarification:} Provide a simple explanation of [placeholder].} \\
%         & & & \\
%         \DboxL{\textcolor{white}{\textbf{Draft}}}~\cite{lu2018inkplanner,chung2022talebrush} & section & sticky note & \makecell[c]{convergent \\ convergent} & \makecell[l]{\underline{Abstract:} [placeholder]\textbackslash n\textbackslash nWrite an abstract of the above outline. \\ \underline{Overview:} [placeholder].\textbackslash n\textbackslash nWrite an overview of the above outline.} \\
%         & & & \\
%         \FboxL{\textcolor{white}{\textbf{Freewrite}}}~\cite{baroudy2008procedural,lu2018inkplanner} & sticky note & sticky note & divergent & \makecell[l]{\underline{Co-creation:} [placeholder].\textbackslash n Continue to write.} \\
%         & & & \\
%         \AboxL{\textcolor{white}{\textbf{Associate}}}~\cite{gero2019metaphoria,chilton2019visiblends,wang2021popblends,faste2012untapped,hope2022scaling,wang2022interpretable} & nodes & keyword & divergent &  \makecell[l]{\underline{Find Relationship:} Clarify the relationship between [placeholder] and [placeholder] in simple words.} \\
%         \bottomrule
%     \end{tabular}}
%     \caption{Six default microtasks of \textit{Polymind}}
%     \label{table:microtasks} 
% \end{table*}

\renewcommand{\arraystretch}{0.75}
\begin{table*}[htb]
    \centering
    \resizebox{\linewidth}{!}{\begin{tabular}{lccl}
        \toprule
        \textbf{Microtask} & \textbf{Input Type} & \textbf{Output Type} &
        \textbf{Prompt} \\
        \midrule
        \renewcommand{\arraystretch}{2.5}
        \BboxL{\textcolor{white}{\textbf{Brainstorm}}}~\cite{huang2020heteroglossia,gero2019stylistic,wang2010idea,teevan2016supporting,lu2018inkplanner,wang2022interpretable} & keyword & keyword & %\makecell[c]{divergent \\ divergent} &
        \makecell[l]{\underline{Find Related:} Brainstorm keywords related to [placeholder]. \\ \underline{Find Synonym:} Find synonyms for [placeholder].} \\
        & & & \\
        \SboxL{\textcolor{white}{\textbf{{Summarise}}}}~\cite{sadauskas2015mining,dang2022beyond} & sticky note & sticky note & \makecell[l]{\underline{TLDR:} Provide a TLDR version of the following:\textbackslash n[placeholder] \\ \underline{Top 3 keywords:} Summarise top 3 keywords of the following:\textbackslash n[placeholder]} \\
        & & & \\
        \EboxL{\textcolor{white}{\textbf{Elaborate}}}~\cite{sadauskas2015mining,uto2015academic,jeon2021fashionq} & concept & concept & \makecell[l]{\underline{Provide Examples:} What are examples of [placeholder]. \\ \underline{Clarification:} Provide a simple explanation of [placeholder].} \\
        & & & \\
        \DboxL{\textcolor{white}{\textbf{Draft}}}~\cite{lu2018inkplanner,chung2022talebrush} & section & sticky note & \makecell[l]{\underline{Abstract:} [placeholder]\textbackslash n\textbackslash nWrite an abstract of the above outline. \\ \underline{Overview:} [placeholder].\textbackslash n\textbackslash nWrite an overview of the above outline.} \\
        & & & \\
        \FboxL{\textcolor{white}{\textbf{Freewrite}}}~\cite{baroudy2008procedural,lu2018inkplanner} & sticky note & sticky note & \makecell[l]{\underline{Co-creation:} [placeholder].\textbackslash n Continue to write.} \\
        & & & \\
        \AboxL{\textcolor{white}{\textbf{Associate}}}~\cite{gero2019metaphoria,chilton2019visiblends,wang2021popblends,faste2012untapped,hope2022scaling,wang2022interpretable} & nodes & keyword &  \makecell[l]{\underline{Find Relationship:} Clarify the relationship between [placeholder] and [placeholder] in simple words.} \\
        \bottomrule
    \end{tabular}}
    \caption{Six default microtasks of \textit{Polymind}}
    \label{table:microtasks} 
\end{table*}

\subsubsection*{\textbf{Goal 1: Scaffold visual-diagramming-based prewriting with microtasks}}
We aim to scaffold the prewriting process with a diagramming tool that supports common strategies such as concept mapping, mind mapping, outlining, etc. To enable natural collaboration with an LLM while diagramming, we are inspired by the concept of ``macros'' \cite{kurlander1992history} and seek to address the uncertainty of collaboration goals by defining default microtasks, and allowing users to customise their requirements or rapidly delegate their own microtasks.

Our formative study inspired us to draw upon existing literature on creativity support tools (CSTs) beyond prewriting itself to define default microtasks, as the collaboration workflow appeared to be a typical creativity process. We conducted a survey of both CST and writing tool literature by searching two academic databases, ACM Digital Library and Google Scholar, using three keywords: ``writing'', ``prewriting'', and ``creativity support''. We reviewed the top 100 entries for each keyword.
%We then performed a thematic analysis of system design papers among top 100 entries of each search result (200 entries in total), recording their key features and functions. Afterwards, we grouped these features or functions into different categories to derive our pre-defined microtasks.
Based on this survey, we identified 6 microtasks: ``Brainstorm'', ``Elaborate'', ``Summarise'', ``Draft'', ``Freewrite'', and ``Associate'', as summarised in \autoref{table:microtasks}. Of these microtasks, ``Brainstorm'' and ``Associate'' are typical divergent thinking tasks seen in existing CSTs (e.g., conceptual blending of \cite{wang2021popblends,chilton2019visiblends}, attribute detection of \cite{jeon2021fashionq}, group ideation of \cite{teevan2016supporting,wang2010idea}, etc.)
``Draft'' and ``Freewrite'' are common features in prewriting tools \cite{lu2018inkplanner,sadauskas2015mining}. ``Elaborate'' and ``Summarise'' are commonly used in the literature of writing support~\cite{uto2015academic,dang2022beyond}, as convergent thinking tasks to help articulate or organise existing ideas.

\subsubsection*{\textbf{Goal 2: Facilitate task management}}
Task management is crucial in human collaboration. In our scenario, a user acts as a leader who determines the goals and progress of the collaboration. Therefore, she should be granted sufficient control to manage microtasks. To this end, we further derive three sub-goals from the literature on human collaboration and our formative study.
\paragraph{\textbf{Goal 2.1: Provide awareness}}
The awareness information about other collaborators while using groupware \cite{gutwin2002descriptive} is vital in tasks such as collaborative writing \cite{birnholtz2013write} and collaborative learning \cite{fransen2011mediating}. On a prewriting interface (e.g., a diagramming canvas), where elements can be loosely organised and often scattered around, it might be hard to notice other collaborators' operations without proper design support. Therefore we seek to provide awareness features so that users can easily track the status of each microtask, and the results returned by each microtask.
As informed by ~\cite{gluck2007matching}, we aim to design different levels of awareness features that match the utility of different interruption types.

\paragraph{\textbf{Goal 2.2: Support progress tracking}}
Progress tracking is an essential aspect of many collaboration tasks, especially collaborative writing \cite{birnholtz2013write,birnholtz2012tracking}. Our formative study suggests that it is also a concern while collaborating with an LLM. We aim to help users manage the results of each microtask in a less demanding way, so that they do not clutter users' diagrams but can be merged into them once accepted.
\paragraph{\textbf{Goal 2.3: Facilitate human feedback}}
Feedback is essential for improvements \cite{dow2011shepherding,haug2021feeasy,huang2018feedback}. In our formative study, users generally preferred feedback-like communication upon unsatisfactory generations. Therefore we aim to facilitate user feedback to enhance LLM-generated content. In our system, the ``feedback'' is provided to an LLM, which implies that it should convert user requirements into actionable prompts.

\subsubsection*{\textbf{Goal 3: Apply mixed initiative}}
Although users preferred to maintain control most of the time in our formative study, they still wanted the LLM to take the initiative when running out of ideas, a common hurdle during prewriting. Sometimes, they even expected the LLM to further clarify or improve its output. Therefore, we aim to apply the principle of ``mixed initiative'' \cite{horvitz1999principles}, and allow a microtasking LLM to infer the focus of attention of the user to determine the timing of suggestions. To keep users in control and minimize the cost of inference errors, we aim to enable users to manage the initiative of each individual microtask. 
\newtcbox{\BboxS}{on line,
  colframe=brainstorm,colback=brainstorm,
  boxrule=0.5pt,arc=1pt,boxsep=0pt,left=2pt,right=2pt,top=2pt,bottom=2pt}
\newtcbox{\SboxS}{on line,
  colframe=summarise,colback=summarise,
  boxrule=0.5pt,arc=1pt,boxsep=0pt,left=2pt,right=2pt,top=2pt,bottom=2pt}
\newtcbox{\EboxS}{on line,
  colframe=elaborate,colback=elaborate,
  boxrule=0.5pt,arc=1pt,boxsep=0pt,left=2pt,right=2pt,top=2pt,bottom=2pt}
\newtcbox{\DboxS}{on line,
  colframe=draft,colback=draft,
  boxrule=0.5pt,arc=1pt,boxsep=0pt,left=2pt,right=2pt,top=2pt,bottom=2pt}
\newtcbox{\FboxS}{on line,
  colframe=freewrite,colback=freewrite,
  boxrule=0.5pt,arc=1pt,boxsep=0pt,left=2pt,right=2pt,top=2pt,bottom=2pt}
\newtcbox{\AboxS}{on line,
  colframe=associate,colback=associate,
  boxrule=0.5pt,arc=1pt,boxsep=0pt,left=2pt,right=2pt,top=2pt,bottom=2pt}

\newtcbox{\CboxS}{on line,
colframe=custom,colback=custom,
boxrule=0.5pt,arc=1pt,boxsep=0pt,left=2pt,right=2pt,top=2pt,bottom=2pt}

\newtcbox{\CCboxS}{on line,
colframe=custom2,colback=custom2,
boxrule=0.5pt,arc=1pt,boxsep=0pt,left=2pt,right=2pt,top=2pt,bottom=2pt}
  
\section{Use Case Scenario}
In this section, we illustrate the workflow of \textit{Polymind} via a use case scenario. Suppose that Bob, an HCI researcher, would like to use fictional narratives to promote his research on social media. He therefore chooses to use \textit{Polymind} to plan his fiction writing.

\subsection*{Task Management}
Bob starts with the key insights of his paper, that parallel inputs from AI agents such as LLMs can significantly increase writers' creativity. His primary goals of using \textit{Polymind} are brainstorming narrative lines, and working out a rough outline. Therefore he keeps three microtasks in the proactive mode to quickly expand his ideas: \BboxS{\textcolor{white}{Brainstorm}}, \EboxS{\textcolor{white}{Elaborate}}, and \AboxS{\textcolor{white}{Associate}}; and switched all other microtasks to the reactive mode so that they will not be intrusive.

\subsection*{Collaborative Brainstorming}
To get some initial ideas, Bob uses \textit{Polymind} to perform mind mapping, which focuses on tracking spontaneous and free-form ideas, and their associations.
He first creates some keywords and concepts on the canvas: such as ``parallel collaboration'', ``creative writing'', etc. The three proactive microtasks, \BboxS{\textcolor{white}{Brainstorm}}, \EboxS{\textcolor{white}{Elaborate}}, \AboxS{\textcolor{white}{Associate}} shortly returns some related keywords, example scenarios, and associations between these diagrams.

Bob accepts three diagrams: ``synchronous tasks'', ``mutual goals'', and ``flash fiction''. These remind him of a story where a former fiction writer revolutionizes the fiction writing industry by leveraging multiple robots in an assembly line to mass produce flash fictions. Each robot is configured to handle a distinct microtask on the assembly line, working synchronously towards a common goal.

\subsection*{Task Delegation}
At this point, Bob feels he has obtained some concrete ideas, and would like to organize them via concept mapping.
This method aims to outline structures and relationships between concepts. He also wants to ask the LLM how to make a story more engaging. Therefore, he delegates a new microtask \CboxS{\textcolor{white}{Improve}}, which returns suggestions for improvements.
He leaves it in the \textit{reactive} mode so that it can provide suggestions based on the ever-changing diagram upon request.

\subsection*{Idea Clarification}
Bob starts concept mapping by specifying elements of the story. He creates some diagram nodes, such as ``Character: Fiction Writer \& Robots'', ``Event: Revolutionizing the industry through assembly lines of flash fictions'', etc. He then creates a section over these diagrams to group them together, and uses the microtask \DboxS{\textcolor{white}{Draft}} to request an outline, and clicks the \FboxS{\textcolor{white}{Freewrite}} microtask several times to continue writing to see different endings.

Bob feels that these results are somewhat bland, and therefore asks the microtask \CboxS{\textcolor{white}{Improve}} to suggest improvements based on these diagrams. The results suggest adding to the beginning the conflict between the fiction writer and his former boss that fired him for lacking creativity. It reminds Bob that he could depict the former boss as a firm advocate of turn-taking conversational robots, and unveil the superiority of a parallel collaboration through the main character's adventure.

% \subsubsection*{Task Management}
% Bob starts with the core concept of using LLMs to support prewriting, and he hopes to develop ideas about system design and potential contributions. His primary goals when using \textit{Polymind} are generating new ideas and organizing existing thoughts.
% %He hopes to track his thought process while using the AI to expand on his ideas.
% Therefore, he chooses to keep three microtasks: \BboxS{\textcolor{white}{Brainstorm}}, \EboxS{\textcolor{white}{Elaborate}}, and \AboxS{\textcolor{white}{Associate}} in the proactive mode to augment divergent thinking, and leave the other three microtasks, \FboxS{\textcolor{white}{Freewrite}}, \SboxS{\textcolor{white}{Summarise}}, and \DboxS{\textcolor{white}{Draft}}, in the reactive mode.
% %Therefore, he  running in the proactive mode so that they can constantly contribute to diagramming, and , generating diagram elements only upon requests.

% \subsubsection*{Collaborative Brainstorming}
% To get some initial ideas, Bob uses \textit{Polymind} to perform mind mapping, which focuses on tracking spontaneous and free-form ideas, and their associations.
% He first creates a keyword on the canvas: ``prewriting''. The \BboxS{\textcolor{white}{Brainstorm}} microtask shortly returns several relevant keyword suggestions. Bob accepts two of them, ``Freewriting'' and ``Clustering''.
% %, which he believes are common but under-explored  prewriting strategies in HCI research. 

% Bob realizes that there might be design opportunities to combine these two strategies, while the \AboxS{\textcolor{white}{Associate}} microtask starts to operate on these newly added nodes.
% Bob expands results and sees three suggestions connecting ``Freewriting'' with ``Clustering'': ``Unstructured writing'', ``Visual organization'', and ``Idea generation''.
% Bob accepts ``Unstructured writing'' and ``Visual organization'', which reminds him of visual summaries of semantic information on a freewriting interface. 
% He then creates a concept ``Hierarchical clustering of freewriting'' to ``Hierarchical freewriting'' and connects it to ``Unstructured writing'' and ``Visual organization''.

% \subsubsection*{Task Delegation}
% At this point, Bob feels he has obtained some concrete ideas, and would like to organize them by concept mapping.
% This method aims to outline structures and relationships between concepts. He also wants to leverage the LLM to evaluate the feasibility of his research ideas. Therefore, he delegates a new microtask \CboxS{\textcolor{white}{Evaluate}}, which returns potential weaknesses for individual ideas. 
% %He leaves it in the reactive mode so that it can provide suggestions based on the ever-changing diagram upon requests.

% \subsubsection*{Idea Clarification}
% Bob starts concept mapping by creating a diagram node ``semantic visualization'', and connects it with ``visual organization'', as he thinks of the feature of semantic visual summary. The \EboxS{\textcolor{white}{Elaborate}} microtask returns three examples of ``semantic visualization'', ``word clouds'', ``tree maps'', and ``network diagrams''. Bob accepts them all as he believes multiple visualization techniques can be used to address different levels of details. He also creates a keyword ``three LoD'' (level of details) and connects it to the three examples.
% Bob moves to evaluate the novelty and significance of his ideas. He creates a sticky note on the canvas ``Contributions of a freewriting interface to HCI that supports visual organization of semantic information'', and asks the \FboxS{\textcolor{white}{Freewrite}} microtask to continue. The microtask returns a paragraph of text suggesting several potential strengths of such systems.
% %including facilitating group collaboration and communication, enhancing reflection of ideas, allowing multiple users to build upon or contribute to one user’s ideas, etc.
% Finally, Bob creates a section over these nodes named ``freewriting interface that supports semantic visualization'', and invokes the \CboxS{\textcolor{white}{Evaluate}} microtask.
% The result reminds him that such a system could overwhelm users cognitively.

\begin{figure}[ht]
% \centering\captionsetup{width=\linewidth,font={small}}
\includegraphics[width=.85\linewidth]{figures/UI_cropped.png}
\caption{The interface of \textit{Polymind} comprises: A. a diagramming canvas B. a toolbar C. a task board}
\label{fig:UI}
\end{figure}
\section{Designing Polymind}
\revision{
\textit{Polymind}'s interface provides a range of diagramming features commonly used in prewriting strategies, and a ``task board'' overlaid on the canvas for microtask management, as shown in ~\autoref{fig:teaser} and ~\autoref{fig:UI}. We map the input and output of each microtask to diagram types on the canvas. To facilitate collaboration and task management, we introduce the notion of ``task header'', and ``task cards''. The former displays notifications and previews of microtask results on a specific diagram, while the latter displays specifications of a microtask on the ``task board''.
To apply mixed initiative, we define two initiative modes: proactive and reactive. We also design a set of workflows to provide awareness information.
In this section, we first provide an overview of the \textit{Polymind} interface, and then elaborate on each of our key features.
}

% \begin{figure}[ht]
% \centering
% \begin{subfigure}{.45\linewidth}
%     \centering
%     \includegraphics[width=.85\linewidth]{figures/keyword.png}
%     \label{fig:keyword}
% \end{subfigure}
% \begin{subfigure}{.45\linewidth}
%     \centering
%     \includegraphics[width=.85\linewidth]{figures/concept.png}
%     \label{fig:concept}
% \end{subfigure}
% \begin{subfigure}{.45\linewidth}
%     \centering
%     \includegraphics[width=.9\linewidth]{figures/sticky_note.png}
%     \label{fig:sticky_note}
% \end{subfigure}
% \begin{subfigure}{.45\linewidth}
%     \centering
%     \includegraphics[width=\linewidth]{figures/section.png}
%     \label{fig:section}
% \end{subfigure}
% \caption{Three basic diagrams (or nodes) in \textit{Polymind}, and the section, which stores multiple diagrams and their connections.}
% \end{figure} \label{fig:diagrams}

\begin{figure}[ht]
\centering
\includegraphics[width=\linewidth]{figures/UIComponents.png}
\caption{\textit{Polymind} supports three basic diagrams (or nodes), and allows users to draw a section over diagrams. The task board interface supports microtask management, and maps microtask input and output to different diagram types on the canvas. The results of microtasks are displayed as notifications and previews on the task header before being expanded and accepted.}
\label{fig:diagrams}
\end{figure}

\subsection{Main Interface}
The interface of \textit{Polymind} comprises a diagramming canvas, a task board, and a toolbar (\autoref{fig:UI}).
\revision{The diagramming canvas includes all diagrams and their topological connections. We defined three basic diagrams: \textbf{\textcolor{keyword}{\textit{keyword}}}, \textbf{\textcolor{concept}{\textit{concept}}}, \textbf{\textcolor{sticky_note}{\textit{sticky note}}}, each with a unique shape affording varying text lengths (keywords the shortest, and sticky notes for the longest pieces of text). Users can connect diagrams through directed or undirected edges, or draw a \textbf{\textcolor{section}{\textit{section}}} over diagrams (see ~\autoref{fig:diagrams}).

The task board displays all microtasks and their specifications in distinct ``task cards'', each including input and output types, prompts, etc. The results of each microtask are displayed as notifications and previews on the ``task header'' over that processed diagram before they are expanded and accepted (see ~\autoref{fig:task_status}).}

\subsubsection{The Diagramming Canvas}
The diagramming canvas supports most common diagrams. Over each of these diagrams, a task header is attached to display the status of microtasks. All resulting diagrams of microtasks are displayed using a hollow shape in contrast to user created diagrams.

\paragraph{\underline{Diagrams}}
To scaffold the diagramming process for \textit{Goal 1}, we define three primitive diagrams that are commonly used in diagramming or prewriting tools (e.g., Inkplanner \cite{lu2018inkplanner}, Figma \footnote{https://www.figma.com/}, etc.): \textbf{\textcolor{keyword}{\textit{keyword}}} (displayed as text labels), \textbf{\textcolor{concept}{\textit{concept}}} (represented by an ellipse), and \textbf{\textcolor{sticky_note}{\textit{sticky note}}} (see \autoref{fig:diagrams}). We use sizes, shapes and the placeholder text upon creation (``Add Keyword'', ``Add Concept'', ``Add text'') to guide users to type text of different lengths and of different functions into different diagrams (e.g, brief words in \textbf{\textcolor{keyword}{\textit{keyword}}}, short phrases in \textbf{\textcolor{concept}{\textit{concept}}}, long paragraphs in \textbf{\textcolor{sticky_note}{\textit{sticky note}}}), so that we can properly define microtasks handling various types of input to support all three stages of LLM usage (see \ref{the_usage_of_LLMs_for_prewriting}).

These three primitive diagrams can be selected, moved, resized or scaled as per users' needs. Additionally, users can establish two types of connection between these diagrams: directed (arrow) or undirected (line). We also allow users to create a \textbf{\textcolor{section}{\textit{section}}} (\autoref{fig:diagrams}) among these diagrams, and assign titles to these sections. On such an interface, users can perform most diagram-based prewriting strategies such as mind mapping, concept mapping, argument mapping, etc.

It is important to note that the diagramming canvas maintains a graph-like (usually tree-like) structure, and we later refer to those primitive diagrams as ``nodes'' at times for simplicity.

\paragraph{\underline{The Task Header}}
To support \textit{Goal 2} (especially \textit{Goal 2.1}) and \textit{Goal 3}, we design a task header that is attached to each diagram (i.e., \textbf{\textcolor{keyword}{\textit{keyword}}}, \textbf{\textcolor{concept}{\textit{concept}}}, \textbf{\textcolor{sticky_note}{\textit{sticky note}}}, and \textbf{\textcolor{section}{\textit{section}}}) on the diagramming canvas to display the status of each microtask on the diagram element (\autoref{fig:task_status}). On the task header, the name of each microtask is displayed as small text labels. By default, all microtasks are activated (see \ref{task_initiative}) and filled with distinctive colours. Each task header also has a preview panel (\autoref{fig:notifications_and_previews}) that displays key points of unread microtask results. It will only pop up when users hover over the task header, and there are unread microtask results.

\begin{figure}[h]
\begin{minipage}{.5\textwidth}
    \captionsetup{width=\linewidth}
    \vspace{0pt}
    \includegraphics[width=\linewidth]{figures/feedback_v2.png}
    \caption{(a) When the user hovers over the three icons over the resulting diagram, suggestions to improve the generation will pop up. The user can click on these suggestions to request a regeneration. (b) If a user clicks on the question mark icon, the system will return an explanation of the generated result.}
    \label{fig:feedback}
\end{minipage}
\begin{minipage}{.45\textwidth}
    \captionsetup{width=.9\linewidth}
    \vspace{0pt}
    \includegraphics[width=\linewidth]{figures/preview_summary2.png}
    \caption{Once the mouse hovers over key points of a microtask on the preview panel for 1.5 seconds, the system will present a summary of the generated results using a news ticker effect.}
    \label{fig:preview_summary}
\end{minipage}
\end{figure}



\paragraph{\underline{Results of Microtasks}} To support \textit{Goal 2.2} we display all resulting diagram nodes using a hollow shape (filled in white), including \textbf{\textcolor{keyword}{\textit{keyword}}}, \textbf{\textcolor{concept}{\textit{concept}}}, and \textbf{\textcolor{sticky_note}{\textit{sticky note}}}. The border colour indicates the specific microtask each resulting node belongs to. A user can choose to discard or accept a resulting node; upon acceptance it will be displayed normally, the same as user-created diagram nodes.

Apart from the two icons for accepting or discarding the diagram node, to support \textit{Goal 2.3}, we design a ``question mark'' icon (see \autoref{fig:feedback}) that users can click on to request an explanation for this specific generation. For each resulting node, we also provide three heuristic suggestions for users to request a regenerated node, ``Be creative'', ``Be more specific'', and ``Be brief''. These suggestions will pop up if users hover over those three icons (see \autoref{fig:feedback}).

\subsubsection{The Task Board}
The task board of \textit{Polymind} enables users to configure and delegate microtasks, to support our \textit{Goal 2}. The task board borrows the design of Trello \footnote{https://trello.com/} board, which consists of a list of ``task cards''. Users can click on the ``add'' icon to delegate a new microtask, or toggle the ``visibility'' switch on the upper-right corner to hide all task headers on the diagramming canvas.

\paragraph{\underline{Task Card}}
A task card contains specifications of a microtask, including the task name, input type, output type, initiative mode (global), visibility (global), and prompts to communicate with the LLM (\autoref{fig:task_card}). The name of the microtask is always displayed in the upper-left corner, each indicated by a distinctive colour (same as resulting diagrams or labels on task headers).

Users can browse through task specifications by clicking on the ``page turn'' icon at the bottom; delete the microtask by clicking on the ``delete'' icon; toggle visibility of all resulting diagrams on the canvas by clicking on the ``visibility'' icon; toggle the initiative modes (see \ref{task_initiative}) by clicking on the task name (the text label). They can also change the specifications of each microtask (see \ref{task_configuration}).

\begin{figure}[t!]
\centering
% \begin{subfigure}{\linewidth}
\begin{subfigure}{.32\linewidth}
    \centering
    % \includegraphics[width=\linewidth]{figures/taskcard1.png}
    \includegraphics[width=\linewidth]{figures/taskcard1.png}
    \subcaption{The first page}
    \label{fig:task_card_page1}
\end{subfigure}
\begin{subfigure}{.32\linewidth}
    \centering
    \includegraphics[width=\linewidth]{figures/taskcard2.png}
    % \includegraphics[width=.8\linewidth]{figures/taskcard2.png}
    \subcaption{The second page}
    \label{fig:task_card_page2}
\end{subfigure}
\begin{subfigure}{.32\linewidth}
    \centering
    % \includegraphics[width=.75\linewidth]{figures/taskcard3.png}
    \includegraphics[width=\linewidth]{figures/taskcard3.png}
    \subcaption{Toggling initiative modes}
    \label{fig:task_card_inactive}
\end{subfigure}
\caption{The task card of a microtask. The text label of the task name is indicated with a distinctive colour. (a) By default, the card shows information of 
% \ref{fig:task_card_page1} 
% the information of 
its input and output types, which are configurable. (b) Users can click on the arrow at the bottom of the card to turn pages, and edit prompts in the second page. The user can also switch between predefined prompts. (c) The user can click on the text label to toggle initiative modes globally, or the ``visibility'' icon to display or hide all resulting diagrams.}
\label{fig:task_card}
\end{figure}



% \begin{figure*}[ht]
% \centering
% \includegraphics[width=\linewidth]{figures/task_cards.png}
% % \centering\captionsetup{width=\linewidth,font={small}}
% % \begin{subfigure}{.32\textwidth}
% %     \centering
% %     \includegraphics[width=.95\linewidth]{figures/task_card1.png}
% %     \subcaption{}
% %     \label{fig:task_card_page1}
% % \end{subfigure}
% % \begin{subfigure}{.32\textwidth}
% %     \centering
% %     \includegraphics[width=.95\linewidth]{figures/task_card2.png}
% %     \subcaption{}
% %     \label{fig:task_card_page2}
% % \end{subfigure}
% % \begin{subfigure}{.32\textwidth}
% %     \centering
% %     \includegraphics[width=.95\linewidth]{figures/task_card_inactive.png}
% %     \subcaption{}
% %     \label{fig:task_card_inactive}
% \caption{The task card of a microtask. The page (a) contains information of its input and output types, and the page (b) and (c) display the prompt. The text label of the task name is indicated with a distinctive colour. A user can 1) change input or output types; 2) manually change prompts; 3) switch between predefined prompts; 4) delete the microtask; 5) click on the ``visibility'' icon the display or hide all resulting diagrams;  6) click on the text label to toggle initiative modes (as shown in page (d) and (e)). }
% \end{figure*} \label{fig:task_card}

\paragraph{\underline{Default Microtasks}} There are 6 predefined microtasks in \textit{Polymind}, which are \BboxS{\textcolor{white}{Brainstorm}}, \EboxS{\textcolor{white}{Elaborate}}, \SboxS{\textcolor{white}{Summarise}}, \DboxS{\textcolor{white}{Draft}}, \FboxS{\textcolor{white}{Freewrite}}, and \AboxS{\textcolor{white}{Associate}}, as specified in our \textit{Goal 1}. Each microtask has a default input and output type, and prompt templates to communicate with the LLM. In practice, when operating on an input node, each microtask will replace the ``[placeholder]'' with the text in that node to prompt the LLM. For details, we refer our readers to \autoref{table:microtasks}. These defaults can all be changed later as per users' needs.

For the input type of ``nodes'', the microtask will, given a node on the canvas, sample another nearby node to perform an operation, and generated diagrams will be linked to both nodes. For the ``section'' input, the microtask will calculate the outline of all nodes (similar to \cite{lu2018inkplanner}) within the section by performing a depth-first search (DFS) of all non-leaf nodes. For example, for a section with a standalone keyword ``creativity'', and a root node ``writing'' connecting to another two keywords, ``drafting'', ``proofreading'', the resulting DFS sequence will be:
\begin{quote}
Writing

 -\ \ \ Drafting
 
 -\ \ \ Proofreading
 
 Creativity
\end{quote}

\subsubsection{The Toolbar}
The toolbar comprises four icons, a pile of sticky notes, and an ellipse representing a concept (see \autoref{fig:teaser}). Users can drag a sticky note or a concept (the ellipse) from the toolbar to the diagramming canvas, or click on the ``text'' icon and then click on the canvas again to add a keyword.
The two leftmost icons are used for connecting diagram nodes using arrows (directed) and lines (undirected). Users can click on the icon and select anchor points of a node to connect with another. There is also an icon for sectioning where users can click on it, and draw a rectangle over diagrams as a section.

\subsection{Processing a Proactive Microtask}
To achieve our \textit{Goal 3}, we introduce two initiative modes: proactive and reactive (\autoref{fig:task_status}). In the reactive mode, the microtask will function similarly to a button, and will not execute until a user clicks on its label on a task header. Those reactive microtasks will be displayed using a gray colour on the task header. By default, all microtasks are activated and run automatically in parallel with user activities.
\revision{
In practice, we sample every \textit{x} ($x=5$ in practice) seconds a diagram node for each input type: \textbf{\textcolor{keyword}{\textit{keyword}}}, \textbf{\textcolor{concept}{\textit{concept}}}, \textbf{\textcolor{sticky_note}{\textit{sticky note}}}, and \textbf{\textcolor{section}{\textit{section}}}.
Each microtask will operate on the sampled diagram corresponding to its input type, if it does not have unchecked notifications and is not displaying results on that diagram.
}
Results of these microtasks will be displayed as notifications and previews before users manually expand all resulting diagrams.

\subsubsection{Inferring User Attention}
In an earlier pilot study of our system, we found that users might lose track of generation results if a microtask was constantly operating on diagram nodes far away from users' focus of attention. Therefore, for our \textit{Goal 2.1}, we want each proactive microtask to infer the user's attention and mainly operate on diagrams of a user's focus. We make the assumption that diagram nodes that can be most easily selected are nodes of users' focus. In other words, the wider the node, or the closer to the mouse cursor, the more likely the node is the focus of users' attention, and operating on it will more likely make users aware. In practice, every \textit{x} seconds, for each node among each of the 5 types of input, we compute an index of difficulty ($ID$) according to Fitts' law \cite{fitts1954information}:
\begin{equation*}
    ID = \log_2 \frac{2D}{W}
\end{equation*}
where $D$ denotes the distance between the current mouse position and the node centre, and $W$ denotes the width of the node.

However, if we choose the node with the lowest $ID$ each time, we will end up always choosing the same node for operation if a user barely moves the mouse within a period of time. Therefore, in practice, we sample a node from a uniform distribution where the probability of a node $i$ being sampled as an input type $T$ is computed as:
\begin{equation*}
p(i, T) = \frac{ID_i}{\sum_{j \in T} ID_j}
\end{equation*}
$T$ includes \textbf{\textcolor{keyword}{\textit{keyword}}}, \textbf{\textcolor{concept}{\textit{concept}}}, \textbf{\textcolor{sticky_note}{\textit{sticky note}}}, and \textbf{\textcolor{section}{\textit{section}}}.
\revision{For microtasks that operate on two \textbf{\textit{nodes}} (e.g., the default \AboxS{\textcolor{white}{Associate}}), we first sample a primitive diagram, and then randomly sample a nearby diagram for prompting.}
%and \textbf{\textit{nodes}}.

\subsubsection{Notifications and Previews of Microtask Results}
We use two levels of attention draw features: notifications and previews of microtask results to support \textit{Goal 2.1}, so that results ready for display will not intrude users' diagramming process.

\begin{figure*}[ht]
% \centering\captionsetup{width=\linewidth,font={small}}
\includegraphics[width=\textwidth]{figures/notifications_previews4.png}
\caption{The processing of a proactive microtask. Once results are obtained from the LLM, it will first ``draw a curtain'' over the task header to notify users, displaying key points of the generation. Users can click on the ``expand'' icon on the ``curtain'' for a quick view. If a user does not expand these result, it will be marked as unread notifications. Once users hovers over the task header, a preview panel will pop up, showing a preview of all unread results.}
\label{fig:notifications_and_previews}
\end{figure*}

\paragraph{\underline{Notifications}}
To notify the user once results of a proactive microtask are ready, it will first ``draw a curtain'' (see \autoref{fig:notifications_and_previews}) over the task header of the node being operated. The curtain will be filled with a distinctive colour indicating which microtask those results are from, and meanwhile, display key points summarising the generated results. Users can click on the ``expand'' icon on the right to display results, and then click on the ``collapse'' icon to hide them. If the user does not perform any operations, the curtain will collapse, and those results will be marked as ``unread'' with a small red circle on the upper-left corner as a sign of notification. (the red circle in \autoref{fig:notifications_and_previews}).

\paragraph{\underline{Previews}}
Once there are unread results, the preview panel of a task header will store a preview of these results. When a user hovers over the task header, the preview panel (should there be any results unread) will pop up, displaying key points of results of each microtask, using labels filled with distinctive colours (see \autoref{fig:notifications_and_previews}). Once a user hovers over a specific label for 1.5 seconds, the key point will turn into a brief summary of the microtask results, displayed using effects resembling a news ticker (see \autoref{fig:preview_summary})

% \begin{figure}[ht]
\centering
\includegraphics[width=.5\linewidth]{figures/preview_summary2.png}
\caption{Once the mouse hovers over key points of a microtask on the preview panel for 1.5 seconds, the system will present a summary of the generated results using a news ticker effect.}
\label{fig:preview_summary}
\end{figure}

\subsection{Managing Microtasks}
To support our \textit{Goal 2} and \textit{Goal 3}, we design a set of features to support configuring existing microtasks, and delegating new microtasks.

\begin{figure*}[ht]
\centering
\includegraphics[width=\textwidth]{figures/microtask_status_v6.png}
% \centering\captionsetup{width=0.64\linewidth}
% \begin{subfigure}{0.32\textwidth}
%     \centering
%     \includegraphics[width=\linewidth]{figures/status1.png}
%     \subcaption{normal mode}
%     \label{fig:status1}
% \end{subfigure}
% \begin{subfigure}{0.32\textwidth}
%     \centering
%     \includegraphics[width=\linewidth]{figures/status2.png}
%     \subcaption{notification}
%     \label{fig:status2}
% \end{subfigure}
% \begin{subfigure}{0.32\textwidth}
%     \centering
%     \includegraphics[width=\linewidth]{figures/status3.png}
%     \subcaption{display mode}
%     \label{fig:status3}
% \end{subfigure}
% \begin{subfigure}{0.45\textwidth}
%     \centering
%     \includegraphics[width=.85\linewidth]{figures/status4.png}
%     \subcaption{inactive mode}
%     \label{fig:status4}
% \end{subfigure}
% \begin{subfigure}{0.45\textwidth}
%     \centering
%     \includegraphics[width=.85\linewidth]{figures/status5.png}
%     \subcaption{displaying results in the inactive mode (when clicked)}
%     \label{fig:status5}
% \end{subfigure}
\caption{Different status of a microtask on a particular node.}
\label{fig:task_status}
\end{figure*}
\subsubsection{Toggling Visibility}
For the purpose of \textit{Goal 2.2}, users can toggle the visibility of microtask results both globally and locally. By clicking on the ``visibility'' icon on each task card \autoref{fig:task_card_page1}, users can display all corresponding generations on the canvas. Users can also click on the task name label on each task header \autoref{fig:task_status} to display generations corresponding to the specific node locally.

For a reactive microtask, generations are only requested once users click on the label on a task header. On such occasions, the label turns black once generations are ready, and users can click on the label again to hide all generations. For a proactive microtask, besides clicking on the ``expand'' icon on the ``curtain'' (\autoref{fig:notifications_and_previews}), users can also click on the label on task headers to see all resulting diagrams of a microtask. In this case, the microtask on this node is in ``display'' mode, and the text on the label is underlined. Should there be unread results, there would also be an ``expand all'' icon that users can click on to show the results of all microtasks. Should there be any microtask in ``display'' mode, there would also be a ``close all'' icon for users to hide the results of all microtasks.

\subsubsection{Toggling Initiative Modes} \label{task_initiative}
The initiative modes (proactive vs. reactive) can also be toggled both globally and locally for our \textit{Goal2} and \textit{Goal 3}. Users can either click on the task name label on the ``task card'' to deactivate it globally (\autoref{fig:task_card_inactive}), or control-click labels on each task header to deactivate the microtask for a specific node (\autoref{fig:task_status}). Once a microtask is globally deactivated, the corresponding label will turn gray and function solely as a button on all task headers, provided that it is not in ``display'' mode and has no notifications. Task headers of newly created diagrams will also have this microtask turned off.

\subsubsection{Task Configuration and Delegation}
To support our \textit{Goal 2}, we allow users to specify microtask requirements, including input and output types, and prompts.
We also enable users to rapidly create and delegate a new microtask to the LLM.

\paragraph{\underline{Configuring Task Specifications}} \label{task_configuration}
Users can change the input and output types of a microtask by clicking on the left and right arrow triangle icon (\autoref{fig:task_card}a). We currently support four input types (\textbf{\textcolor{keyword}{\textit{keyword}}}, \textbf{\textcolor{concept}{\textit{concept}}}, \textbf{\textcolor{sticky_note}{\textit{sticky note}}}, \textbf{\textcolor{section}{\textit{section}}}) and three output types (\textbf{\textcolor{keyword}{\textit{keyword}}}, \textbf{\textcolor{concept}{\textit{concept}}}, \textbf{\textcolor{sticky_note}{\textit{sticky note}}}). Users can also change prompts by double-clicking the text box of the prompt (\autoref{fig:task_card}b). For some microtasks, they can also switch between predefined prompt examples. Each prompt should have a ``[placeholder]'' to be filled with the text of a node (\autoref{fig:task_card}b).

\paragraph{\underline{Delegating a New Microtask}}
For our \textit{Goal 2.3}, a user can click the ``add'' icon on the task board to create a new microtask (\autoref{fig:teaser}B). After clicking, the task board will spawn a new card for users to specify the task name and the example prompt. Users can later change input and output type, or toggle visibility or initiative on a newly added task card after clicking ``confirm''. When a user clicks to specify a task name, the system will prompt the LLM to suggest a task name. Once a task name is specified, the system will also request an example prompt for users from the LLM. \autoref{fig:add_new_task} illustrates the whole process of adding a new microtask.

% \begin{figure*}[ht]
% \begin{subfigure}{0.45\textwidth}
%     \centering
%     \includegraphics[width=.85\linewidth]{figures/new_task1.png}
%     \subcaption{}
%     \label{fig:new_task1}
% \end{subfigure}
% \begin{subfigure}{0.45\textwidth}
%     \centering
%     \includegraphics[width=.85\linewidth]{figures/new_task2.png}
%     \subcaption{}
%     \label{fig:new_task2}
% \end{subfigure}
% \begin{subfigure}{0.45\textwidth}
%     \centering
%     \includegraphics[width=.85\linewidth]{figures/new_task3.png}
%     \subcaption{}
%     \label{fig:new_task3}
% \end{subfigure}
% \begin{subfigure}{0.45\textwidth}
%     \centering
%     \includegraphics[width=.85\linewidth]{figures/new_task4.png}
%     \subcaption{}
%     \label{fig:new_task4}
% \end{subfigure}
% \caption{}
% \end{figure*} \label{fig:add_new_task}

\begin{figure*}[ht]
% \centering\captionsetup{width=\linewidth,font={small}}
\includegraphics[width=\textwidth]{figures/add_new_task_v3.png}
\caption{The workflow of delegating a new microtask. (a) The user first clicks on the ``add'' icon and a new card will appear. (b) After user inputs a task name, the system requests an example prompt from the LLM. (c) User clicks ``confirm'' to add the microtask.}
\label{fig:add_new_task}
\end{figure*}

\subsection{Implementation}
\textit{Polymind} was implemented using Javascript, React \footnote{https://react.dev/} and react-konva \footnote{https://konvajs.org/docs/react/Intro.html}. The backend LLM was ChatGPT (GPT-4), and the temperature was set to $0.7$.
We set $x=5$ based on our pilot study, i.e., every 5 seconds, we sample a diagram node for each of the 5 input types to be processed by proactive microtasks. If \textit{x} were too small, users would be overwhelmed with notifications of new results; if too large, there would be noticeable latency, and users will often end up waiting for results. Once the user confirms an update of a diagram node's text content, a re-sampling of the corresponding input type will be executed immediately.

To reduce the computational cost, we quit prompting ChatGPT with the sampled diagram node if a proactive microtask has unchecked results (i.e., notifications) or expanded diagrams on it.
To make generations less random, in each prompt we added constraints on its output.
\revision{For \textbf{\textcolor{keyword}{\textit{keyword}}}, we request 3 generations, each no more than 3 words; for \textbf{\textcolor{concept}{\textit{concept}}} we request 3 generations no more than 5 words; for \textbf{\textcolor{sticky_note}{\textit{sticky note}}} we request 1 generation no more than 150 words.}
Whenever ChatGPT returns results, we preserve the previous dialogue in each resulting diagram node. Once users request an explanation or a regeneration of the node, we further prompt the ChatGPT based on the saved dialogue (see \ref{appendix_feedback}). When the system requests a microtask name, we use names of predefined microtask as a few-shot template to prompt the ChatGPT. When the system requests a prompt for a new microtask, we use name-prompt pairs of predefined microtasks as a few-shot template. We refer our readers to \ref{appendix_newtask} for more details.
\section{evaluation}
% justification
We conducted an exploratory evaluation of \textit{Polymind}, focusing on the usability, creativity, and usefulness of its parallel collaboration workflow.
More specifically, the study aimed to answer the following two questions:
\begin{enumerate}
    \item Is \textit{Polymind} easy to use, and useful for prewriting?
    \item How effective are \textit{Polymind}'s parallel collaboration workflow and microtasking features for supporting creativity in prewriting?
\end{enumerate}

% 
\begin{table}[htbp!]
\resizebox{\columnwidth}{!}{%
\begin{tabular}{@{}l|ccc|c@{}}
 & Liberal & Moderate & Conservative & Total \\ \hline
Female & 223 & 114 & 45 & 382 \\
Male & 102 & 78 & 53 & 233 \\
Prefer not to say & 2 & 0 & 0 & 2 \\ \hline
Total & 327 & 192 & 98 & 617
\end{tabular}%
}
\caption{Annotator Demographics. All annotators are based in the United States. The table shows the number of annotators across ideology and sex categories, as self-reported to Prolific. The mean age is 38.3 (SD=12.7), and 45 annotators are immigrants (7.3\%).}
\label{tab:demographics}
\end{table}



\subsection{Participants}
We used convenience sampling to recruit 10 participants (4 male, 6 female) mainly from local universities. All participants are L2 English speaker.
\revision{Similar to our formative study, we mainly reached out to participants with creative writing experience or related majors, though not necessarily expert writers. All participants had experience using prewriting or diagramming tools such as Figma or Miro, but reported limited knowledge or experience of AI or programming.}
We refer to them as V1-10. For each participant, we offered a coupon equivalent to 50 HKD.

\subsection{Study Design}

\subsubsection{Tasks}
To evaluate our \textit{Polymind} in both divergent and convergent thinking phases, we divide a creative pre-writing task into two sessions: story ideation, and story outlining. In the story ideation, participants were required to brainstorm as many distinct storylines as possible. Each storyline only needs to be one or two sentences long that specifies main characters, events, locations, time, etc. In the story outlining session, participants were asked to pick one favourite storyline from the previous session, and draft a rough outline with as many details as possible. An outline needs to specify a clear structure (such as the classic beginning-climax-ending structure), and key events along the structure.

\subsubsection{Conditions}
The study compared our system, \textit{Polymind} to a turn-taking, conversational interface, plus \textit{Polymind}'s diagramming canvas, as a baseline using two creative writing prompts. \revision{The baseline allows users to take notes and keep track of conversational results using the canvas, but does not require nor allow users to interact with the AI via diagrams}. Specifically, two system conditions were used:
\begin{itemize}
    \item \textbf{GPT-4 \& Canvas} OpenAI ChatGPT-4 interface plus \textit{Polymind}'s diagramming canvas. All microtasking features were turned off.
    \item \textbf{\textit{Polymind:}} full version of \textit{Polymind} with six predefined default microtasks.
\end{itemize}
Two creative writing prompts were chosen:
\begin{itemize}
    \item Write a story where your character is traveling a road that has no end, either literally or metaphorically.
    \item Write a story in which a character is running away from something, literally or metaphorically.
\end{itemize}
We used a Latin square experimental design~\cite{ryan2007modern} to achieve a balanced sequence of writing prompts and system conditions.

\subsubsection{Study Procedure}
After giving consent to our study, users were invited to use two systems in turn to complete two sessions: story ideation and story outlining, given two different writing prompts. Each session lasted 12 minutes, and users were given time to transfer their prewriting results (generations, diagrams, or merely thoughts and ideas in their minds) to another document after each session concluded. Before using \textit{Polymind}, we walked the participants through all of its features and offered approximately 10 minutes for them to try out the system.

After two sessions concluded for a system condition, each participant was required to complete a survey, including a NASA Task Load Index (NASA-TLX)~\cite{hart1986nasa}, and three dimensions (2 questions each) of Creativity Support Index (CSI)~\cite{cherry2014quantifying}: Enjoyment, Exploration, and Expressiveness. Two dimensions (Collaboration \& Immersion) of the original CSI were dropped because they were irrelevant to the two questions we sought to answer.

To evaluate the final results (outlines), we invited two expert writers to score participants outlines using Torrance Test of Creative Writing (TTCW)~\cite{chakrabarty2023art}, instead of the self-rated score of CSI. \revision{We did a quick interview after the scoring to ask about their general feedback, and to compare results of two conditions and pick out examples with most noticeable differences.}
One of our expert is a professional fiction writer and has a doctoral degree in film studies. She used to be a screenwriter before becoming a fiction writer. The other expert is an AO3 (Archive of Our Own) writer that has posted over 400K words and accumulated over 250K views.

After using two system conditions (4 sessions in total), each participant was then required to complete a survey to rate the usefulness of each \textit{Polymind} features. We then conducted a brief interview (5-10 min) to ask about their overall use experience, feedback on \textit{Polymind}'s workflow, perceptions of creativity support, and perceived differences between the two workflows and their impact on the final results.

The whole study procedure lasted around 2 hours, and was screen recorded. The interviews were audio-taped and transcribed for analysis.

% % \begin{figure}[htb]
% \begin{subfigure}[h]{\linewidth}
%     \centering
%     \includegraphics[width=.64\linewidth]{figures/Polymind_NASA-TLX.png}
% \end{subfigure}
% \begin{subfigure}[h]{\linewidth}
%     \centering
%     \includegraphics[width=.64\linewidth]{figures/Baseline_NASA-TLX.png}
% \end{subfigure}
% \caption{NASA Task Load Index of the \textit{Polymind} and Baseline conditions (the lower, the better).}
% \label{fig:usability}
% \end{figure}

\begin{figure*}[htb]
\centering
\includegraphics[width=.64\linewidth]{figures/NASA_TLX.pdf}
\caption{NASA Task Load Index of \textit{Polymind} and Baseline conditions (the lower, the better).}
\label{fig:usability}
\end{figure*}

\subsection{Study Results}
In this subsection, we report the findings of our study. On balance, \textit{Polymind}'s parallel microtasking workflow granted more customizability and was more controllable. Therefore users reported a stronger sense of agency, ownership of results, and a higher level of expressiveness. The microtasking workflow could also help quickly expand idea trees through ``chaining''-like effects~\cite{wu2022ai}.

% \begin{figure}[htb]
% \begin{subfigure}[h]{\linewidth}
%     \centering
%     \includegraphics[width=.64\linewidth]{figures/Polymind_NASA-TLX.png}
% \end{subfigure}
% \begin{subfigure}[h]{\linewidth}
%     \centering
%     \includegraphics[width=.64\linewidth]{figures/Baseline_NASA-TLX.png}
% \end{subfigure}
% \caption{NASA Task Load Index of the \textit{Polymind} and Baseline conditions (the lower, the better).}
% \label{fig:usability}
% \end{figure}

\begin{figure*}[htb]
\centering
\includegraphics[width=.64\linewidth]{figures/NASA_TLX.pdf}
\caption{NASA Task Load Index of \textit{Polymind} and Baseline conditions (the lower, the better).}
\label{fig:usability}
\end{figure*}

\subsubsection{Usability \& Usefulness}
Despite efforts of microtask and diagram management and the potential learning curve, to our surprise, \textit{Polymind} was almost perceived as easy to use as the ChatGPT interface, and significantly reduced frustration towards generated results (as shown \autoref{fig:usability}). ChatGPT interface was demanding mainly due to efforts of digesting longer text information (e.g., V4-5), and typing and iteratively refining lengthy prompts (e.g., V1-2). \revision{For \textit{Polymind}, the main cause of demand was said by V2-3, \& V10 to be the efforts of mannually managing the canvas, adjusting its layout, and progressing through diagrams.}
Notably, V1 \& V2 said that \textit{Polymind}'s interface was easier to navigate, and easier to read, because its generations were mainly short phrases, and had structures (including lines, sections, \& microtask colours). 
Besides, the randomness of ChatGPT generations also caused higher frustration level and worse perceived performance than \textit{Polymind} among some participants (e.g., V5, V8).

The key features of \textit{Polymind} were mainly perceived useful for prewriting, as shown in \autoref{fig:usefulness}. Of them awareness-related features, such as notifications, previews, and initiative modes, were found most controversial, which revealed the tension of our \textit{Goal 2.1} and being overall non-intrusive. V3 felt a proactive microtask was particularly annoying and intrusive, but we observed that all other participants left key microtasks proactive. Some said (e.g., V5, V7) they would need proactive microtasks in divergent thinking phases for quick ideas, but sometimes did not want to be interrupted while thinking. Therefore, most participants thought the feature of switching intiative modes particularly helpful.

% \begin{figure}[ht!]
\centering
\begin{minipage}[b]{.48\linewidth}
    \vspace{0pt}
    \includegraphics[width=\linewidth]{figures/NASA_TLX.png}
    \caption{NASA Task Load Index of \textit{Polymind} and Baseline conditions (the lower, the better).}
    \label{fig:usability}
\end{minipage}
\begin{minipage}[b]{.5\linewidth}
    \vspace{0pt}
    \includegraphics[width=\linewidth]{figures/perceived_usefulness.png}
    \caption{The perceived usefulness of \textit{Polymind} features}
    \label{fig:usefulness}
\end{minipage}
\end{figure}
\begin{figure*}[htb]
\centering
\includegraphics[width=.85\linewidth]{figures/perceived_usefulness.png}
\caption{The perceived usefulness of \textit{Polymind} features}
\label{fig:usefulness}
\end{figure*}

% \begin{figure*}[htb]
% \begin{subfigure}[h]{.49\textwidth}
%     \includegraphics[width=.975\linewidth]{figures/Statistics of Resulting Diagrams - Task 1.pdf}
% \end{subfigure}
% \begin{subfigure}[h]{.49\textwidth}
%     \includegraphics[width=.975\linewidth]{figures/Statistics of Resulting Diagrams - Task 2.pdf}
% \end{subfigure}
% \caption{Number of resulting nodes and \textit{Polymind} contribution in two tasks}
% \label{fig:result_nodes}
% \end{figure*}

\subsubsection{Creativity Support}
In terms of creativity, participants generally felt \textit{Polymind} was more supportive (see \autoref{fig:CSI}), but the results were not significant ($P_{Enjoyment}=0.67$, $P_{Exploration}=0.19$, $P_{Expressiveness}=0.05$). Notably, the expressiveness dimension has almost shown significance, as many (e.g., V3 \& V5) reported that \textit{Polymind}'s diagramming interface and microtasking workflow put them in dominant roles that encouraged them to freely express their brief ideas. In terms of results, two conditions produced similar number of ideas ($Polymind_{median}=3$, $Polymind_{stdev}=1.06$, $Baseline_{median}=3$, $Baseline_{stdev}=8.51$) in the ideation session.

In addition, experts' scores showed that the baseline condition produced outlines that were able to pass 5.7 TTCW tests, as compared to \textit{Polymind}'s 4 tests ($P=0.23$) (see \autoref{fig:CSI}).
\revision{
Experts did not particularly mention any noticeable differences in quality between two conditions except that \textit{Polymind}'s results were much shorter and lacked details to pass some tests. Besides, they both expressed concern of overused or clichéd results. One expert said she was initially interested by V2's story (baseline), but only to find out that it was from \textit{The Vampire Diaries}
}
This is expected, as ChatGPT interface could quickly generate ``\textit{complete and detailed outlines with simple prompts}'' (V2), while \textit{Polymind} usually encouraged users to make progress in diagrams with limited words.
Although some (e.g., V8) noted that they could still generate a complete outline using \textit{Polymind}, but they simply did not want to, because they would like to take control, and create a story from their own fragmented ideas, instead of borrowing all results from ChatGPT.
\begin{figure}[ht!]
\centering
\begin{minipage}[t]{.4745\linewidth}
    \vspace{0pt}
    \includegraphics[width=\linewidth]{figures/CSI.png}
\end{minipage}
\begin{minipage}[t]{.32\linewidth}
    \vspace{0pt}
    \includegraphics[width=\linewidth]{figures/TTCW.png}
\end{minipage}
\caption{The results of Creativity Support Index (CSI) and Torrance Test of Creative Writing (TTCW)}
\label{fig:CSI}
\end{figure}

\subsubsection{Microtask Usage: Quick Chaining and Idea Expansion}
\revision{All default microtasks have been applied by 10 participants to produce their final results, as shown in \autoref{tab:usage}. In some cases users might have default microtasks slightly edited. For example, V2 changed the prompt of \BboxS{\textcolor{white}{Brainstorm}} and switched the output type to \textbf{\textcolor{sticky_note}{\textit{sticky note}}} to request detailed settings of a story. Four participants have delegated a total of 8 customized microtasks. For example, V8 delegated \CboxS{\textcolor{white}{Beginning}} \& \CCboxS{\textcolor{white}{Climax}} to generate a beginning and climax of a given storyline. V4 delegated \CboxS{\textcolor{white}{Juice}} to juice up a given story in a \textbf{\textcolor{sticky_note}{\textit{sticky note}}} with more details.}

\revision{We also found participants came up with creative and efficient ways of using a combination of microtasks}. By leaving some microtasks in the proactive mode, \textit{Polymind} can easily perform the ``chaining'' operation~\cite{wu2022ai} to expand users' ideas in a tree-like structure. During this process, multiple distinct microtasks could contribute simultaneously in parallel to users' main operations, which made the collaboration more efficient and creative.
Some participants complimented that the parallel microtasks were like ``\textit{a mature pipeline that needs little efforts}'' (V5), ``\textit{as if splitting (brainstorming) indefinitely}'' (V6). For example, V2 mainly used two proactive microtasks: \FboxS{\textcolor{white}{Freewrite}} \& \SboxS{\textcolor{white}{Summarise}} during the story ideation session.
\revision{He later explained that \FboxS{\textcolor{white}{Freewrite}} was used to quickly generate stories in a \textbf{\textcolor{sticky_note}{\textit{sticky note}}} given a few keywords or concepts within a \textbf{\textcolor{section}{\textit{section}}}, while \SboxS{\textcolor{white}{Summarise}} presented brief summaries in a \textbf{\textcolor{sticky_note}{\textit{sticky note}}} of \FboxS{\textcolor{white}{Freewrite}}'s generations so that he would not need to read whole stories.}

V10 instead was mainly using \BboxS{\textcolor{white}{Brainstorm}} and \EboxS{\textcolor{white}{Elaborate}} to expand her ideas in brief keywords and concepts. \revision{She used \BboxS{\textcolor{white}{Brainstorm}} to request related ideas and \EboxS{\textcolor{white}{Elaborate}} to provide concrete examples of an idea.} In a comparison to the ChatGPT interface, she commented that,
\begin{quote}
    ``\textit{I feel that ChatGPT often generated something irrelevant, and missed my expectations. But this system (Polymind) stuck to my main concept by generating relevant ideas. Although the results were only brief keywords, but it was fast. It could produce a huge idea tree within a short period of time, and you could easily find something intriguing and figure out a coherent story.}''.
\end{quote}

\newtcbox{\Bbox}{on line,
  colframe=brainstorm,colback=brainstorm,
  boxrule=0.5pt,arc=1pt,boxsep=0pt,left=2pt,right=2pt,top=2pt,bottom=2pt}
\newtcbox{\Sbox}{on line,
  colframe=summarise,colback=summarise,
  boxrule=0.5pt,arc=1pt,boxsep=0pt,left=2pt,right=2pt,top=2pt,bottom=2pt}
\newtcbox{\Ebox}{on line,
  colframe=elaborate,colback=elaborate,
  boxrule=0.5pt,arc=1pt,boxsep=0pt,left=2pt,right=2pt,top=2pt,bottom=2pt}
\newtcbox{\Dbox}{on line,
  colframe=draft,colback=draft,
  boxrule=0.5pt,arc=1pt,boxsep=0pt,left=2pt,right=2pt,top=2pt,bottom=2pt}
\newtcbox{\Fbox}{on line,
  colframe=freewrite,colback=freewrite,
  boxrule=0.5pt,arc=1pt,boxsep=0pt,left=2pt,right=2pt,top=2pt,bottom=2pt}
\newtcbox{\Abox}{on line,
  colframe=associate,colback=associate,
  boxrule=0.5pt,arc=1pt,boxsep=0pt,left=2pt,right=2pt,top=2pt,bottom=2pt}
\newtcbox{\Cbox}{on line,
  colframe=custom,colback=custom,
  boxrule=0.5pt,arc=1pt,boxsep=0pt,left=2pt,right=2pt,top=2pt,bottom=2pt}
\newtcbox{\CCbox}{on line,
  colframe=custom2,colback=custom2,
  boxrule=0.5pt,arc=1pt,boxsep=0pt,left=2pt,right=2pt,top=2pt,bottom=2pt}

\renewcommand{\arraystretch}{1.2}
\begin{table*}[htb]
    \resizebox{\linewidth}{!}{
    \begin{tabular}{c|cc|cc}
    \toprule
    & \multicolumn{2}{c|}{Task \RNum{1}} & \multicolumn{2}{c}{Task \RNum{2}} \\
    & default & custom & default & custom \\
    \midrule
    V1 & \Ebox{\textcolor{white}{Elaborate}} \Fbox{\textcolor{white}{Freewrite}} & \Cbox{\textcolor{white}{Characteristics}} & \Sbox{\textcolor{white}{Summarise}} \Fbox{\textcolor{white}{Freewrite}} & \\
    V2 & \Bbox{\textcolor{white}{Brainstorm}} \Sbox{\textcolor{white}{Summarise}} \Fbox{\textcolor{white}{Freewrite}} & & \Ebox{\textcolor{white}{Elaborate}} \Fbox{\textcolor{white}{Freewrite}} & \Cbox{\textcolor{white}{Structure}} \\
    V3 & \Bbox{\textcolor{white}{Brainstorm}} \Fbox{\textcolor{white}{Freewrite}} \Dbox{\textcolor{white}{Draft}} \Abox{\textcolor{white}{Associate}} & & \Bbox{\textcolor{white}{Brainstorm}} \Sbox{\textcolor{white}{Summarise}} \Fbox{\textcolor{white}{Freewrite}} & \\
    V4 & \Bbox{\textcolor{white}{Brainstorm}} & \Cbox{\textcolor{white}{Structure}} \CCbox{\textcolor{white}{Juice}} (up) & \Bbox{\textcolor{white}{Brainstorm}} & \Cbox{\textcolor{white}{Structure}} \CCbox{\textcolor{white}{Theme}} \\
    V5 & \Bbox{\textcolor{white}{Brainstorm}} \Fbox{\textcolor{white}{Freewrite}} \Abox{\textcolor{white}{Associate}} & & \Bbox{\textcolor{white}{Brainstorm}} \Fbox{\textcolor{white}{Freewrite}} \Abox{\textcolor{white}{Associate}} & \\
    V6 & \Bbox{\textcolor{white}{Brainstorm}} \Abox{\textcolor{white}{Associate}} & & \Bbox{\textcolor{white}{Brainstorm}} \Abox{\textcolor{white}{Associate}} & \\
    V7 & \Bbox{\textcolor{white}{Brainstorm}} & & \Bbox{\textcolor{white}{Brainstorm}} \Ebox{\textcolor{white}{Elaborate}} & \\
    V8 & \Bbox{\textcolor{white}{Brainstorm}} \Dbox{\textcolor{white}{Draft}} \Abox{\textcolor{white}{Associate}} & & \Bbox{\textcolor{white}{Brainstorm}} \Dbox{\textcolor{white}{Draft}} \Fbox{\textcolor{white}{Freewrite}} & \Cbox{\textcolor{white}{Beginning}} \CCbox{\textcolor{white}{Climax}} \\
    V9 & \Bbox{\textcolor{white}{Brainstorm}} \Dbox{\textcolor{white}{Draft}} \Abox{\textcolor{white}{Associate}} & & \Dbox{\textcolor{white}{Draft}} & \\
    V10 & \Bbox{\textcolor{white}{Brainstorm}} \Ebox{\textcolor{white}{Elaborate}} \Abox{\textcolor{white}{Associate}} & & \Bbox{\textcolor{white}{Brainstorm}} \Abox{\textcolor{white}{Associate}} & \\
    \bottomrule
    \end{tabular}}
    \caption{Microtasks used by each participant in \textit{Polymind} condition during two sessions that produced the final results. The input \& output types and prompts of default microtasks might have been edited.}
    \label{tab:usage}
\end{table*}

% \renewcommand{\arraystretch}{1.2}
% \begin{table*}[htb]
%     \resizebox{\linewidth}{!}{\begin{tabular}{c|cc}
%     \toprule
%     & Task \RNum{1} & Task \RNum{2} \\
%     \midrule
%     S1 & \Bbox{\textcolor{white}{Brainstorm}} \Fbox{\textcolor{white}{Freewrite}} \textbf{+} \Dbox{\textcolor{white}{Draft}} & \Bbox{\textcolor{white}{Brainstorm}} \Abox{\textcolor{white}{Associate}} \\
%     S2 & \Bbox{\textcolor{white}{Brainstorm}} \Ebox{\textcolor{white}{Elaborate}} \Fbox{\textcolor{white}{Freewrite}} \textbf{+} \Dbox{\textcolor{white}{Draft}} & \Bbox{\textcolor{white}{Brainstorm}} \Ebox{\textcolor{white}{Elaborate}} \\
%     S3 & \Bbox{\textcolor{white}{Brainstorm}} \Ebox{\textcolor{white}{Elaborate}} \textbf{+} \Dbox{\textcolor{white}{Draft}} & \Ebox{\textcolor{white}{Elaborate}} \textbf{+} \Dbox{\textcolor{white}{Draft}} \\
%     S4 & \Abox{\textcolor{white}{Associate}} \textbf{+} \Dbox{\textcolor{white}{Draft}} & \Bbox{\textcolor{white}{Brainstorm}} \Ebox{\textcolor{white}{Elaborate}} \Abox{\textcolor{white}{Associate}} \\
    
%     S5 & \Bbox{\textcolor{white}{Brainstorm}} \Abox{\textcolor{white}{Associate}} & \Ebox{\textcolor{white}{Elaborate}} \Fbox{\textcolor{white}{Freewrite}} \\
    
%     S6 & \Ebox{\textcolor{white}{Elaborate}} \Fbox{\textcolor{white}{Freewrite}} \Dbox{\textcolor{white}{Draft}} \textbf{+} \Cbox{\textcolor{white}{Custom}} & \Abox{\textcolor{white}{Associate}} \Bbox{\textcolor{white}{Brainstorm}} \Ebox{\textcolor{white}{Elaborate}} \textbf{+} \Cbox{\textcolor{white}{Custom1}} \CCbox{\textcolor{white}{Custom2}} \\
%     S7 & \Abox{\textcolor{white}{Associate}} \Dbox{\textcolor{white}{Draft}} \textbf{+} \Cbox{\textcolor{white}{Custom}} & \Cbox{\textcolor{white}{Custom}} \\
%     S8 & & \Bbox{\textcolor{white}{Brainstorm}} \textbf{+} \Cbox{\textcolor{white}{Custom}} \\
%     S9 & \Ebox{\textcolor{white}{Elaborate}} \Dbox{\textcolor{white}{Draft}} & \Bbox{\textcolor{white}{Brainstorm}} \Fbox{\textcolor{white}{Freewrite}} \Dbox{\textcolor{white}{Draft}} \\
%     S10 & \Bbox{\textcolor{white}{Brainstorm}} \textbf{+} \Cbox{\textcolor{white}{Custom}} & \Ebox{\textcolor{white}{Elaborate}} \\
%     S11 & \Bbox{\textcolor{white}{Brainstorm}} \Sbox{\textcolor{white}{Summarise}} \textbf{+} \Cbox{\textcolor{white}{Custom}} & \Bbox{\textcolor{white}{Brainstorm}} \Dbox{\textcolor{white}{Draft}} \\
%     S12 & \Bbox{\textcolor{white}{Brainstorm}} \Abox{\textcolor{white}{Associate}} \Fbox{\textcolor{white}{Freewrite}} \textbf{+} \Cbox{\textcolor{white}{Custom}} & \Abox{\textcolor{white}{Associate}} \Ebox{\textcolor{white}{Elaborate}} \Dbox{\textcolor{white}{Draft}} \\
%     S13 & \Bbox{\textcolor{white}{Brainstorm}} \Ebox{\textcolor{white}{Elaborate}} \Sbox{\textcolor{white}{Summarise}} \Fbox{\textcolor{white}{Freewrite}} \textbf{+} \Cbox{\textcolor{white}{Custom}} & \Bbox{\textcolor{white}{Brainstorm}} \Ebox{\textcolor{white}{Elaborate}} \Fbox{\textcolor{white}{Freewrite}} \\
%     S14 & \Bbox{\textcolor{white}{Brainstorm}} \Ebox{\textcolor{white}{Elaborate}} \Sbox{\textcolor{white}{Summarise}} & \Bbox{\textcolor{white}{Brainstorm}} \Ebox{\textcolor{white}{Elaborate}} \textbf{+} \Cbox{\textcolor{white}{Custom}} \\
%     \bottomrule
%     \end{tabular}}
%     \caption{Microtasks used by each participant in two tasks to produce the final results}
%     \label{tab:usage}
% \end{table*}

V3 was one of the three participants that showed clear preference for the ChatGPT interface, but she also added that experimenting ideas with \textit{Polymind} were much easier. She explained that \textit{Polymind} ``\textit{had a structure}'' and could perform chaining-like operations easily ``\textit{by using sections}'' and parallel microtasks, while for ChatGPT, ``\textit{combining elements (like some characters, events, or settings) from its generations to re-prompt it was challenging}''. Similarly, V6 noted that brainstorming associations between key events or scenes was much easier with \textit{Polymind} by ``\textit{using several proactive microtasks operating on nodes or sections}''.

\subsubsection{Polymind is More Controllable}
While ChatGPT-4's conversational interface was able to generate long pieces of text with many ideas and details (V2-4, V7-9), most of our participants (V1-2, V4-5, V7-8, V10) mentioned that \textit{Polymind} felt more controllable in a prewriting task. This is because \textit{Polymind} directly operated on diagrams that were often shorter and \revision{thus easier to digest and re-prompt} than ChatGPT-4's conversations, and the canvas progressed in a structured manner with parallel microtasks handling very specific requirements.

The longer generations of the conversational interface were often criticised for being hallucinatory (e.g., ``\textit{not that creative or sensible as it appears}'' -- V7), too random (e.g., ``\textit{not what I expected}'' -- V5, ``\textit{irrelevant}'' -- V10), or ``\textit{mediocre}'' (V2) during a story pre-writing session. In comparison, \textit{Polymind} generations were perceived by many to be relevant (V2, V5, V10), and its microtasks more responsive to users' requests (V5, V8, V10), although it might require some efforts to manage or configure them (V4). \revision{This echoes with the usability score of the two conditions, where \textit{Polymind} was on the same level with the baseline, despite the efforts of managing a diagramming interface.}
V8 said in retrospect that,
\begin{quote}
    ``\textit{I think this (Polymind) would be very helpful for coming up with a story. Cause you can specify your beginning, you can specify your climax, and the ending too... And I think customizing microtasks is also a nice feature... You can really outline everything. While for GPT, you often don't know what is beginning, what is climax or ending.}''
\end{quote}

Additionally, V1, V2 and V8 noted that \textit{Polymind}'s microtasking workflow made it easier to re-prompt. V8 said,
\begin{quote}
    ``\textit{If I want to change something, I know where the part is. Like the character, I only need to change several keywords, like, Oh I'd like the character to be a dragon... I felt that my prompts were actually considered (by microtasks), while GPT sometimes doesn't process all my prompts.}''
\end{quote}
For the conversational interface, it often took multiple iterations to reach a decent draft (V4-5), and each prompt had to be lengthy to change the context (V1-2, V5), which was demanding. V2 also added that the \textit{Polymind} interface was neater because with some parallel microtasks it required little to no efforts of note taking to ask multiple follow-up questions of different ideas.

It is worth noting that, three participants that disliked \textit{Polymind}'s workflows mentioned that it was quite demanding sometimes to configure microtasks (V3-4), and progress in diagrams (V9). V9 said, ``\textit{GPT could generate a lot with a single prompt, while \textit{Polymind} only little by little.}''
\revision{V10 shared similar sentiments. She noted prompting ChatGPT would be much easier than using \textit{Polymind}'s diagrammatic workflow if its generations were not random. However, she added that \textit{Polymind} was in reality less demanding because you could explore more options and easily drop random results.}

\subsubsection{Polymind Affords Agency}
Our participants almost unanimously said that \textit{Polymind} put users in a dominant role, while with the conversational interface they were completely guided by the GPT. This aligns with our \textit{Goal 2} that aims to put humans in a role of managing all microtasks. Participants without any ideas, such as V3 \& V4, generally did not mind following the ChatGPT. This is expected, and agrees with our formative study. However, same as almost all other participants, they particularly mentioned that they felt these results were not their ideas, as ChatGPT generated almost everything. While using \textit{Polymind}, participants said that they had more freedom and control (V3, V6-8), and needed to think a lot (V4-5, V8). 

One of the participants, V5, particularly said he had no trust in AI because ``\textit{it could not be truly creative}''. He therefore became very annoyed with the ChatGPT interface when it did not generate what he expected, saying it was ``\textit{bad usability}'', while attributing ``\textit{good usability}'' to the task management workflow. V7 also expressed concerns for using the conversational interface for brainstorming,
\begin{quote}
    ``\textit{At first sight, it might seem it had generated everything you could think of, but then you'd find many were indeed non-sensical. But I felt I was confined to these generations after reading them. It was especially hard for a novice writer like me to come up with other possibilities. So it felt like it was GPT that was composing a fiction, rather than me.}''
\end{quote}
She later added that her own results from \textit{Polymind} felt more ``\textit{logical}'', and ``\textit{rigorous}''.

Of the participants that said very positively of the ChatGPT interface, V3 stressed that \textit{Polymind} encouraged her to express her own ideas, while ChatGPT did not. That was why she assigned a very low score of expressiveness in the CSI survey when using ChatGPT.
\section{Discussion}
\label{sec:discussion}
\textsc{WWD} is a socio-technical infrastructure that supports the collection of cultural data, in the form of food, in a bottom-up community-led manner. Community members' needs and experiences actively shaped the architecture of WWD. Our data collection platform was constructed to be compatible with the established digital infrastructure and cultural norms of the communities we worked with. The types of data we collected (e.g., the attributes for each dish) were informed by community members who identified what was important to capture about a dish to accurately represent how the dish is prepared and consumed in their culture. 

Building the \textsc{WWD System} in a bottom-up, community-led manner required an immense amount of labour. Data did not simply flood in once the system architecture was built. Core Organisers and Community Ambassadors engaged in \textit{data work}---a socio-technical process through which data about local cuisines was produced. As many social computing scholars have noted, data work is often overlooked despite its essential role in shaping the epistemology of a dataset and consequently the downstream performance of ML systems~\cite{sambasivan2021everyone,ismailEngagingSolidarityData2018,mollerWhoDoesWork2020,scheuermanProductsPositionalityHow2024}. 

In our discussion, we surface the tensions that occurred during data work. These breakdowns in the data work process help us to reveal deeper structural issues in the AI/ML production pipeline that confound bottom-up, community-led approaches to dataset construction. Communities facing representational harms~\cite{weidinger2021ethical} and disparities in quality of service~\cite{shankar2017allocational, de2019doescvworkallocational} face a catch-22 when participating in efforts to improve dataset coverage: They can shoulder the burden of participation or be excluded from model ontology. New technologies, particularly GenAI tools, have been proposed as a way for communities to preserve their culture representation by participating in efforts to contribute data to model training~\cite{heritage7030070}. However, participation does not necessarily entail improved outcomes for communities~\cite{birhanePowerPeopleOpportunities2022}. We point to a difference in ethical frameworks between communities on the African continent with whom we worked and those of large tech companies that build and control GenAI technologies to illuminate why the promises of participation often fall short. 


\subsection{Tensions in data collection}
 
Our results show that there were tensions around data collection. Specifically, issues surrounding image provenance, the accuracy of information about a dish, and the benefits of participation arose throughout the data collection process. These issues reveal deeper structural problems with the AI/ML pipeline. 
\subsubsection{Establishing a clean bill of data provenance}
Recent efforts in participatory ML research attempt to safeguard the labour and intellectual property of community data contributors by creating dataset licenses restricting the use of the community's dataset, which assign ownership and terms of access and use to these datasets~\cite{birhanePowerPeopleOpportunities2022,longpre2024largelicence}. However, for the license to be effective, the data must have a clean bill of provenance. \textsc{World Wide Dishes} was built to be an open-source dataset with a Creative Commons license that could be used for model evaluation. As a result, the WWD dataset had to fulfil strict requirements for data quality, including ensuring that the dataset creators had a right to the images contained within the dataset.  In other words, the creators of \textsc{WWD} must then be able to claim rightful use and ownership over all the images collected as part of the project. However, many of the images that Contributors submitted during the data collection phase were taken from the Internet and lacked proper licensing. As a result, Community Ambassadors had to engage in extensive consultation and discussion with Contributors to ensure they understood the importance of data provenance in their submissions. Contributions, where the origins of the submitted were unclear, had to be deleted, erasing bits of cultural knowledge from our dataset. Ensuring a clean bill of data provenance was time-intensive and not easily scalable. It was difficult to enforce image upload guidelines in a volunteer effort, resulting in a smaller, less representative, dataset than we would have liked. However, the rigorous process of ensuring a clean bill of data provenance for each submission enabled us to, in good faith, release our dataset as an open-source project. 

The standard of open-sourcing datasets and applying Creative Commons license, while understandable, places massive burdens on small, community-led projects such as \textsc{WWD} to ensure clean bills of data provenance. To be clear, we are not arguing against open-source datasets or Creative Commons licenses, but rather are demonstrating the need to build infrastructures that support and fund the labour needed to verify that data for their projects can be used. As mentioned in~\cref{background}, understanding cultural nuance on a fine-grained, regional scale requires extensive (and non-extractive) consultation with community members who have the capacity to share local expertise. As such, non-exploitative and non-extractive community consultation is an important step in verifying the validity and veracity of cultural information. 

\subsubsection{Verifying cultural information}

Collecting accurate and representative cultural data is exceptionally difficult. Cultures are not bounded by government borders and/or other manufactured systems, but rather extend across larger regions and are often the product of intercultural exchanges~\cite{gupta2008beyondculture}. This makes determining the veracity of a data point in \textsc{WWD} almost impossible without extensive consultation with a community member with local expertise. In \textsc{WWD}, we sought to include as many Contributors as possible to collect a granular representation of cultural data. We accessed Contributors through our community ambassadors who had established relationships and trust with the folks they asked to contribute. Inclusion, however, can be a slippery slope~\cite{epstein2008rise,benjamin2016informed}. 

ML researchers continue to pursue the construction of ever more representative datasets in the name of improving model performance for \textit{everyone}~\cite{luccioni2021everyone, radford2018improvingeveryone}. Often, ML researchers have trouble accessing ``hard-to-reach'' populations, such as the communities we worked with to build \textsc{WWD}. Many recent projects have attempted to solicit engagement from ``hard-to-reach'' populations~\cite{kirk2024prism,ramaswamy2023geodegeographicallydiverseevaluation,singh2024aya_dataset}, yet none of these projects interrogate why these populations might be hard for researchers to access. Drawing on Benjamin's work~\cite{benjamin2016informed}, \textbf{we urge ML researchers to consider how research institutions and industry laboratories may engender distrust within communities that have endured centuries of extractive practices by actors from the Global North.} It is essential that researchers not only endeavour to make participation accessible to members of ``hard-to-reach'' communities but also work towards establishing themselves as trustworthy partners in the research process, in the same way that~\citet{singh2024aya_dataset} do this. 

\subsubsection{Explaining the benefits of participation}

Community Ambassadors wrestled with explaining the benefits of participation in \textsc{WWD} to potential Contributors. Participation was not financially compensated. The research team chose not to make use of professional data centre workers\footnote{Professional data centre workers are those people employed in a centralised manner to perform data collection tasks. Their livelihood is, therefore, connected to the requirement to engage in data contribution, which does not align with \textsc{WWD} goals. Additionally, even had we wished to use data centre workers, we lacked the resources to which a large technology company might have access, such as the ability to engage a business outsourcing company (e.g., Enlabler) to recruit and pay data workers.} because the nature of the data collection process argued for prioritising organic engagement through social networks to collect perspectives from people who do not, and have not, typically contributed to Internet datasets from around the world. We purposely chose a data collection method that would enable the use of social networks and allow us to reach participants other than those employed in a data worker centre, such as older generations and those across a wide socioeconomic range. We also wanted to empower participants to involve their families in the process. 

Although the research team would have preferred to individually compensate each Contributor, because \textsc{WWD} relies on a decentralised, global-scale data collection method, and, crucially, as of the time of data collection, normative standards and infrastructure do not exist to support such a decentralised payment process to effectively \textbf{pay data contributors}, we were \textit{unable} to pay them. The research team explored many possible avenues for paying participants but each time came up against prohibitively expensive and logistically insurmountable barriers. For example, money transfer services such as PayPal~\cite{paypal_countries_2023} and Wise~\cite{wise_usage_2023} were unavailable in many of the regions where \textsc{WWD} operates. These types of services also require that payment recipients have access to digital banking services, which many within our target communities do not. In addition, some of our Core Organisers, who are from the African continent and utilise digital banking services, provided anecdotal evidence of times when their transactions were flagged for seemingly no other reason than their nationality. Infrastructures to support financial remuneration for research participants in the Majority World are simply not commensurate with the many calls from Western researchers to engage participants in these parts of the world. \textbf{Researchers must therefore build the infrastructures to enable equitable participation with communities}; in particular, researchers should investigate how to address breakdowns in participant compensation infrastructures. Other similarly decentralised efforts have remunerated contributors with material items (e.g., sweaters and small gifts)~\cite{singh2024aya_dataset}. Still other researchers point to the limitations of financial compensation for participants and urge researchers to consider what kinds of remuneration would be useful given the context of their research site~\cite{hodge2020relational}.

Despite the lack of extrinsic, financial incentives, the Contributors did exhibit some intrinsic motivation. Contributors shared many different reasons for having participated, such as wanting to make a difference in GenAI outputs, supporting a friend, or contributing to a mission and team they believed in. The majority of data contributions came from the African continent. The authors have speculated why this might be, and have wondered if there is a common focus uniting these Contributors: a central philosophy of ``familyhood'' and unity. This is known by different terms across the continent, including djema’a (Arabic), ubuntu (Zulu), ujamaa (KiSwahili), umuntu (Chichewa), and unhu (Shona). Community Ambassadors also suspected that their positionality as members of the communities from which they were soliciting data contributions further strengthened sentiments of unity among participants who saw \textsc{WWD} as an extension of the growing ``By Africans, for Africans'' movement in ML~\cite{birhane_2024_for_africans}. 

Whilst we can only speculate about why participants engaged in the data contribution process, the authors recognise the responsibility they were given to respect and honour these Contributors and to avoid extractive and exploitative practices. 

\subsection{Participate or be excluded: A catch-22}


Cultural erasure and lack of representation are rooted in deep systemic issues that date back centuries. GenAI, especially T2I models, play an increasingly prominent role in shaping the media ecosystem. However, relying on these models to ``fix'' centuries of intentional cultural erasure overlooks the deeper systemic issues that will likely constrain the efficacy of these technocentric solutions. During the data collecting for \textsc{WWD}, Community Ambassadors often found themselves rationalising the uncompensated nature of data contributions by demonstrating that existing T2I models perform poorly when creating images of local dishes so Contributors should provide accurate data to teach the model what the dish should look like. Regardless, many Contributors, Community Ambassadors recalled, were eager to participate in an African-researcher-led ML effort. 

Through the reflection process, Community Ambassadors shared conflicting feelings about tapping into the shared philosophy of familyhood and unity that they suspected motivated Contributors' participation. On the one hand, local communities were engaging in the dataset creation process through the lens of \textit{Ubuntu} (broadly translated as ``I am because we are'')---an ethical framework that emphasises dignity, reciprocity, and the common good~\cite{ewuoso2019core}. In contrast, the models that would subsequently be trained by these datasets are developed in Western contexts and imbued with utilitarian ethics---a framework that emphasises the best for the greatest number of people~\cite{selbstFairnessAbstractionSociotechnical2019,west2004introduction}. These two distinct, yet interrelated elements of the ML pipeline---dataset production and model development---are therefore produced not only in distinct geographic regions~\cite{sambasivan2021everyone,scheuermanDatasetsHavePolitics2021} but also, in our case, in two distinct ethical frameworks. Contributors who engaged with us out of a sense of \textit{Ubuntu} are unlikely to see their values recognised and preserved in the actual functioning of the downstream T2I model that is optimising for fundamentally different well-being criteria.  

Participation in dataset construction is not a guaranteed way to achieve representational justice in T2I models. Thus, proposing to communities that to avoid being excluded from the future of media representation, they should participate in dataset development, is misleading. This false choice obfuscates (1) the deeper systematic issues that dictate whose culture gets preserved and represented and (2) the disjunction between the value system under which participants may contribute data and that of the models that are then trained on this data. 

Future research efforts should examine how to bring the ethical frameworks of dataset creation and model development into alignment by prioritising local, community ownership over AI. 


% \section{limitations}
\section{Conclusion}

Subgroup analysis is an important, yet under-utilized tool in data science.
Our results suggest that combining algorithm-generated, rule-based insights with human intuition and experimentation in an interactive workflow can help practitioners develop a thorough understanding of complex datasets.
By implementing these interactions in a lightweight notebook-based tool, we hope to lower the barrier for data scientists to try subgroup discovery and to curate unexpected, interesting subpopulations in their data.
Divisi is available as an open-source package so that data scientists and HCI researchers can build on this work, helping to make exploratory subgroup analysis more feasible for a wider range of contexts.

\bibliographystyle{ACM-Reference-Format}
\bibliography{references}

\clearpage
\appendix
\onecolumn
\section{Dataset Examples}
\label{app:dataset-eg}
Figure \ref{fig:dataset-eg} illustrates example data instances from MemeCap, NewYorker, and YesBut.

\begin{figure*}[t]
  \includegraphics[width=\linewidth]{figures/dataset-eg.pdf} \hfill
  \caption {Dataset Examples on MemeCap, NewYorker, and YesBut.}
  \label{fig:dataset-eg}
\end{figure*}


\section{SentenceSHAP}
\label{app:sentence-shap}
In this section, we introduce SentenceSHAP, an adaptation of TokenSHAP \cite{horovicz-goldshmidt-2024-tokenshap}. While TokenSHAP calculates the importance of individual tokens, SentenceSHAP estimates the importance of individual sentences in the input prompt. The importance score is calculated using Monte Carlo Shapley Estimation, following the same principles as TokenSHAP.

Given an input prompt \( X = \{x_1, x_2, \dots, x_n\} \), where \( x_i \) represents a sentence, we generate all possible combinations of \( X \) by excluding each sentence \( x_i \) (i.e., \( X - \{x_i\} \)). Let \( Z \) represent the set of all combinations where each \( x_i \) is removed. To estimate Shapley values efficiently, we randomly sample from \( Z \) with a specified sampling ratio, resulting in a subset \( Z_s = \{X_1, X_2, \dots, X_s\} \), where each \( X_i = X - \{x_i\} \).

Next, we generate a base response \( r_0 \) using a VLM (or LLM) with the original prompt \( X \), and a set of responses \( R_s = \{r_1, r_2, \dots, r_s\} \), each generated by a prompt from one of the sampled combinations in \( Z_s \).

We then compute the cosine similarity between the base response \( r_0 \) and each response in \( R_s \) using Sentence Transformer (\texttt{BAAI/bge-large-en-v1.5}). The average similarity between combinations with and without \( x_i \) is computed, and the difference between these averages gives the Shapley value for sentence \( x_i \). This is expressed as:
\begin{align}
\notag
\phi(x_i) = \\ \notag
&\frac{1}{s} \sum_{j=1}^{s} \left( \text{cos}(r_0, r_j \mid x_i) - \text{cos}(r_0, r_j \mid \neg x_i) \right)
\end{align}
where \( \phi(x_i) \) represents the Shapley value for sentence \( x_i \), $\text{cos}(r_0, r_j \mid x_i)$ is the cosine similarity between the base response and the response that includes sentence $x_i$, $\text{cos}(r_0, r_j \mid \neg x_i)$ is the cosine similarity between the base response and the response that excludes sentence $x_i$, and $s$ is the number of sampled combinations in $Z_s$.

\section{Error Analysis Based on SentenceSHAP}
Figure \ref{fig:error-analysis} presents two examples of negative impacts from implications: dilution of focus and the introduction of irrelevant information.
\label{app:error-analysis-shap}
\begin{figure*}[t]
  \includegraphics[width=\linewidth]{figures/error-analysis.pdf} \hfill
  \caption {Examples of negative impact from implications from Phi (top) and GPT4o (bottom).}
  \label{fig:error-analysis}
\end{figure*}

\section{Details on human anntations}
\label{app:cloudresearch}
We present the annotation interface on CloudResearch used for human evaluation to validate our evaluation metric in Figure \ref{fig:cloud-research}. Refer to Sec.~\ref{sec:ethics} for details on annotator selection criteria and compensation.

\begin{figure*}[t]
  \includegraphics[width=\linewidth]{figures/cloud-research.pdf} \hfill
  \caption {Annotation interface on CloudResearch used for human evaluation to validate our evaluation metric.}
  \label{fig:cloud-research}
\end{figure*}



\section{Generation Prompts for Selection and Refinement}
\label{app:gen-prompts}
Figures \ref{fig:desc-prompt}, \ref{fig:seed-imp-prompt}, and \ref{fig:nonseed-imp-prompt} show the prompts used for generating image descriptions, seed implications (1st hop), and non-seed implications (2nd hop onward). Figure \ref{fig:cand-prompt} displays the prompt used to generate candidate and final explanations. Image descriptions are used for candidate explanations when existing data is insufficient but are not used for final explanations. For calculating Cross Entropy values (used as a relevance term), we use the prompt in Figure \ref{fig:cand-prompt}, substituting the image with image descriptions, as LLM is used to calculate the cross entropies.

\begin{figure*}[h]
\small
\begin{tcolorbox}[
    title=Prompt for Image Descriptions,
    colback=white,
    colframe=CadetBlue,
    arc=0pt,        % Remove rounded corners
    outer arc=0pt   % Remove outer rounded corners (important for some styles)    
]

Describe the image by focusing on the noun phrases that highlight the actions, expressions, and interactions of the main visible objects, facial expressions, and people.\\
\\
Here are some guidelines when generating image descriptions:\\
* Provide specific and detailed references to the objects, their actions, and expressions. Avoid using pronouns in the description.\\
* Do not include trivial details such as artist signatures, autographs, copyright marks, or any unrelated background information.\\
* Focus only on elements that directly contribute to the meaning, context, or main action of the scene.\\
* If you are unsure about any object, action, or expression, do not make guesses or generate made-up elements.\\
* Write each sentence on a new line.\\
* Limit the description to a maximum of 5 sentences, with each focusing on a distinct and relevant aspect that directly contribute to the meaning, context, or main action of the scene.\\
\\
Here are some examples of desired output:
---\\
\text{[Description]} (example of newyorker cartoon image):\\
Through a window, two women with surprised expressions gaze at a snowman with human arms.\\
---\\
\text{[Description]} (example of newyorker cartoon image):\\
A man and a woman are in a room with a regular looking bookshelf and regular sized books on the wall.\\
In the middle of the room the man is pointing to text written on a giant open book which covers the entire floor.\\
He is talking while the woman with worried expression watches from the doorway.\\
---\\
\text{[Description]} (example of meme):\\
The left side shows a woman angrily pointing with a distressed expression, yelling ``You said memes would work!''.\\
The right side shows a white cat sitting at a table with a plate of food in front of it, looking indifferent or smug with the text above the cat reads, ``I said good memes would work''.\\
---\\
\text{[Description]} (example of yesbut image):\\
The left side shows a hand holding a blue plane ticket marked with a price of ``\$50'', featuring an airplane icon and a barcode, indicating it's a flight ticket.\\
The right side shows a hand holding a smartphone displaying a taxi app, showing a route map labeled ``Airport'' and a price of ``\$65''.\\
---\\

Proceed to generate the description.\\
\text{[Description]}:

\end{tcolorbox}
\caption{A prompt used to generate image descriptions.} % Add a caption to the figure
\label{fig:desc-prompt}
\end{figure*}


%%%%%%%%%%%%%%%%%%%%%%%%%%% Prompt for implications %%%%%%%%%%%%%%%%%%%%%%%%%%%
\begin{figure*}[t]
\small
\begin{tcolorbox}[
    title=Prompt for Seed Implications,
    colback=white,
    colframe=Green,
    arc=0pt,        % Remove rounded corners
    outer arc=0pt,  % Remove outer rounded corners (important for some styles)    
    % breakable,
]

You are provided with the following inputs:\\
- \text{[}Image\text{]}: An image (e.g. meme, new yorker cartoon, yes-but image)\\
- \text{[}Caption\text{]}: A caption written by a human.\\
- \text{[}Descriptions\text{]}: Literal descriptions that detail the image.\\
\\
\#\#\# Your Task:\\
\texttt{[ One-sentence description of the ultimate goal of your task. Customize based on the task. ]}\\
Infer implicit meanings, cultural references, commonsense knowledge, social norms, or contrasts that connect the caption to the described objects, concepts, situations, or facial expressions.\\
\\
\#\#\# Guidelines:\\
- If you are unsure about any details in the caption, description, or implication, refer to the original image for clarification.\\
- Identify connections between the objects, actions, or concepts described in the inputs.\\
- Explore possible interpretations, contrasts, or relationships that arise naturally from the scene, while staying grounded in the provided details.\\
- Avoid repeating or rephrasing existing implications. Ensure each new implication introduces fresh insights or perspectives.\\
- Each implication should be concise (one sentence) and avoid being overly generic or vague.\\
- Be specific in making connections, ensuring they align with the details provided in the caption and descriptions.\\
- Generate up to 3 meaningful implications.\\
\\
\#\#\# Example Outputs:\\
\#\#\#\# Example 1 (example of newyorker cartoon image):\\
\text{[}Caption\text{]}: ``This is the most advanced case of Surrealism I've seen.''\\
\text{[}Descriptions\text{]}: A body in three parts is on an exam table in a doctor's office with the body's arms crossed as though annoyed.\\
\text{[}Connections\text{]}:\\
1. The dismembered body is illogical and impossible, much like Surrealist art, which often explores the absurd.\\
2. The body’s angry posture adds a human emotion to an otherwise bizarre scenario, highlighting the strange contrast.\\
\\
\#\#\#\# Example 2 (example of newyorker cartoon image):\\
\text{[}Caption\text{]}: ``He has a summer job as a scarecrow.''\\
\text{[}Descriptions\text{]}: A snowman with human arms stands in a field.\\
\text{[}Connections\text{]}:\\
1. The snowman, an emblem of winter, represents something out of place in a summer setting, much like a scarecrow's seasonal function.\\
2. The human arms on the snowman suggest that the role of a scarecrow is being played by something unexpected and seasonal.\\
\\
\#\#\#\# Example 3 (example of yesbut image):\\
\text{[}Caption\text{]}: ``The left side shows a hand holding a blue plane ticket marked with a price of `\$50'.''\\
\text{[}Descriptions\text{]}: The screen on the right side shows a route map labeled ``Airport'' and a price of `\$65'.\\
\text{[}Connections\text{]}:\\
1. The discrepancy between the ticket price and the taxi fare highlights the often-overlooked costs of travel beyond just booking a flight.\\
2. The image shows the hidden costs of air travel, with the extra fare representing the added complexity of budgeting for transportation.\\
\\
\#\#\#\# Example 4 (example of meme):\\
\text{[}Caption\text{]}: ``You said memes would work!''\\
\text{[}Descriptions\text{]}: A cat smirks with the text ``I said good memes would work.''\\
\text{[}Connections\text{]}:\\
1. The woman's frustration reflects a common tendency to blame concepts (memes) instead of the quality of execution, as implied by the cat’s response.\\
2. The contrast between the angry human and the smug cat highlights how people often misinterpret success as simple, rather than a matter of quality.\\
\\
\#\#\# Now, proceed to generate output:\\
\text{[}Caption\text{]}: \texttt{[ Caption ]}\\
\\
\text{[}Descriptions\text{]}:\\
\texttt{[ Descriptions ]}\\
\\
\text{[}Connections\text{]}:

\end{tcolorbox}
\caption{A prompt used to generate seed implications.} % Add a caption to the figure
\label{fig:seed-imp-prompt}
\end{figure*}


%%%%%%%%%%%%%%%%%%%%%%%%%%% Prompt for nonseed implications %%%%%%%%%%%%%%%%%%%%%%%%%%%
\begin{figure*}[t]
\small
%  \begin{tcolorbox}[
%  width=\textwidth,
%  colback={white},
%  title={Title},
%  colbacktitle={DarkGreen},
%  coltitle=white,
%  colframe={DarkGreen},
%  breakable
% ]
 % \parskip=5pt

\begin{tcolorbox}[
    % breakable,
    title=Prompt for Non-Seed Implications (2nd hop onward),
    colback=white,
    colframe=Green,
    arc=0pt,        % Remove rounded corners
    outer arc=0pt,  % Remove outer rounded corners (important for some styles)    
    % breakable,
]

You are provided with the following inputs:\\
- \text{[}Image\text{]}: An image (e.g. meme, new yorker cartoon, yes-but image)\\
- \text{[}Caption\text{]}: A caption written by a human.\\
- \text{[}Descriptions\text{]}: Literal descriptions that detail the image.\\
- \text{[}Implication\text{]}: A previously generated implication that suggests a possible connection between the objects or concepts in the caption and description.\\
\\
\#\#\# Your Task:\\
\texttt{[ One-sentence description of the ultimate goal of your task. Customize based on the task. ]}\\
Infer implicit meanings across the objects, concepts, situations, or facial expressions found in the caption, description, and implication. Focus on identifying relevant commonsense knowledge, social norms, or underlying connections.\\
\\
\#\#\# Guidelines:\\
- If you are unsure about any details in the caption, description, or implication, refer to the original image for clarification.\\
- Identify potential connections between the objects, actions, or concepts described in the inputs.\\
- Explore interpretations, contrasts, or relationships that naturally arise from the scene while remaining grounded in the inputs.\\
- Avoid repeating or rephrasing existing implications. Ensure each new implication provides fresh insights or perspectives.\\
- Each implication should be concise (one sentence) and avoid overly generic or vague statements.\\
- Be specific in the connections you make, ensuring they align closely with the details provided.\\
- Generate up to 3 meaningful implications that expand on the implicit meaning of the scene.\\
\\
\#\#\# Example Outputs:\\
\#\#\#\# Example 1 (example of newyorker cartoon image):\\
\text{[}Caption\text{]}: "This is the most advanced case of Surrealism I've seen."\\
\text{[}Descriptions\text{]}: A body in three parts is on an exam table in a doctor's office with the body's arms crossed as though annoyed.\\
\text{[}Implication\text{]}: Surrealism is an art style that emphasizes strange, impossible, or unsettling scenes.\\
\text{[}Connections\text{]}:\\
1. A body in three parts creates an unsettling juxtaposition with the clinical setting, which aligns with Surrealist themes.\\
2. The body’s crossed arms add humor by assigning human emotion to an impossible scenario, reflecting Surrealist absurdity.\\
... \\
\texttt{[ We used sample examples from the prompt for generating seed implications (see Figure \ref{fig:seed-imp-prompt}), following the above format, which includes [Implication]:. ]}
\\
---\\
\\
\#\#\# Proceed to Generate Output:\\
\text{[}Caption\text{]}: \texttt{[ Caption ]}\\
\\
\text{[}Descriptions\text{]}:\\
\texttt{[ Descriptions ]}\\
\\
\text{[}Implication\text{]}:\\
\texttt{[ Implication ]}\\
\\
\text{[}Connections\text{]}:
\end{tcolorbox}
\caption{A prompt used to generate non-seed implications.} % Add a caption to the figure
\label{fig:nonseed-imp-prompt}
\end{figure*}


%%%%%%%%%%%%%%%%%%%%%%%%%%% Prompt for nonseed implications %%%%%%%%%%%%%%%%%%%%%%%%%%%
\begin{figure*}[t]
\small
%  \begin{tcolorbox}[
%  width=\textwidth,
%  colback={white},
%  title={Title},
%  colbacktitle={DarkGreen},
%  coltitle=white,
%  colframe={DarkGreen},
%  breakable
% ]
 % \parskip=5pt

\begin{tcolorbox}[
    % breakable,
    title=Prompt for Candidate and Final Explanations,
    colback=white,
    colframe=RedViolet,
    arc=0pt,        % Remove rounded corners
    outer arc=0pt,  % Remove outer rounded corners (important for some styles)    
    % breakable,
]

You are provided with the following inputs:\\
- **\text{[}Image\text{]}:** A New Yorker cartoon image.\\
- **\text{[}Caption\text{]}:** A caption written by a human to accompany the image.\\
- **\text{[}Image Descriptions\text{]}:** Literal descriptions of the visual elements in the image.\\
- **\text{[}Implications\text{]}:** Possible connections or relationships between objects, concepts, or the caption and the image.\\
- **\text{[}Candidate Answers\text{]}:** Example answers generated in a previous step to provide guidance and context.\\
\\
\#\#\# Your Task:\\
Generate **one concise, specific explanation** that clearly captures why the caption is funny in the context of the image. Your explanation must provide detailed justification and address how the humor arises from the interplay of the caption, image, and associated norms or expectations.\\
\\
\#\#\# Guidelines for Generating Your Explanation:\\
1. **Clarity and Specificity:**  \\
   - Avoid generic or ambiguous phrases.  \\
   - Provide specific details that connect the roles, contexts, or expectations associated with the elements in the image and its caption.  \\
\\
2. **Explain the Humor:**  \\
- Clearly connect the humor to the caption, image, and any cultural, social, or situational norms being subverted or referenced.  \\
- Highlight why the combination of these elements creates an unexpected or amusing contrast.\\
\\
3. **Prioritize Clarity Over Brevity:**  \\
- Justify the humor by explaining all important components clearly and in detail.  \\
- Aim to keep your response concise and under 150 words while ensuring no critical details are omitted.  \\
\\
4. **Use Additional Inputs Effectively:**\\
- **\text{[}Image Descriptions\text{]}:** Provide a foundation for understanding the visual elements."   \\
- **\text{[}Implications\text{]}:** Assist in understanding relationships and connections but do not allow them to dominate or significantly alter the central idea.\\
- **\text{[}Candidate Answers\text{]}:** Adapt your reasoning by leveraging strengths or improving upon weaknesses in the candidate answers.\\
\\
Now, proceed to generate your response based on the provided inputs.\\
\\
\#\#\# Inputs:\\
\text{[}Caption\text{]}: \texttt{\text{[} Caption \text{]}}\\
\\
\text{[}Descriptions\text{]}:\\
\texttt{\text{[} Top-K Implications \text{]}}\\
\\
\text{[}Implications\text{]}:\\
\texttt{\text{[} Top-K Implications \text{]}}\\
\\
\text{[}Candidate Anwers\text{]}:\\
\texttt{\text{[} Top-K Candidate Explanations \text{]}}\\
\\
\text{[}Output\text{]}:\\

\end{tcolorbox}
\caption{A prompt used to generate candidate and final explanations.} % Add a caption to the figure
\label{fig:cand-prompt}
\end{figure*}


\section{Evaluation Prompts}
\label{app:eval-prompts}
Figures \ref{fig:recall-prompt} and \ref{fig:precision-prompt} present the prompts used to calculate recall and precision scores in our LLM-based evaluation, respectively.

%%%%%%%%%%%%%%%%%%%%%%%%%%% Prompt for nonseed implications %%%%%%%%%%%%%%%%%%%%%%%%%%%
\begin{figure*}[t]
\small
\begin{tcolorbox}[
    % breakable,
    title=Prompt for Evaluating Recall Score,
    colback=white,
    colframe=MidnightBlue,
    arc=0pt,        % Remove rounded corners
    outer arc=0pt,  % Remove outer rounded corners (important for some styles)    
    % breakable,
]

Your task is to assess whether \text{[}Sentence1\text{]} is conveyed in \text{[}Sentence2\text{]}. \text{[}Sentence2\text{]} may consist of multiple sentences.\\
\\
Here are the evaluation guidelines:\\
1. Mark 'Yes' if \text{[}Sentence1\text{]} is conveyed in \text{[}Sentence2\text{]}.\\
2. Mark 'No' if \text{[}Sentence2\text{]} does not convey the information in \text{[}Sentence1\text{]}.\\
\\
Proceed to evaluate. \\
\\
\text{[}Sentence1\text{]}: \texttt{[ One Atomic Sentence from Decomposed Reference Explanation ]} \\
\\
\text{[}Sentence2\text{]}: \texttt{[ Predicted Explanation ]}\\
\\
\text{[}Output\text{]}:

\end{tcolorbox}
\caption{Prompt for evaluating recall score.} % Add a caption to the figure
\label{fig:recall-prompt}
\end{figure*}


\begin{figure*}[t]
\small
\begin{tcolorbox}[
    % breakable,
    title=Prompt for Evaluating Precision Score,
    colback=white,
    colframe=MidnightBlue,
    arc=0pt,        % Remove rounded corners
    outer arc=0pt,  % Remove outer rounded corners (important for some styles)    
    % breakable,
]

Your task is to assess whether \text{[}Sentence1\text{]} is inferable from \text{[}Sentence2\text{]}. \text{[}Sentence2\text{]} may consist of multiple sentences.\\
\\
Here are the evaluation guidelines:\\
1. Mark "Yes" if \text{[}Sentence1\text{]} can be inferred from \text{[}Sentence2\text{]} — whether explicitly stated, implicitly conveyed, reworded, or serving as supporting information.\\
2. Mark 'No' if \text{[}Sentence1\text{]} is absent from \text{[}Sentence2\text{]}, cannot be inferred, or contradicts it.\\
\\
Proceed to evaluate. \\
\\
\text{[}Sentence1\text{]}: \texttt{[ One Atomic Sentence from Decomposed Predicted Explanation ]}\\
\\
\text{[}Sentence2\text{]}: \texttt{[ Reference Explanation ]}\\
\\
\text{[}Output\text{]}:


\end{tcolorbox}
\caption{Prompt for evaluating precision score.} % Add a caption to the figure
\label{fig:precision-prompt}
\end{figure*}

\section{Prompts for Baselines}
\label{app:base-prompts}

Figure \ref{fig:base-prompt} presents the prompt used for the ZS, CoT, and SR Generator methods. While the format remains largely the same, we adjust it based on the baseline being tested (e.g., CoT requires generating intermediate reasoning, so we add extra instructions for that).
Figure \ref{fig:critic-prompt} shows the prompt used in the SR critic model. The critic's criteria include: (1) \textit{correctness}, measuring whether the explanation directly addresses why the caption is humorous in relation to the image and its caption; (2) \textit{soundness}, evaluating whether the explanation provides a well-reasoned interpretation of the humor; (3) \textit{completeness}, ensuring all important aspects in the caption and image contributing to the humor are considered; (4) \textit{faithfulness}, verifying that the explanation is factually consistency with the image and caption; and (5) \textit{clarity}, ensuring the explanation is clear, concise, and free from unnecessary ambiguity.
\begin{figure*}
\small
\begin{tcolorbox}[
    % breakable,
    title=Prompt for Baselines,
    colback=white,
    colframe=Black,
    arc=0pt,        % Remove rounded corners
    outer arc=0pt,  % Remove outer rounded corners (important for some styles)    
    % breakable,
]

You are provided with the following inputs:\\
- **\text{[}Image\text{]}:** A New Yorker cartoon image.\\
- **\text{[}Caption\text{]}:** A caption written by a human to accompany the image.\\
\texttt{[ if Self-Refine with Critic is True: ]} \\
- **\text{[}Feedback for Candidate Answer\text{]}:** Feedback that points out some weakness in the current candidate responses.\\
\texttt{[ if Self-Refine is True: ]} \\
- **\text{[}Candidate Answers\text{]}:** Example answers generated in a previous step to provide guidance and context.\\
\\
\#\#\# Your Task:\\
Generate **one concise, specific explanation** that clearly captures why the caption is funny in the context of the image. Your explanation must provide detailed justification and address how the humor arises from the interplay of the caption, image, and associated norms or expectations.\\
\\
\#\#\# Guidelines for Generating Your Explanation:\\
1. **Clarity and Specificity:**  \\
   - Avoid generic or ambiguous phrases.  \\
   - Provide specific details that connect the roles, contexts, or expectations associated with the elements in the image and its caption.  \\
\\
2. **Explain the Humor:**  \\
- Clearly connect the humor to the caption, image, and any cultural, social, or situational norms being subverted or referenced.  \\
- Highlight why the combination of these elements creates an unexpected or amusing contrast.\\
\\
3. **Prioritize Clarity Over Brevity:**  \\
- Justify the humor by explaining all important components clearly and in detail.  \\
- Aim to keep your response concise and under 150 words while ensuring no critical details are omitted.  \\
\\
\texttt{[ if Self-Refine is True: ]}\\
4. **Use Additional Inputs Effectively:**\\
- **[Candidate Answers]:** Adapt your reasoning by leveraging strengths or improving upon weaknesses in candidate answers. \\
\texttt{[ if Self-Refine with Critic is True: ]}\\
- **[Feedback for Candidate Answer]:** Feedback that points out some weaknesses in the current candidate responses.\\
\\
\texttt{ [ if CoT is True: ]} \\
Begin by analyzing the image and the given context, and explain your reasoning briefly before generating your final response. \\
\\
Here is an example format of the output: \\
\{\{ \\
    "Reasoning": "...", \\
    "Explanation": "..."   \\
\}\} \\

Now, proceed to generate your response based on the provided inputs.\\
\\
\#\#\# Inputs:\\
\text{[}Caption\text{]}: \texttt{\text{[} Caption \text{]}}\\
\\
\text{[}Candidate Answers\text{]}: \texttt{\text{[} Candidate Explanations \text{]}}\\
\\
\text{[}[Feedback for Candidate Answer]:\text{]}: \texttt{\text{[} Feedback for Candidate Explanations \text{]}}\\
\\
\text{[}Output\text{]}:\\

\end{tcolorbox}
\caption{A prompt used for baseline methods, with conditions added based on the specific baseline being experimented with.} % Add a caption to the figure
\label{fig:base-prompt}
\end{figure*}


\begin{figure*}
\small
\begin{tcolorbox}[
    % breakable,
    title=Prompt for Self-Refine Critic,
    colback=white,
    colframe=Black,
    arc=0pt,        % Remove rounded corners
    outer arc=0pt,  % Remove outer rounded corners (important for some styles)    
    % breakable,
]
\texttt{[ Customize goal text here: ]} \\
\texttt{MemeCap:} You will be given a meme along with its caption, and a candidate response that describes what meme poster is trying to convey. \\
\texttt{NewYorker:} You will be given an image along with its caption, and a candidate response that explains why the caption is funny for the given image. \\
\texttt{YesBut:} You will be given an image and a candidate response that describes why the image is funny or satirical. \\
\\
Your task is to criticize the candidate response based on the following evaluation criteria: \\
- Correctness: Does the explanation directly address why the caption is funny, considering both the image and its caption? \\
- Soundness: Does the explanation provide a meaningful and well-reasoned interpretation of the humor? \\
- Completeness: Does the explanation address all relevant aspects of the caption and image (e.g., visual details, text) that contribute to the humor? \\
- Faithfulness: Is the explanation factually consistent with the details in the image and caption? \\
- Clarity: Is the explanation clear, concise, and free from unnecessary ambiguity? \\
 \\
Proceed to criticize the candidate response ideally using less than 5 sentences:\\
\\
\text{[}Caption\text{]}: \texttt{[ caption ]}\\
\\
\text{[}Candidate Response\text{]}: \\
 \texttt{\text{[} Candidate Response \text{]}}\\
\\
\text{[}Output\text{]}: \\
\end{tcolorbox}
\caption{A prompt used in SR critic model.} % Add a caption to the figure
\label{fig:critic-prompt}
\end{figure*}

% \begin{figure*}[t]
%   \includegraphics[width=\linewidth]{figures/error-analysis.pdf} \hfill
%   \vspace{-20pt}
%   \caption {Examples of negative impact from implications from Phi (top) and GPT4o (bottom).}
%   \label{fig:error-analysis}
% \end{figure*}

\end{document}
\endinput
%%
%% End of file `sample-manuscript.tex'.
