\section{Designing Polymind}
\revision{
\textit{Polymind}'s interface provides a range of diagramming features commonly used in prewriting strategies, and a ``task board'' overlaid on the canvas for microtask management, as shown in ~\autoref{fig:teaser} and ~\autoref{fig:UI}. We map the input and output of each microtask to diagram types on the canvas. To facilitate collaboration and task management, we introduce the notion of ``task header'', and ``task cards''. The former displays notifications and previews of microtask results on a specific diagram, while the latter displays specifications of a microtask on the ``task board''.
To apply mixed initiative, we define two initiative modes: proactive and reactive. We also design a set of workflows to provide awareness information.
In this section, we first provide an overview of the \textit{Polymind} interface, and then elaborate on each of our key features.
}

% \begin{figure}[ht]
% \centering
% \begin{subfigure}{.45\linewidth}
%     \centering
%     \includegraphics[width=.85\linewidth]{figures/keyword.png}
%     \label{fig:keyword}
% \end{subfigure}
% \begin{subfigure}{.45\linewidth}
%     \centering
%     \includegraphics[width=.85\linewidth]{figures/concept.png}
%     \label{fig:concept}
% \end{subfigure}
% \begin{subfigure}{.45\linewidth}
%     \centering
%     \includegraphics[width=.9\linewidth]{figures/sticky_note.png}
%     \label{fig:sticky_note}
% \end{subfigure}
% \begin{subfigure}{.45\linewidth}
%     \centering
%     \includegraphics[width=\linewidth]{figures/section.png}
%     \label{fig:section}
% \end{subfigure}
% \caption{Three basic diagrams (or nodes) in \textit{Polymind}, and the section, which stores multiple diagrams and their connections.}
% \end{figure} \label{fig:diagrams}

\begin{figure}[ht]
\centering
\includegraphics[width=\linewidth]{figures/UIComponents.png}
\caption{\textit{Polymind} supports three basic diagrams (or nodes), and allows users to draw a section over diagrams. The task board interface supports microtask management, and maps microtask input and output to different diagram types on the canvas. The results of microtasks are displayed as notifications and previews on the task header before being expanded and accepted.}
\label{fig:diagrams}
\end{figure}

\subsection{Main Interface}
The interface of \textit{Polymind} comprises a diagramming canvas, a task board, and a toolbar (\autoref{fig:UI}).
\revision{The diagramming canvas includes all diagrams and their topological connections. We defined three basic diagrams: \textbf{\textcolor{keyword}{\textit{keyword}}}, \textbf{\textcolor{concept}{\textit{concept}}}, \textbf{\textcolor{sticky_note}{\textit{sticky note}}}, each with a unique shape affording varying text lengths (keywords the shortest, and sticky notes for the longest pieces of text). Users can connect diagrams through directed or undirected edges, or draw a \textbf{\textcolor{section}{\textit{section}}} over diagrams (see ~\autoref{fig:diagrams}).

The task board displays all microtasks and their specifications in distinct ``task cards'', each including input and output types, prompts, etc. The results of each microtask are displayed as notifications and previews on the ``task header'' over that processed diagram before they are expanded and accepted (see ~\autoref{fig:task_status}).}

\subsubsection{The Diagramming Canvas}
The diagramming canvas supports most common diagrams. Over each of these diagrams, a task header is attached to display the status of microtasks. All resulting diagrams of microtasks are displayed using a hollow shape in contrast to user created diagrams.

\paragraph{\underline{Diagrams}}
To scaffold the diagramming process for \textit{Goal 1}, we define three primitive diagrams that are commonly used in diagramming or prewriting tools (e.g., Inkplanner \cite{lu2018inkplanner}, Figma \footnote{https://www.figma.com/}, etc.): \textbf{\textcolor{keyword}{\textit{keyword}}} (displayed as text labels), \textbf{\textcolor{concept}{\textit{concept}}} (represented by an ellipse), and \textbf{\textcolor{sticky_note}{\textit{sticky note}}} (see \autoref{fig:diagrams}). We use sizes, shapes and the placeholder text upon creation (``Add Keyword'', ``Add Concept'', ``Add text'') to guide users to type text of different lengths and of different functions into different diagrams (e.g, brief words in \textbf{\textcolor{keyword}{\textit{keyword}}}, short phrases in \textbf{\textcolor{concept}{\textit{concept}}}, long paragraphs in \textbf{\textcolor{sticky_note}{\textit{sticky note}}}), so that we can properly define microtasks handling various types of input to support all three stages of LLM usage (see \ref{the_usage_of_LLMs_for_prewriting}).

These three primitive diagrams can be selected, moved, resized or scaled as per users' needs. Additionally, users can establish two types of connection between these diagrams: directed (arrow) or undirected (line). We also allow users to create a \textbf{\textcolor{section}{\textit{section}}} (\autoref{fig:diagrams}) among these diagrams, and assign titles to these sections. On such an interface, users can perform most diagram-based prewriting strategies such as mind mapping, concept mapping, argument mapping, etc.

It is important to note that the diagramming canvas maintains a graph-like (usually tree-like) structure, and we later refer to those primitive diagrams as ``nodes'' at times for simplicity.

\paragraph{\underline{The Task Header}}
To support \textit{Goal 2} (especially \textit{Goal 2.1}) and \textit{Goal 3}, we design a task header that is attached to each diagram (i.e., \textbf{\textcolor{keyword}{\textit{keyword}}}, \textbf{\textcolor{concept}{\textit{concept}}}, \textbf{\textcolor{sticky_note}{\textit{sticky note}}}, and \textbf{\textcolor{section}{\textit{section}}}) on the diagramming canvas to display the status of each microtask on the diagram element (\autoref{fig:task_status}). On the task header, the name of each microtask is displayed as small text labels. By default, all microtasks are activated (see \ref{task_initiative}) and filled with distinctive colours. Each task header also has a preview panel (\autoref{fig:notifications_and_previews}) that displays key points of unread microtask results. It will only pop up when users hover over the task header, and there are unread microtask results.

\begin{figure}[h]
\begin{minipage}{.5\textwidth}
    \captionsetup{width=\linewidth}
    \vspace{0pt}
    \includegraphics[width=\linewidth]{figures/feedback_v2.png}
    \caption{(a) When the user hovers over the three icons over the resulting diagram, suggestions to improve the generation will pop up. The user can click on these suggestions to request a regeneration. (b) If a user clicks on the question mark icon, the system will return an explanation of the generated result.}
    \label{fig:feedback}
\end{minipage}
\begin{minipage}{.45\textwidth}
    \captionsetup{width=.9\linewidth}
    \vspace{0pt}
    \includegraphics[width=\linewidth]{figures/preview_summary2.png}
    \caption{Once the mouse hovers over key points of a microtask on the preview panel for 1.5 seconds, the system will present a summary of the generated results using a news ticker effect.}
    \label{fig:preview_summary}
\end{minipage}
\end{figure}



\paragraph{\underline{Results of Microtasks}} To support \textit{Goal 2.2} we display all resulting diagram nodes using a hollow shape (filled in white), including \textbf{\textcolor{keyword}{\textit{keyword}}}, \textbf{\textcolor{concept}{\textit{concept}}}, and \textbf{\textcolor{sticky_note}{\textit{sticky note}}}. The border colour indicates the specific microtask each resulting node belongs to. A user can choose to discard or accept a resulting node; upon acceptance it will be displayed normally, the same as user-created diagram nodes.

Apart from the two icons for accepting or discarding the diagram node, to support \textit{Goal 2.3}, we design a ``question mark'' icon (see \autoref{fig:feedback}) that users can click on to request an explanation for this specific generation. For each resulting node, we also provide three heuristic suggestions for users to request a regenerated node, ``Be creative'', ``Be more specific'', and ``Be brief''. These suggestions will pop up if users hover over those three icons (see \autoref{fig:feedback}).

\subsubsection{The Task Board}
The task board of \textit{Polymind} enables users to configure and delegate microtasks, to support our \textit{Goal 2}. The task board borrows the design of Trello \footnote{https://trello.com/} board, which consists of a list of ``task cards''. Users can click on the ``add'' icon to delegate a new microtask, or toggle the ``visibility'' switch on the upper-right corner to hide all task headers on the diagramming canvas.

\paragraph{\underline{Task Card}}
A task card contains specifications of a microtask, including the task name, input type, output type, initiative mode (global), visibility (global), and prompts to communicate with the LLM (\autoref{fig:task_card}). The name of the microtask is always displayed in the upper-left corner, each indicated by a distinctive colour (same as resulting diagrams or labels on task headers).

Users can browse through task specifications by clicking on the ``page turn'' icon at the bottom; delete the microtask by clicking on the ``delete'' icon; toggle visibility of all resulting diagrams on the canvas by clicking on the ``visibility'' icon; toggle the initiative modes (see \ref{task_initiative}) by clicking on the task name (the text label). They can also change the specifications of each microtask (see \ref{task_configuration}).

\begin{figure}[t!]
\centering
% \begin{subfigure}{\linewidth}
\begin{subfigure}{.32\linewidth}
    \centering
    % \includegraphics[width=\linewidth]{figures/taskcard1.png}
    \includegraphics[width=\linewidth]{figures/taskcard1.png}
    \subcaption{The first page}
    \label{fig:task_card_page1}
\end{subfigure}
\begin{subfigure}{.32\linewidth}
    \centering
    \includegraphics[width=\linewidth]{figures/taskcard2.png}
    % \includegraphics[width=.8\linewidth]{figures/taskcard2.png}
    \subcaption{The second page}
    \label{fig:task_card_page2}
\end{subfigure}
\begin{subfigure}{.32\linewidth}
    \centering
    % \includegraphics[width=.75\linewidth]{figures/taskcard3.png}
    \includegraphics[width=\linewidth]{figures/taskcard3.png}
    \subcaption{Toggling initiative modes}
    \label{fig:task_card_inactive}
\end{subfigure}
\caption{The task card of a microtask. The text label of the task name is indicated with a distinctive colour. (a) By default, the card shows information of 
% \ref{fig:task_card_page1} 
% the information of 
its input and output types, which are configurable. (b) Users can click on the arrow at the bottom of the card to turn pages, and edit prompts in the second page. The user can also switch between predefined prompts. (c) The user can click on the text label to toggle initiative modes globally, or the ``visibility'' icon to display or hide all resulting diagrams.}
\label{fig:task_card}
\end{figure}



% \begin{figure*}[ht]
% \centering
% \includegraphics[width=\linewidth]{figures/task_cards.png}
% % \centering\captionsetup{width=\linewidth,font={small}}
% % \begin{subfigure}{.32\textwidth}
% %     \centering
% %     \includegraphics[width=.95\linewidth]{figures/task_card1.png}
% %     \subcaption{}
% %     \label{fig:task_card_page1}
% % \end{subfigure}
% % \begin{subfigure}{.32\textwidth}
% %     \centering
% %     \includegraphics[width=.95\linewidth]{figures/task_card2.png}
% %     \subcaption{}
% %     \label{fig:task_card_page2}
% % \end{subfigure}
% % \begin{subfigure}{.32\textwidth}
% %     \centering
% %     \includegraphics[width=.95\linewidth]{figures/task_card_inactive.png}
% %     \subcaption{}
% %     \label{fig:task_card_inactive}
% \caption{The task card of a microtask. The page (a) contains information of its input and output types, and the page (b) and (c) display the prompt. The text label of the task name is indicated with a distinctive colour. A user can 1) change input or output types; 2) manually change prompts; 3) switch between predefined prompts; 4) delete the microtask; 5) click on the ``visibility'' icon the display or hide all resulting diagrams;  6) click on the text label to toggle initiative modes (as shown in page (d) and (e)). }
% \end{figure*} \label{fig:task_card}

\paragraph{\underline{Default Microtasks}} There are 6 predefined microtasks in \textit{Polymind}, which are \BboxS{\textcolor{white}{Brainstorm}}, \EboxS{\textcolor{white}{Elaborate}}, \SboxS{\textcolor{white}{Summarise}}, \DboxS{\textcolor{white}{Draft}}, \FboxS{\textcolor{white}{Freewrite}}, and \AboxS{\textcolor{white}{Associate}}, as specified in our \textit{Goal 1}. Each microtask has a default input and output type, and prompt templates to communicate with the LLM. In practice, when operating on an input node, each microtask will replace the ``[placeholder]'' with the text in that node to prompt the LLM. For details, we refer our readers to \autoref{table:microtasks}. These defaults can all be changed later as per users' needs.

For the input type of ``nodes'', the microtask will, given a node on the canvas, sample another nearby node to perform an operation, and generated diagrams will be linked to both nodes. For the ``section'' input, the microtask will calculate the outline of all nodes (similar to \cite{lu2018inkplanner}) within the section by performing a depth-first search (DFS) of all non-leaf nodes. For example, for a section with a standalone keyword ``creativity'', and a root node ``writing'' connecting to another two keywords, ``drafting'', ``proofreading'', the resulting DFS sequence will be:
\begin{quote}
Writing

 -\ \ \ Drafting
 
 -\ \ \ Proofreading
 
 Creativity
\end{quote}

\subsubsection{The Toolbar}
The toolbar comprises four icons, a pile of sticky notes, and an ellipse representing a concept (see \autoref{fig:teaser}). Users can drag a sticky note or a concept (the ellipse) from the toolbar to the diagramming canvas, or click on the ``text'' icon and then click on the canvas again to add a keyword.
The two leftmost icons are used for connecting diagram nodes using arrows (directed) and lines (undirected). Users can click on the icon and select anchor points of a node to connect with another. There is also an icon for sectioning where users can click on it, and draw a rectangle over diagrams as a section.

\subsection{Processing a Proactive Microtask}
To achieve our \textit{Goal 3}, we introduce two initiative modes: proactive and reactive (\autoref{fig:task_status}). In the reactive mode, the microtask will function similarly to a button, and will not execute until a user clicks on its label on a task header. Those reactive microtasks will be displayed using a gray colour on the task header. By default, all microtasks are activated and run automatically in parallel with user activities.
\revision{
In practice, we sample every \textit{x} ($x=5$ in practice) seconds a diagram node for each input type: \textbf{\textcolor{keyword}{\textit{keyword}}}, \textbf{\textcolor{concept}{\textit{concept}}}, \textbf{\textcolor{sticky_note}{\textit{sticky note}}}, and \textbf{\textcolor{section}{\textit{section}}}.
Each microtask will operate on the sampled diagram corresponding to its input type, if it does not have unchecked notifications and is not displaying results on that diagram.
}
Results of these microtasks will be displayed as notifications and previews before users manually expand all resulting diagrams.

\subsubsection{Inferring User Attention}
In an earlier pilot study of our system, we found that users might lose track of generation results if a microtask was constantly operating on diagram nodes far away from users' focus of attention. Therefore, for our \textit{Goal 2.1}, we want each proactive microtask to infer the user's attention and mainly operate on diagrams of a user's focus. We make the assumption that diagram nodes that can be most easily selected are nodes of users' focus. In other words, the wider the node, or the closer to the mouse cursor, the more likely the node is the focus of users' attention, and operating on it will more likely make users aware. In practice, every \textit{x} seconds, for each node among each of the 5 types of input, we compute an index of difficulty ($ID$) according to Fitts' law \cite{fitts1954information}:
\begin{equation*}
    ID = \log_2 \frac{2D}{W}
\end{equation*}
where $D$ denotes the distance between the current mouse position and the node centre, and $W$ denotes the width of the node.

However, if we choose the node with the lowest $ID$ each time, we will end up always choosing the same node for operation if a user barely moves the mouse within a period of time. Therefore, in practice, we sample a node from a uniform distribution where the probability of a node $i$ being sampled as an input type $T$ is computed as:
\begin{equation*}
p(i, T) = \frac{ID_i}{\sum_{j \in T} ID_j}
\end{equation*}
$T$ includes \textbf{\textcolor{keyword}{\textit{keyword}}}, \textbf{\textcolor{concept}{\textit{concept}}}, \textbf{\textcolor{sticky_note}{\textit{sticky note}}}, and \textbf{\textcolor{section}{\textit{section}}}.
\revision{For microtasks that operate on two \textbf{\textit{nodes}} (e.g., the default \AboxS{\textcolor{white}{Associate}}), we first sample a primitive diagram, and then randomly sample a nearby diagram for prompting.}
%and \textbf{\textit{nodes}}.

\subsubsection{Notifications and Previews of Microtask Results}
We use two levels of attention draw features: notifications and previews of microtask results to support \textit{Goal 2.1}, so that results ready for display will not intrude users' diagramming process.

\begin{figure*}[ht]
% \centering\captionsetup{width=\linewidth,font={small}}
\includegraphics[width=\textwidth]{figures/notifications_previews4.png}
\caption{The processing of a proactive microtask. Once results are obtained from the LLM, it will first ``draw a curtain'' over the task header to notify users, displaying key points of the generation. Users can click on the ``expand'' icon on the ``curtain'' for a quick view. If a user does not expand these result, it will be marked as unread notifications. Once users hovers over the task header, a preview panel will pop up, showing a preview of all unread results.}
\label{fig:notifications_and_previews}
\end{figure*}

\paragraph{\underline{Notifications}}
To notify the user once results of a proactive microtask are ready, it will first ``draw a curtain'' (see \autoref{fig:notifications_and_previews}) over the task header of the node being operated. The curtain will be filled with a distinctive colour indicating which microtask those results are from, and meanwhile, display key points summarising the generated results. Users can click on the ``expand'' icon on the right to display results, and then click on the ``collapse'' icon to hide them. If the user does not perform any operations, the curtain will collapse, and those results will be marked as ``unread'' with a small red circle on the upper-left corner as a sign of notification. (the red circle in \autoref{fig:notifications_and_previews}).

\paragraph{\underline{Previews}}
Once there are unread results, the preview panel of a task header will store a preview of these results. When a user hovers over the task header, the preview panel (should there be any results unread) will pop up, displaying key points of results of each microtask, using labels filled with distinctive colours (see \autoref{fig:notifications_and_previews}). Once a user hovers over a specific label for 1.5 seconds, the key point will turn into a brief summary of the microtask results, displayed using effects resembling a news ticker (see \autoref{fig:preview_summary})

% \begin{figure}[ht]
\centering
\includegraphics[width=.5\linewidth]{figures/preview_summary2.png}
\caption{Once the mouse hovers over key points of a microtask on the preview panel for 1.5 seconds, the system will present a summary of the generated results using a news ticker effect.}
\label{fig:preview_summary}
\end{figure}

\subsection{Managing Microtasks}
To support our \textit{Goal 2} and \textit{Goal 3}, we design a set of features to support configuring existing microtasks, and delegating new microtasks.

\begin{figure*}[ht]
\centering
\includegraphics[width=\textwidth]{figures/microtask_status_v6.png}
% \centering\captionsetup{width=0.64\linewidth}
% \begin{subfigure}{0.32\textwidth}
%     \centering
%     \includegraphics[width=\linewidth]{figures/status1.png}
%     \subcaption{normal mode}
%     \label{fig:status1}
% \end{subfigure}
% \begin{subfigure}{0.32\textwidth}
%     \centering
%     \includegraphics[width=\linewidth]{figures/status2.png}
%     \subcaption{notification}
%     \label{fig:status2}
% \end{subfigure}
% \begin{subfigure}{0.32\textwidth}
%     \centering
%     \includegraphics[width=\linewidth]{figures/status3.png}
%     \subcaption{display mode}
%     \label{fig:status3}
% \end{subfigure}
% \begin{subfigure}{0.45\textwidth}
%     \centering
%     \includegraphics[width=.85\linewidth]{figures/status4.png}
%     \subcaption{inactive mode}
%     \label{fig:status4}
% \end{subfigure}
% \begin{subfigure}{0.45\textwidth}
%     \centering
%     \includegraphics[width=.85\linewidth]{figures/status5.png}
%     \subcaption{displaying results in the inactive mode (when clicked)}
%     \label{fig:status5}
% \end{subfigure}
\caption{Different status of a microtask on a particular node.}
\label{fig:task_status}
\end{figure*}
\subsubsection{Toggling Visibility}
For the purpose of \textit{Goal 2.2}, users can toggle the visibility of microtask results both globally and locally. By clicking on the ``visibility'' icon on each task card \autoref{fig:task_card_page1}, users can display all corresponding generations on the canvas. Users can also click on the task name label on each task header \autoref{fig:task_status} to display generations corresponding to the specific node locally.

For a reactive microtask, generations are only requested once users click on the label on a task header. On such occasions, the label turns black once generations are ready, and users can click on the label again to hide all generations. For a proactive microtask, besides clicking on the ``expand'' icon on the ``curtain'' (\autoref{fig:notifications_and_previews}), users can also click on the label on task headers to see all resulting diagrams of a microtask. In this case, the microtask on this node is in ``display'' mode, and the text on the label is underlined. Should there be unread results, there would also be an ``expand all'' icon that users can click on to show the results of all microtasks. Should there be any microtask in ``display'' mode, there would also be a ``close all'' icon for users to hide the results of all microtasks.

\subsubsection{Toggling Initiative Modes} \label{task_initiative}
The initiative modes (proactive vs. reactive) can also be toggled both globally and locally for our \textit{Goal2} and \textit{Goal 3}. Users can either click on the task name label on the ``task card'' to deactivate it globally (\autoref{fig:task_card_inactive}), or control-click labels on each task header to deactivate the microtask for a specific node (\autoref{fig:task_status}). Once a microtask is globally deactivated, the corresponding label will turn gray and function solely as a button on all task headers, provided that it is not in ``display'' mode and has no notifications. Task headers of newly created diagrams will also have this microtask turned off.

\subsubsection{Task Configuration and Delegation}
To support our \textit{Goal 2}, we allow users to specify microtask requirements, including input and output types, and prompts.
We also enable users to rapidly create and delegate a new microtask to the LLM.

\paragraph{\underline{Configuring Task Specifications}} \label{task_configuration}
Users can change the input and output types of a microtask by clicking on the left and right arrow triangle icon (\autoref{fig:task_card}a). We currently support four input types (\textbf{\textcolor{keyword}{\textit{keyword}}}, \textbf{\textcolor{concept}{\textit{concept}}}, \textbf{\textcolor{sticky_note}{\textit{sticky note}}}, \textbf{\textcolor{section}{\textit{section}}}) and three output types (\textbf{\textcolor{keyword}{\textit{keyword}}}, \textbf{\textcolor{concept}{\textit{concept}}}, \textbf{\textcolor{sticky_note}{\textit{sticky note}}}). Users can also change prompts by double-clicking the text box of the prompt (\autoref{fig:task_card}b). For some microtasks, they can also switch between predefined prompt examples. Each prompt should have a ``[placeholder]'' to be filled with the text of a node (\autoref{fig:task_card}b).

\paragraph{\underline{Delegating a New Microtask}}
For our \textit{Goal 2.3}, a user can click the ``add'' icon on the task board to create a new microtask (\autoref{fig:teaser}B). After clicking, the task board will spawn a new card for users to specify the task name and the example prompt. Users can later change input and output type, or toggle visibility or initiative on a newly added task card after clicking ``confirm''. When a user clicks to specify a task name, the system will prompt the LLM to suggest a task name. Once a task name is specified, the system will also request an example prompt for users from the LLM. \autoref{fig:add_new_task} illustrates the whole process of adding a new microtask.

% \begin{figure*}[ht]
% \begin{subfigure}{0.45\textwidth}
%     \centering
%     \includegraphics[width=.85\linewidth]{figures/new_task1.png}
%     \subcaption{}
%     \label{fig:new_task1}
% \end{subfigure}
% \begin{subfigure}{0.45\textwidth}
%     \centering
%     \includegraphics[width=.85\linewidth]{figures/new_task2.png}
%     \subcaption{}
%     \label{fig:new_task2}
% \end{subfigure}
% \begin{subfigure}{0.45\textwidth}
%     \centering
%     \includegraphics[width=.85\linewidth]{figures/new_task3.png}
%     \subcaption{}
%     \label{fig:new_task3}
% \end{subfigure}
% \begin{subfigure}{0.45\textwidth}
%     \centering
%     \includegraphics[width=.85\linewidth]{figures/new_task4.png}
%     \subcaption{}
%     \label{fig:new_task4}
% \end{subfigure}
% \caption{}
% \end{figure*} \label{fig:add_new_task}

\begin{figure*}[ht]
% \centering\captionsetup{width=\linewidth,font={small}}
\includegraphics[width=\textwidth]{figures/add_new_task_v3.png}
\caption{The workflow of delegating a new microtask. (a) The user first clicks on the ``add'' icon and a new card will appear. (b) After user inputs a task name, the system requests an example prompt from the LLM. (c) User clicks ``confirm'' to add the microtask.}
\label{fig:add_new_task}
\end{figure*}

\subsection{Implementation}
\textit{Polymind} was implemented using Javascript, React \footnote{https://react.dev/} and react-konva \footnote{https://konvajs.org/docs/react/Intro.html}. The backend LLM was ChatGPT (GPT-4), and the temperature was set to $0.7$.
We set $x=5$ based on our pilot study, i.e., every 5 seconds, we sample a diagram node for each of the 5 input types to be processed by proactive microtasks. If \textit{x} were too small, users would be overwhelmed with notifications of new results; if too large, there would be noticeable latency, and users will often end up waiting for results. Once the user confirms an update of a diagram node's text content, a re-sampling of the corresponding input type will be executed immediately.

To reduce the computational cost, we quit prompting ChatGPT with the sampled diagram node if a proactive microtask has unchecked results (i.e., notifications) or expanded diagrams on it.
To make generations less random, in each prompt we added constraints on its output.
\revision{For \textbf{\textcolor{keyword}{\textit{keyword}}}, we request 3 generations, each no more than 3 words; for \textbf{\textcolor{concept}{\textit{concept}}} we request 3 generations no more than 5 words; for \textbf{\textcolor{sticky_note}{\textit{sticky note}}} we request 1 generation no more than 150 words.}
Whenever ChatGPT returns results, we preserve the previous dialogue in each resulting diagram node. Once users request an explanation or a regeneration of the node, we further prompt the ChatGPT based on the saved dialogue (see \ref{appendix_feedback}). When the system requests a microtask name, we use names of predefined microtask as a few-shot template to prompt the ChatGPT. When the system requests a prompt for a new microtask, we use name-prompt pairs of predefined microtasks as a few-shot template. We refer our readers to \ref{appendix_newtask} for more details.