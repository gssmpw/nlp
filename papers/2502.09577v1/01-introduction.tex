\section{Introduction}
In creative ideation, people often apply semi-structured diagrammatic methods, such as mind mapping, concept mapping, and argument mapping~\cite{baroudy2008procedural,mogahed2013planning}.
These thinking strategies support humans' visual exploration of key topics, concepts, arguments, and their relationships~\cite{lu2018inkplanner} by connecting, arranging, and grouping the corresponding graphical representations.

%For example, writers may use mind mapping to spark initial inspirations and discover creative links between ideas.
%As more ideas emerge, writers can employ concept mapping to organize them into a coherent narrative~\cite{davies2011concept}.

%However, most current diagramming tools only help with the graphical organization of existing ideas but do not directly contribute new ideas. 
In addition to using visual thinking tools, creativity-related research has highlighted that certain global thinking strategies can enhance creativity.
One strategy that has been found beneficial for creative thinking in numerous domains~\cite{lo2017critical,dell2020four}, such as writing~\cite{al2018effect,al2015effect} and problem solving~\cite{peterson1998parallel}, is \textit{parallel thinking}~\cite{bono1985six,de2017six}.
It encourages systematic exploration and evaluation of ideas with distinct modes of thought.
For ideation activities, two modes of thinking are of key significance: divergent and convergent thinking~\cite{runco2023creativity}. The former involves generating multiple solutions, while the latter focuses on working out a single correct or best solution to a given problem.
%For prewriting activities, the modes may involve generating new ideas, critiquing these ideas, finding associations between them, and more. 
%, breaking free from conventional thought patterns.
However, successfully applying both convergent and divergent thinking simultaneously often requires considerable training~\cite{de2017six,gregory2012real}, as it breaks away from traditional cognitive patterns by asking for an explicit switch of mental modes.
In practice, people often rely on group collaboration to achieve convergent and divergent thinking in parallel, where individual group members offer separate angles of thought.
Previous creativity literature has further proved that real-time parallel inputs in human collaboration can produce significantly more unique ideas than serial interaction interfaces ~\cite{shih2009groupmind,hymes1992unblocking}, and presenting multiple ideas in parallel can stimulate more productive and creative outcomes than serial presentation ~\cite{dow2010parallel}.
%We are thus interested in enhancing human ideation in prewriting with artificial-intelligence-powered parallel thinking by integrating ``virtual collaborators'' into visual diagramming tools.

We are thus interested in exploring  human-AI collaborative ideation where ``virtual collaborators'' powered by generative artificial intelligence enhance visual diagramming and facilitate parallel thinking.
%Although crucial for the quality of the final draft, prewriting is often overlooked by writing tool developers \cite{lu2018inkplanner}. 
Our effort focused on the creative act of \textit{prewriting}, the first stage of the writing process where writers discover and develop new ideas before they are ``\textit{ready for the words and the page}''~\cite{rohman1965pre}.
Research has shown prewriting strategies~\cite{barnett1989writing,rohman1965pre} can improve the quality of final documents by relieving memory strain~\cite{flower1981cognitive,kellogg1990effectiveness} and incubating new ideas with organised schema of existing knowledge~\cite{lorenz2009using,davies2011concept}.

Authors often apply visual diagramming to expand and organize their ideas in prewriting~\cite{mogahed2013planning}.
There has been a growing interest in leveraging Large Language Models (LLMs) to support writing in general~\cite{chung2022talebrush,clark2018creative,zhang2023visar,suh2023sensecape} and specifically to enhance writers' idea diversity~\cite{benharrak2024writer, shaer2024ai}.
However, conversational LLM interfaces, as exemplified by ChatGPT, fit poorly into diagramming-based ideation workflows.
%Notably, LLMs have shown strong potential for collaborative ideation together with writers for a variety of tasks, including scientific communication~\cite{gero2022sparks}, journalistic reports~\cite{petridis2023anglekindling}, and creative writing~\cite{yuan2022wordcraft}.
%LLM-based ideation support goes beyond scaffolding human thought process, as most existing diagramming tools do, and directly generate new text to follow~\cite{chung2022talebrush,yuan2022wordcraft} or topics to expand. 
%\cite{chung2022talebrush}, provide inspiration and external perspectives \cite{gero2022sparks}, etc., which leaves room for design opportunities to integrate LLMs into a prewriting interface.
%However, adapting LLMs to prewriting workflows in a productive manner requires additional design considerations.
A conversational interface removes its users from the often diagrammatic and divergent thinking context, making it difficult for users to leverage existing information represented in the diagrams.
Furthermore, productive parallel thinking requires systematically applying multiple distinct angles of thoughts over the ideas at hand. 
For a pure-text, turn-taking conversational interface, this entails that a user has to remember the specific prompt for each angle and repeatedly apply them throughout the often iterative prewriting process~\cite{rohman1965pre}.
Although several previous systems have already incorporated diagrammatic interfaces and multiple distinct LLM operations for prewriting (e.g., ~\cite{lu2018inkplanner,zhang2023visar,suh2023sensecape}), they are still far from achieving parallel thinking effects in that the interactions remained a serial manner where each LLM operations waited for users to switch into different modes of thoughts.
%As Rohman put it \cite{rohman1965pre}, prewriting is a stage of ``discovery'' in itself, which implies its informal and iterative nature.
%Compared to many writing tasks, prewriting often does not have a set of clearly defined goals, making it difficult for an intelligent agent to infer the current intention and focus of users~\cite{horvitz1999principles}. %Furthermore, many prewriting strategies are diagram-based and imply multiple iterations to reach a first draft (e.g., \cite{lu2018inkplanner}).

%As such, LLM-augmented diagramming for prewriting calls for new human-AI collaboration workflows that connect plain-text-based LLMs with graph-like information structures in an iterative manner.

%In this paper, we introduce \textit{Polymind}, a visual diagramming tool that incorporates large language models to facilitate prewriting. Visual diagrams contain both semantic information, often writing ideas or thoughts, and their relationships, often organised in a graph-like structure on a canvas, which affords construction of new ideas or knowledge based on existing structural presentations. It is a commonly used prewriting paradigm both in research literature \cite{lu2018inkplanner,zhang2023visar}, and commercial tools (e.g., Figma \footnote{https://www.figma.com/}, Miro \footnote{https://miro.com/}, etc.). With a visual diagramming tool users are able to practice strategies such as argument mapping \cite{twardy2004argument}, concept mapping \cite{novak1990concept}, mind mapping \cite{buzan2002mind}, etc., which already proved to be beneficial in a variety of writing tasks including science writing \cite{barstow2017experimental}, report writing \cite{manalo2022use}, story writing \cite{lan2015computer}, and so forth.
In this paper, we introduce \textit{Polymind}, a visual diagramming tool infused with multiple distinct LLM agents working in parallel to facilitate prewriting.
%We aim to equip diagramming tools with LLMs to help expand existing semantic ideas, and meanwhile maintain and organise semantic relationships. These tasks can seem repetitive and cumbersome for conversational interactions as users need to iteratively translate graph-like diagram data into prompts and vice versa.
\textit{Polymind} supports a writer's initiative in listing, arranging, and connecting ideas via a diagramming interface while augmenting this visual thinking process in-situ with LLM-powered parallel angles of thought.
These angles cover a repertoire of key prewriting techniques involving both convergent and divergent thinking, such as ideation, association, summarisation, drafting, as informed by our literature review of both writing and creativity support tools.
We operationalize these techniques through small, manageable, and independent ``microtasks'' that LLMs execute over diagrams nodes as the writer's virtual collaborators. Microtasking is commonly used in human group collaboration, including crowdsourcing \cite{latoza2014microtask,chen2017retool}, collaborative writing \cite{iqbal2018multitasking,birnholtz2013write,teevan2016supporting}, and group brainstorming \cite{chilton2019visiblends,teevan2016supporting}, as it proves to effectively simplify complex tasks \cite{cheng2015break,kokkalis2013taskgenies}, facilitate recovery from interruptions, and produce higher quality results \cite{cheng2015break}.
In \textit{Polymind} these microtasks process ideas and their relationships as represented in the diagrams to expand, transform, and re-organize them.
The results are reflected back in the diagrams as new or modified nodes and connections. 
Further, \textit{Polymind} allows users to run and orchestrate multiple microtasks simultaneously to achieve efficient expansion and iteration of ideas.  
%In an attempt to replace a chat bot on a diagramming canvas, we obtained the design philosophy of \textit{Polymind} from our formative study and the idea of ``parallel thinking'' proposed by De Bono in his book ``\textit{Six Thinking Hats}'' \cite{bono1985six,de2017six}. The original form of ``parallel thinking'' separates the human thinking process into six clear and distinct functions and roles, and encourages splitting focus of thinking in the six directions one by one. In reality, ``parallel thinking'' can often be practiced through group collaboration to improve efficiency, where each of the six roles can be played in parallel by different individuals or groups independently.
%Similarly, our formative study suggests that LLMs can also be delegated distinct, independent, and context-free tasks, and users often would like them to run concurrently, and in parallel to their own thinking process, so as to quickly explore different possibilities upon requests, as if communicating with a human collaborator.
%Therefore we seek to break down a complex prewriting task into multiple smaller, manageable, and independent tasks, so that LLMs can be delegated specific context-free tasks that contribute concurrently and independently without awareness or efforts from the user, and provide suggestions whenever requested.

%We adopt the notion of ``microtasks'' from human group collaboration, including crowdsourcing \cite{latoza2014microtask,chen2017retool}, collaborative writing \cite{iqbal2018multitasking,birnholtz2013write,teevan2016supporting}, and group brainstorming \cite{chilton2019visiblends,teevan2016supporting}, as microtasking workflows have been shown to effectively simplify complex tasks \cite{cheng2015break,kokkalis2013taskgenies}, facilitate recovery from interruptions, and produce higher quality results \cite{cheng2015break}. 

%To facilitate complex prewriting scenarios more efficiently, we simulate these workflows using multiple microtasks to expand or organise visual diagrams, while users are able to orchestrate these tasks simultaneously.
%To scaffold the prewriting process with well-defined collaboration goals, \textit{Polymind} derives multiple pre-defined microtasks from existing literature of both writing support and creativity support tools.

% In \textit{Polymind}, each microtask can be completed and presented to users independently, requiring little or no awareness from the user or context from other microtasks.
% \textit{Polymind} also allows users to rapidly create and delegate their custom microtasks to the LLM, and provides suggestions of task goals and prompts during task delegation.
% To alleviate users' workload in managing multiple parallel microtasks, while still offering always available assistance, \textit{Polymind} operates in a mixed-initiative manner~\cite{horvitz1999principles}, where a microtask can proactively provide suggestions or execute only upon user request. 
% For each proactive microtask, \textit{Polymind} heuristically infers users' focus of attention and the appropriate timing of their suggestions to provide awareness information such as which diagram has been processed and what results have been generated. 
% Users can manage these microtasks by changing task specifications, and toggling their initiative and visibility modes.
\revision{Our evaluation involved 10 non-expert writers with limited technical background performing two prewriting sessions. Results showed that \textit{Polymind} was overall seen as more creative and customisable for prewriting than a turn-taking conversational interface.}
Parallel LLM assistance allowed participants to efficiently leverage both convergent and divergent modes of thinking.
This lent support to a number of new ideation capabilities, including quick expansion of existing diagrams (e.g., a concept tree).
Meanwhile, our mixed-initiative workflows granted easier and faster control over LLM generation on a prewriting canvas.
Our study results also highlighted more nuanced design considerations around parallel LLM assistance for human-AI co-creativity, suggesting the value of dynamically adjusting AI pro-activity levels and balancing awareness of AI actions and non-intrusiveness to human workflows.   
%In comparison to conventional conversational interfaces, \textit{Polymind} saved the effort of iteratively and repetitively crafting prompts.

%After initial training, most users felt it was easy to leverage the power of LLMs while prewriting using \textit{Polymind}. They spoke highly of key design principles and features of ``microtasking'' as appropriate and helpful to support their prewriting tasks.

Our paper makes the following contributions to HCI:
\begin{itemize}[topsep=0pt]
\item a human-AI collaborative ideation workflow with multiple, heterogeneous AI-based microtaskers to achieve parallel thinking for visual diagramming.
\item a set of user interface designs to support the management of these microtasks in a mixed-initiative manner to support users' creative agency and control.
\item a functional implementation of the workflow and  interface designs as an LLM-based, microtask-enhanced diagramming tool for creative prewriting.
\item a user study that shows the feasibility and potential of the microtasking workflow
\end{itemize}

% The remaining part of the paper is organised as follows. Section 2 briefly reviews related work and situate our contributions in existing literature. Section 3 introduces our formative study where we derive our main design goals. Section 4 introduces the interface and key features of our system \textit{Polymind}. We then report our evaluation in Section 5 and discuss implications and future work in Section 6. We conclude this paper by summarising our key contributions and study results.