





\section{\OurMethod}

\subsection{Architecture}
\label{sec:Architecture}


\begin{figure*}
    \centering
    \includegraphics[width=1\linewidth]{bar2.pdf}
    \caption{(a) shows the bar chart of the raw data, (b) presents the results of applying Moving Average Smoothing to reduce anomalies in prediction percentages, and (c) highlights the reduction of visual clutter and emphasizes sequential behavior patterns after merging behaviors of the same category.}
    \label{fig:bar}
    \Description{(a) shows the bar chart of the raw data, (b) presents the results of applying Moving Average Smoothing to reduce anomalies in prediction percentages, and (c) highlights the reduction of visual clutter and emphasizes sequential behavior patterns after merging behaviors of the same category.}
\end{figure*}

\section{Data Collection and Processing}
\label{sec:data}
\RR{In this section, we provided an overview of the data collection context and introduced the collaborative programming performance framework along with its metric quantification methods.}

\subsection{Data Collection}
We collaborated with Professor E1, an expert in programming education, and teaching assistants (TA1 and TA2), experienced in Python, to collect data from E1's Spring 2023 Python course with 66 non-computer science freshmen in 22 groups. Using non-intrusive methods, we recorded group discussions, screen activities (without audio), and code submissions. Session lengths ranged from 10 to 60 minutes based on question completion. 
Due to data quality issues, we selected data from 19 groups (57 students) for analysis.


\subsection{Data Preprocessing}
In collaborative programming analysis, students' spoken content was key to understanding discussion and evaluating collaboration. We used the Faster-Whisper model~\cite{fasterwhisper} for speech recognition and the Pyannote-audio model~\cite{pyannoteaudio} for speaker diarization. 
For groups lacking clear problem-solving strategies, we used Tesseract OCR~\cite{tesseract} to analyze screen recordings and extract key frames through screenshots.

\subsection{Scope of Collaborative Programming Performance Framework}
Evaluating student and group performance in collaborative programming required considering multiple dimensions~\cite{hawlitschek2023empirical}.  
Building on literature and expert input (E1), we proposed the following comprehensive analytical framework to assess performance. 



\subsubsection{Student Performance Assessment}
\label{shema}
Previous research demonstrated that students' skills, backgrounds, and personalities in the classroom vary significantly, affecting their engagement and learning outcomes~\cite{wu2019analysing}. 
Therefore, we focus on each student's \textit{background} (prior academic performance and major), \textit{role transitions}, \textit{behavioral engagement}, and \textit{cognitive engagement}.






\textbf{Problem-solving Categorization:}
Based on previous frameworks~\cite{wu2019analysing}, team theory~\cite{zhao2023analysing}, and collaborative coding processes~\cite{sun2021three}, we developed a coding scheme (Fig.~\ref{fig:scheme}) to capture group problem-solving in collaborative programming. 
The scheme used four color-coded categories to represent discussion types. 
The first three categories followed a hierarchical structure, indicating discussion depth, while the green category focuses on situation awareness and specific behaviors.

Building on the scheme, we used tailored prompts with the ChatGPT-4o model~\cite{gpt4o} to classify behavioral patterns in transcribed dialogue \RR{(More details are in appendix B)}. 
\RR{The model provided a prediction percentage of uncertainty for each classification, improving result interpretability. }
To minimize anomalies, we applied a ``moving window'' technique with Moving Average Smoothing~\cite{chang2022muse}, stabilizing prediction percentages (Fig.\ref{fig:bar}-b). To reduce visual clutter in long time-series data, we aggregated consecutive instances of the same category, averaging prediction percentages (Fig.\ref{fig:bar}-c). These results were displayed in the timeline panel's progress bar, enabling detailed analysis by zooming into specific behavior categories in Sec.~\ref{barchart}. 




\textbf{Roles Extraction:}
We analyzed each speaker's dynamic roles (Driver, Navigator, and Monitor) during programming~\cite{lewis2011pair}. Using ChatGPT-4o and prompts based on the Thought Chain Model~\cite{wei2022chain}, we guided the model through step-by-step reasoning to generate role classifications. Prompts were iterated for clarity, and the model's responses were structured hierarchically and returned in JSON format. Each query was repeated ten times, with the majority result adopted for classification.

\RR{\textbf{Behavioral Engagement:} reflected the level of effort and participation students invested in learning~\cite{fredricks2022measurement}. 
In our study, we focused on the duration and frequency of student speech.} 
We extracted conversation data, excluding irrelevant chat, and divided each conversation into two parts: the first half and the full conversation. We then measured speaking duration, frequency, and degree centrality using co-occurrence networks~\cite{ng1999toward}. For each question, we created and normalized two networks, followed by Non-negative Matrix Factorization (NMF)~\cite{lee2000algorithms} to identify key behavioral patterns for dynamic group comparison.


\RR{\textbf{Cognitive Engagement:} referred to the cognitive investment students made in their learning. We highlighted the role changes and behavior frequencies of students during the collaborative process. }
To capture dynamic changes in student cognitive engagement, we split the dialogue for each question into two segments: the first half and the full dialogue. We extracted the frequency of each speaker's 14 behavioral categories and their roles at each timestamp. After normalizing these features for consistency, we applied NMF to reduce dimensionality and assess each speaker's cognitive engagement.

\begin{figure*}
  \includegraphics[width=\textwidth]{CPVis.pdf}
  \caption{\RR{A screenshot of Group 10 view.} \textit{CPVis} applies multimodal learning analysis to provide instructors with evidence for evaluating group and student performance. It consists of three views:
Filter View (A) Provides an overview and allows group selection. The selected groups appear in the lasso selection area (A2), and the similarity panel (A3) displays the most similar and different groups based on the search (A1a).
Content View (B) Displays group performance, with the B1 panel showing completed codes, the B3a panel illustrating the behavior sequence, and the B3b panel showing student engagement over time.
Detail View (C) Presents the group's collaborative programming video (C1) and raw conversation data (C2).}
  \Description{A screenshot of Group 10 view. \textit{CPVis} applies multimodal learning analysis to provide instructors with evidence for evaluating group and student performance. It consists of three views:
Filter View (A) Provides an overview and allows group selection. The selected groups appear in the lasso selection area (A2), and the similarity panel (A3) displays the most similar and different groups based on the search (A1a).
Content View (B) Displays group performance, with the B1 panel showing completed codes, the B3a panel illustrating the behavior sequence, and the B3b panel showing student engagement over time.
Detail View (C) Presents the group's collaborative programming video (C1) and raw conversation data (C2).}
  \label{fig:teaser}
  \end{figure*}

\subsubsection{Group Performance Assessment}
We evaluated group performance based on three dimensions: code quality, collaborative problem-solving, and teacher scaffolding. 
Through in-depth discussions with domain experts, we assessed how each dimension was valued and measured in the context of our study.




\label{code}
\textbf{Code quality}, reflecting students' mastery of course concepts, was a key metric for evaluating group performance. To assess student submissions, we used ChatGPT-4o~\cite{gpt4o} to evaluate dimensions such as problem-solving, code integrity, accuracy, and algorithmic innovation, scoring each on a 1–5 scale. After refining evaluation prompts, we ran the assessment ten times per submission, averaging the results to ensure consistency and reliability.





\textbf{Collaborative Problem-Solving (CPS):} 
Earlier studies categorized CPS into team effectiveness and task effectiveness~\cite{rosen2020towards}. Team effectiveness was measured by student engagement, while task effectiveness was assessed through code quality. %Our analysis captured problem-solving behaviors by frequency and sequence.
To evaluate CPS, we examined task effectiveness, represented by the average question score (\(\bar{s}\)), and team effectiveness, assessed through the standard deviation of engagement (\(\sigma_e\)) and the average engagement score (\(\bar{e}\)) as shown in Equation \ref{eq:1}. We then used the coefficient of variation (\(CV_e\)) \RR{to account for both engagement variability and engagement}. Finally, the overall collaboration quality was calculated using Equation \ref{eq:2}, combining question performance and engagement balance. 
\begin{equation}
\sigma_e = \sqrt{\frac{1}{n} \sum_{i=1}^{n} (e_i - \bar{e})^2}, \quad CV_e = \frac{\sigma_e}{\bar{e}}
\label{eq:1}
\end{equation}

\begin{equation}
Quality = \bar{s} \cdot (1 - CV_e)
\label{eq:2}
\end{equation}
As shown in Table \ref{table:comparison}, Group 19, despite achieving a respectable average score, exhibited imbalanced engagement, leading to a lower collaboration quality score. In contrast, Group 20 demonstrated more balanced and higher engagement, resulting in a better overall collaboration quality.
\begin{table}[htbp]
\centering
\begin{tabular}{cccccc}
\toprule
\textbf{Group} & \(\bar{s}\) & \textbf{Engagement Levels} & \(\sigma_e\) & \(\text{CV}_e\) & \textbf{CQ} \\
\midrule
Group 19 & \(4.11\) & (10.515, 9.725, 4.575) & \(2.80\) & \(0.24\) & \(2.80\) \\
Group 20 & \(4.14\) & (10.06, 9.32, 8.62) & \(0.73\) & \(0.08\) & \(3.88\) \\
\bottomrule
\end{tabular}
\caption{Comparison of Group 19 and Group 20 on Collaboration Quality (CQ).}
\label{table:comparison}
\end{table}

\textbf{Teacher Scaffolding,} categorized into cognitive (low, medium, high-control) and metacognitive forms~\cite{ouyang2022applying}, reflected the level of support provided to a group and its impact on programming performance. We evaluated four scaffolding dimensions, leveraging GPT-4o for annotation. By using targeted prompts and examples, we improved classification accuracy, while teacher scaffolding was categorized according to the type of support based on a semantic analysis of interactions.




\begin{algorithm}
\caption{\methodname{} Training}
\KwIn{Coarse-to-fine Autoencoder $\text{Enc}$, $\text{Dec}$}
\KwOut{}
\For{$i \gets 1$ \textbf{to} $n-1$}{
    \For{$j \gets 1$ \textbf{to} $n-i$}{
        \If{$L[j] > L[j+1]$}{
            Swap $L[j]$ and $L[j+1]$
        }
    }
}
\Return $L$
\end{algorithm}



This technical report presents \OurMethod, a new series of open-source vision-language models that explore long-context vision understanding without token compression and sparse local attention.
%
Our multi-modal architecture is constructed around three core components: the Vision Encoder, the Projector, and the LLM.
%


% \begin{itemize}
%     \item
    \textbf{Large Language Model.}
    We choose Qwen2.5-14B-Instruct~\cite{Qwen2.5} as our LLM.
    
    % \item
    \textbf{Vision Encoder.}
    We consider the InternViT-300M~\cite{InternVL} as the visual encoder.
    %
    We introduce a dynamic tiling vision encoding strategy~\cite{InternVL2} that efficiently processes high-resolution images of varying aspect ratios.

    % \item
    \textbf{Vision-Language Projector.}
    We employ a $2$-layer MLP to project image features into the word embedding space.
    We also apply a simple pixel shuffle~\cite{InternVL2} to visual tokens and reduce the number of visual tokens to one-quarter.
    %
% \end{itemize}


\subsection{Data Construction}
\label{sec:Data}

%
\OurMethod is trained on open-source datasets only.
%
As shown in Tab.~\ref{table_data}, the training dataset encompasses a diverse range of sources.
%


\textbf{Image-Text Data.}
%
The datasets employed can be categorized into three groups:
%
\begin{itemize}[leftmargin=2.0em]
    \item
    \textbf{Image Captioning.}
    %
    The visual caption dataset consists of LLaVA-ReCap~\cite{li2024llavanext-ablations}, ALLaVA-4V~\cite{ALLaVA-4V}, ShareGPT4V~\cite{ShareGPT4V} and LLaVA-OneVision-Mid~\cite{LLaVA-OneVision}.
    %

    \item
    \textbf{Visual Question Answering.}
    %
    We combine general VQA from LVIS-Instruct4V~\cite{LVIS-Instruct4V}, the-cauldron~\cite{Idefics2}, Docmatix~\cite{Idefics3}, LLaVA-OneVision~\cite{LLaVA-OneVision}.
    %

    \item
    \textbf{Interleaved Image-Text.}
    %
    To empower all-round multi-image capabilities, we employ the M4-Instruct~\cite{li2024llavanext-interleave}.
    %
    To further enhance multi-image understanding with more than $10$ images, we collect the public comic book with the corresponding detailed synopsis from the web and build the Comic-9k datasets.
    %
    Specifically, Comic-9k contains $200$K images, spanning $9$K comic books, along with a manual-labeled synopsis.
    %
\end{itemize}





\begin{table*}[!htbp]
 \caption{
 Maximal supported sequence length for inference and training.
 }
 % \vspace{-15pt}
 % \footnotesize
 \begin{center}
  \begin{adjustbox}{max width=0.99\textwidth}
   \begin{tabular}{lcc|ccc|c}
    \toprule
    

     \multirow{3}{*}{Name} & \multirow{3}{*}{\# Param.} & \multirow{3}{*}{Device Type} & \multicolumn{4}{c}{Devices Number} \\
     \cmidrule{4-7}
     &&& \multicolumn{3}{c|}{Inference} & Training \\
     \cmidrule{4-7}
     &&& 8 & 16 & 32 & 64 \\

    
    \midrule
    
    LongVILA~\cite{LongVILA}            & 7B & 80G GPU & 276K & -- & -- & 666K \\
    
    \midrule
    \multirow{2}{*}{\OurMethodBF}       & \multirow{2}{*}{14B} & 64G NPU & -- & 1,024K & -- & 1,024K \\
    
                                        & & 96G GPU  & 400K & 800K & 1,600K & 1,024K \\

    \bottomrule%==============================================================================================================
   \end{tabular}
  \end{adjustbox}
 \end{center}
 \label{table_max_length}
 % \vspace{-20pt}
\end{table*}



\begin{table*}[!htbp]
	\caption{
%		Ablation study on logits-masked language modeling head.
		Comparison of different language modeling head.
	}
	% \vspace{-15pt}
	% \footnotesize
	\begin{center}
		\begin{adjustbox}{max width=0.99\textwidth}
			\begin{tabular}{c|r|rl|l}
				\toprule
				
				
				\# GPUs & Method & \# Frames & \# Tokens & Prefill Time (s) \\
				
				\midrule
				\multirow{5}{*}{8}
				
				&\multirow{2}{*}{Original LM Head}
				& 390 & 100K & 19.2 \\
				
				&& 400 & 103K & OOM \\
				
				
				\cmidrule{2-5}
				
				& \multirow{3}{*}{Logits-Masked LM Head}
				& 390 & 100K & 10.1 \color{red}{ ( $\downarrow$ 47.3\% ) } \\
				&& 1,620 & 417K \color{red}{ ( $\uparrow$ 417\% ) } & 102 \\
				&& 1,630 & 420K & OOM \\
				
				\midrule
				
				\multirow{4}{*}{32}
				
				& \multirow{2}{*}{Chunked LM Head}
				& 4,096 & 1,056K & 177 \\
				&& 6,400 & 1,651K & 377 \\
				
				\cmidrule{2-5}
				
				& \multirow{2}{*}{Logits-Masked LM Head}
				& 4,096 & 1,056K & 157 \color{red}{ ( $\downarrow$ 11.3\% ) } \\
				&& 6,400 & 1,651K & 335 \color{red}{ ( $\downarrow$ 11.1\% ) } \\
				
				
				
				
				\bottomrule%==============================================================================================================
			\end{tabular}
		\end{adjustbox}
	\end{center}
	\label{table_lm_head}
	% \vspace{-20pt}
\end{table*}


\textbf{Video-Text Data.}
%
We construct our video understanding data using VideoGPT-plus~\cite{VideoGPT-plus}, ShareGemini~\cite{ShareGemini}, and LLaVA-Video-178K~\cite{LLaVA-Video}.
%
To improve the long-context capability of movie-level video understanding, we build a MovieNet-Summary dataset, which consists of paired movies and synopses from MovieNet~\cite{MovieNet}.
%


%
\textbf{Short Text Data.}
Following~\cite{Idefics3}, the pure text data is collected from OpenHermes-2.5~\cite{OpenHermes-2.5}, LIMA~\cite{LIMA}, databricks-dolly-15k~\cite{databricks-dolly-15k}, MetaMathQA~\cite{MetaMathQA}, MathInstruct~\cite{MathInstruct}, Orca-Math~\cite{Orca-Math}, atlas-math-sets~\cite{atlas-math-sets}, goat~\cite{goat}, and camel-ai-math~\cite{CAMEL}.
%


%
\textbf{Long Text Data.}
To transfer the context length of the language model to the modality-aligned multi-modal models~\cite{LongVA}, we gather several long text datasets, including Long-Instruction-with-Paraphrasing~\cite{Long-Instruction-with-Paraphrasing}, LongForm~\cite{LongForm}, LongAlign-10k~\cite{LongAlign}, LongCite-45k~\cite{LongCite}, LongWriter-6k~\cite{LongWriter}, LongQLoRA~\cite{LongQLoRA}, LongAlpaca~\cite{LongLoRA}, and Long-Data-Collections~\cite{Long-Data-Collections}.
%


%
Note that Comic-9k and MovieNet-Summary are created by this work and are made publicly available.
%
Therefore, \OurMethod is \textbf{only} trained on open data, and we \textbf{do not} use data filtering methods.
%






\subsection{Training Pipelines}
\label{sec:Training}


%
Unlike other models, \OurMethod training is divided into four stages with varying sequence lengths.
%


% \begin{itemize}
    % \item
    \textbf{Stage 1: Vision-Language Alignment.}
    %
    Building upon pre-trained language models, our primary objective is to establish initial connections between visual features and language features.
    %
    We freeze the LLM and the visual encoder, only training the visual projector.
    %
    Therefore, we mainly use caption data for pre-training.
    %
    We also add Docmatix~\cite{Idefics3} in this stage to improve document-based VQA.
    %
 

    % \item
    \textbf{Stage 2: General Knowledge Learning.}
    %
    After establishing the vision-language alignment in the embedding space, we dedicate most of our computational resources to vision-language general knowledge learning.
    %
    This stage leverages all the image-text data for multiple tasks, including image captioning, common VQA, OCR, and multi-model conversations.
    %
    In this stage, we also add text-only general instructions, math problems, and arithmetic calculations.
    %
    For video understanding, we only add VideoGPT-plus~\cite{VideoGPT-plus} and ShareGemini-cap~\cite{ShareGemini}.
    %
    In both Stage 1 and Stage 2, we pack all training data to a fixed sequence length, which effectively trains samples with different lengths of sequences.
    %
    Specifically, we random sample data items from the same source and concatenate them into one training sample with a token length of $32$K and $16$K for Stage 1 and Stage 2, respectively.
    %
    We reset positional embeddings and attention masks for all packed samples so that each text-vision pair only attends to itself.
    %
    This approach helps manage extensive datasets and ensure diverse data segments' coverage.
    %

    % \item
    \textbf{Stage 3: Long-Sequence Fine-Tuning.}
    %
    In this stage, we extend the context length to $128$K.
    %
    We reduce the sampling ratio of the data in Stage 2 to $0.1$ and incorporate additional long-context text instructions, comic book summaries, and video understanding datasets.
    %

    % \item
    \textbf{Stage 4: Long-Sequence Fine-Tuning.}
    %
    In this stage, we extend the context length to $1,024$K and add additional movie summary data.
    %
    In both Stage 3 and Stage 4, we also pack all training data to a fixed sequence length without resetting positional embedding and attention mask.
    %
    Therefore, we impose the model to capture the correlations between these two modalities in long-contextual information.
    %
    
% \end{itemize}

%
We \textbf{do not} use the interpolation technique during training and testing, therefore, the context window of \OurMethod can be extended further when equipped with YaRN~\cite{YaRN}, LongRoPE~\cite{LongRoPE} and NTK-based interpolation.
%
Note that we \textbf{do not} use any parameter-efficient methods such as LoRA~\cite{LoRA} or approximate attention~\cite{LongLoRA}.
%










\begin{table*}[!htbp]
 \caption{
 Comparison with the state-of-the-art models under \textbf{20B} parameters on OpenCompass Leaderboard.
 %
 MMB: the test split of MMBench~\cite{MMBench}, MV: MathVista~\cite{MathVista}, HB: HallusionBench~\cite{HallusionBench}, OCR: OCRBench~\cite{OCRBench}.
 %
 `AVG-6' denotes the average scores of six \textbf{objective} benchmarks, \ie, MMBench, MMStar, MMMU, HallusionBench, AI2D, and OCRBench, which do not use a judge LLM to evaluate.
 `AVG' denotes the average of scores on all eight benchmarks.
 `Internal Data' denotes whether the model is trained with in-house data, which is not publicly available. 
 Results are obtained from the leaderboard of OpenCompass.
 }
 % \vspace{-15pt}
 % \footnotesize
 \begin{center}
 \begin{adjustbox}{max width=0.99\textwidth}
 \begin{tabular}{lc|ccccc}
 \toprule
 
 
 Name
 & Internal Data
 & MMB & MMStar & MMMU & MV & HB \\
 
 \midrule
 \multicolumn{7}{c}{Open weight models \& Partially open models} \\

 Qwen2-VL-7B~\hfilll~\cite{Qwen2-VL} & \checkmarknew & 81.0 & 60.7 & 53.7 & 61.4 & 50.4 \\

 InternVL2-8B~\hfilll~\cite{InternVL2} & \checkmarknew & 79.5 & 61.5 & 51.2 & 58.3 & 45.0 \\
 InternVL2.5-8B~\hfilll~\cite{InternVL2.5} 	 & \checkmarknew & \textbf{83.2} & 62.8 & 56.0 & 64.4 & 50.1 \\

 LLaVA-OneVision-7B~\hfilll~\cite{LLaVA-OneVision}	& \checkmarknew & 80.9 & 61.9 & 47.9 & 62.3 & 31.6 \\

 POINTS1.5-7B~\hfilll~\cite{POINTS1.5} & \checkmarknew & 80.7 & 61.1 & 53.8 & 66.4 & 50.0 \\

 Ovis1.5-Gemma2-9B~\hfilll~\cite{Ovis} & \crossmarknew & 77.3 & 58.1 & 49.7 & 65.6 & 48.2 \\
 Ovis1.6-Gemma2-9B~\hfilll~\cite{Ovis} & \checkmarknew & 80.5 & \textbf{62.9} & 55.0 & \textbf{67.2} & 52.2 \\

 Llama-3.2-11B-Vision-Instruct~\hfilll~\cite{Llama-3.2} & \checkmarknew & 65.8 & 49.8 & 48.0 & 48.6 & 40.3 \\

 Pixtral-12B~\hfilll~\cite{Pixtral-12B}	 & \checkmarknew & 72.7 & 54.5 & 51.1 & 56.9 & 47.0 \\

 OmChat-v2.0-13B~\hfilll~\cite{OmChat} & \checkmarknew & 79.5 & 58.2 & 49.6 & 57.1 & 48.4 \\
 
 bailingMM-mini-17B~\hfilll~\cite{bailingmm} & \checkmarknew & 82.2 & 61.3 & 50.0 & 70.5 & 45.4 \\
 
 CogVLM2-19B-Chat~\hfilll~\cite{CogVLM2}		& \checkmarknew & 70.7 & 50.5 & 42.6 & 38.6 & 41.3 \\


 VITA-1.5-7B~\hfilll~\cite{VITA-1.5}          & \checkmarknew & 76.8 & 60.2 & 52.6 & 66.2 & 44.6 \\	
 
 \midrule
 \multicolumn{7}{c}{Fully open models} \\

 VILA1.5-13B~\hfilll~\cite{VILA} & \crossmarknew & 68.5 & 44.2 & 41.1 & 42.5 & 39.3 \\

% \rowcolor{babyblueeyes}
 \OurMethodBF\textbf{-16K} & \crossmarknew & 79.8 & 61.3 & \textbf{57.0} & 65.3 & \textbf{64.6 }\\

% \rowcolor{babyblueeyes}
 \OurMethodBF\textbf{-128K} & \crossmarknew & 79.5 & 60.5 & 56.7 & 65.5 & \textbf{64.6} \\
 
% \rowcolor{babyblueeyes}
 \OurMethodBF\textbf{-1M }& \crossmarknew & 75.0 & 53.0 & 51.0 & 50.3 & 58.7 \\
 



\bottomrule


\end{tabular}
\end{adjustbox}
\bigskip

\begin{adjustbox}{max width=0.99\textwidth}
\begin{tabular}{lc|ccc|cc}
 \toprule


 Name
 & Internal Data
 & AI2D & OCR & MMVet & {AVG} & {AVG-6} \\
 
 \midrule
 \multicolumn{7}{c}{Open weight models \& Partially open models} \\

 Qwen2-VL-7B~\hfilll~\cite{Qwen2-VL} & \checkmarknew & 83.0 & \textbf{843} & 61.8 & 67.0 & 68.9 \\

 InternVL2-8B~\hfilll~\cite{InternVL2} & \checkmarknew & 83.6 & 794 & 54.3 & 64.1 & 66.7 \\
 InternVL2.5-8B~\hfilll~\cite{InternVL2.5} 	 & \checkmarknew & \textbf{84.5} & 822 & 62.8 & \textbf{68.2} & \textbf{69.8} \\

 LLaVA-OneVision-7B~\hfilll~\cite{LLaVA-OneVision}	& \checkmarknew & 82.4 & 622 & 51.9 & 60.1 & 61.1 \\

 POINTS1.5-7B~\hfilll~\cite{POINTS1.5} & \checkmarknew & 81.4 & 823 & 62.2 & 67.2 & 68.2 \\

 Ovis1.5-Gemma2-9B~\hfilll~\cite{Ovis} & \crossmarknew & \textbf{84.5} & 752 & 53.8 & 64.1 & 65.5 \\
 Ovis1.6-Gemma2-9B~\hfilll~\cite{Ovis} & \checkmarknew & 84.4 & 830	& \textbf{65.0} & \textbf{68.7} & \textbf{69.7} \\

 Llama-3.2-11B-Vision-Instruct~\hfilll~\cite{Llama-3.2} & \checkmarknew & 77.3 & 753 & 57.6 & 57.8 & 59.4 \\

 Pixtral-12B~\hfilll~\cite{Pixtral-12B}	 & \checkmarknew & 79.0 & 685 & 58.5 & 61.0 & 62.1 \\

 OmChat-v2.0-13B~\hfilll~\cite{OmChat} & \checkmarknew & 77.8 & 728 & 52.6 & 62.0 & 64.4 \\
 
 bailingMM-mini-17B~\hfilll~\cite{bailingmm} & \checkmarknew & 83.5 & 835 & 59.2 & 67.0 & 67.7 \\
 
 CogVLM2-19B-Chat~\hfilll~\cite{CogVLM2}		& \checkmarknew & 73.4 & 757 & 57.8 & 56.3 & 59.0 \\

 VITA-1.5-7B~\hfilll~\cite{VITA-1.5}          & \checkmarknew & 79.2 & 741 & 52.7 & 63.3 & 64.5 \\
 
 \midrule
 \multicolumn{7}{c}{Fully open models} \\

 VILA1.5-13B~\hfilll~\cite{VILA} & \crossmarknew & 69.9 & 460 & 45.0 & 49.6 & 51.5 \\

% \rowcolor{babyblueeyes}
 \OurMethodBF\textbf{-16K} & \crossmarknew & 81.5 & 755 & 53.9 & \textbf{67.4} & \textbf{69.9} \\

% \rowcolor{babyblueeyes}
 \OurMethodBF\textbf{-128K} & \crossmarknew & 81.1 & 738 & 53.8 & 66.9 & 69.4 \\
 
% \rowcolor{babyblueeyes}
 \OurMethodBF\textbf{-1M }& \crossmarknew & 78.0 & 702 & 57.4 & 61.7 & 64.3 \\
 



 
 \bottomrule%==============================================================================================================
 \end{tabular}
 \end{adjustbox}
 \end{center}
 \label{table_open_compass}
 % \vspace{-20pt}
\end{table*}










\begin{table*}[!htbp]
 \caption{
 Comparison with the state-of-the-art models under \textbf{20B} parameters on Video-MME (w/o subs).
 `Internal Data' denotes whether the model is trained with in-house data, which is not publicly available.
 Most results are obtained from the leaderboard of OpenCompass.
 }
 % \vspace{-15pt}
 % \footnotesize
 \begin{center}
  \begin{adjustbox}{max width=0.99\textwidth}
   \begin{tabular}{l|c|c|cccc}
    \toprule
    
    
    Name
    & Internal  Data & Frames
    & Overall & Short & Medium & Long \\

    
    \midrule
    \multicolumn{7}{c}{Open weight models \& Partially open models} \\
    
    ARIA~\hfilll~\cite{ARIA}                        & \checkmarknew & 64 & 66.0 & \textbf{77.1} & 64.9 & 56.0 \\

    mPLUG-Owl3~\hfilll~\cite{mPLUG-Owl3}            & \checkmarknew & 16 & 54.0 & 63.3 & 51.8 & 46.8 \\

    PLLaVA-34B~\hfilll~\cite{PLLaVA}                & \checkmarknew & 16 & 53.4 & 62.0 & 52.9 & 45.4 \\
    
    InternVL2-8B~\hfilll~\cite{InternVL2}           & \checkmarknew & 16 & 53.7 & 65.9 & 49.8 & 45.3 \\
    InternVL2.5-8B~\hfilll~\cite{InternVL2.5}       & \checkmarknew & 16 & 64.2 & -- & -- & -- \\

    Qwen2-VL-7B~\hfilll~\cite{Qwen2-VL}             & \checkmarknew & 64 & 59.7 & 71.2 & 57.8 & 50.0 \\

    MiniCPM-V-2.6-7B~\hfilll~\cite{MiniCPM-V}       & \checkmarknew & 64 & 59.7 & 70.4 & 58.1 & 50.4 \\

    Idefics3-8B-Llama3~\hfilll~\cite{Idefics3}      & \checkmarknew & 16 & 54.0 & 56.1 & 45.1 & 43.0 \\

    NVILA-8B~\hfilll~\cite{NVILA}                   & \checkmarknew & 256 & 64.2 & 75.7 & 62.2 & 54.8 \\

    
    \midrule
    \multicolumn{7}{c}{Fully open models} \\

    LLaVA-Video-7B-Qwen2~\hfilll~\cite{LLaVA-Video} & \crossmarknew & 64 & 63.7 & 76.7 & 62.2 & 52.2 \\

    LongVILA-7B~\hfilll~\cite{LongVILA}             & \crossmarknew & 256 & 60.1 & 69.0 & 58.3 & 53.0 \\

    \midrule
    % \rowcolor{babyblueeyes}
    \multirow{2}{*}{\OurMethodBF\textbf{-16K}}         & \crossmarknew & 64 & 62.8 & 74.7 & 59.1 & 54.7 \\
    % \rowcolor{babyblueeyes}
                                            & \crossmarknew & 128 & 64.5 & 74.3 & 63.2 & 56.0 \\


    \midrule
    % \rowcolor{babyblueeyes}
    \multirow{5}{*}{\OurMethodBF\textbf{-128K}}     & \crossmarknew & 64 & 65.6 & 75.0 & 65.7 & 56.0 \\
    % \rowcolor{babyblueeyes}
                                                    & \crossmarknew & 128 & 66.2 & 74.8 & \textbf{66.7} & 57.2 \\
    % \rowcolor{babyblueeyes}
                                                    & \crossmarknew & 256 & \textbf{66.4} & 74.7 & 65.9 & \textbf{58.8} \\
    % \rowcolor{babyblueeyes}
                                                    & \crossmarknew & 512 & 65.7 & 74.7 & 64.6 & 58.0 \\
    % \rowcolor{babyblueeyes}
                                                    % & \crossmarknew & 1024 & 41.8 & 74.9 & 50.4 & 00.1 \\

    \midrule
    % \rowcolor{babyblueeyes}
    \multirow{7}{*}{\OurMethodBF\textbf{-1M}}       & \crossmarknew & 64 & 59.6 & 69.2 & 57.4 & 52.0 \\
    % \rowcolor{babyblueeyes}
                                                    & \crossmarknew & 128 & 60.0 & 68.2 & 59.1 & 52.6 \\
    % \rowcolor{babyblueeyes}
                                                    & \crossmarknew & 256 & 60.7 & 68.6 & 59.7 & 53.8 \\
	% \rowcolor{babyblueeyes}
                                                    & \crossmarknew & 512 & 59.0 & 68.1 & 57.1 & 51.7 \\
    % \rowcolor{babyblueeyes}
                                                    & \crossmarknew &1024 & 57.9 & 68.4 & 57.3 & 48.0 \\
    % \rowcolor{babyblueeyes}
                                                    & \crossmarknew & 2048 & 56.0 & 68.2 & 57.0 & 42.7 \\
    % \rowcolor{babyblueeyes}
                                                    & \crossmarknew & 4096 & 55.8 & 68.2 & 56.7 & 42.6 \\

    \bottomrule%==============================================================================================================
   \end{tabular}
  \end{adjustbox}
 \end{center}
 \label{table_video_mme}
 % \vspace{-20pt}
\end{table*}




\begin{table*}[!htbp]
 \caption{
 Comparison with the state-of-the-art models under \textbf{20B} parameters on video benchmark.
 `Internal Data' denotes whether the model is trained with in-house data, which is not publicly available. 
 }
 % \vspace{-15pt}
 % \footnotesize
 \begin{center}
  \begin{adjustbox}{max width=0.99\textwidth}
   \begin{tabular}{lc|cccccccc|cc}
    \toprule
    
    
    Name
    & Internal  Data & Frames
    & LongVideoBench & MVBench \\

    
    \midrule

    mPLUG-Owl3-7B~\hfilll~\cite{mPLUG-Owl3}          & \crossmarknew & 128 & 52.1 & 54.5 \\
    
    LLaVA-Video-7B-Qwen2~\hfilll~\cite{LLaVA-Video} & \crossmarknew & 64 & 58.2 & 62.1 \\

    InternVL2-8B~\hfilll~\cite{InternVL2}           & \checkmarknew & 16 & 54.6 & 56.5 \\
    InternVL2.5-8B~\hfilll~\cite{InternVL2.5}  	    & \checkmarknew & 16 & 60.0 & 64.5 \\

    Qwen2-VL-7B~\hfilll~\cite{Qwen2-VL}             & \checkmarknew & 64 & 55.6 & 52.0 \\

    MiniCPM-V-2.6~\hfilll~\cite{MiniCPM-V} 	        & \checkmarknew & 64 & 54.9 & 44.7 \\

    Idefics3-8B-Llama3~\hfilll~\cite{Idefics3}	    & \checkmarknew & 16 & -- & 46.1 \\

    NVILA-8B~\hfilll~\cite{NVILA}	                & \checkmarknew & 256 & 57.7 & -- \\

    LongVILA-7B~\hfilll~\cite{LongVILA}	            & \checkmarknew & 256 & 57.1 & \textbf{67.1} \\
    
    \midrule
    % \rowcolor{babyblueeyes}
    \multirow{2}{*}{\OurMethodBF\textbf{-16K}}      & \crossmarknew & 64 & 59.4 & 56.6 \\
    % \rowcolor{babyblueeyes}
                                                    & \crossmarknew & 128 & 59.8 & 57.4 \\


    \midrule
    % \rowcolor{babyblueeyes}
    \multirow{4}{*}{\OurMethodBF\textbf{-128K}}     & \crossmarknew & 64 &  59.2 &	57.4 \\
    % \rowcolor{babyblueeyes}
                                                    & \crossmarknew & 128 & 60.7 &	57.5 \\
    % \rowcolor{babyblueeyes}
                                                    & \crossmarknew & 256 & \textbf{60.9} & 55.4 \\
    % \rowcolor{babyblueeyes}
                                                    & \crossmarknew & 512 & 59.8 & 57.3 \\
    % \rowcolor{babyblueeyes}
                                                    % & \crossmarknew & 1,024 & 38.1 & 57.2 \\

    \midrule
    % \rowcolor{babyblueeyes}
    \multirow{6}{*}{\OurMethodBF\textbf{-1M}}       & \crossmarknew & 64 & 53.9 & 44.7 \\
    % \rowcolor{babyblueeyes}
                                                    & \crossmarknew & 128 & 55.2 & 44.8 \\
    % \rowcolor{babyblueeyes}
                                                    & \crossmarknew & 256 & 54.0 & 44.7 \\
	% \rowcolor{babyblueeyes}
                                                    & \crossmarknew & 512 & 53.1 & 44.7 \\
    % \rowcolor{babyblueeyes}
                                                    & \crossmarknew & 1,024 & 51.8 & 44.7 \\
    % \rowcolor{babyblueeyes}
                                                    & \crossmarknew & 2,048 & 51.8 & 44.5 \\
    % \rowcolor{babyblueeyes}
                                                    % & \crossmarknew & 4,096 & \\


    \bottomrule%==============================================================================================================
   \end{tabular}
  \end{adjustbox}
 \end{center}
 \label{table_open_compass_video}
 % \vspace{-20pt}
\end{table*}





\subsection{Hyper-parameters and Infrastructures}

%
We initially implement training and inference with MindSpeed~\cite{MindSpeed} and MindSpeed-LLM~\cite{MindSpeed-LLM}, which adapt Megatron-LM~\cite{Megatron-LM} to Ascend NPU.
%
We also transfer the training and inference code to the GPU platform.
%
As shown in Tab.~\ref{table_train}, we list the detailed hyper-parameters in \OurMethod.
%



\textbf{Training.}
%
We configure different distributed training strategies for each key module in \OurMethod.
%

\begin{itemize}[leftmargin=2.0em]
    \item
    \textbf{Large Language Model.}
    %
    We employ data, pipeline, tensor, sequence, and context parallelism.
    %
    We enable distribution attention for context parallelism to train long sequences in Stages 3 and 4.
    %
    
    \item
    \textbf{Vision Encoder.}
    %
    We apply data, tensor, and sequence parallelism to the vision module, which is in the first LLM pipeline parallelism stage.
    %
    We do not use context parallelism for the vision encoder.
    %
    
    \item
    \textbf{Vision-Language Projector.}
    %
    The multi-modal projector follows the configuration of the vision encoder during the distributed training.
    %
    
    %
\end{itemize}





\textbf{Inference.}
%
We implement two new designs to scale up the number of tokens for model inference.
%


\begin{itemize}[leftmargin=2.0em]
    \item
    \textbf{Context-Parallelism Distributed Inference.}
    %
    We implement tensor parallelism with context parallelism for model inference, thus supporting distribution attention for infinite-length input tokens.
    %
    Similar to the training mode, the length of inference tokens is fixed during the decoding phase.
    %
    Specifically, we concatenate the input tokens with padding tokens of the maximum output length.
    %
    The system needs to extract the new predicted next token in the fixed-length output tokens and accordingly terminate the generation process during each forward.
    %
    \item
    \textbf{Logits-Masked Language Modeling Head.}
    %
    We observe that the output logits from the language modeling head induce excessive memory footprints.
    %
    For example, given $1$M tokens with $10^5$ vocabularies, the output logit matrix has a shape of $10^6 \times 10^5$ and requires $400$ GB of memory for the float32 data type.
    %
    To address the memory issue, we mask out all hidden features and only feed the one hidden feature that predicts the next tokens to the language modeling head.
    %
    With the above design, the memory consumption of the output logit matrix is $0.0004$ GB with $10^6 \times$ reduction.
    %
    Note that this design can also apply to model training with long-context inputs and the language modeling head only needs to predict the short-context outputs to reduce memory consumption.
    %
\end{itemize}
%

%
We test the maximal sequence length of a fixed number of devices before they raise the out-of-memory error.
%
Tab.~\ref{table_max_length} summarizes the result.
%
Note that activation checkpointing is disabled in LongVILA~\cite{LongVILA}, while our model has much more number of parameters.
%



%
Tab.~\ref{table_lm_head} further shows the effectiveness of logits-masked language modeling head~(logits-masked LM head).
%
All methods are implement with Flash Attention and context-parallelism distributed inference on GPU with $96$G memory and about $150$ TFLOPS for bfloat16.
%
Compared to the original LM head, logits-masked LM head extends the max sequence length by $417\%$, and reduces time cost by $47.3\%$.
%
We also implement a chunked language modeling head~(chunked LM head), which process tokens with a chunk length of $32,768$.
%
Compared to the chunked LM head, logits-masked LM head achieves $11.3$ and $11.1$ speedup under the $1$M  and $1.6$M input lengths, respectively.
%



%
We employ Ring Attention~\cite{RingAttention} to distribute long sequences across multiple devices.
%
We do not use parameter-efficient fine-tuning methods or quantization strategies for both training and inference.
%
Additionally, the temperature is set to $0$ to guarantee consistent performance evaluation.
%
% Unless specified otherwise, ''\OurMethod`` in the subsequent text refers to \OurMethod-16K.
%

