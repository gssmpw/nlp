\section{Notation and Preliminaries.}
\label{sec:notation}
We first introduce necessary concepts from Riemannian geometry; for more detailed treatment and further reading, see \cite{boothby2003introduction,carmo1992riemannian,lee2013smooth,sakai1996riemannian}. 

Throughout this work, we let $\manifold$ denote a $\dimInd$-dimensional symmetric \emph{Riemannian manifold}. The \emph{tangent space} at a point $\mPoint \in \manifold$ is denoted by $\tangent_\mPoint \manifold$, and it is the space of all derivations at $\mPoint$. The \emph{inner product} or \emph{metric tensor} at a point $\mPoint$ is denoted by $(\cdot, \cdot)_\mPoint: \tangent_\mPoint \manifold \times \tangent_\mPoint \manifold \mapsto \Real$ and defines a Riemannian manifold $(\manifold, (\cdot, \cdot))$, and we denote the associated \emph{Riemannian metric} by $\distance_\manifold: \manifold \times \manifold \to \Real_{\geq 0}$. We denote the \emph{geodesic}, or curve of minimal length, between $\mPoint, \mPointB \in \manifold$ by $\geodesic_{\mPoint, \mPointB}:[0,1] \to \manifold$. For a given point $\mPoint \in \manifold$ and tangent vector $\Xi_\mPoint \in \tangent_\mPoint \manifold$, we define the curve $t \to \geodesic_{\mPoint, \Xi_\mPoint} (t)$ to be the geodesic starting at $\mPoint$ with velocity $\dot{\geodesic}_{\mPoint, \Xi_\mPoint} = \Xi_\mPoint$. These geodesics are used to define the \emph{exponential map} $\exp_\mPoint : \mathcal{G}_\mPoint \manifold \to \manifold$ where
\begin{equation}
    \exp_\mPoint (\Xi_\mPoint) \coloneqq \geodesic_{\mPoint, \Xi_\mPoint}(1)
\end{equation}
and $\mathcal{G}_\mPoint \subset \tangent_\mPoint \manifold$ is
\begin{equation}
    \mathcal{G}_\mPoint \coloneqq \{\Xi_\mPoint: \geodesic_{\mPoint, \Xi_\mPoint}(1) \text{ is defined}\}.
\end{equation}
We will consider \emph{geodesically complete} manifolds throughout, which satisfy $\mathcal{G}_\mPoint = \tangent_\mPoint \manifold$. The inverse of $\exp_\mPoint$ is the \emph{logarithmic map} $\log_\mPoint : \exp_\mPoint (\mathcal{G}_\mPoint') \to \mathcal{G}_\mPoint$, where $\mathcal{G}_\mPoint'\subset \mathcal{G}_\mPoint$ is the set on which $\exp_\mPoint$ is a diffeomorphism. Since $\log_\mPoint$ maps from the manifold $\manifold$, which has nonlinear structure in general, to the tangent space $\tangent_\mPoint \manifold$, which is a vector space, application of the logarithmic map can be considered a linearization. Finally, throughout this work we consider a dataset $\Tensor=\{\Tensor^\sumIndA\}_{\sumIndA=1}^\numData$ where each $\Tensor^\sumIndA \in \manifold$. 

