\section{Related Work}

\subsection{Data Videos for Storytelling}

%% Data Storytelling through Video
Storytelling plays a vital role in communicating data, providing a structured and engaging means of sharing insights~\cite{}. 
Data videos combine visual elements, auditory, and raw footages to create compelling stories backed by data. 
Researchers have extensively explored the design space of data videos, 
offering valuable insights for creating and authoring them.

For example, Amini et al.~\cite{} analyzed 50 data videos from diverse sources and used Cohn's visual narrative structure theory~\cite{} to categorize these videos into four narrative structures.
Their findings unveiled design patterns and offered implications for the development of data video authoring tools.
Similarly, Thompson et al.~\cite{} developed a design space for animated data graphics to support the development of future authoring tools. 
Cao et al.~\cite{} analyzed 70 data videos and proposed a taxonomy for characterizing narrative constructs, broadening our understanding of data storytelling.
More recently, researchers have explored incorporating cinematic techniques into data video design.
Xu et al.~\cite{} contributed a series of works, including guidelines for creating cinematic openings~\cite{}, endings~\cite{}, and overarching structures~\cite{} in data videos. 
Their studies, inspired by cinematic storytelling, analyze film and data video examples to provide systematic guidance for practitioners.

While these works are relevant, sports video storytelling introduces unique challenges compared to traditional data video narratives.
Sports video storytelling focus on enhancing real-world footage with additional visual elements, such as overlays, annotations, and real-time data visualizations.
This setting presents a distinct design space with unique requirements and techniques.
Zhu-Tian et al.~\cite{} explored this design space, identifying six narrative orders and common usage scenarios to provide practical guidance for designing authoring tools tailored to this domain.
Building on prior work, Sportsbuddy was developed as an augmented sports video authoring tool, inspired by insights from both traditional data video storytelling research and the unique requirements of sports storytelling.

% Amini et al. [2] systematically analyzed 50 data videos from reputable sources
% and summarized the most common visual elements and attention cues in
% these videos. They further deconstructed each video into four narrative
% categories by using the visual narrative structure theory introduced
% by Cohn [11]. Their findings reveal several design patterns in data
% videos and provide implications for the design of data video authoring
% tools. Thompson et al. [45] contributed a
% design space of data videos for developing future animated data graphic
% authoring tools. Cao et al. [6] analyzed 70 data videos and proposed a
% taxonomy to characterize narrative constructs in data videos.
% more recently, researchers started to explore the way to design data videos using techniques from cinematirc. Typically, Xu et al.~\cite{}
% Is It the End? Guidelines for Cinematic Endings in Data Videos
% From 'Wow' to 'Why': Guidelines for Creating the Opening of a Data Video with Cinematic Styles
% Telling Data Stories with the Hero’s Journey: Design Guidance for Creating Data Videos

% They are relevant but different from sports video storytelling,
% in which it is more about augmenting the video footage with extrac visual elemnents,
% instead of composing using visual elements.
% This settings presents a different design space and requirements than the original data videos.
% Zhu-Tian et al.~\cite{} explore its desing space by analyzing 233 real-world examples and summarize six narrative orders and their common usage scenarios. The results provide guidance
% for designing authoring tools for augmented sports videos.
% Sportsbuddy is designed and developed inspired all these previous work.

% Although these studies provide insights into data video design, our
% scenario is inherently different from theirs and thus yields new challenges. Specifically, these studies use video as a medium to tell the
% story of data, whereas we focus on augmenting existing videos with
% data. An existing video imposes extra constraints on narrative orders,
% visual data forms, etc. With these constraints, how to visually narrate
% data in videos remains underexplored. Perhaps the most relevant work
% is from Stein et al. [40], who annotated the data in soccer videos in a
% linear way. To fully explore the ways for augmenting sports videos with
% data, we 


% Data video, as one of the seven genres of narrative visualization categorized by Segel and Heer [36], is an
% active research topic and has attracted interest from researchers. 
% Build upon that, Amini et al. [3] further contributed DataClips,
% an authoring tool that allows general users to craft data videos with
% predefined templates. 

% VisCommentator [68] proposed a framework for embedding animated visualizations
% in sports videos, supporting the creation of data videos with narrative
% structures like linear and flashback.




% Data storytelling is essential in democratizing data access and analysis. Animation/visualizations are essential in data storytelling, especially in spatiotemporal data.  


\subsection{Video-based Visualizations for Sports}
Given the inherently visual and media-rich nature of sports, 
video-based visualizations have become a crucial format for sports data, 
as highlighted by Perin et al.~\cite{}. 
These visualizations are widely utilized in both professional settings, 
such as coaches' analyses\cite{}, and for engaging general audiences~\cite{}.
Sportsbuddy specifically focuses on embedded visualizations in sports videos~\cite{}, where data visualizations are seamlessly integrated into the video as part of the scene.

In academia, a representative example is the system developed by Stein et al.~\cite{}. 
Their system processes raw footage of soccer games to automatically visualize tactical information as graphical marks within the video. 
Following,
Zhu-Tian et al. developed a series of human-AI collaborative tools that support data visualization in sports videos, ranging from racket-based sports\cite{} to team-based sports~\cite{}. 
Beyond system development, Zhu-Tian et al.~\cite{} and Lin et al.~\cite{} have also explored the design space of embedded visualizations in sports videos.
In industry, the strong market demand has led to the emergence of highly successful commercial systems, as pointed out by Fischer et al.~\cite{}. 
For instance, Piero~\cite{} and Viz Libero~\cite{} are widely used in sportscasting, offering powerful functionalities for editing and annotating sports videos. 
Additionally, CourtVision~\cite{}, developed by Second Spectrum~\cite{}, automatically tracks players' positions and embeds real-time status information to enhance audience engagement.
Yet, these commercial systems often cater to proficient video editors, resulting in a steep learning curve for sports analysts. 

Sportsbuddy seeks to bridge the gap between academic research and industry by developing an easy-to-use tool targeted at real-world users. 
By simplifying workflows and making advanced video-based sports visualizations accessible, 
Sportsbuddy aims to transition academic insights into practical solutions for broader adoption.

% % Video analysis and storytelling is a essential component of sports. Analysts spend a lot of time creating videos for coaching and entertainment purposes.
% Due to the visual, media nature of sports, 
% video-based visualizations is an important format 
% for sports data due to Perin et al.~\cite{}'s survey,
% and  has been always engaging and widely used in both professional settings like coaches' analysis~\cite{} or engage the general audiences~\cite{}.
% Sportsbuddy particularly focuse on embddded visualizations in sports videos~\cite{},
% where the data visualizations are inserted as like a part of the scene.
% % from VisCommentator Due to its advantages of presenting data in actual scenes, video-based sports visualization has been used
% % widely to ease experts’ analysis [41] and engage the audiences [26].
% % Based on the presentation method, video-based visualizations can be
% % divided into side-by-side [62], overlaid [42], and embedded [41]. This
% % work focuses on embedded visualizations in sports videos.
% % Perin et al. [33] comprehensively surveyed the visualization research
% % on sports data, indicating that only a few works can be considered as
% % video-based visualizations. 

% % Among these few works, 
% In academia,
% a representative example was developed by Stein et al. [41] for soccer. 
% Their system takes raw footage as the input and automatically visualizes certain
% tactical information as graphical marks in the video. Later, Stein et
% al. [40] extended their work by proposing a conceptual framework that
% semi-automatically selects proper information to be presented at a given
% moment. 
% Following that
% Zhu-Tian et al. has devleoped a serils works in helping users visualize data into sports videos, ranging from racket-based sports~\cite{} to team based sports~\cite{}.
% In addition to systems, Zhu-Tian et al.~\cite{} and Lin et al.~\cite{} also explored the design space of embedded visualizations in sports videos.
% On the other hand, in industry, 
% the strong market demand has already spawned very successful
% commercial systems as detailed by Fischer et al~\cite{}. 
% For example, Piero [4] and Viz Libero [46] are both
% developed for sportscasting and provide a set of powerful functions to
% edit and annotate a sports video. Besides, CourtVision [10], developed
% by Second Spectrum [39] for basketball, is another industry product that
% automatically tracks players’ positions and embeds status information
% to engage the audiences. 


% Nevertheless, these commercial systems target
% proficient video editors, leading to a steep learning curve for sports
% analysts. 
% Sportsbuddy aims to spinch from the academic reserach and developing an easy-to-use tool that targeteed at real-world users, making an attempts from approaching from academia to industry real users.


% Recently, are actually ahead of most of the
% academic research. 
% [16] found that video-based soccer
% visualization systems from the industry 
% In summary, 
% As detailed by Fischer et al. [16],
% Moreover, they mainly augment the video with graphical
% elements, while the goal of sports analysts is to augment sports videos
% with data. Given that very little is known about the design practices of
% augmented sports videos, it remains unclear how to support augmenting
% sports videos with data. In this work, we explore this direction and aim
% to ease the creation by a systematic study of existing augmented sports
% videos collected from reputable sources


\subsection{Authoring Data Videos}
While data videos are a powerful medium for storytelling, creating them requires significant expertise in data analysis and video crafting. General-purpose video editing tools like Adobe After Effects~\cite{} allow users to manipulate visual elements through animation keyframing and presets, but they often involve extensive manual effort and a steep learning curve. To lower the entry barrier, many researchers have developed systems that simplify the creation process.

Early systems like DataClips~\cite{DBLP:journals/tvcg/AminiRLMI17} introduced template-based approaches, enabling semi-automatic data animation by reusing patterns from existing examples. Building on this foundation, tools such as WonderFlow~\cite{DBLP:journals/corr/abs-2308-04040,} and InfoMotion~\cite{DBLP:journals/cgf/WangGHCZZ21} began automating more complex aspects of storytelling, including synchronizing animations with audio narrations and applying animations to static infographics by analyzing their structures. Systems like Gemini~\cite{DBLP:journals/tvcg/KimH21} and its extension Gemini2~\cite{DBLP:conf/visualization/KimH21} focused on refining the transition process by recommending staged animations, offering designers guidance for creating smooth and semantic transitions. 
Meanwhile, tools like AutoClips~\cite{DBLP:journals/cgf/ShiSXLGC21} and Roslingifier~\cite{DBLP:journals/tvcg/ShinKHXWKKE23} emphasized automation and customization, with the former generating videos from data facts and the latter supporting semi-automated narratives for scatterplots. 
Recent advances, such as Live Charts~\cite{DBLP:journals/corr/abs-2309-02967} and Data Player~\cite{DBLP:journals/tvcg/ShenZZW24}, further integrate intelligent automation, using large language models to link narration and visuals, while Data Playwright~\cite{} pushes the boundary by introducing natural language-driven video synthesis through annotated narration. 
Together, these tools represent a progression from template-based semi-automation to intelligent systems that combine automation, user input, and natural language interaction, making data video authoring increasingly accessible and efficient.

Sportsbuddy differs from these systems in two key aspects: (1) it focuses on authoring augmented sports videos with embedded data visualizations rather than general-purpose data videos; and (2) it emphasizes real-world deployment by collecting real user data to refine its usability and functionality. This approach bridges the gap between academic research and practical application, addressing the unique challenges of sports video storytelling in authentic settings.


% Easy-to-use tool/interfaces are increasingly important in allowing people to capture, annotate, and share their experiences. AI/LLM provides an even more natural way to interact \& author videos.

% Data Playwright: Authoring Data Videos with Annotated Narration

% - Wang, Bryan, et al. "LAVE: LLM-Powered Agent Assistance and Language Augmentation for Video Editing." Proceedings of the 29th International Conference on Intelligent User Interfaces. 2024.
% While data videos are useful, creating them needs significant skills, including data analysis and video crafting.
% General video editing tools~\cite{}, such as Adobe After Effects, 
% require users to control visual elements through animation keyframing and presets,
% but they often require significant manual effort.
%  Thus, many researchers have develop systems to lower the enter barrier.
% For example, DataClips~\cite{} use templates manually summarized from existing examples to enable
% semi-automatic creation of data animation.
% Recently, WonderFlow [66]
% introduced a narration-driven design pipeline, which allows
% users to interactively specify text-visual connections and select a suitable animation preset for the established connections, streamlining the manual effort involved in the creation process.
% To eliminate manual effort, researchers have made significant progress in developing automatic approaches. For instance, InfoMotion[64] empowers the automatic generation
% of animated infographics based on motion graphical properties and structures. 
% Gemini2 [24] enhances Gemini [23] by offering suggestions for keyframe transitions. 
% AutoClips [51] constructs a fact-driven clip library and automatically generates videos from a sequence of data facts. 
% Roslingifier [53] automatically generates visual highlights and playback narratives for animated scatterplots. 
% Live Charts [73] revive static visualizations by decomposing the chart information
% and explaining it with animations and audio narration. 
% Data Player [49] utilizes LLMs to link narration segments and visual
% elements semantically, recommends animations for text-visual
% links using constraint programming, and renders the animation
% sequence with automatically generated audio narration into a
% data video. 
% Data Playwright~\cite{} allow users can effortlessly express
% their narrative and authoring intents with annotated narration.
% Data Playwright then dispatches its automatic interpreter to
% translate users’ inputs into data videos, ensuring a seamless
% transfer of users’ creative vision into the final output


% On top of these findings, various manual authoring and
% programming tools have been developed to facilitate the creation of data videos [9], [60]. 
% These tools employ various authoring paradigms, such as programming languages [17],
% [74], [23], keyframe-based animation generation [61], [16],
% and presets and templates [7], [27]. 
% Additionally, general video creation tools like 


% One widely used approach to ease the visual mapping process is
% templates. For example, 


% Besides automating the visual mapping process, some tools further facilitate
% the data analysis process by automatically suggesting data insights.
% DataShot [50] adopts an auto-insight technique to recommend interesting data insights based on their significance for factsheet generation.
% DataToon [21], an authoring tool for data comic creation, uses a pattern
% detection engine to suggest salient patterns of the input network data.
% However, few, if any, tools exist that provide the aforementioned
% kinds of data-driven support for creating augmented sports videos. 

% The
% challenges of developing such a tool not only exist in the engineering
% implementations but also in the integration between the workflows of
% visualization authoring and video editing. We draw on the line with
% prior visualization design tools and design VisCommentator to support
% visualizing data sports videos in a video editing process.



% Despite the diverse interaction design present in these manual tools, 
% they still require significant manual effort and involve a
% learning curve for tool-specific operations.



% Overall, manual tools tend to be tedious and require expertise, while automatic methods often overlook users’ diverse authoring intents. This paper presents a new data video creation
% tool, Data Playwright, where 