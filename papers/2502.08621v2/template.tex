% $Id: template.tex 11 2007-04-03 22:25:53Z jpeltier $

\documentclass{vgtc}                          % final (conference style)
% \documentclass[review]{vgtc}                 % review
%\documentclass[widereview]{vgtc}             % wide-spaced review
%\documentclass[preprint]{vgtc}               % preprint
%\documentclass[electronic]{vgtc}             % electronic version

%% Uncomment one of the lines above depending on where your paper is
%% in the conference process. ``review'' and ``widereview'' are for review
%% submission, ``preprint'' is for pre-publication, and the final version
%% doesn't use a specific qualifier. Further, ``electronic'' includes
%% hyperreferences for more convenient online viewing.

%% Please use one of the ``review'' options in combination with the
%% assigned online id (see below) ONLY if your paper uses a double blind
%% review process. Some conferences, like IEEE Vis and InfoVis, have NOT
%% in the past.

%% Figures should be in CMYK or Grey scale format, otherwise, colour 
%% shifting may occur during the printing process.

%% These few lines make a distinction between latex and pdflatex calls and they
%% bring in essential packages for graphics and font handling.
%% Note that due to the \DeclareGraphicsExtensions{} call it is no longer necessary
%% to provide the the path and extension of a graphics file:
%% \includegraphics{diamondrule} is completely sufficient.
%%
\ifpdf%                                % if we use pdflatex
  \pdfoutput=1\relax                   % create PDFs from pdfLaTeX
  \pdfcompresslevel=9                  % PDF Compression
  \pdfoptionpdfminorversion=7          % create PDF 1.7
  \ExecuteOptions{pdftex}
  \usepackage{graphicx}                % allow us to embed graphics files
  \DeclareGraphicsExtensions{.pdf,.png,.jpg,.jpeg} % for pdflatex we expect .pdf, .png, or .jpg files
\else%                                 % else we use pure latex
  \ExecuteOptions{dvips}
  \usepackage{graphicx}                % allow us to embed graphics files
  \DeclareGraphicsExtensions{.eps}     % for pure latex we expect eps files
\fi%

%% it is recomended to use ``\autoref{sec:bla}'' instead of ``Fig.~\ref{sec:bla}''
\graphicspath{{figures/}{pictures/}{images/}{./}} % where to search for the images
\usepackage{orcidlink}
\usepackage{microtype}                 % use micro-typography (slightly more compact, better to read)
\PassOptionsToPackage{warn}{textcomp}  % to address font issues with \textrightarrow
\usepackage{xcolor}
\usepackage{textcomp}                  % use better special symbols
\usepackage{mathptmx}                  % use matching math font
\usepackage{times}                     % we use Times as the main font
\renewcommand*\ttdefault{txtt}         % a nicer typewriter font
\usepackage{cite}                      % needed to automatically sort the references
\usepackage{tabu}                      % only used for the table example
\usepackage{booktabs}                  % only used for the table example
%% We encourage the use of mathptmx for consistent usage of times font
%% throughout the proceedings. However, if you encounter conflicts
%% with other math-related packages, you may want to disable it.
\usepackage{enumitem}
\usepackage{changepage}

%% If you are submitting a paper to a conference for review with a double
%% blind reviewing process, please replace the value ``0'' below with your
%% OnlineID. Otherwise, you may safely leave it at ``0''.
\onlineid{0}

%% declare the category of your paper, only shown in review mode
\vgtccategory{Research}

%% allow for this line if you want the electronic option to work properly
\vgtcinsertpkg

%% In preprint mode you may define your own headline.
%\preprinttext{To appear in an IEEE VGTC sponsored conference.}

%% Paper title.
% SportsBuddy: Enhancing Sports Video Storytelling through Multimodal AI and Interactive Authoring 

\title{SportsBuddy: Designing and Evaluating an AI-Powered Sports Video Storytelling Tool Through Real-World Deployment}
%% This is how authors are specified in the conference style

%% Author and Affiliation (single author).
%%\author{Roy G. Biv\thanks{e-mail: roy.g.biv@aol.com}}
%%\affiliation{\scriptsize Allied Widgets Research}


%% Author and Affiliation (multiple authors with multiple affiliations)
\author{
Tica Lin \orcidlink{0000-0002-2860-0871} \thanks{
\textit{
Tica Lin and Ruxun Xiang contributed equally to this work.}
\\ 
\hangindent=4mm \textbullet\ \textit{T. Lin (mlin@g.harvard.edu), R. Xiang (ruxunx@g.harvard.edu), M. Chiang, C. Ye were affiliated with Harvard University during this work. G. Liu and H. Pfister are with Harvard University. }
\\
\hangindent=4mm \textbullet\ \textit{D. Tiwari and Z. Chen (ztchen@umn.edu) are with University of Minnesota. }
\vspace{-1cm}
} ,
Ruxun Xiang \orcidlink{0009-0000-3257-4859} \footnotemark[1] , 
Gardenia Liu \orcidlink{0009-0002-5442-9788}, 
Divyanshu Tiwari \orcidlink{0000-0003-4279-8332}, 
Meng-Chia Chiang, 
Chenjiayi Ye \orcidlink{0009-0006-0030-9296}, \\
Hanspeter Pfister \orcidlink{0000-0002-3620-2582}, Chen Zhu-Tian \orcidlink{0000-0002-2313-0612} 
}




%% A teaser figure can be included as follows, but is not recommended since
%% the space is now taken up by a full width abstract.
\teaser{
\vspace{-2mm}
\begin{center}
  \includegraphics[width=\linewidth, alt={}]{pictures/teaser.png}
  \caption{%
  	\SB{} enables easy highlight creation for video storytelling with intuitive visualization features and workflow. Users can (a) manage clips in \Media{} and choose from \Highlight{} and \Narrative{} features; (b) interact directly with the video canvas to apply features, adjust effects, and edit using the video timeline; and (c) share the final highlight via link or file. 
  }
  \label{fig:teaser}
  \end{center}
}

%% Abstract section.
\abstract{
Video storytelling is essential for sports performance analysis and fan engagement, enabling sports professionals and fans to effectively communicate and interpret the spatial and temporal dynamics of gameplay. Traditional methods rely on manual annotation and verbal explanations, placing significant demands on creators for video editing skills and on viewers for cognitive focus. However, these approaches are time-consuming and often struggle to accommodate individual needs. SportsBuddy addresses this gap with an intuitive, interactive video authoring tool. It combines player tracking, embedded interaction design, and timeline visualizations to seamlessly integrate narratives and visual cues within game contexts. This empowers users to effortlessly create context-driven video stories. Since its launch, over 150 sports users, including coaches, athletes, content creators, parents and fans, have utilized SportsBuddy to produce compelling game highlights for diverse use cases. User feedback highlights its accessibility and ease of use, making video storytelling and insight communication more attainable for diverse audiences. Case studies with collegiate teams and sports creators further demonstrate SportsBuddy’s impact on enhancing coaching communication, game analysis, and fan engagement.
} % end of abstract

%% ACM Computing Classification System (CCS). 
%% See <http://www.acm.org/class/1998/> for details.
%% The ``\CCScat'' command takes four arguments.

% \CCScatlist{ 
%   \CCScat{K.6.1}{Management of Computing and Information Systems}%
% {Project and People Management}{Life Cycle};
%   \CCScat{K.7.m}{The Computing Profession}{Miscellaneous}{Ethics}
% }

%% Copyright space is enabled by default as required by guidelines.
%% It is disabled by the 'review' option or via the following command:
% \nocopyrightspace

%%%%%%%%%%%%%%%%%%%%%%%%%%%%%%%%%%%%%%%%%%%%%%%%%%%%%%%%%%%%%%%%
%%%%%%%%%%%%%%%%%%%%%% START OF THE PAPER %%%%%%%%%%%%%%%%%%%%%%
%%%%%%%%%%%%%%%%%%%%%%%%%%%%%%%%%%%%%%%%%%%%%%%%%%%%%%%%%%%%%%%%%

\begin{document}

\newcommand{\SB}{SportsBuddy}
\newcommand{\Media}{\texttt{Media}}
\newcommand{\Highlight}{\texttt{Highlight}}
\newcommand{\Narrative}{\texttt{Narrative}}
\newcommand{\Circle}{\textit{Circle}}
\newcommand{\Spotlight}{\textit{Spotlight}}
\newcommand{\Connector}{\textit{Connector}}
\newcommand{\Path}{\textit{Path}}
\newcommand{\Zone}{\textit{Zone}}
\newcommand{\Marker}{\textit{Marker}}
\newcommand{\BGFilter}{\textit{BG Filter}}
\newcommand{\Zoom}{\textit{Zoom In}}
\newcommand{\Text}{\textit{Text}}
\newcommand{\Caption}{\textit{Caption}}
\newcommand{\revision}[1]{\textcolor{black}{#1}}
%% The ``\maketitle'' command must be the first command after the
%% ``\begin{document}'' command. It prepares and prints the title block.

%% the only exception to this rule is the \firstsection command
\firstsection{Introduction}

\maketitle

\section{Introduction}
\label{sec:intro}
% Image editing methods in diffusion models depend on user-defined control directions - users can unlock their creativity using these methods by specifying the desired manipulation through prompts~\cite{gandikota2023concept}, reference images~\cite{ruiz2022dreambooth, kumari2022customdiffusion, gal2022image, chen2024trainingfreeregionalpromptingdiffusion}, or attribute vectors~\cite{parmar2023zero,hertz2022prompt}. In this work, we ask a fundamentally different question: \emph{Can we automatically discover the underlying visual structure of a concept within diffusion model's knowledge?} %Rather than requiring user-specified controls, we aim to decompose the model's internal knowledge into meaningful directions.

% This question touches on a fundamental limitation in how we interact with diffusion models. Current control methods ~\cite{zhang2023addingconditionalcontroltexttoimage, gandikota2023concept, ye2023ipadaptertextcompatibleimage,ye2023ipadaptertextcompatibleimage, hertz2024stylealignedimagegeneration, li2023photomaker, shi2024instantbooth, chen2024trainingfreeregionalpromptingdiffusion} require users to specify their desired manipulations in advance, limiting interactive creativity. This contrasts with natural human artistic workflows, where creators dynamically explore creative ideas while jointly refining them toward meaningful artistic outcomes~\cite{hoffmann2016modeling}. This synergy between specification and exploration is not new to generative models. Early GAN architectures naturally developed disentangled latent spaces that enabled continuous\cite{harkonen2020ganspace,radford2015unsupervised, wu2021stylespace, shen2020interfacegan}, compositional control over generated images. Users could explore these spaces to discover interesting variations that would be difficult to describe in words~\cite{wu2021stylespace}, then combine them to achieve their creative goals~\cite{grabe2022towards}. 


% While diffusion models have largely superseded GANs in conditional image synthesis~\cite{dhariwal2021diffusion},  their underlying structure remains less understood. Diffusion models achieve remarkable diversity through high-dimensional latents, unlike GANs' compact latent spaces.  With a single prompt, diffusion models can generate radically different variations through different random initializations of input noise. We ask - Is it possible to discover interpretable structure within this vast space of variations?

Text-to-image diffusion models are capable of generating remarkable visual variations from a single prompt through different random initializations. However, this vast creative potential remains largely opaque to users---while we can generate diverse images, we lack understanding of the underlying structure of these variations. This presents a fundamental challenge: how can we discover and expose the latent visual capabilities encoded within these models?

\let\thefootnote\relax \footnote{$^{*}$Correspondence to \texttt{gandikota.ro@northeastern.edu}}

The challenge touches on a key limitation in how we interact with diffusion models today. Current control methods require users to explicitly specify their desired edits in advance through prompts~\cite{gandikota2023concept}, reference images~\cite{zhang2023addingconditionalcontroltexttoimage, chen2024trainingfreeregionalpromptingdiffusion, ruiz2022dreambooth,kumari2022customdiffusion, Ryu_lora, hu2021lora}, or attribute vectors~\cite{ye2023ipadaptertextcompatibleimage, hertz2024stylealignedimagegeneration, li2023photomaker, shi2024instantbooth,parmar2023zero,hertz2022prompt}. That contrasts sharply with natural human creative workflows, where artists dynamically explore creative ideas and jointly refine them toward meaningful artistic outcomes~\cite{hoffmann2016modeling}. The need for pre-specified controls creates a barrier between users and the full creative potential of these models.

Interestingly, earlier generative models like GANs~\cite{gans,karras2019style,brock2018large} naturally developed more interpretable internal structures. Their compact latent spaces often exhibited emergent disentanglement~\cite{harkonen2020ganspace,radford2015unsupervised, wu2021stylespace, shen2020interfacegan}, enabling continuous and compositional control over generated images. Users could explore these spaces to discover interesting variations that would be difficult to describe in words~\cite{wu2021stylespace}, then combine them to achieve their creative goals~\cite{grabe2022towards}.

Diffusion models have largely superseded GANs in conditional image synthesis~\cite{dhariwal2021diffusion}, achieving greater diversity through much higher-dimensional latents. And yet an understanding of the underlying structure of these larger latent spaces has remained elusive. In this work, we ask a fundamental question: \emph{Can we automatically discover the visual structure within a diffusion model's knowledge of a concept?} Rather than requiring user-specified controls, we aim to decompose the model's internal representations into expressive directions that users can explore and combine.

To address these needs, we present \textbf{SliderSpace}, a framework that brings systematic explorability to diffusion models. Given just a text prompt, SliderSpace discovers a canonical set of meaningful, diverse, and controllable directions within the model's knowledge of that concept. Each direction is implemented as a low-rank adapter~\cite{hu2021lora} that can be scaled and composed with others, allowing users to explore and smoothly combine different aspects of variation, as shown in Figure~\ref{fig:intro}.

We ground SliderSpace discovery in three key requirements for meaningful decomposition of a diffusion model's visual manifold: 
\begin{enumerate}
    \item \textbf{Unsupervised Discovery:} The decomposition process should emerge from the intrinsic structure of the model's learned representation, rather than being guided by predefined attributes. This ensures we capture the true topology of the model's knowledge space rather than projecting our assumptions onto it.
    
    \item \textbf{Semantic Orthogonality:} Each discovered control must represent a distinct semantic direction. This is enforced in a semantic feature space, like CLIP, where every slider has an orthogonal effect in embeddings. This prevents discovering multiple controls that create similar semantic effects, making the system more efficient and easier.
    
    \item \textbf{Distribution Consistency:} Directions must induce consistent transformations across both random seeds and prompt variations. 
\end{enumerate}

These requirements naturally lead to our proposed framework, which we formalize in Section~\ref{sec:method}. As we show in our experiments, SliderSpace is architecture-agnostic, working with both conventional U-Net based models like Stable Diffusion~\cite{rombach2022high, rombach2022sd20, podell2023sdxl, turbo, dmd} and recent transformer-based architectures like Flux~\cite{flux}.

We demonstrate the expressiveness of SliderSpace through three applications: First, we show how SliderSpace can decompose high-level concepts into diverse and expressive components, revealing the natural axes of variation in the model's understanding. Second, we explore artistic style variation, where SliderSpace discovers directions that match or exceed the diversity of manually curated artist lists while being judged more useful by human evaluators. Finally, we show how SliderSpace can help reverse the mode collapse commonly observed in distilled diffusion models, restoring diversity while maintaining generation speed.

Beyond providing practical creative control, SliderSpace opens new avenues for understanding and utilizing the latent capabilities of diffusion models. By mapping these models' visual potential into intuitive, composable directions, we take a step toward making their creative possibilities more accessible and interpretable to users.

% Image editing methods in diffusion models unlock the creativity of users. In this work we ask an alternate question: \emph{Can we organize and expose what of the diffusion model is already capable of?}.
% Existing methods for controlling image generation typically require users to manually specify edit directions for desired changes. This process is time-consuming, requires technical expertise, and limits the spontaneity of the creative process. For instance, if a user wants to adjust the smile of a generated person, they must explicitly request this edit, often through imprecise prompt engineering or model fine-tuning. This approach of predefined controls or manual specifications restricts users from fully exploring the latent capabilities of the model. There may be interesting stylistic variations or attributes that the model can generate, but users have no easy way to discover or utilize these.

% Natural visual disentanglement was an emergent property in the latent space of Generative Adversarial Models (GANs) \cite{harkonen2020ganspace,radford2015unsupervised, wu2021stylespace, shen2020interfacegan}. In particular, it has been observed that StyleGAN~\cite{karras2019style} stylespace neurons offer detailed control over many meaningful aspects of images that would be difficult to describe in words~\cite{wu2021stylespace}. However, diffusion models do not share such a compact latent space~\cite{park2023unsupervised}; and efforts to uncover such a space in the semantic embeddings of the text conditioning have met with limited success \nik{Nick - is there a specific citation you were thinking about?}.

% In this work we introduce \textbf{SliderSpace}, which takes a step towards uncovering an analogous low dimensional representation of diffusion models' visual breadth; in essence treating the diffusion model as many generators sharing parameters, where a particular generator is defined by a specific prompt. For a given prompt we sample many random seeds (and optionally prompt expansions using an LLM), generate the corresponding images, and apply an off the shelf feature extractor (in this work CLIP, but our method can be applied to any differentiable feature extractor). We use PCA to analyze these features, and for each of the leading $k$ principal components we train a LoRA \cite{} which causes the diffusion model to produces images which increase the feature magnitude along that component when passed back through the same feature extractor. This leads to a 'Slider' for each principal component, because each LoRA can be scaled and applied to the original diffusion model, continuously varying those visual features in the generated results (as measured, in our case, by CLIP).

% There are many other works that enhance the controllability of diffusion models. One common approach is enabling users to add spatial constraints to a generation either manually, or via a reference image \cite{zhang2023addingconditionalcontroltexttoimage, chen2024trainingfreeregionalpromptingdiffusion}, a second is leveraging more abstract embeddings (e.g. identity, style) extracted from a reference image \cite{ye2023ipadaptertextcompatibleimage, hertz2024stylealignedimagegeneration, li2023photomaker, shi2024instantbooth}, a third is finetuning a foundation model to better generate a concept important to the user \cite{ruiz2022dreambooth, kumari2022customdiffusion, Ryu_lora, hu2021lora}, and a fourth (most relevant to this work) is finding low-rank adaptors of the model based on a prompt or small training set which can be scaled to provide continous control over one aspect of generated image (e.g. night vs day, basic vs luxury, etc.) \cite{gandikota2023concept}. SliderSpace is complementary to all of these methods and offers something distinct. All of the other methods we are aware require the user (and / or model designer) to know in advance what type of control they want. In contrast SliderSpace assists users in discovering and controlling hidden capabilities present in the diffusion model's distribution of possible generations.

%We propose that truly intuitive creative control in a text-to-image model should meet three key criteria: \emph{discoverability}, \emph{intuitiveness}, and \emph{specificity}. The model should reveal controllable attributes that may not be immediately obvious, offer controls that are easy to understand and manipulate, and ensure each control affects a distinct attribute of the generated image.

% We demonstrate the utility and power of SliderSpace using three applications built on top of SDXL-DMD \cite{dmd}, because its fast generation speed lends itself well to the continuous control offered by SliderSpace.

% First, we study concept decomposition (Section \ref{sec:concept_exp}), where we learn sliders for a specific concept (e.g. 'monster', 'waterfall', 'car'). Through quantitative metrics of diversity and text alignment we demonstrate that the learned sliders dramatically boost the diversity of generations when randomly applied without harming text alignment; we also ask humans to qualitatively judge these results in a user study where they find the SliderSpace results to be more 'Diverse', 'Useful', and 'Creative' than our baselines.

% Second, we attempt to compare the automatic discoveries of SliderSpace to a large scale manual study of artistic styles (Section \ref{sec:art_exp}), open-sourced by ParrotZone \cite{parrotzone}. In this study SDXL was prompted with over 4300 artist names,  and based on visual inspection the cases of successful stylistic mimicry recorded. Quantitatively SliderSpace more closely matches the distribution of artistic variation discovered by ParrotZone than other baselines, and in our user studies was judged to be significantly more 'Diverse' and 'Useful' than the baselines. To our surprise humans even judged SliderSpace results to be slightly more 'Diverse' than the results generated by the manually discovered artist names of \cite{parrotzone}.

% Third, we attempt to use SliderSpace to reverse the mode collapse commonly observed in distilled few-step diffusion models relative to the original teacher model (Section \ref{sec:diverse_exp}). We quantitatively demonstrate that applying SliderSpace to SDXL-DMD leads to more closely matching the distribution of images by the original teacher, SDXL.

%Through extensive experiments on various state-of-the-art text-to-image models, we demonstrate that SliderSpace significantly enhances user control and creative expression in AI-assisted image generation tasks. Our method enables a range of applications, including concept decomposition and control, diversity improvement in generated images, customization dissection and edits, and the exploration of artistic styles inherent in the model.

% SliderSpace goes beyond providing a practical tool for enhanced creative control. By mapping the visual potential of diffusion models it can open new avenues for generative creativity and deepens our understanding of each model's hidden potential.
\section{Related Work}
\label{sec:related_work}

The original investigation \cite{gibson1979ecological} on the relationship between visual perception and human action defines \emph{affordance} as the opportunities for interaction with the surrounding environment. Behavioral studies on regular and cognitively impaired persons have shown evidence that perception results in both visual and motor signals in the human brain. An extended study \cite{anderson2002attentional} shows that visual attention to the spatial characteristics of the perceived objects initiates automatic motor signals for different actions. In computer vision, human affordance learning involves novel pose prediction such that the estimated pose represents a valid human action within the scene context. The task is fundamental to many problems requiring robust semantic reasoning about the environment, such as human motion synthesis \cite{wang2021scene} and scene-aware human pose generation \cite{wang2017binge, roy2016multi, zhang2022inpaint, yao2023scene}.

Earlier methods of affordance learning have explored knowledge mining \cite{zhu2014reasoning} and multimodal feature cues \cite{roy2016multi} to address the problem. In \cite{zhu2014reasoning}, the authors use a Markov Logic Network for constructing a knowledge base by extracting several object attributes from different image and metadata sources, which can perform various downstream visual inference tasks without any additional classifier, including zero-shot affordance prediction. In \cite{roy2016multi}, the authors use depth map, surface normals, and segmentation map as multimodal cues to train a multi-scale convolutional neural network (CNN) for scene-level semantic label assignment associated with specific human actions. In \cite{do2018affordancenet}, the authors design a multi-branch end-to-end CNN with two separate pathways for object detection and affordance label assignment to achieve high real-time inference throughput. Researchers \cite{chuang2018learning} have also explored socially imposed constraints for affordance learning. In \cite{chuang2018learning}, the authors propose a graph neural network (GNN) to propagate contextual scene information from egocentric views for action-object affordance reasoning.

Probabilistic modeling of scene-aware human motion generation also involves semantic reasoning of human interaction with the environment. Initial works on human motion synthesis have taken different architectural approaches, such as sequence-to-sequence models \cite{barsoum2018hp}, generative adversarial networks (GAN) \cite{barsoum2018hp, cai2018deep, yang2018pose}, graph convolutional networks (GCN) \cite{yan2019convolutional}, and variational autoencoders (VAE) \cite{guo2020action2motion}. However, these methods have mostly ignored the role of environmental semantics. Due to potential uncertainty in human motion, in a recent approach \cite{wang2021scene}, the authors address such motion synthesis with a GAN conditioned on scene attributes and motion trajectory to predict probable body pose dynamics.

One key challenge of human affordance generation in 2D scenes is the lack of large-scale datasets with rich pose annotations. In \cite{wang2017binge}, the authors compile the only public dataset of annotated human body poses in complex 2D indoor scenes by extracting frames from sitcom videos. Aiming to generate a contextually valid human affordance at a user-defined location, the authors propose sampling the scale and deformation parameters for an existing human pose template using a VAE conditioned on the localized image patches as scene context. In \cite{zhang2022inpaint}, the authors introduce a two-stage GAN architecture for achieving a similar goal by estimating the affine bounding box parameters to localize a probable human in the scene and then generating a potential body pose at that location. The method uses the input scene, corresponding depth, and segmentation maps as semantic guidance. In \cite{yao2023scene}, the authors propose a transformer-based approach with knowledge distillation for generating human affordances in 2D indoor scenes.


\section{Design of \ourSystem}\label{sec_design}

\begin{figure*}[t]
\centering
{\includegraphics[width=.95\textwidth]{figs/workflow.pdf}}
\caption{Illustration of the \ourSystem architecture, trained end-to-end, consisting of three main building blocks.}
  \Description[]{}
	\label{fig_workflow}
  \vspace{-10pt}
\end{figure*}



Figure~\ref{fig_workflow} illustrates \ourSystem, which synthesizes the received signal at a specific receiver for a transmitter positioned at any location in the scene. 
\textit{i)~Gaussian-based RF scene representation} models the scene with 3D Gaussians, each storing geometric and RF-related attributes.  
\textit{ii)~Gradient-guided attribute learning} optimizes these attributes via gradients while dynamically adjusting the number of Gaussians.  
\textit{iii)~RF-customized CUDA for ray tracing} computes the received signal by emitting rays from the receiver, identifying ray-Gaussian intersections via orthographic projection-based splatting, and aggregating RF attributes along each ray.


\subsection{Gaussian-Based Scene Representation}

Each customized 3D Gaussian distribution carries four attributes:  
i)~mean \(\boldsymbol{\mu}\) and ii)~covariance matrix \(\boldsymbol{\Sigma}\), which define its geometric properties, including position, size, shape, and orientation;
iii)~emission \(\boldsymbol{\psi}\) and iv)~attenuation \(\boldsymbol{\rho}\), which characterize the Gaussian's influence on RF signal propagation.


\textbf{i)~Mean~\(\boldsymbol{\mu}\) and ii)~Covariance Matrix~\(\boldsymbol{\Sigma}\):}  
A 3D Gaussian  distribution resembles an ellipsoid, representing a probability distribution in 3D space, as defined by the probability density function~(PDF) in Equation~(\ref{eqn_gaussian}).
The center of the distribution is a 3D position (mean \(\boldsymbol{\mu}\)), indicating the peak location, while the spread and orientation in space are determined by a full~\(3 \times 3\) covariance matrix \(\boldsymbol{\Sigma}\):
\begin{equation}
\label{eqn_gaussian}
    P\left(\mathbf{x}\right) = e^{-\frac{1}{2} \left(\mathbf{x} - \boldsymbol{\mu}\right)^\text{T} \boldsymbol{\Sigma}^{-1} \left(\mathbf{x} - \boldsymbol{\mu}\right)}
\end{equation}






\textbf{iii)~Emission~\(\boldsymbol{\psi}\):}
Each point on a wavefront serves as a source of wavelets, according to the Huygens–Fresnel principle~\cite{born2013principles}.
Analogously, when an RF signal from a transmitter encounters a 3D Gaussian, the Gaussian acts as a scattering point, re-emitting an RF signal termed emission \(\boldsymbol{\psi}\). 
This emission is characterized by \(\boldsymbol{\psi} = \left|\boldsymbol{\psi}\right| e^{j \angle \boldsymbol{\psi}}\), where \(\left|\boldsymbol{\psi}\right|\) represents the amplitude and \(\angle \boldsymbol{\psi}\) denotes the phase.



Emission \(\boldsymbol{\psi}\) varies due to factors such as surface orientation relative to the incident signal and the material properties at the Gaussian's position.  
Moreover, the transmitter position influences \(\boldsymbol{\psi}\) by altering the incident angle.  
Thus, \(\boldsymbol{\psi}\) depends on both the direction and the transmitter position.
To this end, we employ a small neural network, \(f_\Theta\), within each Gaussian to encode the directional emission signal:
\begin{equation}
\label{eqn_radiance_mlp}
\boldsymbol{\psi} = f_\Theta \left(x_{\text{tx}}, y_{\text{tx}}, z_{\text{tx}}, \alpha, \beta\right)
\end{equation}
where \(f_\Theta\) takes the transmitter position \(\left\{x_{\text{tx}}, y_{\text{tx}}, z_{\text{tx}}\right\}\) and the direction \(\left\{\alpha, \beta\right\}\) as inputs and outputs the corresponding emission \(\boldsymbol{\psi}\). 
Here, \(\left\{\alpha, \beta\right\}\) represents the direction from the 3D Gaussian's position to the receiver.  
Since \(f_\Theta\) models only its own Gaussian's emission, it has a small number of parameters,~\ie~two fully connected layers with ReLU activation.



\textbf{iv) Attenuation~\(\boldsymbol{\rho}\):}
An RF signal passing through a 3D Gaussian undergoes attenuation \(\boldsymbol{\rho}\), resulting in an amplitude reduction \( \left|\boldsymbol{\rho}\right| \) and a phase shift \( \angle \boldsymbol{\rho} \).  
According to Maxwell's equations~\cite{maxwell1873treatise}, attenuation depends on material properties.  
Thus, the attenuation \(\boldsymbol{\rho}\) of a 3D Gaussian is primarily determined by the material properties at its location.



\subsection{Gradient-Guided Attribute Learning}  
We initialize the number of Gaussians and their attributes, then optimize both with gradient-based strategies. 
Gradients are calculated after computing the loss in §\ref{sec_training_loss}.  
Finally, we discuss the efficiency of using 3D Gaussians.


\subsubsection{Cube-Based Initialization}
We partition the scene into equal-sized cubes, each with a side length \( L_{\text{cube}} \), empirically set to six times the wavelength.  
The center of each cube is assigned as a Gaussian's mean.  
The covariance matrix is initialized based on the average distance to the \( N_{\text{cube}} \) nearest cube centers, where \( N_{\text{cube}} \) is set to three, while attenuation and emission are assigned randomly.
Compared to random initialization, this cube-based strategy ensures that the Gaussians cover the entire scene, leading to faster convergence.



\subsubsection{Gradient-Based Updating}
The following two strategies are employed to update the number of Gaussians and their attributes for flexible and efficient scene representation.


\textbf{i)~Attribute Updating:}
Each 3D Gaussian explicitly stores its own attributes and updates them using SGD~\cite{amari1993backpropagation}:
\begin{equation}
\label{eqn_updating}
w^{\left(j+1\right)} = w^{\left(j\right)} - \eta_{w} \cdot \nabla_{w} \mathcal{L}\left(w^{\left(j\right)}\right)
\end{equation}
where \(w\) represents any attribute of a Gaussian, each with its own learning rate \(\eta_{w}\).  
The term \(\nabla_{w} \mathcal{L}\left(w^{(j)}\right)\) denotes the gradient of the loss function \(\mathcal{L}\), defined in Equation~(\ref{eqn_loss}), with respect to \(w\) at iteration \(j\).  
For emission \(\boldsymbol{\psi}\), the updated parameters belong to the network \(f_\Theta\).

The covariance matrix \(\boldsymbol{\Sigma}\) is physically meaningful only when positive semi-definite~\cite{de2011strict}, but the update equation above does not guarantee this property.  
To address this, we adopt the solution proposed in \cite{kerbl20233d}, which represents \(\boldsymbol{\Sigma}\) as \(\Sigma = R S S^{T} R^{T}\), where \(R\) is a rotation matrix and \(S\) is a scaling matrix.  
Updates are applied independently to \(R\) and \(S\), ensuring that \(\boldsymbol{\Sigma}\) remains positive semi-definite.



\textbf{ii)~Number of Gaussian Updating:}
The initial number of Gaussians is set by cube-based initialization.  
However, this number is suboptimal, as some areas require more Gaussians~(\eg object regions), while others need fewer~(\eg free space) to model RF signal propagation effectively.
We observe that such cases lead to large gradients for the Gaussian's mean~\(\boldsymbol{\mu}\), as the existing 3D Gaussians do not adequately capture the area's effect on RF signal propagation.  
The mean~\(\boldsymbol{\mu}\) exhibits larger gradients than other attributes because it represents the position with the highest probability, making it crucial for modeling RF signal behavior.



To this end, we employ a gradient-threshold-based strategy: 
\textit{First}, every \(N_{\boldsymbol{\mu}}\) iterations, we compute the average gradient of the mean \(\boldsymbol{\mu}\) for all Gaussians and select those with a mean gradient exceeding a threshold \(\epsilon_{\boldsymbol{\mu}}\).  
\textit{Second}, we determine the radius of each selected Gaussian, approximated as the average of the diagonal values of its covariance matrix.  
A radius threshold \(\epsilon_{r}\) classifies them as small or large Gaussians.
\textit{Third}, small Gaussians are cloned by duplicating them and shifting the copies in the direction of the gradient.
Large Gaussians are split into two new Gaussians, reducing their scaling matrix \(R\) by a factor of \(\phi\) and initializing their positions by sampling from the original Gaussian's PDF.

Additionally, every \(N_{\boldsymbol{\rho}}\) iterations, we remove Gaussians with attenuation~\(\boldsymbol{\rho}\) below a threshold \(\epsilon_{\boldsymbol{\rho}}\), as they minimally impact signal propagation, \eg in free space.  
A single 3D Gaussian distribution can represent a large free space.




\subsubsection{Efficiency of 3D Gaussian}\label{sec_design_gaussian}


Unlike fixed voxel grids, which require numerous voxels to capture the entire scene's effects on signal propagation, 3D Gaussians adjust their position, shape, size, and orientation to represent these effects.  
This adaptability enables 3D Gaussians to achieve similar representation quality with far fewer voxels, improving computational efficiency and reducing training data requirements.
For example, a typical conference room~\cite{matlab_conference_room} requires learning the attributes of 31,257,628 voxels~(Section~\ref{sec_theoretical_ana}).  
In contrast, only 393,920 Gaussians may be needed, an \(\mathord{\sim}80\)-fold reduction, assuming the optimal count matches the number of points in the conference room's point cloud data. 





\subsection{RF-Customized CUDA for Ray Tracing}

The emitted rays from the receiver are formalized, followed by an orthographic projection-based splatting module, which identifies Gaussians intersected by each ray.  
Next, the complex-valued blending algorithm computes the received signal based on these intersections.  
Finally, computation is parallelized using customized CUDA kernels.


\subsubsection{Definition of Rays}
Rays extend from the receiver in various directions, \eg ray \(\gamma\) in Figure~\ref{fig_blending}:
\begin{equation}
\label{eqn_ray_define}
\gamma(d) = \mathbf{l}_{\text{rx}} + d \hat{\mathbf{v}}, \quad \text{where} \quad d \geq r_{\text{rx}}
\end{equation}
where \(d\) is the distance from the receiver to a point \(r(d)\) on the ray, \(\mathbf{l}_{\text{rx}} = \left(x_{\text{rx}}, y_{\text{rx}}, z_{\text{rx}}\right)\) denotes the receiver position, and the unit vector \(\hat{\mathbf{v}} = \left(\cos\alpha \cos\beta, \sin\alpha \cos\beta, \sin\beta\right)^\top\) defines the ray direction, with \(\alpha\) and \(\beta\) as the azimuthal and elevation angles, respectively.  
The condition \(d \geq r_{\text{rx}}\) indicates that the ray starts at a distance \(r_{\text{rx}}\) from the receiver.
Thus, \(360 * 90\) rays are emitted from a spherical surface centered at the receiver with radius \(r_{\text{rx}}\).
We refer to this surface as the Ray Emitting Spherical Surface~(RESS).



\subsubsection{Orthographic Projection-Based Splatting}\label{sec_ortho}
The uniform voxel grid structure allows straightforward identification of the voxels a ray passes through.  
However, the irregular and discrete placement of 3D Gaussians complicates determining which Gaussians a ray intersects.  
Intuitively, each ray must be checked against all Gaussians, resulting in a computational complexity of \(O(M \times N)\), where \(M\) is the total number of rays and \(N\) is the total number of Gaussians.


In 3DGS, to accelerate the determination of which Gaussians affect each ray~(pixel), 3D Gaussians are projected~(or "splatted") onto a 2D image plane.  
This splatting process utilizes the View Matrix, Projection Matrix, and Jacobian Matrix~\cite{takikawa2021neural} to form 2D Gaussians on the image plane.  
Each projected 2D Gaussian is represented as a circle centered at its mean, with a radius determined by the 2D covariance matrix.  
Pixels~(rays) within this circle are considered affected by the original 3D Gaussian.  
This splatting reduces computational complexity to \(O(N)\) by localizing each Gaussian’s influence to a specific region on the image plane.

 
\begin{figure}[!tp]	{\includegraphics[width=.47\textwidth]{figs/bleding.pdf}}
    \caption{Illustration of the complex-valued blending algorithm, which calculate the received signal in direction \(\gamma\). Four 3D Gaussians are shown, where \(\boldsymbol{\psi}\) and \(\boldsymbol{\rho}\) denote the emission and attenuation of each Gaussian. Each emission \(\boldsymbol{\psi}_i\) is attenuated by \(\boldsymbol{\rho}_m\) from Gaussians \(m\) (from \(1\) to \(i - 1\)). The final received signal in direction \(\gamma\) is the sum of these attenuated emissions.}
\label{fig_blending}
 \Description[]{}
\end{figure}


\textbf{2D RF Plane:}  
The above splatting is not applicable in the RF domain without an image plane.   
Instead, the received signal is measured on the RESS.
To enable splatting, the RESS is transformed into a 2D RF plane.  
Specifically, the Cartesian coordinates \((x, y, z)\) of each point on the RESS are converted to spherical coordinates \((\zeta, \alpha, \beta)\), with \((\alpha, \beta)\) rounded to the nearest integers to achieve one-degree resolution:
\begin{equation}
\label{eqn_projection}
\begin{aligned}
\begin{pmatrix}
\zeta \\
\alpha \\
\beta
\end{pmatrix}
&=
\begin{pmatrix}
\sqrt{x^2 + y^2 + z^2} \\
\arctan2\left(y, x\right) \\
\arccos\left(\frac{z}{\sqrt{x^2 + y^2 + z^2}}\right)
\end{pmatrix} \\
x' &= \lfloor \alpha \rfloor, \quad y' = \lfloor \beta \rfloor
\end{aligned}
\end{equation}
where \(\zeta\) is the radial distance, \(\alpha\) the azimuthal angle, \(\beta\) the elevation angle, and \(\lfloor \cdot \rfloor\) the floor function.  
\(\left(x', y'\right)\) represents the projected coordinates in the 2D RF plane.


\textbf{Splatting:}  
Each 3D Gaussian is splatted onto the RF plane, forming a 2D Gaussian represented as a circle.  
Gaussian mean \(\boldsymbol{\mu}\) is projected onto the 2D plane using Equation~(\ref{eqn_projection}), defining the circle's center.  
Jacobian matrix~\cite{takikawa2021neural} maps the \(3 \times 3\) covariance matrix into a \(2 \times 2\) covariance matrix, whose eigenvalues determine the circle's radius.  
Rays within this circle are considered influenced by the original 3D Gaussian.


\subsubsection{Complex-Valued Blending Algorithm}\label{sec_complex_blending}  
We introduce a complex-valued blending algorithm to process a given ray and its identified intersected Gaussians.  
First, the Gaussians are sorted by their distance to the receiver.  
Then, the received signal for the ray is computed by aggregating their RF attributes, incorporating both amplitude and phase channels:
\begin{equation}
\label{eqn_blending}
S = \sum_{i=1}^{N_{\text{intr}}}  \left|\boldsymbol{\psi_i}\right| e^{j \angle \boldsymbol{\psi_i}} \cdot \prod_{m=1}^{i-1} \left( 1 - \left|\boldsymbol{\rho_m}\right| e^{j \angle \boldsymbol{\rho_m}} \right)
\end{equation}
where \(S\) is the received signal for a ray, \(N_{\text{intr}}\) denotes the number of Gaussians intersecting the ray, and \(\boldsymbol{\psi}_i\) and \(\boldsymbol{\rho}_m\) represent the emission and attenuation of the \(i\)-th and \(m\)-th Gaussians, respectively.  
Emission \(\boldsymbol{\psi}_i\) is attenuated by \(\boldsymbol{\rho}_m\) from preceding Gaussians.
The received signal is the sum of these attenuated emissions.
Figure~\ref{fig_blending} illustrates a ray intersecting four Gaussians, represented as follows:
\begin{equation}
\begin{aligned}
S = & \ \boldsymbol{\psi_1} + 
\boldsymbol{\psi_2} \cdot (1 - \boldsymbol{\rho_1}) + \boldsymbol{\psi_3} \cdot (1 - \boldsymbol{\rho_2}) \cdot (1 - \boldsymbol{\rho_1}) \\
& + \boldsymbol{\psi_4} \cdot (1 - \boldsymbol{\rho_3}) \cdot (1 - \boldsymbol{\rho_2}) \cdot (1 - \boldsymbol{\rho_1})
\end{aligned}
\end{equation}


\textbf{Impact of Gaussian Geometry:}  
In a voxel-based ray tracing algorithm, a ray is assumed to pass through the center of each voxel.  
However, this assumption does not hold in Gaussian-based scene representation.  
For example, in Figure~\ref{fig_blending}, both G2 and G3 intersect the ray, but the intersection point on G2 is closer to its mean than that on G3.  
Even if G2 and G3 share the same emission and attenuation attributes, their contributions to the final received signal differ due to the varying distances of their intersection points from their means.  
These distances affect the probability of each Gaussian influencing the ray.  
Therefore, the blending process in Equation~(\ref{eqn_blending}) should account for Gaussian geometry.




To achieve this, the intersection point is first determined by solving the ray equation~(Equation~\ref{eqn_ray_define}) and the ellipsoid equation~(Equation~\ref{eqn_gaussian}).  
If two solutions exist, their midpoint is taken as the intersection point.  
Next, the distance between the intersection point and the Gaussian’s mean is calculated.  
The intersection probability \(p_{\text{intr}}\) is then determined by evaluating the Gaussian’s PDF at this distance.
Finally, the emission is adjusted by multiplying it by \(p_{\text{intr}}\): \(\boldsymbol{\psi} = p_{\text{intr}} \cdot \boldsymbol{\psi}\).




\subsubsection{CUDA Kernel}
We develop two CUDA kernels for the forward and backward computations in ray tracing.


\textbf{Forward Kernel:}  
Algorithm~\ref{alg_cuda} outlines the forward kernel.  
The inputs include the number of rays in azimuth and elevation, the means, covariance matrices, emissions, and attenuations of all 3D Gaussians, as well as the positions of the receiver and transmitter.  
The output is the received signal computed for all \(360 * 90\) rays.



Specifically, Line 2 projects 3D Gaussians onto the 2D RF plane.  
Line 3 partitions all rays into multiple grids, each containing \(N_{\text{rays}}\) rays in the azimuth and elevation directions, to accelerate processing.  
Line 4 applies the splatting process to identify which Gaussians influence each grid.  
Line 5 records the sorted Gaussians within each grid.  
Finally, Lines 7–12 compute the received signal for each ray in parallel using the complex-valued blending algorithm.



\textbf{Backward Kernel:}
Since the Forward Kernel is invoked for ray tracing forward computation, PyTorch cannot automatically compute the corresponding computation graph gradients.
After computing the received signal \(S\) and the loss \(\mathcal{L}\), PyTorch calculates the gradient \(\frac{\partial \mathcal{L}}{\partial S}\), which is then passed to the Backward Kernel.  
This kernel reverses the computations of the Forward Kernel to compute the gradients for each Gaussian attribute.




\subsection{Training Loss}\label{sec_training_loss}
After calculating the received signal for each ray, the choice of loss function depends on the type of receiver antenna.


\begin{algorithm}[!tp]
\caption{Forward CUDA Kernel for Ray Tracing}
\label{alg_cuda}
\KwIn{$w, h$: numbers of rays in azimuth and elevation}
\KwIn{$M, C$: means \& covariances of all Gaussians}
\KwIn{$E, A$: emissions \& attenuations of all Gaussians}
\KwIn{$L$: positions of receiver and transmitter}
\KwOut{$O$: received signals for all rays}
\SetKwFunction{FMain}{RayTracing}
\SetKwProg{Fn}{Function}{:}{}

\Fn{\FMain{w, h, M, C, E, A, L}}{
    $M'$, $C'$ $\gets$ \text{sphericalGaussian}($M$, $C$, $L$)  \\
    \text{Grids} $\gets$ \text{buildGrid}($w$, $h$) \\
    \text{Idx}, \text{Kys} $\gets$ \text{sphericalSplatting}($M'$, \text{Grids})  \\
    \text{Ranges} $\gets$ \text{computeGridRange}(\text{Idx}, \text{Kys}) \\
    $O$ $\gets$ 0  \\
    \ForAll{grid G in Grids}{
        \ForAll{ray i in G}{
            ra $\gets$ \text{getGridRange}(\text{Ranges}, $g$) \\
            $O$[$i$] $\gets$ \text{Blend}($i$, \text{Idx}, \text{ra} \text{Kys}, $M$', $C'$, $E$, $A$) \\
        }
    }
    \Return $O$
}
\end{algorithm}


\textbf{i) Antenna Array:}
When the receiver is an antenna array, it captures signal power from all directions, represented as a~\((360, 90)\) ground-truth matrix, where 360 and 90 correspond to azimuth and elevation angles, respectively, each with one-degree resolution~\cite{zhao2023nerf}.  
Each matrix entry represents a single real-valued signal power for its corresponding direction.  
The loss function \(\mathcal{L}\) combines the \(\mathcal{L}_{1}\) loss and the Structural Similarity Index Measure (\(\mathcal{L}_{\text{SSIM}}\)) loss: 
\begin{equation}
\label{eqn_loss}
\mathcal{L} = (1 - \lambda) \mathcal{L}_{1} + \lambda \mathcal{L}_{\text{SSIM}}
\end{equation}
where \(\mathcal{L}_{1}\) measures the average difference between the actual and predicted signal power across all rays.  
Since the~\((360, 90)\) matrix can be viewed as an image~(Figure~\ref{fig_vis_d1}), \(\mathcal{L}_{\text{SSIM}}\) evaluates the structural similarity between the predicted and ground-truth "images," helping \ourSystem learn spatial patterns across rays.  
The parameter \(\lambda\) balances these two losses.



\textbf{ii) Single Antenna:}  
For a single antenna, the ground-truth received signal is either a single real-valued signal power or a complex-valued number containing both amplitude and phase, assumed to be the sum of signals from all directions~\cite{zhao2023nerf}.  
Thus, \ourSystem computes the predicted signal by summing the signals from all rays.  
If the received signal is real-valued, the \(\mathcal{L}_{1}\) loss is applied.  
For a complex-valued signal, the \(\mathcal{L}_{1}\) loss is computed separately for amplitude and phase, then averaged to obtain the final loss.
 

\begin{figure}[t]
    \centering
    \includegraphics[width=0.7\linewidth]{Photos/TerraTrace-fig3.pdf}
    \vspace{-0.2cm}
    \caption{\textbf{TerraTrace System.} TerraTrace finds the NDVI coordinates from our dataset, extracts a set of metrics to analyze the data, and passes these to GPT-4 Turbo for additional analysis.}
    \label{fig:Fig3}
    \vspace{-0.5cm}
\end{figure}
\section{TerraTrace Platform}
\label{system}
We combine our insights about NDVI signatures from the dataset developed in Sec.~\ref{signatures} to develop TerraTrace, an end-to-end AI powered land use analysis platform. TerraTrace takes in a set of geographic coordinates that define the target region. From the dataset, we filter coordinates within this geo-polygon by coarse latitude and longitude ranges to identify the dataset region. Next we calculate the Euclidean distance between a target coordinate and points in our dataset. We extract the corresponding signature curves within the prescribed polygon by computing the mean NDVI value per time point across valid coordinates. We then interpolate the NDVI values and fit a 3rd order polynomial. TerraTrace presents users with a GUI that plots the region using the Leaflet interactive map library which creates an interface to adjust the polygon and plot the NDVI signature curve. 

TerraTrace also analyzes the data to extract land use insights. First, we check that the data is valid and the curve has $>$10 points for robust classification. We then extract features such as the annual minimum and maximum NDVI, range, and median. We determine the growth and decline rates by calculating the maximum point difference in NDVI values. We then use these metrics to check if the curve is an annual crop. This means that the NDVI increases above 0.2 indicating healthy vegetation growth, reaches a peak between 0.2 and 0.8, followed by a decline. We check the growth and decline rates are $>$0.005 to help filter out perennial species which only have small NDVI fluctuations across seasons.

TerraTrace complements these metrics with an LLM based analysis. We pass in the statistics such as max, min and average, an image of the NDVI curve, and a classification of whether the region contains vegatation using a JSON format. We determine vegetation presence with thresholds: <0.1 is non-vegetative, 0.1-0.2 as some vegetation, and 0.2 as healthy vegetation \cite{eos}. We pass in images converted to grayscale, resized, and encoded as a base64 string within the JSON. Next we construct prompts like the following to query the model. "The area of interest is defined by the $[{coordinates}]$. Please analyze the land cover type at this location." We use GPT-4 Turbo (version 2024-04-09) which translates the curves and data into a detailed analysis table, providing an additional validation. Further explanation of the algorithm is presented in Fig.~\ref{fig:Fig3}.

%We integrate additional data sources as well. We use the CDL to calculate a percentage of total crop pixels of specific types within the target region and 

%\textcolor{red}{TODO: move Historic Wild-Fire Data to the end}

%We pass in the statistics such as max, min and average, an image of the NDVI curve, and a classification of whether the region contains vegatation. We determine this using a simple threshold defining <0.1 as non-vegetative, 0-0.2 as some vegetation, and 0.2 as healthy vegetation. We pass in images converted to grayscale, resized, and encoded as a base64 string within the JSON. Next we construct prompts like the following to query the model. "The coordinates for the area of interest are {coordinates}. Please analyze the land cover type at this location."


%VI: I'm removing this because I don't know if it's clear that the compute required to serve an LLM is lower than this. Maybe it is but without hard numbers I think it may raise questions.
%After the curves are plotted, a comprehensive data analysis is conducted. We reduce the platform's dependence on extensive data and compute power by combining mathematical modeling with LLM analysis.


%To enhance the robustness of our analysis, we employ a Large Language Model GPT-4 Turbo. Utilizing up to 2000 tokens per query, the LLM translates the visual representation of the curves into a detailed analysis table, providing an additional layer of validation. Further explanation of the algorithm is represent in Fig.~\ref{Fig3}.



\section{Evaluation}
\label{sec:evaluation}
Our experiments aim to investigate whether agents within our framework can produce effective evolution of language strategies. Specifically, our experimental section addresses the following three research questions (RQs):
\begin{enumerate}
    \item RQ1 (Effectiveness): Can participants effectively evade regulatory detection over time, and how does the accuracy of information transmission change? Additionally, how do different LLMs affect the content and effectiveness?
    \item RQ2 (Human Interpretation): Do the evolved language strategies employed by agents effectively align with human understanding? Can they be interpreted and applied in real-world scenarios?
    \item RQ3 (Ablation Study): How does the newly introduced GA impact the evolution process in our framework?
\end{enumerate}

\subsection{Experimental Settings}
In our evaluation, we designed an abstract password game \cite{guess_number02} and a more realistic illicit pet trade scenario\cite{trade01,trade02,trade03}. 
%The password game features a relatively abstract, easily controlled setting, allowing for clear observation of how agents’ strategies evolve. Meanwhile, the illicit pet trade scenario simulates illegal activities on social networks \cite{DiMinin2018MachineLF}, with relevant corpora that more closely resemble real-world conditions, enabling a more direct comparison between evolved strategies and their real-life counterparts.
The overall experimental procedure follows the description in Section~\ref{sec: method}. In each round, the process comprises three stages: initialization, dialogue, and interview. In each round, the information that the participant agents need to convey will be randomly generated within a defined range. Only rounds in which a five-turn dialogue is successfully completed proceed to the interview phase, during which both parties’ successful transmission of information is verified. If any regulation violations are detected by the supervisory agent during the dialogue stage, that round is deemed a failure and is assigned a transmission score of zero. To ensure the robustness of our findings, we conducted 15 independent trials for each experimental condition. Each trial consisted of 50 fixed dialogue rounds. The crossover and mutation probabilities in the framework were set to 0.2 and 0.8.

\subsubsection{Scenario 1: Password Game}
Our first scenario is considered a relatively simple and abstract guessing game that involves a numerical context. In this setup, each participant is assigned a four-digit password, which they must convey to another participant within five rounds of dialogue without directly mentioning the numbers. Simultaneously, they need to extract information from the dialogue to infer the other's password.

The supervisory agent in the dialogue follows the policy of “prohibiting all content related to numbers.” The “password game” scenario is specifically designed to observe and analyze participant agents' language adaptability and strategic evolution in a theoretical and abstract context. Compared to complex scenarios based on real events, it provides a clearer and more easily quantifiable experimental environment.

\subsubsection{Scenario 2: Illicit Pet Trade} 
Simulating and detecting the covert drug transactions prevalent on social platforms is an important and realistic research topic. However, given the ethical considerations, we choose not to proceed with direct simulations, but to adopt a more neutral and less harmful setting—a illicit pet trade scenario—to serve as our experimental context.

In this scenario, we simulate an illegal pet trade through social platform. Participant agents play the roles of buyer and seller. The buyer conveys a request to purchase a randomly selected pet (such as a parrot, lizard, or cat) and specifies the currency they can use for payment (e.g., USD, CNY, JPY). The seller provides details about the transaction, including the time (morning, noon, or midnight) and location (school, park, or station). The supervisory agent enforces a policy that considers discussions related to pet transactions as regulation violations.
The focus of this experiment is to observe the adaptability of the participant agents' language and its evolution at a finer level of information granularity, as well as whether regulators can accurately detect regulation violations.

\subsection{RQ1: Effectiveness}
\subsubsection{Experiment Settings}

To evaluate RQ1, we conducted experiments using four different LLM models provided by various companies: \textit{gpt-4o}, \textit{gpt-4o-mini}, \textit{deepseekV2}, and \textit{qwen-turbo}. We introduced the following metric:
\begin{itemize}
    \item Accuracy of Information Transmission: This metric measures the agents' ability to convey the intended information to their dialogue partner without being detected. Specifically, it calculates the proportion of successfully transmitted information relative to the total intended information. 
    \item Average Dialogue turns: This metric evaluates the number of successful dialogue rounds the agents complete without being flagged by the supervisory agent. A higher dialogue turns indicates greater success in evading regulatory scrutiny.
    \item Average Entropy (Language Complexity):
        Entropy reflects the unpredictability of text and is calculated as:
        \[
        H(X) = -\sum_{i=1}^n P(x_i) \log P(x_i),
        \]
        where \(P(x_i)\) is the probability of each token. Higher entropy suggests a broader range of expressions (indicating greater innovation), but excessively high entropy can lead to incoherence.
    \item Distinct (Lexical Diversity):
        Distinct measures the number of unique n-grams in the text, computed as:
        \[
        \text{Distinct-n} = \frac{\text{Number of unique n-grams}}{\text{Total number of n-grams}}.
        \]
        Specifically, we evaluate Distinct-1 (unique unigrams). A higher distinct score indicates a larger, more varied vocabulary, signifying richer language use. Note that while both metrics assess diversity in language, entropy focuses on unpredictability, whereas distinct emphasizes lexical variety. 
\end{itemize}


\subsubsection{Experiment Results in Password Game}
\begin{figure*}[ht]
    \centering
    % 子图 (a)
    \begin{subfigure}[t]{0.48\textwidth}
        \centering
        \includegraphics[width=\linewidth]{figures/sec1_turn_acc_v5.png}
        \caption{Password Game}
        \label{fig:sce1}
    \end{subfigure}
    \hfill
    % 子图 (b)
    \begin{subfigure}[t]{0.48\textwidth}
        \centering
        \includegraphics[width=\linewidth]{figures/sec2_turn_acc_v5.png}
        \caption{Illicit Pet Trade}
        \label{fig:sce2}
    \end{subfigure}
    \caption{Average Continuous Dialogue Turns and Information Transmission Accuracy Across Dialogue Rounds}
    \label{fig:merged}
\end{figure*}

Figure \ref{fig:sce1} presents our experimental results in the password game. The x-axis corresponds to the increasing number of dialogue rounds, whereas the y-axis captures two primary metrics: (1) the average number of continuous dialogue turns before detection by the supervisory agent, and (2) the accuracy of information transmission, which is defined as the proportion of successfully transmitted information during the post-dialogue interview.

Overall, our findings indicate that as the number of rounds increases, agents gradually learn to evade regulation violation while conveying information with greater accuracy. Notably, most agents exhibit a pronounced local peak around the 20th round, followed by a brief decline and subsequent recovery. These fluctuations can be attributed to the dynamic nature of our simulation framework, which does not converge on a single dominant strategy but rather encourages ongoing exploration of novel language strategies.

We also observe that different LLMs influence the learning trend to varying degrees. Among the models tested, \textit{gpt-4o} demonstrates the strongest performance. Although other models generally share a similar upward trend, their relative performance gaps prove less stable. For instance, while \textit{deepseekV2} achieves the highest number of turns around the 20th round, its performance declines significantly by the 50th round in comparison to other models.

Turning to the accuracy results, we again observe a similar learning trajectory. This parallel arises primarily because if participant agents fail to complete a sufficient number of uninterrupted dialogue turns, the successfully transmitted information in that round is effectively zero. Consequently, especially in the early stages of the experiment, many rounds end with no successful transmissions. Overall, \textit{gpt-4o} still maintains a clear advantage over the other LLMs. However, we do observe subtle differences when comparing the dialogue round trends: for example, at the 20th dialogue round, \textit{deepseekV2} achieves a significantly higher average number of communication cycles than \textit{gpt-4o-mini}, yet their information transmission accuracy remains relatively similar.

Despite the overall positive learning trajectory, the average information accuracy remains low in the password game. We believe this outcome stems primarily from the intrinsic difficulty introduced by the scenario’s abstract nature. Without explicit prompts driving agents to develop symbolic or otherwise encrypted language stratgy, communication largely remains within the realm of everyday language. Consequently, the indirect expression of numeric information is challenging to implement and easily detectible by the supervisory agent.

\begin{table}[h!]
    \centering
    \caption{Performance of Different LLMs in Password Game}
    \label{tab:sce1}
    \renewcommand{\arraystretch}{1.2} % 调整行高
    \begin{tabular}{l S S S}
        \toprule
        \textbf{Model} & \textbf{Total Turns} & \textbf{Avg. Entropy} & \textbf{Avg. Distinct-1} \\
        \midrule
        \rowcolor{gray!10} \textbf{gpt-4o}       & 84.2   & 7.103 & 0.484 \\
        \textbf{gpt-4o-mini}  & 75.5   & 6.998 & 0.354 \\
        \rowcolor{gray!10} \textbf{deepseekV2} & 59.7
        & 5.365 & 0.247 \\
        \textbf{qwen-turbo}   & 50.8  & 6.101 & 0.518 \\
        \bottomrule
    \end{tabular}
\end{table}

Table \ref{tab:sce1} summarizes the performance of the four models in terms of cumulative dialogue turns, entropy, and Distinct-1. As shown, \textit{gpt-4o} achieves the highest values in both Entropy (7.103) and Distinct-1 (0.484), indicating that it employs a broader, more diverse range of vocabulary and more unpredictable expressions—thus having a greater likelihood of evading regulation violation detection when conveying numerical information. In contrast, \textit{deepseekV2} exhibits notably lower Entropy (5.365) and Distinct-1 (0.247), suggesting a more frequent reuse of fixed expressions. Notably, although \textit{qwen-turbo}’s Entropy (6.101) is only moderate, it attains a surprisingly high Distinct-1 (0.518), reflecting greater lexical richness. However, this does not translate into more effective regulatory evasion, as its total turns are only 50.8. Hence, merely having higher lexical diversity and linguistic entropy is insufficient to guarantee successful evasion. A model must also balance the concealment of overall semantics with the adaptation of its language strategy to achieve longer conversation sequences and a higher rate of successful information transmission. In other words, while richer language expression does confer certain advantages in countering regulation, it can still be detected when deeper strategies—such as tailored expression structures and topic evolution—are absent, ultimately resulting in fewer total turns.

\subsubsection{Experiment Results in Illicit Pet Trade Scenario}
Figure~\ref{fig:sce2} presents the experimental results of our framework in the illicit pet trade scenario, which overall resemble those of the password game but also exhibit some notable differences. First, both figures reveal a discernible learning trend, particularly during the initial 10 rounds. Meanwhile, \textit{gpt-4o} continues to demonstrate the strongest overall performance. We note that, because this scenario features a more concrete and complex semantic environment, there is an abundance of relevant linguistic material that can be leveraged for indirect expression. Consequently, under a similar number of turns, the overall accuracy here is noticeably higher compared to the password game.
Nevertheless, performance fluctuations persist. In particular, in the accuracy plot, \textit{deepseekV2} experiences a pronounced increase in accuracy after the 30th round, while \textit{gpt-4o}’s accuracy declines during the same period. As a result, \textit{deepseekV2} ultimately surpasses \textit{gpt-4o}’s accuracy in the final rounds of the experiment.

\begin{table}[h!]
    \centering
    \caption{Performance of Different LLMs in Illicit Pet Trade}
    \label{tab:sce2}
    \renewcommand{\arraystretch}{1.2} % 调整行高
    \begin{tabular}{l S S S}
        \toprule
        \textbf{Model} & \textbf{Total Turns} & \textbf{Avg. Entropy} & \textbf{Avg. Distinct-1} \\
        \midrule
        \rowcolor{gray!10} \textbf{gpt-4o}       & 136.2  & 6.856  & 0.471 \\
        \textbf{gpt-4o-mini}  & 74.4  & 6.595  & 0.387 \\
        \rowcolor{gray!10} \textbf{deepseekV2} & 65.2   & 6.255  & 0.338 \\
        \textbf{qwen-turbo}   & 50.5   & 5.891  & 0.461 \\
        \bottomrule
    \end{tabular}
\end{table}
Table \ref{tab:sce2} presents the performance of various LLMs in the illicit pet trade scenario, measured by total turns, average agent entropy, and Distinct-1. As in the password game, \textit{gpt-4o} maintains a notable lead in total turns (136.2) while also displaying relatively high entropy (6.856) and Distinct-1 (0.471). In contrast, \textit{gpt-4o-mini} reaches roughly half as many total turns (74.4), despite having a comparable entropy score (6.595). Meanwhile, \textit{deepseekV2} (65.2) and \textit{qwen-turbo} (50.5) trail further behind in total turns. Consistent with the results shown in Table 
\ref{tab:sce1}, \textit{qwen-turbo} again achieves a high Distinct-1 score, which we speculate may be linked to its training corpus: it includes extensive data from the Chinese internet, likely giving it an advantage in a Chinese-language environment over more internationally oriented models.

Notably, the range of entropy scores in this scenario—spanning from 5.891 (\textit{qwen-turbo}) to 6.856 (gpt-4o)—is narrower than in the password game (see Table \ref{tab:sce1}), reflecting the more concrete nature of the illicit pet trade setting. This scenario provides richer contextual cues for indirect references, enabling all models to maintain higher semantic complexity. However, as was the case in the password game, having a broader vocabulary or greater unpredictability alone does not guarantee extended evasion: models must integrate their linguistic variety into strategic planning to circumvent regulatory scrutiny, a balance that \textit{gpt-4o} continues to manage most effectively.

\setlength{\fboxrule}{0.5pt} 
\vspace{0.5em}
\noindent
\begin{tcolorbox}[colframe=black!20, colback=gray!10, arc=5pt, boxrule=0.5pt, width=0.99\linewidth]
\textit{Answer to RQ1}: Experimental results indicate that participant agents in our framework progressively improve their ability to evade regulation violation detection through continuous interaction, leading to longer uninterrupted dialogue sequences. Concurrently, the accuracy of information transmission gradually increases over successive rounds, demonstrating that the evolved strategies effectively balance evasion with precise communication.
Moreover, different models also exhibit varying results. For example, \textit{gpt-4o} performs most outstandingly in extending dialogue turns and maintaining language complexity (i.e., high entropy and lexical diversity), while other models such as \textit{gpt-4o-mini}, \textit{deepseekV2}, and \textit{qwen-turbo} demonstrate different fluctuations and localized advantages at different stages.
\end{tcolorbox}

\subsection{RQ2: Human Interpretation}
\subsubsection{Experiment Settings}

To investigate the real-world relevance of both the evolved language strategies and the resulting dialogue, we conducted a human evaluation on a subset of successful dialogue records from the password game and illicit pet trade scenario. The dialogues generated by the \textit{gpt-4o} models are randomly selected, and 40 human participants participated in the experiment to evaluate them. The 40 human reviewers had an average age of approximately 27 (SD = 4). In terms of gender, 75\% of the human reviewers were male, and 25\% were female. Regarding educational background, 67.5\% held a bachelor's degree, 27.5\% held a master's degree or above, and 5\% had an associate degree or lower. All dialogue records were presented in Simplified Chinese.

Each participant rated each dialogue on a 5-point Likert scale on the following five metrics:
\begin{itemize}
    \item Explicit Understanding: Evaluates how effectively the dialogue’s explicit meaning is communicated (1: Extremely vague and confusing; 3: Moderately clear, but some parts may require further interpretation; 5: Crystal clear and precise).
    \item Implicit Understanding: Assesses the reader's ability to grasp the underlying or unstated messages (1: Nearly indecipherable subtext; 3: Some underlying meaning is apparent, but requires effort to fully grasp; 5: Subtext that is immediately apparent).
    \item Realistic Significance: Measures the extent to which the dialogue reflects real-life situations and holds practical relevance (1: Highly unrealistic with little relevance; 3: Generally realistic, though some elements may not align with real-world situations; 5: Deeply rooted in real-world context).
    \item Regulatory Avoidance: Examines the effectiveness of the strategies in evading regulation violation (1: Blatantly ineffective and easily spotted; 3: Partially effective, with the potential for detection in some cases; 5: Exceptionally subtle and effective).
    \item Strategy Existence: Determines how plausible it is for such strategies to be observed in practical, real-world scenarios (1: Extremely implausible; 3: Fairly believable, though may seem impractical in specific situations; 5: Entirely plausible).
\end{itemize}



\subsubsection{Experiment Results}
\begin{figure}
    \centering
    \includegraphics[width=0.9\linewidth]{figures/user_study_v4.png}
    \caption{Box plots of user study scores across different metrics in two scenarios. The red x symbol denotes the mean value.}
    \label{fig:case_study}
\end{figure}
As shown in Fig.\ref{fig:case_study}, our framework consistently achieves average scores of 3.4 or above across most indicators (such as explicit understanding and implicit understanding). This suggests that, both in terms of the generated dialogues and the underlying strategies, it possesses valuable practical applicability.

%Although there are a few exceptions, compared with the old framework (\textit{w/o GA, gpt-4o}), the new version (\textit{w/ GA, gpt-4o}) demonstrates overall advantages in both average scores and score distributions. In the comparison between different versions, under the more realistic illicit pet trade scenario, the new framework shows distinct benefits over the old one in both “regulatory avoidance” and “strategy existence”—both in distribution and mean values. This finding indicates that introducing a genetic algorithm, particularly a fitness‐based strategy selection mechanism, makes strategy adoption more efficient and stable. As for the password game, we speculate that the main reason these two metrics do not show a large distributional gap is that, in an abstract scenario, the range of available strategies is broader.

Comparing distributions between the password game and the illicit pet trade scenario reveals some interesting phenomena. Focusing on “realistic significance” and “regulatory avoidance,” the more abstract password game often yields higher mean values than the more concrete illicit pet trade scenario, while also exhibiting lower dispersion. We speculate this is related to the inherently abstract nature of numeric information: encryption and covert hints can be harder to detect in such contexts, and the growing tendency on Chinese internet platforms to use abstract language \cite{Wu2025HighEnergy} may lead reviewers to have a higher acceptance of “obscure” expressions. Conversely, the illicit pet trade scenario, despite being closely tied to real-world transactions, may suffer if the indirect or euphemistic methods in the dialogues are insufficiently subtle. Human reviewers can find them conspicuous or “forced,” potentially causing lower scores for “realistic significance” and “regulatory avoidance” in terms of both distribution and mean values.
A significant portion of these results can be attributed to inherent biases in commercial LLMs, such as ChatGPT, introduced during their training phases. These general-purpose models undergo fine-tuning via RLHF to align with specific product positioning, which often results in a more standard and safe output style. However, this characteristic poses a limitation for our simulation framework, as it may hinder the model’s ability to capture the nuanced and unconventional expressions typical of online social interactions. Ideally, fine-tuning datasets that are more representative of social platforms could lead to improved performance in our simulations.


In the abstract password game, for instance, a typical conversation might go like this:
\begin{quote}
\textit{
“I've really grown fond of a certain phase of the moon. It's not the brightest or the darkest, but it always carries its own charm. It symbolizes ...... In that green oasis, I felt as if I were catching a glimpse of the golden hues of autumn leaves, much like the soft, warm glow of dusk—calm and serene ...” 
}
\end{quote}
Here, words like “lunar cycle” and “autumn leaves” can subtly hint at larger or smaller digits, or use seasonal imagery to convey key information. Since these references lack an obvious connection, they lend a more literary feel to the dialogue and, to some extent, raise the bar for recognition and detection.

By contrast, in a more concrete setting like illicit pet trade, example conversations may be closer to real‐life buying and selling procedures, which can make them appear more “suspicious”:
\begin{quote}
\textit{
“... about a vibrant 'tropical chatterbird' renowned for its brilliant plumage and uncanny mimicry ... I've also come into possession of a few 'Rising Sun coins' for exchange ...... Perhaps you might know a place where ...”
}
\end{quote}
In this dialogue, the term “tropical chatterbird” serves as an euphemism for a parrot, emphasizing its colorful appearance and mimicking ability without mentioning the animal directly. Meanwhile, “Rising Sun tokens” subtly alludes to the Japanese yen, since the Rising Sun is an iconic symbol of Japan. This coded language allows both parties to communicate their intentions regarding the acquisition of a rare bird and the intended payment method without explicitly revealing sensitive details. However, if these indirect expressions are used excessively, the dialogue may appear artificial or unnatural, potentially reducing its authenticity—thus affecting evaluations of both “regulatory avoidance” and “strategy existence.”
\setlength{\fboxrule}{0.5pt} 
\vspace{0.5em}
\noindent
\begin{tcolorbox}[colframe=black!20, colback=gray!10, arc=5pt, boxrule=0.5pt, width=0.99\linewidth]
\textit{Answer to RQ2}: Our evaluation confirms that the emergent language strategies closely resemble real-world language strategies, effectively employing euphemisms and implicit cues, and are generally understood by human reviewers. However, while these strategies show potential in simulations, they often appear forced or unnatural due to the fine-tuning of LLMs as commercial products, requiring refinement to better mimic the nuanced and fluid communication typical in real-world social interactions.

\end{tcolorbox}

\subsection{RQ3: Ablation Experiment}
\subsubsection{Experiment Settings}

To evaluate the effectiveness of the GA introduced in our framework, we conducted an ablation experiment using \textit{gpt-4o-mini} and \textit{gpt-4o} as the underlying LLM. For comparison, we employed the approach from our initial study \cite{DBLP:conf/cec/CaiLZLWT24}, which primarily differs in its strategy-update mechanism. In that earlier framework, the LLM is provided with both the existing strategy and newly flagged regulation violation records during the reflection stage, prompting the model to propose a new set of strategies that replace the old ones.
In contrast, our new framework employs a GA process where each strategy is treated as a discrete unit and optimized iteratively through GA. 

\subsubsection{Experiment Results}
As shown in Fig.~\ref{fig:ablation}, the GA-based framework demonstrates significant advantages. In the short-term experiment within the first 35 rounds, the w/o GA approach might show slight initial superiority due to the larger changes brought about by replacing the entire strategy. However, overall, w/ GA performs better than w/o GA. This difference increases as the number of rounds grows, particularly after round 35, where the advantages of w/ GA become even more pronounced. The GA process enables effective strategy evolution and adaptation, leading to an increased number of dialogue turns and improved accuracy, highlighting the framework's enhanced adaptability in the long term.
%Despite occasional performance dips during the evolutionary process, the GA framework’s ability to foster strategy diversity and handle complex scenarios makes it a more effective approach for sustained optimization.
\begin{figure}[h!]
    \centering
    \includegraphics[width=\linewidth]{figures/ablation1_v6.png}
    \caption{Performance with/without GA}
    \label{fig:ablation}
\end{figure}
\setlength{\fboxrule}{0.5pt} 
\vspace{0.5em}
\noindent
\begin{tcolorbox}[colframe=black!20, colback=gray!10, arc=5pt, boxrule=0.5pt, width=0.99\linewidth]
\textit{Answer to RQ3}: The results confirm the effectiveness of the GA component in our framework, especially when the number of rounds increases, where it demonstrates greater stability and adaptability. Although the optimization may be slower in the early stages, GA provides stronger adaptability in the long term through effective strategy evolution.
\end{tcolorbox}

\subsection{Discussion and Limitation}
In this study, we leveraged LLM agents to simulate the evolution of language strategies under regulatory pressure. While our results provide initial evidence that agents can adapt and develop covert communication tactics, the simulations also exhibit noteworthy instabilities. First, the inherent randomness of LLM generation can cause significant fluctuations in outcomes: the same prompts may yield different strategic responses, particularly when the experimental scale (number of agents or dialogue rounds) is limited. In our framework, LLMs not only generate dialogues but also determine strategies and regulatory responses; as a result, any stochasticity is compounded across multiple modules, making the final results sensitive to small variations in prompt inputs or random seeds. Although such variability partially reflects the diversity of real-world human behavior to some extent, it complicates the interpretation of findings in a controlled experimental setup.

A second limitation lies in the relatively narrow scope of language strategies observed. The agents predominantly relied on general-purpose evasive methods, such as analogies or implicit references, yet rarely produced fully “encrypted” or specialized code words that might arise in realistic cultural or social contexts. This outcome highlights the challenge that LLMs, pre-trained on broad domains and further refined via RLHF, are predisposed to generate text consistent with mainstream norms, thereby inhibiting the formation of highly unconventional or obscure expressions. Moreover, in scenarios where the training corpus lacks sufficient examples of subcultural or community-specific covert language, the model is less able to invent or adopt specialized linguistic forms. 

Finally, our experiments focused on one-to-one private interactions that emphasize regulatory evasion, without exploring the dynamics of public, many-to-many conversations where language strategies might evolve and propagate differently in a broader social context. While each participant agent does learn and adapt incrementally across dialogue rounds, real-world language evolution involves extensive, long-term propagation across diverse communities. Covert terms or code words may gradually gain acceptance, be modified by different user groups, or fade from use entirely. By contrast, the small-scale nature of our simulated dialogues means that emergent language strategies do not undergo the sustained diffusion and feedback processes characteristic of real social platforms, limiting the ecological validity of our findings.




%对于语言演化的社会类模拟仍然是一个未被开拓的领域,通过借助LLM优秀的自然语言处理能力,为这类自然语言的模拟带来了强大助力。然而伴随着实验也让我们发现LLM也会导致许多局限性。尽管通过实验初步证明了我们的框架的有效性。但同时伴随着实验也为我们带来了许多值得讨论的点。

%实验结果的不稳定性
%首先实验结果本身具有一定的不稳定性,而我们认为整个不稳定性的根源源自于LLM本身生成具有不确定性\cite{},在我们的框架中,LLM几乎参与到了所有环节。同样的violation log让同一个LLM在相同的设置内可能会总结出不同的constraint strategy。尽管这种不稳定性在现实中同样存在(例如不同的人采取不同的策略),同时也是作为模拟框架中非常重要的点,然而在本工作中的数量级的实验中这种不稳定性对结果的影响更为难以过滤。就像\ref{}中也证实的,这种LLM dirven agent的研究中在小数量级上的实验存在着不稳定性,我们认为目前的结果已经足够证实我们的框架可以初步模拟语言动态的学习和演化这一趋势,在今后工作中更大量级的实验中(例如数万数百万agent于更多的round数),我们有理由相信,整体趋势会更加稳定,不同llm的agent之间的性能差距会更加接近llm本身语义理解与生成的综合性能,

%模拟策略的局限性,
%从实验中我们观察到,agent模拟出的策略目前仍然主要集中于比喻类比等较为共通的方式。现实中语言的演化一般根植于当地的文化与经济背景等等因素。例如中文可以利用拼音来将汉字转化为对应的字母从而规避审查,而英文可能会更加积极的利用emoji来作为表达的替代从而规避监管。
%这些较为复杂的策略不仅需要对应环境的大量先验知识,在较为常见的语言中,LLM中训练所需的语料知识可能包含了这些,但是对于训练的数据集中欠缺的语种的知识LLM在不借助prompt的提示的情况下没有能力选择这些既存的策略。


%尽管LLM训练中的数据集可能存在这种更为隐晦的表达方式,首先LLM的RLHF\ref{}本身的训练方法导致了目前绝大多数的LLM为了保证生成文本的泛用性,被训练的更加愿意生成更符合大众的一般化输出文本,在不对LLM进行微调的前提下很难提高在这种特性领域的表现。
%LLM的表现严重依赖prompt的结构设计,提示词工程已经被证明可以有效提高LLM的某一方面能力,单次的基于prompt的模型交互很难实现多步推理或是规划。尽管我们的框架已经将语言演化这一现象解耦,通过多个模块来尽可能模拟人类在该环境中内在的动力学,但是目前的策略生成阶段
%这一部分在不适用复杂prompt工程的前提下LLM很难采用这种小众?特殊领域?的表达。
%对于模拟出的语言策略,我们发现很少的独特加密语言,因为这种需要两边有一套共用的体系,对于我们的模拟情景只有固定turn数的模拟很难形成意思传达。


%\jialong{第二是演化后的语言是如何的存活。我们只考虑了能不能躲避监管。但语言后续的存活和发展其实是更大范围的society的一个动态过程(而不是几个agent之间的交互),这一块可以结合那些上千LLM agent的研究框架来进行拓展}
%\jialong{这边可以多用语言学的角度来说不足之处}
%\jialong{第一个缺点是语言演化一般根植于根植于文化,经济背景,当地的文化背景。但我们的文章没有考虑特定文化背景下的演化。例如中文中可以借用拼音与汉字之间的关系来作为回避监管的方式,日语则可以通过XXX,英语则可以通过XXXX。未来可能要借助persona和role-play之类的设定来进一步拓展}

%更大规模的实验
%策略生成那里增加多步规划
%RAG提供更多语料
%


\section{Discussions}

% \subsection{Bridge the gap between insights and expressions}



\noindent\textbf{Bridge the gap between insights and expressions with AI-powered domain-focused video creation.}
% video creation for different domains
As images and videos continue to dominate communication mediums, visualization and video technologies have become essential tools for enabling diverse domains and the public to express themselves effectively. Emerging generative AI tools, such as Sora~\cite{sora} and Pika~\cite{pika}, exemplify this trend by facilitating creative expression across various fields.

While general AI-driven video creation tools are increasingly popular, our work emphasizes the critical need for domain-specific video creation tools like \SB{} to address unique requirements within specific fields. There are two primary reasons for prioritizing domain-specific video creation over general generative technologies.
% 
First, domain-specific videos, such as sports highlights, rely heavily on human insights. Audiences seek to learn from professionals through these videos, requiring tools that provide greater user control and enable experts to effectively translate their insights into engaging content. 
% \SB{} supports this by enabling users to maintain control over the conveyed insights, ensuring that the final video accurately reflects expert knowledge and user intentions.
% 
Second, the complexity of domain-specific data, such as the intricate motion and strategy analysis, demands advanced data visualization and seamless synchronization of visuals and audio, which general tools may not provide. 
% \SB{} addresses these needs by providing specialized tools that cater to the detailed and dynamic nature of sports content.

\SB{} addresses these needs by integrating automation with customizable visualizations, tailored to the intricate and dynamic nature of sports content. It allows flexible user control through embedded interactions, 
reducing technical barriers and empowering users to effectively communicate their insights. Feedback from users further underscores the importance of balancing automation with user control to accommodate diverse goals and preferences to enhance accessibility across various user groups and use cases, such as tactical analysis, skill development, and profile building. 
% For instance, professional coaches can use \SB{} to create detailed breakdowns of game strategies for training and coaching. Parents and young athletes can produce polished highlight reels for recruitment.
% These examples illustrate how AI-driven tools can empower users across various levels and industries to create videos with meaningful insights, fostering deeper engagement and broader impact. 

Beyond sports, similar tools have the potential to transform fields like healthcare and education, incorporating precise visual aids and step-by-step breakdowns. 
% These applications highlight the transformative potential of tailored video content in amplifying personal expression and benefiting broader audiences.
% 
Future research is required to investigate the balanced integration of AI and intuitive interface design, such as multi-modal interaction~\cite{wang2024lave}, to further advance domain-specific video creation and expression across diverse fields.
% By continuing to develop and refine domain-specific video creation tools, we can unlock new possibilities for effective communication and expression in numerous fields, ultimately bridging the gap between insights and their visual expressions.

% \subsection{Cross sports visualizations - allow different sports domains to leverage other sports' insights}

% \subsection{Enhance human-AI collaboration - creators focus on content while AI helps with editing tasks}


\vspace{1mm}
\noindent\textbf{Promote visualization in practice through real-world system deployment.}
Our work on SportsBuddy advances existing research in sports visualization and video authoring by emphasizing real-world system deployment and evaluation. Through this study, we have identified two significant benefits.

First, deploying SportsBuddy in authentic environments allowed us to validate and refine our design based on genuine use cases and users, uncovering insights that controlled laboratory settings cannot capture. For instance, we discovered that even within a similar user group of content creators, priorities varied significantly—some focused on showcasing player actions, while others emphasized strategic communication. This diversity led to iterative design improvements that balanced the distinct needs of each user group and support customization without complicating user interactions. 

Second, real-world deployment enables the assessment of long-term impacts and the discovery of unique use cases by diverse users. 
For example, some sports experts were hesitant to adopt SportsBuddy initially despite the perceived usefulness they shared. Upon further investigation, this was due to the context-switching costs. This feedback highlighted the necessity for a streamlined workflow tailored to the sports domain, leading to our design of batch processing and web import options. In addition, we observed many users preferred embedded annotation with \Text{} features over typical captions for sharing insights (see Fig.~\ref{fig:case_study}d), suggesting a new form of video storytelling inspired by \SB{}’s design. 
Feedback and insights from our diverse user base has highlighted the value of creating flexible and accessible visualization tools, which offers important external validity of the human-centered system.

This real-world deployment approach not only enhances visualization literacy and accessibility but also ensures that innovative designs translate into practical, widely usable tools, providing a validation for interactive visualization design. Therefore, we advocate for more visualization research to focus on real-world system deployments and to share design learnings, inspiring use cases that are both practical and impactful.

{
\subsection{Future Work}

While SportsBuddy has shown great potential in simplifying sports video storytelling, 
there are key areas for further improvement:

\vspace{1mm}
\noindent\textbf{Enhancing Player Tracking Under Occlusion and Motion Changes.}
The current tracking system faces challenges with occlusions and rapid motion in dynamic scenarios. Future work will refine tracking algorithms using larger domain-specific datasets and multi-view setups to improve accuracy in complex environments.

% The current tracking system struggles with occlusions and rapid motion changes in crowded or dynamic scenarios. Future efforts will focus on refining tracking algorithms using more extensive domain-specific datasets and, where feasible, incorporating multi-view camera setups for improved accuracy. These enhancements aim to ensure reliable tracking in complex sports environments.

\vspace{1mm}
\noindent\textbf{Addressing Perspective and Camera Movement.}
Shifts in camera angles or perspectives cause misalignment issues due to reliance on fixed transformation matrices. Dynamic court mapping and machine learning for real-time adjustments, along with camera metadata integration, will ensure consistent and accurate visualizations.

% Misalignment issues arise when camera angles or perspectives shift, as the system relies on a fixed transformation matrix. Future work will explore dynamic court mapping techniques and machine learning methods for real-time adjustments. Incorporating camera metadata will further enhance visualization accuracy, ensuring effects remain consistent with the game’s context.

\vspace{1mm}
\noindent\textbf{Supporting Longer Videos.}
Longer or higher-resolution videos can strain browser performance. To mitigate this, we will implement dynamic video loading from cloud storage and on-demand decoding, and adopt frame compression during previews to further optimize memory usage and rendering, ensuring smoother video processing.
% Longer or higher-resolution videos may strain browser performance. To address this, dynamic video loading from cloud storage and on-demand decoding will be introduced. Additionally, frame compression during previews will reduce memory usage and rendering time, enabling smoother processing of large and complex videos.



\vspace{1mm}
\noindent\textbf{Extending to Other Sports.}
\SB{} currently focuses on basketball but can expand to sports like soccer and tennis. This requires adapting tracking algorithms and designing sport-specific visualizations to accommodate the unique dynamics and storytelling needs of each sport.

}


% We advocate for more visualization paper that focus on deplyong system in real-world and evaluate their usage for two reasons. 
% 1. In vis research, application paper often address specific domain problems and create a prototype to evaluate with domain experts in a controlled setting. Most projects stop after user evaluation in the lab and the paper is published. With visualization system in real-world that value the practicality of system design and deployment in the wild, it encourages promoting real-world impact brought by novel visualization design, which is crucial in the current visualization community as we promote literacy and accessiblity of visualizations.
% 2. we should also promote long term impact of visualization design, and identify real-wordl use case and learning that might be drastically different from design study that are typically in lab, with a small amount of users, typically university students or academic members.


\section{Conclusion}

We presented \sys, a sparsity-adaptive attention mechanism for efficient long-context LLM inference. Unlike fixed token budget methods, \sys dynamically selects tokens based on cumulative attention scores, adapting to variations in attention sparsity. By leveraging clustering-based sorting and distribution fitting, \sys accurately estimates token importance with low overhead. Our results showed that \sys outperforms existing sparse attention methods, achieving higher accuracy and significant inference speedups, making it a practical solution for long-context LLMs.

\section*{Acknowledgements}
This is acknowledgment.

%% if specified like this the section will be committed in review mode
% \acknowledgments{
% The authors wish to thank A, B, C. This work was supported in part by
% a grant from XYZ.}

%\bibliographystyle{abbrv}
\bibliographystyle{abbrv-doi}
%\bibliographystyle{abbrv-doi-narrow}
%\bibliographystyle{abbrv-doi-hyperref}
%\bibliographystyle{abbrv-doi-hyperref-narrow}

\bibliography{template}
\end{document}
