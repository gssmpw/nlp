% $Id: template.tex 11 2007-04-03 22:25:53Z jpeltier $

\documentclass{vgtc}                          % final (conference style)
% \documentclass[review]{vgtc}                 % review
%\documentclass[widereview]{vgtc}             % wide-spaced review
%\documentclass[preprint]{vgtc}               % preprint
%\documentclass[electronic]{vgtc}             % electronic version

%% Uncomment one of the lines above depending on where your paper is
%% in the conference process. ``review'' and ``widereview'' are for review
%% submission, ``preprint'' is for pre-publication, and the final version
%% doesn't use a specific qualifier. Further, ``electronic'' includes
%% hyperreferences for more convenient online viewing.

%% Please use one of the ``review'' options in combination with the
%% assigned online id (see below) ONLY if your paper uses a double blind
%% review process. Some conferences, like IEEE Vis and InfoVis, have NOT
%% in the past.

%% Figures should be in CMYK or Grey scale format, otherwise, colour 
%% shifting may occur during the printing process.

%% These few lines make a distinction between latex and pdflatex calls and they
%% bring in essential packages for graphics and font handling.
%% Note that due to the \DeclareGraphicsExtensions{} call it is no longer necessary
%% to provide the the path and extension of a graphics file:
%% \includegraphics{diamondrule} is completely sufficient.
%%
\ifpdf%                                % if we use pdflatex
  \pdfoutput=1\relax                   % create PDFs from pdfLaTeX
  \pdfcompresslevel=9                  % PDF Compression
  \pdfoptionpdfminorversion=7          % create PDF 1.7
  \ExecuteOptions{pdftex}
  \usepackage{graphicx}                % allow us to embed graphics files
  \DeclareGraphicsExtensions{.pdf,.png,.jpg,.jpeg} % for pdflatex we expect .pdf, .png, or .jpg files
\else%                                 % else we use pure latex
  \ExecuteOptions{dvips}
  \usepackage{graphicx}                % allow us to embed graphics files
  \DeclareGraphicsExtensions{.eps}     % for pure latex we expect eps files
\fi%

%% it is recomended to use ``\autoref{sec:bla}'' instead of ``Fig.~\ref{sec:bla}''
\graphicspath{{figures/}{pictures/}{images/}{./}} % where to search for the images
\usepackage{orcidlink}
\usepackage{microtype}                 % use micro-typography (slightly more compact, better to read)
\PassOptionsToPackage{warn}{textcomp}  % to address font issues with \textrightarrow
\usepackage{xcolor}
\usepackage{textcomp}                  % use better special symbols
\usepackage{mathptmx}                  % use matching math font
\usepackage{times}                     % we use Times as the main font
\renewcommand*\ttdefault{txtt}         % a nicer typewriter font
\usepackage{cite}                      % needed to automatically sort the references
\usepackage{tabu}                      % only used for the table example
\usepackage{booktabs}                  % only used for the table example
%% We encourage the use of mathptmx for consistent usage of times font
%% throughout the proceedings. However, if you encounter conflicts
%% with other math-related packages, you may want to disable it.
\usepackage{enumitem}
\usepackage{changepage}

%% If you are submitting a paper to a conference for review with a double
%% blind reviewing process, please replace the value ``0'' below with your
%% OnlineID. Otherwise, you may safely leave it at ``0''.
\onlineid{0}

%% declare the category of your paper, only shown in review mode
\vgtccategory{Research}

%% allow for this line if you want the electronic option to work properly
\vgtcinsertpkg

%% In preprint mode you may define your own headline.
%\preprinttext{To appear in an IEEE VGTC sponsored conference.}

%% Paper title.
% SportsBuddy: Enhancing Sports Video Storytelling through Multimodal AI and Interactive Authoring 

\title{SportsBuddy: Designing and Evaluating an AI-Powered Sports Video Storytelling Tool Through Real-World Deployment}
%% This is how authors are specified in the conference style

%% Author and Affiliation (single author).
%%\author{Roy G. Biv\thanks{e-mail: roy.g.biv@aol.com}}
%%\affiliation{\scriptsize Allied Widgets Research}


%% Author and Affiliation (multiple authors with multiple affiliations)
\author{
Tica Lin \orcidlink{0000-0002-2860-0871} \thanks{
\textit{
Tica Lin and Ruxun Xiang contributed equally to this work.}
\\ 
\hangindent=4mm \textbullet\ \textit{T. Lin (mlin@g.harvard.edu), R. Xiang (ruxunx@g.harvard.edu), M. Chiang, C. Ye were affiliated with Harvard University during this work. G. Liu and H. Pfister are with Harvard University. }
\\
\hangindent=4mm \textbullet\ \textit{D. Tiwari and Z. Chen (ztchen@umn.edu) are with University of Minnesota. }
\vspace{-1cm}
} ,
Ruxun Xiang \orcidlink{0009-0000-3257-4859} \footnotemark[1] , 
Gardenia Liu \orcidlink{0009-0002-5442-9788}, 
Divyanshu Tiwari \orcidlink{0000-0003-4279-8332}, 
Meng-Chia Chiang, 
Chenjiayi Ye \orcidlink{0009-0006-0030-9296}, \\
Hanspeter Pfister \orcidlink{0000-0002-3620-2582}, Chen Zhu-Tian \orcidlink{0000-0002-2313-0612} 
}




%% A teaser figure can be included as follows, but is not recommended since
%% the space is now taken up by a full width abstract.
\teaser{
\vspace{-2mm}
\begin{center}
  \includegraphics[width=\linewidth, alt={}]{pictures/teaser.png}
  \caption{%
  	\SB{} enables easy highlight creation for video storytelling with intuitive visualization features and workflow. Users can (a) manage clips in \Media{} and choose from \Highlight{} and \Narrative{} features; (b) interact directly with the video canvas to apply features, adjust effects, and edit using the video timeline; and (c) share the final highlight via link or file. 
  }
  \label{fig:teaser}
  \end{center}
}

%% Abstract section.
\abstract{
Video storytelling is essential for sports performance analysis and fan engagement, enabling sports professionals and fans to effectively communicate and interpret the spatial and temporal dynamics of gameplay. Traditional methods rely on manual annotation and verbal explanations, placing significant demands on creators for video editing skills and on viewers for cognitive focus. However, these approaches are time-consuming and often struggle to accommodate individual needs. SportsBuddy addresses this gap with an intuitive, interactive video authoring tool. It combines player tracking, embedded interaction design, and timeline visualizations to seamlessly integrate narratives and visual cues within game contexts. This empowers users to effortlessly create context-driven video stories. Since its launch, over 150 sports users, including coaches, athletes, content creators, parents and fans, have utilized SportsBuddy to produce compelling game highlights for diverse use cases. User feedback highlights its accessibility and ease of use, making video storytelling and insight communication more attainable for diverse audiences. Case studies with collegiate teams and sports creators further demonstrate SportsBuddy’s impact on enhancing coaching communication, game analysis, and fan engagement.
} % end of abstract

%% ACM Computing Classification System (CCS). 
%% See <http://www.acm.org/class/1998/> for details.
%% The ``\CCScat'' command takes four arguments.

% \CCScatlist{ 
%   \CCScat{K.6.1}{Management of Computing and Information Systems}%
% {Project and People Management}{Life Cycle};
%   \CCScat{K.7.m}{The Computing Profession}{Miscellaneous}{Ethics}
% }

%% Copyright space is enabled by default as required by guidelines.
%% It is disabled by the 'review' option or via the following command:
% \nocopyrightspace

%%%%%%%%%%%%%%%%%%%%%%%%%%%%%%%%%%%%%%%%%%%%%%%%%%%%%%%%%%%%%%%%
%%%%%%%%%%%%%%%%%%%%%% START OF THE PAPER %%%%%%%%%%%%%%%%%%%%%%
%%%%%%%%%%%%%%%%%%%%%%%%%%%%%%%%%%%%%%%%%%%%%%%%%%%%%%%%%%%%%%%%%

\begin{document}

\newcommand{\SB}{SportsBuddy}
\newcommand{\Media}{\texttt{Media}}
\newcommand{\Highlight}{\texttt{Highlight}}
\newcommand{\Narrative}{\texttt{Narrative}}
\newcommand{\Circle}{\textit{Circle}}
\newcommand{\Spotlight}{\textit{Spotlight}}
\newcommand{\Connector}{\textit{Connector}}
\newcommand{\Path}{\textit{Path}}
\newcommand{\Zone}{\textit{Zone}}
\newcommand{\Marker}{\textit{Marker}}
\newcommand{\BGFilter}{\textit{BG Filter}}
\newcommand{\Zoom}{\textit{Zoom In}}
\newcommand{\Text}{\textit{Text}}
\newcommand{\Caption}{\textit{Caption}}
\newcommand{\revision}[1]{\textcolor{black}{#1}}
%% The ``\maketitle'' command must be the first command after the
%% ``\begin{document}'' command. It prepares and prints the title block.

%% the only exception to this rule is the \firstsection command
\firstsection{Introduction}

\maketitle

\section{Introduction}
Backdoor attacks pose a concealed yet profound security risk to machine learning (ML) models, for which the adversaries can inject a stealth backdoor into the model during training, enabling them to illicitly control the model's output upon encountering predefined inputs. These attacks can even occur without the knowledge of developers or end-users, thereby undermining the trust in ML systems. As ML becomes more deeply embedded in critical sectors like finance, healthcare, and autonomous driving \citep{he2016deep, liu2020computing, tournier2019mrtrix3, adjabi2020past}, the potential damage from backdoor attacks grows, underscoring the emergency for developing robust defense mechanisms against backdoor attacks.

To address the threat of backdoor attacks, researchers have developed a variety of strategies \cite{liu2018fine,wu2021adversarial,wang2019neural,zeng2022adversarial,zhu2023neural,Zhu_2023_ICCV, wei2024shared,wei2024d3}, aimed at purifying backdoors within victim models. These methods are designed to integrate with current deployment workflows seamlessly and have demonstrated significant success in mitigating the effects of backdoor triggers \cite{wubackdoorbench, wu2023defenses, wu2024backdoorbench,dunnett2024countering}.  However, most state-of-the-art (SOTA) backdoor purification methods operate under the assumption that a small clean dataset, often referred to as \textbf{auxiliary dataset}, is available for purification. Such an assumption poses practical challenges, especially in scenarios where data is scarce. To tackle this challenge, efforts have been made to reduce the size of the required auxiliary dataset~\cite{chai2022oneshot,li2023reconstructive, Zhu_2023_ICCV} and even explore dataset-free purification techniques~\cite{zheng2022data,hong2023revisiting,lin2024fusing}. Although these approaches offer some improvements, recent evaluations \cite{dunnett2024countering, wu2024backdoorbench} continue to highlight the importance of sufficient auxiliary data for achieving robust defenses against backdoor attacks.

While significant progress has been made in reducing the size of auxiliary datasets, an equally critical yet underexplored question remains: \emph{how does the nature of the auxiliary dataset affect purification effectiveness?} In  real-world  applications, auxiliary datasets can vary widely, encompassing in-distribution data, synthetic data, or external data from different sources. Understanding how each type of auxiliary dataset influences the purification effectiveness is vital for selecting or constructing the most suitable auxiliary dataset and the corresponding technique. For instance, when multiple datasets are available, understanding how different datasets contribute to purification can guide defenders in selecting or crafting the most appropriate dataset. Conversely, when only limited auxiliary data is accessible, knowing which purification technique works best under those constraints is critical. Therefore, there is an urgent need for a thorough investigation into the impact of auxiliary datasets on purification effectiveness to guide defenders in  enhancing the security of ML systems. 

In this paper, we systematically investigate the critical role of auxiliary datasets in backdoor purification, aiming to bridge the gap between idealized and practical purification scenarios.  Specifically, we first construct a diverse set of auxiliary datasets to emulate real-world conditions, as summarized in Table~\ref{overall}. These datasets include in-distribution data, synthetic data, and external data from other sources. Through an evaluation of SOTA backdoor purification methods across these datasets, we uncover several critical insights: \textbf{1)} In-distribution datasets, particularly those carefully filtered from the original training data of the victim model, effectively preserve the model’s utility for its intended tasks but may fall short in eliminating backdoors. \textbf{2)} Incorporating OOD datasets can help the model forget backdoors but also bring the risk of forgetting critical learned knowledge, significantly degrading its overall performance. Building on these findings, we propose Guided Input Calibration (GIC), a novel technique that enhances backdoor purification by adaptively transforming auxiliary data to better align with the victim model’s learned representations. By leveraging the victim model itself to guide this transformation, GIC optimizes the purification process, striking a balance between preserving model utility and mitigating backdoor threats. Extensive experiments demonstrate that GIC significantly improves the effectiveness of backdoor purification across diverse auxiliary datasets, providing a practical and robust defense solution.

Our main contributions are threefold:
\textbf{1) Impact analysis of auxiliary datasets:} We take the \textbf{first step}  in systematically investigating how different types of auxiliary datasets influence backdoor purification effectiveness. Our findings provide novel insights and serve as a foundation for future research on optimizing dataset selection and construction for enhanced backdoor defense.
%
\textbf{2) Compilation and evaluation of diverse auxiliary datasets:}  We have compiled and rigorously evaluated a diverse set of auxiliary datasets using SOTA purification methods, making our datasets and code publicly available to facilitate and support future research on practical backdoor defense strategies.
%
\textbf{3) Introduction of GIC:} We introduce GIC, the \textbf{first} dedicated solution designed to align auxiliary datasets with the model’s learned representations, significantly enhancing backdoor mitigation across various dataset types. Our approach sets a new benchmark for practical and effective backdoor defense.



\section{Related Work}

\subsection{Large 3D Reconstruction Models}
Recently, generalized feed-forward models for 3D reconstruction from sparse input views have garnered considerable attention due to their applicability in heavily under-constrained scenarios. The Large Reconstruction Model (LRM)~\cite{hong2023lrm} uses a transformer-based encoder-decoder pipeline to infer a NeRF reconstruction from just a single image. Newer iterations have shifted the focus towards generating 3D Gaussian representations from four input images~\cite{tang2025lgm, xu2024grm, zhang2025gslrm, charatan2024pixelsplat, chen2025mvsplat, liu2025mvsgaussian}, showing remarkable novel view synthesis results. The paradigm of transformer-based sparse 3D reconstruction has also successfully been applied to lifting monocular videos to 4D~\cite{ren2024l4gm}. \\
Yet, none of the existing works in the domain have studied the use-case of inferring \textit{animatable} 3D representations from sparse input images, which is the focus of our work. To this end, we build on top of the Large Gaussian Reconstruction Model (GRM)~\cite{xu2024grm}.

\subsection{3D-aware Portrait Animation}
A different line of work focuses on animating portraits in a 3D-aware manner.
MegaPortraits~\cite{drobyshev2022megaportraits} builds a 3D Volume given a source and driving image, and renders the animated source actor via orthographic projection with subsequent 2D neural rendering.
3D morphable models (3DMMs)~\cite{blanz19993dmm} are extensively used to obtain more interpretable control over the portrait animation. For example, StyleRig~\cite{tewari2020stylerig} demonstrates how a 3DMM can be used to control the data generated from a pre-trained StyleGAN~\cite{karras2019stylegan} network. ROME~\cite{khakhulin2022rome} predicts vertex offsets and texture of a FLAME~\cite{li2017flame} mesh from the input image.
A TriPlane representation is inferred and animated via FLAME~\cite{li2017flame} in multiple methods like Portrait4D~\cite{deng2024portrait4d}, Portrait4D-v2~\cite{deng2024portrait4dv2}, and GPAvatar~\cite{chu2024gpavatar}.
Others, such as VOODOO 3D~\cite{tran2024voodoo3d} and VOODOO XP~\cite{tran2024voodooxp}, learn their own expression encoder to drive the source person in a more detailed manner. \\
All of the aforementioned methods require nothing more than a single image of a person to animate it. This allows them to train on large monocular video datasets to infer a very generic motion prior that even translates to paintings or cartoon characters. However, due to their task formulation, these methods mostly focus on image synthesis from a frontal camera, often trading 3D consistency for better image quality by using 2D screen-space neural renderers. In contrast, our work aims to produce a truthful and complete 3D avatar representation from the input images that can be viewed from any angle.  

\subsection{Photo-realistic 3D Face Models}
The increasing availability of large-scale multi-view face datasets~\cite{kirschstein2023nersemble, ava256, pan2024renderme360, yang2020facescape} has enabled building photo-realistic 3D face models that learn a detailed prior over both geometry and appearance of human faces. HeadNeRF~\cite{hong2022headnerf} conditions a Neural Radiance Field (NeRF)~\cite{mildenhall2021nerf} on identity, expression, albedo, and illumination codes. VRMM~\cite{yang2024vrmm} builds a high-quality and relightable 3D face model using volumetric primitives~\cite{lombardi2021mvp}. One2Avatar~\cite{yu2024one2avatar} extends a 3DMM by anchoring a radiance field to its surface. More recently, GPHM~\cite{xu2025gphm} and HeadGAP~\cite{zheng2024headgap} have adopted 3D Gaussians to build a photo-realistic 3D face model. \\
Photo-realistic 3D face models learn a powerful prior over human facial appearance and geometry, which can be fitted to a single or multiple images of a person, effectively inferring a 3D head avatar. However, the fitting procedure itself is non-trivial and often requires expensive test-time optimization, impeding casual use-cases on consumer-grade devices. While this limitation may be circumvented by learning a generalized encoder that maps images into the 3D face model's latent space, another fundamental limitation remains. Even with more multi-view face datasets being published, the number of available training subjects rarely exceeds the thousands, making it hard to truly learn the full distibution of human facial appearance. Instead, our approach avoids generalizing over the identity axis by conditioning on some images of a person, and only generalizes over the expression axis for which plenty of data is available. 

A similar motivation has inspired recent work on codec avatars where a generalized network infers an animatable 3D representation given a registered mesh of a person~\cite{cao2022authentic, li2024uravatar}.
The resulting avatars exhibit excellent quality at the cost of several minutes of video capture per subject and expensive test-time optimization.
For example, URAvatar~\cite{li2024uravatar} finetunes their network on the given video recording for 3 hours on 8 A100 GPUs, making inference on consumer-grade devices impossible. In contrast, our approach directly regresses the final 3D head avatar from just four input images without the need for expensive test-time fine-tuning.




\section{Problem Formulation}

In this section, we introduce the problem formulation of \system{}. 
First, based on the retrieve-then-adapt idea~\cite{qian2020retrieve}, we discuss our approach for generating data reports by breaking it down into retrieving a related report, deducing, and reproducing a sequence of data analysis segments, each comprising analytical objectives, data processing, and insights.
Based on the formulation, we further conduct a preliminary study to understand the detailed design requirements of what can be reused from previous reports and how to rectify the new data. 
Based on the findings, we propose \system{}'s design considerations.

\subsection{Definition of Data Reports}
\label{subsec:problem_formulation}

To create a data report, data scientists need to explore and analyze the data, obtain data insights, and organize them into coherent narratives and charts~\cite{li2023wherearewesofar}. 
A shortcut for this process might be referring to an existing report and attempting to adapt it with new datasets.
To begin, it is required to retrieve a report that shares a similar topic with the current dataset.
In this section, we assume the retrieval is performed with a reliable search engine or by the users and focus on the parsing and reconstruction of the report.

Given a report, the key to its reconstruction is to decompose the report into logically coherent segments and validate whether the new data can fulfill the needs of different segments or provide similar insights supporting the argument. 
If not, how can adequate transformation of the original segments or new data be employed to make the whole process successful? 

Previous research~\cite{bar2020automatically, li2023edassistant, batch2017interactive} has outlined the analysis workflow as the three steps: 
1) Given a dataset to analyze, a data scientist usually begins by viewing the data and setting an analytical objective. 
2) Then, the data scientist would perform various data transformation steps, usually writing and executing code, and potentially encode the transformed data into a chart~\cite{wang2023dataFormulator}. 
3) Finally, the scientist inspects the results of the analysis and obtains insights into the data. 
We define an ``analysis segment''  as a triplet:
$$ segment := (objective, transformation, insight).$$

After completing an analysis segment, data scientists continue the analysis by creating new analytical objectives from previous ones, hustling to gain new insights, and generating a subsequent analysis segment. 
By repeating this process, the data scientists create a report with a complete analysis workflow and data insights. 
In this study, the segment is considered a basic building block of a data report.

Based on the definitions, we formulate the analysis workflow as a sequence of interconnected analysis segments $S = \{s_0, s_1, \cdots, s_N\}$, where $N$ is the number of segments in a data report.
Each segment $s_j$ is defined by three key components $(o_j, t_j, i_j)$, where $o_j$ is the analytical objective that guides the inquiry, $t_j$ is the data transformations that process the data, and $i_j$ is the insights that emerge from this exploration. 
Moreover, these analysis segments have interdependent dependencies $D = \{d_0, d_1, \cdots\}$, as the objective for one segment might stem from either insight of a previous segment or the data.
Each dependency $d\in D$ denotes a directed link of a tuple of segments $(s_i, s_j)$, representing that the segment $s_j$ stems from $s_i$.
In this light, the analysis workflow can be organized into a tree~\autoref{fig:formulation}a1, where each node denotes an analysis segment, and each edge denotes a dependency. 
Specifically, the initial analysis segment depends on the data. 
Based on this formulation, a data report is essentially a structural form of insights $S = \{s_0, s_1, ..., s_N\}$ distilled from these segments of analysis. 

Therefore, our method revolves around deducing and reproducing the sequence of analysis segments $S$ from the reference report, including analytical objectives, data processing, and insights. 
Finally, we organize the newly gleaned insights into a coherent and informative report, thereby reviving the original report with fresh data insights.

\begin{figure}[!htb] 
  \centering
  \includegraphics[width=0.5\linewidth]{figs/formulation.png}
  \caption{
  Producing a data report (c1) involves analyzing the data (a) and summarizing the analyzed insights into a data report (b). 
  Specifically, the data analysis workflow (a) includes a series of interdependent analysis segments (a1), each corresponding to an analytical objective, data transformation steps, and data insights (a2). 
  To reuse an existing report on a new dataset, we first deduce the data analysis workflow and reproduce it on the new data (c2). 
  }
  \label{fig:formulation}
\end{figure}

\subsection{Preliminary Study}
\label{subsec:preliminary_study_settings}

Based on the definitions, we aim to decompose a data report into segments and apply them to new data. 
However, the organization of data reports is flexible, highly depending on user preference and experiences. 
Besides, the aspects of a report that can be inherited and need adjustment remain unclear. 
To bridge this gap, we conducted a preliminary study with two main objectives:
(1) Investigate the relationship between the narrative structure of data reports and the corresponding analysis segments. 
(2) Identify the similarities and differences in analysis segments among data reports on the same topic and analyze how the differences stem from the data.

We collected data reports of different topics from well-known organizations that publish data reports, such as ONS~\cite{ons}, YouGov~\cite{yougov}, Pew Research Center~\cite{pewResearchCenter} and PPIC~\cite{ppic}. 
% \todo{based on the collected report, make a report repository. }
For the first objective, we analyzed the narrative structures of these reports to assess their alignment with our analysis segments. 

For the second objective, we observed that these organizations often publish reports on similar datasets, such as epidemic data collected at different times, typically issuing one report per dataset. 
These reports usually look similar but vary subtly in their analysis and content, which can be evidence to inspect which features are inherited and which require adjustment. 
Therefore, we further constructed 39 pairs of data reports that share the same topics and conducted pair-wise analysis on them. 
We identified the similarities and differences in each report pair and analyzed how the differences were sourced from the data. 
Treating a data report containing a series of analysis segments, including \textbf{analytical objectives}, \textbf{data transformations}, and \textbf{report content}, we summarized the patterns of similarities and differences regarding these elements.

\subsubsection{Narrative Structure of Reports}

To extract analysis segments from a report, we identified how the report's content aligns with these segments, defined as \textbf{analytical objectives}, \textbf{data transformations}, and \textbf{report content}. 
We explored whether the narrative structure could be segmented so that each part corresponds to a distinct analysis segment.

Our analysis of 35 reports showed that in most cases (32/35), the analytical content was presented as distinct segments, each focused on a single objective rather than interspersed with multiple topics, with related text and possibly a chart grouped together.
Additionally, 11 out of 35 reports included non-analytical content, such as background information, which could either supplement a specific analysis or the entire report and appeared flexibly throughout. 
Some reports (23/35) also included a summary of key insights at the beginning or end, which we excluded to concentrate on the main analytical content.

\subsubsection{Similarities and Differences in Pair Reports}

We summarize the patterns of similarities and differences between the data reports on similar topics. 
These findings lay the foundation for designing and implementing a method to reuse data reports with new data. 

\paragraph{Analytical Objectives}
Analytical objectives are guidance for the exploration of insights and findings from the data.
For example, the report of internet users~\footnote{https://www.ons.gov.uk/businessindustryandtrade/itandinternetindustry/\\bulletins/internetusers/2018} holds an analytical objective to explore the Internet use among each different age group. 
Therefore, we identified the analytical objectives in the collected reports by inspecting the aspects of the data findings that were discussed. 
After identification, we compared the analytical objectives between each pair of reports and analyzed their alignment and variations. 

As a result, all of our collected pair reports reflect similar analytical objectives.
Specifically, 34/39 of them involve analytical objectives that are exactly the same, while 35/39 of them involve slight differences.
Most slight differences source from different \textbf{data contexts and scopes}. 
For example, the analytical objectives of two data with different time ranges will also focus on distinct time frames. 
Others stem from the \textbf{dependencies to previous data insights}. 
Analytical objectives may be formed based on previous insights through logical dependencies, e.g., exploring the reason for an increasing trend. 
Therefore, the changes in previous insights may also cause adjustments in the latter objectives. 
Additionally, some pair reports involve completely different analytical objectives (18/39), which mainly source from \textbf{different data fields}. 
For example, newer data may introduce additional data fields, thus triggering new analysis objectives.
Reports may also incorporate insights from external data sources (4/39), spanning different contexts, scopes, and fields. 
These insights typically maintain a logical dependency on previously established insights, such as generalizing from local to national trends, which also introduces varied analytical objectives. 


\paragraph{Transformation Operations}
Data transformation operations are not explicitly outlined within the data reports. 
Moreover, most of the source data provided by our collected reports have been processed (34/39), making it harder to infer the specific data processing conducted. 
Nonetheless, two facets of data processing can still be discerned from the reports. 
Firstly, the reports mirror the output of the data processing, as the charts and narratives presenting data insights directly originate from these outputs.
Secondly, they also reflect detailed data processing choices, particularly regarding charts, which involve decisions on chart type, encoding, binning, etc. 

Considering these factors, we compared the content referring to similar analytical objectives between each pair of reports and analyzed their similarities and differences in analysis outputs and data processing choices. 
Consequently, we observed that similar analytical objectives always yield similar \textbf{analysis output forms}.
For example, the objective of analyzing trends always results in a chart with the temporal field on the x-axis, indicating a transformed dataset measuring variables across time periods.
However, the \textbf{detailed data processing choices} may vary to accommodate the data difference (13/39). 
Varied formats and scopes of data fields could potentially result in different chart types or levels of binning granularity to better align with visualization rules.

\paragraph{Report Content}
The report content, comprising both textual information and charts, is directly derived from the result of data processing. 
Since analysis results from different datasets naturally differ, resulting in varying values in the report content, our primary focus lies in the similarities and differences beyond mere numerical distinctions. 

As a result, regarding the similarity of report content, we observe that each pair of reports shares a similar \textbf{narrative and visual style}, such as the formality degree in tone and infographic design. 
As our main objective is to reuse the analysis workflow, the inheritance of content style is not our primary focus and is therefore not considered in this study.
The differences primarily manifest in the textual descriptions of data insights. 
Different reports may describe varying types of data insights. 
For example, one report might focus on detailing an outlier, while another might only describe overall trends. 
This difference stems essentially from \textbf{distinct analysis results}, which not only result in numerical disparities but also lead to variations in the reflected data insights.
Under the same data processing steps, one dataset may exhibit a highly significant outlier in the results, while another dataset may not.

\subsection{Expert Interview}
The findings of the preliminary study revealed that various aspects of existing data reports can be leveraged to generate new reports, but these elements require adjustments to align with the new data. 
The study also provides theoretical guidance on how to identify incompatible aspects and the directions in which adjustments should be made. 

To further understand the user requirements in reusing existing reports to analyze data, we conducted an expert interview with two experienced data analysts, EA and EB. 
EA has over two years of experience working as an actuary at an insurance company, where they frequently analyze data, organize results, and present them to clients to guide future decision-making. 
EB is a seasoned researcher in data storytelling and an experienced data journalist. 

We conducted 45-minute interviews with each expert, during which we discussed the following questions: (1) What formats do they typically use to present data analysis results, such as data reports, data stories, or others? (2) Do they encounter scenarios in which they reuse or refer to existing data analysis materials for analyzing new data and proposing new presentation materials? (3) If so, what is their workflow?

For the first two questions, EA noted that he use different formats depending on the case, including Excel files, slides, and data reports. 
He noted that for all three formats, he often refers to existing analysis materials. 
For example, to analyze and present data for a new insurance product,  he may refer to past reports or related reports from other products. 
In these cases, both the data and analysis goals are usually quite similar, making existing materials particularly useful.
EB, on the other hand, discussed the distinction between data reports and data stories. 
She pointed out that data reports are highly structured and commonly used in formal settings such as official documents, presentations, and communications. 
These reports often follow stable templates, making it easy to refer to existing materials when generating new reports. 
In contrast, EB sees data stories as a more creative format and tends not to reference previous materials. 
She prefers to avoid replicating others' approaches and focuses on originality in crafting data stories.

For the third question, Although EA acknowledged using existing materials in multiple formats, he emphasized that the aspects reused and the workflow vary depending on the format. 
For Excel files, these existing files are often with previous SQL scripts and formulas, which can be seen as preserved analysis code. 
In these cases, small modifications (e.g., updating SQL query conditions for new data) are typically sufficient. 
For data reports and slides, however, the process is different. 
These formats often lack original analysis files, and while existing materials can inspire analysis goals, chart creation, and textual descriptions, the data analysis still needs to be conducted manually.
EA further elaborated on the differences between slides and data reports. 
While slides typically feature relevant charts, they can lack detailed narrative descriptions, as slides are often presented orally. 
In contrast, data reports are expected to contain more comprehensive written content, including detailed explanations.


EA and EB further elaborated on the workflow of reusing and referring to previous data reports. 
For cases where the data fields are similar, or the analysis goals remain consistent, both EA and EB noted that these cases allow simply replacing charts with new charts and replacing the numerical conclusions to align with the new data. 
However, when the fields differ, EA mentioned that adjustments are necessary, either by modifying the analysis goals or removing irrelevant sections of the report. 
For new fields, EA would develop new analysis goals and rewrite the descriptive analysis accordingly. In cases where the insights change, the analysis may need to be entirely redone to accommodate the new findings.

\subsection{Expert Interview}

The findings of the preliminary study reveal that various aspects of existing data reports can be leveraged to generate new reports, but these elements require adjustments to align with the new data. 
The study also provides guidance on how to identify incompatible aspects and the directions in which adjustments should be made. However, it is unclear how users reuse past analysis reports for new scenarios.

Therefore, we conducted an expert interview targeted at data analysts who frequently explore new data and compose reports to communicate insights to leaders or clients. Moreover, we expected the interviewees to have certain experience in using LLMs in their analysis and prototyping process.
We interviewed two experienced data analysts, EA and EB. 
EA has over two years of experience working as an actuary at an insurance company, where they frequently analyze a variety of data, including claims history, policyholder demographics, risk factors, and market trends.
Findings are presented to a range of clients, including internal stakeholders (such as underwriters, product managers, and senior executives) and external clients (such as brokers, corporate policyholders, or regulatory bodies), to guide strategic decision-making and ensure compliance with industry standards. 
EB is an assistant professor in the Department of Journalism of a top-tier university. She also holds a Ph.D. in data science and frequently writes data journalism for news media.


We conducted 45-minute interviews with each expert, during which we raised the following questions: (1) What formats do they typically use to present data analysis results, such as slides, reports, dashboards, and spreadsheets? (2) Do they encounter scenarios in which they reuse or refer to existing data analysis materials for analyzing new data and proposing new presentation materials? (3) If so, what is their workflow?

For the first two questions, EA mentioned that he uses different formats depending on the specific case, including Excel files, slides, and data reports. He emphasized that for all three formats, he frequently refers to existing analysis materials. For instance, when analyzing and presenting data for a new insurance product, he often consults past reports or related analyses from similar products. In such cases, both the data and the analysis objectives tend to align closely, making existing materials particularly valuable for efficiency and consistency. Additionally, EA noted that reports are his primary format for conveying data insights, particularly in formal settings.
On the other hand, EB explained that her approach to data presentation is more scenario-dependent, and her use of previous materials varies accordingly. She highlighted that data reports, which are highly structured and commonly employed in formal contexts such as official documents, presentations, and communications, often follow standardized templates. This structure makes it straightforward to adapt or reference existing materials when creating new reports.
However, EB also expressed her interest in crafting ``data stories,'' which, while similar to data reports in format, are more creative and exploratory. Unlike reports, she tends to avoid relying on specific pre-existing materials for data stories, as doing so might constrain her thinking. Instead, she prioritizes originality and creativity, focusing on developing unique narratives that reflect her individual insights and perspectives.


For the third question, Although EA acknowledged using existing materials in multiple formats, he emphasized that the aspects reused and the workflow vary depending on the format. 
For Excel files, these existing files are often with previous SQL scripts and formulas, which can be seen as preserved analysis code. 
In these cases, small modifications (e.g., updating SQL query conditions for new data) are typically sufficient. 
For data reports and slides, however, the process is different. 
These formats often lack original analysis files, and while existing materials can inspire analysis goals, chart creation, and textual descriptions, the data analysis still needs to be conducted manually.
EA further elaborated on the differences between slides and data reports. 
While slides typically feature relevant charts, they can lack detailed narrative descriptions, as slides are often presented orally. 
In contrast, data reports are expected to contain more comprehensive written content, including detailed explanations.


EA and EB further elaborated on the workflow of reusing and referring to previous data reports. 
For cases where the data fields are similar, or the analysis goals remain consistent, both EA and EB noted that these cases allow simply replacing charts with new charts and replacing the numerical conclusions to align with the new data. 
However, when the fields differ, EA mentioned that adjustments are necessary, either by modifying the analysis goals or removing irrelevant sections of the report. 
For new fields, EA would develop new analysis goals and rewrite the descriptive analysis accordingly. In cases where the insights change, the analysis may need to be entirely redone to accommodate the new findings.

\subsection{Design Considerations}

Based on the problem formulation and the findings of the preliminary study, we summarize five design considerations (\textbf{C1}-\textbf{C5}) for an automatic method of reusing data reports with new data. 

\begin{enumerate}[label=\textbf{C\arabic*}]
\item \textbf{Support analytical objectives extraction, correction, and addition. } 
The key of the method is to extract and re-execute the analysis workflow of the existing report, which corresponds to a series of interdependent analytical objectives. 
To achieve this, the method should first split the report to identify distinct analysis segments and then extract the analytical objectives and their dependencies. 
Additionally, it should support smart objective correction and addition based on the dependencies and data features.

\item \textbf{Generate appropriate data processing steps automatically. } 
Based on the preliminary study, the data processing steps are not reflected directly in the existing report. 
Therefore, the method necessitates autonomous reasoning about appropriate data processing steps that yield outputs similar to those in the existing report while also making informed data processing choices adaptable to the new dataset. 

\item \textbf{Produce insightful report content derived from analysis. } 
The textual and visual content that presents data insights constitutes the primary component of a data report. 
The method should produce content that effectively presents new data insights derived from analysis. 
Our preliminary study also revealed that reports often include non-analytical content, such as background information. 
As LLMs are pre-trained on extensive background knowledge, we allow them to generate this content. 
However, any generated non-analytical content will be highlighted, as it may not always be reliable.

\item \textbf{Enable real-time output observation and report modification. } 
Given the complexity of the method, which involves analytical objectives, data processing, report content, and structure, uncertainties naturally arise, potentially leading to deviations from user expectations. 
To address this, the method should provide an interactive interface, enabling users to observe generated outputs in real-time and make necessary adjustments. This ensures that the final report aligns closely with user expectations. 

\item \textbf{Facilitate report structure organization and modification.}
Since the report is generated based on the reference report's workflow, it will naturally exhibit a similar narrative structure. 
The method should allow for re-organizing this inherited structure, including adding and generating (sub-)titles, to ensure the report is well-structured and tailored to the new content.
\end{enumerate}

Based on the design considerations, we develop an intelligent method, \system{}, to deduce and reproduce the authoring workflow of existing data reports on the new data. 
The pipeline of \system{} consists of three stages. 
\textbf{In the pre-processing stage,} \system{} recommends the most relevant reports from a built repository, ranked by similarity to the user's dataset. 
It then dissects the existing report into interconnected segments, each corresponding to the data insights of an analysis segment (\textbf{C1}). 
Based on the segmentation, it extracts the analytical objectives of each segment and deduces their dependencies (\textbf{C1}). 
\textbf{In the analysis stage,} \system{} executes each segment by reusing the information from the original report, identifying the inconsistencies, and customizing the analytical objectives, approaches, and report contents based on the new data (\textbf{C1-C3}). 
\textbf{In the organization stage,} \system{} inherits the original report structure and enables title re-generation (\textbf{C5}). 
Moreover, to enhance usability, we integrate an interactive interface for \system{}, allowing users to inspect real-time outputs, add new analytical objectives, and modify report content as needed (\textbf{C2, C5}). 



\section{\system{}}
\label{sec:respark}

This section describes the implementation of \system{}, including pre-processing, analysis, and organization. 
\system{} utilizes the GPTs from Azure to incorporate the LLM-driven functionalities. 
Specifically, we employ the ``gpt-4-vision-preview'' model as our input involves chart images. 
The prompts are provided in the supplementary materials. 

\subsection{Pre-processing Stage}

Before proceeding with analysis and organization, we need to pre-process the user dataset and reports to (1) acquire necessary data features, (2) recommend the most relevant reports to the data, and (3) extract the analytical objective and their dependencies of the selected report for the subsequent processes. 

\subsubsection{Data Pre-processing}
\label{subsubsec:data_pre_processing}

Based on the findings in the preliminary study (~\autoref{subsec:preliminary_study}), most adjustments entail considerations of data features such as context, scope, fields, and formats. 
Presenting the entire dataset to LLMs is currently impractical due to token limitations, and it does not facilitate a comprehensive understanding of these data features. 
Therefore, we utilize a similar data summary method to LIDA~\cite{dibia2023lida}. 
This method first extracts scope, data type, and unique value count information and samples some values for each data field. 
Subsequently, it employs LLMs to provide brief semantic descriptions for the dataset and each data field. 
The description of the dataset can also help recommend relevant reports(~\autoref{subsubsec:report_retrieval}). 
These pieces of information are then integrated to form a comprehensive data summary.

\subsubsection{Report Pre-processing}
\label{subsubsec:report_pre_processing}

\begin{figure*}[!htb] 
  \centering
  \includegraphics[width=\linewidth]{figs/system.png}
  \caption{
  The interface of \system{}. 
  \system{} consists of four views: data view (b-c), dependency view (d-e), content view (f-g), and generation view (h-k). 
  The data view displays the overall description and data field information. 
  The dependency view displays the extracted interdependent report segments. 
  The content view shows the analytical objective and content of the selected segment. 
  The generation view demonstrates the generated results in real-time. 
  }
  \label{fig:interface}
\end{figure*}

Based on the analysis workflow formulation in~\autoref{subsec:problem_formulation}, our goal is to deduce the analysis segments and their dependencies from the original report for subsequent execution. 
Our preliminary study showed that most reports present analytical content in distinct segments, each focused on a single objective, with related text and visuals grouped together. 
Therefore, ideally, we can find a segmentation that aligns each report section with a specific analysis segment. 
In this light, \system{} should segment the report accordingly, extract the analytical objectives for each segment, and deduce their dependencies.

To achieve this goal, an appropriate report segmentation criteria is very important, as it directly determines the entire analysis workflow (consisting of a sequence of analysis segments). 
The accuracy of segmentation also affects the quality of the extracted analytical objectives and dependencies.

We were initially inspired by prior work in automatic storytelling and insight-mining, which formalizes data insights and their relationships~\cite{ma2023insightpilot, wang2019datashot}. 
For example, Calliope~\cite{shi2020calliope} defines a data story as a series of interrelated insights, with each insight describing data patterns in specific data fields and subsets. 
For instance, ``The average worldwide gross for action movies is increasing over time'' describes an increasing trend measuring ``average (worldwide gross)'' over the breakdown ``year'' within the subset ``genre = action''. 
Based on this definition, we can potentially segment the report by identifying the insight type, measure, breakdown, subset, etc., and combine the insights into segments based on these labels. 
However, we found this definition challenging for segmenting practical data reports, as it doesn't accommodate the flexibility of analyses that involve deeper data transformations, such as creating new fields and describing patterns in derived variables.

Therefore, we need to define new segmentation criteria that accommodate the flexible analysis in the data report. 
Instead of formally defining a ``segment'' or an ``analytical objective'' with a strict data model, we provide a loose description of how segments can be divided and use LLMs to perform segmentation. 
Our preliminary study indicated that most report segments consist of continuous text and possibly a chart.
Based on the study, we execute report segmentation, extract analytical objectives, and deduce the dependencies between segments through the following approach: 
\begin{itemize}
    \item \textbf{Match. } 
    First, for each chart, we match the related paragraph text to form a segment. 
    Based on our preliminary study, we assume that (1) each paragraph corresponds to the nearest preceding or following chart, or none at all, and (2) all text associated with a single chart is continuous. 
    Starting with the first paragraph, LLMs determine whether it matches the nearest preceding or following chart (e.g., describing insights from the chart) or if it doesn't relate to any chart.
    
    \item \textbf{Categorize. } 
    For text that doesn't match a chart, LLMs categorize it to determine if it involves data analysis or serves another purpose, such as providing background information. 
    For continuous text segments that involve data analysis, we further assess whether they belong to the same segment (describe insights derived from the same transformed data). 
    
    \item \textbf{Summarize. } 
    After matching and categorizing, LLMs summarize the analytical objective of each segment and deduce its dependencies with previous segments. 
    We use the six logical relations defined in Calliope to outline dependencies among report segments. 
    LLMs determine whether new content is logically connected to an existing segment or originates directly from the data.
\end{itemize}


\subsubsection{Relevant Report Retrieval}
\label{subsubsec:report_retrieval}


With the pre-processed data and reports, users can select a reference report to analyze the target dataset. 
Based on findings from our preliminary study, various aspects of existing data reports, such as analytical objectives and report content, can serve as helpful reference material. 
However, since the reference report's data may differ from the target dataset, adjustments are necessary to align with the new dataset. 
The closer the reference report's data is to the target dataset, the more aspects can be reused without modification, making the report more suitable for use as a reference.


To facilitate the retrieval of suitable reports, we aim to identify the reports with data similar to the target dataset. 
The core idea is to convert both the dataset and report information into vector embeddings, compute their cosine similarities, and rank the reports from highest to lowest score.
The key question is: which specific information from the dataset and reports should be embedded?
We propose two mechanisms for extracting the embedding of data and report information: topic relevance and field similarity.
\begin{itemize}
    \item \textbf{Topic relevance} refers to the alignment between the topic of the dataset and the report. 
    Typically, datasets and reports within the same domain (e.g., health, economy) exhibit higher topic relevance. 
    For example, a sales dataset is highly topic-relevant to a report analyzing market sales trends. 
    To compute topic relevance, we extract the embedding of the dataset’s name and description, along with the headings and pre-processed analytical objectives of the report. 
    We hypothesize that these elements are semantically related to the corpus's overall topic, and the cosine similarity of their embeddings can reflect their topic relevance.
    \item \textbf{Field similarity} pertains to the alignment of the data fields described in the report with those contained in the dataset. 
    For example, a report on voting intentions across different gender and age groups would exhibit higher field similarity with a dataset containing gender and age information. 
    To compute field similarity, we embed the names and descriptions of the dataset's fields. 
    For the reports, we use LLMs to infer the data fields discussed in the report and embed these inferred fields along with their descriptions. 
    We hypothesize that the cosine similarity between these reflects the degree of field similarity.
\end{itemize}
Finally, we sum the scores from both mechanisms for each report, ranking them from highest to lowest, allowing users to select the most appropriate reference reports.

\subsection{Analysis Stage}
\label{subsec:analysis_stage}

Through the pre-processing stage, we obtain the summary of the new dataset and the segments of the existing report. 
Each segment corresponds to an analytical objective, a dependency on the previous segment or the data, and pieces of report content, including text and charts. 
The next stage is to reproduce the analysis workflow by re-executing each segment on the new data, encompassing reusing and reconstructing the analytical objective, analysis operations, and report contents. 

\subsubsection{Analytical objective correction and insertion}

The analysis workflow is driven by a series of posed analysis objectives. 
Through the pre-processing stage, we obtain the analysis workflow of the existing report by dividing it into segments and extracting each segment's analytical objectives and dependencies. 
To adapt the workflow to new data, \system{} is required to (1) correct the extracted analytical objectives and (2) support the insertion of new objectives according to the data features and segment dependencies. 

\textbf{Analytical objective correction. }
The preliminary study indicates that while existing analytical objectives often remain applicable, alterations or removals may be necessary due to data fields, dependencies, or the data context and scope. 

First, the original objective might involve data fields absent in the new data. 
Given the pre-processed data summary, we employ LLMs to evaluate if the new dataset's fields sufficiently fulfill the objective, considering semantic similarities despite word-to-word differences in field names. 
For example, an objective mentioning ``earn money'' can be related to the data field ``gross.'' 
LLMs are tasked with explaining their decisions to enhance their reasoning~\cite{mialon2023augmented}. 
If the available fields suffice, LLMs should describe the required fields and analysis operations. 
Otherwise, LLMs must explain what external fields are needed to satisfy the objective. 
In such cases, we correct the objective by replacing missing fields with available alternatives. 
For instance, if a movie dataset lacks geographic data, the objective of locating the highest-grossing movies might shift focus to their directors.

Second, the original objective may derive from insights in a prior segment. 
Adjustments might stem from two scenarios. 
If the insight's nature changes (e.g., from an increasing to a decreasing trend), a corresponding shift is needed in the latter related objective (e.g., from identifying causes of growth to exploring reasons behind the downturn). 
Therefore, we provide the model with the newly generated results of the dependent segment and require it to identify and adapt to such variations. 
Additionally, the dependency may call for a context or scope that the data cannot satisfy, such as from local to national or from a 5-year trend to a 20-year trend. 
LLMs must infer whether changes in scope or context affect the objective's applicability, which could lead to its potential removal if the new data does not support similar adjustments. 

Third, minor adjustments are often required for data context and scope adjustments. 
For example, an objective focusing on a 5-year trend needs adjustment to fit a dataset covering only the past three years. 
LLMs should make these modifications based on the context and scope of the provided data.

\textbf{Analytical objective insertion. }
The uniqueness of new datasets and user-driven queries may necessitate adding fresh analytical objectives based on previous insights and dependencies. 
\system{} enables users to embed new objectives at chosen positions, rooted in the data or reliant on preceding analysis segments. 
Users can define the focus data fields and dependencies of these new objectives, and LLMs can suggest potential objectives based on user input.

\subsubsection{Analysis Operation Generation}

% \TODO{code structure, generate a table and a chart}. 
Once the analytical objective has been refined, \system{} generates the requisite analysis operations to fulfill this objective. 
Since these operations are not explicitly detailed in the report, we utilize the code-generation capabilities of LLMs for this phase. 

LLMs are prompted to generate analysis code that aligns with the clarified objective, provided with the data summary and original report content as guides. 
The model is instructed to refer to the original report to deduce the necessary data transformations. 
We also remind the model to generate the code that accommodates the new data, as the reference report content is from a different dataset and only serves as a reference for expected output. 
The model is required first to plan step by step~\cite{kojima2022large} and then generate the Python code that results in transformed data and a chart using matplotlib~\cite{Hunter2007matplotlib} or Seaborn~\cite{Waskom2021seaborn}. 

Upon code generation, we execute it to procure the transformed data and the accompanying chart. 
The execution may also raise errors. 
We relay any execution results, including the transformed data, chart, and potential errors, back to the LLMs. 
The model then assesses whether the code execution is successful and whether the results accurately address the analytical objective and are adequate for generating report content. 
Should the model deduce that revisions are necessary, the cycle of code generation and execution is repeated until satisfactory results are obtained, paving the way for report content creation.

\subsubsection{Report Content Production}

With the execution results in hand, we proceed to generate new report content. 
Given that the code already produces the chart, the model's task in this step is to generate the accompanying textual narrative. 
We instruct the model to produce a narrative that imitates the writing style of the reference report yet is tailored to fit the new data context and the insights derived from the executed analysis. 
We also enable user modification to the report content. 

\subsection{Organization Stage}

After reproducing the analysis workflow and obtaining the new data insights, the next step is to structure the new report. 
As we generate the sequence of segments based on the order of dependencies, the implicit logical structure is adopted naturally. 
Additionally, we inherit the explicit structural elements (such as titles and sections) from the original report. 
Newly inserted analytical objectives are incorporated along with their dependent segments. 
The report and its sections' titles are re-crafted based on the original ones, incorporating new data insights to guide the title generation process. 
User interventions are also supported, allowing for the reorganization of segments into new sections, thereby tailoring the report structure to meet user needs or preferences better.



\section{Evaluation}
\label{sec:evaluation}
Our experiments aim to investigate whether agents within our framework can produce effective evolution of language strategies. Specifically, our experimental section addresses the following three research questions (RQs):
\begin{enumerate}
    \item RQ1 (Effectiveness): Can participants effectively evade regulatory detection over time, and how does the accuracy of information transmission change? Additionally, how do different LLMs affect the content and effectiveness?
    \item RQ2 (Human Interpretation): Do the evolved language strategies employed by agents effectively align with human understanding? Can they be interpreted and applied in real-world scenarios?
    \item RQ3 (Ablation Study): How does the newly introduced GA impact the evolution process in our framework?
\end{enumerate}

\subsection{Experimental Settings}
In our evaluation, we designed an abstract password game \cite{guess_number02} and a more realistic illicit pet trade scenario\cite{trade01,trade02,trade03}. 
%The password game features a relatively abstract, easily controlled setting, allowing for clear observation of how agents’ strategies evolve. Meanwhile, the illicit pet trade scenario simulates illegal activities on social networks \cite{DiMinin2018MachineLF}, with relevant corpora that more closely resemble real-world conditions, enabling a more direct comparison between evolved strategies and their real-life counterparts.
The overall experimental procedure follows the description in Section~\ref{sec: method}. In each round, the process comprises three stages: initialization, dialogue, and interview. In each round, the information that the participant agents need to convey will be randomly generated within a defined range. Only rounds in which a five-turn dialogue is successfully completed proceed to the interview phase, during which both parties’ successful transmission of information is verified. If any regulation violations are detected by the supervisory agent during the dialogue stage, that round is deemed a failure and is assigned a transmission score of zero. To ensure the robustness of our findings, we conducted 15 independent trials for each experimental condition. Each trial consisted of 50 fixed dialogue rounds. The crossover and mutation probabilities in the framework were set to 0.2 and 0.8.

\subsubsection{Scenario 1: Password Game}
Our first scenario is considered a relatively simple and abstract guessing game that involves a numerical context. In this setup, each participant is assigned a four-digit password, which they must convey to another participant within five rounds of dialogue without directly mentioning the numbers. Simultaneously, they need to extract information from the dialogue to infer the other's password.

The supervisory agent in the dialogue follows the policy of “prohibiting all content related to numbers.” The “password game” scenario is specifically designed to observe and analyze participant agents' language adaptability and strategic evolution in a theoretical and abstract context. Compared to complex scenarios based on real events, it provides a clearer and more easily quantifiable experimental environment.

\subsubsection{Scenario 2: Illicit Pet Trade} 
Simulating and detecting the covert drug transactions prevalent on social platforms is an important and realistic research topic. However, given the ethical considerations, we choose not to proceed with direct simulations, but to adopt a more neutral and less harmful setting—a illicit pet trade scenario—to serve as our experimental context.

In this scenario, we simulate an illegal pet trade through social platform. Participant agents play the roles of buyer and seller. The buyer conveys a request to purchase a randomly selected pet (such as a parrot, lizard, or cat) and specifies the currency they can use for payment (e.g., USD, CNY, JPY). The seller provides details about the transaction, including the time (morning, noon, or midnight) and location (school, park, or station). The supervisory agent enforces a policy that considers discussions related to pet transactions as regulation violations.
The focus of this experiment is to observe the adaptability of the participant agents' language and its evolution at a finer level of information granularity, as well as whether regulators can accurately detect regulation violations.

\subsection{RQ1: Effectiveness}
\subsubsection{Experiment Settings}

To evaluate RQ1, we conducted experiments using four different LLM models provided by various companies: \textit{gpt-4o}, \textit{gpt-4o-mini}, \textit{deepseekV2}, and \textit{qwen-turbo}. We introduced the following metric:
\begin{itemize}
    \item Accuracy of Information Transmission: This metric measures the agents' ability to convey the intended information to their dialogue partner without being detected. Specifically, it calculates the proportion of successfully transmitted information relative to the total intended information. 
    \item Average Dialogue turns: This metric evaluates the number of successful dialogue rounds the agents complete without being flagged by the supervisory agent. A higher dialogue turns indicates greater success in evading regulatory scrutiny.
    \item Average Entropy (Language Complexity):
        Entropy reflects the unpredictability of text and is calculated as:
        \[
        H(X) = -\sum_{i=1}^n P(x_i) \log P(x_i),
        \]
        where \(P(x_i)\) is the probability of each token. Higher entropy suggests a broader range of expressions (indicating greater innovation), but excessively high entropy can lead to incoherence.
    \item Distinct (Lexical Diversity):
        Distinct measures the number of unique n-grams in the text, computed as:
        \[
        \text{Distinct-n} = \frac{\text{Number of unique n-grams}}{\text{Total number of n-grams}}.
        \]
        Specifically, we evaluate Distinct-1 (unique unigrams). A higher distinct score indicates a larger, more varied vocabulary, signifying richer language use. Note that while both metrics assess diversity in language, entropy focuses on unpredictability, whereas distinct emphasizes lexical variety. 
\end{itemize}


\subsubsection{Experiment Results in Password Game}
\begin{figure*}[ht]
    \centering
    % 子图 (a)
    \begin{subfigure}[t]{0.48\textwidth}
        \centering
        \includegraphics[width=\linewidth]{figures/sec1_turn_acc_v5.png}
        \caption{Password Game}
        \label{fig:sce1}
    \end{subfigure}
    \hfill
    % 子图 (b)
    \begin{subfigure}[t]{0.48\textwidth}
        \centering
        \includegraphics[width=\linewidth]{figures/sec2_turn_acc_v5.png}
        \caption{Illicit Pet Trade}
        \label{fig:sce2}
    \end{subfigure}
    \caption{Average Continuous Dialogue Turns and Information Transmission Accuracy Across Dialogue Rounds}
    \label{fig:merged}
\end{figure*}

Figure \ref{fig:sce1} presents our experimental results in the password game. The x-axis corresponds to the increasing number of dialogue rounds, whereas the y-axis captures two primary metrics: (1) the average number of continuous dialogue turns before detection by the supervisory agent, and (2) the accuracy of information transmission, which is defined as the proportion of successfully transmitted information during the post-dialogue interview.

Overall, our findings indicate that as the number of rounds increases, agents gradually learn to evade regulation violation while conveying information with greater accuracy. Notably, most agents exhibit a pronounced local peak around the 20th round, followed by a brief decline and subsequent recovery. These fluctuations can be attributed to the dynamic nature of our simulation framework, which does not converge on a single dominant strategy but rather encourages ongoing exploration of novel language strategies.

We also observe that different LLMs influence the learning trend to varying degrees. Among the models tested, \textit{gpt-4o} demonstrates the strongest performance. Although other models generally share a similar upward trend, their relative performance gaps prove less stable. For instance, while \textit{deepseekV2} achieves the highest number of turns around the 20th round, its performance declines significantly by the 50th round in comparison to other models.

Turning to the accuracy results, we again observe a similar learning trajectory. This parallel arises primarily because if participant agents fail to complete a sufficient number of uninterrupted dialogue turns, the successfully transmitted information in that round is effectively zero. Consequently, especially in the early stages of the experiment, many rounds end with no successful transmissions. Overall, \textit{gpt-4o} still maintains a clear advantage over the other LLMs. However, we do observe subtle differences when comparing the dialogue round trends: for example, at the 20th dialogue round, \textit{deepseekV2} achieves a significantly higher average number of communication cycles than \textit{gpt-4o-mini}, yet their information transmission accuracy remains relatively similar.

Despite the overall positive learning trajectory, the average information accuracy remains low in the password game. We believe this outcome stems primarily from the intrinsic difficulty introduced by the scenario’s abstract nature. Without explicit prompts driving agents to develop symbolic or otherwise encrypted language stratgy, communication largely remains within the realm of everyday language. Consequently, the indirect expression of numeric information is challenging to implement and easily detectible by the supervisory agent.

\begin{table}[h!]
    \centering
    \caption{Performance of Different LLMs in Password Game}
    \label{tab:sce1}
    \renewcommand{\arraystretch}{1.2} % 调整行高
    \begin{tabular}{l S S S}
        \toprule
        \textbf{Model} & \textbf{Total Turns} & \textbf{Avg. Entropy} & \textbf{Avg. Distinct-1} \\
        \midrule
        \rowcolor{gray!10} \textbf{gpt-4o}       & 84.2   & 7.103 & 0.484 \\
        \textbf{gpt-4o-mini}  & 75.5   & 6.998 & 0.354 \\
        \rowcolor{gray!10} \textbf{deepseekV2} & 59.7
        & 5.365 & 0.247 \\
        \textbf{qwen-turbo}   & 50.8  & 6.101 & 0.518 \\
        \bottomrule
    \end{tabular}
\end{table}

Table \ref{tab:sce1} summarizes the performance of the four models in terms of cumulative dialogue turns, entropy, and Distinct-1. As shown, \textit{gpt-4o} achieves the highest values in both Entropy (7.103) and Distinct-1 (0.484), indicating that it employs a broader, more diverse range of vocabulary and more unpredictable expressions—thus having a greater likelihood of evading regulation violation detection when conveying numerical information. In contrast, \textit{deepseekV2} exhibits notably lower Entropy (5.365) and Distinct-1 (0.247), suggesting a more frequent reuse of fixed expressions. Notably, although \textit{qwen-turbo}’s Entropy (6.101) is only moderate, it attains a surprisingly high Distinct-1 (0.518), reflecting greater lexical richness. However, this does not translate into more effective regulatory evasion, as its total turns are only 50.8. Hence, merely having higher lexical diversity and linguistic entropy is insufficient to guarantee successful evasion. A model must also balance the concealment of overall semantics with the adaptation of its language strategy to achieve longer conversation sequences and a higher rate of successful information transmission. In other words, while richer language expression does confer certain advantages in countering regulation, it can still be detected when deeper strategies—such as tailored expression structures and topic evolution—are absent, ultimately resulting in fewer total turns.

\subsubsection{Experiment Results in Illicit Pet Trade Scenario}
Figure~\ref{fig:sce2} presents the experimental results of our framework in the illicit pet trade scenario, which overall resemble those of the password game but also exhibit some notable differences. First, both figures reveal a discernible learning trend, particularly during the initial 10 rounds. Meanwhile, \textit{gpt-4o} continues to demonstrate the strongest overall performance. We note that, because this scenario features a more concrete and complex semantic environment, there is an abundance of relevant linguistic material that can be leveraged for indirect expression. Consequently, under a similar number of turns, the overall accuracy here is noticeably higher compared to the password game.
Nevertheless, performance fluctuations persist. In particular, in the accuracy plot, \textit{deepseekV2} experiences a pronounced increase in accuracy after the 30th round, while \textit{gpt-4o}’s accuracy declines during the same period. As a result, \textit{deepseekV2} ultimately surpasses \textit{gpt-4o}’s accuracy in the final rounds of the experiment.

\begin{table}[h!]
    \centering
    \caption{Performance of Different LLMs in Illicit Pet Trade}
    \label{tab:sce2}
    \renewcommand{\arraystretch}{1.2} % 调整行高
    \begin{tabular}{l S S S}
        \toprule
        \textbf{Model} & \textbf{Total Turns} & \textbf{Avg. Entropy} & \textbf{Avg. Distinct-1} \\
        \midrule
        \rowcolor{gray!10} \textbf{gpt-4o}       & 136.2  & 6.856  & 0.471 \\
        \textbf{gpt-4o-mini}  & 74.4  & 6.595  & 0.387 \\
        \rowcolor{gray!10} \textbf{deepseekV2} & 65.2   & 6.255  & 0.338 \\
        \textbf{qwen-turbo}   & 50.5   & 5.891  & 0.461 \\
        \bottomrule
    \end{tabular}
\end{table}
Table \ref{tab:sce2} presents the performance of various LLMs in the illicit pet trade scenario, measured by total turns, average agent entropy, and Distinct-1. As in the password game, \textit{gpt-4o} maintains a notable lead in total turns (136.2) while also displaying relatively high entropy (6.856) and Distinct-1 (0.471). In contrast, \textit{gpt-4o-mini} reaches roughly half as many total turns (74.4), despite having a comparable entropy score (6.595). Meanwhile, \textit{deepseekV2} (65.2) and \textit{qwen-turbo} (50.5) trail further behind in total turns. Consistent with the results shown in Table 
\ref{tab:sce1}, \textit{qwen-turbo} again achieves a high Distinct-1 score, which we speculate may be linked to its training corpus: it includes extensive data from the Chinese internet, likely giving it an advantage in a Chinese-language environment over more internationally oriented models.

Notably, the range of entropy scores in this scenario—spanning from 5.891 (\textit{qwen-turbo}) to 6.856 (gpt-4o)—is narrower than in the password game (see Table \ref{tab:sce1}), reflecting the more concrete nature of the illicit pet trade setting. This scenario provides richer contextual cues for indirect references, enabling all models to maintain higher semantic complexity. However, as was the case in the password game, having a broader vocabulary or greater unpredictability alone does not guarantee extended evasion: models must integrate their linguistic variety into strategic planning to circumvent regulatory scrutiny, a balance that \textit{gpt-4o} continues to manage most effectively.

\setlength{\fboxrule}{0.5pt} 
\vspace{0.5em}
\noindent
\begin{tcolorbox}[colframe=black!20, colback=gray!10, arc=5pt, boxrule=0.5pt, width=0.99\linewidth]
\textit{Answer to RQ1}: Experimental results indicate that participant agents in our framework progressively improve their ability to evade regulation violation detection through continuous interaction, leading to longer uninterrupted dialogue sequences. Concurrently, the accuracy of information transmission gradually increases over successive rounds, demonstrating that the evolved strategies effectively balance evasion with precise communication.
Moreover, different models also exhibit varying results. For example, \textit{gpt-4o} performs most outstandingly in extending dialogue turns and maintaining language complexity (i.e., high entropy and lexical diversity), while other models such as \textit{gpt-4o-mini}, \textit{deepseekV2}, and \textit{qwen-turbo} demonstrate different fluctuations and localized advantages at different stages.
\end{tcolorbox}

\subsection{RQ2: Human Interpretation}
\subsubsection{Experiment Settings}

To investigate the real-world relevance of both the evolved language strategies and the resulting dialogue, we conducted a human evaluation on a subset of successful dialogue records from the password game and illicit pet trade scenario. The dialogues generated by the \textit{gpt-4o} models are randomly selected, and 40 human participants participated in the experiment to evaluate them. The 40 human reviewers had an average age of approximately 27 (SD = 4). In terms of gender, 75\% of the human reviewers were male, and 25\% were female. Regarding educational background, 67.5\% held a bachelor's degree, 27.5\% held a master's degree or above, and 5\% had an associate degree or lower. All dialogue records were presented in Simplified Chinese.

Each participant rated each dialogue on a 5-point Likert scale on the following five metrics:
\begin{itemize}
    \item Explicit Understanding: Evaluates how effectively the dialogue’s explicit meaning is communicated (1: Extremely vague and confusing; 3: Moderately clear, but some parts may require further interpretation; 5: Crystal clear and precise).
    \item Implicit Understanding: Assesses the reader's ability to grasp the underlying or unstated messages (1: Nearly indecipherable subtext; 3: Some underlying meaning is apparent, but requires effort to fully grasp; 5: Subtext that is immediately apparent).
    \item Realistic Significance: Measures the extent to which the dialogue reflects real-life situations and holds practical relevance (1: Highly unrealistic with little relevance; 3: Generally realistic, though some elements may not align with real-world situations; 5: Deeply rooted in real-world context).
    \item Regulatory Avoidance: Examines the effectiveness of the strategies in evading regulation violation (1: Blatantly ineffective and easily spotted; 3: Partially effective, with the potential for detection in some cases; 5: Exceptionally subtle and effective).
    \item Strategy Existence: Determines how plausible it is for such strategies to be observed in practical, real-world scenarios (1: Extremely implausible; 3: Fairly believable, though may seem impractical in specific situations; 5: Entirely plausible).
\end{itemize}



\subsubsection{Experiment Results}
\begin{figure}
    \centering
    \includegraphics[width=0.9\linewidth]{figures/user_study_v4.png}
    \caption{Box plots of user study scores across different metrics in two scenarios. The red x symbol denotes the mean value.}
    \label{fig:case_study}
\end{figure}
As shown in Fig.\ref{fig:case_study}, our framework consistently achieves average scores of 3.4 or above across most indicators (such as explicit understanding and implicit understanding). This suggests that, both in terms of the generated dialogues and the underlying strategies, it possesses valuable practical applicability.

%Although there are a few exceptions, compared with the old framework (\textit{w/o GA, gpt-4o}), the new version (\textit{w/ GA, gpt-4o}) demonstrates overall advantages in both average scores and score distributions. In the comparison between different versions, under the more realistic illicit pet trade scenario, the new framework shows distinct benefits over the old one in both “regulatory avoidance” and “strategy existence”—both in distribution and mean values. This finding indicates that introducing a genetic algorithm, particularly a fitness‐based strategy selection mechanism, makes strategy adoption more efficient and stable. As for the password game, we speculate that the main reason these two metrics do not show a large distributional gap is that, in an abstract scenario, the range of available strategies is broader.

Comparing distributions between the password game and the illicit pet trade scenario reveals some interesting phenomena. Focusing on “realistic significance” and “regulatory avoidance,” the more abstract password game often yields higher mean values than the more concrete illicit pet trade scenario, while also exhibiting lower dispersion. We speculate this is related to the inherently abstract nature of numeric information: encryption and covert hints can be harder to detect in such contexts, and the growing tendency on Chinese internet platforms to use abstract language \cite{Wu2025HighEnergy} may lead reviewers to have a higher acceptance of “obscure” expressions. Conversely, the illicit pet trade scenario, despite being closely tied to real-world transactions, may suffer if the indirect or euphemistic methods in the dialogues are insufficiently subtle. Human reviewers can find them conspicuous or “forced,” potentially causing lower scores for “realistic significance” and “regulatory avoidance” in terms of both distribution and mean values.
A significant portion of these results can be attributed to inherent biases in commercial LLMs, such as ChatGPT, introduced during their training phases. These general-purpose models undergo fine-tuning via RLHF to align with specific product positioning, which often results in a more standard and safe output style. However, this characteristic poses a limitation for our simulation framework, as it may hinder the model’s ability to capture the nuanced and unconventional expressions typical of online social interactions. Ideally, fine-tuning datasets that are more representative of social platforms could lead to improved performance in our simulations.


In the abstract password game, for instance, a typical conversation might go like this:
\begin{quote}
\textit{
“I've really grown fond of a certain phase of the moon. It's not the brightest or the darkest, but it always carries its own charm. It symbolizes ...... In that green oasis, I felt as if I were catching a glimpse of the golden hues of autumn leaves, much like the soft, warm glow of dusk—calm and serene ...” 
}
\end{quote}
Here, words like “lunar cycle” and “autumn leaves” can subtly hint at larger or smaller digits, or use seasonal imagery to convey key information. Since these references lack an obvious connection, they lend a more literary feel to the dialogue and, to some extent, raise the bar for recognition and detection.

By contrast, in a more concrete setting like illicit pet trade, example conversations may be closer to real‐life buying and selling procedures, which can make them appear more “suspicious”:
\begin{quote}
\textit{
“... about a vibrant 'tropical chatterbird' renowned for its brilliant plumage and uncanny mimicry ... I've also come into possession of a few 'Rising Sun coins' for exchange ...... Perhaps you might know a place where ...”
}
\end{quote}
In this dialogue, the term “tropical chatterbird” serves as an euphemism for a parrot, emphasizing its colorful appearance and mimicking ability without mentioning the animal directly. Meanwhile, “Rising Sun tokens” subtly alludes to the Japanese yen, since the Rising Sun is an iconic symbol of Japan. This coded language allows both parties to communicate their intentions regarding the acquisition of a rare bird and the intended payment method without explicitly revealing sensitive details. However, if these indirect expressions are used excessively, the dialogue may appear artificial or unnatural, potentially reducing its authenticity—thus affecting evaluations of both “regulatory avoidance” and “strategy existence.”
\setlength{\fboxrule}{0.5pt} 
\vspace{0.5em}
\noindent
\begin{tcolorbox}[colframe=black!20, colback=gray!10, arc=5pt, boxrule=0.5pt, width=0.99\linewidth]
\textit{Answer to RQ2}: Our evaluation confirms that the emergent language strategies closely resemble real-world language strategies, effectively employing euphemisms and implicit cues, and are generally understood by human reviewers. However, while these strategies show potential in simulations, they often appear forced or unnatural due to the fine-tuning of LLMs as commercial products, requiring refinement to better mimic the nuanced and fluid communication typical in real-world social interactions.

\end{tcolorbox}

\subsection{RQ3: Ablation Experiment}
\subsubsection{Experiment Settings}

To evaluate the effectiveness of the GA introduced in our framework, we conducted an ablation experiment using \textit{gpt-4o-mini} and \textit{gpt-4o} as the underlying LLM. For comparison, we employed the approach from our initial study \cite{DBLP:conf/cec/CaiLZLWT24}, which primarily differs in its strategy-update mechanism. In that earlier framework, the LLM is provided with both the existing strategy and newly flagged regulation violation records during the reflection stage, prompting the model to propose a new set of strategies that replace the old ones.
In contrast, our new framework employs a GA process where each strategy is treated as a discrete unit and optimized iteratively through GA. 

\subsubsection{Experiment Results}
As shown in Fig.~\ref{fig:ablation}, the GA-based framework demonstrates significant advantages. In the short-term experiment within the first 35 rounds, the w/o GA approach might show slight initial superiority due to the larger changes brought about by replacing the entire strategy. However, overall, w/ GA performs better than w/o GA. This difference increases as the number of rounds grows, particularly after round 35, where the advantages of w/ GA become even more pronounced. The GA process enables effective strategy evolution and adaptation, leading to an increased number of dialogue turns and improved accuracy, highlighting the framework's enhanced adaptability in the long term.
%Despite occasional performance dips during the evolutionary process, the GA framework’s ability to foster strategy diversity and handle complex scenarios makes it a more effective approach for sustained optimization.
\begin{figure}[h!]
    \centering
    \includegraphics[width=\linewidth]{figures/ablation1_v6.png}
    \caption{Performance with/without GA}
    \label{fig:ablation}
\end{figure}
\setlength{\fboxrule}{0.5pt} 
\vspace{0.5em}
\noindent
\begin{tcolorbox}[colframe=black!20, colback=gray!10, arc=5pt, boxrule=0.5pt, width=0.99\linewidth]
\textit{Answer to RQ3}: The results confirm the effectiveness of the GA component in our framework, especially when the number of rounds increases, where it demonstrates greater stability and adaptability. Although the optimization may be slower in the early stages, GA provides stronger adaptability in the long term through effective strategy evolution.
\end{tcolorbox}

\subsection{Discussion and Limitation}
In this study, we leveraged LLM agents to simulate the evolution of language strategies under regulatory pressure. While our results provide initial evidence that agents can adapt and develop covert communication tactics, the simulations also exhibit noteworthy instabilities. First, the inherent randomness of LLM generation can cause significant fluctuations in outcomes: the same prompts may yield different strategic responses, particularly when the experimental scale (number of agents or dialogue rounds) is limited. In our framework, LLMs not only generate dialogues but also determine strategies and regulatory responses; as a result, any stochasticity is compounded across multiple modules, making the final results sensitive to small variations in prompt inputs or random seeds. Although such variability partially reflects the diversity of real-world human behavior to some extent, it complicates the interpretation of findings in a controlled experimental setup.

A second limitation lies in the relatively narrow scope of language strategies observed. The agents predominantly relied on general-purpose evasive methods, such as analogies or implicit references, yet rarely produced fully “encrypted” or specialized code words that might arise in realistic cultural or social contexts. This outcome highlights the challenge that LLMs, pre-trained on broad domains and further refined via RLHF, are predisposed to generate text consistent with mainstream norms, thereby inhibiting the formation of highly unconventional or obscure expressions. Moreover, in scenarios where the training corpus lacks sufficient examples of subcultural or community-specific covert language, the model is less able to invent or adopt specialized linguistic forms. 

Finally, our experiments focused on one-to-one private interactions that emphasize regulatory evasion, without exploring the dynamics of public, many-to-many conversations where language strategies might evolve and propagate differently in a broader social context. While each participant agent does learn and adapt incrementally across dialogue rounds, real-world language evolution involves extensive, long-term propagation across diverse communities. Covert terms or code words may gradually gain acceptance, be modified by different user groups, or fade from use entirely. By contrast, the small-scale nature of our simulated dialogues means that emergent language strategies do not undergo the sustained diffusion and feedback processes characteristic of real social platforms, limiting the ecological validity of our findings.




%对于语言演化的社会类模拟仍然是一个未被开拓的领域,通过借助LLM优秀的自然语言处理能力,为这类自然语言的模拟带来了强大助力。然而伴随着实验也让我们发现LLM也会导致许多局限性。尽管通过实验初步证明了我们的框架的有效性。但同时伴随着实验也为我们带来了许多值得讨论的点。

%实验结果的不稳定性
%首先实验结果本身具有一定的不稳定性,而我们认为整个不稳定性的根源源自于LLM本身生成具有不确定性\cite{},在我们的框架中,LLM几乎参与到了所有环节。同样的violation log让同一个LLM在相同的设置内可能会总结出不同的constraint strategy。尽管这种不稳定性在现实中同样存在(例如不同的人采取不同的策略),同时也是作为模拟框架中非常重要的点,然而在本工作中的数量级的实验中这种不稳定性对结果的影响更为难以过滤。就像\ref{}中也证实的,这种LLM dirven agent的研究中在小数量级上的实验存在着不稳定性,我们认为目前的结果已经足够证实我们的框架可以初步模拟语言动态的学习和演化这一趋势,在今后工作中更大量级的实验中(例如数万数百万agent于更多的round数),我们有理由相信,整体趋势会更加稳定,不同llm的agent之间的性能差距会更加接近llm本身语义理解与生成的综合性能,

%模拟策略的局限性,
%从实验中我们观察到,agent模拟出的策略目前仍然主要集中于比喻类比等较为共通的方式。现实中语言的演化一般根植于当地的文化与经济背景等等因素。例如中文可以利用拼音来将汉字转化为对应的字母从而规避审查,而英文可能会更加积极的利用emoji来作为表达的替代从而规避监管。
%这些较为复杂的策略不仅需要对应环境的大量先验知识,在较为常见的语言中,LLM中训练所需的语料知识可能包含了这些,但是对于训练的数据集中欠缺的语种的知识LLM在不借助prompt的提示的情况下没有能力选择这些既存的策略。


%尽管LLM训练中的数据集可能存在这种更为隐晦的表达方式,首先LLM的RLHF\ref{}本身的训练方法导致了目前绝大多数的LLM为了保证生成文本的泛用性,被训练的更加愿意生成更符合大众的一般化输出文本,在不对LLM进行微调的前提下很难提高在这种特性领域的表现。
%LLM的表现严重依赖prompt的结构设计,提示词工程已经被证明可以有效提高LLM的某一方面能力,单次的基于prompt的模型交互很难实现多步推理或是规划。尽管我们的框架已经将语言演化这一现象解耦,通过多个模块来尽可能模拟人类在该环境中内在的动力学,但是目前的策略生成阶段
%这一部分在不适用复杂prompt工程的前提下LLM很难采用这种小众?特殊领域?的表达。
%对于模拟出的语言策略,我们发现很少的独特加密语言,因为这种需要两边有一套共用的体系,对于我们的模拟情景只有固定turn数的模拟很难形成意思传达。


%\jialong{第二是演化后的语言是如何的存活。我们只考虑了能不能躲避监管。但语言后续的存活和发展其实是更大范围的society的一个动态过程(而不是几个agent之间的交互),这一块可以结合那些上千LLM agent的研究框架来进行拓展}
%\jialong{这边可以多用语言学的角度来说不足之处}
%\jialong{第一个缺点是语言演化一般根植于根植于文化,经济背景,当地的文化背景。但我们的文章没有考虑特定文化背景下的演化。例如中文中可以借用拼音与汉字之间的关系来作为回避监管的方式,日语则可以通过XXX,英语则可以通过XXXX。未来可能要借助persona和role-play之类的设定来进一步拓展}

%更大规模的实验
%策略生成那里增加多步规划
%RAG提供更多语料
%


\section{Discussions}

% \subsection{Bridge the gap between insights and expressions}



\noindent\textbf{Bridge the gap between insights and expressions with AI-powered domain-focused video creation.}
% video creation for different domains
As images and videos continue to dominate communication mediums, visualization and video technologies have become essential tools for enabling diverse domains and the public to express themselves effectively. Emerging generative AI tools, such as Sora~\cite{sora} and Pika~\cite{pika}, exemplify this trend by facilitating creative expression across various fields.

While general AI-driven video creation tools are increasingly popular, our work emphasizes the critical need for domain-specific video creation tools like \SB{} to address unique requirements within specific fields. There are two primary reasons for prioritizing domain-specific video creation over general generative technologies.
% 
First, domain-specific videos, such as sports highlights, rely heavily on human insights. Audiences seek to learn from professionals through these videos, requiring tools that provide greater user control and enable experts to effectively translate their insights into engaging content. 
% \SB{} supports this by enabling users to maintain control over the conveyed insights, ensuring that the final video accurately reflects expert knowledge and user intentions.
% 
Second, the complexity of domain-specific data, such as the intricate motion and strategy analysis, demands advanced data visualization and seamless synchronization of visuals and audio, which general tools may not provide. 
% \SB{} addresses these needs by providing specialized tools that cater to the detailed and dynamic nature of sports content.

\SB{} addresses these needs by integrating automation with customizable visualizations, tailored to the intricate and dynamic nature of sports content. It allows flexible user control through embedded interactions, 
reducing technical barriers and empowering users to effectively communicate their insights. Feedback from users further underscores the importance of balancing automation with user control to accommodate diverse goals and preferences to enhance accessibility across various user groups and use cases, such as tactical analysis, skill development, and profile building. 
% For instance, professional coaches can use \SB{} to create detailed breakdowns of game strategies for training and coaching. Parents and young athletes can produce polished highlight reels for recruitment.
% These examples illustrate how AI-driven tools can empower users across various levels and industries to create videos with meaningful insights, fostering deeper engagement and broader impact. 

Beyond sports, similar tools have the potential to transform fields like healthcare and education, incorporating precise visual aids and step-by-step breakdowns. 
% These applications highlight the transformative potential of tailored video content in amplifying personal expression and benefiting broader audiences.
% 
Future research is required to investigate the balanced integration of AI and intuitive interface design, such as multi-modal interaction~\cite{wang2024lave}, to further advance domain-specific video creation and expression across diverse fields.
% By continuing to develop and refine domain-specific video creation tools, we can unlock new possibilities for effective communication and expression in numerous fields, ultimately bridging the gap between insights and their visual expressions.

% \subsection{Cross sports visualizations - allow different sports domains to leverage other sports' insights}

% \subsection{Enhance human-AI collaboration - creators focus on content while AI helps with editing tasks}


\vspace{1mm}
\noindent\textbf{Promote visualization in practice through real-world system deployment.}
Our work on SportsBuddy advances existing research in sports visualization and video authoring by emphasizing real-world system deployment and evaluation. Through this study, we have identified two significant benefits.

First, deploying SportsBuddy in authentic environments allowed us to validate and refine our design based on genuine use cases and users, uncovering insights that controlled laboratory settings cannot capture. For instance, we discovered that even within a similar user group of content creators, priorities varied significantly—some focused on showcasing player actions, while others emphasized strategic communication. This diversity led to iterative design improvements that balanced the distinct needs of each user group and support customization without complicating user interactions. 

Second, real-world deployment enables the assessment of long-term impacts and the discovery of unique use cases by diverse users. 
For example, some sports experts were hesitant to adopt SportsBuddy initially despite the perceived usefulness they shared. Upon further investigation, this was due to the context-switching costs. This feedback highlighted the necessity for a streamlined workflow tailored to the sports domain, leading to our design of batch processing and web import options. In addition, we observed many users preferred embedded annotation with \Text{} features over typical captions for sharing insights (see Fig.~\ref{fig:case_study}d), suggesting a new form of video storytelling inspired by \SB{}’s design. 
Feedback and insights from our diverse user base has highlighted the value of creating flexible and accessible visualization tools, which offers important external validity of the human-centered system.

This real-world deployment approach not only enhances visualization literacy and accessibility but also ensures that innovative designs translate into practical, widely usable tools, providing a validation for interactive visualization design. Therefore, we advocate for more visualization research to focus on real-world system deployments and to share design learnings, inspiring use cases that are both practical and impactful.

{
\subsection{Future Work}

While SportsBuddy has shown great potential in simplifying sports video storytelling, 
there are key areas for further improvement:

\vspace{1mm}
\noindent\textbf{Enhancing Player Tracking Under Occlusion and Motion Changes.}
The current tracking system faces challenges with occlusions and rapid motion in dynamic scenarios. Future work will refine tracking algorithms using larger domain-specific datasets and multi-view setups to improve accuracy in complex environments.

% The current tracking system struggles with occlusions and rapid motion changes in crowded or dynamic scenarios. Future efforts will focus on refining tracking algorithms using more extensive domain-specific datasets and, where feasible, incorporating multi-view camera setups for improved accuracy. These enhancements aim to ensure reliable tracking in complex sports environments.

\vspace{1mm}
\noindent\textbf{Addressing Perspective and Camera Movement.}
Shifts in camera angles or perspectives cause misalignment issues due to reliance on fixed transformation matrices. Dynamic court mapping and machine learning for real-time adjustments, along with camera metadata integration, will ensure consistent and accurate visualizations.

% Misalignment issues arise when camera angles or perspectives shift, as the system relies on a fixed transformation matrix. Future work will explore dynamic court mapping techniques and machine learning methods for real-time adjustments. Incorporating camera metadata will further enhance visualization accuracy, ensuring effects remain consistent with the game’s context.

\vspace{1mm}
\noindent\textbf{Supporting Longer Videos.}
Longer or higher-resolution videos can strain browser performance. To mitigate this, we will implement dynamic video loading from cloud storage and on-demand decoding, and adopt frame compression during previews to further optimize memory usage and rendering, ensuring smoother video processing.
% Longer or higher-resolution videos may strain browser performance. To address this, dynamic video loading from cloud storage and on-demand decoding will be introduced. Additionally, frame compression during previews will reduce memory usage and rendering time, enabling smoother processing of large and complex videos.



\vspace{1mm}
\noindent\textbf{Extending to Other Sports.}
\SB{} currently focuses on basketball but can expand to sports like soccer and tennis. This requires adapting tracking algorithms and designing sport-specific visualizations to accommodate the unique dynamics and storytelling needs of each sport.

}


% We advocate for more visualization paper that focus on deplyong system in real-world and evaluate their usage for two reasons. 
% 1. In vis research, application paper often address specific domain problems and create a prototype to evaluate with domain experts in a controlled setting. Most projects stop after user evaluation in the lab and the paper is published. With visualization system in real-world that value the practicality of system design and deployment in the wild, it encourages promoting real-world impact brought by novel visualization design, which is crucial in the current visualization community as we promote literacy and accessiblity of visualizations.
% 2. we should also promote long term impact of visualization design, and identify real-wordl use case and learning that might be drastically different from design study that are typically in lab, with a small amount of users, typically university students or academic members.


This work presented \ac{deepvl}, a Dynamics and Inertial-based method to predict velocity and uncertainty which is fused into an EKF along with a barometer to perform long-term underwater robot odometry in lack of extroceptive constraints. Evaluated on data from the Trondheim Fjord and a laboratory pool, the method achieves an average of \SI{4}{\percent} RMSE RPE compared to a reference trajectory from \ac{reaqrovio} with $30$ features and $4$ Cameras. The network contains only $28$K parameters and runs on both GPU and CPU in \SI{<5}{\milli\second}. While its fusion into state estimation can benefit all sensor modalities, we specifically evaluate it for the task of fusion with vision subject to critically low numbers of features. Lastly, we also demonstrated position control based on odometry from \ac{deepvl}.
% \smallskip
% \myparagraph{Acknowledgments} We thank the reviewers for their comments.
% The work by Moshe Tennenholtz was supported by funding from the
% European Research Council (ERC) under the European Union's Horizon
% 2020 research and innovation programme (grant agreement 740435).


%% if specified like this the section will be committed in review mode
% \acknowledgments{
% The authors wish to thank A, B, C. This work was supported in part by
% a grant from XYZ.}

%\bibliographystyle{abbrv}
\bibliographystyle{abbrv-doi}
%\bibliographystyle{abbrv-doi-narrow}
%\bibliographystyle{abbrv-doi-hyperref}
%\bibliographystyle{abbrv-doi-hyperref-narrow}

\bibliography{template}
\end{document}
