\section{\SB{} Design}
\subsection{Gaps in Sports Video Storytelling
}
To design an easy-to-use interactive sports video storytelling tool, we conducted over 50 user interviews with diverse sports roles with the need to share insights through game videos. These roles cover coaches, athletes, trainers, parents, marketing specialists, media producers, self media content creators, and sports fans. 

While their needs for creating and sharing sports videos differ, they all aim to convey their insights in videos to effectively tell stories and engage their audience. The majority of these sports domain users found high entry barriers of existing video editing tools due to lack of proper skills and time. Three gaps were identified: 

\textbf{1. Popular video editing tools have steep learning curve and lack features tailored to sports videos.} Media professionals, such as YouTubers and media specialists, commonly use video editing tools like Adobe Premiere Pro~\cite{premiere}, iMovie~\cite{imovie}, or web platforms like CapCut~\cite{capcut}. However, other sports roles often lack the skills to use these tools, facing a steep learning curve. Even for media professionals, these tools lack sports-specific features, requiring manual customization, such as player tracking or tactical annotations, which is both time-consuming and challenging. For non-video professionals, creating effective sports highlights is often out of reach, despite the significant impact such highlights could have on communication and profile building. This gap makes it difficult for many in the sports industry to leverage video storytelling effectively.
\textbf{Design Implication:} An effective sports highlight creation tool should provide an easy workflow with dedicated sports-specific features.


\textbf{2. Limited access to sports-specific video analysis and creation tools.} 
While tools like Hudl~\cite{hudl} and WSC Sports~\cite{wscsports} offer video clipping and analysis for sports, they are often expensive and designed for team-level access, making them inaccessible to individual coaches, athletes, and creators.
Additionally, these tools are primarily geared toward game analysis or recaps rather than creating sports highlights that effectively convey insights. 
Solutions for sports broadcasters, like Piero~\cite{piero} and Viz Libero~\cite{viz-libero}, allow for the easy addition of sports graphics and annotations, but are typically costly and require complex setups within a media production pipeline. 
As a result, most individuals in sports ecosystem lack access to these specialized tools, limiting their ability to leverage video for their storytelling. 
\textbf{Design Implication:}
A sports video highlight tool tailored to individual users should be accessible, flexible, and function independently without the need for complex team setups.

\textbf{3. Disruptive video creation workflow is distracting and time consuming.} 
Videos play a crucial role in the workflows of many sports professionals, such as coaching and performance analysis. While they often have systems in place to manage game clips, creating highlight videos with personal insights typically requires manual switching between multiple tools and labor-intensive editing to produce meaningful content.
This additional burden diverts professionals away from their primary tasks. Consequently, the effort required to create effective highlights often outweighs the benefits, discouraging regular use of video editing tools to communicate insights.
\textbf{Design Implication:} An effective sports highlight creation tool should seamlessly integrate video management and editing workflows to streamline the creation process.



\subsection{Design Goals}
As \SB{} targets sports domain experts and fans to allow them to convey personal insights through video storytelling effectively, to address the aforementioned gaps,
we aim to design an interactive system that supports easy highlight creation tailored to sports videos and individuals with four design goals below:

\textbf{G1: Intuitive sports features and interactions.} Video editing softwares can be overwhelming for beginners due to unintuitive interfaces and complex steps that distract from the communication intent. For example, adding an animated arrow requires navigating menus, adjusting settings in separate panels, and applying animation in a disconnected interface. This process is far less intuitive than how sports experts simply draw arrows on a tactical board to convey their insights.
To make video editing more accessible for sports experts, tools should provide easily discoverable features and context-embedded interactions that closely align with user intent to lower the entry barriers.

\textbf{G2: Object level visualizations.} Naturally, storytellers describe game insights from the human perspective and apply visual aid to bring attention to the players and their actions, such as highlighting the shooter with a spotlight. Traditional video editing method, which operates at the graphic level, require creators to move a spotlight layer frame-by-frame and manually follow the player position, which is time consuming and tedious. We aim to eliminate such need by automatically attaching visualization to the object based on player tracking and segmentation, allowing easy addition of player-specific visualizations, which aligns better with users' mental model.
% Therefore, visualizations should be applied at the object level to shorten the gap between insights and video storytelling.


\textbf{G3: Seamless integration of visuals and narratives.} Sports insights are typically shared through visual highlights, tactical drawings, and narratives alongside the game footage. Traditionally, coaches and broadcasters rely on voiceovers and pointers or separate tactical boards to explain key moments. However, insights tied to specific players or actions on the court often require direct integration into the video for better clarity and engagement.
To effectively communicate insights in sports highlights, tools must provide features to integrate visual aids and narratives into game footage easily.

% \textbf{G4. Promote human-AI partnerships.}
% AI can significantly simplify sports video creation by automating tedious tasks, such as synchronizing visuals and audio or adding captions. At the same time, it is crucial to provide users with full control to ensure the results align with their intent, especially when AI output fails to meet user needs. 
% Effective tools should combine automation with user input through an intuitive interface, enabling users to leverage sports videos for expression with both ease and control.

\textbf{G4: Streamlined end-to-end video creation process.}
To empower users to leverage videos for sharing their insights effectively, it is crucial to support their workflow with the fewest manual steps possible. From importing video clips and editing to sharing, these steps should have minimal context switching to allow an intuitive and smooth user experience. 


% Gesture UI: use intuitive gesture-based interaction to support editing 
% Click on players
% Drag timeline
% Instant preview
% Language-based UI: support chat feature to create captions, and interactive video authoring 
% Multi-modal interaction: combining the two to support easier human-AI collaboration 


\begin{figure}[t!]
    \centering
    \includegraphics[width=\linewidth]{pictures/visualization.png}
    \vspace{-5mm}
    \caption{Nine visualization features are provided under \Highlight{} Tab. Users can directly interact with the video to apply effects, such as clicking on a player (a) or drawing on canvas (b-d). These features support communicating sports insights, including highlighting players (a)(e), illustrating tactics (b-d)(f), and annotating actions (g).} 
    \label{fig:vis-feature}
    \vspace{-3mm}
\end{figure}

\subsection{Visualization Design}
\label{sec:vis-feature}
% Sports visualizations for depicting insights in game videos need to encode spatial and temporal information. Therefore, we need to support easily adding visualizations in correspondence to the spatial data, and allow easy adjustment of timing/duration of such visualizations.

Based on our user interviews and literature reviews, sports insights in relation to game performance can be categorized into three main types: 
\textbf{1) Emphasize key players}, where the storyteller wants to \textit{bring attention to a player's motion or position};
\textbf{2) Illustrate tactical strategy}, where the user wants to \textit{showcase the detail breakdown of action sequences}, often involving the spatial relationship among players/teams and temporal movement;
\textbf{3) Associate insights to the performance}, where the user wants to \textit{directly annotate decisions or outcomes} to link their insights to the game actions.

To support each insight type with video storytelling, we design nine visualization features under three categories, including player, tactic, and action, as shown in Fig.~\ref{fig:teaser} (a2):

\begin{itemize}[leftmargin=*]
    \item \textbf{Player highlights.} To bring attention to key players, \Circle{} and \Spotlight{} effects apply visual overlay on individual players, while \Connector{} links two or more players to emphasize their spatial relations. For example, \Circle{} and \Spotlight{} are frequently used to focus on key players involved in the play or have outstanding performance, such as the scoring player. \Connector{} is most popular in team sports like soccer, to bring attention to certain attack or defense formation (Fig.~\ref{fig:vis-feature} (e)).

    \item \textbf{Tactic highlights.} To provide deeper insights in sports highlights, it is essential to support tactical drawing, a common practice on sketch boards, directly within the video. As shown in Fig.~\ref{fig:vis-feature} (b)-(d), \Path{}, \Zone{} and \Marker{} allows users to annotate player trajectories, areas, and positions using sports-specific symbols, such as ``O" for offense players and ``X" for defense. To mimic the intuitive interaction of a tactic board, users can draw symbols and lines directly on the video. To prevent visual clutter from excessive annotations,
\BGFilter{} applies a background color to the video, bringing players and tactic visualizations to the forefront with clear contrast (Fig.~\ref{fig:vis-feature} (f)). Additional video effects, such as \textit{Freeze} frame and speed adjustments, serve as complementary tools to further emphasize tactical insights.

    \item \textbf{Action highlights.} 
To connect the insights to game performance, users can directly highlight actions in the video with their annotation. \Zoom{} feature allows direct change of video view to zoom in and follow a selected player. 
\Text{} feature support both adding text at a fix position or track a player, allowing users to directly label actions or insights in-related to the game action, such as a player's move or outcome. Three text label placement options, head, waist, and ground, also provide more flexibility for annotating different insights, such as showing the observation ``Open look'' below the player, the action ``Cut'' on the waist, and the outcome ``Score!'' above the player's head (Fig.~\ref{fig:vis-feature} (g)).
\end{itemize}

Overall, these sports-specific visualizations are designed to address the most common insights related to players, tactics, and actions, drawing inspiration from basketball, soccer, and feedback from coaches and creators. 
In addition, with its easily adjustable timeline tracks, \SB{} highlights features empower users to compose and customize visualization features flexibly, allowing them to craft their own communication toolkit and styles to effectively convey their insights.


\begin{figure*}[ht!]
    \centering
    \includegraphics[width=\linewidth]{pictures/user_journey.png}
    \vspace{-6mm}
    \caption{A demonstration of a coach adding insights with visualization features on \SB{}: starting with uploading a clip (a), highlighting player movements and tactics (b-d), and annotating insights with a video freeze frame, background filter, and text labels (e-f).
    } 
    \label{fig:user_journey}
    \vspace{-3mm}
\end{figure*}


\subsection{User journey}
To illustrate the practical application of \SB{}, we present a coach's user journey in creating an engaging sports highlight to explain a tactical breakdown to the team, as shown in Fig.~\ref{fig:user_journey}.

% \begin{adjustwidth}{0.25cm}{0.25cm}
% (a) 
Coach Mike uploads a game clip to \SB{} under \Media{} (Fig.~\ref{fig:user_journey}a). In the clip, his team executes smooth passes and cuts, resulting in an easy dunk by the center. Mike wants to highlight the timing of player actions and provide positive feedback to the team .
% (b) 
He starts by opening \Highlight{} and selecting the \Spotlight{} feature (Fig.~\ref{fig:user_journey}b). Clicking on player \texttt{P1}, who makes a pass to \texttt{P2} and cuts to the opposite corner, Mike easily applies a spotlight effect that automatically tracks \texttt{P1}. He adjusts the spotlight's duration in the timeline to bring attention on the cutter's movement. 
Next, he uses the \Path{} tool to draw the cutting path, illustrating the correct route (Fig.~\ref{fig:user_journey}c).
To emphasize follow-up actions by the other players, he adds \Circle{} effects to highlight \texttt{P2} and \texttt{P3} as they pass the ball (Fig.~\ref{fig:user_journey}d). 
When \texttt{P3} gains possession, he highlights two options: passing back to \texttt{P2} for a three-point shot or passing to \texttt{P5} for a dunk. To emphasize these options, Mike freezes the frame by clicking on the freeze icon, applies a \BGFilter{} to grey out the background, and uses the \Spotlight{} tool to bring focus on \texttt{P2} and \texttt{P5} (Fig.~\ref{fig:user_journey}e). After the pass, he annotates \texttt{P5} with a \Text{} effect, adding the label ``Score!'' to highlight the successful execution (Fig.~\ref{fig:user_journey}f).

Finally, Mike switches to \Narrative{} to add captions explaining the tactic. He finishes the highlight and clicks on \textit{Export} to easily share this game highlight link with his team. 
The entire creation process took Mike less than 5 minutes.
% \end{adjustwidth}