\section{Related Work}
We review previous work on data videos for storytelling, video-based visualizations in sports, and authoring tools for data videos.

\subsection{Data Videos for Storytelling}

%% Data Storytelling through Video
Storytelling plays a vital role in communicating data, providing a structured and engaging means of sharing insights~\cite{DBLP:journals/tvcg/SegelH10}. 
Data videos combine visual elements, auditory, and raw footages to create compelling stories backed by data. 
Researchers have extensively explored the design space of data videos, 
offering valuable insights for creating and authoring them.

For example, Amini et al.~\cite{DBLP:conf/chi/AminiRLHI15} analyzed 50 data videos from diverse sources and used Cohn's visual narrative structure theory~\cite{DBLP:journals/cogsci/Cohn13} to categorize these videos into four narrative structures.
Their findings unveiled design patterns and offered implications for the development of data video authoring tools.
Similarly, Thompson et al.~\cite{DBLP:journals/cgf/ThompsonLLS20} developed a design space for animated data graphics to support the development of future authoring tools. 
Cao et al.~\cite{DBLP:journals/vi/CaoDCWSZT20} analyzed 70 data videos and proposed a taxonomy for  narrative constructs, broadening our understanding of data storytelling.
More recently, researchers have explored incorporating cinematic techniques into data video design.
Xu et al. contributed a series of works, including guidelines for creating cinematic openings~\cite{DBLP:conf/chi/XuYY0WQ22}, endings~\cite{DBLP:conf/chi/XuWYWHYQ23}, and overarching structures~\cite{wei2024telling} in data videos. 
Their studies, inspired by cinematic storytelling, analyze film and data video examples to provide systematic guidance for practitioners.

While these works are relevant, sports video storytelling introduces unique challenges compared to traditional data video narratives.
Sports video storytelling focus on enhancing real-world footage with visual elements, such as overlays, annotations, and real-time data visualizations.
This setting presents a distinct design space with unique requirements and techniques.
Zhu-Tian et al.~\cite{chen2021augmenting} explored this design space, identifying six narrative orders and common usage scenarios to provide practical guidance for designing authoring tools tailored to this domain.
Building on prior work, \SB{} was developed as an augmented sports video authoring tool, inspired by insights from both traditional data video storytelling research and the unique requirements of sports storytelling.

\subsection{Video-based Visualizations for Sports}
Given the inherently visual and media-rich nature of sports, 
video-based visualizations have become a crucial format for sports data, 
as highlighted by Perin et al.~\cite{DBLP:journals/cgf/PerinVSSWC18}. 
\SB{} specifically focuses on embedded visualizations in videos~\cite{DBLP:journals/tvcg/WillettJD17}, where data visualizations are seamlessly integrated into the video as part of the scene.

In academia, a representative example is the system developed by Stein et al.~\cite{DBLP:journals/tvcg/SteinJLBZGSAGK18}. 
Their system processes raw footage of soccer games to automatically visualize tactical information as graphical marks within the video. 
Following their work,
Zhu-Tian et al. developed a series of human-AI collaborative tools that support data visualization in sports videos, ranging from racket-based sports\cite{chen2021augmenting, chen2022sporthesia} to team-based sports~\cite{chen2023iball}. 
Beyond system development, Zhu-Tian et al.~\cite{chen2021augmenting} and Lin et al.~\cite{lin2022quest} have also explored the design space of embedded visualizations in sports videos.
In industry, the strong market demand has led to the emergence of highly successful commercial systems, as pointed out by Fischer et al.~\cite{DBLP:journals/corr/abs-2105-04875}. 
For instance, Piero~\cite{piero} and Viz Libero~\cite{viz-libero} are widely used in sportscasting, offering powerful functionalities for editing and annotating sports videos. 
Additionally, Second Spectrum~\cite{secondspectrum} automatically tracks players' positions and embeds real-time status information to enhance audience engagement.
Yet, these commercial systems often cater to proficient video editors, resulting in a steep learning curve for sports analysts. 

\SB{} seeks to bridge the gap between academic research and industry by developing an easy-to-use tool targeted at real-world users. 
By simplifying workflows and making advanced video-based sports visualizations accessible, 
\SB{} aims to transition academic insights into practical solutions for broader adoption.

\subsection{Authoring Data Videos}
While data videos are a powerful medium for storytelling, creating them requires significant expertise in data analysis and video crafting. General-purpose video editing tools like Adobe Premiere Pro~\cite{premiere} allow users to manipulate visual elements through animation keyframing and presets, but they often involve extensive manual effort and a steep learning curve. 
To lower the entry barrier, researchers have developed systems to simplify the creation process.

Early systems like DataClips~\cite{DBLP:journals/tvcg/AminiRLMI17} introduced template-based approaches, enabling semi-automatic data animation by reusing patterns from existing examples. Building on this foundation, tools such as WonderFlow~\cite{DBLP:journals/corr/abs-2308-04040} and InfoMotion~\cite{DBLP:journals/cgf/WangGHCZZ21} began automating more complex aspects of storytelling, including synchronizing animations with audio narrations and applying animations to static infographics by analyzing their structures. Systems like Gemini~\cite{DBLP:journals/tvcg/KimH21} and its extension Gemini2~\cite{DBLP:conf/visualization/KimH21} focused on refining the transition process by recommending staged animations, offering designers guidance for creating smooth and semantic transitions. 
Meanwhile, tools like AutoClips~\cite{DBLP:journals/cgf/ShiSXLGC21} and Roslingifier~\cite{DBLP:journals/tvcg/ShinKHXWKKE23} emphasized automation and customization, with the former generating videos from data facts and the latter supporting semi-automated narratives for scatterplots. 
Recent advances, such as Live Charts~\cite{DBLP:journals/corr/abs-2309-02967} and Data Player~\cite{DBLP:journals/tvcg/ShenZZW24}, further integrate intelligent automation, using large language models to link narration and visuals, while Data Playwright~\cite{DBLP:journals/corr/abs-2410-03093} pushes the boundary by introducing natural language-driven video synthesis through annotated narration. 
Together, these tools represent a progression from template-based semi-automation to intelligent systems that combine automation, user input, and natural language interaction, making data video authoring increasingly accessible and efficient.

\SB{} differs from these systems in two key aspects: (1) it focuses on authoring augmented sports videos with embedded data visualizations rather than general-purpose data videos; and (2) it emphasizes real-world deployment by collecting real user data to refine its usability and functionality. This approach bridges the gap between academic research and practical application, addressing the unique challenges of sports video storytelling in authentic settings.

