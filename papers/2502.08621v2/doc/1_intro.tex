\firstsection{Introduction} %for journal use above \firstsection{..} instead

% background - video highlights 
Videos are essential to deliver values in the sports world. Whether through TV broadcasting, social media, film reviews, or highlight reels, they empower sports professionals to record, analyze, and refine their performance while captivating audiences for both educational and entertainment purposes.
Particularly, sports highlights—short videos capturing key moments—provide an effective format for sharing insights with a wide audience. Effective sports highlights are often enhanced with visual aids or narratives to better convey stories and guide viewer attention,
leading to growing demands on social media platforms like Youtube and Tiktok~\cite{genz}. 
% gap
However, creating sports highlights still requires extensive video editing skills, making it challenging to effectively express personal insights for the majority of sports roles, including billions of coaches and athletes globally. 

% problem
Existing solutions, such as popular video editing software like Adobe Premier Pro~\cite{premiere}, have a steep learning curve and are tedious to operate. Due to the dynamic nature of sports, narratives and focal points can shift moment by moment, requiring precise synchronization of insights and videos. As a result, content creators report spending hours to produce even minute-long highlights. 
% 
Furthermore, making effective sports highlights often requires sports-specific graphics and animations to explain tactics and emphasize key targets, typically unavailable to non-media professionals. As a result, sports experts with deep insights but limited video creation skills still rely on manual annotations (e.g., tactic boards) and verbal explanations to convey their ideas, which limits their ability to effectively engage audiences in team or public settings.

% research 
% 
To bridge the gap between sports insights and video storytelling,
prior research leverage visualization and AI techniques and propose new ways of sports highlight authoring. Most prominently, data-driven approach~\cite{chen2021augmenting} detects objects and actions in sports videos and directly map visualizations to the objects, eliminating the need for manual frame-by-frame graphic placement with traditional video editing approach. 
Intuitive interaction such as gesture and natural language could drastically ease the video authoring workflow~\cite{chen2022sporthesia,chen2023iball}. 
In addition, design framework for sports-specific visualizations were also thoroughly explored to support systematic augmented sports video creation~\cite{lin2022quest, yao2023designing}. 
% 
Despite these recent advancements in data-driven sports video creation, there is still a lack of accessible tools for sports practitioners. Our work aims to put these visualization theories in practice to make real-world impact.

% our work
Building upon the data-driven approach in~\cite{chen2021augmenting} and embedded visualization framework in~\cite{lin2022quest}, we develop \SB{}, an AI-powered sports highlight creation tool, with intuitive web interface that allows easy integration of graphic and narratives into videos. It supports various sports, including basketball, soccer, volleyball, lacrosse, and tennis.
Our system leverages multimodal AI models to detect players and events, and provide easy addition of sports visualizations through embedded interaction (i.e. direct interaction with the objects in the video) and interactive video timeline. 
% user flow
\SB{} supports an end-to-end highlight creation journey, 
including importing video to the \Media{}, adding \Highlight{} features to the video, and adding \Narrative{} with manual subtitles or AI-generated captions.
% 
As sports practitioners create sports highlights to 1) emphasize key players,
2) illustrate tactic execution,
and 3) associate data / insights to performance, we design different visualization categories to collectively support these goals, including visualizations focusing on player, tactic, and action, as shown in Fig.~\ref{fig:teaser} (a).

Since launching \SB{} online in August 2024\footnote[1]{\SB{} can be accessed on https://sportsbuddy.online}, 
we have attracted users from a variety of sports roles, each with distinct needs for creating engaging highlights. Over 150 users have signed up, with approximately one-third being coaches focused on game analysis to improve player and team performance, another third consisting of content creators crafting video stories to engage their audiences, and the remainder comprising athletes and parents creating highlights for building profiles.
A case study with Harvard Basketball and Soccer Teams illustrated the ease of use and value of \SB{}'s sports visualization and video features. These tools empowered team marketing managers to efficiently showcase team achievements and in-depth tactical breakdowns through engaging highlights on social media. Another case study with professional and aspiring content creators emphasized how \SB{} lowers the barrier of entry for creating high-quality sports highlights, enabling new creators with limited video editing skills to master sports video storytelling.  
Continuous user feedback has also led to feature additions and usability improvements, suggesting the positive and growing impact \SB{} has in sports community.

Our paper makes four main contributions: 1) the design and implementation of \SB{} for easy sports highlight creation, 2) the integration of state-of-the-art multimodal AI to enable innovative sports video storytelling features, 3) system deployment and testing in real-world settings with active users and 4) insights into the impact of sports video storytelling derived from first-hand user evaluations. 





% Well-crafted sports highlights support goals that are closely aligned with visual analytics and storytelling, including 1) examine key sports moments (e.g., outstanding skill or critical decision making), 2) associate data to the performance (e.g., reason or outcome of the action) , 3) comprehend multivariate data to appreciate the complexity of strategy and execution (e.g., detailed tactical breakdown).

