\section{Real-world system evaluation}
We have made \SB{} publicly available online\footnotemark[1] since August 2024. 
After three months of deployment, \SB{} has attracted 163 registered users. We collect user background and feedback to evaluate the usefulness and benefits of \SB{}. During these three months, we have also made feature and user experience improvements based on user feedback. 

In this section, we analyze user feedback and insights that inform critical system improvements, followed by two case studies on the usefulness of \SB{} in supporting athletic communication and game analysis and video storytelling on social media.

\subsection{User Feedback}

% In this section, we present an analysis of user feedback and insights gathered from both current and potential users of SportsBuddy.ai. These insights, collected over a three-month period, reflect the system's strengths, areas for improvement, and its perceived impact on user workflows and outcomes. Based on this feedback, we also outline the system enhancements implemented during this time.

% After three months of deployment, \SB{} has attracted 156 registered users. 
Of the 163 users, 66 completed a demographic questionnaire. Among them, 38\% identified as coaches, 20\% as content creators, 14\% as athletes,  12\% as parents,  7\% as fans, 2\% marketing specialists, 
and 7\% as others. In terms of age distribution, 40\% of users were aged 20–30, followed by 29\% aged 30–40, 24\% aged over 40, and 7\% aged below 20.
Based on user feedback collected over three months, we found that \SB{} enhances user workflow and video quality, and engagement with efficiency and creativity. Further, we identified the system's strengths and areas for improvement, leading to several enhancements implemented during this period.

\subsubsection{Usefulness \& Impact}




% \textbf{1. Engagement and Marketing Potential}.
% \textbf{1. Usability and Usefulness}.

% feature/impact

% efficiency
Above all, \SB{} significantly enhances video creation efficiency. Many users reported substantial reductions in editing time, with one stating that tasks previously taking over 30 minutes could now be completed in less than 10 minutes. The automation of complex processes, including tracking player location with player features and visual annotations with \Text{} features, was highlighted as a key time-saver, especially for less experienced users like interns and youth coaches.

% feature
\revision{Quantitative insights gathered during this period provide additional evidence of the system's effectiveness. A total of 1,021 videos were successfully uploaded, with an upload success rate of 90.8\%, and 814 highlights were exported, achieving an export success rate of 87.9\%, with no major disruption reported in the user feedback. These metrics reflect the reliability of the video processing pipeline and its ability to handle diverse video inputs and produce high-quality outputs efficiently. Follow-up interviews with users further validated these findings, with users reporting seamless functionality in AI-enabled features like player tracking and pose estimation.}

% While the system does not collect actual video highlights unless shared publicly, we conducted follow-up interviews to gather feedback on the most useful features. 

The most popular features were \Spotlight{}, \Circle{}, and \Text{} annotations, which users praised for enhancing their storytelling with clear visual guidance and embedded narratives. Some users also explored creative usage of less common features, such as \Connector{} and \BGFilter{}. For instance, a content creator used \Connector{}, typically applied in soccer to highlight team formations, to illustrate defensive setups in basketball. Similarly, \BGFilter{}, perceived as a novel visualization effect for sports highlights, enabled users to create strong visual contrasts, further directing the audience's attention to key player actions and moments.


% outcome
Overall, users consistently acknowledged \SB{}'s ability to enhance audience engagement through visually compelling content. 
Coaches particularly appreciated \SB{} for its precision in identifying areas where players need improvement, enabling clearer and more actionable feedback for athletes. This approach aligns with their principle of \textit{``Show, don't tell},'' enabling effective use of visual demonstration over verbal explanation. 
Social media managers and influencers valued the tool’s ability to create engaging highlight clips quickly, which helped increase their reach and follower engagement.
% 
In addition, parents of youth athletes highlighted \SB{}'s potential for creating high-quality player highlights, which are essential for recruitment, scholarships, and applications to collegiate or youth teams. 
% 
% Many users also expressed enthusiasm for the AI-driven features that can streamline the organization and editing of large video libraries, which can significantly reduce the time and effort required compared to manual methods.

% \textbf{2. Impact on Workflow Efficiency}.
% efficency
% Many users reported significant reductions in editing time, with one noting that tasks previously requiring 30+ minutes could now be completed in under 10 minutes. The automation of complex tasks like player tracking and visual annotations was identified as a major time-saver, particularly for less experienced staff such as interns and youth coaches.

% \textbf{3. Usability and Accessibility}.
% \textbf{2. Areas for  Improvement}.
% While users found the interface intuitive, several expressed concerns about long processing times and limited options for batch processing. These issues hindered the tool's potential for high-volume workflows, such as creating multiple highlight reels or analyzing full-game footage. In response, users suggested adding options for batch uploads and processing to streamline their workflows further.

% \textbf{3. Youth Sports Applications}.
% Parents of youth athletes highlighted SportsBuddy's potential for creating high-quality highlight reels, which are essential for recruitment, scholarships, and applications to programs like the NCAA or youth teams. Many parents expressed enthusiasm about using AI to streamline the organization and editing of large video libraries, a time-consuming task when done manually.

% \textbf{Feature Suggestions}.
% Users found the editing features powerful but rigid. For instance, the zoom-in effect lacked dynamic keyframe control, limiting its flexibility. Additionally, some users noted the tool's focus on short clips (e.g., highlights lasting seconds) made it less suited for creating mid-length or longer videos. Several users also expressed interest in features beyond video editing, including tools for live tactical analysis, automatic detection of key plays, and integration with external applications. Requests included: real-time annotations during training sessions, tactical diagrams integrated into video highlights, automated processing of full-game footage into segmented clips, and customizable animations for more sophisticated visual storytelling.

\subsubsection{Insights for System Improvements}

Users feedback has also highlighted areas for improvement in functionality and feature design, leading to improvements in supporting more streamlined user workflow and scalable function extension.

\textbf{1. Enhanced Video Processing Workflow}. While users found \SB{}'s interface intuitive, several raised concerns about long processing times and the lack of batch processing options. These limitations reduced efficiency for high-volume workflows, such as creating multiple highlight reels or analyzing full-game footage. 
In response, users recommended adding batch upload and processing capabilities to streamline their workflows.


To address processing time concerns, we optimized backend systems with multiple thread processing and better GPUs to improve processing speed by 2 to 3 times, and refined the user interface to specify the progress made, such as showing the current video processing steps and estimated remaining time, to allow a smoother and more transparent user experience. 
Additionally, we introduced a batch upload feature, allowing users to upload multiple clips in one session. This enhancement allows users to have a coherent workflow without interruption.
% 










\textbf{2. Customization and Automation}.
Most users found the sports-specific visualization tools intuitive and effective, but some requested greater flexibility. For example, the \Zoom{} effect lacked dynamic keyframe control, limiting its adaptability. 
% Additionally, the tool’s focus on short clips made it less suitable for creating longer or mid-length videos.
On the other hand, several users expressed interest in features that go beyond video editing to deliver deeper sports insights, such as live tactical analysis and automatic detection of key plays.

To balance diverse user needs with our design goal of simplifying highlight creation for sports professionals, we designed \SB{} to incorporate both customization and automation. Beyond the basic sports features in \Highlight{}, feature panels (Fig.~\ref{fig:teaser}(a3)) are designed to accommodate customization options such as styles and templates, allowing users tailor their content to specific needs. 
Additionally, on top of the typical captioning tool under \Narrative{},
we introduced an AI captioning feature, described in Sec.~\ref{sec:video-processing}, to deliver alternative insights through intuitive interactions.  While this represents just one approach to AI automation in sports highlight creation, our interactive design allows users to easily refine and adjust AI-generated content. This foundation is built with scalability in mind, paving the way for future AI model enhancements and fostering seamless human-AI collaboration.




% To accommodate different user needs, we provided options for customizable styles and colors for visual elements and text, allowing users to better visualize their own tactical insights. This enhancement empowers users to tailor their content to specific audiences and purposes. Additionally, to support high-definition video workflows, we began testing increased file size limits, enabling users to work with full-game footage without the need for pre-editing. These updates address key user concerns about flexibility and scalability, making the platform more versatile for diverse use cases.



% \textbf{1. Enhanced Workflow Automation}.
% Based on feedback about efficiency, we introduced an option to process multiple plays in one session. This enhancement allows users to compile longer clips for improved storytelling and analysis.

% \textbf{2. Customization and Flexibility}.
% To accommodate different user needs, we provided options for customizable styles and colors for visual elements and text, allowing users to better visualize their own tactical insights. This enhancement empowers users to tailor their content to specific audiences and purposes. Additionally, to support high-definition video workflows, we began testing increased file size limits, enabling users to work with full-game footage without the need for pre-editing. These updates address key user concerns about flexibility and scalability, making the platform more versatile for diverse use cases.

% \textbf{3. Improved User Experience}.
% In response to concerns about processing times, we optimized backend systems to reduce queue times. Additionally, we refined the user interface to make the workflow more intuitive, addressing common pain points for non-technical users.

% \textbf{4. Broader Application Scenarios}.
% We began exploring features that align with emerging use cases, such as real-time tactical analysis, integration with scouting workflows, and tools tailored for grassroots coaching environments. These efforts aim to expand the platform’s applicability beyond traditional video editing.



\subsection{Case Study}
To understand how \SB{} supports video storytelling in real-world setting, we conducted two case studies. In the first study, we collaborated with Harvard Athletics Department to enhance the diversity of social media content for basketball and soccer teams. In the second study, we collaborated with basketball influencers on Youtube and Instagram to enhance their content quality and simplify the video editing process. In both cases, we discuss the types of video stories they made, and highlight the unique values \SB{} brought to their workflows and outcomes. 


\begin{figure}[t!]
    \centering
    \includegraphics[width=\linewidth]{pictures/case_study.png}
    \vspace{-6mm}
    \caption{\SB{} enhances the effectiveness and diversity of video creation for athletic communication, game analysis, and storytelling, supporting use cases such as highlighting (a) player synergy, (b) tactics, (c) spatial actions, and (d) player performance.} 
    \label{fig:case_study}
    \vspace{-3mm}
\end{figure}


\subsubsection{Athletic Communication}
Amateur and collegiate teams often lack the advanced tools and resources needed to produce high-quality highlight reels, such as those seen in professional leagues like NBA. \SB{} presents a new opportunity to enrich game highlights with more personal insights that enhances team visibility and fan engagement on social media.
Using \SB{}, marketing professionals and coaches at Harvard Athletics created innovative game highlights with engaging visuals and insightful narratives that were previously unavailable. Seven basketball highlights and seven soccer highlights were produced and shared on Instagram. Theses 14 videos have collectively attracted over 150 million views. 
We gathered user feedback and identified three unique types of insights made with \SB{} highlights.

% type of content
\textbf{1. Highlight player synergy}. Before using \SB{}, teams often share the best game moments to showcase team success, such as a winning shot, often focusing on a single player with multiple camera angles and close-ups. While this content is highly engaging due to its dynamic motion,  it often overlooks the interplay among players critical to team success. With \SB{}, teams can draw attention to the synergy between key players. For example, as shown in Fig.~\ref{fig:case_study}a, a smooth rim run paired with a precise pass between two players was highlighted, showcasing a successful transition play driven by teamwork.

\textbf{2. Tactical breakdowns.} \SB{} also enables teams to seamlessly convey coaching strategies in game highlights, enhancing the depth of their storytelling. In the example shown in Fig.~\ref{fig:case_study}b, a coach's verbal commentary was turned into clear, integrated visual highlights. Traditionally, such insights were conveyed verbally or as text alongside raw game footage, often making them difficult to comprehend on social media. With \SB{}, visual highlights and narratives work together effectively, allowing audiences to grasp the complexity of the strategy without feeling overwhelmed.

\textbf{3. Emphasize spatial actions}. The third type of insights target emphasizing the spatial elements of gameplay. For example, in a soccer scoring highlight (Fig.~\ref{fig:case_study}c), a \Zone{} visualization was used to highlight the scoring area on the net, alongside a \Circle{} marking the kicker. In a basketball clip, the distance between the shooter and the hoop was visually emphasized. These visual enhancements draw attention to the spatial attributes of the action, helping audiences appreciate the precision and difficulty involved.


Overall, the feedback from Harvard Athletics was overwhelmingly positive. Collaborators quickly learned to use the tool and applied its features creatively to support their storytelling needs. The innovative ways in which each feature was utilized highlight \SB{}'s effectiveness in reducing operational burdens while empowering users to focus on the creative process.


\subsubsection{Game Analysis and Storytelling}

Individual sports content creators are often sports fans with varying levels of video or image creation skills. Due to limited time, skills and access to professional tools, they face significant challenges in producing high-quality videos to deliver their unique insights. 

To explore how \SB{} can empower individuals in creating engaging highlights for game analysis and storytelling, we collaborated with three basketball influencers, each offering unique perspectives: \textit{HungKu}, a popular basketball YouTuber with over 100K followers covering NBA news~\cite{hungku}; \textit{Hinbasket}, a basketball-focused Instagram account with 12K followers specializing in game analysis and team tactics~\cite{hinbasket}; and \textit{JNC\_bball}, an Instagram account with 7K followers that analyzes player performance~\cite{jnc}.
With distinct focuses in their highlights, each creator found \SB{} supported their workflow and enhanced their content in unique ways. 

\textbf{1. Enhance video quality and engagement.} 
With the goal to delivers timely NBA news through long-format YouTube videos, HungKu integrates multiple short game clips to support his storytelling. In longer videos, smooth transitions and engaging visuals are essential for maintaining audience attention. With\SB{}, HungKu can create high-quality highlight clips with features that spotlight players and guide audience attention. He reported that \SB{} seamlessly integrates into his workflow, enhancing video quality and effectiveness with minimal effort.

\textbf{2. Create engaging tactical short reel.} Hinbasket creates short highlight to cover key moments from NBA games. He focuses on delivering his insights on team tactics and game performance, often requiring annotation of specific players or movement. Previously, this process was highly time-consuming. With \SB{}, he leverage AI to easily track players and add tactical elements like arrows and labels, making his insights stand out. Overall, \SB{} has significantly reduced his editing time and improved clarity and engagement of his highlight reels.

\textbf{3. Unlock new formats for player analysis.} Before using \SB{}, JNC\_bball primarily shared player performance insights through static images and text, which had already garnered them a loyal following due to their thoughtful and fresh perspectives. However, despite their desire to create video content for more engaging storytelling, they struggled with limited video editing skills. 
With \SB{}, they found the process intuitive and began creating dynamic video stories to analyze player strengths and weaknesses in a more engaging format. Each video integrates multiple game clips,  breaking down insights with player and tactical visualizations to reinforce their main message, as illustrated in Fig.~\ref{fig:case_study}d. 

Overall, our collaborations with sports content creators of diverse needs and audiences demonstrated that \SB{} can effectively support their creative workflows and enhance video storytelling quality, whether by maintaining viewer engagement in long-format videos, clearly conveying team tactics and actions in short reels, or presenting compelling stories through detailed player analysis.
