\documentclass{article} % For LaTeX2e
\usepackage{iclr2025_conference,times}

% Optional math commands from https://github.com/goodfeli/dlbook_notation.
%%%%% NEW MATH DEFINITIONS %%%%%

\usepackage{amsmath,amsfonts,bm}
\usepackage{derivative}
% Mark sections of captions for referring to divisions of figures
\newcommand{\figleft}{{\em (Left)}}
\newcommand{\figcenter}{{\em (Center)}}
\newcommand{\figright}{{\em (Right)}}
\newcommand{\figtop}{{\em (Top)}}
\newcommand{\figbottom}{{\em (Bottom)}}
\newcommand{\captiona}{{\em (a)}}
\newcommand{\captionb}{{\em (b)}}
\newcommand{\captionc}{{\em (c)}}
\newcommand{\captiond}{{\em (d)}}

% Highlight a newly defined term
\newcommand{\newterm}[1]{{\bf #1}}

% Derivative d 
\newcommand{\deriv}{{\mathrm{d}}}

% Figure reference, lower-case.
\def\figref#1{figure~\ref{#1}}
% Figure reference, capital. For start of sentence
\def\Figref#1{Figure~\ref{#1}}
\def\twofigref#1#2{figures \ref{#1} and \ref{#2}}
\def\quadfigref#1#2#3#4{figures \ref{#1}, \ref{#2}, \ref{#3} and \ref{#4}}
% Section reference, lower-case.
\def\secref#1{section~\ref{#1}}
% Section reference, capital.
\def\Secref#1{Section~\ref{#1}}
% Reference to two sections.
\def\twosecrefs#1#2{sections \ref{#1} and \ref{#2}}
% Reference to three sections.
\def\secrefs#1#2#3{sections \ref{#1}, \ref{#2} and \ref{#3}}
% Reference to an equation, lower-case.
\def\eqref#1{equation~\ref{#1}}
% Reference to an equation, upper case
\def\Eqref#1{Equation~\ref{#1}}
% A raw reference to an equation---avoid using if possible
\def\plaineqref#1{\ref{#1}}
% Reference to a chapter, lower-case.
\def\chapref#1{chapter~\ref{#1}}
% Reference to an equation, upper case.
\def\Chapref#1{Chapter~\ref{#1}}
% Reference to a range of chapters
\def\rangechapref#1#2{chapters\ref{#1}--\ref{#2}}
% Reference to an algorithm, lower-case.
\def\algref#1{algorithm~\ref{#1}}
% Reference to an algorithm, upper case.
\def\Algref#1{Algorithm~\ref{#1}}
\def\twoalgref#1#2{algorithms \ref{#1} and \ref{#2}}
\def\Twoalgref#1#2{Algorithms \ref{#1} and \ref{#2}}
% Reference to a part, lower case
\def\partref#1{part~\ref{#1}}
% Reference to a part, upper case
\def\Partref#1{Part~\ref{#1}}
\def\twopartref#1#2{parts \ref{#1} and \ref{#2}}

\def\ceil#1{\lceil #1 \rceil}
\def\floor#1{\lfloor #1 \rfloor}
\def\1{\bm{1}}
\newcommand{\train}{\mathcal{D}}
\newcommand{\valid}{\mathcal{D_{\mathrm{valid}}}}
\newcommand{\test}{\mathcal{D_{\mathrm{test}}}}

\def\eps{{\epsilon}}


% Random variables
\def\reta{{\textnormal{$\eta$}}}
\def\ra{{\textnormal{a}}}
\def\rb{{\textnormal{b}}}
\def\rc{{\textnormal{c}}}
\def\rd{{\textnormal{d}}}
\def\re{{\textnormal{e}}}
\def\rf{{\textnormal{f}}}
\def\rg{{\textnormal{g}}}
\def\rh{{\textnormal{h}}}
\def\ri{{\textnormal{i}}}
\def\rj{{\textnormal{j}}}
\def\rk{{\textnormal{k}}}
\def\rl{{\textnormal{l}}}
% rm is already a command, just don't name any random variables m
\def\rn{{\textnormal{n}}}
\def\ro{{\textnormal{o}}}
\def\rp{{\textnormal{p}}}
\def\rq{{\textnormal{q}}}
\def\rr{{\textnormal{r}}}
\def\rs{{\textnormal{s}}}
\def\rt{{\textnormal{t}}}
\def\ru{{\textnormal{u}}}
\def\rv{{\textnormal{v}}}
\def\rw{{\textnormal{w}}}
\def\rx{{\textnormal{x}}}
\def\ry{{\textnormal{y}}}
\def\rz{{\textnormal{z}}}

% Random vectors
\def\rvepsilon{{\mathbf{\epsilon}}}
\def\rvphi{{\mathbf{\phi}}}
\def\rvtheta{{\mathbf{\theta}}}
\def\rva{{\mathbf{a}}}
\def\rvb{{\mathbf{b}}}
\def\rvc{{\mathbf{c}}}
\def\rvd{{\mathbf{d}}}
\def\rve{{\mathbf{e}}}
\def\rvf{{\mathbf{f}}}
\def\rvg{{\mathbf{g}}}
\def\rvh{{\mathbf{h}}}
\def\rvu{{\mathbf{i}}}
\def\rvj{{\mathbf{j}}}
\def\rvk{{\mathbf{k}}}
\def\rvl{{\mathbf{l}}}
\def\rvm{{\mathbf{m}}}
\def\rvn{{\mathbf{n}}}
\def\rvo{{\mathbf{o}}}
\def\rvp{{\mathbf{p}}}
\def\rvq{{\mathbf{q}}}
\def\rvr{{\mathbf{r}}}
\def\rvs{{\mathbf{s}}}
\def\rvt{{\mathbf{t}}}
\def\rvu{{\mathbf{u}}}
\def\rvv{{\mathbf{v}}}
\def\rvw{{\mathbf{w}}}
\def\rvx{{\mathbf{x}}}
\def\rvy{{\mathbf{y}}}
\def\rvz{{\mathbf{z}}}

% Elements of random vectors
\def\erva{{\textnormal{a}}}
\def\ervb{{\textnormal{b}}}
\def\ervc{{\textnormal{c}}}
\def\ervd{{\textnormal{d}}}
\def\erve{{\textnormal{e}}}
\def\ervf{{\textnormal{f}}}
\def\ervg{{\textnormal{g}}}
\def\ervh{{\textnormal{h}}}
\def\ervi{{\textnormal{i}}}
\def\ervj{{\textnormal{j}}}
\def\ervk{{\textnormal{k}}}
\def\ervl{{\textnormal{l}}}
\def\ervm{{\textnormal{m}}}
\def\ervn{{\textnormal{n}}}
\def\ervo{{\textnormal{o}}}
\def\ervp{{\textnormal{p}}}
\def\ervq{{\textnormal{q}}}
\def\ervr{{\textnormal{r}}}
\def\ervs{{\textnormal{s}}}
\def\ervt{{\textnormal{t}}}
\def\ervu{{\textnormal{u}}}
\def\ervv{{\textnormal{v}}}
\def\ervw{{\textnormal{w}}}
\def\ervx{{\textnormal{x}}}
\def\ervy{{\textnormal{y}}}
\def\ervz{{\textnormal{z}}}

% Random matrices
\def\rmA{{\mathbf{A}}}
\def\rmB{{\mathbf{B}}}
\def\rmC{{\mathbf{C}}}
\def\rmD{{\mathbf{D}}}
\def\rmE{{\mathbf{E}}}
\def\rmF{{\mathbf{F}}}
\def\rmG{{\mathbf{G}}}
\def\rmH{{\mathbf{H}}}
\def\rmI{{\mathbf{I}}}
\def\rmJ{{\mathbf{J}}}
\def\rmK{{\mathbf{K}}}
\def\rmL{{\mathbf{L}}}
\def\rmM{{\mathbf{M}}}
\def\rmN{{\mathbf{N}}}
\def\rmO{{\mathbf{O}}}
\def\rmP{{\mathbf{P}}}
\def\rmQ{{\mathbf{Q}}}
\def\rmR{{\mathbf{R}}}
\def\rmS{{\mathbf{S}}}
\def\rmT{{\mathbf{T}}}
\def\rmU{{\mathbf{U}}}
\def\rmV{{\mathbf{V}}}
\def\rmW{{\mathbf{W}}}
\def\rmX{{\mathbf{X}}}
\def\rmY{{\mathbf{Y}}}
\def\rmZ{{\mathbf{Z}}}

% Elements of random matrices
\def\ermA{{\textnormal{A}}}
\def\ermB{{\textnormal{B}}}
\def\ermC{{\textnormal{C}}}
\def\ermD{{\textnormal{D}}}
\def\ermE{{\textnormal{E}}}
\def\ermF{{\textnormal{F}}}
\def\ermG{{\textnormal{G}}}
\def\ermH{{\textnormal{H}}}
\def\ermI{{\textnormal{I}}}
\def\ermJ{{\textnormal{J}}}
\def\ermK{{\textnormal{K}}}
\def\ermL{{\textnormal{L}}}
\def\ermM{{\textnormal{M}}}
\def\ermN{{\textnormal{N}}}
\def\ermO{{\textnormal{O}}}
\def\ermP{{\textnormal{P}}}
\def\ermQ{{\textnormal{Q}}}
\def\ermR{{\textnormal{R}}}
\def\ermS{{\textnormal{S}}}
\def\ermT{{\textnormal{T}}}
\def\ermU{{\textnormal{U}}}
\def\ermV{{\textnormal{V}}}
\def\ermW{{\textnormal{W}}}
\def\ermX{{\textnormal{X}}}
\def\ermY{{\textnormal{Y}}}
\def\ermZ{{\textnormal{Z}}}

% Vectors
\def\vzero{{\bm{0}}}
\def\vone{{\bm{1}}}
\def\vmu{{\bm{\mu}}}
\def\vtheta{{\bm{\theta}}}
\def\vphi{{\bm{\phi}}}
\def\va{{\bm{a}}}
\def\vb{{\bm{b}}}
\def\vc{{\bm{c}}}
\def\vd{{\bm{d}}}
\def\ve{{\bm{e}}}
\def\vf{{\bm{f}}}
\def\vg{{\bm{g}}}
\def\vh{{\bm{h}}}
\def\vi{{\bm{i}}}
\def\vj{{\bm{j}}}
\def\vk{{\bm{k}}}
\def\vl{{\bm{l}}}
\def\vm{{\bm{m}}}
\def\vn{{\bm{n}}}
\def\vo{{\bm{o}}}
\def\vp{{\bm{p}}}
\def\vq{{\bm{q}}}
\def\vr{{\bm{r}}}
\def\vs{{\bm{s}}}
\def\vt{{\bm{t}}}
\def\vu{{\bm{u}}}
\def\vv{{\bm{v}}}
\def\vw{{\bm{w}}}
\def\vx{{\bm{x}}}
\def\vy{{\bm{y}}}
\def\vz{{\bm{z}}}

% Elements of vectors
\def\evalpha{{\alpha}}
\def\evbeta{{\beta}}
\def\evepsilon{{\epsilon}}
\def\evlambda{{\lambda}}
\def\evomega{{\omega}}
\def\evmu{{\mu}}
\def\evpsi{{\psi}}
\def\evsigma{{\sigma}}
\def\evtheta{{\theta}}
\def\eva{{a}}
\def\evb{{b}}
\def\evc{{c}}
\def\evd{{d}}
\def\eve{{e}}
\def\evf{{f}}
\def\evg{{g}}
\def\evh{{h}}
\def\evi{{i}}
\def\evj{{j}}
\def\evk{{k}}
\def\evl{{l}}
\def\evm{{m}}
\def\evn{{n}}
\def\evo{{o}}
\def\evp{{p}}
\def\evq{{q}}
\def\evr{{r}}
\def\evs{{s}}
\def\evt{{t}}
\def\evu{{u}}
\def\evv{{v}}
\def\evw{{w}}
\def\evx{{x}}
\def\evy{{y}}
\def\evz{{z}}

% Matrix
\def\mA{{\bm{A}}}
\def\mB{{\bm{B}}}
\def\mC{{\bm{C}}}
\def\mD{{\bm{D}}}
\def\mE{{\bm{E}}}
\def\mF{{\bm{F}}}
\def\mG{{\bm{G}}}
\def\mH{{\bm{H}}}
\def\mI{{\bm{I}}}
\def\mJ{{\bm{J}}}
\def\mK{{\bm{K}}}
\def\mL{{\bm{L}}}
\def\mM{{\bm{M}}}
\def\mN{{\bm{N}}}
\def\mO{{\bm{O}}}
\def\mP{{\bm{P}}}
\def\mQ{{\bm{Q}}}
\def\mR{{\bm{R}}}
\def\mS{{\bm{S}}}
\def\mT{{\bm{T}}}
\def\mU{{\bm{U}}}
\def\mV{{\bm{V}}}
\def\mW{{\bm{W}}}
\def\mX{{\bm{X}}}
\def\mY{{\bm{Y}}}
\def\mZ{{\bm{Z}}}
\def\mBeta{{\bm{\beta}}}
\def\mPhi{{\bm{\Phi}}}
\def\mLambda{{\bm{\Lambda}}}
\def\mSigma{{\bm{\Sigma}}}

% Tensor
\DeclareMathAlphabet{\mathsfit}{\encodingdefault}{\sfdefault}{m}{sl}
\SetMathAlphabet{\mathsfit}{bold}{\encodingdefault}{\sfdefault}{bx}{n}
\newcommand{\tens}[1]{\bm{\mathsfit{#1}}}
\def\tA{{\tens{A}}}
\def\tB{{\tens{B}}}
\def\tC{{\tens{C}}}
\def\tD{{\tens{D}}}
\def\tE{{\tens{E}}}
\def\tF{{\tens{F}}}
\def\tG{{\tens{G}}}
\def\tH{{\tens{H}}}
\def\tI{{\tens{I}}}
\def\tJ{{\tens{J}}}
\def\tK{{\tens{K}}}
\def\tL{{\tens{L}}}
\def\tM{{\tens{M}}}
\def\tN{{\tens{N}}}
\def\tO{{\tens{O}}}
\def\tP{{\tens{P}}}
\def\tQ{{\tens{Q}}}
\def\tR{{\tens{R}}}
\def\tS{{\tens{S}}}
\def\tT{{\tens{T}}}
\def\tU{{\tens{U}}}
\def\tV{{\tens{V}}}
\def\tW{{\tens{W}}}
\def\tX{{\tens{X}}}
\def\tY{{\tens{Y}}}
\def\tZ{{\tens{Z}}}


% Graph
\def\gA{{\mathcal{A}}}
\def\gB{{\mathcal{B}}}
\def\gC{{\mathcal{C}}}
\def\gD{{\mathcal{D}}}
\def\gE{{\mathcal{E}}}
\def\gF{{\mathcal{F}}}
\def\gG{{\mathcal{G}}}
\def\gH{{\mathcal{H}}}
\def\gI{{\mathcal{I}}}
\def\gJ{{\mathcal{J}}}
\def\gK{{\mathcal{K}}}
\def\gL{{\mathcal{L}}}
\def\gM{{\mathcal{M}}}
\def\gN{{\mathcal{N}}}
\def\gO{{\mathcal{O}}}
\def\gP{{\mathcal{P}}}
\def\gQ{{\mathcal{Q}}}
\def\gR{{\mathcal{R}}}
\def\gS{{\mathcal{S}}}
\def\gT{{\mathcal{T}}}
\def\gU{{\mathcal{U}}}
\def\gV{{\mathcal{V}}}
\def\gW{{\mathcal{W}}}
\def\gX{{\mathcal{X}}}
\def\gY{{\mathcal{Y}}}
\def\gZ{{\mathcal{Z}}}

% Sets
\def\sA{{\mathbb{A}}}
\def\sB{{\mathbb{B}}}
\def\sC{{\mathbb{C}}}
\def\sD{{\mathbb{D}}}
% Don't use a set called E, because this would be the same as our symbol
% for expectation.
\def\sF{{\mathbb{F}}}
\def\sG{{\mathbb{G}}}
\def\sH{{\mathbb{H}}}
\def\sI{{\mathbb{I}}}
\def\sJ{{\mathbb{J}}}
\def\sK{{\mathbb{K}}}
\def\sL{{\mathbb{L}}}
\def\sM{{\mathbb{M}}}
\def\sN{{\mathbb{N}}}
\def\sO{{\mathbb{O}}}
\def\sP{{\mathbb{P}}}
\def\sQ{{\mathbb{Q}}}
\def\sR{{\mathbb{R}}}
\def\sS{{\mathbb{S}}}
\def\sT{{\mathbb{T}}}
\def\sU{{\mathbb{U}}}
\def\sV{{\mathbb{V}}}
\def\sW{{\mathbb{W}}}
\def\sX{{\mathbb{X}}}
\def\sY{{\mathbb{Y}}}
\def\sZ{{\mathbb{Z}}}

% Entries of a matrix
\def\emLambda{{\Lambda}}
\def\emA{{A}}
\def\emB{{B}}
\def\emC{{C}}
\def\emD{{D}}
\def\emE{{E}}
\def\emF{{F}}
\def\emG{{G}}
\def\emH{{H}}
\def\emI{{I}}
\def\emJ{{J}}
\def\emK{{K}}
\def\emL{{L}}
\def\emM{{M}}
\def\emN{{N}}
\def\emO{{O}}
\def\emP{{P}}
\def\emQ{{Q}}
\def\emR{{R}}
\def\emS{{S}}
\def\emT{{T}}
\def\emU{{U}}
\def\emV{{V}}
\def\emW{{W}}
\def\emX{{X}}
\def\emY{{Y}}
\def\emZ{{Z}}
\def\emSigma{{\Sigma}}

% entries of a tensor
% Same font as tensor, without \bm wrapper
\newcommand{\etens}[1]{\mathsfit{#1}}
\def\etLambda{{\etens{\Lambda}}}
\def\etA{{\etens{A}}}
\def\etB{{\etens{B}}}
\def\etC{{\etens{C}}}
\def\etD{{\etens{D}}}
\def\etE{{\etens{E}}}
\def\etF{{\etens{F}}}
\def\etG{{\etens{G}}}
\def\etH{{\etens{H}}}
\def\etI{{\etens{I}}}
\def\etJ{{\etens{J}}}
\def\etK{{\etens{K}}}
\def\etL{{\etens{L}}}
\def\etM{{\etens{M}}}
\def\etN{{\etens{N}}}
\def\etO{{\etens{O}}}
\def\etP{{\etens{P}}}
\def\etQ{{\etens{Q}}}
\def\etR{{\etens{R}}}
\def\etS{{\etens{S}}}
\def\etT{{\etens{T}}}
\def\etU{{\etens{U}}}
\def\etV{{\etens{V}}}
\def\etW{{\etens{W}}}
\def\etX{{\etens{X}}}
\def\etY{{\etens{Y}}}
\def\etZ{{\etens{Z}}}

% The true underlying data generating distribution
\newcommand{\pdata}{p_{\rm{data}}}
\newcommand{\ptarget}{p_{\rm{target}}}
\newcommand{\pprior}{p_{\rm{prior}}}
\newcommand{\pbase}{p_{\rm{base}}}
\newcommand{\pref}{p_{\rm{ref}}}

% The empirical distribution defined by the training set
\newcommand{\ptrain}{\hat{p}_{\rm{data}}}
\newcommand{\Ptrain}{\hat{P}_{\rm{data}}}
% The model distribution
\newcommand{\pmodel}{p_{\rm{model}}}
\newcommand{\Pmodel}{P_{\rm{model}}}
\newcommand{\ptildemodel}{\tilde{p}_{\rm{model}}}
% Stochastic autoencoder distributions
\newcommand{\pencode}{p_{\rm{encoder}}}
\newcommand{\pdecode}{p_{\rm{decoder}}}
\newcommand{\precons}{p_{\rm{reconstruct}}}

\newcommand{\laplace}{\mathrm{Laplace}} % Laplace distribution

\newcommand{\E}{\mathbb{E}}
\newcommand{\Ls}{\mathcal{L}}
\newcommand{\R}{\mathbb{R}}
\newcommand{\emp}{\tilde{p}}
\newcommand{\lr}{\alpha}
\newcommand{\reg}{\lambda}
\newcommand{\rect}{\mathrm{rectifier}}
\newcommand{\softmax}{\mathrm{softmax}}
\newcommand{\sigmoid}{\sigma}
\newcommand{\softplus}{\zeta}
\newcommand{\KL}{D_{\mathrm{KL}}}
\newcommand{\Var}{\mathrm{Var}}
\newcommand{\standarderror}{\mathrm{SE}}
\newcommand{\Cov}{\mathrm{Cov}}
% Wolfram Mathworld says $L^2$ is for function spaces and $\ell^2$ is for vectors
% But then they seem to use $L^2$ for vectors throughout the site, and so does
% wikipedia.
\newcommand{\normlzero}{L^0}
\newcommand{\normlone}{L^1}
\newcommand{\normltwo}{L^2}
\newcommand{\normlp}{L^p}
\newcommand{\normmax}{L^\infty}

\newcommand{\parents}{Pa} % See usage in notation.tex. Chosen to match Daphne's book.

\DeclareMathOperator*{\argmax}{arg\,max}
\DeclareMathOperator*{\argmin}{arg\,min}

\DeclareMathOperator{\sign}{sign}
\DeclareMathOperator{\Tr}{Tr}
\let\ab\allowbreak


\usepackage{hyperref}
\usepackage{url}
\usepackage{soul}
\usepackage{algorithm}
\usepackage{algorithmic}
\usepackage{graphicx}
\usepackage{wrapfig}
\usepackage{multirow}
\usepackage{booktabs}
\usepackage{amsthm}
\usepackage{mathrsfs}
\usepackage{authblk}

\usepackage[font=small]{caption}


\title{NeurFlow: Interpreting Neural Networks through Neuron Groups and Functional Interactions}

% Authors must not appear in the submitted version. They should be hidden
% as long as the \iclrfinalcopy macro remains commented out below.
% Non-anonymous submissions will be rejected without review.
\iclrfinalcopy

% \author{Tue M. Cao \\
% Hanoi University of Science and Technology\\
% Hanoi, Vietnam \\
% \texttt{tue.cm210908@sis.hust.edu.vn} \\
% \And
% \textbf{Nhat X. Hoang} \\
% University of Florida \\
% Gainesville, Florida, USA \\
% \texttt{hoangx@ufl.edu} \\
% \And
% \textbf{Hieu H. Pham} \\
% VinUni-Illinois Smart Health Center, VinUniversity\\
% Hanoi, Vietnam \\
% \texttt{hieu.ph@vinuni.edu.vn} \\
% \And
% \textbf{Phi Le Nguyen} \\
% Hanoi University of Science and Technology \\
% Hanoi, Vietnam \\
% \texttt{lenp@soict.hust.edu.vn}
% \And
% \textbf{My T. Thai} \\
% University of Florida \\
% Gainesville, Florida, USA \\
% \texttt{mythai@cise.ufl.edu} \\
% }

\author{
    \textbf{Tue M. Cao}\textsuperscript{1} \quad
    \textbf{Nhat X. Hoang}\textsuperscript{2} \quad
    \textbf{Hieu H. Pham}\textsuperscript{3} \quad
    \textbf{Phi Le Nguyen}\textsuperscript{1*} \quad
    \textbf{My T. Thai}\textsuperscript{2}\thanks{Corresponding Authors}
}


\newcommand{\fix}{\marginpar{FIX}}
\newcommand{\new}{\marginpar{NEW}}

\theoremstyle{definition}
\newtheorem{mydef}{Definition}
\newcommand{\lenp}[1]{\textcolor{blue}{#1}}
\newcommand{\mt}[1]{\textcolor{red}{#1}}
\newcommand{\tue}[1]{\textcolor{purple}{#1}}
\newcommand{\revise}[1]{\textcolor{red}{\textbf{#1}}}

\begin{document}

\maketitle

% \renewcommand{\thefootnote}{\fnsymbol{\arabic{footnote}}}
\renewcommand{\thefootnote}{\arabic{footnote}}
\footnotetext[1]{Institute for AI Innovation and Societal Impact (AI4LIFE), Hanoi University of Science and Technology, Hanoi, Vietnam \texttt{(tue.cm210908@sis.hust.edu.vn, lenp@soict.hust.edu.vn)}.}
\footnotetext[2]{University of Florida, Gainesville, Florida, USA \texttt{\{hoangx, mythai\}@ufl.edu}.}
\footnotetext[3]{VinUni-Illinois Smart Health Center, VinUniversity, Hanoi, Vietnam (\texttt{hieu.ph@vinuni.edu.vn}).}

% \vspace{-5mm}
\begin{abstract}
\label{abstract}

Understanding the inner workings of neural networks is essential for enhancing model performance and interpretability. Current research predominantly focuses on examining the connection between individual neurons and the model's final predictions, which suffers from challenges in interpreting the internal workings of the model, particularly when neurons encode multiple unrelated features. In this paper, we propose a novel framework that transitions the focus from analyzing individual neurons to investigating groups of neurons, shifting the emphasis from neuron-output relationships to the functional interactions between neurons. Our automated framework, NeurFlow, first identifies core neurons and clusters them into groups based on shared functional relationships, enabling a more coherent and interpretable view of the network’s internal processes. This approach facilitates the construction of a hierarchical circuit representing neuron interactions across layers, thus improving interpretability while reducing computational costs. Our extensive empirical studies validate the fidelity of our proposed NeurFlow. Additionally, we showcase its utility in practical applications such as image debugging and automatic concept labeling, thereby highlighting its potential to advance the field of neural network explainability. \footnote[4]{Source code: \href{https://github.com/tue147/neurflow}{https://github.com/tue147/neurflow}}
\end{abstract}

\section{Introduction}
\label{sec:intro}
%!TEX root = gcn.tex
\section{Introduction}
Graphs, representing structural data and topology, are widely used across various domains, such as social networks and merchandising transactions.
Graph convolutional networks (GCN)~\cite{iclr/KipfW17} have significantly enhanced model training on these interconnected nodes.
However, these graphs often contain sensitive information that should not be leaked to untrusted parties.
For example, companies may analyze sensitive demographic and behavioral data about users for applications ranging from targeted advertising to personalized medicine.
Given the data-centric nature and analytical power of GCN training, addressing these privacy concerns is imperative.

Secure multi-party computation (MPC)~\cite{crypto/ChaumDG87,crypto/ChenC06,eurocrypt/CiampiRSW22} is a critical tool for privacy-preserving machine learning, enabling mutually distrustful parties to collaboratively train models with privacy protection over inputs and (intermediate) computations.
While research advances (\eg,~\cite{ccs/RatheeRKCGRS20,uss/NgC21,sp21/TanKTW,uss/WatsonWP22,icml/Keller022,ccs/ABY318,folkerts2023redsec}) support secure training on convolutional neural networks (CNNs) efficiently, private GCN training with MPC over graphs remains challenging.

Graph convolutional layers in GCNs involve multiplications with a (normalized) adjacency matrix containing $\numedge$ non-zero values in a $\numnode \times \numnode$ matrix for a graph with $\numnode$ nodes and $\numedge$ edges.
The graphs are typically sparse but large.
One could use the standard Beaver-triple-based protocol to securely perform these sparse matrix multiplications by treating graph convolution as ordinary dense matrix multiplication.
However, this approach incurs $O(\numnode^2)$ communication and memory costs due to computations on irrelevant nodes.
%
Integrating existing cryptographic advances, the initial effort of SecGNN~\cite{tsc/WangZJ23,nips/RanXLWQW23} requires heavy communication or computational overhead.
Recently, CoGNN~\cite{ccs/ZouLSLXX24} optimizes the overhead in terms of  horizontal data partitioning, proposing a semi-honest secure framework.
Research for secure GCN over vertical data  remains nascent.

Current MPC studies, for GCN or not, have primarily targeted settings where participants own different data samples, \ie, horizontally partitioned data~\cite{ccs/ZouLSLXX24}.
MPC specialized for scenarios where parties hold different types of features~\cite{tkde/LiuKZPHYOZY24,icml/CastigliaZ0KBP23,nips/Wang0ZLWL23} is rare.
This paper studies $2$-party secure GCN training for these vertical partition cases, where one party holds private graph topology (\eg, edges) while the other owns private node features.
For instance, LinkedIn holds private social relationships between users, while banks own users' private bank statements.
Such real-world graph structures underpin the relevance of our focus.
To our knowledge, no prior work tackles secure GCN training in this context, which is crucial for cross-silo collaboration.


To realize secure GCN over vertically split data, we tailor MPC protocols for sparse graph convolution, which fundamentally involves sparse (adjacency) matrix multiplication.
Recent studies have begun exploring MPC protocols for sparse matrix multiplication (SMM).
ROOM~\cite{ccs/SchoppmannG0P19}, a seminal work on SMM, requires foreknowledge of sparsity types: whether the input matrices are row-sparse or column-sparse.
Unfortunately, GCN typically trains on graphs with arbitrary sparsity, where nodes have varying degrees and no specific sparsity constraints.
Moreover, the adjacency matrix in GCN often contains a self-loop operation represented by adding the identity matrix, which is neither row- nor column-sparse.
Araki~\etal~\cite{ccs/Araki0OPRT21} avoid this limitation in their scalable, secure graph analysis work, yet it does not cover vertical partition.

% and related primitives
To bridge this gap, we propose a secure sparse matrix multiplication protocol, \osmm, achieving \emph{accurate, efficient, and secure GCN training over vertical data} for the first time.

\subsection{New Techniques for Sparse Matrices}
The cost of evaluating a GCN layer is dominated by SMM in the form of $\adjmat\feamat$, where $\adjmat$ is a sparse adjacency matrix of a (directed) graph $\graph$ and $\feamat$ is a dense matrix of node features.
For unrelated nodes, which often constitute a substantial portion, the element-wise products $0\cdot x$ are always zero.
Our efficient MPC design 
avoids unnecessary secure computation over unrelated nodes by focusing on computing non-zero results while concealing the sparse topology.
We achieve this~by:
1) decomposing the sparse matrix $\adjmat$ into a product of matrices (\S\ref{sec::sgc}), including permutation and binary diagonal matrices, that can \emph{faithfully} represent the original graph topology;
2) devising specialized protocols (\S\ref{sec::smm_protocol}) for efficiently multiplying the structured matrices while hiding sparsity topology.


 
\subsubsection{Sparse Matrix Decomposition}
We decompose adjacency matrix $\adjmat$ of $\graph$ into two bipartite graphs: one represented by sparse matrix $\adjout$, linking the out-degree nodes to edges, the other 
by sparse matrix $\adjin$,
linking edges to in-degree nodes.

%\ie, we decompose $\adjmat$ into $\adjout \adjin$, where $\adjout$ and $\adjin$ are sparse matrices representing these connections.
%linking out-degree nodes to edges and edges to in-degree nodes of $\graph$, respectively.

We then permute the columns of $\adjout$ and the rows of $\adjin$ so that the permuted matrices $\adjout'$ and $\adjin'$ have non-zero positions with \emph{monotonically non-decreasing} row and column indices.
A permutation $\sigma$ is used to preserve the edge topology, leading to an initial decomposition of $\adjmat = \adjout'\sigma \adjin'$.
This is further refined into a sequence of \emph{linear transformations}, 
which can be efficiently computed by our MPC protocols for 
\emph{oblivious permutation}
%($\Pi_{\ssp}$) 
and \emph{oblivious selection-multiplication}.
% ($\Pi_\SM$)
\iffalse
Our approach leverages bipartite graph representation and the monotonicity of non-zero positions to decompose a general sparse matrix into linear transformations, enhancing the efficiency of our MPC protocols.
\fi
Our decomposition approach is not limited to GCNs but also general~SMM 
by 
%simply 
treating them 
as adjacency matrices.
%of a graph.
%Since any sparse matrix can be viewed 

%allowing the same technique to be applied.

 
\subsubsection{New Protocols for Linear Transformations}
\emph{Oblivious permutation} (OP) is a two-party protocol taking a private permutation $\sigma$ and a private vector $\xvec$ from the two parties, respectively, and generating a secret share $\l\sigma \xvec\r$ between them.
Our OP protocol employs correlated randomnesses generated in an input-independent offline phase to mask $\sigma$ and $\xvec$ for secure computations on intermediate results, requiring only $1$ round in the online phase (\cf, $\ge 2$ in previous works~\cite{ccs/AsharovHIKNPTT22, ccs/Araki0OPRT21}).

Another crucial two-party protocol in our work is \emph{oblivious selection-multiplication} (OSM).
It takes a private bit~$s$ from a party and secret share $\l x\r$ of an arithmetic number~$x$ owned by the two parties as input and generates secret share $\l sx\r$.
%between them.
%Like our OP protocol, o
Our $1$-round OSM protocol also uses pre-computed randomnesses to mask $s$ and $x$.
%for secure computations.
Compared to the Beaver-triple-based~\cite{crypto/Beaver91a} and oblivious-transfer (OT)-based approaches~\cite{pkc/Tzeng02}, our protocol saves ${\sim}50\%$ of online communication while having the same offline communication and round complexities.

By decomposing the sparse matrix into linear transformations and applying our specialized protocols, our \osmm protocol
%($\prosmm$) 
reduces the complexity of evaluating $\numnode \times \numnode$ sparse matrices with $\numedge$ non-zero values from $O(\numnode^2)$ to $O(\numedge)$.

%(\S\ref{sec::secgcn})
\subsection{\cgnn: Secure GCN made Efficient}
Supported by our new sparsity techniques, we build \cgnn, 
a two-party computation (2PC) framework for GCN inference and training over vertical
%ly split
data.
Our contributions include:

1) We are the first to explore sparsity over vertically split, secret-shared data in MPC, enabling decompositions of sparse matrices with arbitrary sparsity and isolating computations that can be performed in plaintext without sacrificing privacy.

2) We propose two efficient $2$PC primitives for OP and OSM, both optimally single-round.
Combined with our sparse matrix decomposition approach, our \osmm protocol ($\prosmm$) achieves constant-round communication costs of $O(\numedge)$, reducing memory requirements and avoiding out-of-memory errors for large matrices.
In practice, it saves $99\%+$ communication
%(Table~\ref{table:comm_smm}) 
and reduces ${\sim}72\%$ memory usage over large $(5000\times5000)$ matrices compared with using Beaver triples.
%(Table~\ref{table:mem_smm_sparse}) ${\sim}16\%$-

3) We build an end-to-end secure GCN framework for inference and training over vertically split data, maintaining accuracy on par with plaintext computations.
We will open-source our evaluation code for research and deployment.

To evaluate the performance of $\cgnn$, we conducted extensive experiments over three standard graph datasets (Cora~\cite{aim/SenNBGGE08}, Citeseer~\cite{dl/GilesBL98}, and Pubmed~\cite{ijcnlp/DernoncourtL17}),
reporting communication, memory usage, accuracy, and running time under varying network conditions, along with an ablation study with or without \osmm.
Below, we highlight our key achievements.

\textit{Communication (\S\ref{sec::comm_compare_gcn}).}
$\cgnn$ saves communication by $50$-$80\%$.
(\cf,~CoGNN~\cite{ccs/KotiKPG24}, OblivGNN~\cite{uss/XuL0AYY24}).

\textit{Memory usage (\S\ref{sec::smmmemory}).}
\cgnn alleviates out-of-memory problems of using %the standard 
Beaver-triples~\cite{crypto/Beaver91a} for large datasets.

\textit{Accuracy (\S\ref{sec::acc_compare_gcn}).}
$\cgnn$ achieves inference and training accuracy comparable to plaintext counterparts.
%training accuracy $\{76\%$, $65.1\%$, $75.2\%\}$ comparable to $\{75.7\%$, $65.4\%$, $74.5\%\}$ in plaintext.

{\textit{Computational efficiency (\S\ref{sec::time_net}).}} 
%If the network is worse in bandwidth and better in latency, $\cgnn$ shows more benefits.
$\cgnn$ is faster by $6$-$45\%$ in inference and $28$-$95\%$ in training across various networks and excels in narrow-bandwidth and low-latency~ones.

{\textit{Impact of \osmm (\S\ref{sec:ablation}).}}
Our \osmm protocol shows a $10$-$42\times$ speed-up for $5000\times 5000$ matrices and saves $10$-2$1\%$ memory for ``small'' datasets and up to $90\%$+ for larger ones.

\vspace{-8pt}
\section{Related work}
\vspace{-8pt}
\label{sec:related_work}
\section{Related Work}

\subsection{Personalization and Role-Playing}
Recent works have introduced benchmark datasets for personalizing LLM outputs in tasks like email, abstract, and news writing, focusing on shorter outputs (e.g., 300 tokens for product reviews \citep{kumar2024longlamp} and 850 for news writing \citep{shashidhar-etal-2024-unsupervised}). These methods infer user traits from history for task-specific personalization \citep{sun-etal-2024-revealing, sun-etal-2025-persona, pal2024beyond, li2023teach, salemi2025reasoning}. In contrast, we tackle the more subjective problem of long-form story writing, with author stories averaging 1500 tokens. Unlike prior role-playing approaches that use predefined personas (e.g., Tony Stark, Confucius) \citep{wang-etal-2024-rolellm, sadeq-etal-2024-mitigating, tu2023characterchat, xu2023expertprompting}, we propose a novel method to infer story-writing personas from an author’s history to guide role-playing.


\subsection{Story Understanding and Generation}  
Prior work on persona-aware story generation \citep{yunusov-etal-2024-mirrorstories, bae-kim-2024-collective, zhang-etal-2022-persona, chandu-etal-2019-way} defines personas using discrete attributes like personality traits, demographics, or hobbies. Similarly, \citep{zhu-etal-2023-storytrans} explore story style transfer across pre-defined domains (e.g., fairy tales, martial arts, Shakespearean plays). In contrast, we mimic an individual author's writing style based on their history. Our approach differs by (1) inferring long-form author personas—descriptions of an author’s style from their past works, rather than relying on demographics, and (2) handling long-form story generation, averaging 1500 tokens per output, exceeding typical story lengths in prior research.
\vspace{-2pt}
\section{NeurFlow Framework}
\vspace{-5pt}
%\section{NeurFlow - Neuron Group Interaction and Analysis Framework}
\label{sec:proposal}
\subsection{Problem Formulation}
Our goal is to explain the internal mechanisms of deep neural networks (DNNs) by investigating how groups of neurons function and interact to encapsulate concepts, thereby performing a specific task. In particular, we focus on the classification problem, exploring how groups of neurons process visual features to identify a class. Given the exponential number of possible neuron groups, we focus only on core concept neurons. 
In addition, to facilitate human interpretation, we group these neurons through the common visual features they encode.
In essence, we propose a comprehensive framework to address the following questions: 
\textit{(i) Which neurons play a crucial role in each layer? (ii) How can these neurons be clustered, and what visual features does each neuron group encapsulate? (iii) How do groups of neurons in adjacent layers interact?}


Our problem can be formulated as follows:
Given a pretrained classification network \textit{F} and a dataset $\mathcal{D}_c$ composed of exemplars from a specific class $c$, the goal is to construct a hierarchical tree whose vertices represent groups of core concept neurons in each network layer, and the edges capture the relationships between these groups.
Figure \ref{fig:flow} illutrates the workflow of our framework which comprises the following key components: (1) identifying core concept neurons (Section \ref{subsec:identify_node}), (2) determining inter-layer relationships among neurons (Section \ref{sec:indentify_circuit}), (3) clustering the core concept neurons into groups, and analyzing the interactions between these neuron groups (Section \ref{concept_circuit}). 

\begin{figure}[tb] 
\begin{center}
\vspace{-11mm}
\includegraphics[width=0.8\textwidth]{figures/flow3.png} 
\end{center}
\vspace{-15pt} 
\caption{\textbf{Workflow of NeurFlow}, consisting of three main components: identifying core concept neurons in each layer, building the neuron circuit, and constructing the circuit of neuron groups.}
\label{fig:flow}
\vspace{-15pt}
\end{figure}

\subsection{Definitions and Notations}
\label{def_not}
In this paper, the term \emph{neuron} refers to either a unit in a linear layer or a feature map in a convolutional layer. As suggested by \citet{Olah, Invert, Polysemantic}, each neuron is selectively activated by a distinct set of visual features, and by interpreting the neuron as a representation of these features, we can gain insights into the internal representations of a DNN. We refer to these visual features as the concept of the neuron. In the following definitions, let $a$ represent an arbitrary neuron located in layer $l$ of the pretrained network $F$. 
In this study, we do not rely on any predefined concepts. Instead, we enhance the original dataset $\mathcal{D}_c$ by cutting it into smaller patches with varying sizes. These patches serve as visual features for probing the model. We refer to this augmented dataset as $\mathcal{D}$, and denote $v$ as an arbitrary element of $\mathcal{D}$.

\begin{mydef}[Neuron Concept]
\label{def:neuron concept}
    The neuron concept $\mathcal{V}_{a}$ of neuron $a$ is defined as the set of the top-$k$ image patches\footnote{each image patch is a piece cropped from image set.} that most strongly activate neuron $a$. Formally, the neuron concept of $a$ is expressed as $\mathcal{V}_{a} :=\underset{\mathcal{V} \subset \mathcal{D}; |\mathcal{V}| = k}{\arg \max} \sum_{v \in \mathcal{V}} \phi_{a}(v)$, where $\phi_{a}(v): \mathbb{D} \xrightarrow{} \mathbb{R}$ represents the activation of neuron $a$ for a given input $v \in \mathcal{D}$, and $k$ is a hyperparameter.
\end{mydef}
An empirical analysis of the impact of $k$ (Appendix \ref{sec:choice_of_k}) reveals that NeurFlow's performance is relatively insensitive to the selection of $k$.

\begin{mydef}[Neuron Concept with Knockout]
    Let $M$ be the computational graph of the network $F$, $S$ be an arbitrary subset neurons of $M$, and $M \setminus S$ be the sub-graph of $M$ after removing $S$; let $\phi^{\overline{S}}_{a}$ be the activation of neuron $a$ computed from $M \setminus S$. 
    The neuron concept of $a$ when knocking-out $S$ (denoted as $\mathcal{V}^{\overline{S}}_{a}$) is defined as $\mathcal{V}^{\overline{S}}_{a} := \underset{\mathcal{V} \subset \mathcal{D}; |\mathcal{V}| = k}{\arg \max} \sum_{v \in \mathcal{V}} \phi^{\overline{S}}_{a}(v)$. 
\end{mydef}

We hypothesize that for each neuron $a$, only a small subset of neurons from the preceding layer exert the most significant influence on $a$. In particular, knocking out these neurons would lead to a substantial change in the concept associated with $a$. We refer to these neurons as \emph{core concept neurons} and provide a formal definition in the following.

% \mt{How about change ``critical neurons'' to: Core Concept Neurons? Earlier, we have neuron concepts, neuron concepts with knockoff. so core concept neurons is naturally coming...}

\begin{mydef}[Core Concept Neurons]
\label{def:critical neuron} Given a neuron $a$ at layer $l$, core concept neurons of $a$ (denoted as $\mathbb{S}_a$) is the sub-set of neurons at the previous layer $l - 1$ satisfying the following conditions:
    \begin{equation}
    \label{eq:critical neurons}
        \mathbb{S}_a := \underset{S \subseteq \mathbb{S};\left|{S}\right| \leq \tau}{\arg \min} \left| \mathcal{V}^{\overline{{S}}}_a \cap \mathcal{V}_a \right|,
    \end{equation}
    where $\mathbb{S}$ is set of all neurons at layer $l-1$ and $\tau$ is a predefined threshold. 
    Intuitively, the core concept neurons for a target neuron $a$ are those that play an important role in defining the concepts represented by $a$.
    In practice, the value of $\tau$ may vary across the network layers, its impact will be elaborated upon in Sections \ref{sec:analysis}.
\end{mydef}    
% In practice, the value of $\tau$ may vary across the network layers, its impact will be elaborated upon in Sections \ref{sec:analysis}.
In the following, we denote by  $\phi^{1, l-1}(v): \sD \xrightarrow{} \sR^{m\times w \times h}$ the function that maps the input $v$ to the feature maps at the $(l-1)$-th layer of the model, where $m$ represents the number of channels, and $w \times h$ indicates the dimensions of each feature map. Furthermore, we adopt the notation $|.|$ to indicate the cardinality of a set, while $\|.\|$ is employed to represent the absolute value.
We summarize all the notations in Table \ref{fig:notation} (Appendix \ref{sec:appendix_notation}).

\subsection{Identifying core concept Neurons}
\label{subsec:identify_node}
Given a neuron $a$, we describe our algorithm for identifying its core concept neuron set $\mathbb{S}_a$. This process consists of two main steps: determining $a$'s concept $\mathcal{V}_a$ according to Definition \ref{def:neuron concept}, and identifying core concept neurons following Definition \ref{def:critical neuron}. 

Firstly, we generate a set of image patches $\mathcal{D}$ by augmenting the original dataset $\mathcal{D}_c$, which consists of images that the model classifies as class $c$.
%To determine $\mathcal{V}_a$, we create a set of visual features $\mathcal{D}$ by augmenting the original dataset $\mathcal{D}_c$ (images that the model predicts as class $c$). 
Since neurons can detect visual features at different levels of granularity, we divide each image in $\mathcal{D}_c$ into smaller patches using various crop sizes, where smaller patches capture simpler visual features and larger patches represent more complex ones. 
We subsequently evaluate all items in $\mathcal{D}$ to identify the top-$k$ image patches that induce the highest activation in neuron $a$, thereby constructing $\mathcal{V}_a$.

With $\mathcal{V}_a$ identified, one could determine the core concept neurons through a brute-force search over all possible candidates. However, this naive approach is computationally infeasible. 
To this end, we define a metric named \emph{importance score} that quantifies the attribution of a neuron $s_i$ to $a$.
The importance score can be intuitively seen as integrated gradients (\cite{IG}) of $a$ to $s_i$ calculated across all elements of $\mathcal{V}_a$, calculated as follows:
\begin{equation}
     T(a, s_i, \mathcal{V}_a) = \sum_{v\in \mathcal{V}_a} \sum_{ \substack{x \in \phi^{1, l-1}_{s_i}(v); \\ y \in \phi^{l-1,l}_a(\phi^{1,l-1}(v)) }} x \times \frac{1}{N} \left ( \sum_{n=1}^{N} \frac{\partial y(\frac{n}{N}x)}{\partial x} \right ),
\end{equation}
where $\phi^{1,l-1}_{s_i}$ is the element of $\phi^{1,l-1}$ corresponding to neuron $s_i$, $\phi^{l-1,l}_a$ depicts the function mapping from the activation vector of layer \( l-1 \) to the activation of neuron \( a \), and $N$ is the step size. 
Utilizing the \emph{importance scores} of all neurons in the preceding layer, the set of core concept neurons is identified by selecting the top $\tau$ neurons that exhibit the highest absolute scores.
To justify the use of integrated gradients, we empirically show a strong correlation between the absolute values of \( T(a, s_i, \mathcal{V}_a) \) and the change in \( a \)'s concept after knocking out \( s_i \), as demonstrated in Section \ref{sec:analysis}.
Additionally, we compare our method with other attribution techniques in Appendix \ref{sec:compare_scoring}.

\subsection{Constructing core concept Neuron Circuit}
\vspace{-5pt}
\label{sec:indentify_circuit}
For each class of interest \(c\), the neuron circuit $\mathcal{H}_c$ is represented as a \emph{hierarchical hypertree}\footnote{A hypertree is a tree in which each child-parent pair may be connected by multiple edges.}, with the root $a_c$ being the neuron in the logit layer (ouput) associated with class $c$. The nodes in each layer of the tree $\mathcal{H}_c$ are the core concept neurons of those in the layer above, and branches connecting a parent node $a$ and its child $s_i \in \sS_a$ represents the contributions of $s_i$ to $a$'s concept. 

As mentioned in \citep{Olah, Polysemantic}, neurons often exhibit polysemantic behavior, meaning that a single neuron may encode multiple distinct visual features. In other words, the visual features within a concept $\mathcal{V}_a$ of neuron $a$ may not share the same meaning and can be categorized into distinct groups, which we term \emph{semantic groups}.
We hypothesize that each core concept neuron $s_i$ makes a distinct contribution to each semantic group of neuron $a$. To model this relationship, we represent the interaction between $s_i$ and $a$ through multiple connections, where the $j$-th connection reflects $s_i$'s influence on $\mathcal{V}_{a,j}$, the $j$-th semantic group of $a$. 

At a conceptual level, the algorithm for constructing the hypertree $\mathcal{H}_c$ proceeds through the following steps: (1) employing our core concept neuron identification algorithm to determine the children of each node in the tree (Section \ref{subsec:identify_node}), (2) clustering the neuron concept of each parent node into semantic groups, and (3) assigning weights to each branch connecting a child node to the semantic groups of its parent. 
Figure \ref{fig:critical neuron} illustrates our algorithm. 
The complete algorithm for constructing the core concept neuron circuit is presented in Appendix \ref{sec:main_algo}.
We provide a detailed explanation of these steps below.

\textbf{Determining semantic groups.}
 Let the concept $\mathcal{V}_a$ corresponding to $a$ be composed of $k$ elements $\{ v^1_a, \dots, v_a^k \}$.
 For each visual feature $v_a^i$ ($i = 1, \dots, k$), we define its representative vector $r(v_a^i) \in \sR^m$ as:
 \begin{equation}
     r(v_a^i) = \left [ \texttt{mean}\left(\phi^{1, l-1}_1(v_a^i) \right), \dots, \texttt{mean}\left(\phi^{1, l-1}_m(v_a^i) \right) \right ],
 \end{equation}
where $\phi^{1, l-1}_j(v_a^i)$ ($j = 1, \dots, m$) represents the $j$-th feature map and $\texttt{mean}\left(\phi^{1, l-1}_j(v_a^i)\right)$ denotes the average value across its all elements.
Next, we use agglomerative clustering \citep{Agglo_clustering} to divide the set $\{v^1_a, \dots, v_a^k\}$ into clusters, where the distance between two visual features $v^p_a$, $v^q_a$ is defined by the distance between their corresponding representative vectors $r(v^p_a)$, $r(v^q_a)$. 
The Silhouette score \citep{Silhouettes} is employed to ascertain the optimal number of clusters. The complete procedure is detailed in Algorithm \ref{alg:determin_semantic_groups}.

\textbf{Calculating edge weight.} The weight $w(a, s_i, \mathcal{V}_{a,j})$ of the branch connecting a child $s_i$ and its parent $a$'s $j$-th semantic group $\mathcal{V}_{a,j}$ is defined as: 
\begin{equation}\label{equation4}
    w(a, s_i, \mathcal{V}_{a,j}) = \frac{T(a, s_i, \mathcal{V}_{a,j})}{\sum_{s\in \mathbb{S}_a} \|T(a, s, \mathcal{V}_{a,j})\|},
\end{equation}
where $T(a, s_i, \mathcal{V}_{a,j})$ is the importance score of $s_i$ to $a$ calculated over $\mathcal{V}_{a,j}$.

\subsection{Determining Groups of Neurons and Constructing Concept Circuit}
\label{concept_circuit}
\vspace{-5pt}
%Let $\mathbb{S}_a = \{s_1, ..., s_k\}$ be the set of critical neurons of $a$.
This section describes our algorithms to (1) cluster the set of core concept neurons $\mathbb{S}_a = \{s_1, ..., s_k\}$ into distinct groups, (2) identifying the concept associated with each group, and (3) quantifying the interaction between the groups. 

\textbf{Clustering neurons into groups.} 
As mentioned in the previous section, a single neuron can encode multiple distinct visual features, while several neurons may also capture the same visual feature \citet{Olah}. 
We hypothesize that, due to the polysemantic nature of neurons \citep{Olah, Polysemantic}, a model may struggle to accurately determine whether a concept is present in an input image by relying on a single neuron. As a result, the model processes visual features not by considering individual neurons in isolation but rather by operating at the level of neuron groups. 
Intuitively, a group of neurons consists of those that capture similar visual features. This can be interpreted as \emph{two neurons belonging to the same group if they share similar semantic concept groups}. 

Building on this intuition, we develop a neuron clustering algorithm based on the semantic groups of each neuron's concept (Figure \ref{fig:neuron group}). 
Specifically, let $\mathcal{V}_{s_i}$ represent the concept of neuron $s_i$ (i.e., the primary visual features it encodes), which can be decomposed into several semantic groups $\{\mathcal{V}_{s_i, 1}, ..., \mathcal{V}_{s_i, n_i}\}$ (see Section \ref{sec:indentify_circuit}), where $n_i$ is the number of semantic groups encoded by $s_i$. For each semantic group $\mathcal{V}_{s_i, j}$, we calculate its representative activation vector $\overrightarrow{r_{s_i,j}}$ by averaging the feature maps of all its visual features, i.e., $\overrightarrow{r_{s_i,j}}:= \frac{1}{|\mathcal{V}_{s, j}|}\sum_{v_s \in \mathcal{V}_{s, j}} mean(\phi^{1, l-1}(v_s))$.
We then apply the agglomerative clustering algorithm to group the semantic groups $\mathcal{V}_{s_i, j}$ ($i = 1,..., k; j = 1, ..., n_i$), where the distance between any two groups $\mathcal{V}_{s_i, u}$ and $\mathcal{V}_{s_j, w}$ is determined by the distance of their respective representative activation vectors, $\overrightarrow{r_{s_i,u}}$ and $\overrightarrow{r_{s_j,w}}$.
Finally, we assign neurons $s_1, ..., s_k$ to the same groups based on their semantic concept groups. Specifically, neurons $s_i$ and $s_j$ are clustered together if there exists a semantic group $\mathcal{V}_{s_i, u}$ (of $s_i$) and a semantic group $\mathcal{V}_{s_j, w}$ (of $s_j$) belonging to the same group. 

\textbf{Finding neuron group concept automatically.}
We define the concept associated with a group of neurons as the union of all visual features from the corresponding semantic groups. 
Specifically, let $\{\mathcal{V}_{G,1}, \dots, \mathcal{V}_{G,k}\}$ represent the semantic groups categorized into a cluster, with their corresponding neurons $\{s_{G,1}, \dots, s_{G,k}\}$ grouped together in the same set, denoted as $G$. 
The concept of this group, denoted as $\mathbb{V}_G$, is then defined as the union of the sets $\{\mathcal{V}_{G,1}, \dots, \mathcal{V}_{G,k}\}$, i.e., $\mathbb{V}_G := \bigcup_{i=1}^{k}\mathcal{V}_{G,i}$.
We leverage a Multimodal LLM to automatically assign labels to the concept, thereby eliminating the need for a predefined labeled concept dictionary. Further details on the design of the prompts are provided in the Appendix \ref{prompt}.

\begin{figure}
    \centering
    \begin{minipage}{.35\textwidth}
        \vspace{-12mm}
        \includegraphics[width=1\columnwidth]{figures/critical_neuron3.png} 
        \vspace{3mm}
        \caption{The interaction between a neuron $s_i$ and its parent $a$.\label{fig:critical neuron}}   
    \end{minipage}
    \hfill
    \begin{minipage}{0.55\textwidth}
        \vspace{-12mm}
        \includegraphics[width=1\columnwidth]{figures/neuron_group3.png} 
        \vspace{-5mm}
        \caption{Illustration of our algorithm to determine groups of neurons.\label{fig:neuron group}}
    \end{minipage}
\vspace{-12mm}
\end{figure}


\noindent \textbf{Constructing concept circuit.} 
\label{sub_sec:constructing concept circuit}
For a given class $c$, the concept circuit $\mathcal{C}_c$ is a hierarchical tree where each node represents a neuron group concept (\emph{NGC}), and each edge illustrates the contribution of the child neuron group to its parent. 
For a node $G$, we denote by $\sV_{G} = \{\mathcal{V}_{G, 1}, ..., \mathcal{V}_{G, |\sV_{G}|}\}$ the set of semantic groups associated with $G$, and $\sS_{G} = \{ s_{G, 1}, ..., s_{G, |\sS_{G}|}\}$ represent the neurons corresponding to the semantic groups in $\sV_{G}$, i.e., $s_{G, j}$ is the core concept neuron possesses the semantic group $\mathcal{V}_{G, j}$ ($j = 1, ..., |\sV_{G}|$).
Let $G_i$ and $G_j$ be a child-parent pair in the tree, then, the relationship between $G_i$ and $G_j$ (quantified by $W(G_i, G_j))$ is represented by two aspects: the number of edges connecting elements of $G_i$ and $G_j$, and the weights of those connecting edges. The more the edges and the higher the edge weights, the stronger the relationship between $G_i$ and $G_j$. Accordingly, we define
the weight of branch connecting a child $G_i$ to its parent $G_j$ as sum of the attribution of each neuron in $\sS_{G_i}$ with each neuron in $\sS_{G_j}$: $ W(G_i, G_j) := \sum_{\substack{s_{G_i, q} \in \sS_{G_i}; \\ s_{G_j,p} \in \sS_{G_j}}}w(s_{G_j, p}, s_{G_i, q}, \mathcal{V}_{G_j,p})$.

% \begin{equation}\label{equation5}
%      W(G_i, G_j) = \sum_{\substack{s_{G_i, q} \in \sS_{G_i}; \\ s_{G_j,p} \in \sS_{G_j}}}w(s_{G_j, p}, s_{G_i, q}, \mathcal{V}_{G_j,p}).
% \end{equation}
% \begin{equation}\label{equation5}
%      W(G_i, G_j) = \frac{1}{\left | \mathcal{S}_{G_i} \right| \times  \left|\mathcal{S}_{G_j}\right|}\sum_{S^q_{G_i} \in \mathcal{S}_{G_i}; D^p_{G_j} \in \mathcal{D}_{G_j}}^{} w(S^q_{G_i}, D^p_{G_j}).
% \end{equation}
% and $\mathcal{D}^{1}_{G_j}$ the sets of semantic groups associated with $G_i$ and $G_j$, respectively; and $\mathbb{S}_{G_i}$, $\mathbb{S}_{G_j}$ the critical neurons corresponding to semantic groups in $\mathcal{D}^{1}_{G_j}$ and $\mathrm{D}_{G_i}$ re
% Suppose $\mathrm{D}_{G_i} = \{ \mathcal{D}^{1}_{G_i}, ..., \mathcal{D}^{n_i}_{G_i} \}$ is the set of semantic groups associated with $G_i$, and $\mathrm{D}_{G_j} = \{ \mathcal{D}^{1}_{G_j}, ..., \mathcal{D}^{n_j}_{G_j} \}$ are those associated with $G_j$.
% In addition, let us denote by $\mathbb{S}_{G_i} = \{s^1_{G_i}, ..., {s^{n_i}_{G_i}\}$ the for each semantic group $\mathcal{D}^{p}_{G_q}$ ($q \in \{i, j\}, p = 1, ..., n_q$), let us denote by $s^p_q$ the critical neuron possessing $\mathcal{D}^{p}_{G_q}$. 
% representing the way the groups of critical neurons 
% function of neuron groups 
% whose each node represent a semantic group 
% Let $G_1, ..., G_q$ be the groups of critical neurons at layer $l-1$, and 
% We can extend further by taking into account the connection between critical neurons of layer $l+1$ and layer $l$. For a combination $\mathcal{G}_1 = \{s_1^{l+1}, s_2^{l+1}, \dots, s_{g_1}^{l+1} \}$ with the respective semantic groups that share a common concept: $\mathcal{V}_{s_i^{l+1}, j_i}, \: \forall i \in \{1, \dots, g_1\}$ at layer $l+1$ and similar notations for $\mathcal{G}_2$ at layer $l$, we can quantify how strongly $\mathcal{G}_2$ influences $\mathcal{G}_1$ by aggregating the edge weights: $T_{\mathcal{G}_1,\mathcal{G}_2} = \frac{1}{g_1} \frac{1}{g_2} \sum_{i=1}^{g_1} \sum_{t=1}^{g_2} EdgeWeight(s_i^{l+1}, s_t^l, \mathcal{V}_{s_i^{l+1}, j_i})$. This allows us to construct a graph of COC, providing an abstract representation of how the model processes visual features without examining every individual neuron (see Figure \ref{figure1}). We refer to this graph as a \textit{concept circuit}, with applications demonstrated in Section \ref{sec:application}.
\vspace{-10pt}
\section{Experimental Evaluation}
\vspace{-5pt}
\label{sec:analysis}
\section{Analysis}

Our analysis hand-annotated all LLM-generated code for the presence/absence of dark patterns and used those counts to calculate statistical measures of difference. The original response for each prompt pair was a single file using HTML and CSS to create a single component of an ecommerce website. Rather than evaluate the code directly, we developed an automated pipeline to compile the code and screenshot the design. While these visual representations can include minor issues (the most common being that LLMs were prompted to use placeholder image URLs which do not compile), they were generally much easier to assess for the presence of dark patterns than the original code.%is built it is . It is this visual representation of the design 

Three independent, trained designers labeled each output for the presence of dark patterns. In addition, drawing on a taxonomy developed in prior work~\cite{a:44}, we labeled six attributes defined in Table~\ref{tab:darkpattern-definitions} for each LLM-generated component design: asymmetric, covert, deceptive, information hiding, restrictive, and disparate treatment. After an initial 30 components were labeled, the designers met to review any points of disagreement or uncertainty. On the basis of this, the schema was slightly updated, and the designers were able to produce labels more consistently. Nevertheless, there continued to be some opportunities for disagreement. For example, in one instance, there was a debate about whether disparate treatment dark patterns could occur in the LLM-generated designs. One designer initially believed that disparate treatment was unlikely because the LLMs generate only one component at a time, lacking distinct groups of users for comparison. However, another designer pointed out examples like discounts offered only to canceling users or first-time customers, which inherently treat different user groups unequally. This example reflected that the interpretation of these attributes can sometimes depend on personal understanding and tolerance. However, we made every effort to maintain a consistent schema %by keeping 
through real-time communication about controversial attributes, %when a component attribute seemed controversial and d
discussing each collectively as a group. The final label for each component was assigned by majority vote. 

Our analysis included both the presence/absence of dark patterns as well as the mechanisms of those dark patterns. To compare the frequency of producing dark patterns across different models and across different stakeholder interests, we used Chi-squared tests for statistical significance. 
%\vspace{-5pt}
\section{Applications}
\vspace{-5pt}
\label{sec:application}
We outlines some applications of NeurFlow.
We hypothesize that, as one neuron can have multiple meanings, a DNN looks at a group of neurons rather than individually to determine the exact features of the input. Hence, we propose a metric that assesses a model's confidence in determining whether the input contains a specific visual feature. For a group $G$ with core concept neurons $\sS_G = \{s_{G,1}, \dots, s_{G, |\sS_G|}\}$, the metric denoted as $ M(v, \sS_G, \mathcal{D}) = \exp(\frac{1}{|\sS_G|}\sum_{s \in \sS_G}\log(\|\phi_s(v)/\max(\phi_s, \mathcal{D}))\|)$, where $v \in \mathcal{D}$ and $\max(\phi_s, \mathcal{D})$ is the highest value of activation of neuron $s \in \sS_G$ on dataset $\mathcal{D}$.
This returns high score when all neurons in $G$ have high activation (indicating high confidence), while resulting in almost zero if any neuron in the group has low activation (indicating low confidence). We can use this metric to determine how similar the features in the input image are to the predetermined neuron groups concept. 
The specific setup can be found in the Appendix \ref{debugging_setup}. Figures \ref{fig:img debug} and \ref{fig:debias} demonstrate the usage of the metric and the concept circuit.
We use the term \emph{NGC} to denote the concept of a neuron group. 
\begin{figure}[t]
    \centering
    \begin{minipage}{.45\textwidth}
       \vspace{-12mm}
        \includegraphics[width=0.95\columnwidth]{figures/image_debug.png}
     %   \vspace{-3mm}
        \caption{\textbf{Using NeurFlow to reveal the reason behind model's prediction.} The top concepts can be traced throughout the circuit.}\label{fig:img debug}   
        \vspace{-10pt}
    \end{minipage}
    \hfill
    \begin{minipage}{.5\textwidth}
        \vspace{-12mm}
        \includegraphics[width=0.95\columnwidth]{figures/MLLM_captioning.pdf}
    %    \vspace{-5mm}
        \caption{Demonstration for automatically labelling and explaining the relation of NGCs on class ``great white shark" using GPT4-o \citep{Gpt4o}. The captions and the names of the NGCs are highlighted in blue, while the relations are in black.} \label{fig:captioning}
        \vspace{-10pt}
    \end{minipage}
    \vspace{-5pt}
\end{figure}
% Note that, a recent work \citep{VCC}, a concept-based method that also explains the inner mechanism of DNNs, has a similar application of debugging images. They measure how close the activations of misclassified images are to the concept vectors using $l_2$ norm. However, they provide no objective proof. In contrast, one advantage of learning via neurons is that we can edit the neurons related to a concept to see whether it has significant impact on the final predictions (see section \ref{debugging}).

\subsection{Image debugging}
\label{debugging}
We aim to use the concept circuit to identify concepts contributing to false prediction, which we call \textit{image debugging}. If a concept contributes to a class when it should not, we say that the prediction (or equivalently, the model) is \textit{biased} by that concept. \citet{Debias} propose a framework for detecting biases in a vision model by generating captions for the predicted images and tracking the common keywords found in the captions. With this method, they concluded that the pretrained ResNet50 is biased by ``flower pedals" in the class ``bee". However, correlational features do not imply causation and can lead to misjudgments. We verify and enhance the causality of their claim by examining the concept circuit of class ``bee", and conducting experiments on the probabilities of the final predictions with and without neurons that related to ``flowers". Additionally, we discover that the model also suffers from ``green background" bias (resemble ``leaves"), which is not mentioned in \citet{Debias}. 

Figure \ref{fig:debias} shows the process of debugging false positive images. Three different concepts are presented in \textit{layer4.2} of ResNet50, representing ``pink pedals", ``green background", and ``bee" respectively (we choose this layer as it has a small set of NGCs, however, our following experiment is consistent for multiple layers and with different classes). We discover that most of the false positive images have high metric score for ``pedal" and ``green background". 
To further verify the impact of these biased features, we mask all neurons in the groups of the respective concepts and find that the probability of the predictions are distorted drastically (and predictions is no longer ``bee"), as opposed to masking random neurons, which yield negligible changes. 

\begin{figure}
\begin{center}
\vspace{-12mm}
\includegraphics[width=0.9\textwidth]{figures/bias.pdf} 
\end{center}
\vspace{-3mm}
\caption{(left) The metric scores of false positive images for each concept in \textit{layer4.2} of ResNet50. (right) Showing the images that have the greatest drop in the activation of the logit neuron when masking each group concept. Verifying that the neuron groups indeed reflect the concepts.} \label{fig:debias}
\vspace{-15pt}
\end{figure}
This implies the dependence on the biased concept. \textit{But how do we know that the groups reflect the respective visual features?} If these groups indeed represent the visual features, then masking them should hinder the classification probability for images that include those features. We highlight the top images that have the largest decrease in the value of the logit neuron (corresponding to class ``bee") on both validation set of the target class and augmented dataset (see Section \ref{subsec:identify_node}). As shown in Figure \ref{fig:debias}, this process indeed yields the images that contain the respective features.

To demonstrate how NeurFlow's findings differ from those of existing methods, we conduct a qualitative experiment comparing the core concept neurons identified by NeurFlow with those identified by NeuCEPT \citet{NEUCEPT}. Detailed information about this experiment is provided in Appendix \ref{sec:compare_neucept_img_debugging}. Our observations indicate that 
% NeurFlow identifies concepts more closely resembling the original images. Additionally, 
the top logit drop images identified by NeurFlow align better with the representative examples of the corresponding concepts. Moreover, masking the core concept neuron groups identified by NeurFlow resulted in more significant changes to prediction probabilities while utilizing fewer neurons compared to the groups identified by NeuCEPT.


\subsection{Automatic identification of layer-by-layer relations}
\label{labeling}
\vspace{-5pt}
While automatically discovering concepts from inner representation has been a prominent field of research \citep{CRAFT}, automatically explaining the resulting concepts is often ignored, relying on manual annotations. \citet{Invert} utilize label description in ImageNet dataset to generate caption for neurons, however, these annotations is limited and can not be used to label low level concepts. Drawing inspiration from \citet{hoang2024llm, falcon}, we go one step further and not only use MLLM to label the (group of) neurons but also explain the relations between them in consecutive layers. Thus, we show the prospect of completing the whole picture of abstracting and explaining the inner representation in a systematic manner. 

Specifically, for two consecutive layers, we ask MLLM to describe the common visual features in a NGC, then matching with those of the top NGC (with the highest weights) at the preceding layer. This can be done iteratively throughout the concept circuit, generating a comprehensive explanation without human effort. We use a popular technique \citep{Chain_of_thought} to guide GPT4-o \citep{Gpt4o} step by step in captioning and in visual feature matching. Figure \ref{fig:captioning} shows an example of applying this technique to concept circuit of class ``great white shark". We observe that MLLM can correctly identify the common visual features within exemplary images of NGCs. Furthermore, MLLM is able to match the features from lower level NGCs to those at higher level, detailing formation of new features, showing the potential of explaining in automation, capturing the gradual process of constructing the output of the model. The prompt used in this experiment is available in Appendix \ref{prompt}.


\section{Conclusion}
\vspace{-10pt}
\label{sec:conclusion}
\section{Conclusion }
This paper introduces the Latent Radiance Field (LRF), which to our knowledge, is the first work to construct radiance field representations directly in the 2D latent space for 3D reconstruction. We present a novel framework for incorporating 3D awareness into 2D representation learning, featuring a correspondence-aware autoencoding method and a VAE-Radiance Field (VAE-RF) alignment strategy to bridge the domain gap between the 2D latent space and the natural 3D space, thereby significantly enhancing the visual quality of our LRF.
Future work will focus on incorporating our method with more compact 3D representations, efficient NVS, few-shot NVS in latent space, as well as exploring its application with potential 3D latent diffusion models.


\subsubsection*{Acknowledgments}
This work was funded by Vingroup Joint Stock Company (Vingroup JSC),Vingroup, and supported by Vingroup Innovation Foundation (VINIF) under project code VINIF.2021.DA00128.

This work is partially supported by the US National Science Foundation under SCH-2123809 project.


\bibliography{iclr2025_conference}
\bibliographystyle{iclr2025_conference}

\newpage

\appendix
\subsection{Lloyd-Max Algorithm}
\label{subsec:Lloyd-Max}
For a given quantization bitwidth $B$ and an operand $\bm{X}$, the Lloyd-Max algorithm finds $2^B$ quantization levels $\{\hat{x}_i\}_{i=1}^{2^B}$ such that quantizing $\bm{X}$ by rounding each scalar in $\bm{X}$ to the nearest quantization level minimizes the quantization MSE. 

The algorithm starts with an initial guess of quantization levels and then iteratively computes quantization thresholds $\{\tau_i\}_{i=1}^{2^B-1}$ and updates quantization levels $\{\hat{x}_i\}_{i=1}^{2^B}$. Specifically, at iteration $n$, thresholds are set to the midpoints of the previous iteration's levels:
\begin{align*}
    \tau_i^{(n)}=\frac{\hat{x}_i^{(n-1)}+\hat{x}_{i+1}^{(n-1)}}2 \text{ for } i=1\ldots 2^B-1
\end{align*}
Subsequently, the quantization levels are re-computed as conditional means of the data regions defined by the new thresholds:
\begin{align*}
    \hat{x}_i^{(n)}=\mathbb{E}\left[ \bm{X} \big| \bm{X}\in [\tau_{i-1}^{(n)},\tau_i^{(n)}] \right] \text{ for } i=1\ldots 2^B
\end{align*}
where to satisfy boundary conditions we have $\tau_0=-\infty$ and $\tau_{2^B}=\infty$. The algorithm iterates the above steps until convergence.

Figure \ref{fig:lm_quant} compares the quantization levels of a $7$-bit floating point (E3M3) quantizer (left) to a $7$-bit Lloyd-Max quantizer (right) when quantizing a layer of weights from the GPT3-126M model at a per-tensor granularity. As shown, the Lloyd-Max quantizer achieves substantially lower quantization MSE. Further, Table \ref{tab:FP7_vs_LM7} shows the superior perplexity achieved by Lloyd-Max quantizers for bitwidths of $7$, $6$ and $5$. The difference between the quantizers is clear at 5 bits, where per-tensor FP quantization incurs a drastic and unacceptable increase in perplexity, while Lloyd-Max quantization incurs a much smaller increase. Nevertheless, we note that even the optimal Lloyd-Max quantizer incurs a notable ($\sim 1.5$) increase in perplexity due to the coarse granularity of quantization. 

\begin{figure}[h]
  \centering
  \includegraphics[width=0.7\linewidth]{sections/figures/LM7_FP7.pdf}
  \caption{\small Quantization levels and the corresponding quantization MSE of Floating Point (left) vs Lloyd-Max (right) Quantizers for a layer of weights in the GPT3-126M model.}
  \label{fig:lm_quant}
\end{figure}

\begin{table}[h]\scriptsize
\begin{center}
\caption{\label{tab:FP7_vs_LM7} \small Comparing perplexity (lower is better) achieved by floating point quantizers and Lloyd-Max quantizers on a GPT3-126M model for the Wikitext-103 dataset.}
\begin{tabular}{c|cc|c}
\hline
 \multirow{2}{*}{\textbf{Bitwidth}} & \multicolumn{2}{|c|}{\textbf{Floating-Point Quantizer}} & \textbf{Lloyd-Max Quantizer} \\
 & Best Format & Wikitext-103 Perplexity & Wikitext-103 Perplexity \\
\hline
7 & E3M3 & 18.32 & 18.27 \\
6 & E3M2 & 19.07 & 18.51 \\
5 & E4M0 & 43.89 & 19.71 \\
\hline
\end{tabular}
\end{center}
\end{table}

\subsection{Proof of Local Optimality of LO-BCQ}
\label{subsec:lobcq_opt_proof}
For a given block $\bm{b}_j$, the quantization MSE during LO-BCQ can be empirically evaluated as $\frac{1}{L_b}\lVert \bm{b}_j- \bm{\hat{b}}_j\rVert^2_2$ where $\bm{\hat{b}}_j$ is computed from equation (\ref{eq:clustered_quantization_definition}) as $C_{f(\bm{b}_j)}(\bm{b}_j)$. Further, for a given block cluster $\mathcal{B}_i$, we compute the quantization MSE as $\frac{1}{|\mathcal{B}_{i}|}\sum_{\bm{b} \in \mathcal{B}_{i}} \frac{1}{L_b}\lVert \bm{b}- C_i^{(n)}(\bm{b})\rVert^2_2$. Therefore, at the end of iteration $n$, we evaluate the overall quantization MSE $J^{(n)}$ for a given operand $\bm{X}$ composed of $N_c$ block clusters as:
\begin{align*}
    \label{eq:mse_iter_n}
    J^{(n)} = \frac{1}{N_c} \sum_{i=1}^{N_c} \frac{1}{|\mathcal{B}_{i}^{(n)}|}\sum_{\bm{v} \in \mathcal{B}_{i}^{(n)}} \frac{1}{L_b}\lVert \bm{b}- B_i^{(n)}(\bm{b})\rVert^2_2
\end{align*}

At the end of iteration $n$, the codebooks are updated from $\mathcal{C}^{(n-1)}$ to $\mathcal{C}^{(n)}$. However, the mapping of a given vector $\bm{b}_j$ to quantizers $\mathcal{C}^{(n)}$ remains as  $f^{(n)}(\bm{b}_j)$. At the next iteration, during the vector clustering step, $f^{(n+1)}(\bm{b}_j)$ finds new mapping of $\bm{b}_j$ to updated codebooks $\mathcal{C}^{(n)}$ such that the quantization MSE over the candidate codebooks is minimized. Therefore, we obtain the following result for $\bm{b}_j$:
\begin{align*}
\frac{1}{L_b}\lVert \bm{b}_j - C_{f^{(n+1)}(\bm{b}_j)}^{(n)}(\bm{b}_j)\rVert^2_2 \le \frac{1}{L_b}\lVert \bm{b}_j - C_{f^{(n)}(\bm{b}_j)}^{(n)}(\bm{b}_j)\rVert^2_2
\end{align*}

That is, quantizing $\bm{b}_j$ at the end of the block clustering step of iteration $n+1$ results in lower quantization MSE compared to quantizing at the end of iteration $n$. Since this is true for all $\bm{b} \in \bm{X}$, we assert the following:
\begin{equation}
\begin{split}
\label{eq:mse_ineq_1}
    \tilde{J}^{(n+1)} &= \frac{1}{N_c} \sum_{i=1}^{N_c} \frac{1}{|\mathcal{B}_{i}^{(n+1)}|}\sum_{\bm{b} \in \mathcal{B}_{i}^{(n+1)}} \frac{1}{L_b}\lVert \bm{b} - C_i^{(n)}(b)\rVert^2_2 \le J^{(n)}
\end{split}
\end{equation}
where $\tilde{J}^{(n+1)}$ is the the quantization MSE after the vector clustering step at iteration $n+1$.

Next, during the codebook update step (\ref{eq:quantizers_update}) at iteration $n+1$, the per-cluster codebooks $\mathcal{C}^{(n)}$ are updated to $\mathcal{C}^{(n+1)}$ by invoking the Lloyd-Max algorithm \citep{Lloyd}. We know that for any given value distribution, the Lloyd-Max algorithm minimizes the quantization MSE. Therefore, for a given vector cluster $\mathcal{B}_i$ we obtain the following result:

\begin{equation}
    \frac{1}{|\mathcal{B}_{i}^{(n+1)}|}\sum_{\bm{b} \in \mathcal{B}_{i}^{(n+1)}} \frac{1}{L_b}\lVert \bm{b}- C_i^{(n+1)}(\bm{b})\rVert^2_2 \le \frac{1}{|\mathcal{B}_{i}^{(n+1)}|}\sum_{\bm{b} \in \mathcal{B}_{i}^{(n+1)}} \frac{1}{L_b}\lVert \bm{b}- C_i^{(n)}(\bm{b})\rVert^2_2
\end{equation}

The above equation states that quantizing the given block cluster $\mathcal{B}_i$ after updating the associated codebook from $C_i^{(n)}$ to $C_i^{(n+1)}$ results in lower quantization MSE. Since this is true for all the block clusters, we derive the following result: 
\begin{equation}
\begin{split}
\label{eq:mse_ineq_2}
     J^{(n+1)} &= \frac{1}{N_c} \sum_{i=1}^{N_c} \frac{1}{|\mathcal{B}_{i}^{(n+1)}|}\sum_{\bm{b} \in \mathcal{B}_{i}^{(n+1)}} \frac{1}{L_b}\lVert \bm{b}- C_i^{(n+1)}(\bm{b})\rVert^2_2  \le \tilde{J}^{(n+1)}   
\end{split}
\end{equation}

Following (\ref{eq:mse_ineq_1}) and (\ref{eq:mse_ineq_2}), we find that the quantization MSE is non-increasing for each iteration, that is, $J^{(1)} \ge J^{(2)} \ge J^{(3)} \ge \ldots \ge J^{(M)}$ where $M$ is the maximum number of iterations. 
%Therefore, we can say that if the algorithm converges, then it must be that it has converged to a local minimum. 
\hfill $\blacksquare$


\begin{figure}
    \begin{center}
    \includegraphics[width=0.5\textwidth]{sections//figures/mse_vs_iter.pdf}
    \end{center}
    \caption{\small NMSE vs iterations during LO-BCQ compared to other block quantization proposals}
    \label{fig:nmse_vs_iter}
\end{figure}

Figure \ref{fig:nmse_vs_iter} shows the empirical convergence of LO-BCQ across several block lengths and number of codebooks. Also, the MSE achieved by LO-BCQ is compared to baselines such as MXFP and VSQ. As shown, LO-BCQ converges to a lower MSE than the baselines. Further, we achieve better convergence for larger number of codebooks ($N_c$) and for a smaller block length ($L_b$), both of which increase the bitwidth of BCQ (see Eq \ref{eq:bitwidth_bcq}).


\subsection{Additional Accuracy Results}
%Table \ref{tab:lobcq_config} lists the various LOBCQ configurations and their corresponding bitwidths.
\begin{table}
\setlength{\tabcolsep}{4.75pt}
\begin{center}
\caption{\label{tab:lobcq_config} Various LO-BCQ configurations and their bitwidths.}
\begin{tabular}{|c||c|c|c|c||c|c||c|} 
\hline
 & \multicolumn{4}{|c||}{$L_b=8$} & \multicolumn{2}{|c||}{$L_b=4$} & $L_b=2$ \\
 \hline
 \backslashbox{$L_A$\kern-1em}{\kern-1em$N_c$} & 2 & 4 & 8 & 16 & 2 & 4 & 2 \\
 \hline
 64 & 4.25 & 4.375 & 4.5 & 4.625 & 4.375 & 4.625 & 4.625\\
 \hline
 32 & 4.375 & 4.5 & 4.625& 4.75 & 4.5 & 4.75 & 4.75 \\
 \hline
 16 & 4.625 & 4.75& 4.875 & 5 & 4.75 & 5 & 5 \\
 \hline
\end{tabular}
\end{center}
\end{table}

%\subsection{Perplexity achieved by various LO-BCQ configurations on Wikitext-103 dataset}

\begin{table} \centering
\begin{tabular}{|c||c|c|c|c||c|c||c|} 
\hline
 $L_b \rightarrow$& \multicolumn{4}{c||}{8} & \multicolumn{2}{c||}{4} & 2\\
 \hline
 \backslashbox{$L_A$\kern-1em}{\kern-1em$N_c$} & 2 & 4 & 8 & 16 & 2 & 4 & 2  \\
 %$N_c \rightarrow$ & 2 & 4 & 8 & 16 & 2 & 4 & 2 \\
 \hline
 \hline
 \multicolumn{8}{c}{GPT3-1.3B (FP32 PPL = 9.98)} \\ 
 \hline
 \hline
 64 & 10.40 & 10.23 & 10.17 & 10.15 &  10.28 & 10.18 & 10.19 \\
 \hline
 32 & 10.25 & 10.20 & 10.15 & 10.12 &  10.23 & 10.17 & 10.17 \\
 \hline
 16 & 10.22 & 10.16 & 10.10 & 10.09 &  10.21 & 10.14 & 10.16 \\
 \hline
  \hline
 \multicolumn{8}{c}{GPT3-8B (FP32 PPL = 7.38)} \\ 
 \hline
 \hline
 64 & 7.61 & 7.52 & 7.48 &  7.47 &  7.55 &  7.49 & 7.50 \\
 \hline
 32 & 7.52 & 7.50 & 7.46 &  7.45 &  7.52 &  7.48 & 7.48  \\
 \hline
 16 & 7.51 & 7.48 & 7.44 &  7.44 &  7.51 &  7.49 & 7.47  \\
 \hline
\end{tabular}
\caption{\label{tab:ppl_gpt3_abalation} Wikitext-103 perplexity across GPT3-1.3B and 8B models.}
\end{table}

\begin{table} \centering
\begin{tabular}{|c||c|c|c|c||} 
\hline
 $L_b \rightarrow$& \multicolumn{4}{c||}{8}\\
 \hline
 \backslashbox{$L_A$\kern-1em}{\kern-1em$N_c$} & 2 & 4 & 8 & 16 \\
 %$N_c \rightarrow$ & 2 & 4 & 8 & 16 & 2 & 4 & 2 \\
 \hline
 \hline
 \multicolumn{5}{|c|}{Llama2-7B (FP32 PPL = 5.06)} \\ 
 \hline
 \hline
 64 & 5.31 & 5.26 & 5.19 & 5.18  \\
 \hline
 32 & 5.23 & 5.25 & 5.18 & 5.15  \\
 \hline
 16 & 5.23 & 5.19 & 5.16 & 5.14  \\
 \hline
 \multicolumn{5}{|c|}{Nemotron4-15B (FP32 PPL = 5.87)} \\ 
 \hline
 \hline
 64  & 6.3 & 6.20 & 6.13 & 6.08  \\
 \hline
 32  & 6.24 & 6.12 & 6.07 & 6.03  \\
 \hline
 16  & 6.12 & 6.14 & 6.04 & 6.02  \\
 \hline
 \multicolumn{5}{|c|}{Nemotron4-340B (FP32 PPL = 3.48)} \\ 
 \hline
 \hline
 64 & 3.67 & 3.62 & 3.60 & 3.59 \\
 \hline
 32 & 3.63 & 3.61 & 3.59 & 3.56 \\
 \hline
 16 & 3.61 & 3.58 & 3.57 & 3.55 \\
 \hline
\end{tabular}
\caption{\label{tab:ppl_llama7B_nemo15B} Wikitext-103 perplexity compared to FP32 baseline in Llama2-7B and Nemotron4-15B, 340B models}
\end{table}

%\subsection{Perplexity achieved by various LO-BCQ configurations on MMLU dataset}


\begin{table} \centering
\begin{tabular}{|c||c|c|c|c||c|c|c|c|} 
\hline
 $L_b \rightarrow$& \multicolumn{4}{c||}{8} & \multicolumn{4}{c||}{8}\\
 \hline
 \backslashbox{$L_A$\kern-1em}{\kern-1em$N_c$} & 2 & 4 & 8 & 16 & 2 & 4 & 8 & 16  \\
 %$N_c \rightarrow$ & 2 & 4 & 8 & 16 & 2 & 4 & 2 \\
 \hline
 \hline
 \multicolumn{5}{|c|}{Llama2-7B (FP32 Accuracy = 45.8\%)} & \multicolumn{4}{|c|}{Llama2-70B (FP32 Accuracy = 69.12\%)} \\ 
 \hline
 \hline
 64 & 43.9 & 43.4 & 43.9 & 44.9 & 68.07 & 68.27 & 68.17 & 68.75 \\
 \hline
 32 & 44.5 & 43.8 & 44.9 & 44.5 & 68.37 & 68.51 & 68.35 & 68.27  \\
 \hline
 16 & 43.9 & 42.7 & 44.9 & 45 & 68.12 & 68.77 & 68.31 & 68.59  \\
 \hline
 \hline
 \multicolumn{5}{|c|}{GPT3-22B (FP32 Accuracy = 38.75\%)} & \multicolumn{4}{|c|}{Nemotron4-15B (FP32 Accuracy = 64.3\%)} \\ 
 \hline
 \hline
 64 & 36.71 & 38.85 & 38.13 & 38.92 & 63.17 & 62.36 & 63.72 & 64.09 \\
 \hline
 32 & 37.95 & 38.69 & 39.45 & 38.34 & 64.05 & 62.30 & 63.8 & 64.33  \\
 \hline
 16 & 38.88 & 38.80 & 38.31 & 38.92 & 63.22 & 63.51 & 63.93 & 64.43  \\
 \hline
\end{tabular}
\caption{\label{tab:mmlu_abalation} Accuracy on MMLU dataset across GPT3-22B, Llama2-7B, 70B and Nemotron4-15B models.}
\end{table}


%\subsection{Perplexity achieved by various LO-BCQ configurations on LM evaluation harness}

\begin{table} \centering
\begin{tabular}{|c||c|c|c|c||c|c|c|c|} 
\hline
 $L_b \rightarrow$& \multicolumn{4}{c||}{8} & \multicolumn{4}{c||}{8}\\
 \hline
 \backslashbox{$L_A$\kern-1em}{\kern-1em$N_c$} & 2 & 4 & 8 & 16 & 2 & 4 & 8 & 16  \\
 %$N_c \rightarrow$ & 2 & 4 & 8 & 16 & 2 & 4 & 2 \\
 \hline
 \hline
 \multicolumn{5}{|c|}{Race (FP32 Accuracy = 37.51\%)} & \multicolumn{4}{|c|}{Boolq (FP32 Accuracy = 64.62\%)} \\ 
 \hline
 \hline
 64 & 36.94 & 37.13 & 36.27 & 37.13 & 63.73 & 62.26 & 63.49 & 63.36 \\
 \hline
 32 & 37.03 & 36.36 & 36.08 & 37.03 & 62.54 & 63.51 & 63.49 & 63.55  \\
 \hline
 16 & 37.03 & 37.03 & 36.46 & 37.03 & 61.1 & 63.79 & 63.58 & 63.33  \\
 \hline
 \hline
 \multicolumn{5}{|c|}{Winogrande (FP32 Accuracy = 58.01\%)} & \multicolumn{4}{|c|}{Piqa (FP32 Accuracy = 74.21\%)} \\ 
 \hline
 \hline
 64 & 58.17 & 57.22 & 57.85 & 58.33 & 73.01 & 73.07 & 73.07 & 72.80 \\
 \hline
 32 & 59.12 & 58.09 & 57.85 & 58.41 & 73.01 & 73.94 & 72.74 & 73.18  \\
 \hline
 16 & 57.93 & 58.88 & 57.93 & 58.56 & 73.94 & 72.80 & 73.01 & 73.94  \\
 \hline
\end{tabular}
\caption{\label{tab:mmlu_abalation} Accuracy on LM evaluation harness tasks on GPT3-1.3B model.}
\end{table}

\begin{table} \centering
\begin{tabular}{|c||c|c|c|c||c|c|c|c|} 
\hline
 $L_b \rightarrow$& \multicolumn{4}{c||}{8} & \multicolumn{4}{c||}{8}\\
 \hline
 \backslashbox{$L_A$\kern-1em}{\kern-1em$N_c$} & 2 & 4 & 8 & 16 & 2 & 4 & 8 & 16  \\
 %$N_c \rightarrow$ & 2 & 4 & 8 & 16 & 2 & 4 & 2 \\
 \hline
 \hline
 \multicolumn{5}{|c|}{Race (FP32 Accuracy = 41.34\%)} & \multicolumn{4}{|c|}{Boolq (FP32 Accuracy = 68.32\%)} \\ 
 \hline
 \hline
 64 & 40.48 & 40.10 & 39.43 & 39.90 & 69.20 & 68.41 & 69.45 & 68.56 \\
 \hline
 32 & 39.52 & 39.52 & 40.77 & 39.62 & 68.32 & 67.43 & 68.17 & 69.30  \\
 \hline
 16 & 39.81 & 39.71 & 39.90 & 40.38 & 68.10 & 66.33 & 69.51 & 69.42  \\
 \hline
 \hline
 \multicolumn{5}{|c|}{Winogrande (FP32 Accuracy = 67.88\%)} & \multicolumn{4}{|c|}{Piqa (FP32 Accuracy = 78.78\%)} \\ 
 \hline
 \hline
 64 & 66.85 & 66.61 & 67.72 & 67.88 & 77.31 & 77.42 & 77.75 & 77.64 \\
 \hline
 32 & 67.25 & 67.72 & 67.72 & 67.00 & 77.31 & 77.04 & 77.80 & 77.37  \\
 \hline
 16 & 68.11 & 68.90 & 67.88 & 67.48 & 77.37 & 78.13 & 78.13 & 77.69  \\
 \hline
\end{tabular}
\caption{\label{tab:mmlu_abalation} Accuracy on LM evaluation harness tasks on GPT3-8B model.}
\end{table}

\begin{table} \centering
\begin{tabular}{|c||c|c|c|c||c|c|c|c|} 
\hline
 $L_b \rightarrow$& \multicolumn{4}{c||}{8} & \multicolumn{4}{c||}{8}\\
 \hline
 \backslashbox{$L_A$\kern-1em}{\kern-1em$N_c$} & 2 & 4 & 8 & 16 & 2 & 4 & 8 & 16  \\
 %$N_c \rightarrow$ & 2 & 4 & 8 & 16 & 2 & 4 & 2 \\
 \hline
 \hline
 \multicolumn{5}{|c|}{Race (FP32 Accuracy = 40.67\%)} & \multicolumn{4}{|c|}{Boolq (FP32 Accuracy = 76.54\%)} \\ 
 \hline
 \hline
 64 & 40.48 & 40.10 & 39.43 & 39.90 & 75.41 & 75.11 & 77.09 & 75.66 \\
 \hline
 32 & 39.52 & 39.52 & 40.77 & 39.62 & 76.02 & 76.02 & 75.96 & 75.35  \\
 \hline
 16 & 39.81 & 39.71 & 39.90 & 40.38 & 75.05 & 73.82 & 75.72 & 76.09  \\
 \hline
 \hline
 \multicolumn{5}{|c|}{Winogrande (FP32 Accuracy = 70.64\%)} & \multicolumn{4}{|c|}{Piqa (FP32 Accuracy = 79.16\%)} \\ 
 \hline
 \hline
 64 & 69.14 & 70.17 & 70.17 & 70.56 & 78.24 & 79.00 & 78.62 & 78.73 \\
 \hline
 32 & 70.96 & 69.69 & 71.27 & 69.30 & 78.56 & 79.49 & 79.16 & 78.89  \\
 \hline
 16 & 71.03 & 69.53 & 69.69 & 70.40 & 78.13 & 79.16 & 79.00 & 79.00  \\
 \hline
\end{tabular}
\caption{\label{tab:mmlu_abalation} Accuracy on LM evaluation harness tasks on GPT3-22B model.}
\end{table}

\begin{table} \centering
\begin{tabular}{|c||c|c|c|c||c|c|c|c|} 
\hline
 $L_b \rightarrow$& \multicolumn{4}{c||}{8} & \multicolumn{4}{c||}{8}\\
 \hline
 \backslashbox{$L_A$\kern-1em}{\kern-1em$N_c$} & 2 & 4 & 8 & 16 & 2 & 4 & 8 & 16  \\
 %$N_c \rightarrow$ & 2 & 4 & 8 & 16 & 2 & 4 & 2 \\
 \hline
 \hline
 \multicolumn{5}{|c|}{Race (FP32 Accuracy = 44.4\%)} & \multicolumn{4}{|c|}{Boolq (FP32 Accuracy = 79.29\%)} \\ 
 \hline
 \hline
 64 & 42.49 & 42.51 & 42.58 & 43.45 & 77.58 & 77.37 & 77.43 & 78.1 \\
 \hline
 32 & 43.35 & 42.49 & 43.64 & 43.73 & 77.86 & 75.32 & 77.28 & 77.86  \\
 \hline
 16 & 44.21 & 44.21 & 43.64 & 42.97 & 78.65 & 77 & 76.94 & 77.98  \\
 \hline
 \hline
 \multicolumn{5}{|c|}{Winogrande (FP32 Accuracy = 69.38\%)} & \multicolumn{4}{|c|}{Piqa (FP32 Accuracy = 78.07\%)} \\ 
 \hline
 \hline
 64 & 68.9 & 68.43 & 69.77 & 68.19 & 77.09 & 76.82 & 77.09 & 77.86 \\
 \hline
 32 & 69.38 & 68.51 & 68.82 & 68.90 & 78.07 & 76.71 & 78.07 & 77.86  \\
 \hline
 16 & 69.53 & 67.09 & 69.38 & 68.90 & 77.37 & 77.8 & 77.91 & 77.69  \\
 \hline
\end{tabular}
\caption{\label{tab:mmlu_abalation} Accuracy on LM evaluation harness tasks on Llama2-7B model.}
\end{table}

\begin{table} \centering
\begin{tabular}{|c||c|c|c|c||c|c|c|c|} 
\hline
 $L_b \rightarrow$& \multicolumn{4}{c||}{8} & \multicolumn{4}{c||}{8}\\
 \hline
 \backslashbox{$L_A$\kern-1em}{\kern-1em$N_c$} & 2 & 4 & 8 & 16 & 2 & 4 & 8 & 16  \\
 %$N_c \rightarrow$ & 2 & 4 & 8 & 16 & 2 & 4 & 2 \\
 \hline
 \hline
 \multicolumn{5}{|c|}{Race (FP32 Accuracy = 48.8\%)} & \multicolumn{4}{|c|}{Boolq (FP32 Accuracy = 85.23\%)} \\ 
 \hline
 \hline
 64 & 49.00 & 49.00 & 49.28 & 48.71 & 82.82 & 84.28 & 84.03 & 84.25 \\
 \hline
 32 & 49.57 & 48.52 & 48.33 & 49.28 & 83.85 & 84.46 & 84.31 & 84.93  \\
 \hline
 16 & 49.85 & 49.09 & 49.28 & 48.99 & 85.11 & 84.46 & 84.61 & 83.94  \\
 \hline
 \hline
 \multicolumn{5}{|c|}{Winogrande (FP32 Accuracy = 79.95\%)} & \multicolumn{4}{|c|}{Piqa (FP32 Accuracy = 81.56\%)} \\ 
 \hline
 \hline
 64 & 78.77 & 78.45 & 78.37 & 79.16 & 81.45 & 80.69 & 81.45 & 81.5 \\
 \hline
 32 & 78.45 & 79.01 & 78.69 & 80.66 & 81.56 & 80.58 & 81.18 & 81.34  \\
 \hline
 16 & 79.95 & 79.56 & 79.79 & 79.72 & 81.28 & 81.66 & 81.28 & 80.96  \\
 \hline
\end{tabular}
\caption{\label{tab:mmlu_abalation} Accuracy on LM evaluation harness tasks on Llama2-70B model.}
\end{table}

%\section{MSE Studies}
%\textcolor{red}{TODO}


\subsection{Number Formats and Quantization Method}
\label{subsec:numFormats_quantMethod}
\subsubsection{Integer Format}
An $n$-bit signed integer (INT) is typically represented with a 2s-complement format \citep{yao2022zeroquant,xiao2023smoothquant,dai2021vsq}, where the most significant bit denotes the sign.

\subsubsection{Floating Point Format}
An $n$-bit signed floating point (FP) number $x$ comprises of a 1-bit sign ($x_{\mathrm{sign}}$), $B_m$-bit mantissa ($x_{\mathrm{mant}}$) and $B_e$-bit exponent ($x_{\mathrm{exp}}$) such that $B_m+B_e=n-1$. The associated constant exponent bias ($E_{\mathrm{bias}}$) is computed as $(2^{{B_e}-1}-1)$. We denote this format as $E_{B_e}M_{B_m}$.  

\subsubsection{Quantization Scheme}
\label{subsec:quant_method}
A quantization scheme dictates how a given unquantized tensor is converted to its quantized representation. We consider FP formats for the purpose of illustration. Given an unquantized tensor $\bm{X}$ and an FP format $E_{B_e}M_{B_m}$, we first, we compute the quantization scale factor $s_X$ that maps the maximum absolute value of $\bm{X}$ to the maximum quantization level of the $E_{B_e}M_{B_m}$ format as follows:
\begin{align}
\label{eq:sf}
    s_X = \frac{\mathrm{max}(|\bm{X}|)}{\mathrm{max}(E_{B_e}M_{B_m})}
\end{align}
In the above equation, $|\cdot|$ denotes the absolute value function.

Next, we scale $\bm{X}$ by $s_X$ and quantize it to $\hat{\bm{X}}$ by rounding it to the nearest quantization level of $E_{B_e}M_{B_m}$ as:

\begin{align}
\label{eq:tensor_quant}
    \hat{\bm{X}} = \text{round-to-nearest}\left(\frac{\bm{X}}{s_X}, E_{B_e}M_{B_m}\right)
\end{align}

We perform dynamic max-scaled quantization \citep{wu2020integer}, where the scale factor $s$ for activations is dynamically computed during runtime.

\subsection{Vector Scaled Quantization}
\begin{wrapfigure}{r}{0.35\linewidth}
  \centering
  \includegraphics[width=\linewidth]{sections/figures/vsquant.jpg}
  \caption{\small Vectorwise decomposition for per-vector scaled quantization (VSQ \citep{dai2021vsq}).}
  \label{fig:vsquant}
\end{wrapfigure}
During VSQ \citep{dai2021vsq}, the operand tensors are decomposed into 1D vectors in a hardware friendly manner as shown in Figure \ref{fig:vsquant}. Since the decomposed tensors are used as operands in matrix multiplications during inference, it is beneficial to perform this decomposition along the reduction dimension of the multiplication. The vectorwise quantization is performed similar to tensorwise quantization described in Equations \ref{eq:sf} and \ref{eq:tensor_quant}, where a scale factor $s_v$ is required for each vector $\bm{v}$ that maps the maximum absolute value of that vector to the maximum quantization level. While smaller vector lengths can lead to larger accuracy gains, the associated memory and computational overheads due to the per-vector scale factors increases. To alleviate these overheads, VSQ \citep{dai2021vsq} proposed a second level quantization of the per-vector scale factors to unsigned integers, while MX \citep{rouhani2023shared} quantizes them to integer powers of 2 (denoted as $2^{INT}$).

\subsubsection{MX Format}
The MX format proposed in \citep{rouhani2023microscaling} introduces the concept of sub-block shifting. For every two scalar elements of $b$-bits each, there is a shared exponent bit. The value of this exponent bit is determined through an empirical analysis that targets minimizing quantization MSE. We note that the FP format $E_{1}M_{b}$ is strictly better than MX from an accuracy perspective since it allocates a dedicated exponent bit to each scalar as opposed to sharing it across two scalars. Therefore, we conservatively bound the accuracy of a $b+2$-bit signed MX format with that of a $E_{1}M_{b}$ format in our comparisons. For instance, we use E1M2 format as a proxy for MX4.

\begin{figure}
    \centering
    \includegraphics[width=1\linewidth]{sections//figures/BlockFormats.pdf}
    \caption{\small Comparing LO-BCQ to MX format.}
    \label{fig:block_formats}
\end{figure}

Figure \ref{fig:block_formats} compares our $4$-bit LO-BCQ block format to MX \citep{rouhani2023microscaling}. As shown, both LO-BCQ and MX decompose a given operand tensor into block arrays and each block array into blocks. Similar to MX, we find that per-block quantization ($L_b < L_A$) leads to better accuracy due to increased flexibility. While MX achieves this through per-block $1$-bit micro-scales, we associate a dedicated codebook to each block through a per-block codebook selector. Further, MX quantizes the per-block array scale-factor to E8M0 format without per-tensor scaling. In contrast during LO-BCQ, we find that per-tensor scaling combined with quantization of per-block array scale-factor to E4M3 format results in superior inference accuracy across models. 


\end{document}
