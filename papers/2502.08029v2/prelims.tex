% Good


\section{Preliminaries}
\label{sec:prelims}

We use capital bold letters (\mA,\mB,\mC,...) to denote matrices, lowercase bold letters to denote vectors (\va,\vb,\vc,...), and lowercase non-bold letters to denote scalars (a,b,c,...).
\bbR is the set of reals, \bbC is the set of complex numbers, and \bbN is the set of natural numbers.
We will let \cA denote an algorithm.
\(\vx^\intercal\) denotes the transpose and \(\vx^\herm\) denotes the conjugate transpose.
We use bracket notation \([\va]_{i}\) to denote the \(i^{th}\) entry of \va and \([\mA]_{i,j}\) to denote the \((i,j)\) entry of \mA.
\(\norm{\va}_2\) denotes the L2 norm of a vector.
\(\otimes\) denotes the Kronecker product.
\(\tr(\mA)\) is the trace of a matrix.
We let \([n] = \{1,\ldots,n\}\) be the set of integers from 1 to \(n\).
For probability distributions \(\bbP\) and \(\bbQ\) on space \((\Omega,\cF)\), \(D_{TV}(\bbP,\bbQ)\) is the total variation distance between \bbP and \bbQ, and \(D_{KL}(\bbP\,\|\,\bbQ)\) is the Kullback-Liebler divergence.
We will let \(\alphabet \subseteq \bbC\) denote an alphabet.
