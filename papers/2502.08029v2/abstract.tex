% Good
\begin{abstract}

    We study the computational model where we can access a matrix $\mathbf{A}$ only by computing matrix-vector products $\mathbf{A}\mathrm{x}$ for vectors of the form $\mathrm{x} = \mathrm{x}_1 \otimes \cdots \otimes \mathrm{x}_q$.
    We prove exponential lower bounds on the number of queries needed to estimate various properties, including the trace and the top eigenvalue of $\mathbf{A}$.
    Our proofs hold for all adaptive algorithms, modulo a mild conditioning assumption on the algorithm's queries.
    We further prove that algorithms whose queries come from a small alphabet (e.g., $\mathrm{x}_i \in \{\pm1\}^n$) cannot test if $\mathbf{A}$ is identically zero with polynomial complexity, despite the fact that a single query using Gaussian vectors solves the problem with probability 1.
    In steep contrast to the non-Kronecker case, this shows that sketching $\mathbf{A}$ with different distributions of the same subguassian norm can yield exponentially different query complexities.
    Our proofs follow from the observation that random vectors with Kronecker structure have exponentially smaller inner products than their non-Kronecker counterparts.

\end{abstract}
