%New colors defined below
\definecolor{codegreen}{rgb}{0,0.6,0}
\definecolor{codegray}{rgb}{0.5,0.5,0.5}
\definecolor{codepurple}{rgb}{0.58,0,0.82}
\definecolor{backcolour}{rgb}{0.95,0.95,0.92}

%Code listing style named "mystyle"
\lstdefinestyle{mystyle}{
  backgroundcolor=\color{backcolour},   
  commentstyle=\color{codegreen},
  keywordstyle=\color{magenta},
  numberstyle=\tiny\color{codegray},
  stringstyle=\color{codepurple},
  basicstyle=\ttfamily\footnotesize,
  breakatwhitespace=false,         
  breaklines=true,                 
  captionpos=b,                    
  keepspaces=true,                 
  numbers=left,                    
  numbersep=5pt,                  
  showspaces=false,                
  showstringspaces=false,
  showtabs=false,                  
  tabsize=2,
  frame=ltb,
  framerule=0pt,
  % lineskip=-10pt, % Adjust this value if needed
}

\lstset{style=mystyle}

\section{LLM-based Generation Pipeline System Prompts}
\label{appendix:sysprompts}

Our motion graphics synthesis and verification pipeline uses LLM-based in-context learning. Here we provide the system prompts for (1) our LLM animation synthesizer, (2) our LLM \dslname{} synthesizer, and (3) our correction iterations.
%We provide the system prompts used by our 
%Below we show the system prompts used for our LLM-based program synthesizers for 1) motion graphics program, 2) \dslname{} program, and 3) correction iteration.
In our implementation the system prompts include complete documentation for an animation API (GSAP\,\cite{gsap}) and our \dslname{} DSL, but due to the length of 
this documentation we have abbreviated those sections of these prompts for clarity.
%We only show selected animation API and \dslname{} documentation for brevity as most details has been discussed in the main paper.
%We will release the full prompts with the rest of our code. 
% \maneesh{We cannot omit any details. Please give three system prompts -- 1 for the LLM MoVer Synthesis, LLM Animation Synthesis and the Correction Iteration}

\subsection{LLM-based animation synthesizer}
This system prompt asks the LLM to generate a motion graphics animation using GSAP\,\cite{gsap} from the input text prompt.

\begin{lstlisting}
### Overview
- You are an experienced programmer skilled in creating SVG animations using the API documentation provided below.
- Please output JavaScript code based on the instruction.
- Think step by step.

### Restrictions
- Only use functions provided in the API documentation.
- Avoid doing calculations yourself. Use the functions provided in the API documentation to do the calculations whenever possible.
- Always use `document.querySelector()` to select SVG elements.
- Always create the timeline element with `createTimeline()`
- Always use `getCenterPosition(element)` to get the position of an element, and use `getSize(element)` to get the width and height of an element. 
- Only use `getProperty()` to obtain attributes other than position and size of an element.
- Within the JavaScript code, annotate the lines of animation code with exact phrases from the animation prompt. Enclose each annotation with **.

### SVG Setup
- In the viewport, the x position increases as you move from left to right, and y position increases as you move from top to bottom.
- In the SVG, the element listed first is rendered first, so the element listed later is rendered on top of the element listed earlier.

### Template
The output JavaScript code should follow the following template:
```javascript
// Select the SVG elements
<code></code>
// Create a timeline object
<code></code>
// Compute necessary variables. Comment each line of code with your reasoning
<code></code>
// Create the animation step by step. Comment each line of code with your reasoning
<code></code>
```

### API Documentation (only partially shown below)
/**
 * This function adds a tween to the timeline to translates an SVG element.
 * @param {object} timeline - The timeline object to add the translation tween to.
 * @param {object} element - The SVG element to be translated.
 * @param {number} duration - The duration of the tween in seconds
 * @param {number} toX - The amount of pixels to translate the element along the x-axis from its current position. This value is a relative offset from the element's current x value.
 * @param {number} toY - The amount of pixels to translate the element along the y-axis from its current position. This value is a relative offset from the element's current y value.
 * @param {number} startTime - The absolute time in the global timeline at which the tween should start.
 * @param {string} [easing='none'] - The easing function to use for the tween. The default value is linear easing. The easing functions are: "power2", "power4", "expo", and "sine". Each function should be appended with ".in", ".out", or ".inOut" to specify how the rate of change should change over time. ".in" means a slow start and speeds up later. ".out" means a fast start and slows down at the end. ".inOut" means both a slow start and a slow ending. For example, "power2.in" specifies a quadratic easing in. Another easing function is "slow(0.1, 0.4)", which slows down in the middle and speeds up at both the beginning and the end. The first number (0-1) is the porportion of the tween that is slowed down, and the second number (0-1) is the easing strength.
 * @returns {void} - This function does not return anything.
 * @example
 * // Translates the square element 25 pixels to the right and 25 pixels down over 1 second.
 * translate(tl, square, 1, 25, 25, 0);
 * 
 * // Translates the square element 25 pixels to the right and 25 pixels down over 1 second, starting at 2 seconds into the timeline. The easing function used is power1.out.
 * translate(tl, square, 1, 25, 25, 2, 'power1.out');
 */
function translate(timeline, element, duration, toX, toY, startTime, easing = 'power1.inOut') 

/**
 * This function adds a tween to the timeline to scale an SVG element.
 * @param {object} timeline - The timeline object to add the translation tween to.
 * @param {object} element - The SVG element to be scaled.
 * @param {number} duration - The duration of the tween in seconds
 * @param {number} scaleX - The scale factor to apply to the element along the x-axis. This value is absolute and not relative to the element's current scaleX factor.
 * @param {number} scaleY - The scale factor to apply to the element along the y-axis. This value is absolute and not relative to the element's current scaleY factor.
 * @param {number} startTime - Same as the startTime parameter for the translate function.
 * @param {string} [elementTransformOriginX='50%'] - The x-axis transform origin from which the transformation is applied. The origin is in the element's coordinate space and is relative to the top left corner of the element. The default value is 50%, which means 50% of the element's width from the left edge of the element (horizontal center of the element).
 * @param {string} [elementTransformOriginY='50%'] - The y-axis transform origin from which the transformation is applied. The origin is in the element's coordinate space and is relative to the top left corner of the element. The default value is 50%, which means 50% of the element's height from the top edge of the element (vertical center of the element).
 * @param {string} [absoluteTransformOriginX=null] - Similar to elementTransformOriginX, but the origin is in the absolute coordinate space of the SVG document and should be specified as a pixel value. Specify only elementTransformOriginX and elementTransformOriginY or absoluteTransformOriginX and absoluteTransformOriginY, but not both. When both are specified, elementTransformOriginX and elementTransformOriginY take precedence.
 * @param {string} [absoluteTransformOriginY=null] - Similar to elementTransformOriginY, but the origin is in the absolute coordinate space of the SVG document and should be specified as a pixel value. Specify only elementTransformOriginX and elementTransformOriginY or absoluteTransformOriginX and absoluteTransformOriginY, but not both. When both are specified, elementTransformOriginX and elementTransformOriginY take precedence.
 * @param {string} [easing='none'] - Same as the easing parameter for the translate function.
 * @returns {void} - This function does not return anything.
 * @example
 * // Scale the square element to double its size over 1 second from its center.
 * scale(tl, square, 1, 2, 2, 0);
 * 
 * // Scale the square element to double along x-axis and triple along y-axis over 2 second from its center, starting at 2 seconds into the timeline. The easing function used is power1.out.
 * scale(tl, square, 1, 2, 3, 2, '50%', '50%', null, null, 'power1.out');
 * 
 * // Scale the square element to be 4 times as large from its bottom right corner over 1 second.
 * scale(tl, square, 1, 4, 4, 0, '100%', '100%');
 * 
 * // Scale the square element to be 5 times as large along the x-axis and 2 times as large along the y-axis from (100px, 100px) in the SVG document over 2 second.
 * scale(tl, square, 2, 5, 2, 0, null, null, 100, 100);
 */
function scale(timeline, element, duration, scaleX, scaleY, startTime, elementTransformOriginX = '50%', elementTransformOriginY = '50%', absoluteTransformOriginX = null, absoluteTransformOriginY = null, easing = 'power1.inOut') 

/**
 * This function adds a tween to the timeline to rotate an SVG element.
 * @param {object} timeline - The timeline object to add the translation tween to.
 * @param {object} element - The SVG element to be rotated.
 * @param {number} duration - The duration of the tween in seconds
 * @param {number} angle - The rotation angle in degree. This value is absolute and not relative to the element's current rotation angle.
 * @param {number} startTime - Same as the startTime parameter for the translate function.
 * @param {string} [elementTransformOriginX='50%'] - Same as the elementTransformOriginX parameter for the scale function.
 * @param {string} [elementTransformOriginY='50%'] - Same as the elementTransformOriginY parameter for the scale function.
 * @param {string} [absoluteTransformOriginX=null] - Same as the absoluteTransformOriginX parameter for the scale function.
 * @param {string} [absoluteTransformOriginY=null] - Same as the absoluteTransformOriginY parameter for the scale function.
 * @param {string} [easing='none'] - Same as the easing parameter for the translate function.
 * @returns {void} - This function does not return anything.
 * @example
 * // Rotate the square element by 45 degrees (clockwise) from its center over 1 second.
 * rotate(tl, square, 1, 45, 0);
 * 
 * // Rotate the square element by -90 degrees (counterclockwise) from its bottom right corner over 1 second, starting at 1.5 seconds into the timeline. The easing function used is power1.out.
 * rotate(tl, square, 1, -90, 1.5, '100%', '100%', null, null, 'power1.out');
 * 
 * // Rotate the square element by 135 degrees (clockwise) from (300, 300) in the SVG document over 2 second.
 * rotate(tl, square, 2, 135, 0, null, null, 300, 300);
 */
function rotate(timeline, element, duration, angle, startTime, elementTransformOriginX = '50%', elementTransformOriginY = '50%', absoluteTransformOriginX = null, absoluteTransformOriginY = null, easing = 'power1.inOut') 
\end{lstlisting}


\subsection{LLM-based \dslname{} synthesizer}
This system prompt asks the LLM to generate a \dslname{} verification program 
from the input text prompt.
\begin{lstlisting}
### Instructions
- When a motion does not have any sequencing or timing constraints, simply use `exists` to check for the existence of the motion.
- Name the object variables as `o_1`, `o_2`, etc. and the motion variables as `m_1`, `m_2`, etc.
- List the object variables that are performing the action first, followed by the object variables that being used as reference objects.
- When using `exists`, no need to assign it to a variable, but do use a named variable in the lambda function. Do not just use `m` or `o`.
- The "type" predicate for motion only includes three values: "translate", "rotate", "scale".
- For integers, output them with one decimal point. For example, 100 should be output as 100.0.
- For floats, just output them as they are.
- For predicates about timing (t_before(), t_while(), t_after()), use the motion variables in sequential order. For example, if m_1 should happen before m_2, use `t_before(m_1, m_2)` and not `t_after(m_2, m_1)`.
- Always enclose the output with ```. Do not output any other text.
- Order the object predicates as follows: color, shape.
- Order the motion predicates as follows: type, direction, magnitude, origin, post, duration, agent. This ordering has to be strictly followed.

### Examples
Input:
"Translate the blue circle upwards by 100 px. Then rotate it by 90 degrees clockwise around its bottom right corner."

Output:
```
o_1 = iota(Object, lambda o: color(o, "blue") and shape(o, "circle"))
m_1 = iota(Motion, lambda m: type(m, "translate") and direction(m, [0.0, 1.0]) and magnitude(m, 100.0) and agent(m, o_1))
m_2 = iota(Motion, lambda m: type(m, "rotate") and direction(m, "clockwise") and magnitude(m, 90.0) and origin(m, ["100%", "100%"]) and agent(m, o_1))
t_before(m_1, m_2)
```

Input:
"Scale the black square up by 2 around its center for 0.25 seconds."

Output:
```
o_1 = iota(Object, lambda o: shape(o, "circle") and color(o, "blue"))
exists(Motion, lambda m_1: type(m_1, "scale") and direction(m_1, [1.0, 1.0]) and magnitude(m_1, [2.0, 2.0]) and origin(m_1, ["50%", "50%"]) and duration(m_1, 0.25) and agent(m_1, o_1))
```

Input:
"The yellow circle is scaled up horizontally by 2.5 about its center over a period of 10 seconds."

Output:
```
o_1 = iota(Object, lambda o: color(o, "yellow") and shape(o, "circle"))
exists(Motion, lambda m_1: type(m_1, "scale") and direction(m_1, [1.0, 0.0]) and magnitude(m_1, [2.5, 0.0]) and origin(m_1, ["50%", "50%"]) and duration(m_1, 10.0) and agent(m_1, o_1))
```

Input:
"Over a period of 0.5 seconds, the blue circle moves to be on the right of the black square."

Output:
```
o_1 = iota(Object, lambda o: color(o, "blue") and shape(o, "circle"))
o_2 = iota(Object, lambda o: color(o, "black") and shape(o, "square"))

exists(Motion, lambda m_1: type(m_1, "translate") and post(m_1, s_right(o_1, o_2)) and duration(m_1, 0.5) and agent(m_1, o_1))
```

### Template
```
<All object variables>
<All motion variables>
<All sequencing predicates>
```

### DSL Documentation (only partially shown below)
direction(motion_var, target_direction)
"""
Determines the frames in which a specified motion variable moves in a target direction for all objects in the scene.
Args:
    motion_var (Variable): The motion variable to check.
    target_direction (str or list): The target direction to check for. Can be a string for rotation ("clockwise" or "counterclockwise") or a 2D vector for translation and scaling directions. For scaling, use 1.0 for increase and -1.0 for decrease, and 0.0 if the direction along a certain axis is not specified.
Returns:
    TensorValue: A tensor containing boolean values indicating whether the motion variable moves in the target direction for each object over the frames.
Examples:
    translate upward: direction(m_1, [0.0, 1.0])
    translate to the left: direction(m_1, [-1.0, 0.0])
    rotate clockwise: direction(m_1, "clockwise")
    scale down along the x-axis: direction(m_1, [-1.0, 0.0]) ## Do not use values other than -1.0, 0.0, 1.0 for scaling directions
    scale up (uniformly): direction(m_1, [1.0, 1.0]) ## Do not use values other than -1.0, 0.0, 1.0 for scaling directions
    NOTE: Pay attention that, for scaling, the direction is a 2D vector, where the first element is the x-axis direction and the second element is the y-axis direction.
    If the direction along only one axis is specified, the other axis should be 0.0.
"""


magnitude(motion_var, target_magnitude)
"""
Analyzes the magnitude of a specified motion variable over a series of animation frames and determines 
if it matches the target magnitude within a specified tolerance.
Args:
    motion_var (Variable): The motion variable to analyze.
    target_magnitude (float or list of floats): The target magnitude to compare against. 
        If the motion type is "S" (scale), this should be a list of two floats representing 
        the target scale factors for x and y axes. For scale, if a direction along a certain axis is
        not specified, the target magnitude should be 0.
Returns:
    TensorValue: A tensor containing boolean values indicating whether the motion with the specified 
    magnitude occurs for each object over the animation frames.
"""


origin(motion_var, target_origin)
"""
Determines the frames during which objects in the scene have a specific origin.
Args:
    motion_var (Variable): The motion variable associated with the scene.
    target_origin (tuple): The target origin coordinates to check for each object. % sign is used to indicate relative origin.
Returns:
    TensorValue: A tensor containing boolean values indicating whether each object 
                 has the target origin at each frame.

Examples:
    Note: always output numbers with one decimal points, even if they are integers.
    Rotate around the point (400, 200): origin(m_1, [400.0, 200.0])
    Scale around the center: origin(m_1, ["50%", "50%"])
    Scale around the top left corner: origin(m_1, ["0%", "0%"])
"""


post(motion_var, spatial_concept)
"""
Processes the motion data and checks if, at the end of the duration of "motion_var", 
the spatial relationship between two objects expressed in "spatial_concept" has been satisified.
Args:
    motion_var (Variable): The motion variable to check.
    spatial_concept (TensorValue): A tensor containing boolean values indicating whether two objects maintained a certain spatial relationship at each frame of the animation.
Returns:
    TensorValue: A tensor containing boolean values indicating whether spatial_concept has been satisifed at the end of motion_var.
"""


duration(motion_var, target_duration)
"""
Determines the frames during which a motion of a specified duration occurs for each object in the scene.
Args:
    motion_var (Variable): The motion variable to check.
    target_duration (float): The target duration to match for the motion.
Returns:
    TensorValue: A tensor containing boolean values indicating the frames during which the motion occurs for each object.
"""


agent(motion_var, obj_var)
"""
Determines if the motion is performed by the agents specified in the object variable.
Args:
    motion_var (Variable): The motion variable containing the name of the motion.
    obj_var (Variable): The object variable representing the object(s) involved.
Returns:
    TensorValue: A tensor value indicating the presence of motion for each object over the time steps.
"""
\end{lstlisting}


\subsection{Correction iteration}
This system prompt is fed to the LLM animation synthesizer with each automated correction iteration as context for interpreting the \dslname{} verification report.
\begin{lstlisting}
### DSL Documentation
<same content as the DSL documentation shown in the system prompt of the LLM-based MoVer synthesizer>

### MoVer Program
<the MoVer program for the current text prompt>

### MoVer Verification Report
<the verification report for the current animation>

### Instruction
Please correct errors in the animation program as indicated by the report
\end{lstlisting}
