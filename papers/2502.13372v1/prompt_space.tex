\section{Motion Prompt Taxonomy}
% parametrize
% evaluate the space that our approach will work over
% being able to directly use a prompt templates and generate prompt

\maneesh{Need to motivate the need for a prompt space taxonomy in a sentence or two here.}

We characterize the space of motion prompts by adopting Talmy's
theoretical framework that a motion event consists of fundamental
semantic elements like Figure, Ground, and
Path~\cite{talmy1975motion}. \maneesh{This needs to be expanded. I
  don't buy that these things are ``fundamental''. If we are going to
  claim that we need to provide more of an argument about why they are
  fundamental. Also need to define what these terms Figure, Ground,
  Path mean. The definitions below seem long. I think we need to figure out
  how to condense the definitions and really focus on how we produce the prompt
  templates and the algorithmic generation of prompts according to the templates. Also the low level
parameterization doesn't seem to appear here.}


\subsection{Figure}
Figure refers to the moving entity in a motion.
The entity can be an \textit{individual} object, such as ``the black square'', or a \textit{group} of objects, like ``all circles.''

When describing motions, we can refer to individual objects
(e.g. ``The black square moves to the right, and the blue square moves
to the left''), or we can specify them collectively (e.g. ``The
squares move closer to each other'')

\subsection{Ground (Reference Object)}
\jiaju{Frame of reference is NOT ground}
% Frame of reference
Ground refers to the frame of reference.  In linguistics, there are
three main types of frame of reference, which all have their
equivalence in computer graphics terms: \textit{intrinsic} (object
coordinate system), \textit{relative} (camera coordinate system), and
\textit{absolute} (world coordinate system). The relative type is a
ternary system that describes relationships in terms of the position
of an object, the viewpoint of the perceiver, and the position of
another object.  When viewing a motion graphics animation, we assume
that the viewer has a fixed point of view, so we do not consider the
relative type of frame of reference in our taxonomy.

\paragraph{World Coordinate System}
% world system
The world coordinate system (absolute) describes the relationship
between an object and a fixed frame of reference, which in our case is
the SVG coordinate system.  For example, ``translate the black square
to the right'' means to translate the black square object towards the
right side of the SVG scene.

\paragraph{Object Coordinate system}
% relative 
The object coordinate system (intrinsic) describes the relationship between an object and a referential object in the latter object's coordinate system.
For example, in the prompt ``move the black square to the top of the red circle,'' the top of the red circle might not align with the top of the SVG coordinate system (positive y-direction) as the circle could have been rotated.

\subsection{Path}
% how do you describe the path an object should take?
% different ways to describe the same motion
\begin{figure}
    \centering
    \includegraphics[width=\linewidth]{figs/fig_path.pdf}
    \caption{Example scene to demonstrate various ways of specifying a path}
    \label{fig:path}
\end{figure}

Path refers to the course of a motion taken by the Figure.
Based on the seven types of motion attributes that we identified above, we can fully specify a path in the following ways:
\begin{itemize}
    \item Type + Direction + Magnitude + Origin
    
    ``The red circle rotates around the canvas center by 90 degrees clockwise.''
    
    \item Type + Post State + Origin
    
    ``The red circle rotates to coincide with the blue circle around the canvas center.''
\end{itemize}



% It can be categorized into two types based on telicity, which means whether a motion has a specific endpoint or not. There are two types of telicity: \textit{telic} and \textit{atelic}

% \paragraph{Telic}
% A telic motion has a specified endpoint for the motion to be considered completed. For example, ``scale the square up by 2'' is a telic motion as the scaling is considered finished when the square has doubled its size.
% In our space, a telic motion can be specified by describing the magnitude of a motion (e.g. ``rotate the triangle by 90 degrees'') or the post state of motion (e.g. ``shift the circle to the center of the SVG'').

% \paragraph{Atelic}
% An atelic motion does not have a specific endpoint. For example, ``rotate the triangle around the blue circle'' and ``translate the triangle upward'' are atelic because they can continue indefinitely.
% In our space, we can specify atelic motions by providing information on direction and origin, as shown in the examples above.

% source
% goal
% change of direction

% english is motion + manner
% path and ground through prepositional phrases

% path + ground
% he drove to home, to home is path (to home) + ground (home)

% serilizations
% compound verb
% satellite being packed into compound verb

% ground - frames
% intrinsic
% relative
% absolute

% path
% telicity
% granularity. Slobin (2004a)

% scalar change
% non-scalar change


% \subsection{Manner}
% \jiaju{TODO: need to flesh out the categories of Manner}

% Manner can have many subcategories. In our case, the relevant dimensions are: mode (type of motion) and timing (intrinsic speed of a motion and sequencing between motions).

% \paragraph{Mode}

% \paragraph{Timing}
% intrinsic timing
% sequencing

% speed

% compound action
% force exertion

% LocalWords:  Talmy's parameterization perceiver SVG
