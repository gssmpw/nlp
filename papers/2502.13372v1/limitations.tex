\section{Limitations and Future Work}
\label{sec:limitations}
%Our work allows verification of motion graphics animations and
%enables an automatic pipeline for iteratively syntheszing and
%verifying then correcting 

While our method enables verification of motion graphics animations
and thereby supports LLM-based automated synthesis with iterative
correction, it has some limitations.
%
First, because \dslname{} represents motions in terms of low-level
motion attributes, it requires a low-level description of the motion
in the prompt. Prompts that describe complex motion paths (e.g.,
``Move the triangle on a figure eight path'', ``Make the rectangle
dance'') are challenging for the LLM to convert the prompt into the
low-level \dslname{} predicates.  Such prompts are also ambiguous as
it is unclear what set of low-level predicates would make sense to
represent a ``dance'' motion.
%
A direction for future work is to design higher-level predicates to
accommodate such complex motion descriptions.
%
Second, while natural language prompting is an accessible interface for
creating visual content, precisely describing spatio-temporal
trajectories in language can be difficult. Converting other modalities
of user controls (e.g. sketches, gestures, etc.) into motion graphics
animations and corresponding \dslname{} programs could further
facilitate animation creation. The challenge is to map such user
inputs into appropriate spatio-temporal motion constraints.
%
Third, the \dslname{} verification report is passive; it explains
what went wrong, but does not suggest fixes for the
problems. A suggestive interface\,\cite{igarashi2001} giving possible solutions
as part of the report could
further aid users in making their desired animations.
%
Finally, while we have developed verification for animation, it may be
possible to build verification languages for other visual content
domains, such as CAD, video. and image generation to
enable automated iterative optimization with LLMs.




% LocalWords:  LLM spatio
