\begin{table}[t]
% \small
% \footnotesize
% \scriptsize
\centering
\resizebox{\linewidth}{!}{%
\begin{tabular}{lrrrrr}
\toprule
\multirow{2}{*}{\begin{tabular}[c]{@{}c@{}}LLM-based\\generation\\ pipeline\end{tabular}} & \multicolumn{5}{c}{Prompt Type}                     \\ \cline{2-6}
                                 & \begin{tabular}[c]{@{}c@{}}Single\\atomic\end{tabular} & \begin{tabular}[c]{@{}c@{}}Spatially\\relative\end{tabular} & \begin{tabular}[c]{@{}c@{}}Temporally\\relative\end{tabular} & \begin{tabular}[c]{@{}c@{}}Spatio-\\temporally\end{tabular}  & \multicolumn{1}{|r}{Overall} \\
\hline
pass@0                 & 1060 (88.3\%) & 530 (37.9\%)   & 1062 (88.5\%)   & 640 (35.6\%) & \multicolumn{1}{|r}{3292 (58.8\%)}    \\
pass@1+                & 136 (11.3\%) & 718 (51.3\%)    & 134 (11.2\%)    & 962 (53.4\%) & \multicolumn{1}{|r}{1950 (34.8\%)}    \\
fail              & 4 (0.3\%)    & 152 (10.9\%)    & 4 (0.3\%)       & 198 (11.0\%) & \multicolumn{1}{|r}{358 (6.4\%)}    \\
\cline{1-6}
avg. iters.             &  1.1 (1-5)   & 6.2 (1-34)       & 1.8 (1-12)      & 6.7 (1-38) & \multicolumn{1}{|r}{5.8 (1-38)}    \\
\bottomrule
\end{tabular}%
} % end resize
\caption{Performance of our iterative LLM-based %synthesis and  verification
  pipeline with ground-truth \dslname{} programs. For our
  test dataset of 5600 prompts, we compute the number that require 0 correction
  iterations (pass@0), the number that require 1 to 49 iterations
  (pass@1+), and the number that fail after 49 iterations. For the
  pass@1+ prompts we also report the average number of iterations
  required as well as the min-max range of the iteration count.
}
%  
%%   For each test prompt, we iteratively feed the verification report back to the LLM animation synthesizer until all the verification tests pass or it times out at 50 correction iterations.
%% pass@0 means that the LLM synthesizes an animation that passes all checks without any corrections.
%% We also report the range of the number of iterations in addition to the average.
%% %
%%   Performance of the LLM-based generation pipeline.
%% % \maneesh{put stoppping limit in caption of table.}
%% For each test prompt, we iteratively feed the verification report back to the LLM animation synthesizer until all the verification tests pass or it times out at 50 correction iterations.
%% pass@0 means that the LLM synthesizes an animation that passes all checks without any corrections.
%% We also report the range of the number of iterations in addition to the average.
%% % \jiaju{update caption}
%% We notice that the LLM animation synthesizer needs prominently more correction iterations for prompts in the spatially relative and spatio-temporally relative categories.
% }
% \vspace{-3em} %% commented out for arxiv
\label{tab:eval_pipeline}
\end{table}

% LocalWords:  lccccc LLM Spatio iters


% \begin{tabular}{lrr|rr|rr|rr|rr}
% \toprule
% \multirow{2}{*}{\begin{tabular}[c]{@{}c@{}}LLM-based\\generation\\ pipeline\end{tabular}} & \multicolumn{10}{c}{Prompt Type}                     \\ \cline{2-11}
%                                  & \multicolumn{2}{c}{\begin{tabular}[c]{@{}c@{}}Single\\atomic\end{tabular}} & \multicolumn{2}{c}{\begin{tabular}[c]{@{}c@{}}Spatially\\relative\end{tabular}} & \multicolumn{2}{c}{\begin{tabular}[c]{@{}c@{}}Temporally\\relative\end{tabular}} & \multicolumn{2}{c}{\begin{tabular}[c]{@{}c@{}}Spatio-\\temporally\end{tabular}}  & \multicolumn{2}{|c}{Overall} \\
% \hline
% pass@0             & 1060 & 88.3\% & 530 & 37.9\%   & 1062 & 88.5\%   & 640 & 35.6\% & \multicolumn{1}{|r}{3292} & \multicolumn{1}{c}{58.8\%}    \\
% pass@1+            & 136 & 11.3\% & 718 & 51.3\%    & 134 & 11.2\%    & 962 & 53.4\% & \multicolumn{1}{|r}{1950} & \multicolumn{1}{c}{34.8\%}    \\
% time-outs          & 4 & 0.3\%    & 152 & 10.9\%    & 4 & 0.3\%       & 198 & 11.0\% & \multicolumn{1}{|r}{358}  & \multicolumn{1}{c}{6.4\%}    \\
% \cline{1-11}
% avg. iter.         &  2.1 & 2-6   & 7.2 & 2-35       & 2.8 & 2-13      & 7.7 & 2-39  & \multicolumn{1}{|r}{6.8}  & \multicolumn{1}{c}{2-39}    \\
% \bottomrule
% \end{tabular}%
