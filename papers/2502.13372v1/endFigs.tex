%% \begin{figure*}[t]
%%     \centering
%%     \includegraphics[width=\linewidth]{figs/fig_discussion.pdf}
%%     \vspace{-2em}
%%     \caption{\dslname{} can help user's identify ambiguous
%%       prompts. Our pipeline correctly generates a \dslname{} program
%%       reflecting the user's original prompt and it generates a motion that successfully
%%       passes verification. However the user's intention was for the
%%       blue square to make an arcing motion rather
%%       than rotating about its center and simultaneously
%%       translating. Understanding why the animation passed verification
%%       but did not meet his intentions, the user revised the prompt to
%%       ask for a rotation only motion and produced the desired result.
%% %      Note that the \dslname{} program for the revised prompt includes
%%       %      a check that the motion is a rotation only.
%%     }
%%     \label{fig:discussion}
%% \end{figure*}
\begin{figure*}[t]
    \centering
    \includegraphics[width=\linewidth]{figs/fig_example_haha.pdf}
    \vspace{-2em}
    \caption{
        The input text prompt describes an animation in which the letter H and the letter A should maintain contact as they scale up and down, as shown in the frames under Correction Iteration 9 (please refer to the supplemental materials for the animations in action).
        However, the initial animation produced by the LLM animation synthesizer fails to satisfy any of the \texttt{border()} relations as indicated by the verification report.
        In the first correction iteration, the letter H and A are stretched to border the bottom edges of the letters belows them, instead of top edges (frame 60 and 120).
        From there, the synthesizer becomes stuck in a loop, producing the same animation until the eighth correction iteration, where it has corrected the scaling of the yellow H and blue A.
        In the ninth iteration, it eventually changes the sign of the scale factor of the two bottom letters, passing all verification checks.
    }
    \label{fig:haha}
\end{figure*}
\begin{figure*}[t]
    \centering
    \includegraphics[width=\linewidth]{figs/fig_example_wowmom.pdf}
    \vspace{-2em}
    \caption{
        The input text prompt asks the two letter W's to be rotated around the small pink circle and the letter O to be lowered so that the transformed letters form the word ``mom.''
        In the initial animation, the W's are rotated correctly, but the letter O does not move at all, rendering both \texttt{dir()} and \texttt{post()} to false.
        Note that \texttt{bottom\_border()} is true because the bottom edge of O aligns with those of the W's at the beginning of the animation (frame 1).
        In the first correction iteration, the O moves towards the bottom but does not do so enough to align with the W's.
        In the second iteration, the LLM synthesizer correctly computes the translation distance for O, producing an animation that says ``wow mom'' at the end (frame 180).
    }
    \label{fig:wowmom}
\end{figure*}
