\section{\dslname{}: A Motion Verification DSL}
\label{sec:moverDSL}

\dslname{} is a domain specific language for verifying that certain
spatio-temporal concepts appear in a motion graphics animation.
%Figure~\ref{fig:teaser} and \ref{fig:verification} each show an
%example animation and a \dslname{} verification program that checks
%for various concepts we would like the animation to include.
%
%A \dslname{} verification program can be manually coded, or it can be
%generated from a text prompt using LLM-based program synthesis as we
%will describe in Section~\ref{sec:llmsynth}.
We first consider the design of the \dslname{} language (Section~\ref{sec:design}) and then describe
how we execute a \dslname{} verification program (Section~\ref{sec:execution}).
%on a motion graphics
%animation.

\subsection{\dslname{} Language Design}
\label{sec:design}

We designed \dslname{} based on prior work from cognitive psychology
that has analyzed how people think about and describe spatio-temporal properties and relationships in animated
scenes\,\cite{talmy1983language,talmy1975motion,tversky1998space}.
%
%This work has shown that linguistic descriptions of animation focus on
%describing the motion of each object. People break complex
%trajectories into simpler motions, and individually describe the
%translation, rotation and scaling of an object. {\maneesh Can we cite something for prev. sentence}.
%
%
%This work has shown that linguistic descriptions of animation
%schematize the information to emphasize only the most salient
%aspects of object motions.
%
This work has shown that linguistic descriptions of animation
schematize the information to emphasize only the most salient aspects
of the motions of each object.
%
People break complex trajectories into basic motions; they
individually describe the translation, rotation, and scaling of an
object even when all three occur simultaneously. 
%
People also commonly use relative frames of reference rather than absolute frames,
e.g. saying ``Move the circle above the square after rotating the
triangle.'' instead of ``Move the circle to (100,100) starting at second 5.0''.
%


In \dslname{}, spatio-temporal concepts are expressed as
first-order logic predicates (boolean functions)
over variables representing objects {\tt o$_i$} $\in \mathbf{O}$ and
their motions {\tt m$_j$} $\in \mathbf{M}$
(Table~\ref{tab:predicates}).
%
\dslname{} treats seven attributes as 
{\em atomic} descriptions of a motion {\tt m$_j$}:
%At the lowest-level, \dslname{} identifies seven motion attribute predicates as {\em atomic} descriptions of a motion {\tt m$_i$}.
%We briefly discuss each below (more details in Table~\ref{tab:predicates}).
\begin{enumerate}[leftmargin=0.5cm]
    \item {\tt \bf agt } The agent object {\tt o$_i$} performing motion {\tt m$_j$}.
    \item {\tt \bf type}  The motion type of {\tt m$_j$}: ``translation,'' ``rotation,'' or ``scale''. 
    \item {\tt \bf dir } The direction of motion {\tt m$_j$}.
    \item {\tt \bf mag } The magnitude or distance of motion {\tt m$_j$}.
    \item {\tt \bf orig} The origin of motion {\tt m$_j$} when {\tt type} is rotation or scale.
    \item {\tt \bf dur } The duration of motion {\tt m$_j$}.
    \item {\tt \bf post} A postcondition the end state of motion {\tt m$_j$} must satisfy.
\end{enumerate}
\dslname{} includes predicates that can check the values for each of these
attributes. For example, the predicate {\tt dir(m$_j$, [1,0])} returns
true for each frame where motion {\tt m$_j$} is along the direction vector
$[1,0]$ (i.e., rightwards).
%
We can combine individual predicates into a complete logic statement
to check whether an animation contains a motion {\tt m} with a set of \textit{atomic} properties as
\begin{multline}
%{\tt m}_j = \iota
%  \texttt{m.type(m,"trn")}\land\texttt{dir(m,[0,1])}\land\texttt{mag(m,100)}\\ \texttt{agt(m,$\iota$o.clr(o,"black")}\land\texttt{shp(o,"square"))}.
\exists \texttt{m.type(m,"trn")}\land\texttt{dir(m,[0,1])}\land\texttt{mag(m,100)}\\ \texttt{agt(m,$\iota$o.clr(o,"black")}\land\texttt{shp(o,"square"))}.
\end{multline}
This statement returns true for frames containing a motion where a
square-shaped object is translating upwards by 100 px.
%and returns the
%motion as variable {\tt m$_j$}.

%This statement identifies frames of the animation where 
%where a square-shaped object is translating upwards by 100 px and returns the motion as variable {\tt m$_j$}.
% (e.g move the circle to (100,100) over 2 seconds).

\dslname{} also includes predicates that can check relative
spatio-temporal relationships between motions.
%
These predicates are based on Allen's~\shortcite{allen1983interval}
temporal interval algebra and it's 2D analog, the rectangle
algebra~\cite{balbiani1998model,navarrete2013spatial}.
%
These algebras provide a formal representation for 13 temporally
relative relationships and 169 spatially relative relationships (Figure~\ref{fig:allen_rectangle}) and
\dslname{} includes predicates that can check for all of them.
%
In addition, \dslname{} aggregates certain relative predicates together with
disjunction to form higher-level relative predicates to better accommodate for natural
language's inherent ambiguity.
%
For example, the higher-level aggregated temporal predicate
\texttt{while(m$_i$, m$_j$)} (Figure~\ref{fig:allen_rectangle} top)
returns true
for each frame where both {\tt m$_i$} and {\tt m$_j$}  occur at the same time and aggregates
nine lower-level Allen algebra predicates.
%to have some overlap in time, instead
%of additionally restricting their relative start and end times like
%each individual Allen predicates.
%
%Similarly, the aggregated spatial predicate {\tt top(o$_i$,io$_i$)} returns true as long as the top side of object {\tt o$_i$} is above the top side of {\tt o$_j$}, accounting for various sizes that each object might have.
%
%For example, the temporal predicate \texttt{while(m$_i$, m$_j$)}
%returns true for each frame where both $m_i$ and $m_j$ occue
%simultaneously.
Similarly the aggregated spatial predicate {\tt top(o$_i$,o$_j$)}
returns true as long as the top edge of object {\tt o$_i$} is above
the top edge of {\tt o$_j$} and aggregates 52 lower-level rectangle
algebra predicates.


%% spatio temporal predicates
%In addition to using absolute frames of reference and precise quantitative
%metrics like the example above,
%People also commonly use relative frames and imprecise metrics (e.g.,
%``Move the circle above the square after rotating the triangle.'').
%rather than absolute frames and precise metrics (e.g., ``Move the
%$circle to (100,100) starting at second 5.0'').
%
%We build on Allen's~\shortcite{allen1983interval} temporal interval
%algebra and it's 2D analog, the rectangle
%algebra~\cite{balbiani1998model,navarrete2013spatial}
%(Figure~\ref{fig:allen_rectangle}), which provide formal representations for
%such relative and imprecise spatio-temporal relationships.
%
%% \dslname{} provides predicates that can check each of the 13 temporal
%% relationships in Interval Algebra and 169 spatial relationships in the
%% Rectangle Algebra. \jiaju{consider using IA and RA instead of spelling it out.}
%% %
%% % \jiaju{need to be clear here that we have all the low-level predicates as well as high-level ones. The examples below are the high level ones. the high-level mapping allows \dslname{} to accommodate for ambiguities in natural language.}
%% Further, \dslname{} aggregates certain predicates together with
%% disjunction to form higher-level predicates to accommodate for natural
%% language's inherent ambiguity.  For example, the aggregated temporal
%% predicate \texttt{while(m$_i$, m$_j$)}
%% (Table~\ref{fig:allen_rectangle}a) loosely constraints that the motion
%% {\tt m$_i$} and {\tt m$_j$} should have some overlap in time, instead
%% of additionally restricting their relative start and end times like
%% each individual Allen predicates.
%% %
%% Similarly, the aggregated spatial predicate {\tt top(o$_i$,io$_i$)} returns true as long as the top side of object {\tt o$_i$} is above the top side of {\tt o$_j$}, accounting for various sizes that each object might have.
%% % %



%% past edits

%To express various \textit{temporal} relations between atomic motions
%and different \textit{spatial} relations between their objects, we
%build on Allen's~\shortcite{allen1983interval} temporal algebra, which
%provides a formal representation for working with relative temporal
%intervals (Figure~\ref{fig:allen}), and Balbiani et al.'s~\shortcite{}
%rectangle algebra, which extends Allen's algebra to represent relative
%spatial relationships in the 2D space.
%
%Figure~\ref{fig:allen} and \ref{fig:rectangle} show the various
%spatio-temporal relationships these algebras can represent, and our
%DSL provides predicates that can check for each of them.  \jiaju{need
%  to be clear here that we have all the low-level predicates as well
%  as high-level ones. The examples below are the high level ones.}
% example temporal
%For example, \texttt{while(m$_i$, m$_j$)} checks for whether the two
%motions have any amount of overlap across time.
% example spatial
%The spatial predicate {\tt above($o_i$,$o_j$)} returns true for each
%frame of the animation where object $o_i$ is above $o_j$.

%The predicate {\tt above($o_i$,$o_j$)} is a higher-level spatial
%concept that returns true for each frame of the animation where object
%$o_j$ is above $o_i$.

\begin{figure}[t]
  \centering
  \includegraphics[width=\linewidth]{figs/fig_allen_rectangle.pdf}
  \vspace{-2em}
  \caption{
      (top) The 13 low-level temporally relative predicates in Allen's interval algebra and the 3 aggregated predicates (disjunctions of the low level predicates) in \dslname: \texttt{before()}, \texttt{while()}, and \texttt{after()}. 
      (bottom) The 169 spatially relative predicates in the rectangle algebra and 4 of the aggregated predicates in \dslname{}. 
%    Other aggregated spatial relationships, such as \texttt{border()}, all correspond to various sub-regions of this table.
      \dslname{} implements 10 spatially relative aggregated predicates: \texttt{top()}, \texttt{bottom()}, \texttt{left()}, \texttt{right()}, \texttt{border()}, \texttt{intersect()}, \texttt{top\_border()}, \texttt{bottom\_border()}, \texttt{left\_border()}, and \texttt{right\_border()} as sub-regions of this table.
  }
    \vspace{-2em}
  \label{fig:allen_rectangle}
\end{figure}




\subsection{\dslname{} Execution Engine}
\label{sec:execution}
%Our Motion Verifier is responsible for checking whether an SVG motion
%program is consistent with a motion verification program written in our DSL.
%
%It outputs the set of frames that evaluate to false with respect to
%each logical expression in the verification program.
%
%Our Motion Verifier is responsible for executing a motion verification
%program on an SVG motion program.  It outputs the set of frames that
%evaluate to false with respect to each logical expression in the
%verification program.

We have implemented an execution engine for \dslname{} that can check
whether a motion graphics animation is consistent with a \dslname{}
program.
%
It computes the set of frames that evaluate to false with respect to
each logical statement and predicate in the verification program and
outputs a verification report for each such expression
(an example report is Figure~\ref{fig:teaser} bottom left).

%Executing a \dslname{} verification program yields a report specifying
%which logic statements and predicates the animation successfully
%passes and which it fails.
%(bottom left
%Figure~\ref{fig:teaser} and ~\ref{fig:verification}).

%While a motion graphics animation might be written using a variety of
%high-level animation APIs (e.g. GSAP\,\cite{gsap}, \maneesh{add others
%  to cite},\maneesh{Could we demonstrate a physics based API?}),
The \dslname{} execution engine operates on a generic, frame-level SVG
representation of motion graphics animations.
%
Specifically, each foreground object is represented as an SVG {\tt
  <shape>} , {\tt <g>}, or {\tt <image>} node with motion transforms
specified per-frame.
%
This SVG motion program representation is similar to that used by Zhang et
al.\,\shortcite{zhang2023motion}, and many high-level animation APIs,
including CSS~\cite{cssAnimation}, anime.js~\cite{animeJs}, and GSAP~\cite{gsap},
support rendering an animation into this type of
per-frame SVG representation.
% \maneesh{add missing cites.}

Given such an SVG motion program, our execution engine first builds a
2D {\em animation matrix} data structure where each row represents an object and each
column represents a frame of the animation.  In each cell $(i,j)$ of
the animation matrix we store all atomic motion attributes (Section~\ref{sec:design}),
except {\tt post}, for object {\tt o$_i$} at frame $j$.
%\maneesh{Table needs to better describe the
%  7 atomic motion attributes, or we need to pull those out into text.}
%\jiaju{01-18: let me know what you think about reading the updated
%  3.1}
%
%% Specifically we store information about the motion type (translation,
%% rotation or scale), direction (``clockwise'' or ``counterclockwise''
%% for rotations, a normalized tangent vector for translations, and a 2D
%% vector of scalefactors (in $x$ and $y$) for scale), magnitude, and
%% center (of rotation or of scale).
%
We compute these motion attributes by first factoring the per-frame
transformation matrix of each object into a translation, rotation, and
scale~\cite{thomas1991} and then depending on the type of motion,
computing a direction, magnitude, and origin.
%
We separately store static information for each object such as its color,
shape, and bounding box in object local coordinates.


With these data structures in place, we implement our DSL as follows.
%
%{\tt exists(var, expr)}: we recursively execute {\tt expr} 
%
We execute each logical predicate in the verification program on each
cell of the animation matrix. We mark the
cell as true if the
predicate is satisfied and false otherwise. Thus each predicate
produces a boolean matrix with the same shape as the animation matrix.
%
For example, executing the predicate {\tt type(m,"rot")} sets
matrix cell $(i,j)$ to true if object {\tt o$_i$} has a non-zero rotation
angle in frame $j$. Similarly, executing the predicate {\tt shp(o$_i$,"square")}
sets the entire matrix row $(i,*)$ to true if
object {\tt o$_i$} is square-shaped. 
%
A predicate expression (i.e., {\tt expr}) combines the results of
multiple predicates by applying logical boolean operators (e.g., $\land$,
$\lor$, $\lnot$) element-wise on the predicate matrices.
%
We can check logical quantifiers (e.g., $\exists$, $\forall$) by aggregating
the the boolean matrices.
%
Finally, the $\iota$ operator returns an object $o_i$ or motion $m_j$
for which a logical expression {\tt expr} holds. We compute the
expression on the animation matrix, identify a row for which the
expression holds for at least one frame, and return the row index (if an object
is requested) or the row itself (if a motion is requested).
%Note that if a resulting motion contains discontiguos sequences of
%frames we further filter the row to only include true values for the
%first contiguous sequence.


%
%and we can apply boolean operators (e.g. and, or, not) element-wise to
%each entry in the matrix to combine the results of different
%predicates.

While many predicates primarily involve looking up information in the
animation and object data structures, temporal and spatial predicates
based on the interval and rectangle algebras are more complex.
%predicates that compare the
%timing interval of two motions or the bounding boxes of two objects (from the
%Allen or rectangle algebras) are more complex.
These require comparisons between the starting and ending frames of
the motions or between the relative positions and overlaps of the
bounding boxes of the objects (Figure~\ref{fig:teaser}).
%
An interval tree representation~\cite{cormen2022introduction} of the
motions (2D intervals per-frame for spatial predicates, frame
discretized 1D intervals for temporal predicates) enables efficient
execution of these predicates.







% LocalWords:  DSL str dir dur preceds motA motB fastcoref Intra NUM
% LocalWords:  spaCy subtree SetFit finetuned finetune TODO expr pre
% LocalWords:  VidProm intra lemmatized roberta advcl relcl subtrees
% LocalWords:  prev awk LLM SVG Verifier Parsers keyframe GSAP Zhang
% LocalWords:  verifier et al spatio Balbiani allen APIs zhang agt px
% LocalWords:  Maneesh's Jiaju's trn clr shp disjunction js
% LocalWords:  discretized postcondition disjunctions
