\section{Semantic Parser Implementation}
\label{sec:semanticParser}

%% Semantic parsing describes a set of techniques for converting natural
%% language text into logic-based program code that can be executed on an
%% underlying database~\cite{kamath2018survey}.
%% %
%% We have implemented a rule-based semantic parser that can convert a
%% text prompt describing a desired animation into a \dslname{}
%% verification program.
%% %
%% Our approach is to use a dependency
%% parser\,\cite{liu2019roberta,honnibal2020spacy} to build a parse
%% tree from the prompt and then apply templated rules in combination with a transformer-based classifier fine-tuned with SetFit\,\cite{tunstall2022setfit} to map the
%% branches of the tree to predicates and logical expressions in
%% \dslname{}.
%% %
%% This parser serves as a baseline alternative to LLM-based
%% approach for generating \dslname{} verification programs
%% (Section~\ref{sec:llmsynth}).
%% Our parser involves five main steps (Figure~\ref{fig:parser}).

We have implemented a semantic parser that can convert a text prompt
describing a desired animation into a \dslname{} motion verification
program. It uses a rule-based approach that involves five main steps
(Figure~\ref{fig:parser}).


%
%\begin{enumerate}
%\item {\bf \em Co-reference resolution.} We resolve co-references to objects and to motions
%  (e.g. we identify that ``it'' refers to ``the black square'').
%  
%\item {\bf \em Split prompt into atomic motion clauses.} We break the
%  prompt into clauses, each representing a single atomic motion.
%  (e.g. we split ``it rotates ..., while expanding ...'' into the
%  clauses ``it rotates ...'' and ``while [it] expanding ...'')
%  
%\item {\bf \em Map atomic motion clauses to atomic motion predicates.}
%  We map each clause to one or more atomic motion predicates (e.g. we
%  map ``rotates by 90 degrees'' to predicates \texttt{type(m,
%    "rotate")} and \texttt{magnitude(m, 90)}).
%  
%\item {\bf \em Map phrases to relative temporal/spatial
%  predicates.}  We map phrases refering to relative temporal sequencing or
% spatial positioning to predicates based on the Allen and
%  rectangle algebras. (e.g. we
%  map ``Then, '' to the predicate \texttt{after(m1, m2)}).

%\item {\bf \em Compose motion verification program.}  We compose the
%  final verification program by using logical operators to chain
%  predicates together.
%\end{enumerate}
%\maneesh{Not sure we need the enumation.}




\vspace{0.5em}
\noindent
{\bf \em Step 1: Resolve object co-references.}
The first step
identifies segments of text that refer to the same objects
%or motions
For example, in Figure~\ref{fig:parser} we identify that ``it'' refers to
``the black square''.
%
To resolve such co-references %to objects,
we use the fastcoref model\,\cite{otmazgin2022fastcoref} which is primarily
designed to handle noun co-references.
%As shown in Figure~\ref{fig:parser}, this places ``the black square'' and ``it'' in the same co-reference cluster.
%
%To resolve motion co-references we first identify synonymous motion
%verbs (e.g. ``rotate'' and ``spin'') and then check that the other
%attributes of the motion match.

%For motion co-references, we identify \textit{motion verbs}, defined
%as verbs that are synonyms of either ``translate'', ``rotate'', or
%``scale,''. 

%We then compare the content of their subtrees that each acts as
%a modifier for one of the seven attributes of the motion entailed by
%that verb.  If the content of the subtrees describe the same
%attribute match, we mark these verbs as co-references of each other.
%For example, for the prompt ``The black square spins. While the square
%is rotating, the circle scales up.'' we would categorize ``spins'' and
%``is rotating'' as co-references as they share the same motion type
%(rotation) and the same subject (the black square).  Note that verbs
%referencing previous motions need to sit within a subordinate clause
%(e.g. ``is rotating'' is within a adverbial clause), as we assume
%verbs in main clauses always introduce new motions.
%
% matching algorithm
% NOTE: More specifically, we first find pairs of verbs that are synonyms of each other, where the first verb in a pair appears earlier in the prompt and is in a main clause, and the second verb appears later and is in a subordinate clause.

% Take Example Input 2, we are going to compare the dependency parse trees of “the black square spins” and “while the square is rotating”

\vspace{0.5em}
\noindent {\bf \em Step 2: Split prompt into atomic motion clauses.}
In step two we 
break the prompt into clauses, each representing a
single atomic motion.  Our approach uses a Roberta-based transformer
model~\cite{liu2019roberta} trained as a dependency
parser\,\cite{honnibal2020spacy} to convert the input text prompt into
a set of parse trees.
%
Each resulting parse tree contains a verb at its root and represents a
a complete sentence.
%
In English, a sentence may contain more than one verb, so we
further split such trees at each such internal verb.  For example,
in Figure~\ref{fig:parser} we split ``While it lifts, the blue square
rotates by 90 degrees'' into individual clauses ``While it lifts'' and ``the
blue square rotates by 90 degrees''.

%We next check whether the root verb of each tree represents a {\em motion
%  verb}, which we define as a verb that is a synonym of either
%``translate'', ``rotate'', or ``scale,'' -- the three atomic motion
%types in \dslname{}.
%
%To identify synonyms we use SetFit\,\cite{tunstall2022setfit}, a transformer-based classifier we finetune
%with a set of synonyms we obtained from Wikipedia and Thesaurus.com.
%    
%relationship between them.  Other linguistic information like a word’s
%part of speech and lemma is also stored with a node.  For example,
%``the square translates'' will be converted into a tree rooted at the
%verb ``translate''.  The noun ``square`` becomes the child of
%``translate'' with the dependency tag as \textit{nsubj}, the nominal
%subject of the verb.

%For each tree rooted at a motion verb we next identify the nearest
%object the motion is acting on
%
Each original parse tree includes either a subject marked with the dependency tag \textit{nsubj} (nominal subject) or an object marked \textit{dobj} (direct object).
%
% Note however that the tree for ``expanding'' in Figure~\ref{fig:parser} does
% not contain either a \textit{nsubj} or \textit{dobj} node because it was split out from the tree containing the rotation verb.
% %
However, splitting a tree at internal verbs can result in a tree that does not contain a subject or object.
In such cases, we look for the closest \textit{nsubj} or \textit{dobj}
node in the original tree and attach it to the dangling tree.
%
% collapsing verb nodes
English also allows the use of an auxiliary verb in
addition to the main verb to introduce a motion, such as “animate the
square to rotate” and “make the square expand”.  The dependency trees
for such constructions includes non-motion verbs (``animate''
and ``make'') as roots.
We resolve these case by combining the 
main and auxiliary verbs into one node and reattaching their original
subtrees under the combined node.

%At the end of this step, each tree represents an atomic motion for a single object.

\vspace{0.5em}
\noindent
{\bf \em Step 3: Map phrases to \dslname{} predicates.}
In this step, we map phrases in each atomic parse tree to a corresponding predicate or logic operator in \dslname{}.
%
Our approach is to finetune a transformer-based text classifier with SetFit\,\cite{tunstall2022setfit} to build this correspondence.
%
More specifically, for each predicate and logic operator in \dslname{}, we
manually look-up a small set of corresponding synonyms from WordNet~\cite{miller1994wordnet}, thesaurus.com, and synonym.com and use them for finetuning.
% so that it can identify correspondences between the predicate and a much larger set of synonyms.
%
For example, for the predicate \texttt{type(m, ``rot'')} that checks for a rotational motion, we look up synonyms such as ``spin'', ``turn'', etc. and
the after finetuning the classifier with them, it can classify natural language words and phrases that are synonyms of the concept of \texttt{type(m, ``rot'')}.

We process the atomic parse tree in a top-down manner.
For each node of the tree, we apply the finetuned classifier to check if
there is a correspondence between a \dslname{} predicate or operator and the word or phrase spanned by the subtree rooted at that node.
%
Once a correspondence is found, we extract relevant parameters for the predicate by analyzing the content of the mapped subtree.
For example, we extract the number ``90'' from the subtree attached to the verb node ``rotates'' after we map its content to the \texttt{mag()} predicate (Figure~\ref{fig:parser}).
%
As we process each node, we first check if the node is marked as a \textit{nsubj} or \textit{dobj}.
If so, we limit the correspondence match to the \texttt{agt()}
predicate and object attribute predicates such as \texttt{shp()} and
\texttt{clr()}.
% \maneesh{Probably can cut next sentence.}
% Similarly, if a node within the tree is a synonym of
% ``every'', ``each'', or ``all'', we mark the agent of the motion as
% plural and use \texttt{all()}, otherwise we use \texttt{iota()}.

\vspace{0.5em}
\noindent
{\bf \em Step 4: Resolve motion co-references}
There are cases where people might use a subordinate clause to refer
to a previous motion.  For example in Figure~\ref{fig:parser}, the
``while'' clause references an existing translational motion instead
of introducing a new motion.
%
However, our earlier steps treat the ``while'' clause motion as
a standalone atomic motion.
%
To resolve this motion co-reference, we compare every atomic parse tree
$p_{i}$ that comes from a subordinate clause to all atomic trees that
appear before it in the prompt.
% \maneesh{How do we know it is a subordinate clause?}
%
If they match in their {\tt type()} and {\tt agt()}
predicates, we mark $p_{i}$ as a co-reference and remove its
predicates. However, we retain its temporal sequencing predicate
(e.g., \texttt{while()}) and append it to the atomic motion described in the
original main clause.
%, as the subordinate clause, by referencing an
%existing motion, is adding a temporal constraint as opposed to
%introducing a new motion
%(Step 4 in Figure~\ref{fig:parser}).
% \maneesh{We don't have atomic motions, we have atomic motion clauses -- those are the subtrees. Need to rewrite the previous sentence. We never build atomic motions as they are described in the paper. We build predicates in prev step. Also next sentence is confusing as I'm not sure which Allen algebra predicate we are referring to or where it comes from.}


\vspace{0.5em}
\noindent
{\bf \em Step 5: Compose \dslname{} verification program.}
We compose the final \dslname{} program by traversing the tree
top-down and outputting the the predicate or the logic operator that
corresponds to each node as identified in Step 3.


% \vspace{-5mm}
% \begin{description}
%\begin{description}[style=unboxed,leftmargin=0.5cm]
%    \item[\texttt{direction()}]
%    We use another fine-tuned sentence transformation to classify the subtree text into one or more of \{up, down, left, right, clockwise, counterclockwise, x-axis, y-axis\}.
%    If the type of motion is translation, we map up to [0,1], down to [0, -1], left to [-1, 0], and right to [1, 0].
%    If the type of motion is rotation, we map clockwise to ``clockwise'' and counterclockwise to ``counterclockwise''.
%%    If the type of motion is scaling, we map up to +1, down to -1, x-axis to [1,0], and y-axis to [0,1]. If no axis is specified, then we use [1,1]. We then output the final direction is up/down * axis. For example, ``up in x'' is [1, 0]
%    
%    \item[\texttt{magnitude()}]
%    We look for node with the part of speech \textit{NUM} and use that as the parameter if the motion type is translate or rotate
%    If the type is scale, we do NUM $\cdot$ axis with the axis vector obtained from before
%    
%    \item[\texttt{origin()}]
%    We check if the subtree text contains a string pattern of (num1, num2) and use those as absolute parameters.
%    If a reference object has been extracted from this subtree, we use that as the parameter instead.
%    
%    \item[\texttt{duration()}]
%    We look for node with type “nummod” or parts of speech as NUM and use that as the parameter.
%    
%    \item[\texttt{post()}]
%    We use a separate transformer classifier to convert the subtree text to its corresponding spatial concept predicate, with arguments being the current motion agent and the reference object.
%\end{description}


% LocalWords:  templated LLM fastcoref nsubj dobj subtrees SetFit agt
% LocalWords:  finetune finetuning finetuned WordNet subtree shp clr
% LocalWords:  translational
