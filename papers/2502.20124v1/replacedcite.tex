\section{Related work}
\label{Related work}
Continual learning (CL) ____ usually operates under the closed-world assumption, where the system assumes that all test or deployment samples belong to one of the predefined classes seen during training ____.
However, this assumption implies no exposure to novel or previously unseen data during testing, which is far from realistic in dynamic, real-world environments ____. 
In practice, continual learning systems are frequently confronted with new, unknown classes, necessitating the ability to detect, adapt to, and incrementally learn these novelties, continually acquiring new knowledge over time ____.
Hence, it is imperative to detect and incrementally learn novelties while acquiring knowledge without forgetting over time.

More recently, continual learning in an open world or simply \textit{Open-world Continual Learning} (OWCL) has been appealing yet challenging with increasing works ____. 
In order to enable existing CL models to effectively detect open/unknown samples, primary OWCL research utilized open-set recognition (OSR) and out-of-distribution (OOD) detection methods as expansive components into CL baselines to tackle the OWCL tasks. 
____ proposed an OSR framework based on extreme value theory, incorporating incremental tasks to manage dynamic learning environments. ____ introduced an approach leveraging contrastive clustering and an energy-based identification method ____ for handling dynamic data, enabling the system to recognize and accommodate novel inputs during continual learning.
Building on these previous efforts, recent OWCL studies emphasized the integration of OOD detection techniques within the continual learning paradigm. For instance, ____ highlighted the importance of novelty detection as a crucial aspect of open-world learning, suggesting that existing OOD techniques could be effectively adapted to the continual learning setting. 
In a complementary development, frameworks such as SOLA ____ have been proposed, combining OOD detection with incremental task adaptation to facilitate novelty detection and task-specific learning in an open-world context. 

Nevertheless, current research still relies on simplistic approaches by combining CL methods with OOD detection components ____, and several crucial challenges persist in OWCL. 
First, the absence of a standardized and general problem formulation makes it challenging to compare the performance of existing methods in a consistent setting, leading to fairness issues. 
Second, there are still experimental and theoretical gaps in exploring the knowledge transfer for known and unknown samples in OWCL.
Finally, there is a lack of theoretical foundation to guide the design of an OWCL model that supports knowledge transferring and knowledge updating.

To address these limitations, this paper makes several key contributions. 
First, we provide rigorous theoretical analyses with a formal problem construction for understanding the variations and challenges inherent to OWCL. 
By constructing four distinct scenarios, we conduct extensive empirical experiments and systematically explore the factors influencing OWCL model performance under these different scenarios, identifying a significant interplay between open detection and the classification of known samples. 
Finally, we propose a novel framework, termed HoliTrans, designed to effectively address knowns-unknowns knowledge transfer by introducing NRP and proposing DAPs in an adaptive knowledge space.
The proposed HoliTrans serves as a strong baseline for future research, demonstrating its robustness and efficacy across a range of OWCL scenarios with abundant experiments.
% \input{2-Related work}