\documentclass[conference]{IEEEtran}
\IEEEoverridecommandlockouts
% The preceding line is only needed to identify funding in the first footnote. If that is unneeded, please comment it out.
\usepackage{cite}
\usepackage{amsmath,amssymb,amsfonts}
\usepackage{algorithmic}
\usepackage{subcaption}
\usepackage{graphicx}
\usepackage{textcomp}
\usepackage{xcolor}
\usepackage{comment}
\usepackage{hyperref}
%\usepackage{url}
\def\BibTeX{{\rm B\kern-.05em{\sc i\kern-.025em b}\kern-.08em
    T\kern-.1667em\lower.7ex\hbox{E}\kern-.125emX}}

%\newcommand{\se}[1]{{\textcolor{green}{#1}}}
\newcommand{\se}[1]{{\textcolor{black}{#1}}}
%\newcommand{\secor}[2]{{\textcolor{red}{[#1 $\rightarrow$ #2]}}}
\newcommand{\secor}[2]{{\textcolor{black}{#2}}}
%\newcommand{\secmt}[1]{{\textcolor{blue}{[SE: #1]}}}
\newcommand{\secmt}[1]{}
    
\begin{document}

\title{Perceptual Noise-Masking with Music \\through Deep Spectral Envelope Shaping %*\\
%{\footnotesize \textsuperscript{*}Note: Sub-titles are not captured in Xplore and
%should not be used}
\thanks{This work was supported by the Audible project funded by French BPI, and was performed using HPC resources from GENCI-IDRIS (Grant 2023-AD011014883). 
}
}

% \author{\IEEEauthorblockN{Clémentine Berger}
% \IEEEauthorblockA{\textit{LTCI} \\
% \textit{Télécom Paris}\\
% Paris, France \\
% clementine.berger@telecom-paris.fr}
% \and
% \IEEEauthorblockN{Roland Badeau}
% \IEEEauthorblockA{\textit{LTCI} \\
% \textit{Télécom Paris}\\
% Paris, France \\
% roland.badeau@telecom-paris.fr}
% \and
% \IEEEauthorblockN{Slim Essid}
% \IEEEauthorblockA{\textit{LTCI} \\
% \textit{Télécom Paris}\\
% Paris, France \\
% slim.essid@telecom-paris.fr}
% }

\author{\IEEEauthorblockN{Clémentine Berger, Roland Badeau, Slim Essid}
\vspace{0.3cm}
\IEEEauthorblockA{
    \textit{LTCI}, \textit{Télécom Paris}, \textit{Institut Polytechnique de Paris}, Palaiseau, France \\
    \{firstname\}.\{surname\}@telecom-paris.fr
    }}


\maketitle

\begin{abstract}
%This document is a model and instructions for \LaTeX.
%This and the IEEEtran.cls file define the components of your paper [title, text, heads, etc.]. *CRITICAL: Do Not Use Symbols, Special Characters, Footnotes, 
%or Math in Paper Title or Abstract.
People often listen to music in noisy environments, seeking to isolate themselves from ambient sounds. Indeed, a music signal can mask some of the noise's frequency components due to the effect of simultaneous masking. In this article, we propose a neural network based on a psychoacoustic masking model, designed to enhance the music's ability to mask ambient noise by reshaping its spectral envelope with predicted filter frequency responses. The model is trained with a perceptual loss function that balances two constraints: effectively masking the noise while preserving the original music mix and the user's chosen listening level. We evaluate our approach on simulated data replicating a user's experience of listening to music with headphones in a noisy environment. The results, based on defined objective metrics, demonstrate that our system improves the state of the art.
\end{abstract}

\begin{IEEEkeywords}
Ambient noise masking, deep filtering, psychoacoustics
\end{IEEEkeywords}

\section{Introduction}
\section{Introduction}


\begin{figure}[t]
\centering
\includegraphics[width=0.6\columnwidth]{figures/evaluation_desiderata_V5.pdf}
\vspace{-0.5cm}
\caption{\systemName is a platform for conducting realistic evaluations of code LLMs, collecting human preferences of coding models with real users, real tasks, and in realistic environments, aimed at addressing the limitations of existing evaluations.
}
\label{fig:motivation}
\end{figure}

\begin{figure*}[t]
\centering
\includegraphics[width=\textwidth]{figures/system_design_v2.png}
\caption{We introduce \systemName, a VSCode extension to collect human preferences of code directly in a developer's IDE. \systemName enables developers to use code completions from various models. The system comprises a) the interface in the user's IDE which presents paired completions to users (left), b) a sampling strategy that picks model pairs to reduce latency (right, top), and c) a prompting scheme that allows diverse LLMs to perform code completions with high fidelity.
Users can select between the top completion (green box) using \texttt{tab} or the bottom completion (blue box) using \texttt{shift+tab}.}
\label{fig:overview}
\end{figure*}

As model capabilities improve, large language models (LLMs) are increasingly integrated into user environments and workflows.
For example, software developers code with AI in integrated developer environments (IDEs)~\citep{peng2023impact}, doctors rely on notes generated through ambient listening~\citep{oberst2024science}, and lawyers consider case evidence identified by electronic discovery systems~\citep{yang2024beyond}.
Increasing deployment of models in productivity tools demands evaluation that more closely reflects real-world circumstances~\citep{hutchinson2022evaluation, saxon2024benchmarks, kapoor2024ai}.
While newer benchmarks and live platforms incorporate human feedback to capture real-world usage, they almost exclusively focus on evaluating LLMs in chat conversations~\citep{zheng2023judging,dubois2023alpacafarm,chiang2024chatbot, kirk2024the}.
Model evaluation must move beyond chat-based interactions and into specialized user environments.



 

In this work, we focus on evaluating LLM-based coding assistants. 
Despite the popularity of these tools---millions of developers use Github Copilot~\citep{Copilot}---existing
evaluations of the coding capabilities of new models exhibit multiple limitations (Figure~\ref{fig:motivation}, bottom).
Traditional ML benchmarks evaluate LLM capabilities by measuring how well a model can complete static, interview-style coding tasks~\citep{chen2021evaluating,austin2021program,jain2024livecodebench, white2024livebench} and lack \emph{real users}. 
User studies recruit real users to evaluate the effectiveness of LLMs as coding assistants, but are often limited to simple programming tasks as opposed to \emph{real tasks}~\citep{vaithilingam2022expectation,ross2023programmer, mozannar2024realhumaneval}.
Recent efforts to collect human feedback such as Chatbot Arena~\citep{chiang2024chatbot} are still removed from a \emph{realistic environment}, resulting in users and data that deviate from typical software development processes.
We introduce \systemName to address these limitations (Figure~\ref{fig:motivation}, top), and we describe our three main contributions below.


\textbf{We deploy \systemName in-the-wild to collect human preferences on code.} 
\systemName is a Visual Studio Code extension, collecting preferences directly in a developer's IDE within their actual workflow (Figure~\ref{fig:overview}).
\systemName provides developers with code completions, akin to the type of support provided by Github Copilot~\citep{Copilot}. 
Over the past 3 months, \systemName has served over~\completions suggestions from 10 state-of-the-art LLMs, 
gathering \sampleCount~votes from \userCount~users.
To collect user preferences,
\systemName presents a novel interface that shows users paired code completions from two different LLMs, which are determined based on a sampling strategy that aims to 
mitigate latency while preserving coverage across model comparisons.
Additionally, we devise a prompting scheme that allows a diverse set of models to perform code completions with high fidelity.
See Section~\ref{sec:system} and Section~\ref{sec:deployment} for details about system design and deployment respectively.



\textbf{We construct a leaderboard of user preferences and find notable differences from existing static benchmarks and human preference leaderboards.}
In general, we observe that smaller models seem to overperform in static benchmarks compared to our leaderboard, while performance among larger models is mixed (Section~\ref{sec:leaderboard_calculation}).
We attribute these differences to the fact that \systemName is exposed to users and tasks that differ drastically from code evaluations in the past. 
Our data spans 103 programming languages and 24 natural languages as well as a variety of real-world applications and code structures, while static benchmarks tend to focus on a specific programming and natural language and task (e.g. coding competition problems).
Additionally, while all of \systemName interactions contain code contexts and the majority involve infilling tasks, a much smaller fraction of Chatbot Arena's coding tasks contain code context, with infilling tasks appearing even more rarely. 
We analyze our data in depth in Section~\ref{subsec:comparison}.



\textbf{We derive new insights into user preferences of code by analyzing \systemName's diverse and distinct data distribution.}
We compare user preferences across different stratifications of input data (e.g., common versus rare languages) and observe which affect observed preferences most (Section~\ref{sec:analysis}).
For example, while user preferences stay relatively consistent across various programming languages, they differ drastically between different task categories (e.g. frontend/backend versus algorithm design).
We also observe variations in user preference due to different features related to code structure 
(e.g., context length and completion patterns).
We open-source \systemName and release a curated subset of code contexts.
Altogether, our results highlight the necessity of model evaluation in realistic and domain-specific settings.






\section{Neural model}
In this work, we tackle the task of raising the listened music's masking thresholds to better mask the listener's surrounding noise.
The proposed model Deep Perceptual Noise Masking with Music (DPNMM) is based on a deep neural network that predicts the frequency responses of filters to apply to the music to achieve the desired masking effect. A schematic overview of the system is shown in Fig. \ref{fig:system_overview}. The neural network takes as input frequency features derived from both the noise and the music. Their respective power spectral densities (PSDs) are computed using sliding Hanning windows of length $N_{win}=$ 2048 points with a 75\% overlap and a sampling rate $f_s=$ 44100 Hz.  
Both are mapped to the psychoacoustic Bark scale \cite{zwickerPsychoacousticsFactsModels2010} resulting in the PSDs $P^{noise}(n, \nu)$ and $P^{music}(n, \nu)$, where $n$ denotes the frame index and $\nu$ the Bark band. 

The music masking thresholds $T(n, \nu)$ per Bark band are computed using Johnston's model \cite{johnstonTransformCodingAudio1988} as in Estreder's system \cite{estrederPerceptualAudioEqualization2018}. The calculation involves applying a \textit{spreading function} to the PSD in each critical band, which indicates how it masks signals within the same band and in adjacent bands. The contributions from all Bark bands are then summed, following the additivity property of simultaneous masking \cite{lutfiAdditivitySimultaneousMasking1983, johnstonTransformCodingAudio1988, humesModelsAdditivityMasking1989}, and adjusted downward by an offset that depends on whether the music is more tonal or noise-like.

\subsection{Architecture}
\begin{figure}
    \centering
    \includegraphics[width=\linewidth]{figures/system_overview.png}
    \caption{Overview of the proposed system. Bark features from the music and noise signals are computed: PSD per Bark band for both music and noise and music masking thresholds. The features are fed to the U-Net that outputs gains in dB used to scale filter frequency responses. They are applied in the spectral domain to the music and a processed version is generated by inverse STFT.}\vspace{-0.3cm}
    \label{fig:system_overview}
\end{figure}

Our model's architecture is illustrated in Fig. \ref{fig:u-net_architecture}. We design a U-Net encoder-decoder model, inspired by the DeepFilterNet architecture \cite{schroterDeepfilternetLowComplexity2022}, especially its Equivalent Rectangular Bandwidth (ERB) gains prediction branch. The encoder consists of 4 convolutional blocks (separable convolution + BatchNorm + ReLU) followed by a linear layer, designed to process frequency information between the Bark features. 
%while preserving the temporal structure. 
This frequency information is then passed through a Gated Recurrent Unit (GRU) layer to handle the temporal dynamics. The decoder mirrors the encoder and incorporates skip connections, outputting the predicted gains in dB, $g(n,\nu)$, per frame, and Bark band (up to the 24th band $\sim$ 16500 Hz). 
Additionally, a gain smoothing filter is applied to prevent too rapid variations in the filters' frequency responses over time, and the gains are finally clamped as in \cite{estrederPerceptualAudioEqualization2018} with the following threshold values:
\begin{equation}
    g(n, \nu) \leftarrow \max \big( \min( g(n, \nu); -5 \text{ dB} ) ; 10 \text{ dB} \big).
\end{equation} 
\vspace{-0.5cm}

\begin{figure}
    \centering
    \includegraphics[width=0.9\linewidth]{figures/u-net_architecture.png}
    \caption{U-Net architecture of the proposed model. $N$ is the number of time frames, and 26 the number of Bark bands. The encoder is composed of 4 convolutional layers (Conv), a linear layer and a Gated Recurrent Unit (GRU) layer. The decoder follows the inverse path with transposed convolutional layers (TConv) and pathway convolutions (PConv) as add-skip connections.}%\vspace{-0.3cm}
    \label{fig:u-net_architecture}
\end{figure}

\subsection{Filters frequency responses}

The obtained gains are converted into filter frequency responses
that are generated using a pattern $W_{dB}^\nu(f)$ centered on the corresponding Bark band $\nu$ (see Fig. \ref{fig:pattern_filter}). In each band, the pattern is constant and equal to 1. 
To simulate the spreading effect of masking across the Bark bands, the pattern transitions smoothly with a cosine shape across the two adjacent bands below (from 0 to 1) and above (from 1 to 0). For the lowest frequency band, the pattern level is set to 2 instead of 1 at the center. This approach accounts for the fact that adjacent bands also contribute to raising the masking threshold of a given band, thereby distributing the required gain across multiple bands and minimizing the boost needed on the central band. %alone.

\begin{figure}
    \centering
    \includegraphics[width=\linewidth]{figures/pattern_filter.pdf}
    \caption{Pattern $W_{dB}^\nu(f)$ used to shape the filters' frequency responses, for $\nu =$ 5.}\vspace{-0.3cm}
    \label{fig:pattern_filter}
\end{figure}
For each audio frame the overall frequency response $\mathcal{W}_{dB}(n,k)$ is obtained as:
\begin{equation}
    \mathcal{W}_{dB}(n,k) = \sum_{\nu=1}^{B} g(n,\nu) \cdot W_{dB}^\nu(k).
\end{equation}
The input music is then filtered, in the frequency domain, using those frequency responses frame by frame to produce a processed music whose masking properties are enhanced.

\subsection{Loss functions}
Our model DPNMM is trained with a primary loss function designed to raise the music's masking threshold above the noise level in the critical bands where needed. These Bark bands are constrained to exceed specific thresholds, while the energy in other bands is free to change as long as their masking threshold remains above the noise level, thereby supporting masking in adjacent bands: 
\begin{equation}
    \mathcal{L}_0 (\theta_t) = \frac{1}{N} \frac{1}{B}  \sum_{n,\nu} \text{ReLU}\left( P_{dB}^{noise}(n,\nu) - \hat{T}_{dB}(n,\nu) \right) ,
\end{equation}
where $\hat{T}_{dB}(n,\nu)$ is the masking threshold computed with the processed music, $N$ the number of time frames and $B$ the number of Bark bands.
By using a ReLU function to express the constraint, we only set a minimum threshold level for the network to reach, allowing it greater flexibility to find solutions for masking noise across all frequency bands. However, while this choice aims to provide the network with more freedom, it also brings the challenge that the system is not required to output zero gains when no amplification of the music is needed. To address this, we use the knowledge of the masking spreading effect  to compute a mask that identifies, for each band, whether it is close enough to another band (including itself) where the threshold needs to be raised to have an impact on it. If it is not, the gain for that band at the network's output is set to zero.

Even with this gain masking, the system still has an infinite number of potential solutions. To guide the learning process in a desired direction, we add a secondary constraint aimed at preserving the naturalness of the original music by limiting the average power variation: 
\begin{equation}
    \mathcal{L}_{power}(\theta_t) = \frac{1}{N} \sum_n \vert \hat{\mathcal{P}}_{dBA}^{music}(n) - \mathcal{P}_{dBA}^{music}(n) \vert \; ; 
\end{equation}
where the initial music mean power $\mathcal{P}$ of frame $n$ is evaluated in dBA \cite{a_weights}, as well as the processed music mean power $\hat{\mathcal{P}}$. To include this constraint in the training process we use a strategy inspired by the method of multipliers \cite{jonasdegraveHowWeCan2021, plattConstrainedDifferentialOptimization1987}. The goal is to ensure that during training the power variation does not exceed a given value $\Delta \mathcal{P}_{max}$ using a dynamic weight $\lambda_t$ to scale the loss. The overall loss function for this constrained optimization problem is:
\begin{equation}
    L(\theta_t, \lambda_t) = \mathcal{L}_0 - \lambda_t \cdot (\Delta \mathcal{P}_{max} - \mathcal{L}_{power}(\theta_t)).
\end{equation}
At the start of the training, $\lambda_t = 0$. While gradient descent is applied to the overall loss $L(\lambda_t)$, gradient ascent is simultaneously performed on $\lambda_t$ using the gradient $\frac{\partial L(\theta, \lambda_t)}{\partial \lambda_t} = \mathcal{L}_{power}(\theta_t) - \Delta \mathcal{P}_{max}$ with a specific learning rate of $10^{-3}$, keeping $\lambda_t$ always positive. When $\mathcal{L}_{power}(\theta_t)$ exceeds the threshold $\Delta \mathcal{P}_{max}$, $\lambda_t$ increases, giving more weight to the power constraints in the total loss. Conversely, when $\mathcal{L}_{power}(\theta_t)$ drops below $\Delta \mathcal{P}_{max}$, $\lambda_t$ decreases.
This approach allows us to guide the training in a direction that satisfies both constraints without requiring a tedious search for an optimal fixed weight. The degree of compromise between the two constraints is directly controlled by the choice of parameter $\Delta \mathcal{P}_{max}$. In the rest of the article, we explore several values for this parameter. 

The code for the implementation of DPNMM is available on github.\footnote{\href{https://github.com/ClementineBerger/DPNMM}{https://github.com/ClementineBerger/DPNMM}}

\section{Data}
To train and evaluate the proposed model, we generated training, validation, and test datasets that replicate realistic acoustic scenes. These scenes \secor{consist of music listened to by a user through headphones or earphones and ambient noise as perceived by the user through their earphones}{simulate a user listening to music through headphones or earphones while being in a noisy environment}.  

We defined several \textit{environments} to represent a variety of realistic acoustic scenes with different ambient noise levels: urban, indoor office, construction site, beach, transportation (train/plane/boat), and restaurant/bar. 
\se{We chose} the noise recordings from the DNS Challenge dataset \cite{dubeyIcassp2022Deep2022}.
Each environment is \se{created using} the labels from the noise dataset and a realistic noise level distribution in dBA. Noise samples are evenly selected per environment and normalized to levels sampled from the corresponding noise distribution. A pre-processing step is applied to drop the audio signals composed mainly of silence (which are therefore isolated, impulsive noises). After that, all those samples are filtered with one of three headphone frequency responses to reproduce their passive attenuation. 
Each noise sample is then paired with a music track \se{chosen from the FMA dataset \cite{defferrardFMADATASETMUSIC2017}} which covers a large diversity of music genres (Pop, Rock, Classical, Jazz, Hip-Hop, etc.).
Each music is normalized to a dBA level derived from a Signal-to-Noise Ratio (SNR) value, itself sampled from a defined SNR distribution. The resulting music dBA level is constrained within the \secor{45 dBA to 100 dBA range}{range $[45, 100]$ dBA,} reflecting the typical range offered by standard headphones.
\se{We thus generate} 50h of training data, 20h of validation data, and 10h of test data, composed of pairs of 10s mono music and noise excerpts sampled at 44100 Hz.

\section{Results}

\begin{table*}[t]
\centering
\fontsize{11pt}{11pt}\selectfont
\begin{tabular}{lllllllllllll}
\toprule
\multicolumn{1}{c}{\textbf{task}} & \multicolumn{2}{c}{\textbf{Mir}} & \multicolumn{2}{c}{\textbf{Lai}} & \multicolumn{2}{c}{\textbf{Ziegen.}} & \multicolumn{2}{c}{\textbf{Cao}} & \multicolumn{2}{c}{\textbf{Alva-Man.}} & \multicolumn{1}{c}{\textbf{avg.}} & \textbf{\begin{tabular}[c]{@{}l@{}}avg.\\ rank\end{tabular}} \\
\multicolumn{1}{c}{\textbf{metrics}} & \multicolumn{1}{c}{\textbf{cor.}} & \multicolumn{1}{c}{\textbf{p-v.}} & \multicolumn{1}{c}{\textbf{cor.}} & \multicolumn{1}{c}{\textbf{p-v.}} & \multicolumn{1}{c}{\textbf{cor.}} & \multicolumn{1}{c}{\textbf{p-v.}} & \multicolumn{1}{c}{\textbf{cor.}} & \multicolumn{1}{c}{\textbf{p-v.}} & \multicolumn{1}{c}{\textbf{cor.}} & \multicolumn{1}{c}{\textbf{p-v.}} &  &  \\ \midrule
\textbf{S-Bleu} & 0.50 & 0.0 & 0.47 & 0.0 & 0.59 & 0.0 & 0.58 & 0.0 & 0.68 & 0.0 & 0.57 & 5.8 \\
\textbf{R-Bleu} & -- & -- & 0.27 & 0.0 & 0.30 & 0.0 & -- & -- & -- & -- & - &  \\
\textbf{S-Meteor} & 0.49 & 0.0 & 0.48 & 0.0 & 0.61 & 0.0 & 0.57 & 0.0 & 0.64 & 0.0 & 0.56 & 6.1 \\
\textbf{R-Meteor} & -- & -- & 0.34 & 0.0 & 0.26 & 0.0 & -- & -- & -- & -- & - &  \\
\textbf{S-Bertscore} & \textbf{0.53} & 0.0 & {\ul 0.80} & 0.0 & \textbf{0.70} & 0.0 & {\ul 0.66} & 0.0 & {\ul0.78} & 0.0 & \textbf{0.69} & \textbf{1.7} \\
\textbf{R-Bertscore} & -- & -- & 0.51 & 0.0 & 0.38 & 0.0 & -- & -- & -- & -- & - &  \\
\textbf{S-Bleurt} & {\ul 0.52} & 0.0 & {\ul 0.80} & 0.0 & 0.60 & 0.0 & \textbf{0.70} & 0.0 & \textbf{0.80} & 0.0 & {\ul 0.68} & {\ul 2.3} \\
\textbf{R-Bleurt} & -- & -- & 0.59 & 0.0 & -0.05 & 0.13 & -- & -- & -- & -- & - &  \\
\textbf{S-Cosine} & 0.51 & 0.0 & 0.69 & 0.0 & {\ul 0.62} & 0.0 & 0.61 & 0.0 & 0.65 & 0.0 & 0.62 & 4.4 \\
\textbf{R-Cosine} & -- & -- & 0.40 & 0.0 & 0.29 & 0.0 & -- & -- & -- & -- & - & \\ \midrule
\textbf{QuestEval} & 0.23 & 0.0 & 0.25 & 0.0 & 0.49 & 0.0 & 0.47 & 0.0 & 0.62 & 0.0 & 0.41 & 9.0 \\
\textbf{LLaMa3} & 0.36 & 0.0 & \textbf{0.84} & 0.0 & {\ul{0.62}} & 0.0 & 0.61 & 0.0 &  0.76 & 0.0 & 0.64 & 3.6 \\
\textbf{our (3b)} & 0.49 & 0.0 & 0.73 & 0.0 & 0.54 & 0.0 & 0.53 & 0.0 & 0.7 & 0.0 & 0.60 & 5.8 \\
\textbf{our (8b)} & 0.48 & 0.0 & 0.73 & 0.0 & 0.52 & 0.0 & 0.53 & 0.0 & 0.7 & 0.0 & 0.59 & 6.3 \\  \bottomrule
\end{tabular}
\caption{Pearson correlation on human evaluation on system output. `R-': reference-based. `S-': source-based.}
\label{tab:sys}
\end{table*}



\begin{table}%[]
\centering
\fontsize{11pt}{11pt}\selectfont
\begin{tabular}{llllll}
\toprule
\multicolumn{1}{c}{\textbf{task}} & \multicolumn{1}{c}{\textbf{Lai}} & \multicolumn{1}{c}{\textbf{Zei.}} & \multicolumn{1}{c}{\textbf{Scia.}} & \textbf{} & \textbf{} \\ 
\multicolumn{1}{c}{\textbf{metrics}} & \multicolumn{1}{c}{\textbf{cor.}} & \multicolumn{1}{c}{\textbf{cor.}} & \multicolumn{1}{c}{\textbf{cor.}} & \textbf{avg.} & \textbf{\begin{tabular}[c]{@{}l@{}}avg.\\ rank\end{tabular}} \\ \midrule
\textbf{S-Bleu} & 0.40 & 0.40 & 0.19* & 0.33 & 7.67 \\
\textbf{S-Meteor} & 0.41 & 0.42 & 0.16* & 0.33 & 7.33 \\
\textbf{S-BertS.} & {\ul0.58} & 0.47 & 0.31 & 0.45 & 3.67 \\
\textbf{S-Bleurt} & 0.45 & {\ul 0.54} & {\ul 0.37} & 0.45 & {\ul 3.33} \\
\textbf{S-Cosine} & 0.56 & 0.52 & 0.3 & {\ul 0.46} & {\ul 3.33} \\ \midrule
\textbf{QuestE.} & 0.27 & 0.35 & 0.06* & 0.23 & 9.00 \\
\textbf{LlaMA3} & \textbf{0.6} & \textbf{0.67} & \textbf{0.51} & \textbf{0.59} & \textbf{1.0} \\
\textbf{Our (3b)} & 0.51 & 0.49 & 0.23* & 0.39 & 4.83 \\
\textbf{Our (8b)} & 0.52 & 0.49 & 0.22* & 0.43 & 4.83 \\ \bottomrule
\end{tabular}
\caption{Pearson correlation on human ratings on reference output. *not significant; we cannot reject the null hypothesis of zero correlation}
\label{tab:ref}
\end{table}


\begin{table*}%[]
\centering
\fontsize{11pt}{11pt}\selectfont
\begin{tabular}{lllllllll}
\toprule
\textbf{task} & \multicolumn{1}{c}{\textbf{ALL}} & \multicolumn{1}{c}{\textbf{sentiment}} & \multicolumn{1}{c}{\textbf{detoxify}} & \multicolumn{1}{c}{\textbf{catchy}} & \multicolumn{1}{c}{\textbf{polite}} & \multicolumn{1}{c}{\textbf{persuasive}} & \multicolumn{1}{c}{\textbf{formal}} & \textbf{\begin{tabular}[c]{@{}l@{}}avg. \\ rank\end{tabular}} \\
\textbf{metrics} & \multicolumn{1}{c}{\textbf{cor.}} & \multicolumn{1}{c}{\textbf{cor.}} & \multicolumn{1}{c}{\textbf{cor.}} & \multicolumn{1}{c}{\textbf{cor.}} & \multicolumn{1}{c}{\textbf{cor.}} & \multicolumn{1}{c}{\textbf{cor.}} & \multicolumn{1}{c}{\textbf{cor.}} &  \\ \midrule
\textbf{S-Bleu} & -0.17 & -0.82 & -0.45 & -0.12* & -0.1* & -0.05 & -0.21 & 8.42 \\
\textbf{R-Bleu} & - & -0.5 & -0.45 &  &  &  &  &  \\
\textbf{S-Meteor} & -0.07* & -0.55 & -0.4 & -0.01* & 0.1* & -0.16 & -0.04* & 7.67 \\
\textbf{R-Meteor} & - & -0.17* & -0.39 & - & - & - & - & - \\
\textbf{S-BertScore} & 0.11 & -0.38 & -0.07* & -0.17* & 0.28 & 0.12 & 0.25 & 6.0 \\
\textbf{R-BertScore} & - & -0.02* & -0.21* & - & - & - & - & - \\
\textbf{S-Bleurt} & 0.29 & 0.05* & 0.45 & 0.06* & 0.29 & 0.23 & 0.46 & 4.2 \\
\textbf{R-Bleurt} & - &  0.21 & 0.38 & - & - & - & - & - \\
\textbf{S-Cosine} & 0.01* & -0.5 & -0.13* & -0.19* & 0.05* & -0.05* & 0.15* & 7.42 \\
\textbf{R-Cosine} & - & -0.11* & -0.16* & - & - & - & - & - \\ \midrule
\textbf{QuestEval} & 0.21 & {\ul{0.29}} & 0.23 & 0.37 & 0.19* & 0.35 & 0.14* & 4.67 \\
\textbf{LlaMA3} & \textbf{0.82} & \textbf{0.80} & \textbf{0.72} & \textbf{0.84} & \textbf{0.84} & \textbf{0.90} & \textbf{0.88} & \textbf{1.00} \\
\textbf{Our (3b)} & 0.47 & -0.11* & 0.37 & 0.61 & 0.53 & 0.54 & 0.66 & 3.5 \\
\textbf{Our (8b)} & {\ul{0.57}} & 0.09* & {\ul 0.49} & {\ul 0.72} & {\ul 0.64} & {\ul 0.62} & {\ul 0.67} & {\ul 2.17} \\ \bottomrule
\end{tabular}
\caption{Pearson correlation on human ratings on our constructed test set. 'R-': reference-based. 'S-': source-based. *not significant; we cannot reject the null hypothesis of zero correlation}
\label{tab:con}
\end{table*}

\section{Results}
We benchmark the different metrics on the different datasets using correlation to human judgement. For content preservation, we show results split on data with system output, reference output and our constructed test set: we show that the data source for evaluation leads to different conclusions on the metrics. In addition, we examine whether the metrics can rank style transfer systems similar to humans. On style strength, we likewise show correlations between human judgment and zero-shot evaluation approaches. When applicable, we summarize results by reporting the average correlation. And the average ranking of the metric per dataset (by ranking which metric obtains the highest correlation to human judgement per dataset). 

\subsection{Content preservation}
\paragraph{How do data sources affect the conclusion on best metric?}
The conclusions about the metrics' performance change radically depending on whether we use system output data, reference output, or our constructed test set. Ideally, a good metric correlates highly with humans on any data source. Ideally, for meta-evaluation, a metric should correlate consistently across all data sources, but the following shows that the correlations indicate different things, and the conclusion on the best metric should be drawn carefully.

Looking at the metrics correlations with humans on the data source with system output (Table~\ref{tab:sys}), we see a relatively high correlation for many of the metrics on many tasks. The overall best metrics are S-BertScore and S-BLEURT (avg+avg rank). We see no notable difference in our method of using the 3B or 8B model as the backbone.

Examining the average correlations based on data with reference output (Table~\ref{tab:ref}), now the zero-shoot prompting with LlaMA3 70B is the best-performing approach ($0.59$ avg). Tied for second place are source-based cosine embedding ($0.46$ avg), BLEURT ($0.45$ avg) and BertScore ($0.45$ avg). Our method follows on a 5. place: here, the 8b version (($0.43$ avg)) shows a bit stronger results than 3b ($0.39$ avg). The fact that the conclusions change, whether looking at reference or system output, confirms the observations made by \citet{scialom-etal-2021-questeval} on simplicity transfer.   

Now consider the results on our test set (Table~\ref{tab:con}): Several metrics show low or no correlation; we even see a significantly negative correlation for some metrics on ALL (BLEU) and for specific subparts of our test set for BLEU, Meteor, BertScore, Cosine. On the other end, LlaMA3 70B is again performing best, showing strong results ($0.82$ in ALL). The runner-up is now our 8B method, with a gap to the 3B version ($0.57$ vs $0.47$ in ALL). Note our method still shows zero correlation for the sentiment task. After, ranks BLEURT ($0.29$), QuestEval ($0.21$), BertScore ($0.11$), Cosine ($0.01$).  

On our test set, we find that some metrics that correlate relatively well on the other datasets, now exhibit low correlation. Hence, with our test set, we can now support the logical reasoning with data evidence: Evaluation of content preservation for style transfer needs to take the style shift into account. This conclusion could not be drawn using the existing data sources: We hypothesise that for the data with system-based output, successful output happens to be very similar to the source sentence and vice versa, and reference-based output might not contain server mistakes as they are gold references. Thus, none of the existing data sources tests the limits of the metrics.  


\paragraph{How do reference-based metrics compare to source-based ones?} Reference-based metrics show a lower correlation than the source-based counterpart for all metrics on both datasets with ratings on references (Table~\ref{tab:sys}). As discussed previously, reference-based metrics for style transfer have the drawback that many different good solutions on a rewrite might exist and not only one similar to a reference.


\paragraph{How well can the metrics rank the performance of style transfer methods?}
We compare the metrics' ability to judge the best style transfer methods w.r.t. the human annotations: Several of the data sources contain samples from different style transfer systems. In order to use metrics to assess the quality of the style transfer system, metrics should correctly find the best-performing system. Hence, we evaluate whether the metrics for content preservation provide the same system ranking as human evaluators. We take the mean of the score for every output on each system and the mean of the human annotations; we compare the systems using the Kendall's Tau correlation. 

We find only the evaluation using the dataset Mir, Lai, and Ziegen to result in significant correlations, probably because of sparsity in a number of system tests (App.~\ref{app:dataset}). Our method (8b) is the only metric providing a perfect ranking of the style transfer system on the Lai data, and Llama3 70B the only one on the Ziegen data. Results in App.~\ref{app:results}. 


\subsection{Style strength results}
%Evaluating style strengths is a challenging task. 
Llama3 70B shows better overall results than our method. However, our method scores higher than Llama3 70B on 2 out of 6 datasets, but it also exhibits zero correlation on one task (Table~\ref{tab:styleresults}).%More work i s needed on evaluating style strengths. 
 
\begin{table}%[]
\fontsize{11pt}{11pt}\selectfont
\begin{tabular}{lccc}
\toprule
\multicolumn{1}{c}{\textbf{}} & \textbf{LlaMA3} & \textbf{Our (3b)} & \textbf{Our (8b)} \\ \midrule
\textbf{Mir} & 0.46 & 0.54 & \textbf{0.57} \\
\textbf{Lai} & \textbf{0.57} & 0.18 & 0.19 \\
\textbf{Ziegen.} & 0.25 & 0.27 & \textbf{0.32} \\
\textbf{Alva-M.} & \textbf{0.59} & 0.03* & 0.02* \\
\textbf{Scialom} & \textbf{0.62} & 0.45 & 0.44 \\
\textbf{\begin{tabular}[c]{@{}l@{}}Our Test\end{tabular}} & \textbf{0.63} & 0.46 & 0.48 \\ \bottomrule
\end{tabular}
\caption{Style strength: Pearson correlation to human ratings. *not significant; we cannot reject the null hypothesis of zero corelation}
\label{tab:styleresults}
\end{table}

\subsection{Ablation}
We conduct several runs of the methods using LLMs with variations in instructions/prompts (App.~\ref{app:method}). We observe that the lower the correlation on a task, the higher the variation between the different runs. For our method, we only observe low variance between the runs.
None of the variations leads to different conclusions of the meta-evaluation. Results in App.~\ref{app:results}.

\section{Conclusions}
\vspace{-0.2cm}
\section{Impact: Why Free Scientific Knowledge?}
\vspace{-0.1cm}

Historically, making knowledge widely available has driven transformative progress. Gutenberg’s printing press broke medieval monopolies on information, increasing literacy and contributing to the Renaissance and Scientific Revolution. In today's world, open source projects such as GNU/Linux and Wikipedia show that freely accessible and modifiable knowledge fosters innovation while ensuring creators are credited through copyleft licenses. These examples highlight a key idea: \textit{access to essential knowledge supports overall advancement.} 

This aligns with the arguments made by Prabhakaran et al. \cite{humanrightsbasedapproachresponsible}, who specifically highlight the \textbf{ human right to participate in scientific advancement} as enshrined in the Universal Declaration of Human Rights. They emphasize that this right underscores the importance of \textit{ equal access to the benefits of scientific progress for all}, a principle directly supported by our proposal for Knowledge Units. The UN Special Rapporteur on Cultural Rights further reinforces this, advocating for the expansion of copyright exceptions to broaden access to scientific knowledge as a crucial component of the right to science and culture \cite{scienceright}. 

However, current intellectual property regimes often create ``patently unfair" barriers to this knowledge, preventing innovation and access, especially in areas critical to human rights, as Hale compellingly argues \cite{patentlyunfair}. Finding a solution requires carefully balancing the imperative of open access with the legitimate rights of authors. As Austin and Ginsburg remind us, authors' rights are also human rights, necessitating robust protection \cite{authorhumanrights}. Shareable knowledge entities like Knowledge Units offer a potential mechanism to achieve this delicate balance in the scientific domain, enabling wider dissemination of research findings while respecting authors' fundamental rights.

\vspace{-0.2cm}
\subsection{Impact Across Sectors}

\textbf{Researchers:} Collaboration across different fields becomes easier when knowledge is shared openly. For instance, combining machine learning with biology or applying quantum principles to cryptography can lead to important breakthroughs. Removing copyright restrictions allows researchers to freely use data and methods, speeding up discoveries while respecting original contributions.

\textbf{Practitioners:} Professionals, especially in healthcare, benefit from immediate access to the latest research. Quick access to newer insights on the effectiveness of drugs, and alternative treatments speeds up adoption and awareness, potentially saving lives. Additionally, open knowledge helps developing countries gain access to health innovations.

\textbf{Education:} Education becomes more accessible when teachers use the latest research to create up-to-date curricula without prohibitive costs. Students can access high-quality research materials and use LM assistance to better understand complex topics, enhancing their learning experience and making high-quality education more accessible.

\textbf{Public Trust:} When information is transparent and accessible, the public can better understand and trust decision-making processes. Open access to government policies and industry practices allows people to review and verify information, helping to reduce misinformation. This transparency encourages critical thinking and builds trust in scientific and governmental institutions.

Overall, making scientific knowledge accessible supports global fairness. By viewing knowledge as a common resource rather than a product to be sold, we can speed up innovation, encourage critical thinking, and empower communities to address important challenges.

\vspace{-0.2cm}
\section{Open Problems}
\vspace{-0.1cm}

Moving forward, we identify key research directions to further exploit the potential of converting original texts into shareable knowledge entities such as demonstrated by the conversion into Knowledge Units in this work:


\textbf{1. Enhancing Factual Accuracy and Reliability:}  Refining KUs through cross-referencing with source texts and incorporating community-driven correction mechanisms, similar to Wikipedia, can minimize hallucinations and ensure the long-term accuracy of knowledge-based datasets at scale.

\textbf{2. Developing Applications for Education and Research:}  Using KU-based conversion for datasets to be employed in practical tools, such as search interfaces and learning platforms, can ensure rapid dissemination of any new knowledge into shareable downstream resources, significantly improving the accessibility, spread, and impact of KUs.

\textbf{3. Establishing Standards for Knowledge Interoperability and Reuse:}  Future research should focus on defining standardized formats for entities like KU and knowledge graph layouts \citep{lenat1990cyc}. These standards are essential to unlock seamless interoperability, facilitate reuse across diverse platforms, and foster a vibrant ecosystem of open scientific knowledge. 

\textbf{4. Interconnecting Shareable Knowledge for Scientific Workflow Assistance and Automation:} There might be further potential in constructing a semantic web that interconnects publicly shared knowledge, together with mechanisms that continually update and validate all shareable knowledge units. This can be starting point for a platform that uses all collected knowledge to assist scientific workflows, for instance by feeding such a semantic web into recently developed reasoning models equipped with retrieval augmented generation. Such assistance could assemble knowledge across multiple scientific papers, guiding scientists more efficiently through vast research landscapes. Given further progress in model capabilities, validation, self-repair and evolving new knowledge from already existing vast collection in the semantic web can lead to automation of scientific discovery, assuming that knowledge data in the semantic web can be freely shared.

We open-source our code and encourage collaboration to improve extraction pipelines, enhance Knowledge Unit capabilities, and expand coverage to additional fields.

\vspace{-0.2cm}
\section{Conclusion}
\vspace{-0.1cm}

In this paper, we highlight the potential of systematically separating factual scientific knowledge from protected artistic or stylistic expression. By representing scientific insights as structured facts and relationships, prototypes like Knowledge Units (KUs) offer a pathway to broaden access to scientific knowledge without infringing copyright, aligning with legal principles like German \S 24(1) UrhG and U.S. fair use standards. Extensive testing across a range of domains and models shows evidence that Knowledge Units (KUs) can feasibly retain core information. These findings offer a promising way forward for openly disseminating scientific information while respecting copyright constraints.

\section*{Author Contributions}

Christoph conceived the project and led organization. Christoph and Gollam led all the experiments. Nick and Huu led the legal aspects. Tawsif led the data collection. Ameya and Andreas led the manuscript writing. Ludwig, Sören, Robert, Jenia and Matthias provided feedback. advice and scientific supervision throughout the project. 

\section*{Acknowledgements}

The authors would like to thank (in alphabetical order): Sebastian Dziadzio, Kristof Meding, Tea Mustać, Shantanu Prabhat for insightful feedback and suggestions. Special thanks to Andrej Radonjic for help in scaling up data collection. GR and SA acknowledge financial support by the German Research Foundation (DFG) for the NFDI4DataScience Initiative (project number 460234259). AP and MB acknowledge financial support by the Federal Ministry of Education and Research (BMBF), FKZ: 011524085B and Open Philanthropy Foundation funded by the Good Ventures Foundation. AH acknowledges financial support by the Federal Ministry of Education and Research (BMBF), FKZ: 01IS24079A and the Carl Zeiss Foundation through the project "Certification and Foundations of Safe ML Systems" as well as the support from the International Max Planck Research School for Intelligent Systems (IMPRS-IS). JJ acknowledges funding by the Federal Ministry of Education and Research of Germany (BMBF) under grant no. 01IS22094B (WestAI - AI Service Center West), under grant no. 01IS24085C (OPENHAFM) and under the grant DE002571 (MINERVA), as well as co-funding by EU from EuroHPC Joint Undertaking programm under grant no. 101182737 (MINERVA) and from Digital Europe Programme under grant no. 101195233 (openEuroLLM) 

%\section*{Acknowledgment}

%The preferred spelling of the word ``acknowledgment'' in America is without 
%an ``e'' after the ``g''. Avoid the stilted expression ``one of us (R. B. 
%G.) thanks $\ldots$''. Instead, try ``R. B. G. thanks$\ldots$''. Put sponsor 
%acknowledgments in the unnumbered footnote on the first page.

\bibliographystyle{IEEEtran}
%\bibliography{IEEEabrv,bibliography}    
\bibliography{biblio}    

\end{document}
