\documentclass[conference]{IEEEtran}
\IEEEoverridecommandlockouts
% The preceding line is only needed to identify funding in the first footnote. If that is unneeded, please comment it out.
\usepackage{cite}
\usepackage{amsmath,amssymb,amsfonts}
\usepackage{algorithmic}
\usepackage{subcaption}
\usepackage{graphicx}
\usepackage{textcomp}
\usepackage{xcolor}
\usepackage{comment}
\usepackage{hyperref}
%\usepackage{url}
\def\BibTeX{{\rm B\kern-.05em{\sc i\kern-.025em b}\kern-.08em
    T\kern-.1667em\lower.7ex\hbox{E}\kern-.125emX}}

%\newcommand{\se}[1]{{\textcolor{green}{#1}}}
\newcommand{\se}[1]{{\textcolor{black}{#1}}}
%\newcommand{\secor}[2]{{\textcolor{red}{[#1 $\rightarrow$ #2]}}}
\newcommand{\secor}[2]{{\textcolor{black}{#2}}}
%\newcommand{\secmt}[1]{{\textcolor{blue}{[SE: #1]}}}
\newcommand{\secmt}[1]{}
    
\begin{document}

\title{Perceptual Noise-Masking with Music \\through Deep Spectral Envelope Shaping %*\\
%{\footnotesize \textsuperscript{*}Note: Sub-titles are not captured in Xplore and
%should not be used}
\thanks{This work was supported by the Audible project funded by French BPI, and was performed using HPC resources from GENCI-IDRIS (Grant 2023-AD011014883). 
}
}

% \author{\IEEEauthorblockN{Clémentine Berger}
% \IEEEauthorblockA{\textit{LTCI} \\
% \textit{Télécom Paris}\\
% Paris, France \\
% clementine.berger@telecom-paris.fr}
% \and
% \IEEEauthorblockN{Roland Badeau}
% \IEEEauthorblockA{\textit{LTCI} \\
% \textit{Télécom Paris}\\
% Paris, France \\
% roland.badeau@telecom-paris.fr}
% \and
% \IEEEauthorblockN{Slim Essid}
% \IEEEauthorblockA{\textit{LTCI} \\
% \textit{Télécom Paris}\\
% Paris, France \\
% slim.essid@telecom-paris.fr}
% }

\author{\IEEEauthorblockN{Clémentine Berger, Roland Badeau, Slim Essid}
\vspace{0.3cm}
\IEEEauthorblockA{
    \textit{LTCI}, \textit{Télécom Paris}, \textit{Institut Polytechnique de Paris}, Palaiseau, France \\
    \{firstname\}.\{surname\}@telecom-paris.fr
    }}


\maketitle

\begin{abstract}
%This document is a model and instructions for \LaTeX.
%This and the IEEEtran.cls file define the components of your paper [title, text, heads, etc.]. *CRITICAL: Do Not Use Symbols, Special Characters, Footnotes, 
%or Math in Paper Title or Abstract.
People often listen to music in noisy environments, seeking to isolate themselves from ambient sounds. Indeed, a music signal can mask some of the noise's frequency components due to the effect of simultaneous masking. In this article, we propose a neural network based on a psychoacoustic masking model, designed to enhance the music's ability to mask ambient noise by reshaping its spectral envelope with predicted filter frequency responses. The model is trained with a perceptual loss function that balances two constraints: effectively masking the noise while preserving the original music mix and the user's chosen listening level. We evaluate our approach on simulated data replicating a user's experience of listening to music with headphones in a noisy environment. The results, based on defined objective metrics, demonstrate that our system improves the state of the art.
\end{abstract}

\begin{IEEEkeywords}
Ambient noise masking, deep filtering, psychoacoustics
\end{IEEEkeywords}

\section{Introduction}
\section{Introduction}
\label{sec:intro}

\begin{figure*}[tb]
    \centering
    \includegraphics[width=0.848\linewidth]{figs/circuitnn.pdf} 
    \caption{Illustration of differentiable CircuitNN. CircuitNN is designed based on differentiable NAND gates. After DAS is guided by PI and PO pairs of the truth table, CircuitNN can get the precise circuit architecture logic equivalent to the truth table.}
    \label{fig:circuitnn}
\end{figure*}

% 1. Describe the importance of logic synthesis
% 2. Existing Problems
% (a) Neural Architecture Search: Unstable, Predefined Setting, etc.
% (b) Circuit Generation: Probabilistic Model, Logic Equivalence

With the rapid advancement of technology, the scale of integrated circuits (ICs) has expanded exponentially. 
This expansion has introduced significant challenges in chip manufacturing, particularly concerning power and area metrics.
A primary objective in IC design is achieving the same circuit function with fewer transistors, thereby reducing power usage and area occupancy.

Logic synthesis~\cite{hachtel2005logicsynth}, a critical step in electronic design automation (EDA), transforms behavioral-level circuit designs into optimized gate-level circuits, ultimately yielding the final IC layout. 
The primary goal of logic synthesis is to identify the physical implementation with the fewest gates for a given circuit function. 
This task constitutes a challenging NP-hard combinatorial optimization problem. 
Current logic synthesis tools~\cite{brayton2010abc, wolf2013yosys} rely on human-designed heuristics, often leading to sub-optimal outcomes.

Differentiable architecture search (DAS) techniques~\cite{liu2018darts, chu2020darts} offer novel perspectives on addressing challenges in this problem.
Circuit functions can be represented through truth tables, which map binary inputs to their corresponding outputs. 
Truth tables provide a precise representation of input-output relationships, ensuring the design of functionally equivalent circuits.
Inspired by this, researchers~\cite{deepmind2024ai4sys, wang2024tnet} have begun exploring the application of DAS to synthesize circuits directly from truth tables.
Specifically, \citet{deepmind2024ai4sys} proposed CircuitNN, a framework that learns differentiable connection structures with logic gates, enabling the automatic generation of logic circuits from truth tables.
This approach significantly reduces the complexity of traditional circuit generation. 
Building on this, \citet{wang2024tnet} introduced T-Net, a triangle-shaped variant of CircuitNN, incorporating regularization techniques to enhance the efficiency of DAS.

Despite these advancements, several challenges remain. 
The computational complexity of DAS grows quadratically with the number of gates, posing scalability issues.
Although triangle-shaped architecture~\cite{wang2024tnet} partially mitigates this problem, redundancy persists. 
%Additionally, DAS is susceptible to converging to local optima, limiting the ability to search architectures that satisfy the given truth tables~\cite{liu2018darts}. 
%Furthermore, hyperparameters (network depth and layer width) require extensive searches, introducing complexity and prolonging the synthesis process. 
Additionally, DAS is susceptible to converging to local optima~\cite{liu2018darts} and hyperparameters (network depth and layer width) require extensive searches. 
The challenges arise from the vast search space in DAS. 
% Even with predefined settings for CircuitNN, finding a configuration that meets the truth table requires extensive trial and error during the DAS process. 
Intuitively, limiting the search space through predefined parameters (network depth, gates per layer, and connection probabilities) can significantly reduce the complexity.

Recent advances~\cite{openai2023gpt4, abramson2024alphafold3, esser2024sd3, li2024mar} in conditional generative models have demonstrated remarkable performance across language, vision, and graph generation tasks. 
Motivated by these developments, we propose a novel approach to circuit generation that generates preliminary circuit structures to guide DAS in generating refined circuits matching specified truth tables. 
Firstly, we introduce CircuitVQ, a tokenizer with a discrete codebook for circuit tokenization. 
Built upon our Circuit AutoEncoder framework~\cite{hou2022graphmae,li2023maskgae,wu2025mgvga}, CircuitVQ is trained through a circuit reconstruction task. 
Specifically, the CircuitVQ encoder encodes input circuits into discrete tokens using a learnable codebook, while the decoder reconstructs the circuit adjacency matrix based on these tokens.
Subsequently, the CircuitVQ encoder serves as a circuit tokenizer for CircuitAR pretraining, which employs a masked autoregressive modeling paradigm~\cite{chang2022maskgit, li2023mage}. 
In this process, the discrete codes function as supervision signals. 
After training, CircuitAR can generate discrete tokens progressively, which can be decoded into initial circuit structures by the decoder of the CircuitVQ. 
These prior insights can guide DAS in producing refined circuits that match the target truth tables precisely.

Our key contributions can be summarized as follows:
\begin{itemize}
\item We introduce CircuitVQ, a circuit tokenizer that facilitates graph autoregressive modeling for circuit generation, based on our Circuit AutoEncoder framework;
\item Develop CircuitAR, a model trained using masked autoregressive modeling, which generates initial circuit structures conditioned on given truth tables;
\item Propose a refinement framework that integrates differentiable architecture search to produce functionally equivalent circuits guided by target truth tables;
\item Comprehensive experiments demonstrating the scalability and capability emergence of our CircuitAR and the superior performance of the proposed circuit generation approach.
\end{itemize}

% Motivation
% (a) Diffusion (Vision, Graph), Autoregressive (Language, Vision)
% (b) Circuit Generation for Predefined Setting
% (c) Neural Architecture Search for Strict Logic Equivalence

% Contribution
% (a) Circuit Tokenizer (new transformer arch, training strategy)
% (b) CircuitAR (train and gen strategies, post-ar strategy)
% (c) Extensive Evaluation including BitD (Bit Distance) for Scalability


\section{Neural model}
In this work, we tackle the task of raising the listened music's masking thresholds to better mask the listener's surrounding noise.
The proposed model Deep Perceptual Noise Masking with Music (DPNMM) is based on a deep neural network that predicts the frequency responses of filters to apply to the music to achieve the desired masking effect. A schematic overview of the system is shown in Fig. \ref{fig:system_overview}. The neural network takes as input frequency features derived from both the noise and the music. Their respective power spectral densities (PSDs) are computed using sliding Hanning windows of length $N_{win}=$ 2048 points with a 75\% overlap and a sampling rate $f_s=$ 44100 Hz.  
Both are mapped to the psychoacoustic Bark scale \cite{zwickerPsychoacousticsFactsModels2010} resulting in the PSDs $P^{noise}(n, \nu)$ and $P^{music}(n, \nu)$, where $n$ denotes the frame index and $\nu$ the Bark band. 

The music masking thresholds $T(n, \nu)$ per Bark band are computed using Johnston's model \cite{johnstonTransformCodingAudio1988} as in Estreder's system \cite{estrederPerceptualAudioEqualization2018}. The calculation involves applying a \textit{spreading function} to the PSD in each critical band, which indicates how it masks signals within the same band and in adjacent bands. The contributions from all Bark bands are then summed, following the additivity property of simultaneous masking \cite{lutfiAdditivitySimultaneousMasking1983, johnstonTransformCodingAudio1988, humesModelsAdditivityMasking1989}, and adjusted downward by an offset that depends on whether the music is more tonal or noise-like.

\subsection{Architecture}
\begin{figure}
    \centering
    \includegraphics[width=\linewidth]{figures/system_overview.png}
    \caption{Overview of the proposed system. Bark features from the music and noise signals are computed: PSD per Bark band for both music and noise and music masking thresholds. The features are fed to the U-Net that outputs gains in dB used to scale filter frequency responses. They are applied in the spectral domain to the music and a processed version is generated by inverse STFT.}\vspace{-0.3cm}
    \label{fig:system_overview}
\end{figure}

Our model's architecture is illustrated in Fig. \ref{fig:u-net_architecture}. We design a U-Net encoder-decoder model, inspired by the DeepFilterNet architecture \cite{schroterDeepfilternetLowComplexity2022}, especially its Equivalent Rectangular Bandwidth (ERB) gains prediction branch. The encoder consists of 4 convolutional blocks (separable convolution + BatchNorm + ReLU) followed by a linear layer, designed to process frequency information between the Bark features. 
%while preserving the temporal structure. 
This frequency information is then passed through a Gated Recurrent Unit (GRU) layer to handle the temporal dynamics. The decoder mirrors the encoder and incorporates skip connections, outputting the predicted gains in dB, $g(n,\nu)$, per frame, and Bark band (up to the 24th band $\sim$ 16500 Hz). 
Additionally, a gain smoothing filter is applied to prevent too rapid variations in the filters' frequency responses over time, and the gains are finally clamped as in \cite{estrederPerceptualAudioEqualization2018} with the following threshold values:
\begin{equation}
    g(n, \nu) \leftarrow \max \big( \min( g(n, \nu); -5 \text{ dB} ) ; 10 \text{ dB} \big).
\end{equation} 
\vspace{-0.5cm}

\begin{figure}
    \centering
    \includegraphics[width=0.9\linewidth]{figures/u-net_architecture.png}
    \caption{U-Net architecture of the proposed model. $N$ is the number of time frames, and 26 the number of Bark bands. The encoder is composed of 4 convolutional layers (Conv), a linear layer and a Gated Recurrent Unit (GRU) layer. The decoder follows the inverse path with transposed convolutional layers (TConv) and pathway convolutions (PConv) as add-skip connections.}%\vspace{-0.3cm}
    \label{fig:u-net_architecture}
\end{figure}

\subsection{Filters frequency responses}

The obtained gains are converted into filter frequency responses
that are generated using a pattern $W_{dB}^\nu(f)$ centered on the corresponding Bark band $\nu$ (see Fig. \ref{fig:pattern_filter}). In each band, the pattern is constant and equal to 1. 
To simulate the spreading effect of masking across the Bark bands, the pattern transitions smoothly with a cosine shape across the two adjacent bands below (from 0 to 1) and above (from 1 to 0). For the lowest frequency band, the pattern level is set to 2 instead of 1 at the center. This approach accounts for the fact that adjacent bands also contribute to raising the masking threshold of a given band, thereby distributing the required gain across multiple bands and minimizing the boost needed on the central band. %alone.

\begin{figure}
    \centering
    \includegraphics[width=\linewidth]{figures/pattern_filter.pdf}
    \caption{Pattern $W_{dB}^\nu(f)$ used to shape the filters' frequency responses, for $\nu =$ 5.}\vspace{-0.3cm}
    \label{fig:pattern_filter}
\end{figure}
For each audio frame the overall frequency response $\mathcal{W}_{dB}(n,k)$ is obtained as:
\begin{equation}
    \mathcal{W}_{dB}(n,k) = \sum_{\nu=1}^{B} g(n,\nu) \cdot W_{dB}^\nu(k).
\end{equation}
The input music is then filtered, in the frequency domain, using those frequency responses frame by frame to produce a processed music whose masking properties are enhanced.

\subsection{Loss functions}
Our model DPNMM is trained with a primary loss function designed to raise the music's masking threshold above the noise level in the critical bands where needed. These Bark bands are constrained to exceed specific thresholds, while the energy in other bands is free to change as long as their masking threshold remains above the noise level, thereby supporting masking in adjacent bands: 
\begin{equation}
    \mathcal{L}_0 (\theta_t) = \frac{1}{N} \frac{1}{B}  \sum_{n,\nu} \text{ReLU}\left( P_{dB}^{noise}(n,\nu) - \hat{T}_{dB}(n,\nu) \right) ,
\end{equation}
where $\hat{T}_{dB}(n,\nu)$ is the masking threshold computed with the processed music, $N$ the number of time frames and $B$ the number of Bark bands.
By using a ReLU function to express the constraint, we only set a minimum threshold level for the network to reach, allowing it greater flexibility to find solutions for masking noise across all frequency bands. However, while this choice aims to provide the network with more freedom, it also brings the challenge that the system is not required to output zero gains when no amplification of the music is needed. To address this, we use the knowledge of the masking spreading effect  to compute a mask that identifies, for each band, whether it is close enough to another band (including itself) where the threshold needs to be raised to have an impact on it. If it is not, the gain for that band at the network's output is set to zero.

Even with this gain masking, the system still has an infinite number of potential solutions. To guide the learning process in a desired direction, we add a secondary constraint aimed at preserving the naturalness of the original music by limiting the average power variation: 
\begin{equation}
    \mathcal{L}_{power}(\theta_t) = \frac{1}{N} \sum_n \vert \hat{\mathcal{P}}_{dBA}^{music}(n) - \mathcal{P}_{dBA}^{music}(n) \vert \; ; 
\end{equation}
where the initial music mean power $\mathcal{P}$ of frame $n$ is evaluated in dBA \cite{a_weights}, as well as the processed music mean power $\hat{\mathcal{P}}$. To include this constraint in the training process we use a strategy inspired by the method of multipliers \cite{jonasdegraveHowWeCan2021, plattConstrainedDifferentialOptimization1987}. The goal is to ensure that during training the power variation does not exceed a given value $\Delta \mathcal{P}_{max}$ using a dynamic weight $\lambda_t$ to scale the loss. The overall loss function for this constrained optimization problem is:
\begin{equation}
    L(\theta_t, \lambda_t) = \mathcal{L}_0 - \lambda_t \cdot (\Delta \mathcal{P}_{max} - \mathcal{L}_{power}(\theta_t)).
\end{equation}
At the start of the training, $\lambda_t = 0$. While gradient descent is applied to the overall loss $L(\lambda_t)$, gradient ascent is simultaneously performed on $\lambda_t$ using the gradient $\frac{\partial L(\theta, \lambda_t)}{\partial \lambda_t} = \mathcal{L}_{power}(\theta_t) - \Delta \mathcal{P}_{max}$ with a specific learning rate of $10^{-3}$, keeping $\lambda_t$ always positive. When $\mathcal{L}_{power}(\theta_t)$ exceeds the threshold $\Delta \mathcal{P}_{max}$, $\lambda_t$ increases, giving more weight to the power constraints in the total loss. Conversely, when $\mathcal{L}_{power}(\theta_t)$ drops below $\Delta \mathcal{P}_{max}$, $\lambda_t$ decreases.
This approach allows us to guide the training in a direction that satisfies both constraints without requiring a tedious search for an optimal fixed weight. The degree of compromise between the two constraints is directly controlled by the choice of parameter $\Delta \mathcal{P}_{max}$. In the rest of the article, we explore several values for this parameter. 

The code for the implementation of DPNMM is available on github.\footnote{\href{https://github.com/ClementineBerger/DPNMM}{https://github.com/ClementineBerger/DPNMM}}

\section{Data}
To train and evaluate the proposed model, we generated training, validation, and test datasets that replicate realistic acoustic scenes. These scenes \secor{consist of music listened to by a user through headphones or earphones and ambient noise as perceived by the user through their earphones}{simulate a user listening to music through headphones or earphones while being in a noisy environment}.  

We defined several \textit{environments} to represent a variety of realistic acoustic scenes with different ambient noise levels: urban, indoor office, construction site, beach, transportation (train/plane/boat), and restaurant/bar. 
\se{We chose} the noise recordings from the DNS Challenge dataset \cite{dubeyIcassp2022Deep2022}.
Each environment is \se{created using} the labels from the noise dataset and a realistic noise level distribution in dBA. Noise samples are evenly selected per environment and normalized to levels sampled from the corresponding noise distribution. A pre-processing step is applied to drop the audio signals composed mainly of silence (which are therefore isolated, impulsive noises). After that, all those samples are filtered with one of three headphone frequency responses to reproduce their passive attenuation. 
Each noise sample is then paired with a music track \se{chosen from the FMA dataset \cite{defferrardFMADATASETMUSIC2017}} which covers a large diversity of music genres (Pop, Rock, Classical, Jazz, Hip-Hop, etc.).
Each music is normalized to a dBA level derived from a Signal-to-Noise Ratio (SNR) value, itself sampled from a defined SNR distribution. The resulting music dBA level is constrained within the \secor{45 dBA to 100 dBA range}{range $[45, 100]$ dBA,} reflecting the typical range offered by standard headphones.
\se{We thus generate} 50h of training data, 20h of validation data, and 10h of test data, composed of pairs of 10s mono music and noise excerpts sampled at 44100 Hz.

\section{Results}
\begin{table}[ht!]
\centering
\caption{\textbf{Super Resolution Performance Results.} Our proposed WGAN EEG Spatial Upsampling method significantly outperforms a baseline of Bicubic Interpolation commonly used in EEG upsampling pipelines.}
\label{tab:results}
\resizebox{0.8\linewidth}{!}{%
\begin{tabular}{@{}cccccc@{}}
\toprule
\multirow{2}{*}{\textbf{Dataset}} & \multirow{2}{*}{\textbf{Scale}} & \multicolumn{2}{c}{\textbf{Bicubic}} & \multicolumn{2}{c}{\textbf{WGAN}} \\ \cmidrule(l){3-6} 
                      &   & \textbf{MSE} & \textbf{MAE} & \textbf{MSE}    & \textbf{MAE}   \\
\toprule
\multirow{2}{*}{Val}  & 2 & 3.71E7       & 3.89E3       & \textbf{2.01E3} & \textbf{24.38} \\
                      & 4 & 7.23E7       & 6.42E3       & \textbf{8.53E3} & \textbf{63.83} \\
\midrule
\multirow{2}{*}{Test} & 2 & 3.75E7       & 3.91E3       & \textbf{2.06E3} & \textbf{24.66} \\
                      & 4 & 7.30E7       & 6.45E3       & \textbf{8.68E3} & \textbf{64.39} \\
\bottomrule
\end{tabular}%
}
\end{table}

\section{Conclusions}
\section{Conclusion Remarks}
This work proposes a RBG graph model for disease spreading via hubs. We study the joint effect of the agent density, hub density, and connection function. The existence of a critical hub density depends only on the boundedness of the support of the connection function, which relates to curbing the traveling distance of individuals. When it comes to dispersion, both the degree distribution and the percolation threshold suggest that increasing dispersion helps spread the disease. The percolation properties of RBG graphs relate to unipartite graphs with modified connection functions. 
An interesting question in this direction is if and when the properties of the RBG graphs can be well represented by unipartite graphs with some modified connection functions. Our conjecture is that for independent connections between different pairs of agents, such representation is unlikely due to the oblivion of the local dependence (present in the RBG models). 
 Another direction is to consider hybrid models where agents may get infected either through common hubs or direct interactions between agents. The former infection mechanism is more centralized than the latter. 

%\section*{Acknowledgment}

%The preferred spelling of the word ``acknowledgment'' in America is without 
%an ``e'' after the ``g''. Avoid the stilted expression ``one of us (R. B. 
%G.) thanks $\ldots$''. Instead, try ``R. B. G. thanks$\ldots$''. Put sponsor 
%acknowledgments in the unnumbered footnote on the first page.

\bibliographystyle{IEEEtran}
%\bibliography{IEEEabrv,bibliography}    
\bibliography{biblio}    

\end{document}
