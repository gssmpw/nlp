\subsection{Baseline and evaluation metrics}

We use as a baseline the perceptual equalizer used by Estreder et al. \cite{estrederPerceptualAudioEqualization2018} that is based on a Bark band graphic equalizer employing second-order peak filters \cite{abelFilterDesignUsing2004a, valimakiAllAudioEqualization2016}.
The gains per Bark band are computed in order to raise the music level in the bands where the noise PSD is above its masking threshold,
\begin{equation}
    g(n,\nu) = \max \left( P_{dB}^{noise}(n,\nu) - T_{dB}(n,\nu); 0 \right) .
    \label{eq:estreder}
\end{equation}

This computation approach does not consider the additivity property of simultaneous masking.  The level in the $\nu$-th Bark band is raised to bring the masking threshold at the noise level by adjusting only this band while raising the levels in adjacent bands would also contribute to increasing the masking threshold in the target band via the effect of the spreading function.

Two objective metrics are considered to perform the evaluation of the models. We compute a mean Noise-to-Mask Ratio (NMR) per audio sample of the test set, only selecting the Bark bands where the initial music masking threshold is below the noise level :
\begin{equation}
\text{NMR} = \frac{1}{M} \sum_{n, \nu} (1-m_\nu(n)) | P_{dB}^{noise}(n,\nu) - \hat{T}_{dB}(n,\nu) |,
\end{equation} 
where $M = \sum_{n, \nu} (1-m_\nu(n))$ with $m_\nu$ a mask such that $m_\nu(n) = 0$ if the initial threshold is below the noise, and $m_\nu(n) = 1$ otherwise. The obtained NMR is compared to the initial NMR with the unprocessed music to evaluate how much the system can improve the masking effect on the bands where it is required. However, the system may as well induce power variations in the other bands. To evaluate this effect we also compute a mean Global Level Difference (GLD) :

\begin{equation}
    \text{GLD} = \frac{1}{N} \sum | \hat{\mathcal{P}}_{dBA}^{music}(n) - \mathcal{P}_{dBA}^{music}(n) | .
\end{equation}
Both metrics are computed by frequency \secor{bands}{ranges}: broadband, first \secor{tier}{third} of Bark bands (low), second third (medium), and last third (high). 

\subsection{\se{Discussion}}
We evaluate our approach by testing different configurations of the power constraint: no power constraint, $\Delta \mathcal{P}_{max} =$ 2, 1, and 0.5 dB. In all configurations, the model is trained for 50 epochs with a learning rate of $10^{-3}$ and a batch size of 64. The statistical difference between each of DPNMM configurations and the baseline model is evaluated by calculating the mean \textit{p}-value with the Wilcoxon test over 100 batches of 50 samples from the test set, with a Bonferroni correction applied.
The results are presented in Fig. ~\ref{fig:results}. 

\begin{figure}
\centering
\begin{subfigure}{\linewidth}
    \includegraphics[width=\textwidth]{figures/nmr.pdf}
    \label{fig:nmr}
\end{subfigure}
%\hfill
\begin{subfigure}{\linewidth}
    \includegraphics[width=0.98\textwidth]{figures/gld.pdf}
    \label{fig:gld}
\end{subfigure}
        
\caption{Obtained NMR and GLD on the test set for the Estreder model and four versions of the proposed neural model with different degrees of power constraint during training : no constraint and $\Delta \mathcal{P}_{max} = 2, 1, 0.5$ dBA.}\vspace{-0.3cm}
\label{fig:results}
\end{figure}



In terms of NMR, all three versions of the neural model outperform Estreder's model on the broadband metric statistically significantly, except DPNMM with $\Delta \mathcal{P}_{max} = 0.5$ dBA. The version of the neural model without any power constraint performs the best compared to the baseline (\textit{p}-value $= 7\cdot 10^{-8}$). Applying a power constraint results in a decrease in performance all the more important the stricter the constraint (low $\Delta \mathcal{P}_{max}$), particularly in the low-frequency range to the point of becoming less performant than Estreder's PEQ. This outcome is expected, given the relatively low weight of high frequencies in the power measurement. When the power constraint is strict, the low and mid frequencies are more significantly affected. This trend is confirmed when examining the GLD. Without a power constraint, the neural model achieves excellent NMR performance by significantly amplifying the musical signal compared to Estreder's model. Adding the power constraint has then a clear beneficial effect on the GLD measure, \secor{even allowing}{thus achieving significantly} better results compared to the baseline model\secmt{you need to do statistical testing and provide p-value; use Wicoxon signed rank, see mattermost}, except at high frequencies where the model is less affected by the constraint. In particular, both neural models with constraints $\Delta \mathcal{P}_{max} =$ 2, 1 dBA achieve a better NMR than Estreder's model (\textit{p}-value of $10^{-6}$ and 0.01) and a better broadband GLD (\textit{p}-value of $1.5 \cdot 10^{-4}$ and $3.3 \cdot 10^{-9}$).

Audio examples and further results are available on this article's companion website.\footnote{\href{https://clementineberger.github.io/DPNMM/}{https://clementineberger.github.io/DPNMM/}}