    % This must be in the first 5 lines to tell arXiv to use pdfLaTeX, which is strongly recommended.
\pdfoutput=1
% In particular, the hyperref package requires pdfLaTeX in order to break URLs across lines.

\documentclass[11pt]{article}

% Change "review" to "final" to generate the final (sometimes called camera-ready) version.
% Change to "preprint" to generate a non-anonymous version with page numbers.
\usepackage[preprint]{acl}

% Standard package includes
\usepackage{times}
\usepackage{latexsym}
\usepackage{amsmath} 
\usepackage{amssymb}
\usepackage{bm}

% For proper rendering and hyphenation of words containing Latin characters (including in bib files)
\usepackage[T1]{fontenc}
% For Vietnamese characters
% \usepackage[T5]{fontenc}
% See https://www.latex-project.org/help/documentation/encguide.pdf for other character sets

% This assumes your files are encoded as UTF8
\usepackage[utf8]{inputenc}

% This is not strictly necessary, and may be commented out,
% but it will improve the layout of the manuscript,
% and will typically save some space.
\usepackage{microtype}

% This is also not strictly necessary, and may be commented out.
% However, it will improve the aesthetics of text in
% the typewriter font.
\usepackage{inconsolata}

%Including images in your LaTeX document requires adding
%additional package(s)
\usepackage{graphicx}
\usepackage{booktabs}
\usepackage{multirow}
\usepackage{subcaption}


\usepackage[most]{tcolorbox}

\usepackage{listings}
\usepackage{setspace}
\usepackage{color}

\definecolor{isarblue}{HTML}{006699}
\definecolor{isarfaintblue}{rgb}{0.0, 0.75, 1.0}
\definecolor{isargreen}{HTML}{009966}
\definecolor{red}{HTML}{990000}
\definecolor{patriarch}{rgb}{0.5, 0.0, 0.5}

\lstdefinelanguage{isabelle}{%
    keywords=[1]{theory,type_synonym,datatype,fun,abbreviation,definition,proof,lemma,theorem,qed,corollary,have,hence,also,finally,ultimately,moreover,using,\{},
    keywordstyle=[1]\bfseries\color{isarblue},
    keywords=[2]{where,assumes,shows,fixes,and,begin,end,imports},
    keywordstyle=[2]\bfseries\color{isargreen},
    keywords=[3]{if,then,else,case,SOME,let,in,O},
    keywordstyle=[3]\color{isarblue},
    keywords=[4]{ATP},
    keywordstyle=[4]\it\color{patriarch},
    keywords=[5]{show,assume,obtain},
    keywordstyle=[5]\bfseries\color{isarfaintblue},
    keywords=[6]{<proof>},
    keywordstyle=[6]\color{yellow},
}


\lstdefinestyle{isabelle}{%
  language=isabelle,
  escapeinside={&}{&},
  columns=fixed,
  extendedchars,
  basewidth={0.5em,0.45em},
  basicstyle=\singlespacing\ttfamily\small,
  mathescape,
  morecomment=[s][\bfseries\color{red}]{(*}{*)},
  morecomment=[l][\bfseries]{####},
}

\lstset{
    language=isabelle,
    mathescape=true,
    escapeinside={--"}{"},
    basicstyle={\itshape},
    keywordstyle=\rm\ttfamily\fontseries{b}\selectfont,
    keywordstyle=[2]\rm\ttfamily\fontseries{m}\selectfont,
    keywordstyle=[3]\rm,
    keywordstyle=[4]\rm,
    showstringspaces=false,
    keepspaces=true,
    columns=[c]fullflexible}
\lstset{literate=
  {'}{{${}^\prime\!$}}1
  {\\<^sup>*}{{$^*$}}1
  {\\<^sub>*}{{$_*$}}1
  {\\<^sub>A}{{$_A$}}1
  {\\<^sub>M}{{$_M$}}1
  {\\<^sub>r}{{$_r$}}1
  {\\<^sub>a}{{$_a$}}1
  {\\<A>}{{$\mathcal{A}$}}1
  {\\<O>}{{\sf o}}1
  {\\<lambda>}{{$\lambda$}}1
  {\\<Lambda>}{{$\Lambda$}}1
  {\\<phi>}{{$\phi$}}1
  {\\<Phi>}{{$\Phi$}}1
  {\\<psi>}{{$\psi$}}1
  {\\<Psi>}{{$\Psi$}}1
  {\\<theta>}{{$\theta$}}1
  {\\<Theta>}{{$\Theta$}}1
  {\\<sigma>}{{$\sigma$}}1
  {\\<Sigma>}{{$\Sigma$}}1
  {\\<gamma>}{{$\gamma$}}1
  {\\<Gamma>}{{$\Gamma$}}1
  {\\<alpha>}{{$\alpha$}}1
  {\\<beta>}{{$\beta$}}1
  {\\<omega>}{{$\omega$}}1
  {\\<cdot>}{{$\cdot$}}1
  {\\<in>}{{$\in$}}1
  {\\<le>}{{$\le$}}1
  {\\<ge>}{{$\ge$}}1
  {\\<noteq>}{{$\ne$}}1
  {\\<longrightarrow>}{{$\longrightarrow$}}1
  {\\<longleftrightarrow>}{{$\longleftrightarrow$}}1
  {\\<Rightarrow>}{{$\Rightarrow$}}1
  {\\<Longrightarrow>}{{$\Longrightarrow$}}1
  {\\<rightarrow>}{{$\rightarrow$}}1
  {\\<leftarrow>}{{$\leftarrow$}}1
  {\\<mapsto>}{{$\mapsto$}}1
  {\\<leftrightarrow>}{{$\leftrightarrow$}}1
  {\\<equiv>}{{$\equiv$}}1
  {\\<and>}{{$\wedge$}}1
  {\\<or>}{{$\vee$}}1
  {\\<And>}{{$\bigwedge$}}1
  {\\<Up>}{{$\Uparrow$}}1
  {\\<Down>}{{$\Downarrow$}}1
  {\\<Union>}{{$\bigcup$}}1
  {\\<up>}{{$\uparrow$}}1
  {\\<down>}{{$\downarrow$}}1
  {\\<times>}{{$\times$}}1
  {\\<forall>}{{$\forall$}}1
  {\\<exists>}{{$\exists$}}1
  {\\<nexists>}{{$\nexists$}}1
  {\\<union>}{{$\cup$}}1
  {\\<inter>}{{$\cap$}}1
  {\\<subset>}{{$\subset$}}1
  {\\<subseteq>}{{$\subseteq$}}1
  {\\<supset>}{{$\supset$}}1
  {\\<supseteq>}{{$\supseteq$}}1
  {\\<langle>}{{$\langle$}}1
  {\\<rangle>}{{$\rangle$}}1
  {\\<not>}{{$\neg$}}1
  {\\<box>}{{$\oblong$}}1
  {\\<bot>}{{$\bot$}}1
  {\\<top>}{{$\top$}}1
  {\\<notin>}{{$\notin$}}1
  {\\<guillemotright>}{{$\gg$}}1
  {\\<open>}{{\rm\guilsinglleft}}1
  {\\<close>}{{\rm\guilsinglright}}1
  {\\<integral>}{{$\int$}}1
  {\\<partial>}{{$\partial$}}1
  {\\<Sum>}{{$\sum$}}2
  {⇒}{{$\Rightarrow$}}1
  {⟷}{{$\leftrightarrow$}}1
  {∈}{{$\in$}}1
  {∧}{{$\wedge$}}1
  {∀}{{$\forall$}}1
  {λ}{{$\lambda$}}1
  {σ}{{$\sigma$}}1
  {β}{{$\beta$}}1
  {ω}{{$\omega$}}1
  {∑}{{$\Sigma$}}1
  {⟶}{{$\rightarrow$}}1
  {×}{{$\times$}}1
  {≡}{{$\equiv$}}1
  {≥}{{$\geq$}}1
  {≤}{{$\leq$}}1
  {⋃}{{$\bigcup$}}2
}
\definecolor{mybrown}{RGB}{128,64,0}

% If the title and author information does not fit in the area allocated, uncomment the following
%
%\setlength\titlebox{<dim>}
%
% and set <dim> to something 5cm or larger.

\title{
Formalizing Complex Mathematical Statements with LLMs: \\A Study on Mathematical Definitions
}

% Author information can be set in various styles:
% For several authors from the same institution:
% \author{Author 1 \and ... \and Author n \\
%         Address line \\ ... \\ Address line}
% if the names do not fit well on one line use
%         Author 1 \\ {\bf Author 2} \\ ... \\ {\bf Author n} \\
% For authors from different institutions:
% \author{Author 1 \\ Address line \\  ... \\ Address line
%         \And  ... \And
%         Author n \\ Address line \\ ... \\ Address line}
% To start a separate ``row'' of authors use \AND, as in
% \author{Author 1 \\ Address line \\  ... \\ Address line
%         \AND
%         Author 2 \\ Address line \\ ... \\ Address line \And
%         Author 3 \\ Address line \\ ... \\ Address line}

\author{
  \textbf{Lan Zhang\textsuperscript{1}},
  \textbf{Marco Valentino\textsuperscript{2}},
  \textbf{Andr\'e Freitas\textsuperscript{1,2,3}}\\
  \textsuperscript{1}Department of Computer Science, University of Manchester, United Kingdom\\
  \textsuperscript{2}Idiap Research Institute, Switzerland\\
  \textsuperscript{3}National Biomarker Centre, CRUK Manchester Institute, United Kingdom\\
  \texttt{lan.zhang-6@postgrad.manchester.ac.uk}\\
  \texttt{\{marco.valentino, andre.freitas\}@idiap.ch}
}

%\author{
%  \textbf{First Author\textsuperscript{1}},
%  \textbf{Second Author\textsuperscript{1,2}},
%  \textbf{Third T. Author\textsuperscript{1}},
%  \textbf{Fourth Author\textsuperscript{1}},
%\\
%  \textbf{Fifth Author\textsuperscript{1,2}},
%  \textbf{Sixth Author\textsuperscript{1}},
%  \textbf{Seventh Author\textsuperscript{1}},
%  \textbf{Eighth Author \textsuperscript{1,2,3,4}},
%\\
%  \textbf{Ninth Author\textsuperscript{1}},
%  \textbf{Tenth Author\textsuperscript{1}},
%  \textbf{Eleventh E. Author\textsuperscript{1,2,3,4,5}},
%  \textbf{Twelfth Author\textsuperscript{1}},
%\\
%  \textbf{Thirteenth Author\textsuperscript{3}},
%  \textbf{Fourteenth F. Author\textsuperscript{2,4}},
%  \textbf{Fifteenth Author\textsuperscript{1}},
%  \textbf{Sixteenth Author\textsuperscript{1}},
%\\
%  \textbf{Seventeenth S. Author\textsuperscript{4,5}},
%  \textbf{Eighteenth Author\textsuperscript{3,4}},
%  \textbf{Nineteenth N. Author\textsuperscript{2,5}},
%  \textbf{Twentieth Author\textsuperscript{1}}
%\\
%\\
%  \textsuperscript{1}Affiliation 1,
%  \textsuperscript{2}Affiliation 2,
%  \textsuperscript{3}Affiliation 3,
%  \textsuperscript{4}Affiliation 4,
%  \textsuperscript{5}Affiliation 5
%\\
%  \small{
%    \textbf{Correspondence:} \href{mailto:email@domain}{email@domain}
%  }
%}

\begin{document}
\maketitle
\begin{abstract}
Thanks to their linguistic capabilities, LLMs offer an opportunity to bridge the gap between informal mathematics and formal languages through \emph{autoformalization}. However, it is still unclear how well LLMs generalize to sophisticated and naturally occurring mathematical statements. To address this gap, we investigate the task of autoformalizing real-world \emph{mathematical definitions} -- a critical component of mathematical discourse. Specifically, we introduce two novel resources for autoformalisation, collecting \emph{definitions} from Wikipedia (Def\_Wiki) and arXiv papers (Def\_ArXiv). We then systematically evaluate a range of LLMs, analyzing their ability to formalize definitions into Isabelle/HOL. Furthermore, we investigate strategies to enhance LLMs' performance including \emph{refinement through external feedback} from Proof Assistants, and \emph{formal definition grounding}, where we guide LLMs through relevant contextual elements from formal mathematical libraries.
Our findings reveal that definitions present a greater challenge compared to existing benchmarks, such as miniF2F. In particular, we found that LLMs still struggle with self-correction, and aligning with relevant mathematical libraries. At the same time, structured refinement methods and definition grounding strategies yield notable improvements of up to 16\% on self-correction capabilities and 43\% on the reduction of undefined errors, highlighting promising directions for enhancing LLM-based autoformalization in real-world scenarios.\footnote{Code and datasets are available at \url{https://github.com/lanzhang128/definition_autoformalization}}
%At the same time, structured refinement methods and definition grounding strategies yield notable improvements, highlighting promising directions for enhancing LLM-based autoformalization in real-world scenarios.
\end{abstract}

\section{Introduction}

Large Language Models (LLMs) have demonstrated remarkable potential in assisting with mathematical reasoning on different downstream tasks ~\citep{wei2022chain,meadows2023generating,meadows-etal-2024-symbolic,valentino-etal-2022-textgraphs,lu-etal-2023-survey,meadows2023introduction,mishra-etal-2022-lila,ferreira-etal-2022-integer,ferreira2020premise,welleck2021naturalproofs,mishra2022numglue,petersen-etal-2023-neural}. In the context of mathematics, formal languages play a crucial role by providing a precise, logic-based framework for verifying the correctness and logical validity of mathematical statements and proofs~\citep{survey2020}. 
\begin{figure}[!t]
    \centering
    \includegraphics[width=\columnwidth]{figures/intro.pdf}
    \caption{ \emph{Can LLMs formalize complex mathematical statements?} This paper investigates the task of translating \emph{real-world mathematical definitions} into a formal language. We introduce a new resource collecting definitions from \emph{Wikipedia} and \emph{ArXiv} papers,  exploring different strategies for autoformalization through the interaction between \emph{LLMs} and \emph{Proof Assistants}.}
    \label{fig:framework}
\end{figure}
Consequently, one promising application of LLMs is \emph{autoformalization}, the task of translating informal statements into formal languages~\citep{wu2022autoformalization}. Given their advanced linguistic capabilities, LLMs offer an opportunity to bridge the gap between informal mathematics, natural language, and machine-verifiable logic, potentially streamlining and scaling the process of formal mathematical reasoning~\citep{jiang2023draft,tarrach2024more}.

The task of autoformalization has garnered increasing attention in recent years, leading to the development of benchmarks and evaluation methodologies~\citep{azerbayev2023proofnet,zhang-etal-2024-consistent,li2024autoformalize}. Despite this progress, however, existing benchmarks for autoformalization often focus on relatively simple mathematical problems, limiting our understanding of how well LLMs generalize to more sophisticated and naturally occurring mathematical statements.

To address this gap, this paper investigates the task of autoformalizing \emph{mathematical definitions} -- a critical component of mathematical discourse~\citep{Moschkovich2003WhatCA}. Definitions serve as foundational building blocks in mathematical reasoning, yet they are often intricate, context-dependent, and difficult to formalize. Evaluating LLMs on this subset of mathematical statements, therefore, allows for assessing their ability to formally represent fine-grained mathematical concepts, highlighting persisting challenges and limitations for real-world applications.

%The task of autoformalization has garnered increasing attention in recent years, leading to the development of benchmarks and evaluation methodologies~\citep{azerbayev2023proofnet,zhang-etal-2024-consistent,li2024autoformalize}. Despite this progress, however, several challenges remain -- particularly in assessing how state-of-the-art LLMs perform on autoformalization \emph{in the wild}, where mathematical statements are complex, nuanced, and extracted from real-world scientific literature \textbf{(cite)}.

%Existing benchmarks for autoformalization, in fact, often focus on relatively simple mathematical problems, limiting our understanding of how well LLMs generalize to more sophisticated and naturally occurring mathematical statements. To address this gap, this paper investigates the task of autoformalizing \emph{mathematical definitions} -- a critical component of mathematical discourse~\citep{Moschkovich2003WhatCA}. Definitions serve as foundational building blocks in mathematical reasoning, yet they are often intricate, context-dependent, and difficult to formalize. Evaluating LLMs on this subset of mathematical statements, therefore, allows assessing their ability to capture and represent fine-grained mathematical concepts, highlighting persisting challenges and limitations for real-world applications.

Specifically, this paper introduces two new benchmarks for autoformalization by collecting \emph{real-world mathematical definitions} into two distinct resources:
 (1) \emph{Def\_Wiki}, including definitions extracted from Wikipedia articles, and (2) 
  \emph{Def\_ArXiv}, including definitions collected from machine learning research papers.
Using these resources, we first evaluate LLMs in a zero-shot setting, analyzing their ability to translate definitions into Isabelle/HOL~\citep{Nipkow-Paulson-Wenzel:2002}. 

Furthermore, to address observed limitations, we investigate two key strategies to enhance performance: (1) \emph{Refinement via external feedback}, investigating the self-correction capabilities of LLMs by incorporating errors found by the supporting Proof Assistant. In particular, we show that while LLMs exhibit limited ability to refine outputs based on binary feedback (error vs. non-error), a more structured categorical refinement implemented via additional instructional constraints can improve performance. (2) \emph{Formal definition grounding}. Many mathematical definitions require references to formal objects in external mathematical libraries. To guide LLMs in the process of autoformalization, we investigate the impact of introducing additional contextual control mechanisms, which add contextual elements from formal mathematical libraries as auxiliary premises.

Overall, our findings reveal that the proposed benchmarks present a greater challenge compared to existing autoformalization datasets, such as miniF2F~\citep{zheng2022miniff}. In particular, LLMs struggle with self-correction and particularly with incorporating relevant mathematical libraries as preambles. Proposed structured refinement methods and definition grounding strategies both yield notable improvements, highlighting promising directions for enhancing LLM-based autoformalization in real-world scenarios.

%In particular, LLMs struggle with self-correction and particularly with handling preambles and incorporating relevant mathematical libraries. However, structured refinement methods and definition grounding strategies both yield notable improvements, highlighting promising directions for enhancing LLM-based autoformalization in real-world scenarios.

Our contributions can be summarized as follows:

\begin{enumerate}
\item We introduce and release two novel datasets for autoformalization: Def\_Wiki (definitions from Wikipedia) and Def\_ArXiv (definitions from research papers on arXiv), designed to assess LLMs performance on complex, real-world mathematical definitions.
\item We perform a comprehensive error analysis on Isabelle/HOL, identifying key failures in formalizations generated by a range of LLMs, including GPT-4o~\citep{openai2024gpt4o}, Llama3~\citep{grattafiori2024llama3herdmodels} and DeepSeekMath~\citep{shao2024deepseekmath}.
\item We investigate refinement-based strategies, including structured feedback mechanisms from Proof Assistants and instruction-based categorical refinements.%, leading to an improvement of up to 16\% on self-correction capabilities.
\item We explore the role of formal definition grounding, investigating how the inclusion of relevant mathematical libraries impacts the ability of LLMs to connect the formalized statements with contextual mathematical elements and relevant premises. We found that definition grounding is fundamental for complex autoformalization.%, leading to a reduction in undefined errors of up to 43\%.
\end{enumerate}

%The datasets and code for reproducing our results are fully available online to support future research in the field.\footnote{\url{https://github.com/lanzhang128/formal_definition_grounding}}. 

\begin{table*}[!t]
    \tiny
    \centering
    \begin{tabular}{p{0.22\textwidth}| p{0.35\textwidth} | p{0.35\textwidth}}
        \toprule
        miniF2F & Def\_Wiki & Def\_ArXiv\\
        \midrule
        1. Suppose that $\sec x+\tan x=\frac{22}7$ and that $\csc x+\cot x=\frac mn,$ where $\frac mn$ is in lowest terms.  Find $m+n^{}_{}.$ Show that it is 044.\newline
        2. What is the sum of the two values of $x$ for which $(x+3)^2 = 121$? Show that it is -6.\newline
        3. The product of two positive whole numbers is 2005. If neither number is 1, what is the sum of the two numbers? Show that it is 406.\newline
        4. The expression $10x^2-x-24$ can be written as $(Ax-8)(Bx+3),$ where $A$ and $B$ are integers. What is $AB + B$? Show that it is 12.
        & 
        1. Definition of Rademacher Complexity: Given a set \(A\subseteq \mathbb{R}^m\), the Rademacher complexity of A is defined as follows: \[\operatorname{Rad}(A):=\frac{1}{m}\mathbb{E}_\sigma\left[\sup_{a \in A}\sum_{i=1}^m \sigma_i a_i\right]\] where \(\sigma_1, \sigma_2, \dots, \sigma_m\) are independent random variables drawn from the Rademacher distribution (i.e. \(\Pr(\sigma_i = +1) = \Pr(\sigma_i = -1) = 1/2\) for \(i=1,2,\dots,m\)), and \(a=(a_1,\dots,a_m)\).\newline 
        2. Definition of Polynomial Kernel: For degree-\(d\) polynomials, the polynomial kernel is defined as \(K(\mathbf{x},\mathbf{y}) = (\mathbf{x}^\mathsf{T} \mathbf{y} + c)^{d}\) where \(\mathbf{x}\) and \(\mathbf{y}\) are vectors of size \(n\) in the input space, i.e. vectors of features computed from training or test samples and \(c\geq 0\) is a free parameter trading off the influence of higher-order versus lower-order terms in the polynomial.
        & 
        1. Definition of Covering Number: Given a metric space $(\mathcal{S}, \rho)$, and a subset $\tilde{\mathcal{S}} \subset \mathcal{S}$, we say that a subset $\hat{\mathcal{S}}$ of $\tilde{\mathcal{S}}$ is a $\epsilon$-cover of $\tilde{\mathcal{S}}$, if $\forall \tilde{s} \in \tilde{\mathcal{S}}$, $\exists \hat{s} \in \hat{\mathcal{S}}$ such that $\rho(\tilde{s}, \hat{s}) \leq \epsilon$. The $\epsilon$-covering number of $\tilde{\mathcal{S}}$ is
        \begin{displaymath}
        \mathcal{N}_{\epsilon}(\tilde{\mathcal{S}}, \rho) = \min\{ |\hat{\mathcal{S}}|:
        \hat{\mathcal{S}} \text{ is an } \epsilon\text{-covering of } \tilde{\mathcal{S}} \} .
        \end{displaymath}\newline
        2. Definition of Trimmed Mean: Consider $n$ copies $X_1, ..., X_n$ of a heavy-tailed random variable $X$ such that $\mathbb E[X] = \mu, \mathbb E[X^{1+\varepsilon}]\leq u$ for some $\varepsilon\in(0, 1]$. The online trimmed mean, for some $\delta \in (0, 1)$ is defined as
        \begin{equation*}
        \hat{\mu}_O = \frac{1}{n}\sum_{i=1}^n X_i \bm{1}\left\{|X_i| \leq \left(\frac{ui}{\log \delta^{-1}}\right)^{\frac{1}{1+\varepsilon}}\right\}.
        \end{equation*}\\
        \bottomrule
    \end{tabular}
    \caption{Examples of instances from Def\_Wiki and Def\_ArXiv and comparison with miniF2F.}
    \label{tab:data_example}
\end{table*}

%in this research, we target the following research question: \textit{Can we guide autoformalization with LLMs better via systematically analyzing errors according to the theorem prover?} Our approach is illustrated in Figure~\ref{fig:framework}. To facilitate our research, we propose two datasets focusing on definitions rather than theorems. On these datasets, we start the error analysis on zero-shot autoformalization with modern LLMs and summarize the error patterns. From these patterns, we propose several methods and interventions named Categorical Refinement, Symbolic Refinement and Formal Definition Grounding to improve autoformalization performance and highlight existing problems in autoformalization with LLMs. In summary, the contributions of the paper are:
%\begin{enumerate}
%    \item Proposal of two datasets targeting definition autoformalization with sufficient complexity;
%    \item Error analysis on Isabelle/HOL, which can facilitate future autformalization research;
%    \item Proposal of several novel techniques for improving autoformalization performance;
%    \item Highlighting remaining problems and future directions in the area of autoformalization with LLMs.
%\end{enumerate}

\section{Autoformalization with LLMs}

\begin{table}[!t]
    \tiny
    \centering
    \begin{tabular}{c c c c}
        \toprule
        Property & miniF2F-Test & Def\_Wiki & Def\_ArXiv\\
        \midrule
        No. Samples & 244 & 56 & 30\\
        No. Tokens & 70.25(47.70) & 200.18(112.98) & 164.40(71.47)\\
        No. Objects &  4.76(1.68)& 7.63(2.71) & 7.10(2.64)\\
        No. Formulae & 2.71(1.74) & 2.84(2.05)& 3.17(1.97)\\
        \bottomrule
    \end{tabular}
    \caption{Dataset properties. The number of tokens per sample is calculated using the GPT-2 tokenizer. The number of directly mentioned mathematical objects—excluding explicit numbers and variables—and the number of mathematical formulae per sample are estimated through prompting with GPT-4o. The mean (standard deviation) is reported for each dataset.}
    \label{tab:data_property}
\end{table}


The task of autoformalization can be defined as a transformation function from natural language and LaTeX symbols $\mathcal{S}$ to a formal language $\mathcal{F}$, $f: \mathcal{S} \to \mathcal{F}$, such that for every informal mathematical statement $s \in \mathcal{S}$, there exists a formal mathematical statement $\phi \in \mathcal{F}$ where $f(s) = \phi$~\citep{zhang-etal-2024-consistent}. Autoformalization via LLMs reifies the transformation function as:
\[
f(s)=\text{LLM}(p_\text{auto},\{(s_i,\phi_i)\},s),
\]
\noindent where $p_\text{auto}$ is a prompt for autoformalization and $\{(s_i,\phi_i)\}$ is an optional set of exemplars.

\subsection{Limitations of Existing Benchmarks}

Naturally occurring mathematical statements typically involve complex and abstract mathematical concepts. However, the statements in existing datasets, such as miniF2F~\citep{zheng2022miniff}, primarily consist of basic arithmetic operations and elementary mathematical objects, such as integers, fractions, and real numbers (as shown in Table~\ref{tab:data_example}). Such mathematical objects are relatively simple compared to the complex and abstract concepts found in naturally occurring mathematical statements and scientific papers, which may involve higher-level structures like vectors, matrices, and probability. The operations are also limited to simple arithmetic, such as addition, subtraction, multiplication, division, and exponentiation. Studying autoformalization on such datasets, therefore, does not necessarily reflect the challenges of autoformalization in realistic scenarios.

In addition, the ground-truth formal code in publicly available datasets may have been exposed to LLMs whose training corpora are not disclosed. Fundamental mathematical definitions also have a high likelihood of already being formalized in theorem prover libraries. This risk raises concerns about data leakage when LLMs perform autoformalization and could lead to biased results when analyzing performance. However, few benchmarks focus on complex mathematical definitions. To address this, we propose constructing data samples of mathematical definitions in the machine learning domain, as concepts in this area are sufficiently complex and less likely to have been formalized.

%, which depend on certain \emph{preliminaries} (i.e. assumed statements). Preliminaries include axioms, definitions and lemmas. For a mathematical entity in a document, preliminaries can be categorized as \textit{explicit} or \textit{implicit}. Explicit preliminaries are ``local'' preliminaries stated within the same document, usually introduced near the relevant item. Implicit preliminaries are assumed to be known by the reader. An autoformalization system that can formalize any mathematical statement must master the advanced reasoning ability to infer and formalize preliminaries. However, the statements in existing datasets, such as miniF2F~\citep{zheng2022miniff}, often lack explicit preliminaries, and the implicit preliminaries are typically restricted to common low-level mathematical objects, such as real numbers. Studying autoformalization on these datasets, therefore, does not necessarily reflect the problem of autoformalization in realistic scenarios. %Furthermore, even state-of-the-art large language models may generate unsolicited proofs during theorem autoformalization~\citep{zhang-etal-2024-consistent}, complicating the distinction between errors arising from autoformalization and those from automated theorem proving during the error analysis. These challenges inspire us to propose two new datasets focused on novel definitions and conduct error analyses of autoformalization on them.



\subsection{Mathematical Definitions in Machine Learning Domain}
We obtain mathematical definitions in machine learning domain from two sources: Wikipedia (Def\_Wiki) and Arxiv Papers (Def\_ArXiv). Definitions from these two sources are likely to have already been validated and exhibit sufficient variety. For Def\_Wiki, definitions are from pages under the Machine Learning category\footnote{\url{https://en.wikipedia.org/wiki/Category:Machine_learning}} and its sub-categories. We manually browsed each page, identified well-defined definitions (i.e., formal descriptions with mathematical symbols), and converted the chosen definitions into LaTeX format. In total, we obtained 56 qualified natural language definitions in LaTeX and divided them into development and test sets, containing 10 and 46 samples, respectively. For Def\_ArXiv, we used the advanced search tool on ArXiv’s website, filtering for papers in the subcategories cs.LG and stat.ML, with comments including "ICML." We restricted the search to papers published in 2019, 2020, and 2021 and manually reviewed the first 25 papers from each year. We shortlisted papers that were accepted to the ICML conference and contained formally described definitions with mathematical symbols to ensure reliability. We then filtered out definitions that were less straightforward or formal in their expressions, extracted the LaTeX versions, and ultimately obtained 30 definitions from 7 papers.

Statements in definition datasets are more abstract and complex, as intuitively shown in the randomly chosen examples in Table~\ref{tab:data_example}. The data properties are summarized in Table~\ref{tab:data_property}. Although small in scale, definition datasets exhibit higher means for the number of tokens, mathematical objects, and formulae per example, indicating that they are more challenging. Additionally, definition datasets have higher standard deviations, suggesting greater diversity among the samples.

%While our benchmarks are currently small-scale, we believe they provide sufficient evidence for analyzing the current bottlenecks in autoformalization with LLMs. The process of obtaining definitions is reproducible and can be easily generalized to domains beyond machine learning for further increasing the size and variety of the benchmarks.

The data samples in our benchmarks contain only definitions in LaTeX format. We did not include ground-truth formal codes for the following reasons: 1. Including ground-truth formal codes could increase the risk of the aforementioned data leakage problem. 2. A single mathematical statement can have multiple correct formalizations. An autoformalized code that differs from the ground-truth does not necessarily indicate incorrect formalization. 3. The purpose of ground-truth formal codes is to evaluate autoformalization. However, the syntactic correctness of formalized code can be rigorously and automatically verified using theorem provers~\citep{zhang-etal-2024-consistent}, and semantic consistency can be evaluated in a reference-free manner~\citep{li2024autoformalize}. Manual inspection of autoformalized code also does not require ground-truth formal codes.

\begin{table*}[!t]
    \centering
    \tiny
    \begin{tabular}{l l c c| c c c c c}
        \toprule
        Prompt Strategy & Model & Pass$\uparrow$ & FEO$\uparrow$ & TRO$\downarrow$ & IVI$\downarrow$  & SYN$\downarrow$ & UDF$\downarrow$ & TUF$\downarrow$\\
        \midrule
        \multicolumn{9}{l}{\textbf{miniF2F-Test}}\\
        \midrule
        ZS & DeepSeekMath-7B & 3.28 & 12.79 & 18.44 & \textbf{0.00} & 50.00 & 14.34 & 9.43\\
        ZS + Binary & & 2.05 & 6.73 & 2.46 & \textbf{0.00} & 79.91 & \textbf{5.33} & \textbf{2.05}\\
        ZS & Llama3-8B & 4.92 & 20.70 & 4.51 & 0.41 & 29.51 & 38.52 & 18.85\\
        ZS + Binary & & 3.69 & 20.52 & 3.28 & 0.41 & 33.20 & 39.75 & 20.49\\
        ZS & GPT-4o & 25.41 & 48.90 & \textbf{1.23} & 1.23 & \textbf{6.15} & 23.77 & 7.38\\
        ZS + Binary & & \textbf{29.10} & \textbf{53.90} & 2.05 & 1.23 & \textbf{6.15} & 21.72 & 8.20\\
        \midrule
        \multicolumn{9}{l}{\textbf{Def\_Wiki-Test}}\\
        \midrule
        ZS & DeepSeekMath-7B & 10.87 & 17.75 & 34.78 & 2.17 & 30.43 & 26.09 & \textbf{2.17}\\
        ZS + Binary & & 6.52 & 7.73 & 8.70 & \textbf{0.00} & 69.57 & \textbf{21.74} & \textbf{2.17}\\
        ZS & Llama3-8B & 0.00 & 2.80 & \textbf{0.00} & 23.91 & 56.52 & 32.61 & 4.35\\
        ZS + Binary & & 2.17 & 3.71 & \textbf{0.00} & 26.09 & 52.17 & 30.43 & \textbf{2.17}\\
        ZS & GPT-4o & 10.87 & 16.12 & 8.70 & 8.70 & 19.57 & 50.00 & 13.04\\
        ZS + Binary & & \textbf{13.04} & \textbf{18.30} & 8.70 & 6.52 & \textbf{17.39} & 50.00 & 13.04\\
        \midrule
        \multicolumn{9}{l}{\textbf{Def\_ArXiv}}\\
        \midrule
        ZS & DeepSeekMath-7B & 13.33 & 14.69 & 16.67 & \textbf{0.00} & 40.00 & 36.67 & 13.33\\
        ZS + Binary & & 3.33 & 3.33 & 6.67 & \textbf{0.00} & 66.67 & \textbf{33.33} & \textbf{3.33}\\
        ZS & Llama3-8B & 0.00 & 2.67 & \textbf{0.00} & 13.33 & 70.00 & 40.00 & 6.67\\
        ZS + Binary & & 3.33 & 5.83 & \textbf{0.00} & 20.00 & 60.00 & \textbf{33.33} & 6.67\\
        ZS & GPT-4o & 13.33 & 19.30 & \textbf{0.00} & \textbf{0.00} & 40.00 & 56.66 & 6.67\\
        ZS + Binary & GPT-4o & \textbf{16.67} & \textbf{24.30} & \textbf{0.00} & \textbf{0.00} & \textbf{33.33} & 53.33 & 6.67\\
        \bottomrule
    \end{tabular}
    \caption{Autoformalization results. Prompt strategies include: (\textbf{ZS}): zero-shot autoformalization; (\textbf{ZS + Binary}): refinement given (zero-shot) formalized code and binary syntactic correctness state. Pass rate (\textbf{Pass}), the place of first error occurrence in the main body of the code (\textbf{FEO}), and percentage of occurrence of each error category are recorded here. Errors in each error category are: (\textbf{TRO}): Time Run-Out for checking; (\textbf{IVI}): Fake Non-Existent Theory, Invalid structural format; (\textbf{SYN}): Inner syntax error, Outer syntax error, Inner lexical error, Malformed command syntax, Bad name, Bad number of arguments for type constructor, Extra free type variable(s); (\textbf{UDF}): Undefined type names, Undeclared class, Undefined locale, No type arity list, Extra variables on rhs; (\textbf{TUF}) Type unification failed.}
    \label{tab:zs}
\end{table*}


\section{Empirical Evaluation}

\paragraph{Empirical Setup.} Isabelle/HOL is chosen as the representative formal language due to its wide adoption and support for formal mathematical reasoning. We evaluate three LLMs with different features: DeepSeekMath-7B~\citep{shao2024deepseekmath}, Llama3-8B~\citep{grattafiori2024llama3herdmodels} and GPT-4o~\citep{openai2024gpt4o}. DeepSeekMath-7B is an open-sourced LLM trained specifically for mathematics using mathematical contents from Common Crawl. As a smaller model, it has demonstrated comparable mathematical reasoning performance as in GPT-4~\citep{openai2024gpt4}, and strong few-shot autoformalization performance on miniF2F with Isabelle. This superiority makes it a good representative of smaller but specialized LLMs. LLama3-8B is a smaller open-sourced foundation LLM with no specific emphasis on math. GPT-4o is widely acknowledged as one of the state-of-the-art LLMs.
For reproducibility, greedy decoding is used for generation in all settings. 

\paragraph{Evaluation Metrics.} The success rate of passing the check by the Isabelle Proof Assistant across the tested dataset is used as the first metric. We assume that a formalized code instance with the first error occurring later in the code reflects a more complete autoformalization. Thus, we evaluate such by calculating the proportion of correct lines (up to the first error) within the main body of the code. For syntactically correct instances, this value is equal to 1. To better monitor the occurrence of errors, we group them into three categories: Syntax Errors (SYN), Undefined Item Errors (UDF), and Type Unification Failed Errors (TUF). For each category, we calculate the percentage of incorrect formalized codes caused by errors in that category.

\subsection{Zero-Shot Prompting \& Binary Refinement}

In order to understand the challenges in autoformalising mathematical definitions with LLMs, we perform a preliminary analysis on miniF2F~\citep{zheng2022miniff}, Def\_Wiki and Def\_ArXiv using zero-shot prompting (ZS) and binary refinement. With binary refinement, we aim to assess the capabilities of LLMs for error correction, providing them with the formal code generated via ZS, along with the syntactic correctness evaluated using the proof assistant (i.e., ``correct'', ``incorrect''). From the results reported in Table~\ref{tab:zs}, we can derive the following observations:

\paragraph{Def\_Wiki and Def\_ArXiv are more challenging than miniF2F.} When performing autoformalization on Def\_Wiki and Def\_ArXiv, GPT-4o achieves a significantly lower success rates (-13.78\% on average) and FEO (-31.90\% on average) compared to results on miniF2F-Test. %In addition, GPT-4o exhibits a lower percentage of errors for each category on miniF2F-Test, indicating that miniF2F is generally less challenging for modern LLMs with larger capacity.

\paragraph{LLMs can provide false preambles when performing autoformalization.} In Table~\ref{tab:zs}, we observe that the percentage of Invalid Inputs errors (IVI) can be non-zero. Errors in this category are caused by either non-existent preambles or invalid theory file formats in structure. For Llama3-8B the latter is more common whereas for GPT-4o, we observe that the dominant cause is the generation of non-existent preambles. % This is a specific type of hallucination, since it is unlikely that such false information would appear in the training data. 
This behaviour shows that LLMs do not perfectly generalize in recognizing the names of preambles.

\paragraph{Specialized smaller models can reach the same level of success rate as larger LLMs.} As a model designed specifically for mathematics, DeepSeekMath with 7B parameters can achieve a similar success rate as GPT-4o. Although Llama3-8B has a larger model size, its generalization ability on definitions is limited. Additionally, DeepSeekMath-7B exhibits a lower percentage of undefined type names errors (UDF). However, one disadvantage of the specialized model is that its formalizations have a higher percentage of time run-out issues (TRO). This is likely caused by the bias introduced during the fine-tuning phase on theorem proving which can lead the model to generate unsolicited proofs.

\paragraph{Small LLMs possess limited binary self-correction capabilities.} With binary refinement, GPT-4o produces formal codes with a higher success rate on all three datasets, whereas for DeepSeekMath-7B this mechanism leads to a performance decrease. LLama3-8B also fails to self-correct its autoformalization results on miniF2F. This behavior suggests that self-refinement exceeds the capabilities of smaller LLMs.


\subsubsection{Error Analysis \& Interventions}

\begin{figure*}[!t]
    \centering
    \begin{subfigure}{0.245\textwidth} 
      \centering
      \includegraphics[width=\textwidth]{figures/ref_Success.pdf}
      \caption{Overall Error Rate}
      \label{fig:success}
    \end{subfigure}
    \begin{subfigure}{0.245\textwidth} 
      \centering
      \includegraphics[width=\textwidth]{figures/ref_SYN.pdf}
      \caption{SYN Error Rate}
      \label{fig:syn}
    \end{subfigure}
     \begin{subfigure}{0.245\textwidth} 
      \centering
      \includegraphics[width=\textwidth]{figures/ref_UDF.pdf}
      \caption{UDF Error Rate}
      \label{fig:udf}
    \end{subfigure}
    \begin{subfigure}{0.245\textwidth} 
      \centering
      \includegraphics[width=\textwidth]{figures/ref_TUF.pdf}
      \caption{TUF Error Rate}
      \label{fig:tuf}
    \end{subfigure}
    \caption{Error rates of different refinement methods on GPT-4o. Variants include: (\textbf{ZS}): zero-shot autoformalization; (\textbf{(ZS)+Binary}): binary refinement on (zero-shot) formal codes; (\textbf{(ZS)+Detailed}): detailed refinement on (zero-shot) formal codes; (\textbf{(ZS)+CR-SYN/UDF/TUF}): plain refinement on (zero-shot) formal codes with SYN/UDF/TUF categorical refinement instructions; (\textbf{(ZS)+Detailed+CR-SYN/UDF/TUF}): detailed refinement on (zero-shot) formal codes with SYN/UDF/TUF categorical refinement instructions.}
    \label{fig:ref}
\end{figure*}

To understand potential interventions for improving autoformalization, we qualitatively analyze error patterns on the development set of Def\_Wiki. Our analysis is based on the results obtained via GPT-4o, given its better performances on ZS and binary refinement. The main reasons for failure identified through our analysis are summarized in Table~\ref{tab:failure}, with additional examples reported in Appendix.

We observe that syntactic errors (SYN) exhibit the most variety, suggesting that GPT-4o may struggle to follow syntactic rules in Isabelle/HOL if not explicitly instructed. Type unification errors (TUF) suggest that GPT-4o my struggle with the exact usage of defined Isabelle items. To improve these issues, we investigate a \textbf{Categorical Refinement} (CR) method. CR involves designing specific additive instructions that constraint the behaviors leading to errors identified in the qualitative analysis.

Similarly, for syntactic errors (SYN), causes 1, 2, and 3 in Table 4 can be addressed with rule-based algorithms that refine formal codes at the symbolic level (\textbf{Symbolic Refinement}, SR).

Undefined errors (UDF), on the other hand, indicate that although GPT-4o can refer to external formal mathematical items, it remains unaware of the location or existence of relevant libraries. To alleviate UDF errors, we propose the process of \textbf{Formal Definition Grounding} (FDG) based on two methods: 1. \textit{Postprocessing} (Post-FDG): explicitly augment preambles generated by LLMs with relevant libraries; 2. \textit{Prompting} (Prompt-FDG): provide LLMs with grounded formal items and preambles in context to guide autoformalization.

% \begin{figure*}[!t]
%     \centering
%     \includegraphics[width=\textwidth]{figures/ref.pdf}
%     \caption{Error rate of variants on (\textbf{Top}): miniF2F-test set; (\textbf{Middle}): Def\_Wiki test set; (\textbf{Bottom}): Def\_ArXiv set. Variants include: (\textbf{ZS}): zero-shot autoformalization; (\textbf{Binary}): binary refinement instructions; (\textbf{Detailed}): detailed refinement instructions; (\textbf{All}): instructions highlighting the reasons of category errors. For prompts requiring previous formalized codes, the source of such codes is marked by round brackets.}
%     \label{fig:ref}
% \end{figure*}

\subsection{Categorical Refinement}
%LLMs have demonstrated the ability to refine formalized codes with syntax error feedback to improve syntactic correctness~\citep{zhang-etal-2024-consistent}. 


In order to better understand the refinement capabilities of GPT4-o, we investigate a set of error correction strategies: (i) Plain: provide LLMs with previously generated formal codes; (ii) Binary: additionally, provide LLMs with the correctness status of the formal code; (iii) Detailed: instead of just the binary correctness status, provide LLMs with the details of type, message, and line location of individual errors in the code.

In addition, to evaluate categorical refinement, we design specific instructions for each category of errors based on our qualitative analysis (Table~\ref{tab:failure}). We report the error rate results of different refinement methods on GPT-4o in bar charts in Figure~\ref{fig:ref}. All prompts used for categorical refinement along with additional empirical results are provided in Appendix.

\paragraph{Providing LLMs with more information about individual errors is more effective than simply indicating binary correctness.} As shown in Figure~\ref{fig:success}, both binary and detailed refinements can reduce the overall error rate across all the datasets, with detailed refinement fixing more errors on miniF2F-Test and Def\_Wiki-Test. For SYN errors, although there is no clear trend indicating that one refinement outperforms the other, both refinements lead to a lower error rate compared to zero-shot autoformalization. Detailed refinement also decreases the percentage of UDF errors as shown in Figure~\ref{fig:udf}. These performance gains suggest that detailed refinement improves the quality of autoformalized codes. For TUF errors, applying both refinements does not consistently result in a lower error rate, indicating that errors in this category are more difficult for LLMs to fix.

\paragraph{Categorical refinement demonstrates superiority in reducing error rates.} As shown in Figure~\ref{fig:success}, across all datasets, the refinement method that achieves the lowest overall error rate incorporates one of the instructions for categorical refinement, highlighting its superiority. However, when categorical refinement is applied without error details, such improvements do not occur. We hypothesize that this is because categorical instructions serve as constraints, making it more difficult for the LLM to follow them without more detailed error information for individual instances. Once such information is provided, the LLM receives sufficient information to adhere to the categorical refinement instructions.

\begin{figure*}[!t]
    \centering
    \begin{subfigure}{0.25\textwidth} 
      \centering
      \includegraphics[width=\textwidth]{figures/fdg_miniF2F_zs.pdf}
      \caption{ZS on miniF2F-Test}
      \label{fig:miniF2F_zs}
    \end{subfigure}
    \begin{subfigure}{0.25\textwidth} 
      \centering
      \includegraphics[width=\textwidth]{figures/fdg_wiki_zs.pdf}
      \caption{ZS on Def\_Wiki-Test}
      \label{fig:wiki_zs}
    \end{subfigure}
    \begin{subfigure}{0.25\textwidth} 
      \centering
      \includegraphics[width=\textwidth]{figures/fdg_arxiv_zs.pdf}
      \caption{ZS on Def\_ArXiv}
      \label{fig:arxiv_zs}
    \end{subfigure}
    \begin{subfigure}{0.25\textwidth} 
      \centering
      \includegraphics[width=\textwidth]{figures/fdg_miniF2F_zs_det.pdf}
      \caption{ZS+Detailed on miniF2F}
      \label{fig:miniF2F_zs_det}
    \end{subfigure}
    \begin{subfigure}{0.25\textwidth} 
      \centering
     \includegraphics[width=\textwidth]{figures/fdg_wiki_zs_det.pdf}
      \caption{ZS+Detailed on Def\_Wiki}
      \label{fig:wiki_zs_det}
    \end{subfigure}
    \begin{subfigure}{0.25\textwidth} 
      \centering
    \includegraphics[width=\textwidth]{figures/fdg_arxiv_zs_det.pdf}
      \caption{ZS+Detailed on Def\_ArXiv}
      \label{fig:arxiv_zs_det}
    \end{subfigure}
    \caption{Gain of error rates when testing autoformalization with different methods compared to direct test. We evaluate results on zero-shot autoformalized codes and (zero-shot) formal codes with detailed refinement. Testing variants include: (\textbf{SR}): Symbolic Refinement; (\textbf{Post-FDG}): Postprocessing with Formal Definition Grounding.}
    \label{fig:fdg}
\end{figure*}

\paragraph{Categorical refinement can effectively reduce errors for specific categories.} As shown in Figure~\ref{fig:syn}, the method with the lowest SYN error rate on miniF2F-Test is plain refinement with SYN categorical refinement instructions, whereas on the other two datasets the best performing method is SYN categorical refinement with error details. In Figure~\ref{fig:udf}, UDF categorical refinement with error details also leads to the lowest UDF error rate on all three datasets. Similarly in Figure~\ref{fig:tuf}, TUF categorical refinement with error details achieves the lowest TUF error rate on two out of the three datasets. These results collectively demonstrate the effectiveness of the design of categorical refinement. The only exceptional is TUF errors on the Def\_ArXiv dataset, which again highlights the difficulty of fixing TUF errors.

% \begin{figure*}[!t]
% \begin{tcolorbox}[colback=blue!5!white,colframe=mybrown!75!black]
% \begin{small}
% \textbf{Definition of Rademacher Complexity:} Given a set \(A\subseteq \mathbb{R}^m\), the Rademacher complexity of A is defined as follows: \[\operatorname{Rad}(A):=\frac{1}{m}\mathbb{E}_\sigma\left[\sup_{a \in A}\sum_{i=1}^m \sigma_i a_i\right]\] where \(\sigma_1, \sigma_2, \dots, \sigma_m\) are independent random variables drawn from the Rademacher distribution (i.e. \(\Pr(\sigma_i = +1) = \Pr(\sigma_i = -1) = 1/2\) for \(i=1,2,\dots,m\)), and \(a=(a_1,\dots,a_m)\).
% \textbf{FDG Generated Formal Statement (GPT-4o, incorrect):}
% \vspace{-1.5em}
% \begin{lstlisting}[style=isabelle]
% theory test_8
% imports
% Main
% "HOL.Real_Vector_Spaces"
% "HOL.Real"
% "HOL-Probability.Probability_Measure"
% "HOL-Probability.Independent_Family"
% begin

% definition rademacher_distribution :: "prob_space \<Rightarrow> nat \<Rightarrow> real measure" where
%   "rademacher_distribution M m = distr M (count_space {1, -1}) (\<lambda>_. 1 / 2)"

% definition rademacher_complexity :: "real set \<Rightarrow> nat \<Rightarrow> real" where
%   "rademacher_complexity A m = 
%     (1 / real m) * 
%     (expectation (distr (count_space UNIV) (count_space {1, -1}) (\<lambda>_. 1 / 2)) 
%       (\<lambda>\<sigma>. Sup {sum (\<lambda>i. \<sigma i * a i) {1..m} | a. a \<in> A}))"

% end

% \end{lstlisting}
% \vspace{-1.5em}
% \end{small}
% \end{tcolorbox}
% \end{figure*}


% \begin{table*}[!t]
%     \centering
%     \begin{tabular}{l l c c c c}
%         \toprule
%         Prompt Strategy & Intervention & Success$\uparrow$ & SYN$\downarrow$ & UDF$\downarrow$ & TUF$\downarrow$\\
%         \hline
%         \multicolumn{6}{l}{\textbf{miniF2F-Test}}\\
%         \hline
%         ZS & None & 25.41 & 6.15 & 23.77 & 7.38\\
%         & +SR & +0.00 & +0.00 & +0.00 & +0.00\\
%         & +FDG & +41.80 & -2.87 & -20.90 & -2.05\\
%         & +FDG+SR & +41.80 & -2.87 & -20.90 & -2.05\\
%         \hline
%         (ZS) + Detailed & None & 37.30 & 5.74 & 9.02 & 8.61\\
%         & +SR & +0.00 & +0.00 & +0.00 & +0.00\\
%         & +FDG & +46.31 & -3.69 & -8.20 & -5.33\\
%         & +FDG+SR & +46.31 & -3.69 & -8.20 & -5.33\\
%         \hline
%         \hline
%         \multicolumn{6}{l}{\textbf{Def\_Wiki-Test}}\\
%         \hline
%         ZS & None & 10.87 & 19.57 & 50.00 & 13.04\\
%         & +SR & +0.00 & -4.35 & +2.17 & +0.00\\
%         & +FDG & +23.91 & +10.86 & -32.61 & +10.87\\
%         & +FDG+SR & +23.91 & +4.34 & -30.43 & +15.22\\
%         \hline
%         (ZS) + Detailed & None & 19.57 & 10.87 & 47.83 & 10.87\\
%         & +SR & +0.00 & -2.17 & +0.00 & +0.00\\
%         & +FDG & +23.91 & +10.87 & -36.96 & +13.04\\
%         & +FDG+SR & +23.91 & +6.52 & -36.96 & +17.39\\
%         \hline
%         \hline
%         \multicolumn{6}{l}{\textbf{Def\_ArXiv}}\\
%         \hline
%         ZS & None & 13.33 & 40.00 & 56.66 & 6.67\\
%         & +SR & +0.00 & -16.67 & +10.01 & +0.00\\
%         & +FDG & +10.00 & +20.00 & -43.33 & +6.66\\
%         & +FDG+SR & +10.00 & +20.00 & -43.33 & +6.66\\
%         \hline
%         (ZS) + Detailed & None & 16.67 & 36.67 & 43.33 & 16.67\\
%         & +SR & +0.00 & -13.34 & +3.34 & +3.33\\
%         & +FDG & +13.33 & +20.00 & -30.00 & -13.34\\
%         & +FDG+SR & +13.33 & +20.00 & -30.00 & -13.34\\
%         \bottomrule
%     \end{tabular}
%     \caption{Gain of error rates when evaluating with formal definition grounding.}
%     \label{tab:fdg}
% \end{table*}


\subsection{Symbolic Refinement}
Based on reasons 1 and 2 of SYN errors in Table~\ref{tab:failure}, we defined two rules for implementing Symbolic Refinement: (1) if a symbol in the formal code is likely to be an Isabelle symbol (i.e., it starts with ``\textbackslash<'' but misses ``>''), we add ``>'' at its end to ensure that the symbol follows Isabelle's format; (2) for non-existent symbols of mathematical fonts, we replace them with relevant symbols in Isabelle.

The differences in error rates between our methods and direct testing results are shown as bar charts in Figure~\ref{fig:fdg}. Detailed numbers and additional results are provided in \textit{Appendix}.

\paragraph{Symbolic Refinement can effectively reduce SYN errors in the generated formal codes on definition datasets.} In Figures~\ref{fig:wiki_zs} and \ref{fig:wiki_zs_det}, both applying SR alone and in combination with Post-FDG lead to a lower SYN error rate on Def\_Wiki-Test. On Def\_ArXiv, Figures~\ref{fig:arxiv_zs} and \ref{fig:arxiv_zs_det} similarly shows that applying SR alone results in a reduction of SYN errors. These results suggest that SR is an effective approach for addressing SYN errors. On miniF2F-Test, however, SR does not influence the error rates. This is because SR is closely tied to specific error patterns in the dataset.

\subsection{Formal Definition Grounding (FDG)}

\subsubsection{FDG via Post-Processing}
For FDG, we first extracted external formal definitions of mathematical items and their sources from the Isabelle/HOL library. Then we filtered the extracted definitions to retain only those likely relevant to the autoformalization task on the datasets. Finally, for each individual instance in Def\_Wiki and Def\_ArXiv, we manually determined which formal definitions should be provided as contextual elements. For miniF2F, we simply selected the definitions of real and complex numbers as the relevant definitions.

\noindent\textbf{Autoformalization performance can be underestimated without including contextual information.} In Figure~\ref{fig:fdg}, without modifying the main body of the formalization, replacing the preambles with possible preambles via FDG (Post-FDG) directly leads to higher overall syntactic correctness. On miniF2F-Test, this setting only considers sources containing formal definitions of real and complex numbers, yet it increases overall syntactic correctness by more than 40\%.

\paragraph{FDG can reduce the occurrence of errors caused by referring to undefined mathematical objects.} In Figure~\ref{fig:fdg}, the UDF error category has the most significant improvement from Post-FDG. Even when LLMs do not include the exact library that contains relevant mathematical items, they tend to use conventional names for the autoformalization task. By importing the appropriate theory files, these previously undefined items can be linked to the formalization, thereby reducing UDF errors.


\paragraph{Errors in autoformalized codes for definition datasets are more likely to be entangled than those in the miniF2F dataset.} In Figure~\ref{fig:miniF2F_zs} and Figure~\ref{fig:miniF2F_zs_det}, Post-FDG leads to positive performance gains across all error categories. However, in Figures~\ref{fig:wiki_zs}, \ref{fig:arxiv_zs}, \ref{fig:wiki_zs_det} and \ref{fig:arxiv_zs_det}, while UDF error rates decrease, error rates in other categories can increase. A similar trend is observed when applying SR, where a reduction in SYN errors can coincide with increases in errors from the other two categories. This phenomenon suggests that because definition datasets are more complex, LLMs are more prone to generating entangled errors during the autoformalization process.


\subsubsection{FDG via Prompting}

\begin{table}[!t]
    \centering
    \small
    \begin{tabular}{l l | c c c}
        \toprule
        Prompt Strategy & Pass$\uparrow$ & SYN$\downarrow$ & UDF$\downarrow$ & TUF$\downarrow$\\
        \midrule
        ZS & 34.78 & 30.43 & 17.39 & 23.91\\
        \midrule
        Soft-IFDC & 19.57 & 34.78 & 30.43 & 26.09\\
        \midrule
        Hard-IFDC & 19.57 & 36.96 & 21.74 & 39.13\\
        \bottomrule
    \end{tabular}
    \caption{GPT-4o Error results of Prompt-FDG on Def\_Wiki-Test with Post-FDG applied. \textbf{IFDC}: provide LLM with formal definition codes from FDG and force (\textbf{Hard}) or not force (\textbf{Soft}) LLM to use them.}
    \label{tab:wiki_fdc}
\end{table}
We designed two prompts to include external formal definitions for FDG: 1. Soft: allow the LLM some flexibility in whether to use in-context formal definitions for autoformalization; 2. Hard: explicitly instruct the LLM to use the in-context formal definitions if they are related. We tested these prompts on GPT-4o and Def\_Wiki-Test to evaluate whether it can correctly refer to formalised items in context. The results are reported in Table~\ref{tab:wiki_fdc}.

\paragraph{Including relevant formal definitions in the prompt does not boost the performance of autoformalization.} Intuitively, LLMs should perform better when more relevant information is provided within the prompt. However, directly including grounded formal definitions does not positively impact the formalisation. This behaviour indicates that current state-of-the-art LLMs cannot effectively link to relevant in-context formal items for autoformalization.

\section{Related Work}

Autoformalization builds connections between natural language and formal languages \cite{quan-etal-2024-verification,quan-etal-2024-enhancing}. It also plays an important role in formal mathematical reasoning. For instance, proof autoformalization has been used as an intermediate step in automated theorem proving~\citep{jiang2023draft,tarrach2024more}. Deep learning models, such as transformers, have been applied to autoformalization. For example, \citet{cunningham-etal-2022-towards} developed a transformer-based model for autoformalizing of theorems along with their proofs in Coq. In recent years, with the increasing capabilities of LLMs, prompting-based methods have also demonstrated the ability to autoformalize mathematical statements in Isabelle~\citep{wu2022autoformalization,zhang-etal-2024-consistent,li2024autoformalize} and Lean~\citep{lu2024processdrivenautoformalizationlean4}. Despite recent progress in autoformalization, few studies have analyzed this task from an error perspective. Our work aim to address this gap.

There are a few benchmarks providing informal-formal mathemtical statement pairs. miniF2F dataset~\citep{zheng2022miniff} provided 488 mathematical statement pairs from high school and undergraduate level to Olympiad problems in Lean, Metamath, and Isabelle. ProofNet~\citep{azerbayev2023proofnet} benchmark contains 371 parallel formal theorem statements, natural language theorem statements, and natural language proofs in Lean. However, informal-Formal mathematical statement pairs are still scarce. Obtaining ground-truth formal codes requires specialists and there are many ways to formalize mathematical statements. In our work, we only provide the datasets in the natural language version and aim to develop methods without ground-truth formal codes.

%Error analysis is a common practice in machine learning. It aims to identify error patterns in model predictions and help developers enhance performance based on these patterns. In natural language processing, error analysis has been applied to natural language generation tasks such as machine translation and text summarization~\citep{lu-etal-2023-toward}, as well as code generation tasks~\citep{wang2024largelanguagemodelsfail}. Additionally, error analysis with prompting has been used to aid evaluation in translation tasks~\citep{lu-etal-2024-error}. Our work focuses on error analysis in the context of autoformalization to address the lack of research on error patterns in this task.

\section{Conclusion}
This paper explored the challenges and advancements in autoformalization of complex mathematical statements. To this end, two datasets collecting real-world definitions in machine learning were introduced for systematic evaluation. By assessing autoformalization performance across three modern LLMs on newly introduced datasets, we identify key failure patterns including syntactic inconsistency, undefined references, and type mismatch. To address these, we proposed interventions such as Formal Definition Grounding and Categorical Refinement to enhance performance. Our results suggest that while modern LLMs exhibit potential in converting natural language mathematical definitions into formal representations, they still require improved guidance mechanisms and structured refinement techniques to enhance accuracy. Future research should focus on strengthening self-correction capabilities and integrating more robust contextual understanding into LLM-based formalization systems.

\section{Limitations}
Despite its contributions, this study has several limitations. First, the error analysis was conducted in Isabelle/HOL, and results may not directly generalize to other formal proof assistants such as Lean. Second, the definition datasets proposed, though diverse, are relatively small scale. Additionally, while the proposed refinements improve formalization performance, they do not fully eliminate semantic inconsistencies between natural language definitions and their formalized counterparts. More advanced methods are still needed to be developed.

\section*{Acknowledgements}
This work was funded by the Swiss National Science Foundation (SNSF) project ``NeuMath'' (200021\_204617),  by the CRUK National Biomarker Centre, and supported by the Manchester Experimental Cancer Medicine Centre and the NIHR Manchester Biomedical Research Centre.



% Bibliography entries for the entire Anthology, followed by custom entries
%\bibliography{anthology,custom}
% Custom bibliography entries only
\bibliography{custom, anthology}

\appendix
% \section{Dataset Creation}
% \subsection{Arxiv Papers}
% Papers from arxiv are first shortlisted via using arxiv's advanced search with subcategories cs.LG and stat.ML and published in ICML 2020.


\section{Case Study for Formal Definition Grounding}


The following example shows an example of using GPT-4o in a zero-shot setting to formalize the definition of Bradley--Terry model\footnote{\url{https://en.wikipedia.org/wiki/Bradley\%E2\%80\%93Terry_model}}.

\begin{tcolorbox}[colback=blue!5!white,colframe=mybrown!75!black]
\begin{small}
\textbf{Definition of Bradley--Terry model:} Given a pair of items \(i\) and \(j\) drawn from some population, the Bradley--Terry model estimates the probability that the pairwise comparison turns out true, as \[\Pr(i > j) = \frac{p_i}{p_i + p_j}\] where \(p_i\) is a positive real-valued score assigned to individual \(i\).

\textbf{Generated Formal Code:}
\vspace{-1.5em}
\begin{lstlisting}[style=isabelle]
theory test
imports Main
begin
definition bradley_terry :: "real \<Rightarrow> real 
  \<Rightarrow> real" where "bradley_terry p_i p_j = 
    p_i / (p_i + p_j)"
end
\end{lstlisting}
\end{small}
\end{tcolorbox}

\begin{table*}[t]
    \small
    \centering
    \begin{tabular}{p{0.08\textwidth} p{0.85\textwidth}}
        \toprule
        Category & Reasons\\
        \midrule
        SYN & 
        1. \textbf{Invalid Symbol Format.} Isabelle uses symbols like ``\textbackslash<Rightarrow>'' to represent ``\textbackslash Rightarrow ($\Rightarrow$)'' in LaTeX. GPT-4o does not strictly follow this behaviour. A symbol in its formalized code starting with ``\textbackslash<'' can miss ``>'' at the end so that the relevant symbol is not valid.\newline
        2. \textbf{Confusion of Mapping between LaTeX Mathematical Symbols and Isabelle Symbols.} Not all natural language symbols in LaTeX have a similar corresponding version in Isabelle symbols. In natural language mathematics we use different mathematical fonts such as ``\textbackslash mathcal ($\mathcal{A}$)'' to distinguish items. Isabelle uses ``\textbackslash<A>'' to represent this LaTeX symbol. However, GPT-4o would pretend the existence of a symbol named \textbackslash<mathcal> and use it for autoformalization.\newline 
        3. \textbf{Unaware of Name Conflict.} Some keywords such as ``instance'' are reserved by Isabelle/HOL and they cannot be used as the name of a new item.\newline
        4. \textbf{Incorrect Stylistic Usage of Symbols or Operators.} Some symbols or operators require specific usage which is not in the same style as in natural language. The incorrect usage of them in formalized code generated by GPT-4o can lead to syntax errors.\\
        \midrule
        UDF & 1. \textbf{Items not defined.} Formalization requires every mentioned item  to be clearly defined in the local context or preambles. For one piece of generated formal code, GPT-4o could refer to items that are not defined in both sources.\\
        \midrule
        TUF & 1. \textbf{Mismatch between Types in Definition and Types in Actual Usage.} There are some operators or functions which have been clearly defined about the types of their operands or parameters. When using these operators or functions, the types of actual operands or parameters need to match the types in the definitions exactly. GPT-4o would produce mismatched types in the formalized codes and introduce TUF errors.\\
        \bottomrule
    \end{tabular}
    \caption{Reasons of failure in each error category during autoformalization with GPT-4o.}
    \label{tab:failure}
\end{table*}

The preamble in the generated formal code is ``Main''. However, ``Main'' does not contain the formalization of ``real'', making the formal code invalid. After applying Post-FDG, the preamble is updated to ``HOL.Real'', and the formal code becomes valid. One might suggest creating a universal preamble that imports all source files from the library, applying this common preamble to solve such issues. However, this approach would not align with how a human expert would perform formalization. This failure to identify the correct preambles exposes limitations in the autoformalization capabilities of LLMs. Another issue, which is outside the scope of this paper but an important future direction, is that while Post-FDG can correct the formal code, the semantics of the generated code still do not fully match the original natural language version. For instance, the term ``probability'' does not appear in the formal code, and the phrase ``$p_i$'' is a positive real number” is omitted. The challenge of measuring semantic consistency between the generated formal code and its corresponding natural language version remains an open problem.

\section{Examples of Incorrect Formal Codes}

In this section, we provide some examples of incorrect formal codes generated by GPT-4o to support our summarized reasons in Table~\ref{tab:failure}. All examples of definitions are from Def\_Wiki development set.

Example 1 is about autoformalization on the definition of Rademacher Complexity\footnote{\url{https://en.wikipedia.org/wiki/Rademacher_complexity}}. In this example, the missing of ``>'' in ``\textbackslash<sigma''leads to a SYN error, which contributes to reason 1 of SYN category. In addition, this piece of formal code uses ``real'' but ``real'' is not imported by ``imports Main''. This problem contributes to reason 1 of UDF category.
\begin{tcolorbox}[colback=blue!5!white,colframe=mybrown!75!black]
\begin{small}
\textbf{Example 1:}

\textbf{Definition of Rademacher Complexity:} Given a set \(A\subseteq \mathbb{R}^m\), the Rademacher complexity of A is defined as follows: \[\operatorname{Rad}(A):=\frac{1}{m}\mathbb{E}_\sigma\left[\sup_{a \in A}\sum_{i=1}^m \sigma_i a_i\right]\] where \(\sigma_1, \sigma_2, \dots, \sigma_m\) are independent random variables drawn from the Rademacher distribution (i.e. \(\Pr(\sigma_i = +1) = \Pr(\sigma_i = -1) = 1/2\) for \(i=1,2,\dots,m\)), and \(a=(a_1,\dots,a_m)\).

\textbf{Generated Formal Code:}
\vspace{-1.5em}
\begin{lstlisting}[style=isabelle]
theory test
imports Main
begin

definition rademacher_distribution :: "nat 
  \<Rightarrow> real pmf" where "rademacher_distribution 
    m = pmf_of_set {x. x = 1 \<or> x = -1}"

definition rademacher_complexity :: "(real 
  vector set) \<Rightarrow> nat \<Rightarrow> real" where
  "rademacher_complexity A m = 
    (1 / real m) * (expectation (pmf_of_set 
    {x. x = 1 \<or> x = -1}) (\<lambda>\<sigma>. Sup {sum (\<lambda>i. 
    \<sigma i * a i) {1..m} | a. a \<in> A}))"
    
end
\end{lstlisting}
\end{small}
\end{tcolorbox}


Example 2 is about autoformalization on the definition of Algorithm Selection\footnote{\url{https://en.wikipedia.org/wiki/Algorithm_selection}}. In this example, the problem of missing ``>'' and using ``real'' also occurs. Even if missing ``>'' is solved, ``\textbackslash<mathcal> is not a valid symbol in Isabelle. This problem contributes to reason 2 of SYN category. In addition, ``instance'' is a reserved keyword in Isabelle so it cannot be used in ``type\_synonym''. This contributes to reason 3 of SYN category.
\begin{tcolorbox}[colback=blue!5!white,colframe=mybrown!75!black]
\begin{small}
\textbf{Example 2:}

\textbf{Definition of Algorithm Selection:} Given a portfolio \(\mathcal{P}\) of algorithms \(\mathcal{A} \in \mathcal{P}\), a set of instances \(i \in \mathcal{I}\) and a cost metric \(m: \mathcal{P} \times \mathcal{I} \to \mathbb{R}\), the algorithm selection problem consists of finding a mapping \(s: \mathcal{I} \to \mathcal{P}\) from instances \(\mathcal{I}\) to algorithms \(\mathcal{P}\) such that the cost \(\sum_{i \in \mathcal{I}} m(s(i),i)\) across all instances is optimized.

\textbf{Generated Formal Code:}
\vspace{-1.5em}
\begin{lstlisting}[style=isabelle]
theory test
imports Main
begin

type_synonym 'a instance = 'a
type_synonym 'a algorithm = 'a
type_synonym cost_metric = "'a algorithm 
  \<Rightarrow> 'a instance \<Rightarrow> real"

definition algorithm_selection ::
  "'a algorithm set \<Rightarrow> 'a instance set 
  \<Rightarrow> cost_metric \<Rightarrow> ('a instance 
  \<Rightarrow> 'a algorithm) \<Rightarrow> real" where
  "algorithm_selection \<mathcal{P} 
    \<mathcal{I} m s = (\<Sum>i \<in> \<mathcal{I}. 
    m (s i) i)"
    
end
\end{lstlisting}
\end{small}
\end{tcolorbox} 

Example 3 is about autoformalization on the definition of Set-Family Growth Function\footnote{\url{https://en.wikipedia.org/wiki/Growth_function}}. In this example, there is no SYN or UDF errors. The error occurs in the definition of ``growth\_function'', where ``card ()'' is used on the right side of operator ``$\longrightarrow$''. However, the left and right sides of operator ``$\longrightarrow$'' are explicitly defined as type ``bool'' and the type of ``card ()'' is explicitly defined as ``nat'' in the Isabelle/HOL library. This incorrect usage of them causes a type mismatch and leads to TUF error.
\begin{tcolorbox}[colback=blue!5!white,colframe=mybrown!75!black]
\begin{small}
\textbf{Example 3:}

\textbf{Definition of Set-Family Growth Function:} Let \(H\) be a set family (a set of sets) and \(C\) a set. Their intersection is defined as the following set-family: \(H\cap C := \{h\cap C\mid h\in H\}\). The intersection-size (also called the index) of \(H\) with respect to \(C\) is \(|H\cap C|\). The growth function measures the size of \(H\cap C\) as a function of \(|C|\). Formally: \(\operatorname{Growth}(H,m) := \max_{C: |C|=m} |H\cap C|\).

\textbf{Generated Formal Code:}
\vspace{-1.5em}
\begin{lstlisting}[style=isabelle]
theory test
imports Main
begin
definition intersection_set_family :: 
  "'a set set \<Rightarrow> 'a set \<Rightarrow> 'a set set" 
  (infixl "\<inter>\<^sub>H" 70) where
  "H \<inter>\<^sub>H C = {h \<inter> C | h. h \<in> H}"

definition intersection_size :: 
  "'a set set \<Rightarrow> 'a set \<Rightarrow> nat" where
  "intersection_size H C = card (
    H \<inter>\<^sub>H C)"

definition growth_function :: "'a set set \<Rightarrow> 
  nat \<Rightarrow> nat" where
  "growth_function H m = (SUP C. card C = 
    m \<and> finite C \<and> (\<forall>h \<in> H. finite h) 
    \<and> finite H \<longrightarrow> card (H \<inter>\<^sub>H C))"
end
\end{lstlisting}
\end{small}
\end{tcolorbox}


\section{Prompts and Detailed Results}
The prompts used for the estimation of dataset statistics are provided in Table~\ref{tab:prompt}. The instructions used in the prompts of experiments are provided in Table~\ref{tab:inst_1}. Detailed numbers of autoformalization results on miniF2F test set, Def\_Wiki test set and Def\_ArXiv are provided in Table~\ref{tab:miniF2F_1}, \ref{tab:wiki_1}, \ref{tab:arxiv_1}, respectively. Symbolic refinement results and Post-FDG results on Def\_Wiki test set are provided in Table~\ref{tab:sr} and Table~\ref{tab:wiki_fdc_full}, respectively.  

\begin{table*}[t]
    \centering
    \begin{tabular}{p{0.25\textwidth} p{0.7\textwidth}}
        \toprule
        Purpose & Content\\
        \hline
        Mathematical Objects & Given the following statement written in LaTeX: \{\{latex\}\} How many mathematical objects excluding explicit numbers and variables are mentioned directly in this statement? You can think it step by step. Give me the final number as NUMBER=\{the number\}\\
        \hline
        Mathematical Formulae & Given the following statement written in LaTeX: \{\{latex\}\} How many mathematical formulae are mentioned directly in this statement? You can think it step by step. Give me the final number as NUMBER=\{the number\}\\
        \bottomrule
    \end{tabular}
    \caption{Prompts for the estimation of dataset statistics.}
    \label{tab:prompt}
\end{table*}


\begin{table*}[t]
    \centering
    \begin{tabular}{p{0.12\textwidth} p{0.8\textwidth}}
        \toprule
        Instruction & Content\\
        \hline
        General & You are an expert in \emph{machine learning} and \emph{formal language Isabelle/HOL}. Given the following definition in LaTeX: \{\{latex\}\}, your task is to provide the formal code of this definition in Isabelle/HOL. The following text might contain some preliminaries to explain the given definition: \{\{preliminary\}\}. In case that you need to import any necessary dependent theory files, you should not import any fake theory files.\\
        \hline
        Stylistic & To represent the math symbols, you must use the textual full name of symbols in Isabelle instead of direct symbols. For example you should use \textbackslash<Rightarrow> instead of $\Rightarrow$, \textbackslash<lambda> instead of $\lambda$.\\
        \hline
        Output & Give the results directly without any additional explanations.\\
        \hline
        Refinement & \textbf{Plain}: For your reference, there are some previous formal codes generated by you: \{\{previous\}\}. You can choose to refine this piece of code for your task.\newline
        \textbf{Binary}: For your reference, there are some previous formal codes generated by you: \{\{previous\}\}. The syntactic correctness for this piece of code is: \{\{correctness\}\}. You can choose to refine this piece of code for your task.\newline
        \textbf{Detailed}: For your reference, there are some previous formal codes generated by you: \{\{previous\}\}. The provided code might have some errors according to the Isabelle prover. The error details and where the error code is located in the code are: \{\{error\_details\}\}. You should refine this piece of code for your task.\\
        \hline
        SYN & You should make sure that every symbol you use is a valid Isabelle symbol. If an Isabelle symbol starts with \textbackslash<, then it must end with >. Isabelle reserves some words as keywords. You should be careful with this and avoid to use them to define new names. You should make sure that the usage of symbols and operators is correct in your final output as the incorrect usage will lead to syntax errors.\\
        \hline
        UDF & You should make sure that every item you mentioned in your code has a clear reference either in the local context or the theory files that you decide to import.\\
        \hline
        TUF & You should make sure that in your code, the types of operands of operators or the types of parameters of functions match the types in their definitions exactly. Failure to maintain such compatibility will lead to type mismatch errors.\\
        \hline
        Include Formal Definition Codes & \textbf{Soft}: You can use the following Isabelle/HOL codes to support your task: \{\{formal\_defs\}\} but you should not restate these codes in your final output. You need to formalize everything that is not provided in the given code. In this case, you should assume that you can only use things from HOL.Main. You only need to provide the main body of formal codes for the given definition. You may not import any theory files.\newline
         \textbf{Hard}: The following Isabelle/HOL codes define some mathematical concepts which might be related to your task: \{\{formal\_defs\}\}. If a mathematical concept in your task has been defined in the above codes, you are required to use this version of formal codes but you should not restate these codes in your final output. You need to formalize everything that is not provided in the given code. In this case, you should assume that you can only use things from HOL.Main. You only need to provide the main body of formal codes for the given definition. You may not import any theory files.\\
        \bottomrule
    \end{tabular}
    \caption{Instructions used in prompts.}
    \label{tab:inst_1}
\end{table*}

% \begin{table*}[t]
%     \centering
%     \begin{tabular}{p{0.2\textwidth} p{0.7\textwidth}}
%         \toprule
%         Setting & Prompt\\
%         \hline
%         ZS & General Instruction + Stylistic Instruction + Output Instruction\\
%         \hline
%         (ZS) + Binary & General Instruction + Stylistic Instruction + Binary Refinement Instruction with (ZS) Results + Output Instruction\\
%         \hline
%         (ZS) + Detailed & General Instruction + Stylistic Instruction + Detailed Refinement Instruction with (ZS) Results + Output Instruction\\
%         \hline
%         (ZS) + Detailed + CR-SYN & General Instruction + Stylistic Instruction + Detailed Refinement Instruction with (ZS) Results + CR-SYN Instruction + Output Instruction\\
%         \hline
%         (ZS) + Detailed + CR-UDF & General Instruction + Stylistic Instruction + Detailed Refinement Instruction with (ZS) Results + CR-UDF Instruction + Output Instruction\\
%         \hline
%         (ZS) + Detailed + CR-TUF & General Instruction + Stylistic Instruction + Detailed Refinement Instruction with (ZS) Results + CR-TUF Instruction + Output Instruction\\
%         \hline
%         Soft-IFDC & General Instruction + Stylistic Instruction + Formal Definition Grounding Instruction + Output Instruction\\
%         \hline
%         (ZS) + Soft-IFDC + Binary & General Instruction + Stylistic Instruction + Formal Definition Grounding Instruction + Binary Refinement Instruction with (ZS) Results + Output Instruction\\
%         \hline
%         (ZS) + Soft-IFDC + Detailed & General Instruction + Stylistic Instruction + Formal Definition Grounding Instruction + Detailed Refinement Instruction with (ZS) Results + Output Instruction\\
%         \bottomrule
%     \end{tabular}
%     \caption{Prompts.}
%     \label{tab:inst_2}
% \end{table*}


\begin{table*}[!t]
    \centering
    \small
    \begin{tabular}{l l c c| c c c c c}
        \toprule
        Prompt Strategy & Preamble & Pass$\uparrow$ & FEO$\uparrow$ & TRO$\downarrow$ & IVI$\downarrow$  & SYN$\downarrow$ & UDF$\downarrow$ & TUF$\downarrow$\\
        \midrule
        \multicolumn{9}{l}{\textbf{\textbf{DeepSeekMath-7B}}}\\
        \midrule
        \multirow{2}{*}{ZS} & Direct & 3.28 & 12.79 & 18.44 & 0.00 & 50.00 & 14.34 & 9.43\\
          & Post-FDG & 12.30 & 23.60 & 15.98 & 0.00 & 47.13 & 1.23 & 9.02\\
        \midrule
        \multirow{2}{*}{(ZS) + Binary} & Direct & 2.05 & 6.73 & 2.46 & 0.00 & 79.91 & 5.33 & 2.05\\
          & Post-FDG & 4.10 & 9.39 & 2.46 & 0.00 & 80.33 & 0.41 & 1.23\\
        \midrule
        \multirow{2}{*}{(ZS) + Detailed} & Direct & 3.28 & 10.03 & 5.74 & 0.00 & 70.49 & 10.66 & 4.10\\
          & Post-FDG & 5.74 & 15.57 & 5.74 & 0.00 & 69.67 & 0.82 & 0.41\\
        \midrule
        \multirow{2}{*}{(ZS) + Detailed + CR-All} & Direct & 3.28 & 9.11 & 6.15 & 0.00 & 73.77 & 6.15 & 3.28\\
          & Post-FDG & 5.33 & 13.08 & 6.15 & 0.00 & 72.95 & 0.41 & 3.28\\
        \midrule
        \multicolumn{9}{l}{\textbf{Llama3-8B}}\\
        \midrule
        \multirow{2}{*}{ZS} & Direct & 4.92 & 20.70 & 4.51 & 0.41 & 29.51 & 38.52 & 18.85\\
          & Post-FDG & 10.66 & 31.17 & 4.92 & 0.00 & 28.69 & 20.08 & 21.31\\
        \midrule
        \multirow{2}{*}{(ZS) + Binary} & Direct & 3.69 & 20.52 & 3.28 & 0.41 & 33.20 & 39.75 & 20.49\\
          & Post-FDG & 9.43 & 30.57 & 3.69 & 0.00 & 31.97 & 22.95 & 22.13\\
        \midrule
        \multirow{2}{*}{(ZS) + Detailed} & Direct & 4.10 & 24.33 & 3.69 & 0.82 & 29.51 & 35.25 & 18.44\\
          & Post-FDG & 9.02 & 33.36 & 4.10 & 0.00 & 27.46 & 18.44 & 22.13\\
        \midrule
        \multirow{2}{*}{(ZS) + Detailed + CR-All} & Direct & 4.92 & 24.16 & 6.97 & 0.82 & 27.46 & 35.25 & 20.08\\
          & Post-FDG & 9.43 & 32.41 & 7.79 & 0.00 & 27.46 & 18.85 & 22.54\\
        \midrule
        \multicolumn{9}{l}{\textbf{GPT-4o}}\\
        \midrule
        \multirow{2}{*}{ZS} & Direct & 25.41 & 48.90 & 1.23 & 1.23 & 6.15 & 23.77 & 7.38\\
         & Post-FDG & 67.21 & 81.88 & 0.00 & 0.00 & 3.28 & 2.87 & 5.33\\
        \midrule
        \multirow{2}{*}{ZS + CR-SYN} & Direct & 24.18 & 45.31 & 2.46 & 0.00 & 9.02 & 27.46 & 7.79\\
         & Post-FDG & 52.46 & 73.96 & 0.41 & 0.00 & 7.79 & 3.69 & 3.69\\
        \midrule
        \multirow{2}{*}{ZS + CR-UDF} & Direct & 25.82 & 50.75 & 2.05 & 2.46 & 6.56 & 22.54 & 6.97\\
         & Post-FDG & 61.48 & 80.41 & 0.41 & 0.00 & 5.33 & 2.87 & 2.87\\
        \midrule
        \multirow{2}{*}{ZS + CR-TUF} & Direct & 27.87 & 50.62 & 2.05 & 1.64 & 5.33 & 26.64 & 5.74\\
         & Post-FDG & 54.10 & 78.79 & 0.00 & 0.00 & 3.28 & 4.10 & 2.87\\
        \midrule
        \multirow{2}{*}{(ZS)} & Direct & 25.41 & 53.15 & 1.64 & 1.23 & 6.56 & 22.13 & 7.79\\
         & Post-FDG & 67.21 & 84.05 & 0.00 & 0.00 & 3.28 & 2.46 & 4.92\\
        \midrule
        \multirow{2}{*}{(ZS) + Binary} & Direct & 29.10 & 53.90 & 2.05 & 1.23 & 6.15 & 21.72 & 8.20\\
         & Post-FDG & 67.21 & 83.60 & 0.00 & 0.00 & 4.10 & 2.05 & 4.92\\
        \midrule
        \multirow{2}{*}{(ZS) + Detailed} & Direct & 37.30 & 63.28 & 2.05 & 1.23 & 5.74 & 9.02 & 8.61\\
          & Post-FDG & 83.61 & 91.47 & 0.00 & 0.00 & 2.05 & 0.82 & 3.28\\
        \midrule
        \multirow{2}{*}{(ZS) + CR-SYN} & Direct & 25.41 & 52.72 & 2.05 & 1.23 & 5.74 & 22.13 & 8.61\\
         & Post-FDG & 67.21 & 83.73 & 0.00 & 0.00 & 2.87 & 2.46 & 5.74\\
        \midrule
        \multirow{2}{*}{(ZS) + CR-UDF} & Direct & 26.64 & 54.06 & 1.64 & 1.23 & 6.15 & 21.72 & 6.97\\
         & Post-FDG & 67.21 & 83.78 & 0.00 & 0.00 & 3.69 & 2.05 & 4.92\\
         \midrule
        \multirow{2}{*}{(ZS) + CR-TUF} & Direct & 25.41 & 51.18 & 2.46 & 1.23 & 6.56 & 24.18 & 7.38\\
         & Post-FDG & 67.21 & 83.94 & 0.00 & 0.00 & 3.28 & 2.87 & 4.10\\
        \midrule
        \multirow{2}{*}{(ZS) + Detailed + CR-SYN} & Direct & 38.52 & 64.42 & 2.05 & 1.23 & 7.79 & 8.20 & 7.79\\
          & Post-FDG & 82.79 & 90.32 & 0.00 & 0.00 & 3.28 & 0.82 & 2.05\\
        \midrule
        \multirow{2}{*}{(ZS) + Detailed + CR-UDF} & Direct & 38.11 & 63.95 & 2.05 & 2.46 & 5.74 & 6.56 & 6.97\\
          & Post-FDG & 82.38 & 90.48 & 0.00 & 0.00 & 2.46 & 1.23 & 2.87\\
        \midrule
        \multirow{2}{*}{(ZS) + Detailed + CR-TUF} & Direct & 41.39 & 64.76 & 3.28 & 1.23 & 6.15 & 11.07 & 6.15\\
          & Post-FDG & 83.20 & 90.71 & 0.00 & 0.00 & 2.87 & 1.64 & 2.05\\
        \midrule
        \multirow{2}{*}{(ZS) + Detailed + CR-All} & Direct & 38.52 & 65.73 & 2.05 & 1.23 & 6.15 & 5.74 & 7.79\\
          & Post-FDG & 81.97 & 90.65 & 0.00 & 0.00 & 2.46 & 0.41 & 2.46 \\
        \bottomrule
    \end{tabular}
    \caption{Error results on miniF2F test set.}
    \label{tab:miniF2F_1}
\end{table*}

% \begin{table*}[!t]
%     \centering
%     \begin{tabular}{l l c c c c}
%         \toprule
%         Prompt Strategy & Preamble & SYN$\downarrow$ & UDF$\downarrow$  & TUF$\downarrow$\\
%         \hline
%         \multicolumn{5}{l}{\textbf{DeepSeekMath-7B}}\\
%         \hline
%         \multirow{2}{*}{ZS} & Direct & 4.57$\pm$8.46 & 0.28$\pm$0.91 & 0.60$\pm$2.94\\
%           & Post-FDG & 4.71$\pm$8.69 & 0.02$\pm$0.18 & 0.59$\pm$2.95\\
%         \hline
%         \multirow{2}{*}{(ZS) + Binary} & Direct & $\pm$ & $\pm$ & $\pm$\\
%           & Post-FDG & $\pm$ & $\pm$ & $\pm$\\
%         \hline
%         \multirow{2}{*}{(ZS) + Detailed} & Direct & $\pm$ & $\pm$ & $\pm$\\
%           & Post-FDG & $\pm$ & $\pm$ & $\pm$\\
%         \hline
%         \multirow{2}{*}{(ZS) + Detailed + CR-All} & Direct & $\pm$ & $\pm$ & $\pm$\\
%           & Post-FDG & $\pm$ & $\pm$ & $\pm$\\
%         \hline
%         \multicolumn{5}{l}{\textbf{Llama3-8B}}\\
%         \hline
%         \multirow{2}{*}{ZS} & Direct & 0.47$\pm$0.85 & 0.79$\pm$1.19 & 0.30$\pm$0.71\\\
%           & Post-FDG & 0.49$\pm$0.86 & 0.44$\pm$0.99 & 0.38$\pm$0.82\\
%         \hline
%         \multirow{2}{*}{(ZS) + Binary} & Direct & $\pm$ & $\pm$ & $\pm$\\
%           & Post-FDG & $\pm$ & $\pm$ & $\pm$\\
%         \hline
%         \multirow{2}{*}{(ZS) + Detailed} & Direct & $\pm$ & $\pm$ & $\pm$\\
%           & Post-FDG & $\pm$ & $\pm$ & $\pm$\\
%         \hline
%         \multirow{2}{*}{(ZS) + Detailed + CR-All} & Direct & $\pm$ & $\pm$ & $\pm$\\
%           & Post-FDG & $\pm$ & $\pm$ & $\pm$\\
%         \hline
%         \multicolumn{5}{l}{\textbf{GPT-4o}}\\
%         \hline
%         \multirow{2}{*}{ZS} & Direct & 0.10$\pm$0.33 & 0.51$\pm$0.95 & 0.13$\pm$0.53\\
%          & Post-FDG & 0.11$\pm$0.35 & 0.14$\pm$0.54 & 0.28$\pm$0.85\\
%         \hline
%         \multirow{2}{*}{(ZS) + Binary} & Direct & 0.11$\pm$0.38 & 0.47$\pm$0.90 & 0.17$\pm$0.63\\
%          & Post-FDG & 0.15$\pm$0.45 & 0.08$\pm$0.31 & 0.21$\pm$0.74\\
%         \hline
%         \multirow{2}{*}{(ZS) + Detailed} & Direct & 0.12$\pm$0.38 & 0.21$\pm$0.62 & 0.31$\pm$1.05\\
%           & Post-FDG & 0.18$\pm$0.49 & 0.05$\pm$0.22 & 0.43$\pm$1.34\\
%         \hline
%         \multirow{2}{*}{(ZS) + Detailed + CR-All} & Direct & 0.12$\pm$0.37 & 0.17$\pm$0.63 & 0.24$\pm$0.78\\
%           & Post-FDG & 0.16$\pm$0.42 & 0.02$\pm$0.15 & 0.27$\pm$0.96\\
%         \hline
%         \multirow{2}{*}{(ZS) + Detailed + CR-SYN} & Direct & 0.15$\pm$0.42 & 0.23$\pm$0.68 & 0.30$\pm$1.06\\
%           & Post-FDG & 0.24$\pm$0.53 & 0.05$\pm$0.21 & 0.33$\pm$1.30\\
%         \hline
%         \multirow{2}{*}{(ZS) + Detailed + CR-UDF} & Direct & 0.12$\pm$0.39 & 0.14$\pm$0.44 & 0.23$\pm$0.92\\
%           & Post-FDG & 0.19$\pm$0.49 & 0.07$\pm$0.25 & 0.33$\pm$1.23\\
%         \hline
%         \multirow{2}{*}{(ZS) + Detailed + CR-TUF} & Direct & 0.13$\pm$0.38 & 0.29$\pm$0.70 & 0.32$\pm$1.60\\
%           & Post-FDG & 0.20$\pm$0.45 & 0.10$\pm$0.30 & 0.32$\pm$1.28\\
%         \bottomrule
%     \end{tabular}
%     \caption{Mean and standard deviation of error occurrences in failure cases on miniF2F test set.}
%     \label{tab:miniF2F_2}
% \end{table*}


\begin{table*}[!t]
    \centering
    \small
    \begin{tabular}{l l c c| c c c c c}
        \toprule
        Prompt Strategy & Preamble & Pass$\uparrow$ & FEO$\uparrow$ & TRO$\downarrow$ & IVI$\downarrow$  & SYN$\downarrow$ & UDF$\downarrow$ & TUF$\downarrow$\\
        \midrule
        \multicolumn{9}{l}{\textbf{\textbf{DeepSeekMath-7B}}}\\
        \midrule
        \multirow{2}{*}{ZS} & Direct & 10.87 & 17.75 & 34.78 & 2.17 & 30.43 & 26.09 & 2.17\\
          & Post-FDG & 26.09 & 30.98 & 34.78 & 0.00 & 21.74 & 10.87 & 13.04\\
        \midrule
        \multirow{2}{*}{(ZS) + Binary} & Direct & 6.52 & 7.73 & 8.70 & 0.00 & 69.57 & 21.74 & 2.17\\
          & Post-FDG & 10.87 & 12.56 & 8.70 & 0.00 & 65.22 & 15.22 & 6.52\\
        \midrule
        \multirow{2}{*}{(ZS) + Detailed} & Direct & 10.87 & 13.27 & 15.22 & 2.17 & 43.48 & 34.78 & 6.52\\
          & Post-FDG & 26.09 & 29.21 & 13.04 & 0.00 & 36.96 & 17.39 & 19.57\\
        \midrule
        \multirow{2}{*}{(ZS) + Detailed + CR-All} & Direct & 4.35 & 7.66 & 13.04 & 2.17 & 47.83 & 32.61 & 8.70\\
          & Post-FDG & 17.39 & 21.43 & 13.04 & 0.00 & 41.30 & 15.22 & 21.74\\
        \midrule
        \multicolumn{9}{l}{\textbf{Llama3-8B}}\\
        \midrule
        \multirow{2}{*}{ZS} & Direct & 0.00 & 2.80 & 0.00 & 23.91 & 56.52 & 32.61 & 4.35\\
          & Post-FDG & 0.00 & 2.80 & 21.74 & 0.00 & 58.70 & 23.91 & 15.22\\
        \midrule
        \multirow{2}{*}{(ZS) + Binary} & Direct & 2.17 & 3.71 & 0.00 & 26.09 & 52.17 & 30.43 & 2.17\\
          & Post-FDG & 0.00 & 1.53 & 23.91 & 0.00 & 56.52 & 28.26 & 13.04\\
        \midrule
        \multirow{2}{*}{(ZS) + Detailed} & Direct & 2.17 & 3.80 & 0.00 & 26.09 & 50.00 & 30.43 & 6.52\\
          & Post-FDG & 4.35 & 5.98 & 23.91 & 0.00 & 52.17 & 26.09 & 15.22\\
        \midrule
        \multirow{2}{*}{(ZS) + Detailed + CR-All} & Direct & 2.17 & 3.71 & 0.00 & 26.09 & 52.17 & 32.61 & 4.35\\
          & Post-FDG & 2.17 & 3.71 & 23.91 & 0.00 & 54.35 & 23.91 & 15.22\\
        \midrule
        \multicolumn{9}{l}{\textbf{GPT-4o}}\\
        \midrule
        \multirow{2}{*}{ZS} & Direct & 10.87 & 16.12 & 8.70 & 8.70 & 19.57 & 50.00 & 13.04\\
         & Post-FDG & 34.78 & 42.56 & 6.52 & 0.00 & 30.43 & 17.39 & 23.91\\
        \midrule
        \multirow{2}{*}{ZS + CR-SYN} & Direct & 10.87 & 15.18 & 8.70 & 2.17 & 15.22 & 58.70 & 13.04\\
         & Post-FDG & 34.78 & 40.27 & 8.70 & 0.00 & 28.26 & 13.04 & 26.09\\
        \midrule
        \multirow{2}{*}{ZS + CR-UDF} & Direct & 2.17 & 11.59 & 6.52 & 6.52 & 19.57 & 60.87 & 19.57\\
         & Post-FDG & 30.43 & 42.66 & 2.17 & 0.00 & 34.78 & 23.91 & 23.91\\
        \midrule
        \multirow{2}{*}{ZS + CR-TUF} & Direct & 8.70 & 14.55 & 8.70 & 6.52 & 21.74 & 56.52 & 15.22\\
         & Post-FDG & 30.43 & 40.51 & 6.52 & 0.00 & 34.78 & 17.39 & 28.26\\
        \midrule
        \multirow{2}{*}{(ZS)} & Direct & 10.87 & 16.21 & 8.70 & 8.70 & 19.57 & 50.00 & 13.04\\
         & Post-FDG & 39.13 & 47.23 & 6.52 & 0.00 & 28.26 & 15.22 & 23.91\\
        \midrule
        \multirow{2}{*}{(ZS) + Binary} & Direct & 13.04 & 18.30 & 8.70 & 6.52 & 17.39 & 50.00 & 13.04\\
         & Post-FDG & 39.13 & 48.00 & 6.52 & 0.00 & 26.09 & 8.70 & 28.26\\
        \midrule
        \multirow{2}{*}{(ZS) + Detailed} & Direct & 19.57 & 23.46 & 8.70 & 8.70 & 10.87 & 47.83 & 10.87\\
          & Post-FDG & 43.48 & 50.13 & 6.52 & 0.00 & 21.74 & 10.87 & 23.91\\
        \midrule
        \multirow{2}{*}{(ZS) + CR-SYN} & Direct & 10.87 & 16.12 & 8.70 & 8.70 & 17.39 & 52.17 & 13.04\\
         & Post-FDG & 36.96 & 44.97 & 6.52 & 0.00 & 30.43 & 15.22 & 23.91\\
        \midrule
        \multirow{2}{*}{(ZS) + CR-UDF} & Direct & 10.87 & 16.12 & 8.70 & 8.70 & 19.57 & 50.00 & 13.04\\
         & Post-FDG & 36.96 & 44.97 & 6.52 & 0.00 & 30.43 & 15.22 & 23.91\\
         \midrule
        \multirow{2}{*}{(ZS) + CR-TUF} & Direct & 10.87 & 16.21 & 8.70 & 8.70 & 21.74 & 47.83 & 13.04\\
         & Post-FDG & 36.96 & 45.06 & 6.52 & 0.00 & 32.61 & 15.22 & 21.74\\
        \midrule
        \multirow{2}{*}{(ZS + Detailed) + Detailed} & Direct & 19.57 & 24.09 & 8.70 & 8.70 & 13.04 & 43.48 & 10.87\\
          & Post-FDG & 43.48 & 50.32 & 6.52 & 0.00 & 19.57 & 8.70 & 26.09\\
        \midrule
        \multirow{2}{*}{(ZS) + Detailed + CR-SYN} & Direct & 21.74 & 25.63 & 8.70 & 10.87 & 10.87 & 41.30 & 8.70\\
          & Post-FDG & 45.65 & 52.31 & 6.52 & 0.0 & 21.74 & 8.70 & 21.74\\
        \midrule
        \multirow{2}{*}{(ZS) + Detailed + CR-UDF} & Direct & 17.39 & 21.83 & 8.70 & 13.04 & 17.39 & 39.13 & 8.70\\
          & Post-FDG & 43.48 & 50.24 & 6.52 & 0.0 & 21.74 & 10.87 & 21.74\\
        \midrule
        \multirow{2}{*}{(ZS) + Detailed + CR-TUF} & Direct & 19.57 & 23.46 & 8.70 & 8.70 & 17.39 & 43.48 & 8.70\\
          & Post-FDG & 45.65 & 52.31 & 6.52 & 0.0 & 23.91 & 10.87 & 19.57\\
        \midrule
        \multirow{2}{*}{(ZS) + Detailed + CR-All} & Direct & 21.74 & 25.63 & 8.70 & 8.70 & 10.87 & 43.48 & 13.04\\
          & Post-FDG & 43.48 & 50.13 & 6.52 & 0.00 & 21.74 & 10.87 & 23.91\\
        \bottomrule
    \end{tabular}
    \caption{Error results on Def\_Wiki test set.}
    \label{tab:wiki_1}
\end{table*}

% \begin{table*}[!t]
%     \centering
%     \begin{tabular}{l l c c c}
%         \toprule
%         Prompt Strategy & Header & SYN$\downarrow$ & UDF$\downarrow$  & TUF$\downarrow$\\
%         \hline
%         \multicolumn{5}{l}{\textbf{DeepSeekMath-7B}}\\
%         \hline
%         \multirow{2}{*}{ZS} & Direct & 6.54$\pm$16.05 & 0.50$\pm$0.50 & 0.04$\pm$0.20\\
%           & Post-FDG & 7.94$\pm$18.26 & 0.28$\pm$0.45 & 0.33$\pm$0.47\\
%         \hline
%         \multirow{2}{*}{(ZS) + Binary} & Direct & 5.54$\pm$7.61 & 0.36$\pm$0.66 & 0.03$\pm$0.16\\
%           & Post-FDG & 5.59$\pm$7.76 & 0.22$\pm$0.47 & 0.14$\pm$0.47\\
%         \hline
%         \multirow{2}{*}{(ZS) + Detailed} & Direct & 7.27$\pm$18.93 & 0.70$\pm$0.94 & 0.09$\pm$0.29\\
%           & Post-FDG & 8.14$\pm$20.37 & 0.39$\pm$0.72 & 0.43$\pm$0.82\\
%         \hline
%         \multirow{2}{*}{(ZS) + Detailed + CR-All} & Direct & 5.14$\pm$13.37 & 0.51$\pm$0.79 & 0.14$\pm$0.41\\
%           & Post-FDG & 5.56$\pm$14.25 & 0.31$\pm$0.77 & 0.34$\pm$0.54\\
%         \hline
%         \multicolumn{5}{l}{\textbf{Llama3-8B}}\\
%         \hline
%         \multirow{2}{*}{ZS} & Direct & 1.69$\pm$1.97 & 0.80$\pm$1.19 & 0.06$\pm$0.23\\
%           & Post-FDG & 1.67$\pm$1.94 & 0.61$\pm$1.14 & 0.22$\pm$0.48\\
%         \hline
%         \multirow{2}{*}{(ZS) + Binary} & Direct & 1.70$\pm$2.04 & 0.94$\pm$1.56 & 0.03$\pm$0.17\\
%           & Post-FDG & $\pm$ & $\pm$ & $\pm$\\
%         \hline
%         \multirow{2}{*}{(ZS) + Detailed} & Direct & $\pm$ & $\pm$ & $\pm$\\
%           & Post-FDG & $\pm$ & $\pm$ & $\pm$\\
%         \hline
%         \multirow{2}{*}{(ZS) + Detailed + CR-All} & Direct & $\pm$ & $\pm$ & $\pm$\\
%           & Post-FDG & $\pm$ & $\pm$ & $\pm$\\
%         \hline
%         \multicolumn{5}{l}{\textbf{GPT-4o}}\\
%         \hline
%         \multirow{2}{*}{ZS} & Direct & 0.36$\pm$0.77 & 1.18$\pm$1.19 & 0.24$\pm$0.60\\
%          & Post-FDG & 0.70$\pm$0.90 & 0.33$\pm$0.54 & 0.48$\pm$0.69\\
%         \hline
%         \multirow{2}{*}{(ZS) + Binary} & Direct & 0.45$\pm$1.10 & 1.24$\pm$1.33 & 0.24$\pm$0.60\\
%          & Post-FDG & 0.84$\pm$1.25 & 0.16$\pm$0.37 & 0.60$\pm$0.69\\
%         \hline
%         \multirow{2}{*}{(ZS) + Detailed} & Direct & 0.21$\pm$0.48 & 1.45$\pm$1.30 & 0.17$\pm$0.38\\
%           & Post-FDG & 0.57$\pm$0.71 & 0.39$\pm$0.92 & 0.48$\pm$0.50\\
%         \hline
%         \multirow{2}{*}{(ZS + Detailed) + Detailed} & Direct & 0.21$\pm$0.41 & 1.41$\pm$1.40 & 0.17$\pm$0.38\\
%           & Post-FDG & 0.48$\pm$0.65 & 0.35$\pm$0.91 & 0.65$\pm$0.87\\
%         \hline
%         \multirow{2}{*}{(ZS) + Detailed + CR-All} & Direct & 0.29$\pm$0.80 & 1.25$\pm$1.18 & 0.25$\pm$0.51\\
%           & Post-FDG & 0.65$\pm$0.96 & 0.26$\pm$0.53 & 0.52$\pm$0.58\\
%         \hline
%         \multirow{2}{*}{(ZS) + Detailed + CR-SYN} & Direct & 0.30$\pm$0.81 & 1.33$\pm$1.33 & 0.15$\pm$0.36\\
%           & Post-FDG & 0.68$\pm$0.97 & 0.32$\pm$0.87 & 0.45$\pm$0.50\\
%         \hline
%         \multirow{2}{*}{(ZS) + Detailed + CR-UDF} & Direct & 0.32$\pm$0.54 & 1.32$\pm$1.44 & 0.14$\pm$0.35\\
%           & Post-FDG & 0.52$\pm$0.65 & 0.48$\pm$1.14 & 0.43$\pm$0.50\\
%         \hline
%         \multirow{2}{*}{(ZS) + Detailed + CR-TUF} & Direct & 0.34$\pm$0.60 & 1.34$\pm$1.35 & 0.14$\pm$0.34\\
%           & Post-FDG & 0.64$\pm$0.71 & 0.41$\pm$0.94 & 0.41$\pm$0.49\\
%         \hline
%         \multirow{2}{*}{Soft-IFDC} & Direct & 0.21$\pm$0.40 & 1.23$\pm$1.05 & 0.03$\pm$0.16\\
%          & Post-FDG & 0.43$\pm$0.50 & 0.46$\pm$0.64 & 0.35$\pm$0.53\\
%         \hline
%         \multirow{2}{*}{Hard-IFDC} & Direct & 0.18$\pm$0.55 & 0.97$\pm$0.62 & 0.08$\pm$0.27\\
%           & Post-FDG & 0.54$\pm$0.68 & 0.30$\pm$0.51 & 0.54$\pm$0.64\\
%         \hline
%         \multirow{2}{*}{(ZS) + Soft-IFDC + Binary} & Direct & 0.29$\pm$0.75 & 1.38$\pm$1.35 & 0.21$\pm$0.58\\
%           & Post-FDG & 0.71$\pm$0.93 & 0.21$\pm$0.41 & 0.58$\pm$0.70\\
%         \hline
%         \multirow{2}{*}{(ZS) + Soft-IFDC + Detailed} & Direct & 0.18$\pm$0.38 & 1.35$\pm$1.23 & 0.21$\pm$0.47\\
%           & Post-FDG & 0.54$\pm$0.64 & 0.38$\pm$0.90 & 0.54$\pm$0.58\\ 
%         \bottomrule
%     \end{tabular}
%     \caption{Mean and standard deviation of error occurrences in failure cases on Def\_Wiki test set.}
%     \label{tab:wiki_2}
% \end{table*}

\begin{table*}[!t]
    \centering
    \small
    \begin{tabular}{l l c c| c c c c c c}
        \toprule
        Prompt Strategy & Preamble & Pass$\uparrow$ & FEO$\uparrow$ & TRO$\downarrow$ & IVI$\downarrow$  & SYN$\downarrow$ & UDF$\downarrow$ & TUF$\downarrow$\\
        \midrule
        \multicolumn{9}{l}{\textbf{\textbf{DeepSeekMath-7B}}}\\
        \midrule
        \multirow{2}{*}{ZS} & Direct & 13.33 & 14.69 & 16.67 & 0.00 & 40.00 & 36.67 & 13.33\\
          & Post-FDG & 16.67 & 18.02 & 13.33 & 0.00 & 43.33 & 30.00 & 16.67\\
        \midrule
        \multirow{2}{*}{(ZS) + Binary} & Direct & 3.33 & 3.33 & 6.67 & 0.00 & 66.67 & 33.33 & 3.33\\
          & Post-FDG & 6.67 & 7.41 & 3.33 & 0.00 & 70.00 & 23.33 & 10.00\\
        \midrule
        \multirow{2}{*}{(ZS) + Detailed} & Direct & 6.67 & 7.36 & 13.33 & 0.00 & 46.67 & 43.33 & 13.33\\
          & Post-FDG & 13.33 & 14.02 & 10.00 & 0.00 & 46.67 & 33.33 & 20.00\\
        \midrule
        \multirow{2}{*}{(ZS) + Detailed + CR-All} & Direct & 6.67 & 7.59 & 13.33 & 0.00 & 46.67 & 43.33 & 13.33\\
          & Post-FDG & 13.33 & 14.26 & 10.00 & 0.00 & 46.67 & 33.33 & 20.00\\
        \midrule
        \multicolumn{9}{l}{\textbf{Llama3-8B}}\\
        \midrule
        \multirow{2}{*}{ZS} & Direct & 0.00 & 2.67 & 0.00 & 13.33 & 70.00 & 40.00 & 6.67\\
          & Post-FDG & 0.00 & 2.67 & 13.33 & 0.00 & 66.67 & 26.67 & 20.00\\
        \midrule
        \multirow{2}{*}{(ZS) + Binary} & Direct & 3.33 & 5.83 & 0.00 & 20.00 & 60.00 & 33.33 & 6.67\\
          & Post-FDG & 3.33 & 5.83 & 20.00 & 0.00 & 60.00 & 26.67 & 16.67\\
        \midrule
        \multirow{2}{*}{(ZS) + Detailed} & Direct & 0.00 & 1.41 & 0.00 & 20.00 & 63.33 & 33.33 & 6.67\\
          & Post-FDG & 0.00 & 4.22 & 20.00 & 0.00 & 56.67 & 26.67 & 20.00\\
        \midrule
        \multirow{2}{*}{(ZS) + Detailed + CR-All} & Direct & 0.00 & 2.33 & 0.00 & 16.67 & 66.67 & 36.67 & 6.67\\
          & Post-FDG & 3.33 & 7.00 & 16.67 & 0.00 & 63.33 & 26.67 & 23.33\\
        \midrule
        \multicolumn{9}{l}{\textbf{GPT-4o}}\\
        \midrule
        \multirow{2}{*}{ZS} & Direct & 13.33 & 19.30 & 0.00 & 0.00 & 40.00 & 56.66 & 6.67\\
         & Post-FDG & 23.33 & 36.02 & 0.00 & 0.00 & 60.00 & 13.33 & 13.33\\
        \midrule
        \multirow{2}{*}{ZS + CR-SYN} & Direct & 10.00 & 17.14 & 0.00 & 0.00 & 26.67 & 66.67 & 6.67\\
         & Post-FDG & 26.67 & 39.11 & 0.00 & 0.00 & 50.00 & 20.00 & 16.67\\
        \midrule
        \multirow{2}{*}{ZS + CR-UDF} & Direct & 10.00 & 18.54 & 0.00 & 10.00 & 33.33 & 46.67 & 16.67\\
         & Post-FDG & 23.33 & 36.52 & 0.00 & 0.00 & 46.67 & 23.33 & 16.67\\
        \midrule
        \multirow{2}{*}{ZS + CR-TUF} & Direct & 6.67 & 14.05 & 0.00 & 3.33 & 23.33 & 63.33 & 10.00\\
         & Post-FDG & 23.33 & 35.03 & 0.00 & 0.00 & 56.67 & 13.33 & 10.00\\
        \midrule
        \multirow{2}{*}{(ZS)} & Direct & 16.67 & 23.28 & 0.00 & 0.00 & 36.67 & 53.33 & 6.67\\
         & Post-FDG & 30.00 & 40.83 & 0.00 & 0.00 & 56.67 & 10.00 & 10.00\\
        \midrule
        \multirow{2}{*}{(ZS) + Binary} & Direct & 16.67 & 24.30 & 0.00 & 0.00 & 33.33 & 53.33 & 6.67\\
         & Post-FDG & 26.67 & 41.02 & 0.00 & 0.00 & 60.00 & 10.00 & 6.67\\
        \midrule
        \multirow{2}{*}{(ZS) + Detailed} & Direct & 16.67 & 28.91 & 0.00 & 0.00 & 36.67 & 43.33 & 16.67\\
          & Post-FDG & 30.00 & 44.15 & 0.00 & 0.00 & 56.67 & 13.33 & 3.33\\
        \midrule
        \multirow{2}{*}{(ZS) + CR-SYN} & Direct & 20.00 & 24.12 & 0.00 & 0.00 & 36.67 & 53.33 & 3.33\\
         & Post-FDG & 30.00 & 40.83 & 0.00 & 0.00 & 60.00 & 10.00 & 6.67\\
        \midrule
        \multirow{2}{*}{(ZS) + CR-UDF} & Direct & 20.00 & 24.12 & 0.00 & 0.00 & 30.00 & 56.67 & 6.67\\
         & Post-FDG & 30.00 & 40.83 & 0.00 & 0.00 & 56.67 & 10.00 & 10.00\\
         \midrule
        \multirow{2}{*}{(ZS) + CR-TUF} & Direct & 16.67 & 23.07 & 0.00 & 0.00 & 33.33 & 53.33 & 10.00\\
         & Post-FDG & 26.67 & 37.47 & 0.00 & 0.00 & 60.00 & 13.33 & 6.67\\
        \midrule
        \multirow{2}{*}{(ZS) + Detailed + CR-SYN} & Direct & 23.33 & 29.74 & 0.00 & 0.00 & 30.00 & 50.00 & 10.00\\
          & Post-FDG & 30.00 & 43.12 & 0.00 & 0.00 & 53.33 & 16.67 & 3.33\\
        \midrule
        \multirow{2}{*}{(ZS) + Detailed + CR-UDF} & Direct & 26.67 & 34.18 & 0.00 & 0.00 & 33.33 & 43.33 & 10.00\\
          & Post-FDG & 30.00 & 44.23 & 0.00 & 0.00 & 53.33 & 13.33 & 6.67\\
        \midrule
        \multirow{2}{*}{(ZS) + Detailed + CR-TUF} & Direct & 13.33 & 25.41 & 0.00 & 0.00 & 33.33 & 46.67 & 16.67\\
          & Post-FDG & 30.00 & 43.98 & 0.00 & 0.00 & 56.67 & 13.33 & 3.33\\
        \midrule
        \multirow{2}{*}{(ZS) + Detailed + CR-All} & Direct & 13.33 & 24.54 & 0.00 & 0.00 & 33.33 & 50.00 & 16.67\\
          & Post-FDG & 33.33 & 46.45 & 0.00 & 0.00 & 50.00 & 13.33 & 6.67\\
        \bottomrule
    \end{tabular}
    \caption{Error results on Def\_ArXiv set.}
    \label{tab:arxiv_1}
\end{table*}

% \begin{table*}[!t]
%     \centering
%     \begin{tabular}{l l c c c}
%         \toprule
%         Prompt Strategy & Preamble & SYN$\downarrow$ & UDF$\downarrow$  & TUF$\downarrow$\\
%         \hline
%         \multicolumn{5}{l}{\textbf{DeepSeekMath-7B}}\\
%         \hline
%         \multirow{2}{*}{ZS} & Direct & $\pm$ & $\pm$ & $\pm$\\
%           & Post-FDG & $\pm$ & $\pm$ & $\pm$\\
%         \hline
%         \multirow{2}{*}{(ZS) + Binary} & Direct & $\pm$ & $\pm$ & $\pm$\\
%           & Post-FDG & $\pm$ & $\pm$ & $\pm$\\
%         \hline
%         \multirow{2}{*}{(ZS) + Detailed} & Direct & $\pm$ & $\pm$ & $\pm$\\
%           & Post-FDG & $\pm$ & $\pm$ & $\pm$\\
%         \hline
%         \multirow{2}{*}{(ZS) + Detailed + CR-All} & Direct & $\pm$ & $\pm$ & $\pm$\\
%           & Post-FDG & $\pm$ & $\pm$ & $\pm$\\
%         \hline
%         \multicolumn{5}{l}{\textbf{Llama3-8B}}\\
%         \hline
%         \multirow{2}{*}{ZS} & Direct & $\pm$ & $\pm$ & $\pm$\\
%           & Post-FDG & $\pm$ & $\pm$ & $\pm$\\
%         \hline
%         \multirow{2}{*}{(ZS) + Binary} & Direct & $\pm$ & $\pm$ & $\pm$\\
%           & Post-FDG & $\pm$ & $\pm$ & $\pm$\\
%         \hline
%         \multirow{2}{*}{(ZS) + Detailed} & Direct & $\pm$ & $\pm$ & $\pm$\\
%           & Post-FDG & $\pm$ & $\pm$ & $\pm$\\
%         \hline
%         \multirow{2}{*}{(ZS) + Detailed + CR-All} & Direct & $\pm$ & $\pm$ & $\pm$\\
%           & Post-FDG & $\pm$ & $\pm$ & $\pm$\\
%         \hline
%         \multicolumn{5}{l}{\textbf{GPT-4o}}\\
%         \hline
%         \multirow{2}{*}{ZS} & Direct & 0.62$\pm$0.79 & 0.88$\pm$0.80 & 0.08$\pm$0.27\\
%          & Post-FDG & 1.00$\pm$0.72 & 0.17$\pm$0.38 & 0.17$\pm$0.38\\
%         \hline
%         \multirow{2}{*}{(ZS) + Binary} & Direct & 0.56$\pm$0.80 & 0.88$\pm$0.82 & 0.08$\pm$0.27\\
%          & Post-FDG & 1.05$\pm$0.71 & 0.14$\pm$0.34 & 0.09$\pm$0.29\\
%         \hline
%         \multirow{2}{*}{(ZS) + Detailed} & Direct & 0.64$\pm$0.77 & 0.86$\pm$0.87 & 0.09$\pm$0.29\\
%           & Post-FDG & 1.00$\pm$0.69 & 0.24$\pm$0.53 & 0.05$\pm$0.21\\
%         \hline
%         \multirow{2}{*}{(ZS) + Detailed + CR-All} & Direct & 0.52$\pm$0.75 & 0.84$\pm$0.83 & 0.16$\pm$0.37\\
%           & Post-FDG & 0.95$\pm$0.74 & 0.25$\pm$0.54 & 0.10$\pm$0.30\\
%         \hline
%         \multirow{2}{*}{(ZS) + Detailed + CR-SYN} & Direct & 0.52$\pm$0.77 & 0.91$\pm$0.83 & 0.13$\pm$0.34\\
%           & Post-FDG & 0.95$\pm$0.72 & 0.29$\pm$0.55 & 0.05$\pm$0.21\\
%         \hline
%         \multirow{2}{*}{(ZS) + Detailed + CR-UDF} & Direct & 0.59$\pm$0.78 & 0.86$\pm$0.87 & 0.14$\pm$0.34\\
%           & Post-FDG & 0.95$\pm$0.72 & 0.24$\pm$0.53 & 0.10$\pm$0.29\\
%         \hline
%         \multirow{2}{*}{(ZS) + Detailed + CR-TUF} & Direct & 0.46$\pm$0.58 & 0.96$\pm$1.17 & 0.13$\pm$0.33\\
%           & Post-FDG & 1.00$\pm$0.69 & 0.24$\pm$0.53 & 0.05$\pm$0.21\\
%         \bottomrule
%     \end{tabular}
%     \caption{Mean and standard deviation of error occurrences in failure cases on Def\_ArXiv set.}
%     \label{tab:arxiv_2}
% \end{table*}

\begin{table*}[!t]
    \centering
    \small
    \begin{tabular}{l l c c| c c c c c}
        \toprule
        Prompt Strategy & Preamble & Pass$\uparrow$ & FEO$\uparrow$ & TRO$\downarrow$ & IVI$\downarrow$  & SYN$\downarrow$ & UDF$\downarrow$ & TUF$\downarrow$\\
        \midrule
        \multicolumn{9}{l}{\textbf{\textbf{miniF2F-Test}}}\\
        \midrule
        \multirow{2}{*}{ZS} & Direct & 25.41 & 48.90 & 1.23 & 1.23 & 6.15 & 23.77 & 7.38\\
          & Post-FDG & 67.21 & 81.88 & 0.00 & 0.00 & 3.28 & 2.87 & 5.33\\
        \midrule
        \multirow{2}{*}{(ZS) + Detailed} & Direct & 37.30 & 63.28 & 2.05 & 1.23 & 5.74 & 9.02 & 8.61\\
          & Post-FDG & 83.61 & 91.47 & 0.00 & 0.00 & 2.05 & 0.82 & 3.28\\
        \midrule
        \multicolumn{9}{l}{\textbf{Def\_Wiki-Test}}\\
        \midrule
        \multirow{2}{*}{ZS} & Direct & 10.87 & 16.43 & 8.70 & 8.70 & 15.22 & 52.17 & 13.04\\
          & Post-FDG & 34.78 & 43.19 & 6.52 & 0.00 & 23.91 & 19.57 & 28.26\\
        \midrule
        \multirow{2}{*}{(ZS) + Detailed} & Direct & 19.57 & 23.77 & 8.70 & 8.70 & 8.70 & 47.83 & 10.87\\
          & Post-FDG & 43.48 & 50.76 & 6.52 & 0.00 & 17.39 & 10.87 & 28.26\\
        \midrule
        \multicolumn{9}{l}{\textbf{Def\_ArXiv}}\\
        \midrule
        \multirow{2}{*}{ZS} & Direct & 13.33 & 19.30 & 0.00 & 0.00 & 23.33 & 66.67 & 6.67\\
         & Post-FDG & 23.33 & 36.02 & 0.00 & 0.00 & 60.00 & 13.33 & 13.33\\
        \midrule
        \multirow{2}{*}{(ZS) + Detailed} & Direct & 16.67 & 28.91 & 0.00 & 0.00 & 23.33 & 46.67 & 20.00\\
          & Post-FDG & 30.00 & 44.15 & 0.00 & 0.00 & 56.67 & 13.33 & 3.33\\
        \bottomrule
    \end{tabular}
    \caption{Symbolic refinement of GPT-4o results on three dataset.}
    \label{tab:sr}
\end{table*}


\begin{table*}[!t]
    \centering
    \small
    \begin{tabular}{l l c c| c c c c c}
        \toprule
        Prompt Strategy & Preamble & Pass$\uparrow$ & FEO$\uparrow$ & TRO$\downarrow$ & IVI$\downarrow$  & SYN$\downarrow$ & UDF$\downarrow$ & TUF$\downarrow$\\
        \midrule
        \multicolumn{9}{l}{\textbf{GPT-4o}}\\
        \midrule
        \multirow{2}{*}{Soft-IFDC} & Direct & 6.52 & 11.45 & 8.70 & 0.00 & 17.39 & 71.74 & 2.17\\
         & Post-FDG & 19.57 & 29.65 & 0.00 & 0.00 & 34.78 & 30.43 & 26.09\\
        \midrule
        \multirow{2}{*}{Hard-IFDC} & Direct & 4.35 & 11.86 & 10.87 & 0.00 & 10.87 & 69.57 & 6.52\\
          & Post-FDG & 19.57 & 26.95 & 0.00 & 0.00 & 36.96 & 21.74 & 39.13\\
        \midrule
        \multirow{2}{*}{(ZS) + Soft-IFDC + Binary} & Direct & 15.22 & 20.47 & 8.70 & 2.17 & 15.22 & 58.70 & 10.87\\
          & Post-FDG & 41.30 & 51.09 & 6.52 & 0.00 & 26.09 & 10.87 & 26.09\\
        \midrule
        \multirow{2}{*}{(ZS) + Soft-IFDC + Detailed} & Direct & 15.22 & 20.20 & 8.70 & 2.17 & 13.04 & 56.52 & 13.04\\
          & Post-FDG & 41.30 & 51.26 & 6.52 & 0.0 & 23.91 & 10.87 & 26.09\\
        \bottomrule
    \end{tabular}
    \caption{Prompt-FDG results on Def\_Wiki test set.}
    \label{tab:wiki_fdc_full}
\end{table*}

\end{document}
