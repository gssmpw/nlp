\section{Introduction}
\label{sec:main:introduction}

\begin{figure}[!t]
\centering
\includegraphics[width=0.48\textwidth,]{figure/concept.pdf}
\vspace{-5mm}
\caption{Conceptual illustration of our proposed \textbf{Brain Dynamics with Optimal control (BDO)}. The ROI signals observed at \textit{discrete} time points are encoded into an optimal control policy, which steers the \textit{continuous} latent state dynamics. The pre-trained optimal control policy is then utilized for various downstream tasks.}
\vspace{-6mm}
\end{figure}


Functional Magnetic Resonance Imaging (fMRI) measures changes in the blood-oxygen-level-dependent (BOLD) signal, an indirect and noisy observation of underlying neural activity~\citep{doi:10.1073/pnas.87.24.9868}. These signals reflect latent brain dynamics that are fundamental to understanding human cognition and psychopathology~\citep{LeeEtAl2022b, CaiEtAl2021, TaghiaEtAl2018}. A central goal of fMRI analysis is to extract, interpret, and understand this unobserved latent signal, as it provides valuable insights into brain function and its perturbations in disease states.

State-Space Models (SSMs) are a natural choice for modeling the latent processes underlying fMRI data, as they explicitly account for the dynamics of unobserved states and their relationship to noisy observations~\citep{FRISTON20031273, 10.1093/nsr/nwae079}. In neuroscience, SSMs have been extensively employed in methods like Dynamic Causal Modeling (DCM)~\citep{FRISTON20031273,triantafyllopoulos2021bayesian} to infer effective connectivity through Bayesian filtering. Other applications include modeling dynamic functional connectivity and capturing time-varying patterns in resting-state fMRI. However, traditional SSM approaches often impose strong simplifying assumptions, such as linearity in the state dynamics and observation models, which limit their ability to capture the complex, non-linear, and high-dimensional nature of brain activity. Moreover, they do not fully leverage modern machine learning techniques, leaving significant potential untapped. As a result, conventional SSMs may be unsuitable for building foundation models for various real-world applications.

\vspace{3mm}
Recently, the field has seen a surge in interest in self-supervised learning (SSL)~\cite{lecun2022path,he2022masked} approaches for fMRI data, which aim to learn transferrable representations from brain signals. Notable models, such as BrainLM~\citep{caro2024brainlm} and BrainJEPA~\citep{dong2024brain}, have showcased the potential of SSL in extracting representations that generalize well across diverse tasks and datasets. These models rely on SSL objectives such as masked prediction~\citep{he2022masked} or joint-embedding frameworks~\citep{assran2023self} to uncover structure in the data without requiring explicit labels. While these methods excel at learning global representations, they inherently lack the inductive biases necessary to capture key properties of the fMRI signal, particularly its temporal structure and the uncertainty arising from its noisy and indirect nature.

The absence of a principled approach to modeling temporal dynamics in SSL frameworks is a critical limitation for fMRI data. Unlike natural images, fMRI recordings are time-series, where the observed BOLD signal evolves over time and reflects latent neural activity. Purely data-driven SSL methods~\citep{caro2024brainlm, dong2024brain} often treat these signals as independent or use heuristics to aggregate information across time, which may overlook crucial temporal dependencies. This limitation restricts the ability of SSL models to fully capture the dynamic nature of brain activity, potentially missing fine-grained patterns that are essential for understanding underlying neural mechanisms.

In this work, we propose \textbf{Brain Dynamics with Optimal control (BDO)}, a novel approach that bridges the strengths of state-space modeling and modern representation learning. BDO introduces a continuous-discrete SSM framework powered by stochastic optimal control (SOC)~\citep{fleming2006controlled, carmona2016lectures} and amortized inference. To ensure scalability and utility as a foundation model, BDO incorporates SSL principles, enabling it to extract transferrable representations from large-scale datasets. The resulting model achieves state-of-the-art performance on a wide range of downstream tasks, including demographic prediction, trait analysis, and clinical diagnosis, while demonstrating robust scalability, efficiency, and interpretability. By addressing the limitations of traditional SSMs and leveraging the latest advances in machine learning, BDO sets a new standard for modeling brain dynamics from fMRI data. We summarize our contributions as follows:
\vspace{-2mm}
\begin{itemize}[leftmargin=10pt]
    \item We combine continuous-discrete SSMs under SOC theory with SSL to capture transferable representations.
    \item Built on the SOC formulation, our amortized inference scheme enables efficient and scalable learning.
    \item We demonstrate that our approach outperforms baselines on a variety of downstream tasks, maintaining both computational efficiency and robust scalability.
\end{itemize}
