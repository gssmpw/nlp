\begin{table*}[t!]
\caption{An example of differences in the citation from baseline vs BoN. Related information are highlighted in the context/response.}
\label{tab:q2}
\centering
\small
\resizebox{\linewidth}{!}{
\renewcommand{\arraystretch}{1.0} % row height
\begin{tabular}{p{0.15\textwidth} p{0.85\textwidth}}
\toprule
\bf Sent. ID & \textbf{Context Sentences} (only showing cited sentences due to space) \\
\midrule
\textbf{299 (\ding{55})} 
  & Ribosomes link amino acids together in the order specified by the codons of messenger RNA (mRNA) molecules to form polypeptide chains.\\
\midrule
\textbf{300 (\ding{51})} 
  & \hl{Ribosomes consist of two major components: the small and large ribosomal subunits.}\\
\midrule
\textbf{301 (\ding{51})} 
  & \hl{Each subunit consists of one or more ribosomal RNA (rRNA) molecules and many ribosomal proteins (RPs or r-proteins).}\\
\midrule
\textbf{302 (\ding{51})} 
  & \hl{The ribosomes and associated molecules are also known as the translational apparatus.}\\
\midrule
\textbf{383 (\ding{55})} 
  & Ribosomes are minute particles consisting of RNA and associated proteins that function to synthesize proteins. \\
  \midrule
\textbf{384 (\ding{55})} 
  & Proteins are needed for many cellular functions such as repairing damage or directing chemical processes. \\
  \midrule
\textbf{385 (\ding{55})} 
  & Ribosomes can be found floating within the cytoplasm or attached to the endoplasmic reticulum.\\
\midrule
\textbf{390 (\ding{51})} 
  & \hl{Each subunit is made of one or more rRNAs and many r-proteins.}\\
\midrule
\textbf{391 (\ding{55})} 
  & The small subunit (30S in bacteria and archaea, 40S in eukaryotes) has the decoding function, whereas the large subunit (50S in bacteria and archaea, 60S in eukaryotes) catalyzes the formation of peptide bonds, referred to as the peptidyl-transferase activity.\\
\midrule
\textbf{392 (\ding{55})} 
  & The bacterial (and archaeal) small subunit contains the 16S rRNA and 21 r-proteins (Escherichia coli), whereas the eukaryotic small subunit contains the 18S rRNA and 32 r-proteins (Saccharomyces cerevisiae; although the numbers vary between species).\\
\midrule
\textbf{393 (\ding{55})} 
  & The bacterial large subunit contains the 5S and 23S rRNAs and 34 r-proteins (E. coli), with the eukaryotic large subunit containing the 5S, 5.8S, and 25S / 28S rRNAs and 46 r-proteins (S. cerevisiae; again, the exact numbers vary between species).\\
\midrule
\midrule
\bf Query & The large subunit and small subunit that use two types of RNA are major components that make up what? \\
\midrule
\textbf{Response} (single statement due to space)
 & [...] The ribosome consists of \hl{two major components: the small and large ribosomal subunits.} \hl{Each subunit consists of one or more ribosomal RNA (rRNA) molecules and many ribosomal proteins (RPs or r-proteins).} \hl{The ribosomes and associated molecules are also known as the translational apparatus}. [...]\\
\midrule
\multicolumn{2}{l}{\textbf{Citation Strings (\textcolor{ForestGreen}{green: correct}; \textcolor{red}{red: wrong})}} \\
\midrule
\textbf{Baseline} 
 & \texttt{\textcolor{red}{[299-}\textcolor{ForestGreen}{302]}\textcolor{red}{[383-385]}\textcolor{ForestGreen}{[390}\textcolor{red}{-393]}} \\
\textbf{\ours BoN} 
 & \texttt{\textcolor{ForestGreen}{[300-302]}\textcolor{ForestGreen}{[390}\textcolor{red}{-393]}} \\
\midrule
\end{tabular}
}
\end{table*}