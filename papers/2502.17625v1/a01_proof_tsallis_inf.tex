This appendix provides the proof of Theorem~\ref{thm:Tsallis-INF}.
We include the proof of Theorem~\ref{thm:Tsallis-INF},  
as the corresponding proof is not provided in \citet{zimmert2021tsallis},  
and several aspects differ from their setting:  
the range of the loss is different,  
noisy feedback $r_t$ such that $\E[r_t \mid i_t, j_t] = A(i_t, j_t)$ is observed,  
and a negative term is introduced to ensure last-iterate convergence.

We begin by providing the following standard regret upper bound of FTRL.
By refining an analysis of FTRL,
we obtain a negative term of $- \frac{1}{\eta_{T+1}} D (x^*, x_{T+1} )$,
which is used to provide the last-iterate convergence result of  Proposition~\ref{prop:last-iterate}.
\begin{lemma}\label{lem:ftrl_bound}
    Let $\mathcal{X} \in \mathbb{R}^n$ be a non-empty compact convex set. 
    Let $\psi$ be a continuously differentiable convex function over $\mathcal{X}$.
    Suppose that $x_t$ is given by FTRL with the regularizer function $\psi$ 
    and learning rates $\eta_1 \ge \eta_2 \ge \cdots \ge \eta_{T+1} > 0$,
    as follows:
    \begin{align}
        \label{eq:defFTRL}
        x_{t} \in \argmin_{ x \in \cX } \left\{ 
        \sum_{s=1}^{t-1} \ell_s^\top x
        +
        \frac{1}{\eta_t} \psi(x)
        \right\}.
    \end{align}
    Then,
    for any $x^* \in \mathcal{X}$,
    we have
    \begin{align}
        \sum_{t=1}^T  \ell_t^\top \left( x_t - x^* \right)
        &
        \le
        \sum_{t=1}^T
        \rbrm[\Big]{
            \ell_t^\top (x_t - x_{t+1})
            -
            \frac{1}{\eta_t} D(x_{t+1}, x_t)
        }
        +
        \sum_{t=1}^T
            \rbrm[\Big]{
            \frac{1}{\eta_{t+1}}
            -
            \frac{1}{\eta_{t}}
            }
            \left(
            \psi(x^*)
            -
            \psi( x_{t+1} )
            \right)
        \nonumber
        \\
        &
        \quad
        \quad
        +
        \frac{1}{\eta_1} \left(
        \psi(x^*)
        -
        \psi(x_1)
        \right)
        -
        \frac{1}{\eta_{T+1}}
        D(x^*, x_{T+1}),
        \label{eq:bound_FTRL_with_negative}
    \end{align}
    where $D(\cdot, \cdot )$ is the Bregman divergence associated with $\psi$:
    $D(y, x) = \psi(y) - \psi(x) - \nabla \psi(x)^\top (y - x)$.
\end{lemma}
\begin{proof}
    We have
    \begin{align}
        &
        \sum_{t=1}^T \ell_t^\top x^* + \frac{1}{\eta_{T+1}} \psi(x^*)
        \nonumber \\
        &
        \ge
        \sum_{t=1}^T \ell_t^\top x_{T+1} + \frac{1}{\eta_{T+1}} \psi(x_{T+1})
        +
        \frac{1}{\eta_{T+1}} D(x^*, x_{T+1})
        \nonumber \\
        &
        =
        \sum_{t=1}^{T-1} \ell_t^\top x_{T+1} + 
        \frac{1}{\eta_{T}} \psi(x_{T+1})
        +
        \ell_{T}^\top x_{T+1}
        +
        \left(
        \frac{1}{\eta_{T+1}} 
        -
        \frac{1}{\eta_{T}} 
        \right)
        \psi(x_{T+1})
        +
        \frac{1}{\eta_{T+1}} D(x^*, x_{T+1})
        \nonumber \\
        &
        \ge
        \sum_{t=1}^{T-1} \ell_t^\top x_{T} + 
        \frac{1}{\eta_{T}} \psi(x_{T})
        +
        \frac{1}{\eta_{T}} D(x_{T+1},x_T)
        +
        \ell_{T}^\top x_{T+1}
        \nonumber\\
        &\qquad+
        \rbrm[\Big]{
        \frac{1}{\eta_{T+1}} 
        -
        \frac{1}{\eta_{T}} 
        }
        \psi(x_{T+1})
        +
        \frac{1}{\eta_{T+1}} D(x^*, x_{T+1})
        \nonumber \\
        &
        \ge
        \cdots
        \nonumber \\
        &\ge
        \frac{1}{\eta_1} \psi(x_{1})
        +
        \sum_{t=1}^{T} 
        \left(
        \frac{1}{\eta_t}
        D(x_{t+1}, x_t)
        +
        \ell_t^\top x_{t+1}
        +
        \rbrm[\Big]{
        \frac{1}{\eta_{t+1}} 
        -
        \frac{1}{\eta_{t}} 
        }
        \psi(x_{t+1})
        \right)
        +
        \frac{1}{\eta_{T+1}} D(x^*, x_{T+1}),
        \nonumber 
    \end{align}
    where each inequality follows from the definition of FTRL \eqref{eq:defFTRL}
    and the first-order optimality condition.
    This immediately leads to the desired inequality.
\end{proof}

We next provide lemmas to the upper bound the first summation in the RHS of \eqref{eq:bound_FTRL_with_negative} for the $1/2$-Tsallis entropy.
\begin{lemma}\label{lem:tsalis_stab_onedim}
Let $\phi(x) = - 2 \sqrt{x}$
and
$D_\phi(y, x) = - 2 \sqrt{y} + 2 \sqrt{x} + (y-x)/\sqrt{x} = (\sqrt{y} - \sqrt{x})^2 / \sqrt{x}$ 
be the Bregman divergence associated with $\phi$.
Then, for any 
$x \in (0,1)$
and 
$a > -1/\sqrt{x}$,
\begin{equation}
    \max_{y \in (0, \infty)} 
    \left\{
        a (x - y) - D_\phi(y, x) 
    \right\}
    \leq 
    \sqrt{x} \, \xi(a \sqrt{x})
    \nonumber
\end{equation}
for $\xi(z) = z^2 / (1 + z)$ for $z \geq 0$.
If $a \geq - 1/(2\sqrt{x})$, then it also holds that 
\begin{equation}
    \max_{y \in (0, \infty)} 
    \left\{
        a (x - y) - D_\phi(y, x) 
    \right\}
    \leq 
    2 x^{3/2} a^2
    .
    \nonumber
\end{equation}
\end{lemma}
\begin{proof}
We have
\begin{align}
  a (x - y) - D_\phi(y, x)
  &=
  a (x - y)
  -
  \frac{(\sqrt{y} - \sqrt{x})^2}{\sqrt{x}}
  \nonumber \\
  &=
  (\sqrt{x} - \sqrt{y})
  \left\{
      a (\sqrt{x} + \sqrt{y})
      -
      \frac{ \sqrt{x} - \sqrt{y} }{\sqrt{x}}
  \right\}
  \nonumber \\
  &=
  (\sqrt{x} - \sqrt{y})
  \left\{
      a \left(2 \sqrt{x} - (\sqrt{x} - \sqrt{y}) \right)
      -
      \frac{ \sqrt{x} - \sqrt{y} }{\sqrt{x}}
  \right\}
  \nonumber \\
  &=
  2 a \sqrt{x}
  (\sqrt{x} - \sqrt{y})
  -
  \left(
    a + \frac{1}{\sqrt{x}}
  \right)
  (\sqrt{x} - \sqrt{y})^2
  \nonumber \\
  &\leq
  \frac{(2 a \sqrt{x})^2}{4 \left(a + \frac{1}{\sqrt{x}}\right)}
  =
  \frac{a^2 x^{3/2}}{a \sqrt{x} + 1}
  =
  \sqrt{x} \xi(a \sqrt{x})
  ,
  \nonumber
\end{align}
where we used $c_1 z - c_2 z^2 \leq c_1^2 / (4 c_2)$ for $c_1 \geq 0$ and $c_2 > 0$ with $a > - 1/\sqrt{x}$.
The second statement of the lemma follows since 
$\xi(z) \leq 2 z^2$ for any $z \geq - 1/2$.
This completes the proof.
\end{proof}

Define 
\begin{equation}
    \tilde{\ell}_t(i)
    =
    \frac{\one[i_t=i](1 - r_t)}{x_t(i)} \in \left[0, \frac{2}{x_t(i)} \right]
    .
    \nonumber
\end{equation}
Then, using Lemma~\ref{lem:tsalis_stab_onedim}, we can prove the following lemma, which plays a key role in proving Theorem~\ref{thm:Tsallis-INF}.
\begin{lemma}\label{lem:tsalis_stab_multidim}
Suppose that $\eta_t \leq 1/4$.
Then it holds that
\begin{equation}
        \tilde{\ell}_t^\top 
    (x_t - x_{t+1})
    -
    \frac{1}{\eta_t} D(x_{t+1}, x_t)
    \leq
    4
    \eta_t
    \sum_{i=1}^m
    x_t(i)^{3/2} (1 - x_t(i))
    \tilde{\ell}_t(i)^2
    .
    \nonumber
\end{equation}
\end{lemma}
\begin{proof}
Let $k \in \argmax_{i \in [m]} x_t(i)$.
We then have 
\begin{align}
    \tilde{\ell}_t^\top (x_t - x_{t+1})
    -
    \frac{1}{\eta_t} D(x_{t+1}, x_t)
    &=
    \left(
        \tilde{\ell}_t - x_t(k) \tilde{\ell}_t(k) \mathbf{1}
    \right)^\top 
    (x_t - x_{t+1})
    -
    \frac{1}{\eta_t} D(x_{t+1}, x_t)
    \nonumber \\
    &\leq
    2
    \eta_t
    \sum_{i=1}^m 
    x_t(i)^{3/2} \left( 
        \tilde{\ell}_t(i) - x_t(k) \tilde{\ell}_t(k)
    \right)^2
    ,
    \label{eq:stab_0}
\end{align}
where in the inequality we used  
the second statement in Lemma~\ref{lem:tsalis_stab_onedim}
with
\begin{equation}
    \sqrt{x_t(i)} \eta_t 
    \left( \tilde{\ell}_t(i) -  x_t(k) \tilde{\ell}_t(k) \right)
    \geq
    - \sqrt{x_t(i)} \eta_t x_t(k) \tilde{\ell}_t(k)
    \geq
    - \eta_t (1 - r_t)
    \geq 
    - \frac12
    \nonumber
\end{equation}
for each $i \in [m]$,
which is due to the assumption that $\eta_t \leq 1/4$ and $1 - r_t \in [0, 2]$.
We will upper bound the RHS of~\eqref{eq:stab_0} below.
First we have
\begin{align}
    &
    \sum_{i=1}^m 
    x_t(i)^{3/2} \left( 
        \tilde{\ell}_t(i) - x_t(k) \tilde{\ell}_t(k)
    \right)^2
    \nonumber \\
    &=
    x_t(k)^{3/2}
    \left( 
        1 - x_t(k)
    \right)^2
    \tilde{\ell}_t(k)^2
    +
    \sum_{i \neq k}
    x_t(i)^{3/2} \left( 
        \tilde{\ell}_t(i) - x_t(k) \tilde{\ell}_t(k)
    \right)^2
    .
    \label{eq:stab_1}
\end{align}
The second term in the last equality is upper bounded by
\begin{align}
    \sum_{i \neq k}
    x_t(i)^{3/2} \left( 
        \tilde{\ell}_t(i) - x_t(k) \tilde{\ell}_t(k)
    \right)^2
    \leq
    \sum_{i \neq k}
    x_t(i)^{3/2} 
        \tilde{\ell}_t(i)^2 
    +
    x_t(k)^2 \tilde{\ell}_t(k)^2
    \sum_{i \neq k}
    x_t(i)^{3/2} 
    \label{eq:stab_2}
    .
\end{align}
The second term in the last inequality is further upper bounded by
\begin{align}
    x_t(k)^2 \tilde{\ell}_t(k)^2
    \sum_{i \neq k}
    x_t(i)^{3/2} 
    &\leq
    x_t(k)^2 \tilde{\ell}_t(k)^2
    \rbrm[\bigg]{
        \sum_{i \neq k}
        x_t(i)
    }^{3/2}
    \leq
    x_t(k)^{3/2} \tilde{\ell}_t(k)^2
    \rbrm[\bigg]{
        \sum_{i \neq k}
        x_t(i)
    }
    \nonumber \\
    &
    =
    x_t(k)^{3/2} \tilde{\ell}_t(k)^2
    \left(
        1 - x_t(k)
    \right)
    ,
    \label{eq:stab_3}
\end{align}
where the first inequality follows from the superadditivity of $z \mapsto z^{3/2}$ for $z \geq 0$
and the second inequality follows from $\sum_{i \neq k} x_t(i) \in [0,1]$.
Combining \eqref{eq:stab_1}, \eqref{eq:stab_2}, and \eqref{eq:stab_3},
we have
\begin{align}
    &
    \sum_{i=1}^m 
    x_t(i)^{3/2} \left( 
        \tilde{\ell}_t(i) - x_t(k) \tilde{\ell}_t(k)
    \right)^2    
    \nonumber \\
    &\leq
    x_t(k)^{3/2}
    \left( 
        1 - x_t(k)
    \right)^2
    \tilde{\ell}_t(k)^2
    +
    \sum_{i \neq k}
    x_t(i)^{3/2}
    \tilde{\ell}_t(i)^2
    +
    x_t(k)^{3/2}
    \tilde{\ell}_t(k)^2
    (1 - x_t(k))
    \nonumber \\
    &\leq
    2
    x_t(k)^{3/2}
    \left( 
        1 - x_t(k)
    \right)
    \tilde{\ell}_t(k)^2
    +
    2
    \sum_{i \neq k}
    x_t(i)^{3/2} (1 - x_t(i))
    \tilde{\ell}_t(i)^2
    \nonumber \\
    &=
    2
    \sum_{i=1}^m
    x_t(i)^{3/2} (1 - x_t(i))
    \tilde{\ell}_t(i)^2
    ,
    \nonumber
\end{align}
where the second inequality follows from $1 - x_t(i) \geq 1/2$ for $i \neq k$ since $x_t(i) \leq 1/2$ for $i \neq k$.
Finally, combining \eqref{eq:stab_0} with the last inequality, we obtain the desired bound.
\end{proof}

Finally, we are ready to prove Theorem \ref{thm:Tsallis-INF}.
We will show that Theorem~\ref{thm:Tsallis-INF} holds with $C_1 = 19$ and $C_2 = 2$.
\begin{proof}[Proof of Theorem \ref{thm:Tsallis-INF}]
Let $x \in \cP_m$ and $T_0 = 4$.
When $m = 1$, the LHS and RHS of~\eqref{eq:Tsallis-INF-upper} are $0$, and thus we consider the case of $m \geq 2$ below.
Recall 
$
\tilde{\ell}_t(i)
=
\frac{\one[i_t=i](1 - r_t)}{x_t(i)}.
$
Then, the regret of the row player can be rewritten as
\begin{align}
    &
    \Rr_T(x)
    =
    \E\sbrm[\bigg]{
        \sum_{t=1}^T (x_t - x)^\top \ell_t
    }
    \leq
    \E\sbrm[\bigg]{
        \sum_{t=T_0+1}^T (x_t - x)^\top \ell_t
    }
    +
    2 T_0
    =
    \E\sbrm[\bigg]{
        \sum_{t=T_0+1}^T (x_t - x)^\top \hat{\ell}_t
    }
    +
    2 T_0
    \nonumber \\
    &=
    \E\sbrm[\bigg]{
        \sum_{t=T_0+1}^T (x_t - x)^\top \tilde{\ell}_t
        +
        \sum_{t=T_0+1}^T (x_t - x)^\top \left( \hat{\ell}_t - \tilde{\ell}_t \right)
    }
    +
    2 T_0
    =
    \E\sbrm[\bigg]{
        \sum_{t=T_0+1}^T (x_t - x)^\top \tilde{\ell}_t
    }
    +
    2 T_0
    ,
    \label{eq:ftrl_tsallis_0}
\end{align}
where the second equality follows from the unbiasedness of $\hat{\ell}_t$
and the last equality follows from $\hat{\ell}_t - \tilde{\ell}_t = \mathbf{1}$.
From the fact that the outputs of FTRL with loss estimator $\hat{\ell}_t$ and $\tilde{\ell}_t$ are the same and Lemma~\ref{lem:ftrl_bound}, the inside of the expectation in \eqref{eq:ftrl_tsallis_0} is upper bounded by
\begin{align}
    \sum_{t=T_0+1}^T \tilde{\ell}_t^\top \left( x_t - x \right)
    &\leq
    \sum_{t=T_0+1}^T
    \rbrm[\Big]{
        \tilde{\ell}_t^\top (x_t - x_{t+1})
        -
        \frac{1}{\eta_t} D(x_{t+1}, x_t)
        }
    \nonumber \\
    &\qquad+
    \sum_{t=T_0+1}^T
    \rbrm[\Big]{
        \frac{1}{\eta_{t+1}}
        -
        \frac{1}{\eta_{t}}
    }
    \rbrm[\Big]{
        \psi(x^*)
        -
        \psi( x_{t+1} )
    }
    \nonumber \\
    &\qquad+
    \frac{1}{\eta_{T_0+1}} \left(
        \psi(x^*)
        -
        \psi(x_{T_0+1})
    \right)
    -
    \frac{1}{\eta_{T+1}}
    D(x^*, x_{T+1}),
    \label{eq:ftrl_tsallis_1}
\end{align}
We first consider the first term in \eqref{eq:ftrl_tsallis_1}.

For $t \geq T_0 = 4$, 
we have $\eta_t = 1/(2\sqrt{t}) \leq 1/4$,
and thus from Lemma~\ref{lem:tsalis_stab_multidim},
\begin{align}
    \tilde{\ell}_t^\top (x_t - x_{t+1})
    -
    \frac{1}{\eta_t} D(x_{t+1}, x_t)
    \leq
    4
    \eta_t
    \sum_{i=1}^m
    x_t(i)^{3/2} (1 - x_t(i))
    \tilde{\ell}_t(i)^2
    .
    \label{eq:ftrl_tsallis_stab2}
\end{align}
Let $i^* \in [m]$.
Then,
using $\E_{r_t,i_t,j_t}[ \tilde{\ell}_t(i)^2 \mid x_t] \leq 4 / x_t(i)$,
we have
\begin{align}
    &
    \E\nolimits_{r_t, i_t, j_t} \sbrm[\bigg]{
        \sum_{i=1}^m
        x_t(i)^{3/2} (1 - x_t(i))
        \tilde{\ell}_t(i)^2
        \mid
        x_t
    }
    \leq
    4
    \sum_{i=1}^m
    \sqrt{x_t(i)} (1 - x_t(i))
    \nonumber \\
    &\leq
    4
    \sum_{i \neq i^*}
    \sqrt{x_t(i)}
    +
    4
    (1 - x_t(i^*))
    \leq
    8
    \sum_{i \neq i^*}
    \sqrt{x_t(i)}
    ,
    \label{eq:ftrl_tsallis_stab_final}
\end{align}
where the last inequality follows from
$1 - x_t(i^*) = \sum_{i \neq i^*} x_t(i) \leq \sum_{i \neq i^*} \sqrt{x_t(i)}$.


We next consider the second and third terms in \eqref{eq:ftrl_tsallis_1}.
We first observe that 
$
{1}/{\eta_{t+1}}
-
{1}/{\eta_{t}}
=
2 (\sqrt{t+1} - \sqrt{t}) 
\leq
1/\sqrt{t}
\leq
\sqrt{2 / (t+1)}
$
and 
$\psi(x^*) - \psi(x_{t+1})
\leq
2 \sum_{i=1}^m \sqrt{x_{t+1}(i)} - 2
\leq
2 \sum_{i \in [m] \setminus \{i^*\}} \sqrt{x_{t+1}(i)}.
$
Using these inequalities, we have
\begin{align}
    &
    \sum_{t=T_0+1}^{T-1}
    \rbrm[\Big]{
        \frac{1}{\eta_{t+1}}
        -
        \frac{1}{\eta_{t}}
    }
    \left(
        \psi(x^*)
        -
        \psi( x_{t+1} )
    \right)
    \nonumber \\
    &\leq
    2 \sqrt{2} \sum_{t=T_0+1}^{T}
    \frac{1}{\sqrt{t+1}}
    \sum_{i \in [m] \setminus \{i^*\}} \sqrt{x_{t+1}(i)}
    \leq
    2 \sqrt{2} \sum_{t=T_0+2}^{T}
    \frac{1}{\sqrt{t}}
    \sum_{i \in [m] \setminus \{i^*\}} \sqrt{x_t(i)}
    +
    2 \sqrt{2} \sqrt{\frac{m}{T+1}}
    ,
    \label{eq:ftrl_tsallis_penalty_final}
\end{align}
where in the last inequality we used the Cauchy-Schwarz inequality.
The remaining term in \eqref{eq:ftrl_tsallis_1} is at most
\begin{align}
    \frac{1}{\eta_{T_0+1}} \left(
        \psi(x^*)
        -
        \psi(x_{T_0+1})
    \right)    
    \leq
    2 \sqrt{2} \sqrt{T_0 + 1} \sum_{i \neq i^* } \sqrt{x_{T_0+1}(i)}
    \leq
    2 \sqrt{10} \sum_{i \neq i^* } \sqrt{x_{T_0+1}(i)}
    .
    \label{eq:ftrl_tsallis_remaining}
\end{align}
Finally, by combining \eqref{eq:ftrl_tsallis_0} with \eqref{eq:ftrl_tsallis_1}, \eqref{eq:ftrl_tsallis_stab_final}, \eqref{eq:ftrl_tsallis_penalty_final}, and \eqref{eq:ftrl_tsallis_remaining},
we obtain that for any $i^* \in [m]$,
\begin{align}
    \Rr_T(x)
    &\leq
    2 T_0
    +
    2 \sqrt{2}
    \sqrt{\frac{m}{T+1}}
    +
    \E \sbrm[\bigg]{
        19
        \sum_{t=T_0+1}^T
        \sum_{i \in [m] \setminus \{i^*\}}
        \sqrt{\frac{x_t(i)}{t}}
        -
        2
        \sqrt{T+1}
        \cdot 
        D(x, x_{T+1})
    }.
    \label{eq:theorem1_pre}
\end{align}
Since we have 
$\sum_{t=1}^{T_0} \sum_{i \in [m] \setminus \{i^*\}} \sqrt{x_t(i)} 
\geq 
(\sqrt{m} - 1) + (T_0 - 1)
=
\sqrt{m} + 2
$,
the last inequality implies that the choice of 
$C_1 = 19$, which is larger than $\frac{2 T_0 + 2 \sqrt{2} \sqrt{m / (T+1)}}{\sqrt{m} + 2} (\leq 4)$,
implies that 
the first three terms in \eqref{eq:theorem1_pre} is upper bounded by
\begin{equation}
    2 T_0
    +
    2 \sqrt{2}
    \sqrt{\frac{m}{T+1}}
    \leq
    C_1 (\sqrt{m} + 2)
    \leq 
    C_1 
    \sum_{t=1}^{T_0}
    \sum_{i \in [m] \setminus \{i^*\}} \sqrt{x_t(i)}    
    .
    \nonumber
\end{equation}
Combining this inequality with \eqref{eq:theorem1_pre} implies that 
Theorem~\ref{thm:Tsallis-INF} holds with $C_1 = 19$ and $C_2 = 2$ as desired.
\end{proof}







