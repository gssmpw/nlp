\section{Proof of Theorem~\ref{thm:RegLB}}
\label{sec:pfRegLB}
When $A$ is given by \eqref{eq:defADelta},
we can see that
$(\istar, \jstar)$ is a Nash equilibrium of the game with payoff matrix $A$.
In fact,
if $\xstar$ and $\ystar$ are the indicator vectors of $\istar$ and $\jstar$,
it holds for any $x \in \cP_m$ and $y \in \cP_n$ that
\begin{align*}
    x^\top A \ystar
    -
    \xstar^\top A y
    &
    =
    {\Deltac}^\top \ystar
    -
    x^\top \Deltar
    -
    {\Deltac}^\top y
    +
    \xstar^\top \Deltar
    =
    (\xstar - x)^\top \Deltar
    +
    (\ystar - y)^\top \Deltac
    \\
    &
    =
    - x^\top \Deltar
    - y^\top \Deltac
    =
    - 
    \rbrm[\Big]{
        \sum_{i \in [m]} \Deltar(i) x(i)
        +
        \sum_{j \in [n]} \Deltac(j) y(j)
    }
    \le
    0,
    \yestag\label{eq:NE}
\end{align*}
which means that $\DG(\xstar, \ystar)=0$.

In this section,
let $x_t \in \cP_m$ and $y_t \in \cP_n$ denote indicator vectors of $i_t \in [m]$ and $j_t \in [n]$,
respectively.
For any fixed algorithm and the true payoff matrix $A$,
we denote the regret of the algorithm as
\begin{align*}
    R_T (A)
    =
    \Rr_T(\xstar)
    +
    \Rc_T(\ystar)
    =
    \E \sbrm[\bigg]{
    \sum_{t=1}^T
    \rbrm[\big]{
    \xstar^\top A y_t
    -
    x_t^\top A \ystar
    }
    } .
\end{align*}
Then,
if $A$ is given by \eqref{eq:defADelta},
from \eqref{eq:NE},
we have
\begin{align}
    R_T(A)
    =
    \E \sbrm[\bigg]{
        \sum_{t=1}^T
        \rbrm[\big]{
            x_t^\top \Deltar
            +
            y_t^\top \Deltac
        }
    }
    =
    \sum_{i=1}^m
    \Deltar(i) \Tr_{T,i}(A)
    +
    \sum_{j=1}^n
    \Deltac(j) \Tc_{T,j}(A).
    \label{eq:RDeltaLB}
\end{align}

We can show Theorem~\ref{thm:RegLB} by using the following lemma:
\begin{lemma}
    \label{lem:TiLB}
    Suppose $A$ is given by \eqref{eq:defADelta}.
    Fix an arbitrary $i \in [m] \setminus \{ \istar \}$.
    Let 
    $\tilde{\Deltar} = \Deltar - 2 \Deltar_i \chi_i$
    and
    $
    \tilde{A}
    =
    \mathbf{1}_m {\Deltac}^\top 
    -
    \tilde{\Deltar} \mathbf{1}_n^\top 
    $.
    We then have
    \begin{align*}
        ({\Deltar}(i))^2
        \Tr_{T,i}(A)
        \ge
        \frac{1}{5}
        \ln
        {
        \frac{T}{2 \rbrm[\big]{
        \Tr_{T,i}(A) +  T - \Tr_{T,i}(\tilde{A})
        }}
        }.
    \end{align*}
\end{lemma}
\begin{proof}
    Note first that,
    for $p \in [3/8, 1/2]$ and $\delta \in [0, 1/4]$,
    we have
    \begin{align*}
        \KL( p , p + \delta )
        &
        =
        p \ln \frac{p}{p+\delta}
        +
        (1 - p ) \ln \frac{ 1-p}{1-p-\delta}
        \\
        &
        =
        -
        p \ln \rbrm[\Big]{
        1
        +
        \frac{\delta}{p}
        }
        -
        (1 - p ) \ln \rbrm[\Big]{
        1
        -
        \frac{\delta}{1-p}
        }
        \\
        &
        \le
        p \ln \rbrm[\Big]{
        -
        \frac{\delta}{p}
        +
        \rbrm[\big]{\frac{\delta}{p}}^2
        }
        +
        (1 - p ) \ln \rbrm[\Big]{
        \frac{\delta}{1-p}
        +
        \rbrm[\big]{
        \frac{\delta}{1-p}
        }^2
        }
        \\
        &
        =
        \frac{\delta^2}{p(1-p)}
        \le
        5 \delta^2.
        \yestag\label{eq:KLpD}
    \end{align*}
    Let $P$ and $\tilde{P}$ be distributions of
    $\cbrm{ (i_t, j_t, \ell_t) }_{t\in [T]}$ for $A$ and $\tilde{A}$,
    respectively.
    Then,
    from the Bretagnolle-Huber inequality (e.g., \citealp[Corollary 4]{canonne2022short}),
    we have
    \begin{align*}
        \TV ( P, \tilde{P} )
        \le
        1 - \frac{1}{2} \exp ( -\KL (P, \tilde{P}) ) .
    \end{align*}
    From the chain rule for the KL divergence (e.g., \citealp[Lemma 15.1]{lattimore2020bandit}),
    we have
    \begin{align*}
        \KL ( P, \tilde{P} )
        &
        =
        \E_{\{(i_t, j_t, \ell_t)\} \sim P} 
        \sbrm[\bigg]{
        \sum_{t=1}^T 
        \KL ( \Berpm( A_{i_t, j_t} ), \Berpm( \tilde{A}_{i_t, j_t} ) ) 
        }
        \\
        &
        \le
        \E_{\{(i_t, j_t, \ell_t)\} \sim P} 
        \sbrm[\bigg]{ 
        \sum_{t=1}^T 
        \mathbf{1}[i_t = i]
        \cdot
        5 (\Deltar(i))^2
        }
        =
        5 \Tr_{T, i} (A) (\Deltar(i))^2 ,
    \end{align*}
    where the inequality follows from the definition of $\tilde{A}$ and \eqref{eq:KLpD}.
    By combining above inequalities,
    we obtain
    \begin{align*}
        \frac{1}{T}|\Tr_{T, i}(A) - \Tr_{T, i}(\tilde{A})|
        \le
        \TV ( P, \tilde{P} )
        \le
        1 - \frac{1}{2} \exp ( -\KL (P, \tilde{P}) )
        \le
        1 - \frac{1}{2} \exp ( -
        5 \Tr_{T, i} (A) (\Deltar(i))^2 
        ),
    \end{align*}
    which implies that
    \begin{align*}
        \Tr_{T, i} (A) (\Deltar(i))^2 
        \ge
        \frac{1}{5}
        \ln 
        {
        \frac{T}{2 \rbrm[\big]{
        \Tr_{T, i}(A) +  T - \Tr_{T, i}(\tilde{A})
        }}
        }.
    \end{align*}
\end{proof}

\begin{proof}[of Theorem~\ref{thm:RegLB}]
    If $\tilde{A}$ is given as in Lemma~\ref{lem:TiLB},
    from \eqref{eq:RDeltaLB},
    we have
    \begin{align*}
        R_T(A)
        \ge
        \Deltar(i) \Tr_{T, i}(A),
        \quad
        R_T(\tilde{A})
        \ge
        \Deltar(i) (T - \Tr_{T,i}(\tilde{A})).
    \end{align*}
    From this and Lemma~\ref{lem:TiLB},
    we have
    \begin{align*}
        (\Deltar(i))^2 \Tr_{T,i}(A)
        &
        \ge
        \frac{1}{5}
        \ln
        {
        \frac{T}{2 \rbrm[\big]{
        R_T(A) / \Deltar(i) +  R_T(\tilde{A}) / \Deltar(i)
        }}
        }.
    \end{align*}
    From the assumption that $R_T(\hat{A}) \le g(m,n) T^{1 - \epsilon}$ for any $\hat{A}$,
    we have
    \begin{align*}
        \frac{T}{
        R_T(A) / \Deltar(i) +  R_T(\tilde{A}) / \Deltar(i)
        }
        \ge
        \frac{T}{2 g(m, n) T^{1-c} /\Deltar(i) }
        =
        \frac{\Deltar(i) T^c}{2 g(m, n) },
    \end{align*}
    which implies
    \begin{align*}
        \Tr_{T,i}(A)
        \ge
        \frac{1}{5 (\Deltar(i))^2}
        {
            \ln {
                \frac{\Delta_i T^c}{4g(m, n)}
            }
        }.
    \end{align*}
    Consequently,
    we have
    \begin{align*}
        &
        \liminf_{T \rightarrow \infty}
        \frac{R_T(A)}{\ln T}
        =
        \liminf_{T \rightarrow \infty}
        \frac{1}{\ln T}
        \rbrm[\bigg]{
            \sum_{\substack{i \in [m]\\ \Deltar(i) > 0}}
            \Tr_{T,i}(A)
            +
            \sum_{\substack{j \in [n]\\ \Deltac(j) > 0}}
            \Deltac(j)
            \Tc_{T,j}(A)
        }
        \\
        &
        \ge
        \liminf_{T \rightarrow \infty}
        \rbrm[\bigg]{
            \sum_{\substack{i \in [m]\\ \Deltar(i) > 0}}
            \frac{1}{5\Deltar(i)}
            +
            \sum_{\substack{j \in [n]\\ \Deltac(j) > 0}}
            \frac{1}{5\Deltac(j)}
        }
        \rbrm[\bigg]{
            c+
            \frac{1}{\ln T}
            \ln {
                \frac{\Delta_i}{4g(m, n)}
            }
        }
        \\
        &
        =
        \frac{c}{5}
        \rbrm[\bigg]{
        \sum_{\substack{i \in [m]\\ \Deltar(i) > 0}}
        \frac{1}{\Deltar(i)}
        +
        \sum_{\substack{j \in [n]\\ \Deltac(j) > 0}}
        \frac{1}{\Deltac(j)}
        },
    \end{align*}
    which completes the proof.
\end{proof}
