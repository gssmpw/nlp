In this appendix section, we will prove our main regret bound theorem by first presenting a generalized formulation that encompasses both inequalities stated in the main text. We do this by first defining a unified notation of the gap parameters.

\begin{definition}[Admissible $(I, \Deltar, \pir)$ and $(J, \Deltac, \pic)$]\label{def:admissble}
    Denote $v = \max_{x \in \cP_m}\left\{ \min_{y \in \cP_n} x^\top A y \right\}$.
    An action subset $I\subseteq [m]$, a gap vector $\Deltar\in \fR_{\geq 0}^m$ and a mapping $\pir : \cP_m\to \xstarset$ are together called {admissible} for the row player if
    \begin{itemize}[leftmargin=*]
    \item The entries $\Deltar(i)$ are positive for every $i\not\in I$. %
    \item For any $x\in \cP_m$, the NE strategy $\xstar=\pir(x)\in \xstarset$ must satisfy:
    \begin{equation}
        \DG(x, \ystar) = v - \min_{y\in \cP_n}\cbrm[\big]{x^\top A y} \geq \Deltar \cdot \rbrm{ x - \xstar }_+, \label{eq:def-Delta-r}
    \end{equation}
where $\ystar\in\ystarset$ is an arbitrary NE strategy, and we define $(x)_+ = \max\{ x, 0 \}$ which applies entrywise to vectors.


The admissibility for a subset of actions $J\subseteq [n]$, a gap vector $\Deltac\in \fR_{\geq 0}^n$, and a mapping $\pic : \cP_n\to\ystarset$ can be analogously defined for the column player, with
\begin{equation}
        \DG(\xstar, y) = \min_{x\in \cP_m}\cbrm[\big]{x^\top A y} - v  \geq \Deltac \cdot \rbrm{ y - \ystar }_+, \label{eq:def-Delta-c}
\end{equation}
\end{itemize}
\end{definition}



The following is the full version of our main theorem.

\begin{theorem}\label{thm:general-bound-together-appx}
If both players follow the Tsallis-INF algorithm, then for any admissible $(I, \Deltar, \pir)$ and $ (J, \Deltac, \pic)$ (Definition~\ref{def:admissble}) such that $I \neq \emptyset$, $J \neq \emptyset $, we have
    \begin{align*}
       & \max\cbrm[\Big]{
            \Rr_T(x)
            +
            \sqrt{T}\constd{\E\sbrm[\big]{
                D(x, x_{T+1})
            }},
            \Rc_T(y)
            +
            \sqrt{T}\constd{\E\sbrm[\big]{
                D(y, y_{T+1})
            }}
        }\\
        &{}={} 
        O\rbrm[\Big]{
            \sqrt{T} \rbrm[\big] {
                \sqrt{\abs{I}-1}
                + \sqrt{\abs{J}-1}
                + \gammar \sqrt{\Logr}
                + \gammac \sqrt{\Logc}
            }
            +
            \omegar \Logr
            +
            \omegac \Logc
        } \yestag\label{eq:general-bound-together}
    \end{align*}
    for any $x \in \cP_m$ and $y \in \cP_n$,
    where
\begin{equation}
    \begin{aligned}
    \omegar & = \sum_{i \not\in I}
        \frac{1}{\Deltar(i)}, & 
    \gammar & = \max_{\xstar \in \xstarset} \sum_{i \not\in I} \sqrt{\xstar(i)}, &
    \Logr & = \min\cbrm[\Big]{
            \log_+ \frac{
                    T\rbrm{m-\absm{I}}
                }{
                    \omegar^2
                },
            \log_+ \frac{
                    m-\absm{I}
                }{
                    \gammar^2
                }
        },
        \\
    \omegac & = \sum_{j \not\in J}
        \frac{1}{\Deltac(j)}, & 
    \gammac & = \max_{\ystar \in \ystarset} \sum_{j \not\in J} \sqrt{\ystar(j)}, &
     \Logc & = \min\cbrm[\Big]{
            \log_+ \frac{
                    T\rbrm{n-\absm{J}}
                }{
                    {\omegac}^2
                },
            \log_+ \frac{
                    n-\absm{J}
                }{
                    \gammac^2
                }
        }.
    \end{aligned}    
    \label{eq:def-omega}
\end{equation}
\end{theorem}


We note that the first part of Theorem~\ref{thm:general-bound-together} is a special cases of this theorem.
In fact,
if $(\xstar, \ystar, I,J,\Deltar,\Deltac)$ are given by the first condition in Theorem~\ref{thm:general-bound-together}, we can verify the admissibility of $(I,\Deltar)$ by directly plugging the definition $\Deltar=\rbrm[\big]{\xstar^\top A \ystar}\one - A \ystar$ into \eqref{eq:def-Delta-r}; and similarly admissibility can be proven for $(J, \Deltac)$.

The first bound in Theorem~\ref{thm:general-bound-together} can be recovered by observing that $\gammar=\gammac=0$ as we define $I$ and $J$ to be the support for $\xstar$ and $\ystar$, and by taking the second branch in the definitions of $\Logr$ and $\Logc$.





\begin{proof}[of Theorem~\ref{thm:general-bound-together-appx}]
From Equation~\eqref{eq:def-Delta-r} and \eqref{eq:def-Delta-c}, we have
\begin{align*}
    \Rr_T + \Rc_T
    & = \max_{ x \in \cP_m, y \in \cP_n } {
            \E \sbrm[\bigg]{
                \sum_{t=1}^T \rbrm[\big]{
                    x^\top A y_t
                    -
                    x_t^\top A y
                }
            }
        } \\
    & = \max_{x \in \cP_m} \cbrm[\bigg]{
            x^\top A
            \E \sbrm[\bigg]{
                \sum_{t=1}^T y_t
            }
        }
      - \min_{ y \in \cP_n } \cbrm[\bigg]{
            \E \sbrm[\bigg]{
                \sum_{t=1}^T x_t
            }
            A y
        } \\
    & = T \max_{x \in \cP_m} \cbrm[\bigg]{
            x^\top A
            \E [\ybar_T]
        }
      - T \min_{ y \in \cP_n } \cbrm[\bigg]{
            \E [\xbar_T]
            A y
        } \\
    & \ge
    T \Deltar \cdot \rbrm{ \E[\xbar_T] - \xstar }_+
    +
    T \Deltac \cdot \rbrm{ \E[\ybar_T] - \ystar }_+,
    \yestag\label{eq:general-regret-bounded-below}
\end{align*}
where we define $\xbar_T=\frac{1}{T}\sum_{t=1}^T x_t$ and 
$\ybar_T=\frac{1}{T}\sum_{t=1}^T y_t$.

From Theorem~\ref{thm:Tsallis-INF},
we know that the following bound holds:
\begin{align}
    &
    \Rr_T(x) +
    \constd\sqrt{T}\E\sbrm[\big]{
        D(x, x_{T+1})
    }
    \le
    \consti\E\sbrm[\bigg]{
        \sum_{t=1}^{T} {
            \frac{1}{\sqrt{t}}
            \sum_{i \neq \istar} {
                \sqrt{x_t(i)}
            }
        }
    }
    , \label{eq:general-regret-bounded-above-specific}
\end{align}
for an arbitrary $\istar\in I$,
and specifically
\begin{align}
    &
    \Rr_T
    \le
    \consti\E\sbrm[\bigg]{
        \sum_{t=1}^{T} {
            \frac{1}{\sqrt{t}}
            \sum_{i \neq \istar} {
                \sqrt{x_t(i)}
            }
        }
    }
    \defeq
    \consti \E[S]
    . \label{eq:general-regret-bounded-above}
\end{align}
Define $\Sr$ as the summation inside the expectation bracket above.
We can split it into two parts:
\begin{align}
    \Sr = 
    \sum_{t=1}^{T} {
        \frac{1}{\sqrt{t}}
        \sum_{i \in I \setminus \cbr{\istar}} {
            \sqrt{x_t(i)}
        }
    }
    +
    \sum_{t=1}^{T} {
        \frac{1}{\sqrt{t}}
        \sum_{i \not\in I} {
            \sqrt{x_t(i)}
        }
    }. \label{eq:general-i-bound-by-I}
\end{align}
The sum within $I$ can be bounded with a Cauchy-Schwarz inequality:
\begin{align*}
    \sum_{t=1}^{T} {
        \frac{1}{\sqrt{t}}
        \sum_{i \in I \setminus \cbr{\istar}} {
            \sqrt{1 \cdot x_t(i)}
        }
    } & \leq \sum_{t=1}^{T} {
        \frac{1}{\sqrt{t}}
        \sqrt{
            \sum_{i \in I \setminus \cbr{\istar}} {
                1
            }
            \sum_{i \in I \setminus \cbr{\istar}} {
                \sqrt{x_t(i)}^2
            }
        }
    } \\
    & \leq
    2\sqrt{T}\sqrt{\abs{I}-1}. \yestag\label{eq:general-i-within-I}
\end{align*}
We define $\xbar(i)=\frac{1}{T}\sum_{t=1}^T x_t(i)$ as a notational shorthand.
To handle the sum outside $I$, we have due to the Cauchy-Schwarz inequality,
\begin{align*}
    \sum_{t=1}^{T} {
        \frac{1}{\sqrt{t}}
        \sum_{i \not\in I} {
            \sqrt{x_t(i)}
        }
    } & 
    = \sum_{t=1}^{s} {
        \frac{1}{\sqrt{t}}
        \sum_{i \not\in I} {
            \sqrt{x_t(i)}
        }
    } + \sum_{i \not\in I} {
        \sum_{t=s+1}^{T} {
            \frac{1}{\sqrt{t}}
            \sqrt{x_t(i)}
        }
    } \\
    & \leq
    \sum_{t=1}^{s} {
        \frac{1}{\sqrt{t}}
        \sqrt{m-\absm{I}}
    }
    +
    \sum_{i \not\in I} \sqrt{
        \rbrm[\Big]{ \sum\nolimits_{t=s+1}^{T} {
            1/t
        } }
        \rbrm[\Big]{ \sum\nolimits_{t=s+1}^{T} {
            x_t(i)
        } }
    } \\
    &\leq 2 \sqrt{s \rbrm{m-\absm{I}}}
    +
    \sqrt{T \log \rbrm{T/s}}
    \sum_{i \not\in I} \sqrt{
        \xbar_T(i)
    },
    \yestag\label{eq:ng-outside-expand-i}
\end{align*}
in which $s\in[T]$ is a parameter yet to be determined. 
We bound the last summation above in expectation. 
Definition~\ref{def:admissble} guarantees that, for $\E[\xbar_T]$, it is possible to find a Nash equilibrium strategy $\xstar$ such that the following holds:
\begin{align*}
    \E\sbrm[\Big]{\sum_{i \not\in I} \sqrt{
        \xbar_T(i)
    }} & \leq
    \sum_{i \not\in I} \sqrt{
        \E[\xbar_T(i)]
    }
    \\
    & \leq
    \sum_{i \not\in I} \sqrt{
        \xstar(i)
    }
    +
    \sum_{i \not\in I} \sqrt{
        \rbrm{\E[\xbar_T(i)] - \xstar(i)}_{+}
    }\\
    & \leq
    \gamma
    +
    \sum_{i \not\in I} \sqrt{
        \frac{1}{\Deltar(i)}
        \rbrm[\Big]{
            \Deltar(i) \cdot
            \rbrm{\E[\xbar_T(i)] - \xstar(i)}_{+}
        }
    } \\
    & \leq
    \gamma
    +
    \sqrt{
        \rbrm[\Big]{
            \sum_{i \not\in I}
            \frac{1}{\Deltar(i)}
        }
        \rbrm[\Big]{
            \sum_{i \not\in I}
            \Deltar(i) \cdot
            \rbrm{\E[\xbar_T(i)] - \xstar(i)}_{+}
        }
    } \\
    & = 
    \gamma + \sqrt{\omega \Deltar \cdot \rbrm{ \E[\xbar_T] - \xstar }_+}.
\end{align*}
We put \eqref{eq:general-i-bound-by-I}, \eqref{eq:general-i-within-I}, \eqref{eq:ng-outside-expand-i} together, and take the expectation on both sides to get
\begin{align*}
    \E[\Sr]
    & \leq 
    2 \sqrt{T}\sqrt{\abs{I}-1}
    +
    2 \sqrt{\rbrm{m-\abs{I}}s} 
    + \sqrt{T \log \rbrm{T/s}} \rbrm[\big]{
        \gamma + 
        \sqrt{\omegar \Deltar \cdot 
              \rbrm{ \E\sbrm{\xbar} - \xstar }_+}
    }, \yestag\label{eq:general-bound-together-expand-i}
\end{align*}
where the last inequality is due to Jensen's inequality and the concavity of square root.
For similarly defined 
$\Sc$ and $s'$, we also have a similar bound:
\begin{align*}
    \E[\Sc]
    & \leq 
    2 \sqrt{T}\sqrt{\abs{J}-1}
    +
    2 \sqrt{\rbrm{n-\abs{J}}s'} 
    + \sqrt{T \log \rbrm{T/s'}} \rbrm[\big]{
        \gamma + 
        \sqrt{\omegac \Deltac \cdot 
              \rbrm{ \E\sbrm{\ybar} - \ystar }_+}
    } .
    \yestag\label{eq:general-bound-together-expand-j}
\end{align*}
Now, note that from Jensen's inequality, we have
\begin{align*}
    &
    \sqrt{T \log \rbrm{T/s} \omegar}
    \sqrt{ \Deltar \cdot 
          \rbrm{ \E\sbrm{\xbar_T} - \xstar }_+}
    +
    \sqrt{T \log \rbrm{T/s'} \omegac}
    \sqrt{ \Deltac \cdot 
          \rbrm{ \E\sbrm{\ybar_T} - \ystar }_+} \\
    & \leq
    \sqrt{
        T \rbrm[\big]{
          \log \rbrm{T/s} \omegar 
        + \log \rbrm{T/s'} \omegac
        } \rbrm[\big] {
            \Deltar \cdot 
            \rbrm{ \E\sbrm{\xbar_T} - \xstar }_+
            +
            \Deltac \cdot 
            \rbrm{ \E\sbrm{\ybar_T} - \ystar }_+
        }
    }\\
    & \leq
    \sqrt{
        \rbrm[\big]{
          \omegar \log \rbrm{T/s} 
        + \omegac \log \rbrm{T/s'} 
        } \rbrm[\big] {
            \Rr_T + \Rc_T
        }
    } \tag*{\textlangle\,from \eqref{eq:general-regret-bounded-below}\,\textrangle}\\
    & \leq
    \sqrt{
        \consti \rbrm[\big]{
          \omegar \log \rbrm{T/s} 
        + \omegac \log \rbrm{T/s'} 
        } \rbrm[\big] {
            \E[\Sr]+\E[\Sc]
        }
    } \tag*{\textlangle\,from \eqref{eq:general-regret-bounded-above} and its counterpart\,\textrangle}
\end{align*}
If we add \eqref{eq:general-bound-together-expand-i} and \eqref{eq:general-bound-together-expand-j}, we get the following bound:
\begin{align*}
    \E[\Sr+\Sc]
    \leq {} &
    2\sqrt{T}\rbrm[\big]{\sqrt{\abs{I}-1} + \sqrt{\abs{J}-1}}
    + 2 \sqrt{\rbrm{m-\abs{I}}s} 
    + 2 \sqrt{\rbrm{n-\abs{J}}s'}\\
    &
    {} +
    \gamma\sqrt{T}
    \sqrt{
        \log \rbrm{T/s}
    }
    +
    \gamma\sqrt{T}
    \sqrt{
        \log \rbrm{T/s'}
    } \\
    &
    {} +
    \sqrt{
        \consti \rbrm[\big]{
            \omegar \log \rbrm{T/s}
            +
            \omegac \log \rbrm{T/s'}
        }
        \rbrm{
            \E[S]+\E[S']
        }
    }\\
    \leq{} &
    4\sqrt{T}\rbrm[\big]{\sqrt{\abs{I}-1} + \sqrt{\abs{J}-1}}
    + 4 \sqrt{\rbrm{m-\abs{I}}s} 
    + 4 \sqrt{\rbrm{n-\abs{J}}s'}\\
    &
    {} +
    2\gamma\sqrt{T}
    \sqrt{
        \log \rbrm{T/s}
    }
    +
    2\gamma\sqrt{T}
    \sqrt{
        \log \rbrm{T/s'}
    } \\
    &
    +
        2 \consti
        \omegar \log \rbrm{T/s}
    +
        2 \consti
        \omegac \log \rbrm{T/s'}
    , \yestag\label{eq:general-bound-together-expand}
\end{align*}
where in the last inequality we apply Lemma~\ref{lem:sumszt-2}.
We take $s=\ceilm[\big]{
    \min\cbrm[\big]{
        \frac{T}{2},
        \frac{
            \max\cbrm{\omega^2,\, \gamma^2 T}
        }{
            m-\absm{I}
        }
    }
}$; since $\Deltar_i\leq 2$ for every $i$, we know that $\omegar \geq \frac{1}{2} \rbrm{m-\absm{I}}$, 
so we have $\frac{\omegar^2}{m-\absm{I}}\geq \frac{1}{4}$,
and thus the rounding-up increases $s$ by a factor of at most 4. This implies that $4 \sqrt{\rbrm{m-\abs{I}}s} \leq 16 \gamma \sqrt{T}+16\omegar$.

We also have
\begin{align*}
s & \geq 
    \min\cbrm[\Big]{
        \frac{T}{2},
        \frac{
            \max\cbrm{\omega^2, \gamma^2 T}
        }{
            m-\absm{I}
        }
    }, \\
\frac{T}{s} & \leq
    \max\cbrm[\Big]{
        2,
        \min\cbrm[\Big]{
        \frac{
            T\rbrm{m-\absm{I}}
        }{
            \omega^2
        },
        \frac{
            m-\absm{I}
        }{
            \gamma^2
        }
    }
}, \\
\log \frac{T}{s} & \leq 
\min\cbrm[\Big]{
    \log_+ \frac{
            T\rbrm{m-\absm{I}}
        }{
            \omega^2
        },
    \log_+ \frac{
            m-\absm{I}
        }{
            \gamma^2
        }
} \defeq L.
\end{align*}
A similar definition and respective inequalities are omitted for $s'$. Plugging these bounds into \eqref{eq:general-bound-together-expand} yields
\begin{align*}
    \E[\Sr+\Sc]
    \leq {} & 
    4\sqrt{T\rbrm{\absm{I}-1}} +
    18\gammar\sqrt{
        T \Logr
    }
    +
        (2\consti+16)
        \omegar \Logr
    \\
    &
    {} 
    +
    4\sqrt{T\rbrm{\absm{J}-1}}
    +
    18 \gammac \sqrt{
        T \Logc
    } 
    +
        (2\consti+16)
        \omegac \Logc
    .
\end{align*}
Together with \eqref{eq:general-regret-bounded-above} and the definition of $S$ and $S'$, this completes the proof.
\end{proof}

The second part of Theorem~\ref{thm:general-bound-together}
is a corollary of the following theorem:
\begin{theorem}
	\label{thm:general-bound3}
    Suppose that $c, c'\in (0, 1]$ satisfy 
    \begin{align}
        \label{eq:LBDG}
        \DG({x}, \hat{y})
        \ge
        c \min_{\xstar \in \NEr}\| {x} - \xstar \|_1
        +
        c' \min_{\ystar \in \NEc} \| {y} - \ystar   \|_1
    \end{align}
    for all $x\in \cP_m$ and $y \in \cP_n$.
    Define
    $\gamma \ge 0, \gamma' \ge 0 , \rhor > 0$ and
    $\rhoc > 0$ by
    \begin{align}
        \label{eq:defgamma}
        \gamma
        &
        =
        \max_{x \in \NEr}
        \cbrm[\bigg]{
        \sum_{i = 1}^m
        \sqrt{x(i)}
        }
        -1
        ,
        \quad
        \gamma'
        =
        \max_{y \in \NEc}
        \cbrm[\bigg]{
        \sum_{j = 1}^n
        \sqrt{y(j)}
        }
        -1,
        \\
        \rhor 
        &
        = 
        \gamma\sqrt{T \log_+ \rbrm[\bigg]{ \frac{m-1}{\gamma^2} }} 
        +
        \frac{m-1}{c}
        \log_+\rbrm[\bigg]{
            \frac{c^2 T}{m-1}
        },
        \\
        \rhoc 
        &
        = 
        \gamma'\sqrt{T \log_+ \rbrm[\bigg]{\frac{n-1}{\gamma'^2} }} 
        +
        \frac{n-1}{c'}
        \log_+\rbrm[\bigg]{
            \frac{c'^2 T}{n-1}
        }.
    \end{align}
    If both players follow the Tsallis-INF algoirthm,
    we have
    \begin{align*}
    \Rr_T(x)
    +
    \sqrt{T}\constd{\E\sbrm{
        D(x, x_{T+1})
    }}
    &= O\mleft(\rhor + \sqrt{(\rhor + \rhoc)\frac{m-1}{c} \log_+ \rbrm[\bigg]{\frac{c^2T}{m-1} }}\mright),
    \\
    \Rc_T(y)
    +
    \sqrt{T}\constd{\E\sbrm{
        D(y, y_{T+1})
    }}
    &= O\mleft(\rhoc + \sqrt{(\rhor + \rhoc)\frac{n-1}{c'} \log_+ \rbrm[\bigg]{\frac{c'^2T}{n-1} }}\mright)
    \end{align*}
    for any $x \in \cP_m$ and $y \in \cP_n$.
    Consequently,
    we have
    \begin{align*}
        \limsup_{T \to \infty}
        \frac{\Rr_T}{\sqrt{T}}
        =
        O\mleft(
        \gamma\sqrt{\log_+ \mleft( \frac{m-1}{\gamma^2} \mright)} 
        \mright),
        \quad
        \limsup_{T \to \infty}
        \frac{\Rc_T}{\sqrt{T}}
        =
        O\mleft(
        \gamma'\sqrt{\log_+ \mleft( \frac{n-1}{\gamma'^2} \mright)} 
        \mright).
    \end{align*}
\end{theorem}
From this theorem and the AM-GM inequality,
we have
\begin{align*}
    \max\mleft\{
        \Rr_T,
        \Rc_T
    \mright\}
    &
    =
    O\mleft(
        \rhor
        +
        \rhoc + \sqrt{(\rhor + \rhoc)\frac{n-1}{c'} 
        \mleft( 
            \log_+ \mleft(\frac{c^2T}{m-1} \mright)
            +
            \log_+ \mleft(\frac{c'^2T}{n-1} \mright)
        \mright)
        }
    \mright)
    \\
    &
    =
    O\mleft(
        \rhor + \rhoc
    \mright),
\end{align*}
which implies that the second part of Theorem~\ref{thm:general-bound-together} holds.
\begin{proof}
From Theorem~\ref{thm:Tsallis-INF},
for any $s \in [T]$ any $i^* \in [m]$,
and any $x \in \cP_m$,
we have
\begin{align}
    &
    \nonumber
    \Rr_T(x)
    +
    \sqrt{T}\constd{\E\sbrm{
        D(x, x_{T+1})
    }}
    \\
    &
    \nonumber
    =
    O\rbr{
        \E\sbrm[\bigg]{
        \sum_{t=1}^T
        \frac{1}{\sqrt{t}}
        \sum_{i \in [m] \setminus \{ i^* \}}
        \sqrt{x_t(i)}
        }
    }
    \\
    &
    \nonumber
    =
    O\mleft(
        \E\sbrm[\bigg]{
        \sum_{t=1}^s
        \frac{1}{\sqrt{t}}
        \sum_{i \in [m] \setminus \{ i^* \}}
        \sqrt{x_t(i)}
        +
        \sum_{t=s+1}^T
        \frac{1}{\sqrt{t}}
        \sum_{i \in [m] \setminus \{ i^* \}}
        \sqrt{x_t(i)}
        }
    \mright)
    \\
    &
    \nonumber
    =
    O\mleft(
        \sqrt{(m-1)s}
        +
        \sum_{i \in [m] \setminus \{ i^* \}}
        \sqrt{
            \E\sbrm[\bigg]{
            \sum_{t=s+1}^T
            x_t(i)
            }
            \log \frac{T}{s}
        }
    \mright)
    \\
    &
    =
    O\mleft(
        \sqrt{(m-1)s}
        +
        \sqrt{
        T
        \log \frac{T}{s}
        }
        \sum_{i \in [m] \setminus \{ i^* \}}
        \sqrt{
            \E \left[
            \bar{x}_T(i)
            \right]
        }
    \mright).
    \label{eq:Rrmx}
\end{align}
Denote
\begin{align}
    \label{eq:deftildex}
    \tilde{x}_T
    \in
    \argmin_{\xstar \in \NEr}\| \E [ \bar{x}_T] - \xstar \|_1,
    \quad
    \tilde{y}_T
    \in
    \argmin_{\ystar \in \NEc}\| \E[\bar{y}_T] - \ystar \|_1.
\end{align}
We then have
\begin{align}
    \nonumber
    &
    \sum_{i \in [m] \setminus \{ i^* \}}
    \sqrt{ \E[ \bar{x}_T(i) ]  }
    \\
    &
    \nonumber
    \le
    \sum_{i \in [m] \setminus \{ i^* \}}
    \sqrt{ \tilde{x}_T(i)} + 
    \sum_{i \in [m] \setminus \{ i^* \}}
    \sqrt{ | \E[ \bar{x}_T(i) ] - \tilde{x}_T(i)|  }
    \\
    &
    \nonumber
    \le
    \sum_{i \in [m] \setminus \{ i^* \}}
    \sqrt{ \tilde{x}_T(i)} 
    +
    \sqrt{
        (m-1)
        \sum_{i \in [m] \setminus \{ i^* \} }
    \abs{ \E[ \bar{x}_T(i) ] - \tilde{x}_T(i) }  }
    &
    (\mbox{Cauchy-Schwarz})
    \\
    &
    \nonumber
    \le
    \frac{1}{2}
    \left(
    \sum_{i =1}^m
    \sqrt{ \tilde{x}_T(i)} 
    -
    1
    \right)
    +
    \sqrt{
        (m-1)
        \nbr{ \E[ \bar{x}_T ] - \tilde{x}_T }_1
    }
    \\
    &
    \nonumber
    \le
    \frac{1}{2}
    \gamma
    +
    \sqrt{
        (m-1)
        \nbr{ \E[ \bar{x}_T ] - \tilde{x}_T }_1
    }
    &(\mbox{From \eqref{eq:defgamma} and \eqref{eq:deftildex}})
    \\
    &
    \le
    \frac{1}{2}
    \gamma
    +
    \sqrt{
        \frac{m-1}{c} \DG(\E[\bar{x}_T], \E[\bar{y}_T])
    }.
    &(\mbox{From \eqref{eq:LBDG} and \eqref{eq:deftildex}})
    \label{eq:sqrtxDG}
\end{align}
The third inequality can be shown 
by setting $i^* \in \argmax_{i \in [m]} \left\{ \tilde{x}_T(i) \right\}$.
From \eqref{eq:Rrmx} and \eqref{eq:sqrtxDG},
we have
\begin{align}
    \nonumber
    &
    \Rr_T(x)
    +
    \sqrt{T}\constd{\E\sbrm{
        D(x, x_{T+1})
    }}
    \\
    &
    =
    O\mleft(
        \sqrt{(m-1)s}
        +
        \sqrt{
        T
        \log \frac{T}{s}
        }
        \left(
        \gamma
        +
        \sqrt{
            \frac{m-1}{c} \DG(\E[\bar{x}_T], \E[\bar{y}_T])
        }
        \right)
    \mright).
    \label{eq:RrDG}
\end{align}
Similarly,
for any $s' \in [T]$,
we have
\begin{align*}
    &
    \Rc_T(y)
    +
    \sqrt{T}\constd{\E\sbrm{
        D(y, y_{T+1})
    }}
    \\
    &
    =
    O\mleft(
        \sqrt{(n-1)s'}
        +
        \sqrt{
        T
        \log \frac{T}{s'}
        }
        \left(
        \gamma'
        +
        \sqrt{
            \frac{n-1}{c'} \DG(\E[\bar{x}_T], \E[ \bar{y}_T])
        }
        \right)
    \mright).
\end{align*}
Here,
as we have
\begin{align*}
    T \cdot
    \DG( \E[\bar{x}_T],\E[ \bar{y}_T] )
    &
    =
    \Rr_T
    +
    \Rc_T,
\end{align*}
the value of
$ \DG( \E[\bar{x}_T],\E[ \bar{y}_T] ) $
is bounded as
\begin{align*}
    T \cdot
    \DG( \E[\bar{x}_T ],\E[ \bar{y}_T] )
    &
    =
    O\mleft(
    \sqrt{(m-1)s}
    +
    \sqrt{(n-1)s'}
    +
    \gamma
    \sqrt{
    T
    \log \frac{T}{s}
    }
    +
    \gamma'
    \sqrt{
    T
    \log \frac{T}{s'}
    }
    \mright.
    \\
    &
    \quad
    \quad
    \quad
    \left.
    +
    \sqrt{
        \left(
        \frac{m-1}{c} \log\frac{T}{s}
        +
        \frac{n-1}{c'} \log\frac{T}{s'}
        \right)
        T
        \cdot
        \DG(\E[\bar{x}_T], \E[ \bar{y}_T])
    }
    \right)
\end{align*}
for any $s, s' \in [T]$,
which implies
\begin{align*}
    &
    T \cdot
    \DG( \E[\bar{x}_T],\E[ \bar{y}_T] )
    \\
    &
    =
    O\mleft(
    \sqrt{(m-1)s}
    +
    \sqrt{(n-1)s'}
    +
    \gamma
    \sqrt{
    T
    \log \frac{T}{s}
    }
    +
    \gamma'
    \sqrt{
    T
    \log \frac{T}{s'}
    }
    +
        \frac{m-1}{c} \log\frac{T}{s}
        +
        \frac{n-1}{c'} \log\frac{T}{s'}
    \mright).
\end{align*}
By choosing 
\begin{align}
    \label{eq:tunesvalue}
    s = 
    \left\lceil
    \min\left\{ T,
        \max \left\{ 
            \frac{\gamma^2 T}{m-1},
            \frac{m-1}{c^2}
        \right\}
        \right\}
    \right\rceil
    \quad
    s' = 
    \left\lceil
    \min\left\{ T,
        \max \left\{ 
            \frac{\gamma'^2 T}{n-1},
            \frac{n-1}{c'^2}
        \right\}
        \right\}
    \right\rceil
\end{align}
we have
\begin{align*}
    &
    T \cdot
    \DG(\E[ \bar{x}_T], \E[ \bar{y}_T ] )
    \\
    &
    =
    O\mleft(
        \gamma\sqrt{T \log_+ \left( \frac{m-1}{\gamma^2} \right)} 
        +
        \frac{m-1}{c}
        \log_+\left( 
            \frac{c^2 T}{m-1}
        \right)
        +
        \gamma'\sqrt{T \log_+ \left( \frac{n-1}{\gamma'^2} \right)} 
        +
        \frac{n-1}{c'}
        \log_+\left( 
            \frac{c'^2 T}{n-1}
        \right)
    \mright)
    \\
    &
    =
    O\mleft(
        \rhor
        +
        \rhoc
    \mright).
\end{align*}
From this and \eqref{eq:RrDG}
with \eqref{eq:tunesvalue},
we have
\begin{align*}
    \Rr_T(x)
    +
    \sqrt{T}\constd{\E\sbrm{
        D(x, x_{T+1})
    }}
    &
    =
    O\mleft(
        \rhor
        +
        \sqrt{
            (\rhor + \rhoc)
            \frac{m-1}{c}
            \log\frac{T}{s}
        }
    \mright)
    \\
    &
    =
    O\mleft(
        \rhor
        +
        \sqrt{
            (\rhor + \rhoc)
            \frac{m-1}{c}
            \log_+\left( 
                \frac{c^2 T}{m-1}
            \right)
        }
    \mright).
\end{align*}
Similarly,
we obtain the desired upper bound on $
    \Rc_T(y)
    +
    \sqrt{T}\constd{\E\sbrm{
        D(y, y_{T+1})
    }}$,
which completes the proof.
\end{proof}
