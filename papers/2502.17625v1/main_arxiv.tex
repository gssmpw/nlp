\documentclass[11pt,a4paper]{article}
\pdfoutput=1

\usepackage{fullpage}
\usepackage{amsmath}
\usepackage{amssymb}
\usepackage{natbib}
\usepackage{authblk}
\usepackage{graphicx}
\usepackage{url}
\usepackage{algorithm2e}
\usepackage{xcolor}
\usepackage[utf8]{inputenc}       %
\usepackage[T1]{fontenc}          %
\usepackage{hyperref}             %
\usepackage{cleveref}             %
\usepackage{booktabs}             %
\usepackage{amsfonts}             %
\usepackage{nicefrac}             %
\usepackage{microtype}            %
\usepackage{enumitem} 
\usepackage{mathtools}            %
\usepackage{mleftright}           %
\usepackage{nicefrac}             %
\usepackage{textcomp}
\usepackage{nicematrix}
\usepackage{tikz}

\let\todo\undefined               %
\usepackage[obeyFinal]{easy-todo} %

\usepackage{comment}             

\usepackage{amsthm}
\newtheorem{theorem}{Theorem}
\newtheorem{proposition}{Proposition}
\newtheorem{corollary}{Corollary}
\newtheorem{lemma}{Lemma}
\theoremstyle{definition}
\newtheorem{observation}{Observation}
\newtheorem{assumption}{Assumption}
\newtheorem{definition}{Definition}
\newtheorem{remark}{Remark}


\newcommand{\ve}{}
\let\svthefootnote\thefootnote
\newcommand\blfootnote[1]{%
  \let\thefootnote\relax%
  \footnotetext{#1}%
  \let\thefootnote\svthefootnote%
}



\makeatletter

\patchcmd{\NAT@test}{\else \NAT@nm}{\else \NAT@nmfmt{\NAT@nm}}{}{}

\DeclareRobustCommand\citepos
  {\begingroup
   \let\NAT@nmfmt\NAT@posfmt%
   \NAT@swafalse\let\NAT@ctype\z@\NAT@partrue
   \@ifstar{\NAT@fulltrue\NAT@citetp}{\NAT@fullfalse\NAT@citetp}}

\let\NAT@orig@nmfmt\NAT@nmfmt
\def\NAT@posfmt#1{\NAT@orig@nmfmt{#1's}}

\makeatother


\newcommand{\wh}{\widehat}


\newcommand{\alphab}{\bar{\alpha}}
\newcommand{\psib}{\bar{\psi}}
\newcommand{\betab}{\bar{\beta}}
\renewcommand{\epsilon}{\varepsilon}
\newcommand{\cO}{\mathcal{O}}
\newcommand{\cS}{\mathcal{S}}
\newcommand{\cK}{\mathcal{K}}
\newcommand{\cE}{\mathcal{E}}
\newcommand{\cU}{\mathcal{U}}
\newcommand{\cA}{\mathcal{A}}
\newcommand{\cT}{\mathcal{T}}
\newcommand{\cX}{\mathcal{X}}
\newcommand{\cY}{\mathcal{Y}}
\newcommand{\cP}{\mathcal{P}}
\newcommand{\cD}{\mathcal{D}}
\newcommand{\fR}{\mathbb{R}}
\newcommand{\one}{\operatorname{\boldsymbol{1}}}
\newcommand{\eps}{\varepsilon}



\newcommand*{\defeq}{\stackrel{\text{def}}{=}}

\newcommand{\nicerfrac}[2]{\text{\LARGE\nicefrac{#1}{#2}}}

\newcommand{\conv}{\operatorname{conv}}
\newcommand{\Cov}{\operatorname{Cov}}
\newcommand{\DG}{\operatorname{DGap}}
\newcommand{\SR}{\operatorname{SR}}
\newcommand{\KL}{\mathop{D_{\mathrm{KL}}}\limits}
\newcommand{\TV}{\mathop{D_{\mathrm{TV}}}\limits}
\newcommand{\prob}{\mathop{\mathrm{Prob}}\limits}
\newcommand{\Ber}{\mathrm{Ber}}
\newcommand{\Berpm}{\mathrm{Ber}^{\pm}}
\DeclareMathOperator*{\argmin}{arg\,min}
\DeclareMathOperator*{\argmax}{arg\,max}
\DeclareMathOperator*{\E}{\mathbf{E}}
\DeclareMathOperator*{\Var}{\mathbf{Var}}
\newcommand{\re}{\mathbb{R}}
\newcommand{\F}{F}


\newcommand{\linner}{\mleft\langle}
\newcommand{\rinner}{\mright\rangle}

\newcommand{\paren}[1]{\mleft({#1}\mright)}
\newcommand{\rbr}{\paren}    %
\newcommand{\brackets}[1]{\mleft[{#1}\mright]}
\newcommand{\sbr}{\brackets} %
\newcommand{\braces}[1]{\mleft\{{#1}\mright\}}
\newcommand{\cbr}{\braces}   %

\newcommand{\floor}[1]{\mleft\lfloor{#1}\mright\rfloor}
\newcommand{\ceil}[1]{\mleft\lceil{#1}\mright\rceil}

\newcommand{\inner}[2]{\mleft\langle {#1},{#2} \mright\rangle}
\newcommand{\tuple}[1]{\mleft\langle{#1}\mright\rangle}

\newcommand{\abs}[1]{\mleft|{#1}\mright|}
\newcommand{\nbr}[1]{\mleft\|{#1}\mright\|}

\newcommand\RemoveAmpNL[1]{
    \bgroup         %
    \catcode`\&=9   %
    \let\\\relax    %
    \scantokens{#1} %
    \egroup         %
}
\newcommand{\NegSpaceDot}{\kern-\nulldelimiterspace}
\newcommand{\nomleft}{\mleft.\NegSpaceDot}
\newcommand{\nomright}{\NegSpaceDot\mright.}
\newcommand{\brkXbr}[3]{\mleft#1\vphantom{\RemoveAmpNL{#2}}\nomright#2\nomleft\vphantom{\RemoveAmpNL{#2}}\mright#3}
\newcommand{\brkrbr}[1]{\brkXbr({#1})}
\newcommand{\brksbr}[1]{\brkXbr[{#1}]}

\DeclarePairedDelimiter{\rbrm}()
\DeclarePairedDelimiter{\sbrm}[]
\DeclarePairedDelimiter{\cbrm}\{\}
\DeclarePairedDelimiter{\absm}||
\DeclarePairedDelimiter{\ceilm}\lceil\rceil
\DeclarePairedDelimiter{\nbrm}\|\|

\newcommand{\Deltar}{\Delta}
\newcommand{\Deltac}{\Delta'}
\newcommand{\Rr}{\mathrm{Reg}}
\newcommand{\Rc}{\mathrm{Reg}'}
\newcommand{\Rcb}{\bar{\mathrm{Reg}}}
\newcommand{\Rrb}{\bar{\mathrm{Reg}}'}
\newcommand{\Tr}{N}
\newcommand{\Tc}{N'}
\newcommand{\lr}{\ell}
\newcommand{\lc}{\ell'}
\newcommand{\PS}{\mathcal{P}}
\newcommand{\xstar}{x_{\star}}
\newcommand{\ystar}{y_{\star}}
\newcommand{\xbar}{\bar{x}}
\newcommand{\ybar}{\bar{y}}
\newcommand{\xstarset}{\mathcal{X}_{\star}}
\newcommand{\ystarset}{\mathcal{Y}_{\star}}
\newcommand{\istar}{i_{\star}}
\newcommand{\jstar}{j_{\star}}
\newcommand{\omegar}{\omega}
\newcommand{\omegac}{\omega'}
\newcommand{\Sr}{S}
\newcommand{\Sc}{S'}
\newcommand{\Logrs}{L_\star}
\newcommand{\Logcs}{L'_\star}
\newcommand{\Logr}{L}
\newcommand{\Logc}{L'}
\newcommand{\vecdelta}{\mathbf{\delta}}
\newcommand{\proj}{\Pi}
\newcommand{\pir}{\pi}
\newcommand{\pic}{\pi'}
\newcommand{\NEr}{\mathcal{X}_{\star}}
\newcommand{\NEc}{\mathcal{Y}_{\star}}

\newcommand{\consti}{C_1}
\newcommand{\constd}{C_2}
\newcommand{\gammar}{\gamma}
\newcommand{\gammac}{\gamma'}
\newcommand{\rhor}{\rho}
\newcommand{\rhoc}{\rho'}
\newcommand{\gammabar}{\overline{\gamma}}
\newcommand{\OPT}{\mathrm{OPT}}

\newcommand\yestag{\addtocounter{equation}{1}\tag{\theequation}} %

\newcommand{\hl}[1]{\todo{(Haipeng: #1)}}
\newcommand{\inote}[1]{\todo{(Shinji: #1)}}
\newcommand{\tnote}[1]{\todo{(Tsuchiya: #1)}}
\newcommand{\wnote}[1]{\todo{\textcolor{purple}{(Yue: #1)}}}
\makeatletter
\newcommand\usemm[1]{%
  \strip@pt\dimexpr0.3514598\dimexpr #1\relax\relax mm%
}
\newcommand\usein[1]{%
  \strip@pt\dimexpr0.013837\dimexpr #1\relax\relax in%
}
\makeatother


\title{Instance-Dependent Regret Bounds for Learning \\ Two-Player Zero-Sum Games with Bandit Feedback}


\date{February 20, 2025}

\author[1,2]{Shinji Ito}
\author[3]{Haipeng Luo}
\author[1,2]{Taira Tsuchiya}
\author[3]{Yue Wu}

\affil[1]{The University of Tokyo\\ \texttt{\{shinji,tsuchiya\}@mist.i.u-tokyo.ac.jp}}
\affil[2]{RIKEN AIP}
\affil[3]{University of Southern California\\ \texttt{\{haipengl, wu.yue\}@usc.edu}}

\hypersetup{
    pdftitle={Instance-Dependent Regret Bounds for Learning Two-Player Zero-Sum Games with Bandit Feedback},
    pdfauthor={Shinji Ito, Haipeng Luo, Taira Tsuchiya, and Yue Wu},
    pdfcreator={LaTeX}
}

\begin{document}

\maketitle
\blfootnote{Authors are listed in alphabetical order.}


\begin{abstract}%
No-regret self-play learning dynamics have become one of the premier ways to solve large-scale games in practice.
Accelerating their convergence via improving the regret of the players over the naive $O(\sqrt{T})$ bound after $T$ rounds has been extensively studied in recent years, but almost all studies assume access to exact gradient feedback.
We address the question of whether acceleration is possible under bandit feedback only and provide an affirmative answer for two-player zero-sum normal-form games.
Specifically, we show that if both players apply the Tsallis-INF algorithm of~\citet{zimmert2021tsallis}, then their regret is at most $O(c_1 \log T +  \sqrt{c_2 T})$, where $c_1$ and $c_2$ are game-dependent constants that characterize the difficulty of learning ----- $c_1$ resembles the complexity of learning a stochastic multi-armed bandit instance and depends inversely on some gap measures,
while $c_2$ can be much smaller than the number of actions when the Nash equilibria have a small support or are close to the boundary. 
In particular, for the case when a pure strategy Nash equilibrium exists, $c_2$ becomes zero, leading to an optimal instance-dependent regret bound as we show.
We additionally prove that in this case our algorithm also enjoys last-iterate convergence and can identify the pure strategy Nash equilibrium with near-optimal sample complexity.
\end{abstract}


\section{Introduction}
\label{section:introduction}

% redirection is unique and important in VR
Virtual Reality (VR) systems enable users to embody virtual avatars by mirroring their physical movements and aligning their perspective with virtual avatars' in real time. 
As the head-mounted displays (HMDs) block direct visual access to the physical world, users primarily rely on visual feedback from the virtual environment and integrate it with proprioceptive cues to control the avatar’s movements and interact within the VR space.
Since human perception is heavily influenced by visual input~\cite{gibson1933adaptation}, 
VR systems have the unique capability to control users' perception of the virtual environment and avatars by manipulating the visual information presented to them.
Leveraging this, various redirection techniques have been proposed to enable novel VR interactions, 
such as redirecting users' walking paths~\cite{razzaque2005redirected, suma2012impossible, steinicke2009estimation},
modifying reaching movements~\cite{gonzalez2022model, azmandian2016haptic, cheng2017sparse, feick2021visuo},
and conveying haptic information through visual feedback to create pseudo-haptic effects~\cite{samad2019pseudo, dominjon2005influence, lecuyer2009simulating}.
Such redirection techniques enable these interactions by manipulating the alignment between users' physical movements and their virtual avatar's actions.

% % what is hand/arm redirection, motivation of study arm-offset
% \change{\yj{i don't understand the purpose of this paragraph}
% These illusion-based techniques provide users with unique experiences in virtual environments that differ from the physical world yet maintain an immersive experience. 
% A key example is hand redirection, which shifts the virtual hand’s position away from the real hand as the user moves to enhance ergonomics during interaction~\cite{feuchtner2018ownershift, wentzel2020improving} and improve interaction performance~\cite{montano2017erg, poupyrev1996go}. 
% To increase the realism of virtual movements and strengthen the user’s sense of embodiment, hand redirection techniques often incorporate a complete virtual arm or full body alongside the redirected virtual hand, using inverse kinematics~\cite{hartfill2021analysis, ponton2024stretch} or adjustments to the virtual arm's movement as well~\cite{li2022modeling, feick2024impact}.
% }

% noticeability, motivation of predicting a probability, not a classification
However, these redirection techniques are most effective when the manipulation remains undetected~\cite{gonzalez2017model, li2022modeling}. 
If the redirection becomes too large, the user may not mitigate the conflict between the visual sensory input (redirected virtual movement) and their proprioception (actual physical movement), potentially leading to a loss of embodiment with the virtual avatar and making it difficult for the user to accurately control virtual movements to complete interaction tasks~\cite{li2022modeling, wentzel2020improving, feuchtner2018ownershift}. 
While proprioception is not absolute, users only have a general sense of their physical movements and the likelihood that they notice the redirection is probabilistic. 
This probability of detecting the redirection is referred to as \textbf{noticeability}~\cite{li2022modeling, zenner2024beyond, zenner2023detectability} and is typically estimated based on the frequency with which users detect the manipulation across multiple trials.

% version B
% Prior research has explored factors influencing the noticeability of redirected motion, including the redirection's magnitude~\cite{wentzel2020improving, poupyrev1996go}, direction~\cite{li2022modeling, feuchtner2018ownershift}, and the visual characteristics of the virtual avatar~\cite{ogawa2020effect, feick2024impact}.
% While these factors focus on the avatars, the surrounding virtual environment can also influence the users' behavior and in turn affect the noticeability of redirection.
% One such prominent external influence is through the visual channel - the users' visual attention is constantly distracted by complex visual effects and events in practical VR scenarios.
% Although some prior studies have explored how to leverage user blindness caused by visual distractions to redirect users' virtual hand~\cite{zenner2023detectability}, there remains a gap in understanding how to quantify the noticeability of redirection under visual distractions.

% visual stimuli and gaze behavior
Prior research has explored factors influencing the noticeability of redirected motion, including the redirection's magnitude~\cite{wentzel2020improving, poupyrev1996go}, direction~\cite{li2022modeling, feuchtner2018ownershift}, and the visual characteristics of the virtual avatar~\cite{ogawa2020effect, feick2024impact}.
While these factors focus on the avatars, the surrounding virtual environment can also influence the users' behavior and in turn affect the noticeability of redirection.
This, however, remains underexplored.
One such prominent external influence is through the visual channel - the users' visual attention is constantly distracted by complex visual effects and events in practical VR scenarios.
We thus want to investigate how \textbf{visual stimuli in the virtual environment} affect the noticeability of redirection.
With this, we hope to complement existing works that focus on avatars by incorporating environmental visual influences to enable more accurate control over the noticeability of redirected motions in practical VR scenarios.
% However, in realistic VR applications, the virtual environment often contains complex visual effects beyond the virtual avatar itself. 
% We argue that these visual effects can \textbf{distract users’ visual attention and thus affect the noticeability of redirection offsets}, while current research has yet taken into account.
% For instance, in a VR boxing scenario, a user’s visual attention is likely focused on their opponent rather than on their virtual body, leading to a lower noticeability of redirection offsets on their virtual movements. 
% Conversely, when reaching for an object in the center of their field of view, the user’s attention is more concentrated on the virtual hand’s movement and position to ensure successful interaction, resulting in a higher noticeability of offsets.

Since each visual event is a complex choreography of many underlying factors (type of visual effect, location, duration, etc.), it is extremely difficult to quantify or parameterize visual stimuli.
Furthermore, individuals respond differently to even the same visual events.
Prior neuroscience studies revealed that factors like age, gender, and personality can influence how quickly someone reacts to visual events~\cite{gillon2024responses, gale1997human}. 
Therefore, aiming to model visual stimuli in a way that is generalizable and applicable to different stimuli and users, we propose to use users' \textbf{gaze behavior} as an indicator of how they respond to visual stimuli.
In this paper, we used various gaze behaviors, including gaze location, saccades~\cite{krejtz2018eye}, fixations~\cite{perkhofer2019using}, and the Index of Pupil Activity (IPA)~\cite{duchowski2018index}.
These behaviors indicate both where users are looking and their cognitive activity, as looking at something does not necessarily mean they are attending to it.
Our goal is to investigate how these gaze behaviors stimulated by various visual stimuli relate to the noticeability of redirection.
With this, we contribute a model that allows designers and content creators to adjust the redirection in real-time responding to dynamic visual events in VR.

To achieve this, we conducted user studies to collect users' noticeability of redirection under various visual stimuli.
To simulate realistic VR scenarios, we adopted a dual-task design in which the participants performed redirected movements while monitoring the visual stimuli.
Specifically, participants' primary task was to report if they noticed an offset between the avatar's movement and their own, while their secondary task was to monitor and report the visual stimuli.
As realistic virtual environments often contain complex visual effects, we started with simple and controlled visual stimulus to manage the influencing factors.

% first user study, confirmation study
% collect data under no visual stimuli, different basic visual stimuli
We first conducted a confirmation study (N=16) to test whether applying visual stimuli (opacity-based) actually affects their noticeability of redirection. 
The results showed that participants were significantly less likely to detect the redirection when visual stimuli was presented $(F_{(1,15)}=5.90,~p=0.03)$.
Furthermore, by analyzing the collected gaze data, results revealed a correlation between the proposed gaze behaviors and the noticeability results $(r=-0.43)$, confirming that the gaze behaviors could be leveraged to compute the noticeability.

% data collection study
We then conducted a data collection study to obtain more accurate noticeability results through repeated measurements to better model the relationship between visual stimuli-triggered gaze behaviors and noticeability of redirection.
With the collected data, we analyzed various numerical features from the gaze behaviors to identify the most effective ones. 
We tested combinations of these features to determine the most effective one for predicting noticeability under visual stimuli.
Using the selected features, our regression model achieved a mean squared error (MSE) of 0.011 through leave-one-user-out cross-validation. 
Furthermore, we developed both a binary and a three-class classification model to categorize noticeability, which achieved an accuracy of 91.74\% and 85.62\%, respectively.

% evaluation study
To evaluate the generalizability of the regression model, we conducted an evaluation study (N=24) to test whether the model could accurately predict noticeability with new visual stimuli (color- and scale-based animations).
Specifically, we evaluated whether the model's predictions aligned with participants' responses under these unseen stimuli.
The results showed that our model accurately estimated the noticeability, achieving mean squared errors (MSE) of 0.014 and 0.012 for the color- and scale-based visual stimili, respectively, compared to participants' responses.
Since the tested visual stimuli data were not included in the training, the results suggested that the extracted gaze behavior features capture a generalizable pattern and can effectively indicate the corresponding impact on the noticeability of redirection.

% application
Based on our model, we implemented an adaptive redirection technique and demonstrated it through two applications: adaptive VR action game and opportunistic rendering.
We conducted a proof-of-concept user study (N=8) to compare our adaptive redirection technique with a static redirection, evaluating the usability and benefits of our adaptive redirection technique.
The results indicated that participants experienced less physical demand and stronger sense of embodiment and agency when using the adaptive redirection technique. 
These results demonstrated the effectiveness and usability of our model.

In summary, we make the following contributions.
% 
\begin{itemize}
    \item 
    We propose to use users' gaze behavior as a medium to quantify how visual stimuli influences the noticebility of redirection. 
    Through two user studies, we confirm that visual stimuli significantly influences noticeability and identify key gaze behavior features that are closely related to this impact.
    \item 
    We build a regression model that takes the user's gaze behavioral data as input, then computes the noticeability of redirection.
    Through an evaluation study, we verify that our model can estimate the noticeability with new participants under unseen visual stimuli.
    These findings suggest that the extracted gaze behavior features effectively capture the influence of visual stimuli on noticeability and can generalize across different users and visual stimuli.
    \item 
    We develop an adaptive redirection technique based on our regression model and implement two applications with it.
    With a proof-of-concept study, we demonstrate the effectiveness and potential usability of our regression model on real-world use cases.

\end{itemize}

% \delete{
% Virtual Reality (VR) allows the user to embody a virtual avatar by mirroring their physical movements through the avatar.
% As the user's visual access to the physical world is blocked in tasks involving motion control, they heavily rely on the visual representation of the avatar's motions to guide their proprioception.
% Similar to real-world experiences, the user is able to resolve conflicts between different sensory inputs (e.g., vision and motor control) through multisensory integration, which is essential for mitigating the sensory noise that commonly arises.
% However, it also enables unique manipulations in VR, as the system can intentionally modify the avatar's movements in relation to the user's motions to achieve specific functional outcomes,
% for example, 
% % the manipulations on the avatar's movements can 
% enabling novel interaction techniques of redirected walking~\cite{razzaque2005redirected}, redirected reaching~\cite{gonzalez2022model}, and pseudo haptics~\cite{samad2019pseudo}.
% With small adjustments to the avatar's movements, the user can maintain their sense of embodiment, due to their ability to resolve the perceptual differences.
% % However, a large mismatch between the user and avatar's movements can result in the user losing their sense of embodiment, due to an inability to resolve the perceptual differences.
% }

% \delete{
% However, multisensory integration can break when the manipulation is so intense that the user is aware of the existence of the motion offset and no longer maintains the sense of embodiment.
% Prior research studied the intensity threshold of the offset applied on the avatar's hand, beyond which the embodiment will break~\cite{li2022modeling}. 
% Studies also investigated the user's sensitivity to the offsets over time~\cite{kohm2022sensitivity}.
% Based on the findings, we argue that one crucial factor that affects to what extent the user notices the offset (i.e., \textit{noticeability}) that remains under-explored is whether the user directs their visual attention towards or away from the virtual avatar.
% Related work (e.g., Mise-unseen~\cite{marwecki2019mise}) has showcased applications where adjustments in the environment can be made in an unnoticeable manner when they happen in the area out of the user's visual field.
% We hypothesize that directing the user's visual attention away from the avatar's body, while still partially keeping the avatar within the user's field-of-view, can reduce the noticeability of the offset.
% Therefore, we conduct two user studies and implement a regression model to systematically investigate this effect.
% }

% \delete{
% In the first user study (N = 16), we test whether drawing the user's visual attention away from their body impacts the possibility of them noticing an offset that we apply to their arm motion in VR.
% We adopt a dual-task design to enable the alteration of the user's visual attention and a yes/no paradigm to measure the noticeability of motion offset. 
% The primary task for the user is to perform an arm motion and report when they perceive an offset between the avatar's virtual arm and their real arm.
% In the secondary task, we randomly render a visual animation of a ball turning from transparent to red and becoming transparent again and ask them to monitor and report when it appears.
% We control the strength of the visual stimuli by changing the duration and location of the animation.
% % By changing the time duration and location of the visual animation, we control the strengths of attraction to the users.
% As a result, we found significant differences in the noticeability of the offsets $(F_{(1,15)}=5.90,~p=0.03)$ between conditions with and without visual stimuli.
% Based on further analysis, we also identified the behavioral patterns of the user's gaze (including pupil dilation, fixations, and saccades) to be correlated with the noticeability results $(r=-0.43)$ and they may potentially serve as indicators of noticeability.
% }

% \delete{
% To further investigate how visual attention influences the noticeability, we conduct a data collection study (N = 12) and build a regression model based on the data.
% The regression model is able to calculate the noticeability of the offset applied on the user's arm under various visual stimuli based on their gaze behaviors.
% Our leave-one-out cross-validation results show that the proposed method was able to achieve a mean-squared error (MSE) of 0.012 in the probability regression task.
% }

% \delete{
% To verify the feasibility and extendability of the regression model, we conduct an evaluation study where we test new visual animations based on adjustments on scale and color and invite 24 new participants to attend the study.
% Results show that the proposed method can accurately estimate the noticeability with an MSE of 0.014 and 0.012 in the conditions of the color- and scale-based visual effects.
% Since these animations were not included in the dataset that the regression model was built on, the study demonstrates that the gaze behavioral features we extracted from the data capture a generalizable pattern of the user's visual attention and can indicate the corresponding impact on the noticeability of the offset.
% }

% \delete{
% Finally, we demonstrate applications that can benefit from the noticeability prediction model, including adaptive motion offsets and opportunistic rendering, considering the user's visual attention. 
% We conclude with discussions of our work's limitations and future research directions.
% }

% \delete{
% In summary, we make the following contributions.
% }
% % 
% \begin{itemize}
%     \item 
%     \delete{
%     We quantify the effects of the user's visual attention directed away by stimuli on their noticeability of an offset applied to the avatar's arm motion with respect to the user's physical arm. 
%     Through two user studies, we identified gaze behavioral features that are indicative of the changes in noticeability.
%     }
%     \item 
%     \delete{We build a regression model that takes the user's gaze behavioral data and the offset applied to the arm motion as input, then computes the probability of the user noticing the offset.
%     Through an evaluation study, we verified that the model needs no information about the source attracting the user's visual attention and can be generalizable in different scenarios.
%     }
%     \item 
%     \delete{We demonstrate two applications that potentially benefit from the regression model, including adaptive motion offsets and opportunistic rendering.
%     }

% \end{itemize}

\begin{comment}
However, users will lose the sense of embodiment to the virtual avatars if they notice the offset between the virtual and physical movements.
To address this, researchers have been exploring the noticing threshold of offsets with various magnitudes and proposing various redirection techniques that maintain the sense of embodiment~\cite{}.

However, when users embody virtual avatars to explore virtual environments, they encounter various visual effects and content that can attract their attention~\cite{}.
During this, the user may notice an offset when he observes the virtual movement carefully while ignoring it when the virtual contents attract his attention from the movements.
Therefore, static offset thresholds are not appropriate in dynamic scenarios.

Past research has proposed dynamic mapping techniques that adapted to users' state, such as hand moving speed~\cite{frees2007prism} or ergonomically comfortable poses~\cite{montano2017erg}, but not considering the influence of virtual content.
More specifically, PRISM~\cite{frees2007prism} proposed adjusting the C/D ratio with a non-linear mapping according to users' hand moving speed, but it might not be optimal for various virtual scenarios.
While Erg-O~\cite{montano2017erg} redirected users' virtual hands according to the virtual target's relative position to reduce physical fatigue, neglecting the change of virtual environments. 

Therefore, how to design redirection techniques in various scenarios with different visual attractions remains unknown.
To address this, we investigate how visual attention affects the noticing probability of movement offsets.
Based on our experiments, we implement a computational model that automatically computes the noticing probability of offsets under certain visual attractions.
VR application designers and developers can easily leverage our model to design redirection techniques maintaining the sense of embodiment adapt to the user's visual attention.
We implement a dynamic redirection technique with our model and demonstrate that it effectively reduces the target reaching time without reducing the sense of embodiment compared to static redirection techniques.

% Need to be refined
This paper offers the following contributions.
\begin{itemize}
    \item We investigate how visual attractions affect the noticing probability of redirection offsets.
    \item We construct a computational model to predict the noticing probability of an offset with a given visual background.
    \item We implement a dynamic redirection technique adapting to the visual background. We evaluate the technique and develop three applications to demonstrate the benefits. 
\end{itemize}



First, we conducted a controlled experiment to understand how users perceived the movement offset while subjected to various distractions.
Since hand redirection is one of the most frequently used redirections in VR interactions, we focused on the dynamic arm movements and manually added angular offsets to the' elbow joint~\cite{li2022modeling, gonzalez2022model, zenner2019estimating}. 
We employed flashing spheres in the user's field of view as distractions to attract users' visual attention.
Participants were instructed to report the appearing location of the spheres while simultaneously performing the arm movements and reporting if they perceived an offset during the movement. 
(\zhipeng{Add the results of data collection. Analyze the influence of the distance between the gaze map and the offset.}
We measured the visual attraction's magnitude with the gaze distribution on it.
Results showed that stronger distractions made it harder for users to notice the offset.)
\zhipeng{Need to rewrite. Not sure to use gaze distribution or a metric obtained from the visual content.}
Secondly, we constructed a computational model to predict the noticing probability of offsets with given visual content.
We analyzed the data from the user studies to measure the influence of visual attractions on the noticing probability of offsets.
We built a statistical model to predict the offset's noticing probability with a given visual content.
Based on the model, we implement a dynamic redirection technique to adjust the redirection offset adapted to the user's current field of view.
We evaluated the technique in a target selection task compared to no hand redirection and static hand redirection.
\zhipeng{Add the results of the evaluation.}
Results showed that the dynamic hand redirection technique significantly reduced the target selection time with similar accuracy and a comparable sense of embodiment.
Finally, we implemented three applications to demonstrate the potential benefits of the visual attention adapted dynamic redirection technique.
\end{comment}

% This one modifies arm length, not redirection
% \citeauthor{mcintosh2020iteratively} proposed an adaptation method to iteratively change the virtual avatar arm's length based on the primary tasks' performance~\cite{mcintosh2020iteratively}.



% \zhipeng{TO ADD: what is redirection}
% Redirection enables novel interactions in Virtual Reality, including redirected walking, haptic redirection, and pseudo haptics by introducing an offset to users' movement.
% \zhipeng{TO ADD: extend this sentence}
% The price of this is that users' immersiveness and embodiment in VR can be compromised when they notice the offset and perceive the virtual movement not as theirs~\cite{}.
% \zhipeng{TO ADD: extend this sentence, elaborate how the virtual environment attracts users' attention}
% Meanwhile, the visual content in the virtual environment is abundant and consistently captures users' attention, making it harder to notice the offset~\cite{}.
% While previous studies explored the noticing threshold of the offsets and optimized the redirection techniques to maintain the sense of embodiment~\cite{}, the influence of visual content on the probability of perceiving offsets remains unknown.  
% Therefore, we propose to investigate how users perceive the redirection offset when they are facing various visual attractions.


% We conducted a user study to understand how users notice the shift with visual attractions.
% We used a color-changing ball to attract the user's attention while instructing users to perform different poses with their arms and observe it meanwhile.
% \zhipeng{(Which one should be the primary task? Observe the ball should be the primary one, but if the primary task is too simple, users might allocate more attention on the secondary task and this makes the secondary task primary.)}
% \zhipeng{(We need a good and reasonable dual-task design in which users care about both their pose and the visual content, at least in the evaluation study. And we need to be able to control the visual content's magnitude and saliency maybe?)}
% We controlled the shift magnitude and direction, the user's pose, the ball's size, and the color range.
% We set the ball's color-changing interval as the independent factor.
% We collect the user's response to each shift and the color-changing times.
% Based on the collected data, we constructed a statistical model to describe the influence of visual attraction on the noticing probability.
% \zhipeng{(Are we actually controlling the attention allocation? How do we measure the attracting effect? We need uniform metrics, otherwise it is also hard for others to use our knowledge.)}
% \zhipeng{(Try to use eye gaze? The eye gaze distribution in the last five seconds to decide the attention allocation? Basically constructing a model with eye gaze distribution and noticing probability. But the user's head is moving, so the eye gaze distribution is not aligned well with the current view.)}

% \zhipeng{Saliency and EMD}
% \zhipeng{Gaze is more than just a point: Rethinking visual attention
% analysis using peripheral vision-based gaze mapping}

% Evaluation study(ideal case): based on the visual content, adjusting the redirection magnitude dynamically.

% \zhipeng{(The risk is our model's effect is trivial.)}

% Applications:
% Playing Lego while watching demo videos, we can accelerate the reaching process of bricks, and forbid the redirection during the manipulation.

% Beat saber again: but not make a lot of sense? Difficult game has complicated visual effects, while allows larger shift, but do not need large shift with high difficulty




\section{Preliminaries}
In this section, we formally describe concepts related to two-player zero-sum games, self-play learning dynamics, and our main algorithm.

\paragraph{Notations}
Throughout this paper, we will use $\log(\cdot)$ to denote base-$2$ logarithm, $\ln(\cdot)$ to denote base-$e$ logarithm, and use $\log_+ x = \max \{ 1, \log x \}$. We use $\tilde{O}(\cdot)$ to hide logarithmic factors; formally, $f(x)=\tilde{O}(g(x))$ means that there exists a positive integer $k$ such that $f(x)=O(g(x) \log^k g(x))$.

\paragraph{Two-Player Zero-Sum Normal-Form Games}
A two-player zero-sum normal-form game is defined via a payoff matrix $A \in [-1, 1]^{m \times n}$,
where $m$ and $n$ are the number of actions for the row player and the column player respectively.
When the row player plays action $i$ and the column player plays action $j$,
the entry $A(i, j) \in [-1, 1]$ is the expected reward for the row player and also the expected loss for the column player (hence zero-sum).

The players also have the option to play according to a probability distribution over their actions, or a \emph{mixed strategy}.
Let $\cP_m = \{ x \in [0, 1]^m \mid \| x \|_1 = 1 \}$ be the probability simplex of size $m$.
Given mixed strategies $x \in \cP_m$ and $y \in \cP_n$ of the row and column players,
the expected reward for the row player is given by
$x^\top A y$, which is also the expected loss for the column player.

A pair of mixed strategies $(\xstar, \ystar) \in \cP_m \times \cP_n$ is a \textit{Nash equilibrium} (NE) if
$
x^{\top} A \ystar
\le
\xstar^{ \top} A \ystar
\le
\xstar^{\top} A y 
$
hold for all $x \in \cP_m$ and $y \in \cP_n$.
The celebrated Minimax theorem \citep{vonneumann1928theory} implies that $(x_\star, y_\star)$ is an NE if and only if $x_\star \in  \xstarset = \argmax_{x} \cbrm[\big]{\min_y x^\top A y}$ and $y_\star \in \ystarset = \argmin_{y} \cbrm[\big]{\max_x x^\top A y}$.

A pure-strategy Nash equilibrium (PSNE) is a Nash equilibrium $(\xstar, \ystar)$ where both players choose a pure strategy,
i.e.,
$\xstar \in \{ 0, 1 \}^m$ and $\ystar \in \{ 0, 1 \}^n$.
A PSNE is also denoted by $(\istar, \jstar)$ where $\istar \in [m]$ and $\jstar \in [n]$ are the indices of the non-zero entries of $\xstar$ and $\ystar$,
respectively.
The \textit{duality gap} for $(\hat{x}, \hat{y}) \in \cP_m \times \cP_n$ is defined by
\begin{align}
    \label{eq:DG}
    \DG(\hat{x}, \hat{y}) = \max_{x \in \cP_m, y \in \cP_n} \left\{ x^\top A \hat{y} - \hat{x}^\top A y \right\} \ge 0,
\end{align}
which measures how far $(\hat{x}, \hat{y})$ is from a Nash equilibrium.
Indeed,
$(\xstar, \ystar)$ is a Nash equilibrium if and only if $\DG(\xstar, \ystar) = 0$.


\paragraph{Learning Dynamics with Bandit Feedback}
We consider a realistic setting where both players have no prior information about the game and repeatedly play the game with bandit feedback for $T$ rounds.
Specifically, 
in each round $t = 1, 2, \ldots, T$,
the row player chooses a mixed strategy $x_t \in \cP_m$, and the column player chooses $y_t \in \cP_n$.
They each draw their action $i_t \in [m]$ and $j_t \in [n]$ from their mixed strategy,
independently from each other. %
The nature then draws an outcome $r_t \in [-1,1]$ with expectation $\E \sbrm{r_t \mid i_t,j_t } = A(i_t,j_t)$ and reveals it to the row player as their realized reward and to the column player as their realized loss.\footnote{Our results hold for the more general setting where the observations for the two players are two different samples with mean $A(i_t, j_t)$.}
Note that this is a strongly uncoupled learning dynamic as defined by~\citet{daskalakis2011near}, where the players do not need to know the mixed strategy or the realized action of the opponent (in fact, not even their existence).
This property sets us apart from previous works such as \citet{zhou2017identify,o2021matrix} that use the realized action of both players to gain insight about the matrix $A$.

From each player's perspective, they are essentially facing an MAB problem with time-varying loss vectors: $\lr_t = -A y_t$ for the row player and $\lc_t = A^\top x_t$ for the column player, with noisy feedback for the coordinate they choose.
The standard performance measure in MAB is the (pseudo-)regret, defined for the row player and the column player respectively as


\begin{equation}
\begin{aligned}
    \Rr_T &= \max_{x \in \cP_m} \Rr_T(x),
    \quad\text{where}\;\;
    \Rr_T(x)
    =
    \E \sbrm[\bigg]{
    \sum_{t=1}^T
    (x - x_t)^\top
    A
    y_t
    },
    \\
    \Rc_T &= \max_{y \in \cP_n} \Rc_T(y),
    \quad\text{where}\;\;
    \Rc_T(y)
    =
    \E \sbrm[\bigg]{
    \sum_{t=1}^T
    x_t^\top
    A
    (y_t - y)
    }.  
\end{aligned}
    \label{eq:defR}
\end{equation}
We say that an algorithm achieves no-regret if $\Rr_T$ and $\Rc_T$ grow sublinearly as $o(T)$, which has
an important game-theoretic implication, since the duality gap of the average-iterate strategy $(\bar{x}_T, \bar{y}_T)$ where
$\bar{x}_T= \frac{1}{T}{\sum_{t=1}^T x_t}$ and
$\bar{y}_T= \frac{1}{T}{\sum_{t=1}^T y_t}$ is equal to the average regret:
\begin{align*}
    \DG(\E[\bar{x}_T], \E[\bar{y}_T])
    =
    \max_{x \in \cP_m, y \in \cP_n}
    \E \sbrm[\bigg]{
        x^\top A \rbrm[\Big]{ \frac{1}{T} \sum_{t=1}^T y_t }
        -
        \rbrm[\Big]{ \frac{1}{T} \sum_{t=1}^T x_t }^\top
        A 
        y
    }
    =
    \frac{1}{T}
    \left(
    \Rr_T
    +
    \Rc_T
    \right).
\end{align*}
 Therefore, the average-iterate strategy converges to a Nash equilibrium, with the convergence rate governed by the average regret.
By simply deploying standard adversarial MAB algorithms such as Exp3~\citep{auer2002nonstochastic},
one can obtain a convergence rate of $\tilde{O}(\sqrt{(m+n)/T})$, which is not improvable in the worst case even in this game setting~\citep{klein1999number}.
The goal of this work is thus to improve the regret/convergence rate in an instance-dependent manner.

\paragraph{Tsallis-INF Algorithm}

Throughout the paper, we let both players apply the $\frac{1}{2}$-Tsallis-INF algorithm~\citep{zimmert2021tsallis},
which is based on the Follow-the-Regularized-Leader (FTRL) framework and chooses its strategy by solving the following optimization problem:
\begin{align}
    \label{eq:Tsallis-INF}
    x_t
    =
    \argmin_{x \in \cP_m}
    \cbrm[\bigg]{
    \sum_{s=1}^{t-1}
    \hat{\lr}_s^\top x
    +
    \frac{1}{\eta_t}
    \psi(x)
    },
    \quad
    y_t
    =
    \argmin_{y \in \cP_n}
    \cbrm[\bigg]{
    \sum_{s=1}^{t-1}
    \hat{\lc}_s^\top x
    +
    \frac{1}{\eta_t}
    \psi(y)
    },
\end{align}
where $\eta_t = \frac{1}{2 \sqrt{t}}$ is the learning rate, $\psi(x) = - 2 \sum_{i =1}^m \sqrt{x(i)}$ (or $- 2 \sum_{j =1}^n \sqrt{y(j)}$ for the column player, with a slight abuse of the notation) is the $\frac{1}{2}$-Tsallis entropy regularizer, and $\hat{\lr}_s$ and $\hat{\lc_s}$ are importance-weighted (IW) unbiased estimators for the loss vector $\lr_s$ and $\lc_s$ respectively, defined via\footnote{
    Shifting the loss values uniformly does not affect the behavior of the algorithm, so the \(-1\) in this equation can be removed in implementation.
    We add it here just to ensure that these indeed serve as unbiased estimators of \(\lr_t, \lc_t\).
}
\begin{equation}
\begin{aligned}
    \hat{\lr_t}(i)=\frac{\one[i_t=i](1 - r_t)}{x_t(i)} - 1, 
    \quad
    \hat{\lc_t}(j)=
    \frac{\one[j_t=j] (1 + r_t)}{y_t(j)} - 1.
\end{aligned}
    \tag{IW}\label{eq:defIW}
\end{equation}


Tsallis-INF is an algorithm that achieves the optimal
instance-dependent bound in stochastic MAB and simultaneously the optimal worst-case bound in adversarial MAB.
Directly applying its guarantee for adversarial MAB shows that both players enjoy $\sqrt{T}$-type regret always, \emph{even when their opponent behaves arbitrarily}.
We summarize this in the following theorem and omit further mention in the rest of the paper.
On the other hand, note that one cannot directly apply the guarantee of Tsallis-INF for stochastic MAB since the players are not facing a stochastic MAB instance with a fixed expected loss vector.\footnote{In fact,
\citet{zimmert2021tsallis} provide instance-dependent regret in a setting more general than the standard stochastic setting, 
but still, that does not directly apply to the game setting, especially when a PSNE does not exist.
}
Instead, we will utilize an immediate regret bound, also summarized in the theorem below, along with the self-play nature and the zero-sum game structure to prove our results.
For completeness,
we provide the proof of this theorem in Appendix~\ref{sec:app:proof-tsallis-inf}.
\begin{theorem}[\citealp{zimmert2021tsallis}]
    \label{thm:Tsallis-INF}
    For any $x \in \cP_m$,
    the pseudo-regret of the Tsallis-INF algorithm against $x$ is bounded as follows for the row player (and similarly for the column players):
    \begin{align}\label{eq:Tsallis-INF-upper}
        \Rr_T(x)
        \le
        \min_{i^* \in [m]}
        \cbrm[\bigg]{
        \E \sbrm[\bigg]{
            C_1
            \sum_{t=1}^T
            \frac{1}{\sqrt{t}}
            \sum_{i \in [m] \setminus \{i^*\}}
            \sqrt{x_t(i)}
            -
            C_2
            \sqrt{T}
            \cdot
            D(x, x_{T+1})
        }
        },
    \end{align}
    where $C_1$ and $C_2$ are positive universal constants
    and
    $D(x', x) = 
    \sum_{i=1}^m \frac{1}{\sqrt{x(i)}}(\sqrt{x'(i)} - \sqrt{x(i)})^2
    $ 
    is the Bregman divergence associated with the $\frac{1}{2}$-Tsallis entropy.
    In particular, we always have $\Rr_T = O(\sqrt{mT})$ even if the opponent behaves arbitrarily.
\end{theorem}

We will show in Appendix~\ref{sec:app:proof-tsallis-inf} that Theorem~\ref{thm:Tsallis-INF} holds with $C_1 = 19$ and $C_2 = 2$.
It is worth noting that by using a more refined analysis similar to \citet{zimmert2021tsallis},  
the values of $C_1$ and $C_2$ could be further improved.
Additionally, replacing the IW estimator~\eqref{eq:defIW} with their more sophisticated reduced variance estimator could further improve the values of $C_1$ and $C_2$.
However, such precise analysis introduces extra terms like $O(m \log T)$,  
which unnecessarily complicates the upper bound.  
To avoid such unnecessary complexity and to handle noisy observations $r_t$,  
we provide an analysis that differs from theirs.




\section{Instance-Dependent Regret Bounds for General Zero-Sum Games}\label{sec:MSNE}
{
\allowdisplaybreaks

We now provide and discuss our main theorem for general zero-sum games,
which states two regret bounds both in the form of $c_1\log T + \sqrt{c_2T}$ for some game-dependent constants $c_1$ and $c_2$.
\begin{theorem}\label{thm:general-bound-together}
If both players follow the Tsallis-INF algorithm, then for any $x \in \cP_m$ and $y\in \cP_n$, the following two upper bounds simultaneously hold for the quantity: %
\begin{equation}
\max\cbrm[\Big]{
            \Rr_T(x)
            +
            \sqrt{T}\constd{\E\sbrm[\big]{
                D(x, x_{T+1})
            }},
            \Rc_T(y)
            +
            \sqrt{T}\constd{\E\sbrm[\big]{
                D(y, y_{T+1})
            }}}
    \tag{$\star$}\label{eq:thm2-lhs}
\end{equation}
\begin{itemize}[leftmargin=*]
\item \label{enum:main-bound-omega} \hspace{1in}
$\displaystyle
\rbr{\text{\ref{eq:thm2-lhs}}}=
O\rbrm[\Big]{\sqrt{T(|I|+|J|-2)} + 
\omegar
\log_{+}\frac{mT}{\omegar^2}
+
\omegac
\log_{+}\frac{nT}{\omegac^2}},
$


where $(\xstar, \ystar)$ is an NE with maximum support, $I$ and $J$ are the support of them respectively,
$\Deltar = \rbrm[\big]{\xstar^\top A \ystar}\one - A \ystar$, %
$\Deltac = A^\top \xstar - \rbrm[\big]{\xstar^\top A \ystar}\one$, $\omegar=\sum_{i \notin I}\frac{1}{\Deltar(i)}$, and $\omegac=\sum_{j \notin J}\frac{1}{\Deltac(j)}$
;

\item \label{enum:main-bound-gamma} \hspace{1in}
$\displaystyle
\rbr{\text{\ref{eq:thm2-lhs}}}=
O\rbrm[\bigg]{\sqrt{T}\rbrm[\Big]{
    \gammar\sqrt{\log_{+}\frac{m}{\gammar^2}}+\gammac\sqrt{\log_{+}\frac{n}{\gammac^2}}
} + 
\frac{m+n}{c}\log T
},
$

where $\gammar = \max_{\xstar \in \xstarset} \sum_{i\in [m]} \sqrt{\xstar(i)} - 1$, $\gammac = \max_{\ystar \in \ystarset} \sum_{j\in [n]} \sqrt{\ystar(j)} - 1$,
and $c >0$ is a game-dependent constant such that $\DG(x,y)\geq c \min_{\xstar\in\xstarset}\nbr{x-\xstar}_1 +c \min_{\ystar\in\ystarset}\nbr{y-\ystar}_1$ holds for all $(x,y) \in \cP_m \times \cP_n$ (which always exists).

\end{itemize}

\end{theorem}



While the key of the proof also relies on the self-bounding technique that is common in the analysis of Tsallis-INF,
some new ideas are required; see details in Appendix~\ref{sec:app:proof-thm2}.
We note that Eq.~\eqref{eq:thm2-lhs} is an upper bound on $\max\{\Rr_T(x), \Rc_T(x)\}$ since Bregman divergence is non-negative,
and we include the Bregman divergence terms in Eq.~\eqref{eq:thm2-lhs} because they are crucial for proving the last-iterate convergence result in Section~\ref{sec:last-iterate}.

In both bounds of Theorem~\ref{thm:general-bound-together}, the coefficients for $\sqrt{T}$ are smaller than the trivial bound $\max\{\sqrt{m}, \sqrt{n}\}$ and reflect the proximity of the NE to a pure strategy; specifically, $\sqrt{\abs{I}+\abs{J}-2}=\gammar=\gammac=0$ when the game has a unique PSNE. This case will be further elaborated in Section~\ref{sec:PSNE}.
More generally, the coefficient $\sqrt{\abs{I}+\abs{J}-2}$ in the first bound is small when the support of the NE is small, and this bound can be seen as a generalization of that in~\citet{balsubramani2016instance} for the special case of dueling bandits.
On the other hand, the coefficients $\gammar$ and $\gammac$ in the second bound are small when the NE are close to the boundary so that some actions have much larger weight than others.
This kind of problem dependence resembles that of~\citet{maiti2023instance} who study sample complexity of finding approximate NE in the special case of $2\times n$ games.
Indeed, their sample complexity to reach $\epsilon$ duality gap is at a high-level of order $1/\epsilon^2$ multiplied with a qualitatively similar problem-dependent constant, which exactly corresponds to our $\sqrt{T}$ regret term.

The inverse coefficients for the $\log T$ term, $\Deltar$ ($\Deltac$) and $c$, quantify the relative suboptimality of alternative actions compared to the NE. 
In particular, $\Deltar$ and $\Deltac$ are exactly the standard suboptimality gaps for a stochastic MAB instance with loss vector $-A\ystar$ and $A^\top\xstar$ respectively.
Very roughly speaking, this $\log T$ term can then be interpreted as the overhead of finding the non-support of the NE, which is relatively small and is as if playing an MAB with the opponent fixed to a minimax or maximin strategy.
On the other hand, the meaning of the inverse coefficient $c$ is less clear, but its existence is guaranteed by~\citet[Theorem~5]{wei2021linear}, and we also refer the reader to their work for more details on this constant.
It only approaches zero when a strategy sufficiently different from the NE has a disproportionately small duality gap. We demonstrate this with an example:
\begin{equation}
    A=\sbrm[\bigg]{
        \begin{array}{cc}
            0 &             3\eps \\
            1 - \eps &      2\eps
        \end{array}
    },\label{eq:example-2x2-game-matrix-simplified}
\end{equation}
where $0<\eps<\frac{1}{3}$. This game has a unique NE $\xstar=(1-3\eps,3\eps), \ystar=(\eps,1-\eps)$. Direct calculation shows that $c=\eps$ satisfies the requirement for $c$. When $\eps$ approaches zero, $\gamma\approx \sqrt{\eps}$ vanishes while $\frac 1c=\frac 1\eps$ explodes.
In particular, when $\epsilon \approx 1/T^{1/3}$, our regret bound is of order $T^{1/3}$, thus provably smaller than the worst-case $\sqrt{T}$ regret.
We will revisit this example in numerical experiments in Section~\ref{sec:experiments}.


    

    



    

}


\section{Games with Pure-Strategy Nash Equilibria}\label{sec:PSNE}
In this section,
we further discuss the case with a unique PSNE denoted as $(\istar, \jstar)$. 
Using the first bound in Theorem~\ref{thm:general-bound-together}, we immediately obtain the following regret bound since $|I|=|J|=1$.
\begin{corollary}\label{cor:PSNE}
For a game with a unique PSNE, if both players follow the Tsallis-INF algorithm, then the following regret bound holds:  
\begin{align*}
    \max\cbrm[\big]{\Rr_T,\Rc_T} = O \rbrm[\Big]{
    {
        \omegar
        \log_+ \frac{
            mT
        }{
            \omegar^2
        }
    }
    +
    { 
        \omegac
        \log_+ \frac{
            nT
        }{
            {\omegac}^2
        }
    }
    }
    = O \rbrm[\big]{
        \rbrm{\omegar + \omegac} \log T
    },
    \yestag\label{eq:psne-bound-formula}
\end{align*}
where $\omegar = \sum_{i \neq \istar}\frac{1}{\Deltar(i)}$, $\omegac  = \sum_{j \neq \jstar } \frac{1}{\Deltac(j)}$, $\Deltar(i)=A(\istar,\jstar)-A(i,\jstar)$, and $\Deltac(j)=A(\istar,j)-A(\istar,\jstar)$.
\end{corollary}


This is a generalization of the standard instance-dependent regret bound for stochastic MAB and also similar to those from the dueling bandit literature (e.g., \citealp{yue2012k, zoghi2014relative, saha2022versatile}).
We next show that this bound is asymptotically optimal in Section~\ref{sec:PSNE_lower_bound}.
After that, we present two other results:  the last-iterate convergence behavior of our algorithm (Section~\ref{sec:last-iterate})
and using our algorithm to identify the PSNE with high probability (Section~\ref{sec:PSNE_complexity}).





\subsection{Regret lower bound}\label{sec:PSNE_lower_bound}
In this section, 
we show that the regret bound in Corollary~\ref{cor:PSNE} is tight up to some constants.
In fact,
for any $\Deltar$ and $\Deltac$,
there exists a problem instance such that
$\liminf_{T \to \infty} \frac{\Rr_T + \Rc_T}{\log T} = \Omega( \omegar + \omegac )$
for any \textit{consistent} algorithms \citep[Definition 16.1]{lattimore2020bandit}.
Note that this lower bound is also applicable to \textit{coupled} algorithms,
i.e.,
this applies to situations where a single algorithm determines both \( i_t \) and \( j_t \) based on the observation of \( \{ r_s \}_{s<t} \).

We consider problem instances in which $r_t$ follows a Bernoulli distribution over $\{ -1, 1 \}$,
i.e.,
$r_t \sim \Berpm( A(i_t, j_t) )$
given $(i_t, j_t)$,
where $\Berpm( a )$ for a parameter $a \in [-1, 1]$ is a distribution that takes values $1$ and $-1$
with probability $(1+a)/2$ and $(1-a)/2$,
respectively.
Fix an algorithm for choosing $(i_t, j_t)$ given the observations of $\{ r_s \}_{s<t}$.

Let $\Deltar \in [0, 1/4]^m$ and $\Deltac \in [0,1/4]^n$ be such that
$\Deltar_{\istar} = 0$ and $\Deltac_{\jstar} = 0$ for some $\istar \in [m]$ and $\jstar \in [n]$.
Suppose $A$ is given by
\begin{align}
    \label{eq:defADelta}
    A = 
    \mathbf{1}_m {\Deltac}^\top 
    -
    \Deltar \mathbf{1}_n^\top .
\end{align}
Then,
$(\istar, \jstar)$ is a Nash equilibrium of the game with payoff matrix $A$
as we have
$A(\istar, \jstar) - A(i, \jstar) = \Delta(i) \ge 0$
and
$A(\istar, j) - A(\istar, \jstar) = \Delta(j) \ge 0$
for all $i \in [m]$ and $j \in [n]$.
Let $\Tr_{T,i}(A)$ and $\Tc_{T,j}(A)$ denote
the expected numbers of times the $i$-th row and $j$-th column are chosen:
\begin{align*}
    \Tr_{T,i}(A) =
    \E \sbrm[\bigg]{
    \sum_{t=1}^T
    \mathbf{1}[ i_t = i ]
    }
    =
    \E \sbrm[\bigg]{
    \sum_{t=1}^T
    x_t(i)
    },
    \quad
    \Tc_{T,j}(A) =
    \E \sbrm[\bigg]{
    \sum_{t=1}^T
    \mathbf{1}[ j_t = j ]
    }
    =
    \E \sbrm[\bigg]{
    \sum_{t=1}^T
    y_t(j)
    }.
\end{align*}
We then have the following lower bound:
\begin{theorem}
    \label{thm:RegLB}
    Suppose that there exist a function $g(m, n) > 0$ and a constant $c \in (0, 1)$ such that
    $\Rr_T + \Rc_T \le g(m,n) T^{1 - c}$ for any $\hat{A} \in [-1, 1]^{m \times n}$.
    Then,
    if $A$ is given by \eqref{eq:defADelta},
    we have
    \begin{align*}
        \Tr_{T,i}(A)
        =
        \Omega\rbrm[\bigg]{
            \frac{1}{(\Deltar(i))^2}
            \log 
            \rbrm[\Big]{
            \frac{\Deltar(i) T^c}{4 g(m,n)}
            }
        },
        \quad
        \Tc_{T,j}(A)
        =
        \Omega\rbrm[\bigg]{
            \frac{1}{(\Deltac(j))^2}
            \log 
            \rbrm[\Big]{
            \frac{\Deltar(j) T^c}{4 g(m,n)}
            }
        }
    \end{align*}
    for any $i \in [m]$ and $j \in [n]$ such that
    $\Deltar_i \neq 0$ and $\Deltac_j \neq 0$.
    Consequently,
    we have
    \begin{align*}
        \liminf_{T \to \infty} \frac{\Rr_T + \Rc_T}{\log T}
        =
        \Omega\rbrm[\bigg]{
            \sum_{\substack{i \in [m] \\\Deltar(i) > 0}} \frac{c}{\Deltar(i)}
            +
            \sum_{\substack{j \in [n]\\ \Deltac(j) > 0}} \frac{c}{\Deltac(j)}
        }
        =
        \Omega\left(
            c \cdot (\omegar + \omegac)
        \right).
    \end{align*}
\end{theorem}
\begin{remark}
    \label{rem:dueling-LB}
    \upshape
    For the special case in which $A$ is skew-symmetric and $\Deltar = \Deltac$,
    the regret lower bound for the dueling bandit problem
    \citep[Theorem 2]{komiyama2015regret} leads to the same asymptotic lower bound as Theorem~\ref{thm:RegLB} above.
    Our Theorem~\ref{thm:RegLB} is more general in that it relaxes this symmetry condition; however, the underlying idea used in their proofs are shared.
    That said, while \citet{komiyama2015regret} follow the proof structure of \citet[Theorem 1]{lai1985asymptotically}, our proof, provided in Appendix~\ref{sec:pfRegLB}, adopts a simplified analytical approach based on the Bretagnolle-Huber inequality (see, e.g.,~\citealp[Chapter 17]{lattimore2020bandit}).
\end{remark}
\begin{remark}
    \upshape
    Under the assumption that bandit algorithms are \textit{minimax optimal},
    i.e.,
    if there exists a universal constant $C$ such that
    $\Rr_T + \Rc_T \le C \sqrt{(m+n)T} $ holds for all $\hat{A}\in [-1, 1]^{m \times n}$,
    which corresponds to $g(m,n) = C\sqrt{m + n}$ and $c = 1/2$,
    we obtain the following finite-time lower bound:
    \begin{align*}
        R_T(A)
        =
        \Omega
        \rbrm[\bigg]{
            \sum_{\substack{i \in [m] \\\Deltar(i) > 0}} \frac{1}{\Deltar(i)}
            \log {
                \frac{(\Deltar(i))^2 T}{16 C^2 (m+n)}
            }
            +
            \sum_{\substack{j \in [n]\\ \Deltac(j) > 0}} \frac{1}{\Deltac(j)}
            \log {
                \frac{(\Deltac(j))^2 T}{16 C^2 (m+n)}
            }
        }.
    \end{align*}
    This matches upper bound in \eqref{eq:psne-bound-formula} up to a constant factor,
    under the conditions that
    $n = \Theta(m)$
    and that
    the values of non-zero $\Deltar(i)$'s and $\Deltac(j)$'s are equivalent up to a constant factor.
\end{remark}

\subsection{Last-iterate convergence}\label{sec:last-iterate}
Somewhat surprisingly, we show that Tsallis-INF also ensures 
the following last-iterate convergence guarantee.
\begin{proposition}
    \label{prop:last-iterate}
    For a game with a unique PSNE, if both players use the Tsallis-INF algorithm,
    the output distributions converge to the PSNE (in expectation) as follows: for any $t$,
    \begin{align}
        \E \sbr{
        D(x_{\star}, x_t)
        +
        D(y_{\star}, y_t)
        }
        =
        O \rbrm[\Big]{
            \frac{1}{\sqrt{t}}
            \rbrm[\big]{
                \omegar \log_{+}{
                    \frac{mt}{\omegar^2}
                }
                +
                \omegac \log_{+}{
                    \frac{nt}{\omegac^2}
                }
            }
        },
        \label{eq:li1}
    \end{align}
    where $D(\cdot, \cdot)$ represents the Bregman divergence associated with the $(1/2)$-Tsallis entropy.
    Consequently,
    we have
    \begin{align}
        \E \sbr{
            \sqrt{
                \DG \rbr{
                    x_t,
                    y_t
                }
            }
        }
        =
        O \rbrm[\Big]{
            \frac{1}{\sqrt{t}}
            \rbrm[\big]{
                \omegar \log_{+}{
                    \frac{mt}{\omegar^2}
                }
                +
                \omegac \log_{+}{
                    \frac{nt}{\omegac^2}
                }
            }
        }.
        \label{eq:li2}
    \end{align}
\end{proposition}
Even though $\E[\sqrt{\DG(x_t, y_t)}] = O(1/\sqrt{t})$ (ignoring other factors) only imply $\E[\DG(x_t, y_t)] = O(1/\sqrt{t})$ but not necessarily $\E[\DG(x_t, y_t)] = O(1/t)$ (so the last-iterate convergence might be slower than the average-iterate convergence),
this rate is already much better than the generic $O(1/t^{1/6})$ rate of~\citet{cai2023uncoupled} for general zero-sum games.
Our proof is also particularly simple and is in fact a simple corollary of the regret bound of Theorem~\ref{thm:general-bound-together}. \\

\begin{proof}
    Fix arbitrary $T \in \mathbb{N}$.
    From the first bound of Theorem~\ref{thm:general-bound-together},
    we have
    \begin{align*}
        &
        \Rr_T(\xstar)
        +
        \Rc_T(\ystar)
        +
        C_2 \sqrt{T} \E \left[
            D(x_{\star}, x_{T+1})
            +
            D(y_{\star}, y_{T+1})
        \right]
        =
        O \rbrm[\Big]{
        {
            \omegar
            \log_+ \frac{
                mT
            }{
                \omegar^2
            }
        }
        +
        { 
            \omegac
            \log_+ \frac{
                nT
            }{
                {\omegac}^2
            }
        }
        }.
    \end{align*}
    Since $(\xstar, \ystar)$ is a Nash equilibrium, we know that 
    $\Rr_T(\xstar) + \Rc_T(\ystar) \geq 0$.
    This implies
    \begin{align*}
        \E \left[
            D(x_{\star}, x_{T+1})
            +
            D(y_{\star}, y_{T+1})
        \right]
        &
        =
        O \rbrm[\Big]{
            \frac{1}{\sqrt{T}}
            \rbrm[\big]{
                \omegar \log_{+} {
                    \frac{mT}{\omegar^2}
                }
                +
                \omegac \log_{+} {
                    \frac{nT}{\omegac^2}
                }
            }
        },
    \end{align*}
    which completes the proof of \eqref{eq:li1}.
    The Bregmann divergence associated with $(1/2)$-Tsallis entropy 
    is bounded as
    \begin{align*}
            D( \xstar, x_t )
        =
            \sum_{i=1}^m
            \frac{1}{\sqrt{x_t(i)}}\rbrm[\Big]{
                \sqrt{\xstar(i)} - \sqrt{x_t(i)}
            }^2
        \ge
            \sum_{i \in [m] \setminus \{ \istar \}}
            \sqrt{x_t(i)}
        \ge
        \frac{1}{2}
            \sqrt{
                \| x_t - \xstar \|_1.
            }
    \end{align*}
    As $\DG$ is a $1$-Lipschitz function w.r.t.~the $L^1$ norm (Lemma~\ref{lem:DGLipschitz} in Appndix~\ref{sec:app:duality-gap}),
    we have
    \begin{align*}
        \E \sbr{
            \sqrt{
                \DG( x_t, y_t )
            }
        }
        &
        \le
        \E \sbr{
            \sqrt{
                \DG( \xstar, \ystar )
                +
                \| x_t - \xstar \|_1
                +
                \| y_t - \ystar \|_1
            }
        }
        \\
        &
        \le
        2 
        \E \sbr{
            D(\xstar, x_t)
            +
            D(\ystar, y_t)
        }.
    \end{align*}
    From this and \eqref{eq:li1},
    we obtain \eqref{eq:li2} as desired.
\end{proof}


\subsection{Sample complexity of identifying PSNE}\label{sec:PSNE_complexity}
While the main focus of our work is regret minimization, we show that our algorithm can also find the exact PSNE with high probability, again using its regret guarantee.
Specifically,
define $\Delta_{\min} = \min\cbr{ \min_{i \in [m] \setminus \{ \istar \}} \Deltar(i), \min_{j \in [n] \setminus \{ \jstar \}} \Deltac(j) }$.
We prove the following.
\begin{theorem}
    For output sequences $\{ i_t \}_{t=1}^T$ and $\{ j_t \}_{t=1}^T$ generated by the Tsallis-INF algorithm,
    let $\hat{i}_T$ and $\hat{j}_T$ be the most frequently chosen arms in these sequences,
    i.e.,
    $\hat{i}_T \in \argmax_{ i \in [m] } \absm[\big]{\{ t \in [T] \mid i_t = i \} }$ and
    $\hat{j}_T \in \argmax_{ j \in [n] } \absm[\big]{ \{ t \in [T] \mid j_t = j \} }$.
    Then,
    there exists a universal constant $\alpha > 0$ such that
    $(\hat{i}_T, \hat{j}_T) = (\istar, \jstar)$
    holds with probability at least $3/4$ for
    $T \ge \alpha \frac{\omegar + \omegac}{\Delta_{\min}}$.
\end{theorem}



\begin{proof}
    From the definition of $\hat{i}_T$
    and Markov's inequality,
    we have
    \begin{align*}
        \Pr \sbrm[\big]{ \hat{i}_T \neq i_{\star} }
        &
        \le
        \Pr \sbrm[\Big]{ \sum_{t=1}^T \mathbf{1}[i_t = i_{\star}] \le \frac{T}{2}  }
        =
        \Pr \sbrm[\Big]{ T - \sum_{t=1}^T \mathbf{1}[i_t = i_{\star}] \ge \frac{T}{2}  }
        \\
        &
        \le
        \frac{2}{T}
        \E \sbrm[\Big]{ T - \sum_{t=1}^T \mathbf{1}[i_t = i_{\star}]  }
        =
        2
        -
        \frac{2}{T}
        \E \sbrm[\Big]{ \sum_{t=1}^T x_{t}(\istar)  }
        =
        2(1 - \bar{x}_{T}(\istar)).
        \yestag\label{eq:Prx}
    \end{align*}
    As we have
        $
        \bar{x}_T \cdot
        \Deltar
        \ge
        \sum_{i \neq i_{\star}}
        \bar{x}_{t}(i)
        \Deltar(i)
        \ge
        \sum_{i \neq i_{\star}}
        \bar{x}_{T}(i)
        \Deltar_{\min}
        =
        \Deltar_{\min} ( 1 - \bar{x}_{T}(\istar)),
        $
    by combining this with \eqref{eq:Prx},
    we obtain
    $
        \Pr \sbrm[\big]{ \hat{i}_T \neq i_{\star} }
        \le
        2(1 - \bar{x}_{T}( i_{\star}))
        \le
        \frac{2}{\Deltar_{\min}}
        \bar{x}_T \cdot\Deltar.
    $
    As a similar bound holds for $\Pr \sbrm[\big]{ \hat{j}_T \neq j_{\star} }$,
    we have
    \begin{align}
        \Pr \sbrm[\Big]{ (\hat{i}_T, \hat{j}_T)  \neq ( \istar, \jstar) }
        \le
        \Pr \sbrm[\big]{ \hat{i}_T \neq i_{\star}} + \Pr\sbrm[\big]{ \hat{j}_T \neq j_{\star} }
        \le
        \frac{2}{\Delta_{\min}}
        \rbrm[\Big]{
            \bar{x}_T \cdot \Deltar
            +
            \bar{y}_T \cdot \Deltac
        }
        \label{eq:PrDelta}
    \end{align}
    From the definition of $\Deltar$
    and Theorem~\ref{thm:general-bound-together},
    we have
    \begin{align*}
        \frac{2}{\Delta_{\min}}
        \rbrm[\Big]{
            \bar{x}_T \cdot \Deltar
            +
            \bar{y}_T \cdot \Deltac
        }
        & \le
        \frac{2}{\Delta_{\min}} \frac{\Rr_T + \Rc_T}{T}
        \\
        &
        \leq 2C_3
        \rbrm[\Big]{
            {
                \frac{\omegar}{\Delta_{\min}T}
                \log_+ \frac{ mT }{ \omegar^2 }
            }
            +
            { 
                \frac{\omegac}{\Delta_{\min}T}
                \log_+ \frac{ nT }{ \omegac^2 }
            }
        } \\
        &
        \leq 2C_3
        \rbrm[\Big]{
            {
                \frac{\omegar}{\Delta_{\min}T}
                \log_+ \frac{ \Delta_{\min}T }{ \omegar }
            }
            +
            { 
                \frac{\omegac}{\Delta_{\min}T}
                \log_+ \frac{ \Delta_{\min}T }{ \omegac }
            }
        },\\
        &
        \leq 2C_3
        \rbrm[\Big]{
            {
                \frac{1}{\alpha}
                \log_+ \alpha
            }
            +
            { 
                \frac{1}{\alpha}
                \log_+ \alpha
            }
        } \leq \frac{1}{4},
    \end{align*}
    where $C_3$ is the contant factor hidden by the $O(\cdot)$ symbol in \eqref{eq:psne-bound-formula},
    and in the third inequality we used the fact that $\omegar \ge m \Delta_{\min}$ and $\omegac \ge n \Delta_{\min}$.
    The last inequality holds if we take $\alpha=8C_3+4$.
    By combining this with \eqref{eq:PrDelta},
    we obtain
    $\Pr \sbrm[\big]{ (\hat{i}_T, \hat{j}_T)  \neq ( \istar, \jstar) } \le 1/4$,
    which completes the proof.
\end{proof}
From this,
we can further boost the confidence and identify the PSNE with probability at least $(1-\delta)$ with 
$O\rbrm[\big]{\frac{\omegar + \omegac}{\Delta_{\min}} \log \rbrm{ 1/\delta } }$ samples.
More concretely,
consider repeating $S > 1$ independent trials of calculating $\hat{i}_T$.
Let $\hat{i}_{T, s}$ be the result for the $s$-th trial.
Let $\tilde{i}_{T,S} \in \argmax_{i \in [m]} \absm[\big]{\{ s \in [S] \mid \hat{i}_{T, s} = i \} }$
be the arm most frequently chosen in these $S$ trials.
We then have
\begin{align*}
    \Pr \sbrm[\big]{
    \tilde{i}_{T,S}
    \neq 
    i_{\star}
    }
    &
    \le
    \Pr \sbrm[\bigg]{
    \sum_{s=1}^S
    \mathbf{1}\sbrm[\big]{
    \hat{i}_{T, s}
    =
    i_{\star}
    }
    \le
    S/2
    }
    \\
    &
    \le
    \Pr\sbrm[\bigg]{
    \sum_{s=1}^S
    X_s
    \le
    S/2
    \mid
    X_s \sim \mathrm{Ber}(3/4),
    ~
    \mbox{i.i.d. for $s \in [S]$}
    }
    \le \exp \rbr{ \Omega( - S ) }.
\end{align*}
Hence,
for any $\delta \in (0, 1)$,
by setting $S = \Theta ( 1 / \delta )$,
we have 
$
\Pr \left[
(\tilde{i}_{T,S}, \tilde{j}_{T,S})
=
(\istar, \jstar)
\right]
\ge 1 - \delta
$.
We note that,
to perform this procedure, it is necessary to know an approximate value of $\frac{\omegar + \omegac}{\Delta_{\min}}$.


We also note that due to Lemma~\ref{lem:sqrtk-ratio},
our sample complexity is at most $O\rbrm[\big]{\sqrt{\max\cbrm{n,m}}}$ times the information-theoretic optimal,
which is $O\rbrm[\big] { 
\sum_{i \in [m] \setminus \{ \istar \}} \frac{1}{{\Deltar}^2_i}
+
\sum_{j \in [n] \setminus \{ \jstar \}} \frac{1}{{\Deltac}^2_j}
}
$ and is achieved by the Midsearch algorithm of~\citet{maiti2024midsearch}.
However, their algorithm is coupled; that is, Midsearch requires the algorithm to control both players at the same time,
while our algorithm is a no-regret uncoupled learning dynamic.


\section{Numerical Experiments}\label{sec:experiments}



\begin{figure}[t]
    \centering
    \includegraphics[width=0.8\textwidth]{img/regret_vs_iter.pdf}
    \caption{
    Regret scaling for Tsallis-INF and two other bandit algorithms. Each configuration $(T)$ is run for 512 trials. The interval between the 10th and 90th percentile is overlaid. The thicker dashed line represents a linear fit on the $T\geq 10^5$ subset of the log-log data.}
    \label{fig:regret-comparison}
\end{figure}


To validate our theoretical results,
we conduct a few numerical experiments.

The first experiment compares
Tsallis-INF against two baselines in terms of the regret: 
the classical UCB1~\citep{auer2002finite} and Exp3~\citep{auer2002nonstochastic} algorithms, 
which are known to have $O(T)$ and $\tilde{O}(\sqrt{T})$ regret bounds respectively in the adversarial setting.
We compare them on the game associated with $A$ defined in \eqref{eq:example-2x2-game-matrix-simplified},
with varying $T$ and $\eps=T^{-1/3}$,
where feedback $r_t$ follows a Bernoulli distribution over $\{ -1, 1 \}$ such that $ \E[r_t \mid i_t, j_t] = A(i_t, j_t)$.
As discussed, Theorem~\ref{thm:general-bound-together} predicts a regret of $\tilde{O}(T^{1/3})$ for Tsallis-INF.
The result of the experiment agrees with all these bounds in Figure~\ref{fig:regret-comparison},
where the asymptotic slope in the log-log plot (shown with a linear fit on the $T\geq 10^5$ region) is close to the theoretical prediction.


\begin{figure}[t]
    \centering
    \includegraphics[width=0.8\textwidth]{img/identify_P_vs_iterations_by_H1.pdf}
    \caption{
        Experimental validation of Tsallis-INF's PSNE identification capability.
        The plot shows the algorithm's success rate in correctly identifying PSNE
        against the number of itrations.
        We use a hard instance of a $256\times 256$ matrix and $\Delta_1=0.1$,
        running 512 trials for each $\Delta_{\min}$ values
        over a horizon of $128\OPT$ iterations,
        where $\OPT$ is the theoretical lower bound for PSNE identification.
        The $x$-axis is scaled by $1/\OPT$.
        }
    \label{fig:PSNE-id-rate}
\end{figure}

We have discussed in Section~\ref{sec:PSNE_complexity} that Tsallis-INF needs $\frac{\omegar+\omegac}{\Delta_{\min}}$ iterations to identify the PSNE of a game. To validate our theoretical bounds, we conduct our second experiment using the following hard instance  introduced by \citet{maiti2024midsearch}:
\begin{equation}
    A=\begin{bNiceArray}{ccccc}[nullify-dots, margin, custom-line = {letter=I, tikz=dashed}, cell-space-limits = 4pt]
        0 & 2{\Delta_{\min}} & \Block{1-3}{} 2{\Delta_1} &\Cdots& 2{\Delta_1} \\
        -2{\Delta_{\min}} & \Block{4-4}{} 
                       0      & 1      & \Cdots & 1      \\
        \Block{3-1}{}
        -2{\Delta_1} & -1     & \Ddots & \Ddots & \Vdots \\
        \Vdots       & \Vdots & \Ddots & \Ddots & 1      \\
        -2{\Delta_1} & -1     & \Cdots & -1     & 0      \\
    \end{bNiceArray},
    \label{eq:psne-experiment-array}
\end{equation}
where the top-left entry is the PSNE. We set the number of actions $n=m=256$ and the gap $\Delta_1=0.1$, and vary the value of $\Delta_{\min}$.
Let $\OPT$ represent the theoretical optimal bound for identifying PSNE (ignoring log terms), defined as 
$\OPT=
\sum_{i \in [m] \setminus \{ \istar \}} \frac{1}{{\Deltar}^2_i}
+
\sum_{j \in [n] \setminus \{ \jstar \}} \frac{1}{{\Deltac}^2_j}
$,
which simplifies to $\frac{1}{2\Delta_{\min}^2}+\frac{m-2}{2\Delta_1^2}$ in this experiment.
\citepos{maiti2024midsearch} achieve the optimal $\tilde{O}(\OPT)$ sample complexity,
and their Figure~2 suggests that the sample complexity of Tsallis-INF divided by $\OPT$ is unbounded as $\Delta_{\min}$ decreases,
but our analysis in Section~\ref{sec:PSNE_complexity} disagrees with this trend.
As shown in Figure~\ref{fig:PSNE-id-rate},
the number of iterations needed to identify the PSNE divided by $\OPT$
decreases and then increases
as $\Delta_{\min}$ varies.
Lemma~\ref{lem:sqrtk-ratio} predicts the minimum ratio occurs when $\frac{\Delta_{\min}}{\Delta_1}=\frac{1}{\sqrt{m}+1}=1/17$,
and among the values we tested,
the minimum is reached when $\frac{\Delta_{\min}}{\Delta_1}=0.005/0.1=1/20$,
closely matching the prediction.
This supports our derived bound of $\tilde{O}\rbrm[\big]{\sqrt{m}\cdot \OPT}$.

The code for reproducing the experiments is available on 
\url{https://github.com/EtaoinWu/instance-dependent-game-learning}.



\section{Conclusion}
\label{sec:Conclusion}
In this paper, we proposed a complete real-time planning and control approach for continuous, reliable, and fast online generation of dynamically feasible Bernstein trajectories and control for FW aircrafts. The generated trajectories span kilometers, navigating through multiple waypoints. By leveraging differential flatness equations for coordinated flight, we ensure precise trajectory tracking. Our approach guarantees smooth transitions from simulation to real-world applications, enabling timely field deployment. The system also features a user-friendly mission planning interface. Continuous replanning  maintains the rajectory curvature 
$\kappa$ within limits, preventing abrupt roll changes.

Future works will include the ability to add  a higher-level kinodynamic path planner to optimize waypoint spatial allocation and improve replanning success, and enhancing the trajectory-tracking algorithm by refining the aerodynamic coefficient estimation. 





\bibliographystyle{abbrvnat}
\bibliography{reference.bib}


\subsection{Lloyd-Max Algorithm}
\label{subsec:Lloyd-Max}
For a given quantization bitwidth $B$ and an operand $\bm{X}$, the Lloyd-Max algorithm finds $2^B$ quantization levels $\{\hat{x}_i\}_{i=1}^{2^B}$ such that quantizing $\bm{X}$ by rounding each scalar in $\bm{X}$ to the nearest quantization level minimizes the quantization MSE. 

The algorithm starts with an initial guess of quantization levels and then iteratively computes quantization thresholds $\{\tau_i\}_{i=1}^{2^B-1}$ and updates quantization levels $\{\hat{x}_i\}_{i=1}^{2^B}$. Specifically, at iteration $n$, thresholds are set to the midpoints of the previous iteration's levels:
\begin{align*}
    \tau_i^{(n)}=\frac{\hat{x}_i^{(n-1)}+\hat{x}_{i+1}^{(n-1)}}2 \text{ for } i=1\ldots 2^B-1
\end{align*}
Subsequently, the quantization levels are re-computed as conditional means of the data regions defined by the new thresholds:
\begin{align*}
    \hat{x}_i^{(n)}=\mathbb{E}\left[ \bm{X} \big| \bm{X}\in [\tau_{i-1}^{(n)},\tau_i^{(n)}] \right] \text{ for } i=1\ldots 2^B
\end{align*}
where to satisfy boundary conditions we have $\tau_0=-\infty$ and $\tau_{2^B}=\infty$. The algorithm iterates the above steps until convergence.

Figure \ref{fig:lm_quant} compares the quantization levels of a $7$-bit floating point (E3M3) quantizer (left) to a $7$-bit Lloyd-Max quantizer (right) when quantizing a layer of weights from the GPT3-126M model at a per-tensor granularity. As shown, the Lloyd-Max quantizer achieves substantially lower quantization MSE. Further, Table \ref{tab:FP7_vs_LM7} shows the superior perplexity achieved by Lloyd-Max quantizers for bitwidths of $7$, $6$ and $5$. The difference between the quantizers is clear at 5 bits, where per-tensor FP quantization incurs a drastic and unacceptable increase in perplexity, while Lloyd-Max quantization incurs a much smaller increase. Nevertheless, we note that even the optimal Lloyd-Max quantizer incurs a notable ($\sim 1.5$) increase in perplexity due to the coarse granularity of quantization. 

\begin{figure}[h]
  \centering
  \includegraphics[width=0.7\linewidth]{sections/figures/LM7_FP7.pdf}
  \caption{\small Quantization levels and the corresponding quantization MSE of Floating Point (left) vs Lloyd-Max (right) Quantizers for a layer of weights in the GPT3-126M model.}
  \label{fig:lm_quant}
\end{figure}

\begin{table}[h]\scriptsize
\begin{center}
\caption{\label{tab:FP7_vs_LM7} \small Comparing perplexity (lower is better) achieved by floating point quantizers and Lloyd-Max quantizers on a GPT3-126M model for the Wikitext-103 dataset.}
\begin{tabular}{c|cc|c}
\hline
 \multirow{2}{*}{\textbf{Bitwidth}} & \multicolumn{2}{|c|}{\textbf{Floating-Point Quantizer}} & \textbf{Lloyd-Max Quantizer} \\
 & Best Format & Wikitext-103 Perplexity & Wikitext-103 Perplexity \\
\hline
7 & E3M3 & 18.32 & 18.27 \\
6 & E3M2 & 19.07 & 18.51 \\
5 & E4M0 & 43.89 & 19.71 \\
\hline
\end{tabular}
\end{center}
\end{table}

\subsection{Proof of Local Optimality of LO-BCQ}
\label{subsec:lobcq_opt_proof}
For a given block $\bm{b}_j$, the quantization MSE during LO-BCQ can be empirically evaluated as $\frac{1}{L_b}\lVert \bm{b}_j- \bm{\hat{b}}_j\rVert^2_2$ where $\bm{\hat{b}}_j$ is computed from equation (\ref{eq:clustered_quantization_definition}) as $C_{f(\bm{b}_j)}(\bm{b}_j)$. Further, for a given block cluster $\mathcal{B}_i$, we compute the quantization MSE as $\frac{1}{|\mathcal{B}_{i}|}\sum_{\bm{b} \in \mathcal{B}_{i}} \frac{1}{L_b}\lVert \bm{b}- C_i^{(n)}(\bm{b})\rVert^2_2$. Therefore, at the end of iteration $n$, we evaluate the overall quantization MSE $J^{(n)}$ for a given operand $\bm{X}$ composed of $N_c$ block clusters as:
\begin{align*}
    \label{eq:mse_iter_n}
    J^{(n)} = \frac{1}{N_c} \sum_{i=1}^{N_c} \frac{1}{|\mathcal{B}_{i}^{(n)}|}\sum_{\bm{v} \in \mathcal{B}_{i}^{(n)}} \frac{1}{L_b}\lVert \bm{b}- B_i^{(n)}(\bm{b})\rVert^2_2
\end{align*}

At the end of iteration $n$, the codebooks are updated from $\mathcal{C}^{(n-1)}$ to $\mathcal{C}^{(n)}$. However, the mapping of a given vector $\bm{b}_j$ to quantizers $\mathcal{C}^{(n)}$ remains as  $f^{(n)}(\bm{b}_j)$. At the next iteration, during the vector clustering step, $f^{(n+1)}(\bm{b}_j)$ finds new mapping of $\bm{b}_j$ to updated codebooks $\mathcal{C}^{(n)}$ such that the quantization MSE over the candidate codebooks is minimized. Therefore, we obtain the following result for $\bm{b}_j$:
\begin{align*}
\frac{1}{L_b}\lVert \bm{b}_j - C_{f^{(n+1)}(\bm{b}_j)}^{(n)}(\bm{b}_j)\rVert^2_2 \le \frac{1}{L_b}\lVert \bm{b}_j - C_{f^{(n)}(\bm{b}_j)}^{(n)}(\bm{b}_j)\rVert^2_2
\end{align*}

That is, quantizing $\bm{b}_j$ at the end of the block clustering step of iteration $n+1$ results in lower quantization MSE compared to quantizing at the end of iteration $n$. Since this is true for all $\bm{b} \in \bm{X}$, we assert the following:
\begin{equation}
\begin{split}
\label{eq:mse_ineq_1}
    \tilde{J}^{(n+1)} &= \frac{1}{N_c} \sum_{i=1}^{N_c} \frac{1}{|\mathcal{B}_{i}^{(n+1)}|}\sum_{\bm{b} \in \mathcal{B}_{i}^{(n+1)}} \frac{1}{L_b}\lVert \bm{b} - C_i^{(n)}(b)\rVert^2_2 \le J^{(n)}
\end{split}
\end{equation}
where $\tilde{J}^{(n+1)}$ is the the quantization MSE after the vector clustering step at iteration $n+1$.

Next, during the codebook update step (\ref{eq:quantizers_update}) at iteration $n+1$, the per-cluster codebooks $\mathcal{C}^{(n)}$ are updated to $\mathcal{C}^{(n+1)}$ by invoking the Lloyd-Max algorithm \citep{Lloyd}. We know that for any given value distribution, the Lloyd-Max algorithm minimizes the quantization MSE. Therefore, for a given vector cluster $\mathcal{B}_i$ we obtain the following result:

\begin{equation}
    \frac{1}{|\mathcal{B}_{i}^{(n+1)}|}\sum_{\bm{b} \in \mathcal{B}_{i}^{(n+1)}} \frac{1}{L_b}\lVert \bm{b}- C_i^{(n+1)}(\bm{b})\rVert^2_2 \le \frac{1}{|\mathcal{B}_{i}^{(n+1)}|}\sum_{\bm{b} \in \mathcal{B}_{i}^{(n+1)}} \frac{1}{L_b}\lVert \bm{b}- C_i^{(n)}(\bm{b})\rVert^2_2
\end{equation}

The above equation states that quantizing the given block cluster $\mathcal{B}_i$ after updating the associated codebook from $C_i^{(n)}$ to $C_i^{(n+1)}$ results in lower quantization MSE. Since this is true for all the block clusters, we derive the following result: 
\begin{equation}
\begin{split}
\label{eq:mse_ineq_2}
     J^{(n+1)} &= \frac{1}{N_c} \sum_{i=1}^{N_c} \frac{1}{|\mathcal{B}_{i}^{(n+1)}|}\sum_{\bm{b} \in \mathcal{B}_{i}^{(n+1)}} \frac{1}{L_b}\lVert \bm{b}- C_i^{(n+1)}(\bm{b})\rVert^2_2  \le \tilde{J}^{(n+1)}   
\end{split}
\end{equation}

Following (\ref{eq:mse_ineq_1}) and (\ref{eq:mse_ineq_2}), we find that the quantization MSE is non-increasing for each iteration, that is, $J^{(1)} \ge J^{(2)} \ge J^{(3)} \ge \ldots \ge J^{(M)}$ where $M$ is the maximum number of iterations. 
%Therefore, we can say that if the algorithm converges, then it must be that it has converged to a local minimum. 
\hfill $\blacksquare$


\begin{figure}
    \begin{center}
    \includegraphics[width=0.5\textwidth]{sections//figures/mse_vs_iter.pdf}
    \end{center}
    \caption{\small NMSE vs iterations during LO-BCQ compared to other block quantization proposals}
    \label{fig:nmse_vs_iter}
\end{figure}

Figure \ref{fig:nmse_vs_iter} shows the empirical convergence of LO-BCQ across several block lengths and number of codebooks. Also, the MSE achieved by LO-BCQ is compared to baselines such as MXFP and VSQ. As shown, LO-BCQ converges to a lower MSE than the baselines. Further, we achieve better convergence for larger number of codebooks ($N_c$) and for a smaller block length ($L_b$), both of which increase the bitwidth of BCQ (see Eq \ref{eq:bitwidth_bcq}).


\subsection{Additional Accuracy Results}
%Table \ref{tab:lobcq_config} lists the various LOBCQ configurations and their corresponding bitwidths.
\begin{table}
\setlength{\tabcolsep}{4.75pt}
\begin{center}
\caption{\label{tab:lobcq_config} Various LO-BCQ configurations and their bitwidths.}
\begin{tabular}{|c||c|c|c|c||c|c||c|} 
\hline
 & \multicolumn{4}{|c||}{$L_b=8$} & \multicolumn{2}{|c||}{$L_b=4$} & $L_b=2$ \\
 \hline
 \backslashbox{$L_A$\kern-1em}{\kern-1em$N_c$} & 2 & 4 & 8 & 16 & 2 & 4 & 2 \\
 \hline
 64 & 4.25 & 4.375 & 4.5 & 4.625 & 4.375 & 4.625 & 4.625\\
 \hline
 32 & 4.375 & 4.5 & 4.625& 4.75 & 4.5 & 4.75 & 4.75 \\
 \hline
 16 & 4.625 & 4.75& 4.875 & 5 & 4.75 & 5 & 5 \\
 \hline
\end{tabular}
\end{center}
\end{table}

%\subsection{Perplexity achieved by various LO-BCQ configurations on Wikitext-103 dataset}

\begin{table} \centering
\begin{tabular}{|c||c|c|c|c||c|c||c|} 
\hline
 $L_b \rightarrow$& \multicolumn{4}{c||}{8} & \multicolumn{2}{c||}{4} & 2\\
 \hline
 \backslashbox{$L_A$\kern-1em}{\kern-1em$N_c$} & 2 & 4 & 8 & 16 & 2 & 4 & 2  \\
 %$N_c \rightarrow$ & 2 & 4 & 8 & 16 & 2 & 4 & 2 \\
 \hline
 \hline
 \multicolumn{8}{c}{GPT3-1.3B (FP32 PPL = 9.98)} \\ 
 \hline
 \hline
 64 & 10.40 & 10.23 & 10.17 & 10.15 &  10.28 & 10.18 & 10.19 \\
 \hline
 32 & 10.25 & 10.20 & 10.15 & 10.12 &  10.23 & 10.17 & 10.17 \\
 \hline
 16 & 10.22 & 10.16 & 10.10 & 10.09 &  10.21 & 10.14 & 10.16 \\
 \hline
  \hline
 \multicolumn{8}{c}{GPT3-8B (FP32 PPL = 7.38)} \\ 
 \hline
 \hline
 64 & 7.61 & 7.52 & 7.48 &  7.47 &  7.55 &  7.49 & 7.50 \\
 \hline
 32 & 7.52 & 7.50 & 7.46 &  7.45 &  7.52 &  7.48 & 7.48  \\
 \hline
 16 & 7.51 & 7.48 & 7.44 &  7.44 &  7.51 &  7.49 & 7.47  \\
 \hline
\end{tabular}
\caption{\label{tab:ppl_gpt3_abalation} Wikitext-103 perplexity across GPT3-1.3B and 8B models.}
\end{table}

\begin{table} \centering
\begin{tabular}{|c||c|c|c|c||} 
\hline
 $L_b \rightarrow$& \multicolumn{4}{c||}{8}\\
 \hline
 \backslashbox{$L_A$\kern-1em}{\kern-1em$N_c$} & 2 & 4 & 8 & 16 \\
 %$N_c \rightarrow$ & 2 & 4 & 8 & 16 & 2 & 4 & 2 \\
 \hline
 \hline
 \multicolumn{5}{|c|}{Llama2-7B (FP32 PPL = 5.06)} \\ 
 \hline
 \hline
 64 & 5.31 & 5.26 & 5.19 & 5.18  \\
 \hline
 32 & 5.23 & 5.25 & 5.18 & 5.15  \\
 \hline
 16 & 5.23 & 5.19 & 5.16 & 5.14  \\
 \hline
 \multicolumn{5}{|c|}{Nemotron4-15B (FP32 PPL = 5.87)} \\ 
 \hline
 \hline
 64  & 6.3 & 6.20 & 6.13 & 6.08  \\
 \hline
 32  & 6.24 & 6.12 & 6.07 & 6.03  \\
 \hline
 16  & 6.12 & 6.14 & 6.04 & 6.02  \\
 \hline
 \multicolumn{5}{|c|}{Nemotron4-340B (FP32 PPL = 3.48)} \\ 
 \hline
 \hline
 64 & 3.67 & 3.62 & 3.60 & 3.59 \\
 \hline
 32 & 3.63 & 3.61 & 3.59 & 3.56 \\
 \hline
 16 & 3.61 & 3.58 & 3.57 & 3.55 \\
 \hline
\end{tabular}
\caption{\label{tab:ppl_llama7B_nemo15B} Wikitext-103 perplexity compared to FP32 baseline in Llama2-7B and Nemotron4-15B, 340B models}
\end{table}

%\subsection{Perplexity achieved by various LO-BCQ configurations on MMLU dataset}


\begin{table} \centering
\begin{tabular}{|c||c|c|c|c||c|c|c|c|} 
\hline
 $L_b \rightarrow$& \multicolumn{4}{c||}{8} & \multicolumn{4}{c||}{8}\\
 \hline
 \backslashbox{$L_A$\kern-1em}{\kern-1em$N_c$} & 2 & 4 & 8 & 16 & 2 & 4 & 8 & 16  \\
 %$N_c \rightarrow$ & 2 & 4 & 8 & 16 & 2 & 4 & 2 \\
 \hline
 \hline
 \multicolumn{5}{|c|}{Llama2-7B (FP32 Accuracy = 45.8\%)} & \multicolumn{4}{|c|}{Llama2-70B (FP32 Accuracy = 69.12\%)} \\ 
 \hline
 \hline
 64 & 43.9 & 43.4 & 43.9 & 44.9 & 68.07 & 68.27 & 68.17 & 68.75 \\
 \hline
 32 & 44.5 & 43.8 & 44.9 & 44.5 & 68.37 & 68.51 & 68.35 & 68.27  \\
 \hline
 16 & 43.9 & 42.7 & 44.9 & 45 & 68.12 & 68.77 & 68.31 & 68.59  \\
 \hline
 \hline
 \multicolumn{5}{|c|}{GPT3-22B (FP32 Accuracy = 38.75\%)} & \multicolumn{4}{|c|}{Nemotron4-15B (FP32 Accuracy = 64.3\%)} \\ 
 \hline
 \hline
 64 & 36.71 & 38.85 & 38.13 & 38.92 & 63.17 & 62.36 & 63.72 & 64.09 \\
 \hline
 32 & 37.95 & 38.69 & 39.45 & 38.34 & 64.05 & 62.30 & 63.8 & 64.33  \\
 \hline
 16 & 38.88 & 38.80 & 38.31 & 38.92 & 63.22 & 63.51 & 63.93 & 64.43  \\
 \hline
\end{tabular}
\caption{\label{tab:mmlu_abalation} Accuracy on MMLU dataset across GPT3-22B, Llama2-7B, 70B and Nemotron4-15B models.}
\end{table}


%\subsection{Perplexity achieved by various LO-BCQ configurations on LM evaluation harness}

\begin{table} \centering
\begin{tabular}{|c||c|c|c|c||c|c|c|c|} 
\hline
 $L_b \rightarrow$& \multicolumn{4}{c||}{8} & \multicolumn{4}{c||}{8}\\
 \hline
 \backslashbox{$L_A$\kern-1em}{\kern-1em$N_c$} & 2 & 4 & 8 & 16 & 2 & 4 & 8 & 16  \\
 %$N_c \rightarrow$ & 2 & 4 & 8 & 16 & 2 & 4 & 2 \\
 \hline
 \hline
 \multicolumn{5}{|c|}{Race (FP32 Accuracy = 37.51\%)} & \multicolumn{4}{|c|}{Boolq (FP32 Accuracy = 64.62\%)} \\ 
 \hline
 \hline
 64 & 36.94 & 37.13 & 36.27 & 37.13 & 63.73 & 62.26 & 63.49 & 63.36 \\
 \hline
 32 & 37.03 & 36.36 & 36.08 & 37.03 & 62.54 & 63.51 & 63.49 & 63.55  \\
 \hline
 16 & 37.03 & 37.03 & 36.46 & 37.03 & 61.1 & 63.79 & 63.58 & 63.33  \\
 \hline
 \hline
 \multicolumn{5}{|c|}{Winogrande (FP32 Accuracy = 58.01\%)} & \multicolumn{4}{|c|}{Piqa (FP32 Accuracy = 74.21\%)} \\ 
 \hline
 \hline
 64 & 58.17 & 57.22 & 57.85 & 58.33 & 73.01 & 73.07 & 73.07 & 72.80 \\
 \hline
 32 & 59.12 & 58.09 & 57.85 & 58.41 & 73.01 & 73.94 & 72.74 & 73.18  \\
 \hline
 16 & 57.93 & 58.88 & 57.93 & 58.56 & 73.94 & 72.80 & 73.01 & 73.94  \\
 \hline
\end{tabular}
\caption{\label{tab:mmlu_abalation} Accuracy on LM evaluation harness tasks on GPT3-1.3B model.}
\end{table}

\begin{table} \centering
\begin{tabular}{|c||c|c|c|c||c|c|c|c|} 
\hline
 $L_b \rightarrow$& \multicolumn{4}{c||}{8} & \multicolumn{4}{c||}{8}\\
 \hline
 \backslashbox{$L_A$\kern-1em}{\kern-1em$N_c$} & 2 & 4 & 8 & 16 & 2 & 4 & 8 & 16  \\
 %$N_c \rightarrow$ & 2 & 4 & 8 & 16 & 2 & 4 & 2 \\
 \hline
 \hline
 \multicolumn{5}{|c|}{Race (FP32 Accuracy = 41.34\%)} & \multicolumn{4}{|c|}{Boolq (FP32 Accuracy = 68.32\%)} \\ 
 \hline
 \hline
 64 & 40.48 & 40.10 & 39.43 & 39.90 & 69.20 & 68.41 & 69.45 & 68.56 \\
 \hline
 32 & 39.52 & 39.52 & 40.77 & 39.62 & 68.32 & 67.43 & 68.17 & 69.30  \\
 \hline
 16 & 39.81 & 39.71 & 39.90 & 40.38 & 68.10 & 66.33 & 69.51 & 69.42  \\
 \hline
 \hline
 \multicolumn{5}{|c|}{Winogrande (FP32 Accuracy = 67.88\%)} & \multicolumn{4}{|c|}{Piqa (FP32 Accuracy = 78.78\%)} \\ 
 \hline
 \hline
 64 & 66.85 & 66.61 & 67.72 & 67.88 & 77.31 & 77.42 & 77.75 & 77.64 \\
 \hline
 32 & 67.25 & 67.72 & 67.72 & 67.00 & 77.31 & 77.04 & 77.80 & 77.37  \\
 \hline
 16 & 68.11 & 68.90 & 67.88 & 67.48 & 77.37 & 78.13 & 78.13 & 77.69  \\
 \hline
\end{tabular}
\caption{\label{tab:mmlu_abalation} Accuracy on LM evaluation harness tasks on GPT3-8B model.}
\end{table}

\begin{table} \centering
\begin{tabular}{|c||c|c|c|c||c|c|c|c|} 
\hline
 $L_b \rightarrow$& \multicolumn{4}{c||}{8} & \multicolumn{4}{c||}{8}\\
 \hline
 \backslashbox{$L_A$\kern-1em}{\kern-1em$N_c$} & 2 & 4 & 8 & 16 & 2 & 4 & 8 & 16  \\
 %$N_c \rightarrow$ & 2 & 4 & 8 & 16 & 2 & 4 & 2 \\
 \hline
 \hline
 \multicolumn{5}{|c|}{Race (FP32 Accuracy = 40.67\%)} & \multicolumn{4}{|c|}{Boolq (FP32 Accuracy = 76.54\%)} \\ 
 \hline
 \hline
 64 & 40.48 & 40.10 & 39.43 & 39.90 & 75.41 & 75.11 & 77.09 & 75.66 \\
 \hline
 32 & 39.52 & 39.52 & 40.77 & 39.62 & 76.02 & 76.02 & 75.96 & 75.35  \\
 \hline
 16 & 39.81 & 39.71 & 39.90 & 40.38 & 75.05 & 73.82 & 75.72 & 76.09  \\
 \hline
 \hline
 \multicolumn{5}{|c|}{Winogrande (FP32 Accuracy = 70.64\%)} & \multicolumn{4}{|c|}{Piqa (FP32 Accuracy = 79.16\%)} \\ 
 \hline
 \hline
 64 & 69.14 & 70.17 & 70.17 & 70.56 & 78.24 & 79.00 & 78.62 & 78.73 \\
 \hline
 32 & 70.96 & 69.69 & 71.27 & 69.30 & 78.56 & 79.49 & 79.16 & 78.89  \\
 \hline
 16 & 71.03 & 69.53 & 69.69 & 70.40 & 78.13 & 79.16 & 79.00 & 79.00  \\
 \hline
\end{tabular}
\caption{\label{tab:mmlu_abalation} Accuracy on LM evaluation harness tasks on GPT3-22B model.}
\end{table}

\begin{table} \centering
\begin{tabular}{|c||c|c|c|c||c|c|c|c|} 
\hline
 $L_b \rightarrow$& \multicolumn{4}{c||}{8} & \multicolumn{4}{c||}{8}\\
 \hline
 \backslashbox{$L_A$\kern-1em}{\kern-1em$N_c$} & 2 & 4 & 8 & 16 & 2 & 4 & 8 & 16  \\
 %$N_c \rightarrow$ & 2 & 4 & 8 & 16 & 2 & 4 & 2 \\
 \hline
 \hline
 \multicolumn{5}{|c|}{Race (FP32 Accuracy = 44.4\%)} & \multicolumn{4}{|c|}{Boolq (FP32 Accuracy = 79.29\%)} \\ 
 \hline
 \hline
 64 & 42.49 & 42.51 & 42.58 & 43.45 & 77.58 & 77.37 & 77.43 & 78.1 \\
 \hline
 32 & 43.35 & 42.49 & 43.64 & 43.73 & 77.86 & 75.32 & 77.28 & 77.86  \\
 \hline
 16 & 44.21 & 44.21 & 43.64 & 42.97 & 78.65 & 77 & 76.94 & 77.98  \\
 \hline
 \hline
 \multicolumn{5}{|c|}{Winogrande (FP32 Accuracy = 69.38\%)} & \multicolumn{4}{|c|}{Piqa (FP32 Accuracy = 78.07\%)} \\ 
 \hline
 \hline
 64 & 68.9 & 68.43 & 69.77 & 68.19 & 77.09 & 76.82 & 77.09 & 77.86 \\
 \hline
 32 & 69.38 & 68.51 & 68.82 & 68.90 & 78.07 & 76.71 & 78.07 & 77.86  \\
 \hline
 16 & 69.53 & 67.09 & 69.38 & 68.90 & 77.37 & 77.8 & 77.91 & 77.69  \\
 \hline
\end{tabular}
\caption{\label{tab:mmlu_abalation} Accuracy on LM evaluation harness tasks on Llama2-7B model.}
\end{table}

\begin{table} \centering
\begin{tabular}{|c||c|c|c|c||c|c|c|c|} 
\hline
 $L_b \rightarrow$& \multicolumn{4}{c||}{8} & \multicolumn{4}{c||}{8}\\
 \hline
 \backslashbox{$L_A$\kern-1em}{\kern-1em$N_c$} & 2 & 4 & 8 & 16 & 2 & 4 & 8 & 16  \\
 %$N_c \rightarrow$ & 2 & 4 & 8 & 16 & 2 & 4 & 2 \\
 \hline
 \hline
 \multicolumn{5}{|c|}{Race (FP32 Accuracy = 48.8\%)} & \multicolumn{4}{|c|}{Boolq (FP32 Accuracy = 85.23\%)} \\ 
 \hline
 \hline
 64 & 49.00 & 49.00 & 49.28 & 48.71 & 82.82 & 84.28 & 84.03 & 84.25 \\
 \hline
 32 & 49.57 & 48.52 & 48.33 & 49.28 & 83.85 & 84.46 & 84.31 & 84.93  \\
 \hline
 16 & 49.85 & 49.09 & 49.28 & 48.99 & 85.11 & 84.46 & 84.61 & 83.94  \\
 \hline
 \hline
 \multicolumn{5}{|c|}{Winogrande (FP32 Accuracy = 79.95\%)} & \multicolumn{4}{|c|}{Piqa (FP32 Accuracy = 81.56\%)} \\ 
 \hline
 \hline
 64 & 78.77 & 78.45 & 78.37 & 79.16 & 81.45 & 80.69 & 81.45 & 81.5 \\
 \hline
 32 & 78.45 & 79.01 & 78.69 & 80.66 & 81.56 & 80.58 & 81.18 & 81.34  \\
 \hline
 16 & 79.95 & 79.56 & 79.79 & 79.72 & 81.28 & 81.66 & 81.28 & 80.96  \\
 \hline
\end{tabular}
\caption{\label{tab:mmlu_abalation} Accuracy on LM evaluation harness tasks on Llama2-70B model.}
\end{table}

%\section{MSE Studies}
%\textcolor{red}{TODO}


\subsection{Number Formats and Quantization Method}
\label{subsec:numFormats_quantMethod}
\subsubsection{Integer Format}
An $n$-bit signed integer (INT) is typically represented with a 2s-complement format \citep{yao2022zeroquant,xiao2023smoothquant,dai2021vsq}, where the most significant bit denotes the sign.

\subsubsection{Floating Point Format}
An $n$-bit signed floating point (FP) number $x$ comprises of a 1-bit sign ($x_{\mathrm{sign}}$), $B_m$-bit mantissa ($x_{\mathrm{mant}}$) and $B_e$-bit exponent ($x_{\mathrm{exp}}$) such that $B_m+B_e=n-1$. The associated constant exponent bias ($E_{\mathrm{bias}}$) is computed as $(2^{{B_e}-1}-1)$. We denote this format as $E_{B_e}M_{B_m}$.  

\subsubsection{Quantization Scheme}
\label{subsec:quant_method}
A quantization scheme dictates how a given unquantized tensor is converted to its quantized representation. We consider FP formats for the purpose of illustration. Given an unquantized tensor $\bm{X}$ and an FP format $E_{B_e}M_{B_m}$, we first, we compute the quantization scale factor $s_X$ that maps the maximum absolute value of $\bm{X}$ to the maximum quantization level of the $E_{B_e}M_{B_m}$ format as follows:
\begin{align}
\label{eq:sf}
    s_X = \frac{\mathrm{max}(|\bm{X}|)}{\mathrm{max}(E_{B_e}M_{B_m})}
\end{align}
In the above equation, $|\cdot|$ denotes the absolute value function.

Next, we scale $\bm{X}$ by $s_X$ and quantize it to $\hat{\bm{X}}$ by rounding it to the nearest quantization level of $E_{B_e}M_{B_m}$ as:

\begin{align}
\label{eq:tensor_quant}
    \hat{\bm{X}} = \text{round-to-nearest}\left(\frac{\bm{X}}{s_X}, E_{B_e}M_{B_m}\right)
\end{align}

We perform dynamic max-scaled quantization \citep{wu2020integer}, where the scale factor $s$ for activations is dynamically computed during runtime.

\subsection{Vector Scaled Quantization}
\begin{wrapfigure}{r}{0.35\linewidth}
  \centering
  \includegraphics[width=\linewidth]{sections/figures/vsquant.jpg}
  \caption{\small Vectorwise decomposition for per-vector scaled quantization (VSQ \citep{dai2021vsq}).}
  \label{fig:vsquant}
\end{wrapfigure}
During VSQ \citep{dai2021vsq}, the operand tensors are decomposed into 1D vectors in a hardware friendly manner as shown in Figure \ref{fig:vsquant}. Since the decomposed tensors are used as operands in matrix multiplications during inference, it is beneficial to perform this decomposition along the reduction dimension of the multiplication. The vectorwise quantization is performed similar to tensorwise quantization described in Equations \ref{eq:sf} and \ref{eq:tensor_quant}, where a scale factor $s_v$ is required for each vector $\bm{v}$ that maps the maximum absolute value of that vector to the maximum quantization level. While smaller vector lengths can lead to larger accuracy gains, the associated memory and computational overheads due to the per-vector scale factors increases. To alleviate these overheads, VSQ \citep{dai2021vsq} proposed a second level quantization of the per-vector scale factors to unsigned integers, while MX \citep{rouhani2023shared} quantizes them to integer powers of 2 (denoted as $2^{INT}$).

\subsubsection{MX Format}
The MX format proposed in \citep{rouhani2023microscaling} introduces the concept of sub-block shifting. For every two scalar elements of $b$-bits each, there is a shared exponent bit. The value of this exponent bit is determined through an empirical analysis that targets minimizing quantization MSE. We note that the FP format $E_{1}M_{b}$ is strictly better than MX from an accuracy perspective since it allocates a dedicated exponent bit to each scalar as opposed to sharing it across two scalars. Therefore, we conservatively bound the accuracy of a $b+2$-bit signed MX format with that of a $E_{1}M_{b}$ format in our comparisons. For instance, we use E1M2 format as a proxy for MX4.

\begin{figure}
    \centering
    \includegraphics[width=1\linewidth]{sections//figures/BlockFormats.pdf}
    \caption{\small Comparing LO-BCQ to MX format.}
    \label{fig:block_formats}
\end{figure}

Figure \ref{fig:block_formats} compares our $4$-bit LO-BCQ block format to MX \citep{rouhani2023microscaling}. As shown, both LO-BCQ and MX decompose a given operand tensor into block arrays and each block array into blocks. Similar to MX, we find that per-block quantization ($L_b < L_A$) leads to better accuracy due to increased flexibility. While MX achieves this through per-block $1$-bit micro-scales, we associate a dedicated codebook to each block through a per-block codebook selector. Further, MX quantizes the per-block array scale-factor to E8M0 format without per-tensor scaling. In contrast during LO-BCQ, we find that per-tensor scaling combined with quantization of per-block array scale-factor to E4M3 format results in superior inference accuracy across models. 



\end{document}
