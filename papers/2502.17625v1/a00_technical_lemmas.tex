\begin{lemma}\label{lem:sqrt-bound-ineq}
    For any real numbers $a,b,c$, if $a>0, c>0$, then $a\leq b+\sqrt{a c}$ implies $a\leq 2b+c$.
\end{lemma}
\begin{proof}
    The function $g(t) = t-\sqrt{t}-\frac 12(t-1)$ defined on $\fR_{\geq 0}$ is convex and has a minimum of $0$ at $t=1$. This implies that
    \[
        a-\sqrt{ac}-\frac 12\rbr{a-c}=c \cdot g\rbr{\frac{a}{c}}\geq 0.
    \]
    Therefore, when $b\geq a-\sqrt{ac}$, we also have $2b\geq a-c$.
\end{proof}

\begin{lemma}
    \label{lem:sumszt-2}
    Let $a, b > 0$
    and suppose that $0\le z_t \le b$ holds for all $t = 1, 2, \ldots, T$.
    Also, assume that $\sum_{t=1}^T z_t^2\leq a$.
    We then have
    \begin{align}
        \sum_{t=1}^T \frac{z_t}{\sqrt{t}}
        \le
        f\rbr{a,b},
        \quad
        \mbox{where}
        \quad
        f(a, b) :=
            \min_{s \in [T]} \cbrm[\Big]{
            \sqrt{ a \log \frac{T}{s} } + 2 b \sqrt{s}
            }. \label{eq:sumszt-2}
    \end{align}
    Note that $f$ is a concave function.
\end{lemma}

\begin{proof}
    We split the sum into the first $n$ terms and the last $T-s$ terms:
    \begin{align}
        \sum_{t=1}^T \frac{z_t}{\sqrt{t}}
        =
        \sum_{t=1}^{s} \frac{z_t}{\sqrt{t}}
        +
        \sum_{t=s+1}^{T} \frac{z_t}{\sqrt{t}}.\label{eq:lem2-sum-two-parts}
    \end{align}
    The first part can be bounded with the fact that $z_t\leq b$:
    \begin{align*}
        \sum_{t=1}^{s} \frac{z_t}{\sqrt{t}}
        & \leq
        \sum_{t=1}^{s} \frac{b}{\sqrt{t}}\\
        & =
        b \sum_{t=1}^{s} \frac{1}{\sqrt{t}} \\
        & \leq
        b \cdot 2\sqrt{s}.\yestag\label{eq:lem2-sum-part-1}
    \end{align*}
    The other part can be bounded with the Cauchy-Schwarz inequality:
    \begin{align*}
        \sum_{t=s+1}^{T} \frac{z_t}{\sqrt{t}}
        & =
        \sum_{t=s+1}^{T} z_t\frac{1}{\sqrt{t}} \\
        & \leq
        \sqrt{\rbrm[\bigg]{\sum_{t=s+1}^{T} z_t^2}\rbrm[\bigg]{\sum_{t=s+1}^{T} \frac{1}{t}}} \\
        & \leq
        \sqrt{a \log{\frac{T}{s}}}.\yestag\label{eq:lem2-sum-part-2}
    \end{align*}
    Note that our $\log(\cdot)$ is of base 2. Our desired inequality can be obtained by plugging~\eqref{eq:lem2-sum-part-1} and \eqref{eq:lem2-sum-part-2} into \eqref{eq:lem2-sum-two-parts}.
\end{proof}

\begin{lemma}
    \label{lem:sqrtk-ratio}
    For $n\geq 2$ and an arbitrary sequence of nonnegative real numbers $x_1,\dots,x_n$, the following inequality holds:
    \begin{equation*}
        \max\cbr{x_1,\dots,x_n}\sum_{i=1}^n x_i \leq \rbrm[\Big]{\frac{1}{2}+\frac{1}{2}\sqrt{n}} \sum_{i=1}^n x_i^2.
    \end{equation*}
    Equality holds when all but one $x_i$ values are equal to $\frac{1}{\sqrt{n}+1}$ times the single outlier.
\end{lemma}
\begin{proof}
    Without loss of generality we assume that $x_1\geq x_2\geq \dots \geq x_n$. Let $\bar{x}=\frac{1}{n-1} \sum_{i=2}^n x_i$ be the average of the last $n-1$ numbers. Due to the strict convexity of the square function, we see from Jensen's inequality that
    \begin{equation}
        \sum_{i=1}^n x_i^2 = x_1^2 + \sum_{i=2}^n x_i^2 \geq x_1^2 + (n-1) \bar{x}^2,\label{eq:square-bounded-by-average}
    \end{equation}
    with equality when all $x_2, \dots, x_n$ are equal to $\bar{x}$. Consider the function $f(k)=\frac{1+(n-1)k}{1+(n-1)k^2}$ defined on $k\in [0, 1]$. Note that
    \begin{equation}
        f(k)
        =    \frac{1+(n-1)k}{1+(n-1)k^2}
        =    1+\frac{n-1}{\frac{1}{k}-1+\frac{n}{\frac{1}{k}-1}+2}
        \leq 1+\frac{n-1}{2\sqrt{n}+2}
        =    \frac12 \rbrm[\big]{\sqrt{n}+1}\label{eq:fk-bounded-by-sqrtn} 
    \end{equation}
    where the inequality is due to AM-GM, and is tight when $k=\frac{1}{\sqrt{n}+1}$.
    If we plug in $k=\frac{\bar{x}}{x_1}$, we get
    \begin{align*}
        1+(n-1)\frac{\bar{x}}{x_1}
        &\leq
        \frac12 \rbrm[\big]{\sqrt{n}+1} \rbrm[\Big]{1+(n-1)\rbrm[\big]{\frac{\bar{x}}{x_1}}^2}.\\
    \end{align*}
    We multiply both sides by $x_1^2$ and get
    \begin{align*}
        x_1(x_1+(n-1)\bar{x})
        &\leq
        \frac12 \rbrm[\big]{\sqrt{n}+1} \rbrm[\big]{x_1^2+(n-1)\bar{x}^2},\\
    \end{align*}
    which means that
    \begin{align*}
        x_1\sum_{i=1}^n x_i
        &\leq
        \frac12 \rbrm[\big]{\sqrt{n}+1} \rbrm[\big]{x_1^2+(n-1)\bar{x}^2}.\\
    \end{align*}
    Together with \eqref{eq:square-bounded-by-average} this completes the proof.
    Equality holds when both \eqref{eq:square-bounded-by-average} and \eqref{eq:fk-bounded-by-sqrtn} are tight, i.e., $x_2=\dots=x_n=\bar{x}=\frac{1}{\sqrt{n}+1}x_1$.
\end{proof}
