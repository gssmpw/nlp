%%%%%%%%%%%%
% ABSTRACT %
%%%%%%%%%%%%

% \begin{abstract}
% Ensuring the reliability and safety of automated decision-making is crucial. 
% It is well-known that data distribution shifts in machine learning can produce unreliable outcomes. 
% This paper proposes a new approach for measuring the reliability of predictions under distribution shifts. 
% We analyze how the outputs of a trained neural network change using clustering to measure distances between outputs and class centroids. 
% We propose this distance as a metric to evaluate the confidence of predictions under distribution shifts. 
% We assign each prediction to a cluster with centroid representing the mean softmax output for all correct predictions of a given class. 
% We then define a safety threshold for a class as the smallest distance from an incorrect prediction to the given class centroid. 
% We evaluate the approach on the MNIST and CIFAR-10 datasets using a Convolutional Neural Network and a Vision Transformer, respectively. 
% The results show that our approach is consistent across these data sets and network models, and indicate that the proposed metric can offer an efficient way of determining when automated predictions are acceptable and when they should be deferred to human operators given a distribution shift. %The approach is being evaluated using the CARLA simulator in the context of self-driving decision making under noisy conditions as when sunshine and rain at the same time will distort the input.
% \end{abstract}

% Abstract without mentions to distribution shift, which is the current state of the paper
\begin{abstract}
Ensuring the reliability and safety of automated decision-making is crucial.
This paper proposes a new approach for measuring the reliability of predictions in machine learning models.
We analyze how the outputs of a trained neural network change using clustering to measure distances between outputs and class centroids.
We propose this distance as a metric to evaluate the confidence of predictions.
We assign each prediction to a cluster with centroid representing the mean softmax output for all correct predictions of a given class.
We then define a safety threshold for a class as the smallest distance from an incorrect prediction to the given class centroid.
We evaluate the approach on the MNIST and CIFAR-10 datasets using a Convolutional Neural Network and a Vision Transformer, respectively.
The results show that our approach is consistent across these data sets and network models, and indicate that the proposed metric can offer an efficient way of determining when automated predictions are acceptable and when they should be deferred to human operators.    
\end{abstract}