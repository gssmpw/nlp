%%%%%%%%%%%%%%%
% CONCLUSIONS %
%%%%%%%%%%%%%%%

\section{Conclusions and Future Work}

We presented a simple approach for quantifying the reliability of neural networks, validating it across distinct network architectures and datasets, with results presented for a representative scenario. Future work will extend this approach to autonomous systems, using simulators for OOD detection.

% We presented a simple approach to thresholding accuracy in an image classification setting, by training two distinct network architectures, on two distinct datasets, and showing our results are consistent across both scenarios.
% We demonstrated the steps required to effectively create class clusters by initialising centroids with the mean of correct class predictions for each distinct class, where the K-Means algorithm quickly converges with good cluster separation.
% A threshold was proposed to ensure predictions below threshold are safe, at a small cost of labelling as incorrect a small number of correct predictions at or above threshold.

% We posit that the optimal threshold is domain-dependent and should be determined by domain experts who are equipped to balance the efficiency of automated decision-making against the necessity for time-intensive human judgment, taking into account the specific stakes involved.

% We are applying the concepts discussed in this study to autonomous system safety, in the context of self-driving cars using the CARLA simulator, and constituent tasks in the decision-making stack, such as image classification and segmentation, where our methodology aims to enhance system safety by identifying scenarios, including OOD scenarios, where human judgment is preferable to the autonomous system’s assessment.

%our approach may help make such systems safer, by deferring to human judgement what may not be safely assessed by the autonomous system.

% Finally, we are considering the softmax output as a training dataset in itself, for a simple multi-layer perceptron binary classifier, to compare and contrast the softmax distance approach, and a regressor network combined with our current approach, to provide a quantity e.g. sigmoid function output as a weight to further study optimal threshold values.