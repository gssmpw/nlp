\section{Related Works}
In budget-constrained auctions, 
it is well known that no mechanism can achieve a constant approximation to the optimal SW while satisfying IC and IR. 
____ pioneered the use of PO as an alternative efficiency goal. 
____ characterized the the space of mechanisms that satisfies these desirable properties 
in the market with two bidders and heterogeneous goods. 
LW ____ is now widely recognized as a standard efficiency objective for budget-constrained auctions (e.g., ____).
For clinching auctions, ____ showed that the adaptive clinching auction achieves a 1/2-approximation to the optimal LW and recent studies ____ have extended this result to polymatroidal settings. Furthermore,  ____ and ____ presented stronger efficiency guarantees for clinching auctions in symmetric settings. The monotonicity of the LW was not addressed in these studies.

____ initiated the study of auctions in social networks, proposing a mechanism that achieves higher revenue than the VCG mechanism____ extended to this setting. For efficiency guarantees, ____ showed that any efficient mechanism satisfying IC and IR results in a seller's deficit, necessitating a weaker notion of efficiency. Although these studies focused on the case of a single unit, the frameworks have been applied to more general settings, including multi-unit auctions____, budget-constrained auctions____, and double auctions____. For a more detailed overview of this research trend, see ____. Also, 
the arrival of new bidders has been a central topic in online and dynamic market settings, 
see ____ for more details.

Revenue monotonicity (e.g., ____) is a well-studied property.  
It ensures that revenue increases monotonically in response to changes in inputs or other factors.  
Among various notions, bidder revenue monotonicity is the most relevant for auctions in social networks,  
as ____ considered this property in this context.  
For the VCG mechanism, ____ and ____ have characterized  
the conditions under which this property holds.  
To the best of our knowledge, this property has not been considered for clinching auctions.






\paragraph{Organization of this paper.}
The remainder of the paper is organized as follows.  
Section~2 presents our model.  
Section~3 introduces the adaptive clinching auction and provides some useful properties under the symmetric setting.  
Section~4 investigates monotonicity under the addition of a new bidder.  
Section~5 extends the results of Section~4 to a setting where bidders arrive online.  
Section~6 discusses the challenges of the asymmetric setting.  
Section~7 provides some concluding remarks.  
All omitted proofs are given in the appendix.



\paragraph{Notation.}
Let $\mathbf R_+$ denote the set of non-negative real numbers. 
We often denote a singleton $\{i\}$ by $i$ and a set $\{1,2,\ldots,k\}$ by $[k]$. For a vector $x\in \mathbf R^N_+$, we often denote $x(i)$ by $x_i$, and write as $x= (x_i)_{i\in N}$. We also define $x_{-i}$ by $x_{-i}:=(x_j)_{j\in N\setminus i}$.