\section{Experiment}
\label{sec:Experiment}

In accordance with existing multimodal out-of-context misinformation detection methods \cite{abdelnabi2022open, qi2024sniffer}, we conduct a multitude of experiments to demonstrate the effectiveness of the proposed E2LVLM on the public benchmark dataset NewsCLIPpings~\cite{luo2021newsclippings}. Typically, we focus on the six evaluation questions as follows:
\begin{enumerate}[label=\textbf{Q\arabic*:}]
    \item  How does E2LVLM perform in the task of multimodal OOC misinformation detection?
    \item  How does each procedure contribute to the E2LVLM's performance in detection?
    \item  Does E2LVLM provide accurate detections and compelling rationales for their judgments?
    \item  How impact are different sizes of the base LVLM on E2LVLM in detection?
    \item  Can E2LVLM be rapidly deployed at the stage of early detection?
    \item  How does E2LVLM perform on the other dataset?
\end{enumerate}

\subsection{Experimental Setup}

\textbf{Dataset.} We evaluate the efficacy of the proposed E2LVLM on the dataset NewsCLIPpings~\cite{luo2021newsclippings}. This dataset serves as the largest real-world multimodal misinformation detection benchmark. We follow the standard protocol \cite{abdelnabi2022open, yuan2023support, qi2024sniffer}, and report experimental results on the Merged/Balance subset. This subset consists of 71,072 training, 7,024 validation, and 7,264 testing, respectively.

\noindent \textbf{Compared Baselines.} To make a comprehensive performance evaluation, we compare the proposed E2LVLM with a series of representative methods. (1) A line of research focuses on attached classifiers trained from scratch, including SAFE~\cite{massarelli2019safe} and EANN~\cite{wang2018eann}. (2) Another line of research underlines the use of pre-trained models, containing VisualBERT~\cite{li2019visualbert}, CLIP~\cite{radford2021learning}, Neu-Sym detector~\cite{zhang2023detecting}, DT-Transformer~\cite{papadopoulos2023synthetic}, CCN~\cite{abdelnabi2022open}, SEN~\cite{yuan2023support}, and ECENet~\cite{zhang2023ecenet}. (3) Furthermore, in the era of LVLMs, SNIFFER~\cite{qi2024sniffer} is the first attempt to adopt a multimodal large language model for addressing the OOC task. More details of these methods can be provided in their official papers. 

\noindent \textbf{Evaluation Metrics.} We regard the multimodal OOC misinformation detection issue as a binary classification task. Following the standard process~\cite{abdelnabi2022open}, the accuracy over all samples (All), the accuracy over the OOC (Falsified), and not OOC (Pristine) are reported as the metrics during evaluation for a fair comparison.

\noindent \textbf{Implementation Details.} We choose Qwen2-VL-7B~\cite{wang2024qwen2} as the base LVLM, unless otherwise specified. We implement E2LVLM on PyTorch~\cite{paszke2019pytorch} version 2.3.1 with CUDA 12.2, and train it for 2 epochs on 4 NVIDIA GeForce RTX 3090 GPUs with 24G of memory. We adopt FlashAttention - 2~\cite{dao2024flashattention} for efficient training on the visual encoder and large language model. We use a batch size of 8 and a learning rate of $2\times 10^{-4}$. The models are optimized using AdamW~\cite{loshchilov2017decoupled} optimizer with a linear warmup and a cosine learning rate scheduler. Additionally, all experimental results are the average of three runs with no hyper-parameter searching.

% \clearpage
\subsection{RQ1: Comparison with SOTA Methods}

%-------------main results
% \begin{table}[t]
% \newcommand{\hlinebold}{\noalign{\hrule height 0.2mm}}
% \centering
% \caption{Performance for accuracy compared to existing OOC methods on the benchmark dataset NewsCLIPpings~\cite{luo2021newsclippings}. The best performances are indicated in \textbf{bold}.}
% \resizebox{\columnwidth}{!}{%
% \begin{tabular}{l|c|ccc}
% % \toprule
% Methods & Venue & \textbf{All} & \textbf{Falsified} & \textbf{Pristine}  \\
% \hlinebold
% SAFE~\cite{massarelli2019safe}  & PAKDD20 & 52.8 & 54.8 & 52.0  \\
% EANN~\cite{wang2018eann}  & SIGKDD18 & 58.1 & 61.8 & 56.2  \\
% \hline
% VisualBERT~\cite{li2019visualbert}  & arXiv19 & 58.6 & 38.9 & 78.4 \\
% CLIP~\cite{radford2021learning} & ICML21  & 66.0 & 64.3 & 67.7  \\
% Neu-Sym detector~\cite{zhang2023detecting} & arXiv23 & 68.2 & - & - \\
% DT-Transformer~\cite{papadopoulos2023synthetic} & MAD23 & 77.1 & 78.6 & 75.6  \\
% CCN~\cite{abdelnabi2022open} & CVPR22 & 84.7 & 84.8 & 84.5  \\
% SEN~\cite{yuan2023support} & EMNLP23 & 87.1 & 85.5 & 88.6  \\
% ECENet~\cite{zhang2023ecenet} & MM23 & 87.7 & - & - \\
% \hline
% SNIFFER~\cite{qi2024sniffer} & CVPR24 & 88.4 & 86.9 & \textbf{91.8}  \\
% \rowcolor{lightgreen} E2LVLM (\textit{Ours}) & & \textbf{89.9} & \textbf{90.3} & 89.4  \\
% % \hlinebold
% % \bottomrule
% \end{tabular}%
% }
% \label{tab:tab_1}
% \end{table}


% \begin{table}[t]
% \newcommand{\hlinebold}{\noalign{\hrule height 0.2mm}}
% \centering
% \small
% \caption{Performance for accuracy compared to existing OOC methods on the benchmark dataset NewsCLIPpings~\cite{luo2021newsclippings}. The best performances are indicated in \textbf{bold}.}
% \begin{adjustbox}{valign=c,max width=\columnwidth}
% \begin{tabular}{l|c|ccc}
% Methods & Venue & \textbf{All} & \textbf{Falsified} & \textbf{Pristine}  \\
% \hlinebold
% SAFE~\cite{massarelli2019safe}  & PAKDD20 & 52.8 & 54.8 & 52.0  \\
% EANN~\cite{wang2018eann}  & SIGKDD18 & 58.1 & 61.8 & 56.2  \\
% \hline
% VisualBERT~\cite{li2019visualbert}  & arXiv19 & 58.6 & 38.9 & 78.4 \\
% CLIP~\cite{radford2021learning} & ICML21  & 66.0 & 64.3 & 67.7  \\
% Neu-Sym detector~\cite{zhang2023detecting} & arXiv23 & 68.2 & - & - \\
% DT-Transformer~\cite{papadopoulos2023synthetic} & MAD23 & 77.1 & 78.6 & 75.6  \\
% CCN~\cite{abdelnabi2022open} & CVPR22 & 84.7 & 84.8 & 84.5  \\
% SEN~\cite{yuan2023support} & EMNLP23 & 87.1 & 85.5 & 88.6  \\
% ECENet~\cite{zhang2023ecenet} & MM23 & 87.7 & - & - \\
% \hline
% SNIFFER~\cite{qi2024sniffer} & CVPR24 & 88.4 & 86.9 & \textbf{91.8}  \\
% \rowcolor{lightgreen} E2LVLM (\textit{Ours}) & & \textbf{89.9} & \textbf{90.3} & 89.4  \\
% \end{tabular}%
% \end{adjustbox}
% \label{tab:tab_1}
% \end{table}

\Cref{tab:tab_1} presents the detailed comparison of E2LVLM with existing OOC methods on NewsCLIPpings. We use ``-'' for partial methods that do not release source codes or results. As shown in these results, we summarize the following findings: (1) As for the accuracy over ``All'', E2LVLM outperforms all methods by a large margin. Even for the state-of-the-art (SNIFFER), E2LVLM still outperforms it by around 1.5\% accuracy. (2) E2LVLM owns strong discriminatory powers, with an improvement of around 3.4\% over ``Falsified'', increasing the SOTA from 86.9\% to 90.3\%. (3) E2LVLM has a trade-off between ``Falsified'' and ``Pristine''. As reported by CCN~\cite{abdelnabi2022open}, a professional OOC misinformation detector should accurately identify both ``Falsified'' and ``Pristine'' samples. E2LVLM provides a closer distance between them, compared with the SOTA. This confirms that the improvement in the E2LVLM's performance on ``Falsified'' does not come at the cost of its performance on ``Pristine'', or vice versa. (4) With the increasing of architectural complexity, the performance of OOC misinformation detectors has been significantly enhanced, which is consistent with previous statements. In the context of LVLMs-based OOC methods, E2LVLM with Qwen2-VL-7B~\cite{wang2024qwen2} is superior to SNIFFER that depends on GPT-4~\cite{achiam2023gpt} and Vicuna-13B~\cite{vicuna2023}. These findings suggest the superiority of the proposed method E2LVLM on the OOC detection.


\begin{table}[t]
    \caption{Performance for accuracy compared to existing methods on NewsCLIPpings~\cite{luo2021newsclippings}. The best results are indicated in \textbf{bold}.}
      \centering
      \begin{adjustbox}{valign=c, max width=\columnwidth}
      \begin{tabular}{l|c|ccc}
        \toprule
        Methods & Venue & \textbf{All} & \textbf{Falsified} & \textbf{Pristine}  \\
        \hline
        SAFE~\cite{massarelli2019safe}  & PAKDD20 & 52.8 & 54.8 & 52.0  \\
        EANN~\cite{wang2018eann}  & SIGKDD18 & 58.1 & 61.8 & 56.2  \\
        \hline
        VisualBERT~\cite{li2019visualbert}  & arXiv19 & 58.6 & 38.9 & 78.4 \\
        CLIP~\cite{radford2021learning} & ICML21  & 66.0 & 64.3 & 67.7  \\
        Neu-Sym detector~\cite{zhang2023detecting} & arXiv23 & 68.2 & - & - \\
        DT-Transformer~\cite{papadopoulos2023synthetic} & MAD23 & 77.1 & 78.6 & 75.6  \\
        CCN~\cite{abdelnabi2022open} & CVPR22 & 84.7 & 84.8 & 84.5  \\
        SEN~\cite{yuan2023support} & EMNLP23 & 87.1 & 85.5 & 88.6  \\
        ECENet~\cite{zhang2023ecenet} & MM23 & 87.7 & - & - \\
        \hline
        SNIFFER~\cite{qi2024sniffer} & CVPR24 & 88.4 & 86.9 & \textbf{91.8}  \\
        \rowcolor{lightgreen} E2LVLM (\textit{Ours}) & & \textbf{89.9} & \textbf{90.3} & 89.4  \\
        \bottomrule
  \end{tabular}
  \end{adjustbox}
  \label{tab:tab_1}
\end{table}



% %------------------main results
% \begin{table}[t]
% \newcommand{\hlinebold}{\noalign{\hrule height 0.2mm}}
% \centering
% \small
% \caption{Performance for accuracy compared to existing OOC methods on the benchmark dataset NewsCLIPpings~\cite{luo2021newsclippings}. The best performances are indicated in \textbf{bold}.}
% \begin{adjustbox}{valign=c,max width=\columnwidth}
% \begin{tabular}{l|c|ccc}
% \toprule
% Methods & Venue & \textbf{All} & \textbf{Falsified} & \textbf{Pristine}  \\
% \hlinebold
% SAFE~\cite{massarelli2019safe}  & PAKDD20 & 52.8 & 54.8 & 52.0  \\
% EANN~\cite{wang2018eann}  & SIGKDD18 & 58.1 & 61.8 & 56.2  \\
% \hline
% VisualBERT~\cite{li2019visualbert}  & arXiv19 & 58.6 & 38.9 & 78.4 \\
% CLIP~\cite{radford2021learning} & ICML21  & 66.0 & 64.3 & 67.7  \\
% Neu-Sym detector~\cite{zhang2023detecting} & arXiv23 & 68.2 & - & - \\
% DT-Transformer~\cite{papadopoulos2023synthetic} & MAD23 & 77.1 & 78.6 & 75.6  \\
% CCN~\cite{abdelnabi2022open} & CVPR22 & 84.7 & 84.8 & 84.5  \\
% SEN~\cite{yuan2023support} & EMNLP23 & 87.1 & 85.5 & 88.6  \\
% ECENet~\cite{zhang2023ecenet} & MM23 & 87.7 & - & - \\
% \hline
% SNIFFER~\cite{qi2024sniffer} & CVPR24 & 88.4 & 86.9 & \textbf{91.8}  \\
% \rowcolor{lightgreen} E2LVLM (\textit{Ours}) & & \textbf{89.9} & \textbf{90.3} & 89.4  \\
% \end{tabular}%
% \end{adjustbox}
% \label{tab:tab_1}
% \end{table}
% \clearpage
\subsection{RQ2: Ablation Study}

\begin{table}[t]
\caption{Ablation experiments for the E2LVLM. Evaluated on the NewsCLIPpings~\cite{luo2021newsclippings}. ``\#Evid.'' and ``\#Expla.'' represent the use of textual evidence and the supervised signal with explanations.}
\centering
\setlength{\tabcolsep}{1.5pt} % 调整列间距,减小为4pt
\renewcommand{\arraystretch}{1.2} % 调整行高(默认为1)
    \begin{adjustbox}{valign=c,max width=\columnwidth}
        \begin{tabular}{cccccc|ccc}
            \toprule
            Qwen2-VL & \#Evid. & Rerank & Rewrite& \#Expla. & Tuning & \textbf{All} & \textbf{Falsified} & \textbf{Pristine} \\
            \hline
            \cmark & \xmark & \xmark & \xmark & \xmark &\xmark & 69.1 & 54.4 & 83.9 \\
            \cmark & \cmark & \xmark & \xmark & \xmark &\xmark & 76.7 &	68.0 &	85.3 \\
            \hline
            \cmark & \xmark & \xmark & \xmark & \xmark &\cmark   & 78.9 & 73.8 & 84.2 \\
            \cmark & \cmark & \xmark & \xmark & \xmark &\cmark   & 83.0 & 77.1 & 88.9 \\
            \cmark & \cmark & \cmark & \xmark & \xmark &\cmark   & 87.7 & 86.5 & 88.7  \\
            \cmark & \cmark & \cmark & \cmark & \xmark &\cmark   & 88.5 & 87.7 & 89.1  \\
            \rowcolor{lightgreen} \cmark & \cmark & \cmark & \cmark &  \cmark &\cmark  & \textbf{89.9} & \textbf{90.3} & \textbf{89.4} \\
            \bottomrule
        \end{tabular}
    \end{adjustbox}
\label{tab:tab_2}
\end{table}









We conduct ablation experiments to analyze the effectiveness of primary procedures in E2LVLM. The results are shown in \Cref{tab:tab_2}, we can obtain the following findings:

\begin{itemize}
    \item We evaluate the impact of the use of textual evidence in zero-shot scenarios. As shown in the first two rows of \Cref{tab:tab_2}, the introduction of evidence achieves significant gains in performance, especially over ``Falsified''. Typically, the raw Qwen2-VL~\cite{wang2024qwen2} provides an accuracy of 69.1$\%$ over ``All'', which is higher than random guessing. It is attributed to the multimodal understanding capabilities of LVLMs. However, the base model struggles to discern falsified information, which necessitates robust methods to handle this challenge. Introducing textual evidence into the base model provides a performance improvement of 13.6$\%$ over the OOC. Besides, the accuracy on ``All'' reaches 76.7$\%$ (a 7.6$\%$ improvement in performance). This suggests the importance of the use of textual evidence in LVLMs for the OOC detection.
    \item In the second study, as shown in the remainder of \Cref{tab:tab_2}, each procedure contributes to the performance of E2LVLM. Typically, the task-specific fine-tuning technology extends LVLMs to the OOC, leading to promising results compared with zero-shot scenarios. Next, we conduct the textual evidence reranking strategy on E2LVLM, focusing on the salient item for debunking OOC misinformation. This results in an accuracy of 87.7$\%$ on ``All'', and shows a 9.4$\%$ improvement on ``Falsified'', while avoiding performance degradation on ``Pristine''. The reason is that such strategy appropriately eliminates the noise in the retrieved evidence. Further, we enforce the textual evidence rewriting strategy to generate coherent and contextually attuned content for better understanding, leading to an accuracy of 88.5$\%$ on ``All''. This suggests the importance of the reranking and rewriting of the retrieved textual evidence in E2LVLM for the OOC detection.
    \item Additionally, to understand the importance of the model’s explainability, we introduce the supervised signal with both judgment and explanation to the training phase, as shown in the last row of \Cref{tab:tab_2}. This results in an accuracy of 90.3$\%$ over ``Falsified'', higher than others. This is caused by the fact that such supervised signal enhances the model's discriminatory powers, thereby making the model provide accurate detections and attach them with compelling rationales for judgments. These results indicate that the supervised signal with both judgment and explanation contributes to the performance of E2LVLM.
\end{itemize}

% \newpage




\subsection{RQ3: Detection and Beyond}

%---------------------------------
% \begin{figure}[t]
%   \centering
%     \fbox{\rule{0pt}{1.8in} \rule{.8\linewidth}{0pt}}
%    \caption{Reranking performance of E2LVLM.  }
%    \label{fig:4}
% \end{figure}

\begin{figure}[t]
  \centering
   \includegraphics[width=.9\linewidth]{fig4.jpg}
   \caption{Comparison of E2LVLM with various reranking ways.}
   \label{fig:4}
\end{figure}

% \begin{figure}[t]
%   \centering
%    \includegraphics[width=1\linewidth]{sec/fig5.jpg}
%    \caption{Visualization of various data distributions.}
%    \label{fig:5}
% \end{figure}


Considering authentic images in the OOC realm, we further evaluate the effectiveness of the textual evidence reranking and rewriting in E2LVLM for detection and beyond. In subsequent experiments, we uniformly adopt the accuracy over ``All'' samples for comparison, unless otherwise specified.

\noindent \textbf{Reranking Analysis.} Upon analysis experiments above, we have observed a significant performance of E2LVLM in the top-1 textual evidence. To understand the impact of this design, we start an investigation of E2LVLM's predictions, as shown in \Cref{fig:4}. We add other reranking ways for comparison, \ie, cosine similarity and random choice. For $k$, we set the range of it as $\{1,2,3\}$. As shown in these results, we can note that the change trends of accuracy are generally consistent across distinct ways, presenting degradations as the growth of textual evidence. This is caused by the fact that not all textual evidence is effective. This necessitates the textual evidence reranking to prevent the introduction of irrelevant items. Further, these results suggest that E2LVLM lies in the multimodal understanding and reranking capabilities of LVLMs, serving as a professional OOC misinformation detector that incorporates internal and external information for revealing misinformation.

\begin{figure}[t]
  \centering
   \includegraphics[width=1\linewidth]{fig5.jpg}
   \caption{Visualization of various data distributions.}
   \label{fig:5}
\end{figure}

\noindent \textbf{Explainability Analysis.} To illustrate the E2LVLM's explainability, we conduct a qualitative analysis around data distributions to comprehend the decision-making phase, as shown in \Cref{fig:5}. From the results, the following findings are drawn: 1) As for an image-claim pair with ``Falsified'' in subfigure (a) of \Cref{fig:5}, the authentic image and its rewritten content are far from the falsified claim. Simultaneously, the rewritten content is closer to the authentic image, which supports it and refutes its claim. This makes the proposed E2LVLM provide an accurate decision. 2) As for an image-claim pair with ``Pristine'' in subfigure (b) of \Cref{fig:5}, the representation distributions of the authentic image, rewritten content, and claim are close together. This promotes E2LVLM to accurately discern this sample. These cases indicate that E2LVLM can effectively provide both judgment and explanation for debunking OOC misinformation.

\subsection{RQ4: Discussion of Different LVLMs}




% \begin{table}[t]
% \caption{A multimodal OOC misinformation detectors in distinct settings. Evaluated on NewsCLIPpings~\cite{luo2021newsclippings}. ``\#Params.'' refers to the parameters of Large Language Models in LVLMs.}
% \centering
% % \setlength{\tabcolsep}{4pt} % 调整列间距,减小为4pt
% \renewcommand{\arraystretch}{1.1} % 调整行高(默认为1)
%     \begin{adjustbox}{valign=c,max width=\columnwidth}
%         \begin{tabular}{lc|c|ccc}
%             \toprule
%             Settings & Tuning& \#Params. & \textbf{All} & \textbf{Falsified} & \textbf{Pristine} \\
%             \hline
%             Random & \xmark & -  & 50.3	& 49.2 & 51.5 \\
%             Qwen2-VL-2B & \xmark & 1.5B  & 65.7 & 88.6 & 42.9 \\
%             Qwen2-VL-7B & \xmark & 7.6B  &  79.1 & 75.7  & 82.5  \\
%             Qwen2-VL-72B & \xmark & 72B  & 79.8 & 87.2 & 72.5  \\
%             \hline
%             Qwen2-VL-2B & \cmark & 1.5B  & 81.4	& 80.2 & 82.6 \\
%             \rowcolor{lightgreen} Qwen2-VL-7B & \cmark & 7.6B  & 89.9	& 90.3 &89.4 \\
%             \bottomrule
%         \end{tabular}
%     \end{adjustbox}
% \label{tab:tab_4}
% \end{table}


% \begin{table}[t]
% \caption{A multimodal OOC misinformation detectors with distinct LVLMs. Evaluated on the NewsCLIPpings~\cite{luo2021newsclippings} dataset.}
% \centering
% \small
% % \setlength{\tabcolsep}{0.5pt} % 调整列间距,减小为4pt
% % \renewcommand{\arraystretch}{1.1} % 调整行高(默认为1)
%     \begin{adjustbox}{valign=c,max width=\columnwidth}
%         \begin{tabular}{lc|c|c}
%             \toprule
%             Settings & Tuning & Parameters & \textbf{All} \\
%             \hline
%             Random & \xmark & -  & 50.3	\\
%             Qwen2-VL-2B & \xmark & 1.5B  & 65.7 \\
%             Qwen2-VL-7B & \xmark & 7.6B  &  79.1 \\
%             Qwen2-VL-72B & \xmark & 72B  & 79.8 \\
%             \hline
%             Qwen2-VL-2B & \cmark & 1.5B  & 81.4	\\
%             \rowcolor{lightgreen} Qwen2-VL-7B & \cmark & 7.6B  & \textbf{89.9} \\
%             \bottomrule
%         \end{tabular}
%     \end{adjustbox}
% \label{tab:tab_4}
% \end{table}


% \begin{table}[t]
% \caption{Performance comparison on distinct LVLMs between instructional data construction and model tuning. Evaluated on the NewsCLIPpings~\cite{luo2021newsclippings} dataset.}
% \centering
% \small
% % \setlength{\tabcolsep}{5.5pt} % 调整列间距,减小为4pt
% % \renewcommand{\arraystretch}{1.1} % 调整行高(默认为1)
%     \begin{adjustbox}{valign=c,max width=\columnwidth}
%         \begin{tabular}{lcl|c}
%         \toprule
%          Data Construction &  \ding{223} & Model  Tuning & \textbf{All} \\
%         \hline
%         Qwen2-VL-2B & \ding{223} & Qwen2-VL-2B   & 81.1 \\
%         Qwen2-VL-2B & \ding{223} & Qwen2-VL-7B   & 87.6 \\
%         \hline
%         Qwen2-VL-7B & \ding{223} & Qwen2-VL-2B   & 81.4 \\
%         \rowcolor{lightgreen} Qwen2-VL-7B & \ding{223} & Qwen2-VL-7B   & \textbf{89.9} \\
%         \bottomrule
%         \end{tabular}
%     \end{adjustbox}
% \label{tab:tab_5}
% \end{table}

To analyze the impact of distinct LVLMs on E2LVLM in the OOC detection, we conduct analysis experiments around Qwen2-VL family~\cite{wang2024qwen2} in \Cref{tab:tab_4} and \Cref{tab:tab_5}. As depicted in these results, we achieve the following observations:



% The former is to evaluate the impact of the chosen base LVLM during training, and the latter is to measure the mutual synergy between instructional data construction and model tuning. 

(1) As shown in \Cref{tab:tab_4}, we employ different LVLMs for evidence-enhanced fine-tuning, apart from random guessing. The increasing of model's parameters provides better performance, which is consistent with previous research. Although the LVLM model Qwen2-VL-72B outperforms others (\eg, Qwen2-VL-7B) in a zero-shot scenario, it achieves a performance improvement of 0.7$\%$ with introducing around 10 times parameters. This ignores the compromise between computation burden and detection accuracy. Furthermore, the introduction of instruction tuning extends general-purpose LVLMs to the task of OOC, leading to significant improvements in performance (as shown in the fifth and last rows of this table).

% \newpage
(2) As depicted in \Cref{tab:tab_5}, we use different LVLMs at two stages, \ie, instructional data construction and model tuning. As for the former, we can observe that although LVLMs on a larger scale provide higher performance in detection accuracy, their improvements are finite (\eg, a performance improvement of 2.3$\%$ even in Qwen2-VL-7B). As for the latter, larger-scale LVLMs provide significant improvements in performance, leading to performance improvements of 6.5$\%$ and 8.5$\%$, respectively. This indicates the importance of the adopted LVLM in E2LVLM.
% \newpage
% \clearpage
% \section{The Proposed Method: GDDSG}\label{sec4}
\begin{figure*}
    \centering
    \includegraphics[width=\linewidth]{figures/framework.pdf}
    \caption{Illustration of The Overall Framework. [best view in color]}
    \label{fig: framework}
\end{figure*}
\textbf{Overview.} \autoref{fig: framework} provides an overview of our proposed method. 
Using task \( t \) as an example, we begin by projecting all training samples into an embedding space utilizing a pre-trained backbone. In this space, we compute the centroids for each class. Next, we evaluate whether a new centroid \( \mathbf{c}_i \) should be integrated into an existing class group \( G_j \).
If \( \mathbf{c}_i \) is dissimilar to all classes within \( G_j \), it is added to the group. If it is similar to any class in an existing group, it remains unassigned.
For unassigned centroids, we construct new similarity graphs (SimGraphs) based on their pairwise similarities. We then apply graph coloring theory to these SimGraphs, forming new class groups by clustering dissimilar categories together.
Finally, we update the NCM-based classifier with all class groups, facilitating efficient model updates with minimal computational overhead.

\subsection{Class Grouping Based on Similarity}

\autoref{Corollary: cor} provides guidance for constructing a sequence of dissimilar tasks. A key idea is to dynamically assign each new class to a group during CIL, ensuring that the similarity between the new class and other classes within the group is minimized. This approach helps maintain the robustness of each group's incremental learning process to the order of tasks. For each group, a separate adapter can be trained, and the results from different adapters can be merged during prediction to enhance the model's overall performance. 

In a given CIL task sequence, we organize the classes into several groups. The group list is denoted as \( G = [G_1, \dots, G_k] \), where each \( G_i \) represents a distinct group of classes. For a specified task \( t \) and each class \( C \in CLS^t \), our objective is to assign class \( C \) to an optimal group \( G^* \), ensuring that the new class is dissimilar to all existing classes in that group.

To achieve this objective, we first define the similarity between classes.
The similarity between any two classes, \( CLS_i \) and \( CLS_j \), is determined using an adaptive similarity threshold \( \eta_{i,j} \).
This threshold is computed based on the mean distance between the training samples of each class and their respective centroids in a learned embedding space, as shown below:

\begin{align}
    \eta_{i,j} = \max [
    & \frac{\sum_{k = 1}^{|X^t|} \mathbb{I}(y^t_k = i) \, d(h(x_k^t), \mathbf{c_i}) }{\sum_{k = 1}^{|X^t|} \mathbb{I}(y^t_k = i)}, \nonumber \\
    & \frac{\sum_{k = 1}^{|X^t|} \mathbb{I}(y^t_k = j) \, d(h(x_k^t), \mathbf{c_j}) }{\sum_{k = 1}^{|X^t|} \mathbb{I}(y^t_k = j)} 
    ],
\end{align}
where \( \mathbf{x}^{(t)} \) denotes the t-th task instance, \( h(\cdot) \) is the feature extraction function defined in Equation \autoref{eq: feature}, \( d: \mathcal{X} \times \mathcal{X} \to \mathbb{R}^+ \) specifies the distance metric space, \( \mathbb{I}(\cdot) \) represents the characteristic function, and the class centroid \( \mathbf{c}_i \in \mathbb{R}^d \) is computed as \( \mathbf{c}_i = \frac{1}{|C_i|} \sum_{x_j \in C_i} \mathbf{x}_j \).


Building upon this framework, we define the condition under which two classes, \( CLS_i \) and \( CLS_j \), are considered dissimilar. Specifically, they are deemed dissimilar if the following condition holds:

\begin{equation}
    d(\mathbf{c_i}, \mathbf{c_j}) > \eta_{i,j}.
    \label{eq: sim}
\end{equation}

Thus, class \( C \) is assigned to group \( G^* \) only if it is dissimilar to all classes within \( G^* \), and \( G^* \) is the choice with the lowest average similarity:

\begin{equation}
    G^* = \arg\min_{G} \frac{1}{|G|} \sum_{C' \in G} d(C, C').
\end{equation}

This approach is consistent with the principles outlined in \autoref{Corollary: cor} and ensures the robustness of the model across the entire task sequence.


\subsection{Graph-Driven Class Grouping}

Graph algorithms provide an efficient method for dynamically grouping classes while minimizing intra-group similarity.
In a graph-theoretic framework, classes are represented as nodes, with edge weights quantifying the similarity between them.
The flexibility and analytical power of graph structures allow for dynamic adjustment of class assignments in CIL, facilitating optimal grouping in polynomial time.
This approach significantly enhances the model's robustness and adaptability in incremental learning tasks.

Therefore, we can leverage the similarity between classes to construct a SimGraph, defined as follows:
\begin{definition} \textbf{(SimGraph.)}
A SimGraph can be defined as an undirect graph $SimG = (V, E)$, where $V$ is the set of nodes that represent each class's centroid and $E$ is the set of edges connecting pair of nodes that represent classes that are determined as similar by \autoref{eq: sim}.
\label{SimGraph}
\end{definition}

Then, we aim to partition the vertex set of this graph into subsets, with each subset forming a maximal subgraph with no edges between vertices. This problem can be abstracted as the classic NP-hard combinatorial optimization problem of finding a minimum coloring of the graphs. Let $G^{-1}(\cdot)$ be an assignment of class group identities to each vertex of a graph such that no edge connects two identically labeled vertices (i.e. $G^{-1}(i) \neq G^{-1}(j)$ for all $(i,j) \in E$). We can formulate the minimum coloring for graph $SimG$ as follows:
\begin{equation}
    \mathcal{X}(SimG) = \min | \{ G^{-1}(k) | k \in V\} |, 
    \label{eq: graph}
\end{equation}
where $\mathcal{X}(SimG)$ is called the chromatic number of $SimG$ and $|\cdot|$ denotes the size of the set.

Brooks' theorem \cite{brooks1941colouring} offers an upper bound for the graph coloring problem. To apply this in our context, we must demonstrate that the similarity graphs constructed in CIL meet the conditions required by Brooks' theorem. By doing so, we can establish that the problem is solvable and that the solution converges, ensuring the effectiveness of our grouping and class coloring process in class incremental learning. Without loss of generality, we can make the following assumptions:

\begin{assumption} In the CIL task, class \( C_i \) is randomly sampled without replacement from the set \( \mathcal{U} = \bigcup_{i=1}^{\infty} C_i \), ensuring that \( C_i \neq C_j \) for all \( i \neq j \). The probability that any two classes \( C_i \) and \( C_j \) within the set \( \mathcal{U} \) meet the similarity condition (as described in \autoref{eq: sim}) is denoted by \( p \).
\end{assumption}

In the CIL scenario with \( N \) classes, the probability of forming an odd cycle is given by \(\left( p^2(1-p)^{(N-2)} \right)^N = p^{2N}(1-p)^{N^2-2N}\). Similarly, the probability of forming a complete graph is \(p^{\binom{N}{2}} = p^{\frac{1}{2}N(N-1)}\).
Thus, the probability that the CIL scenario satisfies Brooks' theorem can be expressed as:
\begin{equation}
    P_{\text{Satisfy Brooks}'} = 1 - p^{2N}(1-p)^{N^2-2N} - p^{\frac{1}{2}N(N-1)}.
\end{equation}
\begin{figure}[t]
    \centering
    \includegraphics[width=\linewidth]{figures/contour_plot.pdf}
    \caption{Contour plot delineating the subthreshold region where \( P_{\text{Satisfy Brooks}'} < 0.99 \). The horizontal axis spans \( p \in [0.9, 1.0] \), representing probability values, while the vertical axis specifies sample sizes \( N \in [10, 40] \). In regions not displayed, the corresponding \( P_{\text{Satisfy Brooks}'} \) values exceed 0.99.}
    \label{fig: probability}
\end{figure}

\autoref{fig: probability} illustrates the various values of \( N \) and \( p \) that satisfy Brooks' theorem with a probability of less than 0.99. Our findings indicate that when \( N > 35 \), the CIL scenario adheres to Brooks' theorem. Furthermore, even with fewer classes, as long as \( p \) does not exceed 0.9, the CIL scenario can still ensure that the similarity graph complies with Brooks' theorem at a confidence level of 0.99. We conclude that class grouping based on the similarity graph is convergent and can be solved efficiently in polynomial time.

For \autoref{eq: graph}, while no algorithm exists that can compute \(\mathcal{X}(SimG)\) in polynomial time for all cases, efficient algorithms have been developed that can handle most problems involving small to medium-sized graphs, particularly the similarity graph \(SimG\) discussed here. In practical scenarios, such graphs are typically sparse. Notably, in conjunction with the above analysis, the similarity graph \(SimG\) in the CIL scenario satisfies the non-odd cycle assumption in Brooks' theorem \cite{brooks1941colouring}. For non-complete similarity graphs \(SimG\), we have \(\mathcal{X}(SimG) \le \Delta(SimG)\), where \(\Delta(SimG)\) represents the maximum vertex degree in \(SimG\).

Therefore, we can apply a simple yet effective greedy method, the Welsh-Powell graph coloring algorithm \cite{welsh1967upper}. This algorithm first sorts all nodes in the graph in descending order based on their degree and then assigns a color to each node, prioritizing those with higher degrees. During the coloring process, the algorithm selects the minimum available color for each node that differs from its neighbors, creating new color classes when necessary. The time complexity of this algorithm is \( O(|V|^2) \), primarily due to the color conflict check between each node and its neighbors. In theory, the maximum number of groupings produced by this algorithm is \( \max_{i = 1}^n \min\{ \deg(v_i') + 1, i \} \), with an error margin of no more than 1, where \( V' \) is the sequence of nodes sorted by degree, derived from \( V \).

\subsection{Overall Process}

\noindent \textbf{Training Pipeline.}
Building upon the theoretical foundations in Section 3.1, we now formalize the complete training procedure. Our framework leverages a frozen pre-trained feature extractor $\phi(\cdot)$, augmented with trainable random projections $W \in \mathbb{R}^{L \times M}$ where $M \gg L$, to enhance representation capacity. For each input $x_i^t$ from class group $s$, we compute its expanded feature:
\begin{equation}
\label{eq: feature}
h(x_i^t) = g(\phi(x_i^t) W) \in \mathbb{R}^M,
\end{equation}
where $g(\cdot)$ denotes the nonlinear activation.

The core learning paradigm reframes classification as regularized least-squares regression. Let $H_s^t \in \mathbb{R}^{N_s^t \times M}$ be the feature matrix and $Y_s^t \in \mathbb{R}^{N_s^t \times L_s^t}$ denote the one-hot label matrix for class group $s$. We optimize the projection matrix $\Theta_s^t \in \mathbb{R}^{M \times L_s^t}$ through:
\begin{equation}
\label{eq: loss}
\min_{\Theta} \|Y_s^t - H_s^t \Theta_s^t\|_F^2 + \lambda \|\Theta_s^t\|_F^2,
\end{equation}
where $\lambda$ controls regularization strength. The closed-form solution is:
\begin{equation}
\label{eq: analytic}
\Theta_s^t = ( {H_s^t}^\top H_s^t + \lambda I )^{-1} {H_s^t}^\top Y_s^t.
\end{equation}

For incremental updates, we maintain two key components: the Gram matrix $Gram_s^t$ capturing feature correlations, and the prototype matrix $C_s^t$ encoding class centroids. When new task $t$ arrives with $N_s^t$ samples:
\begin{equation}
    \label{eq: gram}
    Gram_{s}^t = Gram_{s}^{t-1} + \sum_{n = 1}^{N_{s}^t} h(x^t_i)^\top h(x^t_i),
\end{equation}
\begin{equation}
    \label{eq: pro}
    C_{s}^t = \begin{bmatrix}C_{s}^{t-1} & \underbrace{\mathbf{0}_M \ \mathbf{0}_M \ \ldots \ \mathbf{0}_M}_{(L_{s}^t - L_s^{t-1})\text{ times}} \end{bmatrix} + \sum_{n = 1}^{N_s^t} h(x^t_i)^\top y(x^t_i).
\end{equation}

The regularization parameter $\lambda$ is adaptively selected from a candidate pool $\Lambda$ through cross-validation on a held-out calibration set, minimizing the empirical risk:
\begin{equation}
\lambda^* = \arg \min_{\lambda \in \Lambda} \|Y_{\text{val}} - H_{\text{val}} (Gram_{\text{val}} + \lambda I)^{-1} C_{\text{val}} \|_F^2.
\end{equation}

Additionally, group descriptors are constructed through prototype similarity analysis. For each training instance $(x, y) \in \mathcal{D}^t$, we generate meta-features dataset as:
\begin{equation}
\mathcal{D}_g = \left\{ \left( \rho(x),\ G^{-1}(y) \right) \right\}_{(x,y)\in \mathcal{D}^t},
\end{equation}
where $\rho(x) = \big[ d(h(x),\mathbf{c}_1), \ldots, d(h(x),\mathbf{c}_k) \big]^\top$ denotes the concatenated distance vector measuring similarity between the sample embedding and prototype vectors.

\noindent \textbf{Inference Pipeline.}
Given test sample $x^*$, its group identification can be learned via $\hat{g} = \mathcal{M}_g(\rho(x^*))$, where $\mathcal{M}_g$ is the class group predict model trained with $ \mathcal{D}_g$.
Then, the prediction will be performed within the selected group via $\hat{y} = \underset{c \in \mathcal{C}_{\hat{g}}}{\arg\max}\ ( g(\phi(x^*) W) (Gram_{\hat{g}} + \lambda I)^{-1} C_{\hat{g}}[:,c]$.
        
\iffalse
In the previous section, we introduced the motivation and core concepts behind the proposed algorithm. In this section, we will describe the entire training process in detail. Recent years have seen CIL methods based on pre-trained models achieve remarkable results \cite{panos2023first,zhou2023revisiting,zhou2023revisiting,mcdonnell2024ranpac}, largely due to their robust representation capabilities. Since our proposed class grouping method also relies heavily on the model's representation ability, we utilize a widely-adopted pre-trained model as a feature extractor. For each class group, we train independent classification heads, which enhances the model’s adaptability and generalization to different class groups.

As outlined above, we utilize a frozen random projection matrix \( W \in \mathbb{R}^{L \times M} \) to enhance features across all class groups, where \( L \) is the output dimension of the pre-trained model and \( M \gg L \) is the expanded dimensionality. Given a task \( t \) and a sample \( x^t_i \) belonging to a class group \( s \), the feature vector of the sample is denoted as \( h(x^t_i) \), and its one-hot encoded label as \( y(x^t_i) \). Specifically,

\begin{equation}
    \label{eq: feature}
    h(x^t_i) = g(\phi(x)^T W),
\end{equation}
where \( \phi(\cdot) \) represents the feature extractor, and \( g(\cdot) \) is a nonlinear activation function.
We define \( H_{s}^t \in \mathbb{R}^{N_{s}^t \times M} \) as the matrix containing feature vectors of \( N_{s}^t \) samples from group \( s \). The corresponding Gram matrix is defined as:
\begin{equation}
    \label{eq: grammatrix}
    Gram_{s}^t = {H_{s}^t}^T H_{s}^t \in \mathbb{R}^{M \times M}.
\end{equation}
Additionally, the matrix \( C_{s}^t \) consists of the concatenated column vectors of all classes within group \( s \), with dimensions \( M \times L_{s}^t \), where \( L_s^t \) represents the number of classes in group \( s \) for task \( t \). When a new task arrives, the model applies the GDDSG algorithm to assign new classes to their respective groups. The Gram matrix \( Gram \) and matrix \( C \) for each group are updated according to the following formulas:
\begin{equation}
    \label{eq: gram}
    Gram_{s}^t = Gram_{s}^{t-1} + \sum_{n = 1}^{N_{s}^t} h(x^t_i) \otimes h(x^t_i),
\end{equation}
\begin{equation}
    \label{eq: pro}
    C_{s}^t = \begin{bmatrix}C_{s}^{t-1} & \underbrace{\mathbf{0}_M \ \mathbf{0}_M \ \ldots \ \mathbf{0}_M}_{(L_{s}^t - L_s^{t-1})\text{ times}} \end{bmatrix} + \sum_{n = 1}^{N_s^t} h(x^t_i) \otimes y(x^t_i),
\end{equation}
where \( \mathbf{0}_M \) denotes a zero vector with \( M \) dimensions.

During the test phase, we combine the classification heads of all groups \( G = [G_1, G_2, \dots, G_k] \) to make a joint prediction for a given sample \( x \). For each class \( c' \) in a group, the score is computed as follows:
\begin{equation}
    \label{eq: predict}
    s_{c'} = g(\phi(x)^T W)(Gram_{i} + \lambda I)^{-1} C_{c'},
\end{equation}
where \( i = 1,\dots,k \) denotes the indices of each groups, and \( \lambda \) is the regularization parameter used to ensure that the \( Gram \) matrix remains invertible. The final classification result is then obtained by applying the following formula:
\begin{equation}
    \label{eq: predict_joint}
    \hat{c} = \mathop{\arg\max}\limits_{c' \in \cup_{i = 1}^k CLS^{G_i}} s_{c'},
\end{equation}
where \( \cup_{i = 1}^k CLS^{G_i} \) represents the set of possible classes across all class groups.
\fi

% \newpage
\subsection{RQ5: Data Scaling Exploration}

\begin{table}[t]
\caption{A multimodal OOC misinformation detectors with distinct LVLMs. Evaluated on the NewsCLIPpings~\cite{luo2021newsclippings} dataset.}
\centering
\small
% \setlength{\tabcolsep}{0.5pt} % 调整列间距,减小为4pt
% \renewcommand{\arraystretch}{1.1} % 调整行高(默认为1)
    \begin{adjustbox}{valign=c,max width=\columnwidth}
        \begin{tabular}{lc|c|c}
            \toprule
            Settings & Tuning & Parameters & \textbf{All} \\
            \hline
            Random & \xmark & -  & 50.3	\\
            Qwen2-VL-2B & \xmark & 1.5B  & 65.7 \\
            Qwen2-VL-7B & \xmark & 7.6B  &  79.1 \\
            Qwen2-VL-72B & \xmark & 72B  & 79.8 \\
            \hline
            Qwen2-VL-2B & \cmark & 1.5B  & 81.4	\\
            \rowcolor{lightgreen} Qwen2-VL-7B & \cmark & 7.6B  & \textbf{89.9} \\
            \bottomrule
        \end{tabular}
    \end{adjustbox}
\label{tab:tab_4}
\end{table}


\begin{table}[t]
\caption{Comparison on distinct LVLMs between data construction and model tuning. Evaluated on NewsCLIPpings~\cite{luo2021newsclippings}.}
\centering
\small
% \setlength{\tabcolsep}{5.5pt} % 调整列间距,减小为4pt
% \renewcommand{\arraystretch}{1.1} % 调整行高(默认为1)
    \begin{adjustbox}{valign=c,max width=\columnwidth}
        \begin{tabular}{lcl|c}
        \toprule
         Data Construction &  \ding{223} & Model  Tuning & \textbf{All} \\
        \hline
        Qwen2-VL-2B & \ding{223} & Qwen2-VL-2B   & 81.1 \\
        Qwen2-VL-2B & \ding{223} & Qwen2-VL-7B   & 87.6 \\
        \hline
        Qwen2-VL-7B & \ding{223} & Qwen2-VL-2B   & 81.4 \\
        \rowcolor{lightgreen} Qwen2-VL-7B & \ding{223} & Qwen2-VL-7B   & \textbf{89.9} \\
        \bottomrule
        \end{tabular}
    \end{adjustbox}
\label{tab:tab_5}
\end{table}


% \begin{figure}[t]
%   \centering
%    \includegraphics[width=.9\linewidth]{sec/fig6.jpg}
%    \caption{Performance of E2LVM on NewsCLIPpings~\cite{luo2021newsclippings} using different training data proportions.}
%    \label{fig:6}
% \end{figure}

% \newpage
To evaluate the feasibility of rapid deployment of E2LVLM at the stage of early detection, we randomly choose 10$\%$, 25$\%$, 50$\%$, and 75$\%$ samples for experiment analysis. As shown in \Cref{fig:6}, E2LVLM achieves remarkable performance on different proportions of training data. Typically, in a zero-shot scenario, E2LVLM has outperformed the previous method (\eg, DT-Transformer~\cite{papadopoulos2023synthetic}), suggesting the effectiveness of E2LVLM in the task of OOC. With the increasing of samples, the discriminatory powers of E2LVLM have a significant rising tendency. This demonstrates that the proposed E2LVLM can be rapidly deployed and ensure detection accuracy, even at the stage of early detection.

\subsection{RQ6: Robustness Analysis}

% \begin{figure}[t]
%   \centering
%    \includegraphics[width=.9\linewidth]{sec/fig6.jpg}
%    \caption{Performance of E2LVM on NewsCLIPpings~\cite{luo2021newsclippings} using different training data proportions.}
%    \label{fig:6}
% \end{figure}


% \begin{table}[t]
% \caption{Comparison of ``True vs OOC'' accuracy on the dataset VERITE~\cite{papadopoulos2024verite}. Trained on NewsCLIPpings~\cite{luo2021newsclippings} and evaluated on ``All''. The partial results are reported from SNIFFER~\cite{qi2024sniffer}.}
% \centering
% % \Huge
% % \setlength{\tabcolsep}{10pt} % 调整列间距,减小为4pt
% \renewcommand{\arraystretch}{1.1} % 调整行高(默认为1)
%     \begin{adjustbox}{valign=c,max width=\columnwidth}
%         \begin{tabular}{l|c|ll}
%             \toprule
%             Methods & Venue & \textbf{All} & \textbf{True vs OOC} \\
%             \hline
%             RED-DOT~\cite{papadopoulos2023red} & arXiv23 & 84.5 & 73.9 \\
%             SNIFFER~\cite{qi2024sniffer} & CVPR24 & 88.4 ($\uparrow$ 3.9) & 74.0 ($\uparrow$ 0.1)  \\
%             \rowcolor{lightgreen} E2LVLM &  & \textbf{89.9 ($\uparrow$ 5.4)} & \textbf{74.4 ($\uparrow$ 0.5)}  \\
%             \bottomrule
%         \end{tabular}
%     \end{adjustbox}
% \label{tab:tab_6}
% \end{table}

To further verify the robustness of E2LVLM towards real-world scenarios, we implement extended experiments on VERITE~\cite{papadopoulos2024verite}. This dataset serves as a novel real-world OOC misinformation, including 1,000 annotated samples stemming from fact-checking websites. Following~\cite{qi2024sniffer}, we also compare E2LVLM with RED-DOT~\cite{papadopoulos2023red} in \Cref{tab:tab_6}. From the results, we observe that E2LVLM shows the best performance on large-scale NewsCLIPpings~\cite{luo2021newsclippings} and more challenging VERITE~\cite{papadopoulos2024verite}. Typically, E2LVLM achieves a ``True vs OOC'' accuracy of 74.4$\%$, which outperforms RED-DOT by 0.5$\%$ accuracy, and the SOTA SNIFFER by 0.4$\%$ accuracy. The evaluations on these two multimodal OOC misinformation datasets exhibit the detection accuracy and robustness of E2LVLM in practical scenarios.

% \section{The Proposed Method: GDDSG}\label{sec4}
\begin{figure*}
    \centering
    \includegraphics[width=\linewidth]{figures/framework.pdf}
    \caption{Illustration of The Overall Framework. [best view in color]}
    \label{fig: framework}
\end{figure*}
\textbf{Overview.} \autoref{fig: framework} provides an overview of our proposed method. 
Using task \( t \) as an example, we begin by projecting all training samples into an embedding space utilizing a pre-trained backbone. In this space, we compute the centroids for each class. Next, we evaluate whether a new centroid \( \mathbf{c}_i \) should be integrated into an existing class group \( G_j \).
If \( \mathbf{c}_i \) is dissimilar to all classes within \( G_j \), it is added to the group. If it is similar to any class in an existing group, it remains unassigned.
For unassigned centroids, we construct new similarity graphs (SimGraphs) based on their pairwise similarities. We then apply graph coloring theory to these SimGraphs, forming new class groups by clustering dissimilar categories together.
Finally, we update the NCM-based classifier with all class groups, facilitating efficient model updates with minimal computational overhead.

\subsection{Class Grouping Based on Similarity}

\autoref{Corollary: cor} provides guidance for constructing a sequence of dissimilar tasks. A key idea is to dynamically assign each new class to a group during CIL, ensuring that the similarity between the new class and other classes within the group is minimized. This approach helps maintain the robustness of each group's incremental learning process to the order of tasks. For each group, a separate adapter can be trained, and the results from different adapters can be merged during prediction to enhance the model's overall performance. 

In a given CIL task sequence, we organize the classes into several groups. The group list is denoted as \( G = [G_1, \dots, G_k] \), where each \( G_i \) represents a distinct group of classes. For a specified task \( t \) and each class \( C \in CLS^t \), our objective is to assign class \( C \) to an optimal group \( G^* \), ensuring that the new class is dissimilar to all existing classes in that group.

To achieve this objective, we first define the similarity between classes.
The similarity between any two classes, \( CLS_i \) and \( CLS_j \), is determined using an adaptive similarity threshold \( \eta_{i,j} \).
This threshold is computed based on the mean distance between the training samples of each class and their respective centroids in a learned embedding space, as shown below:

\begin{align}
    \eta_{i,j} = \max [
    & \frac{\sum_{k = 1}^{|X^t|} \mathbb{I}(y^t_k = i) \, d(h(x_k^t), \mathbf{c_i}) }{\sum_{k = 1}^{|X^t|} \mathbb{I}(y^t_k = i)}, \nonumber \\
    & \frac{\sum_{k = 1}^{|X^t|} \mathbb{I}(y^t_k = j) \, d(h(x_k^t), \mathbf{c_j}) }{\sum_{k = 1}^{|X^t|} \mathbb{I}(y^t_k = j)} 
    ],
\end{align}
where \( \mathbf{x}^{(t)} \) denotes the t-th task instance, \( h(\cdot) \) is the feature extraction function defined in Equation \autoref{eq: feature}, \( d: \mathcal{X} \times \mathcal{X} \to \mathbb{R}^+ \) specifies the distance metric space, \( \mathbb{I}(\cdot) \) represents the characteristic function, and the class centroid \( \mathbf{c}_i \in \mathbb{R}^d \) is computed as \( \mathbf{c}_i = \frac{1}{|C_i|} \sum_{x_j \in C_i} \mathbf{x}_j \).


Building upon this framework, we define the condition under which two classes, \( CLS_i \) and \( CLS_j \), are considered dissimilar. Specifically, they are deemed dissimilar if the following condition holds:

\begin{equation}
    d(\mathbf{c_i}, \mathbf{c_j}) > \eta_{i,j}.
    \label{eq: sim}
\end{equation}

Thus, class \( C \) is assigned to group \( G^* \) only if it is dissimilar to all classes within \( G^* \), and \( G^* \) is the choice with the lowest average similarity:

\begin{equation}
    G^* = \arg\min_{G} \frac{1}{|G|} \sum_{C' \in G} d(C, C').
\end{equation}

This approach is consistent with the principles outlined in \autoref{Corollary: cor} and ensures the robustness of the model across the entire task sequence.


\subsection{Graph-Driven Class Grouping}

Graph algorithms provide an efficient method for dynamically grouping classes while minimizing intra-group similarity.
In a graph-theoretic framework, classes are represented as nodes, with edge weights quantifying the similarity between them.
The flexibility and analytical power of graph structures allow for dynamic adjustment of class assignments in CIL, facilitating optimal grouping in polynomial time.
This approach significantly enhances the model's robustness and adaptability in incremental learning tasks.

Therefore, we can leverage the similarity between classes to construct a SimGraph, defined as follows:
\begin{definition} \textbf{(SimGraph.)}
A SimGraph can be defined as an undirect graph $SimG = (V, E)$, where $V$ is the set of nodes that represent each class's centroid and $E$ is the set of edges connecting pair of nodes that represent classes that are determined as similar by \autoref{eq: sim}.
\label{SimGraph}
\end{definition}

Then, we aim to partition the vertex set of this graph into subsets, with each subset forming a maximal subgraph with no edges between vertices. This problem can be abstracted as the classic NP-hard combinatorial optimization problem of finding a minimum coloring of the graphs. Let $G^{-1}(\cdot)$ be an assignment of class group identities to each vertex of a graph such that no edge connects two identically labeled vertices (i.e. $G^{-1}(i) \neq G^{-1}(j)$ for all $(i,j) \in E$). We can formulate the minimum coloring for graph $SimG$ as follows:
\begin{equation}
    \mathcal{X}(SimG) = \min | \{ G^{-1}(k) | k \in V\} |, 
    \label{eq: graph}
\end{equation}
where $\mathcal{X}(SimG)$ is called the chromatic number of $SimG$ and $|\cdot|$ denotes the size of the set.

Brooks' theorem \cite{brooks1941colouring} offers an upper bound for the graph coloring problem. To apply this in our context, we must demonstrate that the similarity graphs constructed in CIL meet the conditions required by Brooks' theorem. By doing so, we can establish that the problem is solvable and that the solution converges, ensuring the effectiveness of our grouping and class coloring process in class incremental learning. Without loss of generality, we can make the following assumptions:

\begin{assumption} In the CIL task, class \( C_i \) is randomly sampled without replacement from the set \( \mathcal{U} = \bigcup_{i=1}^{\infty} C_i \), ensuring that \( C_i \neq C_j \) for all \( i \neq j \). The probability that any two classes \( C_i \) and \( C_j \) within the set \( \mathcal{U} \) meet the similarity condition (as described in \autoref{eq: sim}) is denoted by \( p \).
\end{assumption}

In the CIL scenario with \( N \) classes, the probability of forming an odd cycle is given by \(\left( p^2(1-p)^{(N-2)} \right)^N = p^{2N}(1-p)^{N^2-2N}\). Similarly, the probability of forming a complete graph is \(p^{\binom{N}{2}} = p^{\frac{1}{2}N(N-1)}\).
Thus, the probability that the CIL scenario satisfies Brooks' theorem can be expressed as:
\begin{equation}
    P_{\text{Satisfy Brooks}'} = 1 - p^{2N}(1-p)^{N^2-2N} - p^{\frac{1}{2}N(N-1)}.
\end{equation}
\begin{figure}[t]
    \centering
    \includegraphics[width=\linewidth]{figures/contour_plot.pdf}
    \caption{Contour plot delineating the subthreshold region where \( P_{\text{Satisfy Brooks}'} < 0.99 \). The horizontal axis spans \( p \in [0.9, 1.0] \), representing probability values, while the vertical axis specifies sample sizes \( N \in [10, 40] \). In regions not displayed, the corresponding \( P_{\text{Satisfy Brooks}'} \) values exceed 0.99.}
    \label{fig: probability}
\end{figure}

\autoref{fig: probability} illustrates the various values of \( N \) and \( p \) that satisfy Brooks' theorem with a probability of less than 0.99. Our findings indicate that when \( N > 35 \), the CIL scenario adheres to Brooks' theorem. Furthermore, even with fewer classes, as long as \( p \) does not exceed 0.9, the CIL scenario can still ensure that the similarity graph complies with Brooks' theorem at a confidence level of 0.99. We conclude that class grouping based on the similarity graph is convergent and can be solved efficiently in polynomial time.

For \autoref{eq: graph}, while no algorithm exists that can compute \(\mathcal{X}(SimG)\) in polynomial time for all cases, efficient algorithms have been developed that can handle most problems involving small to medium-sized graphs, particularly the similarity graph \(SimG\) discussed here. In practical scenarios, such graphs are typically sparse. Notably, in conjunction with the above analysis, the similarity graph \(SimG\) in the CIL scenario satisfies the non-odd cycle assumption in Brooks' theorem \cite{brooks1941colouring}. For non-complete similarity graphs \(SimG\), we have \(\mathcal{X}(SimG) \le \Delta(SimG)\), where \(\Delta(SimG)\) represents the maximum vertex degree in \(SimG\).

Therefore, we can apply a simple yet effective greedy method, the Welsh-Powell graph coloring algorithm \cite{welsh1967upper}. This algorithm first sorts all nodes in the graph in descending order based on their degree and then assigns a color to each node, prioritizing those with higher degrees. During the coloring process, the algorithm selects the minimum available color for each node that differs from its neighbors, creating new color classes when necessary. The time complexity of this algorithm is \( O(|V|^2) \), primarily due to the color conflict check between each node and its neighbors. In theory, the maximum number of groupings produced by this algorithm is \( \max_{i = 1}^n \min\{ \deg(v_i') + 1, i \} \), with an error margin of no more than 1, where \( V' \) is the sequence of nodes sorted by degree, derived from \( V \).

\subsection{Overall Process}

\noindent \textbf{Training Pipeline.}
Building upon the theoretical foundations in Section 3.1, we now formalize the complete training procedure. Our framework leverages a frozen pre-trained feature extractor $\phi(\cdot)$, augmented with trainable random projections $W \in \mathbb{R}^{L \times M}$ where $M \gg L$, to enhance representation capacity. For each input $x_i^t$ from class group $s$, we compute its expanded feature:
\begin{equation}
\label{eq: feature}
h(x_i^t) = g(\phi(x_i^t) W) \in \mathbb{R}^M,
\end{equation}
where $g(\cdot)$ denotes the nonlinear activation.

The core learning paradigm reframes classification as regularized least-squares regression. Let $H_s^t \in \mathbb{R}^{N_s^t \times M}$ be the feature matrix and $Y_s^t \in \mathbb{R}^{N_s^t \times L_s^t}$ denote the one-hot label matrix for class group $s$. We optimize the projection matrix $\Theta_s^t \in \mathbb{R}^{M \times L_s^t}$ through:
\begin{equation}
\label{eq: loss}
\min_{\Theta} \|Y_s^t - H_s^t \Theta_s^t\|_F^2 + \lambda \|\Theta_s^t\|_F^2,
\end{equation}
where $\lambda$ controls regularization strength. The closed-form solution is:
\begin{equation}
\label{eq: analytic}
\Theta_s^t = ( {H_s^t}^\top H_s^t + \lambda I )^{-1} {H_s^t}^\top Y_s^t.
\end{equation}

For incremental updates, we maintain two key components: the Gram matrix $Gram_s^t$ capturing feature correlations, and the prototype matrix $C_s^t$ encoding class centroids. When new task $t$ arrives with $N_s^t$ samples:
\begin{equation}
    \label{eq: gram}
    Gram_{s}^t = Gram_{s}^{t-1} + \sum_{n = 1}^{N_{s}^t} h(x^t_i)^\top h(x^t_i),
\end{equation}
\begin{equation}
    \label{eq: pro}
    C_{s}^t = \begin{bmatrix}C_{s}^{t-1} & \underbrace{\mathbf{0}_M \ \mathbf{0}_M \ \ldots \ \mathbf{0}_M}_{(L_{s}^t - L_s^{t-1})\text{ times}} \end{bmatrix} + \sum_{n = 1}^{N_s^t} h(x^t_i)^\top y(x^t_i).
\end{equation}

The regularization parameter $\lambda$ is adaptively selected from a candidate pool $\Lambda$ through cross-validation on a held-out calibration set, minimizing the empirical risk:
\begin{equation}
\lambda^* = \arg \min_{\lambda \in \Lambda} \|Y_{\text{val}} - H_{\text{val}} (Gram_{\text{val}} + \lambda I)^{-1} C_{\text{val}} \|_F^2.
\end{equation}

Additionally, group descriptors are constructed through prototype similarity analysis. For each training instance $(x, y) \in \mathcal{D}^t$, we generate meta-features dataset as:
\begin{equation}
\mathcal{D}_g = \left\{ \left( \rho(x),\ G^{-1}(y) \right) \right\}_{(x,y)\in \mathcal{D}^t},
\end{equation}
where $\rho(x) = \big[ d(h(x),\mathbf{c}_1), \ldots, d(h(x),\mathbf{c}_k) \big]^\top$ denotes the concatenated distance vector measuring similarity between the sample embedding and prototype vectors.

\noindent \textbf{Inference Pipeline.}
Given test sample $x^*$, its group identification can be learned via $\hat{g} = \mathcal{M}_g(\rho(x^*))$, where $\mathcal{M}_g$ is the class group predict model trained with $ \mathcal{D}_g$.
Then, the prediction will be performed within the selected group via $\hat{y} = \underset{c \in \mathcal{C}_{\hat{g}}}{\arg\max}\ ( g(\phi(x^*) W) (Gram_{\hat{g}} + \lambda I)^{-1} C_{\hat{g}}[:,c]$.
        
\iffalse
In the previous section, we introduced the motivation and core concepts behind the proposed algorithm. In this section, we will describe the entire training process in detail. Recent years have seen CIL methods based on pre-trained models achieve remarkable results \cite{panos2023first,zhou2023revisiting,zhou2023revisiting,mcdonnell2024ranpac}, largely due to their robust representation capabilities. Since our proposed class grouping method also relies heavily on the model's representation ability, we utilize a widely-adopted pre-trained model as a feature extractor. For each class group, we train independent classification heads, which enhances the model’s adaptability and generalization to different class groups.

As outlined above, we utilize a frozen random projection matrix \( W \in \mathbb{R}^{L \times M} \) to enhance features across all class groups, where \( L \) is the output dimension of the pre-trained model and \( M \gg L \) is the expanded dimensionality. Given a task \( t \) and a sample \( x^t_i \) belonging to a class group \( s \), the feature vector of the sample is denoted as \( h(x^t_i) \), and its one-hot encoded label as \( y(x^t_i) \). Specifically,

\begin{equation}
    \label{eq: feature}
    h(x^t_i) = g(\phi(x)^T W),
\end{equation}
where \( \phi(\cdot) \) represents the feature extractor, and \( g(\cdot) \) is a nonlinear activation function.
We define \( H_{s}^t \in \mathbb{R}^{N_{s}^t \times M} \) as the matrix containing feature vectors of \( N_{s}^t \) samples from group \( s \). The corresponding Gram matrix is defined as:
\begin{equation}
    \label{eq: grammatrix}
    Gram_{s}^t = {H_{s}^t}^T H_{s}^t \in \mathbb{R}^{M \times M}.
\end{equation}
Additionally, the matrix \( C_{s}^t \) consists of the concatenated column vectors of all classes within group \( s \), with dimensions \( M \times L_{s}^t \), where \( L_s^t \) represents the number of classes in group \( s \) for task \( t \). When a new task arrives, the model applies the GDDSG algorithm to assign new classes to their respective groups. The Gram matrix \( Gram \) and matrix \( C \) for each group are updated according to the following formulas:
\begin{equation}
    \label{eq: gram}
    Gram_{s}^t = Gram_{s}^{t-1} + \sum_{n = 1}^{N_{s}^t} h(x^t_i) \otimes h(x^t_i),
\end{equation}
\begin{equation}
    \label{eq: pro}
    C_{s}^t = \begin{bmatrix}C_{s}^{t-1} & \underbrace{\mathbf{0}_M \ \mathbf{0}_M \ \ldots \ \mathbf{0}_M}_{(L_{s}^t - L_s^{t-1})\text{ times}} \end{bmatrix} + \sum_{n = 1}^{N_s^t} h(x^t_i) \otimes y(x^t_i),
\end{equation}
where \( \mathbf{0}_M \) denotes a zero vector with \( M \) dimensions.

During the test phase, we combine the classification heads of all groups \( G = [G_1, G_2, \dots, G_k] \) to make a joint prediction for a given sample \( x \). For each class \( c' \) in a group, the score is computed as follows:
\begin{equation}
    \label{eq: predict}
    s_{c'} = g(\phi(x)^T W)(Gram_{i} + \lambda I)^{-1} C_{c'},
\end{equation}
where \( i = 1,\dots,k \) denotes the indices of each groups, and \( \lambda \) is the regularization parameter used to ensure that the \( Gram \) matrix remains invertible. The final classification result is then obtained by applying the following formula:
\begin{equation}
    \label{eq: predict_joint}
    \hat{c} = \mathop{\arg\max}\limits_{c' \in \cup_{i = 1}^k CLS^{G_i}} s_{c'},
\end{equation}
where \( \cup_{i = 1}^k CLS^{G_i} \) represents the set of possible classes across all class groups.
\fi
