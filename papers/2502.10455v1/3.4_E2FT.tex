\subsection{Evidence-enhanced Fine-tuning}

Regarding the identified challenge in the LVLM model Qwen2-VL~\cite{wang2024qwen2} when it comes to the OOC detection task, we suggest a one-stage multimodal instruction tuning solution extending the general-purpose LVLM, to the news domain for discerning multimodal OOC misinformation.

Our evidence-enhanced fine-tuning strategy is to provide both judgments and explanations for each image-claim pair. Simply put, we introduce both questions and candidate answers~\cite{shao2023prompting} into the tuning prompt $\mathcal{P}_\mathrm{OOC}(v,t,\widetilde{v}^r_e,{t}^r_e)$ that serves as informative inputs of the model. This makes the LVLM model unleash the potential knowledge behind itself~\cite{liu2024fka}, which explicitly analyzes the discrepancy between candidate answers for more accurate decisions. An illustration of the model response is shown in subfigure (d) of \Cref{fig:3}. For an image-claim pair with the ``Falsified'' label, we utilize the image $v$, claim $t$, rewritten textual evidence $\widetilde{v}^r_e$, and reranked visual evidence ${t}^r_e$ to format the input prompt of the LVLM model for response generation. The model provides the response likewise ``$<$Judgment$>$ $<$Explanation$>$'', where $<$Judgment$>$ indicates the symbol (\ie, ``No'') associated with candidate answers, and $<$Explanation$>$ is a coherent sentence serving as the compelling rationale for supporting its assessment.

In order to align the initial training way of the LVLM model, we employ LoRA technology~\cite{hu2022lora} and the next token prediction loss for assessing the model’s output error. Therefore, the learning objective of the proposed model E2LVLM over the OOC task, which can be described as,
\begin{equation}\label{eq:vcg_5}
    \mathcal{L}_{\mathrm{ooc}} = \sum_{i=1}^{N} -\log P(\epsilon_i \mid v_i,t_i,\widetilde{v_i}^r_e, {t_i}^r_e, \theta_{\mathrm{ooc}}),
\end{equation}
where, $\epsilon_i$ is considered as the generated tokens of both judgment and explanation corresponding to the formatted input $(v_i,t_i,\widetilde{v_i}^r_e,{t_i}^r_e)$, and $\theta_{\mathrm{ooc}}$ refers to the learnable parameters of E2LVLM during training. Besides, $N$ is the size of $\widetilde{\mathcal{D}}$.