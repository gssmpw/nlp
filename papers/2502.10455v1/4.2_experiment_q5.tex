\subsection{RQ5: Data Scaling Exploration}

\begin{table}[t]
\caption{A multimodal OOC misinformation detectors with distinct LVLMs. Evaluated on the NewsCLIPpings~\cite{luo2021newsclippings} dataset.}
\centering
\small
% \setlength{\tabcolsep}{0.5pt} % 调整列间距,减小为4pt
% \renewcommand{\arraystretch}{1.1} % 调整行高(默认为1)
    \begin{adjustbox}{valign=c,max width=\columnwidth}
        \begin{tabular}{lc|c|c}
            \toprule
            Settings & Tuning & Parameters & \textbf{All} \\
            \hline
            Random & \xmark & -  & 50.3	\\
            Qwen2-VL-2B & \xmark & 1.5B  & 65.7 \\
            Qwen2-VL-7B & \xmark & 7.6B  &  79.1 \\
            Qwen2-VL-72B & \xmark & 72B  & 79.8 \\
            \hline
            Qwen2-VL-2B & \cmark & 1.5B  & 81.4	\\
            \rowcolor{lightgreen} Qwen2-VL-7B & \cmark & 7.6B  & \textbf{89.9} \\
            \bottomrule
        \end{tabular}
    \end{adjustbox}
\label{tab:tab_4}
\end{table}


\begin{table}[t]
\caption{Comparison on distinct LVLMs between data construction and model tuning. Evaluated on NewsCLIPpings~\cite{luo2021newsclippings}.}
\centering
\small
% \setlength{\tabcolsep}{5.5pt} % 调整列间距,减小为4pt
% \renewcommand{\arraystretch}{1.1} % 调整行高(默认为1)
    \begin{adjustbox}{valign=c,max width=\columnwidth}
        \begin{tabular}{lcl|c}
        \toprule
         Data Construction &  \ding{223} & Model  Tuning & \textbf{All} \\
        \hline
        Qwen2-VL-2B & \ding{223} & Qwen2-VL-2B   & 81.1 \\
        Qwen2-VL-2B & \ding{223} & Qwen2-VL-7B   & 87.6 \\
        \hline
        Qwen2-VL-7B & \ding{223} & Qwen2-VL-2B   & 81.4 \\
        \rowcolor{lightgreen} Qwen2-VL-7B & \ding{223} & Qwen2-VL-7B   & \textbf{89.9} \\
        \bottomrule
        \end{tabular}
    \end{adjustbox}
\label{tab:tab_5}
\end{table}


% \begin{figure}[t]
%   \centering
%    \includegraphics[width=.9\linewidth]{sec/fig6.jpg}
%    \caption{Performance of E2LVM on NewsCLIPpings~\cite{luo2021newsclippings} using different training data proportions.}
%    \label{fig:6}
% \end{figure}

% \newpage
To evaluate the feasibility of rapid deployment of E2LVLM at the stage of early detection, we randomly choose 10$\%$, 25$\%$, 50$\%$, and 75$\%$ samples for experiment analysis. As shown in \Cref{fig:6}, E2LVLM achieves remarkable performance on different proportions of training data. Typically, in a zero-shot scenario, E2LVLM has outperformed the previous method (\eg, DT-Transformer~\cite{papadopoulos2023synthetic}), suggesting the effectiveness of E2LVLM in the task of OOC. With the increasing of samples, the discriminatory powers of E2LVLM have a significant rising tendency. This demonstrates that the proposed E2LVLM can be rapidly deployed and ensure detection accuracy, even at the stage of early detection.

\subsection{RQ6: Robustness Analysis}

% \begin{figure}[t]
%   \centering
%    \includegraphics[width=.9\linewidth]{sec/fig6.jpg}
%    \caption{Performance of E2LVM on NewsCLIPpings~\cite{luo2021newsclippings} using different training data proportions.}
%    \label{fig:6}
% \end{figure}


% \begin{table}[t]
% \caption{Comparison of ``True vs OOC'' accuracy on the dataset VERITE~\cite{papadopoulos2024verite}. Trained on NewsCLIPpings~\cite{luo2021newsclippings} and evaluated on ``All''. The partial results are reported from SNIFFER~\cite{qi2024sniffer}.}
% \centering
% % \Huge
% % \setlength{\tabcolsep}{10pt} % 调整列间距,减小为4pt
% \renewcommand{\arraystretch}{1.1} % 调整行高(默认为1)
%     \begin{adjustbox}{valign=c,max width=\columnwidth}
%         \begin{tabular}{l|c|ll}
%             \toprule
%             Methods & Venue & \textbf{All} & \textbf{True vs OOC} \\
%             \hline
%             RED-DOT~\cite{papadopoulos2023red} & arXiv23 & 84.5 & 73.9 \\
%             SNIFFER~\cite{qi2024sniffer} & CVPR24 & 88.4 ($\uparrow$ 3.9) & 74.0 ($\uparrow$ 0.1)  \\
%             \rowcolor{lightgreen} E2LVLM &  & \textbf{89.9 ($\uparrow$ 5.4)} & \textbf{74.4 ($\uparrow$ 0.5)}  \\
%             \bottomrule
%         \end{tabular}
%     \end{adjustbox}
% \label{tab:tab_6}
% \end{table}

To further verify the robustness of E2LVLM towards real-world scenarios, we implement extended experiments on VERITE~\cite{papadopoulos2024verite}. This dataset serves as a novel real-world OOC misinformation, including 1,000 annotated samples stemming from fact-checking websites. Following~\cite{qi2024sniffer}, we also compare E2LVLM with RED-DOT~\cite{papadopoulos2023red} in \Cref{tab:tab_6}. From the results, we observe that E2LVLM shows the best performance on large-scale NewsCLIPpings~\cite{luo2021newsclippings} and more challenging VERITE~\cite{papadopoulos2024verite}. Typically, E2LVLM achieves a ``True vs OOC'' accuracy of 74.4$\%$, which outperforms RED-DOT by 0.5$\%$ accuracy, and the SOTA SNIFFER by 0.4$\%$ accuracy. The evaluations on these two multimodal OOC misinformation datasets exhibit the detection accuracy and robustness of E2LVLM in practical scenarios.