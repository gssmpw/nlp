\begin{abstract}
     Existing Multimodal Large Language Models (MLLMs) are predominantly trained and tested on consistent visual-textual inputs, leaving open the question of whether they can handle inconsistencies in real-world, layout-rich content. To bridge this gap, we propose the Multimodal Inconsistency Reasoning (MMIR) benchmark to assess MLLMs' ability to detect and reason about semantic mismatches in artifacts such as webpages, presentation slides, and posters. MMIR comprises 534 challenging samples, each containing synthetically injected errors across five reasoning-heavy categories: Factual Contradiction, Identity Misattribution, Contextual Mismatch, Quantitative Discrepancy, and Temporal/Spatial Incoherence.
     We evaluate six state-of-the-art MLLMs, showing that models with dedicated multimodal reasoning capabilities, such as o1, substantially outperform their counterparts while open-source models remain particularly vulnerable to inconsistency errors.
     Detailed error analyses further show that models excel in detecting pairwise inconsistencies but struggle with inconsistencies confined to single elements in complex layouts. 
     Probing experiments reveal that single-modality prompting, including Chain-of-Thought (CoT) and Set-of-Mark (SoM) methods, yields marginal gains, revealing a key bottleneck in cross-modal reasoning. Our findings highlight the need for advanced multimodal reasoning and point to future research on multimodal inconsistency.
     % \eric{abstract is a bit long. }

          % To tackle these limitations, we propose an Interleaved Multimodal CoT (MM-CoT) framework that iteratively refines reasoning by jointly leveraging visual and textual modalities—yielding notable performance improvements for both proprietary and open-source models.
\end{abstract}