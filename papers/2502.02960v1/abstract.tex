Large Language Models (LLMs) represent a transformative leap in artificial intelligence, enabling the comprehension, generation, and nuanced interaction with human language on an unparalleled scale. However, LLMs are increasingly vulnerable to a range of adversarial attacks that threaten their privacy, reliability, security, and trustworthiness. These attacks can distort outputs, inject biases, leak sensitive information, or disrupt the normal functioning of LLMs, posing significant challenges across various applications. 

In this paper, we provide a novel comprehensive analysis of the adversarial landscape of LLMs, framed through the lens of {\em attack objectives}. By concentrating on the core goals of adversarial actors, we offer a fresh perspective that examines threats from the angles of {\em privacy}, {\em integrity}, {\em availability}, and {\em misuse}, moving beyond conventional taxonomies that focus solely on attack techniques. This objective-driven adversarial landscape not only highlights the strategic intent behind different adversarial approaches but also sheds light on the evolving nature of these threats and the effectiveness of current defenses. Our analysis aims to guide researchers and practitioners in better understanding, anticipating, and mitigating these attacks, ultimately contributing to the development of more resilient and robust LLM systems.




