\documentclass[11pt]{article}
\usepackage{graphicx} % Required for inserting images
\usepackage[utf8]{inputenc}
\usepackage{amsmath,amssymb}
\usepackage[top=1in,bottom=1in,left=1in,right=1in]{geometry}
\usepackage{sfmath}

\usepackage{titlesec}
\usepackage{authblk}


 \renewcommand{\Authfont}{\fontfamily{phv}\fontsize{11}{13}\fontfamily{phv}\selectfont}
 \renewcommand{\Affilfont}{\fontfamily{phv}\fontsize{10}{12}\selectfont}
 
\title{PATCH: a deep learning method to assess heterogeneity of
artistic practice in historical paintings}

\author[1,2]{Andrew Van Horn}
\author[2]{Lauryn Smith}
\author[3]{Mahamad Mahmoud}
\author[1]{Michael McMaster}
\author[2]{Clara Pinchbeck}
\author[1,4]{Ina Martin}
\author[4]{Andrew Lininger}
\author[5]{Anthony Ingrisano}
\author[6]{Adam Lowe}
\author[6]{Carlos Bayod}
\author[2]{Elizabeth Bolman}
\author[1]{Kenneth Singer}
\author[1]{Michael Hinczewski}

\affil[1]{Department of Physics, Case Western Reserve University, Cleveland, OH,
USA}

\affil[2]{Department of Art History and Art, Case Western Reserve University,
Cleveland, OH, USA}

\affil[3]{Department of Computer and Data Sciences, Case Western Reserve
University, Cleveland, OH, USA}

\affil[4]{MORE Center, Case Western Reserve University, Cleveland, OH, USA}

\affil[5]{Cleveland Institute of Art, Painting Department, Cleveland, OH,USA}

\affil[6]{Factum Foundation, Madrid, Spain}




\date{}

\renewcommand{\figurename}{\fontfamily{phv}\fontsize{9}{12}\selectfont \textbf{Figure}}
\renewcommand{\thetable}{\textbf{\arabic{table}}}
\renewcommand{\thefigure}{\textbf{\arabic{figure}}}
\renewcommand{\tablename}{\fontfamily{phv}\fontsize{9}{12}\selectfont \textbf{Table}}
%\renewcommand{\theequation}{S\arabic{equation}}

\begin{document}
\fontfamily{phv}\selectfont
\maketitle


\begin{abstract}
The history of art has seen significant shifts in the manner in which
artworks are created, driven by changes in technology and the art
market. Thus, understanding creative processes is a central question in
technical art history. In the Renaissance and Early Modern period,
paintings were largely produced by master painters who directed
workshops of apprentices and others who often contributed to the work
produced. The masters varied significantly in both their artistic and
managerial styles, meaning different combinations of artists and
implements might be seen both between masters and within workshops or
even individual canvases. Information about the way different workshops
were managed and the processes by which artworks were created remains
elusive, however. Machine learning methods have potential to unearth new
information about artists' creative processes by extending the analysis
of brushwork to a microscopic scale. Analysis of paintings created in a
workshop setting, however, presents a challenge in that documentation of
the artists and materials involved in individual works is sparse,
meaning external examples are not available to train networks to
recognize their contributions. Here we present a novel machine learning
approach we call pairwise assignment training for classifying
heterogeneity (PATCH) that is capable of identifying individual artistic
practice regimes with no external training data, or ``ground truth.''
The method achieves unsupervised results by supervised means, and
outperforms both simple statistical procedures and unsupervised machine
learning methods. We apply this method to two historical paintings by
the Spanish Renaissance master, El Greco: \emph{The Baptism of Christ}
and \emph{Christ on the Cross with Landscape}, and our findings
regarding the former potentially challenge previous work that has
assigned the painting to workshop members. Further, the results of our
analyses create a measure of heterogeneity of artistic practice that can
be used to characterize artworks across time and space.
\end{abstract}

\titleformat*{\section}{\fontsize{11}{13}\bfseries\selectfont}


\section*{Introduction}

There are many ways to create a painting. The process can be
collaborative or solitary, and there is no end to the styles, tools,
materials and techniques that can be implemented. The evolution of what
we call artistic practice---the technical, stylistic, material, and even
physiological aspects of an artist's creative process---over time and its
variation between artists remains a key element of technical art history
scholarship. For example, in the last two centuries, Western painters,
aided (and in many ways, enabled) by concomitant changes in the tools of
the trade such as the invention of the paint tube\textsuperscript{1},
have often chosen more solitary practice in the studio or even outdoors.
From the late Middle Ages through the Early Modern period, on the other
hand, painting was typically done in a bustling
workshop\textsuperscript{2}. A master painter would establish a workshop
and assemble a group of apprentices, journeymen and specialist painters
and even collaborate with associates to help fulfill commissions and
create smaller works to support the enterprise between larger contracts.

Workshop practice adds a second layer of complexity to artistic
practice. Master painters varied considerably in their managerial
styles. Some performed much of the work themselves, while others
dictated the composition but left many elements to the hands of workshop
members (e.g., van Dyck\textsuperscript{3}). Still others involved
apprentices in the conceptualization of a piece\textsuperscript{4}.
Often, there was a hierarchy, with young apprentices typically assigned
more mundane tasks such as the mixing of paints and the construction of
canvases and more senior workshop members directed to paint portions of
larger works in the master's style\textsuperscript{5}.

Because workshops varied so widely in their organization and management
yet produced works with cohesive styles, deeper understanding of
variation in artistic practice can shed light on workshop practice and
vice versa. Research into these key aspects of Renaissance and Early
Modern art production is hindered by a lack of surviving textual sources
regarding the structure, size, and membership of individual workshops as
well as their day-to-day operation and the variation in artistic
practices\textsuperscript{5}. While scholars have been able to
reconstruct individual artists' workshops through a variety of methods,
our collective understanding of the inner workings of workshops is still
fragmentary\textsuperscript{6}. Artistic practice varies widely,
encompassing materials, methods, styles, and individual contributions.
Understanding who interacted with individual paintings and how they did
so---discerning the artists at work in a painting, the interplay of their
individual ``hands,'' and the interaction of the artist's hand and eye
with paint and brush\textsuperscript{7}---is an important open question
in art history\textsuperscript{8}. Answering these questions requires
new methods of analysis.

Machine learning has recently emerged as a promising complement to
traditional art historical analyses in applications including
preservation, conservation and the detection of forgeries\textsuperscript{9--12} as well as in fighting illegal trafficking in
antiquities\textsuperscript{13}. Much work involving style is focused on
broad classification of works\textsuperscript{14,15}, such as the use of
a neural network to arrange works of the Western canon in the correct
chronological order\textsuperscript{16}. Recently published work by some
of the authors of the present study demonstrated the efficacy of ML for
attribution of paintings by analyzing high-resolution topographic images
of the surface texture of oil paintings by known authors (student
painters) in a controlled experimental setting. The topographic images,
created using optical profilometry, captured variations in the height of
the paint with microscopic resolution. While existing methods of study
and conservation can employ microscopic methods to small areas obtained
using a variety of imaging techniques, optical profilometry is unique in
that it creates a topographic image at microscopic resolution over the
entire painting. Though time consuming, the method provides a wealth of
data on the surface topography and even the topography of pigmented
layers beneath a transparent varnish layer. In a supervised machine
learning analysis, a convolutional neural network (CNN) was trained to
sort microscopic surface textures among artists, assigning 1
cm\textsuperscript{2} patches of paintings to the correct authors with
$\sim$95\% accuracy\textsuperscript{17}. Thus, the method has
shown that microscopic analysis can complement traditional art
historical analyses of macroscopic features.

But can deep learning on topographic images be used for such analyses
when there is no ground truth to test against? When the number of
artistic practice regimes in a given piece is unknown, known examples of
the included artists' work may not exist, and knowledge of the painting
practices employed is incomplete---as in a painting produced by a
Renaissance or Early Modern artist's workshop? Here, we demonstrate a
novel technique, to our knowledge, for ML-based attribution of paintings
by unknown authors to the correct artist(s). We then combine this method
with network analysis to identify artistic practice regimes
(combinations of artists and materials) and create a measure of the
heterogeneity of artistic practice (HAP) within a given painting or set
of paintings. We apply this method to two important historical paintings
by El Greco, one which is considered to be entirely by the master
himself, and one previously thought to feature the work of members of
his workshop.

ML methods can be divided into two classes based on the way the network
is trained and the output of the network: supervised and unsupervised.
In unsupervised learning, the machine analyzes individual objects (e.g.,
images or patches of an image) and creates classes based on the
statistical properties of those objects. In supervised learning, the
classes are already known, and the network learns by attempting to
correctly classify objects in a training set, adjusting the weights of
the interconnections between its nodes as it succeeds (or fails) in
doing so. A successfully trained network will be able to assign novel
objects to the correct classes. Supervised learning is thus particularly
useful for image recognition tasks. The method employed in the previous
publication described above\textsuperscript{17} was a standard
application of supervised learning for image recognition to topographic
images. The network was trained on known topographic images (in this
case, two paintings by each artist) and then tested on novel objects (a
third painting by each artist). The trained network was able to
distinguish between four different artists, correctly assigning
1 cm\textsuperscript{2} patches of each painting to its respective artist
with 95\% accuracy.

The Pairwise Assignment Training for Classifying Heterogeneity (PATCH) method described here uses supervised learning toward an
unsupervised end. Rather than train a network to recognize individual
classes, we test whether the network is \emph{capable} of learning to
distinguish between two objects and then use that information to build
\emph{post-hoc} classes. The PATCH method relies on the inability of the
network to correctly sort patches of paintings or regions of a painting
painted by the same artist. Given patches of two paintings by different
painters---a different-artist pair---the network will learn to assign
those patches to the correct painting with a high degree of accuracy, as
demonstrated in our previous publication. But if the network is given
patches of two paintings by the same artist---a same-artist pair---it
will not be able to learn to assign those patches to the correct
painting; it will perform no better than a coin flip. We then create a
network of same-artist pairs and use network analysis to construct
classes corresponding to artistic practice regimes. In this way, we
obviate the need for separate ground truth information for individual
artists and artistic practice regimes, enabling successful applications
in contexts such as workshop paintings where the ground truth is not
known with certainty.

\section*{Study 1: Method development}

The PATCH method (Figure 1) consists of two phases. The first phase, pairwise
assignment training, identifies pairs of paintings or pairs of regions
of a single painting that are the ``same,'' that is, created by the same
artist under the same conditions. The second phase, community finding,
creates \emph{post hoc} classes based on these pairings.

\begin{figure}[t]
\includegraphics[width=\textwidth]{Fig1.pdf}
\caption{Overview of the PATCH method: A) Paint height data is
gathered using optical profilometry, generating a 16-bit image file
encoding the height values, which is subsequently divided into
1x1 cm\textsuperscript{2} patches. B, C) A CNN is repeatedly trained
to sort patches of two images, recording its accuracy as a percentage of
patches correctly sorted from the test set. This distribution of
observed accuracies is compared to the random assignment distribution
(RAD) of accuracies if the patches were sorted at random. If the
observed accuracy is no better than the RAD accuracy, then the images
were likely painted by the same artist, whereas if the observed accuracy
is better than expected by chance, the images were likely painted by
different artists.}
\end{figure}


The first phase was developed on a set of paintings by known artists. A
group of nine artists each individually painted three paintings, all
using the same tools, materials, and subject (a photograph of lilies).
The materials used for the paintings were described
previously\textsuperscript{17}. While the artists were instructed to
realistically depict the subject, choices of style and technique
(including the use of brushes and alteration of paint viscosity) were
left up to the artists. One artist was only able to produce two
paintings, and one painting was removed from the sample due to the
optical profilometry data being corrupted, for a total of 25 paintings.
Topographical information for each painting was recorded using high
resolution spectral confocal optical profilometry with a spatial
resolution of 50 microns, and a height repeatability of 200 nm. The resulting data were processed to remove any large scale warping
of the canvas by subtracting a mean-filtered version of the height map
(with filter radius of 0.5 cm)\textsuperscript{17}.

The network architecture chosen was VGG-16, implemented in TensorFlow
and pretrained on the ImageNet dataset. Given the computational
complexity of the training---the combinatorics of many region-to-region
comparisons, repeated over many folds, described in detail below---the
relatively light-weight but accurate VGG-16 architecture was ideal for
the PATCH approach. Output from VGG-16 was pooled and flattened and
passed through two dense layers with dropout rates of 0.25. The learning
rate was set to 0.0001 and the batch size was set to 32 patches (see the
\emph{Supplementary Materials} for more information on hyperparameter
optimization). In order to measure the ability of the network to learn
to distinguish between two paintings or regions of a painting, we ran 26
training folds. Each fold comprised the following steps: 1
cm\textsuperscript{2} patches from each painting/region were randomly
selected, with replacement. Sampling with replacement creates a
bootstrapping effect that drives down the success rate of same-artist
pairs and drives up the success rate for different-artist pairs. In each
fold, the network was trained for 25 epochs with validation set size set
to 30\% of patches, and the maximum validation accuracy (percentage of
validation patches correctly identified) was recorded. At the end of 26
folds, the distribution of the 26 maximum validation accuracies was
compared to the distribution expected if the network were assigning
patches at random.


If the network has failed to learn to assign objects to their correct
classes, it should perform this task no better than if it were assigning
them randomly. The prospect of randomly choosing from two options
invokes the coin flip and, by extension, the binomial distribution.
However, we are measuring the maximum validation accuracy from 25
training epochs in each of 26 folds. Thus, the distribution expected if
the network fails to learn to distinguish between painters would be the
random assignment distribution (RAD) \(p_{n,k}^{\max}(m)\), the
probability that you will see a maximum of \emph{m} heads (i.e., correct
assignments) over \emph{k} repetitions (\emph{k} epochs) of an
experiment where you flip a fair coin \emph{n} times (the number of
patches in the test set). The \emph{Supplementary Materials} contains an
analytical derivation of this distribution.

Two decision criteria were selected to determine whether the
distribution of maximum validation accuracies indicated that the network
had failed to learn to distinguish between the input paintings: 1) if
the observed mean had a z-score less than 2 relative to the RAD mean and
2) if the largest of the observed maximum validation accuracies (the
right edge of the observed distribution) among the 26 folds was less
than an empirically determined threshold based on finite sampling from
RAD, plus 10\% to account for the effects of bootstrapping (selection
with replacement, see \emph{Supplementary Information} for more
details). Using these decision criteria, the accuracy in determining
whether the artist was the same or different was exceptional. We
identified same-artist pairs (\emph{n} = 23) and different-artist pairs
(\emph{n} = 277) with 96.0\% overall accuracy (87\% for same and 99.3\%
for different). This translates to F$_1$ scores of 0.991 for
different-artist pairs and 0.889 for same-artist pairs, an average of
0.940 (Table 1). The pairwise assignment training phase outperformed
both an unsupervised method and a statistical method based on surface
roughness (see details in SI Appendix C) in precision, recall and total
F$_1$ for both same-artist and different-artist pairs.

\begin{table}
\centering
\renewcommand{\arraystretch}{1.05}
\fontfamily{phv}\selectfont
\begin{tabular}{|l|c|c|c|} 
\hline
 &\textbf{Precision}&\textbf{Recall} & \textbf{F$_1$}\\
\hline
\textbf{Same-artist pairs}& 0.909 & 0.87 & 0.889\\
\textbf{Different-artist pairs}&0.989&0.993&0.991\\
\textbf{\em Average}&\textit{0.949}&\textit{0.931}&\textit{0.940}\\ 
\hline
\end{tabular}
\caption{Precision, recall and F$_1$ score for the pairwise
assignment training process implemented on a group of paintings from a
controlled experiment where the attribution was known. Same-artist and
different-artist pairs are treated as ``positives'' in turn and the
results are averaged.}
\end{table}

While pairwise assignment training performs exceptionally well in
identifying artists, situations where ground truth about the artistic
practices employed in a particular painting is not available require
some means to create the \emph{post hoc} classes to which regions of a
painting will be assigned. To this end, we use network analysis. We
construct a network where same-artist pairs (or, for historical
paintings, same-practice pairs) of paintings or regions are connected by
an edge (lines connecting regions in Figure 3) and
different-artist/practice pairs are unconnected.

The PATCH method is concerned with ``sameness,'' meaning we need a way
to remove spurious edges from the network. To accomplish this, we employ
an edge pruning process based on uniqueness. We assume that regions with
a high degree (number of connections) have more common features and
nodes with a low degree have more unique features. Therefore, a
connection between regions with more unique features is more likely to
be correct. We operationalize this assumption by assigning each edge a
score based on the average uniqueness of its nodes, where uniqueness is
defined as the percentage of nodes a node does not connect with:
$W_{ij} = \frac{1}{2}((\text{max} - \text{degree}_{i})/\text{max} + (\text{max} -
\text{degree}_{j})/\text{max})$ where max = the maximum possible degree
(i.e., number of connections) and degree = the degree
of one of the two nodes in question. There were two erroneous
same-artist pairs in our experimental dataset (9\% of edges in the
network), and these edges had the lowest uniqueness scores. Thus, we
trim the 9\% of edges (or more if there are ties) in a given network
with the lowest uniqueness scores.

With potentially spurious edges removed, and the remaining edges
weighted equally, we then employ a community finding algorithm to
identify groups of regions that are the most similar. If a set of
regions has a large number of internal links within the set, and fewer
links to outside regions, it forms a ``community'' within the
network\textsuperscript{18--20}, suggesting that the regions in a given
community were likely painted under different circumstances (artists,
materials, etc.) than those in other communities. To establish the
degree of difference between communities, we can characterize the
community structure of the network via measures such as modularity
($Q$)---the fraction of edges in the network that are internal to
communities minus the mean fraction in a network with the same
communities but where the edges are completely
randomized\textsuperscript{19}. We implemented the Louvain community
finding algorithm\textsuperscript{21} in Gephi\textsuperscript{22}.
With the resolution set to 1.0, the algorithm returns the partition of
the network with the maximum modularity. The Louvain algorithm returned
a correct partition of the experimental dataset with $Q=0.875$.

\section*{El Greco, \emph{The Baptism}, and \emph{Christ on the Cross}}

El Greco, born Domenikos Theotokópoulos (1541-1614), is regarded as a
pillar of the Renaissance in Spain and as an early progenitor of
modernism\textsuperscript{23}. El Greco became a master of icon painting
in Crete before journeying through Venice (where he may have worked with
Titian) and Rome, eventually settling in Toledo,
Spain\textsuperscript{24,25}. It was in Toledo that the master married
Byzantine and Venetian motifs\textsuperscript{26} and developed the
peculiar style for which he is most widely recognized, with its
distorted human figures, ``expressive hands''\textsuperscript{27}, and
"exploitation" of "pure colors to their limits"\textsuperscript{28}. Two
examples of that style are \emph{The Baptism of Christ} (1624, Hospital
Tavera, Toledo, Esp., henceforth, \emph{The Baptism}), shown in Figure
2A, and \emph{Christ on the Cross with Landscape} (ca. 1600-1610,
Cleveland Museum of Art, Cleveland, OH, henceforth \emph{Christ on the
Cross}), shown in Figure 2B.

These works were chosen for this analysis because of the difference in
how art historians have characterized their authorship. \emph{The
Baptism} has long been thought to feature the work of El Greco and at
least one other artist. Primary historical evidence indicates that El
Greco began the painting under a contract with the Hospital Tavera but
retained it at his death in 1614. It was delivered to the Hospital
nearly a decade later\textsuperscript{29}, during which period art
historians have proposed the painting was finished by workshop members,
particularly the master's son, Jorge Manuel\textsuperscript{24}.
Previous art historical studies have attempted to attribute regions of
\emph{The Baptism} to the master and others (illustrated in Figure S2)
largely through connoisseurship---visual analysis of artists' styles and
artistic choices\textsuperscript{30}. El Greco himself is proposed to
have painted the entire top, with the possible exception of the robe of
the angel on the right (which was not scanned for this study due to its
poor condition)\textsuperscript{29,31}, as well as the angel in green
(bottom left), with the exception of the wings. Jorge Manuel is proposed
to have been painted John the Baptist (bottom right) and the adjacent
figure in red. Lopera\textsuperscript{31} and Wethey\textsuperscript{29}
ascribe the image of Christ (bottom center) to Jorge Manuel and El
Greco, respectively, suggesting that it could represent the work of both
artists. Lopera believes that a third hand may have been involved in
rendering the faces of the angels in the background on the bottom. The
landscapes at the bottom were extended and ``transformed into a river''
by an unknown artist sometime between receipt of the painting and its
installation in the epistle-side altarpiece ca. 1660, which was not an
unusual practice at the time. Restoration of the landscape on the right
at the very bottom was performed in 1936\textsuperscript{29,31,32}. In
contrast, \emph{Christ on the Cross} has been entirely attributed to El
Greco himself. It is generally accepted that El Greco created many works
in their entirety without the aid of his workshop, including
\emph{Christ on the Cross.} Francis\textsuperscript{33} states that most
of El Greco's approximately twenty renditions of the Crucifixion were
``by his own hand,'' and notes the ``lightning dexterity of brushwork''
in this particular example.

\begin{figure}
\includegraphics[width=\textwidth]{Fig2.pdf}
\caption{A) El Greco's \emph{The Baptism of Christ}, begun
in approximately 1608 and suspected to have been finished by his
workshop before delivery to the Hospital Tavera in 1624. B) El Greco's
\emph{Christ on the Cross with Landscape,} ca. 1600--1610.}
\end{figure}

\section*{Study 2: Application to El Greco's {\em The Baptism}}

Two large sections of \emph{The Baptism} were scanned \emph{in situ}
using the Factum Foundation's Lucida scanner. The Lucida scanner is a
non-contact laser triangulation scanner with 100-micron lateral
resolution and is capable of acquiring scans in situ over a large area
by stitching together 48 cm square tiles. The scanned regions used in
this study are indicated in Figure 3. The height data was processed in
the same way as in study 1, including correction for possible canvas
warping.

Regions of the topographic image of \emph{The Baptism} were selected by
hand in order to minimize variation in subject matter (e.g., faces,
robes) within each region and to constrain region size to between 180
and 540 cm\textsuperscript{2}. The lower bound was chosen because the
method had been developed using 180-patch paintings in study 1 and the
upper bound was selected to ensure adequate representation of patches
when regions of significantly different size were compared. We avoided
selecting areas with significant cracking or damage, including the robes
of the large angels to the left and right of God, and the red swatch
held by the ``green angel'' in the bottom half of the painting. Each
fold proceeded as described in study 1, above. Where regions were of
different sizes, the sample size was set equal to the number of patches
in the smaller region.

Pairwise comparisons of all 55 regions (a total of 1485 pairwise tests)
of \emph{The Baptism} yielded a network with 356 edges, which was
reduced to 314 edges after pruning. The Louvain community finding
algorithm\textsuperscript{21} with resolution set to 1.0, returned a
partition with $Q=0.341$ that featured four communities (Figure
3a). A value of $Q \ge  0.3$ is often considered evidence
of structure in real-world networks\textsuperscript{19}.

\begin{figure}
\includegraphics[width=\textwidth]{Fig3.pdf}
\caption{Results of PATCH analysis on El Greco's \emph{The Baptism of Christ}
(1624, Hospital Tavera, Toledo, Esp.). Areas that were scanned are
outlined with dashed white lines. A) Network diagram showing the
maximum modularity ($Q$) partition. Four communities are present. B) the
communities mapped onto the corresponding regions of the painting that
were analyzed.}
\end{figure}

The four communities are mapped onto the painting in Figure 3b. The
first community overlaps much of the top portion of the painting,
including the image of God the Father. The second overlaps the image of
John the Baptist and Christ's torso and hands. The third covers much of
the center of the painting, the face of Christ and the faces of many of
the background angels. The final community is confined to Jesus' legs
and the rocks at the bottom of the painting.

\section*{Study 3: Application to El Greco's {\em Christ on the Cross}}

As in study 2, regions of the topographic image of \emph{Christ on the
Cross} were selected by hand in order to minimize variation in subject
matter and to constrain region size. Areas with apparent cracking or
damage, such as the background and trees on the lower left and the cross
beam to the immediate left of Christ's face, were not included.

Pairwise comparison of all 24 regions (276 pairwise tests) yielded a
network with 123 edges, which was reduced to 108 edges after pruning.
The Louvain community finding algorithm\textsuperscript{21} at
resolution 1.0 generated a partition with $Q=0.231$ that featured
two communities. Two is the smallest possible number of communities for
a non-fully connected graph, as $Q$ compares inter- and
intra-community connections. Communities are mapped onto the painting in
Figure 4. Constituent regions of both communities are not randomly
distributed, with one community overlapping much of Christ's torso and
face, while the other community overlaps Christ's legs, the lower
portion of the cross and regions of the background.

\begin{figure}
\includegraphics[width=\textwidth]{Fig4.pdf}
\caption{Results of PATCH analysis on El Greco's \emph{Christ
on the Cross with Landscape.} A) Network diagram showing the maximum
modularity ($Q$) partition. Two communities are identified, though it
should be noted that this is the minimum number of communities
discoverable using a maximum-modularity partition. B) The communities
mapped onto the corresponding regions of the painting that were
analyzed.}
\end{figure}

\section*{Analysis of PATCH-generated communities}

Further analysis of the community structure in the PATCH generated
networks can provide additional insight into the community cohesiveness
beyond the modularity index. We analyzed the average community degree,
defined as the count of all edges between nodes within two communities.
This is shown for all possible community pairs in Figure 5A and C, and
within a single community (internal degree) or between nodes within and
external to a given community (external degree) in Figure 5B and D. Each
degree is normalized by the theoretical maximum degree based on the
number of edges in the classification. The degree between community
pairs and external degree can be taken as a measure of the PATCH
inter-community similarity between specific communities and between a
community and all others, respectively. Likewise, the internal degree
can be taken as a measure of intra-community cohesiveness.

In \emph{The Baptism}, all of the connections between the first three
communities are significantly stronger than those involving the fourth
community, with the average external degree over four times larger. This
indicates a significant isolation of the fourth community (colored orange in Figure 3, and represented by orange bars in Figure 5B), indicating a relatively strong disparity in the topographical data. The strongest connection is between communities 1 (yellow) and 3 (green) and the weakest is between 3
and 4, which has no connecting edges. Overall the 3rd community has a
slightly greater connection with the remainder of the communities,
indicating greater similarity. Interestingly, the internal degree is
relatively similar for all communities, indicating that the differences
within the observed topographical signal are evenly distributed across
communities. A significant difference in the internal degree could, for
instance, indicate a community with above or below average range of
stylistic practice.

\begin{figure}
\includegraphics[width=\textwidth]{Fig5.pdf}
\caption{Analysis of the network community structure generated
from the PATCH method for: A, B) El Greco's \emph{The Baptism of Christ}
and C, D) \emph{Christ on the Cross with Landscape}. A, C) Normalized
network degree between each pair of identified communities. The color
scale is logarithmic. B, D) Normalized internal degree (considering all
edges within a given community) and external degree (considering all
edges between two distinct communities) averaged over all external
community pairs in the network. Colors of the bars correspond to the communities in
the network representations of Figures 3-4. Note that \emph{Baptism} community 4 (orange) in panel B has
significantly less external connectivity, and the external degree for
the \emph{Christ on the Cross} communities in panel D is significantly higher than for any \emph{Baptism} community.}
\end{figure}

Although the external degree analysis is largely redundant in the case
of \emph{Christ on the Cross}, since only two communities
are proposed, it should be noted that the external degree between the
two communities is significantly larger than for any other two
communities in \emph{The Baptism} except the connection between communities 1
and 3. This indicates the two proposed communities are on average more
closely connected than those proposed in \emph{The Baptism}, which is reflected
in the low modularity value. Additionally, the internal degree in each
community is similar, which can indicate a similar intra-community
stylistic distribution.

\section*{Discussion}

To our knowledge, ours is the first study to demonstrate this learning
method and apply it both to paintings from a controlled experiment and
to historical paintings by a well-known artist with a workshop. Among
AI/ML-based methods designed to aid in the analysis of historical
paintings, PATCH is notable in that it requires no external, ground
truth data. Networks trained on known examples can accomplish tasks
associated with workshop practice and authorship with exceptional
accuracy. For example, Ugail and colleagues recently used transfer
learning to authenticate paintings by the Renaissance master Raphael
(Raffello Sanzio)\textsuperscript{34}. By combining edge detection with
typical feature extraction in a residual neural network, they were able
to recognize works by Raphael with 98\% accuracy. However, known
examples of artists' work may not be available, as exemplified by the
workshop context. Our method shows exceptional promise as a tool for ML
analysis of complex historical works where known samples of work by the
artists involved do not exist. Indeed, it may have extensive application
in other image analysis tasks where there is little to no ground truth
information available.

Further, the second phase of PATCH creates what we consider a measure of
the heterogeneity of artistic practice (HAP) within a given dataset. Communities found within a dataset
represent different artists, materials or both. The degree to which
those communities are interconnected provides information about the
possibility of shared authorship, materials, or other practices. A fully
connected network with no distinct communities has $Q = 0$. As the number
of communities increases and the interconnection between communities
decreases, modularity increases toward 1 (a completely disconnected
network has undefined modularity, but would effectively represent the
maximum heterogeneity). The networks from our three studies provide
empirical examples of lower and higher modularity networks. \emph{Christ
on the Cross} and \emph{The Baptism} have $Q$ values that cluster around
0.3, considered the threshold for evidence of structure, indicating low
heterogeneity. The network for the experimental student paintings has
much greater heterogeneity at $Q=0.875$ (see Figure S3). The
maximum modularity depends on the number of edges in the network\textsuperscript{35}, but also on the number of communities in the
network. A network with two fully connected and completely disjoint
communities has $Q = 0.5$, for example. As such, network modularity can
function as a measure of absolute observed heterogeneity (Figure 6).

\begin{figure}
\centering
\includegraphics[width=0.95\textwidth]{Fig6.pdf}
\caption{Networks and modularity ($Q$) values for {\em Christ on the Cross}, {\em The Baptism}, and the experimental student paintings along the axis of $Q$ from
least to most heterogeneous.}
\end{figure}

HAP informs interpretation of our analyses of \emph{Christ on the Cross}
and \emph{The Baptism}. For the former, our analysis identifies two
communities of regions, but with $Q$ below the threshold generally
accepted to indicate structure (0.231 vs 0.3). The regions in each
community do cluster in space, however, which is worth considering. One
community overlaps Christ's head and torso and surrounding areas.
Conservation files generously provided by the Cleveland Museum of Art
indicate that the figure of Christ (areas overlapped by the community in
yellow in Figure 4) contains inpainting from previous conservation,
which could create a distinguishable signal, though damage to the canvas
is another, more likely culprit. Further mapping of damage and past
conservation efforts will improve our understanding of this work.
However, the results comport with the notion of \emph{Christ on the
Cross} as the work of a single artist with some variation introduced by
artistic practice or possibly early conservation practice.

At first glance, the communities identified in our analysis of El
Greco's \emph{Baptism} appear to roughly correspond to the proposed
attributions by previous authors discussed above. The community in
yellow in Figure 3 overlaps much of the top of the painting, previously
attributed to El Greco, while the community in blue overlaps John the
Baptist and the figure in red, assigned to Jorge Manuel. We also find a
combination of communities in the face and torso of Christ, which was
attributed to both artists. A single, small and particularly unique
community (in orange in Figure 3) overlaps the landscape at the bottom
of the painting and Christ's legs.

Overall, however, the PATCH findings suggest a different story
underlying the existing art historical interpretation. The partition of
\emph{The Baptism} has $Q = 0.341$, which is only slightly above the
threshold for evidence of structure. Nearly a third of all edges in the
network are interconnections between three of the four communities we
have identified. The yellow and green groups have 49 intercommunity
edges, blue and green 30, and blue and yellow 22. If the blue, green and
yellow communities indeed represented three distinct artists working
with the same or similar materials, we would expect the communities to
be more insular and the modularity of the network to be substantially
higher, similar to what we observed in our controlled test. Yet the
modularity of \emph{The Baptism} is much closer to that of \emph{Christ
on the Cross} ($Q = 0.231$), which appears to feature one artist\emph{.}
It is worth noting that, as shown in the analysis of student paintings,
the CNN is particularly proficient at recognizing different-artist pairs
and is more likely to incorrectly label a same-artist pair as different
than the converse. Thus, the intercommunity edges in \emph{The Baptism}
network that survived our initial pruning are not liable to be spurious.

This suggests that some unifying factor connects these three
communities. One possibility is a single artist working with different
brushes. In El Greco's time, brushes were made with either bristles
(which are coarse) or hair (which is fine and soft). Each served
distinct technical purposes and often came in material-specific
shapes\textsuperscript{36}. Variation could arise from using the
hair/bristles of different species in the same type of brush. Key
sources were hog bristles (especially white hog) and ``minever'' (gray
squirrel tail fur)\textsuperscript{37}, though dog and mongoose often
replaced minever due to its rarity\textsuperscript{36,38}. Hog's bristles likewise may have
been replaced with ox\textsuperscript{39}. Such changes could affect
paint application. For example, hog's bristles are naturally flagged
(the ends of the hairs are split), allowing them to pick up more
paint\textsuperscript{40}. Bristle substitution could decrease brush
capacity. Further experimentation to understand the effect of materials
on PATCH analysis is needed.

The evidence could also indicate the work of one artist with changing
style or technique over time. Interestingly, Biannuci and
colleagues\textsuperscript{41} present evidence that El Greco suffered a
series of ischemic events, one in the 1590s and a second in 1608 (the
year the contract for the altarpieces for the Hospital Tavera was
initiated), ``resulting in progressive disabilities with fluctuating
course characterized by temporary improvements and worsening before his
death.'' The regions that were previously identified as the work of
different individual painters could conceivably represent variation in
the master's individual style (however this may have arisen) over the
course of his final years. Investigation of changes in style and
technique and their effect on PATCH analysis will be an important
direction for future research, as well.

PATCH has the capability to make a substantial contribution to research
as a complement to existing art historical methods. The accuracy of the
method has been demonstrated in the validation on student paintings. Our analyses of \emph{Christ on the Cross} and \emph{The Baptism} show the true
potential of PATCH by contributing important new information to art
historical scholarship on the creation of objects in El Greco's workshop
in Toledo.

In the analysis of \emph{The Baptism}, the fact that the orange group,
which overlaps a region known to have been altered after the delivery of
the painting, is so disconnected from the remainder of the communities
speaks to the method's capability of recognizing a unique contribution
or substantial departure from the technique used in the rest of the
painting. That the other three communities roughly correspond to art
historical attributions suggests PATCH possesses a kind of semantic
fidelity\textsuperscript{42}---our method recognizes areas that appear
different to the trained eye of an art historian. However, the
connections between those areas revealed by PATCH call this attribution
into question. Traditional attribution studies rely on the inspection of
painters' unconscious tendencies and individual styles. For example, in
attributing the painting of John the Baptist to El Greco's son,
Wethey\textsuperscript{29} cites the ``hardness'' of the details.
Computer vision allows us to view the surface of a painting at a
different scale. Previous analyses indicate that features as small as
the diameter of a single paint brush bristle (length scales of 0.2--0.4
mm) may be integral to the network's identification of an
artist\textsuperscript{17}. By analyzing spatial correlations at this
microscopic scale, the network may be revealing aspects of the physics
of paint application and of the physiology of hand movements. PATCH adds
micro-scale features to the macro (brushstrokes) and meta-scale
(historical and material) data, allowing for a full-scale analysis and
the discovery of heretofore unseen evidence of the processes by which
paintings were created.

Future research with PATCH will focus on teasing apart the influences of
materials and artists on the heterogeneity of artistic practice.
Expansion of PATCH applications to include \emph{inter-}painting
comparisons and other art historical applications will be important as
well. However, uses of PATCH are not limited to those contexts: it
provides a general-purpose approach to clustering data that can be
encoded in the form of image-like arrays, as occurs in a wide variety of
applications across the social and natural sciences. Underlying the
effectiveness of pairwise assignment training is the power of
convolutional neural networks to find extremely subtle differences among
such data points, and the counter-intuitively useful information gained
when the network consistently fails to succeed in this task. As such,
PATCH can be deployed where supervised learning is impossible (because
the ground truth is unknown) and when alternative unsupervised
approaches are ineffective.

\section*{Acknowledgments}

The authors would like to thank Fundación Casa Ducal de Medinaceli
for generously providing data for The Baptism, and Sr. Juan Manuel
Albendea Solis for his insightful criticism that helped us reevaluate
our conclusions and improve the manuscript. We would also like to thank
the Cleveland Museum of Art for use of the data from Christ on the
Cross, and the Lapis Senior Conservator of Paintings and Head of
Paintings Conservation, Dean Yoder, for his help and feedback. The
authors gratefully acknowledge support from the Expanding Horizons
Initiative of Case Western Research University (CWRU) College of Arts
and Sciences and the Jack, Joseph and Morton Mandel Foundation.
Computational support was provided by the CWRU High Performance
Computing cluster and a workstation kindly donated through the Dell Seed
Program.

%\setlength{\belowcaptionskip}{-10pt}

\bibliographystyle{unsrt}

%\fontsize{9}{11}\selectfont

\begin{thebibliography}{42}

\bibitem{1}
Rand, J. G. Improvement in the Construction of Vessels or Apparatus
for Preserving Paint, \& c. (1841).

\bibitem{2}
Alpers, S. \emph{The Vexations of Art: Velazquez and Others}. (Yale
University Press, New Haven, 2005).

\bibitem{3}
Brown, C. \emph{Van Dyck}. (Ithaca, N.Y.\.: Cornell University Press,
1983).

\bibitem{4}
Talvacchia, B. \emph{Raphael}. (Phaidon Press, New York, 2007).

\bibitem{5}
Bambach, C. \emph{Drawing and Painting in the Italian Renaissance
Workshop}. (Cambridge University Press, Cambridge, UK, 1999).

\bibitem{6}
Brooks, J., Allen, D. \& Salomon, X. F. \emph{Andrea Del Sarto: The
Renaissance Workshop in Action}. (Getty Publications, 2015).

\bibitem{7}
Hoogstraten, S. van. \emph{Inleyding tot de hooge schoole der
schilderkonst: anders de zichtbaere werelt.\,: verdeelt in negen
leerwinkels, yder bestiert door eene der zanggodinnen. Ten hoogsten
noodzakelijk, tot onderwijs, voor alle die deeze edele, vrye, en hooge
konst oeffenen, of met yver zoeken te leeren, of anders eenigzins
beminnen.} (Tot Rotterdam.\,: By Fransois van Hoogstraeten, 1678).

\bibitem{8}
van de Wetering, E. The search for the master's hand: an anachronism?
(A summary). \emph{Int Kongr Kunstgesch} 627--630 (1993).

\bibitem{9}
Hughes, J. M., Graham, D. J. \& Rockmore, D. N. Quantification of
artistic style through sparse coding analysis in the drawings of Pieter
Bruegel the Elder. \emph{Proceedings of the National Academy of
Sciences} \textbf{107}, 1279--1283 (2010).

\bibitem{10}
Friedman, T., Lurie, D. J. \& Mha, A. S. Authentication of
Rembrandt's Self-Portraits through the Use of Facial Aging Analysis.
\emph{Israel Medical Association Journal} \textbf{14}, (2012).

\bibitem{11}
Kogou, S. \emph{et al.} From remote sensing and machine learning to
the history of the Silk Road: large scale material identification on
wall paintings. \emph{Sci. Rep.} \textbf{10}, 19312 (2020).

\bibitem{12}
Assael, Y. \emph{et al.} Restoring and attributing ancient texts
using deep neural networks. \emph{Nature} \textbf{603}, 280--283 (2022).

\bibitem{13}
Winterbottom, T., Leone, A. \& Al Moubayed, N. A deep learning
approach to fight illicit trafficking of antiquities using artefact
instance classification. \emph{Sci. Rep.} \textbf{12}, 13468 (2022).

\bibitem{14}
Sandoval, C., Pirogova, E. \& Lech, M. Two-Stage Deep Learning
Approach to the Classification of Fine-Art Paintings. \emph{IEEE Access}
\textbf{7}, 41770--41781 (2019).

\bibitem{15}
Bigerelle, M. \emph{et al.} Fractal and statistical characterization
of brushstroke on paintings. \emph{Surf. Topogr.: Metrol. Prop.}
\textbf{11}, 015019 (2023).

\bibitem{16}
Elgammal, A., Mazzone, M., Liu, B., Kim, D. \& Elhoseiny, M. The
Shape of Art History in the Eyes of the Machine. Preprint at
https://doi.org/10.48550/arXiv.1801.07729 (2018).

\bibitem{17}
Ji, F. \emph{et al.} Discerning the painter's hand: machine learning
on surface topography. \emph{Heritage Science} \textbf{9}, 152 (2021).

\bibitem{18}
Girvan, M. \& Newman, M. E. J. Community structure in social and
biological networks. \emph{Proceedings of the National Academy of
Sciences} \textbf{99}, 7821--7826 (2002).

\bibitem{19}
Newman, M. E. J. \& Girvan, M. Finding and evaluating community
structure in networks. \emph{Phys. Rev. E} \textbf{69}, 026113 (2004).

\bibitem{20}
Fortunato, S. \& Hric, D. Community detection in networks: A user
guide. \emph{Physics Reports} \textbf{659}, 1--44 (2016).

\bibitem{21}
Blondel, V. D., Guillaume, J.-L., Lambiotte, R. \& Lefebvre, E. Fast
unfolding of communities in large networks. \emph{J. Stat. Mech.}
\textbf{2008}, P10008 (2008).

\bibitem{22}
Bastian, M., Heymann, S. \& Jacomy, M. Gephi: an open source
software for exploring and manipulating networks. (2009).

\bibitem{23}
Hadjinicolaou, N. He Is, Indeed, A Prophet of the Moderns. in
\emph{El Greco of Toledo: Painter of the Visible and the Invisible} (ed.
Marias, F.) (Ediciones El Viso, Madrid, Esp., 2014).

\bibitem{24}
Marias, F. \emph{El Greco: Life and Work - A New History}. (Thames
\& Hudson, London, UK, 2013).

\bibitem{25}
Marias, F. The Greek, Between Invention and History. in \emph{El
Greco of Toledo: Painter of the Visible and the Invisible} (ed. Marias,
F.) (Ediciones El Viso, Madrid, Esp., 2014).

\bibitem{26}
Mann, R. Tradition and Originality in El Greco's Work: His Synthesis
of Byzantine and Renaissance Conceptions of Art. \emph{Quidditas}
\textbf{23}, (2002).

\bibitem{27}
Sanchez Canton, F. J. \emph{Spanish Drawings From the 10th to the
19th Century}. (Shorewood Publishers, New York, NY, 1964).

\bibitem{28}
Krumrine, M. L. Color in El Greco. \emph{Arte Lombarda} 42--48
(1993).

\bibitem{29}
Wethey, H. E. \emph{El Greco and His School}. (Princeton University
Press, Princeton, NJ, 1962).

\bibitem{30}
Scallen, C. B. \emph{Rembrandt, Reputation, and the Practice of
Connoisseurship}. (Amsterdam University Press, 2004).

\bibitem{31}
Lopera, J. A. \emph{El Greco: Estudio y Catalogo}. (Fundacion De
Apoyo a La Historia Del Arte Hispanico, Madrid, 2005).

\bibitem{32}
Cossío, M. B. \emph{El greco}. (V. Suárez, 1908).

\bibitem{33}
Francis, H. A Crucifixion by El Greco. \emph{The Bulletin of the
Cleveland Museum of Art} \textbf{40}, (1953).

\bibitem{34}
Ugail, H., Stork, D. G., Edwards, H., Seward, S. C. \& Brooke, C.
Deep transfer learning for visual analysis and attribution of paintings
by Raphael. \emph{Heritage Science} \textbf{11}, 268 (2023).

\bibitem{35}
Fortunato, S. \& Barthelemy, M. Resolution limit in community
detection. \emph{Proc. Natl. Acad. Sci. U.S.A.} \textbf{104}, 36--41
(2007).

\bibitem{36}
Gettens, R. J. (Rutherford J. \& Stout, G. \emph{Painting
Materials\,: A Short Encyclopaedia}. (New York\,: Dover Publications,
1966).

\bibitem{37}
Cennini, C. \& Herringham, C. J. P. \emph{The Book of the Art of
Cennino Cennini\,: A Contemporary Practical Treatise on Quattrocento
Painting}. (London\,: George Allen \& Unwin, 1922).

\bibitem{38}
Palomino-y-Velasco, A. The Pictorial Museum and Optical Scale. in
\emph{Artists' Techniques in Golden Age Spain: Six Treatises in
Translation} (ed. Veliz, Z.) 141--191 (Cambridge University Press,
Cambridge, 1986).

\bibitem{39}
Fuga, A. \emph{Artists' Techniques and Materials}. (Los Angeles\,:
J. Paul Getty Museum, 2006).

\bibitem{40}
Tuman, M. \emph{The Science of Art \textbar{} Anatomy of a Bristle Brush}.
(2024).

\bibitem{41}
Bianucci, R., Marías Franco, F. \& Appenzeller, O. Historical
evidence supports El Greco's depiction of a neurological condition in
his attributed self-portrait. \emph{Journal of the Neurological
Sciences} \textbf{372}, 316--317 (2017).

\bibitem{42}
Takagi, Y. \& Nishimoto, S. High-resolution image reconstruction
with latent diffusion models from human brain activity.
bioRxiv 2022.11.18.517004, preprint at https://doi.org/10.1101/2022.11.18.517004v3
(2023).

\end{thebibliography}

\newpage
\setcounter{page}{1}
\setcounter{table}{0}
\setcounter{figure}{0}

\renewcommand{\figurename}{\fontfamily{phv}\fontsize{9}{12}\selectfont Figure}
\renewcommand{\thefigure}{S\arabic{figure}}
\renewcommand{\tablename}{\fontfamily{phv}\fontsize{9}{12}\selectfont Table}
\renewcommand{\thetable}{S\arabic{table}}
\renewcommand{\thepage}{S\arabic{page}}

\begin{center}
\LARGE Supplemental Materials:\\
``PATCH: a deep learning method to assess heterogeneity of artistic practice in historical paintings''
\end{center}

\section*{\fontfamily{phv}\selectfont Appendix A: Analytical derivation of the random assignment distribution}

In order to determine whether the network has learned to correctly sort patches of both images, we generated the distribution of correct assignments expected if the network were assigning each patch to a class at random (as though it were simply flipping a coin). The network is trained for 26 folds of 25 epochs each. In each epoch, the network is trained on 70\% of the patches and tested on the remaining 30\% (there are equal numbers of patches from each image). The percentage of this test set that is assigned correctly is called the “validation accuracy.” At the end of each fold, the maximum validation accuracy is recorded, the dataset is resampled, and a new fold begins. After all folds are completed, we have a distribution of 26 maximum validation accuracies that represent how well the network has learned to distinguish between the two images in each of its attempts. The derivation of this expected distribution is as follows:

The goal is to find the distribution $p^{\text{max}}_{n,k}(m)$, the probability that you will see a maximum of $m$ heads over $k$ repetitions (``$k$ epochs'') of an experiment where you flip a fair coin $n$ times. We start with the distribution of finding $m$ heads in a single experiment, which is the binomial distribution for a fair coin,
\begin{equation}
    p_n(m) = 2^{-n} \binom{n}{m},
\end{equation}
where $\binom{n}{m} \equiv \frac{n!}{m!(n-m)!}$ is the binomial coefficient. We also define a tail distribution, which is the cumulative probability that you would find $m$ or more coin flips in a single experiment:
\begin{equation}
    p^{\text{tail}}_n(m) = \sum_{j=m}^{n} p_n(j) = \binom{n}{m} {}_{2}F_1(m,n+1,m+1; -1).
\end{equation}
Here ${}_2 F_1(a,b,c;z)$ is a hypergeometric function, defined by the series:
\begin{equation}
{}_2 F_1(a,b,c;z) = \sum_{j=0}^\infty\frac{ (a)_j (b)_j}{ (c)_j}\frac{z^j}{j!},
\end{equation}
where $(x)_j \equiv x (x+1) \cdots (x+j-1)$ for $j>0$ and $(x)_0 \equiv 1$.  There are other equivalent ways of expressing the tail probability of a binomial distribution (using for example incomplete beta functions), but all of them are similarly complicated.

If we repeat the experiment $k$ times, the probability that every single experiment had fewer than $m$ heads is:
\begin{equation}
    U_{n,k}(m) = (1- p^{\text{tail}}_n(m))^k.
\end{equation}
The difference between consecutive values of this probability, $U_{n,k}(m+1) - U_{n,k}(m)$, is the probability that we would find a maximum exactly equal to $m$ over $k$ repetitions.  Hence we get our final answer:
\begin{equation}
    p^{\text{max}}_{n,k}(m) = U_{n,k}(m+1) - U_{n,k}(m).
\end{equation}
It is easy to check that this probability is properly normalized:
\begin{equation}
    \sum_{m=0}^n p^{\text{max}}_{n,k}(m) = U_{n,k}(n+1) - U_{n,k}(0) = 1.
\end{equation}

In practice, we compared the mean and maximum of the observed distribution to the random assignment distribution derived above. However, the theoretical maximum of the distribution is \textit{m} = \textit{n}. Given enough tries (infinite sampling), a random assignment engine would eventually achieve perfect accuracy. Thus, we developed an empirical “maximum” value based on finite sampling using the image size of our student paintings (180 patches, 360 total, \textit{n} = 108 patches in the test set). We created a simple coin-flipping script that gathered the maximum “validation accuracies” (i.e., percentage of flips that land heads) for 25x108 flips and ran 250,000 iterations of this process. This generated a maximum value of \textit{m} = 80 “correct assignments,” or 74.074\%. The probability of 80 correct assignments is ~0.000004676581 (truncated for clarity). As the distribution is a probability mass function based on discrete events, for tests with larger numbers of patches, we used the number of correct assignments with a probability closest to this probability, even if it was less probable. 

\section*{\fontfamily{phv}\selectfont Appendix B: Hyperparameter Optimization}

The model hyperparameters for the neural network artistic practice discriminator model are optimized with a commercially available Bayesian hyperparameter optimization package built around the tree parzen optimization algorithm\textsuperscript{1}. The optimization loss is based on the model performance after initial training on a restricted subset of the training data, because using the full dataset for this initial optimization step is computationally infeasible. In particular each model was trained for: 6 folds, 10 epochs, and patch size 200 on a reduced student painting dataset consisting of 5 paintings with 4 same artist combinations and 6 different artist combinations. Training loss was calculated as the RMSE distance between the predicted artist similarity and the ground truth, averaged over all painting combinations. The restricted training scheme was employed to reduce the total optimization time on finite computational resources. The considered hyperparameters and ranges are shown in Table S1. The model structure consists of a variable number of dense, fully-connected layers following the VGG16 model, where the number of nodes per layer decreases linearly from a variable maximum at the first dense layer to 2 at the output layer. Maximum posterior estimation was performed on the generated set of all trained models (130 instances) to predict the optimal hyperparameter set. Final optimization of the model shown in the manuscript utilized a full training scheme on the full dataset, as described in the manuscript, and is performed by hand. This is necessary since the full training scheme leads to unreasonably long convergence time in the optimization algorithm. The final hyperparameters align closely with the maximum posterior estimation result, as shown in Figure S1. Note that the largest deviation between the optimized and maximum posterior hyperparameters occurs for the learning rate, which is expected since the longer number of epochs in the full training process allows the model to converge with a smaller learning rate.

\begin{table}
\fontfamily{phv}\selectfont
\renewcommand{\arraystretch}{1.05}
\centering
\begin{tabular}{|l|c|c|c|c|}
\hline
\textbf{Hyperparameter}&\textbf{Min search val.}&\textbf{Max search val.}&\textbf{Max posterior}&\textbf{Final optimization}\\
\hline
Learning rate&0.0001&0.02&0.0025&0.0001\\
Dense layers&1&4&1&1\\
Max nodes&25&200&97&64\\
Dropout rate&0.0&0.25&0.20&0.25\\
L2 reg. rate&0.0001&0.13&0.0007&0.001\\ 
\hline
\end{tabular}
\caption{\fontfamily{phv}\selectfont Optimization ranges for the searched hyperparameters.}
\label{table:S1}

\end{table}

\begin{figure}[] \centering \includegraphics[width=\textwidth]{FigS1.pdf}
\caption{Results of the hyperparameter optimization on a reduced dataset compared to the final model trained on the full dataset, shown in the manuscript. The posterior distribution over all optimization training is shown for each hyperparameter (blue curve), with the maximum posterior from the optimization process (blue dashed line) and the final model hyperparameters (red dashed line) indicated for comparison. The final model hyperparameters align closely with the optimization maximum posterior in most cases. The distribution of the loss for all trained model instances in the optimization is shown in the bottom right panel.}
\end{figure}

\newpage
\section*{\fontfamily{phv}\selectfont Appendix C: Comparison to alternative methods}

\noindent\textit{Surface roughness method}\\

Pairwise comparison of large numbers of regions via ML is computationally expensive, so a simpler method would be preferable if it were equally accurate. One simple way of distinguishing between regions of a painting using topography would be to compare the “roughness” of the surfaces. We applied a roughness-based method to the student paintings to determine its accuracy relative to the PATCH method. We used the standard deviation of height values as a measure of roughness, and calculated the standard deviation of height values for each 1x1 cm$^2$ patch of each student painting . We then compared paintings pairwise using the pairwise Wilcoxon rank sum test implemented in R using the stats package. A significant result on the pairwise Wilcoxon test means the distributions (specifically the medians) are significantly different, which in this context would indicate a different-artist pair. Thus, same-artist pairs would be tests where the null hypothesis could not be rejected. Using the Bonferroni correction for multiple tests with $\alpha$=0.01, any pairwise comparison with a \textit{p}-value greater than 0.00003 was considered a same-artist pair. This method had an overall accuracy of 81.0\%,  identifying 19 of 23 same-artist pairs (82.6\%) and 220 of 277 different artist pairs (79.4\%), a significant reduction in accuracy relative to PATCH. This translates to F$_1$ scores of 0.384 for same-artist pairs and 0.878 for different-artist pairs, an average of 0.631 (Table S2). We conclude that PATCH substantially outperforms roughness for identification of painters. This also indicates that the network is not simply sorting patches of images based on their roughness.

\begin{table}[htb]
\renewcommand{\arraystretch}{1.05}
\centering
\fontfamily{phv}\selectfont
\begin{tabular}{|l|c|c|c|}
\hline
 &\textbf{Precision}&\textbf{Recall}&\textbf{F$_1$}\\
\hline
\textbf{Same-artist pairs}&0.250&0.826&0.384\\
\textbf{Different-artist pairs}&0.982&0.794&0.878\\
\textbf{\em Average}&\textit{0.616}&\textit{0.810}&\textit{0.631}\\
\hline
\end{tabular}

\caption{Precision, recall, and overall F$_1$ for the surface roughness method.}
\label{table:S2}

\end{table}

\noindent\textit{Unsupervised method}\\


As described in the main text, PATCH achieves unsupervised ends by supervised means. How would PATCH fair against fully unsupervised learning on the surface topography, using clustering of dimension-reduced feature vectors for each patch? To facilitate this comparison, we developed an unsupervised pairwise region analysis method, in the spirit of the PATCH approach. We used the student paintings for validation, dividing them up into 1x1 cm$^2$ (200 x 200 px) patches, giving 180 patches per painting. We then took every possible pair of paintings, to determine whether each pair could be distinguished (marked as “different artist”) or not (marked as “same artist”). For a given pair of paintings (360 patches total) we ran each patch through the following unsupervised workflow:

\begin{itemize}

\item {\em Feature extraction:} each patch was converted to a feature vector using a 2D Fourier transform, which created a Fourier domain 200x200 patch with the origin in the center. We took the absolute values of the upper half of the domain (the bottom was identical by symmetry), leaving out the 13 central columns. These columns correspond to the largest wavelengths (comparable to the patch size), and deemphasizing them puts the focus on brushstroke details at smaller wavelengths. We flattened the absolute values into a vector of size 17578. Other parameter choices gave comparable or worse final performance. We also tried an alternative feature extractor based on the VGG16 convolutional neural network (CNN) that had been trained to classify height data in our earlier student painting controlled experiment\textsuperscript{2}, with the bottom layers of the network cut out. This performed worse than the Fourier feature extractor, so we stayed with the latter.

\item {\em Dimension reduction:} to allow eventual clustering of the feature vectors, we used the UMAP dimension reduction algorithm\textsuperscript{3}, with default parameters, to map the full feature vectors onto 4-dimensional vectors.

\item {\em Clustering:} Finally, we used the flat clustering module of HDBSCAN\textsuperscript{4} (minimum cluster size = 25, cluster selection method = ‘leaf’) to force the algorithm to group the 4-dimensional vectors into two clusters. 

\end{itemize}

If there was a statistically significant difference in the numbers of patches assigned to each cluster among the two paintings (chi-squared test with \textit{p} $<$ 0.05), then the pair of paintings were marked as different artist. For example, if the labeled patches in one painting were mostly assigned to cluster 1, and the labeled patches in the other painting to cluster 2, this would be taken as evidence that different artists were involved. Otherwise (i.e. if cluster assignments were fairly evenly distributed among the two paintings) the pair was marked as same artist. The algorithm occasionally failed to find two clusters at all, which we also treated as evidence that the pair was made by the same artist. The final results were quite similar to the surface roughness approach described above: the method correctly identified 18 out of 23 same-artist pairs (78.3\%), and 225 out of 277 different-artist pairs (81.2\%), yielding F$_1$ scores of 0.387 and 0.888 for the same-artist and different-artist cases respectively. Changing the HDBSCAN parameters like minimum cluster size and the cluster selection method did not improve the results, and in all cases the approach performed significantly worse than the PATCH method.


\begin{table}[htb]
\centering
\renewcommand{\arraystretch}{1.05}
\fontfamily{phv}\selectfont
\begin{tabular}{|l|c|c|c|} 
\hline
 &\textbf{Precision}&\textbf{Recall} & \textbf{F$_1$}\\
\hline
\textbf{Same-artist pairs}&0.257&0.783&0.387\\
\textbf{Different-artist pairs}&0.978&0.812&0.888\\
\textbf{\em Average}&\textit{0.618}&\textit{0.797}&\textit{0.637}\\ 
\hline
\end{tabular}
\caption{Precision, recall, and overall F$_1$ for the unsupervised method.}
\label{table:S32}
\end{table}





\newpage
\section*{\fontfamily{phv}\selectfont Appendix D: Additional Figures}

\begin{figure}[htb] \centering \includegraphics[scale=1.65]{FigS2.png}
\caption{Art historical attributions of portions of the painting. Areas attributed to El Greco are overlaid in purple, Jorge Manuel in gold, both artists in red, and a third hand or hands in light blue. }
\end{figure}
\clearpage
\makeatletter
\setlength{\@fptop}{0pt}
\makeatother
\begin{figure}[] \centering \includegraphics[width=0.8\textwidth]{FigS3.png}
\caption{Network diagram for the student paintings with spurious edges trimmed. Nine disjoint communities are identified, corresponding to the individual artists.}
\end{figure}

\renewcommand{\refname}{Supplementary References}

\begin{thebibliography}{4}

\bibitem{1}
Bergstra, J., Komer, B., Eliasmith, C., Yamins, D. \& Cox, D. D. Hyperopt: a Python library for model selection and hyperparameter optimization. Comput. Sci. Discov. 8, 014008 (2015).

\bibitem{2}
Ji, F. et al. Discerning the painter’s hand: machine learning on surface topography. Heritage Science 9, 152 (2021).

\bibitem{3}
McInnes, L., Healy, J., Saul, N. \& Großberger, L. UMAP: Uniform Manifold Approximation and Projection. Journal of Open Source Software 3, 861 (2018).

\bibitem{4}
McInnes, L., Healy, J. \& Astels, S. HDBSCAN: Hierarchical density based clustering. Journal of Open Source Software 2, 205 (2017).

\end{thebibliography}

\end{document}

