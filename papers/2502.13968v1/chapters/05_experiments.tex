\section{Experiments} 





\par
\noindent 
\textbf{Metrics:} We report novel-view synthesis evaluations on commonly used metrics such as peak signal-to-noise ratio (PSNR), structural similarity index (SSIM) and learned perceptual image patch similarity (LPIPS).
We use bi-directional Chamfer distance to evaluate the 3D reconstruction quality. 
\par
As our method is agnostic to the object types, we demonstrate its effectiveness on four kinds of interactions: Human-object, hand-object, and object-object and human-human interaction. 
For the hand-object and object-object scenarios, we use scenes from the AffordPose~\cite{affordpose} and WildRGB-D~\cite{xia2024rgbd} datasets, respectively.
AffordPose is a synthetic dataset with ground-truth geometry. 
We use these ground-truth meshes to generate a synthetic multi-view dataset of 60 views for six objects. 
Human-object interaction is especially challenging due to the large differences between the scales of the two entities. 
To evaluate our method comprehensively, we capture and record a new dataset with various human-object interactions with objects of different scales and complexity (further details in supplementary \cref{sec:human_object_dataset}) and also evaluate our method on a few scenes of human-human interaction from ReMoCap\cite{ghosh2024remos} dataset.
The evaluation datasets also differ in the relative scales of the objects in the scene. 
For example, the human-object evaluation dataset consists of small scale objects in a large capture dome, as opposed to the Wild-RGBD dataset.
\par 
\noindent \textbf{Comparisons:} We compare our method against ObjectSDF++~\cite{Wu2023objectsdfplus} and ``Segmented Neus2''. 
As NeuS2~\cite{neus2} is not designed to be instance-specific, we train it separately for each object in the scene by providing the corresponding segmentation masks (henceforth referred to as ``Segmented NeuS2''). 
ObjectSDF++, on the other hand, is a state-of-the-art method that reconstructs objects in a scene separately.%

For human-object, human-human and object-object datasets, we evaluate 3D reconstruction quality for the overall scene. 
As we do not have a ground truth here, we consider a NeuS2~\cite{neus2} model trained on the entire scene (agnostic of the objects) as the pseudo ground truth. 
This allows us to compare the compositionally reconstructed scene with the non-compositional reconstruction of the scene, which can be treated as an upper bound on the reconstruction quality.
In the case of human-object interaction, we also evaluate the object reconstruction separately with respect to 3D scanned templates--we first align the pre-scanned 3D object shape with the reconstructed mesh (extracted using marching cubes) using the rigid ICP~\cite{icp} algorithm and then compute the Chamfer distance. 
We use the ground-truth meshes provided in AffordPose to compute the Chamfer distance metric for both the hand and the object separately. 


\subsection{Geometric Evaluation}
\begin{figure}
    \centering \includegraphics[width=1.0\linewidth]{figures/reconstruction/human_object_geometric.pdf}
    \caption{Qualitative comparison of the reconstructed geometry. In most scenes, we obtain better geometry, with fewer deformations near the contact regions. Best viewed when zoomed.
    }
    \label{fig:human_object_geometry}
    \vspace{-1em}
\end{figure}
\begin{figure}
    \centering
    \includegraphics[width=\linewidth]{figures/reconstruction/human_human_geometric.pdf}
    \caption{Qualitative comparison of 3D scene reconstructions with human-human interaction along with selected multi-view (MV) input images. 
    Digital zoom recommended. 
    } 
    \label{fig:human_human_geometry}
    \vspace{-1em}
\end{figure}
\begin{figure}
    \centering    
    \includegraphics[width=\linewidth]{figures/reconstruction/object_object_geometric.pdf}
    \caption{Qualitative comparison of reconstruction of scenes involving two objects in proximity, along with samples from the multi-view (MV) input images. 
    Digital zoom recommended. 
    } \label{fig:object_object_geometry}
    \vspace{-1em}
\end{figure}

\begin{table}[t]
\resizebox{\linewidth}{!} {
    \begin{tabular}{|l|c|cc|cc|cc|}
\hline
\textbf{Scene}                         & \textbf{Seq ID} & \multicolumn{2}{c|}{\textbf{Segmented NeuS2}}                 & \multicolumn{2}{c|}{\textbf{ObjectSDF++}}                     & \multicolumn{2}{c|}{\textbf{Ours}}                            \\ \hline
                                       & \textbf{}       & \multicolumn{1}{c|}{\textbf{Overall scene}} & \textbf{Object} & \multicolumn{1}{c|}{\textbf{Overall scene}} & \textbf{Object} & \multicolumn{1}{c|}{\textbf{Overall scene}} & \textbf{Object} \\ \hline
\multirow{2}{*}{\textbf{Box}}          & 1               & \multicolumn{1}{c|}{\textbf{4.86}}          & 26.06           & \multicolumn{1}{c|}{11.08}                  & 10.14           & \multicolumn{1}{c|}{5.65}                   & \textbf{7.9}    \\ \cline{2-8} 
                                       & 2               & \multicolumn{1}{c|}{\textbf{11.18}}         & 42.87           & \multicolumn{1}{c|}{18.96}                  & \textbf{12.62}  & \multicolumn{1}{c|}{11.41}                  & 38.46           \\ \hline
\multirow{2}{*}{\textbf{Book}}         & 1               & \multicolumn{1}{c|}{6.78}                   & -               & \multicolumn{1}{c|}{13.31}                  & 9.13            & \multicolumn{1}{c|}{\textbf{5.35}}          & \textbf{7.33}   \\ \cline{2-8} 
                                       & 2               & \multicolumn{1}{c|}{7.30}                   & -               & \multicolumn{1}{c|}{11.96}                  & 9.35            & \multicolumn{1}{c|}{\textbf{5.95}}          & \textbf{8.28}   \\ \hline
\multirow{2}{*}{\textbf{Birdhouse}}    & 1               & \multicolumn{1}{c|}{4.70}                   & -               & \multicolumn{1}{c|}{9.27}                   & 13.2            & \multicolumn{1}{c|}{\textbf{4.50}}          & \textbf{10.99}  \\ \cline{2-8} 
                                       & 2               & \multicolumn{1}{c|}{4.96}                   & -               & \multicolumn{1}{c|}{8.74}                   & 14.33           & \multicolumn{1}{c|}{\textbf{4.38}}          & \textbf{11.13}  \\ \hline
\multirow{2}{*}{\textbf{Spray Bottle}} & 1               & \multicolumn{1}{c|}{\textbf{3.21}}          & -               & \multicolumn{1}{c|}{9.23}                   & 10.89           & \multicolumn{1}{c|}{4.40}                   & \textbf{7.72}   \\ \cline{2-8} 
                                       & 2               & \multicolumn{1}{c|}{\textbf{3.04}}          & -               & \multicolumn{1}{c|}{9.60}                   & 9.67            & \multicolumn{1}{c|}{4.19}                   & \textbf{7.48}   \\ \hline
\multirow{2}{*}{\textbf{Hanoi Tower}}  & 1               & \multicolumn{1}{c|}{\textbf{4.13}}          & -               & \multicolumn{1}{c|}{10.06}                  & 10.86           & \multicolumn{1}{c|}{4.30}                   & \textbf{8.59}   \\ \cline{2-8} 
                                       & 2               & \multicolumn{1}{c|}{\textbf{3.66}}          & -               & \multicolumn{1}{c|}{9.47}                   & 10.53           & \multicolumn{1}{c|}{4.30}                   & \textbf{8.64}   \\ \hline
\multirow{2}{*}{\textbf{Cupid}}        & 1               & \multicolumn{1}{c|}{8.11}                   & 119.20          & \multicolumn{1}{c|}{36.16}                  & 16.76           & \multicolumn{1}{c|}{\textbf{5.79}}          & \textbf{11.71}  \\ \cline{2-8} 
                                       & 2               & \multicolumn{1}{c|}{6.96}                   & 51.84           & \multicolumn{1}{c|}{18.81}                  & 63.34           & \multicolumn{1}{c|}{\textbf{5.92}}          & \textbf{8.66}   \\ \hline\hline
\textbf{Mean}        &                & \multicolumn{1}{c|}{5.74}                   & 59.99          & \multicolumn{1}{c|}{13.89}                  & 15.90           & \multicolumn{1}{c|}{\textbf{5.51}}          & \textbf{11.40} \\ \hline
\end{tabular}
}
    \caption{
    3D reconstruction accuracy of the \textit{overall scene} and individual \textit{object}, on human-object evaluation dataset, using Chamfer Distance (lower the better).
    }
    \label{table:human_object_geometric}
\end{table}



\textbf{Human-Object Interaction:} Table~\ref{table:human_object_geometric} tabulates the quantitative comparisons of our method with ObjectSDF++ and Segmented Neus2 for the whole scenes and also individual objects separately. 
While we outperform ObjectSDF++ on most scenes, an interesting pattern emerges: 
The \textit{Overall scene} in ~\cref{table:human_object_geometric} shows Segmented NeuS2 achieves consistently lower Chamfer distance for the scenes involving small objects like \textit{Spray Bottle} and \textit{Hanoi Tower}. 
The full-scene results are dominated by the reconstruction of the human. 
However, these results deteriorate 
with a relatively larger object like \textit{Cupid}. 
As the occlusions on the human body grow (due to the larger object size), the Segmented NeuS2 struggles to maintain artefact-free reconstruction. 
Further results in the \textit{Object} in ~\cref{table:human_object_geometric} indicate that indeed, the Segmented NeuS2 does not recover the object geometries in most cases from the supervision using segmented objects alone.
We believe that this is because Neus2 tries to \textit{carve away} regions that are segmented out because of occlusion. This causes conflicting optimisation goals for different camera views depending on whether the object is visible. Since this can be a significant volume relative to the total volume for smaller objects, Neus2 fails to converge.
We show a qualitative comparison of our 3D reconstructions in~\cref{fig:human_object_geometry}.
\begin{table}[t]
\resizebox{\columnwidth}{!}{
\begin{tabular}{|l|ll|ll|ll|}
\hline
\multirow{2}{*}{\textbf{Scene}}    & \multicolumn{2}{c|}{\textbf{Segmented Neus2}}                   & \multicolumn{2}{c|}{\textbf{ObjectSDF++}}             & \multicolumn{2}{c|}{\textbf{Ours}}                    \\ \cline{2-7}
                  & \multicolumn{1}{l|}{\textbf{Hand}} & \textbf{Object} & \multicolumn{1}{l|}{\textbf{Hand}} & \textbf{Object} & \multicolumn{1}{l|}{\textbf{Hand}} & \textbf{Object} \\ \hline
\textbf{Bag}      & \multicolumn{1}{l|}{7.51}       & \gold{4.16}        & \multicolumn{1}{l|}{5.86}       & 11.30        & \multicolumn{1}{l|}{\gold{5.83}}       & 4.62        \\ \hline
\textbf{Bottle}   & \multicolumn{1}{l|}{8.28}       & 8.28        & \multicolumn{1}{l|}{6.28}       & 2.76        & \multicolumn{1}{l|}{\gold{5.40}}       & \gold{1.77}        \\ \hline
\textbf{Earphone} & \multicolumn{1}{l|}{7.18}               &  7.15 & \multicolumn{1}{l|}{5.85}               &  6.18               & \multicolumn{1}{l|}{\gold{5.44}}               & \gold{6.17}                \\ \hline
\textbf{Knife}    & \multicolumn{1}{l|}{6.06}       & 4.10        & \multicolumn{1}{l|}{\gold{5.64}}       & 4.32        & \multicolumn{1}{l|}{5.68}       & \gold{3.24}        \\ \hline
\textbf{Pot}      & \multicolumn{1}{l|}{-}              & 3.13        & \multicolumn{1}{l|}{\gold{6.94}}       & 11.97        & \multicolumn{1}{l|}{7.49}      & \gold{2.79}        \\ \hline
\textbf{Scissors} & \multicolumn{1}{l|}{6.00}       & 28.75        & \multicolumn{1}{l|}{6.52}       & 4.24        & \multicolumn{1}{l|}{\gold{5.42}}       & \gold{4.20}        \\ \hline\hline
\textbf{Mean} & \multicolumn{1}{l|}{7.00}       & 9.26        & \multicolumn{1}{l|}{6.18}       & 6.79        & \multicolumn{1}{l|}{\gold{5.87}}       & \gold{3.80}        \\ \hline
\end{tabular}
}
\caption{
3D reconstruction accuracy for different object sequences on the AffordPose dataset. Chamfer distance (lower the better) is calculated against the ground-truth meshes provided. Segmented Neus2 reconstruction fails for the hand in the \textit{Pot} scene, which is represented with ``-''.} 
\label{table:hand_object_geometry}
\vspace{-1em}
\end{table}
\par
\noindent \textbf{Hand-Object Interaction:} 
We show quantitative and qualitative comparisons in \cref{table:hand_object_geometry} and  \cref{fig:hand_object_geometry} in the supplement, respectively. 
ObjectSDF++ suffers from denting artefacts in the occluded areas, whereas our method generates a smoother surface. 
Both methods are at par when the object is only mildly in contact (as in the case of comparably thin scissors). 
\par
\noindent \textbf{Human-Human Interaction:} We show qualitative and quantitative comparisons in \cref{fig:human_human_geometry} 
and \cref{table:human_human_geometry}, respectively. 
Similarly to the previous section, the Segmented Neus2 reconstructs humans with missing sections, while reconstruction near contact areas in ObjectSDF++ shows severe artefacts. 
Note that for the case of Seq 1---even though numerically Segmented Neus2 appears to be slightly better---we can see in the qualitative results that our method reconstructs the hand near occlusions significantly better. 

\begin{table}[]
\resizebox{\columnwidth}{!}{
\begin{tabular}{|l|c|c|c|}
\hline
\textbf{Seq ID} & \multicolumn{1}{l|}{\textbf{Segmented Neus2}} & \multicolumn{1}{l|}{\textbf{ObjectSDF++}} & \multicolumn{1}{l|}{\textbf{Ours}} \\ \hline
\textbf{1}    & \gold{4.36}                                          & 25.05                                     & 4.73                               \\ \hline
\textbf{2}    & 9.8                                           & 21.14                                     & \gold{6.19}                               \\ \hline
\textbf{3}    & 13.71                                         & 43.26                                     & \gold{6.67}                               \\ \hline
\end{tabular}}
\vspace{-1em}
\caption{Comparison of 3D reconstruction quality for human-human interaction. Chamfer distance metric (lower the better) is calculated against the overall scene reconstructed using Neus2.} 
\label{table:human_human_geometry}
\vspace{-1em}
\end{table}
\par
\noindent \textbf{Object-Object Reconstruction:} The qualitative and quantitative results for two interesting objects are shown in \cref{fig:object_object_geometry} and \cref{table:object_object_geometric}, respectively. 
Our method excels in two cases out of three. 
\begin{table}[]
\resizebox{\columnwidth}{!}{
\begin{tabular}{|l|c|c|c|}
\hline
\textbf{}        & \multicolumn{1}{l|}{\textbf{Segmented Neus2}} & \multicolumn{1}{l|}{\textbf{ObjectSDF++}} & \multicolumn{1}{l|}{\textbf{Ours}} \\ \hline
\textbf{Doll and Box}    & 7.44                                         & 11.88                                    & \gold{4.44}                              \\ \hline
\textbf{Pikachu and Bottle} & 4.11                                       & 3.96                                     & \gold{2.99}                              \\ \hline
\textbf{Apple and Wallet}   & \gold{2.72}                                         & 6.28                                     & 3.83                              \\ \hline
\end{tabular}}
\vspace{-1em}
\caption{Comparison of the 3D reconstruction quality for overall scene, on the WilDRGBD dataset using Chamfer Distance metric (lower the better).}
\label{table:object_object_geometric}
\vspace{-1em}
\end{table}
\subsection{Appearance Evaluation}
\begin{figure}
    \centering    \includegraphics[width=\linewidth]{figures/reconstruction/human_object_appearance.pdf}
    \caption{Qualitative comparison of novel view synthesis on human-object scenes. 
    The results of ObjectSDF++ are blurrier than our rendered views, especially around the object. 
    Digital zoom recommended. 
    } \label{fig:human_object_appearance}
    \vspace{-1em}
\end{figure}
To evaluate the quality of novel view synthesis, we render the entire scenes into a held-out set of views. 
\begin{table}[t]
\resizebox{\columnwidth}{!}{
\begin{tabular}{|l|l|ll|ll|ll|}
\hline
\textbf{Scene}          & \multicolumn{1}{c|}{\shortstack{\textbf{Seq}\\\textbf{ID}}} & \multicolumn{2}{c|}{\textbf{PSNR}$\uparrow$}                                             & \multicolumn{2}{c|}{\textbf{SSIM} $\uparrow$}                                             & \multicolumn{2}{c|}{\textbf{LPIPS} $\downarrow$}                                            \\ \hline
\textbf{}               & \multicolumn{1}{c|}{\textbf{}}       & \multicolumn{1}{c|}{\shortstack{\textbf{Object-}\\\textbf{SDF++}}} & \multicolumn{1}{c|}{\textbf{Ours}} & \multicolumn{1}{c|}{\shortstack{\textbf{Object-}\\\textbf{SDF++}}} & \multicolumn{1}{c|}{\textbf{Ours}} & \multicolumn{1}{c|}{\shortstack{\textbf{Object-}\\\textbf{SDF++}}} & \multicolumn{1}{c|}{\textbf{Ours}} \\ \hline
\multicolumn{1}{|l|}{\textbf{Box}}                             & \multicolumn{1}{c|}{1} & \multicolumn{1}{c|}{27.28} & \multicolumn{1}{c|}{\textbf{30.39}} & \multicolumn{1}{c|}{0.95}          & \multicolumn{1}{c|}{\textbf{0.96}} & \multicolumn{1}{c|}{0.08}          & \multicolumn{1}{c|}{\textbf{0.07}} \\ \hline
\multicolumn{1}{|l|}{\textbf{}}                                & \multicolumn{1}{c|}{2} & \multicolumn{1}{c|}{28.26} & \multicolumn{1}{c|}{\textbf{29.64}} & \multicolumn{1}{c|}{0.95}          & \multicolumn{1}{c|}{\textbf{0.96}} & \multicolumn{1}{c|}{0.08}          & \multicolumn{1}{c|}{\textbf{0.07}} \\ \hline
\multicolumn{1}{|l|}{\multirow{2}{*}{\textbf{Book}}}           & \multicolumn{1}{c|}{1} & \multicolumn{1}{c|}{25.61} & \multicolumn{1}{c|}{\textbf{33.06}} & \multicolumn{1}{c|}{0.95}          & \multicolumn{1}{c|}{\textbf{0.96}} & \multicolumn{1}{c|}{0.08}          & \multicolumn{1}{c|}{\textbf{0.06}} \\ \cline{2-8} 
\multicolumn{1}{|l|}{}                                         & \multicolumn{1}{c|}{2} & \multicolumn{1}{c|}{28.70} & \multicolumn{1}{c|}{\textbf{32.12}} & \multicolumn{1}{c|}{0.95}          & \multicolumn{1}{c|}{\textbf{0.96}} & \multicolumn{1}{c|}{0.08}          & \multicolumn{1}{c|}{\textbf{0.07}} \\ \hline
\multicolumn{1}{|l|}{\multirow{2}{*}{\textbf{Birdhouse}}}      & \multicolumn{1}{c|}{1} & \multicolumn{1}{c|}{29.82} & \multicolumn{1}{c|}{\textbf{33.71}} & \multicolumn{1}{c|}{0.96}          & \multicolumn{1}{c|}{\textbf{0.97}} & \multicolumn{1}{c|}{0.08}          & \multicolumn{1}{c|}{\textbf{0.05}} \\ \cline{2-8} 
\multicolumn{1}{|l|}{}                                         & \multicolumn{1}{c|}{2} & \multicolumn{1}{c|}{26.37} & \multicolumn{1}{c|}{\textbf{32.69}} & \multicolumn{1}{c|}{0.93}          & \multicolumn{1}{c|}{\textbf{0.96}} & \multicolumn{1}{c|}{0.10}          & \multicolumn{1}{c|}{\textbf{0.06}} \\ \hline
\multicolumn{1}{|l|}{\multirow{2}{*}{\textbf{\shortstack{Spray \\ Bottle}}}}   & \multicolumn{1}{c|}{1} & \multicolumn{1}{c|}{29.90} & \multicolumn{1}{c|}{\textbf{32.73}} & \multicolumn{1}{c|}{0.97}          & \multicolumn{1}{c|}{\textbf{0.98}} & \multicolumn{1}{c|}{0.07}          & \multicolumn{1}{c|}{\textbf{0.04}} \\ \cline{2-8} 
\multicolumn{1}{|l|}{}                                         & \multicolumn{1}{c|}{2} & \multicolumn{1}{c|}{26.32} & \multicolumn{1}{c|}{\textbf{36.38}} & \multicolumn{1}{c|}{0.94}          & \multicolumn{1}{c|}{\textbf{0.98}} & \multicolumn{1}{c|}{0.10}          & \multicolumn{1}{c|}{\textbf{0.04}} \\ \hline
\multicolumn{1}{|l|}{\multirow{2}{*}{\textbf{\shortstack{Hanoi \\ Tower}}}} & \multicolumn{1}{c|}{1} & \multicolumn{1}{c|}{27.25} & \multicolumn{1}{c|}{\textbf{30.36}} & \multicolumn{1}{c|}{0.95}          & \multicolumn{1}{c|}{\textbf{0.96}} & \multicolumn{1}{c|}{0.09}          & \multicolumn{1}{c|}{\textbf{0.07}} \\ \cline{2-8} 
\multicolumn{1}{|l|}{}                                         & \multicolumn{1}{c|}{2} & \multicolumn{1}{c|}{26.30} & \multicolumn{1}{c|}{\textbf{29.63}} & \multicolumn{1}{c|}{\textbf{0.96}} & \multicolumn{1}{c|}{\textbf{0.96}} & \multicolumn{1}{c|}{\textbf{0.07}} & \multicolumn{1}{c|}{\textbf{0.07}} \\ \hline
\multicolumn{1}{|l|}{\multirow{2}{*}{\textbf{Cupid}}}          & \multicolumn{1}{c|}{1} & \multicolumn{1}{c|}{29.92} & \multicolumn{1}{c|}{\textbf{34.10}} & \multicolumn{1}{c|}{0.96}          & \multicolumn{1}{c|}{\textbf{0.98}} & \multicolumn{1}{c|}{0.10}          & \multicolumn{1}{c|}{\textbf{0.05}} \\ \cline{2-8} 
\multicolumn{1}{|l|}{}                                         & \multicolumn{1}{c|}{2} & \multicolumn{1}{c|}{30.50} & \multicolumn{1}{c|}{\textbf{34.20}} & \multicolumn{1}{c|}{\textbf{0.97}} & \multicolumn{1}{c|}{\textbf{0.97}} & \multicolumn{1}{c|}{\textbf{0.08}} & \multicolumn{1}{c|}{\textbf{0.08}}          \\ \hline\hline
\multicolumn{1}{|l|}{\textbf{Mean}}                            & \multicolumn{1}{c|}{}  & \multicolumn{1}{c|}{28.02} & \multicolumn{1}{c|}{\textbf{32.42}} & \multicolumn{1}{c|}{0.95}          & \multicolumn{1}{c|}{\textbf{0.97}} & \multicolumn{1}{c|}{0.08}          & \multicolumn{1}{c|}{\textbf{0.06}} \\ \hline
\end{tabular}
}
\vspace{-1em}
\caption{Quantitative comparison of view synthesis on held-out views for the human-object dataset. 
We consistently outperform ObjectSDF++. 
}
\label{table:human_object_appearance}
\vspace{-1em}
\end{table}
We report the human-object novel-view results in~\cref{table:human_object_appearance}.
Again, we achieve consistently better performance than ObjectSDF++; see~\cref{fig:human_object_appearance} for the visualisations. 
One can observe blurring artefacts in ObjectSDF++ renderings, which are especially pronounced around the object.
As it supervises only the individual opacities, we hypothesize that the colour network of Wu \textit{et al.}~\cite{Wu2023objectsdfplus} assigns colours to any residual opacity, which is more likely to exist at the transition boundaries.
We also provide appearance evaluation for human-human interaction scenes in the supplementary \cref{sec:human_human_appearance_suppl}. 


\subsection{Ablations}
\begin{figure}
    \centering
    \includegraphics[width=0.98\linewidth]{figures/others/ablations.pdf}
    \caption[Ablations]{Qualitative comparison (ablations). We observe that the overall scene reconstruction largely remains the same, though the individual object and human reconstruction quality deteriorates because of phantom blobs formed underneath the surface (as highlighted inside the transparent surface) when we have a shared MLP or we do not use the alpha-regularisation.} 
    \label{fig:ablations}
\end{figure}

\begin{figure}
    \centering
    \includegraphics[width=\linewidth]{figures/graphs/ablations_1.pdf}
    \caption[Ablations]{Quantitative evaluation with ablated components shows that having separate MLPs for human and object, and the proposed alpha-regularisation are important for high-quality reconstruction.}
    \label{table:ablations} 
\end{figure}


\begin{figure}
    \centering
    \includegraphics[width=\linewidth]{figures/graphs/ablations_2.pdf}
    \caption[Ablations]{3D IOU to assess the intersection of segmented objects for the ablated components.}
    \label{table:3d_iou_intersection}
\end{figure}


We also perform an ablation study to evaluate the design choices.
In particular, we evaluate the importance of having separate MLPs for the human and the object, instead of a single, shared MLP predicting both the SDFs as done in ObjectSDF++.
We also compare the proposed alpha-regularisation against SDF level regularisation $\sum_p \left(\exp(\frac{\beta}{\lambda_t} \cdot \max(-\Phi_1, 0) \cdot \max(-\Phi_2, 0) \right)$, such that both SDFs are not negative at the same position as mentioned in \cref{sec:losses}. 
The differences in the results are shown in~\cref{fig:ablations}, and the quantitative results are shown in~\cref{table:ablations}.
The overall scene reconstruction largely remains the same, but the individual object and human reconstruction qualities deteriorate because of phantom blobs that get formed underneath the surface of the complimentary SDF when we have shared MLP or we do not use the alpha-regularisation. 
Since these blobs are underneath the surface, they are invisible in renderings, thereby satisfying rendering losses. 
To, demonstrate that the high Chamfer distance is due to the intersecting phantom blobs, we also show results by calculating the 3D IOU between the human and the object. 
Ideally, we do not want any penetrations, hence the IOU should be as low as possible. 
Thus, we can see that having separate MLPs for humans and objects and the proposed alpha-regularisation are important for high-quality reconstruction.


