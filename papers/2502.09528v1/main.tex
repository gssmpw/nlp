\documentclass[tinyml]{acmart}

\AtBeginDocument{%
  \providecommand\BibTeX{{%
    \normalfont B\kern-0.5em{\scshape i\kern-0.25em b}\kern-0.8em\TeX}}}

\setcopyright{rightsretained}
\copyrightyear{2025}
\acmYear{2025}

%%%%%%%%%%%%%%%%%%%%%%%%%%%%%%%%%%%%
% Packages that Jack added for misc stuff
%%%%%%%%%%%%%%%%%%%%%%%%%%%%%%%%%%%%
\usepackage{multirow}
\usepackage{enumitem}
\usepackage{url}
\usepackage{soul}

\begin{document}

\def\projname{SteROI-D}
\title{\projname: System Design and Mapping for Stereo Depth Inference on Regions of Interest}

\author{Jack Erhardt}
\email{erharj@umich.edu}
\affiliation{%
    \institution{University of Michigan}
    \city{Ann Arbor}
    \state{Michigan}
    \country{USA}
}

\author{Ziang Li}
\email{ziangli@umich.edu}
\affiliation{%
    \institution{University of Michigan}
    \city{Ann Arbor}
    \state{Michigan}
    \country{USA}
}

\author{Reid Pinkham}
\email{pinkhamr@meta.com}
\affiliation{%
    \institution{Reality Labs - Research}
    \city{Redmond}
    \state{Washington}
    \country{USA}
}

\author{Andrew Berkovich}
\email{andrew.berkovich@meta.com}
\affiliation{%
    \institution{Reality Labs - Research}
    \city{Redmond}
    \state{Washington}
    \country{USA}
}

\author{Zhengya Zhang}
\email{zhengya@umich.edu}
\affiliation{%
    \institution{University of Michigan}
    \city{Ann Arbor}
    \state{Michigan}
    \country{USA}
}

\renewcommand{\shortauthors}{Erhardt et al.}

\newcommand{\je}[1]{\textcolor{black}{#1}}
\newcommand{\zz}[1]{\textcolor{black}{#1}}
\newcommand{\ca}[1]{\textcolor{black}{#1}}

\begin{abstract}
    Machine learning algorithms have enabled high quality stereo depth estimation to run on Augmented and Virtual Reality (AR/VR) devices.
    However, high energy consumption across the full image processing stack prevents stereo depth algorithms from running effectively on battery-limited devices.
    This paper introduces \textit{\projname{}}, a full stereo depth system paired with a mapping methodology.
    \projname{} exploits Region-of-Interest (ROI) and temporal sparsity at the system level to save energy. \projname{}'s flexible and heterogeneous compute fabric supports diverse ROIs. Importantly, we introduce a systematic mapping methodology to effectively handle dynamic ROIs, thereby maximizing energy savings.
    Using these techniques, our 28nm prototype \projname{} design achieves up to 4.35$\times$ reduction in total system energy compared to a baseline ASIC.
\end{abstract}

\keywords{Augmented Reality, Low Power Computing, Hardware-Software Co-Design}

\maketitle

\IEEEPARstart{H}{yperspectral} image (HSI) can record the spectral characteristics of ground objects~\cite{6555921}. As a key technology of HSI processing, HSI classification is aimed at assigning a unique category label to each pixel based on the spectral and spatial characteristics of this pixel~\cite{FPGA,10078841,10696913,10167502}, which is widely used in agriculture~\cite{WHU-Hi}, forest~\cite{ITreeDet}, city~\cite{WANG2022113058}, ocean~\cite{WHU-Hi} studies and so on.

The existing HSI classifiers~\cite{FPGA,10325566,10047983,9573256} typically assume the closed-set setting, where all HSI pixels are presumed to belong to one of the \textit{known} classes. However, due to the practical limitations of field investigations across wide geographical areas and the high annotation costs associated with the limited availability of domain experts, it is inevitable to have outliers in the vast study area~\cite{MDL4OW,Fang_OpenSet,Kang_OpenSet}. These outliers do not belong to any known classes and will be referred as \textit{unknown} classes hereafter. A classifier based on closed-set assumption will misclassify the unknown class as one of the known classes. For example, in the University of Pavia HSI dataset (Fig.~\ref{fig:open_set_example}), objects such as vehicles, buildings with red roofs, carports, and swimming pools are ignored from the original annotations~\cite{MDL4OW}. These objects are misclassified as one of predefined known classes.

\begin{figure}[!t]
    \centering
    \includegraphics[width=0.98\columnwidth]{example_paviau.png}
    \caption{Comparison of classification results between closed-set based classifier and open-set based classifier for the University of Pavia dataset. The dataset originally contains nine \textit{known} land cover classes, however, significant misclassifications occur in the \textit{unknown} classes in closed-set based results. For instance, these unknown buildings with red roofs are misclassified as Bare S., Meadows, and other known materials by closed-set based classifier~\cite{FPGA}. Note that there is a significant overlap in the distribution of spectral curves between known and unknown classes in HSI datasets, which poses a major problem to open-set HSI classification.}
    \label{fig:open_set_example}
\end{figure}

Open-set classification (Fig.~\ref{fig:open_set_example}), as a critical task for safely deploying models in real-world scenarios, addresses the above problem by accurately classifying known class samples and rejecting unknown class outliers~\cite{OpenMax,MDL4OW,Fang_OpenSet}. Moreover, the recent advanced researches have explored training with an auxiliary unknown classes dataset to regularize the classifiers to produce lower confidence~\cite{Entropy,WOODS} or higher energies~\cite{Energy} on these unknown classes samples.

Despite its promise, there are some limitations when open-set classification meets HSI. First, the limited number of training samples, combined with significant spectral overlap between known and unknown classes (see Fig.~\ref{fig:open_set_example}), causes the classifier to overfit on the training samples. Second, the distribution of the auxiliary unknown classes dataset may not align well with the distribution of real-world unknown classes, potentially leading to the misclassification of the test-time data. Finally, it is labor-intensive to ensure the collected extra unknown classes dataset does not overlap with the known classes.

To mitigate these limitations, this paper leverages unlabeled ``in-the-wild'' hyperspectral data (referred to as ``wild data''), which can be collected \textit{freely} during deploying HSI classifiers in the open real-world environments, and has been largely neglected for open-set HSI classification purposes. Such data is abundant, has a better match to the test-time distribution than the collected auxiliary unknown classes dataset, and does not require any annotation workloads. Moreover, the information about unknown classes stored in the wild data can be leveraged to promote the rejection of unknown classes in the case of spectral overlap. While leveraging wild data naturally suits open-set HSI classification, it also poses a unique challenge: wild data is not pure and consists of both known and unknown classes. This challenge originates from the marginal distribution of wild data, which can be modeled by the Huber contamination model~\cite{Huber}:
\begin{equation}
    \mathbb{P}_{wild}=\pi\mathbb{P}_{k}+(1-\pi)\mathbb{P}_{u},
    \label{eq:huber_contamination_model}
\end{equation}
where $\mathbb{P}_{k}$ and $\mathbb{P}_{u}$ represent the distributions of known and unknown classes, respectively. Here, $\pi=\pi_{1}+\dots+\pi_{C}$, and $\pi_{c}$ refers to the probability (or class prior~\cite{DistPU}) of the known class $c \in [1,C]$ in $\mathbb{P}_{wild}$.
The known component of wild data acts as noise, potentially disrupting the training process (further analysis can be found in Section~\ref{sec:Methodology}). 

\begin{center}
    \fbox{\begin{minipage}{23em}
        This paper aims to propose a novel framework---\textit{HOpenCls}---to effectively leverage wild data for open-set HSI classification. Wild data is easily available as it's naturally generated during classifier deployment in real-world environments. This framework can be regarded as training open-set HSI classifiers in their \textit{living environments}.
    \end{minipage}}
\end{center}

To handle the lack of ``clean'' unknown classes datasets, the insight of this paper is to formulate a positive-unlabeled (PU) learning problem~\cite{DistPU,T-HOneCls} in the rejection of unknown classes: learning a binary classifier to classify positive (known) and negative (unknown) classes only from positive and unlabeled (wild) data. What's more, the high intra-class variance of positive class and the high class prior of positive class are potential factors that limit the ability of PU learning methods to address unknown class rejection task. To overcome these limitations, the multi-label strategy is introduced to the \textit{HOpenCls} to decouple the original unknown classes rejection task into multiple sub-PU learning tasks, where the $c$-th sub-PU learning task is responsible for classifying the known class $c$ against all other classes. Compared to the original unknown classes rejection task, each sub-PU learning task exhibits reduced intra-class variance and class prior in the positive class.

Beyond the mathematical reformulation, a key contribution of this paper is a novel PU learning method inspired by the abnormal gradient weights found in wild data. First of all, when the auxiliary unknown classes dataset is replaced by the wild data, this paper demonstrates that the adverse effects impeding the rejection of unknown classes originate from the larger gradient weights associated with the component of known classes in the wild data. Therefore, a gradient contraction (Grad-C) module is designed to reduce the gradient weights associated with all training wild data, and then, the gradient weights of wild unknown samples are recovered by the gradient expansion (Grad-E) module to enhance the fitting capability of the classifier. Compared to other PU learning methods~\cite{nnPU,DistPU,PUET,HOneCls}, the combination of Grad-C and Grad-E modules provides the capability to reject unknown classes in a class prior-free manner. Given the spectral overlap characteristics in HSI, estimating class priors for each known class is highly challenging~\cite{T-HOneCls}, and the class prior-free PU learning method is more suitable for open-set HSI classification.

Extensive experiments have been conducted to evaluate the proposed \textit{HOpenCls}. For thorough comparison, two groups of methods are compared: (1) trained with only $\mathbb{P}_{k}$ data, and (2) trained with both $\mathbb{P}_{k}$ data and an additional dataset. The experimental results demonstrate that the proposed framework substantially enhances the classifier's ability to reject unknown classes, leading to a marked improvement in open-set HSI classification performance. Taking the challenging WHU-Hi-HongHu dataset as an example, \textit{HOpenCls} boosts the overall accuracy in open-set classification (Open OA) by 8.20\% compared to the strongest baseline, with significantly improving the metric of unknown classes rejection (F1\textsuperscript{U}) by 38.91\%. The key contributions of this paper can be summarized as follows:
\begin{itemize}
    \item[1)] This paper proposes a novel framework, \textit{HOpenCls}, for open-set HSI classification, designed to effectively leverage wild data. To the best of our knowledge, this paper pioneers the exploration of PU learning for open-set HSI classification.
    \item[2)] The multi-PU head is designed to incorporate the multi-label strategy into \textit{HOpenCls}, decoupling the original unknown classes rejection task into multiple sub-PU learning tasks. As demonstrated in the experimental section, the multi-PU strategy is crucial for bridging PU learning with open-set HSI classification.
    \item[3)] The Grad-C and Grad-E modules, derived from the theoretical analysis of abnormal gradient weights, are proposed for the rejection of unknown classes. The combination of these modules forms a novel class prior-free PU learning method.
    \item[4)] Extensive comparisons and ablations are conducted across: (1) a diverse range of datasets, and (2) varying assumptions about the relationship between the auxiliary dataset distribution and the test-time distribution. The proposed \textit{HOpenCls} achieves state-of-the-art performance, demonstrating significant improvements over existing methods.
\end{itemize}
\section{Background}\label{sec:background}

\subsection{\je{Low Power Algorithms}}
Stereo depth processing consumes significant energy on AR/VR platforms.
For instance, on a Jetson Orin Nano, we measure stereo depth on a 90k pixel crop at 30~FPS consumes 5.6~W of power, or 400~mJ per inference.
% This energy can be explained by examining the complexity of recent stereo depth models, shown in Fig.~\ref{fig:ops_compare}.
% While multiply-accumulate operations decrease in recent models, stereo-depth specific non-parameterized layers have increased in size, raising the floor of the overall computational complexity and energy consumption.
As AR/VR devices employ higher resolution sensors to achieve more immersive experiences, it is anticipated that the computational intensity will escalate even more.

\subsection{Stereo Depth Processing}
Stereo depth has been the focus of many algorithmic works; as of writing, deep learning based approaches achieve the best inference quality. 
StereoNet \cite{stereonet} is an early attempt at an algorithm that can be accelerated on edge hardware in real time, and shares many foundational traits with subsequent networks. 
It uses twin Siamese feature extraction layers to initialize multi-resolution disparity estimates, which are hierarchically refined to produce a final disparity estimate. 
HITNet \cite{hitnet} iterates on this algorithm structure by introducing tile based iterative refinement, and local slant predictions alongside disparity estimation, to improve inference quality. 
This network also performs disparity processing without explicitly evaluating a 3D cost volume. 
More recently, the monocular depth network Tiefenrausch and it's stereo depth cousin Argos \cite{tiefenrausch} have been proposed using building blocks inspired by MobileNetv2 \cite{mobilenetv2}, and trained on 8-bit quantized weights for highly efficient inference.

% One strategy to reduce computational load \je{of ML workloads} is to exploit Regions of Interest (ROIs) \je{\cite{eyecod}}.
% In % typical stereo depth, 
% \je{many AR} applications, only specific %sections of 
% \je{objects in} an image are of interest\je{; bounding boxes around these objects can be used as ROIs.}
% Fig.~\ref{fig:roi_distribution} illustrates the typical sizes of bounding boxes across various object classes in %the 
% \je{egocentric datasets,} KITTI \cite{kitti2012, kitti2015} and Epic Kitchens \cite{epic_kitchens}. %datasets. 
% \je{Typical ROI sizes often multiple orders of magnitude smaller than the full image resolution. }
% Unlike classification CNNs, stereo depth models can handle variable-sized ROIs.
% This is because they are designed for regression tasks, producing output with the same spatial dimensions as the input images.
% % With typical ROI sizes often multiple orders of magnitude smaller than the full image resolution, significant processing savings are possible.
% While this approach can degrade the quality of resulting depth maps, we find that this degradation is acceptable for certain tasks, particularly in Augmented Reality where limited portions of sensor data are of interest\je{; thus, significant processing savings are possible with this approach.}

% While ROIs can reduce stereo depth processing, %their extraction
% \je{finding them} can introduce overhead costs.
% Extracting ROIs requires object detection \cite{redmon_yolov3_2018}, which demands comparable MACs and weight storage to stereo depth processing itself \cite{hitnet, tiefenrausch}.
% Minimizing the overhead of ROI extraction is critical to reduce the loss in efficiency.
% \je{One approach to combat this inefficiency is to interleave expensive object detection with fast and efficient object tracking \cite{asu_fpga}, such as a correlation filter \cite{correlation_filter}.}
% \je{Such algorithms have been demonstrated to be efficient on low power platforms \cite{vota}.}
% \je{For egocentric tasks, where objects move continuously with respect to the observer, such an approach can accurately track objects with greatly reduced computational cost.}

\subsection{\je{AR Systems and Compute}}
\je{AR platforms incorporate many system and architectural techniques to achieve low power, low latency, and tight form factors.}
\je{Typical AR Systems on Chip (SoCs), such as the Qualcomm Snapdragon \cite{qualcomm_soc}, feature heterogeneously integrated accelerator, CPU, GPU, and memory units for diverse tasks.}
\je{Near-sensor computing \cite{eyecod, ansa, gomez_distributed_2022, sony_coprocessor, siracusa} is also commonly proposed as an energy-efficient computing technique for AR.}
\je{This is because near-sensor processing can be used to reduce raw sensor data sizes, saving expensive communication energy over MIPI or other protocols.}

\je{Several accelerator designs have been proposed for AR SoCs.}
\je{These accelerators must enable efficient inference at low batch sizes, while providing real-time latency, and small core areas to satisfy the stringent space limitations of AR platforms.}
\ca{For this work, we use a baseline real-time throughput and latency requirement of 30FPS, which is common for off-the-shelf image sensors and applications.}
\je{In particular, we consider the architecture proposed in ANSA \cite{ansa}, which enables efficient real-time processing with Vector-Matrix Multiplier (VMM) based compute units.}
\je{This architecture is also organized hierarchically and with parameterized scale, which enables design space exploration.}

\je{While this accelerator design works well for classification-based CNNs, stereo depth networks present new compute morphologies which must be addressed.}
\je{Modern stereo-depth models increasingly utilize non-parameterized special layer types, as illustrated in Figure ~\ref{fig:ops_compare}.}
\je{These operations cannot leverage VMM parallelism, and thus present a potential latency bottleneck.}
\je{The large network weights in these networks also make DRAM I/O a potential bottleneck when storing weights in off-accelerator DRAMs.}

% --- JACK --- My goal with this section is to establish (1) what typical AR platform SoCs look like (heterogeneous accelerators and some local memory) and (2) to establish on or near-sensor co-processors as a thing that AR platforms are either actively doing, or are interested in doing. This way, when I say we will be operating in this framework, it is coming from somewhere.
% I can also use this section to establish the drivers for latency, power, and form factor constraints.

% \begin{figure*}
%     \centering
%     \includegraphics[width=.8\linewidth]{Figures/System_2024_06_03.pdf}
%     \caption{\projname{} system design.}
%     \label{fig:steroi_compare}
% \end{figure*}

% \textbf{\projname{} System Design}:
% The system-level design of \projname{} is depicted in Fig.~\ref{fig:steroi_compare}, while the processing pipeline is illustrated in Fig.~\ref{fig:alg_design}.
% \projname{} uses a detect-and-track strategy for energy-efficient ROI extraction.
% Since the compute complexity of object tracking is significantly less than object detection, \projname{} employs only periodic object detection to detect the objects (ROIs) and uses lightweight object tracking to efficiently follow these ROIs.
% The large difference in computing demands between object tracking and object detection means that we can achieve energy savings even when object detection is run frequently.
% \projname{} incorporates a small level-1 (L1) processor co-packaged with each image sensor in the AR/VR system for efficient object tracking using correlation filters \cite{correlation_filter, vota}. 
% This approach significantly reduces the need for costly object detection, which is performed only occasionally using the level-2 (L2) processor.
% Aside from occasional object detection, the L2 processor is responsible for ROI-based stereo depth processing, which is the most significant challenge.

% \subsection{Computing on Dynamic ROIs}
% The ROI-based stereo depth processing presents a demanding computational workload due to the variability in ROI sizes.
% A highly flexible system is required to accommodate this variability.
% However, mapping this dynamic workload, with its continuous range of ROI sizes, onto a flexible system to achieve optimal energy savings while satisfying the latency and area constraints of AR/VR systems is a significant challenge.
% Traditionally, a static design space exploration can identify the optimal mapping. 
% However, in our application, the ROI and by extension the workload change in runtime; and runtime design space exploration would be impractical.
% Without an optimal mapping strategy, it would be impossible to realize the energy efficiency potential offered by this ROI-based approach.

% \begin{figure}
%     \centering
%     \includegraphics[width=.9\linewidth]{Figures/Algorithm_2024_07_04.pdf}
%     \caption{Illustration of the \projname{} processing pipeline. Object detection is run on the L2 processor and run infrequently; object tracking is run on intermediate frames on the L1 processors.}
%     \label{fig:alg_design}
% \end{figure}

% \textbf{\projname{} L2 Processor Mapping}:
% The \projname{} L2 processor is our main focus, as it is responsible for the challenging ROI-based stereo depth processing.
% The L2 processor is designed to support variable-size ROI-based stereo depth processing
% (Section~\ref{sec:architecture}).
% Importantly, the L2 processor is co-designed with workload mapping using an efficient design space exploration (Section ~\ref{sec:mapping}).
% This approach involves three phases: indexing the hyperdimensional mapping space using a set of axes, ROI binning to facilitate runtime mapping, and iterations of system parameter sweeps.
% This approach enables a complete system-mapping co-design that achieves optimal energy efficiency.

\subsection{\je{Mapping and Dynamic ROIs}}
\je{A final design challenge lies in mapping algorithms running on dynamic ROIs to hardware.}
\je{As established in \cite{ansa}, mapping networks to compute can significantly effect energy efficiency and latency.}
\je{However, processing runtime-dynamic ROIs complicates the generation of these mappings.}
\je{As the range of ROI sizes to be supported is very large, storing a mapping for every possible size is impractical, and therefore generating these mappings entirely offline is imfeasible.}
\je{Simultaneously, the modeling and optimization needed to generate these mappings also prohibits entirely online mapping.}
\je{An intermediate solution is necessary to realize ROI-based stereo depth processing.}

% \je{Mapping presents a further complication to design-space exploration and hardware optimization.}
% \je{In particular, conventional mapping assumes a static compute architecture for which to optimize an algorithm for.}
% \je{However, when searching for an optimal architecture for this application, it is impractical to simply sweep a range of possible processor designs, due to the size of this space.}
% \je{An alternate approach would be to reduce the size of the processor design space by extracting some architecture design parameters from mappings, and jointly optimizing the mapping for the accelerator's static and dynamic energy.}
% --- JACK --- A possible solution for me to do design space exploration here would have been to just iterate over possible SRAM allocations (e.g. do a grid search of vector sram in {1kB, 4kB, 16kB, ...}, and so on).
% I didn't do this because I didn't have a sense of what reasonable amounts of SRAM would be, and I wanted to get that from the mappings.
% So instead, I extracted the system SRAM from the mappings, and co-optimized the mapping dynamic and static energy.
% I'm hesitant to mention this here, because I don't know if there's a big problem with doing it the other way, or if I just avoided it for stupid reasons.
% But the long and the short of it is that I did it to reduce the number of design space loops I iterated over.


% --- JACK --- I plan to move the discussion of the SteROI system to the beginning of Section 3. I'll be careful to specify when I'm talking about the system vs. when I'm talking about the L2 processor (SoC Accelerator) design.
% I'll also move my discussion of my binned mapping to section 4 (and optimization) to section 4.
\section{ROI-Based Stereo Depth}\label{sec:algorithm}

\begin{figure}
    \centering
    \includegraphics[width=0.9\linewidth]{Figures/Algorithm_2024_07_04.pdf}
    \caption{Illustration of the \projname{} processing pipeline. Object detection is run on the L2 processor and run infrequently; object tracking is run on intermediate frames on the L1 processors.}
    \label{fig:alg_design}
\end{figure}

\begin{figure}
    \centering
    \includegraphics[width=0.9\linewidth]{Figures/Distributions_2024_07_10.pdf}
    \caption{Distribution of ROI sizes across various object tracking datasets and object classes.}
    \label{fig:roi_distribution}
\end{figure}

\je{We propose the use of Regions-of-Interest (ROIs) in AR platforms to augment conventional Stereo Depth Algorithms.}
\je{We illustrate this proposed augmented algorithm in Figure ~\ref{fig:alg_design}.}
In \je{many AR} applications, only specific \je{objects in} an image are of interest\je{; bounding boxes around these objects can be used as ROIs. \cite{eyecod}}
Fig.~\ref{fig:roi_distribution} illustrates the typical sizes of bounding boxes across various object classes in %the 
\je{egocentric datasets,} KITTI \cite{kitti2012, kitti2015} and Epic Kitchens \cite{epic_kitchens}. %datasets. 
\je{Typical ROI sizes often multiple orders of magnitude smaller than the full image resolution. }
Unlike classification CNNs, stereo depth models can handle variable-sized ROIs.
This is because they are designed for regression tasks, producing output with the same spatial dimensions as the input images.
With typical ROI sizes often multiple orders of magnitude smaller than the full image resolution, significant processing savings are possible.

\begin{figure}
    \centering
    \includegraphics[width=0.9\linewidth]{Figures/ROIQuality_2024_11_24.pdf}
    \caption{EPE (left) and 3-pixel error (right) for HITNet \cite{hitnet} evaluated on 'Car' ROIs in the KITTI Object Tracking \cite{kitti2012} dataset, as measured against inference on full frames.}
    \label{fig:alg_quality}
\end{figure}

\je{One concern with this method is the degradation of stereo depth quality incurred by processing only ROIs.}
\je{To assess this concern, we evaluate HITNet \cite{hitnet} on crops from the KITTI dataset \cite{kitti2012}, shown in Figure ~\ref{fig:alg_quality}.}
\ca{We choose this dataset for it's availability of full frame stereo depth and object tracking labels.}
\je{We evaluate the degradation of results from full-frame inference by computing End-Point Error (EPE) and 3-pixel error against the full frame results\footnote{Dataset access and model processing took place at the University of Michigan.}; on each graph, lower errors imply less degradation.}
\je{In general, we find that ROI width is most strongly correlated to ROI degradation, with narrow ROIs with little spatial context suffering more compared to broader ROIs.}
\je{However, the tolerability of this degradation has an application dependence.}
\je{While simple algorithms, such as enforcing a minimum ROI size, can be used to address these challenges; in this work, we explore the ramifications of designing a compute system for the full dynamic range of ROI sizes extracted from these datasets.}

While ROIs can reduce stereo depth processing, \je{finding them} can introduce overhead \je{computational} costs.
Extracting ROIs requires object detection \cite{redmon_yolov3_2018}, which demands comparable MACs and weight storage to stereo depth processing itself \cite{hitnet, tiefenrausch}.
Minimizing the overhead of ROI extraction is critical to reduce the loss in efficiency.
\je{We propose to combat this inefficiency by interleaving expensive object detection with fast and efficient object tracking \cite{asu_fpga}, such as a correlation filter \cite{correlation_filter}.}
\je{Such algorithms have been demonstrated to be efficient on low power platforms \cite{vota}.}
\je{For egocentric tasks, where objects move continuously with respect to the observer, such an approach can accurately track objects with greatly reduced computational cost.}
\je{In this work, we specifically use YOLOv3 \cite{redmon_yolov3_2018} and Correlation Filters \cite{correlation_filter} for object detection and tracking, respectively.}
% \section{\projname{} L2 Processor Architecture}\label{sec:architecture}
\section{System and Architecture Design}\label{sec:architecture}

\begin{figure*}
    \centering
    \includegraphics[width=0.8\linewidth]{Figures/System_2024_06_03.pdf}
    \caption{\projname{} system design.}
    \label{fig:steroi_compare}
\end{figure*}

\textbf{System Design}:
The system-level design of \projname{} is depicted in Fig.~\ref{fig:steroi_compare}. %, while the processing pipeline is illustrated in Fig.~\ref{fig:alg_design}.
\je{The design features stereo sensors and a custom accelerator integrated into an SoC, which is a typical design for AR platforms \cite{aria}.}
% \projname{} uses a detect-and-track strategy for energy-efficient ROI extraction.
% Since the compute complexity of object tracking is significantly less than object detection, \projname{} employs only periodic object detection to detect the objects (ROIs) and uses lightweight object tracking to efficiently follow these ROIs.
% The large difference in computing demands between object tracking and object detection means that we can achieve energy savings even when object detection is run frequently.
\je{Furthermore, each sensor is co-packaged with a lightweight L1 processor, based on \cite{siracusa}}.
\je{These processors are responsible for handling the object tracking algorithms used to extract ROIs; in this way, they enable saving energy when transmitting ROIs from the L1 processor to the SoC and the accelerator (heretoafter referred to as the \textit{L2 Processor}.}
% \projname{} incorporates a small level-1 (L1) processor co-packaged with each image sensor in the AR/VR system for efficient object tracking using correlation filters \cite{correlation_filter, vota}. 
% This approach significantly reduces the need for costly object detection, which is performed only occasionally using the level-2 (L2) processor.

\begin{figure}
    \centering
    \includegraphics[width=0.8\linewidth]{Figures/Architecture_2024_07_04.pdf}
    \caption{\projname{} L2 processor architecture. The Special Compute Unit (SCU) accelerates non-parameterized compute patterns.}
    %The design is flexible to permit efficient adaptation to dynamic ROIs.}
    \label{fig:arch_sketch}
\end{figure}

\je{\textbf{Accelerator Architecture}:}
The L2 processor is responsible for object detection and ROI-based stereo depth processing %, which is the most significant challenge.
The \je{accelerator} uses a parameterized hierarchical \je{architecture}, pictured in Fig.~\ref{fig:arch_sketch}, which draws inspiration from ANSA \cite{ansa}.
The architecture is composed of tiles, with each tile consisting of a collection of PEs. 
Each PE is responsible for executing convolutions and activation functions through a Vector-Matrix Multiplier (VMM), local SRAMs, and a shuffle buffer for depthwise convolutions.
Flexibility is achieved through both the hierarchical structure, allowing for dynamic reconfiguration of compute resources for different compute tasks\je{;} and the compute units themselves, which support multiple dataflows to enable optimization at the mapping level.
\je{Power gating is further used to capture further energy savings by disabling unused resources on a per-frame basis.}
We have expanded on ANSA to accommodate the diverse sizes of ROIs while maintaining energy efficiency. 

\begin{figure}
    \centering
    \includegraphics[width=0.75\linewidth]{Figures/Operations_2024_04_16.pdf}
    \caption{Conventional CNN operation counts (e.g. convolution MACs) and stereo depth specific operations in stereo depth networks in recent years on 384$\times$1280 images.}
    \label{fig:ops_compare}
\end{figure}

\textbf{Special Compute Unit (SCU)}: 
% Non-parameterized operations specific to stereo depth must be supported efficiently and with low latency to prevent them from becoming bottlenecks.
Stereo depth networks consist of both CNN layers and stereo-depth-specific non-parameterized layers, as illustrated in Fig.~\ref{fig:ops_compare}.
% Non-parameterized operations specific to stereo depth must be supported efficiently and with low latency to enable real time, high frame rate operation.
These operations, seen in many networks \cite{stereonet, hitnet, tiefenrausch}, are used for processing disparity estimates.
\je{They use operations and data broadcast patterns for which conventional linear algebra engines are ill-suited, such as vector L1 norm and list argmin.}
\je{While not the majority of operations in any network, they constitute a sufficient portion of the network to necessitate hardware support to enable low latency, high framerate processing.}
Supporting these operations directly in PEs would increase their complexity and footprint, and reduce efficiency for both CNN and special operations.
Therefore, the \projname{} L2 processor uses both PEs for CNN compute, and special compute units (SCUs) for special operations.

To design an efficient SCU, we first observe that the cost volume processing in \cite{stereonet, hitnet, tiefenrausch} employs a limited set of compute primitives: vector-vector difference, L1 norm, sequence minimum, and argmin.
We propose a SCU pipeline with bypass options to handle these operations as well as their compositions.
The resulting SCU design captures various variants of cost volume processing for each stereo depth network, along with the warp and aggregate operations from \cite{hitnet}, and the maxpool operation in \cite{redmon_yolov3_2018}. 
\ca{A single SCU is allocated per Tile; the correct balance of SCUs to PEs is then realized through design space exploration over the mapping design.}

\textbf{Mutipacket NoC Routing}: 
The \projname{} L2 processor uses hierarchically arranged mesh NoCs connecting tiles globally and PEs locally.
These NoCs allow for fine-grained partitioning of compute tasks. 
However, this flexibility also costs redundant data movements.
To mitigate this cost, the \projname{} L2 processor uses a multipacket NoC. %, illustrated in Fig.~\ref{fig:multipacket}.
A multipacket consists of a data packet and a list of destination nodes. 
When a NoC node receives a multipacket, it forwards the data to the remaining destination nodes in the network.
To minimize the overhead of this scheme, we use simple Direction Order Routing (DOR) \cite{eecs570_dor}.
With DOR, each data packet is sent over a given link only once.
This approach reduces data movement while maintaining high flexibility in compute unit allocation.

% \begin{figure}
%     \centering
%     \includegraphics[width=0.6\linewidth]{Figures/Multipacket_2024_07_04.pdf}
%     \caption{Multipackets address multiple destination nodes with a single packet.}
%     %This allows efficient transfer through bottlenecks, such as DRAM I/Os or tile-tile links.}
%     \label{fig:multipacket}
% \end{figure}
\section{\projname{} L2 Processor Mapping}\label{sec:mapping}

% \begin{figure*}
%     \centering
%     \includegraphics[width=0.9\linewidth]{Figures/Binning_2024_07_04.pdf}
%     \caption{We perform a multidimensional design space sweep along each axis, and evaluate these mappings across the range of ROI sizes in the target application (Phase \#1). From here, optimal binning can be computed by multiphase annealing over this cost volume, respecting the target ROI probability distribution and area and latency constraints (Phase \#2). Note that the plots have been reduced from 4D to 2D for the purpose of simplifying visualization.}
%     \label{fig:binning_explanation}
% \end{figure*}

% To efficiently support a broad range of ROI sizes, we need to navigate the vast design space of possible mappings onto the L2 processor. 
\je{To leverage the energy savings made possible by our algorithm and architecture, we must do a good job of mapping compute onto the processor.}
\je{However, there are two key challenges that must be overcome to achieve this.}
\je{The first is the vast space of possible mappings that are possible; to find low energy, low latency mappings within this space, we must either reduce the dimensionality of this space, or develop algorithms to efficiently traverse it.}
\je{The variable ROI-size in this application adds the second challenge of providing mapping support across this range of possible ROIs.}
\je{Purely offline solutions to this problem are impractical, due to the storage requirements for supporting the full ROI range; and purely online solutions are equally impractical, due to the complexity of generating these mappings.}

\subsection{Single-ROI Mapping}\label{subsec:single_roi}

\je{We first consider the space of mappings for algorithms running on a fixed ROI size.}
\je{To traverse the large space of possible mappings, we first generate a set of higher level mapping descriptors, which can be used to generate low-level control signals for the processor.}

\textbf{DRAM I/O Options}:
\je{The off-chip DRAM storage in the \projname{} system can be used to store intermediate activations in neural networks.}
\je{This helps to address two levels of memory utilization variance.}
\je{Within a single frame, different activations in the network can have vastly different sizes; to enable processing on form-factor limited processors, it is beneficial to not rely on local SRAMs to store these activations.}
\je{Across multiple frames, different ROI sizes result in similarly variable activation sizes.}
% The variations in ROI result in vastly varying activation sizes from frame to frame. 
% Furthermore, for a single ROI, each layer of the model requires significantly different intermediate activation sizes.
DRAM provides a means to trade off dynamic energy and latency with SRAM utilization for handling these large ROIs and layers: firstly, by choosing which activations are stored in DRAM; and secondly, by choosing how these activations are accessed.
Activations may be streamed directly from DRAM to local buffers to reduce SRAM utilization. 
Alternatively, activations can be buffered in SRAM to reduce energy and latency.

\begin{figure}
    \centering
    \includegraphics[width=0.75\linewidth]{Figures/DRAMModes_2024_07_10.pdf}
    \caption{By successively buffering or streaming large activations from DRAM, a progression of \textit{DRAM Modes} are formed which reduce SRAM utilization.}
    \label{fig:dram_modes}
\end{figure}

\je{To choose which activations to store in DRAM and which to store in SRAM, we consider the minimal amount of off-device storage necessary to reduce an algorithms peak SRAM utilization, as seen in Figure ~\ref{fig:dram_modes}.}
\je{Iterating on this process, we generate a sequence of \textit{DRAM Modes}, or sets of activations to be streamed or buffered in DRAM.}
\je{By forming this sequence, we reduce the problem of determining DRAM I/O for a mapping from making a per-activation ternary choice, to selecting selecting the appropriate DRAM mode to fit within a given architecture's memory budget.}
\je{In practice, a combination of dataflow selection and DRAM I/O can be used to reduce SRAM utilization; therefore, we generate mappings for a few DRAM modes, and evaluate each on latency, energy, and memory utilization when generating mappings.}
% The DRAM option is used to reduce peak SRAM utilization.
% This utilization is usually dominated by large intermediate activations, which can instead be streamed from DRAM.
% By applying DRAM I/O to the ranked list of largest activations, we generate a sequence of activation sets to be streamed or loaded to and from DRAM, as shown in Fig.~\ref{fig:dram_modes}.
% We refer to these sets of activations as \textit{DRAM Modes}. 
% Indexing these DRAM Modes simplifies the selection of DRAM I/O to a single parameter.

\textbf{Dataflow Options}: The \projname{} L2 processor supports per-layer dataflows for both PEs and SCUs.
These dataflows include Weight and Input Stationary, which define the patterns of data reuse for weights and activations\je{;} and Channel First and Last, which divide input channel accumulation either spatially or temporally.
While the choice between Weight and Input Stationary is conventionally based on data reuse, the relative sizes of activations and weights for variably sized ROIs must also be considered.
Data movement in the NoC must also be managed. Different dataflows affect the location where activations are produced, thereby impacting the efficiency of data movement between PEs.

\je{Prior works \cite{ansa} have used greedy algorithms to assign per-layer dataflows to a network.}
\je{While this method is useful for minimizing latency and dynamic energy; we find it is insufficient to satisfy SRAM utilization constraints, as minimizing peak memory utilization requires global optimization.}
\je{Therefore, we use an augmented greedy algorithm to assign per-layer dataflows, which consists of a preliminary greedy assignment to minimize dynamic energy, and a subsequent optimization step which identifies dataflow choices that exceed memory useage limits.}
% In every elementary step in the above procedure, a greedy algorithm is employed to assign layer-wise dataflow options.
% However, relying solely on the greedy algorithm may lead to suboptimal solutions, particularly when operating within a fixed SRAM budget.
% To address this, we further augment this greedy algorithm by implementing an SRAM utilization refinement step, which identifies layers in the network exceeding a fixed SRAM allocation, and fine-tunes the mapping to bring these layers within the SRAM budget.

\je{\textbf{Tile Shutoff}:}
\je{We also consider turning off tiles, PEs, or SCUs within a processor to be an aspect of mapping.}
\je{Partial processor shutoff can be used on a per-frame basis to conform the compute capacity of the processor to the current task.}
\je{This allows dynamic energy efficiency to be sacrificed in order to reduce static power draw for the duration of processing a given frame.}
\je{Like with DRAM Modes, we limit our evaluation to a small number of processor sub-configurations to simplify the mapping design space.}

% \subsection{Indexing Mapping Design Space}\label{subsec:phase1}

% To efficiently describe the mapping design space, we identify three orthogonal axes to index the design space. 
% We then sweep these axes in relation to the ROI sizes to generate mappings. 
% Finally, we analyze key metrics of these mappings, including runtime latency, SRAM utilization, and dynamic energy.

% \textbf{Design-Time ROI Axis}: The choice of per-layer dataflow impacts both SRAM utilization and dynamic energy consumption.
% Tradeoffs between SRAM utilization (and by proxy, static energy) and dynamic energy must be made to minimize total energy.
% To simplify layer dataflow selection, we describe a series of mappings with different tradeoffs between SRAM utilization and dynamic energy, indexed by \textit{design-time ROI} sizes.
% For a given processor, we first find the absolute minimum SRAM needed to support the largest expected ROI size.
% Next, we apply our augmented greedy mapping algorithm using this SRAM budget.
% For much smaller ROIs, this SRAM budget has minimal impact on per-layer dataflow selection, as the small activations do not approach the SRAM budget.
% However, for larger ROIs, compromises need to be made to meet the SRAM budget.
% This sequence of network dataflow selections is indexed by a single design-time ROI axis.
% As we sweep through the design-time ROI axis, the corresponding dataflow selections
% are applied to various \textit{runtime ROI sizes} to form low-level mappings, realizing variable tradeoffs between mapping dynamic energy and static energy.

% \textbf{Processor Utilization Axis}:
% We may wish to utilize the entire L2 processor when processing large ROIs, but power gate all but a small portion when processing small ROIs to save energy.
% We thus enumerate the possible compute subconfigurations, ordered in terms of compute capacity to produce the third axis.

% \subsection{Binning Runtime ROIs}\label{subsec:phase2}
\subsection{Multiple-ROI Binning}\label{subsec:multi_roi}

% \je{\textbf{Mapping Descriptors}:}
\je{We consider the combination of a DRAM Mode, a dataflow assignment for each network layer, and a processor sub-configuration to constitute a \textit{mapping descriptor}.}
\je{From this mapping descriptor, a \textit{low level mapping} may be trivially generated by assigning compute operations and activation storage to the PEs, SCUs, and SRAMs within a processor.}
\je{While generating a low-level mapping requires knowledge of the ROI and intermediate feature sizes, a mapping descriptor is agnostic of this detail.}
\je{For HITNet, storing a mapping descriptor takes on the order of 100s of Bytes of memory.}

\je{To support the full range of possible ROI sizes efficiently, we propose to divide this range of sizes into a finite number of intervals, and assigning a separate high level mapping descriptor to each interval.}
\je{Low level mapping can then be performed at runtime with minimal overhead.}
\je{This scheme allows for the mapping descriptors to be optimized offline while still enabling runtime ROI variability.}
\je{Additionally, using separate mapping descriptors for different ROI size intervals allows for mappings to be customized for each size; for example, larger ROIs may use mappings that focus more heavily on reducing SRAM utilization, while mappings for smaller ROIs can focus entirely on maximizing energy efficiency.}
\je{We combine optimized mapping design with architecture architectural design exploration.}
\je{In this way, we are able to determine the optimal architecture design for different ROI probability distributions.}

% We propose ROI range binning to process the continuous range of ROI sizes efficiently.
% A binning divides the range of runtime ROI sizes into $n$ intervals.
% The ROIs within an interval are then mapped according to a single point in the mapping design space.
% This phase also determines the allocation of system SRAM allocation based on the usage of SRAM within each bin.

% By representing a bin as the coordinates of the interval boundaries and 
% the coordinates of the chosen mapping in the mapping design space, we can use simulated annealing to optimize binning.
% This representation can be visualized geometrically, as seen in Fig.~\ref{fig:binning_explanation}, with a bin represented by a straight line in the mapping design space.
% However, as the dimensionality of this representation increases with the number of intervals, we face the challenge of simulated annealing getting stuck in local minima.
% To address this, we use \textit{multiphase binning annealing}.
% This process first optimizes a single interval; then splits this interval in two sub-intervals and optimizes each; and the process repeats as needed.

% \subsection{Sweeping Processor Configurations}\label{subsec:phase3}

% \textbf{Design Space Exploration}: To systematically and efficiently navigate this vast and complex mapping design space, we propose a three-phase design space exploration.
% In \textbf{Phase 1} (Sec. ~\ref{subsec:phase1}), we use three orthogonal axes to traverse the design space of mappings for a set of runtime ROIs onto a processor of known compute capacity.
% In \textbf{Phase 1} (Section~\ref{subsec:phase1}), we use three orthogonal axes to generate high level descriptions of mappings for different ROI sizes on a processor of known compute capacity.
% In \textbf{Phase 2} (Section~\ref{subsec:phase2}), we partition the range of ROI sizes into intervals, or ``bins''.
% This enables runtime mapping, where each runtime ROI within a particular bin is assigned the same mapping.
% Each interval is then assigned a high level mapping descriptor offline; mappings for ROIs within an interval can then be quickly generated at runtime.
% Finally, in \textbf{Phase 3} (Section~\ref{subsec:phase3}), we perform Phases 1 and 2 for many processor configurations, evaluating core area, average energy, and latency. 
% This process yields a pareto-optimal curve of potential L2 processor configurations and ROI binnings.

% The design so far has assumed fixed processor compute resources in terms of the number of PEs and SCUs.
% We expand the design space sweep over a range of potential L2 processor designs to perform a complete design space exploration.
% This sweep allows us to evaluate tradeoffs between fine and coarse-grained compute structures in the processor, as well as to determine many candidate processors and optimal binnings for them.

\section{Results}\label{sec:results}

We focus our analysis on the L2 processor design and performance.
For the evaluations, HITNet \cite{hitnet} has been chosen as a representative of the latest stereo depth algorithms.
For comparison with the L2 processor, we evaluate HITNet on the Jetson Orin Nano, which is a representative off-the-shelf mobile compute system.
A public implementation of HITNet is used \cite{tinyhitnet}, and it is compiled on a per-ROI size basis using ONNX and TensorRT. This approach provides optimistic estimates for the device's performance.
To measure system power, we utilize Jetson Stats and isolate the power consumption of the GPU and CPU for comparison.

Our system simulator is made of multiple parts. 
Firstly, we implement a counter-based architecture model to estimate the L2 processor performance.
The dynamic and static energy of PE, SCU, \ca{and} SRAM I/O are estimated based on post-APR simulation using the TSMC 28nm PDK and 16-bit operations.
This L2 simulator also accounts for DRAM I/O \cite{lpddr4x, lpddr5_est} and NoC \cite{ansa} energy and latency.
% In our energy estimation, we take into account all parts of the system including DRAM I/O \cite{lpddr4x, lpddr5_est}, NoC \cite{ansa}, sensor \cite{gs_cis1}, uTSV, and MIPI interface \cite{gomez_distributed_2022}.
We base the L1 energy and latency on \cite{marsellus} and estimate the energy required for object detection on images of size $384\times1280$.
We also evaluate system level sensor \cite{gs_cis1}, uTSV, and MIPI interface \cite{gomez_distributed_2022} energy.
This complete simulator is used to conduct design space sweeps of the \projname{} system running HITNet for stereo depth and TinyYOLOv3 for object detection, according to Section ~\ref{sec:mapping}.

\subsection{Ablation Studies}

% \begin{figure}
%     \centering
%     \includegraphics[width=0.8\linewidth]{Figures/ResultAxisAblation_2024_07_04.pdf}
%     \caption{Binning with each design space axis disabled.}
%     \label{fig:result_axis_ablation}
% \end{figure}

% \textbf{Design Space Axes}: In Fig.~\ref{fig:result_axis_ablation}, we evaluate binnings generated by disabling each of the design axes from Section ~\ref{subsec:phase1}, and compare these to the ones generated with all axes enabled.
% For example, disabling the processor utilization axis forces all bins to use one uniform processor utilization, i.e., the entire L2 processor without power gating.
% By comparing the uniform processor utilization curve with the independent bins curve, one can recognize the significance of disabling the processor utilization axis.
% From the comparisons in Fig.~\ref{fig:result_axis_ablation}, we can see that the design-time ROI axis has the least impact on energy.
% The DRAM modes axis is situationally important.
% Large processors have sufficient on-chip SRAM and do not need DRAM I/O caching.
% On the other hand, for very small processors, other mapping inefficiencies dominate the energy for average ROI sizes.
% However, medium-sized processors rely on DRAM caching to support large ROIs, but suffer up to $3.15\times$ inefficiency by using it for average ROIs.
% Finally, the processor utilization axis is the most significant axis across processor sizes, showing that the ability to operate efficiently on small and large ROI sizes by tuning processor utilization within the same silicon is critical for this system design.

\begin{figure}
    \centering
    \includegraphics[width=0.8\linewidth]{Figures/ResultBinCountAblation_2024_07_04.pdf}
    \caption{Effect of number of bins. Increasing bin count results in marginal gains, with highest energy savings on intermediate sized processors.}
    \label{fig:result_bincount_ablation}
\end{figure}

\textbf{Bin Count}: In Fig.~\ref{fig:result_bincount_ablation}, we evaluate the results using different number of bins, which corresponds to the number of runtime ROI intervals.
Moving from one to two bins, the processor can use different mappings for large and small ROI sizes, optimized for different objectives.
This results in a significant gain in energy.
However, as the number of bins increases beyond 2 bins, the marginal gain per additional bin diminishes and varies slightly across processor areas.
In some cases, adding an extra bin can still be beneficial to better adapt to the specific ROI distribution being used.

\begin{figure}
    \centering
    \includegraphics[width=0.8\linewidth]{Figures/ResultDistributionCompare_2024_07_04.pdf}
    \caption{Results of different ROI distributions. The pareto curve of an ROI distribution is dictated primarily by the ROI mean, though variance also plays a minor role.}
    \label{fig:result_distribution_compare}
\end{figure}

\textbf{ROI Distributions}: We compare results for multiple ROI probability distributions to evaluate the generality of our design methodology, as seen in Fig.~\ref{fig:result_distribution_compare}.
Interestingly, the average energy required to run the various ROI distributions is ordered in the same sequence as their mean ROI size, as reported in Fig.~\ref{fig:roi_distribution}.
This suggests that our design method minimizes nonlinear overheads caused by the variable ROI sizes.
Even high variance bimodal distributions, such as ``Vegetables'' \cite{epic_kitchens}, can be efficiently handled by adequately parameterized ROI binnings on \projname{}.

\subsection{Design Benchmarking}

Next, we compare designs generated by this methodology, with existing edge system and baseline ASIC designs.

\begin{figure}
    \centering
    \includegraphics[width=0.8\linewidth]{Figures/ResultJetsonCompare_2024_07_04.pdf}
    \caption{Energy consumption of Jetson Orin Nano running different ROI sizes and \projname{} systems optimized for different ROI distributions. \projname{} achieves superior granularity in energy and lower overall energy.}
    \label{fig:jetson_compare}
    % Distribution: KITTI
    %    Config: ((4, 16), 16, 4)
    %    SRAM Dictionary: {'core_v0': 164864.0, 'core_v1': 123648.0, 'vmm_vec': 61632.0, 'vmm_mat': 248112.0}
    % Distribution: Aubergine
    %    Config: ((8, 16), 6, 2)
    %    SRAM Dictionary: {'core_v0': 245760.0, 'core_v1': 491520.0, 'vmm_vec': 256000.0, 'vmm_mat': 1024000.0}
    % Distribution: Olive
    %    Config: ((4, 16), 6, 2)
    %    SRAM Dictionary: {'core_v0': 1310720.0, 'core_v1': 491520.0, 'vmm_vec': 109568.0, 'vmm_mat': 1382400.0}
    % Distribution: Chopping Board
    %    Config: ((16, 16), 4, 1)
    %    SRAM Dictionary: {'core_v0': 370944.0, 'core_v1': 491520.0, 'vmm_vec': 1966080.0, 'vmm_mat': 4748928.0}
    % Distribution: Left Hand
    %    Config: ((16, 16), 4, 1)
    %    SRAM Dictionary: {'core_v0': 513280.0, 'core_v1': 491520.0, 'vmm_vec': 1966080.0, 'vmm_mat': 4799664.0}
    % Distribution: Pan
    %    Config: ((16, 16), 4, 1)
    %    SRAM Dictionary: {'core_v0': 368640.0, 'core_v1': 491520.0, 'vmm_vec': 1966080.0, 'vmm_mat': 4748928.0}
    % Distribution: Vegetable
    %    Config: ((8, 16), 4, 1)
    %    SRAM Dictionary: {'core_v0': 1966080.0, 'core_v1': 491520.0, 'vmm_vec': 61440.0, 'vmm_mat': 4055040.0}
\end{figure}

\textbf{Comparison with Jetson Orin Nano}: 
In Fig.~\ref{fig:jetson_compare}, we compare the per-ROI energy of the Jetson Orin Nano with \projname{} systems optimized for different ROI distributions.
The \projname{} designs demonstrate two prominent advantages.
Firstly, a \projname{} design can be optimized based on the ROI distribution, whereas the Jetson Orin Nano requires statically compiled binaries for each ROI size in the distribution, which makes it less practical. 
Secondly, 
\projname{} processor dynamically scales performance and energy according to the size of ROI being processed.
In contrast, the Jetson Orin Nano appears to suffer from a coarse-grained reconfigurability of its tensor cores, resulting in an energy pattern characterized by stair steps\ca{; compute latency also suffers, and the Jetson Orin Nano does not exceed 15 FPS operation.}
\projname{}, on the other hand, uses tiles, PEs and SCUs to enable fine-grained optimization.

\begin{figure}
    \centering
    \includegraphics[width=0.9\linewidth]{Figures/ResultBinningBreakdown_2024_07_11_KITTI.NORM.pdf}
    \caption{Breakdown of \projname{} L2 processor energy by ROI size. Dashed lines represent the boundaries of mapping bins.}
    \label{fig:binning_breakdown}
    % 29 January 2025
    % 4 Tiles, 16 PEs / Tile, 1 Core / Tile
    % 16 Vectors, 4 Long / PE
    % 90kB Vector1 SRAM / Core
    % 480kB Vector2 SRAM / Core
    % 18kB Vector SRAM / PE
    % 240kB Matrix SRAM / PE
\end{figure}

\textbf{Energy Breakdown by ROI Size}: We analyze the breakdown of energy by ROI size in a \projname{} L2 processor, as illustrated in Fig.~\ref{fig:binning_breakdown}.
The energy consumption is primarily influenced by static power and DRAM access, with their proportions varying accordingly.
For very small ROIs, static power draw is dominant. 
For extremely large ROIs, the energy is dominated by DRAM I/O.
DRAM I/O is used to minimize the L2 SRAM and reduce static power for small ROIs.
This insight underscores the challenge of optimizing energy across ROI distribution in the L2 processor design. 

\begin{figure}
    \centering
    \includegraphics[width=0.9\linewidth]{Figures/Piechart_2024_07_11.pdf}
    \caption{Comparison of energy in baseline design (no ROI exploitation) with \projname{} design, assuming object detection runs every 5 frames and KITTI ROI distribution.}
    \label{fig:baseline_compare}
    % 29 January 2025
    % Naive Pipeline:
    %   8 Tiles, 16 PEs / Tile, 1 Core / Tile
    %   16 Vectors, 4 Long / PE
    %   960kB Vector1 SRAM / Core
    %   240kB Vector2 SRAM / Core
    %   16kB Vector SRAM / PE
    %   600kB Matrix SRAM / PE
    % ROI Sparsity Exploitation:
    %   4 Tiles, 16 PEs / Tile, 1 Core / Tile
    %   16 Vectors, 4 Long / PE
    %   480kB Vector1 SRAM / Core
    %   241.5kB Vector2 SRAM / Core
    %   288kB Vector SRAM / PE
    %   487.5kB Matrix SRAM / PE
\end{figure}

\textbf{Comparison with Baseline System}: In Fig.~\ref{fig:baseline_compare}, we compare a baseline system with no ROI or temporal sparsity with a \projname{} system.
In this comparison, we optimize both processors to have an area under 100 mm\textsuperscript{2} and a frame rate of at least 30 FPS.
The exploitation of ROI requires object detection and object tracking.
Energy savings are limited by these two costs, which do not scale with the ROI.
Nevertheless, \projname{} still achieves \textbf{$4.35\times$} per-inference energy savings by effectively leveraging ROI-based processing.

\section{Conclusions}
This study explores how multilingual LLMs affect user behaviour and advertisement persuasiveness in a co-writing task, with a focus on English and Spanish. Our findings reveal a clear ``rationality gap'', where prior exposure to a relatively lower-resource language (Spanish) diminishes subsequent reliance on an LLM writing assistant in a higher-resource language (English). We then evaluate the consequence of these patterns for the persuasiveness of generated ads via a charitable donation task. Here, peoples' donation behaviour appears largely unaffected by the specific advertisement, alleviating the potential negative consequences of under-utilization due to irrational behavioural adaptions. However, donations are strongly related to participants' beliefs about the source of the advertisement. Those who think that it was written by an AI are (1) significantly less likely to donate, and (2) donate less. Our results have strong implications for a number of important stakeholders, including companies deploying global multilingual AI assistants, the dissemination of LLMs across linguistically different parts of the world, marketing practitioners, and societal stakeholders concerned about inequality. Heterogeneity in AI performance across tasks can lead to substantial behavioural second-order effects that asymmetrically affect appropriate reliance and utilization.  



\begin{acks}
This project was funded by the NDSEG Fellowship, and by Meta Reality Labs Research.
\end{acks}

%%
%% The next two lines define the bibliography style to be used, and
%% the bibliography file.
\bibliographystyle{ACM-Reference-Format}
\bibliography{refs}

\end{document}