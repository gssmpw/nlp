\section{Introduction}\label{sec:intro}

Augmented and Virtual Reality (AR/VR) has made significant advancements in recent years in terms of quality and affordability \cite{meta_motiv1, meta_motiv2}, through the use of machine learning algorithms.
One crucial algorithm is depth estimation from stereo sensors, which plays a vital role in spatial computing, hand tracking \cite{depth_gesture_recog}, and %spatial and
passthrough rendering.
Conventional DNN based stereo depth algorithms use expensive hierarchical processing \cite{stereonet, mobilestereonet, hitnet}, 
% resulting in high storage and processing energy
\je{which are challenging to accelerate on power constrained platforms}.
Increased resolutions and frame rates in newer systems further increase these costs \cite{near_sensor_distributed}.
Consequently, accelerating these networks on AR/VR devices while meeting real-time latency requirements and operating within the energy budget of limited battery devices presents a challenge.

In this work, we propose \textit{\projname{}}, an AR/VR stereo depth system comprising a flexible architecture for processing dynamic Regions of Interest (ROIs), and a comprehensive mapping methodology to optimize ROI processing for energy efficiency while maintaining real-time performance. Our contributions are as follows:

\begin{itemize}[leftmargin=*]
    % \item Analysis of stereo depth compute across variable ROI sizes;
    \item \je{The \textit{SteROI-D Algorithm}, which leverages Region-of-Interest (ROI) Sparsity to reduce per-frame depth extraction cost, and interleaved object detection and tracking to reduce ROI detection cost;}
    % \item A flexible ROI-based stereo depth processing system and a comprehensive mapping methodology to achieve near-ideal average inference energy by balancing resource availability for the largest ROIs while maintaining high efficiency for average ROIs;
    % \item Special Compute Unit (SCU) and multipacket routing to enable real time, high framerate operation; and
    \item \je{Special Compute Units (SCUs) and NoC Multipackets, to address compute and communication challenges in accelerating stereo depth networks;}
    % \item An evaluation of this end-to-end system design across a range of algorithm components and candidate datasets.
    \item \je{\textit{Binned Mapping}, a method for split online-offline algorithm mapping to enable efficient processing for a continuous range of ROI sizes; and}
    \item \je{A design space exploration framework for jointly optimizing an accelerator's SRAM allocation with it's Binned Mapping.}
\end{itemize}

To our knowledge, this is the first study to achieve ROI-based stereo depth processing.
This is also the first work to address variable ROI processing through a mapping-system co-design approach via an efficient design space exploration.
While prior work has exploited ROIs for eye tracking on AR devices, it was limited to a static architecture \cite{eyecod}.
Furthermore, although prior work has also proposed lightweight stereo depth systems for AR devices, they have not leveraged ROI sparsity \cite{tiefenrausch}.