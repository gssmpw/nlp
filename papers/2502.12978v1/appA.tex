\section{Proof of Theorem~\ref{theo:main}}
\label{app:proof_theo_main}
%
Firstly, we show that the conditional distribution
%
\begin{equation}
    T(\bm{Y}) \mid
    \left\{
    \mathcal{N}_{\bm{Y}} = \mathcal{N}_{\bm{y}},
    \mathcal{K}_{\bm{Y}} = \mathcal{K}_{\bm{y}},
    \mathcal{S}_{\bm{Y}} = \mathcal{S}_{\bm{y}},
    \mathcal{Q}_{\bm{Y}} = \mathcal{Q}_{\bm{y}}
    \right\}
\end{equation}
%
is a truncated normal distribution.
%
Let we define the two vectors $\bm{a},\bm{b}\in\mathbb{R}^{(n+1)d}$ as $\bm{a}=\mathcal{Q}_{\bm{y}}$ and $\bm{b}=\tilde{\Sigma}\bm{\eta}/\bm{\eta}^\top\tilde{\Sigma}\bm{\eta}$, respectively.
%
Then, from the condition on $\mathcal{Q}_{\bm{Y}}=\mathcal{Q}_{\bm{y}}$, we have
%
\begin{equation}
    \mathcal{Q}_{\bm{Y}} = \mathcal{Q}_{\bm{y}} \Leftrightarrow
    \left(
    I_{(n+1)d} -
    \frac{\tilde{\Sigma}\bm{\eta}\bm{\eta}^\top}{\bm{\eta}^\top\tilde{\Sigma}\bm{\eta}}
    \right)
    \bm{Y} =
    \mathcal{Q}_{\bm{y}}
    \Leftrightarrow
    \bm{Y} = \bm{a} + \bm{b}z,
\end{equation}
%
where $z=T(\bm{Y})=\bm{\eta}^\top\bm{Y}\in\mathbb{R}$. Thus, we have
%
\begin{align}
      &
    \{
    \bm{Y}\in\mathbb{R}^{(n+1)d}\mid
    \mathcal{N}_{\bm{Y}} = \mathcal{N}_{\bm{y}},
    \mathcal{K}_{\bm{Y}} = \mathcal{K}_{\bm{y}},
    \mathcal{S}_{\bm{Y}} = \mathcal{S}_{\bm{y}},
    \mathcal{Q}_{\bm{Y}} = \mathcal{Q}_{\bm{y}}
    \}  \\
    = &
    \{
    \bm{Y}\in\mathbb{R}^{(n+1)d}\mid
    \mathcal{N}_{\bm{Y}} = \mathcal{N}_{\bm{y}},
    \mathcal{K}_{\bm{Y}} = \mathcal{K}_{\bm{y}},
    \mathcal{S}_{\bm{Y}} = \mathcal{S}_{\bm{y}},
    \bm{Y} = \bm{a} + \bm{b}z, z\in\mathbb{R}
    \}  \\
    = &
    \{
    \bm{a} + \bm{b}z\in\mathbb{R}^{(n+1)d}\mid
    \mathcal{N}_{\bm{a}+\bm{b}z} = \mathcal{N}_{\bm{y}},
    \mathcal{K}_{\bm{a}+\bm{b}z} = \mathcal{K}_{\bm{y}},
    \mathcal{S}_{\bm{a}+\bm{b}z} = \mathcal{S}_{\bm{y}},
    z\in\mathbb{R}
    \}  \\
    = &
    \{
    \bm{a} + \bm{b}z\in\mathbb{R}^{(n+1)d}\mid
    z\in \mathcal{Z}
    \},
\end{align}
%
where truncated interval $\mathcal{Z}$ is defined as
%
\begin{equation}
    \mathcal{Z} =
    \{
    z\in\mathbb{R}\mid
    \mathcal{N}_{\bm{a}+\bm{b}z} = \mathcal{N}_{\bm{y}},
    \mathcal{K}_{\bm{a}+\bm{b}z} = \mathcal{K}_{\bm{y}},
    \mathcal{S}_{\bm{a}+\bm{b}z} = \mathcal{S}_{\bm{y}}
    \}.
\end{equation}
%
Therefore, we obtain
%
\begin{equation}
    T(\bm{Y}) \mid
    \{
    \mathcal{N}_{\bm{Y}} = \mathcal{N}_{\bm{y}},
    \mathcal{K}_{\bm{Y}} = \mathcal{K}_{\bm{y}},
    \mathcal{S}_{\bm{Y}} = \mathcal{S}_{\bm{y}},
    \mathcal{Q}_{\bm{Y}} = \mathcal{Q}_{\bm{y}}
    \}
    \sim
    \mathrm{TN}(\bm{\eta}^\top\bm{\mu}, \bm{\eta}^\top\tilde{\Sigma}\bm{\eta}, \mathcal{Z}),
\end{equation}
%
which is the truncated normal distribution with the mean $\bm{\eta}^\top\bm{\mu}$, the variance $\bm{\eta}^\top\tilde{\Sigma}\bm{\eta}$, and the truncation intervals $\mathcal{Z}$.
%

From the above result, we can compute the selective $p$-value as defined in Eq.~\eqref{eq:selective_pvalue} by using the truncated normal distribution.
%
Therefore, by probability integral transformation, under the null hypothesis, we have
%
\begin{equation}
    p_\mathrm{selective} \mid
    \{
    \mathcal{N}_{\bm{Y}} = \mathcal{N}_{\bm{y}},
    \mathcal{K}_{\bm{Y}} = \mathcal{K}_{\bm{y}},
    \mathcal{S}_{\bm{Y}} = \mathcal{S}_{\bm{y}},
    \mathcal{Q}_{\bm{Y}} = \mathcal{Q}_{\bm{y}}
    \}
    \sim
    \mathrm{Unif}(0, 1),
\end{equation}
%
which leads to
%
\begin{equation}
    \mathbb{P}_{\mathrm{H}_0}
    \left(
    p_\mathrm{selective} \leq \alpha \mid
    \mathcal{N}_{\bm{Y}} = \mathcal{N}_{\bm{y}},
    \mathcal{K}_{\bm{Y}} = \mathcal{K}_{\bm{y}},
    \mathcal{S}_{\bm{Y}} = \mathcal{S}_{\bm{y}},
    \mathcal{Q}_{\bm{Y}} = \mathcal{Q}_{\bm{y}}
    \right)
    =\alpha,\
    \forall\alpha\in(0,1).
\end{equation}
%
For any $\alpha\in(0,1)$, by marginalizing over all the values of the nuisance parameters, we obtain
%
\begin{align}
      &
    \mathbb{P}_{\mathrm{H}_0}
    \left(
    p_\mathrm{selective} \leq \alpha \mid
    \mathcal{N}_{\bm{Y}} = \mathcal{N}_{\bm{y}},
    \mathcal{K}_{\bm{Y}} = \mathcal{K}_{\bm{y}},
    \mathcal{S}_{\bm{Y}} = \mathcal{S}_{\bm{y}}
    \right)                                                                                                       \\
    = &
    \begin{multlined}
        \int_{\mathbb{R}^{n^\prime}}
        \mathbb{P}_{\mathrm{H}_0}
        \left(
        p_\mathrm{selective} \leq \alpha \mid
        \mathcal{N}_{\bm{Y}} = \mathcal{N}_{\bm{y}},
        \mathcal{K}_{\bm{Y}} = \mathcal{K}_{\bm{y}},
        \mathcal{S}_{\bm{Y}} = \mathcal{S}_{\bm{y}},
        \mathcal{Q}_{\bm{Y}} = \mathcal{Q}_{\bm{y}}
        \right) \\
        \mathbb{P}_{\mathrm{H}_0}
        \left(
        \mathcal{Q}_{\bm{Y}} = \mathcal{Q}_{\bm{y}} \mid
        \mathcal{N}_{\bm{Y}} = \mathcal{N}_{\bm{y}},
        \mathcal{K}_{\bm{Y}} = \mathcal{K}_{\bm{y}},
        \mathcal{S}_{\bm{Y}} = \mathcal{S}_{\bm{y}},
        \right)
        d\mathcal{Q}_{\bm{y}}
    \end{multlined} \\
    = & \alpha \int_{\mathbb{R}^{(n+1)d}}
    \mathbb{P}_{\mathrm{H}_0}
    \left(
    \mathcal{Q}_{\bm{Y}} = \mathcal{Q}_{\bm{y}} \mid
    \mathcal{N}_{\bm{Y}} = \mathcal{N}_{\bm{y}},
    \mathcal{K}_{\bm{Y}} = \mathcal{K}_{\bm{y}},
    \mathcal{S}_{\bm{Y}} = \mathcal{S}_{\bm{y}}
    \right)
    d\mathcal{Q}_{\bm{y}} = \alpha.
\end{align}
%
Therefore, we also obtain
%
\begin{align}
      &
    \mathbb{P}_{\mathrm{H}_0}(p_{\mathrm{selective}}\leq \alpha)                                              \\
    = &
    \sum_{\mathcal{N}_{\bm{y}}\in 2^{[n]}}
    \sum_{\mathcal{K}_{\bm{y}}\in \{0, 1\}}
    \sum_{\mathcal{S}_{\bm{y}}\in \{-1, 1\}^d}
    \begin{multlined}
        \mathbb{P}_{\mathrm{H}_0}
        (
        \mathcal{N}_{\bm{Y}} = \mathcal{N}_{\bm{y}},
        \mathcal{K}_{\bm{Y}} = \mathcal{K}_{\bm{y}},
        \mathcal{S}_{\bm{Y}} = \mathcal{S}_{\bm{y}}
        ) \\
        \mathbb{P}_{\mathrm{H}_0}
        \left(
        p_\mathrm{selective} \leq \alpha \mid
        \mathcal{N}_{\bm{Y}} = \mathcal{N}_{\bm{y}},
        \mathcal{K}_{\bm{Y}} = \mathcal{K}_{\bm{y}},
        \mathcal{S}_{\bm{Y}} = \mathcal{S}_{\bm{y}}
        \right)
    \end{multlined} \\
    = &
    \alpha
    \sum_{\mathcal{N}_{\bm{y}}\in 2^{[n]}}
    \sum_{\mathcal{K}_{\bm{y}}\in \{0, 1\}}
    \sum_{\mathcal{S}_{\bm{y}}\in \{-1, 1\}^d}
    \mathbb{P}_{\mathrm{H}_0}(\mathcal{N}_{\bm{y}}, \mathcal{K}_{\bm{y}}, \mathcal{S}_{\bm{y}})
    = \alpha.
\end{align}
%
