% \appendix

% \section{Appendix}
% \label{sec:appendix}

% \subsection{Common Concepts in Scenario 1}
% \label{sec:sce_1}
% Specific concepts in different semantic groups are listed in Table~\ref{tab:concepts_SC}. Each semantic group has ten common concepts.

% \begin{table*}[h!]
%     \centering
%     % \small
%     \begin{tabular}{cc}
%         \toprule
%         \textbf{ Group} & \textbf{Value} \\
%         \midrule
%         NUMBER & one, two, three, four, five, six, seven, eight, nine, ten \\
%         NAME & Alice, Bob, Carol, Dave, Francis, Grace, Hans, Ivan, Zach, Mike \\
%         MONTH & January, February, March, April, May, June, July, August, September, October \\
%         COLOR & red, orange, yellow, green, blue, brown, black, white, grey, gray \\
%         CITY & Taiwan, York, Cambridge, Oxford, Berlin, Paris, Washington, Rome, Tokyo, Toronto \\
%         NATION & China, America, England, UK, Germany, France, USA, Italy, Japan, Spain \\
%         PLACE & factory, concert, museum, library, bar, zoo, park, Theater, hospital, church \\
%         HUMAN & female, male, man, woman, human, boy, girl, elder, gentleman, guys \\
%         FURNITURE & chair, desktop, table, bed, cabinet, computer, lamp, mirror, house, room \\
%         % Random & \#ateien, \#╦, \#tatywna, \#don, \#it, \#on, \#invånare, \#Савезне, \#надморској, \#Российской \\
%         \bottomrule
%     \end{tabular}
%     \caption{Specific concepts in different semantic groups.}
%     \label{tab:concepts_SC}
% \end{table*}
% %

% \subsection{Concept Spaces in Scenario 3}
% \label{app:CS}
% Scenario 3 provides a human-annotated conceptual space in the domain of qualitative words, as shown in~\ref{fig:SMM_human}. Each concept is represented by an English word. If a pair of concepts happen to appear at a polysemous word in at least three languages, the corresponding nodes are connected. Node names marked with red shows the federative words - a common shared concept indicated by a large number of degree.

% \begin{figure*}
%     \centering
%     \includegraphics[width=1.1\linewidth]{figs/graph4_GT2.pdf}
%     \caption{A semantic map for the domain of qualitative words. Red shows the federative notions with high degrees.}
%     \label{fig:SMM_human}
% \end{figure*}

% \subsection{Degree Distribution}
% \label{app: dd}
% We draw the distribution of node degrees for spaces generated by two models, Llama2-7B and Llama2-70B, respectively. Figure~\ref{fig:degree_dist} (a) shows the unweighted degree while (b) demonstrates the weighted one. The figures show 70B has less long-tailed shape, with more nodes attributed to the relatively larger one.

% \begin{figure*}
%     \centering
%     \includegraphics[width=1.0\linewidth]{figs/compare_degree_dist.pdf}
%     \caption{Comparison of the degree distribution for both models. (a) shows the unweighted degree, while (b) shows the weighted one.}
%     \label{fig:degree_dist}
% \end{figure*}


\appendix

\section{Appendix}
\label{sec:appendix}

\subsection{Common Concepts in Scenario 1}
\label{sec:sce_1}
Table~\ref{tab:concepts_SC} lists specific concepts from different semantic groups, with each group containing ten common concepts.

\begin{table*}[h!]
    \centering
    % \small
    \begin{tabular}{cc}
        \toprule
        \textbf{Group} & \textbf{Concepts} \\
        \midrule
        NUMBER & one, two, three, four, five, six, seven, eight, nine, ten \\
        NAME & Alice, Bob, Carol, Dave, Francis, Grace, Hans, Ivan, Zach, Mike \\
        MONTH & January, February, March, April, May, June, July, August, September, October \\
        COLOR & red, orange, yellow, green, blue, brown, black, white, grey, gray \\
        CITY & Taiwan, York, Cambridge, Oxford, Berlin, Paris, Washington, Rome, Tokyo, Toronto \\
        NATION & China, America, England, UK, Germany, France, USA, Italy, Japan, Spain \\
        PLACE & factory, concert, museum, library, bar, zoo, park, theater, hospital, church \\
        HUMAN & female, male, man, woman, human, boy, girl, elder, gentleman, guys \\
        FURNITURE & chair, desk, table, bed, cabinet, computer, lamp, mirror, house, room \\
        \bottomrule
    \end{tabular}
    \caption{Specific concepts from different semantic groups.}
    \label{tab:concepts_SC}
\end{table*}

\subsection{Conceptual Spaces in Scenario 3}
\label{app:CS}
Scenario 3 presents a human-annotated conceptual space for qualitative words, as shown in Figure~\ref{fig:SMM_human}. Each concept is represented by an English word. Nodes are connected if a pair of concepts co-occur as a polysemous word in at least three languages. Nodes marked in red represent federative words, indicating a shared concept with a higher degree.

\begin{figure*}
    \centering
    \includegraphics[width=1.1\linewidth]{figs/graph4_GT2.pdf}
    \caption{A semantic map for the domain of qualitative words, with federative notions which have a higher degree highlighted in red.}
    \label{fig:SMM_human}
\end{figure*}

\subsection{Degree Distribution}
\label{app: dd}
We show the distribution of node degrees for spaces generated by two models, Llama2-7B and Llama2-70B. Figure~\ref{fig:degree_dist}(a) displays the unweighted degree distribution, while (b) shows the weighted distribution. The results indicate that the 70B model has a less pronounced long-tail distribution, with more nodes having relatively larger degrees.

\begin{figure*}
    \centering
    \includegraphics[width=1.0\linewidth]{figs/compare_degree_dist.pdf}
    \caption{Comparison of the degree distribution for both models: (a) unweighted degree and (b) weighted degree.}
    \label{fig:degree_dist}
\end{figure*}
