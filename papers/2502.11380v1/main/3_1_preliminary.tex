% \section{Preliminary}
% \label{sec:preliminary}

% \subsection{Basic Notions}
% % We define a conceptual space $\mathcal{G} = \{V, E\}$, where $V$ is a set of nodes, and $E$ is a set of edges. 
% % Each node $v \in V$ represents a concept, which can be realized by a token, a word, a sense, etc. Each edge $e(u,v) \in E$ connects a pair of nodes $(u,v)$, reflecting the degree of association~\cite{}. Any two nodes are connected if there exists a path $p(u,v)$ from node $u$ to $v$, where the length $L$ is the number of edges along with the path. If they are unconnected, we define $L = \infty$. If any node
% % in $\mathcal{G}$ is connected, we call $\mathcal{G}$ connected.
% \lky{We define a conceptual space $\mathcal{G} = \{V, E\}$, where $V$ and $E$ are the sets of nodes and edges, respectively. Each node $v \in V$ represents a concept, which can be realized by a token, a word, a sense, etc. Each edge $e(u,v) \in E$ connects a pair of nodes $(u,v)$, reflecting their degree of association~\cite{guo2012concepts}. If a path $p(u,v)$ exists between nodes $u$ and $v$, they are connected, with the path length $L$ defined as the number of edges along the path. If no such path exists, we define $L = \infty$. If every pair of nodes in $\mathcal{G}$ is connected, the conceptual space is deemed connected.} 
% \lky{Notably,} a subgraph $\mathcal{G'} = \{V', E'\}$, where $V' \subset V$ and $E' \subset E$ shows the local topology of $\mathcal{G}$. This usually indicates a specific semantic domain, such as repetitive adverbs~\cite{zhang2017semantic}, color adjectives~\cite{gardenfors2014geometry}, qualitative words~\cite{perrin2010polysemous}. Again, there exists a path for any pair of nodes in a connected subgraph $\mathcal{G}'$.
% % a metric $M$ is defined on $\mathcal{G}$, which reflects the degree of association or similarity of any two nodes.
% \lky{Moreover, we define the metric $M$ on $\mathcal{G}$ to measure the degree of association or similarity of any two nodes.} 
% A common strategy is cosine similarity~\footnote{We note that the cosine distance violates the triangle inequality which is required by a strict distance. However we relax this constraint for its simplicity and prevalence. }, which has been widely applied in the similarity-related tasks.

\section{Preliminary}
\label{sec:preliminary}

\subsection{Basic Notions}
We define a conceptual space $\mathcal{G} = \{V, E\}$, where $V$ and $E$ are sets of nodes and edges, respectively. Each node $v \in V$ represents a concept, which can be realized by a token, word, or sense. Each edge $e(u,v) \in E$ connects a pair of nodes $(u,v)$, reflecting their degree of association~\cite{guo2012concepts}. If a path $p(u,v)$ exists between nodes $u$ and $v$, they are connected, with path length $L$ defined as the number of edges along the path. If no path exists, $L = \infty$. A conceptual space is considered connected if every pair of nodes is connected.

A subgraph $\mathcal{G'} = \{V', E'\}$, where $V' \subset V$ and $E' \subset E$, reflects the local topology of $\mathcal{G}$ and typically represents a specific semantic domain, such as adverbs~\cite{zhang2017semantic}, color adjectives~\cite{gardenfors2014geometry}, or qualitative words~\cite{perrin2010polysemous}. A subgraph is connected if every pair of nodes has a path.

We define a metric $M$ on $\mathcal{G}$ to measure the association or similarity between nodes. A common metric is cosine similarity~\footnote{While cosine distance violates the triangle inequality required by strict distances, we relax this constraint due to its simplicity and widespread use.}, widely applied in similarity-related tasks.


% \subsection{Semantic Map Models}

% Semantic map models (SMMs) construct a cross-lingual conceptual space, adhering to the connectivity hypothesis \cite{croft2001radical}. 
% % This hypothesis posits that the subgraph $\mathcal{G'(x)}$ consisted of concepts associated with the same form $x$ in a specific language must be connected. 
% \lky{This hypothesis posits that the subgraph $\mathcal{G'(x)}$, consisted of concepts associated with the same form $x$ in a specific language, must be connected.}
% This connected region is 
% % called 
% \lky{termed}
% the semantic map of $x$. 
% % By creating the semantic map of each appearing forms, which are often selected by linguists, a conceptual space is constructed in a bottom-up manner.
% \lky{By creating the semantic map of each form (typically selected by linguists), a conceptual space is constructed in a bottom-up manner.}
% \lky{However,} this manner is always manual and demanding when the number of form instances and concepts becomes large. To handle this issue,~\citet{liu2024top} relax this hypothesis and convert the graph topology into a maximum spanning tree. This guarantees the overall connectivity while reducing the number of edges as less as possible. Therefore, this top-down method provides an efficient way to automatically generate the network.