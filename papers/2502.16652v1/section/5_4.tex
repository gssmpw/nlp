\subsection{Ablation study}
\label{subsec:5_4}
We conduct an ablation study using the ScanNet dataset on different hyper-parameters of \nickname to measure the contribution of each component.

\noindentbold{Product Quantization}
PQ introduces a trade-off between memory usage, computational efficiency, and accuracy. 
To better understand the balance between computational cost and localization quality, we conduct an ablation study by varying the number of sub-vectors. We evaluate performance at sub-vector sizes of 64, 128, and 256. Notably, these settings correspond to bit-size reductions of 1/32, 1/16, and 1/8 of the original CLIP feature, respectively. We measure the query distance computation time for one million data points, averaging results over 100 iterations for efficiency measure. Our findings reveal a favorable trade-off between quantization performance and accuracy (see~\cref{fig:ablation}-(b)) in the Pareto front with our PQ configurations. This achieves a balance that maximizes memory and computational efficiency while minimizing any loss in accuracy.

\noindentbold{Top-$k$ Gaussians}
We examine the influence of the number of Gaussians assigned per ray. This parameter affects both memory requirements and computation, serving as a critical factor in overall performance. The ratio of pruned Gaussians and the mIoU results from different $k$ are presented in~\cref{fig:ablation-a}. We observe that increasing the aggregating number of Gaussians per ray improves localization performance; however, it results in higher memory consumption and the number of occupied Gaussians, indicating a clear trade-off.

