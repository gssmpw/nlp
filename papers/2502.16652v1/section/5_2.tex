\subsection{3D object localization}
\label{subsec:5_2}

\noindentbold{Settings}
Similar to the 3D object selection task, we calculate the cosine similarity between text query and 3D features embedded in each Gaussian. By thresholding the similarity, we identify the 3D Gaussians relevant to the given text query. To measure volume-aware localization evaluation, we propose a protocol to measure the IoU of 3D Gaussians that expands the traditional metric of point cloud-based approaches by incorporating volumetric information of 3D Gaussians.

\noindentbold{Novel evaluation protocol for 3D localization in 3DGS}
Unlike conventional evaluation protocol for the 3D localization task in point clouds, it is tricky to evaluate 3D localization performance in 3D Gaussians~\cite{3dgs}. This is primarily due to the un-deterministic structure of Gaussian distribution. To address this issue, we compute 3DGS pseudo-labels for evaluating the 3DGS localization in a volume-aware way. The details can be found in the supplementary material.

Given the ground truth, we measure IoU considering the spatial significance of each Gaussian and define a significant score $d_{i}$ for each Gaussian $\theta_i$ with its scale $\mathbf{s}_i = [s_{ix}, s_{iy}, s_{iy}]$ and opacity $\alpha_i$ as $d_{i} = s_{ix}s_{iy}s_{iz} \alpha_i$, where $s_{ix}s_{iy}s_{iz}$ denotes a relative ellipsoid volume of a Gaussian $\theta_i$. With the obtained significant scores $\mathbf{d}=[d_1, d_2, ..., d_{N}]$, we compute weighted IoU of 3D Gaussians to approximate volumes. 
The proposed metric is designed to assign a larger weight to the Gaussians with higher significant scores, when measuring IoU. Figure \ref{fig:significant_score} shows that the impact of each Gaussian on the scene extremely varies depending on their significant scores, which demonstrates the necessity of the proposed IoU metric on 3D Gaussians that regards unequal contributions of each Gaussian. 

\noindentbold{Results}
We report the 3D localization performance on the Scannet dataset in~\Tref{table:3d_loc}. 
The 2D rasterization-based methods~\cite{langsplat, legaussian} struggle to achieve precise activations for 3D localization.
They inherently face challenges when applying for 3D tasks because they need to render 2D images for the scene interaction.
Even with the 3D space search method, OpenGaussian~\cite{open_gaussian}, our model consistently demonstrates superior performance and achieves higher accuracy in localization. 
Figure \ref{fig:6_seg} also shows that LangSplat-m and LEGaussians-m fail to properly localize the objects, and OpenGaussian misses queried objects in the scene.
