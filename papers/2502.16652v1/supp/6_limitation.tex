\section{Broader Applications and Limitations}
\label{discussion}
\paragraph{Broader application}
The proposed method offers the potential for broader applications across diverse scenarios. 
Similar to works that explore the application in point cloud~\cite{concept_fusion, open_scene} and MLP-based methods~\cite{lerftogo2023}, our approach, using 3DGS, can be extended to support various input modalities, such as click or image queries, by leveraging a self-referencing mechanism. Additionally, integrating our method with Large Language Models (LLMs) could facilitate dialogue-based interactions, allowing users to dynamically issue commands or explore the environment.
This integration suggests promising avenues for developing 3D interactive systems that go beyond simple search tasks.

Furthermore, applying the method to canonical forms could support dynamic 3D scenes~\cite{deformable3dgs, gagaussian}. This adaptation would extend the applicability of our approach beyond static environments, demonstrating its versatility in handling complex, real-world scenarios.

\paragraph{Limitation}
While our method has demonstrated robust performance across diverse combinations of nouns and adjectives (\eg, ``tea in a glass,'' wavy noodles,'' and red light'' in~\Fref{fig:3d_obj_sel} and~\Fref{fig:large_scene}) as well as unfamiliar nouns (e.g., nori,'' waldo,'' and safety cone''), without additional training, generalization remains an area for improvement. Exploring additional training techniques for Product Quantization (PQ) could further enhance the method’s capabilities. 
Further exploration of Product Quantization (PQ) training, such as using more diverse datasets or finer-grained query representations, could enhance adaptability across varied contexts.

Despite its advantages, some limitations of the proposed method have also been identified, particularly related to CLIP features. Occasionally, related but distinct objects are simultaneously activated for a given query. For instance, the query ``red apple'' might activate non-red apples or unrelated red objects. This stems from CLIP's semantic associations and could be mitigated with post-processing techniques like re-ranking to improve query specificity.

Lastly, similar to previous methods~\cite{langsplat, legaussian, open_gaussian}, ours also requires to set an appropriate threshold. In this study, we utilize a fixed similarity threshold employed a fixed similarity threshold across all scenes, ensuring stable and reproducible results. However, optimizing thresholds for specific scenarios or implementing dynamic adjustments could further refine localization accuracy in diverse environments.
