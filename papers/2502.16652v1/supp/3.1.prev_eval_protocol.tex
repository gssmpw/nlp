\section{Evaluation Protocols}
\label{metric}

\paragraph{Limitations of existing evaluation protocols}
Compared to the previous works, such as LERF~\cite{lerf}, LEGaussian~\cite{legaussian}, and LangSplat~\cite{langsplat}, our method challenges to leverage the 3D Gaussian representation into the 3D scene understanding tasks. Similar to ours, OpenGaussian~\cite{open_gaussian} is a concurrent work that aims at the open-vocabulary 3D semantic segmentation task as well. However, unlike OpenGaussian, we introduce a new evaluation criterion specialized for the 3D Gaussians, instead of using point cloud-specific evaluations.

OpenGaussian~\cite{open_gaussian} computes evaluation metrics directly from 3D Gaussians, using ScanNet~\cite{dai2017scannet} ground truth point clouds with semantic labels. It aligns Gaussian centers~${ \bmu }$ with dataset points~${ [x,y,z] }$ and keeps both~${ \bmu }$ and the number of Gaussians~$N$ fixed during parameter optimization. This differs from vanilla 3D Gaussian Splatting~\cite{3dgs}. As shown in~\cref{fig:metric_qualitative}, their approach introduces significant quality issues, influenced by the evaluation metric. However, the reason behind this optimization trick is related to evaluation.

The evaluation by OpenGaussian involves predicting labels for each Gaussian and measuring their alignment with the ground truth point cloud using Intersection over Union (IoU). To compute IoU, the overlap (intersection) and total extent (union) of the points are calculated between the 3D Gaussians' center locations~$\{ \bmu \}$ and the ground truth point clouds at fixed positions. As we discussed, since OpenGaussian does not update the locations of the 3D Gaussians, which is identical to the locations of the 3D ground truth points, they simply count the overlap and union without considering the volumetric properties of the 3D Gaussians.

We claim that such an evaluation protocol has two dominant issues. 
First, by pre-defining the number of Gaussians as well as the center locations of the 3D Gaussians, the optimized 3D Gaussians produce degraded rendering quality as shown in~\cref{fig:metric_qualitative}, which is not a practical solution. 
Second, the aforementioned IoU is calculated only with the number of 3D Gaussians, which does not consider the significance of each Gaussian having different shapes and densities.

