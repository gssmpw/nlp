\section{Introduction}

Over the past decades, platforms such as Facebook, Instagram, and Twitter have played pivotal roles in online social interactions, significantly influencing political campaigns, information dissemination, digital community building, and shaping cultural dynamics. Previous studies have underscored the necessity of analyzing social platforms to understand social phenomena \cite{hate-speech-reddit-savvas, moderation-importance-qanon, facebook-1}. In this context, Discord emerges as a unique platform due to its decentralized nature, where moderation is explicitly delegated to users themselves\footnote{https://discord.com/community/your-responsibilities-as-a-discord-moderator-discord}. This model creates a potentially volatile environment for exposing social and cultural phenomena that might be overlooked on more centralized platforms. As other social networks have recently begun adopting trends of user-driven moderation, such as the use of community notes for fact-checking, Discord, with its longstanding user-moderation approach, offers valuable insights into what the future of decentralized moderation systems might evolve into.

The study of the phenomena generated by these online interactions was only possible due to data availability, and the emergence of new research areas interested in them opened up discussions on the pros and cons of the wide adoption of online social media. However, access to data from these platforms has become increasingly restricted. Recently, Twitter has severely limited its public Application Programming Interface (API)\footnote{https://docs.x.com/x-api/introduction}, while Meta (the parent company of Facebook and Instagram) also tightened data availability through its APIs\footnote{https://developers.facebook.com/docs/}. These changes have created significant challenges for researchers relying on these data sources to study social and cultural phenomena. 

\begin{figure*} [!h]
    \centering
    \includegraphics[width=1\linewidth]{gaming2.png}
    \caption{Discord interface, with the list of servers on the left, the distinct channels within the selected server, the central panel displaying messages from the active channel, and the list of connected users on the right.}
    \label{fig:disc}
\end{figure*} 

In contrast to these restrictive trends, data from Discord is available through its API. APIs provide a secure and standardized method for accessing data, ensuring compliance with user privacy guidelines and protection against unauthorized data collection practices. Discord's API allows researchers to collect data from public servers in a structured and ethical manner, adhering to the platform’s policies. This enables the study of large-scale interactions without compromising user privacy or data integrity.

While its API provides researchers with valuable access to public data, Discord’s versatility as a platform further amplifies its research potential. Initially developed as a communication tool for gamers, Discord has transformed into a dynamic platform that brings together communities with a wide range of interests \cite{johnson2022embracing}. Alongside private communication, its public servers host thousands of members, facilitating open conversations. This structure makes it a valuable context for studying social dynamics and digital communities, particularly in public settings where data can be collected. Despite being an emerging platform with an enormous user base and highly active communities, there remains a significant gap in works and analyses regarding Discord, with far fewer studies compared to other social networks.

In this regard, this paper introduces the most extensive Discord dataset available to date, comprising 2,052,206,308 messages from 4,735,057 unique users across 3,167 servers -- approximately 10\% of the servers listed in Discord's Discovery tab, a feature designed to highlight public servers that users can join. The dataset spans messages from 2015 to the end of 2024, capturing a wide variety of user interactions and community dynamics. By focusing on public servers, this dataset provides a robust and diverse foundation for exploring Discord as a unique social platform, offering insights into its distinctive structure and vibrant community-driven interactions. All data collection adhered strictly to Discord's API guidelines, and anonymization techniques were applied to ensure compliance with privacy standards.

This dataset marks a significant contribution to the study of digital communities, particularly in unmoderated and decentralized social platforms. Its scale and diversity enable comprehensive investigations into key topics such as governance models, moderation strategies, and the dissemination of information within dynamic online environments. Furthermore, it facilitates meaningful comparisons between Discord and traditional social networks, highlighting the unique interaction patterns fostered by community-driven moderation. The availability of this dataset opens new avenues for both theoretical and practical advancements, including the development of tools to analyze moderation policies, detect harmful behaviors, and model large-scale social interactions. By presenting this resource, we aim to cover the existing gap in research and studies on Discord, providing a valuable foundation for advancing the understanding of this increasingly relevant platform within computational social science.


%By providing this dataset, we contribute to the understanding of emerging social dynamics in decentralized platforms, enabling comparisons with traditional social networks and fostering research into community governance, moderation practices, and information dissemination. This study also highlights the role of secure and accessible APIs as a critical tool for computational social science research, especially as access to data on traditional platforms becomes increasingly limited. \yan{acho q da pra refazer esse final pra falar mais da relevancia do dataset e possíveis usos}

