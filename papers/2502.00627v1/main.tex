%File: anonymous-submission-latex-2025.tex
\documentclass[letterpaper]{article} % DO NOT CHANGE THIS
\usepackage{aaai25}  % DO NOT CHANGE THIS
\usepackage{times}  % DO NOT CHANGE THIS
\usepackage{helvet}  % DO NOT CHANGE THIS
\usepackage{courier}  % DO NOT CHANGE THIS
\usepackage[hyphens]{url}  % DO NOT CHANGE THIS
\usepackage{graphicx} % DO NOT CHANGE THIS
\usepackage{xcolor}
\usepackage{afterpage}
\usepackage{multicol}
\usepackage{longtable}
\usepackage{booktabs}
\usepackage{lscape} % Para paisagem, caso seja necessário

% \usepackage{ulem} % Para riscar texto
% \normalem % Evita que o ulem modifique o comportamento de \emph
\urlstyle{rm} % DO NOT CHANGE THIS
\def\UrlFont{\rm}  % DO NOT CHANGE THIS
\usepackage{natbib}  % DO NOT CHANGE THIS AND DO NOT ADD ANY OPTIONS TO IT
\usepackage{caption} % DO NOT CHANGE THIS AND DO NOT ADD ANY OPTIONS TO IT
\frenchspacing  % DO NOT CHANGE THIS
\setlength{\pdfpagewidth}{8.5in} % DO NOT CHANGE THIS
\setlength{\pdfpageheight}{11in} % DO NOT CHANGE THIS


\usepackage{aaai25}  % DO NOT CHANGE THIS
\usepackage{times}  % DO NOT CHANGE THIS
\usepackage{helvet}  % DO NOT CHANGE THIS
\usepackage{courier}  % DO NOT CHANGE THIS
\usepackage[hyphens]{url}  % DO NOT CHANGE THIS
\usepackage{graphicx} % DO NOT CHANGE THIS
\urlstyle{rm} % DO NOT CHANGE THIS
\def\UrlFont{\rm}  % DO NOT CHANGE THIS
\usepackage{natbib}  % DO NOT CHANGE THIS AND DO NOT ADD ANY OPTIONS TO IT
\usepackage{caption} % DO NOT CHANGE THIS AND DO NOT ADD ANY OPTIONS TO IT
\DeclareCaptionStyle{ruled}{labelfont=normalfont,labelsep=colon,strut=off} % DO NOT CHANGE THIS
\frenchspacing  % DO NOT CHANGE THIS
\setlength{\pdfpagewidth}{8.5in}  % DO NOT CHANGE THIS
\setlength{\pdfpageheight}{11in}  % DO NOT CHANGE THIS
%
% These are recommended to typeset algorithms but not required. See the subsubsection on algorithms. Remove them if you don't have algorithms in your paper.
\usepackage{algorithm}
\usepackage{algorithmic}
% Checklist macros
\usepackage{xcolor}
\newcommand{\answerYes}[1]{\textcolor{blue}{#1}} 
\newcommand{\answerNo}[1]{\textcolor{teal}{#1}} 
\newcommand{\answerNA}[1]{\textcolor{gray}{#1}} 
\newcommand{\answerTODO}[1]{\textcolor{red}{#1}} 
%
% These are recommended to typeset algorithms but not required. See the subsubsection on algorithms. Remove them if you don't have algorithms in your paper.
\usepackage{algorithm}
\usepackage{algorithmic}

%
% These are are recommended to typeset listputgs but not required. See the subsubsection on listing. Remove this block if you don't have listings in your paper.
\usepackage{newfloat}
\usepackage{listings}
\DeclareCaptionStyle{ruled}{labelfont=normalfont,labelsep=colon,strut=off} % DO NOT CHANGE THIS
\lstset{%
	basicstyle={\footnotesize\ttfamily},% footnotesize acceptable for monospace
	numbers=left,numberstyle=\footnotesize,xleftmargin=2em,% show line numbers, remove this entire line if you don't want the numbers.
	aboveskip=0pt,belowskip=0pt,%
	showstringspaces=false,tabsize=2,breaklines=true}
\floatstyle{ruled}
\newfloat{listing}{tb}{lst}{}
\floatname{listing}{Listing}
%
% Keep the \pdfinfo as shown here. There's no need
% for you to add the /Title and /Author tags.
\pdfinfo{
/TemplateVersion (2025.1)
}


% DISALLOWED PACKAGES
% \usepackage{authblk} -- This package is specifically forbidden
% \usepackage{balance} -- This package is specifically forbidden
% \usepackage{color (if used in text)
% \usepackage{CJK} -- This package is specifically forbidden
% \usepackage{float} -- This package is specifically forbidden
% \usepackage{flushend} -- This package is specifically forbidden
% \usepackage{fontenc} -- This package is specifically forbidden
% \usepackage{fullpage} -- This package is specifically forbidden
% \usepackage{geometry} -- This package is specifically forbidden
% \usepackage{grffile} -- This package is specifically forbidden
% \usepackage{hyperref} -- This package is specifically forbidden
% \usepackage{navigator} -- This package is specifically forbidden
% (or any other package that embeds links such as navigator or hyperref)
% \indentfirst} -- This package is specifically forbidden
% \layout} -- This package is specifically forbidden
% \multicol} -- This package is specifically forbidden
% \nameref} -- This package is specifically forbidden
% \usepackage{savetrees} -- This package is specifically forbidden
% \usepackage{setspace} -- This package is specifically forbidden
% \usepackage{stfloats} -- This package is specifically forbidden
% \usepackage{tabu} -- This package is specifically forbidden
% \usepackage{titlesec} -- This package is specifically forbidden
% \usepackage{tocbibind} -- This package is specifically forbidden
% \usepackage{ulem} -- This package is specifically forbidden
% \usepackage{wrapfig} -- This package is specifically forbidden
% DISALLOWED COMMANDS
\nocopyright % -- Your paper will not be published if you use this command
% \addtolength -- This command may not be used
% \balance -- This command may not be used
% \baselinestretch -- Your paper will not be published if you use this command
% \clearpage -- No page breaks of any kind may be used for the final version of your paper
% \columnsep -- This command may not be used
% \newpage -- No page breaks of any kind may be used for the final version of your paper
% \pagebreak -- No page breaks of any kind may be used for the final version of your paperr
% \pagestyle -- This command may not be used
% \tiny -- This is not an acceptable font size.
% \vspace{- -- No negative value may be used in proximity of a caption, figure, table, section, subsection, subsubsection, or reference
% \vskip{- -- No negative value may be used to alter spacing above or below a caption, figure, table, section, subsection, subsubsection, or reference

\setcounter{secnumdepth}{0} %May be changed to 1 or 2 if section numbers are desired.

% The file aaai25.sty is the style file for AAAI Press
% proceedings, working notes, and technical reports.
%

\newcommand{\yan}[1]{{\emph{\color{magenta}#1 -- \textbf{Yan}}}}
\newcommand{\caio}[1]{{\emph{\color{blue}#1 -- \textbf{Caio}}}}
\newcommand{\gi}[1]{{\emph{\color{red}#1 -- \textbf{Gi}}}}
\newcommand{\pr}[1]{{\emph{\color{purple}#1 -- \textbf{Robles}}}}
\newcommand{\lu}[1]{{\emph{\color{brown}#1 -- \textbf{Luísa}}}}
\newcommand{\vic}[1]{{\emph{\color{pink}#1 -- \textbf{vic}}}}


% Title
\title{Discord Unveiled: A Comprehensive Dataset of Public Communication (2015-2024)}

% Your title must be in mixed case, not sentence case.
% That means all verbs (including short verbs like be, is, using,and go),
% nouns, adverbs, adjectives should be capitalized, including both words in hyphenated terms, while
% articles, conjunctions, and prepositions are lower case unless they
% directly follow a colon or long dash
\author{
    %Authors
    % All authors must be in the same font size and format.
    Yan Aquino \textsuperscript{\rm 1},
    Pedro Bento \textsuperscript{\rm 1}, 
    Arthur Buzelin \textsuperscript{\rm 1},
    Lucas Dayrell \textsuperscript{\rm 1}, 
    Samira Malaquias \textsuperscript{\rm 1}, \\
    Caio Santana \textsuperscript{\rm 1}, 
    Victoria Estanislau \textsuperscript{\rm 1},
    Pedro Dutenhefner \textsuperscript{\rm 1},
    Guilherme H. G. Evangelista \textsuperscript{\rm 1},\\
    Luisa G. Porfírio \textsuperscript{\rm 1},
    Caio Souza Grossi \textsuperscript{\rm 1},
    Pedro B. Rigueira \textsuperscript{\rm 1},\\
    Virgilio Almeida \textsuperscript{\rm 1},
    Gisele L. Pappa \textsuperscript{\rm 1},
    Wagner Meira Jr. \textsuperscript{\rm 1}
}
\affiliations{
    %Afiliations
    \textsuperscript{\rm 1}Universidade Federal de Minas Gerais - UFMG\\
    % If you have multiple authors and multiple affiliations
    % use superscripts in text and roman font to identify them.
    % For example,

    % Sunil Issar\textsuperscript{\rm 2},
    % J. Scott Penberthy\textsuperscript{\rm 3},
    % George Ferguson\textsuperscript{\rm 4},
    % Hans Guesgen\textsuperscript{\rm 5}
    % Note that the comma should be placed after the superscript

    
    Belo Horizonte, Brazil\\
    % email address must be in roman text type, not monospace or sans serif
    \{yanaquino, pedro.bento, arthurbuzelin, lucasdayrell, samiramalaquias, caiosantana, victoria.estanislau, guilherme.evangelhista, luisagontijo, caio.grossi, pedrobacelar.rigueira, virgilio, glpappa, meira\}@dcc.ufmg.br, pedroroblesduten@ufmg.br
    
%
% See more examples next
}

%Example, Single Author, ->> remove \iffalse,\fi and place them surrounding AAAI title to use it
\iffalse
\title{My Publication Title --- Single Author}
\author {
    Author Name
}
\affiliations{
    Affiliation\\
    Affiliation Line 2\\
    name@example.com
}
\fi

\iffalse
%Example, Multiple Authors, ->> remove \iffalse,\fi and place them surrounding AAAI title to use it
\title{My Publication Title --- Multiple Authors}
\author {
    % Authors
    First Author Name\textsuperscript{\rm 1},
    Second Author Name\textsuperscript{\rm 2},
    Third Author Name\textsuperscript{\rm 1}
}
\affiliations {
    % Affiliations
    \textsuperscript{\rm 1}Affiliation 1\\
    \textsuperscript{\rm 2}Affiliation 2\\
    firstAuthor@affiliation1.com, secondAuthor@affilation2.com, thirdAuthor@affiliation1.com, caio.grossi
}
\fi


% REMOVE THIS: bibentry
% This is only needed to show inline citations in the guidelines document. You should not need it and can safely delete it.
\usepackage{bibentry}
% END REMOVE bibentry

\begin{document}

\maketitle

\begin{abstract}
Discord has evolved from a gaming-focused communication tool into a versatile platform supporting diverse online communities. Despite its large user base and active public servers, academic research on Discord remains limited due to data accessibility challenges. This paper introduces \textbf{Discord Unveiled: A Comprehensive Dataset of Public Communication (2015-2024)}, the most extensive Discord public server's data to date. The dataset comprises over \textbf{2.05 billion messages} from \textbf{4.74 million users} across \textbf{3,167 public servers}, representing approximately 10\% of servers listed in Discord’s Discovery feature. Spanning from Discord’s launch in 2015 to the end of 2024, it offers a robust temporal and thematic framework for analyzing decentralized moderation, community governance, information dissemination, and social dynamics. Data was collected through Discord’s public API, adhering to ethical guidelines and privacy standards via anonymization techniques. Organized into structured JSON files, the dataset facilitates seamless integration with computational social science methodologies. Preliminary analyses reveal significant trends in user engagement, bot utilization, and linguistic diversity, with English predominating alongside substantial representations of Spanish, French, and Portuguese. Additionally, prevalent community themes such as social, art, music, and memes highlight Discord’s expansion beyond its gaming origins.


\end{abstract}

\documentclass[../main.tex]{subfiles}
\graphicspath{{../images/}}
\makeatletter
\def\input@path{{../images/}}
\makeatother
\begin{document}
\section{Introduction}
\begin{figure}
\centering
\begin{tikzpicture}
\node[inner sep=0pt] (ws) at (0, 0) {
\includegraphics[height=.4\textwidth, trim={10cm 0 10cm 0},clip]{world_space.png}};
\node[inner sep=0pt] (cs) at (6,0) {\includegraphics[height=.4\textwidth, trim={10cm 1cm 10cm 4cm},clip]{conf_space.png}};
\end{tikzpicture}
\vspace{-5pt}
\label{fig:pbrm_intro}
\caption{\textbf{Left}: Shows world space obstacles as grey spheres. Robots start and goal configuration is colored red and green, respectively. Configurations along the computed path are colored transparent blue. \textbf{Right:} Mapped world space scenario to configuration space. Obstacle region is the grey mesh. Red spheres are collision-free regions computed by the neural SCDF. The optimized shortest path in the convex corridor is the blue curve.}
\vspace{-25pt}
\end{figure}
Motion planning is the problem of finding a collision-free trajectory that connects a given start and goal configuration. The planning takes place in the configuration space of the robot. For single body robots, like mobile robots or drones, the configuration space and the world space are usually the same. This simplifies the planning, since explicit obstacle representations are available which enables geometrical tools like separating hyperplanes, smallest distance to obstacles etc., to be used when designing motion planning algorithms. For multi-body robots like manipulators, the situation is completely different. The world space obstacles are usually mapped to non-convex regions, and to make the problem even harder, the mapping is usually not known. Forming explicit representations of the obstacle region in the configuration space is usually too expensive or intractable. Despite all of this, sampling based planners are used with great success, which mainly is due to their use of implicit representations of the obstacle region. The basic idea is to construct a graph in the configuration space that covers and connects the collision-free region. From this graph, a path can be extracted that connects a given start and goal configuration. The approach is computationally expensive, since the graph is constructed with the smallest geometrical building block available, points, which represents a collision-check. Furthermore, the extracted paths from the graph are non-smooth and jagged due to the stochastic nature of the approach. This adds an additional post-processing step to the process, where the paths are shortcutted and smoothened, before the path can be used for tracking. Clearly a lot of time is invested to form this graph and produce smooth paths. Thus, if the obstacles start to move, then all of this work is done in no use, since all points that make up this graph need to be re-verified, which is simply too time consuming to be done in real time.
\\\\
In this work, we want to address the existing drawbacks of the sampling based planners. Our main contribution is an improved motion planner where each vertex in the graph covers a collision-free region in the form of a sphere instead of a point and where the edges are formed with neighboring intersecting spheres. This representation has the advantage of instead of returning piecewise linear paths, returning a sequence of overlapping spheres, i.e. a convex corridor, that connects a given start and goal configuration, illustrated in Figure \ref{fig:pbrm_intro}. This convex corridor allows us to use convex optimization to produce smooth trajectories, instead of computationally expensive post-processing methods. The representation further allows us to estimate the coverage of the collision-free space, which gives us awareness and feedback in the offline roadmap construction phase. Finally, our representation is simple to adapt to moving obstacles, simply requery for the new radii and recheck for intersections. 
\\\\
The spherical collision-free regions are formed using a signed distance function (SDF), which is a function that returns the smallest distance from an arbitrary point to the boundary of an obstacle. As the name implies, the distance is signed, thus if the point is inside the obstacle it is negative otherwise positive. If the distance is positive, a sphere with radius equal to the distance is guaranteed to cover a collision-free region. Using an SDF in motion planning is not new, but what is novel about our approach is that we express the distance in the configuration space instead of the world space and by doing so allows us to form these convex collision-free regions. We refer to the resulting SDF as a signed configuration distance function (SCDF). Computing an SCDF analytically is non-trivial, our approach is therefore to parameterize the SCDF with a deep neural network and learn the mapping by supervised learning. Our resulting neural SCDF can compute distances for different parameter values of obstacle shapes and we also show how multiple distances can be combined, thus making our approach flexible.
\section{Related work}
Motion planning algorithms can roughly be divided into three families, grid-based, sampling based and optimization based methods. Grid-based methods (GBM) discretize the planning space from which a graph is then compiled. A standard search method is A$^\star$ \citep{a_star}, which is classified as an \textit{informed} search method, since it employs a heuristic function to speed up the search. A$^\star$ guarantees to return an optimal path at the level of discretization used. GBMs usually discretize the planning space by a regular lattice and this limits the GBMs to problems with low dimensionality due to the curse of dimensionality. Thus, GBMs are usually limited to single-body robots where the degrees of freedom (DOF) are low. To overcome the inherent scaling problem with the GBMs, stochastic methods are usually used for multi-body robots. These methods are termed as sampling-based methods (SBM) and core members within this family are the rapidly-exploring random trees (RRT) \citep{rrt} and the probabilistic roadmap (PRM) \citep{prm}. RRT grows a tree from the start configuration and explores the collision-free region in a rapid way until it is able to connect to the goal region. RRT is usually improved by bi-directional planning \citep{rrt_connect}, i.e. an additional tree is grown from the goal configuration and the trees are tested for connection after any tree has been expanded. RRT is a single-query method, thus it searches for a path from scratch each time it is queried. Contrary to this, PRM is a multi-query method, which solves for multiple queries without starting from scratch. PRM does this by creating a roadmap (graph) that covers the collision-free space as an offline step. The graph is then used to solve for multiple queries. PRMs are used in cases where the environment does not change since the extra offline step is too computationally costly and needs to be re-done if the environment is changed. In our work, we address this inherent issue by using a different roadmap representation. Our vertices in the graph cover a collision-free region in the form of spheres and we form the edges by checking for intersecting spheres. If something in the environment changes, we recompute the spheres radii and recheck the intersections, without relying on collision detection. We use a trained neural network to compute the sphere radius, therefore querying for the radius can be done fast, hence our representation enables the PRM for dynamic environments.
\\\\
In the recent decades, optimization based methods (OBM) \citep{chomp, schulman, itomp, stomp} have been introduced as an alternative to SBM for multi-body robots. Like the SBM, the OBMs scale well to higher dimensional problems and produce smoother motion. It is common to use a SDF in the optimization since it is a smooth function, thus enabling gradient-based methods. However, the standard way of expressing the SDF is in world space. The distance therefore needs to be mapped to the configuration space by the forward kinematics. This mapping makes the optimization problem a non-linear program (NLP), which is computationally expensive to solve. Recently, a different approach has been proposed. In \cite{mp_gcs} motion planning is formulated as a convex optimization problem by using the graph of convex sets framework \citep{gcs}. The underlying idea is to decompose the collision-free space into intersecting convex sets from which a convex optimization problem is formulated. In cases where an explicit representation of the obstacles in the configuration space exists, like for single-body robots, creating collision-free convex regions can be done fast \citep{iris}. For multi-body robots, this is non-trivial. Existing work does this successfully \citep{iris_nlp, iris_c} by an optimization based approach, but the methods are still too time consuming to be used in the presence of moving obstacles. Our approach is instead to use deep learning to learn an SDF expressed in the configuration space. With this, we can query for shortest distances to the collision boundary, which allows us to expand spherical regions which are collision-free. Our approach is fast and therefore enables our suggested roadmap planner to be used in dynamic environments.
\\\\
Recent research has focused on learning collision detection \citep{fk_kernel_distance, diffco, graphdistnet} by predicting the signed distance between the robot links and the surrounding obstacles in the world space. The learned SDF is used in trajectory optimization but since the distance is expressed in the world space, the problem becomes an NLP and therefore takes a long time to solve. We take a novel approach and suggest to instead express the signed distance in the configuration space. This allows us to improve the PRM at the same time as it enables convex optimization for trajectory optimization, which runs faster and is more reliable than NLP solvers. In \cite{cspf} a learned signed distance function in the configuration space is proposed similar to our approach. However, their approach is restricted to point cloud representations, while we propose to represent the obstacles as parameterized geometric shapes, e.g. spheres. Furthermore, we also show how to use our learned SCDF to improve an existing roadmap planner.
\section{Problem formulation}
A robot is located in the world space, $\W \subset \R^3 $. The unique location of the robot is given by its configuration $\q \in \C$, where $\C$ is the configuration space. The set of points covered by the robots bodies at a certain configuration is expressed as $\B(\q) \subset \W$. The robot is surrounded by $\NrObst$ obstacles $\O = \bigcup_{i=1}^{\NrObst} \O_i$, where  $\O_i \subset \W$. The representation of the obstacle in the configuration space is the set $\C\O_i = \{\q \in \C \: |\: \B(\q) \cap \O_i \neq \emptyset \}$. The obstacle space is formed as $\Co = \bigcup_{i=1}^{\NrObst} \C \O_i$. The complement is referred to as the free space, $\Cf = \C \setminus \Co$. The path planning problem is a tuple, ($\Cf$, $\qStart$, $\qGoal$), where we want to connect a query pair, consisting of a start, $\qStart$, and goal configuration, $\qGoal$, with a geometric path, $\q(s): [0, 1] \mapsto \Cf$, such that $\q(0)=\qStart$ and $\q(1)=\qGoal$, or report correctly when such a path does not exist.
\end{document}



\section{What is Discord?}

Discord is a communication platform launched in 2015 by Discord Inc., designed to enable real-time interaction through text, voice, and video. Initially popular within gaming communities, Discord has expanded to accommodate diverse user groups, including educators, hobbyists, and professional teams\footnote{https://discord.com/safety/360044149331-what-is-discord}. Similar to platforms like Telegram and WhatsApp, Discord allows users to create and join customizable group spaces for synchronous and asynchronous communication, which can be either public or private.


Discord is organized into ``servers", virtual spaces designed for communities to connect, share content, and engage in conversations. Each server is divided into ``channels," which are dedicated spaces for specific types of conversations or activities. Text channels are used for written discussions, enabling members to exchange messages, share links, images, and other media. Voice channels, on the other hand, facilitate real-time audio conversations, often used for meetings, gaming sessions, or casual chats, and can also support video and screen sharing. All of these dynamics can be seen in Figure \ref{fig:disc}. Additionally, servers can be customized with ``roles" which are permission-based assignments that define what actions members can perform within the server. Roles can grant or restrict access to specific channels, allow users to manage messages, ban or mute members, or even adjust server settings. Combined with bots to automate tasks and server-specific rules to guide member behavior, this flexible architecture supports a wide range of communities, from small friend groups to large public forums.


A distinguishing feature of Discord is its user-driven moderation. While traditional social networks such as Twitter, Facebook, or YouTube have historically relied on centralized moderation managed by the platform itself \cite{wilson2020hate}, there is a very recent trend toward decentralizing moderation, exemplified by initiatives like community-based fact-checking \cite{balasubramanian2024publicdatasettrackingsocial}. Discord has always explicitly delegated the responsibility of rule-setting, behavior management, and access control to server administrators and moderators. This long-standing approach has fostered a well-established environment of non-centralized moderation, offering valuable insights into the dynamics and challenges of this model as other platforms begin to adopt similar strategies. For instance, as shown by \cite{moderation-challenges}, this flexibility can also be exploited to facilitate the presence of extremist and hateful groups and networks.

Bots are another key element of Discord’s ecosystem, setting it apart from other social networks. Unlike bots on other platforms, which are often limited to automated content posting or analytics, Discord bots are highly customizable and play an interactive role \cite{moderation-discord-bots, bots-discord}. They can automate moderation tasks, provide community engagement features (e.g., games, polls, or reminders), and integrate external services like music streaming, analytics tools, and generative AI services like Midjourney and ChatGPT. Discord actively encourages the creation and integration of bots by providing developers with extensive API documentation, support, and a bot-friendly environment, fostering innovation and customization within the platform \footnote{https://discord.com/developers/docs/intro}. Their extensive use has become central to how users interact within servers and how these communities are structured and managed.

To facilitate the discovery of public servers, Discord introduced the \textit{Server Discovery} feature. This feature allows users to search, explore and join public servers that align with their interests\footnote{https://support.discord.com/hc/en-us/articles/360023968311-Server-Discovery}. Servers listed in the Discovery tab must adhere to specific guidelines, including maintaining a welcoming environment, avoiding graphic or sexual content, and having an accurate server name and description\footnote{https://support.discord.com/hc/en-us/articles/4409308485271-Discovery-Guidelines}. These servers are the focus of our research, as they offer a curated look at communities
built around a wide variety of topics, each with at least 1000 members.
%with public servers being discoverable through Discord’s Discovery feature, provided they meet specific guidelines regarding content and moderation. 
% This work focuses exclusively on public Discord servers available in the Discovery feature
 %, from gaming and technology to education and entertainment.

%\yan{add um ultimo paragrafo?}

\section{Related Work}
% \subsection{Vision Language Model}
% 시각장애인에서 상황을 설명할 DB가 없으니 만들었다. 그리고 이를 VLM에 튜닝했다.
\subsection{Technical approaches for assisting the visually-impaired}


\subsection{Datasets for visual instruction tuning}


\section{Dataset Generation}
\label{sec:dataset}
\revise{
To train the proposed GNN, we constructed a dataset of building structures and a subset of these structures were subjected to fire simulations using FEA. The dataset generation process is illustrated in \figref{fig:dataset_generation_procedure}. Initially, a total of 33,000 building structures with geometrical details, material properties, and gravity loads were created. Due to randomness in generating these structures, a filter is applied to remove unreasonable data after gravity load simulation, which included 15,377 structures. A trade-off between computational feasibility and model performance is made among the remaining 17,623 structures. As further labeling structures with MIDR requires resource-intensive fire simulations via OpenSeesRT, a large proportion of 16,050 structures is selected as unlabeled dataset. On the other hand, each of the other 1,573 structures was further subjected to 30 different fire simulations, forming the labeled dataset containing $1,573\times 30 = 47,190$ fire cases.} This section details the step-by-step process for generating the dataset, including geometry creation, material property assignment, and simulations due to gravity loads and fire scenarios. 
% To train the proposed neural network, we constructed a dataset comprising building structure data and a subset of fire scenario data. The dataset generation process is illustrated in \figref{fig:dataset_generation_procedure}. 
% A total of 33,000 building structures with geometric details, material properties, and gravity loads were initially created. Out of these, 3,000 structures were selected as labeled data, and the remaining 30,000 were designated as unlabeled data. Further, about half of them filtered out due to instability under gravity loads only. 
\begin{figure*}[h!]
    \centering
    \includegraphics[width=0.8\linewidth]{figures/dataset_filter_procedure.pdf}
    \caption{Workflow for dataset generation (geometry, material property, gravity loads, and fire scenarios).}
    \label{fig:dataset_generation_procedure}
\end{figure*}

\subsection{Geometry Generation}
\label{subsec:geometry_generation}
The geometry of the building structures forms the foundation of the dataset. Regular 
\revise{3D structures} resembling multi-story parking structures or shopping malls were generated, with parameters such as building floor dimensions and story heights selected randomly. Each building structure is composed of multiple rooms, which serve as the basic unit in this study. A room herein is a cuboid space defined by specific length, width, and height. Within a structure, rooms of the same dimensions are uniformly arranged along the length, width, and height, corresponding to the $x$-, $y$-, and $z$-axes, respectively. Structures vary in room size and number of rooms along each axis. Specifically, the room length, width, and height are independently sampled from a uniform distribution within the interval $[2, 5]$ meters along the three directions of the structure. Similarly, the room number along each axis is uniformly sampled independently as an integer within the interval $[2, 7]$, i.e., the maximum number of stories of the buildings simulated in this study is 7.

To introduce variability and simulate real-world scenarios, approximately $8\%$ of structural elements (beams or columns) are randomly removed after initial geometry creation. 
\revise{Such removal is not fire-induced damage, but reflects functional diversity often observed in real buildings, such as open spaces designed for activities in shopping malls, e.g., ice skating rinks. Examples of the generated geometries are illustrated in \figref{fig:example_generated_geometry}, showcasing the diversity and realism of the dataset. This element removal does not affect the definition of room's geometry in the structure and nor does it affect the number of considered fire scenarios.} 

\revise{A range of coefficient of variation values ($3.3\%$ to $17.5\%$) was derived from prior studies that investigated the statistics of geometrical and material properties of structural components of buildings (e.g., \cite{mirza1979variations, lee2004probabilistic}). These studies provide empirical data on the natural variability in parameters such as Young's modulus, yield strength, and dimensions of structural elements due to manufacturing tolerances and material inconsistencies. By selecting $8\%$ for the removal of structural elements in our database, we aimed to maintain a level of variability that is representative of real-world uncertainties while ensuring computational feasibility. This choice ensures that the database captures realistic deviations without introducing extreme cases that may not be commonly encountered in practice.}

\begin{figure*}[h!]
    \centering
    \includegraphics[width=\linewidth]{figures/example_generated_geometry.pdf}
    \caption{Examples of generated structural geometry of different sizes (all dimensions in meters).}
    \label{fig:example_generated_geometry} 
\end{figure*}

{\blockRevise

In this study, we opted for a deterministic square, dimension of $0.1$ m, solid cross-sectional steel elements due to their simplicity in modeling and analysis. Square sections exhibit uniform geometrical properties in all directions, simplifying the computation of structural responses and avoiding complications associated with more complex shapes, such as wide-flange sections, facilitating the computational efficiency and scalability to generate a large dataset. This choice also helps to mitigate issues related to stress concentrations and facilitates a more straightforward representation of structural behavior under thermal loads. 

\textit{Remark:} The selected cross-section provides a comparable flexural rigidity to a $W 130 \times 130 \times 28.1$ wide-flange section (metric units), albeit with significantly higher axial rigidity. This cross-section is acceptable for gravity-load-designed frames under service loading conditions where the models assume fully rigid, moment-resisting beam-column connections for the evaluation of the IDR under thermal loading. This assumption is reasonable in this computational study where the primary interest is to understand the global deformation response of frames under fire conditions. The selection of uniform square cross-sections for both beams and columns, rather than adherence to standard capacity design principles, was made here primarily for computational efficiency and to reduce design parameters in the database generation process. This choice allows for simplified and scalable approach to analyze the fire-induced response of generic steel frames without the need for large section variations, where this study mainly focuses on the fire vulnerability assessment using ML-based predictions. However, if additional loading conditions, e.g., seismic or wind loads, were to be considered, larger sections, strong-column/weak-beam principle, and ductile detailing would be required in the generated buildings for realistic structural behavior under combined loading conditions. Future studies may also consider investigating the influence of variable cross-sectional dimensions and semi-rigid connections on the structural performance under fire conditions. 
} % blockRevise

\subsection{Material Properties}
Steel is chosen as the material for the structures. To reflect real-world variations, we randomly assign one of five slightly different steel material types to each structural element. \revise{
The ranges of material properties are provided in \tabref{tab:material_property_ranges} and the properties are sampled from uniform distributions of the corresponding ranges. These variations simulate differences arising from manufacturing batches or regional material properties. That these properties are at ambient temperature and change when the temperature rises due to a fire. The selection of materials with varying properties is aimed at increasing the diversity of the data. Our goal is to represent as wide a range of data as possible with a limited amount of building structure data, thereby enhancing the generalization ability of the GNN. Our assumed material property ranges are expected to be wider than the real-world conditions based on findings in \cite{mirza1979variations, lee2004probabilistic}. Therefore, we are essentially tackling a more challenging and general task. If we can solve this problem, we are confident that our method will perform equally well or even better in real-world scenarios.
}
\begin{table}[h!]
    \centering
    \caption{Material properties ranges for considered steel structures.}
    \begin{tabular}{lc}
        \toprule
        Property & Range \\
        \midrule
        Young's modulus & [168, 252] GPa \\
        Yield strength & [220, 330] MPa \\
        Strain-hardening ratio & [0.8, 1.2] \% \\
        \bottomrule
    \end{tabular}
    \label{tab:material_property_ranges}
\end{table}

\subsection{Gravity Loads}
Gravity loads are applied to columns and beams based on their \revise{influence (tributary) areas as typically conducted in structural analysis. The considered ``service'' load conditions include the column self-weight and the additional loads directly supported on the beams from their self-weight and weights of the reinforced concrete slabs, people as live load, and building content. An edge beam typically carries approximately half the gravity load supported by a parallel interior beam}. The ranges of gravity loads are listed in \tabref{tab:gravity_load_ranges}. \revise{The loads are sampled from uniform distributions of the corresponding ranges.} Structures that failed to meet an MIDR threshold of $1\%$ under gravity loads were deemed unacceptable designs and filtered out, as such configurations of randomly chosen geometry, material, and gravity load combinations were considered unrealistic from a regulatory and practicality points of view.
\begin{table}[h!]
    \centering
    \caption{Gravity load ranges for considered beams and columns.}
    \begin{tabular}{lc}
        \toprule
        Element & Range (kN/m)  \\
        \midrule
        Column & [0.5, 1.0]  \\
        Edge beam & [1.5, 4.5]  \\
        Interior beam & [3.0, 7.5]  \\
        \bottomrule
    \end{tabular}
    \label{tab:gravity_load_ranges}
\end{table} 

\subsection{Rule-based Thermal Load Generation}
\label{subsec:thermal_load_generation}
To evaluate a building's structural response during a fire event, we employed a simplified rule-based approach for thermal load generation. 
% Previous studies \cite{nan_structuralfire_2023} have demonstrated that steel structures rapidly equilibrate with surrounding gases temperatures due to efficient heat exchange. Consequently, gas temperatures can be directly used as inputs for FEA tools, e.g., OpenSees, simplifying the process of modeling thermal loads. 
% Accurately simulating temperature fields in fire scenarios poses significant challenges. Advanced thermodynamic simulations, such as those performed using Fire Dynamics Simulator (FDS) \cite{mcgrattan_fire_2000}, provide precise temperature predictions. However, these methods are hindered by high computational costs, prolonging execution times, and limited scalability, making them impractical for generating large datasets. Additionally, real-world fire loads often display substantial spatial variability across different rooms \cite{dundar_fire_2023}, resulting in scenario-specific temperature fields with limited generalizability. For example, studies on bridge fires \cite{he_study_2024} have demonstrated that environmental factors, such as wind speeds, can significantly influence temperature distributions. Furthermore, even within identical scenarios, variations in fire modeling methodologies can produce distinctly different temperature fields \cite{zhang_temperature_2020, du_new_2012}. These challenges emphasize the need for efficient and adaptable methods to generate fire temperature data.
% To address these issues, we adopted a rule-based approach to model temperature variations. 
According to \cite{spearpoint_fire_2008}, a typical fire development follows a predictable pattern. During the {\em{growth stage}}, the temperature rises slowly and approximately linearly after ignition. This is followed by the {\em{flashover stage}}, where temperatures increase rapidly to peak values. After reaching the peak, the temperature either stabilizes or continues to rise slowly until the {\em{decay stage}} begins. Inspired by this fire development pattern, we describe the temperature evolution in time, $t$, prior to the decay stage in two distinct stages:
\begin{enumerate}
    \item {\bf{Initial linear increase stage}}: For $t \in [0, t_1)$, temperature increases gradually and linearly as the fire spreads through the building. This stage represents the time before the fire directly affects a structural element.  
    \item {\bf{ISO 834 fire curve stage}}: For $t \in [t_1, t_{\thre}]$, temperature rises rapidly following the ISO 834 curve \cite{ISO834}, modeling the direct impact of the fire on the structural element. 
\end{enumerate}
The slope of the linear temperature increase, $c$, and the transition time, $t_1$, are influenced by the spatial relationship between the fire source and the structural element. For the second stage of temperature evolution, we utilize the ISO 834 curve, a widely accepted standard for fire resistance testing. This standardized fire curve describes the temperature rise over time, enabling rapid and consistent thermal fields across various scenarios. The duration of fire simulation in this study is set to $t_{\thre}=60$ minutes. This value represents the upper limit for the temperature evolution of each structural element, providing a consistent basis for analyzing the structural response to fire.

Let $(x, y, z)$ represents the midpoint of a structural element and $(x_{\subfire}, y_{\subfire}, z_{\subfire})$ the fire source point. \revise{Integer parameters $h$ and $h_{\subfire}$ correspond to the respective floor levels of the element and the fire source}. The temperature evolution for each element is expressed as follows:
\begin{enumerate}
    \item Linear increase stage ($0 < t < t_1$):
    \begin{equation}
    T(t) = c \cdot t,
    \end{equation}
    where $c$, the rate of temperature increase ($^\circ\mathrm{C}/\mathrm{min}$), depends on the height difference between the element, $h$, and the fire source, $h_{\subfire}$:
    \begin{equation}
        c = 
        \begin{cases} 
        5\left/\left(h - h_{\subfire} + 1\right)\right., & h \geq h_{\subfire}, \\
        2\left/\left(h_{\subfire} - h\right)\right., & h < h_{\subfire}.
        \end{cases}
    \end{equation}
     \item ISO 834 stage ($t \geq t_1$):
\begin{equation}
    T(t) = c \cdot t_1 + 345 \log_{10} \left(8 \left(t - t_1\right) + 1\right).
\end{equation}
\end{enumerate}

The transition (arrival) time $t_1$, marking the end of the linear stage, depends on the spatial distance between the fire source and the element. We define the following two Euclidean distances $L_p$ in the $xy$ plane and $L_s$ in the $xyz$ space:
\begin{eqnarray}
L_p & \triangleq & \sqrt{(x - x_{\subfire})^2 + (y - y_{\subfire})^2}, \\
\label{eq:Lp}
L_s & \triangleq & \sqrt{(x - x_{\subfire})^2 + (y - y_{\subfire})^2 + (z - z_{\subfire})^2}.
\label{eq:Ls}
\end{eqnarray}
Accordingly, the transition time, $t_1$, is expressed as follows:
\begin{equation}
    t_1 = 
    \begin{cases}
    \beta_{1} \cdot \left(1 - \exp\left\{- L_s\left/\alpha_{1}\right.\right\}\right), & h > h_{\subfire}, \\
    \beta_{2} \cdot \left(1 - \exp\left\{- L_p\left/\alpha_{2}\right.\right\}\right), & h = h_{\subfire}, \\
    \beta_{3} \cdot \left(1 - \exp\left\{- L_s\left/\alpha_{3}\right.\right\}\right), & h < h_{\subfire} .
    \end{cases}
    \label{eq:t1}
\end{equation}
The parameters $\beta_i$ and $\alpha_i$ for determining $t_1$ are summarized in Table~\ref{tab:fire_spread_parameters}. In this study, we take $r_{\mathrm{up}}=0.95$ and $r_{\mathrm{down}}=0.97$.
\begin{table}[ht]
    \centering
    \caption{Fire spread parameters for $t_1$ calculations.}
    \begin{tabular}{lcc}
        \toprule
        Case  & $\beta_i$ & $\alpha_i$  \\
        \midrule
        $i=1$, Upward spread & $16 \left.\left(1-r_{\mathrm{up}}^{\left|h-h_{\subfire}\right|}\right)\right/\left(1-r_{\mathrm{up}}\right)$ & $10$  \\
        $i=2$, Horizontal spread & $18$ & $18$  \\
        $i=3$, Downward spread & $30 \left.\left(1-r_{\mathrm{down}}^{\left|h-h_{\subfire}\right|}\right)\right/\left(1-r_{\mathrm{down}}\right)$ & $5$  \\
        \bottomrule
    \end{tabular}
    \label{tab:fire_spread_parameters}
\end{table}

\figref{fig:t1_curve} illustrates the $t_1$ curves for various fire scenarios: (1) fire originating on the lower floor, $h-h_{\subfire}=1$ with rapid upward spread, (2) fire on the same floor, $h=h_{\subfire}$ with the fastest spread, and (3) fire on the upper floor, $h_{\subfire}-h=1$ with slow downward spread. The exponential decay in $t_1$ reflects the accelerating fire propagation speed as the distance increases. \figref{fig:t1_curve} also indicates that the employed simplified model is consistent with the Markov chain-based dynamic model given by \cite{cheng_dynamic_2011}, where the rooms at the same floor of the fire point start flashover slightly before the corresponding upper floors. Additionally, $\beta_{1}$ and $\beta_{3}$ are the summation of a geometric sequence, where story level $h$ is the index. The common ratios $r_{\mathrm{up}}<1$ in $\beta_{1}$ and $r_{\mathrm{down}}<1$ in $\beta_{3}$ indicate that the fire speeds up to spread through the next story, which is consistent with the real-world fire spread mechanism given in \cite{hokugo_mechanism_2000}. The temperature profile within the range $t \in [0, t_{\thre}]$ is subsequently used as the thermal load in OpenSeesRT simulations to compute displacements at each structural node at time $t_{\thre}$.
\begin{figure}[h!]
    \centering
    \includegraphics[width=0.8\linewidth]{figures/m204_t1_curve.pdf}
    \caption{Three examples for the $t_1$ curve.}
    \label{fig:t1_curve}
\end{figure}

\revise{
\textit{Remark:} The effects of structural elements, such as concrete floor slabs and partitions, are not explicitly modeled in our approach. Instead, their influence is implicitly captured through the careful selection of the parameters $ \alpha, \beta, r_\mathrm{up} $, and $ r_\mathrm{down} $. This parameterization provides a unified framework for generating temperature fields. Indeed, fire propagation is governed by a multitude of factors and remains an open research question. For instance, if the fire resistance of a floor slab is enhanced by fire protective coating, the corresponding model can account for this by decreasing $\alpha_1$ \& $\alpha_3$, increasing $\beta_1$ \& $\beta_3$, and adopting larger values for $r_\mathrm{up}$ \& $r_\mathrm{down}$, which collectively slow down the vertical spread of fire. Conversely, scenarios involving higher amounts of combustible materials would warrant the opposite adjustments. This flexible and integrated approach avoids the need to design separate models for different fire propagation scenarios while still capturing the essential effects.
}

\revise{
In conclusion, our rule-based approach is a computationally efficient method for approximating fire temperature fields, enabling large-scale dataset generation to train predictive models. By combining ISO 834 fire curves with spatial considerations and embedding structural effects through parameter calibration, the method achieves a balanced trade-off between accuracy and scalability, making it a practical solution for thermal load modeling in fire scenarios. After generating the temperature of each beam or column according to the middle point, the temperature is applied as uniform thermal load to the elements of the structure in question using OpenSeesRT. 
}

% In conclusion, this rule-based approach is a computationally efficient method to approximate fire temperature fields, enabling large-scale dataset generation to train predictive models. By combining ISO 834 fire curves with spatial considerations, the method balances accuracy and scalability, making it a practical solution for thermal load modeling in fire scenarios.

% \subsection{Interstory Drift Ratio}
\subsection{OpenSeesRT Simulation}
\label{subsec:opensees_simulation}

The thermal and mechanical responses of 3D frame structures under combined fire and gravity loads are simulated using OpenSeesRT \cite{perez2024openseesrt}. \revise{In the simulation, the IDR of each node at $t_{\thre}$ is computed using the computed nodal displacements. Each structural model features six degrees of freedom per node (3 translational  and 3 rotational), with linear geometrical transformations (\texttt{geomTransf: Linear}) defining how the element local coordinate systems are mapped to the global coordinate system and assuming small displacements and rotations. Although OpenSeesRT allows a variety of options for modeling finite deformations, in the present simulations and mainly for simplicity, we did not consider large deformations. All bottom nodes (nodes on the ground) are fully constrained in all six degrees of freedom, while degrees of freedom os all other nodes are free.} Material behavior is temperature-dependent and modeled with \texttt{Steel01Thermal}, while fiber-based sections (\texttt{FiberThermal}) capture nonlinear interactions between thermal and mechanical responses at the cross-section level. \revise{Structural elements are represented as displacement-based Euler-Bernoulli beam-columns (\texttt{dispBeamColumnThermal}). This element  formulation accounts for thermal strains (temperature gradients) in the section, which is discretized into fibers. Numerical integration is used along the length of each element using three integration (Gauss) points, one at each end and the third in the middle of the element.}

{\revise{Thermal expansion of steel members plays a crucial role in IDR development. In reality, reinforced concrete floor slabs heat at a different rate than steel members due to their higher thermal mass and lower thermal conductivity. This differential heating can lead to restrained thermal expansion, introducing axial compression in beams and affecting the overall structural response. In this study, explicit {\em{composite action}} between steel members and concrete slabs is not modeled. Instead, our approach focuses on isolating the response of the steel structural frame, which is often the critical load-bearing component in fire scenarios. This assumption aligns with prior studies \cite{Possidente_2024} demonstrating that steel structures reach thermal equilibrium with surrounding gases quickly, allowing the use of uniform thermal loading in fire analysis. Future work could enhance this framework by incorporating slab-beam interaction effects, through a refined FEA for an extended dataset where constraints imposed by floor slabs are explicitly considered.}

The analysis begins with the application of gravity loads, followed by incremental thermal loads simulating the fire exposure. A static nonlinear solver using  \texttt{ExpressNewton} algorithm ensures convergence, while the \texttt{NormDispIncr} test maintains accuracy. An incremental \texttt{LoadControl} scheme with small step sizes is employed to guarantee numerical stability, using 10\% for gravity loads and 1\% for thermal loads. 

\revise{
In the thermal load analysis, uniform thermal load is applied to each beam or column, i.e., the temperature of each element is set to be that at the middle point, according to \secref{subsec:thermal_load_generation}. The \texttt{Steel01Thermal} material allows the properties (e.g., Young's modulus and yield strength) to be adjusted at increasing temperatures according to \cite{EN1993} using its Table 3.1: Reduction factors for the stress-strain relationship of carbon steel at elevated temperatures. For example, if the Young’s modulus at ambient temperature is $E_0$, then as the temperature ($T$) increases, the modulus changes as $E(T) = \eta (T) \times E_0$. \cite{EN1993} directly provides the values of $\eta(T) \in \left[0,1\right] $ at every $100 ^\circ\mathrm{C}$ interval and recommends using linear interpolation to obtain $\eta(T)$ for intermediate values of $T$.
} OpenSeesRT documentation \cite{OpenSeesThermalExamples} provides several examples of thermal analyses.

This modeling framework accommodates variations in material properties, cross-sectional geometries, and temperature profiles, providing robust simulations of structural behavior under fire conditions. The primary settings and configurations for the OpenSeesRT simulations are summarized in \tabref{tab:ops_detail}.
\begin{table}[h!]
    \centering
        \caption{Key settings of OpenSeesRT simulations.}
    \begin{tabular}{l|>{\raggedright\arraybackslash}p{0.6\linewidth}} %
    \toprule
    Modeling Aspect     & Details \\
    \midrule
    Geometry            & 3D models; 6 degrees of freedom per node \\
    Transformation      & geomTransf: Linear \\ 
    Material            & Steel01Thermal \\
    Section             & FiberThermal; Cross-section: $0.1$ m $\times$ $0.1$ m \\ 
    Element type        & {dispBeamColumnThermal} \\ 
    Loading             & Gravity loads: {beamUniform}; Thermal loads: {beamThermal} \\
    Integration scheme  & Incremental {LoadControl}; Step size: $10\%$ (gravity analysis), $1\%$ (thermal analysis) \\
    Nonlinear solver    & {ExpressNewton} algorithm; {UmfPack} solver; Convergence test: {NormDispIncr} tolerance: $10^{-8}$; Maximum \# iterations per step: $1000$. \\ 
    \bottomrule
    \end{tabular}
    \label{tab:ops_detail}
\end{table}

For each structure in the labeled dataset, 30 fire points are selected using a dual-granularity approach, \revise{i.e., two-stage sampling strategy,} to ensure they are well-distributed. Specifically, rooms are sequentially selected, with one fire point randomly chosen within each selected room. If a building is large and contains more than 30 rooms, we randomly select 30 rooms without replacement, i.e., ensuring that no more than one fire point is located in the same room. Conversely, if the building is small and has fewer than 30 rooms, all rooms are initially selected, with one fire point randomly assigned to each room. Additionally, rooms are then selected with replacement until a total of 30 fire points are assigned. \revise{The room-level sampling prioritizes selecting distinct rooms to avoid spatial clustering of fire points, while the point-level sampling ensures intra-room variability. This approach aligns with stratified sampling principles commonly used for efficient spatial representation, where multi-stage sampling strategies optimize coverage and variability, e.g., \cite{arunachalam_generalized_2023}, and enables a more comprehensive characterizing of how the structures respond under fire conditions.}
% This selection method prevents fire points from clustering too closely while maintaining an element of randomness. By distributing fire points in this manner, the 30 fire scenarios are effectively utilized, enabling a more comprehensive characterizing of how the structures respond under fire conditions.

\subsection{Summary of the Dataset Generation}
As discussed in this section and related to  \figref{fig:dataset_generation_procedure}, three key steps were considered in the development of the dataset: 
\begin{enumerate}
    \item {\bf{Filtering process}}: Structures with MIDR exceeding $1\%$ under gravity loads were excluded,  resulting in $1,573$ labeled structures retained for fire simulation and $16,050$ unlabeled structures for training the MFSP predictor.
    \item {\bf{Fire simulations}}: For each retained labeled structure, 30 fire scenarios were simulated using OpenSeesRT, yielding $47,190$ fire cases.
    \item {\bf{Data distribution check}}: MIDR distributions for labeled and unlabeled data under gravity loads were highly similar, because both datasets were generated using the same method. Under fire conditions, the MIDR distribution shifted, reflecting significant structural deformation with values reaching a maximum of about 6\%, an average of 1.70\%, and a standard deviation of 1.12\%. This step ensured a diverse and comprehensive dataset for the proposed predictive framework.
\end{enumerate}
The statistical distribution histograms for MIDR (after applying the $1\%$ filtering threshold \revise{for gravity load responses}) under different loading conditions are plotted in \figref{fig:histogram_mdr}. Figures \ref{fig:histogram_mdr}(a) and \ref{fig:histogram_mdr}(b) show the MIDR distributions of the labeled and unlabeled data, respectively, under gravity loads only. \figref{fig:histogram_mdr}(c) shows the MIDR distribution of the labeled data under the combined effects of gravity and fire loads. Fire load causes the structures to significantly deform, leading to a noticeably \revise{right-skewed} MIDR distribution.

\begin{figure*}[h!]
    \centering
    \includegraphics[width=\linewidth]{figures/histogram_mdr.pdf}
    \caption{Histograms of MIDR for labeled and unlabeled structures with gravity loads and fire cases.}
    \label{fig:histogram_mdr}
\end{figure*}

\revise{
This dataset provides the basis for training and testing the performance of the GNN-based framework. Although we employed a simplified rule-based thermal load generation method compared with conventional CFD-based simulations, the temperature field, the changes of the material properties, and the response of the structures, are all still highly nonlinear and complex. Therefore, it is still a challenging task for the NN to predict the MIDRs based on this dataset.
}

\section{Data Availability}
The dataset presented in this study has been made publicly available and can be accessed via DOI: 10.5281/zenodo.14658505\footnote{https://zenodo.org/records/14658505}. The data is provided in a compressed format, which can be decompressed for analysis. Detailed instructions for accessing and utilizing the dataset are provided in this article and the platform.


\section{Dataset Characterization}

\begin{figure*} [t]
    \centering
    \includegraphics[width=1\linewidth]{mensagens_por_dia_comparativo.pdf}
    \caption{Evolution of the daily number of messages sent over time. The top panel presents the complete time series, distinguishing between messages sent by regular users (in blue) and bots (in red). The bottom-left panel focuses on the initial period, covering up to October 2022, while the bottom-right panel highlights the most recent period, from June 2022 to January 2024. Notice the graph scales are different.}
    \label{fig:messages_per_time}
\end{figure*}

% \gi{sugestao: fazer dos proximos paragrafos o inicio da secao de caracterizacao.}

The \textbf{3,167 collected servers} yielded a total of \textbf{2,052,206,308 unique messages} sent by \textbf{4,735,057 distinct users}. From the total number of messages,   \textbf{364,447,569 (17\%) originated from bots}. The data spans from Discord's launch, on \textbf{May 13, 2015}, to \textbf{December 17, 2024}, when the data collection process began.

Figure \ref{fig:messages_per_time} shows the number of messages over time. 
Note that Discord's retroactive approach to chat allows the dataset to encompass a timeframe broader than the actual data collection period.
Upon joining a server, users gain access to all non-deleted historical content within public channels, and the same is valid for data retrieval using their API.
Notably, 2024 stands out as the most active year across all servers, reflecting a growing network that offers ample opportunities for further exploration.

Although it is interesting to be able to access older data, it is important to note that the servers listed in the \textit{Discovery} tab most likely represent those active at the time of data collection. Servers featured in this tab typically emphasize ongoing engagement and relevance. Additionally, while Discord provides access to historical content, some servers may periodically delete older messages, which could introduce variability in the completeness of the temporal data available. These factors may influence the representativeness of older content and should be considered when analysing the dataset.

Next, we present simple but relevant analyses of the dataset. The first aspect we examined is \textbf{Bots.} As previously mentioned, Discord bots play a crucial role in enhancing user experience by automating tasks, facilitating engagement, and offering specialized functionalities. They serve as virtual assistants that can perform various tasks, such as moderation, entertainment, and utility services. Table \ref{tab:top10bots} illustrates the diverse applications of bots, highlighting their role in generating messages, being mentioned by users, and eliciting reactions, which are key indicators of their impact and utility in Discord.

Bots like \textit{MEE6} and \textit{Dyno} are widely recognized as powerful moderators, enabling server administrators to enforce rules, assign roles, and monitor activity effectively. These bots are essential for maintaining order in larger communities where manual moderation would be impractical. Similarly, \textit{Arcane Premium} and \textit{Loritta} excel in general server management, providing features such as levelling systems, customizable commands, and automated event handling to enhance user engagement and server functionality. On the other hand, entertainment-focused bots like \textit{Pokétwo}, \textit{Mudae}, \textit{Karuta} and \textit{OwO} captivate users with gaming and collectable experiences, encouraging interaction through game mechanics such as Pokémon battles, waifu collections, trading cards and interacting with fictional animals. These bots create an environment where users actively participate, forming micro-communities within the larger server ecosystem. Finally, \textit{Lucky VR}, despite being the bot with the highest volume of messages, lacks extensive documentation online. Its primary focus is on gambling in virtual reality, particularly in facilitating poker games.

\textbf{Languages.} To identify the most frequent languages in our dataset, we analysed the values present in the \textit{preferred locale} field within the server metadata. Figure \ref{fig:hist-linguas} shows English (US) as the primary language for most servers, with 1,705 servers. Disregarding the ``unknown" field, the second most frequent language is Spanish (Spain), with 144 servers, followed by French, with 136 servers. Also note there is a notable linguistic diversity, including Portuguese, Russian, and German.

\begin{figure}[t]
    \centering
    \includegraphics[width=1\linewidth]{language_histogram_avancado.pdf}
    \caption{Bar plot of the number of servers by language, with the majority of servers being in English.}
    \label{fig:hist-linguas}
\end{figure}


% For instance, bots like MEE6 and Dyno are widely used for community management, handling tasks like assigning roles, enforcing rules, and moderating content. On the other hand, entertainment bots such as Mudae, Karuta, and Pokétwo engage users through games, collectibles, and interactive features, exhibiting higher levels of participation and interaction. Additionally, niche bots like AniLibria.TV provide specialized services, such as streaming updates or content delivery for targeted audiences. The bots listed in Table \ref{tab:top10bots} illustrate the diverse applications of Discord bots, highlighting their role in generating messages, being mentioned by users, and eliciting reactions, which are key indicators of their impact and utility in various online communities.

\begin{table*}[t]
\caption{Top 10 Bots by Messages, Mentions, and Reactions.}
\centering
\label{tab:top10bots}
\begin{tabular}{ccc}
\toprule
\textbf{Messages} & \textbf{Mentions} & \textbf{Reactions} \\
\midrule
\hline
\begin{tabular}{lr}
Lucky VR & 3,684,600 \\
Mudae & 2,955,892 \\
AniLibria.TV & 2,844,343 \\
Pok\'etwo & 2,005,250 \\
RoM & 1,716,654 \\
OwO & 1,523,404 \\
Mimu & 1,430,137 \\
Wan Shi Tong & 1,297,316 \\
BTE France Minecraft & 1,077,553 \\
MEE6 & 904,432 \\
\end{tabular} &
\begin{tabular}{lr}
Pok\'etwo & 304,657 \\
Dyno & 290,387 \\
AniLibria.TV & 290,286 \\
ProBot  & 278,653 \\
Loritta & 251,208 \\
Arcane Premium & 224,681 \\
Mimu & 206,157 \\
MEE6 & 198,534 \\
Dank Memer & 178,813 \\
Liquid Esports & 134,816 \\
\end{tabular} &
\begin{tabular}{lr}
Karuta & 984,693 \\
Mudae & 281,929 \\
MEE6 & 237,370 \\
XenosPD.dk & 217,634 \\
Bongo & 216,779 \\
OwO & 193,493 \\
YAGPDB.xyz & 137,052 \\
Dyno & 131,881 \\
Suggester & 88,981 \\
Emps-World & 85,889 \\
\end{tabular} \\
\bottomrule
\hline
\end{tabular}
\end{table*}

\textbf{Servers.} Discord, as a platform for community interactions, supports a wide range of themes that reflect different user interests. To analyze the thematic distribution of servers, we examined server metadata, focusing on keywords that describe their content. These keywords also improve the platform's search functionality through the Discovery feature.

Table \ref{tab:top_40_keywords} presents the 40 most frequent keywords appearing in the description field of the servers collected. Despite Discord's evolution into a versatile platform, gaming-related content remains significant, with \textit{gaming} appearing in over 15\% of descriptions. Related keywords like \textit{minecraft}, \textit{roblox}, and \textit{twitch} reinforce its gaming's central role. Other prominent themes include \textit{anime} (5.68\%), \textit{roleplay} (6.44\%), and \textit{fivem} (4.39\%), reflecting niche communities and interactive multiplayer experiences. 

Emerging interests like \textit{esports} (4.04\%) and \textit{social} (4.14\%) highlight the importance of competitive gaming and socialization. Creative and recreational themes, such as \textit{art} and \textit{music}, and educational discussions, such as \textit{programming}, demonstrate the platform's versatility beyond gaming.

\begin{table}[t]
\caption{Top 40 most frequent keywords in Discord servers description and their coverage.}
\centering
\begin{tabular}{lr|lr}
%\hline
%\multicolumn{4}{c}{\textbf{Keywords}} \\
\hline
\textbf{Keyword} & \textbf{\%} & \textbf{Keyword} & \textbf{\%} \\
\hline
gaming & 15.28 & giveaways & 2.31 \\
youtube & 15.00 & art & 2.02 \\
minecraft & 12.28 & rp & 2.02 \\
roblox & 10.83 & tiktok & 1.89 \\
twitch & 8.68 & manga & 1.83 \\
community & 8.65 & chill & 1.74 \\
roleplay & 6.44 & leagueoflegends & 1.71 \\
anime & 5.68 & rpg & 1.71 \\
fivem & 4.39 & pvp & 1.58 \\
social & 4.14 & music & 1.48 \\
esports & 4.04 & server & 1.39 \\
valorant & 3.79 & chatting & 1.36 \\
memes & 2.97 & survival & 1.36 \\
fortnite & 2.78 & gta5 & 1.33 \\
streamer & 2.59 & modding & 1.20 \\
events & 2.53 & mobile & 1.20 \\
game & 2.49 & gtav & 1.20 \\
fun & 2.46 & programming & 1.17 \\
gta & 2.43 & pc & 1.17 \\
games & 2.43 & csgo & 1.14 \\
\hline
\label{tab:top_40_keywords}
\end{tabular}
\end{table}

However, note that these keywords are only related to the server's descriptions. The context of messages goes beyond these subjects, with messages concerning topics that include mental health, political debates and web dating, for example. \looseness=-1


\section{Potential Applications}

Our dataset represents the largest publicly available collection of textual data from Discord servers, offering an unprecedented resource for studying online interactions. The dataset spans a wide range of languages, cultures, and topics, making it uniquely suited to support a diverse cross-section of research efforts and laying a robust foundation for examining the complexities of online communities. Its breadth and depth enable researchers to explore critical areas such as discourse analysis, community governance, and political debate, making it an invaluable asset for interdisciplinary studies.\looseness=-1

\textbf{Online Community Governance.} Discord's user-driven moderation model stands apart from those used in platforms like Facebook and YouTube, which rely heavily on centralized moderation enforced by platform administrators or automated systems \cite{doi:10.1177/1461444818821316}. This decentralized approach enables server owners and moderators to define and enforce community-specific rules, offering an opportunity to explore how self-governance influences online interactions, social dynamics, and conflict resolution \cite{moderationbook}. Researchers can examine the efficacy of decentralized moderation in fostering inclusive and safe environments, the strategies employed by communities to address toxic behavior, and the broader impact of granting users autonomy in governing digital spaces \cite{succesfulcommunitybook}.

\textbf{Discourse analysis.} The dataset also provides a valuable resource for advancing research across multiple scientific domains, particularly in Natural Language Processing (NLP) and Machine Learning (ML). It enables the development and evaluation of models for tasks such as sentiment analysis, intent recognition, topic modeling, and toxic or abusive language detection. Moreover, the temporal structure of the data supports studies on conversational dynamics, social network analysis, and community behavior in digital environments. This dataset can also facilitate the creation of domain-specific chatbots, recommendation systems, and tools for automated moderation, fostering innovations that bridge computational techniques with the study of human communication and online interaction.

\textbf{Political Debate.} The political debate research community faces substantial challenges in understanding how both mainstream and decentralized platforms influence political opinions and behaviors. Social networks have become increasingly critical in shaping political landscapes, not only by amplifying voices and ideologies but also by contributing to the rapid dissemination of fake news and misinformation \cite{Aimeur2023}. While much of the existing research focuses on platforms like Facebook and Twitter \cite{10.1145/3578503.3583597, doi:10.1126/science.adk3451}, Discord’s semi-private and community-driven architecture offers a distinct and underexplored environment for political conversations. 
%This dynamic has raised significant concerns about the potential for unregulated spaces to facilitate polarization, radicalization, and the erosion of trust in democratic processes \cite{moderation-challenges}. At the same time, Discord's user-moderated model provides a valuable opportunity to study these phenomena in a decentralized context, contrasting with the centralized moderation models of mainstream platforms. -> Achei que estava redundante com outra parte do texto
Our dataset enables researchers to explore the impact of digital platforms on political discourse, the propagation of misinformation, and the development of effective moderation and regulation strategies tailored to such environments.

\textbf{Mental health.} The relationship between social media usage and the prevalence of mental health issues, including self-harm and suicidal behavior, has been extensively documented, particularly among younger demographics \cite{elia2020, pater2016characterizations}. Discord, as a platform with diverse communities and user-generated content, represents a critical environment for understanding these phenomena. Recent studies in Brazil have already looked at mental health in the context of Discord, underscoring the importance of examining how the platform’s semi-private and community-oriented spaces may influence mental health outcomes \cite{webmedia}. Our multilingual dataset significantly expands the scope for such research by enabling cross-cultural and cross-linguistic analyses of mental health trends and discourse on Discord, providing a valuable foundation for identifying patterns of at-risk behavior and explore critical questions such as the prevalence of harmful behaviors or supportive interactions.

% \subsection{Social Network Analysis}
% This dataset provides a unique opportunity to study user interactions, group dynamics, and information dissemination patterns within a controlled environment. Researchers can use it to examine the spread of trends, topics, and opinions over time, as well as analyze patterns of collaboration and conflict resolution in online communities. Such studies can reveal the underlying mechanisms of information flow and social influence within digital ecosystems.

% \subsection{Behavioral and Sentiment Analysis}
% With anonymized message content and metadata, the dataset enables researchers to detect behavioral trends and shifts in online activity. Sentiment analysis can be performed to measure emotional responses to events or announcements, offering insights into public sentiment on various issues. Additionally, the dataset can facilitate the study of how moderation policies affect user behavior, providing data-driven insights into the effectiveness of such interventions.

% \subsection{Machine Learning and Natural Language Processing (NLP)}
% This dataset offers a robust foundation for developing and testing algorithms in machine learning and NLP. It can be used to train chatbots and virtual assistants for applications in customer service or education. Algorithms designed to detect toxic behavior or policy violations on online platforms can be refined using this dataset. Furthermore, models for tasks such as summarization, translation, or sentiment classification can be tested on real-world data, bridging the gap between theoretical development and practical application.

% \subsection{Platform Policy Evaluation and Development}
% The data can guide platform administrators and policymakers in evaluating the effectiveness of moderation techniques in reducing harmful behavior. It can inform the design of inclusive and user-friendly community guidelines that foster positive interactions. Additionally, the data can be used to assess the long-term impact of platform updates and feature rollouts, enabling evidence-based decision-making for better user experiences.

% \subsection{Educational and Training Purposes}
% The dataset is an invaluable resource for academic settings. It can be used to train students in data science, machine learning, or social science methods, providing real-world scenarios for hands-on learning. It also serves as a practical tool to demonstrate ethical considerations in data collection and processing, offering a controlled environment for testing hypotheses in online communication studies.

% \subsection{Ethics and Privacy Research}
% Finally, this dataset itself can serve as a case study in ethical anonymization practices. Researchers can explore novel anonymization techniques and evaluate their effectiveness, contributing to the ongoing dialogue on privacy preservation. It also provides an opportunity to study the trade-offs between data utility and user privacy in real-world scenarios, advancing the field of ethical data management.








%%%%%%%%%%%%%%%%%%%%%%%%%%%%%%%%%%%%%%%%%%%%%%%%%%%%%%%%%%%%%%%%%%%%%%%%%%%%%%%%%%%%%%%%%%%%%%%%%%%%%%%%%
\iffalse

\section{Potential Applications}

The anonymized dataset created through the process described in this work has the potential to support a variety of research areas and applications while adhering to ethical and legal standards for data protection. Below, we outline key domains and examples of potential use cases:

\subsection{Social Network Analysis}
This dataset provides a unique opportunity to study user interactions, group dynamics, and information dissemination patterns within a controlled environment. Applications include:
\begin{itemize}
    \item Examining the spread of trends, topics, and opinions over time.
    \item Analyzing patterns of collaboration and conflict resolution in online communities.
\end{itemize}

\subsection{Behavioral and Sentiment Analysis}
With anonymized message content and metadata, the dataset enables researchers to:
\begin{itemize}
    \item Detect behavioral trends and shifts in online activity.
    \item Perform sentiment analysis to measure emotional responses to events or announcements.
    \item Study the effects of moderation policies on user behavior.
\end{itemize}

\subsection{Machine Learning and Natural Language Processing (NLP)}
This dataset offers a foundation for developing and testing algorithms in machine learning and NLP:
\begin{itemize}
    \item Training chatbots and virtual assistants for customer service or educational purposes.
    \item Developing algorithms to detect toxic behavior or policy violations in online platforms.
    \item Testing models for summarization, translation, or sentiment classification on real-world data.
\end{itemize}

\subsection{Platform Policy Evaluation and Development}
The anonymized data can guide platform administrators and policymakers by:
\begin{itemize}
    \item Evaluating the effectiveness of moderation techniques in reducing harmful behavior.
    \item Informing the design of inclusive and user-friendly community guidelines.
    \item Assessing the long-term impact of platform updates and feature rollouts.
\end{itemize}

\subsection{Educational and Training Purposes}
The dataset, with sensitive information removed, can be used in academic settings to:
\begin{itemize}
    \item Train students in data science, machine learning, or social science methods.
    \item Demonstrate ethical considerations in data collection and processing.
    \item Offer a controlled environment for testing hypotheses in online communication studies.
\end{itemize}

\subsection{Ethics and Privacy Research}
Finally, this dataset itself can serve as a case study in ethical anonymization practices:
\begin{itemize}
    \item Exploring novel anonymization techniques and evaluating their effectiveness.
    \item Studying the trade-off between data utility and user privacy in real-world scenarios.
\end{itemize}

\fi







\section*{Conclusion}
This paper aims to enhance our understanding of the computational complexity of computing various Shapley value variants. We found that for various ML models --- including decision trees, regression tree ensembles, weighted automata, and linear regression --- both local and global interventional and baseline SHAP can be computed in polynomial time under HMM modeled distributions. This extends popular algorithms, such as TreeSHAP, beyond their empirical distributional scope. We also establish strict complexity gaps between the various SHAP variants (baseline, interventional, and conditional) and prove the intractability of computing SHAP for tree ensembles and neural networks in simplified scenarios. Overall, we present SHAP as a versatile framework whose complexity depends on four key factors: \begin{inparaenum}[(i)] \item model type, \item SHAP variant, \item distribution modeling approach, \item and local vs. global explanations\end{inparaenum}. We believe this perspective provides deeper insight into the computational complexity of SHAP, paving the way for future work.




%We believe that our framework provides a more intricate understanding of SHAP computation complexity across different models, distributions, and variants, paving the way for further research.

Our work opens promising directions for future research. First, expanding our computational analysis to other SHAP-related metrics, such as asymmetric SHAP~\citep{frye20} and SAGE~\citep{covert2020understanding}, would be valuable. Additionally, we aim to explore more expressive distribution classes and relaxed assumptions beyond those in Section \ref{sec:tractable} while maintaining tractable SHAP computation. Finally, when exact computation is intractable (Section \ref{sec:intractable}), investigating the approximability of SHAP metrics through approximation and parameterized complexity theory~\citep{downey2012parameterized} is an important direction.

%Our work opens several promising avenues for future research on the computational properties of explainable AI methods, with a particular focus on SHAP. First, it would be interesting to broaden the computational analysis conducted in this work to include other popular SHAP-related metrics in the literature, such as asymmetric SHAP \cite{frye20} and SAGE \cite{covert2020understanding}. Also, in the future, we aim to explore more expressive distribution classes and relaxed distributional assumptions—extending beyond those examined in Section \ref{sec:tractable} —that still yield tractable SHAP computation. Finally, when exact computation proves intractable (Section \ref{sec:intractable}), it is worthwhile to theoretically investigate the question of the approximability of computing the SHAP metrics across various configurations, through the lens of approximation and parametrized complexity theory \cite{arora2009computational}.

%This paper aims to deepen our understanding of the computational complexity involved in obtaining different Shapley value variants. We found that for a variety of ML models, including decision trees, tree ensembles for regression, weighted automata, and linear regression models — computing both local and global interventional and baseline SHAP can be done in polynomial time when distributions are modeled by HMMs. This extends the distributional scope of popular algorithms like TreeSHAP, which is limited to empirical distributions. Additionally, we demonstrate a strict complexity gap between SHAP variants, showing that interventional and baseline SHAP can be strictly easier to compute than conditional SHAP. Despite these positive results, we uncovered intractability for various SHAP variants in neural networks and tree ensembles. Finally, we provided generalized complexity relations across SHAP variants. We believe that our framework offers a deeper understanding of the complexity involved in computing SHAP across various variants, models, distributions, as well as in both local and global computations, laying the groundwork for future research.
\section{Acknowledgments}

This work was partially funded by CNPq, CAPES, FAPEMIG, and IAIA - INCT on AI.

\bibliography{main}


\subsection{Paper Checklist}

\begin{enumerate}

\item For most authors...
\begin{enumerate}
    \item  Would answering this research question advance science without violating social contracts, such as violating privacy norms, perpetuating unfair profiling, exacerbating the socio-economic divide, or implying disrespect to societies or cultures?
    \answerYes{Yes}
  \item Do your main claims in the abstract and introduction accurately reflect the paper's contributions and scope?
    \answerYes{Yes}
   \item Do you clarify how the proposed methodological approach is appropriate for the claims made? 
    \answerYes{Yes}
   \item Do you clarify what are possible artifacts in the data used, given population-specific distributions?
    \answerYes{Yes}
  \item Did you describe the limitations of your work?
    \answerYes{Yes}
  \item Did you discuss any potential negative societal impacts of your work?
    \answerYes{Yes}
      \item Did you discuss any potential misuse of your work?
    \answerYes{Yes}
    \item Did you describe steps taken to prevent or mitigate potential negative outcomes of the research, such as data and model documentation, data anonymization, responsible release, access control, and the reproducibility of findings?
    \answerYes{Yes}
  \item Have you read the ethics review guidelines and ensured that your paper conforms to them?
    \answerYes{Yes}
\end{enumerate}

\item Additionally, if your study involves hypotheses testing...
\begin{enumerate}
  \item Did you clearly state the assumptions underlying all theoretical results?
    \answerNA{NA}
  \item Have you provided justifications for all theoretical results?
    \answerNA{NA}
  \item Did you discuss competing hypotheses or theories that might challenge or complement your theoretical results?
    \answerNA{NA}
  \item Have you considered alternative mechanisms or explanations that might account for the same outcomes observed in your study?
    \answerNA{NA}
  \item Did you address potential biases or limitations in your theoretical framework?
    \answerNA{NA}
  \item Have you related your theoretical results to the existing literature in social science?
    \answerNA{NA}
  \item Did you discuss the implications of your theoretical results for policy, practice, or further research in the social science domain?
    \answerNA{NA}
\end{enumerate}

\item Additionally, if you are including theoretical proofs...
\begin{enumerate}
  \item Did you state the full set of assumptions of all theoretical results?
    \answerNA{NA}
	\item Did you include complete proofs of all theoretical results?
    \answerNA{NA}
\end{enumerate}

\item Additionally, if you ran machine learning experiments...
\begin{enumerate}
  \item Did you include the code, data, and instructions needed to reproduce the main experimental results (either in the supplemental material or as a URL)?
    \answerNA{NA}
  \item Did you specify all the training details (e.g., data splits, hyperparameters, how they were chosen)?
    \answerNA{NA}
     \item Did you report error bars (e.g., with respect to the random seed after running experiments multiple times)?
    \answerNA{NA}
	\item Did you include the total amount of compute and the type of resources used (e.g., type of GPUs, internal cluster, or cloud provider)?
    \answerNA{NA}
     \item Do you justify how the proposed evaluation is sufficient and appropriate to the claims made? 
    \answerNA{NA}
     \item Do you discuss what is ``the cost`` of misclassification and fault (in)tolerance?
    \answerNA{NA}
  
\end{enumerate}

\item Additionally, if you are using existing assets (e.g., code, data, models) or curating/releasing new assets, \textbf{without compromising anonymity}...
\begin{enumerate}
  \item If your work uses existing assets, did you cite the creators?
    \answerNA{NA}
  \item Did you mention the license of the assets?
    \answerNA{NA}
  \item Did you include any new assets in the supplemental material or as a URL?
    \answerNA{NA}
  \item Did you discuss whether and how consent was obtained from people whose data you're using/curating?
    \answerYes{Yes}
  \item Did you discuss whether the data you are using/curating contains personally identifiable information or offensive content?
    \answerYes{Yes}
\item If you are curating or releasing new datasets, did you discuss how you intend to make your datasets FAIR?
\answerYes{Yes}
\item If you are curating or releasing new datasets, did you create a Datasheet for the Dataset (see \citet{gebru2021datasheets})? 
\answerYes{Yes}
\end{enumerate}

\item Additionally, if you used crowdsourcing or conducted research with human subjects, \textbf{without compromising anonymity}...
\begin{enumerate}
  \item Did you include the full text of instructions given to participants and screenshots?
    \answerNA{NA}
  \item Did you describe any potential participant risks, with mentions of Institutional Review Board (IRB) approvals?
    \answerNA{NA}
  \item Did you include the estimated hourly wage paid to participants and the total amount spent on participant compensation?
    \answerNA{NA}
   \item Did you discuss how data is stored, shared, and deidentified?
   \answerYes{Yes} 
   
\end{enumerate}

\end{enumerate}

\onecolumn

\appendix

\section*{Appendix}

\begin{table*}[ht!]
    \caption{Fields in the Discord Server Object. These fields provide metadata for each server included in the dataset.}
    \centering
    \renewcommand{\arraystretch}{1.3} % Espaçamento vertical
    \setlength{\tabcolsep}{5pt} % Espaçamento horizontal
    \begin{tabular}{|p{3.5cm}|p{3.5cm}|p{8cm}|}
        \hline
        \textbf{Field} & \textbf{Type} & \textbf{Description} \\ 
        \hline
        slug & string & A unique identifier combining the server name and ID, used as a shorthand or URL-friendly reference. \\ 
        \hline
        id & string & A unique numerical identifier for the server. \\ 
        \hline
        name & string & The human-readable name of the server. \\ 
        \hline
        description & string & A brief description or identifier for the server. \\ 
        \hline
        icon & string & URL to the server's icon image. \\ 
        \hline
        splash & string & URL to the splash image, typically used for larger visual displays or promotional purposes. \\ 
        \hline
        banner & string & Identifier for the server's banner image, used for branding or visual purposes. \\ 
        \hline
        approximate\_presence\_ count & number & The estimated number of currently active users on the server. \\ 
        \hline
        approximate\_member\_ count & number & The estimated total number of members in the server. \\ 
        \hline
        premium\_subscription\_ count & number & The number of members with premium subscriptions (e.g., Discord Nitro) on the server. \\ 
        \hline
        preferred\_locale & string & The server's preferred language, represented by its locale code. \\ 
        \hline
        auto\_removed & boolean & Indicates whether the server was automatically removed from a directory or platform. \\ 
        \hline
        discovery\_splash & string & An identifier for the discovery splash image, used for promoting the server in public directories. \\ 
        \hline
        primary\_category\_id & number & ID representing the primary category or classification of the server. \\ 
        \hline
        vanity\_url\_code & string & Custom short URL code for the server. \\ 
        \hline
        is\_published & boolean & Indicates whether the server is published and discoverable in public directories. \\ 
        \hline
        keywords & array of strings & A list of keywords associated with the server for discoverability. \\ 
        \hline
        features & array of strings & A list of special features enabled for the server, providing enhanced functionality or customization. \\ 
        \hline
        created\_at & string (ISO 8601) & The timestamp indicating when the server was created. \\ 
        \hline
        reasons\_to\_join & array of strings & A list of promotional highlights or unique selling points of the server. \\ 
        \hline
        social\_links & array of strings & Links to external social media or community pages associated with the server. \\ 
        \hline
        about & string & A detailed description of the server's purpose, history, and features. \\ 
        \hline
        category\_ids & array of numbers & A list of IDs representing the categories the server is associated with. \\ 
        \hline
    \end{tabular}
    \label{tab:discord_server_fields}
\end{table*}

\begin{table*}[ht!]
    \caption{Fields in the Discord Message Object. Fields marked with '?' are optional. References [1], [2], etc., refer to detailed specifications in the Discord API documentation.}
    \centering
    \renewcommand{\arraystretch}{1.3
    } % Espaçamento vertical
    \setlength{\tabcolsep}{5pt} % Espaçamento horizontal
    \begin{tabular}{|p{3.5cm}|p{3.5cm}|p{8cm}|}
        \hline
        \textbf{Field} & \textbf{Type} & \textbf{Description} \\ 
        \hline
        id & snowflake & ID of the message \\ 
        \hline
        channel\_id & snowflake & ID of the channel the message was sent in \\ 
        \hline
        author & user object & The author of this message (not guaranteed to be a valid user, see below) \\ 
        \hline
        content & string & Contents of the message \\ 
        \hline
        timestamp & ISO8601 timestamp & When this message was sent \\ 
        \hline
        edited\_timestamp & ?ISO8601 timestamp & When this message was edited (or null if never) \\ 
        \hline
        tts & boolean & Whether this was a TTS message \\ 
        \hline
        mention\_everyone & boolean & Whether this message mentions everyone \\ 
        \hline
        mentions & array of user objects & Users specifically mentioned in the message \\ 
        \hline
        mention\_roles & array of role object IDs & Roles specifically mentioned in this message \\ 
        \hline
        mention\_channels? & array of channel mention objects & Channels specifically mentioned in this message \\ 
        \hline
        attachments & array of attachment objects & Any attached files \\ 
        \hline
        embeds & array of embed objects & Any embedded content \\ 
        \hline
        reactions? & array of reaction objects & Reactions to the message \\ 
        \hline
        nonce? & integer or string & Used for validating a message was sent \\ 
        \hline
        pinned & boolean & Whether this message is pinned \\ 
        \hline
        webhook\_id? & snowflake & If the message is generated by a webhook, this is the webhook's ID \\ 
        \hline
        type & integer & Type of message \\ 
        \hline
        activity? & message activity object & Sent with Rich Presence-related chat embeds \\ 
        \hline
        application? & partial application object & Sent with Rich Presence-related chat embeds \\ 
        \hline
        application\_id? & snowflake & If the message is an Interaction or application-owned webhook, this is the ID of the application \\ 
        \hline
        flags? & integer & Message flags combined as a bitfield \\ 
        \hline
        message\_reference? & message reference object & Data showing the source of a crosspost, channel follow add, pin, or reply message \\ 
        \hline
        referenced\_message? [4] & ?message object & The message associated with the message\_reference \\ 
        \hline
        thread? & channel object & The thread that was started from this message, includes thread member object \\ 
        \hline
        components? & array of message components & Sent if the message contains components like buttons, action rows, or other interactive components \\ 
        \hline
        sticker\_items? & array of message sticker item objects & Sent if the message contains stickers \\ 
        \hline
        stickers? & array of sticker objects & Deprecated: the stickers sent with the message \\ 
        \hline
        position? & integer & A generally increasing integer (there may be gaps or duplicates) that represents the approximate position of the message in a thread \\ 
        \hline
        role\_subscription\_data? & role subscription data object & Data of the role subscription purchase or renewal that prompted this ROLE\_SUBSCRIPTION\_PURCHASE message \\ 
        \hline
        resolved? & resolved data & Data for users, members, channels, and roles in the message's auto-populated select menus \\ 
        \hline
        poll? [2] & poll object & A poll! \\ 
        \hline
        call? & message call object & The call associated with the message \\ 
        \hline
    \end{tabular}
    \label{tab:discord_message_fields}
\end{table*}

\twocolumn


\end{document}
