%File: anonymous-submission-latex-2025.tex
\documentclass[letterpaper]{article} % DO NOT CHANGE THIS
\usepackage{aaai25}  % DO NOT CHANGE THIS
\usepackage{times}  % DO NOT CHANGE THIS
\usepackage{helvet}  % DO NOT CHANGE THIS
\usepackage{courier}  % DO NOT CHANGE THIS
\usepackage[hyphens]{url}  % DO NOT CHANGE THIS
\usepackage{graphicx} % DO NOT CHANGE THIS
\usepackage{xcolor}
\usepackage{afterpage}
\usepackage{multicol}
\usepackage{longtable}
\usepackage{booktabs}
\usepackage{lscape} % Para paisagem, caso seja necessário

% \usepackage{ulem} % Para riscar texto
% \normalem % Evita que o ulem modifique o comportamento de \emph
\urlstyle{rm} % DO NOT CHANGE THIS
\def\UrlFont{\rm}  % DO NOT CHANGE THIS
\usepackage{natbib}  % DO NOT CHANGE THIS AND DO NOT ADD ANY OPTIONS TO IT
\usepackage{caption} % DO NOT CHANGE THIS AND DO NOT ADD ANY OPTIONS TO IT
\frenchspacing  % DO NOT CHANGE THIS
\setlength{\pdfpagewidth}{8.5in} % DO NOT CHANGE THIS
\setlength{\pdfpageheight}{11in} % DO NOT CHANGE THIS


\usepackage{aaai25}  % DO NOT CHANGE THIS
\usepackage{times}  % DO NOT CHANGE THIS
\usepackage{helvet}  % DO NOT CHANGE THIS
\usepackage{courier}  % DO NOT CHANGE THIS
\usepackage[hyphens]{url}  % DO NOT CHANGE THIS
\usepackage{graphicx} % DO NOT CHANGE THIS
\urlstyle{rm} % DO NOT CHANGE THIS
\def\UrlFont{\rm}  % DO NOT CHANGE THIS
\usepackage{natbib}  % DO NOT CHANGE THIS AND DO NOT ADD ANY OPTIONS TO IT
\usepackage{caption} % DO NOT CHANGE THIS AND DO NOT ADD ANY OPTIONS TO IT
\DeclareCaptionStyle{ruled}{labelfont=normalfont,labelsep=colon,strut=off} % DO NOT CHANGE THIS
\frenchspacing  % DO NOT CHANGE THIS
\setlength{\pdfpagewidth}{8.5in}  % DO NOT CHANGE THIS
\setlength{\pdfpageheight}{11in}  % DO NOT CHANGE THIS
%
% These are recommended to typeset algorithms but not required. See the subsubsection on algorithms. Remove them if you don't have algorithms in your paper.
\usepackage{algorithm}
\usepackage{algorithmic}
% Checklist macros
\usepackage{xcolor}
\newcommand{\answerYes}[1]{\textcolor{blue}{#1}} 
\newcommand{\answerNo}[1]{\textcolor{teal}{#1}} 
\newcommand{\answerNA}[1]{\textcolor{gray}{#1}} 
\newcommand{\answerTODO}[1]{\textcolor{red}{#1}} 
%
% These are recommended to typeset algorithms but not required. See the subsubsection on algorithms. Remove them if you don't have algorithms in your paper.
\usepackage{algorithm}
\usepackage{algorithmic}

%
% These are are recommended to typeset listputgs but not required. See the subsubsection on listing. Remove this block if you don't have listings in your paper.
\usepackage{newfloat}
\usepackage{listings}
\DeclareCaptionStyle{ruled}{labelfont=normalfont,labelsep=colon,strut=off} % DO NOT CHANGE THIS
\lstset{%
	basicstyle={\footnotesize\ttfamily},% footnotesize acceptable for monospace
	numbers=left,numberstyle=\footnotesize,xleftmargin=2em,% show line numbers, remove this entire line if you don't want the numbers.
	aboveskip=0pt,belowskip=0pt,%
	showstringspaces=false,tabsize=2,breaklines=true}
\floatstyle{ruled}
\newfloat{listing}{tb}{lst}{}
\floatname{listing}{Listing}
%
% Keep the \pdfinfo as shown here. There's no need
% for you to add the /Title and /Author tags.
\pdfinfo{
/TemplateVersion (2025.1)
}


% DISALLOWED PACKAGES
% \usepackage{authblk} -- This package is specifically forbidden
% \usepackage{balance} -- This package is specifically forbidden
% \usepackage{color (if used in text)
% \usepackage{CJK} -- This package is specifically forbidden
% \usepackage{float} -- This package is specifically forbidden
% \usepackage{flushend} -- This package is specifically forbidden
% \usepackage{fontenc} -- This package is specifically forbidden
% \usepackage{fullpage} -- This package is specifically forbidden
% \usepackage{geometry} -- This package is specifically forbidden
% \usepackage{grffile} -- This package is specifically forbidden
% \usepackage{hyperref} -- This package is specifically forbidden
% \usepackage{navigator} -- This package is specifically forbidden
% (or any other package that embeds links such as navigator or hyperref)
% \indentfirst} -- This package is specifically forbidden
% \layout} -- This package is specifically forbidden
% \multicol} -- This package is specifically forbidden
% \nameref} -- This package is specifically forbidden
% \usepackage{savetrees} -- This package is specifically forbidden
% \usepackage{setspace} -- This package is specifically forbidden
% \usepackage{stfloats} -- This package is specifically forbidden
% \usepackage{tabu} -- This package is specifically forbidden
% \usepackage{titlesec} -- This package is specifically forbidden
% \usepackage{tocbibind} -- This package is specifically forbidden
% \usepackage{ulem} -- This package is specifically forbidden
% \usepackage{wrapfig} -- This package is specifically forbidden
% DISALLOWED COMMANDS
\nocopyright % -- Your paper will not be published if you use this command
% \addtolength -- This command may not be used
% \balance -- This command may not be used
% \baselinestretch -- Your paper will not be published if you use this command
% \clearpage -- No page breaks of any kind may be used for the final version of your paper
% \columnsep -- This command may not be used
% \newpage -- No page breaks of any kind may be used for the final version of your paper
% \pagebreak -- No page breaks of any kind may be used for the final version of your paperr
% \pagestyle -- This command may not be used
% \tiny -- This is not an acceptable font size.
% \vspace{- -- No negative value may be used in proximity of a caption, figure, table, section, subsection, subsubsection, or reference
% \vskip{- -- No negative value may be used to alter spacing above or below a caption, figure, table, section, subsection, subsubsection, or reference

\setcounter{secnumdepth}{0} %May be changed to 1 or 2 if section numbers are desired.

% The file aaai25.sty is the style file for AAAI Press
% proceedings, working notes, and technical reports.
%

\newcommand{\yan}[1]{{\emph{\color{magenta}#1 -- \textbf{Yan}}}}
\newcommand{\caio}[1]{{\emph{\color{blue}#1 -- \textbf{Caio}}}}
\newcommand{\gi}[1]{{\emph{\color{red}#1 -- \textbf{Gi}}}}
\newcommand{\pr}[1]{{\emph{\color{purple}#1 -- \textbf{Robles}}}}
\newcommand{\lu}[1]{{\emph{\color{brown}#1 -- \textbf{Luísa}}}}
\newcommand{\vic}[1]{{\emph{\color{pink}#1 -- \textbf{vic}}}}


% Title
\title{Discord Unveiled: A Comprehensive Dataset of Public Communication (2015-2024)}

% Your title must be in mixed case, not sentence case.
% That means all verbs (including short verbs like be, is, using,and go),
% nouns, adverbs, adjectives should be capitalized, including both words in hyphenated terms, while
% articles, conjunctions, and prepositions are lower case unless they
% directly follow a colon or long dash
\author{
    %Authors
    % All authors must be in the same font size and format.
    Yan Aquino \textsuperscript{\rm 1},
    Pedro Bento \textsuperscript{\rm 1}, 
    Arthur Buzelin \textsuperscript{\rm 1},
    Lucas Dayrell \textsuperscript{\rm 1}, 
    Samira Malaquias \textsuperscript{\rm 1}, \\
    Caio Santana \textsuperscript{\rm 1}, 
    Victoria Estanislau \textsuperscript{\rm 1},
    Pedro Dutenhefner \textsuperscript{\rm 1},
    Guilherme H. G. Evangelista \textsuperscript{\rm 1},\\
    Luisa G. Porfírio \textsuperscript{\rm 1},
    Caio Souza Grossi \textsuperscript{\rm 1},
    Pedro B. Rigueira \textsuperscript{\rm 1},\\
    Virgilio Almeida \textsuperscript{\rm 1},
    Gisele L. Pappa \textsuperscript{\rm 1},
    Wagner Meira Jr. \textsuperscript{\rm 1}
}
\affiliations{
    %Afiliations
    \textsuperscript{\rm 1}Universidade Federal de Minas Gerais - UFMG\\
    % If you have multiple authors and multiple affiliations
    % use superscripts in text and roman font to identify them.
    % For example,

    % Sunil Issar\textsuperscript{\rm 2},
    % J. Scott Penberthy\textsuperscript{\rm 3},
    % George Ferguson\textsuperscript{\rm 4},
    % Hans Guesgen\textsuperscript{\rm 5}
    % Note that the comma should be placed after the superscript

    
    Belo Horizonte, Brazil\\
    % email address must be in roman text type, not monospace or sans serif
    \{yanaquino, pedro.bento, arthurbuzelin, lucasdayrell, samiramalaquias, caiosantana, victoria.estanislau, guilherme.evangelhista, luisagontijo, caio.grossi, pedrobacelar.rigueira, virgilio, glpappa, meira\}@dcc.ufmg.br, pedroroblesduten@ufmg.br
    
%
% See more examples next
}

%Example, Single Author, ->> remove \iffalse,\fi and place them surrounding AAAI title to use it
\iffalse
\title{My Publication Title --- Single Author}
\author {
    Author Name
}
\affiliations{
    Affiliation\\
    Affiliation Line 2\\
    name@example.com
}
\fi

\iffalse
%Example, Multiple Authors, ->> remove \iffalse,\fi and place them surrounding AAAI title to use it
\title{My Publication Title --- Multiple Authors}
\author {
    % Authors
    First Author Name\textsuperscript{\rm 1},
    Second Author Name\textsuperscript{\rm 2},
    Third Author Name\textsuperscript{\rm 1}
}
\affiliations {
    % Affiliations
    \textsuperscript{\rm 1}Affiliation 1\\
    \textsuperscript{\rm 2}Affiliation 2\\
    firstAuthor@affiliation1.com, secondAuthor@affilation2.com, thirdAuthor@affiliation1.com, caio.grossi
}
\fi


% REMOVE THIS: bibentry
% This is only needed to show inline citations in the guidelines document. You should not need it and can safely delete it.
\usepackage{bibentry}
% END REMOVE bibentry

\begin{document}

\maketitle

\begin{abstract}
Discord has evolved from a gaming-focused communication tool into a versatile platform supporting diverse online communities. Despite its large user base and active public servers, academic research on Discord remains limited due to data accessibility challenges. This paper introduces \textbf{Discord Unveiled: A Comprehensive Dataset of Public Communication (2015-2024)}, the most extensive Discord public server's data to date. The dataset comprises over \textbf{2.05 billion messages} from \textbf{4.74 million users} across \textbf{3,167 public servers}, representing approximately 10\% of servers listed in Discord’s Discovery feature. Spanning from Discord’s launch in 2015 to the end of 2024, it offers a robust temporal and thematic framework for analyzing decentralized moderation, community governance, information dissemination, and social dynamics. Data was collected through Discord’s public API, adhering to ethical guidelines and privacy standards via anonymization techniques. Organized into structured JSON files, the dataset facilitates seamless integration with computational social science methodologies. Preliminary analyses reveal significant trends in user engagement, bot utilization, and linguistic diversity, with English predominating alongside substantial representations of Spanish, French, and Portuguese. Additionally, prevalent community themes such as social, art, music, and memes highlight Discord’s expansion beyond its gaming origins.


\end{abstract}

\section{Introduction}
\label{sec:introduction}
The business processes of organizations are experiencing ever-increasing complexity due to the large amount of data, high number of users, and high-tech devices involved \cite{martin2021pmopportunitieschallenges, beerepoot2023biggestbpmproblems}. This complexity may cause business processes to deviate from normal control flow due to unforeseen and disruptive anomalies \cite{adams2023proceddsriftdetection}. These control-flow anomalies manifest as unknown, skipped, and wrongly-ordered activities in the traces of event logs monitored from the execution of business processes \cite{ko2023adsystematicreview}. For the sake of clarity, let us consider an illustrative example of such anomalies. Figure \ref{FP_ANOMALIES} shows a so-called event log footprint, which captures the control flow relations of four activities of a hypothetical event log. In particular, this footprint captures the control-flow relations between activities \texttt{a}, \texttt{b}, \texttt{c} and \texttt{d}. These are the causal ($\rightarrow$) relation, concurrent ($\parallel$) relation, and other ($\#$) relations such as exclusivity or non-local dependency \cite{aalst2022pmhandbook}. In addition, on the right are six traces, of which five exhibit skipped, wrongly-ordered and unknown control-flow anomalies. For example, $\langle$\texttt{a b d}$\rangle$ has a skipped activity, which is \texttt{c}. Because of this skipped activity, the control-flow relation \texttt{b}$\,\#\,$\texttt{d} is violated, since \texttt{d} directly follows \texttt{b} in the anomalous trace.
\begin{figure}[!t]
\centering
\includegraphics[width=0.9\columnwidth]{images/FP_ANOMALIES.png}
\caption{An example event log footprint with six traces, of which five exhibit control-flow anomalies.}
\label{FP_ANOMALIES}
\end{figure}

\subsection{Control-flow anomaly detection}
Control-flow anomaly detection techniques aim to characterize the normal control flow from event logs and verify whether these deviations occur in new event logs \cite{ko2023adsystematicreview}. To develop control-flow anomaly detection techniques, \revision{process mining} has seen widespread adoption owing to process discovery and \revision{conformance checking}. On the one hand, process discovery is a set of algorithms that encode control-flow relations as a set of model elements and constraints according to a given modeling formalism \cite{aalst2022pmhandbook}; hereafter, we refer to the Petri net, a widespread modeling formalism. On the other hand, \revision{conformance checking} is an explainable set of algorithms that allows linking any deviations with the reference Petri net and providing the fitness measure, namely a measure of how much the Petri net fits the new event log \cite{aalst2022pmhandbook}. Many control-flow anomaly detection techniques based on \revision{conformance checking} (hereafter, \revision{conformance checking}-based techniques) use the fitness measure to determine whether an event log is anomalous \cite{bezerra2009pmad, bezerra2013adlogspais, myers2018icsadpm, pecchia2020applicationfailuresanalysispm}. 

The scientific literature also includes many \revision{conformance checking}-independent techniques for control-flow anomaly detection that combine specific types of trace encodings with machine/deep learning \cite{ko2023adsystematicreview, tavares2023pmtraceencoding}. Whereas these techniques are very effective, their explainability is challenging due to both the type of trace encoding employed and the machine/deep learning model used \cite{rawal2022trustworthyaiadvances,li2023explainablead}. Hence, in the following, we focus on the shortcomings of \revision{conformance checking}-based techniques to investigate whether it is possible to support the development of competitive control-flow anomaly detection techniques while maintaining the explainable nature of \revision{conformance checking}.
\begin{figure}[!t]
\centering
\includegraphics[width=\columnwidth]{images/HIGH_LEVEL_VIEW.png}
\caption{A high-level view of the proposed framework for combining \revision{process mining}-based feature extraction with dimensionality reduction for control-flow anomaly detection.}
\label{HIGH_LEVEL_VIEW}
\end{figure}

\subsection{Shortcomings of \revision{conformance checking}-based techniques}
Unfortunately, the detection effectiveness of \revision{conformance checking}-based techniques is affected by noisy data and low-quality Petri nets, which may be due to human errors in the modeling process or representational bias of process discovery algorithms \cite{bezerra2013adlogspais, pecchia2020applicationfailuresanalysispm, aalst2016pm}. Specifically, on the one hand, noisy data may introduce infrequent and deceptive control-flow relations that may result in inconsistent fitness measures, whereas, on the other hand, checking event logs against a low-quality Petri net could lead to an unreliable distribution of fitness measures. Nonetheless, such Petri nets can still be used as references to obtain insightful information for \revision{process mining}-based feature extraction, supporting the development of competitive and explainable \revision{conformance checking}-based techniques for control-flow anomaly detection despite the problems above. For example, a few works outline that token-based \revision{conformance checking} can be used for \revision{process mining}-based feature extraction to build tabular data and develop effective \revision{conformance checking}-based techniques for control-flow anomaly detection \cite{singh2022lapmsh, debenedictis2023dtadiiot}. However, to the best of our knowledge, the scientific literature lacks a structured proposal for \revision{process mining}-based feature extraction using the state-of-the-art \revision{conformance checking} variant, namely alignment-based \revision{conformance checking}.

\subsection{Contributions}
We propose a novel \revision{process mining}-based feature extraction approach with alignment-based \revision{conformance checking}. This variant aligns the deviating control flow with a reference Petri net; the resulting alignment can be inspected to extract additional statistics such as the number of times a given activity caused mismatches \cite{aalst2022pmhandbook}. We integrate this approach into a flexible and explainable framework for developing techniques for control-flow anomaly detection. The framework combines \revision{process mining}-based feature extraction and dimensionality reduction to handle high-dimensional feature sets, achieve detection effectiveness, and support explainability. Notably, in addition to our proposed \revision{process mining}-based feature extraction approach, the framework allows employing other approaches, enabling a fair comparison of multiple \revision{conformance checking}-based and \revision{conformance checking}-independent techniques for control-flow anomaly detection. Figure \ref{HIGH_LEVEL_VIEW} shows a high-level view of the framework. Business processes are monitored, and event logs obtained from the database of information systems. Subsequently, \revision{process mining}-based feature extraction is applied to these event logs and tabular data input to dimensionality reduction to identify control-flow anomalies. We apply several \revision{conformance checking}-based and \revision{conformance checking}-independent framework techniques to publicly available datasets, simulated data of a case study from railways, and real-world data of a case study from healthcare. We show that the framework techniques implementing our approach outperform the baseline \revision{conformance checking}-based techniques while maintaining the explainable nature of \revision{conformance checking}.

In summary, the contributions of this paper are as follows.
\begin{itemize}
    \item{
        A novel \revision{process mining}-based feature extraction approach to support the development of competitive and explainable \revision{conformance checking}-based techniques for control-flow anomaly detection.
    }
    \item{
        A flexible and explainable framework for developing techniques for control-flow anomaly detection using \revision{process mining}-based feature extraction and dimensionality reduction.
    }
    \item{
        Application to synthetic and real-world datasets of several \revision{conformance checking}-based and \revision{conformance checking}-independent framework techniques, evaluating their detection effectiveness and explainability.
    }
\end{itemize}

The rest of the paper is organized as follows.
\begin{itemize}
    \item Section \ref{sec:related_work} reviews the existing techniques for control-flow anomaly detection, categorizing them into \revision{conformance checking}-based and \revision{conformance checking}-independent techniques.
    \item Section \ref{sec:abccfe} provides the preliminaries of \revision{process mining} to establish the notation used throughout the paper, and delves into the details of the proposed \revision{process mining}-based feature extraction approach with alignment-based \revision{conformance checking}.
    \item Section \ref{sec:framework} describes the framework for developing \revision{conformance checking}-based and \revision{conformance checking}-independent techniques for control-flow anomaly detection that combine \revision{process mining}-based feature extraction and dimensionality reduction.
    \item Section \ref{sec:evaluation} presents the experiments conducted with multiple framework and baseline techniques using data from publicly available datasets and case studies.
    \item Section \ref{sec:conclusions} draws the conclusions and presents future work.
\end{itemize}


\section{What is Discord?}

Discord is a communication platform launched in 2015 by Discord Inc., designed to enable real-time interaction through text, voice, and video. Initially popular within gaming communities, Discord has expanded to accommodate diverse user groups, including educators, hobbyists, and professional teams\footnote{https://discord.com/safety/360044149331-what-is-discord}. Similar to platforms like Telegram and WhatsApp, Discord allows users to create and join customizable group spaces for synchronous and asynchronous communication, which can be either public or private.


Discord is organized into ``servers", virtual spaces designed for communities to connect, share content, and engage in conversations. Each server is divided into ``channels," which are dedicated spaces for specific types of conversations or activities. Text channels are used for written discussions, enabling members to exchange messages, share links, images, and other media. Voice channels, on the other hand, facilitate real-time audio conversations, often used for meetings, gaming sessions, or casual chats, and can also support video and screen sharing. All of these dynamics can be seen in Figure \ref{fig:disc}. Additionally, servers can be customized with ``roles" which are permission-based assignments that define what actions members can perform within the server. Roles can grant or restrict access to specific channels, allow users to manage messages, ban or mute members, or even adjust server settings. Combined with bots to automate tasks and server-specific rules to guide member behavior, this flexible architecture supports a wide range of communities, from small friend groups to large public forums.


A distinguishing feature of Discord is its user-driven moderation. While traditional social networks such as Twitter, Facebook, or YouTube have historically relied on centralized moderation managed by the platform itself \cite{wilson2020hate}, there is a very recent trend toward decentralizing moderation, exemplified by initiatives like community-based fact-checking \cite{balasubramanian2024publicdatasettrackingsocial}. Discord has always explicitly delegated the responsibility of rule-setting, behavior management, and access control to server administrators and moderators. This long-standing approach has fostered a well-established environment of non-centralized moderation, offering valuable insights into the dynamics and challenges of this model as other platforms begin to adopt similar strategies. For instance, as shown by \cite{moderation-challenges}, this flexibility can also be exploited to facilitate the presence of extremist and hateful groups and networks.

Bots are another key element of Discord’s ecosystem, setting it apart from other social networks. Unlike bots on other platforms, which are often limited to automated content posting or analytics, Discord bots are highly customizable and play an interactive role \cite{moderation-discord-bots, bots-discord}. They can automate moderation tasks, provide community engagement features (e.g., games, polls, or reminders), and integrate external services like music streaming, analytics tools, and generative AI services like Midjourney and ChatGPT. Discord actively encourages the creation and integration of bots by providing developers with extensive API documentation, support, and a bot-friendly environment, fostering innovation and customization within the platform \footnote{https://discord.com/developers/docs/intro}. Their extensive use has become central to how users interact within servers and how these communities are structured and managed.

To facilitate the discovery of public servers, Discord introduced the \textit{Server Discovery} feature. This feature allows users to search, explore and join public servers that align with their interests\footnote{https://support.discord.com/hc/en-us/articles/360023968311-Server-Discovery}. Servers listed in the Discovery tab must adhere to specific guidelines, including maintaining a welcoming environment, avoiding graphic or sexual content, and having an accurate server name and description\footnote{https://support.discord.com/hc/en-us/articles/4409308485271-Discovery-Guidelines}. These servers are the focus of our research, as they offer a curated look at communities
built around a wide variety of topics, each with at least 1000 members.
%with public servers being discoverable through Discord’s Discovery feature, provided they meet specific guidelines regarding content and moderation. 
% This work focuses exclusively on public Discord servers available in the Discovery feature
 %, from gaming and technology to education and entertainment.

%\yan{add um ultimo paragrafo?}

\section{RELATED WORK}
\label{sec:relatedwork}
In this section, we describe the previous works related to our proposal, which are divided into two parts. In Section~\ref{sec:relatedwork_exoplanet}, we present a review of approaches based on machine learning techniques for the detection of planetary transit signals. Section~\ref{sec:relatedwork_attention} provides an account of the approaches based on attention mechanisms applied in Astronomy.\par

\subsection{Exoplanet detection}
\label{sec:relatedwork_exoplanet}
Machine learning methods have achieved great performance for the automatic selection of exoplanet transit signals. One of the earliest applications of machine learning is a model named Autovetter \citep{MCcauliff}, which is a random forest (RF) model based on characteristics derived from Kepler pipeline statistics to classify exoplanet and false positive signals. Then, other studies emerged that also used supervised learning. \cite{mislis2016sidra} also used a RF, but unlike the work by \citet{MCcauliff}, they used simulated light curves and a box least square \citep[BLS;][]{kovacs2002box}-based periodogram to search for transiting exoplanets. \citet{thompson2015machine} proposed a k-nearest neighbors model for Kepler data to determine if a given signal has similarity to known transits. Unsupervised learning techniques were also applied, such as self-organizing maps (SOM), proposed \citet{armstrong2016transit}; which implements an architecture to segment similar light curves. In the same way, \citet{armstrong2018automatic} developed a combination of supervised and unsupervised learning, including RF and SOM models. In general, these approaches require a previous phase of feature engineering for each light curve. \par

%DL is a modern data-driven technology that automatically extracts characteristics, and that has been successful in classification problems from a variety of application domains. The architecture relies on several layers of NNs of simple interconnected units and uses layers to build increasingly complex and useful features by means of linear and non-linear transformation. This family of models is capable of generating increasingly high-level representations \citep{lecun2015deep}.

The application of DL for exoplanetary signal detection has evolved rapidly in recent years and has become very popular in planetary science.  \citet{pearson2018} and \citet{zucker2018shallow} developed CNN-based algorithms that learn from synthetic data to search for exoplanets. Perhaps one of the most successful applications of the DL models in transit detection was that of \citet{Shallue_2018}; who, in collaboration with Google, proposed a CNN named AstroNet that recognizes exoplanet signals in real data from Kepler. AstroNet uses the training set of labelled TCEs from the Autovetter planet candidate catalog of Q1–Q17 data release 24 (DR24) of the Kepler mission \citep{catanzarite2015autovetter}. AstroNet analyses the data in two views: a ``global view'', and ``local view'' \citep{Shallue_2018}. \par


% The global view shows the characteristics of the light curve over an orbital period, and a local view shows the moment at occurring the transit in detail

%different = space-based

Based on AstroNet, researchers have modified the original AstroNet model to rank candidates from different surveys, specifically for Kepler and TESS missions. \citet{ansdell2018scientific} developed a CNN trained on Kepler data, and included for the first time the information on the centroids, showing that the model improves performance considerably. Then, \citet{osborn2020rapid} and \citet{yu2019identifying} also included the centroids information, but in addition, \citet{osborn2020rapid} included information of the stellar and transit parameters. Finally, \citet{rao2021nigraha} proposed a pipeline that includes a new ``half-phase'' view of the transit signal. This half-phase view represents a transit view with a different time and phase. The purpose of this view is to recover any possible secondary eclipse (the object hiding behind the disk of the primary star).


%last pipeline applies a procedure after the prediction of the model to obtain new candidates, this process is carried out through a series of steps that include the evaluation with Discovery and Validation of Exoplanets (DAVE) \citet{kostov2019discovery} that was adapted for the TESS telescope.\par
%



\subsection{Attention mechanisms in astronomy}
\label{sec:relatedwork_attention}
Despite the remarkable success of attention mechanisms in sequential data, few papers have exploited their advantages in astronomy. In particular, there are no models based on attention mechanisms for detecting planets. Below we present a summary of the main applications of this modeling approach to astronomy, based on two points of view; performance and interpretability of the model.\par
%Attention mechanisms have not yet been explored in all sub-areas of astronomy. However, recent works show a successful application of the mechanism.
%performance

The application of attention mechanisms has shown improvements in the performance of some regression and classification tasks compared to previous approaches. One of the first implementations of the attention mechanism was to find gravitational lenses proposed by \citet{thuruthipilly2021finding}. They designed 21 self-attention-based encoder models, where each model was trained separately with 18,000 simulated images, demonstrating that the model based on the Transformer has a better performance and uses fewer trainable parameters compared to CNN. A novel application was proposed by \citet{lin2021galaxy} for the morphological classification of galaxies, who used an architecture derived from the Transformer, named Vision Transformer (VIT) \citep{dosovitskiy2020image}. \citet{lin2021galaxy} demonstrated competitive results compared to CNNs. Another application with successful results was proposed by \citet{zerveas2021transformer}; which first proposed a transformer-based framework for learning unsupervised representations of multivariate time series. Their methodology takes advantage of unlabeled data to train an encoder and extract dense vector representations of time series. Subsequently, they evaluate the model for regression and classification tasks, demonstrating better performance than other state-of-the-art supervised methods, even with data sets with limited samples.

%interpretation
Regarding the interpretability of the model, a recent contribution that analyses the attention maps was presented by \citet{bowles20212}, which explored the use of group-equivariant self-attention for radio astronomy classification. Compared to other approaches, this model analysed the attention maps of the predictions and showed that the mechanism extracts the brightest spots and jets of the radio source more clearly. This indicates that attention maps for prediction interpretation could help experts see patterns that the human eye often misses. \par

In the field of variable stars, \citet{allam2021paying} employed the mechanism for classifying multivariate time series in variable stars. And additionally, \citet{allam2021paying} showed that the activation weights are accommodated according to the variation in brightness of the star, achieving a more interpretable model. And finally, related to the TESS telescope, \citet{morvan2022don} proposed a model that removes the noise from the light curves through the distribution of attention weights. \citet{morvan2022don} showed that the use of the attention mechanism is excellent for removing noise and outliers in time series datasets compared with other approaches. In addition, the use of attention maps allowed them to show the representations learned from the model. \par

Recent attention mechanism approaches in astronomy demonstrate comparable results with earlier approaches, such as CNNs. At the same time, they offer interpretability of their results, which allows a post-prediction analysis. \par



\section{Dataset}
\label{sec:dataset}

\subsection{Data Collection}

To analyze political discussions on Discord, we followed the methodology in \cite{singh2024Cross-Platform}, collecting messages from politically-oriented public servers in compliance with Discord's platform policies.

Using Discord's Discovery feature, we employed a web scraper to extract server invitation links, names, and descriptions, focusing on public servers accessible without participation. Invitation links were used to access data via the Discord API. To ensure relevance, we filtered servers using keywords related to the 2024 U.S. elections (e.g., Trump, Kamala, MAGA), as outlined in \cite{balasubramanian2024publicdatasettrackingsocial}. This resulted in 302 server links, further narrowed to 81 English-speaking, politics-focused servers based on their names and descriptions.

Public messages were retrieved from these servers using the Discord API, collecting metadata such as \textit{content}, \textit{user ID}, \textit{username}, \textit{timestamp}, \textit{bot flag}, \textit{mentions}, and \textit{interactions}. Through this process, we gathered \textbf{33,373,229 messages} from \textbf{82,109 users} across \textbf{81 servers}, including \textbf{1,912,750 messages} from \textbf{633 bots}. Data collection occurred between November 13th and 15th, covering messages sent from January 1st to November 12th, just after the 2024 U.S. election.

\subsection{Characterizing the Political Spectrum}
\label{sec:timeline}

A key aspect of our research is distinguishing between Republican- and Democratic-aligned Discord servers. To categorize their political alignment, we relied on server names and self-descriptions, which often include rules, community guidelines, and references to key ideologies or figures. Each server's name and description were manually reviewed based on predefined, objective criteria, focusing on explicit political themes or mentions of prominent figures. This process allowed us to classify servers into three categories, ensuring a systematic and unbiased alignment determination.

\begin{itemize}
    \item \textbf{Republican-aligned}: Servers referencing Republican and right-wing and ideologies, movements, or figures (e.g., MAGA, Conservative, Traditional, Trump).  
    \item \textbf{Democratic-aligned}: Servers mentioning Democratic and left-wing ideologies, movements, or figures (e.g., Progressive, Liberal, Socialist, Biden, Kamala).  
    \item \textbf{Unaligned}: Servers with no defined spectrum and ideologies or opened to general political debate from all orientations.
\end{itemize}

To ensure the reliability and consistency of our classification, three independent reviewers assessed the classification following the specified set of criteria. The inter-rater agreement of their classifications was evaluated using Fleiss' Kappa \cite{fleiss1971measuring}, with a resulting Kappa value of \( 0.8191 \), indicating an almost perfect agreement among the reviewers. Disagreements were resolved by adopting the majority classification, as there were no instances where a server received different classifications from all three reviewers. This process guaranteed the consistency and accuracy of the final categorization.

Through this process, we identified \textbf{7 Republican-aligned servers}, \textbf{9 Democratic-aligned servers}, and \textbf{65 unaligned servers}.

Table \ref{tab:statistics} shows the statistics of the collected data. Notably, while Democratic- and Republican-aligned servers had a comparable number of user messages, users in the latter servers were significantly more active, posting more than double the number of messages per user compared to their Democratic counterparts. 
This suggests that, in our sample, Democratic-aligned servers attract more users, but these users were less engaged in text-based discussions. Additionally, around 10\% of the messages across all server categories were posted by bots. 

\subsection{Temporal Data} 

Throughout this paper, we refer to the election candidates using the names adopted by their respective campaigns: \textit{Kamala}, \textit{Biden}, and \textit{Trump}. To examine how the content of text messages evolves based on the political alignment of servers, we divided the 2024 election year into three periods: \textbf{Biden vs Trump} (January 1 to July 21), \textbf{Kamala vs Trump} (July 21 to September 20), and the \textbf{Voting Period} (after September 20). These periods reflect key phases of the election: the early campaign dominated by Biden and Trump, the shift in dynamics with Kamala Harris replacing Joe Biden as the Democratic candidate, and the final voting stage focused on electoral outcomes and their implications. This segmentation enables an analysis of how discourse responds to pivotal electoral moments.

Figure \ref{fig:line-plot} illustrates the distribution of messages over time, highlighting trends in total messages volume and mentions of each candidate. Prior to Biden's withdrawal on July 21, mentions of Biden and Trump were relatively balanced. However, following Kamala's entry into the race, mentions of Trump surged significantly, a trend further amplified by an assassination attempt on him, solidifying his dominance in the discourse. The only instance where Trump’s mentions were exceeded occurred during the first debate, as concerns about Biden’s age and cognitive abilities temporarily shifted the focus. In the final stages of the election, mentions of all three candidates rose, with Trump’s mentions peaking as he emerged as the victor.

\section{Data Availability}
The dataset presented in this study has been made publicly available and can be accessed via DOI: 10.5281/zenodo.14658505\footnote{https://zenodo.org/records/14658505}. The data is provided in a compressed format, which can be decompressed for analysis. Detailed instructions for accessing and utilizing the dataset are provided in this article and the platform.


\section{Dataset Characterization}

\begin{figure*} [t]
    \centering
    \includegraphics[width=1\linewidth]{mensagens_por_dia_comparativo.pdf}
    \caption{Evolution of the daily number of messages sent over time. The top panel presents the complete time series, distinguishing between messages sent by regular users (in blue) and bots (in red). The bottom-left panel focuses on the initial period, covering up to October 2022, while the bottom-right panel highlights the most recent period, from June 2022 to January 2024. Notice the graph scales are different.}
    \label{fig:messages_per_time}
\end{figure*}

% \gi{sugestao: fazer dos proximos paragrafos o inicio da secao de caracterizacao.}

The \textbf{3,167 collected servers} yielded a total of \textbf{2,052,206,308 unique messages} sent by \textbf{4,735,057 distinct users}. From the total number of messages,   \textbf{364,447,569 (17\%) originated from bots}. The data spans from Discord's launch, on \textbf{May 13, 2015}, to \textbf{December 17, 2024}, when the data collection process began.

Figure \ref{fig:messages_per_time} shows the number of messages over time. 
Note that Discord's retroactive approach to chat allows the dataset to encompass a timeframe broader than the actual data collection period.
Upon joining a server, users gain access to all non-deleted historical content within public channels, and the same is valid for data retrieval using their API.
Notably, 2024 stands out as the most active year across all servers, reflecting a growing network that offers ample opportunities for further exploration.

Although it is interesting to be able to access older data, it is important to note that the servers listed in the \textit{Discovery} tab most likely represent those active at the time of data collection. Servers featured in this tab typically emphasize ongoing engagement and relevance. Additionally, while Discord provides access to historical content, some servers may periodically delete older messages, which could introduce variability in the completeness of the temporal data available. These factors may influence the representativeness of older content and should be considered when analysing the dataset.

Next, we present simple but relevant analyses of the dataset. The first aspect we examined is \textbf{Bots.} As previously mentioned, Discord bots play a crucial role in enhancing user experience by automating tasks, facilitating engagement, and offering specialized functionalities. They serve as virtual assistants that can perform various tasks, such as moderation, entertainment, and utility services. Table \ref{tab:top10bots} illustrates the diverse applications of bots, highlighting their role in generating messages, being mentioned by users, and eliciting reactions, which are key indicators of their impact and utility in Discord.

Bots like \textit{MEE6} and \textit{Dyno} are widely recognized as powerful moderators, enabling server administrators to enforce rules, assign roles, and monitor activity effectively. These bots are essential for maintaining order in larger communities where manual moderation would be impractical. Similarly, \textit{Arcane Premium} and \textit{Loritta} excel in general server management, providing features such as levelling systems, customizable commands, and automated event handling to enhance user engagement and server functionality. On the other hand, entertainment-focused bots like \textit{Pokétwo}, \textit{Mudae}, \textit{Karuta} and \textit{OwO} captivate users with gaming and collectable experiences, encouraging interaction through game mechanics such as Pokémon battles, waifu collections, trading cards and interacting with fictional animals. These bots create an environment where users actively participate, forming micro-communities within the larger server ecosystem. Finally, \textit{Lucky VR}, despite being the bot with the highest volume of messages, lacks extensive documentation online. Its primary focus is on gambling in virtual reality, particularly in facilitating poker games.

\textbf{Languages.} To identify the most frequent languages in our dataset, we analysed the values present in the \textit{preferred locale} field within the server metadata. Figure \ref{fig:hist-linguas} shows English (US) as the primary language for most servers, with 1,705 servers. Disregarding the ``unknown" field, the second most frequent language is Spanish (Spain), with 144 servers, followed by French, with 136 servers. Also note there is a notable linguistic diversity, including Portuguese, Russian, and German.

\begin{figure}[t]
    \centering
    \includegraphics[width=1\linewidth]{language_histogram_avancado.pdf}
    \caption{Bar plot of the number of servers by language, with the majority of servers being in English.}
    \label{fig:hist-linguas}
\end{figure}


% For instance, bots like MEE6 and Dyno are widely used for community management, handling tasks like assigning roles, enforcing rules, and moderating content. On the other hand, entertainment bots such as Mudae, Karuta, and Pokétwo engage users through games, collectibles, and interactive features, exhibiting higher levels of participation and interaction. Additionally, niche bots like AniLibria.TV provide specialized services, such as streaming updates or content delivery for targeted audiences. The bots listed in Table \ref{tab:top10bots} illustrate the diverse applications of Discord bots, highlighting their role in generating messages, being mentioned by users, and eliciting reactions, which are key indicators of their impact and utility in various online communities.

\begin{table*}[t]
\caption{Top 10 Bots by Messages, Mentions, and Reactions.}
\centering
\label{tab:top10bots}
\begin{tabular}{ccc}
\toprule
\textbf{Messages} & \textbf{Mentions} & \textbf{Reactions} \\
\midrule
\hline
\begin{tabular}{lr}
Lucky VR & 3,684,600 \\
Mudae & 2,955,892 \\
AniLibria.TV & 2,844,343 \\
Pok\'etwo & 2,005,250 \\
RoM & 1,716,654 \\
OwO & 1,523,404 \\
Mimu & 1,430,137 \\
Wan Shi Tong & 1,297,316 \\
BTE France Minecraft & 1,077,553 \\
MEE6 & 904,432 \\
\end{tabular} &
\begin{tabular}{lr}
Pok\'etwo & 304,657 \\
Dyno & 290,387 \\
AniLibria.TV & 290,286 \\
ProBot  & 278,653 \\
Loritta & 251,208 \\
Arcane Premium & 224,681 \\
Mimu & 206,157 \\
MEE6 & 198,534 \\
Dank Memer & 178,813 \\
Liquid Esports & 134,816 \\
\end{tabular} &
\begin{tabular}{lr}
Karuta & 984,693 \\
Mudae & 281,929 \\
MEE6 & 237,370 \\
XenosPD.dk & 217,634 \\
Bongo & 216,779 \\
OwO & 193,493 \\
YAGPDB.xyz & 137,052 \\
Dyno & 131,881 \\
Suggester & 88,981 \\
Emps-World & 85,889 \\
\end{tabular} \\
\bottomrule
\hline
\end{tabular}
\end{table*}

\textbf{Servers.} Discord, as a platform for community interactions, supports a wide range of themes that reflect different user interests. To analyze the thematic distribution of servers, we examined server metadata, focusing on keywords that describe their content. These keywords also improve the platform's search functionality through the Discovery feature.

Table \ref{tab:top_40_keywords} presents the 40 most frequent keywords appearing in the description field of the servers collected. Despite Discord's evolution into a versatile platform, gaming-related content remains significant, with \textit{gaming} appearing in over 15\% of descriptions. Related keywords like \textit{minecraft}, \textit{roblox}, and \textit{twitch} reinforce its gaming's central role. Other prominent themes include \textit{anime} (5.68\%), \textit{roleplay} (6.44\%), and \textit{fivem} (4.39\%), reflecting niche communities and interactive multiplayer experiences. 

Emerging interests like \textit{esports} (4.04\%) and \textit{social} (4.14\%) highlight the importance of competitive gaming and socialization. Creative and recreational themes, such as \textit{art} and \textit{music}, and educational discussions, such as \textit{programming}, demonstrate the platform's versatility beyond gaming.

\begin{table}[t]
\caption{Top 40 most frequent keywords in Discord servers description and their coverage.}
\centering
\begin{tabular}{lr|lr}
%\hline
%\multicolumn{4}{c}{\textbf{Keywords}} \\
\hline
\textbf{Keyword} & \textbf{\%} & \textbf{Keyword} & \textbf{\%} \\
\hline
gaming & 15.28 & giveaways & 2.31 \\
youtube & 15.00 & art & 2.02 \\
minecraft & 12.28 & rp & 2.02 \\
roblox & 10.83 & tiktok & 1.89 \\
twitch & 8.68 & manga & 1.83 \\
community & 8.65 & chill & 1.74 \\
roleplay & 6.44 & leagueoflegends & 1.71 \\
anime & 5.68 & rpg & 1.71 \\
fivem & 4.39 & pvp & 1.58 \\
social & 4.14 & music & 1.48 \\
esports & 4.04 & server & 1.39 \\
valorant & 3.79 & chatting & 1.36 \\
memes & 2.97 & survival & 1.36 \\
fortnite & 2.78 & gta5 & 1.33 \\
streamer & 2.59 & modding & 1.20 \\
events & 2.53 & mobile & 1.20 \\
game & 2.49 & gtav & 1.20 \\
fun & 2.46 & programming & 1.17 \\
gta & 2.43 & pc & 1.17 \\
games & 2.43 & csgo & 1.14 \\
\hline
\label{tab:top_40_keywords}
\end{tabular}
\end{table}

However, note that these keywords are only related to the server's descriptions. The context of messages goes beyond these subjects, with messages concerning topics that include mental health, political debates and web dating, for example. \looseness=-1


\section{Potential Applications}

Our dataset represents the largest publicly available collection of textual data from Discord servers, offering an unprecedented resource for studying online interactions. The dataset spans a wide range of languages, cultures, and topics, making it uniquely suited to support a diverse cross-section of research efforts and laying a robust foundation for examining the complexities of online communities. Its breadth and depth enable researchers to explore critical areas such as discourse analysis, community governance, and political debate, making it an invaluable asset for interdisciplinary studies.\looseness=-1

\textbf{Online Community Governance.} Discord's user-driven moderation model stands apart from those used in platforms like Facebook and YouTube, which rely heavily on centralized moderation enforced by platform administrators or automated systems \cite{doi:10.1177/1461444818821316}. This decentralized approach enables server owners and moderators to define and enforce community-specific rules, offering an opportunity to explore how self-governance influences online interactions, social dynamics, and conflict resolution \cite{moderationbook}. Researchers can examine the efficacy of decentralized moderation in fostering inclusive and safe environments, the strategies employed by communities to address toxic behavior, and the broader impact of granting users autonomy in governing digital spaces \cite{succesfulcommunitybook}.

\textbf{Discourse analysis.} The dataset also provides a valuable resource for advancing research across multiple scientific domains, particularly in Natural Language Processing (NLP) and Machine Learning (ML). It enables the development and evaluation of models for tasks such as sentiment analysis, intent recognition, topic modeling, and toxic or abusive language detection. Moreover, the temporal structure of the data supports studies on conversational dynamics, social network analysis, and community behavior in digital environments. This dataset can also facilitate the creation of domain-specific chatbots, recommendation systems, and tools for automated moderation, fostering innovations that bridge computational techniques with the study of human communication and online interaction.

\textbf{Political Debate.} The political debate research community faces substantial challenges in understanding how both mainstream and decentralized platforms influence political opinions and behaviors. Social networks have become increasingly critical in shaping political landscapes, not only by amplifying voices and ideologies but also by contributing to the rapid dissemination of fake news and misinformation \cite{Aimeur2023}. While much of the existing research focuses on platforms like Facebook and Twitter \cite{10.1145/3578503.3583597, doi:10.1126/science.adk3451}, Discord’s semi-private and community-driven architecture offers a distinct and underexplored environment for political conversations. 
%This dynamic has raised significant concerns about the potential for unregulated spaces to facilitate polarization, radicalization, and the erosion of trust in democratic processes \cite{moderation-challenges}. At the same time, Discord's user-moderated model provides a valuable opportunity to study these phenomena in a decentralized context, contrasting with the centralized moderation models of mainstream platforms. -> Achei que estava redundante com outra parte do texto
Our dataset enables researchers to explore the impact of digital platforms on political discourse, the propagation of misinformation, and the development of effective moderation and regulation strategies tailored to such environments.

\textbf{Mental health.} The relationship between social media usage and the prevalence of mental health issues, including self-harm and suicidal behavior, has been extensively documented, particularly among younger demographics \cite{elia2020, pater2016characterizations}. Discord, as a platform with diverse communities and user-generated content, represents a critical environment for understanding these phenomena. Recent studies in Brazil have already looked at mental health in the context of Discord, underscoring the importance of examining how the platform’s semi-private and community-oriented spaces may influence mental health outcomes \cite{webmedia}. Our multilingual dataset significantly expands the scope for such research by enabling cross-cultural and cross-linguistic analyses of mental health trends and discourse on Discord, providing a valuable foundation for identifying patterns of at-risk behavior and explore critical questions such as the prevalence of harmful behaviors or supportive interactions.

% \subsection{Social Network Analysis}
% This dataset provides a unique opportunity to study user interactions, group dynamics, and information dissemination patterns within a controlled environment. Researchers can use it to examine the spread of trends, topics, and opinions over time, as well as analyze patterns of collaboration and conflict resolution in online communities. Such studies can reveal the underlying mechanisms of information flow and social influence within digital ecosystems.

% \subsection{Behavioral and Sentiment Analysis}
% With anonymized message content and metadata, the dataset enables researchers to detect behavioral trends and shifts in online activity. Sentiment analysis can be performed to measure emotional responses to events or announcements, offering insights into public sentiment on various issues. Additionally, the dataset can facilitate the study of how moderation policies affect user behavior, providing data-driven insights into the effectiveness of such interventions.

% \subsection{Machine Learning and Natural Language Processing (NLP)}
% This dataset offers a robust foundation for developing and testing algorithms in machine learning and NLP. It can be used to train chatbots and virtual assistants for applications in customer service or education. Algorithms designed to detect toxic behavior or policy violations on online platforms can be refined using this dataset. Furthermore, models for tasks such as summarization, translation, or sentiment classification can be tested on real-world data, bridging the gap between theoretical development and practical application.

% \subsection{Platform Policy Evaluation and Development}
% The data can guide platform administrators and policymakers in evaluating the effectiveness of moderation techniques in reducing harmful behavior. It can inform the design of inclusive and user-friendly community guidelines that foster positive interactions. Additionally, the data can be used to assess the long-term impact of platform updates and feature rollouts, enabling evidence-based decision-making for better user experiences.

% \subsection{Educational and Training Purposes}
% The dataset is an invaluable resource for academic settings. It can be used to train students in data science, machine learning, or social science methods, providing real-world scenarios for hands-on learning. It also serves as a practical tool to demonstrate ethical considerations in data collection and processing, offering a controlled environment for testing hypotheses in online communication studies.

% \subsection{Ethics and Privacy Research}
% Finally, this dataset itself can serve as a case study in ethical anonymization practices. Researchers can explore novel anonymization techniques and evaluate their effectiveness, contributing to the ongoing dialogue on privacy preservation. It also provides an opportunity to study the trade-offs between data utility and user privacy in real-world scenarios, advancing the field of ethical data management.








%%%%%%%%%%%%%%%%%%%%%%%%%%%%%%%%%%%%%%%%%%%%%%%%%%%%%%%%%%%%%%%%%%%%%%%%%%%%%%%%%%%%%%%%%%%%%%%%%%%%%%%%%
\iffalse

\section{Potential Applications}

The anonymized dataset created through the process described in this work has the potential to support a variety of research areas and applications while adhering to ethical and legal standards for data protection. Below, we outline key domains and examples of potential use cases:

\subsection{Social Network Analysis}
This dataset provides a unique opportunity to study user interactions, group dynamics, and information dissemination patterns within a controlled environment. Applications include:
\begin{itemize}
    \item Examining the spread of trends, topics, and opinions over time.
    \item Analyzing patterns of collaboration and conflict resolution in online communities.
\end{itemize}

\subsection{Behavioral and Sentiment Analysis}
With anonymized message content and metadata, the dataset enables researchers to:
\begin{itemize}
    \item Detect behavioral trends and shifts in online activity.
    \item Perform sentiment analysis to measure emotional responses to events or announcements.
    \item Study the effects of moderation policies on user behavior.
\end{itemize}

\subsection{Machine Learning and Natural Language Processing (NLP)}
This dataset offers a foundation for developing and testing algorithms in machine learning and NLP:
\begin{itemize}
    \item Training chatbots and virtual assistants for customer service or educational purposes.
    \item Developing algorithms to detect toxic behavior or policy violations in online platforms.
    \item Testing models for summarization, translation, or sentiment classification on real-world data.
\end{itemize}

\subsection{Platform Policy Evaluation and Development}
The anonymized data can guide platform administrators and policymakers by:
\begin{itemize}
    \item Evaluating the effectiveness of moderation techniques in reducing harmful behavior.
    \item Informing the design of inclusive and user-friendly community guidelines.
    \item Assessing the long-term impact of platform updates and feature rollouts.
\end{itemize}

\subsection{Educational and Training Purposes}
The dataset, with sensitive information removed, can be used in academic settings to:
\begin{itemize}
    \item Train students in data science, machine learning, or social science methods.
    \item Demonstrate ethical considerations in data collection and processing.
    \item Offer a controlled environment for testing hypotheses in online communication studies.
\end{itemize}

\subsection{Ethics and Privacy Research}
Finally, this dataset itself can serve as a case study in ethical anonymization practices:
\begin{itemize}
    \item Exploring novel anonymization techniques and evaluating their effectiveness.
    \item Studying the trade-off between data utility and user privacy in real-world scenarios.
\end{itemize}

\fi







\section{Conclusion}
In this work, we propose a simple yet effective approach, called SMILE, for graph few-shot learning with fewer tasks. Specifically, we introduce a novel dual-level mixup strategy, including within-task and across-task mixup, for enriching the diversity of nodes within each task and the diversity of tasks. Also, we incorporate the degree-based prior information to learn expressive node embeddings. Theoretically, we prove that SMILE effectively enhances the model's generalization performance. Empirically, we conduct extensive experiments on multiple benchmarks and the results suggest that SMILE significantly outperforms other baselines, including both in-domain and cross-domain few-shot settings.
\section{Acknowledgments}

This work was partially funded by CNPq, CAPES, FAPEMIG, and IAIA - INCT on AI.

\bibliography{main}


\subsection{Paper Checklist}

\begin{enumerate}

\item For most authors...
\begin{enumerate}
    \item  Would answering this research question advance science without violating social contracts, such as violating privacy norms, perpetuating unfair profiling, exacerbating the socio-economic divide, or implying disrespect to societies or cultures?
    \answerYes{Yes}
  \item Do your main claims in the abstract and introduction accurately reflect the paper's contributions and scope?
    \answerYes{Yes}
   \item Do you clarify how the proposed methodological approach is appropriate for the claims made? 
    \answerYes{Yes}
   \item Do you clarify what are possible artifacts in the data used, given population-specific distributions?
    \answerYes{Yes}
  \item Did you describe the limitations of your work?
    \answerYes{Yes}
  \item Did you discuss any potential negative societal impacts of your work?
    \answerYes{Yes}
      \item Did you discuss any potential misuse of your work?
    \answerYes{Yes}
    \item Did you describe steps taken to prevent or mitigate potential negative outcomes of the research, such as data and model documentation, data anonymization, responsible release, access control, and the reproducibility of findings?
    \answerYes{Yes}
  \item Have you read the ethics review guidelines and ensured that your paper conforms to them?
    \answerYes{Yes}
\end{enumerate}

\item Additionally, if your study involves hypotheses testing...
\begin{enumerate}
  \item Did you clearly state the assumptions underlying all theoretical results?
    \answerNA{NA}
  \item Have you provided justifications for all theoretical results?
    \answerNA{NA}
  \item Did you discuss competing hypotheses or theories that might challenge or complement your theoretical results?
    \answerNA{NA}
  \item Have you considered alternative mechanisms or explanations that might account for the same outcomes observed in your study?
    \answerNA{NA}
  \item Did you address potential biases or limitations in your theoretical framework?
    \answerNA{NA}
  \item Have you related your theoretical results to the existing literature in social science?
    \answerNA{NA}
  \item Did you discuss the implications of your theoretical results for policy, practice, or further research in the social science domain?
    \answerNA{NA}
\end{enumerate}

\item Additionally, if you are including theoretical proofs...
\begin{enumerate}
  \item Did you state the full set of assumptions of all theoretical results?
    \answerNA{NA}
	\item Did you include complete proofs of all theoretical results?
    \answerNA{NA}
\end{enumerate}

\item Additionally, if you ran machine learning experiments...
\begin{enumerate}
  \item Did you include the code, data, and instructions needed to reproduce the main experimental results (either in the supplemental material or as a URL)?
    \answerNA{NA}
  \item Did you specify all the training details (e.g., data splits, hyperparameters, how they were chosen)?
    \answerNA{NA}
     \item Did you report error bars (e.g., with respect to the random seed after running experiments multiple times)?
    \answerNA{NA}
	\item Did you include the total amount of compute and the type of resources used (e.g., type of GPUs, internal cluster, or cloud provider)?
    \answerNA{NA}
     \item Do you justify how the proposed evaluation is sufficient and appropriate to the claims made? 
    \answerNA{NA}
     \item Do you discuss what is ``the cost`` of misclassification and fault (in)tolerance?
    \answerNA{NA}
  
\end{enumerate}

\item Additionally, if you are using existing assets (e.g., code, data, models) or curating/releasing new assets, \textbf{without compromising anonymity}...
\begin{enumerate}
  \item If your work uses existing assets, did you cite the creators?
    \answerNA{NA}
  \item Did you mention the license of the assets?
    \answerNA{NA}
  \item Did you include any new assets in the supplemental material or as a URL?
    \answerNA{NA}
  \item Did you discuss whether and how consent was obtained from people whose data you're using/curating?
    \answerYes{Yes}
  \item Did you discuss whether the data you are using/curating contains personally identifiable information or offensive content?
    \answerYes{Yes}
\item If you are curating or releasing new datasets, did you discuss how you intend to make your datasets FAIR?
\answerYes{Yes}
\item If you are curating or releasing new datasets, did you create a Datasheet for the Dataset (see \citet{gebru2021datasheets})? 
\answerYes{Yes}
\end{enumerate}

\item Additionally, if you used crowdsourcing or conducted research with human subjects, \textbf{without compromising anonymity}...
\begin{enumerate}
  \item Did you include the full text of instructions given to participants and screenshots?
    \answerNA{NA}
  \item Did you describe any potential participant risks, with mentions of Institutional Review Board (IRB) approvals?
    \answerNA{NA}
  \item Did you include the estimated hourly wage paid to participants and the total amount spent on participant compensation?
    \answerNA{NA}
   \item Did you discuss how data is stored, shared, and deidentified?
   \answerYes{Yes} 
   
\end{enumerate}

\end{enumerate}

\onecolumn

\appendix

\section*{Appendix}

\begin{table*}[ht!]
    \caption{Fields in the Discord Server Object. These fields provide metadata for each server included in the dataset.}
    \centering
    \renewcommand{\arraystretch}{1.3} % Espaçamento vertical
    \setlength{\tabcolsep}{5pt} % Espaçamento horizontal
    \begin{tabular}{|p{3.5cm}|p{3.5cm}|p{8cm}|}
        \hline
        \textbf{Field} & \textbf{Type} & \textbf{Description} \\ 
        \hline
        slug & string & A unique identifier combining the server name and ID, used as a shorthand or URL-friendly reference. \\ 
        \hline
        id & string & A unique numerical identifier for the server. \\ 
        \hline
        name & string & The human-readable name of the server. \\ 
        \hline
        description & string & A brief description or identifier for the server. \\ 
        \hline
        icon & string & URL to the server's icon image. \\ 
        \hline
        splash & string & URL to the splash image, typically used for larger visual displays or promotional purposes. \\ 
        \hline
        banner & string & Identifier for the server's banner image, used for branding or visual purposes. \\ 
        \hline
        approximate\_presence\_ count & number & The estimated number of currently active users on the server. \\ 
        \hline
        approximate\_member\_ count & number & The estimated total number of members in the server. \\ 
        \hline
        premium\_subscription\_ count & number & The number of members with premium subscriptions (e.g., Discord Nitro) on the server. \\ 
        \hline
        preferred\_locale & string & The server's preferred language, represented by its locale code. \\ 
        \hline
        auto\_removed & boolean & Indicates whether the server was automatically removed from a directory or platform. \\ 
        \hline
        discovery\_splash & string & An identifier for the discovery splash image, used for promoting the server in public directories. \\ 
        \hline
        primary\_category\_id & number & ID representing the primary category or classification of the server. \\ 
        \hline
        vanity\_url\_code & string & Custom short URL code for the server. \\ 
        \hline
        is\_published & boolean & Indicates whether the server is published and discoverable in public directories. \\ 
        \hline
        keywords & array of strings & A list of keywords associated with the server for discoverability. \\ 
        \hline
        features & array of strings & A list of special features enabled for the server, providing enhanced functionality or customization. \\ 
        \hline
        created\_at & string (ISO 8601) & The timestamp indicating when the server was created. \\ 
        \hline
        reasons\_to\_join & array of strings & A list of promotional highlights or unique selling points of the server. \\ 
        \hline
        social\_links & array of strings & Links to external social media or community pages associated with the server. \\ 
        \hline
        about & string & A detailed description of the server's purpose, history, and features. \\ 
        \hline
        category\_ids & array of numbers & A list of IDs representing the categories the server is associated with. \\ 
        \hline
    \end{tabular}
    \label{tab:discord_server_fields}
\end{table*}

\begin{table*}[ht!]
    \caption{Fields in the Discord Message Object. Fields marked with '?' are optional. References [1], [2], etc., refer to detailed specifications in the Discord API documentation.}
    \centering
    \renewcommand{\arraystretch}{1.3
    } % Espaçamento vertical
    \setlength{\tabcolsep}{5pt} % Espaçamento horizontal
    \begin{tabular}{|p{3.5cm}|p{3.5cm}|p{8cm}|}
        \hline
        \textbf{Field} & \textbf{Type} & \textbf{Description} \\ 
        \hline
        id & snowflake & ID of the message \\ 
        \hline
        channel\_id & snowflake & ID of the channel the message was sent in \\ 
        \hline
        author & user object & The author of this message (not guaranteed to be a valid user, see below) \\ 
        \hline
        content & string & Contents of the message \\ 
        \hline
        timestamp & ISO8601 timestamp & When this message was sent \\ 
        \hline
        edited\_timestamp & ?ISO8601 timestamp & When this message was edited (or null if never) \\ 
        \hline
        tts & boolean & Whether this was a TTS message \\ 
        \hline
        mention\_everyone & boolean & Whether this message mentions everyone \\ 
        \hline
        mentions & array of user objects & Users specifically mentioned in the message \\ 
        \hline
        mention\_roles & array of role object IDs & Roles specifically mentioned in this message \\ 
        \hline
        mention\_channels? & array of channel mention objects & Channels specifically mentioned in this message \\ 
        \hline
        attachments & array of attachment objects & Any attached files \\ 
        \hline
        embeds & array of embed objects & Any embedded content \\ 
        \hline
        reactions? & array of reaction objects & Reactions to the message \\ 
        \hline
        nonce? & integer or string & Used for validating a message was sent \\ 
        \hline
        pinned & boolean & Whether this message is pinned \\ 
        \hline
        webhook\_id? & snowflake & If the message is generated by a webhook, this is the webhook's ID \\ 
        \hline
        type & integer & Type of message \\ 
        \hline
        activity? & message activity object & Sent with Rich Presence-related chat embeds \\ 
        \hline
        application? & partial application object & Sent with Rich Presence-related chat embeds \\ 
        \hline
        application\_id? & snowflake & If the message is an Interaction or application-owned webhook, this is the ID of the application \\ 
        \hline
        flags? & integer & Message flags combined as a bitfield \\ 
        \hline
        message\_reference? & message reference object & Data showing the source of a crosspost, channel follow add, pin, or reply message \\ 
        \hline
        referenced\_message? [4] & ?message object & The message associated with the message\_reference \\ 
        \hline
        thread? & channel object & The thread that was started from this message, includes thread member object \\ 
        \hline
        components? & array of message components & Sent if the message contains components like buttons, action rows, or other interactive components \\ 
        \hline
        sticker\_items? & array of message sticker item objects & Sent if the message contains stickers \\ 
        \hline
        stickers? & array of sticker objects & Deprecated: the stickers sent with the message \\ 
        \hline
        position? & integer & A generally increasing integer (there may be gaps or duplicates) that represents the approximate position of the message in a thread \\ 
        \hline
        role\_subscription\_data? & role subscription data object & Data of the role subscription purchase or renewal that prompted this ROLE\_SUBSCRIPTION\_PURCHASE message \\ 
        \hline
        resolved? & resolved data & Data for users, members, channels, and roles in the message's auto-populated select menus \\ 
        \hline
        poll? [2] & poll object & A poll! \\ 
        \hline
        call? & message call object & The call associated with the message \\ 
        \hline
    \end{tabular}
    \label{tab:discord_message_fields}
\end{table*}

\twocolumn


\end{document}
