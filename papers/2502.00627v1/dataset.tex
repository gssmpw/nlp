\section{Dataset Construction}

\begin{figure*} [!ht]
    \centering
    \includegraphics[width=1\linewidth]{Diagrama2.pdf}
    \caption{Diagram showcasing each step of the data collection process.}
    \label{fig:diagrama}
\end{figure*}

The dataset was built in three main phases: data collection, anonymization and organization, as detailed in this section.

%\subsection{Data Collection Process}
%\label{data-collection}

% To collect data, we first used Discord's Discovery feature, which allows users to browse public servers within the app, i.e., any user can view the server content without joining it. From this feature, we gathered all available server IDs, names, and descriptions of public servers with at least 1,000 members. 
% These server IDs served as parameters for accessing data through the Discord API. With this process, a total of 31,128 unique servers were saved. We collected a random sample of approximately \textbf{10\% of all global servers listed in Discord's  \textit{Discovery tab}} from a snapshot made in November 2024. During the sampling, the platform reported a total of 30,093 servers available through this feature. From this population, we collected all accessible text messages from \textbf{3,113 servers}.\gi{pq em um lugar aparece 31,128 e no outro 30,093?}

% Using the API, we retrieved publicly accessible channels and their associated messages. The data collection process spanned \caio{conferir datas}  from December 1, 2024, to January 12, 2025.
% We used Python alongside the official Discord API library to automate the collection process, managing rate limits and ensuring adherence to Discord's API terms of service. Ethical safeguards were implemented to ensure only public data was collected. Data was also anonymized to protect user privacy.

% The decision to limit the sample to 10\% was driven by storage and computational constraints, as well as the need to adhere to Discord's API Guidelines\footnote{\url{https://support-dev.discord.com/hc/en-us/categories/360000656491-Developer-Policies-Terms}}. Collecting data from the entire population of servers would require significantly more resources and time. Nonetheless, we believe this sample size is sufficient to provide a representative and insightful characterization of the community.


\subsection{Data Collection}
\label{data-collection}
To access data from Discord servers, we first used Discord's Discovery feature, which allows users to browse public servers, and even see the public messages, without joining them. By querying the URL \textit{https://discord.com/servers?query=\{query\}}, where \textit{\{query\}} represents all possible combinations of letters and numbers of length up to four (which was the smallest length sufficient to retrieve all listed servers)
we systematically gathered server IDs, names, descriptions, and further metadata (e.g., server categories, keywords, and member count) for all servers available in Discovery. All of the metadata collected from each server is described in detail in the Appendix. Using this approach, we identified a total of 31,673 servers available in Discovery as of November 17, 2024.

% To access data from Discord servers, we first utilized the platform's Discovery feature, which allows users to explore public servers without needing to join them, to collect server IDs. Using this feature, we repeatedly queried the URL \textit{https://discord.com/servers?query=\textless query\textgreater}, where \textit{\textless query\textgreater} represents all possible combinations of letters and numbers of length four. This process allowed us to gather server IDs, names, descriptions, and additional metadata(e.g., server categories, keywords, and member count) for all public servers with at least 1,000 members. The specific metadata attributes collected for each server are detailed in Appendix \ref{tal}. Using this approach, we identified and recorded a total of 31,673 servers available in Discovery as of November 17, 2024.

Given the scale of the collected data, we implemented a random sampling strategy, selecting 10\% of all servers listed in Discord's Discovery tab. This sample, consisting of 3,167 servers, was chosen to balance feasibility with representativeness. The decision to limit the scope was motivated by computational and storage constraints, as well as adherence to Discord's API guidelines. Despite these limitations, we believe the sample size is sufficient to provide a meaningful characterization of the communities in Discord.

Using the server IDs from the sampled servers, we initiated the data collection process via the Discord API. First, we retrieved the publicly accessible channels for each server using the endpoint \textit{https://discord.com/api/v9/guilds/ \{server\_id\}/channels}. For each identified channel, we accessed all associated text messages by querying \textit{https://discord.com/api/v9/channels/\{channel\_id\}/messages}. The collection spanned from December 17, 2024, to January 2, 2025, ensuring that the dataset captured recent server activity during this period.

The complete workflow for data collection is illustrated in Figure \ref{fig:diagrama}. Further details on the attributes of the collected data are provided in the Data Instance Description section and in the Appendix.

\subsection{Anonymization Process}

Working with sensitive data that involves individuals requires addressing several ethical concerns. Even when data originate exclusively from public servers and messages, ensuring user anonymity remains a critical priority. This process aligns with principles such as GDPR and other privacy regulations, aiming to safeguard individual's identities while maintaining the utility of the dataset.

To address these concerns, our solution employs anonymization techniques to obscure sensitive information while preserving the dataset's analytical utility. Usernames are replaced with consistent pseudonyms generated by the \textit{mimesis library}, ensuring that identifiers remain unique and contextually meaningful across records. Similarly, user IDs and message IDs are hashed using the SHA-256 algorithm and truncated to 12 characters. This deterministic hashing approach maintains linkage between related records while effectively masking the original identifiers. The \textit{global\_name} field, deemed unnecessary for analysis, is entirely removed. Additionally, user IDs embedded within the content field are identified via regular expressions and replaced with their corresponding hash values. 

While the methods used significantly reduce the risk of reidentification, no anonymization process can guarantee absolute anonymity, particularly in datasets with rich contextual information. By sharing the detailed methodology, we aim to encourage transparency, enable validation, and promote ethical data usage. This anonymization process strikes a balance between preserving user privacy and maintaining the analytical value of the data set, ensuring that it can be safely used in future research.

\subsection{Dataset Organization}

The dataset was organized into individual \textbf{JSON files}, where each file corresponds to a specific server. The files are named after the servers ID's, ensuring reliability and consistency, and avoiding ambiguity or possible errors due to special characters.

Within each file, messages are stored in a format that prioritizes both channel organization and chronological order. Messages from the same channel are grouped, and within each channel, ordered from the most recent to the oldest.

This organization was designed to facilitate the analysis of temporal and channel-specific communication dynamics across servers. The details of the content within each message are described next.


\subsection{Data Description}

We opted to collect every single \textit{``Field"} made available by Discord's API. Each message is treated as an object with several attributes on our JSON files, and we also added a lone file to the dataset named ``servers\_metadata", which aggregates server related information from every server we had a message collected from. 

In this section, we introduce the structure and some details on those data instances to promote better interpretability.

\subsubsection{Message Object}

The main attributes of the message object are those commonly available in most chat-related datasets, such as: \textit{Message Content, Author (User ID, Username), Channel (Name, ID), Timestamp}.

However, many more sophisticated fields give access to other insightful information from each message, revealing aspects such as user behaviour or interactions. Examples include: \textit{``isBot"} indicates whether a message was sent from a bot or a personal account, \textit{``reactions?"} provides an array of reactions, a feature from the platform that allows users to interact with a particular message, \textit{``pinned"} reveals if a message was pinned or not, and \textit{``tts"} whether the message was transcribed through   `` text to speech" technology. All fields are listed in Table \ref{tab:discord_message_fields} in the Appendix. 

Messages are also classified according to the categories \textit{Message Types} and \textit{Flags}, including \textit{``DEFAULT"}, \textit{``REPLY"}, \textit{``USER\_JOIN"}, \textit{``IS\_CROSSPOST"} and many more. 


\subsubsection{Server Metadata File}

Each server is also represented as an object with several attributes, capturing a wide range of metadata provided by Discord's Discovery Feature.

The main attributes include those essential for server identification and activity analysis, such as \textit{Server ID, Name, Description, Icon URL, Member Count}. Additional attributes allow for a deeper exploration of server characteristics and functionalities. For example, \textit{``slug"} provides a unique, URL-friendly identifier for the server, \textit{``preferred\_locale"} specifies the server's primary language, and \textit{``features"} lists special functionalities enabled for the server. Furthermore, fields like \textit{``reasons\_to\_join"} and \textit{``social\_links"} highlight promotional aspects and external connections of the server. 

Servers also include activity-related attributes such as \textit{``approximate\_presence\_count"} (the estimated number of currently active users) and \textit{``premium\_subscription\_count"} (indicating the number of premium subscriptions within the server). 
A complete list of fields and their descriptions can be found in Table \ref{tab:discord_server_fields} in the Appendix.

\subsubsection{Documentation} 

We highly encourage anyone interested in working with the dataset to go further and access Discord's official Documentation\footnote{https://discord.com/developers/docs/resources/message}, which provides more detailed information on the fields and attributes, ensuring a clearer understanding of the data structure and enabling more effective and thoughtful usage of the dataset for various purposes.

\subsection{Ethical Concerns}

Throughout every step of our data collection process, we prioritized adherence to ethical standards. Precautions were taken to collect data responsibly. All data was sourced from groups that are explicitly considered public according to Discord's terms of use, which every user agrees to upon signing up. The data was anonymized, and the methodology was detailed to promote reproducibility and transparency. Additionally, the data collection methods align with practices described in the current literature \cite{webmedia, Singh_Ghafouri_Such_Suarez-Tangil_2024, hateful_messages_adolescents}.

%All collected groups had at least 1,000 members and were accessible via Discord's ``Discovery" feature, which allows any user to view server content without joining. This ensures that the servers are unequivocally public. Furthermore, no interactions were made during the data collection process. We did not engage with any server or their members. Instead, Discord's API and server IDs were used to collect the data passively and non-intrusively.

Finally, the dataset released in this study adheres to the FAIR guiding principles for scientific data management and stewardship:

\textbf{Findable:} A unique and persistent Digital Object Identifier (DOI) is assigned to our dataset, ensuring it can be easily located and cited.

\textbf{Accessible:} The dataset is openly available for download, subject to the appropriate terms of use.

\textbf{Interoperable:} The dataset is released in JSON format, ensuring compatibility with a wide range of programming languages and operating systems.

\textbf{Reusable:} A detailed schema is provided in the Data Description section, clearly defining each field's purpose and facilitating reuse across diverse applications. 
