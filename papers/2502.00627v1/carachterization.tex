\section{Dataset Characterization}

\begin{figure*} [t]
    \centering
    \includegraphics[width=1\linewidth]{mensagens_por_dia_comparativo.pdf}
    \caption{Evolution of the daily number of messages sent over time. The top panel presents the complete time series, distinguishing between messages sent by regular users (in blue) and bots (in red). The bottom-left panel focuses on the initial period, covering up to October 2022, while the bottom-right panel highlights the most recent period, from June 2022 to January 2024. Notice the graph scales are different.}
    \label{fig:messages_per_time}
\end{figure*}

% \gi{sugestao: fazer dos proximos paragrafos o inicio da secao de caracterizacao.}

The \textbf{3,167 collected servers} yielded a total of \textbf{2,052,206,308 unique messages} sent by \textbf{4,735,057 distinct users}. From the total number of messages,   \textbf{364,447,569 (17\%) originated from bots}. The data spans from Discord's launch, on \textbf{May 13, 2015}, to \textbf{December 17, 2024}, when the data collection process began.

Figure \ref{fig:messages_per_time} shows the number of messages over time. 
Note that Discord's retroactive approach to chat allows the dataset to encompass a timeframe broader than the actual data collection period.
Upon joining a server, users gain access to all non-deleted historical content within public channels, and the same is valid for data retrieval using their API.
Notably, 2024 stands out as the most active year across all servers, reflecting a growing network that offers ample opportunities for further exploration.

Although it is interesting to be able to access older data, it is important to note that the servers listed in the \textit{Discovery} tab most likely represent those active at the time of data collection. Servers featured in this tab typically emphasize ongoing engagement and relevance. Additionally, while Discord provides access to historical content, some servers may periodically delete older messages, which could introduce variability in the completeness of the temporal data available. These factors may influence the representativeness of older content and should be considered when analysing the dataset.

Next, we present simple but relevant analyses of the dataset. The first aspect we examined is \textbf{Bots.} As previously mentioned, Discord bots play a crucial role in enhancing user experience by automating tasks, facilitating engagement, and offering specialized functionalities. They serve as virtual assistants that can perform various tasks, such as moderation, entertainment, and utility services. Table \ref{tab:top10bots} illustrates the diverse applications of bots, highlighting their role in generating messages, being mentioned by users, and eliciting reactions, which are key indicators of their impact and utility in Discord.

Bots like \textit{MEE6} and \textit{Dyno} are widely recognized as powerful moderators, enabling server administrators to enforce rules, assign roles, and monitor activity effectively. These bots are essential for maintaining order in larger communities where manual moderation would be impractical. Similarly, \textit{Arcane Premium} and \textit{Loritta} excel in general server management, providing features such as levelling systems, customizable commands, and automated event handling to enhance user engagement and server functionality. On the other hand, entertainment-focused bots like \textit{Pokétwo}, \textit{Mudae}, \textit{Karuta} and \textit{OwO} captivate users with gaming and collectable experiences, encouraging interaction through game mechanics such as Pokémon battles, waifu collections, trading cards and interacting with fictional animals. These bots create an environment where users actively participate, forming micro-communities within the larger server ecosystem. Finally, \textit{Lucky VR}, despite being the bot with the highest volume of messages, lacks extensive documentation online. Its primary focus is on gambling in virtual reality, particularly in facilitating poker games.

\textbf{Languages.} To identify the most frequent languages in our dataset, we analysed the values present in the \textit{preferred locale} field within the server metadata. Figure \ref{fig:hist-linguas} shows English (US) as the primary language for most servers, with 1,705 servers. Disregarding the ``unknown" field, the second most frequent language is Spanish (Spain), with 144 servers, followed by French, with 136 servers. Also note there is a notable linguistic diversity, including Portuguese, Russian, and German.

\begin{figure}[t]
    \centering
    \includegraphics[width=1\linewidth]{language_histogram_avancado.pdf}
    \caption{Bar plot of the number of servers by language, with the majority of servers being in English.}
    \label{fig:hist-linguas}
\end{figure}


% For instance, bots like MEE6 and Dyno are widely used for community management, handling tasks like assigning roles, enforcing rules, and moderating content. On the other hand, entertainment bots such as Mudae, Karuta, and Pokétwo engage users through games, collectibles, and interactive features, exhibiting higher levels of participation and interaction. Additionally, niche bots like AniLibria.TV provide specialized services, such as streaming updates or content delivery for targeted audiences. The bots listed in Table \ref{tab:top10bots} illustrate the diverse applications of Discord bots, highlighting their role in generating messages, being mentioned by users, and eliciting reactions, which are key indicators of their impact and utility in various online communities.

\begin{table*}[t]
\caption{Top 10 Bots by Messages, Mentions, and Reactions.}
\centering
\label{tab:top10bots}
\begin{tabular}{ccc}
\toprule
\textbf{Messages} & \textbf{Mentions} & \textbf{Reactions} \\
\midrule
\hline
\begin{tabular}{lr}
Lucky VR & 3,684,600 \\
Mudae & 2,955,892 \\
AniLibria.TV & 2,844,343 \\
Pok\'etwo & 2,005,250 \\
RoM & 1,716,654 \\
OwO & 1,523,404 \\
Mimu & 1,430,137 \\
Wan Shi Tong & 1,297,316 \\
BTE France Minecraft & 1,077,553 \\
MEE6 & 904,432 \\
\end{tabular} &
\begin{tabular}{lr}
Pok\'etwo & 304,657 \\
Dyno & 290,387 \\
AniLibria.TV & 290,286 \\
ProBot  & 278,653 \\
Loritta & 251,208 \\
Arcane Premium & 224,681 \\
Mimu & 206,157 \\
MEE6 & 198,534 \\
Dank Memer & 178,813 \\
Liquid Esports & 134,816 \\
\end{tabular} &
\begin{tabular}{lr}
Karuta & 984,693 \\
Mudae & 281,929 \\
MEE6 & 237,370 \\
XenosPD.dk & 217,634 \\
Bongo & 216,779 \\
OwO & 193,493 \\
YAGPDB.xyz & 137,052 \\
Dyno & 131,881 \\
Suggester & 88,981 \\
Emps-World & 85,889 \\
\end{tabular} \\
\bottomrule
\hline
\end{tabular}
\end{table*}

\textbf{Servers.} Discord, as a platform for community interactions, supports a wide range of themes that reflect different user interests. To analyze the thematic distribution of servers, we examined server metadata, focusing on keywords that describe their content. These keywords also improve the platform's search functionality through the Discovery feature.

Table \ref{tab:top_40_keywords} presents the 40 most frequent keywords appearing in the description field of the servers collected. Despite Discord's evolution into a versatile platform, gaming-related content remains significant, with \textit{gaming} appearing in over 15\% of descriptions. Related keywords like \textit{minecraft}, \textit{roblox}, and \textit{twitch} reinforce its gaming's central role. Other prominent themes include \textit{anime} (5.68\%), \textit{roleplay} (6.44\%), and \textit{fivem} (4.39\%), reflecting niche communities and interactive multiplayer experiences. 

Emerging interests like \textit{esports} (4.04\%) and \textit{social} (4.14\%) highlight the importance of competitive gaming and socialization. Creative and recreational themes, such as \textit{art} and \textit{music}, and educational discussions, such as \textit{programming}, demonstrate the platform's versatility beyond gaming.

\begin{table}[t]
\caption{Top 40 most frequent keywords in Discord servers description and their coverage.}
\centering
\begin{tabular}{lr|lr}
%\hline
%\multicolumn{4}{c}{\textbf{Keywords}} \\
\hline
\textbf{Keyword} & \textbf{\%} & \textbf{Keyword} & \textbf{\%} \\
\hline
gaming & 15.28 & giveaways & 2.31 \\
youtube & 15.00 & art & 2.02 \\
minecraft & 12.28 & rp & 2.02 \\
roblox & 10.83 & tiktok & 1.89 \\
twitch & 8.68 & manga & 1.83 \\
community & 8.65 & chill & 1.74 \\
roleplay & 6.44 & leagueoflegends & 1.71 \\
anime & 5.68 & rpg & 1.71 \\
fivem & 4.39 & pvp & 1.58 \\
social & 4.14 & music & 1.48 \\
esports & 4.04 & server & 1.39 \\
valorant & 3.79 & chatting & 1.36 \\
memes & 2.97 & survival & 1.36 \\
fortnite & 2.78 & gta5 & 1.33 \\
streamer & 2.59 & modding & 1.20 \\
events & 2.53 & mobile & 1.20 \\
game & 2.49 & gtav & 1.20 \\
fun & 2.46 & programming & 1.17 \\
gta & 2.43 & pc & 1.17 \\
games & 2.43 & csgo & 1.14 \\
\hline
\label{tab:top_40_keywords}
\end{tabular}
\end{table}

However, note that these keywords are only related to the server's descriptions. The context of messages goes beyond these subjects, with messages concerning topics that include mental health, political debates and web dating, for example. \looseness=-1