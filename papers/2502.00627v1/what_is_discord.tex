
\section{What is Discord?}

Discord is a communication platform launched in 2015 by Discord Inc., designed to enable real-time interaction through text, voice, and video. Initially popular within gaming communities, Discord has expanded to accommodate diverse user groups, including educators, hobbyists, and professional teams\footnote{https://discord.com/safety/360044149331-what-is-discord}. Similar to platforms like Telegram and WhatsApp, Discord allows users to create and join customizable group spaces for synchronous and asynchronous communication, which can be either public or private.


Discord is organized into ``servers", virtual spaces designed for communities to connect, share content, and engage in conversations. Each server is divided into ``channels," which are dedicated spaces for specific types of conversations or activities. Text channels are used for written discussions, enabling members to exchange messages, share links, images, and other media. Voice channels, on the other hand, facilitate real-time audio conversations, often used for meetings, gaming sessions, or casual chats, and can also support video and screen sharing. All of these dynamics can be seen in Figure \ref{fig:disc}. Additionally, servers can be customized with ``roles" which are permission-based assignments that define what actions members can perform within the server. Roles can grant or restrict access to specific channels, allow users to manage messages, ban or mute members, or even adjust server settings. Combined with bots to automate tasks and server-specific rules to guide member behavior, this flexible architecture supports a wide range of communities, from small friend groups to large public forums.


A distinguishing feature of Discord is its user-driven moderation. While traditional social networks such as Twitter, Facebook, or YouTube have historically relied on centralized moderation managed by the platform itself \cite{wilson2020hate}, there is a very recent trend toward decentralizing moderation, exemplified by initiatives like community-based fact-checking \cite{balasubramanian2024publicdatasettrackingsocial}. Discord has always explicitly delegated the responsibility of rule-setting, behavior management, and access control to server administrators and moderators. This long-standing approach has fostered a well-established environment of non-centralized moderation, offering valuable insights into the dynamics and challenges of this model as other platforms begin to adopt similar strategies. For instance, as shown by \cite{moderation-challenges}, this flexibility can also be exploited to facilitate the presence of extremist and hateful groups and networks.

Bots are another key element of Discord’s ecosystem, setting it apart from other social networks. Unlike bots on other platforms, which are often limited to automated content posting or analytics, Discord bots are highly customizable and play an interactive role \cite{moderation-discord-bots, bots-discord}. They can automate moderation tasks, provide community engagement features (e.g., games, polls, or reminders), and integrate external services like music streaming, analytics tools, and generative AI services like Midjourney and ChatGPT. Discord actively encourages the creation and integration of bots by providing developers with extensive API documentation, support, and a bot-friendly environment, fostering innovation and customization within the platform \footnote{https://discord.com/developers/docs/intro}. Their extensive use has become central to how users interact within servers and how these communities are structured and managed.

To facilitate the discovery of public servers, Discord introduced the \textit{Server Discovery} feature. This feature allows users to search, explore and join public servers that align with their interests\footnote{https://support.discord.com/hc/en-us/articles/360023968311-Server-Discovery}. Servers listed in the Discovery tab must adhere to specific guidelines, including maintaining a welcoming environment, avoiding graphic or sexual content, and having an accurate server name and description\footnote{https://support.discord.com/hc/en-us/articles/4409308485271-Discovery-Guidelines}. These servers are the focus of our research, as they offer a curated look at communities
built around a wide variety of topics, each with at least 1000 members.
%with public servers being discoverable through Discord’s Discovery feature, provided they meet specific guidelines regarding content and moderation. 
% This work focuses exclusively on public Discord servers available in the Discovery feature
 %, from gaming and technology to education and entertainment.

%\yan{add um ultimo paragrafo?}