\section{Related Work}
% \pr{talvez o related work esteja grande, se nao tiver espaço no final, da pra tirar daqui}

% \subsection{Datasets in Social Media Research}

The availability of comprehensive datasets has underpinned advancements in computational social science, allowing researchers to explore user behavior, discourse, and network structures. Initially, research was strongly based on Twitter and Facebook data, given its initial openness to data access \cite{dooms2013movietweetings,lewis2008tastes}. After these platforms restricted access to most of their data, other platforms such as Reddit and WhatsApp became the focus of online social research. 

The Pushshift Reddit Dataset \cite{baumgartner2020pushshiftredditdataset}, for example, provides a detailed archive of Reddit posts and comments, facilitating studies of public discourse and online community dynamics. Similarly, GDELT \cite{Leetaru13gdelt:global} tracks global news events, offering tools for studying media trends, geopolitical developments, and event-based analyses. Also, WhatsApp datasets \cite{Seufert2023} have facilitated analyses of private communication in mobile-based social environments, contributing to the understanding of interpersonal dynamics.

Efforts have also focused on datasets from niche platforms, which provide unique insights into specific communities and communication structures. For instance, datasets from Twitter have been pivotal during specific events, such as the COVID-19 pandemic, enabling studies on information dissemination, public sentiment, and the spread of misinformation \cite{https://doi.org/10.5281/zenodo.7834392}. Similarly, recent datasets from Telegram public channels \cite{https://doi.org/10.5281/zenodo.7640712} and Koo posts, comments and profiles \cite{koo_Mekacher_Falkenberg_Baronchelli_2024} offer valuable perspectives on large-scale group interactions and content sharing. 

Social and news media datasets have further highlighted the potential for leveraging large-scale textual and visual data for societal impact studies. For instance, drought-related datasets \cite{Shang_Chen_Vora_Zhang_Cai_Wang_2024} integrate social media and news media to analyse environmental and socioeconomic effects. Similarly, datasets collected during the Russia-Ukraine conflict demonstrate the role of social media in information warfare and propaganda dissemination \cite{Ai_Gupta_Oak_Hui_Liu_Hirschberg_2024}. These datasets illustrate how social media data can be harnessed to address global and societal challenges.

Despite the breadth of available datasets, platforms like Discord remain underexplored. Discord’s focus on semi-private, community-oriented communication presents a distinct research opportunity. Unlike Twitter or Reddit, where public or semi-public interactions dominate, Discord facilitates structured, real-time conversations within server-based communities.

While prior works, such as \citet{Singh_Ghafouri_Such_Suarez-Tangil_2024} and \citet{10.1007/978-3-031-42171-6_5}, have introduced datasets of Discord messages, our work significantly expands the scope. We present the first general-purpose dataset, offering a far broader and more representative foundation for research. Building on a similar methodology, our dataset provides a more general and holistic perspective, addressing a critical gap in the research landscape. It is designed to enable studies of digital communication, community governance, moderation practices, cross-platform comparisons, and much more, thus unlocking new avenues for social media research.

% \subsection{Models and Methods for Leveraging Social Media Datasets}

% \gi{eu removeria essa secao ou  moveria pro fim do paper, onde vcs falam de potential applications. ou simplesmente citaria esses papers na secao acima.}

% The development and application of robust analytical models are essential for extracting meaningful insights from social media datasets. Traditional methods such as natural language processing (NLP) and network analysis have been extensively applied to tasks like sentiment analysis, topic modeling, and community structure exploration \cite{devarajan2023ai, cheong2011social}. These methods provide foundational tools for examining text-based communication and relational dynamics within online networks.

% The advent of deep learning has significantly expanded the analytical capabilities available to researchers. Transformer-based models like BERT \cite{devlin2019bertpretrainingdeepbidirectional} and GPT \cite{brown2020languagemodelsfewshotlearners} have set benchmarks in text classification, language generation, and intent prediction. Similarly, graph neural networks (GNNs) have emerged as powerful tools for modeling the complex relational data inherent in social networks, enabling deeper analysis of community behavior, influence propagation, and interaction patterns \cite{han2020graph,hamid2020fake}.

% More recently, multimodal and hybrid approaches have been applied to integrate textual, temporal, and visual data, reflecting the increasing complexity of social media content. For instance, models such as CLIP and ALIGN \cite{jia2021scalingvisualvisionlanguagerepresentation} have demonstrated the ability to align multimodal inputs for content analysis. Prompt engineering has also gained prominence as a means to guide large language models (LLMs) for specific social media tasks, such as detecting misinformation and analyzing user-generated content \cite{wang2024wordflow}.

% Discord’s unique characteristics—threaded conversations, multi-layered server structures, and real-time dynamics—require specialized approaches that integrate NLP for text analysis, GNNs for community modeling, and temporal methods to capture longitudinal changes. Tailored models can unlock insights into governance structures, moderation practices, and the evolution of digital communities, making Discord an exciting frontier for social media research.