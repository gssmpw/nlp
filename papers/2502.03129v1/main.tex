% NAACL 2025 instructions:
% https://aclrollingreview.org/cfp#long-papers
%Long papers must describe substantial, original, completed and unpublished work. Wherever appropriate, concrete evaluation and analysis should be included. Long papers may consist of:
%up to eight (8) pages of content
%plus up to one page for limitations (required, see below) and optionally ethical considerations
%plus unlimited pages of references
%Submissions that exceed the length requirements, or are missing a limitations section, will be desk rejected.
% Final versions of accepted papers will be given one additional page of content (up to 9 pages for long papers, up to 5 pages for short papers) to address reviewers’ comments.

% This must be in the first 5 lines to tell arXiv to use pdfLaTeX, which is strongly recommended.
\pdfoutput=1
% In particular, the hyperref package requires pdfLaTeX in order to break URLs across lines.

\documentclass[11pt]{article}

% Change "review" to "final" to generate the final (sometimes called camera-ready) version.
% Change to "preprint" to generate a non-anonymous version with page numbers.
%\usepackage[review]{acl}
\usepackage[final]{acl}

% Standard package includes
\usepackage{times}
\usepackage{latexsym}

% For proper rendering and hyphenation of words containing Latin characters (including in bib files)
\usepackage[T1]{fontenc}
% For Vietnamese characters
% \usepackage[T5]{fontenc}
% See https://www.latex-project.org/help/documentation/encguide.pdf for other character sets

% This assumes your files are encoded as UTF8
\usepackage[utf8]{inputenc}

% This is not strictly necessary, and may be commented out,
% but it will improve the layout of the manuscript,
% and will typically save some space.
\usepackage{microtype}

% This is also not strictly necessary, and may be commented out.
% However, it will improve the aesthetics of text in
% the typewriter font.
\usepackage{inconsolata}

%Including images in your LaTeX document requires adding
%additional package(s)
\usepackage{graphicx}
\usepackage{epstopdf} 

% If the title and author information does not fit in the area allocated, uncomment the following
%
%\setlength\titlebox{<dim>}
%
% and set <dim> to something 5cm or larger.
\usepackage{float}
\usepackage{stfloats}
\usepackage{array}
\usepackage{svg}
% \usepackage[table]{xcolor}
\usepackage{xcolor}
% for \rowcolor
\usepackage{colortbl}
% for \todo
\usepackage{todonotes}

%\title{Fine-tuning Large Language Models with Chain of Thought Teaching Signals for Number-Focus Headline Generation}
%\title{Optimizing Rationales via Constructing Topic and numerical Preferences to Teach LLMs for Number-Focused Headline Generation}
%\title{Generating Topic and numerical-based Chain-of-Thought Rationales to Teach LLMs for Number-focused Headline Generation}
%\title{Generating Topic-Entity-numerical-reasoning Rationales to Teach LLMs for Number-focused Headline Generation}
% Avoid two "generate" in the above version
\title{Teaching Large Language Models Number-Focused Headline Generation With Key Element Rationales}

% Author information can be set in various styles:
% For several authors from the same institution:
% \author{Author 1 \and ... \and Author n \\
%         Address line \\ ... \\ Address line}
% if the names do not fit well on one line use
%         Author 1 \\ {\bf Author 2} \\ ... \\ {\bf Author n} \\
\author{Zhen Qian \and Xiuzhen Zhang\thanks{Corresponding author} \and Xiaofei Xu \and Feng Xia \\
         School of Computing Technologies, RMIT University, Australia \\ 
         \texttt{\{s3888611,xiaofei.xu2\}@student.rmit.edu.au;} \\
         \texttt{\{xiuzhen.zhang,feng.xia\}@rmit.edu.au}
}


% For authors from different institutions:
% \author{Author 1 \\ Address line \\  ... \\ Address line
%         \And  ... \And
%         Author n \\ Address line \\ ... \\ Address line}
% To start a separate ``row'' of authors use \AND, as in
% \author{Author 1 \\ Address line \\  ... \\ Address line
%         \AND
%         Author 2 \\ Address line \\ ... \\ Address line \And
%         Author 3 \\ Address line \\ ... \\ Address line}

% %\author{First Author \\
%   Affiliation / Address line 1 \\
%   Affiliation / Address line 2 \\
%   Affiliation / Address line 3 \\
%   \texttt{email@domain} \\\And
%   Second Author \\
%   Affiliation / Address line 1 \\
%   Affiliation / Address line 2 \\
%   Affiliation / Address line 3 \\
%   \texttt{email@domain} \\}

% \author{
%  \textbf{First Author\textsuperscript{1}},
%  \textbf{Second Author\textsuperscript{1,2}},
%  \textbf{Third T. Author\textsuperscript{1}},
%  \textbf{Fourth Author\textsuperscript{1}},
% \\
%  \textbf{Fifth Author\textsuperscript{1,2}},
%  \textbf{Sixth Author\textsuperscript{1}},
%  \textbf{Seventh Author\textsuperscript{1}},
%  \textbf{Eighth Author \textsuperscript{1,2,3,4}},
% \\
%  \textbf{Ninth Author\textsuperscript{1}},
%  \textbf{Tenth Author\textsuperscript{1}},
%  \textbf{Eleventh E. Author\textsuperscript{1,2,3,4,5}},
%  \textbf{Twelfth Author\textsuperscript{1}},
% \\
%  \textbf{Thirteenth Author\textsuperscript{3}},
%  \textbf{Fourteenth F. Author\textsuperscript{2,4}},
%  \textbf{Fifteenth Author\textsuperscript{1}},
%  \textbf{Sixteenth Author\textsuperscript{1}},
% \\
%  \textbf{Seventeenth S. Author\textsuperscript{4,5}},
%  \textbf{Eighteenth Author\textsuperscript{3,4}},
%  \textbf{Nineteenth N. Author\textsuperscript{2,5}},
%  \textbf{Twentieth Author\textsuperscript{1}}
% \\
% \\
%  \textsuperscript{1}Affiliation 1,
%  \textsuperscript{2}Affiliation 2,
%  \textsuperscript{3}Affiliation 3,
%  \textsuperscript{4}Affiliation 4,
%  \textsuperscript{5}Affiliation 5
% \\
%  \small{
%    \textbf{Correspondence:} \href{mailto:email@domain}{email@domain}
%  }
% }

\begin{document}
\maketitle
\begin{abstract}
%While state-of-the-art pre-trained language models (PLMs) and large language models (LLMs) have proven exceptional in 
% While Large Language Models (LLMs) have proven powerful for news headline generation in terms of textual quality,
% %when measured by semantic similarity, 
% the numbers in the generated headlines are often inaccurate. 
% On the other hand, studies for solving word math problems are focused on only numerical reasoning to obtain the correct number, without needing to generate texts. 
% %The challenge in number-focused headline generation lies in maintaining both textual quality and numerical accuracy. 
%To address this challenge, 
%Number-focused headline generation poses a unique challenge for Large Language Models (LLMs), requiring both high-quality text and precise numerical accuracy.
Number-focused headline generation is a summarization task 
that requires both high textual quality and precise numerical accuracy, which poses a unique challenge for Large Language Models (LLMs). 
Existing studies in the literature focus only on either textual quality or numerical reasoning and thus are inadequate to address this challenge. 
In this paper, we propose a novel chain-of-thought framework for using rationales comprising key elements of the Topic, Entities, and Numerical reasoning (TEN) in news articles to enhance the capability  
for LLMs to generate topic-aligned high-quality texts with precise numerical accuracy. 
Specifically, a teacher LLM is employed to generate TEN rationales as supervision data, which are then used to teach and fine-tune a student LLM. 
%This process aims to improve automatic generation of rationales, ultimately enhancing the student LLM's ability to generate numerical headlines. 
Our approach teaches the student LLM automatic generation of rationales with enhanced capability for numerical reasoning and topic-aligned numerical headline generation.  
%jz3: I explained earlier directly. 
%In our framework, a rationale refers to the key intermediate steps and elements that must be considered before generating a headline.
%to refine and optimize the student LLM’s performance  
%\todo{XF: maybe we are optimizing the student LLM?}, 
%Importantly, we automatically construct topic and numerical-based preferences and leverage direct preference optimization for automatic rationale generation. 
Experiments show that our approach achieves superior performance %for number-focused headline generation 
in both textual quality and numerical accuracy.  
%jz0: I'm trying to clone this into my account and make it public.
Our implementation is publicly available at 
\texttt{https://github.com/TEN-Sum/TEN}.
%\footnote{https://anonymous.4open.science/r/TEN-4664}. 
\end{abstract}

\section{Introduction}

%jz1: below please cite a few papers in the text below. You can borrow from the NumHG paper. 
Headline generation, an important task in abstractive summarization, aims to condense a news article into a single line of text. 
%Text summarization models have proven effective for this purpose. 
%Text summarization models have been developed for headline generation.  
In the literature, text summarization models employ pre-trained language models \citep{lewis_bart_2019, raffel_exploring_2023, zhang_pegasus_2020} and large language models (LLMs)~\cite{jin_comprehensive_2024} have shown 
%effective for generation of headlines  high textual quality. 
high textual quality for headline generation. 

Numerical facts are crucial elements for modern news articles, and headlines often include numerals to enhance conciseness and attract readers' attention. A headline like "Pink Floyd reaches deal with Sony to sell music catalogue for \underline{\$400m}"\footnote{https://www.theguardian.com/music/2024/oct/02/pink-floyd-catalog-sony} immediately grabs readers' interest. 

%jz1:The current example in Fig. 1 can not show how the correct number is obtained from the news article. Please include relevent textual context in the source document.  
\begin{figure}[t]
  \centering
  \includegraphics[width=\columnwidth]{f1}
 % \caption{Example of Auto-Generated Rationale for Teaching LLM Headline Generation using the TEN Scheme}
 \caption{An example TEN Rationale for key elements of \underline{T}opic (green), \underline{E}ntities (blue) and \underline{N}umerical reasoning (purple).}
  \label{f1}
\end{figure}

Research shows that generating headlines with correct numbers requires mathematical reasoning capabilities in text summarization models~\cite{huang-etal-2024-numhg}. 
Obtaining correct numbers in headlines can involve mathematical operations such as addition, subtraction, and rounding of numbers from the source news articles. 
%\todo[color=green]{numhg paper now published in LREC-COLING 2024 --> done}
% jz1: some text to describe the example in Fig. 1. needed -- explain how the correct number is obtained (what operation) and what text summarization. 
As shown in Fig.~\ref{f1}, the news article covers the St. Louis protests, mentioning various entities and numbers. To generate an accurate headline, the language model must first identify the most newsworthy aspect of the event—the number of injured police officers. It then needs to calculate the correct number based on the information provided. In this case, the headline's number, 10, is not explicitly stated but requires addition (9 plus 1).
%jz1:below, citation needed. 
Number-focused headline generation requires 
not only text summarization to produce high quality text but also numerical reasoning within the textual context to generate the correct numbers. 

Existing studies on text summarization and numerical reasoning are inadequate for this challenging numerical headline generation problem~\citep{huang-etal-2024-numhg}. For text summarization, state-of-the-art pre-trained language models (PLMs)~\citep{lewis_bart_2019,zhang_pegasus_2020,liu_brio_2022} have relied on supervised fine-tuning to develop their summarization abilities. Researchers have also applied Chain-of-Thought (CoT) prompting~\citep{wang_element-aware_2023}, reinforcement learning~\citep{stiennon_learning_2022}, and direct preference optimization (DPO)~\citep{rajpoot_team_2024} to large language models (LLMs), aiming to improve their summarization quality. However, these methods focus on textual quality and do not address numerical accuracy.
%in the summaries. 

On the other hand, numerical reasoning models mainly focus on tasks that require producing a single numerical answer, such as solving word math problems, rather than generating text that includes numbers~\citep{ling_program_2017,amini_mathqa_2019,chiang_semantically-aligned_2019, cobbe_training_2021,wei_chain--thought_2023}. 
%jz3: below sentence "rationale" means "explanation", different from our definition. As rationale is only defined later, so rewrite and avoid using the word "rationale". 
Researchers have shown that language models' proficiency in solving these tasks can be enhanced through 
%rationale-augmented training 
explanation of intermediate steps
~\citep{amini_mathqa_2019, chiang_semantically-aligned_2019, cobbe_training_2021, wang_t-sciq_2023}, verification~\citep{cobbe_training_2021, wang_math-shepherd_2024}, and reinforcement learning~\citep{wang_math-shepherd_2024}. 
However, these techniques are developed in a setting where the question is given and they only need to infer the correct number as the final output. 

%When it comes to generating a numerically accurate headline, the challenge is twofold: it requires generation of concise text that captures both the essential topic and accurate numerical facts about the topic from the source news article. 
%This task involves not only text summarization but also the ability to extract and contextualize precise numbers within the textual context. 
%The model for number-focused headline generation needs to not only generate high quality text but also perform numerical reasoning within the textual context to generate the correct numbers. 

%jz3: below sentence needs more details. See my re-write.
In this paper, we propose a novel Chain-of-Thought (CoT) framework that uses rationales comprising key elements of \underline{T}opic, \underline{E}ntity and \underline{N}umerical reasoning (TEN) to teach and fine-tune LLMs for number-focused headline generation. 
Here rationales refer to textual descriptions for the key elements in a news article -- topics, entities, and numbers and their intermediate reasoning steps. These key element rationales can be used to enhance LLMs for the generation of topic-aligned headlines with higher numerical accuracy.  
%and critical elements that must be carefully considered before generating a headline.

Instead of costly manual annotation of TEN rationales, we propose to fine-tune open-source LLMs (e.g. Mistral 7B) to automatically generate such rationales for numerical headline generation.   
%In our approach, we first generates TEN reasoning rationales and then the news headline; for feasibility, ideally the base LLM should be an open-source LLM. 
%Building upon the teacher-student fine-tuning framework \citep{wang_t-sciq_2023}, our approach decomposes the rationales and optimizes the student LLMs via automatically generated Topic-Entity-Number-reasoning preferences. 
To enhance the capability for an open-source LLM to generate TEN rationales, we employ the teacher-student knowledge distillation framework~\cite{wang_t-sciq_2023} and leverage a powerful teacher LLM (e.g. GPT 4o) to generate TEN rationals as supervision data to fine-tune the open-source LLM as a student. 
Experiments show that our approach can achieve significant improvement over strong baselines in both textual quality and numerical accuracy. 

% jz3: training comprises two phases; inference is not a phase of training.
% jz3: but the description below is duplicated in Section 3. Such details are not needed for Introduction and so I removed it. 
% Training of the system comprises two phases -- 
% teacher generation of supervision data and fine-tuning student LLMs.
% %, and inference with the student LLMs. 
% In the supervision data generation phase, we instruct the teacher LLM to generate TEN rationales. 
% %that addresses two key issues: topic alignment and numerical reasoning. 
% %These rationales comprise three essential elements for headline generation: (1) Topic [T]; (2) Entities [E]; and (3) numerical Reasoning [N]. 
% In the fine-tuning phase, we develop two independent student LLMs: the rationale generator is fine-tuned to generate TEN reasoning rationales, while the headline generator is to generate headlines. 
% %jz3: below citations needed for DPO. 
% We further refine the rationale generator using DPO. The preference data for DPO are automatically generated to favour rationales that lead to headlines with matching topics and accurate numbers. 
% At the inference stage, the two fine-tuned student LLMs are applied sequentially. First, the rationale generator produces TEN rationales and then the headline generator uses these rationales together with the news articles as inputs to create the final headlines.

Contributions of our research are three fold:
\begin{itemize}
    \item We propose a CoT framework that uses rationales of key elements Topic, Entities and Numerical reasoning for LLMs to generate number-focused headlines. 
    %scheme to generate rationales automatically and use them as teaching signals to train student models to produce CoT-based headlines via fine-tuning.
    \item To enhance the capability for LLMs to generate topic-aligned headline text with high numerical accuracy, we apply the teacher-student framework to distill knowledge from a powerful teacher LLM to fine-tune an open-source LLM for automatic generation of TEN rationales.   
    %We distill TEN rationales from the teacher LLM GPT-4o to fine-tune student LLMs Mistral-7B and Llama-3.1-8B on the benchmark dataset NumHG and XSum. Our models achieve higher numerical accuracy than existing methodologies by 2.26 on NumHG, and 2.56 on XSum.
    \item We further develop a strategy based on Direct Preference Optimization \citep{rafailov_direct_2023} for LLMs to refine generation of TEN rationales. 
    %Our experiment results show that both ROUGE scores and accuracy scores are improved for the generated headlines.
\end{itemize}


\section{Related Work}

\subsection{Headline Generation}

Headline generation, a form of extreme text summarization, requires producing highly condensed, single-sentence summaries that capture the key information in a news article. 
%most important idea of the input text. 
% In the pioneering studies~\citep{narayan_dont_2018} and \citep{rush_neural_2015} at an early stage, 
%are pioneering work for extreme summarization. 
%They developed 
In early studies~\citep{rush_neural_2015,narayan_dont_2018}, models are supervise-trained on datasets containing single-sentence summaries (XSum)~\citep{narayan_dont_2018} and news-headline pairs (Gigaword) \citep{rush_neural_2015} However, their CNN-based and RNN-based approaches have since been outperformed by transformer-based models. 
%in the extreme summarization task. 
Recent studies show that transformer-based PLMs such as BART~\citep{lewis_bart_2019}, PEGASUS~\citep{zhang_pegasus_2020}, and BRIO~\citep{liu_brio_2022} can be fine-tuned on XSum and Gigaword to achieve promising results for extreme summarization and headline generation. 

While PLMs laid the bedrock for summarization, the advancement of LLMs has pushed the boundaries further. Several LLM-based approaches have emerged for general text summarization. Recent works have leveraged CoT prompting for summarization, proposing a "Summary Chain-of-Thought" method that guides LLMs to focus on key elements and generate summaries step-by-step~\citep{wang_element-aware_2023}. To further enhance summary quality, reinforcement learning methods have been employed to optimize LLMs based on human preferences~\citep{stiennon_learning_2022}. LLM-based approaches have also been tailored specifically for headline generation. For instance, leveraging reinforcement learning, \citet{tan_enhancing_2024} focus on creating personalized headlines for content recommendation. 
These approaches, whether PLM-based or LLM-based, focus on the text quality of summarization and the numerical accuracy is overlooked.
%\todo[color=green]{XF: maybe we need some related work to highlight the challenges from headline generation: 2020 extreme headline --> tbc}. 

%There are some closely related works. 
%Closely related 
Research on number-focused headline generation is reported recently. 
\citet{huang-etal-2024-numhg} assess the performance of PLMs in number-focused headline generation, but they do not provide strategies to enhance the models' numerical accuracy. \citet{rajpoot_team_2024} apply DPO to optimize headline generation using a preference dataset designed to train the model to favor headlines with correct numbers. While this preference for correct numbers can improve numerical accuracy, solely relying on it may degrade the textual quality of the generation. 
%Unlike previous approaches, our approach aims to enhance textual quality in headline generation as well as improve language models' numerical reasoning ability when generating headlines. 
% includes components explicitly designed to 

% \subsection{numerical Reasoning}
% \todo{XF: Do we discuss approaches other than providing rationales?} 
% Learning from human crafted rationales helps the performance, but requires human efforts
% \noindent{\textbf{Learning from human-crafted rationales.}} Researchers have demonstrated that language models' numerical reasoning abilities can be enhanced through learning from human-crafted rationales. 
% % rationale-augmented training. 
% One line of research employs humans to create natural language rationales for training~\citep{ling_program_2017}. They were the first to propose that language models could achieve higher accuracy in solving word math problems when required to generate both final answers and intermediate reasoning steps. 
% %Other researchers have shown that creating 
% Another line of research utilizes symbolic rationales for improving numerical reasoning abilities. For instance, symbolic equations can be used as intermediate steps to help solve word math problems~\citep{chiang_semantically-aligned_2019}. And in~\citep{amini_mathqa_2019}, the authors introduce a new representation language to model the intermediate steps so as to improve both the performance and the interpretability of the learned models. While effective, these methods rely heavily on human input, making them resource-intensive and potentially less scalable. 

% Learning from rationales generated by teacher LLMs helps the performance, and less human efforts
% \noindent{\textbf{Learning from teacher LLMs.}} 
% \subsection{Learning from LLM Generated TEN rationales.}
\subsection{CoT Prompting for Rationale Generation}

CoT prompting has gained great popularity due to its potential to unlock LLMs' reasoning capabilities by simply instructing them to generate intermediate steps as rationales before reaching a final answer~\citep{wei_chain--thought_2023}. For example, one can utilize CoT reasoning by simply adding the phrase "let's think step by step" to the end of each question~\citep{kojima_large_2023}. This approach is improved by a two-step process to generate rationales~\citep{zhang_automatic_2022}: first, selecting representative questions to generate exemplar rationales, and then using these representative rationales as demonstrations for LLMs to generate reasoning steps for other questions in the dataset. This idea is further enhanced by including the correct solution in prompts can enhance the quality of LLM-generated rationales~\citep{magister_teaching_2023}. 
 

%, specifically addressing the numerical accuracy issues in headline generation. 
% Our work also follows a two-step process to instruct an LLM to generate TEN rationales automatically, but we focus on creating rationales suitable for number-focused headline generation.

% Better rationales gives better performance
% \noindent{\textbf{Generating rationales through CoT prompting.}} In the LLM era, CoT prompting has gained great popularity due to its potential to unlock LLMs' reasoning capabilities by simply instructing them to generate intermediate steps before reaching a final answer~\citep{wei_chain--thought_2023}. CoT prompting has been applied to various tasks, including automatic rational generation. \citep{kojima_large_2023} demonstrate that CoT reasoning steps can be generated in a zero-shot manner by simply adding the phrase "let's think step by step" to the end of each question. \citep{zhang_automatic_2022} propose a two-step process to generate rationales automatically: first, selecting representative questions to generate exemplar rationales, and then using these representative rationales as demonstrations for LLMs to generate reasoning steps for other questions in the dataset. This idea is further enhanced by including the correct solution in prompts can enhance the quality of LLM-generated rationales~\citep{magister_teaching_2023}. Our work also follows a two-step process to instruct an LLM to generate TEN rationales automatically, but we focus on creating rationales suitable for number-focused headline generation. 

Especially for word math problems, research shows that LLM's numerical reasoning ability can be improved by learning from human-crafted rationales, including natural language intermediate reasoning steps~\citep{ling_program_2017} and symbolic representations like equations~\citep{chiang_semantically-aligned_2019,amini_mathqa_2019}. 
%While these methods have proven effective in enhancing model performance for tasks such as solving word math problems, 
But these methods rely on human annotations and therefore are costly. 
% jz3:I reworded and I think it sits here better. 
%\todo{XF: I feel this paragraph does not fit in here, can we put this paragraph at the beginning of this subsection?}

%potentially less scalable. 
% Better rationales gives better performance
\subsection{Learning from Teacher LLM Generated Rationales}
% summarize non-CoT in one or two sentences
% \noindent{\textbf{Learning from LLM generated TEN rationales.}} 

%An alternative approach that requires less human efforts involves 
% Learning from TEN rationales generated by teacher LLMs is a scalable alternative to reduce human efforts. 
Learning from rationales generated by teacher LLMs is a scalable alternative to human annotation. 
Research has shown that CoT reasoning steps generated from teacher LLMs can be used to fine-tune smaller student language models~\cite{ho_large_2023,hsieh_distilling_2023} that 
%. The CoT fine-tuned student models sometimes even outperform the teacher model in certain tasks. 
may even outperform the teacher LLM for some tasks.
% They validated their approach on four types of complex reasoning including arithmetic, symbolic, common sense, and other reasoning. 
%Similarly, in~\citep{hsieh_distilling_2023}, the authors suggest that the student model can achieve comparable performance with reduced training data when fine-tuned with LLM-generated rationales. They test their approach on several NLP tasks including arithmetic word math problems. 
Such teacher-student knowledge distillation has also been applied to multi-modal training for science QA~\cite{wang_t-sciq_2023}, which involves numerical reasoning.
The authors propose that mixing simple and complex reasoning in supervision data can enhance student LLMs' performance~\cite{wang_t-sciq_2023}. 
%Our work also follows a two-step process to instruct an LLM to generate TEN rationales automatically. But unlike existing works, our approach focuses on creating rationales that address the topic alignment and numerical reasoning so that two independent student LLMs can be further fine-tuned for number-focused headline generation tasks. 
Our approach also leverages the teacher-student knowledge distillation framework. 
Unlike existing work, we focus on rationales for topic alignment as well as numerical reasoning in numerical text generation.
%, which is not studied in existing work. 

Researchers have explored various approaches to enhance rationale quality, including the use of verifiers~\cite{cobbe_training_2021}, majority voting~\cite{wang_self-consistency_2023} and reinforcement learning~\cite{wang_math-shepherd_2024}. 
% In \citep{cobbe_training_2021}, the authors propose to train a verifier to rank the probabilities of correctness for model-generated rationales and select the most likely one. Similarly, the majority vote has been utilized to choose the most consistent output from several output rationales generated by LLMs~\citep{wang_self-consistency_2023}. 
Our approach, which leverages DPO for refining rationale generation, is closely related to reinforcement learning strategies~\citep{wang_math-shepherd_2024}. 
%the authors propose to automatically assess the step-wise correctness of model-generated rationales, 
%train a reward model using the labelled data, and further
%the authors propose to improve %LLMs' performance 
%LLM rationale generation through reinforcement learning. 
%Our approach of using DPO to refine also 
%explores the effectiveness of DPO in enhancing the quality of the generated rationales. 
%jz3: pelase check the my re-write below. 
However, unlike previous work that focuses solely on reward models for numerical reasoning rationales, our approach develops preference datasets tailored to both nuanced topic alignment and complex numerical reasoning. 

\begin{figure*}[t]
  \centering
  \includegraphics[width=\textwidth, trim={2cm 2.45cm 0cm 1.7cm}, clip]{f2}
%  \caption {Three-phase pipeline of the proposed TEN fine-tuning scheme: (1) Generating supervision data, (2) Fine-tuning student LLMs, and (3) Inferring with fine-tuned student LLMs for headline generation.}
  \caption {Our TEN approach for automatic generation of rationales to enhance numerical headline generation.}
  \label{fig:three_phase_of_TEN}
\end{figure*}

\section{Methodology}
%\todo[color=green]{XF: figure 2 ECNC need to be updated to TEN --> done}

%This section presents our TEN reasoning rationales  fine-tuning scheme TEN, as shown in Figure~\ref{fig:three_phase_of_TEN}. 
This section presents our framework for leveraging the teacher-student knowledge distillation framework to fine-tune LLMs for automatic generation of TEN rationales to enhance LLM headline generation.
%jz3: below, CoT rationale > TEN rationale
We employ a teacher LLM (e.g. GPT 4o) to generate TEN rationales 
%for number-focused headline generation and 
%jz3: below, teaching data > supervision data 
and use these rationales as supervision data to fine-tune a student LLM (e.g. Mistral-7B), including a rationale generator for automatic generation of TEN rationales and
a headline generator for headline generation. 

As shown in Fig.~\ref{fig:three_phase_of_TEN}, our approach adopts a teacher-student framework to fine-tune a (student) LLM to automatically generate TEN rationales.  
%framework comprises three phases: teacher generation of supervision data, fine-tuning student LLMs for rationale and headline generation, and inference with fine-tuned student LLMs. 
When generating supervision data to fine-tune a rationale generator, 
we prompt a teacher LLM to generate rationales for each news-headline pair in the dataset.  
The rationales are aimed to enhance the topic alignment and numerical reasoning capabilities for numerical headline generation, comprising key elements \underline{T}opic and \underline{E}ntities, 
as well as \underline{N}umbers in the news article and the intermediate reasoning steps to calculate the correct number in the headline. 
%These rationales are further decomposed into three essential elements (TEN) for headline generation. 

The teacher LLM generated rationales are used as supervision data to 
%When fine-tuning student LLM, we use the supervision data along with the input (news-article, headline) pairs to two independent student LLMs, 
%one fine-tuned for generating TEN rationales (the rationale generator) and another for generating headlines (the headline generator). 
fine-tune a student LLM as the rationale generator. 
We further refine the rationale generator using DPO. The preference data for DPO are automatically generated to favour rationales that lead to headlines with matching topics and accurate numerals. 
The news article and teacher LLM generated TEN rationales are then used to fine-tune another student LLM for headline generation. 
In the inference phase, the two fine-tuned student LLMs are used sequentially. The rationale generator will first produce TEN rationales for the input news article. The headline generator will then use the rationales together with the news articles as input to generate final headlines. 

%we generate the final headline in two steps.
\begin{figure*}[t]
  \includegraphics[width=\textwidth]{f3ab}
%  \parbox[b]{0.45\textwidth}{\centering Step 1: Generating representative TEN rationales via zero-shot prompting.}
  % \parbox[b]{0.45\textwidth}{\centering (a) An example zero-shot prompt for generating TEN rationales.}
  % \hspace{0.05\textwidth}
%  \parbox[b]{0.45\textwidth}{\centering Step 2: Generating teaching signals for the entire dataset via few-shot prompting.}  
  % \parbox[b]{0.45\textwidth}{\centering (b) The five-shot prompt to generate rationales for supervised fine-tuning student LLMs}  
  \caption {The process for Teacher LLM to generate TEN rationales for fine-tuning student LLMs}
  \label{fig:pre_tuning}
\end{figure*}

\subsection{Teacher LLM generation of TEN rationales}
In this phase, we focus on utilizing a teacher LLM to generate TEN rationales as supervision data. 
Figure~\ref{fig:pre_tuning} shows the process for generating this data. It is generated through a two-step process~\citep{zhang_automatic_2022}. 
% As shown in Figure~\ref{fig:pre_tuning}, 
%jz0: below, with the five examples, I believe we did 5-shot prompting here rather than zero-shot prompting? 
%jz0: I reworded the setence to avoid confusion. 
%In the first step, we carefully select representative examples and instruct a teacher LLM to generate rationales for these examples using zero-shot prompting.
In the first step, we instruct a teacher LLM with zero-shot prompting to generate demonstration TEN rationales for a few (five) representative examples for calculating numbers in headlines. 
In the second step, we employ these demonstration rationales as context to generate rationales for other examples in the whole training dataset using few-shot prompting. Specifically, we create demonstration rationales for five representative examples from the NumHG dataset \citep{huang-etal-2024-numhg}. 
%This dataset not only provides news-headline pairs but also includes human annotations detailing the types of mathematical operations required to derive the numerals in the headlines. 
The five examples are selected to represent five types of mathematical operations annotated in this dataset: (a) Copying: The numeral is directly copied from the article. (b) Addition: Numerals from the article are added to get the final numeral. (c) Subtraction: One numeral is subtracted from another. (d) Paraphrasing: The digits of the numeral are rewritten (e.g., changing 6,000 to 6k). and (e) Rounding: Only certain digits after the decimal point are retained. The details of the five demonstrations are shown in Appendix~\ref{appendix: five-demonstrations}.

%\todo{XF: It feels we can include those 5 samples in the appendix (e.g. using lstlisting).}

% \begin{enumerate}
%     \item Copying: The numeral is directly copied from the article.
%     \item Addition: Numerals from the article are added to get the final numeral.
%     \item Subtraction: One numeral is subtracted from another.
%     \item Paraphrasing: The digits of the numeral are rewritten (e.g., changing 6,000 to 6k).
%     \item Rounding: Only certain digits after the decimal point are retained.
% \end{enumerate}

We next instruct the teacher LLM to elaborate on the intermediate steps that lead to high-quality headlines with accurate numerals for the five selected examples. The prompts we use are shown in Figure~\ref{fig:pre_tuning}. 
%We structure these intermediate steps to focus on three key elements for the news article -- The Topic of the headline (T), the Entities mentioned (E), and the Numbers mentioned (N) -- as well as the intermediate reasoning steps to calculate the number in the reference headline. 
Note that the TEN rationale comprise key elements for the news article, including numerical reasoning steps. 
Note also that to enhance the reliability of the teacher LLM-generated rationales, we also provide the reference headlines and the correct numbers to the teacher LLM in prompts. The numbers in the reference headlines are masked to ensure the teacher LLM focuses on the topic when generating topic-alignment rationales. The correct numbers are provided separately as hints to improve accuracy in numerical reasoning rationales. We then manually review and refine these five rationales to adhere to the template, ensuring consistency across the entire dataset. Using these five example TEN rationales as demonstrations, we instruct the teacher LLM to generate TEN rationales for the complete training datasets through five-shot prompting. 

% \begin{enumerate}
%     \item Topic of headline (E)
%     \item Characters mentioned (C)
%     \item Numerals mentioned (N)
%     \item Calculations of Numerals (C)
% \end{enumerate}

\subsection{Fine-Tuning Student LLMs}
Inspired by~\citep{zhang_multimodal_2024}, we fine-tune two student LLMs independently, as illustrated in Figure~\ref{fig:three_phase_of_TEN}. The first student LLM (rationale generator) generates TEN rationales from news articles. The second model (headline generator), initialized from the same student LLM, is fine-tuned to predict headlines using both the news articles and the TEN rationales generated by the teacher LLM as inputs. 

We also apply DPO to the rationale generator to enhance its output quality. To construct the preference dataset for DPO, we first use the fine-tuned rationale generator to sample multiple rationales for each news article in the training data using a high temperature. Next, we use the headline generator to complete headlines based on the news articles and sampled rationales. We then build a pair of chosen and rejected rationales for each instance in the training data based on the following criteria: (a) Choose the rationale that leads to the headline with correct numerals and reject the one that results in a headline with incorrect numerals. (b) Choose the rationale with a high ROUGE score compared to the reference rationale and reject the one with a low ROUGE score. The flowchart for automatically constructing preference data is shown in Appendix~\ref{appendix:preference_data}. 
%\todo[color=green]{XF: Maybe an algorithm can better explain how preference is made. --> Done with a flowchart in appendix}

% \begin{itemize}
%     \item Choose the rationale that leads to the headline with correct numerals and reject the one that results in a headline with incorrect numerals.
%     \item Choose the rationale with a high ROUGE score compared to the reference rationale and reject the one with low ROUGE score.
% \end{itemize}

% \subsection{Inferring with Fine-Tuned Student LLMs}
% In this phase, we have obtained the fine-tuned student LLMs: the rationale generator and the headline generator. The two fine-tuned student LLMs are used sequentially. As shown in Figure~\ref{fig:three_phase_of_TEN}, the rationale generator will first produce TEN rationales for the input news article. Then, the headline generator will use the rationales together with the news articles as input to generate final headlines. 

% During the deployment phase, we use the rationale generator and headline generator in a pipeline. First, the rationale generator predicts TEN rationales based on the news articles. Then, the headline generator takes these predicted TEN rationales along with the original news articles to produce the final headlines. 

% \section{Experiment Setup}
\section{Experiments}
We evaluate our approach TEN against state-of-the-art baselines on benchmark datasets for number-focused headline generation. 
%from one representative PLM-based method and recent LLM-based methods for headline generation tasks. 
%\todo[color=green]{Experiment environment and software used need to be updated --> done}
%Experiments on NumHG are conducted on a system which consists of 32 cores, 128G memory and is equipped with an NVIDIA A100 (40G) GPU. Experiments on XSum are conducted on a system which consists of 32 cores, 128G memory and is equipped with 8 NVIDIA L40S (45G) GPU. 
In all experiments, GPT 4o is the teacher LLM for our TEN approach. 
All experiments were conducted on a system with 32 cores, 128GB memory and NVIDIA A100(40G) GPU. The estimated GPU usage for our experiments is approximately 2,000 hours.
All deep neural networks are implemented using Transformers~\citep{wolf2019huggingface} (distributed with the Apache-2.0 license) under the support of PyTorch~\citep{paszke2019pytorch} (distributed with the modified BSD-3-Clause license). 
%jz3: @Zhen: the real url for the anonymous github repo is needed below. 
% the repo content can be done after paper submission but the url must be real. 
Implementation details, including parameter efficient fine-tuning settings and hyperparameter settings are in Appendix~\ref{appendix:implementation}. 

\subsection{Datasets and Evaluation Metrics}
We evaluate our proposed approach using two real-world datasets: 
\begin{itemize}
	\item The NumHG dataset~\citep{huang-etal-2024-numhg} is a large dataset for number-focused headline generation that is also used for SemEval-2024 Task 7 ``NumEval: Numeral-aware language understanding and generation'' task.~\footnote{https://sites.google.com/view/numeval/numeval} 
 %It contains approximately 
 %\todo[color=green]{exact number maybe better --> done}27,74 news-headline pairs sourced from the news platform Newser\footnote{https://www.newser.com/}. 
 Each news article contains 200--300 words, and all headlines include numerals. This dataset provides human annotations of mathematical operations required to derive the numerals in each headline. 
 %To focus on validating the model's numerical reasoning ability, 
 We apply a pre-processing step to the dataset by removing duplicate samples and retaining only those with one number in the headlines. 
 %After this data pre-process, 
 In the end, we obtained 18,315 samples for training and 3,579 for testing.
    \item The XSum dataset~\citep{narayan_dont_2018} is 
    %\todo[color=green]{intro to XSum needed --> done} 
    an extreme summarization dataset comprising 226,771 BBC articles from 2010 to 2017, each accompanied by a single-sentence summary. 
    %These summaries, serving as introductory sentences for the articles, are crafted by the authors themselves. 
    We applied pre-processing and selected articles containing 200--500 words and summaries containing only a whole number. This resulted in 9,052 samples for training and 1,605 for testing. 
    % with one single whole number, , as headlines with multiple numerals are challenging to evaluate for numerical accuracy
\end{itemize}

% \subsection{Evaluation Metrics}
We adopt the evaluation metrics commonly used in existing studies~\citep{huang-etal-2024-numhg} to assess both the textual quality and numerical accuracy for headline generation. 
%For textual quality, 
We adopt ROUGE~\citep{lin2004rouge}, BERTScore~\citep{zhang2019bertscore}, and MoverScore~\citep{zhao2019moverscore} for textual quality. 
%They are calculated against ground truth headlines. 
For numerical accuracy, 
a generated headline's numeral is considered correct if it matches the numeral
in the reference headline. 
%and it matches the ground truth. 
We use the evaluation code~\footnote{https://github.com/ChunJiChen/NumEval\_Evaluation} from~\citet{huang-etal-2024-numhg} to automatically calculate these metrics. 

%jz3: below should move to later closer to the result table.
%The NumHG dataset includes annotations for the type of operations needed to obtain the correct numeral in headlines. This allows us to break down the numerical accuracy into three categories: overall accuracy, copy accuracy, and reasoning accuracy. Copy accuracy is the numerical accuracy when the correct numeral can be directly copied from the news article. Reasoning accuracy is the numerical accuracy when mathematical operations are required. For the XSum dataset, which lacks operation type annotations, we only report the overall accuracy. 

\subsection{Baselines}

We compare TEN against three baseline methods, including one representative PLM-based method and two recent LLM-based methods. 
\begin{itemize}
	\item BART~\citep{lewis-etal-2020-bart,huang-etal-2024-numhg} 
 %is a PLM widely applied for various tasks. As reported in the NumHG dataset~\citep{huang-etal-2024-numhg}, 
 is a PLM-based representative baseline for numerical headline generation that is shown to outperform other PLM-based methods like T5, Pegasus, SEASON, and BRIO in terms of numerical accuracy while maintaining comparable textual quality~\cite{huang-etal-2024-numhg}. 
 %Thus BART is selected as the representative PLM-based method. 
%jz3: below, NLC_NLP > NLC, please update the table. 
\item NCL (NCL\_NLP)~\citep{zhao_ncl_nlp_2024} is an LLM-based method from SemEval-2024 Task 7 that achieves reasonable result. 
% jz3: our implementation of NLC should use the same teacher and student LLMs for TEN. 
Similar to TEN, they also employ the teacher-student framework to generate CoT rationales to fine-tune student LLMs for headline generation. 
%a teacher LLM (GPT-3.5-Turbo) to generate TEN rationales as supervision data and fine-tune a student LLM (Mistral-7B) for headline generation. 
%We implement this baseline by instructing the teacher LLM to generate rationales based on the given news articles, using the correct headlines as hints. 
Different from TEN, NCL does not do the structured, element-wise rationales. 
Comparing TEN against NCL will help us understand the effectiveness of our TEN reasoning strategy. 
 %jz3: below, NP-problm > NPP, please update the table. 
    \item NPP (NP-Problem)~\citep{rajpoot_team_2024} is another LLM-based baseline from SemEval-2024 Task 7. NPP achieved the highest numerical accuracy and comparable text quality among all submissions. 
%    an LLM-based method that came from SemEval-2024's task 7 which is a shared task specifically for the number-focused headline generation. 
They fine-tune Mistral-7B for headline generation and further align the model-generated headlines through DPO. 
%Their method achieved the highest numerical accuracy in headline generation among the participants in SemEval-2024's task 7. 
As both TEN and NPP use DPO to refine headline generation, their comparison can reveal the utility of our strategy of preference data generation for DPO. 
    % Unlike our TEN scheme, which breaks down TEN rationales into four essential elements, this method doesn't specify particular focus areas. We then use these rationales as supervision data to fine-tune Mistral-7B. 
\end{itemize}

% The first is fine-tuned BART. \citet{huang_numhg_2023} have evaluated state-of-the-art PLMs for number-focused headline generation, including BART, T5, Pegasus, SEASON, and BRIO. Their experimental results show that fine-tuned BART achieves the highest numerical accuracy in headline generation while maintaining comparable textual quality. 

% We also choose \citet{rajpoot_team_2024}'s method as a baseline. They have fine-tuned Mistral-7B for headline generation and further aligned the model-generated headlines through DPO. Their method achieved the highest numerical accuracy in headline generation among the participants in SemEval-2024's task 7. 

% The final baseline method we select is proposed by \citet{zhao_ncl_nlp_2024}. They employ a teacher LLM (GPT-3.5-Turbo) to generate TEN rationales as supervision data and fine-tune a student LLM (Mistral-7B) for headline generation. This knowledge distillation approach has shown state-of-the-art performance in reasoning tasks \citep{wang_t-sciq_2023} but is rarely used in headline generation. We choose this baseline because we also use CoT fine-tuning and want to compare our TEN rationale decomposition scheme with the existing method. We implement this baseline by instructing the teacher LLM to generate rationales based on the given news articles, using the correct headlines as hints. Unlike our TEN scheme, which breaks down TEN rationales into four key aspects, this method doesn't specify particular focus areas. We then use these rationales as supervision data to fine-tune Mistral-7B. 



\begin{table*}[t]
  \centering
  \resizebox{\textwidth}{!}{
  \begin{tabular}{lccc|ccc|ccc|c}
    \hline
    &&\textbf{Num Acc}&&&\textbf{ROUGE}&&&\textbf{BERTScore}&&\textbf{MoverScore}\\
    \cline{2-11}
     & Overall & Copy & Reasoning & 1 & 2 & L & P & R & F1 & \\
    \hline
    BART & 71.59 & 76.54 & 61.82 & 48.13 & 22.76 & 43.36 & 49.29 & 50.81 & 50.60 & 60.34 \\
    NPP & 74.57 & 77.43 & 68.93 & 49.24 & 23.44 & 44.08 & 50.17 & 50.57 & 50.36 & 60.54 \\
    NCL & 74.94 & 78.43 & 68.06 & 50.03 & 24.72 & 45.39 & 53.44 & 51.19 & 52.31 & \textbf{60.97} \\
    % \rowcolor{lightgray}
    TEN (Ours) & \textbf{77.20} & \textbf{81.11} & \textbf{69.49} & \textbf{51.14} & \textbf{25.46} & \textbf{46.29} & \textbf{54.57} & \textbf{51.84} & \textbf{53.21} & \textbf{61.23} \\
    \hline
    % \hline
  \end{tabular}
  }
  % \caption{\label{tbl:main_result_numhg}
  %   Comparative analysis of numerical accuracy and textual quality metrics for baseline methods and our proposed approach on the \textbf{NumHG} dataset.
  % }
  \caption{\label{tbl:main_result_numhg}
    Numerical accuracy (\%) and textual quality score (\%) for TEN against baselines on \textbf{NumHG}. Higher numbers indicate better performance. Best results are in bold, where results within 0.5\% difference are deemed comparable.
  }
\end{table*}

\begin{table*}[t]
  \centering
  \resizebox{0.82\textwidth}{!}{
  \begin{tabular}{lc|ccc|ccc|c}
    \hline
    &\textbf{Num Acc}&&\textbf{ROUGE}&&&\textbf{BERTScore}&&\textbf{MoverScore}\\
    \cline{2-9}
     & Overall & 1 & 2 & L & P & R & F1 & \\
    \hline
    BART & 29.34 & 43.83 & 19.81 & 35.46 & 52.54 & 54.16 & 53.38 & 60.46 \\
    NPP & 30.15 & 46.32 & \textbf{22.58} & \textbf{38.08} & 55.60 & \textbf{55.82} & 55.73 & \textbf{61.35} \\
    NCL & 36.76 & 45.17 & 21.47 & 37.07 & 57.30 & 53.49 & 55.41 & \textbf{61.02} \\
    % \rowcolor{lightgray}
    TEN (Ours) & \textbf{39.07} & \textbf{46.63} & \textbf{22.50} & \textbf{38.36} & \textbf{58.81} & 54.57 & \textbf{56.70} & \textbf{61.36} \\
    \hline
    % \hline
  \end{tabular}
  }
  % \caption{\label{tbl:main_result_xsum}
  %   Comparative analysis of numerical accuracy and textual quality metrics for baseline methods and our proposed approach on the \textbf{XSum} dataset.
  % }
  \caption{\label{tbl:main_result_xsum}
    Numerical accuracy (\%) and textual quality metric score (\%) for TEN against baseline methods on  \textbf{XSum}. Higher numbers indicate better performance. The best results are in bold, where results within 0.5\% difference are deemed comparable.
  }
\end{table*}

% \begin{table}[t]
%   \centering
%   \resizebox{0.5\textwidth}{!}{
%   \begin{tabular}{lcccc}
%     \hline
%     & \multicolumn{2}{c}{\textbf{NumHG}} & \multicolumn{2}{c}{\textbf{Xsum}} \\
%     \hline
%     & NumAcc & ROUGE-1 & NumAcc & ROUGE-1 \\
%     \hline
%     Mistral-7B-v0.3 & & & & \\
%     \hline
%     \: TEN w/o DPO & 76.24 & 50.95 & 37.69 & 46.26 \\
%     \: TEN (Ours) & \textbf{77.20} & \textbf{51.14} & \textbf{39.07} & \textbf{46.63}\\
%     % \hline
%     \hline
%     Llama-3.1-8B & & & & \\
%     \hline
%     \: TEN w/o DPO & 76.06 & 50.69 & 36.76 & 45.89 \\
%     \: TEN (Ours) & \textbf{77.89} & \textbf{51.15} & \textbf{37.51} & \textbf{46.03} \\
%     \hline
%   \end{tabular}
%   }
%   \caption{\label{tbl:dpo_ablation}
%     Comparative analysis of headline generation performance: impact of refining model-generated rationales through Direct Preference Optimization (DPO) on two student models Mistral-7B-v0.3 and Llama-3.1-8B.
%   }
% \end{table}

% \section{Main Results}
\subsection{Experiment Results}

%\noindent \textbf{TEN vs Baselines.} Tables~\ref{tbl:main_result_numhg} and~\ref{tbl:main_result_xsum} show the performance of TEN against baseline methods on the NumHG and XSum datasets respectively. 
For fair comparison, all baselines use Mistral-7B, and we also use Mistral-7B as the student LLM for TEN.
Following existing studies in the literature~\cite{huang-etal-2024-numhg}, we employ numerical accuracy and textual similarity metrics ROUGE~\cite{lin2004rouge}, BERTScore~\cite{zhang2019bertscore} and MoverScore~\cite{zhao_moverscore_2019} for evaluation. 
Note that textual similarity metrics evaluate the complete headline text, including numbers as tokens, and therefore can  be seen as measuring the overall quality for number-focused headlines.

The NumHG dataset includes annotations for the type of operations needed to obtain the correct numeral in headlines. 
%This allows us to break down 
The numerical accuracy thus includes overall accuracy, copy accuracy (when numbers can be directly extracted from the news), and reasoning accuracy (when mathematical operations are required).   
%-- Copy accuracy is the numerical accuracy when the correct numeral can be directly copied from the news article. Reasoning accuracy is the numerical accuracy when mathematical operations are required. 
On the XSum dataset, which lacks operation type annotations, we only report the overall accuracy. 
%The results demonstrate that our approach outperforms existing methodologies. Note that the LLM baselines all utilize Mistral-7B
%\todo[color=green]{XF: suggest we use Mistral-7B instead of Mistral-7B-v0.3 in main content --> already done?}, 

Observe from Table~\ref{tbl:main_result_numhg} that on the NumHG dataset, TEN achieves an overall numerical accuracy of 77.20\%, surpassing BART by 5.61\%, NPP by 2.63\%, and NCL by 2.26\% (in absolute percentage points).
In Table~\ref{tbl:main_result_xsum}, on the XSum dataset, TEN reaches an overall accuracy of 
%\todo[color=green]{XF: these numbers should be updated --> done}
39.07\%, outperforming BART by 9.73\%, NPP by 8.92\%, and NCL by 2.31\%. While improving numerical accuracy, TEN also maintains mostly higher textual quality by textual similarity metrics ROUGE, BERTScore, and MoverScore. 

% jz3: some discussion is needed below. 
%Here: compare TEN against NP; compare TEN against NCL. Follow the description of baselines.
Models trained with our TEN framework outperform NCL, demonstrating that our TEN rationales are more effective than NCL's rationales that only explain how the correct number in the headline is obtained. Our approach also outperforms NPP, demonstrating that enhancing the intermediate rationale generation process is a more effective strategy for improving headline's numerical accuracy.

%\noindent \textbf{LLM-based Quality Evaluation.}
We further evaluated the quality of the generated news headlines using an LLM-based metric G-Eval \cite{liu_g-eval_2023}.
Recent research shows that LLMs can be used for evaluation of quality of generated texts and demonstrate strong correlation with human judgements. 
G-Eval leverages the capability of LLMs and Chain-of-Thoughts prmopts to assess the quality of model-generated texts. 
We employed G-Eval(GPT4) to evaluate four aspects of generated headlines: coherence, consistency, fluency, and relevance. As shown in Table~\ref{tbl:llm_based_quality_evaluation}, 
%TEN achieved the highest scores across all four dimensions.
TEN outperforms all baselines on NumHG and achieves comparable results on XSum. 


%jz0: Any results on XSum?
\begin{table*}[htbp]
  \centering
  \resizebox{\textwidth}{!}{

  % \begin{tabular}{lcccc}
  %   \hline
  %   &\textbf{Coherence}&\textbf{Consistency}&\textbf{Fluency}&\textbf{Relevance}\\
  %   BART&4.04&4.36&2.82&4.11\\
  %   NPP&4.17&4.65&2.91&4.22\\
  %   NCL&4.17&4.71&\textbf{2.96}&4.23\\
  %   TEN&\textbf{4.18}&\textbf{4.72}&\textbf{2.96}&\textbf{4.24}\\
  %   \hline
  % \end{tabular}  
  
  \begin{tabular}{l|cccc|cccc}
    \hline
    &\multicolumn{4}{c|}{\textbf{NumHG}} & \multicolumn{4}{c}{\textbf{XSum}}\\
    \hline    
    &\textbf{Coherence}&\textbf{Consistency}&\textbf{Fluency}&\textbf{Relevance}&\textbf{Coherence}&\textbf{Consistency}&\textbf{Fluency}&\textbf{Relevance}\\ \hline
    BART&4.0361&4.3647&2.8209&4.1068&3.3740&2.9179&2.6836&3.2840\\
    NPP&4.1734&4.6550&2.9068&4.2184&3.3782&2.9048&2.6758&3.2853\\
    NCL&4.1739&4.7108&\textbf{2.9616}&4.2264&3.3795&2.9037&2.6830&3.2767\\
    TEN&\textbf{4.1853}&\textbf{4.7210}&2.9594&\textbf{4.2436}&3.3733&2.9125&2.6802&3.2824\\
    \hline
  \end{tabular}
  }
  % \caption{\label{tbl:main_result_numhg}
  %   Comparative analysis of numerical accuracy and textual quality metrics for baseline methods and our proposed approach on the \textbf{NumHG} dataset.
  % }
  \caption{\label{tbl:llm_based_quality_evaluation}
    G-Eval scores for TEN against baselines on \textbf{NumHG} and \textbf{XSum}. Headlines are assessed in terms of Coherence (1-5), Consistency (1-5), Fluency (1-3), and Relevance (1-5). The higher numbers indicate better performance. Best results are in bold.}
\end{table*}

\subsection{Performance of TEN rationales and Teacher-student knowledge distillation}
%\noindent \textbf{Effect of TEN on GPT-4o.}
The structured rationales and teacher-student paradigm to distill knowledge from a teacher LLM (GPT-4o) to a student LLM (Mistral 7B and Llama 3.1) are important parts of our TEN framework. 
%Effectiveness of the system depends on the performance of the teacher LLM. 
To evaluate the effectiveness of TEN rationales and if GPT-4o is a good teacher LLM, we evaluated the performance of GPT-4o with and without TEN structured rationales under zero-shot prompting. %Specifically, GPT-4o is evaluated using NumHG test data in two settings. Setting 1 "without TEN" is to prompt GPT-4o to directly generate headlines based on news articles. Setting 2 "with TEN" is to include the instructions for GPT-4o to generate both TEN-structured rationales and headlines. 
The results are shown in Table~\ref{tbl:main_result_gpt4o_both}. It can be seen that by prompting GPT-4o to generate TEN rationales, performance improved significantly both for numerical accuracy and textual quality.

%jz0: below overall table replace the two separate tables on NumHG and XSum. 
%jz0: The headings "NumHG w/o TEN" is not quite right though --> "w/o TEN" as the method. Please reformat the table to match the other tables using two datasets NumHG and Xsum as columns. You have to use ROUGE-1 and BERSTScore-F1
\begin{table*}[htbp]
  \centering
  \resizebox{\textwidth}{!}{
  % \begin{tabular}{l|c|ccc|ccc|c}
  %   \hline
  %   & \textbf{Num Acc}&\textbf{ROUGE}&&&\textbf{BERTScore}&&&\textbf{MoverScore}\\
  %   & & 1 & 2 & L & P & R & F1 & \\
  %   \hline
  %   NumHG w/o TEN & 21.39 & 35.59 & 13.63 & 30.82 & 30.33 & 48.38 & 39.21 & 56.53 \\
  %   NumHG w TEN & 33.01 & 35.47 & 13.14 & 30.96 & 33.06 & 46.82 & 39.86 & 56.72 \\   
  %   XSum w/o TEN & 6.23 & 22.59 & 4.98 & 17.77 & 19.70 & 25.71 & 22.76 & 54.58 \\
  %   XSum w TEN & 9.91 & 22.18 & 5.14 & 17.65 & 21.43 & 24.73 & 23.14 & 54.63 \\
  %   \hline
  %   % \hline
  % \end{tabular}
  \begin{tabular}{l|cccc|cccc}
    \hline
    &\multicolumn{4}{c|}{\textbf{NumHG}} & \multicolumn{4}{c}{\textbf{XSum}}\\
    \hline
     & Num Acc & ROUGE-1 & BERTScore-F1 & MoverScore & Numm Acc & ROUGE-1 & BERTScore-F1 & MoverScore\\
    \hline
    w/o TEN & 21.39 & 35.59 & 39.21 & 56.53 & 6.23 & 22.59 & 22.76 & 54.58\\
    w TEN & 33.01 & 35.47 & 39.86 & 56.72 & 9.91 & 22.18 & 23.14 & 54.63\\
    \hline
  \end{tabular}
  
  }
  % \caption{\label{tbl:main_result_numhg}
  %   Comparative analysis of numerical accuracy and textual quality metrics for baseline methods and our proposed approach on the \textbf{NumHG} dataset.
  % }
\caption{\label{tbl:main_result_gpt4o_both}
%    The effectiveness of TEN rationales: Numerical accuracy (\%) and textual quality score (\%) for GPT-4o under zero-shot prompting on \textbf{NumHG} and \textbf{XSum}. 
Performance of GPT-4o with/without TEN rationales under zero-shot prompting on \textbf{NumHG} and \textbf{XSum}.
  }
\end{table*}

% \begin{table*}[htbp]
%   \centering
%   \resizebox{\textwidth}{!}{
%   \begin{tabular}{lccc|ccc|ccc|c}
%     \hline
%     &&\textbf{Num Acc}&&&\textbf{ROUGE}&&&\textbf{BERTScore}&&\textbf{MoverScore}\\
%     \cline{2-11}
%      & Overall & Copy & Reasoning & 1 & 2 & L & P & R & F1 & \\
%     \hline
%     Zero-shot w/o TEN & 21.39 & 28.35 & 7.67 & 35.59 & 13.63 & 30.82 & 30.33 & 48.38 & 39.21 & 56.53 \\
%     Zero-shot w TEN & 33.01 & 44.11 & 11.15 & 35.47 & 13.14 & 30.96 & 33.06 & 46.82 & 39.86 & 56.72 \\    
%     Five-shot w/o TEN & 43.55 & 52.00 & 26.88 & 41.37 & 16.93 & 36.27 & 38.46 & 50.30 & 44.32 & 57.95 \\
%     Five-shot w TEN & 51.16 & 58.74 & 36.21 & 41.51 & 17.25 & 36.50 & 45.19 & 49.07 & 47.14 & 57.86 \\    
%     \hline
%     % \hline
%   \end{tabular}
%   }
  % \caption{\label{tbl:main_result_numhg}
  %   Comparative analysis of numerical accuracy and textual quality metrics for baseline methods and our proposed approach on the \textbf{NumHG} dataset.
  % }
% \caption{\label{tbl:main_result_numhg_inc_gpt4o}
%     Numerical accuracy (\%) and textual quality score (\%) for GPT-4o using zero-shot and five-shot prompting in \textbf{NumHG}. Experimental results shows the performance of GPT-4o with and without TEN rationales in prompts.
%   }
% \end{table*}



% \begin{table*}[htbp]
%   \centering
%   \resizebox{0.82\textwidth}{!}{
%   \begin{tabular}{lc|ccc|ccc|c}
%     \hline
%     &\textbf{Num Acc}&&\textbf{ROUGE}&&&\textbf{BERTScore}&&\textbf{MoverScore}\\
%     \cline{2-9}
%      & Overall & 1 & 2 & L & P & R & F1 & \\
%     \hline
%     Zero-shot w/o TEN & 6.23 & 22.59 & 4.98 & 17.77 & 19.70 & 25.71 & 22.76 & 54.58 \\
%     Zero-shot w TEN & 9.91 & 22.18 & 5.14 & 17.65 & 21.43 & 24.73 & 23.14 & 54.63 \\
%     Five-shot w/o TEN & 16.76 & 29.09 & 8.88 & 23.18 & 31.78 & 35.63 & 33.75 & 56.74 \\
%     Five-shot w TEN & 10.41 & 25.82 & 7.04 & 20.37 & 34.75 & 34.20 & 34.53 & 55.83 \\   
%     \hline
%     % \hline
%   \end{tabular}
%   }
%   % \caption{\label{tbl:main_result_xsum}
%   %   Comparative analysis of numerical accuracy and textual quality metrics for baseline methods and our proposed approach on the \textbf{XSum} dataset.
%   % }
%   \caption{\label{tbl:main_result_xsum_inc_gpt4o}
%     Numerical accuracy (\%) and textual quality score (\%) for GPT-4o using zero-shot and five-shot prompting in \textbf{XSum}. Experimental results shows the performance of GPT-4o with and without TEN rationales in prompts.
%   }
% \end{table*}



%\noindent \textbf{How effective student LLM has learned from teacher LLM.}
We further conducted experiments to evaluate if the student LLM can effectively learn rationale generation from the teacher LLM. 
On the test data, we computed the textual and semantic similarity scores for the rationales automatically generated by Mistral-7B-v0.3 and Llama-3.1-8B as the student model, against the supervision data generated by the teacher LLM GPT-4o. Table~\ref{tbl:rationale_evaluation} illustrates the evaluation results. The high textual similarity and semantic similarity scores demonstrate that the student model can learn from the teacher model to generate rationales to enhance its reasoning capability for number-focused headline generation.

\begin{table*}[htbp]
  \centering
  \resizebox{\textwidth}{!}{
  \begin{tabular}{l|ccc|ccc}
    \hline
    &\multicolumn{3}{c|}{\textbf{NumHG}} & \multicolumn{3}{c}{\textbf{XSum}}\\
    \hline
     & ROUGE-1 & BERTScore-F1 & MoverScore & ROUGE-1 & BERTScore-F1 & MoverScore\\
    \hline
    Mistral-7B-v0.3 & 84.12 & 81.13 & 71.03 & 75.46 & 70.43 & 66.68\\
    Llama-3.1-8B & 84.02 & 80.90 & 70.91 & 75.35 & 70.18 & 66.69\\
    \hline
  \end{tabular}
  }
  % \caption{\label{tbl:main_result_xsum}
  %   Comparative analysis of numerical accuracy and textual quality metrics for baseline methods and our proposed approach on the \textbf{XSum} dataset.
  % }
  \caption{\label{tbl:rationale_evaluation}
    textual quality metric score (\%) for Student LLM generated rationales against teache LLM generated supervision rationales on \textbf{NumHG} and \textbf{XSum}.
  }
\end{table*}


\subsection{Ablation study}
% Due to Mistral-7B-v0.3's more consistent performance across different benchmark datasets, we selected it as our final student LLM. 

\noindent \textbf{Effect of refining rationales through DPO.} In TEN, we apply DPO to enhance the capability of the student LLM rationale generator for topic alignment and numerical reasoning. 
%to enhance the quality of their outputs. 
To understand the effectiveness of DPO we tested TEN minus($-$) DPO on both Mistral-7B and Llama-3.1-8B. Table~\ref{tbl:ablation} illustrates the results. It can be seen that using a rationale generator improved through DPO leads to higher numerical accuracy and textual quality. On the NumHG dataset, DPO improves the numerical accuracy by 0.96\% with a Mistral-7B-v0.3-based rationale generator, and by 1.83\% with Llama-3.1-8B. On the XSum dataset, DPO enhances the numerical accuracy of Mistral-7B-v0.3 and Llama-3.1-8B by 1.38\% and 0.75\%, respectively. Additionally, DPO enhances the ROUGE scores marginally for both student models across both benchmark datasets. 

\begin{table*}[t]
  \centering
  \resizebox{\textwidth}{!}{
  \begin{tabular}{lcccccccc}
    \hline
    \textbf{Method} & \multicolumn{4}{c}{\textbf{NumHG}} & \multicolumn{4}{c}{\textbf{XSum}} \\
    \hline
    & NumAcc & ROUGE-1 & BERTScore-F1 & MoverScore & NumAcc & ROUGE-1 & BERTScore-F1 & MoverScore \\
    \hline
    Mistral-7B-v0.3 & & & & & & & \\
    \hline
    \: TEN (Ours) & \textbf{77.20} & \textbf{51.14} & \textbf{53.21} & \textbf{61.23} & \textbf{39.07} & 46.63 & \textbf{56.70} & \textbf{61.36} \\
    \: $-$ DPO & 76.24 & 50.95 & 53.12 & \textbf{61.24} & 37.69 & 46.26 & 56.35 & \textbf{61.27}  \\
    % \hline
    \: $-$ DPO $-$ N & 74.04 & 50.23 & 52.59 & \textbf{61.08} & 35.58 & 46.18 & 56.13 & \textbf{61.21}  \\
    \: $-$ DPO $-$ TE & 75.55 & \textbf{51.63} & \textbf{53.67} & \textbf{61.37} & 31.71 & 45.94 & 55.55 & \textbf{61.11}  \\
    \: $-$ DPO $-$ TEN & 70.33 & \textbf{51.27} & \textbf{53.43} & \textbf{61.42} & 30.41 & \textbf{47.37} & \textbf{56.88} & \textbf{61.58} \\
    \hline
    \hline
    Llama-3.1-8B & & & & & & & &  \\
    \hline
    \: TEN (Ours) & \textbf{77.89} & \textbf{51.15} & \textbf{52.83} & \textbf{61.14} & \textbf{37.51} & 46.11 & \textbf{56.09} & \textbf{61.18} \\
    \: $-$ DPO & 76.06 & \textbf{50.69} & 52.59 & \textbf{61.09} & 36.76 & 45.89 & 55.80 & \textbf{61.06}  \\
    % \hline
    \: $-$ DPO $-$ N & 73.80 & 50.05 & 52.24 & \textbf{60.90} & 36.51 & 45.83 & \textbf{56.03} & \textbf{61.11} \\
    \: $-$ DPO $-$ TE  & 74.86 & \textbf{50.98} & \textbf{53.21} & \textbf{61.22} & 32.02 & 45.62 & 55.50 & \textbf{61.03}  \\
    \: $-$ DPO $-$ TEN & 70.71 & \textbf{51.01} & \textbf{53.21} & \textbf{61.34} & 29.35 & \textbf{46.62} & \textbf{56.44} & \textbf{61.43} \\    
    \hline
  \end{tabular}
  }
    % Comparative analysis of LLM performance in headline generation: ablation study. 
  % \caption{\label{tbl:ablation}
  %   Ablation study results comparing the proposed TEN scheme with student models fine-tuned without DPO, topic alignment, or numerical reasoning signals.     
  % }
  \caption{\label{tbl:ablation}
    Results (\%) for ablation study of TEN. Higher numbers indicate better performance. Best results are in bold, where results within 0.5 (\%) difference are deemed comparable.     
  }
\end{table*}


% \begin{table*}[t]
%   \centering
%   \resizebox{0.8\textwidth}{!}{
%   \begin{tabular}{lcccc}
%     \hline
%     \textbf{Method} & \multicolumn{2}{c}{\textbf{NumHG}} & \multicolumn{2}{c}{\textbf{Xsum}} \\
%     \hline
%     & NumAcc & ROUGE-1 & NumAcc & ROUGE-1 \\
%     \hline
%     Mistral-7B-v0.3 & & & & \\
%     \hline
%     \: + No Teaching Signal & 70.33 & 51.27 & 30.41 & 47.37 \\
%     \: + Topic Alignment Only & 74.04 & 50.23 & 35.58 & 46.18 \\
%     \: + Calculation Enhancement Only & 75.56 & 51.63 & 31.71 & 45.94 \\
%     \hline
%     \: + Topic Alignment \& Calculation Enhancement (Ours) & 76.24 & 50.95 & 37.69 & 46.26 \\
%     \hline
%     % \hline
%     Llama-3.1-8B & & & & \\
%     \hline
%     \: + No Teaching Signal & 70.71 & 51.01 & 29.35 & 46.62 \\
%     \: + Topic Alignment & 73.80 & 50.05 & 36.51 & 45.83 \\
%     \: + Calculation Enhancement & 74.86 & 50.98 & 32.02 & 45.62 \\
%     \hline
%     \: + Topic Alignment \& Calculation Enhancement (Ours) & 76.06 & 50.69 & 36.76 & 45.89 \\
%     \hline
%   \end{tabular}
%   }
%   \caption{\label{t4}
%     Comparative analysis of LLM performance in headline generation: impact of different teaching signals.
%   }
% \end{table*}

% \section{Further Analysis}

\noindent \textbf{Effect of different supervision signals}. In TEN, we have developed two types of CoT supervision signals. One focuses on aligning the topic of the generated headline, while the other enhances numerical calculation accuracy. As illustrated in Figure~\ref{fig:three_phase_of_TEN}, the rationales under "Topic" and "Entities" contribute to topic alignment, whereas those under "Numbers mentioned" and "Reasoning steps" boost numerical reasoning. We've assessed the impact of these supervision signals, with results for ``TEN minus Number'' ($-$ N), ``TEN minus Topic and Entity ''($-$ TE), and ''no supervision''($-$ TEN) presented in Table~\ref{tbl:ablation}. It can be seen that both types of signals independently improve numerical accuracy in headline generation. However, their effectiveness varies across the two benchmark datasets: numerical reasoning signals show a more pronounced effect on NumHG, while topic alignment signals have a greater impact on XSum. Notably, combining both types of supervision signals yields optimal model performance with highest numerical accuracy and comparable textual quality. 
% \todo{XF: are we going to discuss -TEN or the result is complicated? If so, ignore my suggestion here.}

% For the NumHG dataset, the overall numerical accuracy of headlines generated by Mistral-7B-v0.3 drops 5.91\% when fine-tuned without any teaching signals, compared to TEN. The accuracy decreases by 2.2\% without number reasoning signals and by 0.68\% without topic alignment teaching signals. Llama-3.1-8B exhibits similar trends, with numerical accuracy declining by 5.35\%, 2.26\%, and 1.2\% when fine-tuned without teaching signals, calculation signals, and topic alignment signals, respectively. For the XSum dataset, unlike the results on NumHG, topic alignment teaching signals have a greater impact on the models' numerical accuracy. Mistral-7B-v0.3 and Llama-3.1-8B experience a decrease in numerical accuracy by 7.28\% and 7.41\% respectively when fine-tuned without any teaching signals. Their numerical accuracy only drops by 2.21\% and 0.25\% when fine-tuned without number reasoning signals. However, they show a more significant decline of 5.98\% and 4.74\% when fine-tuned without topic alignment signals, compared to TEN. 

%\todo[color=green]{XF: suggest mentioned the trend of TEN, not the details. Also, we can highlight the difference between the two datasets. -->done}

\noindent \textbf{TEN performance with different student LLMs.} We also want to highlight the performance of TEN with different base student LLMs. All results in Table~\ref{tbl:ablation} are obtained using GPT-4o as the teacher LLM and two different student LLMs: Mistral-7B and Llama-3.1-8B. 
Observe that Mistral-7B and Llama-3.1 demonstrate similar performance for both numerical accuracy and textual quality. 
It can also be seen that the supervision signals and DPO show their effectiveness for both student LLMs. 
% We also conducted experiments using two different types of student models: Mistral-7B-v0.3 and Llama-3.1-8B. The results are shown in Table~\ref{tbl:ablation}. When fine-tuned with TEN, Mistral-7B-v0.3 achieved an overall accuracy of 77.20\% on NumHG with a ROUGE-1 score of 51.14; and 39.07\% on XSum with a ROUGE-1 score of 46.63. Llama-3.1-8B, on the other hand, attained an overall accuracy of 77.89\% on NumHG with a ROUGE-1 score of 51.15; and 37.51\% on XSum with a ROUGE-1 score of 46.03. It can be found that our proposed TEN scheme persists in its efficiency across different LLMs.


% \begin{figure}[t!]
%   \includegraphics[width=\columnwidth]{latex/error_analysis_numhg.pdf} 
%   % Comparative analysis of error rates across different mathematical operations in number-focused headline generation
% %  \caption {The error analysis across different mathematical operations in number-focused headline generation.}
%   \caption {Error analysis across different mathematical operations on NumHG}
%   \label{fig:error_analysis}
% \end{figure}

\begin{figure*}[t!]
  \includegraphics[width=0.48\linewidth]{case_study_2} \hfill
  \includegraphics[width=0.48\linewidth]{case_study_3}
  \parbox[b]{0.45\textwidth}{\centering (a) Topic alignment}
  \hspace{0.05\textwidth}
  \parbox[b]{0.45\textwidth}{\centering (b) Numerical reasoning }  
%  \caption {Comparison between NCL\_NLP (Baseline) and our proposed TEN approach for rationale and headline generation}
  \caption {\label{fig:case_study} TEN vs. NCL (Baseline) for rationale and headline generation}
\end{figure*}

%\subsection{Discussions}
\subsection{Error analysis and case study}
\begin{table*}[!h]
    \centering
    \resizebox{0.9\textwidth}{!}{
    \begin{tabular}{ccccccccccc}
    \hline
    Operation & Overall & Copy & Trans & Paraphrase & Round & Subtract & Add & Span & Divide & Multiply \\
    Count & 3996 & 2,494 & 682 & 376 & 133 & 89 & 76 & 85 & 28 & 33 \\
    \hline
    BART (Err\%) & 31.53 & 23.46 & 34.02 & 26.60 & 60.90 & 96.63 & 78.95 & 68.24 & 96.43 & 93.94 \\
    NPP (Err\%) & 27.55 & 22.57 & 30.06 & 21.81 & \textbf{40.60} & \textbf{68.54} & \textbf{56.58} & \textbf{49.41} & \textbf{82.14} & 84.85 \\
    NCL (Err\%) & 28.08 & 21.57 & 29.62 & 20.74 & 48.12 & 79.78 & 75.00 & 64.71 & 96.43 & 90.91 \\
    TEN (Err\%) & \textbf{25.40} & \textbf{18.89} & \textbf{27.57} & \textbf{20.21} & 48.12 & 75.28 & 60.53 & 58.82 & 92.86 & \textbf{81.82} \\
    \hline
    \end{tabular}
}
  \caption{\label{tbl:error_analysis}
    Error analysis across different mathematical operations on test data from NumHG. Lower numbers indicate better performance. Best results are in bold, where results within 0.5\% difference are deemed comparable. 
  }
\end{table*}

\noindent \textbf{Error analysis.} Utilizing the annotations from the NumHG dataset, which outlines nine types of operations necessary for calculating numerals in headlines, we conducted an error analysis for TEN in comparison to the baselines. We present the error rates in Table~\ref{tbl:error_analysis}. 
%models trained with our proposed approach. 
%We examined errors related to the nine operation types. 
The experimental results demonstrate that our approach significantly reduces errors in copying, translating, and paraphrasing, achieving the lowest error rates compared to baseline methods. These three operations represent over 88\% of the total. For the remaining less frequent operations, our approach achieves error rates comparable to the best-performing baseline.
%\todo[color=green]{Figure~\ref{fig:error_analysis} np\_problem to NP-Problem --> done}
%jz3: some more detailed discussions are needed here. 
%jz3: Give some numbers. How many "copy" operations in total and the errors should be relative to the total operations, explain the total operations for each type and the percentage. 
% TEN has the largest number (%) of errors in Copy but NP, NCL has more errors in X, Y operations.

\noindent \textbf{Case study.} Two examples are selected from the test dataset to %demonstrate the improvements in topic alignment and numerical reasoning brought by our proposed TEN approach. 
illustrate the benefits of the TEN reasoning strategy, compared against the NCL baseline, which generates rationale without the TEN structured rationales. 
%We compare the different outputs between NCL (baseline) and TEN (ours). 
Figure~\ref{fig:case_study} (a) shows that TEN correctly identifies the topic the headline should focus on in the rationale, which is the rank of the tornado in this case, while NCL mistakenly focuses on the elevation. In Figure~\ref{fig:case_study} (b), TEN successfully calculates the number of people who died by adding 1 Australian tourist and 3 Tibetans, while NCL fails to count the Australian tourist. 

\section{Conclusion}
In this paper
%we presented  
%a novel fine-tuning scheme for number-focused headline generation named TEN (Topic, Entities, and numerical reasoning). 
we studied number-focused news headline generation, 
a problem presenting the unique challenge of high textual quality with precise number accuracy for LLM generation. 
We proposed a novel framework of using rationales of key elements Topic, Entity, and Numerical reasoning (TEN) to enhance the capability of LLMs for topic alignment and numerical reasoning in headline generation.   
We developed an approach to fine-tune LLMs to automatically generate TEN rationales for numerical headlines generation. 
Especially our TEN approach builds upon the teacher-student rationale-augmented training framework, where a teacher LLM automatically generate TEN rationales as supervision data to teach a student LLM rationale generator and a student LLM headline generator.    
%extends it to further refine automatically generated rationales specifically designed for the headline generation task. 
Experiments on popular numerical news headline generation datasets showed that TEN outperforms existing approaches, achieving higher numerical accuracy and mostly better textual quality for headline generation. 
%across both NumHG and XSum datasets. 
%The decomposition of rationales into topic alignment and numerical reasoning components proves highly effective, with their combination yielding optimal results. Further refinement of generated rationales through DPO leads to additional improvements in numerical accuracy and textual quality. 
%By effectively addressing the challenges of topic alignment and numerical calculation, our method paves the way for more accurate and reliable headline generation. 

\section*{Acknowledgements}
This research is supported in part by the Australian Research Council Discovery Project DP200101441.

\section*{Limitations}

%We employ only one teacher LLM (GPT-4o), so our proposed approach heavily relies on GPT-4o's performance when generating the supervision data. 
%For the student LLMs, 
One limitation of our study is that due to computing resource limitation, we have only applied parameter-efficient technique QLoRA \citep{dettmers_qlora_2023} to fine-tune student LLMs, 
and as such
%As we have not conducted full-parameter tuning,  
it is possible that we 
have not fully elicited the capability of student LLMs. 
%On For refining the rationales, we've tested DPO techniques but haven't explored other approaches such as reinforcement learning and verification. 
Another limitation of our study is the limited data for experiments. 
To our best knowledge NumHG is the only public benchmark dataset for numerical headline generation, and we constructed one more dataset based on XSum for extreme summarization.  
%There's also a limitation with the XSum benchmark dataset. This dataset is designed for extreme summarization, not specifically for headline generation, and contains considerable noise, which isn't ideal for machine learning.

% Bibliography entries for the entire Anthology, followed by custom entries
%\bibliography{anthology,custom}
% Custom bibliography entries only
% \bibliography{main}
% This must be in the first 5 lines to tell arXiv to use pdfLaTeX, which is strongly recommended.
\pdfoutput=1
% In particular, the hyperref package requires pdfLaTeX in order to break URLs across lines.

\documentclass[11pt]{article}

% Change "review" to "final" to generate the final (sometimes called camera-ready) version.
% Change to "preprint" to generate a non-anonymous version with page numbers.
\usepackage{acl}

% Standard package includes
\usepackage{times}
\usepackage{latexsym}

% Draw tables
\usepackage{booktabs}
\usepackage{multirow}
\usepackage{xcolor}
\usepackage{colortbl}
\usepackage{array} 
\usepackage{amsmath}

\newcolumntype{C}{>{\centering\arraybackslash}p{0.07\textwidth}}
% For proper rendering and hyphenation of words containing Latin characters (including in bib files)
\usepackage[T1]{fontenc}
% For Vietnamese characters
% \usepackage[T5]{fontenc}
% See https://www.latex-project.org/help/documentation/encguide.pdf for other character sets
% This assumes your files are encoded as UTF8
\usepackage[utf8]{inputenc}

% This is not strictly necessary, and may be commented out,
% but it will improve the layout of the manuscript,
% and will typically save some space.
\usepackage{microtype}
\DeclareMathOperator*{\argmax}{arg\,max}
% This is also not strictly necessary, and may be commented out.
% However, it will improve the aesthetics of text in
% the typewriter font.
\usepackage{inconsolata}

%Including images in your LaTeX document requires adding
%additional package(s)
\usepackage{graphicx}
% If the title and author information does not fit in the area allocated, uncomment the following
%
%\setlength\titlebox{<dim>}
%
% and set <dim> to something 5cm or larger.

\title{Wi-Chat: Large Language Model Powered Wi-Fi Sensing}

% Author information can be set in various styles:
% For several authors from the same institution:
% \author{Author 1 \and ... \and Author n \\
%         Address line \\ ... \\ Address line}
% if the names do not fit well on one line use
%         Author 1 \\ {\bf Author 2} \\ ... \\ {\bf Author n} \\
% For authors from different institutions:
% \author{Author 1 \\ Address line \\  ... \\ Address line
%         \And  ... \And
%         Author n \\ Address line \\ ... \\ Address line}
% To start a separate ``row'' of authors use \AND, as in
% \author{Author 1 \\ Address line \\  ... \\ Address line
%         \AND
%         Author 2 \\ Address line \\ ... \\ Address line \And
%         Author 3 \\ Address line \\ ... \\ Address line}

% \author{First Author \\
%   Affiliation / Address line 1 \\
%   Affiliation / Address line 2 \\
%   Affiliation / Address line 3 \\
%   \texttt{email@domain} \\\And
%   Second Author \\
%   Affiliation / Address line 1 \\
%   Affiliation / Address line 2 \\
%   Affiliation / Address line 3 \\
%   \texttt{email@domain} \\}
% \author{Haohan Yuan \qquad Haopeng Zhang\thanks{corresponding author} \\ 
%   ALOHA Lab, University of Hawaii at Manoa \\
%   % Affiliation / Address line 2 \\
%   % Affiliation / Address line 3 \\
%   \texttt{\{haohany,haopengz\}@hawaii.edu}}
  
\author{
{Haopeng Zhang$\dag$\thanks{These authors contributed equally to this work.}, Yili Ren$\ddagger$\footnotemark[1], Haohan Yuan$\dag$, Jingzhe Zhang$\ddagger$, Yitong Shen$\ddagger$} \\
ALOHA Lab, University of Hawaii at Manoa$\dag$, University of South Florida$\ddagger$ \\
\{haopengz, haohany\}@hawaii.edu\\
\{yiliren, jingzhe, shen202\}@usf.edu\\}



  
%\author{
%  \textbf{First Author\textsuperscript{1}},
%  \textbf{Second Author\textsuperscript{1,2}},
%  \textbf{Third T. Author\textsuperscript{1}},
%  \textbf{Fourth Author\textsuperscript{1}},
%\\
%  \textbf{Fifth Author\textsuperscript{1,2}},
%  \textbf{Sixth Author\textsuperscript{1}},
%  \textbf{Seventh Author\textsuperscript{1}},
%  \textbf{Eighth Author \textsuperscript{1,2,3,4}},
%\\
%  \textbf{Ninth Author\textsuperscript{1}},
%  \textbf{Tenth Author\textsuperscript{1}},
%  \textbf{Eleventh E. Author\textsuperscript{1,2,3,4,5}},
%  \textbf{Twelfth Author\textsuperscript{1}},
%\\
%  \textbf{Thirteenth Author\textsuperscript{3}},
%  \textbf{Fourteenth F. Author\textsuperscript{2,4}},
%  \textbf{Fifteenth Author\textsuperscript{1}},
%  \textbf{Sixteenth Author\textsuperscript{1}},
%\\
%  \textbf{Seventeenth S. Author\textsuperscript{4,5}},
%  \textbf{Eighteenth Author\textsuperscript{3,4}},
%  \textbf{Nineteenth N. Author\textsuperscript{2,5}},
%  \textbf{Twentieth Author\textsuperscript{1}}
%\\
%\\
%  \textsuperscript{1}Affiliation 1,
%  \textsuperscript{2}Affiliation 2,
%  \textsuperscript{3}Affiliation 3,
%  \textsuperscript{4}Affiliation 4,
%  \textsuperscript{5}Affiliation 5
%\\
%  \small{
%    \textbf{Correspondence:} \href{mailto:email@domain}{email@domain}
%  }
%}

\begin{document}
\maketitle
\begin{abstract}
Recent advancements in Large Language Models (LLMs) have demonstrated remarkable capabilities across diverse tasks. However, their potential to integrate physical model knowledge for real-world signal interpretation remains largely unexplored. In this work, we introduce Wi-Chat, the first LLM-powered Wi-Fi-based human activity recognition system. We demonstrate that LLMs can process raw Wi-Fi signals and infer human activities by incorporating Wi-Fi sensing principles into prompts. Our approach leverages physical model insights to guide LLMs in interpreting Channel State Information (CSI) data without traditional signal processing techniques. Through experiments on real-world Wi-Fi datasets, we show that LLMs exhibit strong reasoning capabilities, achieving zero-shot activity recognition. These findings highlight a new paradigm for Wi-Fi sensing, expanding LLM applications beyond conventional language tasks and enhancing the accessibility of wireless sensing for real-world deployments.
\end{abstract}

\section{Introduction}

In today’s rapidly evolving digital landscape, the transformative power of web technologies has redefined not only how services are delivered but also how complex tasks are approached. Web-based systems have become increasingly prevalent in risk control across various domains. This widespread adoption is due their accessibility, scalability, and ability to remotely connect various types of users. For example, these systems are used for process safety management in industry~\cite{kannan2016web}, safety risk early warning in urban construction~\cite{ding2013development}, and safe monitoring of infrastructural systems~\cite{repetto2018web}. Within these web-based risk management systems, the source search problem presents a huge challenge. Source search refers to the task of identifying the origin of a risky event, such as a gas leak and the emission point of toxic substances. This source search capability is crucial for effective risk management and decision-making.

Traditional approaches to implementing source search capabilities into the web systems often rely on solely algorithmic solutions~\cite{ristic2016study}. These methods, while relatively straightforward to implement, often struggle to achieve acceptable performances due to algorithmic local optima and complex unknown environments~\cite{zhao2020searching}. More recently, web crowdsourcing has emerged as a promising alternative for tackling the source search problem by incorporating human efforts in these web systems on-the-fly~\cite{zhao2024user}. This approach outsources the task of addressing issues encountered during the source search process to human workers, leveraging their capabilities to enhance system performance.

These solutions often employ a human-AI collaborative way~\cite{zhao2023leveraging} where algorithms handle exploration-exploitation and report the encountered problems while human workers resolve complex decision-making bottlenecks to help the algorithms getting rid of local deadlocks~\cite{zhao2022crowd}. Although effective, this paradigm suffers from two inherent limitations: increased operational costs from continuous human intervention, and slow response times of human workers due to sequential decision-making. These challenges motivate our investigation into developing autonomous systems that preserve human-like reasoning capabilities while reducing dependency on massive crowdsourced labor.

Furthermore, recent advancements in large language models (LLMs)~\cite{chang2024survey} and multi-modal LLMs (MLLMs)~\cite{huang2023chatgpt} have unveiled promising avenues for addressing these challenges. One clear opportunity involves the seamless integration of visual understanding and linguistic reasoning for robust decision-making in search tasks. However, whether large models-assisted source search is really effective and efficient for improving the current source search algorithms~\cite{ji2022source} remains unknown. \textit{To address the research gap, we are particularly interested in answering the following two research questions in this work:}

\textbf{\textit{RQ1: }}How can source search capabilities be integrated into web-based systems to support decision-making in time-sensitive risk management scenarios? 
% \sq{I mention ``time-sensitive'' here because I feel like we shall say something about the response time -- LLM has to be faster than humans}

\textbf{\textit{RQ2: }}How can MLLMs and LLMs enhance the effectiveness and efficiency of existing source search algorithms? 

% \textit{\textbf{RQ2:}} To what extent does the performance of large models-assisted search align with or approach the effectiveness of human-AI collaborative search? 

To answer the research questions, we propose a novel framework called Auto-\
S$^2$earch (\textbf{Auto}nomous \textbf{S}ource \textbf{Search}) and implement a prototype system that leverages advanced web technologies to simulate real-world conditions for zero-shot source search. Unlike traditional methods that rely on pre-defined heuristics or extensive human intervention, AutoS$^2$earch employs a carefully designed prompt that encapsulates human rationales, thereby guiding the MLLM to generate coherent and accurate scene descriptions from visual inputs about four directional choices. Based on these language-based descriptions, the LLM is enabled to determine the optimal directional choice through chain-of-thought (CoT) reasoning. Comprehensive empirical validation demonstrates that AutoS$^2$-\ 
earch achieves a success rate of 95–98\%, closely approaching the performance of human-AI collaborative search across 20 benchmark scenarios~\cite{zhao2023leveraging}. 

Our work indicates that the role of humans in future web crowdsourcing tasks may evolve from executors to validators or supervisors. Furthermore, incorporating explanations of LLM decisions into web-based system interfaces has the potential to help humans enhance task performance in risk control.






\section{Related Work}
\label{sec:relatedworks}

% \begin{table*}[t]
% \centering 
% \renewcommand\arraystretch{0.98}
% \fontsize{8}{10}\selectfont \setlength{\tabcolsep}{0.4em}
% \begin{tabular}{@{}lc|cc|cc|cc@{}}
% \toprule
% \textbf{Methods}           & \begin{tabular}[c]{@{}c@{}}\textbf{Training}\\ \textbf{Paradigm}\end{tabular} & \begin{tabular}[c]{@{}c@{}}\textbf{$\#$ PT Data}\\ \textbf{(Tokens)}\end{tabular} & \begin{tabular}[c]{@{}c@{}}\textbf{$\#$ IFT Data}\\ \textbf{(Samples)}\end{tabular} & \textbf{Code}  & \begin{tabular}[c]{@{}c@{}}\textbf{Natural}\\ \textbf{Language}\end{tabular} & \begin{tabular}[c]{@{}c@{}}\textbf{Action}\\ \textbf{Trajectories}\end{tabular} & \begin{tabular}[c]{@{}c@{}}\textbf{API}\\ \textbf{Documentation}\end{tabular}\\ \midrule 
% NexusRaven~\citep{srinivasan2023nexusraven} & IFT & - & - & \textcolor{green}{\CheckmarkBold} & \textcolor{green}{\CheckmarkBold} &\textcolor{red}{\XSolidBrush}&\textcolor{red}{\XSolidBrush}\\
% AgentInstruct~\citep{zeng2023agenttuning} & IFT & - & 2k & \textcolor{green}{\CheckmarkBold} & \textcolor{green}{\CheckmarkBold} &\textcolor{red}{\XSolidBrush}&\textcolor{red}{\XSolidBrush} \\
% AgentEvol~\citep{xi2024agentgym} & IFT & - & 14.5k & \textcolor{green}{\CheckmarkBold} & \textcolor{green}{\CheckmarkBold} &\textcolor{green}{\CheckmarkBold}&\textcolor{red}{\XSolidBrush} \\
% Gorilla~\citep{patil2023gorilla}& IFT & - & 16k & \textcolor{green}{\CheckmarkBold} & \textcolor{green}{\CheckmarkBold} &\textcolor{red}{\XSolidBrush}&\textcolor{green}{\CheckmarkBold}\\
% OpenFunctions-v2~\citep{patil2023gorilla} & IFT & - & 65k & \textcolor{green}{\CheckmarkBold} & \textcolor{green}{\CheckmarkBold} &\textcolor{red}{\XSolidBrush}&\textcolor{green}{\CheckmarkBold}\\
% LAM~\citep{zhang2024agentohana} & IFT & - & 42.6k & \textcolor{green}{\CheckmarkBold} & \textcolor{green}{\CheckmarkBold} &\textcolor{green}{\CheckmarkBold}&\textcolor{red}{\XSolidBrush} \\
% xLAM~\citep{liu2024apigen} & IFT & - & 60k & \textcolor{green}{\CheckmarkBold} & \textcolor{green}{\CheckmarkBold} &\textcolor{green}{\CheckmarkBold}&\textcolor{red}{\XSolidBrush} \\\midrule
% LEMUR~\citep{xu2024lemur} & PT & 90B & 300k & \textcolor{green}{\CheckmarkBold} & \textcolor{green}{\CheckmarkBold} &\textcolor{green}{\CheckmarkBold}&\textcolor{red}{\XSolidBrush}\\
% \rowcolor{teal!12} \method & PT & 103B & 95k & \textcolor{green}{\CheckmarkBold} & \textcolor{green}{\CheckmarkBold} & \textcolor{green}{\CheckmarkBold} & \textcolor{green}{\CheckmarkBold} \\
% \bottomrule
% \end{tabular}
% \caption{Summary of existing tuning- and pretraining-based LLM agents with their training sample sizes. "PT" and "IFT" denote "Pre-Training" and "Instruction Fine-Tuning", respectively. }
% \label{tab:related}
% \end{table*}

\begin{table*}[ht]
\begin{threeparttable}
\centering 
\renewcommand\arraystretch{0.98}
\fontsize{7}{9}\selectfont \setlength{\tabcolsep}{0.2em}
\begin{tabular}{@{}l|c|c|ccc|cc|cc|cccc@{}}
\toprule
\textbf{Methods} & \textbf{Datasets}           & \begin{tabular}[c]{@{}c@{}}\textbf{Training}\\ \textbf{Paradigm}\end{tabular} & \begin{tabular}[c]{@{}c@{}}\textbf{\# PT Data}\\ \textbf{(Tokens)}\end{tabular} & \begin{tabular}[c]{@{}c@{}}\textbf{\# IFT Data}\\ \textbf{(Samples)}\end{tabular} & \textbf{\# APIs} & \textbf{Code}  & \begin{tabular}[c]{@{}c@{}}\textbf{Nat.}\\ \textbf{Lang.}\end{tabular} & \begin{tabular}[c]{@{}c@{}}\textbf{Action}\\ \textbf{Traj.}\end{tabular} & \begin{tabular}[c]{@{}c@{}}\textbf{API}\\ \textbf{Doc.}\end{tabular} & \begin{tabular}[c]{@{}c@{}}\textbf{Func.}\\ \textbf{Call}\end{tabular} & \begin{tabular}[c]{@{}c@{}}\textbf{Multi.}\\ \textbf{Step}\end{tabular}  & \begin{tabular}[c]{@{}c@{}}\textbf{Plan}\\ \textbf{Refine}\end{tabular}  & \begin{tabular}[c]{@{}c@{}}\textbf{Multi.}\\ \textbf{Turn}\end{tabular}\\ \midrule 
\multicolumn{13}{l}{\emph{Instruction Finetuning-based LLM Agents for Intrinsic Reasoning}}  \\ \midrule
FireAct~\cite{chen2023fireact} & FireAct & IFT & - & 2.1K & 10 & \textcolor{red}{\XSolidBrush} &\textcolor{green}{\CheckmarkBold} &\textcolor{green}{\CheckmarkBold}  & \textcolor{red}{\XSolidBrush} &\textcolor{green}{\CheckmarkBold} & \textcolor{red}{\XSolidBrush} &\textcolor{green}{\CheckmarkBold} & \textcolor{red}{\XSolidBrush} \\
ToolAlpaca~\cite{tang2023toolalpaca} & ToolAlpaca & IFT & - & 4.0K & 400 & \textcolor{red}{\XSolidBrush} &\textcolor{green}{\CheckmarkBold} &\textcolor{green}{\CheckmarkBold} & \textcolor{red}{\XSolidBrush} &\textcolor{green}{\CheckmarkBold} & \textcolor{red}{\XSolidBrush}  &\textcolor{green}{\CheckmarkBold} & \textcolor{red}{\XSolidBrush}  \\
ToolLLaMA~\cite{qin2023toolllm} & ToolBench & IFT & - & 12.7K & 16,464 & \textcolor{red}{\XSolidBrush} &\textcolor{green}{\CheckmarkBold} &\textcolor{green}{\CheckmarkBold} &\textcolor{red}{\XSolidBrush} &\textcolor{green}{\CheckmarkBold}&\textcolor{green}{\CheckmarkBold}&\textcolor{green}{\CheckmarkBold} &\textcolor{green}{\CheckmarkBold}\\
AgentEvol~\citep{xi2024agentgym} & AgentTraj-L & IFT & - & 14.5K & 24 &\textcolor{red}{\XSolidBrush} & \textcolor{green}{\CheckmarkBold} &\textcolor{green}{\CheckmarkBold}&\textcolor{red}{\XSolidBrush} &\textcolor{green}{\CheckmarkBold}&\textcolor{red}{\XSolidBrush} &\textcolor{red}{\XSolidBrush} &\textcolor{green}{\CheckmarkBold}\\
Lumos~\cite{yin2024agent} & Lumos & IFT  & - & 20.0K & 16 &\textcolor{red}{\XSolidBrush} & \textcolor{green}{\CheckmarkBold} & \textcolor{green}{\CheckmarkBold} &\textcolor{red}{\XSolidBrush} & \textcolor{green}{\CheckmarkBold} & \textcolor{green}{\CheckmarkBold} &\textcolor{red}{\XSolidBrush} & \textcolor{green}{\CheckmarkBold}\\
Agent-FLAN~\cite{chen2024agent} & Agent-FLAN & IFT & - & 24.7K & 20 &\textcolor{red}{\XSolidBrush} & \textcolor{green}{\CheckmarkBold} & \textcolor{green}{\CheckmarkBold} &\textcolor{red}{\XSolidBrush} & \textcolor{green}{\CheckmarkBold}& \textcolor{green}{\CheckmarkBold}&\textcolor{red}{\XSolidBrush} & \textcolor{green}{\CheckmarkBold}\\
AgentTuning~\citep{zeng2023agenttuning} & AgentInstruct & IFT & - & 35.0K & - &\textcolor{red}{\XSolidBrush} & \textcolor{green}{\CheckmarkBold} & \textcolor{green}{\CheckmarkBold} &\textcolor{red}{\XSolidBrush} & \textcolor{green}{\CheckmarkBold} &\textcolor{red}{\XSolidBrush} &\textcolor{red}{\XSolidBrush} & \textcolor{green}{\CheckmarkBold}\\\midrule
\multicolumn{13}{l}{\emph{Instruction Finetuning-based LLM Agents for Function Calling}} \\\midrule
NexusRaven~\citep{srinivasan2023nexusraven} & NexusRaven & IFT & - & - & 116 & \textcolor{green}{\CheckmarkBold} & \textcolor{green}{\CheckmarkBold}  & \textcolor{green}{\CheckmarkBold} &\textcolor{red}{\XSolidBrush} & \textcolor{green}{\CheckmarkBold} &\textcolor{red}{\XSolidBrush} &\textcolor{red}{\XSolidBrush}&\textcolor{red}{\XSolidBrush}\\
Gorilla~\citep{patil2023gorilla} & Gorilla & IFT & - & 16.0K & 1,645 & \textcolor{green}{\CheckmarkBold} &\textcolor{red}{\XSolidBrush} &\textcolor{red}{\XSolidBrush}&\textcolor{green}{\CheckmarkBold} &\textcolor{green}{\CheckmarkBold} &\textcolor{red}{\XSolidBrush} &\textcolor{red}{\XSolidBrush} &\textcolor{red}{\XSolidBrush}\\
OpenFunctions-v2~\citep{patil2023gorilla} & OpenFunctions-v2 & IFT & - & 65.0K & - & \textcolor{green}{\CheckmarkBold} & \textcolor{green}{\CheckmarkBold} &\textcolor{red}{\XSolidBrush} &\textcolor{green}{\CheckmarkBold} &\textcolor{green}{\CheckmarkBold} &\textcolor{red}{\XSolidBrush} &\textcolor{red}{\XSolidBrush} &\textcolor{red}{\XSolidBrush}\\
API Pack~\cite{guo2024api} & API Pack & IFT & - & 1.1M & 11,213 &\textcolor{green}{\CheckmarkBold} &\textcolor{red}{\XSolidBrush} &\textcolor{green}{\CheckmarkBold} &\textcolor{red}{\XSolidBrush} &\textcolor{green}{\CheckmarkBold} &\textcolor{red}{\XSolidBrush}&\textcolor{red}{\XSolidBrush}&\textcolor{red}{\XSolidBrush}\\ 
LAM~\citep{zhang2024agentohana} & AgentOhana & IFT & - & 42.6K & - & \textcolor{green}{\CheckmarkBold} & \textcolor{green}{\CheckmarkBold} &\textcolor{green}{\CheckmarkBold}&\textcolor{red}{\XSolidBrush} &\textcolor{green}{\CheckmarkBold}&\textcolor{red}{\XSolidBrush}&\textcolor{green}{\CheckmarkBold}&\textcolor{green}{\CheckmarkBold}\\
xLAM~\citep{liu2024apigen} & APIGen & IFT & - & 60.0K & 3,673 & \textcolor{green}{\CheckmarkBold} & \textcolor{green}{\CheckmarkBold} &\textcolor{green}{\CheckmarkBold}&\textcolor{red}{\XSolidBrush} &\textcolor{green}{\CheckmarkBold}&\textcolor{red}{\XSolidBrush}&\textcolor{green}{\CheckmarkBold}&\textcolor{green}{\CheckmarkBold}\\\midrule
\multicolumn{13}{l}{\emph{Pretraining-based LLM Agents}}  \\\midrule
% LEMUR~\citep{xu2024lemur} & PT & 90B & 300.0K & - & \textcolor{green}{\CheckmarkBold} & \textcolor{green}{\CheckmarkBold} &\textcolor{green}{\CheckmarkBold}&\textcolor{red}{\XSolidBrush} & \textcolor{red}{\XSolidBrush} &\textcolor{green}{\CheckmarkBold} &\textcolor{red}{\XSolidBrush}&\textcolor{red}{\XSolidBrush}\\
\rowcolor{teal!12} \method & \dataset & PT & 103B & 95.0K  & 76,537  & \textcolor{green}{\CheckmarkBold} & \textcolor{green}{\CheckmarkBold} & \textcolor{green}{\CheckmarkBold} & \textcolor{green}{\CheckmarkBold} & \textcolor{green}{\CheckmarkBold} & \textcolor{green}{\CheckmarkBold} & \textcolor{green}{\CheckmarkBold} & \textcolor{green}{\CheckmarkBold}\\
\bottomrule
\end{tabular}
% \begin{tablenotes}
%     \item $^*$ In addition, the StarCoder-API can offer 4.77M more APIs.
% \end{tablenotes}
\caption{Summary of existing instruction finetuning-based LLM agents for intrinsic reasoning and function calling, along with their training resources and sample sizes. "PT" and "IFT" denote "Pre-Training" and "Instruction Fine-Tuning", respectively.}
\vspace{-2ex}
\label{tab:related}
\end{threeparttable}
\end{table*}

\noindent \textbf{Prompting-based LLM Agents.} Due to the lack of agent-specific pre-training corpus, existing LLM agents rely on either prompt engineering~\cite{hsieh2023tool,lu2024chameleon,yao2022react,wang2023voyager} or instruction fine-tuning~\cite{chen2023fireact,zeng2023agenttuning} to understand human instructions, decompose high-level tasks, generate grounded plans, and execute multi-step actions. 
However, prompting-based methods mainly depend on the capabilities of backbone LLMs (usually commercial LLMs), failing to introduce new knowledge and struggling to generalize to unseen tasks~\cite{sun2024adaplanner,zhuang2023toolchain}. 

\noindent \textbf{Instruction Finetuning-based LLM Agents.} Considering the extensive diversity of APIs and the complexity of multi-tool instructions, tool learning inherently presents greater challenges than natural language tasks, such as text generation~\cite{qin2023toolllm}.
Post-training techniques focus more on instruction following and aligning output with specific formats~\cite{patil2023gorilla,hao2024toolkengpt,qin2023toolllm,schick2024toolformer}, rather than fundamentally improving model knowledge or capabilities. 
Moreover, heavy fine-tuning can hinder generalization or even degrade performance in non-agent use cases, potentially suppressing the original base model capabilities~\cite{ghosh2024a}.

\noindent \textbf{Pretraining-based LLM Agents.} While pre-training serves as an essential alternative, prior works~\cite{nijkamp2023codegen,roziere2023code,xu2024lemur,patil2023gorilla} have primarily focused on improving task-specific capabilities (\eg, code generation) instead of general-domain LLM agents, due to single-source, uni-type, small-scale, and poor-quality pre-training data. 
Existing tool documentation data for agent training either lacks diverse real-world APIs~\cite{patil2023gorilla, tang2023toolalpaca} or is constrained to single-tool or single-round tool execution. 
Furthermore, trajectory data mostly imitate expert behavior or follow function-calling rules with inferior planning and reasoning, failing to fully elicit LLMs' capabilities and handle complex instructions~\cite{qin2023toolllm}. 
Given a wide range of candidate API functions, each comprising various function names and parameters available at every planning step, identifying globally optimal solutions and generalizing across tasks remains highly challenging.



\section{Preliminaries}
\label{Preliminaries}
\begin{figure*}[t]
    \centering
    \includegraphics[width=0.95\linewidth]{fig/HealthGPT_Framework.png}
    \caption{The \ourmethod{} architecture integrates hierarchical visual perception and H-LoRA, employing a task-specific hard router to select visual features and H-LoRA plugins, ultimately generating outputs with an autoregressive manner.}
    \label{fig:architecture}
\end{figure*}
\noindent\textbf{Large Vision-Language Models.} 
The input to a LVLM typically consists of an image $x^{\text{img}}$ and a discrete text sequence $x^{\text{txt}}$. The visual encoder $\mathcal{E}^{\text{img}}$ converts the input image $x^{\text{img}}$ into a sequence of visual tokens $\mathcal{V} = [v_i]_{i=1}^{N_v}$, while the text sequence $x^{\text{txt}}$ is mapped into a sequence of text tokens $\mathcal{T} = [t_i]_{i=1}^{N_t}$ using an embedding function $\mathcal{E}^{\text{txt}}$. The LLM $\mathcal{M_\text{LLM}}(\cdot|\theta)$ models the joint probability of the token sequence $\mathcal{U} = \{\mathcal{V},\mathcal{T}\}$, which is expressed as:
\begin{equation}
    P_\theta(R | \mathcal{U}) = \prod_{i=1}^{N_r} P_\theta(r_i | \{\mathcal{U}, r_{<i}\}),
\end{equation}
where $R = [r_i]_{i=1}^{N_r}$ is the text response sequence. The LVLM iteratively generates the next token $r_i$ based on $r_{<i}$. The optimization objective is to minimize the cross-entropy loss of the response $\mathcal{R}$.
% \begin{equation}
%     \mathcal{L}_{\text{VLM}} = \mathbb{E}_{R|\mathcal{U}}\left[-\log P_\theta(R | \mathcal{U})\right]
% \end{equation}
It is worth noting that most LVLMs adopt a design paradigm based on ViT, alignment adapters, and pre-trained LLMs\cite{liu2023llava,liu2024improved}, enabling quick adaptation to downstream tasks.


\noindent\textbf{VQGAN.}
VQGAN~\cite{esser2021taming} employs latent space compression and indexing mechanisms to effectively learn a complete discrete representation of images. VQGAN first maps the input image $x^{\text{img}}$ to a latent representation $z = \mathcal{E}(x)$ through a encoder $\mathcal{E}$. Then, the latent representation is quantized using a codebook $\mathcal{Z} = \{z_k\}_{k=1}^K$, generating a discrete index sequence $\mathcal{I} = [i_m]_{m=1}^N$, where $i_m \in \mathcal{Z}$ represents the quantized code index:
\begin{equation}
    \mathcal{I} = \text{Quantize}(z|\mathcal{Z}) = \arg\min_{z_k \in \mathcal{Z}} \| z - z_k \|_2.
\end{equation}
In our approach, the discrete index sequence $\mathcal{I}$ serves as a supervisory signal for the generation task, enabling the model to predict the index sequence $\hat{\mathcal{I}}$ from input conditions such as text or other modality signals.  
Finally, the predicted index sequence $\hat{\mathcal{I}}$ is upsampled by the VQGAN decoder $G$, generating the high-quality image $\hat{x}^\text{img} = G(\hat{\mathcal{I}})$.



\noindent\textbf{Low Rank Adaptation.} 
LoRA\cite{hu2021lora} effectively captures the characteristics of downstream tasks by introducing low-rank adapters. The core idea is to decompose the bypass weight matrix $\Delta W\in\mathbb{R}^{d^{\text{in}} \times d^{\text{out}}}$ into two low-rank matrices $ \{A \in \mathbb{R}^{d^{\text{in}} \times r}, B \in \mathbb{R}^{r \times d^{\text{out}}} \}$, where $ r \ll \min\{d^{\text{in}}, d^{\text{out}}\} $, significantly reducing learnable parameters. The output with the LoRA adapter for the input $x$ is then given by:
\begin{equation}
    h = x W_0 + \alpha x \Delta W/r = x W_0 + \alpha xAB/r,
\end{equation}
where matrix $ A $ is initialized with a Gaussian distribution, while the matrix $ B $ is initialized as a zero matrix. The scaling factor $ \alpha/r $ controls the impact of $ \Delta W $ on the model.

\section{HealthGPT}
\label{Method}


\subsection{Unified Autoregressive Generation.}  
% As shown in Figure~\ref{fig:architecture}, 
\ourmethod{} (Figure~\ref{fig:architecture}) utilizes a discrete token representation that covers both text and visual outputs, unifying visual comprehension and generation as an autoregressive task. 
For comprehension, $\mathcal{M}_\text{llm}$ receives the input joint sequence $\mathcal{U}$ and outputs a series of text token $\mathcal{R} = [r_1, r_2, \dots, r_{N_r}]$, where $r_i \in \mathcal{V}_{\text{txt}}$, and $\mathcal{V}_{\text{txt}}$ represents the LLM's vocabulary:
\begin{equation}
    P_\theta(\mathcal{R} \mid \mathcal{U}) = \prod_{i=1}^{N_r} P_\theta(r_i \mid \mathcal{U}, r_{<i}).
\end{equation}
For generation, $\mathcal{M}_\text{llm}$ first receives a special start token $\langle \text{START\_IMG} \rangle$, then generates a series of tokens corresponding to the VQGAN indices $\mathcal{I} = [i_1, i_2, \dots, i_{N_i}]$, where $i_j \in \mathcal{V}_{\text{vq}}$, and $\mathcal{V}_{\text{vq}}$ represents the index range of VQGAN. Upon completion of generation, the LLM outputs an end token $\langle \text{END\_IMG} \rangle$:
\begin{equation}
    P_\theta(\mathcal{I} \mid \mathcal{U}) = \prod_{j=1}^{N_i} P_\theta(i_j \mid \mathcal{U}, i_{<j}).
\end{equation}
Finally, the generated index sequence $\mathcal{I}$ is fed into the decoder $G$, which reconstructs the target image $\hat{x}^{\text{img}} = G(\mathcal{I})$.

\subsection{Hierarchical Visual Perception}  
Given the differences in visual perception between comprehension and generation tasks—where the former focuses on abstract semantics and the latter emphasizes complete semantics—we employ ViT to compress the image into discrete visual tokens at multiple hierarchical levels.
Specifically, the image is converted into a series of features $\{f_1, f_2, \dots, f_L\}$ as it passes through $L$ ViT blocks.

To address the needs of various tasks, the hidden states are divided into two types: (i) \textit{Concrete-grained features} $\mathcal{F}^{\text{Con}} = \{f_1, f_2, \dots, f_k\}, k < L$, derived from the shallower layers of ViT, containing sufficient global features, suitable for generation tasks; 
(ii) \textit{Abstract-grained features} $\mathcal{F}^{\text{Abs}} = \{f_{k+1}, f_{k+2}, \dots, f_L\}$, derived from the deeper layers of ViT, which contain abstract semantic information closer to the text space, suitable for comprehension tasks.

The task type $T$ (comprehension or generation) determines which set of features is selected as the input for the downstream large language model:
\begin{equation}
    \mathcal{F}^{\text{img}}_T =
    \begin{cases}
        \mathcal{F}^{\text{Con}}, & \text{if } T = \text{generation task} \\
        \mathcal{F}^{\text{Abs}}, & \text{if } T = \text{comprehension task}
    \end{cases}
\end{equation}
We integrate the image features $\mathcal{F}^{\text{img}}_T$ and text features $\mathcal{T}$ into a joint sequence through simple concatenation, which is then fed into the LLM $\mathcal{M}_{\text{llm}}$ for autoregressive generation.
% :
% \begin{equation}
%     \mathcal{R} = \mathcal{M}_{\text{llm}}(\mathcal{U}|\theta), \quad \mathcal{U} = [\mathcal{F}^{\text{img}}_T; \mathcal{T}]
% \end{equation}
\subsection{Heterogeneous Knowledge Adaptation}
We devise H-LoRA, which stores heterogeneous knowledge from comprehension and generation tasks in separate modules and dynamically routes to extract task-relevant knowledge from these modules. 
At the task level, for each task type $ T $, we dynamically assign a dedicated H-LoRA submodule $ \theta^T $, which is expressed as:
\begin{equation}
    \mathcal{R} = \mathcal{M}_\text{LLM}(\mathcal{U}|\theta, \theta^T), \quad \theta^T = \{A^T, B^T, \mathcal{R}^T_\text{outer}\}.
\end{equation}
At the feature level for a single task, H-LoRA integrates the idea of Mixture of Experts (MoE)~\cite{masoudnia2014mixture} and designs an efficient matrix merging and routing weight allocation mechanism, thus avoiding the significant computational delay introduced by matrix splitting in existing MoELoRA~\cite{luo2024moelora}. Specifically, we first merge the low-rank matrices (rank = r) of $ k $ LoRA experts into a unified matrix:
\begin{equation}
    \mathbf{A}^{\text{merged}}, \mathbf{B}^{\text{merged}} = \text{Concat}(\{A_i\}_1^k), \text{Concat}(\{B_i\}_1^k),
\end{equation}
where $ \mathbf{A}^{\text{merged}} \in \mathbb{R}^{d^\text{in} \times rk} $ and $ \mathbf{B}^{\text{merged}} \in \mathbb{R}^{rk \times d^\text{out}} $. The $k$-dimension routing layer generates expert weights $ \mathcal{W} \in \mathbb{R}^{\text{token\_num} \times k} $ based on the input hidden state $ x $, and these are expanded to $ \mathbb{R}^{\text{token\_num} \times rk} $ as follows:
\begin{equation}
    \mathcal{W}^\text{expanded} = \alpha k \mathcal{W} / r \otimes \mathbf{1}_r,
\end{equation}
where $ \otimes $ denotes the replication operation.
The overall output of H-LoRA is computed as:
\begin{equation}
    \mathcal{O}^\text{H-LoRA} = (x \mathbf{A}^{\text{merged}} \odot \mathcal{W}^\text{expanded}) \mathbf{B}^{\text{merged}},
\end{equation}
where $ \odot $ represents element-wise multiplication. Finally, the output of H-LoRA is added to the frozen pre-trained weights to produce the final output:
\begin{equation}
    \mathcal{O} = x W_0 + \mathcal{O}^\text{H-LoRA}.
\end{equation}
% In summary, H-LoRA is a task-based dynamic PEFT method that achieves high efficiency in single-task fine-tuning.

\subsection{Training Pipeline}

\begin{figure}[t]
    \centering
    \hspace{-4mm}
    \includegraphics[width=0.94\linewidth]{fig/data.pdf}
    \caption{Data statistics of \texttt{VL-Health}. }
    \label{fig:data}
\end{figure}
\noindent \textbf{1st Stage: Multi-modal Alignment.} 
In the first stage, we design separate visual adapters and H-LoRA submodules for medical unified tasks. For the medical comprehension task, we train abstract-grained visual adapters using high-quality image-text pairs to align visual embeddings with textual embeddings, thereby enabling the model to accurately describe medical visual content. During this process, the pre-trained LLM and its corresponding H-LoRA submodules remain frozen. In contrast, the medical generation task requires training concrete-grained adapters and H-LoRA submodules while keeping the LLM frozen. Meanwhile, we extend the textual vocabulary to include multimodal tokens, enabling the support of additional VQGAN vector quantization indices. The model trains on image-VQ pairs, endowing the pre-trained LLM with the capability for image reconstruction. This design ensures pixel-level consistency of pre- and post-LVLM. The processes establish the initial alignment between the LLM’s outputs and the visual inputs.

\noindent \textbf{2nd Stage: Heterogeneous H-LoRA Plugin Adaptation.}  
The submodules of H-LoRA share the word embedding layer and output head but may encounter issues such as bias and scale inconsistencies during training across different tasks. To ensure that the multiple H-LoRA plugins seamlessly interface with the LLMs and form a unified base, we fine-tune the word embedding layer and output head using a small amount of mixed data to maintain consistency in the model weights. Specifically, during this stage, all H-LoRA submodules for different tasks are kept frozen, with only the word embedding layer and output head being optimized. Through this stage, the model accumulates foundational knowledge for unified tasks by adapting H-LoRA plugins.

\begin{table*}[!t]
\centering
\caption{Comparison of \ourmethod{} with other LVLMs and unified multi-modal models on medical visual comprehension tasks. \textbf{Bold} and \underline{underlined} text indicates the best performance and second-best performance, respectively.}
\resizebox{\textwidth}{!}{
\begin{tabular}{c|lcc|cccccccc|c}
\toprule
\rowcolor[HTML]{E9F3FE} &  &  &  & \multicolumn{2}{c}{\textbf{VQA-RAD \textuparrow}} & \multicolumn{2}{c}{\textbf{SLAKE \textuparrow}} & \multicolumn{2}{c}{\textbf{PathVQA \textuparrow}} &  &  &  \\ 
\cline{5-10}
\rowcolor[HTML]{E9F3FE}\multirow{-2}{*}{\textbf{Type}} & \multirow{-2}{*}{\textbf{Model}} & \multirow{-2}{*}{\textbf{\# Params}} & \multirow{-2}{*}{\makecell{\textbf{Medical} \\ \textbf{LVLM}}} & \textbf{close} & \textbf{all} & \textbf{close} & \textbf{all} & \textbf{close} & \textbf{all} & \multirow{-2}{*}{\makecell{\textbf{MMMU} \\ \textbf{-Med}}\textuparrow} & \multirow{-2}{*}{\textbf{OMVQA}\textuparrow} & \multirow{-2}{*}{\textbf{Avg. \textuparrow}} \\ 
\midrule \midrule
\multirow{9}{*}{\textbf{Comp. Only}} 
& Med-Flamingo & 8.3B & \Large \ding{51} & 58.6 & 43.0 & 47.0 & 25.5 & 61.9 & 31.3 & 28.7 & 34.9 & 41.4 \\
& LLaVA-Med & 7B & \Large \ding{51} & 60.2 & 48.1 & 58.4 & 44.8 & 62.3 & 35.7 & 30.0 & 41.3 & 47.6 \\
& HuatuoGPT-Vision & 7B & \Large \ding{51} & 66.9 & 53.0 & 59.8 & 49.1 & 52.9 & 32.0 & 42.0 & 50.0 & 50.7 \\
& BLIP-2 & 6.7B & \Large \ding{55} & 43.4 & 36.8 & 41.6 & 35.3 & 48.5 & 28.8 & 27.3 & 26.9 & 36.1 \\
& LLaVA-v1.5 & 7B & \Large \ding{55} & 51.8 & 42.8 & 37.1 & 37.7 & 53.5 & 31.4 & 32.7 & 44.7 & 41.5 \\
& InstructBLIP & 7B & \Large \ding{55} & 61.0 & 44.8 & 66.8 & 43.3 & 56.0 & 32.3 & 25.3 & 29.0 & 44.8 \\
& Yi-VL & 6B & \Large \ding{55} & 52.6 & 42.1 & 52.4 & 38.4 & 54.9 & 30.9 & 38.0 & 50.2 & 44.9 \\
& InternVL2 & 8B & \Large \ding{55} & 64.9 & 49.0 & 66.6 & 50.1 & 60.0 & 31.9 & \underline{43.3} & 54.5 & 52.5\\
& Llama-3.2 & 11B & \Large \ding{55} & 68.9 & 45.5 & 72.4 & 52.1 & 62.8 & 33.6 & 39.3 & 63.2 & 54.7 \\
\midrule
\multirow{5}{*}{\textbf{Comp. \& Gen.}} 
& Show-o & 1.3B & \Large \ding{55} & 50.6 & 33.9 & 31.5 & 17.9 & 52.9 & 28.2 & 22.7 & 45.7 & 42.6 \\
& Unified-IO 2 & 7B & \Large \ding{55} & 46.2 & 32.6 & 35.9 & 21.9 & 52.5 & 27.0 & 25.3 & 33.0 & 33.8 \\
& Janus & 1.3B & \Large \ding{55} & 70.9 & 52.8 & 34.7 & 26.9 & 51.9 & 27.9 & 30.0 & 26.8 & 33.5 \\
& \cellcolor[HTML]{DAE0FB}HealthGPT-M3 & \cellcolor[HTML]{DAE0FB}3.8B & \cellcolor[HTML]{DAE0FB}\Large \ding{51} & \cellcolor[HTML]{DAE0FB}\underline{73.7} & \cellcolor[HTML]{DAE0FB}\underline{55.9} & \cellcolor[HTML]{DAE0FB}\underline{74.6} & \cellcolor[HTML]{DAE0FB}\underline{56.4} & \cellcolor[HTML]{DAE0FB}\underline{78.7} & \cellcolor[HTML]{DAE0FB}\underline{39.7} & \cellcolor[HTML]{DAE0FB}\underline{43.3} & \cellcolor[HTML]{DAE0FB}\underline{68.5} & \cellcolor[HTML]{DAE0FB}\underline{61.3} \\
& \cellcolor[HTML]{DAE0FB}HealthGPT-L14 & \cellcolor[HTML]{DAE0FB}14B & \cellcolor[HTML]{DAE0FB}\Large \ding{51} & \cellcolor[HTML]{DAE0FB}\textbf{77.7} & \cellcolor[HTML]{DAE0FB}\textbf{58.3} & \cellcolor[HTML]{DAE0FB}\textbf{76.4} & \cellcolor[HTML]{DAE0FB}\textbf{64.5} & \cellcolor[HTML]{DAE0FB}\textbf{85.9} & \cellcolor[HTML]{DAE0FB}\textbf{44.4} & \cellcolor[HTML]{DAE0FB}\textbf{49.2} & \cellcolor[HTML]{DAE0FB}\textbf{74.4} & \cellcolor[HTML]{DAE0FB}\textbf{66.4} \\
\bottomrule
\end{tabular}
}
\label{tab:results}
\end{table*}
\begin{table*}[ht]
    \centering
    \caption{The experimental results for the four modality conversion tasks.}
    \resizebox{\textwidth}{!}{
    \begin{tabular}{l|ccc|ccc|ccc|ccc}
        \toprule
        \rowcolor[HTML]{E9F3FE} & \multicolumn{3}{c}{\textbf{CT to MRI (Brain)}} & \multicolumn{3}{c}{\textbf{CT to MRI (Pelvis)}} & \multicolumn{3}{c}{\textbf{MRI to CT (Brain)}} & \multicolumn{3}{c}{\textbf{MRI to CT (Pelvis)}} \\
        \cline{2-13}
        \rowcolor[HTML]{E9F3FE}\multirow{-2}{*}{\textbf{Model}}& \textbf{SSIM $\uparrow$} & \textbf{PSNR $\uparrow$} & \textbf{MSE $\downarrow$} & \textbf{SSIM $\uparrow$} & \textbf{PSNR $\uparrow$} & \textbf{MSE $\downarrow$} & \textbf{SSIM $\uparrow$} & \textbf{PSNR $\uparrow$} & \textbf{MSE $\downarrow$} & \textbf{SSIM $\uparrow$} & \textbf{PSNR $\uparrow$} & \textbf{MSE $\downarrow$} \\
        \midrule \midrule
        pix2pix & 71.09 & 32.65 & 36.85 & 59.17 & 31.02 & 51.91 & 78.79 & 33.85 & 28.33 & 72.31 & 32.98 & 36.19 \\
        CycleGAN & 54.76 & 32.23 & 40.56 & 54.54 & 30.77 & 55.00 & 63.75 & 31.02 & 52.78 & 50.54 & 29.89 & 67.78 \\
        BBDM & {71.69} & {32.91} & {34.44} & 57.37 & 31.37 & 48.06 & \textbf{86.40} & 34.12 & 26.61 & {79.26} & 33.15 & 33.60 \\
        Vmanba & 69.54 & 32.67 & 36.42 & {63.01} & {31.47} & {46.99} & 79.63 & 34.12 & 26.49 & 77.45 & 33.53 & 31.85 \\
        DiffMa & 71.47 & 32.74 & 35.77 & 62.56 & 31.43 & 47.38 & 79.00 & {34.13} & {26.45} & 78.53 & {33.68} & {30.51} \\
        \rowcolor[HTML]{DAE0FB}HealthGPT-M3 & \underline{79.38} & \underline{33.03} & \underline{33.48} & \underline{71.81} & \underline{31.83} & \underline{43.45} & {85.06} & \textbf{34.40} & \textbf{25.49} & \underline{84.23} & \textbf{34.29} & \textbf{27.99} \\
        \rowcolor[HTML]{DAE0FB}HealthGPT-L14 & \textbf{79.73} & \textbf{33.10} & \textbf{32.96} & \textbf{71.92} & \textbf{31.87} & \textbf{43.09} & \underline{85.31} & \underline{34.29} & \underline{26.20} & \textbf{84.96} & \underline{34.14} & \underline{28.13} \\
        \bottomrule
    \end{tabular}
    }
    \label{tab:conversion}
\end{table*}

\noindent \textbf{3rd Stage: Visual Instruction Fine-Tuning.}  
In the third stage, we introduce additional task-specific data to further optimize the model and enhance its adaptability to downstream tasks such as medical visual comprehension (e.g., medical QA, medical dialogues, and report generation) or generation tasks (e.g., super-resolution, denoising, and modality conversion). Notably, by this stage, the word embedding layer and output head have been fine-tuned, only the H-LoRA modules and adapter modules need to be trained. This strategy significantly improves the model's adaptability and flexibility across different tasks.


\section{Experiment}
\label{s:experiment}

\subsection{Data Description}
We evaluate our method on FI~\cite{you2016building}, Twitter\_LDL~\cite{yang2017learning} and Artphoto~\cite{machajdik2010affective}.
FI is a public dataset built from Flickr and Instagram, with 23,308 images and eight emotion categories, namely \textit{amusement}, \textit{anger}, \textit{awe},  \textit{contentment}, \textit{disgust}, \textit{excitement},  \textit{fear}, and \textit{sadness}. 
% Since images in FI are all copyrighted by law, some images are corrupted now, so we remove these samples and retain 21,828 images.
% T4SA contains images from Twitter, which are classified into three categories: \textit{positive}, \textit{neutral}, and \textit{negative}. In this paper, we adopt the base version of B-T4SA, which contains 470,586 images and provides text descriptions of the corresponding tweets.
Twitter\_LDL contains 10,045 images from Twitter, with the same eight categories as the FI dataset.
% 。
For these two datasets, they are randomly split into 80\%
training and 20\% testing set.
Artphoto contains 806 artistic photos from the DeviantArt website, which we use to further evaluate the zero-shot capability of our model.
% on the small-scale dataset.
% We construct and publicly release the first image sentiment analysis dataset containing metadata.
% 。

% Based on these datasets, we are the first to construct and publicly release metadata-enhanced image sentiment analysis datasets. These datasets include scenes, tags, descriptions, and corresponding confidence scores, and are available at this link for future research purposes.


% 
\begin{table}[t]
\centering
% \begin{center}
\caption{Overall performance of different models on FI and Twitter\_LDL datasets.}
\label{tab:cap1}
% \resizebox{\linewidth}{!}
{
\begin{tabular}{l|c|c|c|c}
\hline
\multirow{2}{*}{\textbf{Model}} & \multicolumn{2}{c|}{\textbf{FI}}  & \multicolumn{2}{c}{\textbf{Twitter\_LDL}} \\ \cline{2-5} 
  & \textbf{Accuracy} & \textbf{F1} & \textbf{Accuracy} & \textbf{F1}  \\ \hline
% (\rownumber)~AlexNet~\cite{krizhevsky2017imagenet}  & 58.13\% & 56.35\%  & 56.24\%& 55.02\%  \\ 
% (\rownumber)~VGG16~\cite{simonyan2014very}  & 63.75\%& 63.08\%  & 59.34\%& 59.02\%  \\ 
(\rownumber)~ResNet101~\cite{he2016deep} & 66.16\%& 65.56\%  & 62.02\% & 61.34\%  \\ 
(\rownumber)~CDA~\cite{han2023boosting} & 66.71\%& 65.37\%  & 64.14\% & 62.85\%  \\ 
(\rownumber)~CECCN~\cite{ruan2024color} & 67.96\%& 66.74\%  & 64.59\%& 64.72\% \\ 
(\rownumber)~EmoVIT~\cite{xie2024emovit} & 68.09\%& 67.45\%  & 63.12\% & 61.97\%  \\ 
(\rownumber)~ComLDL~\cite{zhang2022compound} & 68.83\%& 67.28\%  & 65.29\% & 63.12\%  \\ 
(\rownumber)~WSDEN~\cite{li2023weakly} & 69.78\%& 69.61\%  & 67.04\% & 65.49\% \\ 
(\rownumber)~ECWA~\cite{deng2021emotion} & 70.87\%& 69.08\%  & 67.81\% & 66.87\%  \\ 
(\rownumber)~EECon~\cite{yang2023exploiting} & 71.13\%& 68.34\%  & 64.27\%& 63.16\%  \\ 
(\rownumber)~MAM~\cite{zhang2024affective} & 71.44\%  & 70.83\% & 67.18\%  & 65.01\%\\ 
(\rownumber)~TGCA-PVT~\cite{chen2024tgca}   & 73.05\%  & 71.46\% & 69.87\%  & 68.32\% \\ 
(\rownumber)~OEAN~\cite{zhang2024object}   & 73.40\%  & 72.63\% & 70.52\%  & 69.47\% \\ \hline
(\rownumber)~\shortname  & \textbf{79.48\%} & \textbf{79.22\%} & \textbf{74.12\%} & \textbf{73.09\%} \\ \hline
\end{tabular}
}
\vspace{-6mm}
% \end{center}
\end{table}
% 

\subsection{Experiment Setting}
% \subsubsection{Model Setting.}
% 
\textbf{Model Setting:}
For feature representation, we set $k=10$ to select object tags, and adopt clip-vit-base-patch32 as the pre-trained model for unified feature representation.
Moreover, we empirically set $(d_e, d_h, d_k, d_s) = (512, 128, 16, 64)$, and set the classification class $L$ to 8.

% 

\textbf{Training Setting:}
To initialize the model, we set all weights such as $\boldsymbol{W}$ following the truncated normal distribution, and use AdamW optimizer with the learning rate of $1 \times 10^{-4}$.
% warmup scheduler of cosine, warmup steps of 2000.
Furthermore, we set the batch size to 32 and the epoch of the training process to 200.
During the implementation, we utilize \textit{PyTorch} to build our entire model.
% , and our project codes are publicly available at https://github.com/zzmyrep/MESN.
% Our project codes as well as data are all publicly available on GitHub\footnote{https://github.com/zzmyrep/KBCEN}.
% Code is available at \href{https://github.com/zzmyrep/KBCEN}{https://github.com/zzmyrep/KBCEN}.

\textbf{Evaluation Metrics:}
Following~\cite{zhang2024affective, chen2024tgca, zhang2024object}, we adopt \textit{accuracy} and \textit{F1} as our evaluation metrics to measure the performance of different methods for image sentiment analysis. 



\subsection{Experiment Result}
% We compare our model against the following baselines: AlexNet~\cite{krizhevsky2017imagenet}, VGG16~\cite{simonyan2014very}, ResNet101~\cite{he2016deep}, CECCN~\cite{ruan2024color}, EmoVIT~\cite{xie2024emovit}, WSCNet~\cite{yang2018weakly}, ECWA~\cite{deng2021emotion}, EECon~\cite{yang2023exploiting}, MAM~\cite{zhang2024affective} and TGCA-PVT~\cite{chen2024tgca}, and the overall results are summarized in Table~\ref{tab:cap1}.
We compare our model against several baselines, and the overall results are summarized in Table~\ref{tab:cap1}.
We observe that our model achieves the best performance in both accuracy and F1 metrics, significantly outperforming the previous models. 
This superior performance is mainly attributed to our effective utilization of metadata to enhance image sentiment analysis, as well as the exceptional capability of the unified sentiment transformer framework we developed. These results strongly demonstrate that our proposed method can bring encouraging performance for image sentiment analysis.

\setcounter{magicrownumbers}{0} 
\begin{table}[t]
\begin{center}
\caption{Ablation study of~\shortname~on FI dataset.} 
% \vspace{1mm}
\label{tab:cap2}
\resizebox{.9\linewidth}{!}
{
\begin{tabular}{lcc}
  \hline
  \textbf{Model} & \textbf{Accuracy} & \textbf{F1} \\
  \hline
  (\rownumber)~Ours (w/o vision) & 65.72\% & 64.54\% \\
  (\rownumber)~Ours (w/o text description) & 74.05\% & 72.58\% \\
  (\rownumber)~Ours (w/o object tag) & 77.45\% & 76.84\% \\
  (\rownumber)~Ours (w/o scene tag) & 78.47\% & 78.21\% \\
  \hline
  (\rownumber)~Ours (w/o unified embedding) & 76.41\% & 76.23\% \\
  (\rownumber)~Ours (w/o adaptive learning) & 76.83\% & 76.56\% \\
  (\rownumber)~Ours (w/o cross-modal fusion) & 76.85\% & 76.49\% \\
  \hline
  (\rownumber)~Ours  & \textbf{79.48\%} & \textbf{79.22\%} \\
  \hline
\end{tabular}
}
\end{center}
\vspace{-5mm}
\end{table}


\begin{figure}[t]
\centering
% \vspace{-2mm}
\includegraphics[width=0.42\textwidth]{fig/2dvisual-linux4-paper2.pdf}
\caption{Visualization of feature distribution on eight categories before (left) and after (right) model processing.}
% 
\label{fig:visualization}
\vspace{-5mm}
\end{figure}

\subsection{Ablation Performance}
In this subsection, we conduct an ablation study to examine which component is really important for performance improvement. The results are reported in Table~\ref{tab:cap2}.

For information utilization, we observe a significant decline in model performance when visual features are removed. Additionally, the performance of \shortname~decreases when different metadata are removed separately, which means that text description, object tag, and scene tag are all critical for image sentiment analysis.
Recalling the model architecture, we separately remove transformer layers of the unified representation module, the adaptive learning module, and the cross-modal fusion module, replacing them with MLPs of the same parameter scale.
In this way, we can observe varying degrees of decline in model performance, indicating that these modules are indispensable for our model to achieve better performance.

\subsection{Visualization}
% 


% % 开始使用minipage进行左右排列
% \begin{minipage}[t]{0.45\textwidth}  % 子图1宽度为45%
%     \centering
%     \includegraphics[width=\textwidth]{2dvisual.pdf}  % 插入图片
%     \captionof{figure}{Visualization of feature distribution.}  % 使用captionof添加图片标题
%     \label{fig:visualization}
% \end{minipage}


% \begin{figure}[t]
% \centering
% \vspace{-2mm}
% \includegraphics[width=0.45\textwidth]{fig/2dvisual.pdf}
% \caption{Visualization of feature distribution.}
% \label{fig:visualization}
% % \vspace{-4mm}
% \end{figure}

% \begin{figure}[t]
% \centering
% \vspace{-2mm}
% \includegraphics[width=0.45\textwidth]{fig/2dvisual-linux3-paper.pdf}
% \caption{Visualization of feature distribution.}
% \label{fig:visualization}
% % \vspace{-4mm}
% \end{figure}



\begin{figure}[tbp]   
\vspace{-4mm}
  \centering            
  \subfloat[Depth of adaptive learning layers]   
  {
    \label{fig:subfig1}\includegraphics[width=0.22\textwidth]{fig/fig_sensitivity-a5}
  }
  \subfloat[Depth of fusion layers]
  {
    % \label{fig:subfig2}\includegraphics[width=0.22\textwidth]{fig/fig_sensitivity-b2}
    \label{fig:subfig2}\includegraphics[width=0.22\textwidth]{fig/fig_sensitivity-b2-num.pdf}
  }
  \caption{Sensitivity study of \shortname~on different depth. }   
  \label{fig:fig_sensitivity}  
\vspace{-2mm}
\end{figure}

% \begin{figure}[htbp]
% \centerline{\includegraphics{2dvisual.pdf}}
% \caption{Visualization of feature distribution.}
% \label{fig:visualization}
% \end{figure}

% In Fig.~\ref{fig:visualization}, we use t-SNE~\cite{van2008visualizing} to reduce the dimension of data features for visualization, Figure in left represents the metadata features before model processing, the features are obtained by embedding through the CLIP model, and figure in right shows the features of the data after model processing, it can be observed that after the model processing, the data with different label categories fall in different regions in the space, therefore, we can conclude that the Therefore, we can conclude that the model can effectively utilize the information contained in the metadata and use it to guide the model for classification.

In Fig.~\ref{fig:visualization}, we use t-SNE~\cite{van2008visualizing} to reduce the dimension of data features for visualization.
The left figure shows metadata features before being processed by our model (\textit{i.e.}, embedded by CLIP), while the right shows the distribution of features after being processed by our model.
We can observe that after the model processing, data with the same label are closer to each other, while others are farther away.
Therefore, it shows that the model can effectively utilize the information contained in the metadata and use it to guide the classification process.

\subsection{Sensitivity Analysis}
% 
In this subsection, we conduct a sensitivity analysis to figure out the effect of different depth settings of adaptive learning layers and fusion layers. 
% In this subsection, we conduct a sensitivity analysis to figure out the effect of different depth settings on the model. 
% Fig.~\ref{fig:fig_sensitivity} presents the effect of different depth settings of adaptive learning layers and fusion layers. 
Taking Fig.~\ref{fig:fig_sensitivity} (a) as an example, the model performance improves with increasing depth, reaching the best performance at a depth of 4.
% Taking Fig.~\ref{fig:fig_sensitivity} (a) as an example, the performance of \shortname~improves with the increase of depth at first, reaching the best performance at a depth of 4.
When the depth continues to increase, the accuracy decreases to varying degrees.
Similar results can be observed in Fig.~\ref{fig:fig_sensitivity} (b).
Therefore, we set their depths to 4 and 6 respectively to achieve the best results.

% Through our experiments, we can observe that the effect of modifying these hyperparameters on the results of the experiments is very weak, and the surface model is not sensitive to the hyperparameters.


\subsection{Zero-shot Capability}
% 

% (1)~GCH~\cite{2010Analyzing} & 21.78\% & (5)~RA-DLNet~\cite{2020A} & 34.01\% \\ \hline
% (2)~WSCNet~\cite{2019WSCNet}  & 30.25\% & (6)~CECCN~\cite{ruan2024color} & 43.83\% \\ \hline
% (3)~PCNN~\cite{2015Robust} & 31.68\%  & (7)~EmoVIT~\cite{xie2024emovit} & 44.90\% \\ \hline
% (4)~AR~\cite{2018Visual} & 32.67\% & (8)~Ours (Zero-shot) & 47.83\% \\ \hline


\begin{table}[t]
\centering
\caption{Zero-shot capability of \shortname.}
\label{tab:cap3}
\resizebox{1\linewidth}{!}
{
\begin{tabular}{lc|lc}
\hline
\textbf{Model} & \textbf{Accuracy} & \textbf{Model} & \textbf{Accuracy} \\ \hline
(1)~WSCNet~\cite{2019WSCNet}  & 30.25\% & (5)~MAM~\cite{zhang2024affective} & 39.56\%  \\ \hline
(2)~AR~\cite{2018Visual} & 32.67\% & (6)~CECCN~\cite{ruan2024color} & 43.83\% \\ \hline
(3)~RA-DLNet~\cite{2020A} & 34.01\%  & (7)~EmoVIT~\cite{xie2024emovit} & 44.90\% \\ \hline
(4)~CDA~\cite{han2023boosting} & 38.64\% & (8)~Ours (Zero-shot) & 47.83\% \\ \hline
\end{tabular}
}
\vspace{-5mm}
\end{table}

% We use the model trained on the FI dataset to test on the artphoto dataset to verify the model's generalization ability as well as robustness to other distributed datasets.
% We can observe that the MESN model shows strong competitiveness in terms of accuracy when compared to other trained models, which suggests that the model has a good generalization ability in the OOD task.

To validate the model's generalization ability and robustness to other distributed datasets, we directly test the model trained on the FI dataset, without training on Artphoto. 
% As observed in Table 3, compared to other models trained on Artphoto, we achieve highly competitive zero-shot performance, indicating that the model has good generalization ability in out-of-distribution tasks.
From Table~\ref{tab:cap3}, we can observe that compared with other models trained on Artphoto, we achieve competitive zero-shot performance, which shows that the model has good generalization ability in out-of-distribution tasks.


%%%%%%%%%%%%
%  E2E     %
%%%%%%%%%%%%


\section{Conclusion}
In this paper, we introduced Wi-Chat, the first LLM-powered Wi-Fi-based human activity recognition system that integrates the reasoning capabilities of large language models with the sensing potential of wireless signals. Our experimental results on a self-collected Wi-Fi CSI dataset demonstrate the promising potential of LLMs in enabling zero-shot Wi-Fi sensing. These findings suggest a new paradigm for human activity recognition that does not rely on extensive labeled data. We hope future research will build upon this direction, further exploring the applications of LLMs in signal processing domains such as IoT, mobile sensing, and radar-based systems.

\section*{Limitations}
While our work represents the first attempt to leverage LLMs for processing Wi-Fi signals, it is a preliminary study focused on a relatively simple task: Wi-Fi-based human activity recognition. This choice allows us to explore the feasibility of LLMs in wireless sensing but also comes with certain limitations.

Our approach primarily evaluates zero-shot performance, which, while promising, may still lag behind traditional supervised learning methods in highly complex or fine-grained recognition tasks. Besides, our study is limited to a controlled environment with a self-collected dataset, and the generalizability of LLMs to diverse real-world scenarios with varying Wi-Fi conditions, environmental interference, and device heterogeneity remains an open question.

Additionally, we have yet to explore the full potential of LLMs in more advanced Wi-Fi sensing applications, such as fine-grained gesture recognition, occupancy detection, and passive health monitoring. Future work should investigate the scalability of LLM-based approaches, their robustness to domain shifts, and their integration with multimodal sensing techniques in broader IoT applications.


% Bibliography entries for the entire Anthology, followed by custom entries
%\bibliography{anthology,custom}
% Custom bibliography entries only
\bibliography{main}
\newpage
\appendix

\section{Experiment prompts}
\label{sec:prompt}
The prompts used in the LLM experiments are shown in the following Table~\ref{tab:prompts}.

\definecolor{titlecolor}{rgb}{0.9, 0.5, 0.1}
\definecolor{anscolor}{rgb}{0.2, 0.5, 0.8}
\definecolor{labelcolor}{HTML}{48a07e}
\begin{table*}[h]
	\centering
	
 % \vspace{-0.2cm}
	
	\begin{center}
		\begin{tikzpicture}[
				chatbox_inner/.style={rectangle, rounded corners, opacity=0, text opacity=1, font=\sffamily\scriptsize, text width=5in, text height=9pt, inner xsep=6pt, inner ysep=6pt},
				chatbox_prompt_inner/.style={chatbox_inner, align=flush left, xshift=0pt, text height=11pt},
				chatbox_user_inner/.style={chatbox_inner, align=flush left, xshift=0pt},
				chatbox_gpt_inner/.style={chatbox_inner, align=flush left, xshift=0pt},
				chatbox/.style={chatbox_inner, draw=black!25, fill=gray!7, opacity=1, text opacity=0},
				chatbox_prompt/.style={chatbox, align=flush left, fill=gray!1.5, draw=black!30, text height=10pt},
				chatbox_user/.style={chatbox, align=flush left},
				chatbox_gpt/.style={chatbox, align=flush left},
				chatbox2/.style={chatbox_gpt, fill=green!25},
				chatbox3/.style={chatbox_gpt, fill=red!20, draw=black!20},
				chatbox4/.style={chatbox_gpt, fill=yellow!30},
				labelbox/.style={rectangle, rounded corners, draw=black!50, font=\sffamily\scriptsize\bfseries, fill=gray!5, inner sep=3pt},
			]
											
			\node[chatbox_user] (q1) {
				\textbf{System prompt}
				\newline
				\newline
				You are a helpful and precise assistant for segmenting and labeling sentences. We would like to request your help on curating a dataset for entity-level hallucination detection.
				\newline \newline
                We will give you a machine generated biography and a list of checked facts about the biography. Each fact consists of a sentence and a label (True/False). Please do the following process. First, breaking down the biography into words. Second, by referring to the provided list of facts, merging some broken down words in the previous step to form meaningful entities. For example, ``strategic thinking'' should be one entity instead of two. Third, according to the labels in the list of facts, labeling each entity as True or False. Specifically, for facts that share a similar sentence structure (\eg, \textit{``He was born on Mach 9, 1941.''} (\texttt{True}) and \textit{``He was born in Ramos Mejia.''} (\texttt{False})), please first assign labels to entities that differ across atomic facts. For example, first labeling ``Mach 9, 1941'' (\texttt{True}) and ``Ramos Mejia'' (\texttt{False}) in the above case. For those entities that are the same across atomic facts (\eg, ``was born'') or are neutral (\eg, ``he,'' ``in,'' and ``on''), please label them as \texttt{True}. For the cases that there is no atomic fact that shares the same sentence structure, please identify the most informative entities in the sentence and label them with the same label as the atomic fact while treating the rest of the entities as \texttt{True}. In the end, output the entities and labels in the following format:
                \begin{itemize}[nosep]
                    \item Entity 1 (Label 1)
                    \item Entity 2 (Label 2)
                    \item ...
                    \item Entity N (Label N)
                \end{itemize}
                % \newline \newline
                Here are two examples:
                \newline\newline
                \textbf{[Example 1]}
                \newline
                [The start of the biography]
                \newline
                \textcolor{titlecolor}{Marianne McAndrew is an American actress and singer, born on November 21, 1942, in Cleveland, Ohio. She began her acting career in the late 1960s, appearing in various television shows and films.}
                \newline
                [The end of the biography]
                \newline \newline
                [The start of the list of checked facts]
                \newline
                \textcolor{anscolor}{[Marianne McAndrew is an American. (False); Marianne McAndrew is an actress. (True); Marianne McAndrew is a singer. (False); Marianne McAndrew was born on November 21, 1942. (False); Marianne McAndrew was born in Cleveland, Ohio. (False); She began her acting career in the late 1960s. (True); She has appeared in various television shows. (True); She has appeared in various films. (True)]}
                \newline
                [The end of the list of checked facts]
                \newline \newline
                [The start of the ideal output]
                \newline
                \textcolor{labelcolor}{[Marianne McAndrew (True); is (True); an (True); American (False); actress (True); and (True); singer (False); , (True); born (True); on (True); November 21, 1942 (False); , (True); in (True); Cleveland, Ohio (False); . (True); She (True); began (True); her (True); acting career (True); in (True); the late 1960s (True); , (True); appearing (True); in (True); various (True); television shows (True); and (True); films (True); . (True)]}
                \newline
                [The end of the ideal output]
				\newline \newline
                \textbf{[Example 2]}
                \newline
                [The start of the biography]
                \newline
                \textcolor{titlecolor}{Doug Sheehan is an American actor who was born on April 27, 1949, in Santa Monica, California. He is best known for his roles in soap operas, including his portrayal of Joe Kelly on ``General Hospital'' and Ben Gibson on ``Knots Landing.''}
                \newline
                [The end of the biography]
                \newline \newline
                [The start of the list of checked facts]
                \newline
                \textcolor{anscolor}{[Doug Sheehan is an American. (True); Doug Sheehan is an actor. (True); Doug Sheehan was born on April 27, 1949. (True); Doug Sheehan was born in Santa Monica, California. (False); He is best known for his roles in soap operas. (True); He portrayed Joe Kelly. (True); Joe Kelly was in General Hospital. (True); General Hospital is a soap opera. (True); He portrayed Ben Gibson. (True); Ben Gibson was in Knots Landing. (True); Knots Landing is a soap opera. (True)]}
                \newline
                [The end of the list of checked facts]
                \newline \newline
                [The start of the ideal output]
                \newline
                \textcolor{labelcolor}{[Doug Sheehan (True); is (True); an (True); American (True); actor (True); who (True); was born (True); on (True); April 27, 1949 (True); in (True); Santa Monica, California (False); . (True); He (True); is (True); best known (True); for (True); his roles in soap operas (True); , (True); including (True); in (True); his portrayal (True); of (True); Joe Kelly (True); on (True); ``General Hospital'' (True); and (True); Ben Gibson (True); on (True); ``Knots Landing.'' (True)]}
                \newline
                [The end of the ideal output]
				\newline \newline
				\textbf{User prompt}
				\newline
				\newline
				[The start of the biography]
				\newline
				\textcolor{magenta}{\texttt{\{BIOGRAPHY\}}}
				\newline
				[The ebd of the biography]
				\newline \newline
				[The start of the list of checked facts]
				\newline
				\textcolor{magenta}{\texttt{\{LIST OF CHECKED FACTS\}}}
				\newline
				[The end of the list of checked facts]
			};
			\node[chatbox_user_inner] (q1_text) at (q1) {
				\textbf{System prompt}
				\newline
				\newline
				You are a helpful and precise assistant for segmenting and labeling sentences. We would like to request your help on curating a dataset for entity-level hallucination detection.
				\newline \newline
                We will give you a machine generated biography and a list of checked facts about the biography. Each fact consists of a sentence and a label (True/False). Please do the following process. First, breaking down the biography into words. Second, by referring to the provided list of facts, merging some broken down words in the previous step to form meaningful entities. For example, ``strategic thinking'' should be one entity instead of two. Third, according to the labels in the list of facts, labeling each entity as True or False. Specifically, for facts that share a similar sentence structure (\eg, \textit{``He was born on Mach 9, 1941.''} (\texttt{True}) and \textit{``He was born in Ramos Mejia.''} (\texttt{False})), please first assign labels to entities that differ across atomic facts. For example, first labeling ``Mach 9, 1941'' (\texttt{True}) and ``Ramos Mejia'' (\texttt{False}) in the above case. For those entities that are the same across atomic facts (\eg, ``was born'') or are neutral (\eg, ``he,'' ``in,'' and ``on''), please label them as \texttt{True}. For the cases that there is no atomic fact that shares the same sentence structure, please identify the most informative entities in the sentence and label them with the same label as the atomic fact while treating the rest of the entities as \texttt{True}. In the end, output the entities and labels in the following format:
                \begin{itemize}[nosep]
                    \item Entity 1 (Label 1)
                    \item Entity 2 (Label 2)
                    \item ...
                    \item Entity N (Label N)
                \end{itemize}
                % \newline \newline
                Here are two examples:
                \newline\newline
                \textbf{[Example 1]}
                \newline
                [The start of the biography]
                \newline
                \textcolor{titlecolor}{Marianne McAndrew is an American actress and singer, born on November 21, 1942, in Cleveland, Ohio. She began her acting career in the late 1960s, appearing in various television shows and films.}
                \newline
                [The end of the biography]
                \newline \newline
                [The start of the list of checked facts]
                \newline
                \textcolor{anscolor}{[Marianne McAndrew is an American. (False); Marianne McAndrew is an actress. (True); Marianne McAndrew is a singer. (False); Marianne McAndrew was born on November 21, 1942. (False); Marianne McAndrew was born in Cleveland, Ohio. (False); She began her acting career in the late 1960s. (True); She has appeared in various television shows. (True); She has appeared in various films. (True)]}
                \newline
                [The end of the list of checked facts]
                \newline \newline
                [The start of the ideal output]
                \newline
                \textcolor{labelcolor}{[Marianne McAndrew (True); is (True); an (True); American (False); actress (True); and (True); singer (False); , (True); born (True); on (True); November 21, 1942 (False); , (True); in (True); Cleveland, Ohio (False); . (True); She (True); began (True); her (True); acting career (True); in (True); the late 1960s (True); , (True); appearing (True); in (True); various (True); television shows (True); and (True); films (True); . (True)]}
                \newline
                [The end of the ideal output]
				\newline \newline
                \textbf{[Example 2]}
                \newline
                [The start of the biography]
                \newline
                \textcolor{titlecolor}{Doug Sheehan is an American actor who was born on April 27, 1949, in Santa Monica, California. He is best known for his roles in soap operas, including his portrayal of Joe Kelly on ``General Hospital'' and Ben Gibson on ``Knots Landing.''}
                \newline
                [The end of the biography]
                \newline \newline
                [The start of the list of checked facts]
                \newline
                \textcolor{anscolor}{[Doug Sheehan is an American. (True); Doug Sheehan is an actor. (True); Doug Sheehan was born on April 27, 1949. (True); Doug Sheehan was born in Santa Monica, California. (False); He is best known for his roles in soap operas. (True); He portrayed Joe Kelly. (True); Joe Kelly was in General Hospital. (True); General Hospital is a soap opera. (True); He portrayed Ben Gibson. (True); Ben Gibson was in Knots Landing. (True); Knots Landing is a soap opera. (True)]}
                \newline
                [The end of the list of checked facts]
                \newline \newline
                [The start of the ideal output]
                \newline
                \textcolor{labelcolor}{[Doug Sheehan (True); is (True); an (True); American (True); actor (True); who (True); was born (True); on (True); April 27, 1949 (True); in (True); Santa Monica, California (False); . (True); He (True); is (True); best known (True); for (True); his roles in soap operas (True); , (True); including (True); in (True); his portrayal (True); of (True); Joe Kelly (True); on (True); ``General Hospital'' (True); and (True); Ben Gibson (True); on (True); ``Knots Landing.'' (True)]}
                \newline
                [The end of the ideal output]
				\newline \newline
				\textbf{User prompt}
				\newline
				\newline
				[The start of the biography]
				\newline
				\textcolor{magenta}{\texttt{\{BIOGRAPHY\}}}
				\newline
				[The ebd of the biography]
				\newline \newline
				[The start of the list of checked facts]
				\newline
				\textcolor{magenta}{\texttt{\{LIST OF CHECKED FACTS\}}}
				\newline
				[The end of the list of checked facts]
			};
		\end{tikzpicture}
        \caption{GPT-4o prompt for labeling hallucinated entities.}\label{tb:gpt-4-prompt}
	\end{center}
\vspace{-0cm}
\end{table*}
% \section{Full Experiment Results}
% \begin{table*}[th]
    \centering
    \small
    \caption{Classification Results}
    \begin{tabular}{lcccc}
        \toprule
        \textbf{Method} & \textbf{Accuracy} & \textbf{Precision} & \textbf{Recall} & \textbf{F1-score} \\
        \midrule
        \multicolumn{5}{c}{\textbf{Zero Shot}} \\
                Zero-shot E-eyes & 0.26 & 0.26 & 0.27 & 0.26 \\
        Zero-shot CARM & 0.24 & 0.24 & 0.24 & 0.24 \\
                Zero-shot SVM & 0.27 & 0.28 & 0.28 & 0.27 \\
        Zero-shot CNN & 0.23 & 0.24 & 0.23 & 0.23 \\
        Zero-shot RNN & 0.26 & 0.26 & 0.26 & 0.26 \\
DeepSeek-0shot & 0.54 & 0.61 & 0.54 & 0.52 \\
DeepSeek-0shot-COT & 0.33 & 0.24 & 0.33 & 0.23 \\
DeepSeek-0shot-Knowledge & 0.45 & 0.46 & 0.45 & 0.44 \\
Gemma2-0shot & 0.35 & 0.22 & 0.38 & 0.27 \\
Gemma2-0shot-COT & 0.36 & 0.22 & 0.36 & 0.27 \\
Gemma2-0shot-Knowledge & 0.32 & 0.18 & 0.34 & 0.20 \\
GPT-4o-mini-0shot & 0.48 & 0.53 & 0.48 & 0.41 \\
GPT-4o-mini-0shot-COT & 0.33 & 0.50 & 0.33 & 0.38 \\
GPT-4o-mini-0shot-Knowledge & 0.49 & 0.31 & 0.49 & 0.36 \\
GPT-4o-0shot & 0.62 & 0.62 & 0.47 & 0.42 \\
GPT-4o-0shot-COT & 0.29 & 0.45 & 0.29 & 0.21 \\
GPT-4o-0shot-Knowledge & 0.44 & 0.52 & 0.44 & 0.39 \\
LLaMA-0shot & 0.32 & 0.25 & 0.32 & 0.24 \\
LLaMA-0shot-COT & 0.12 & 0.25 & 0.12 & 0.09 \\
LLaMA-0shot-Knowledge & 0.32 & 0.25 & 0.32 & 0.28 \\
Mistral-0shot & 0.19 & 0.23 & 0.19 & 0.10 \\
Mistral-0shot-Knowledge & 0.21 & 0.40 & 0.21 & 0.11 \\
        \midrule
        \multicolumn{5}{c}{\textbf{4 Shot}} \\
GPT-4o-mini-4shot & 0.58 & 0.59 & 0.58 & 0.53 \\
GPT-4o-mini-4shot-COT & 0.57 & 0.53 & 0.57 & 0.50 \\
GPT-4o-mini-4shot-Knowledge & 0.56 & 0.51 & 0.56 & 0.47 \\
GPT-4o-4shot & 0.77 & 0.84 & 0.77 & 0.73 \\
GPT-4o-4shot-COT & 0.63 & 0.76 & 0.63 & 0.53 \\
GPT-4o-4shot-Knowledge & 0.72 & 0.82 & 0.71 & 0.66 \\
LLaMA-4shot & 0.29 & 0.24 & 0.29 & 0.21 \\
LLaMA-4shot-COT & 0.20 & 0.30 & 0.20 & 0.13 \\
LLaMA-4shot-Knowledge & 0.15 & 0.23 & 0.13 & 0.13 \\
Mistral-4shot & 0.02 & 0.02 & 0.02 & 0.02 \\
Mistral-4shot-Knowledge & 0.21 & 0.27 & 0.21 & 0.20 \\
        \midrule
        
        \multicolumn{5}{c}{\textbf{Suprevised}} \\
        SVM & 0.94 & 0.92 & 0.91 & 0.91 \\
        CNN & 0.98 & 0.98 & 0.97 & 0.97 \\
        RNN & 0.99 & 0.99 & 0.99 & 0.99 \\
        % \midrule
        % \multicolumn{5}{c}{\textbf{Conventional Wi-Fi-based Human Activity Recognition Systems}} \\
        E-eyes & 1.00 & 1.00 & 1.00 & 1.00 \\
        CARM & 0.98 & 0.98 & 0.98 & 0.98 \\
\midrule
 \multicolumn{5}{c}{\textbf{Vision Models}} \\
           Zero-shot SVM & 0.26 & 0.25 & 0.25 & 0.25 \\
        Zero-shot CNN & 0.26 & 0.25 & 0.26 & 0.26 \\
        Zero-shot RNN & 0.28 & 0.28 & 0.29 & 0.28 \\
        SVM & 0.99 & 0.99 & 0.99 & 0.99 \\
        CNN & 0.98 & 0.99 & 0.98 & 0.98 \\
        RNN & 0.98 & 0.99 & 0.98 & 0.98 \\
GPT-4o-mini-Vision & 0.84 & 0.85 & 0.84 & 0.84 \\
GPT-4o-mini-Vision-COT & 0.90 & 0.91 & 0.90 & 0.90 \\
GPT-4o-Vision & 0.74 & 0.82 & 0.74 & 0.73 \\
GPT-4o-Vision-COT & 0.70 & 0.83 & 0.70 & 0.68 \\
LLaMA-Vision & 0.20 & 0.23 & 0.20 & 0.09 \\
LLaMA-Vision-Knowledge & 0.22 & 0.05 & 0.22 & 0.08 \\

        \bottomrule
    \end{tabular}
    \label{full}
\end{table*}




\end{document}


\clearpage

\onecolumn
\appendix

\section{Flowchart for Constructing Preference Data}
\label{appendix:preference_data}
Figure~\ref{appendix:constructing_preference_data} shows the flowchart for the construction of a pair of chosen-rejected rationales for a given news article in the training data. We apply this process and construct a dataset containing preferences for all the instance in the training data. Then the preference data is used for DPO. 
\begin{figure}[!h]
  \centering
  \includegraphics[width=\textwidth]{f5}
  \caption{Flowchart for constructing preference data}
  \label{appendix:constructing_preference_data}
\end{figure}

% \clearpage

\section{Implementation Details}
\label{appendix:implementation}
We employ GPT-4o as the teacher LLM to generate supervision data and experiment with two student LLMs: Mistral-7B-v0.3 and Llama-3.1-8B. All our rationale generators and headline generators are fine-tuned on the training data for three epochs. We apply QLoRA \citep{dettmers_qlora_2023} techniques to fine-tune the student LLMs efficiently. For all rationale generators, we set the LoRA rank and LoRA alpha to 128 and 64, respectively. For all headline generators, we set these values to 64 and 32. We fine-tune Mistral-7B-v0.3 with a learning rate of 2e-4 and Llama-3.1-8B with 8e-4. To create the preference dataset of chosen and rejected rationales, we fine-tune Mistral-7B-v0.3 and Llama-3.1-8B on the training data for a single epoch, sample 15 TEN rationales from each model for every news article, remove duplicate generations, and create a pair of chosen and rejected rationales for each training sample. During DPO, we set the LoRA rank to 256, LoRA alpha to 128, and DPO beta to 0.8. We set the learning rates for Mistral-7B-v0.3 and Llama-3.1-8B to 2e-6 and 8e-6 when doing experiments on NumHG, while setting them to 5e-6 and 2e-5 on XSum.
% \clearpage

% \section{Error Analysis}
% \label{appendix:error_analysis}


\clearpage
\section{Generation of Supervision Data}
\label{appendix: five-demonstrations}

The following five examples are demonstrations for five distinct operations needed to calculate the correct numbers in the headlines. The outputs (TEN rationales) are generated by the GPT-4o via zero-shot prompting respectively. Then the five examples, including generated TEN rationales, are used as demonstrations for the GPT-4o to generate supervision data for the entire training dataset.

\begin{table}[htbp]
    \centering
    \begin{tabular}{p{0.9\textwidth}}
    \hline
    \textbf{Example 1: Copy} \\
    \hline
    \textbf{Input:} \\
    \textbf{News:} (Oct 29, 2013 8:15 AM CDT) Dax Shepard and Kristen Bell got married at the Beverly Hills courthouse, in a ceremony about as different from Kim Kardashian's last wedding extravaganza as it is possible to be. As Shepard revealed last night on Jimmy Kimmel Live, the whole thing—including the fuel it took to get to the courthouse—cost \$142. It was just Kristen and I at this lonely courthouse, he said, so friends showed up afterward with a cake reading, in icing, The World's Worst Wedding. How many people can say they threw the world's worst wedding? Shepard asked. \\
    \textbf{Headline:} Dax Shepard: Wedding to Kristen Bell Cost \$\_\_\_\_. \\
    \textbf{Correct Number:} 142 \\
    \textbf{System Instruction: } You will be given a piece of news after prefix "News: ". You will be given a masked headline of the news after prefix "Headline:". You will be provided the correct missing numeral after prefix "Correct Number:". Please emphasise what Topic the headline should focus on, list all Entities and Numbers mentioned in the news, and elaborate on the Reasoning steps for how to calculate the correct number in the headline. \\
    \hline
    \textbf{Output:} \\
    \***Topic the headline should focus on** \\
    The headline should focus on the cost of Dax Shepard and Kristen Bell's wedding. \\
    \\
    \***Entities Mentioned** \\
    1. Dax Shepard \\
    2. Kristen Bell \\
    3. Kim Kardashian \\
    4. Jimmy Kimmel \\
    \\
    \***Numbers Mentioned** \\
    1. \$142 (cost of the wedding) \\
    \\
    \***Reasoning steps** \\
    The news content directly states that the entire wedding, including the fuel to get to the courthouse, cost \$142. \\
    
    Therefore, the numeral in the headline should be 142. \\
    \hline
    \end{tabular}
  % \caption{\label{tbl:demonstration_copy}
  % }
  
\end{table}

\begin{table}[htbp]
    \centering
    \begin{tabular}{p{0.9\textwidth}}
    \hline
    \textbf{Example 2: Add} \\
    \hline
    \textbf{Input:} \\
    \textbf{News:} (Sep 16, 2017 8:26 AM CDT) Police in St. Louis used tear gas to break up protests Friday night after around 1,000 demonstrators surrounded the home of Mayor Lyda Krewson, hurling red paint and breaking a front window. Police say at least nine St. Louis officers and a Highway Patrol trooper were injured in the protests, which followed the acquittal of white former police officer Jason Stockley in the 2011 shooting death of black motorist Anthony Lamar Smith, the St. Louis Post-Dispatch reports. Police say 32 people were arrested when a demonstration that had been largely peaceful turned violent after dark and agitators began to destroy property and assault police officers. Activists have promised more protests against the acquittal of Stockley, who, during a police chase, was caught on dashcam video saying I'm going to kill this motherf*****, don't you know it, the AP reports. Less than a minute later, he shot Smith five times.....  \\
    \textbf{Headline:} \_\_\_\_ Cops Injured in St. Louis Protests. \\
    \textbf{Correct Number:} 10 \\
    \textbf{System Instruction: } You will be given a piece of news after prefix "News: ". You will be given a masked headline of the news after prefix "Headline:". You will be provided the correct missing numeral after prefix "Correct Number:". Please emphasise what Topic the headline should focus on, list all Entities and Numbers mentioned in the news, and elaborate on the Reasoning steps for how to calculate the correct number in the headline. \\
    \hline
    \textbf{Output:} \\
    \***Topic the headline should focus on** \\
    The headline should focus on the number of police officers injured during the St. Louis protests. \\
    \\
    \***Entities Mentioned:**\\
    1. Mayor Lyda Krewson \\
    2. Jason Stockley (white former police officer) \\
    3. Anthony Lamar Smith (black motorist) \\
    4. Reverend Clinton Stancil \\
    5. St. Louis officers \\
    6. Highway Patrol trooper \\
    \\
    \***Numbers Mentioned** \\
    1. 1,000 (demonstrators) \\
    2. 9 (St. Louis officers injured) \\
    3. 1 (Highway Patrol trooper injured) \\
    4. 2011 (year of the shooting) \\
    5. 32 (people arrested) \\
    6. 3 (years since Ferguson unrest) \\
    7. 18 (age of Michael Brown) \\
    8. 5 (times Smith was shot) \\
    9. 24 (age of Anthony Lamar Smith) \\
    \\
    \***Reasoning steps** \\
    To find the total number of cops injured in the St. Louis protests, we need to sum the number of St. Louis officers injured and the Highway Patrol trooper injured. \\
    
    - St. Louis officers injured: 9 \\
    - Highway Patrol trooper injured: 1 \\

    Total injured cops = 9 (St. Louis officers) + 1 (Highway Patrol trooper) = 10 \\
    
    Therefore, the numeral in the headline should be 10. \\
    \hline
    \end{tabular}
  % \caption{\label{tbl:demonstration_add}
  % }
  
\end{table}

\begin{table}[htbp]
    \centering
    \begin{tabular}{p{0.9\textwidth}}
    \hline
    \textbf{Example 3: Subtract} \\
    \hline
    \textbf{Input:} \\
    \textbf{News:} (Jun 26, 2017 9:30 AM CDT) After three decades on the lam, an Arkansas man has been quickly reacquainted with life behind bars. Steven Dishman, 60—who was five months into a seven-year sentence for burglary and property theft when he escaped from the Cummins Unit in Lincoln County, Ark., on May 28, 1985, reports KTHV—was arrested Sunday at a home in Springdale in the northwestern part of the state, report Arkansas Online and KFSM. Arkansas State Police and local cops raided the home after receiving a tip. A rep for the Arkansas Department of Correction tells Fox 16 that Dishman will now be required to serve the remainder of his sentence. Dishman may also face additional charges related to his escape, the details of which were not immediately explained. Had Dishman stayed in jail, he would have been eligible for parole in 1987 and freed by 1991.\\
    \textbf{Headline:} He Spent \_\_\_\_ Years on the Run. Then Police Received a Tip. \\
    \textbf{Correct Number:} 32 \\
    \textbf{System Instruction: } You will be given a piece of news after prefix "News: ". You will be given a masked headline of the news after prefix "Headline:". You will be provided the correct missing numeral after prefix "Correct Number:". Please emphasise what Topic the headline should focus on, list all Entities and Numbers mentioned in the news, and elaborate on the Reasoning steps for how to calculate the correct number in the headline. \\
    \hline
    \textbf{Output:} \\
    \***Topic the headline should focus on** \\
    The headline should focus on the number of years Steven Dishman spent on the run before being recaptured. \\
    \\
    \***Entities Mentioned** \\
    1. Steven Dishman \\
    2. Arkansas State Police \\
    3. Local cops \\
    4. Arkansas Department of Correction representative \\
    \\
    \***Numbers Mentioned** \\
    1. 60 (age of Steven Dishman) \\
    2. 7 (years sentence for burglary and property theft) \\
    3. 5 (months into his sentence when he escaped) \\
    4. May 28, 1985 (date of escape) \\
    5. 1987 (year he would have been eligible for parole) \\
    6. 1991 (year he would have been freed) \\
    7. 2017 (current year) \\
     \\
    \***Reasoning steps** \\
    To find the number of years Steven Dishman spent on the run, we need to calculate the difference between the year he escaped (1985) and the year he was recaptured (2017). \\
    
    - Year of escape: 1985 \\
    - Year of recapture: 2017 \\
    
    Years on the run = 2017 - 1985 = 32 \\
    
    Therefore, the numeral in the headline should be 32. \\
    \hline
    \end{tabular}
  % \caption{\label{tbl:demonstration_subtract}
  % }
  
\end{table}


\begin{table}[htbp]
    \centering
    \begin{tabular}{p{0.9\textwidth}}
    \hline
    \textbf{Example 4: Paraphrase} \\
    \hline
    \textbf{Input:} \\
    \textbf{News:} News: (Oct 7, 2014 12:40 PM CDT) As of Jan. 1, Walmart will no longer offer 30,000 of its employees health insurance. Bloomberg notes that's about 2\% of its workforce. The move comes as a reaction to the company's rising health care costs as far more of its employees and their families enrolled in its health care plans than it had expected following the ObamaCare rollout. The AP reports those costs will surge \$500 million this fiscal year, \$170 million more than had been estimated. Those affected are employees who average fewer than 30 hours of work per week; the Wall Street Journal explains they were grandfathered in when Walmart in 2012 stopped offering insurance to new hires who didn't exceed the 30-hour threshold. A benefits expert says Walmart is actually late to the game in terms of cutting insurance to some part-time workers; Target, the Home Depot, and others have already done so. Meanwhile, Walmart's full time workers will see their premiums rise in 2015. Premiums for the basic plan, which 40\% of its workforce is on, will increase 19\% to \$21.90 per pay period come Jan. 1. \\
    \textbf{Headline:} Headline: \_\_\_\_K Walmart Part-Timers to Lose Health Insurance. \\
    \textbf{Correct Number:} 30. \\
    \textbf{System Instruction: } You will be given a piece of news after prefix "News: ". You will be given a masked headline of the news after prefix "Headline:". You will be provided the correct missing numeral after prefix "Correct Number:". Please emphasise what Topic the headline should focus on, list all Entities and Numbers mentioned in the news, and elaborate on the Reasoning steps for how to calculate the correct number in the headline. \\
    \hline
    \textbf{Output:} \\
    \***Topic the headline should focus on** \\
    The headline should focus on the number of Walmart part-time employees who will lose their health insurance. \\
    \\
    \***Entitiess Mentioned** \\
    1. Walmart employees \\
    2. Walmart's full-time workers \\
    3. Target \\
    4. The Home Depot \\
    5. ObamaCare \\
    \\
    \***Numbers Mentioned** \\
    1. 30,000 (employees losing health insurance) \\
    2. 2\% (of Walmart's workforce) \\
    3. \$500 million (surge in health care costs) \\
    4. \$170 million (more than estimated) \\
    5. 30 hours (work per week threshold) \\
    6. 2012 (year when Walmart stopped offering insurance to new hires under 30 hours) \\
    7. 2015 (year when premiums will rise) \\
    8. 40\% (of workforce on the basic plan) \\
    9. 19\% (increase in premiums) \\
    10. \$21.90 (new premium per pay period) \\
    \\
    \***Reasoning steps** \\
    The news content directly states that Walmart will no longer offer health insurance to 30,000 of its employees. \\

    To convert this number into thousands (K): \\
    - 30,000 employees = 30K employees \\
    
    Therefore, the numeral in the headline should be 30. \\    \hline
    \end{tabular}
  % \caption{\label{tbl:demonstration_paraphrase}
  % }
  
\end{table}

\begin{table}[htbp]
    \centering
    \begin{tabular}{p{0.9\textwidth}}
    \hline
    \textbf{Example 5: Rounding} \\
    \hline
    \textbf{Input:} \\
    \textbf{News:} (Apr 6, 2011 10:10 AM CDT) Well that was fast. Scott Walker did an about-face yesterday and demoted the son of a major campaign donor from his \$81,500-per-year Commerce Department gig—to his former \$64,728 job. Brian Deschane will return to his previous post as a bureau director at the Department of Regulation and Licensing, the Milwaukee Journal-Sentinel reports. The 27-year-old, who has no college degree and little experience (but does have two drunk-driving convictions!), had been in that job just two months when he was promoted to head of Wisconsin's environmental and regulatory affairs. Among those Deschane beat out for the job: a former Cabinet secretary with a doctoral degree and eight years experience overseeing oil contamination cleanup, and a professional engineer who had held the job since 2003 under the last governor. Neither was even interviewed. When Governor Walker learned the details of this agency staffing decision, he directed his administration to move in another direction, a Walker spokesman says (last week, this same rep called Deschane a natural fit for the position). But the demotion didn't satisfy some Democrats. I want to make sure he was properly hired for that job in the first place, said Assembly Minority Leader Peter Barca. Click for more on the brouhaha. \\
    \textbf{Headline:}  Headline: Walker Demotes Donor's Kid— to Puny \$\_\_\_\_K Job. \\
    \textbf{Correct Number:} 65. \\
    \textbf{System Instruction: } You will be given a piece of news after prefix "News: ". You will be given a masked headline of the news after prefix "Headline:". You will be provided the correct missing numeral after prefix "Correct Number:". Please emphasise what Topic the headline should focus on, list all Entities and Numbers mentioned in the news, and elaborate on the Reasoning steps for how to calculate the correct number in the headline. \\
    \hline
    \textbf{Output:} \\
    \***Topic the headline should focus on** \\
    The headline should focus on the new, lower salary of the demoted son of a major campaign donor. \\
    \\
    \***Entities Mentioned** \\
    1. Scott Walker \\
    2. Brian Deschane \\
    3. Assembly Minority Leader Peter Barca \\
    4. Former Cabinet secretary (unnamed) \\
    5. Professional engineer (unnamed) \\
    \\
    \***Numbers Mentioned** \\
    1. \$81,500 (initial salary at Commerce Department) \\
    2. \$64,728 (new salary after demotion) \\
    3. 27 (age of Brian Deschane) \\
    4. 2 (months in the initial job before promotion) \\
    5. 2003 (year since the professional engineer held the job) \\
    6. 8 (years of experience of the former Cabinet secretary) \\
    \\
    \***Reasoning Steps** \\
    The news content states that Brian Deschane was demoted from his \$81,500-per-year job to his former \$64,728 job. \\
    
    To convert this new salary into thousands (K): \\
    $\$64,728 \approx \$65,000$ \\
    
    Therefore, the numeral in the headline should be 65. \\
    \hline
    \end{tabular}
  
\end{table}



\end{document}
