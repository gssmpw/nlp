% NAACL 2025 instructions:
% https://aclrollingreview.org/cfp#long-papers
%Long papers must describe substantial, original, completed and unpublished work. Wherever appropriate, concrete evaluation and analysis should be included. Long papers may consist of:
%up to eight (8) pages of content
%plus up to one page for limitations (required, see below) and optionally ethical considerations
%plus unlimited pages of references
%Submissions that exceed the length requirements, or are missing a limitations section, will be desk rejected.
% Final versions of accepted papers will be given one additional page of content (up to 9 pages for long papers, up to 5 pages for short papers) to address reviewers’ comments.

% This must be in the first 5 lines to tell arXiv to use pdfLaTeX, which is strongly recommended.
\pdfoutput=1
% In particular, the hyperref package requires pdfLaTeX in order to break URLs across lines.

\documentclass[11pt]{article}

% Change "review" to "final" to generate the final (sometimes called camera-ready) version.
% Change to "preprint" to generate a non-anonymous version with page numbers.
%\usepackage[review]{acl}
\usepackage[final]{acl}

% Standard package includes
\usepackage{times}
\usepackage{latexsym}

% For proper rendering and hyphenation of words containing Latin characters (including in bib files)
\usepackage[T1]{fontenc}
% For Vietnamese characters
% \usepackage[T5]{fontenc}
% See https://www.latex-project.org/help/documentation/encguide.pdf for other character sets

% This assumes your files are encoded as UTF8
\usepackage[utf8]{inputenc}

% This is not strictly necessary, and may be commented out,
% but it will improve the layout of the manuscript,
% and will typically save some space.
\usepackage{microtype}

% This is also not strictly necessary, and may be commented out.
% However, it will improve the aesthetics of text in
% the typewriter font.
\usepackage{inconsolata}

%Including images in your LaTeX document requires adding
%additional package(s)
\usepackage{graphicx}
\usepackage{epstopdf} 

% If the title and author information does not fit in the area allocated, uncomment the following
%
%\setlength\titlebox{<dim>}
%
% and set <dim> to something 5cm or larger.
\usepackage{float}
\usepackage{stfloats}
\usepackage{array}
\usepackage{svg}
% \usepackage[table]{xcolor}
\usepackage{xcolor}
% for \rowcolor
\usepackage{colortbl}
% for \todo
\usepackage{todonotes}

%\title{Fine-tuning Large Language Models with Chain of Thought Teaching Signals for Number-Focus Headline Generation}
%\title{Optimizing Rationales via Constructing Topic and numerical Preferences to Teach LLMs for Number-Focused Headline Generation}
%\title{Generating Topic and numerical-based Chain-of-Thought Rationales to Teach LLMs for Number-focused Headline Generation}
%\title{Generating Topic-Entity-numerical-reasoning Rationales to Teach LLMs for Number-focused Headline Generation}
% Avoid two "generate" in the above version
\title{Teaching Large Language Models Number-Focused Headline Generation With Key Element Rationales}

% Author information can be set in various styles:
% For several authors from the same institution:
% \author{Author 1 \and ... \and Author n \\
%         Address line \\ ... \\ Address line}
% if the names do not fit well on one line use
%         Author 1 \\ {\bf Author 2} \\ ... \\ {\bf Author n} \\
\author{Zhen Qian \and Xiuzhen Zhang\thanks{Corresponding author} \and Xiaofei Xu \and Feng Xia \\
         School of Computing Technologies, RMIT University, Australia \\ 
         \texttt{\{s3888611,xiaofei.xu2\}@student.rmit.edu.au;} \\
         \texttt{\{xiuzhen.zhang,feng.xia\}@rmit.edu.au}
}


% For authors from different institutions:
% \author{Author 1 \\ Address line \\  ... \\ Address line
%         \And  ... \And
%         Author n \\ Address line \\ ... \\ Address line}
% To start a separate ``row'' of authors use \AND, as in
% \author{Author 1 \\ Address line \\  ... \\ Address line
%         \AND
%         Author 2 \\ Address line \\ ... \\ Address line \And
%         Author 3 \\ Address line \\ ... \\ Address line}

% %\author{First Author \\
%   Affiliation / Address line 1 \\
%   Affiliation / Address line 2 \\
%   Affiliation / Address line 3 \\
%   \texttt{email@domain} \\\And
%   Second Author \\
%   Affiliation / Address line 1 \\
%   Affiliation / Address line 2 \\
%   Affiliation / Address line 3 \\
%   \texttt{email@domain} \\}

% \author{
%  \textbf{First Author\textsuperscript{1}},
%  \textbf{Second Author\textsuperscript{1,2}},
%  \textbf{Third T. Author\textsuperscript{1}},
%  \textbf{Fourth Author\textsuperscript{1}},
% \\
%  \textbf{Fifth Author\textsuperscript{1,2}},
%  \textbf{Sixth Author\textsuperscript{1}},
%  \textbf{Seventh Author\textsuperscript{1}},
%  \textbf{Eighth Author \textsuperscript{1,2,3,4}},
% \\
%  \textbf{Ninth Author\textsuperscript{1}},
%  \textbf{Tenth Author\textsuperscript{1}},
%  \textbf{Eleventh E. Author\textsuperscript{1,2,3,4,5}},
%  \textbf{Twelfth Author\textsuperscript{1}},
% \\
%  \textbf{Thirteenth Author\textsuperscript{3}},
%  \textbf{Fourteenth F. Author\textsuperscript{2,4}},
%  \textbf{Fifteenth Author\textsuperscript{1}},
%  \textbf{Sixteenth Author\textsuperscript{1}},
% \\
%  \textbf{Seventeenth S. Author\textsuperscript{4,5}},
%  \textbf{Eighteenth Author\textsuperscript{3,4}},
%  \textbf{Nineteenth N. Author\textsuperscript{2,5}},
%  \textbf{Twentieth Author\textsuperscript{1}}
% \\
% \\
%  \textsuperscript{1}Affiliation 1,
%  \textsuperscript{2}Affiliation 2,
%  \textsuperscript{3}Affiliation 3,
%  \textsuperscript{4}Affiliation 4,
%  \textsuperscript{5}Affiliation 5
% \\
%  \small{
%    \textbf{Correspondence:} \href{mailto:email@domain}{email@domain}
%  }
% }

\begin{document}
\maketitle
\begin{abstract}
%While state-of-the-art pre-trained language models (PLMs) and large language models (LLMs) have proven exceptional in 
% While Large Language Models (LLMs) have proven powerful for news headline generation in terms of textual quality,
% %when measured by semantic similarity, 
% the numbers in the generated headlines are often inaccurate. 
% On the other hand, studies for solving word math problems are focused on only numerical reasoning to obtain the correct number, without needing to generate texts. 
% %The challenge in number-focused headline generation lies in maintaining both textual quality and numerical accuracy. 
%To address this challenge, 
%Number-focused headline generation poses a unique challenge for Large Language Models (LLMs), requiring both high-quality text and precise numerical accuracy.
Number-focused headline generation is a summarization task 
that requires both high textual quality and precise numerical accuracy, which poses a unique challenge for Large Language Models (LLMs). 
Existing studies in the literature focus only on either textual quality or numerical reasoning and thus are inadequate to address this challenge. 
In this paper, we propose a novel chain-of-thought framework for using rationales comprising key elements of the Topic, Entities, and Numerical reasoning (TEN) in news articles to enhance the capability  
for LLMs to generate topic-aligned high-quality texts with precise numerical accuracy. 
Specifically, a teacher LLM is employed to generate TEN rationales as supervision data, which are then used to teach and fine-tune a student LLM. 
%This process aims to improve automatic generation of rationales, ultimately enhancing the student LLM's ability to generate numerical headlines. 
Our approach teaches the student LLM automatic generation of rationales with enhanced capability for numerical reasoning and topic-aligned numerical headline generation.  
%jz3: I explained earlier directly. 
%In our framework, a rationale refers to the key intermediate steps and elements that must be considered before generating a headline.
%to refine and optimize the student LLM’s performance  
%\todo{XF: maybe we are optimizing the student LLM?}, 
%Importantly, we automatically construct topic and numerical-based preferences and leverage direct preference optimization for automatic rationale generation. 
Experiments show that our approach achieves superior performance %for number-focused headline generation 
in both textual quality and numerical accuracy.  
%jz0: I'm trying to clone this into my account and make it public.
Our implementation is publicly available at 
\texttt{https://github.com/TEN-Sum/TEN}.
%\footnote{https://anonymous.4open.science/r/TEN-4664}. 
\end{abstract}

\section{Introduction}

%jz1: below please cite a few papers in the text below. You can borrow from the NumHG paper. 
Headline generation, an important task in abstractive summarization, aims to condense a news article into a single line of text. 
%Text summarization models have proven effective for this purpose. 
%Text summarization models have been developed for headline generation.  
In the literature, text summarization models employ pre-trained language models \citep{lewis_bart_2019, raffel_exploring_2023, zhang_pegasus_2020} and large language models (LLMs)~\cite{jin_comprehensive_2024} have shown 
%effective for generation of headlines  high textual quality. 
high textual quality for headline generation. 

Numerical facts are crucial elements for modern news articles, and headlines often include numerals to enhance conciseness and attract readers' attention. A headline like "Pink Floyd reaches deal with Sony to sell music catalogue for \underline{\$400m}"\footnote{https://www.theguardian.com/music/2024/oct/02/pink-floyd-catalog-sony} immediately grabs readers' interest. 

%jz1:The current example in Fig. 1 can not show how the correct number is obtained from the news article. Please include relevent textual context in the source document.  
\begin{figure}[t]
  \centering
  \includegraphics[width=\columnwidth]{f1}
 % \caption{Example of Auto-Generated Rationale for Teaching LLM Headline Generation using the TEN Scheme}
 \caption{An example TEN Rationale for key elements of \underline{T}opic (green), \underline{E}ntities (blue) and \underline{N}umerical reasoning (purple).}
  \label{f1}
\end{figure}

Research shows that generating headlines with correct numbers requires mathematical reasoning capabilities in text summarization models~\cite{huang-etal-2024-numhg}. 
Obtaining correct numbers in headlines can involve mathematical operations such as addition, subtraction, and rounding of numbers from the source news articles. 
%\todo[color=green]{numhg paper now published in LREC-COLING 2024 --> done}
% jz1: some text to describe the example in Fig. 1. needed -- explain how the correct number is obtained (what operation) and what text summarization. 
As shown in Fig.~\ref{f1}, the news article covers the St. Louis protests, mentioning various entities and numbers. To generate an accurate headline, the language model must first identify the most newsworthy aspect of the event—the number of injured police officers. It then needs to calculate the correct number based on the information provided. In this case, the headline's number, 10, is not explicitly stated but requires addition (9 plus 1).
%jz1:below, citation needed. 
Number-focused headline generation requires 
not only text summarization to produce high quality text but also numerical reasoning within the textual context to generate the correct numbers. 

Existing studies on text summarization and numerical reasoning are inadequate for this challenging numerical headline generation problem~\citep{huang-etal-2024-numhg}. For text summarization, state-of-the-art pre-trained language models (PLMs)~\citep{lewis_bart_2019,zhang_pegasus_2020,liu_brio_2022} have relied on supervised fine-tuning to develop their summarization abilities. Researchers have also applied Chain-of-Thought (CoT) prompting~\citep{wang_element-aware_2023}, reinforcement learning~\citep{stiennon_learning_2022}, and direct preference optimization (DPO)~\citep{rajpoot_team_2024} to large language models (LLMs), aiming to improve their summarization quality. However, these methods focus on textual quality and do not address numerical accuracy.
%in the summaries. 

On the other hand, numerical reasoning models mainly focus on tasks that require producing a single numerical answer, such as solving word math problems, rather than generating text that includes numbers~\citep{ling_program_2017,amini_mathqa_2019,chiang_semantically-aligned_2019, cobbe_training_2021,wei_chain--thought_2023}. 
%jz3: below sentence "rationale" means "explanation", different from our definition. As rationale is only defined later, so rewrite and avoid using the word "rationale". 
Researchers have shown that language models' proficiency in solving these tasks can be enhanced through 
%rationale-augmented training 
explanation of intermediate steps
~\citep{amini_mathqa_2019, chiang_semantically-aligned_2019, cobbe_training_2021, wang_t-sciq_2023}, verification~\citep{cobbe_training_2021, wang_math-shepherd_2024}, and reinforcement learning~\citep{wang_math-shepherd_2024}. 
However, these techniques are developed in a setting where the question is given and they only need to infer the correct number as the final output. 

%When it comes to generating a numerically accurate headline, the challenge is twofold: it requires generation of concise text that captures both the essential topic and accurate numerical facts about the topic from the source news article. 
%This task involves not only text summarization but also the ability to extract and contextualize precise numbers within the textual context. 
%The model for number-focused headline generation needs to not only generate high quality text but also perform numerical reasoning within the textual context to generate the correct numbers. 

%jz3: below sentence needs more details. See my re-write.
In this paper, we propose a novel Chain-of-Thought (CoT) framework that uses rationales comprising key elements of \underline{T}opic, \underline{E}ntity and \underline{N}umerical reasoning (TEN) to teach and fine-tune LLMs for number-focused headline generation. 
Here rationales refer to textual descriptions for the key elements in a news article -- topics, entities, and numbers and their intermediate reasoning steps. These key element rationales can be used to enhance LLMs for the generation of topic-aligned headlines with higher numerical accuracy.  
%and critical elements that must be carefully considered before generating a headline.

Instead of costly manual annotation of TEN rationales, we propose to fine-tune open-source LLMs (e.g. Mistral 7B) to automatically generate such rationales for numerical headline generation.   
%In our approach, we first generates TEN reasoning rationales and then the news headline; for feasibility, ideally the base LLM should be an open-source LLM. 
%Building upon the teacher-student fine-tuning framework \citep{wang_t-sciq_2023}, our approach decomposes the rationales and optimizes the student LLMs via automatically generated Topic-Entity-Number-reasoning preferences. 
To enhance the capability for an open-source LLM to generate TEN rationales, we employ the teacher-student knowledge distillation framework~\cite{wang_t-sciq_2023} and leverage a powerful teacher LLM (e.g. GPT 4o) to generate TEN rationals as supervision data to fine-tune the open-source LLM as a student. 
Experiments show that our approach can achieve significant improvement over strong baselines in both textual quality and numerical accuracy. 

% jz3: training comprises two phases; inference is not a phase of training.
% jz3: but the description below is duplicated in Section 3. Such details are not needed for Introduction and so I removed it. 
% Training of the system comprises two phases -- 
% teacher generation of supervision data and fine-tuning student LLMs.
% %, and inference with the student LLMs. 
% In the supervision data generation phase, we instruct the teacher LLM to generate TEN rationales. 
% %that addresses two key issues: topic alignment and numerical reasoning. 
% %These rationales comprise three essential elements for headline generation: (1) Topic [T]; (2) Entities [E]; and (3) numerical Reasoning [N]. 
% In the fine-tuning phase, we develop two independent student LLMs: the rationale generator is fine-tuned to generate TEN reasoning rationales, while the headline generator is to generate headlines. 
% %jz3: below citations needed for DPO. 
% We further refine the rationale generator using DPO. The preference data for DPO are automatically generated to favour rationales that lead to headlines with matching topics and accurate numbers. 
% At the inference stage, the two fine-tuned student LLMs are applied sequentially. First, the rationale generator produces TEN rationales and then the headline generator uses these rationales together with the news articles as inputs to create the final headlines.

Contributions of our research are three fold:
\begin{itemize}
    \item We propose a CoT framework that uses rationales of key elements Topic, Entities and Numerical reasoning for LLMs to generate number-focused headlines. 
    %scheme to generate rationales automatically and use them as teaching signals to train student models to produce CoT-based headlines via fine-tuning.
    \item To enhance the capability for LLMs to generate topic-aligned headline text with high numerical accuracy, we apply the teacher-student framework to distill knowledge from a powerful teacher LLM to fine-tune an open-source LLM for automatic generation of TEN rationales.   
    %We distill TEN rationales from the teacher LLM GPT-4o to fine-tune student LLMs Mistral-7B and Llama-3.1-8B on the benchmark dataset NumHG and XSum. Our models achieve higher numerical accuracy than existing methodologies by 2.26 on NumHG, and 2.56 on XSum.
    \item We further develop a strategy based on Direct Preference Optimization \citep{rafailov_direct_2023} for LLMs to refine generation of TEN rationales. 
    %Our experiment results show that both ROUGE scores and accuracy scores are improved for the generated headlines.
\end{itemize}


\section{Related Work}

\subsection{Headline Generation}

Headline generation, a form of extreme text summarization, requires producing highly condensed, single-sentence summaries that capture the key information in a news article. 
%most important idea of the input text. 
% In the pioneering studies~\citep{narayan_dont_2018} and \citep{rush_neural_2015} at an early stage, 
%are pioneering work for extreme summarization. 
%They developed 
In early studies~\citep{rush_neural_2015,narayan_dont_2018}, models are supervise-trained on datasets containing single-sentence summaries (XSum)~\citep{narayan_dont_2018} and news-headline pairs (Gigaword) \citep{rush_neural_2015} However, their CNN-based and RNN-based approaches have since been outperformed by transformer-based models. 
%in the extreme summarization task. 
Recent studies show that transformer-based PLMs such as BART~\citep{lewis_bart_2019}, PEGASUS~\citep{zhang_pegasus_2020}, and BRIO~\citep{liu_brio_2022} can be fine-tuned on XSum and Gigaword to achieve promising results for extreme summarization and headline generation. 

While PLMs laid the bedrock for summarization, the advancement of LLMs has pushed the boundaries further. Several LLM-based approaches have emerged for general text summarization. Recent works have leveraged CoT prompting for summarization, proposing a "Summary Chain-of-Thought" method that guides LLMs to focus on key elements and generate summaries step-by-step~\citep{wang_element-aware_2023}. To further enhance summary quality, reinforcement learning methods have been employed to optimize LLMs based on human preferences~\citep{stiennon_learning_2022}. LLM-based approaches have also been tailored specifically for headline generation. For instance, leveraging reinforcement learning, \citet{tan_enhancing_2024} focus on creating personalized headlines for content recommendation. 
These approaches, whether PLM-based or LLM-based, focus on the text quality of summarization and the numerical accuracy is overlooked.
%\todo[color=green]{XF: maybe we need some related work to highlight the challenges from headline generation: 2020 extreme headline --> tbc}. 

%There are some closely related works. 
%Closely related 
Research on number-focused headline generation is reported recently. 
\citet{huang-etal-2024-numhg} assess the performance of PLMs in number-focused headline generation, but they do not provide strategies to enhance the models' numerical accuracy. \citet{rajpoot_team_2024} apply DPO to optimize headline generation using a preference dataset designed to train the model to favor headlines with correct numbers. While this preference for correct numbers can improve numerical accuracy, solely relying on it may degrade the textual quality of the generation. 
%Unlike previous approaches, our approach aims to enhance textual quality in headline generation as well as improve language models' numerical reasoning ability when generating headlines. 
% includes components explicitly designed to 

% \subsection{numerical Reasoning}
% \todo{XF: Do we discuss approaches other than providing rationales?} 
% Learning from human crafted rationales helps the performance, but requires human efforts
% \noindent{\textbf{Learning from human-crafted rationales.}} Researchers have demonstrated that language models' numerical reasoning abilities can be enhanced through learning from human-crafted rationales. 
% % rationale-augmented training. 
% One line of research employs humans to create natural language rationales for training~\citep{ling_program_2017}. They were the first to propose that language models could achieve higher accuracy in solving word math problems when required to generate both final answers and intermediate reasoning steps. 
% %Other researchers have shown that creating 
% Another line of research utilizes symbolic rationales for improving numerical reasoning abilities. For instance, symbolic equations can be used as intermediate steps to help solve word math problems~\citep{chiang_semantically-aligned_2019}. And in~\citep{amini_mathqa_2019}, the authors introduce a new representation language to model the intermediate steps so as to improve both the performance and the interpretability of the learned models. While effective, these methods rely heavily on human input, making them resource-intensive and potentially less scalable. 

% Learning from rationales generated by teacher LLMs helps the performance, and less human efforts
% \noindent{\textbf{Learning from teacher LLMs.}} 
% \subsection{Learning from LLM Generated TEN rationales.}
\subsection{CoT Prompting for Rationale Generation}

CoT prompting has gained great popularity due to its potential to unlock LLMs' reasoning capabilities by simply instructing them to generate intermediate steps as rationales before reaching a final answer~\citep{wei_chain--thought_2023}. For example, one can utilize CoT reasoning by simply adding the phrase "let's think step by step" to the end of each question~\citep{kojima_large_2023}. This approach is improved by a two-step process to generate rationales~\citep{zhang_automatic_2022}: first, selecting representative questions to generate exemplar rationales, and then using these representative rationales as demonstrations for LLMs to generate reasoning steps for other questions in the dataset. This idea is further enhanced by including the correct solution in prompts can enhance the quality of LLM-generated rationales~\citep{magister_teaching_2023}. 
 

%, specifically addressing the numerical accuracy issues in headline generation. 
% Our work also follows a two-step process to instruct an LLM to generate TEN rationales automatically, but we focus on creating rationales suitable for number-focused headline generation.

% Better rationales gives better performance
% \noindent{\textbf{Generating rationales through CoT prompting.}} In the LLM era, CoT prompting has gained great popularity due to its potential to unlock LLMs' reasoning capabilities by simply instructing them to generate intermediate steps before reaching a final answer~\citep{wei_chain--thought_2023}. CoT prompting has been applied to various tasks, including automatic rational generation. \citep{kojima_large_2023} demonstrate that CoT reasoning steps can be generated in a zero-shot manner by simply adding the phrase "let's think step by step" to the end of each question. \citep{zhang_automatic_2022} propose a two-step process to generate rationales automatically: first, selecting representative questions to generate exemplar rationales, and then using these representative rationales as demonstrations for LLMs to generate reasoning steps for other questions in the dataset. This idea is further enhanced by including the correct solution in prompts can enhance the quality of LLM-generated rationales~\citep{magister_teaching_2023}. Our work also follows a two-step process to instruct an LLM to generate TEN rationales automatically, but we focus on creating rationales suitable for number-focused headline generation. 

Especially for word math problems, research shows that LLM's numerical reasoning ability can be improved by learning from human-crafted rationales, including natural language intermediate reasoning steps~\citep{ling_program_2017} and symbolic representations like equations~\citep{chiang_semantically-aligned_2019,amini_mathqa_2019}. 
%While these methods have proven effective in enhancing model performance for tasks such as solving word math problems, 
But these methods rely on human annotations and therefore are costly. 
% jz3:I reworded and I think it sits here better. 
%\todo{XF: I feel this paragraph does not fit in here, can we put this paragraph at the beginning of this subsection?}

%potentially less scalable. 
% Better rationales gives better performance
\subsection{Learning from Teacher LLM Generated Rationales}
% summarize non-CoT in one or two sentences
% \noindent{\textbf{Learning from LLM generated TEN rationales.}} 

%An alternative approach that requires less human efforts involves 
% Learning from TEN rationales generated by teacher LLMs is a scalable alternative to reduce human efforts. 
Learning from rationales generated by teacher LLMs is a scalable alternative to human annotation. 
Research has shown that CoT reasoning steps generated from teacher LLMs can be used to fine-tune smaller student language models~\cite{ho_large_2023,hsieh_distilling_2023} that 
%. The CoT fine-tuned student models sometimes even outperform the teacher model in certain tasks. 
may even outperform the teacher LLM for some tasks.
% They validated their approach on four types of complex reasoning including arithmetic, symbolic, common sense, and other reasoning. 
%Similarly, in~\citep{hsieh_distilling_2023}, the authors suggest that the student model can achieve comparable performance with reduced training data when fine-tuned with LLM-generated rationales. They test their approach on several NLP tasks including arithmetic word math problems. 
Such teacher-student knowledge distillation has also been applied to multi-modal training for science QA~\cite{wang_t-sciq_2023}, which involves numerical reasoning.
The authors propose that mixing simple and complex reasoning in supervision data can enhance student LLMs' performance~\cite{wang_t-sciq_2023}. 
%Our work also follows a two-step process to instruct an LLM to generate TEN rationales automatically. But unlike existing works, our approach focuses on creating rationales that address the topic alignment and numerical reasoning so that two independent student LLMs can be further fine-tuned for number-focused headline generation tasks. 
Our approach also leverages the teacher-student knowledge distillation framework. 
Unlike existing work, we focus on rationales for topic alignment as well as numerical reasoning in numerical text generation.
%, which is not studied in existing work. 

Researchers have explored various approaches to enhance rationale quality, including the use of verifiers~\cite{cobbe_training_2021}, majority voting~\cite{wang_self-consistency_2023} and reinforcement learning~\cite{wang_math-shepherd_2024}. 
% In \citep{cobbe_training_2021}, the authors propose to train a verifier to rank the probabilities of correctness for model-generated rationales and select the most likely one. Similarly, the majority vote has been utilized to choose the most consistent output from several output rationales generated by LLMs~\citep{wang_self-consistency_2023}. 
Our approach, which leverages DPO for refining rationale generation, is closely related to reinforcement learning strategies~\citep{wang_math-shepherd_2024}. 
%the authors propose to automatically assess the step-wise correctness of model-generated rationales, 
%train a reward model using the labelled data, and further
%the authors propose to improve %LLMs' performance 
%LLM rationale generation through reinforcement learning. 
%Our approach of using DPO to refine also 
%explores the effectiveness of DPO in enhancing the quality of the generated rationales. 
%jz3: pelase check the my re-write below. 
However, unlike previous work that focuses solely on reward models for numerical reasoning rationales, our approach develops preference datasets tailored to both nuanced topic alignment and complex numerical reasoning. 

\begin{figure*}[t]
  \centering
  \includegraphics[width=\textwidth, trim={2cm 2.45cm 0cm 1.7cm}, clip]{f2}
%  \caption {Three-phase pipeline of the proposed TEN fine-tuning scheme: (1) Generating supervision data, (2) Fine-tuning student LLMs, and (3) Inferring with fine-tuned student LLMs for headline generation.}
  \caption {Our TEN approach for automatic generation of rationales to enhance numerical headline generation.}
  \label{fig:three_phase_of_TEN}
\end{figure*}

\section{Methodology}
%\todo[color=green]{XF: figure 2 ECNC need to be updated to TEN --> done}

%This section presents our TEN reasoning rationales  fine-tuning scheme TEN, as shown in Figure~\ref{fig:three_phase_of_TEN}. 
This section presents our framework for leveraging the teacher-student knowledge distillation framework to fine-tune LLMs for automatic generation of TEN rationales to enhance LLM headline generation.
%jz3: below, CoT rationale > TEN rationale
We employ a teacher LLM (e.g. GPT 4o) to generate TEN rationales 
%for number-focused headline generation and 
%jz3: below, teaching data > supervision data 
and use these rationales as supervision data to fine-tune a student LLM (e.g. Mistral-7B), including a rationale generator for automatic generation of TEN rationales and
a headline generator for headline generation. 

As shown in Fig.~\ref{fig:three_phase_of_TEN}, our approach adopts a teacher-student framework to fine-tune a (student) LLM to automatically generate TEN rationales.  
%framework comprises three phases: teacher generation of supervision data, fine-tuning student LLMs for rationale and headline generation, and inference with fine-tuned student LLMs. 
When generating supervision data to fine-tune a rationale generator, 
we prompt a teacher LLM to generate rationales for each news-headline pair in the dataset.  
The rationales are aimed to enhance the topic alignment and numerical reasoning capabilities for numerical headline generation, comprising key elements \underline{T}opic and \underline{E}ntities, 
as well as \underline{N}umbers in the news article and the intermediate reasoning steps to calculate the correct number in the headline. 
%These rationales are further decomposed into three essential elements (TEN) for headline generation. 

The teacher LLM generated rationales are used as supervision data to 
%When fine-tuning student LLM, we use the supervision data along with the input (news-article, headline) pairs to two independent student LLMs, 
%one fine-tuned for generating TEN rationales (the rationale generator) and another for generating headlines (the headline generator). 
fine-tune a student LLM as the rationale generator. 
We further refine the rationale generator using DPO. The preference data for DPO are automatically generated to favour rationales that lead to headlines with matching topics and accurate numerals. 
The news article and teacher LLM generated TEN rationales are then used to fine-tune another student LLM for headline generation. 
In the inference phase, the two fine-tuned student LLMs are used sequentially. The rationale generator will first produce TEN rationales for the input news article. The headline generator will then use the rationales together with the news articles as input to generate final headlines. 

%we generate the final headline in two steps.
\begin{figure*}[t]
  \includegraphics[width=\textwidth]{f3ab}
%  \parbox[b]{0.45\textwidth}{\centering Step 1: Generating representative TEN rationales via zero-shot prompting.}
  % \parbox[b]{0.45\textwidth}{\centering (a) An example zero-shot prompt for generating TEN rationales.}
  % \hspace{0.05\textwidth}
%  \parbox[b]{0.45\textwidth}{\centering Step 2: Generating teaching signals for the entire dataset via few-shot prompting.}  
  % \parbox[b]{0.45\textwidth}{\centering (b) The five-shot prompt to generate rationales for supervised fine-tuning student LLMs}  
  \caption {The process for Teacher LLM to generate TEN rationales for fine-tuning student LLMs}
  \label{fig:pre_tuning}
\end{figure*}

\subsection{Teacher LLM generation of TEN rationales}
In this phase, we focus on utilizing a teacher LLM to generate TEN rationales as supervision data. 
Figure~\ref{fig:pre_tuning} shows the process for generating this data. It is generated through a two-step process~\citep{zhang_automatic_2022}. 
% As shown in Figure~\ref{fig:pre_tuning}, 
%jz0: below, with the five examples, I believe we did 5-shot prompting here rather than zero-shot prompting? 
%jz0: I reworded the setence to avoid confusion. 
%In the first step, we carefully select representative examples and instruct a teacher LLM to generate rationales for these examples using zero-shot prompting.
In the first step, we instruct a teacher LLM with zero-shot prompting to generate demonstration TEN rationales for a few (five) representative examples for calculating numbers in headlines. 
In the second step, we employ these demonstration rationales as context to generate rationales for other examples in the whole training dataset using few-shot prompting. Specifically, we create demonstration rationales for five representative examples from the NumHG dataset \citep{huang-etal-2024-numhg}. 
%This dataset not only provides news-headline pairs but also includes human annotations detailing the types of mathematical operations required to derive the numerals in the headlines. 
The five examples are selected to represent five types of mathematical operations annotated in this dataset: (a) Copying: The numeral is directly copied from the article. (b) Addition: Numerals from the article are added to get the final numeral. (c) Subtraction: One numeral is subtracted from another. (d) Paraphrasing: The digits of the numeral are rewritten (e.g., changing 6,000 to 6k). and (e) Rounding: Only certain digits after the decimal point are retained. The details of the five demonstrations are shown in Appendix~\ref{appendix: five-demonstrations}.

%\todo{XF: It feels we can include those 5 samples in the appendix (e.g. using lstlisting).}

% \begin{enumerate}
%     \item Copying: The numeral is directly copied from the article.
%     \item Addition: Numerals from the article are added to get the final numeral.
%     \item Subtraction: One numeral is subtracted from another.
%     \item Paraphrasing: The digits of the numeral are rewritten (e.g., changing 6,000 to 6k).
%     \item Rounding: Only certain digits after the decimal point are retained.
% \end{enumerate}

We next instruct the teacher LLM to elaborate on the intermediate steps that lead to high-quality headlines with accurate numerals for the five selected examples. The prompts we use are shown in Figure~\ref{fig:pre_tuning}. 
%We structure these intermediate steps to focus on three key elements for the news article -- The Topic of the headline (T), the Entities mentioned (E), and the Numbers mentioned (N) -- as well as the intermediate reasoning steps to calculate the number in the reference headline. 
Note that the TEN rationale comprise key elements for the news article, including numerical reasoning steps. 
Note also that to enhance the reliability of the teacher LLM-generated rationales, we also provide the reference headlines and the correct numbers to the teacher LLM in prompts. The numbers in the reference headlines are masked to ensure the teacher LLM focuses on the topic when generating topic-alignment rationales. The correct numbers are provided separately as hints to improve accuracy in numerical reasoning rationales. We then manually review and refine these five rationales to adhere to the template, ensuring consistency across the entire dataset. Using these five example TEN rationales as demonstrations, we instruct the teacher LLM to generate TEN rationales for the complete training datasets through five-shot prompting. 

% \begin{enumerate}
%     \item Topic of headline (E)
%     \item Characters mentioned (C)
%     \item Numerals mentioned (N)
%     \item Calculations of Numerals (C)
% \end{enumerate}

\subsection{Fine-Tuning Student LLMs}
Inspired by~\citep{zhang_multimodal_2024}, we fine-tune two student LLMs independently, as illustrated in Figure~\ref{fig:three_phase_of_TEN}. The first student LLM (rationale generator) generates TEN rationales from news articles. The second model (headline generator), initialized from the same student LLM, is fine-tuned to predict headlines using both the news articles and the TEN rationales generated by the teacher LLM as inputs. 

We also apply DPO to the rationale generator to enhance its output quality. To construct the preference dataset for DPO, we first use the fine-tuned rationale generator to sample multiple rationales for each news article in the training data using a high temperature. Next, we use the headline generator to complete headlines based on the news articles and sampled rationales. We then build a pair of chosen and rejected rationales for each instance in the training data based on the following criteria: (a) Choose the rationale that leads to the headline with correct numerals and reject the one that results in a headline with incorrect numerals. (b) Choose the rationale with a high ROUGE score compared to the reference rationale and reject the one with a low ROUGE score. The flowchart for automatically constructing preference data is shown in Appendix~\ref{appendix:preference_data}. 
%\todo[color=green]{XF: Maybe an algorithm can better explain how preference is made. --> Done with a flowchart in appendix}

% \begin{itemize}
%     \item Choose the rationale that leads to the headline with correct numerals and reject the one that results in a headline with incorrect numerals.
%     \item Choose the rationale with a high ROUGE score compared to the reference rationale and reject the one with low ROUGE score.
% \end{itemize}

% \subsection{Inferring with Fine-Tuned Student LLMs}
% In this phase, we have obtained the fine-tuned student LLMs: the rationale generator and the headline generator. The two fine-tuned student LLMs are used sequentially. As shown in Figure~\ref{fig:three_phase_of_TEN}, the rationale generator will first produce TEN rationales for the input news article. Then, the headline generator will use the rationales together with the news articles as input to generate final headlines. 

% During the deployment phase, we use the rationale generator and headline generator in a pipeline. First, the rationale generator predicts TEN rationales based on the news articles. Then, the headline generator takes these predicted TEN rationales along with the original news articles to produce the final headlines. 

% \section{Experiment Setup}
\section{Experiments}
We evaluate our approach TEN against state-of-the-art baselines on benchmark datasets for number-focused headline generation. 
%from one representative PLM-based method and recent LLM-based methods for headline generation tasks. 
%\todo[color=green]{Experiment environment and software used need to be updated --> done}
%Experiments on NumHG are conducted on a system which consists of 32 cores, 128G memory and is equipped with an NVIDIA A100 (40G) GPU. Experiments on XSum are conducted on a system which consists of 32 cores, 128G memory and is equipped with 8 NVIDIA L40S (45G) GPU. 
In all experiments, GPT 4o is the teacher LLM for our TEN approach. 
All experiments were conducted on a system with 32 cores, 128GB memory and NVIDIA A100(40G) GPU. The estimated GPU usage for our experiments is approximately 2,000 hours.
All deep neural networks are implemented using Transformers~\citep{wolf2019huggingface} (distributed with the Apache-2.0 license) under the support of PyTorch~\citep{paszke2019pytorch} (distributed with the modified BSD-3-Clause license). 
%jz3: @Zhen: the real url for the anonymous github repo is needed below. 
% the repo content can be done after paper submission but the url must be real. 
Implementation details, including parameter efficient fine-tuning settings and hyperparameter settings are in Appendix~\ref{appendix:implementation}. 

\subsection{Datasets and Evaluation Metrics}
We evaluate our proposed approach using two real-world datasets: 
\begin{itemize}
	\item The NumHG dataset~\citep{huang-etal-2024-numhg} is a large dataset for number-focused headline generation that is also used for SemEval-2024 Task 7 ``NumEval: Numeral-aware language understanding and generation'' task.~\footnote{https://sites.google.com/view/numeval/numeval} 
 %It contains approximately 
 %\todo[color=green]{exact number maybe better --> done}27,74 news-headline pairs sourced from the news platform Newser\footnote{https://www.newser.com/}. 
 Each news article contains 200--300 words, and all headlines include numerals. This dataset provides human annotations of mathematical operations required to derive the numerals in each headline. 
 %To focus on validating the model's numerical reasoning ability, 
 We apply a pre-processing step to the dataset by removing duplicate samples and retaining only those with one number in the headlines. 
 %After this data pre-process, 
 In the end, we obtained 18,315 samples for training and 3,579 for testing.
    \item The XSum dataset~\citep{narayan_dont_2018} is 
    %\todo[color=green]{intro to XSum needed --> done} 
    an extreme summarization dataset comprising 226,771 BBC articles from 2010 to 2017, each accompanied by a single-sentence summary. 
    %These summaries, serving as introductory sentences for the articles, are crafted by the authors themselves. 
    We applied pre-processing and selected articles containing 200--500 words and summaries containing only a whole number. This resulted in 9,052 samples for training and 1,605 for testing. 
    % with one single whole number, , as headlines with multiple numerals are challenging to evaluate for numerical accuracy
\end{itemize}

% \subsection{Evaluation Metrics}
We adopt the evaluation metrics commonly used in existing studies~\citep{huang-etal-2024-numhg} to assess both the textual quality and numerical accuracy for headline generation. 
%For textual quality, 
We adopt ROUGE~\citep{lin2004rouge}, BERTScore~\citep{zhang2019bertscore}, and MoverScore~\citep{zhao2019moverscore} for textual quality. 
%They are calculated against ground truth headlines. 
For numerical accuracy, 
a generated headline's numeral is considered correct if it matches the numeral
in the reference headline. 
%and it matches the ground truth. 
We use the evaluation code~\footnote{https://github.com/ChunJiChen/NumEval\_Evaluation} from~\citet{huang-etal-2024-numhg} to automatically calculate these metrics. 

%jz3: below should move to later closer to the result table.
%The NumHG dataset includes annotations for the type of operations needed to obtain the correct numeral in headlines. This allows us to break down the numerical accuracy into three categories: overall accuracy, copy accuracy, and reasoning accuracy. Copy accuracy is the numerical accuracy when the correct numeral can be directly copied from the news article. Reasoning accuracy is the numerical accuracy when mathematical operations are required. For the XSum dataset, which lacks operation type annotations, we only report the overall accuracy. 

\subsection{Baselines}

We compare TEN against three baseline methods, including one representative PLM-based method and two recent LLM-based methods. 
\begin{itemize}
	\item BART~\citep{lewis-etal-2020-bart,huang-etal-2024-numhg} 
 %is a PLM widely applied for various tasks. As reported in the NumHG dataset~\citep{huang-etal-2024-numhg}, 
 is a PLM-based representative baseline for numerical headline generation that is shown to outperform other PLM-based methods like T5, Pegasus, SEASON, and BRIO in terms of numerical accuracy while maintaining comparable textual quality~\cite{huang-etal-2024-numhg}. 
 %Thus BART is selected as the representative PLM-based method. 
%jz3: below, NLC_NLP > NLC, please update the table. 
\item NCL (NCL\_NLP)~\citep{zhao_ncl_nlp_2024} is an LLM-based method from SemEval-2024 Task 7 that achieves reasonable result. 
% jz3: our implementation of NLC should use the same teacher and student LLMs for TEN. 
Similar to TEN, they also employ the teacher-student framework to generate CoT rationales to fine-tune student LLMs for headline generation. 
%a teacher LLM (GPT-3.5-Turbo) to generate TEN rationales as supervision data and fine-tune a student LLM (Mistral-7B) for headline generation. 
%We implement this baseline by instructing the teacher LLM to generate rationales based on the given news articles, using the correct headlines as hints. 
Different from TEN, NCL does not do the structured, element-wise rationales. 
Comparing TEN against NCL will help us understand the effectiveness of our TEN reasoning strategy. 
 %jz3: below, NP-problm > NPP, please update the table. 
    \item NPP (NP-Problem)~\citep{rajpoot_team_2024} is another LLM-based baseline from SemEval-2024 Task 7. NPP achieved the highest numerical accuracy and comparable text quality among all submissions. 
%    an LLM-based method that came from SemEval-2024's task 7 which is a shared task specifically for the number-focused headline generation. 
They fine-tune Mistral-7B for headline generation and further align the model-generated headlines through DPO. 
%Their method achieved the highest numerical accuracy in headline generation among the participants in SemEval-2024's task 7. 
As both TEN and NPP use DPO to refine headline generation, their comparison can reveal the utility of our strategy of preference data generation for DPO. 
    % Unlike our TEN scheme, which breaks down TEN rationales into four essential elements, this method doesn't specify particular focus areas. We then use these rationales as supervision data to fine-tune Mistral-7B. 
\end{itemize}

% The first is fine-tuned BART. \citet{huang_numhg_2023} have evaluated state-of-the-art PLMs for number-focused headline generation, including BART, T5, Pegasus, SEASON, and BRIO. Their experimental results show that fine-tuned BART achieves the highest numerical accuracy in headline generation while maintaining comparable textual quality. 

% We also choose \citet{rajpoot_team_2024}'s method as a baseline. They have fine-tuned Mistral-7B for headline generation and further aligned the model-generated headlines through DPO. Their method achieved the highest numerical accuracy in headline generation among the participants in SemEval-2024's task 7. 

% The final baseline method we select is proposed by \citet{zhao_ncl_nlp_2024}. They employ a teacher LLM (GPT-3.5-Turbo) to generate TEN rationales as supervision data and fine-tune a student LLM (Mistral-7B) for headline generation. This knowledge distillation approach has shown state-of-the-art performance in reasoning tasks \citep{wang_t-sciq_2023} but is rarely used in headline generation. We choose this baseline because we also use CoT fine-tuning and want to compare our TEN rationale decomposition scheme with the existing method. We implement this baseline by instructing the teacher LLM to generate rationales based on the given news articles, using the correct headlines as hints. Unlike our TEN scheme, which breaks down TEN rationales into four key aspects, this method doesn't specify particular focus areas. We then use these rationales as supervision data to fine-tune Mistral-7B. 



\begin{table*}[t]
  \centering
  \resizebox{\textwidth}{!}{
  \begin{tabular}{lccc|ccc|ccc|c}
    \hline
    &&\textbf{Num Acc}&&&\textbf{ROUGE}&&&\textbf{BERTScore}&&\textbf{MoverScore}\\
    \cline{2-11}
     & Overall & Copy & Reasoning & 1 & 2 & L & P & R & F1 & \\
    \hline
    BART & 71.59 & 76.54 & 61.82 & 48.13 & 22.76 & 43.36 & 49.29 & 50.81 & 50.60 & 60.34 \\
    NPP & 74.57 & 77.43 & 68.93 & 49.24 & 23.44 & 44.08 & 50.17 & 50.57 & 50.36 & 60.54 \\
    NCL & 74.94 & 78.43 & 68.06 & 50.03 & 24.72 & 45.39 & 53.44 & 51.19 & 52.31 & \textbf{60.97} \\
    % \rowcolor{lightgray}
    TEN (Ours) & \textbf{77.20} & \textbf{81.11} & \textbf{69.49} & \textbf{51.14} & \textbf{25.46} & \textbf{46.29} & \textbf{54.57} & \textbf{51.84} & \textbf{53.21} & \textbf{61.23} \\
    \hline
    % \hline
  \end{tabular}
  }
  % \caption{\label{tbl:main_result_numhg}
  %   Comparative analysis of numerical accuracy and textual quality metrics for baseline methods and our proposed approach on the \textbf{NumHG} dataset.
  % }
  \caption{\label{tbl:main_result_numhg}
    Numerical accuracy (\%) and textual quality score (\%) for TEN against baselines on \textbf{NumHG}. Higher numbers indicate better performance. Best results are in bold, where results within 0.5\% difference are deemed comparable.
  }
\end{table*}

\begin{table*}[t]
  \centering
  \resizebox{0.82\textwidth}{!}{
  \begin{tabular}{lc|ccc|ccc|c}
    \hline
    &\textbf{Num Acc}&&\textbf{ROUGE}&&&\textbf{BERTScore}&&\textbf{MoverScore}\\
    \cline{2-9}
     & Overall & 1 & 2 & L & P & R & F1 & \\
    \hline
    BART & 29.34 & 43.83 & 19.81 & 35.46 & 52.54 & 54.16 & 53.38 & 60.46 \\
    NPP & 30.15 & 46.32 & \textbf{22.58} & \textbf{38.08} & 55.60 & \textbf{55.82} & 55.73 & \textbf{61.35} \\
    NCL & 36.76 & 45.17 & 21.47 & 37.07 & 57.30 & 53.49 & 55.41 & \textbf{61.02} \\
    % \rowcolor{lightgray}
    TEN (Ours) & \textbf{39.07} & \textbf{46.63} & \textbf{22.50} & \textbf{38.36} & \textbf{58.81} & 54.57 & \textbf{56.70} & \textbf{61.36} \\
    \hline
    % \hline
  \end{tabular}
  }
  % \caption{\label{tbl:main_result_xsum}
  %   Comparative analysis of numerical accuracy and textual quality metrics for baseline methods and our proposed approach on the \textbf{XSum} dataset.
  % }
  \caption{\label{tbl:main_result_xsum}
    Numerical accuracy (\%) and textual quality metric score (\%) for TEN against baseline methods on  \textbf{XSum}. Higher numbers indicate better performance. The best results are in bold, where results within 0.5\% difference are deemed comparable.
  }
\end{table*}

% \begin{table}[t]
%   \centering
%   \resizebox{0.5\textwidth}{!}{
%   \begin{tabular}{lcccc}
%     \hline
%     & \multicolumn{2}{c}{\textbf{NumHG}} & \multicolumn{2}{c}{\textbf{Xsum}} \\
%     \hline
%     & NumAcc & ROUGE-1 & NumAcc & ROUGE-1 \\
%     \hline
%     Mistral-7B-v0.3 & & & & \\
%     \hline
%     \: TEN w/o DPO & 76.24 & 50.95 & 37.69 & 46.26 \\
%     \: TEN (Ours) & \textbf{77.20} & \textbf{51.14} & \textbf{39.07} & \textbf{46.63}\\
%     % \hline
%     \hline
%     Llama-3.1-8B & & & & \\
%     \hline
%     \: TEN w/o DPO & 76.06 & 50.69 & 36.76 & 45.89 \\
%     \: TEN (Ours) & \textbf{77.89} & \textbf{51.15} & \textbf{37.51} & \textbf{46.03} \\
%     \hline
%   \end{tabular}
%   }
%   \caption{\label{tbl:dpo_ablation}
%     Comparative analysis of headline generation performance: impact of refining model-generated rationales through Direct Preference Optimization (DPO) on two student models Mistral-7B-v0.3 and Llama-3.1-8B.
%   }
% \end{table}

% \section{Main Results}
\subsection{Experiment Results}

%\noindent \textbf{TEN vs Baselines.} Tables~\ref{tbl:main_result_numhg} and~\ref{tbl:main_result_xsum} show the performance of TEN against baseline methods on the NumHG and XSum datasets respectively. 
For fair comparison, all baselines use Mistral-7B, and we also use Mistral-7B as the student LLM for TEN.
Following existing studies in the literature~\cite{huang-etal-2024-numhg}, we employ numerical accuracy and textual similarity metrics ROUGE~\cite{lin2004rouge}, BERTScore~\cite{zhang2019bertscore} and MoverScore~\cite{zhao_moverscore_2019} for evaluation. 
Note that textual similarity metrics evaluate the complete headline text, including numbers as tokens, and therefore can  be seen as measuring the overall quality for number-focused headlines.

The NumHG dataset includes annotations for the type of operations needed to obtain the correct numeral in headlines. 
%This allows us to break down 
The numerical accuracy thus includes overall accuracy, copy accuracy (when numbers can be directly extracted from the news), and reasoning accuracy (when mathematical operations are required).   
%-- Copy accuracy is the numerical accuracy when the correct numeral can be directly copied from the news article. Reasoning accuracy is the numerical accuracy when mathematical operations are required. 
On the XSum dataset, which lacks operation type annotations, we only report the overall accuracy. 
%The results demonstrate that our approach outperforms existing methodologies. Note that the LLM baselines all utilize Mistral-7B
%\todo[color=green]{XF: suggest we use Mistral-7B instead of Mistral-7B-v0.3 in main content --> already done?}, 

Observe from Table~\ref{tbl:main_result_numhg} that on the NumHG dataset, TEN achieves an overall numerical accuracy of 77.20\%, surpassing BART by 5.61\%, NPP by 2.63\%, and NCL by 2.26\% (in absolute percentage points).
In Table~\ref{tbl:main_result_xsum}, on the XSum dataset, TEN reaches an overall accuracy of 
%\todo[color=green]{XF: these numbers should be updated --> done}
39.07\%, outperforming BART by 9.73\%, NPP by 8.92\%, and NCL by 2.31\%. While improving numerical accuracy, TEN also maintains mostly higher textual quality by textual similarity metrics ROUGE, BERTScore, and MoverScore. 

% jz3: some discussion is needed below. 
%Here: compare TEN against NP; compare TEN against NCL. Follow the description of baselines.
Models trained with our TEN framework outperform NCL, demonstrating that our TEN rationales are more effective than NCL's rationales that only explain how the correct number in the headline is obtained. Our approach also outperforms NPP, demonstrating that enhancing the intermediate rationale generation process is a more effective strategy for improving headline's numerical accuracy.

%\noindent \textbf{LLM-based Quality Evaluation.}
We further evaluated the quality of the generated news headlines using an LLM-based metric G-Eval \cite{liu_g-eval_2023}.
Recent research shows that LLMs can be used for evaluation of quality of generated texts and demonstrate strong correlation with human judgements. 
G-Eval leverages the capability of LLMs and Chain-of-Thoughts prmopts to assess the quality of model-generated texts. 
We employed G-Eval(GPT4) to evaluate four aspects of generated headlines: coherence, consistency, fluency, and relevance. As shown in Table~\ref{tbl:llm_based_quality_evaluation}, 
%TEN achieved the highest scores across all four dimensions.
TEN outperforms all baselines on NumHG and achieves comparable results on XSum. 


%jz0: Any results on XSum?
\begin{table*}[htbp]
  \centering
  \resizebox{\textwidth}{!}{

  % \begin{tabular}{lcccc}
  %   \hline
  %   &\textbf{Coherence}&\textbf{Consistency}&\textbf{Fluency}&\textbf{Relevance}\\
  %   BART&4.04&4.36&2.82&4.11\\
  %   NPP&4.17&4.65&2.91&4.22\\
  %   NCL&4.17&4.71&\textbf{2.96}&4.23\\
  %   TEN&\textbf{4.18}&\textbf{4.72}&\textbf{2.96}&\textbf{4.24}\\
  %   \hline
  % \end{tabular}  
  
  \begin{tabular}{l|cccc|cccc}
    \hline
    &\multicolumn{4}{c|}{\textbf{NumHG}} & \multicolumn{4}{c}{\textbf{XSum}}\\
    \hline    
    &\textbf{Coherence}&\textbf{Consistency}&\textbf{Fluency}&\textbf{Relevance}&\textbf{Coherence}&\textbf{Consistency}&\textbf{Fluency}&\textbf{Relevance}\\ \hline
    BART&4.0361&4.3647&2.8209&4.1068&3.3740&2.9179&2.6836&3.2840\\
    NPP&4.1734&4.6550&2.9068&4.2184&3.3782&2.9048&2.6758&3.2853\\
    NCL&4.1739&4.7108&\textbf{2.9616}&4.2264&3.3795&2.9037&2.6830&3.2767\\
    TEN&\textbf{4.1853}&\textbf{4.7210}&2.9594&\textbf{4.2436}&3.3733&2.9125&2.6802&3.2824\\
    \hline
  \end{tabular}
  }
  % \caption{\label{tbl:main_result_numhg}
  %   Comparative analysis of numerical accuracy and textual quality metrics for baseline methods and our proposed approach on the \textbf{NumHG} dataset.
  % }
  \caption{\label{tbl:llm_based_quality_evaluation}
    G-Eval scores for TEN against baselines on \textbf{NumHG} and \textbf{XSum}. Headlines are assessed in terms of Coherence (1-5), Consistency (1-5), Fluency (1-3), and Relevance (1-5). The higher numbers indicate better performance. Best results are in bold.}
\end{table*}

\subsection{Performance of TEN rationales and Teacher-student knowledge distillation}
%\noindent \textbf{Effect of TEN on GPT-4o.}
The structured rationales and teacher-student paradigm to distill knowledge from a teacher LLM (GPT-4o) to a student LLM (Mistral 7B and Llama 3.1) are important parts of our TEN framework. 
%Effectiveness of the system depends on the performance of the teacher LLM. 
To evaluate the effectiveness of TEN rationales and if GPT-4o is a good teacher LLM, we evaluated the performance of GPT-4o with and without TEN structured rationales under zero-shot prompting. %Specifically, GPT-4o is evaluated using NumHG test data in two settings. Setting 1 "without TEN" is to prompt GPT-4o to directly generate headlines based on news articles. Setting 2 "with TEN" is to include the instructions for GPT-4o to generate both TEN-structured rationales and headlines. 
The results are shown in Table~\ref{tbl:main_result_gpt4o_both}. It can be seen that by prompting GPT-4o to generate TEN rationales, performance improved significantly both for numerical accuracy and textual quality.

%jz0: below overall table replace the two separate tables on NumHG and XSum. 
%jz0: The headings "NumHG w/o TEN" is not quite right though --> "w/o TEN" as the method. Please reformat the table to match the other tables using two datasets NumHG and Xsum as columns. You have to use ROUGE-1 and BERSTScore-F1
\begin{table*}[htbp]
  \centering
  \resizebox{\textwidth}{!}{
  % \begin{tabular}{l|c|ccc|ccc|c}
  %   \hline
  %   & \textbf{Num Acc}&\textbf{ROUGE}&&&\textbf{BERTScore}&&&\textbf{MoverScore}\\
  %   & & 1 & 2 & L & P & R & F1 & \\
  %   \hline
  %   NumHG w/o TEN & 21.39 & 35.59 & 13.63 & 30.82 & 30.33 & 48.38 & 39.21 & 56.53 \\
  %   NumHG w TEN & 33.01 & 35.47 & 13.14 & 30.96 & 33.06 & 46.82 & 39.86 & 56.72 \\   
  %   XSum w/o TEN & 6.23 & 22.59 & 4.98 & 17.77 & 19.70 & 25.71 & 22.76 & 54.58 \\
  %   XSum w TEN & 9.91 & 22.18 & 5.14 & 17.65 & 21.43 & 24.73 & 23.14 & 54.63 \\
  %   \hline
  %   % \hline
  % \end{tabular}
  \begin{tabular}{l|cccc|cccc}
    \hline
    &\multicolumn{4}{c|}{\textbf{NumHG}} & \multicolumn{4}{c}{\textbf{XSum}}\\
    \hline
     & Num Acc & ROUGE-1 & BERTScore-F1 & MoverScore & Numm Acc & ROUGE-1 & BERTScore-F1 & MoverScore\\
    \hline
    w/o TEN & 21.39 & 35.59 & 39.21 & 56.53 & 6.23 & 22.59 & 22.76 & 54.58\\
    w TEN & 33.01 & 35.47 & 39.86 & 56.72 & 9.91 & 22.18 & 23.14 & 54.63\\
    \hline
  \end{tabular}
  
  }
  % \caption{\label{tbl:main_result_numhg}
  %   Comparative analysis of numerical accuracy and textual quality metrics for baseline methods and our proposed approach on the \textbf{NumHG} dataset.
  % }
\caption{\label{tbl:main_result_gpt4o_both}
%    The effectiveness of TEN rationales: Numerical accuracy (\%) and textual quality score (\%) for GPT-4o under zero-shot prompting on \textbf{NumHG} and \textbf{XSum}. 
Performance of GPT-4o with/without TEN rationales under zero-shot prompting on \textbf{NumHG} and \textbf{XSum}.
  }
\end{table*}

% \begin{table*}[htbp]
%   \centering
%   \resizebox{\textwidth}{!}{
%   \begin{tabular}{lccc|ccc|ccc|c}
%     \hline
%     &&\textbf{Num Acc}&&&\textbf{ROUGE}&&&\textbf{BERTScore}&&\textbf{MoverScore}\\
%     \cline{2-11}
%      & Overall & Copy & Reasoning & 1 & 2 & L & P & R & F1 & \\
%     \hline
%     Zero-shot w/o TEN & 21.39 & 28.35 & 7.67 & 35.59 & 13.63 & 30.82 & 30.33 & 48.38 & 39.21 & 56.53 \\
%     Zero-shot w TEN & 33.01 & 44.11 & 11.15 & 35.47 & 13.14 & 30.96 & 33.06 & 46.82 & 39.86 & 56.72 \\    
%     Five-shot w/o TEN & 43.55 & 52.00 & 26.88 & 41.37 & 16.93 & 36.27 & 38.46 & 50.30 & 44.32 & 57.95 \\
%     Five-shot w TEN & 51.16 & 58.74 & 36.21 & 41.51 & 17.25 & 36.50 & 45.19 & 49.07 & 47.14 & 57.86 \\    
%     \hline
%     % \hline
%   \end{tabular}
%   }
  % \caption{\label{tbl:main_result_numhg}
  %   Comparative analysis of numerical accuracy and textual quality metrics for baseline methods and our proposed approach on the \textbf{NumHG} dataset.
  % }
% \caption{\label{tbl:main_result_numhg_inc_gpt4o}
%     Numerical accuracy (\%) and textual quality score (\%) for GPT-4o using zero-shot and five-shot prompting in \textbf{NumHG}. Experimental results shows the performance of GPT-4o with and without TEN rationales in prompts.
%   }
% \end{table*}



% \begin{table*}[htbp]
%   \centering
%   \resizebox{0.82\textwidth}{!}{
%   \begin{tabular}{lc|ccc|ccc|c}
%     \hline
%     &\textbf{Num Acc}&&\textbf{ROUGE}&&&\textbf{BERTScore}&&\textbf{MoverScore}\\
%     \cline{2-9}
%      & Overall & 1 & 2 & L & P & R & F1 & \\
%     \hline
%     Zero-shot w/o TEN & 6.23 & 22.59 & 4.98 & 17.77 & 19.70 & 25.71 & 22.76 & 54.58 \\
%     Zero-shot w TEN & 9.91 & 22.18 & 5.14 & 17.65 & 21.43 & 24.73 & 23.14 & 54.63 \\
%     Five-shot w/o TEN & 16.76 & 29.09 & 8.88 & 23.18 & 31.78 & 35.63 & 33.75 & 56.74 \\
%     Five-shot w TEN & 10.41 & 25.82 & 7.04 & 20.37 & 34.75 & 34.20 & 34.53 & 55.83 \\   
%     \hline
%     % \hline
%   \end{tabular}
%   }
%   % \caption{\label{tbl:main_result_xsum}
%   %   Comparative analysis of numerical accuracy and textual quality metrics for baseline methods and our proposed approach on the \textbf{XSum} dataset.
%   % }
%   \caption{\label{tbl:main_result_xsum_inc_gpt4o}
%     Numerical accuracy (\%) and textual quality score (\%) for GPT-4o using zero-shot and five-shot prompting in \textbf{XSum}. Experimental results shows the performance of GPT-4o with and without TEN rationales in prompts.
%   }
% \end{table*}



%\noindent \textbf{How effective student LLM has learned from teacher LLM.}
We further conducted experiments to evaluate if the student LLM can effectively learn rationale generation from the teacher LLM. 
On the test data, we computed the textual and semantic similarity scores for the rationales automatically generated by Mistral-7B-v0.3 and Llama-3.1-8B as the student model, against the supervision data generated by the teacher LLM GPT-4o. Table~\ref{tbl:rationale_evaluation} illustrates the evaluation results. The high textual similarity and semantic similarity scores demonstrate that the student model can learn from the teacher model to generate rationales to enhance its reasoning capability for number-focused headline generation.

\begin{table*}[htbp]
  \centering
  \resizebox{\textwidth}{!}{
  \begin{tabular}{l|ccc|ccc}
    \hline
    &\multicolumn{3}{c|}{\textbf{NumHG}} & \multicolumn{3}{c}{\textbf{XSum}}\\
    \hline
     & ROUGE-1 & BERTScore-F1 & MoverScore & ROUGE-1 & BERTScore-F1 & MoverScore\\
    \hline
    Mistral-7B-v0.3 & 84.12 & 81.13 & 71.03 & 75.46 & 70.43 & 66.68\\
    Llama-3.1-8B & 84.02 & 80.90 & 70.91 & 75.35 & 70.18 & 66.69\\
    \hline
  \end{tabular}
  }
  % \caption{\label{tbl:main_result_xsum}
  %   Comparative analysis of numerical accuracy and textual quality metrics for baseline methods and our proposed approach on the \textbf{XSum} dataset.
  % }
  \caption{\label{tbl:rationale_evaluation}
    textual quality metric score (\%) for Student LLM generated rationales against teache LLM generated supervision rationales on \textbf{NumHG} and \textbf{XSum}.
  }
\end{table*}


\subsection{Ablation study}
% Due to Mistral-7B-v0.3's more consistent performance across different benchmark datasets, we selected it as our final student LLM. 

\noindent \textbf{Effect of refining rationales through DPO.} In TEN, we apply DPO to enhance the capability of the student LLM rationale generator for topic alignment and numerical reasoning. 
%to enhance the quality of their outputs. 
To understand the effectiveness of DPO we tested TEN minus($-$) DPO on both Mistral-7B and Llama-3.1-8B. Table~\ref{tbl:ablation} illustrates the results. It can be seen that using a rationale generator improved through DPO leads to higher numerical accuracy and textual quality. On the NumHG dataset, DPO improves the numerical accuracy by 0.96\% with a Mistral-7B-v0.3-based rationale generator, and by 1.83\% with Llama-3.1-8B. On the XSum dataset, DPO enhances the numerical accuracy of Mistral-7B-v0.3 and Llama-3.1-8B by 1.38\% and 0.75\%, respectively. Additionally, DPO enhances the ROUGE scores marginally for both student models across both benchmark datasets. 

\begin{table*}[t]
  \centering
  \resizebox{\textwidth}{!}{
  \begin{tabular}{lcccccccc}
    \hline
    \textbf{Method} & \multicolumn{4}{c}{\textbf{NumHG}} & \multicolumn{4}{c}{\textbf{XSum}} \\
    \hline
    & NumAcc & ROUGE-1 & BERTScore-F1 & MoverScore & NumAcc & ROUGE-1 & BERTScore-F1 & MoverScore \\
    \hline
    Mistral-7B-v0.3 & & & & & & & \\
    \hline
    \: TEN (Ours) & \textbf{77.20} & \textbf{51.14} & \textbf{53.21} & \textbf{61.23} & \textbf{39.07} & 46.63 & \textbf{56.70} & \textbf{61.36} \\
    \: $-$ DPO & 76.24 & 50.95 & 53.12 & \textbf{61.24} & 37.69 & 46.26 & 56.35 & \textbf{61.27}  \\
    % \hline
    \: $-$ DPO $-$ N & 74.04 & 50.23 & 52.59 & \textbf{61.08} & 35.58 & 46.18 & 56.13 & \textbf{61.21}  \\
    \: $-$ DPO $-$ TE & 75.55 & \textbf{51.63} & \textbf{53.67} & \textbf{61.37} & 31.71 & 45.94 & 55.55 & \textbf{61.11}  \\
    \: $-$ DPO $-$ TEN & 70.33 & \textbf{51.27} & \textbf{53.43} & \textbf{61.42} & 30.41 & \textbf{47.37} & \textbf{56.88} & \textbf{61.58} \\
    \hline
    \hline
    Llama-3.1-8B & & & & & & & &  \\
    \hline
    \: TEN (Ours) & \textbf{77.89} & \textbf{51.15} & \textbf{52.83} & \textbf{61.14} & \textbf{37.51} & 46.11 & \textbf{56.09} & \textbf{61.18} \\
    \: $-$ DPO & 76.06 & \textbf{50.69} & 52.59 & \textbf{61.09} & 36.76 & 45.89 & 55.80 & \textbf{61.06}  \\
    % \hline
    \: $-$ DPO $-$ N & 73.80 & 50.05 & 52.24 & \textbf{60.90} & 36.51 & 45.83 & \textbf{56.03} & \textbf{61.11} \\
    \: $-$ DPO $-$ TE  & 74.86 & \textbf{50.98} & \textbf{53.21} & \textbf{61.22} & 32.02 & 45.62 & 55.50 & \textbf{61.03}  \\
    \: $-$ DPO $-$ TEN & 70.71 & \textbf{51.01} & \textbf{53.21} & \textbf{61.34} & 29.35 & \textbf{46.62} & \textbf{56.44} & \textbf{61.43} \\    
    \hline
  \end{tabular}
  }
    % Comparative analysis of LLM performance in headline generation: ablation study. 
  % \caption{\label{tbl:ablation}
  %   Ablation study results comparing the proposed TEN scheme with student models fine-tuned without DPO, topic alignment, or numerical reasoning signals.     
  % }
  \caption{\label{tbl:ablation}
    Results (\%) for ablation study of TEN. Higher numbers indicate better performance. Best results are in bold, where results within 0.5 (\%) difference are deemed comparable.     
  }
\end{table*}


% \begin{table*}[t]
%   \centering
%   \resizebox{0.8\textwidth}{!}{
%   \begin{tabular}{lcccc}
%     \hline
%     \textbf{Method} & \multicolumn{2}{c}{\textbf{NumHG}} & \multicolumn{2}{c}{\textbf{Xsum}} \\
%     \hline
%     & NumAcc & ROUGE-1 & NumAcc & ROUGE-1 \\
%     \hline
%     Mistral-7B-v0.3 & & & & \\
%     \hline
%     \: + No Teaching Signal & 70.33 & 51.27 & 30.41 & 47.37 \\
%     \: + Topic Alignment Only & 74.04 & 50.23 & 35.58 & 46.18 \\
%     \: + Calculation Enhancement Only & 75.56 & 51.63 & 31.71 & 45.94 \\
%     \hline
%     \: + Topic Alignment \& Calculation Enhancement (Ours) & 76.24 & 50.95 & 37.69 & 46.26 \\
%     \hline
%     % \hline
%     Llama-3.1-8B & & & & \\
%     \hline
%     \: + No Teaching Signal & 70.71 & 51.01 & 29.35 & 46.62 \\
%     \: + Topic Alignment & 73.80 & 50.05 & 36.51 & 45.83 \\
%     \: + Calculation Enhancement & 74.86 & 50.98 & 32.02 & 45.62 \\
%     \hline
%     \: + Topic Alignment \& Calculation Enhancement (Ours) & 76.06 & 50.69 & 36.76 & 45.89 \\
%     \hline
%   \end{tabular}
%   }
%   \caption{\label{t4}
%     Comparative analysis of LLM performance in headline generation: impact of different teaching signals.
%   }
% \end{table*}

% \section{Further Analysis}

\noindent \textbf{Effect of different supervision signals}. In TEN, we have developed two types of CoT supervision signals. One focuses on aligning the topic of the generated headline, while the other enhances numerical calculation accuracy. As illustrated in Figure~\ref{fig:three_phase_of_TEN}, the rationales under "Topic" and "Entities" contribute to topic alignment, whereas those under "Numbers mentioned" and "Reasoning steps" boost numerical reasoning. We've assessed the impact of these supervision signals, with results for ``TEN minus Number'' ($-$ N), ``TEN minus Topic and Entity ''($-$ TE), and ''no supervision''($-$ TEN) presented in Table~\ref{tbl:ablation}. It can be seen that both types of signals independently improve numerical accuracy in headline generation. However, their effectiveness varies across the two benchmark datasets: numerical reasoning signals show a more pronounced effect on NumHG, while topic alignment signals have a greater impact on XSum. Notably, combining both types of supervision signals yields optimal model performance with highest numerical accuracy and comparable textual quality. 
% \todo{XF: are we going to discuss -TEN or the result is complicated? If so, ignore my suggestion here.}

% For the NumHG dataset, the overall numerical accuracy of headlines generated by Mistral-7B-v0.3 drops 5.91\% when fine-tuned without any teaching signals, compared to TEN. The accuracy decreases by 2.2\% without number reasoning signals and by 0.68\% without topic alignment teaching signals. Llama-3.1-8B exhibits similar trends, with numerical accuracy declining by 5.35\%, 2.26\%, and 1.2\% when fine-tuned without teaching signals, calculation signals, and topic alignment signals, respectively. For the XSum dataset, unlike the results on NumHG, topic alignment teaching signals have a greater impact on the models' numerical accuracy. Mistral-7B-v0.3 and Llama-3.1-8B experience a decrease in numerical accuracy by 7.28\% and 7.41\% respectively when fine-tuned without any teaching signals. Their numerical accuracy only drops by 2.21\% and 0.25\% when fine-tuned without number reasoning signals. However, they show a more significant decline of 5.98\% and 4.74\% when fine-tuned without topic alignment signals, compared to TEN. 

%\todo[color=green]{XF: suggest mentioned the trend of TEN, not the details. Also, we can highlight the difference between the two datasets. -->done}

\noindent \textbf{TEN performance with different student LLMs.} We also want to highlight the performance of TEN with different base student LLMs. All results in Table~\ref{tbl:ablation} are obtained using GPT-4o as the teacher LLM and two different student LLMs: Mistral-7B and Llama-3.1-8B. 
Observe that Mistral-7B and Llama-3.1 demonstrate similar performance for both numerical accuracy and textual quality. 
It can also be seen that the supervision signals and DPO show their effectiveness for both student LLMs. 
% We also conducted experiments using two different types of student models: Mistral-7B-v0.3 and Llama-3.1-8B. The results are shown in Table~\ref{tbl:ablation}. When fine-tuned with TEN, Mistral-7B-v0.3 achieved an overall accuracy of 77.20\% on NumHG with a ROUGE-1 score of 51.14; and 39.07\% on XSum with a ROUGE-1 score of 46.63. Llama-3.1-8B, on the other hand, attained an overall accuracy of 77.89\% on NumHG with a ROUGE-1 score of 51.15; and 37.51\% on XSum with a ROUGE-1 score of 46.03. It can be found that our proposed TEN scheme persists in its efficiency across different LLMs.


% \begin{figure}[t!]
%   \includegraphics[width=\columnwidth]{latex/error_analysis_numhg.pdf} 
%   % Comparative analysis of error rates across different mathematical operations in number-focused headline generation
% %  \caption {The error analysis across different mathematical operations in number-focused headline generation.}
%   \caption {Error analysis across different mathematical operations on NumHG}
%   \label{fig:error_analysis}
% \end{figure}

\begin{figure*}[t!]
  \includegraphics[width=0.48\linewidth]{case_study_2} \hfill
  \includegraphics[width=0.48\linewidth]{case_study_3}
  \parbox[b]{0.45\textwidth}{\centering (a) Topic alignment}
  \hspace{0.05\textwidth}
  \parbox[b]{0.45\textwidth}{\centering (b) Numerical reasoning }  
%  \caption {Comparison between NCL\_NLP (Baseline) and our proposed TEN approach for rationale and headline generation}
  \caption {\label{fig:case_study} TEN vs. NCL (Baseline) for rationale and headline generation}
\end{figure*}

%\subsection{Discussions}
\subsection{Error analysis and case study}
\begin{table*}[!h]
    \centering
    \resizebox{0.9\textwidth}{!}{
    \begin{tabular}{ccccccccccc}
    \hline
    Operation & Overall & Copy & Trans & Paraphrase & Round & Subtract & Add & Span & Divide & Multiply \\
    Count & 3996 & 2,494 & 682 & 376 & 133 & 89 & 76 & 85 & 28 & 33 \\
    \hline
    BART (Err\%) & 31.53 & 23.46 & 34.02 & 26.60 & 60.90 & 96.63 & 78.95 & 68.24 & 96.43 & 93.94 \\
    NPP (Err\%) & 27.55 & 22.57 & 30.06 & 21.81 & \textbf{40.60} & \textbf{68.54} & \textbf{56.58} & \textbf{49.41} & \textbf{82.14} & 84.85 \\
    NCL (Err\%) & 28.08 & 21.57 & 29.62 & 20.74 & 48.12 & 79.78 & 75.00 & 64.71 & 96.43 & 90.91 \\
    TEN (Err\%) & \textbf{25.40} & \textbf{18.89} & \textbf{27.57} & \textbf{20.21} & 48.12 & 75.28 & 60.53 & 58.82 & 92.86 & \textbf{81.82} \\
    \hline
    \end{tabular}
}
  \caption{\label{tbl:error_analysis}
    Error analysis across different mathematical operations on test data from NumHG. Lower numbers indicate better performance. Best results are in bold, where results within 0.5\% difference are deemed comparable. 
  }
\end{table*}

\noindent \textbf{Error analysis.} Utilizing the annotations from the NumHG dataset, which outlines nine types of operations necessary for calculating numerals in headlines, we conducted an error analysis for TEN in comparison to the baselines. We present the error rates in Table~\ref{tbl:error_analysis}. 
%models trained with our proposed approach. 
%We examined errors related to the nine operation types. 
The experimental results demonstrate that our approach significantly reduces errors in copying, translating, and paraphrasing, achieving the lowest error rates compared to baseline methods. These three operations represent over 88\% of the total. For the remaining less frequent operations, our approach achieves error rates comparable to the best-performing baseline.
%\todo[color=green]{Figure~\ref{fig:error_analysis} np\_problem to NP-Problem --> done}
%jz3: some more detailed discussions are needed here. 
%jz3: Give some numbers. How many "copy" operations in total and the errors should be relative to the total operations, explain the total operations for each type and the percentage. 
% TEN has the largest number (%) of errors in Copy but NP, NCL has more errors in X, Y operations.

\noindent \textbf{Case study.} Two examples are selected from the test dataset to %demonstrate the improvements in topic alignment and numerical reasoning brought by our proposed TEN approach. 
illustrate the benefits of the TEN reasoning strategy, compared against the NCL baseline, which generates rationale without the TEN structured rationales. 
%We compare the different outputs between NCL (baseline) and TEN (ours). 
Figure~\ref{fig:case_study} (a) shows that TEN correctly identifies the topic the headline should focus on in the rationale, which is the rank of the tornado in this case, while NCL mistakenly focuses on the elevation. In Figure~\ref{fig:case_study} (b), TEN successfully calculates the number of people who died by adding 1 Australian tourist and 3 Tibetans, while NCL fails to count the Australian tourist. 

\section{Conclusion}
In this paper
%we presented  
%a novel fine-tuning scheme for number-focused headline generation named TEN (Topic, Entities, and numerical reasoning). 
we studied number-focused news headline generation, 
a problem presenting the unique challenge of high textual quality with precise number accuracy for LLM generation. 
We proposed a novel framework of using rationales of key elements Topic, Entity, and Numerical reasoning (TEN) to enhance the capability of LLMs for topic alignment and numerical reasoning in headline generation.   
We developed an approach to fine-tune LLMs to automatically generate TEN rationales for numerical headlines generation. 
Especially our TEN approach builds upon the teacher-student rationale-augmented training framework, where a teacher LLM automatically generate TEN rationales as supervision data to teach a student LLM rationale generator and a student LLM headline generator.    
%extends it to further refine automatically generated rationales specifically designed for the headline generation task. 
Experiments on popular numerical news headline generation datasets showed that TEN outperforms existing approaches, achieving higher numerical accuracy and mostly better textual quality for headline generation. 
%across both NumHG and XSum datasets. 
%The decomposition of rationales into topic alignment and numerical reasoning components proves highly effective, with their combination yielding optimal results. Further refinement of generated rationales through DPO leads to additional improvements in numerical accuracy and textual quality. 
%By effectively addressing the challenges of topic alignment and numerical calculation, our method paves the way for more accurate and reliable headline generation. 

\section*{Acknowledgements}
This research is supported in part by the Australian Research Council Discovery Project DP200101441.

\section*{Limitations}

%We employ only one teacher LLM (GPT-4o), so our proposed approach heavily relies on GPT-4o's performance when generating the supervision data. 
%For the student LLMs, 
One limitation of our study is that due to computing resource limitation, we have only applied parameter-efficient technique QLoRA \citep{dettmers_qlora_2023} to fine-tune student LLMs, 
and as such
%As we have not conducted full-parameter tuning,  
it is possible that we 
have not fully elicited the capability of student LLMs. 
%On For refining the rationales, we've tested DPO techniques but haven't explored other approaches such as reinforcement learning and verification. 
Another limitation of our study is the limited data for experiments. 
To our best knowledge NumHG is the only public benchmark dataset for numerical headline generation, and we constructed one more dataset based on XSum for extreme summarization.  
%There's also a limitation with the XSum benchmark dataset. This dataset is designed for extreme summarization, not specifically for headline generation, and contains considerable noise, which isn't ideal for machine learning.

% Bibliography entries for the entire Anthology, followed by custom entries
%\bibliography{anthology,custom}
% Custom bibliography entries only
% \bibliography{main}
\documentclass{MITstyle}

%\usepackage[table]{xcolor}
\usepackage{chngcntr}
\usepackage{hyperref}
\usepackage{microtype}

\title{A Lightweight and Extensible Cell Segmentation and Classification Model for Whole Slide Images}

\author{Nikita Shvetsov~$^{1, }$\footnote{Correspondence e-mail: nikita.shvetsov@uit.no}, Thomas K. Kilvaer~$^{2, 3}$, Masoud Tafavvoghi~$^{4}$, Anders Sildnes~$^{1}$, \\ Kajsa Møllersen~$^{4}$, Lill-Tove Rasmussen Busund~$^{5, 6}$, Lars Ailo Bongo~$^{1}$ \\
%
\vspace{1em} % Space between authors and afilliations
%
\normalfont{\small $^{1}$Department of Computer Science, UiT The Arctic University of Norway}\\
\normalfont{\small $^{2}$Department of Oncology, University Hospital of North Norway}\\
\normalfont{\small $^{3}$Department of Clinical Medicine, UiT The Arctic University of Norway}\\
\normalfont{\small $^{4}$Department of Community Medicine, UiT The Arctic University of Norway}\\
\normalfont{\small $^{5}$Department of Medical Biology, UiT The Arctic University of Norway} \\
\normalfont{\small $^{6}$Department of Clinical Pathology, University Hospital of North Norway} %\vspace{2em}
}

\begin{document}
\maketitle

\section*{Abstract}

% \begin{abstract}
% Developing clinically useful cell-level analysis tools in digital pathology remains challenging due to limitations in dataset granularity, inconsistent annotations, computational demands of advanced models, and difficulties in integrating new technologies into clinical workflows. To address these challenges, we propose a multi-faceted solution that enhances data quality, model performance, and usability to create a lightweight and extensible cell segmentation and classification model.

% First, we update data labels by employing a cross-relabeling process that refines the labels of two existing datasets, PanNuke and MoNuSAC, to create a new unified dataset with enhanced granularity, encompassing seven distinct cell types. Second, we leverage the H-Optimus foundation model as a fixed encoder to improve feature representation for simultaneous cell segmentation and classification tasks. Third, to address the computational demands of foundation models, we employ knowledge distillation to reduce model size and complexity while maintaining comparable performance. Finally, to facilitate integration into clinical workflows, we integrate the distilled model into the QuPath software, a widely used open-source platform in digital pathology.

% Our results demonstrate improvements in cell segmentation and classification performance using the H‑Optimus-based model compared to a CNN-based model. Specifically, the average $R^2$ improved from 0.575 to 0.871, and the average $PQ$ score improved from 0.450 to 0.492, indicating better alignment with actual cell counts and enhanced segmentation and classification quality. Furthermore, the distilled student model maintains performance comparable to the larger foundation model while reducing the parameter count by a factor of 48.
% Overall, by reducing computational complexity and integrating it into existing workflows, the proposed approach may significantly impact diagnostic processes, reduce the workload of pathologists, and contribute to improved patient outcomes. Though our approach shows potential enhancements in efficiency and usability of cell segmentation and classification models in digital pathology, extensive validation is needed to deploy these models in clinical practice.
% \end{abstract}

%%% shortened abstract
\begin{abstract}
Developing clinically useful cell-level analysis tools in digital pathology remains challenging due to limitations in dataset granularity, inconsistent annotations, high computational demands, and difficulties integrating new technologies into workflows. To address these issues, we propose a solution that enhances data quality, model performance, and usability by creating a lightweight, extensible cell segmentation and classification model. 

First, we update data labels through cross-relabeling to refine annotations of PanNuke and MoNuSAC, producing a unified dataset with seven distinct cell types. Second, we leverage the H-Optimus foundation model as a fixed encoder to improve feature representation for simultaneous segmentation and classification tasks. Third, to address foundation models' computational demands, we distill knowledge to reduce model size and complexity while maintaining comparable performance. Finally, we integrate the distilled model into QuPath, a widely used open-source digital pathology platform. 

Results demonstrate improved segmentation and classification performance using the H-Optimus-based model compared to a CNN-based model. Specifically, average $R^2$ improved from 0.575 to 0.871, and average $PQ$ score improved from 0.450 to 0.492, indicating better alignment with actual cell counts and enhanced segmentation quality. The distilled model maintains comparable performance while reducing parameter count by a factor of 48. By reducing computational complexity and integrating into workflows, this approach may significantly impact diagnostics, reduce pathologist workload, and improve outcomes. Although the method shows promise, extensive validation is necessary prior to clinical deployment.
\end{abstract}
\clearpage

\section{Introduction}
In digital pathology, accurate segmentation and classification of cells are crucial for many diagnostic, prognostic, and predictive analyses \cite{Jaber_Beziaeva_etal._2019,Lin_Pan_etal._2022,Park_Ock_etal._2022,Shen_Choi_etal._2024}. Nowadays, developments in computational pathology offer multiple solutions \cite{H._Qu_P._Wu_etal._2020,Javed_Mahmood_etal._2020} to utilize cell-level datasets to train machine learning models that solve these problems. The quality and specificity of training datasets are critical for robust and accurate models. Adhering to the principle of "garbage in, garbage out", it is essential to ensure that these datasets are extensively and accurately labeled with distinct classes that reflect the diverse biological characteristics of different cell types. Unfortunately, the number of open-source datasets comprising such high-quality annotations is limited. Existing cell segmentation datasets \cite{Gamper_Koohbanani_etal._2019,Graham_Vu_etal._2019,Verma_Kumar_etal._2021} may offer extensive annotations for certain cell types while providing more general labels for others. For example, in PanNuke, which is one of the largest open-source datasets comprising labeled cells, various types of morphologically and functionally different inflammatory cells like macrophages and lymphocytes are clustered in a broad "inflammatory" class. Consequently, these classes are frequently omitted from analyses or aggregated into broader meta-classes \cite{Gamper_Koohbanani_etal._2020} and likely interfere with other cell classes included in the dataset. This and similar inconsistencies in annotation granularity limit the ability of machine learning models to learn the comprehensive and nuanced features necessary for accurate cell segmentation and classification. To address these challenges, methods for refining and standardizing dataset annotations are essential to enhance the quality of training data.

A complementary approach to mitigate the absence of high-quality training data is the use of foundation models. Foundation models as encoders are defined as large-scale, versatile networks pre-trained on vast, diverse datasets using self-supervised learning, contrasting with convolutional neural network (CNN) pre-trained encoders that rely on supervised learning with labeled data. In practice, foundation models leverage enormous amounts of weakly or unlabeled data from millions of whole slide images (WSIs) and employ self-attention mechanisms to capture long-range dependencies and global context \cite{Chen_Ding_etal._2024,Saillard_Jenatton_etal._2024,Vorontsov_Bozkurt_etal._2024,Xu_Usuyama_etal._2024}. As a consequence, foundation models are able to produce transferable feature representations across different cell types and tissue environments. The feature representations can be leveraged by decoder networks to produce segmentation masks and pixel-level classifications. Because foundation models have comprehensive feature representations, they can be effectively fine-tuned using much smaller amounts of cell-level data compared to the large datasets needed to train models from scratch. Furthermore, foundation models incorporate adversarial training elements or contrastive learning \cite{Chen_Ding_etal._2024,Xu_Usuyama_etal._2024}, enhancing their resilience and adaptability by exposing them to challenging and varied scenarios during training. This may result in more generalizable models, often making them well-suited for diverse and complex tasks in digital pathology.

Despite the inherent advantages of foundation models, their deployment for practical use faces its own obstacles. In particular, they require substantial computational power, financial investments and rigorous testing to ensure reliability and efficacy for a given task \cite{Akkus_Dangott_etal._2022,Dragomir_Cocuz_etal._2022,Go_2022,Jafri_Farooqui_etal._2024}. Moreover, while foundation models enhance feature representation and performance, they depend on the quality of available annotations for decoder fine-tuning and, like any other model, cannot resolve existing inconsistencies or ambiguities in data labels. Therefore, there remains a critical need for solutions that address both data quality and practical deployment considerations.
Further, integrating new technologies into existing clinical workflows often encounters resistance, as it necessitates adjustments to established diagnostic processes. So, there is a need to develop solutions that could be integrated into current practices, minimizing the burden on medical professionals to adopt new tools \cite{King_Williams_etal._2023}.

Existing solutions \cite{Goldsborough_Philps_etal._2024,Hörst_Rempe_etal._2024}, while addressing some aspects of these challenges, fall short in providing a comprehensive approach. To address the data quality and clinical deployment issues, we propose a multi-faceted solution that encompasses data refinement, model optimization, and integration with existing pathology tools (\hyperref[fig:fig1]{Figure 1}). The outcome is a lightweight cell segmentation and classification model that can be integrated into digital pathology workflows for practical clinical use.

\begin{figure}[h!]
    \centering
    \includegraphics[width=\textwidth, height=0.82\textheight, keepaspectratio]{images/Figure_1.pdf}
    \caption{Overview of the proposed solution, including 1) Data refinement using cross-relabeling, 2) Teacher model development and fine tuning, 3) Student model optimization with knowledge distillation and 4) Student model and QuPath integration}
    \label{fig:fig1}
\end{figure}
\clearpage

Our approach begins with preparing the data for the fine-tuning and training of the machine learning models. We create a refined dataset, acquired via cross-relabeling two cell-level datasets, enhancing annotation specificity and consistency of the labeled data. Subsequently, we create a cell segmentation and classification model based on the foundation model. We leverage the foundation model as a fixed encoder and fine-tune a decoder using the refined dataset to improve generalization across diverse tissue- and cell types.
To ensure that the model remains lightweight and deployable in a possibly resource-constrained environment, we employ knowledge distillation to approximate the functionality of the foundation model. Finally, to facilitate the practical application of our model in digital pathology workflows, we integrate it with the QuPath \cite{Bankhead_Loughrey_etal._2017} application. Each methodological component contributes to the overarching goal of enhancing model performance, generalizability, and usability in clinical settings.

The primary contributions of this paper are:
\begin{enumerate}
    \item \textit{Data labels refinement through cross-relabeling:}
    
    We propose a new method for refining labels of cell-level datasets through cross-relabeling. This method employs classification models to re-label broad and ambiguous instances, resulting in a more diverse dataset. Our evaluation demonstrates that these classification models achieve high accuracy on test subsets, indicating the reliability of the method for label refinement.

    \item \textit{Enhanced model performance via foundation models:}
    
    We employ a foundation model as a feature extractor for the cell segmentation and classification task. In comparison with training a CNN model from scratch, the foundation model backbone only needs fine-tuning, which significantly reduces training time, computational resources and data requirements. We show that using a foundation model encoder leads to better performance in cell segmentation and classification networks than using a CNN-based encoder. This improvement may enable the model to generalize more effectively across various tissue types and imaging methods.
    
    \item \textit{Model optimization through knowledge distillation:}
    
    We show that a smaller student model trained using knowledge distillation on the refined dataset obtained via our cross-relabeling approach from a foundation model achieves comparable performance in cell segmentation and quantification tasks. As a result, this model is more suitable for deployment in environments without high-performance computing resources.
    
    \item \textit{Integration with QuPath:}
    
    We integrate the distilled cell segmentation and classification model into QuPath, a widely used open-source digital pathology platform, to accelerate clinical adaptation by enabling pathologists to more easily incorporate advanced computational tools into their existing workflows.
\end{enumerate}

Through these methodological steps, we aim to bridge the gap between advanced machine learning techniques and practical clinical applications, making accurate and efficient digital pathology accessible in a broader range of healthcare settings.

\section{Refining Existing Datasets Using Cross-Relabeling}
To address the limitations of sparse and ambiguous labeling of cell-level datasets, we propose a generalizable cross-relabeling strategy that can be applied to any dataset containing broadly categorized or imprecisely labeled cell types. This approach involves training and subsequently leveraging classification models to refine broad categories into more specific or biologically relevant classes.
When applied to cell-level data, the methodology includes extracting individual cell images from the dataset patches, preprocessing these images to standardize the size and accommodate partial cells, and then training deep learning classifiers capable of distinguishing between the finer cell subtypes within the coarser categories. 
To illustrate our approach, we focus on the PanNuke \cite{Gamper_Koohbanani_etal._2020, Gamper_Koohbanani_etal._2019} and MoNuSAC \cite{Verma_Kumar_etal._2021} datasets that we have used to train models for cell quantification in our previous works \cite{Shvetsov_Grønnesby_etal._2022,Shvetsov_Sildnes_etal._2024}. We find that for better cell differentiation we have to introduce more granular labels. PanNuke includes a broad classification of "inflammatory" cells, encompassing lymphocytes, macrophages, and neutrophils. Each cell type differs significantly in structure, function, and clinical relevance. Conversely, MoNuSAC uses the label "epithelial" for a class that comprises both benign epithelial cells and malignant neoplastic cells. This practice makes it challenging to differentiate between benign and malignant epithelial cells in the dataset, which is a critical distinction when identifying tumor areas within tissue samples. To address these issues, we implement a cross-relabeling strategy as shown in \hyperref[fig:fig2]{Figure 2}. The key components are two classification models: one is trained on singular cell images from PanNuke data to classify the epithelial meta-class into epithelial and neoplastic classes. The other is trained on MoNuSAC to refine the inflammatory class into lymphocytes, neutrophils, and macrophages.

\begin{figure}[h!]
    \centering
    \includegraphics[width=\textwidth]{images/Figure_2.pdf}
    \caption{Refined dataset generation via cross relabeling}
    \label{fig:fig2}
\end{figure}

The refining approach consists of three consecutive steps. The first is the preprocessing step, in which we extract individual cells from both datasets (\hyperref[fig:fig3]{Figure 3}). The specifics of PanNuke and MoNuSAC patch preparation before cell preprocessing are provided in \hyperref[chap:S1]{Appendix S1}.

\begin{figure}[h!]
    \centering
    \includegraphics[width=\textwidth]{images/Figure_3.pdf}
    \caption{Cell instances preprocessing including (1) cell map extraction, (2) bounding box delineation, (3) adjusting cell boxes and (4) cropping and resizing of cell images}
    \label{fig:fig3}
\end{figure}

During preprocessing, we extract cell type maps from the ground truth label mask and calculate bounding boxes around each cell instance. To accommodate partial cells at patch borders, a common issue in cropped patch images, we employ mirror padding and extend the field of view of the cell label by 15 pixels to capture adjacent cells. We then crop and resize the identified regions to $64 \times 64$ pixels using bicubic interpolation.

The preprocessed PanNuke dataset comprises 68,031 neoplastic and 23,207 epithelial cell images, while MoNuSAC comprises  33,104 lymphocytes, 1,252 neutrophils, and 1,695 macrophages, which we subsequently use in training cell classification models and classifying the cell image data \hyperref[fig:S2]{Appendix Figure S2 (1)}. 

The next step is to train two distinct ResNet50-based classifiers tailored to address the specific labeling challenges inherent in each dataset. We use ResNet50 for classification models due to its proven effectiveness for image classification tasks in histopathology \cite{pan2022reviewmachinelearningapproaches}, and its compatibility with small images. For the PanNuke dataset, we design the classifier, trained on MoNuSAC data, to disaggregate the heterogeneous "inflammatory" cell category into distinct subtypes: lymphocytes, macrophages, and neutrophils. Similarly, for the MoNuSAC dataset, the classifier is trained on PanNuke data and distinguishes between benign and malignant epithelial cells within the overarching "epithelial" label. By applying these targeted classifiers to their respective datasets, we assign more specific labels to individual cell instances, thus enabling us to create a unified dataset.
To ensure a balanced representation of classes, we train both models on datasets that had been equalized to match the size of the least represented class. Thus, we obtain datasets comprising 23,207 samples per class for PanNuke and 1,252 samples per class for MoNuSAC data. Next, we partition both of them into training (70\%), validation (20\%), and testing (10\%) subsets. To mitigate the risk of overfitting, we use a single dropout layer with a rate of p=0.5 in both models and data augmentation using randomized color perturbations, rotation, and horizontal and vertical flipping. We employ AdamW optimizer and the cross-entropy loss function for the training criterion.

To evaluate the two trained models, we measure the classification accuracy on the respective test subsets. The accuracies on the test subset for both classifiers are presented in \hyperref[tab:1]{Table 1}. The PanNuke model achieves an average accuracy of 93.57\%, with higher accuracy for neoplastic cells (96.06\%) compared to epithelial cells (86.26\%). The confusion matrix in Figure A3.1 shows that the model predominantly distinguishes accurately between epithelial and neoplastic tissues, with a substantial number of correct classifications and relatively few misclassifications. The MoNuSAC model demonstrates an average accuracy of 98.92\%, excelling in classifying lymphocytes (99.67\%) and macrophages (94.12\%), with lower performance for neutrophils (85.71\%). The confusion matrix in Figure A3.2 shows that the model identifies lymphocytes and performs reasonably well with macrophages and neutrophils.

\begin{table}[h!]
\renewcommand{\arraystretch}{1.5}
  \centering
  \caption{Cell classification results for PanNuke and MoNuSAC trained models (CI 95\%).}
  \label{tab:1}
  \begin{tabular}{|l|c|c|}
   \hline
   %\rowcolor{gray!30}
    Accuracy               & PanNuke model              & MoNuSAC model              \\
    \hline
    Average      & 0.936 (0.931--0.941)         & 0.989 (0.986--0.993)        \\
    \hline
    Neoplastic   & 0.961 (0.956--0.965)         & -                          \\
    \hline
    Epithelial   & 0.863 (0.849--0.877)         & -                          \\
    \hline
    Lymphocytes  & -                          & 0.997 (0.995--0.999)        \\
    \hline
    Neutrophils  & -                          & 0.857 (0.796--0.918)        \\
    \hline
    Macrophages  & -                          & 0.941 (0.906--0.976)        \\
    \hline
  \end{tabular}
\end{table}

Finally, during the last step, we use the model trained on PanNuke data for epithelial cells in MoNuSAC and the model trained on MoNuSAC for the inflammatory cells class in PanNuke. Specifically, we use classifier models to relabel epithelial cells in MoNuSAC and inflammatory cells in PanNuke data. Then we combine cells with refined labels and the rest of the cells in both datasets to create a refined dataset (\hyperref[fig:S2]{Appendix Figure S2 (2)}). The process of relabeling cells and visualizing them on a patch is shown in \hyperref[fig:fig4]{Figure 4}. The cell counts in the refined dataset are provided in \hyperref[tab:S4]{Appendix Table S4}.

\begin{figure}[h!]
    \centering
    \includegraphics[width=\textwidth, height=0.42\textheight, keepaspectratio]{images/Figure_4.pdf}
    \caption{Cell relabeling procedure for epithelial and inflammatory cell classes}
    \label{fig:fig4}
\end{figure}

%\hfill

Relabeling and combining datasets have been explored in a prior study \cite{Parulekar_Kanwat_etal._2023}, where consecutive fine-tuning on multiple datasets was employed to account for hierarchical class label structures. While the method presented in \cite{Parulekar_Kanwat_etal._2023} is intuitive, it often lacks consistency and requires multiple fine-tuning runs, which can be cumbersome and time-consuming. 
In contrast, cross-relabeling simplifies this process by using specialized classification models tailored to each dataset's specific labeling challenges. This approach provides better transparency and produces a unified dataset encompassing seven distinct cell types across multiple tissue samples, enhancing data diversity for further model training or fine-tuning.

Despite these improvements, cross-relabeling does not entirely resolve issues related to poor labeling quality or the amount of labeled data. Specifically, our results show lower accuracies persist for underrepresented classes, such as macrophages, which may stem from a limited sample availability and intrinsic challenges in distinguishing these cells based solely on H\&E staining. Furthermore, while our method enhances label specificity, it relies on the initial quality of the broad labels; thus, any fundamental inaccuracies in the original annotations can propagate through the relabeling process. Addressing the overall problem of limited data labels may require integrating additional data sources or utilizing complementary immunohistochemical staining methods.
Although the reported performance metrics are obtained from evaluations on the native test sets of each dataset, it is important to note that the primary application of these classifiers is to perform cross-relabeling, where a model trained on one dataset (e.g., PanNuke) is applied to another (e.g., MoNuSAC) and vice versa. We acknowledge that a more systematic evaluation of cross-dataset generalization is needed and could be performed in future work.

Overall, the refined dataset produced by our approach can enhance the supervised training or fine-tuning of cell segmentation and classification models, especially those that utilize pre-trained foundation models to improve feature extraction robustness. In addition, these models can detect nuanced classes that enable researchers to conduct more detailed analyses of biological processes in computational pathology.

\section{Foundation models for robust cell segmentation and classification}

Accurate cell segmentation and classification in digital pathology are hindered by limited labeled data and the fact that conventional CNNs are unable to capture global contextual information due to their local receptive field constraints \cite{Gheflati_Rivaz_2022,Yang_Marcus_etal.}. Traditional approaches in cell quantification have predominantly relied on CNN encoders, such as ResNet50, given their proven effectiveness in semantic segmentation tasks \cite{Deshmane_2023,Graham_Vu_etal._2019,Mukasheva_Koishiyeva_etal._2024,Stringer_Wang_etal._2021}. However, approaches that include fine-tuning of pretrained CNNs, data augmentation, and stain normalization to partially increase data variability and address staining differences often fail to achieve the necessary generalization and robustness across diverse tissue types and staining conditions \cite{G._Wang_W._Li_etal._2018,Gao_Bagci_etal._2018,Karim_El_Khoury_Martin_Fockedey_etal._2021}.

To overcome these challenges, we leverage an encoder-decoder network that uses a foundation model as the encoder and a CNN upsampling decoder (\hyperref[fig:fig5]{Figure 5}) for simultaneous cell segmentation and classification in 2D patches extracted from WSIs. Foundation models with transformer-based architectures are viable alternatives to CNN-based encoders \cite{Shamshad_Khan_etal._2023,Sourget_2023}. They enable the creation of more advanced architectures that can decode or transform learned features more effectively \cite{Chen_Duan_etal._2023,Cheng_Misra_etal._2022,Xie_Wang_etal._2021}.

\begin{figure}[h!]
    \centering
    \includegraphics[width=\textwidth]{images/Figure_5.pdf}
    \caption{UNETR-like model with foundational model as backbone}
    \label{fig:fig5}
\end{figure}

By utilizing a transformer-based encoder, we incorporate global contextual information into the feature extraction process, which is a key advantage of such architectures \cite{Chen_Lu_etal._2021}. This foundation model integration facilitates accurate pixel-wise segmentation and classification without the need for extensive encoder training, thereby potentially improving generalization across varied cellular structures and tissue types.
In our implementation, we employ a modified UNETR \cite{Hatamizadeh_Tang_etal._2021} architecture that combines a vision transformer (ViT) \cite{Dosovitskiy_Beyer_etal._2021} encoder with a CNN-based decoder. The encoder utilizes the pretrained H-Optimus foundation model, which contains 1.1 billion parameters and is trained on over 500,000 H\&E stained WSIs \cite{Saillard_Jenatton_etal._2024}. We extract outputs from four evenly spaced transformer blocks $Z_i$, where $i \in [1, 14, 26, 38]$, to serve as residual connections for the CNN decoder. We select these blocks based on our observation that features from non-adjacent levels of the encoder lead to better overall performance on the test subset.

The CNN decoder upsamples the feature representations, acquired from the transformer blocks, to generate an intermediate vector that is handled by two task-specific layers that generate cell segmentation and classification masks. The first task-specific layer is the ‘Cellpose head’,  which is used to delineate cell instances. The layer generates horizontal and vertical gradient maps to form vector fields that are refined through gradient tracking in a post-processing step using the Cellpose algorithm \cite{Stringer_Wang_etal._2021}, known for its efficacy in cell segmentation tasks and generalizability across multiple domains \cite{Pachitariu_Stringer_2022,Stringer_Pachitariu_2024}. The second task-specific layer is the "Cell type head", which assigns labels to individual pixels. In the post-processing step, we determine the output classification label of each segmented cell instance by majority voting over the labeled pixels that comprise the cell in the segmentation map.

To evaluate model performance and measure the impact of adding a foundation model as backbone, we compare it to a ResNet50-based model. ResNet50 is a widely used solution for encoders in segmentation architectures in the medical domain \cite{Deshmane_2023,Graham_Vu_etal._2019,Mukasheva_Koishiyeva_etal._2024,Stringer_Wang_etal._2021}. For the H-Optimus-based model, we utilize frozen weights for the encoder and only fine-tune the decoder to take advantage of the extensive pre-training of the foundation model. For the ResNet50-based model we start with ImageNet \cite{Deng_Dong_etal.} weights and train both encoder and decoder parts. Hyperparameters for the training step are set to be identical, where possible, for comparable evaluation. 
For this evaluation, we deliberately use the PanNuke dataset to provide a standardized and controlled comparison between the H‑Optimus and ResNet50-based models (\hyperref[fig:S2]{Appendix Figure S2 (3)}). Specifically, we use two of the default PanNuke dataset splits (66\%) for training and validation, and reserve the third split (33\%) for testing.

To address the challenge of cell class imbalance in the PanNuke dataset, which is a common characteristic in most cell-level H\&E patch datasets, both models’ training processes employ a weighted loss function comprising cross-entropy and focal loss \cite{Lin_Goyal_etal._2018}. The focal loss component is adjusted with coefficients derived from each cell class' instance frequency, emphasizing learning from underrepresented classes and enhancing the model's sensitivity to rare but significant cellular patterns. The cross-entropy loss is augmented with spectral decoupling regularization \cite{Pezeshki_Kaba_etal._2021,Pohjonen_Stürenberg_etal._2022} and spatially varying label smoothing \cite{Islam_Glocker_2021}, which potentially stabilizes training and improves generalization in case of complex tissue morphologies. For optimization, we employ the \textit{AdamW} \cite{Loshchilov_Hutter_2019} to counter unbalanced class scenarios, with cosine annealing learning rate scheduler.

We utilize the scikit-learn library \cite{Van_der_Walt_Schönberger_etal._2014} and HoVer-Net \cite{Graham_Vu_etal._2019} implementations of $R^2$ (the coefficient of determination) and $PQ$ (panoptic quality) to evaluate our experiments. Complete mathematical formulations and detailed explanations of these metrics are provided in \hyperref[chap:S5]{Appendix S5}. To compute confidence intervals, we use nonparametric bootstrapping, where after calculating the metric on the full sample, we generated 1000 bootstrap replicates by resampling with replacement and then determined the 95\% confidence intervals as the 2.5th and 97.5th percentiles of the resulting empirical distribution.

%\hfill

The model comparisons are summarized in \hyperref[tab:2]{Table 2}. The H‑Optimus-based model achieves higher $R^2$ across all cell classes compared to the ResNet50-based model, which means that its predictions are more closely aligned with the PanNuke cell counts, indicating a stronger correlation with the observed data. Notably, the improvement of $R^2_{dead}$ may be an indicator of better global contextual representations provided by the foundation model backbone. In terms of segmentation and classification quality combined, measured by the PQ score, the H‑Optimus-based model demonstrates notable improvements across most cell classes. Overall, the average $R^2$ improved from 0.575 to 0.871, while the average $PQ$ score improved from 0.450 to 0.492, demonstrating better performance of the H-Optimus-based model.

\begin{table}[h!]
\renewcommand{\arraystretch}{1.5}
  \centering
  \caption{Cell quantification metrics for baseline and proposed models (CI 95\%).}
  \label{tab:2}
  \begin{tabular}{|l|c|c|}
    \hline
    %\rowcolor{gray!30}
    Metric             & Resnet50-based            & H-optimus-based              \\
    \hline
    $R^2_{neoplastic}$    & 0.681 (0.576--0.769)       & \textbf{0.941 (0.917--0.960)} \\
    \hline
    $R^2_{inflammatory}$  & 0.863 (0.778--0.903)       & \textbf{0.949 (0.918--0.966)} \\
    \hline
    $R^2_{connective}$    & 0.600 (0.488--0.698)       & 0.609 (0.436--0.772)          \\
    \hline
    $R^2_{dead}$          & 0.097 (-11.389--0.669)     & 0.925 (0.404--0.982)          \\
    \hline
    $R^2_{epithelial}$    & 0.635 (0.490--0.747)       & \textbf{0.930 (0.886--0.964)} \\
    \hline
    $PQ_{neoplastic}$       & 0.517 (0.499--0.535)       & \textbf{0.589 (0.575--0.604)} \\
    \hline
    $PQ_{inflammatory}$     & 0.455 (0.429--0.482)       & \textbf{0.528 (0.507--0.549)} \\
    \hline
    $PQ_{connective}$       & 0.416 (0.400--0.431)       & \textbf{0.451 (0.436--0.465)} \\
    \hline
    $PQ_{dead}$             & 0.374 (0.342--0.408)       & 0.292 (0.209--0.365)          \\
    \hline
    $PQ_{epithelial}$       & 0.488 (0.460--0.519)       & \textbf{0.599 (0.579--0.618)} \\
    \hline
  \end{tabular}
\end{table}

Our results  show that integrating the H‑Optimus foundation model within the UNETR architecture enhances the model's ability to segment and classify cells across diverse tissues from PanNuke data. The pretrained transformer encoder provides robust feature representations, resulting in higher average $R^2$ and $PQ$ scores compared to the CNN-based model. This leads to more reliable cell quantification and more accurate downstream analysis. Additionally, the streamlined fine-tuning process reduces computational overhead and training time, making the model more adaptable for new data.

Despite these advancements, the foundation model-based approach does not fully resolve all challenges related to cell segmentation and classification. We observe lower metric scores for underrepresented classes in the training data. Furthermore, foundation models typically encompass billions of parameters, resulting in substantial computational and memory requirements. It therefore poses challenges for deployment in resource-constrained environments, limiting their practical applicability in certain clinical settings.

\section{Model optimization via Knowledge Distillation}

To address the limitations posed by the extensive size of foundation models, we implement knowledge distillation — a model compression technique that leverages the teacher-student paradigm \cite{Hinton_Vinyals_etal._2015}. By training a smaller, more efficient student model to replicate the output of a larger, pre-trained teacher model, we retain performance while significantly reducing the model's complexity and resource requirements (\hyperref[fig:fig6]{Figure 6}).

\begin{figure}[h!]
    \centering
    \includegraphics[width=\textwidth, height=0.45\textheight, keepaspectratio]{images/Figure_6.pdf}
    \caption{Knowledge distillation framework for training a student model using a pre-trained teacher}
    \label{fig:fig6}
\end{figure}

We employ knowledge distillation to compress the H‑Optimus-based teacher model into a more efficient student model. The teacher model is the modified UNETR architecture with the H‑Optimus foundation model described in the previous chapter. The student model is based on a UNet architecture augmented with residual connections and incorporates a smaller ViT encoder with 9 million parameters \cite{Steiner_Kolesnikov_etal._2022,Wightman_2019}. 

First, we fine-tune the teacher model using the refined dataset from the cross-relabeling procedure (Section 2). Initially we train the decoder of the teacher model while keeping the encoder weights frozen. We split the refined dataset into train (70\%), validation (20\%) and test (10\%) subsets (\hyperref[fig:S2]{Appendix Figure S2 (4)}). During fine-tuning, we use the train and validation subsets, while leaving the test subset for model evaluation. We set the training procedure and model hyperparameters to be identical to those that were used to demonstrate the utility of foundation models for the simultaneous cell segmentation and classification task.

Next, we perform knowledge distillation from teacher to student using the refined dataset used to fine-tune the teacher model. The student model is trained to replicate the teacher model's outputs. We utilize a specialized loss function that aligns the student's predicted probability distribution with the teacher's, incorporating the teacher's class probability distribution derived from the output. Following the methodology of Hinton et al. \cite{Hinton_Vinyals_etal._2015}, we experiment with various hyperparameter settings for the temperature ($T$) and the balancing coefficients ($\alpha$ and $\beta$) in the loss function. We vary $T$ from 1 to 20 and adjust $\alpha$ and $\beta$ to balance the distillation and student losses. Through iterative tuning and evaluation, we identify that setting $T=14$, $\alpha=0.3$, and $\beta=0.7$ yields a configuration that converges and closely approximates the teacher model's performance during training.

Finally, we assess the performance of both models using the $R^2$ and $PQ$ (defined in \hyperref[chap:S5]{Appendix S5}) on the test set of the refined dataset (\hyperref[tab:3]{Table 3}). We observe that the 95\% confidence intervals overlap for most cell types, so we cannot claim statistically significant performance differences between the teacher and student models. One exception appears in the neoplastic class. The teacher model produces an $R^2$ of 0.919, while the student model shows an $R^2$ of 0.852. In addition, the student model achieves higher $PQ$ values for the neoplastic and connective classes, though the confidence intervals show overlap.

\begin{table}[h!]
\renewcommand{\arraystretch}{1.5}
  \centering
  \caption{Cell quantification metrics for teacher and distilled student models (CI 95\%).}
  \label{tab:3}
  \begin{tabular}{|l|c|c|}
    \hline
    %\rowcolor{gray!30}
    Metric & Teacher & Student \\
    \hline
    $R^2_{neoplastic}$    & \textbf{0.919} (0.898--0.939) & 0.852 (0.800--0.891) \\
    \hline
    $R^2_{lymphocyte}$    & 0.969 (0.956--0.977)         & 0.969 (0.956--0.978) \\
    \hline
    $R^2_{connective}$    & 0.694 (0.548--0.809)         & 0.618 (0.469--0.741) \\
    \hline
    $R^2_{dead}$          & 0.755 (0.400--0.908)         & 0.424 (0.100--0.731) \\
    \hline
    $R^2_{epithelial}$    & 0.922 (0.870--0.958)         & 0.843 (0.738--0.917) \\
    \hline
    $R^2_{macrophage}$    & 0.384 (-0.369--0.724)        & 0.704 (0.352--0.859) \\
    \hline
    $R^2_{neutrofil}$     & 0.854 (0.578--0.929)         & 0.833 (0.502--0.925) \\
    \hline
    $PQ_{neoplastic}$       & 0.581 (0.569--0.593)         & 0.601 (0.588--0.613) \\
    \hline
    $PQ_{lymphocyte}$       & 0.536 (0.520--0.553)         & 0.563 (0.544--0.579) \\
    \hline
    $PQ_{connective}$       & 0.436 (0.421--0.451)         & 0.457 (0.441--0.474) \\
    \hline
    $PQ_{dead}$             & 0.272 (0.235--0.315)         & 0.279 (0.201--0.369) \\
    \hline
    $PQ_{epithelial}$       & 0.522 (0.500--0.545)         & 0.530 (0.506--0.555) \\
    \hline
    $PQ_{macrophage}$       & 0.524 (0.459--0.588)         & 0.474 (0.405--0.543) \\
    \hline
    $PQ_{neutrofil}$        & 0.541 (0.490--0.592)         & 0.565 (0.522--0.607) \\
    \hline
  \end{tabular}
\end{table}


We further decompose the $PQ$ metric into its $SQ$ and $DQ$ components (\hyperref[tab:S6]{Appendix Table S6}). Both models produce nearly identical $SQ$ values, which indicates that they predict instance boundaries with similar precision. Although the student model shows some improvement in $DQ$ scores for certain classes, the confidence intervals overlap and do not confirm a statistically significant difference.

We observe that the student and teacher models yield comparable detection performance despite the student model using a much smaller and simpler architecture. A model with fewer parameters reduces the risk of overfitting when training data are scarce relative to the model’s complexity \cite{Farias_Ludermir_etal._2022}. The knowledge distillation process also encourages the student model to focus on the most generalizable detection features learned from the teacher. These factors enable the student model to achieve similar detection performance across different cell types.

Additionally, considering the model sizes reported in \hyperref[tab:4]{Table 4}, the distilled model achieves a significant reduction compared to the teacher model, with a 48-fold decrease in parameter count and a 5.5-fold reduction in on-disk size. In inference mode, the teacher model requires 16 GB of VRAM for a batch size of 32, while the distilled model only needs 3 GB of VRAM for the same batch size. These reductions make the distilled model significantly more practical for fine-tuning and deployment in resource-constrained environments.

\begin{table}[h!]
\renewcommand{\arraystretch}{1.5}
  \centering
  \caption{Parameter counts and size of teacher and distilled model}
  \label{tab:4}
  \adjustbox{max width=\textwidth}{%
  \begin{tabular}{|l|c|c|c|}
    \hline
    %\rowcolor{gray!30}
    Metric & H-optimus-based (Teacher) & mobileViT-based (Student) & Magnitude of difference \\
    \hline
    Parameters count       & 1,158,917,906   & \textbf{24,093,393}   & \textbf{48x}  \\
    \hline
    Estimated Total Size (MB) & 87,912       & \textbf{15,935}    & \textbf{5.5x} \\
    \hline
  \end{tabular}%
}
\end{table}

%\hfill

With recent advancements in complex network architectures and the use of pretrained encoders to achieve state-of-the-art performance \cite{Baumann_Dislich_etal._2024,Hörst_Rempe_etal._2024} in cell segmentation and classification tasks, model size, computational complexity, and processing times have increased. This limits the scalability and accessibility of these models. As we demonstrate, this may be mitigated using knowledge distillation. Studies in the field of natural language processing have demonstrated the efficacy of knowledge distillation in retaining the capabilities of the teacher model while achieving significant reductions in size and complexity \cite{Huangpu_Gao_2024,Sun_Yu_etal.}. 

We demonstrate the feasibility of knowledge distillation in digital pathology, specifically for cell segmentation and classification tasks. Moreover, we achieve this performance while also significantly reducing the parameter count. In addressing the challenge of knowledge transfer, we found that distillation from a transformer-based model to a smaller transformer is more straightforward than attempting to map transformer features to CNN blocks. In our experiments, using a CNN-based network as a student results in worse cell quantification performance due to the structural constraints of CNN feature space dimensions. 

Although our primary approach relies on a transformer-based student model that performs well, it can be further optimized to incorporate advantages from CNN architectures. For example, employing alternative techniques such as using ViT adapters \cite{Chen_Duan_etal._2023} or $1 \times 1$ convolutions to adjust feature map sizes may be beneficial for harnessing CNN advantages like enhanced local feature extraction. Moreover, if additional performance improvements are desired, the process can be further enhanced by applying supplementary knowledge distillation techniques, such as self-distillation \cite{Zhang_Song_etal._2019} or online distillation \cite{Houyon_Cioppa_etal._2023}.

Despite these promising results, further validation on independent datasets is necessary to fully understand the model's limitations. Underrepresented classes may pose challenges when addressing complex cases. Pathologists need to validate these models to adopt them in clinical settings. While the distilled models are smaller and more deployable, a technological gap persists because pathologists traditionally rely on established methods for inspecting WSIs and diagnosing diseases. Addressing the complexities involved in deploying models for inference and supporting pathologists in adopting new tools is essential for integrating these models into clinical workflows.

\section{Model integration with QuPath}
Digital pathology tools with graphical user interfaces are essential for visualizing and analyzing WSIs. To make our student model useful in clinical pathology workflows, it needs to be integrated into a tool that enables inspecting regions, creating annotations, and providing quantitative analyses of biomarkers. Therefore, we integrate the trained student model from the previous chapter into the QuPath open‑source platform \cite{Bankhead_Loughrey_etal._2017}. QuPath provides the required annotation, visualization, and analysis tools to interpret complex histological data, including workflows for cell segmentation, classification, and quantification (\hyperref[fig:fig7]{Figure 7}). 

\begin{figure}[h!]
    \centering
    \includegraphics[width=\textwidth]{images/Figure_7.pdf}
    \caption{Visualization of model-generated cell quantification annotations (left) and the corresponding unannotated slide (right) in QuPath}
    \label{fig:fig7}
\end{figure}

To identify the regions in a WSI critical for prognosticating tumor development, such as specific tumor areas or border regions without overlapping healthy tissue, the pathologist uses QuPath to outline these regions. Then, the pathologist initiates a cell segmentation and classification script through the QuPath interface for the selected regions. The resulting annotations and quantified cell information are then directly overlaid onto the WSI in the QuPath interface. Additional design and implementation details are in \hyperref[chap:S7]{Appendix S7}. 

Two common approaches for integrating deep learning models into QuPath are Java‑based native QuPath extensions \cite{Goldsborough_Philps_etal._2024} and the execution of RESTful API requests to a model server coupled with handling the response via an extension, as demonstrated in the application of cell segmentation models applied to immunofluorescence images \cite{Sugawara_2023}. While the community is actively working on these integration strategies, there is currently no universal solution that fully addresses all integration and performance requirements.

Extensions may offer better integration with QuPath, allowing slightly improved performance and more widespread usage of the built-in QuPath models, but they lack the flexibility to customize models and modify their behavior. For example, the newest version of QuPath includes models such as StarDist \cite{Weigert_Schmidt} and InstanSeg \cite{Goldsborough_Philps_etal._2024} that can perform cell segmentation. Both models pose limitations when applied to simultaneous cell segmentation and classification. StarDist performs well only on convex, round shapes by design, whereas some neoplastic, inflammatory, and connective cells exhibit complex and non-convex shapes. InstanSeg provides only semantic segmentation without assigning classes to the segmented cells.

%\hfill

In contrast, our approach offers an alternative integration strategy. It utilizes the paquo library to directly interact with QuPath’s internal application programming interface from within Python. This enables data exchange and processing without the need for intermediate conversion steps and provides greater control over model customization, retraining, and the incorporation of custom processing steps.

The integration of our custom model with QuPath underscores its potential to significantly enhance the diagnostic process by reducing the time burden on pathologists and enabling them to focus on more complex interpretative tasks using familiar software. Leveraging a tool that is already well-established among pathologists increases the likelihood of its adoption into daily clinical workflows. The quantitative data generated through the automated workflow is critical for both clinical decision-making and research, facilitating more accurate biomarker analysis, enabling robust statistical evaluations, and supporting hypothesis generation and testing. Additionally, by streamlining cell segmentation and classification, the tool enhances the scalability and reproducibility of pathological assessments, ultimately contributing to improved diagnostic accuracy and patient outcomes.

\section{Conclusion and future work}

In this study, we address critical challenges in digital pathology and tackle the usability and deployment issues of the developed models in standard computing environments without the need for high-performance computing systems. Our multi-faceted approach encompasses data refinement through cross-relabeling, leveraging foundation models for robust cell segmentation and classification, optimizing model performance via knowledge distillation, and integrating the optimized model into the QuPath software for practical application. This approach is used to construct a capable, versatile, and adjustable model for cell segmentation and classification, with enhanced performance and usability.

\begin{sloppypar}
While our approach shows potential in the field of computational pathology, certain limitations persist. 
For example, our implementation currently exhibits lower performance in detecting macrophages. 
This serves as an instance of the broader challenge of accurately identifying complex cell types. In order to address this issue, extending our approach to incorporate additional data sources, exploring alternative modeling approaches, and integrating other imaging modalities such as immunohistochemical staining may help improve detection accuracy. Moreover, although the distilled model reduces computational demands, integrating advanced deep learning models into clinical practice requires addressing technological gaps and potential resistance to adopting new tools within established diagnostic processes.
\end{sloppypar}

Future work could focus on several key areas to refine the proposed approach and facilitate its adoption in clinical environments. Enhancing the cell-relabeling process with additional datasets \cite{Graham_Jahanifar_etal._2021} could improve the representation of underrepresented cell types and enhance overall model performance. Also, incorporating additional data sources, such as multi-modal imaging or complementary staining methods, may address limitations related to cell type differentiation and class imbalance. Exploring other foundation models \cite{Vorontsov_Bozkurt_etal._2024,Zimmermann_Vorontsov_etal._2024} or introducing additional modalities \cite{Ding_Wagner_etal._2024,Vaidya_Zhang_etal._2025} may provide alternative architectures better suited to specific tasks or offer improved efficiency. Implementing more complex knowledge distillation techniques \cite{Houyon_Cioppa_etal._2023,Zhang_Song_etal._2019} could further optimize the model's performance and adaptability. Additionally, deeper integration with QuPath or other digital pathology software could provide pathologists more control over cell quantification analysis directly within the QuPath interface, thereby increasing accessibility and usability. Such enhancements would not only refine model performance but also ensure greater adaptability and scalability within various clinical environments. Finally, extensive validation of the model by pathologists and benchmarking against independent datasets are essential steps toward establishing the model's reliability and fostering confidence in its clinical utility.

\section*{Acknowledgments} 
This work was funded in part by the Research Council of Norway grant no. 309439 SFI Visual Intelligence, and the North Norwegian Health Authority grant no. HNF1521-20.

\bibliographystyle{IEEEtran}
\begin{sloppypar}
\begin{thebibliography}{99}

\bibitem{chaplot2020neural} Chaplot, Devendra Singh, et al. "Neural topological slam for visual navigation." Proceedings of the IEEE/CVF conference on computer vision and pattern recognition. 2020.

\bibitem{maksymets2021thda} Maksymets, Oleksandr, et al. "Thda: Treasure hunt data augmentation for semantic navigation." Proceedings of the IEEE/CVF International Conference on Computer Vision. 2021.

\bibitem{mezghan2022memory} Mezghan, Lina, et al. "Memory-augmented reinforcement learning for image-goal navigation." 2022 IEEE/RSJ International Conference on Intelligent Robots and Systems (IROS). IEEE, 2022.

\bibitem{al2022zero} Al-Halah, Ziad, Santhosh Kumar Ramakrishnan, and Kristen Grauman. "Zero experience required: Plug \& play modular transfer learning for semantic visual navigation." Proceedings of the IEEE/CVF Conference on Computer Vision and Pattern Recognition. 2022.

\bibitem{ye2021auxiliary} Ye, Joel, et al. "Auxiliary tasks and exploration enable objectgoal navigation." Proceedings of the IEEE/CVF international conference on computer vision. 2021.

\bibitem{chaplot2020object} Chaplot, Devendra Singh, et al. "Object goal navigation using goal-oriented semantic exploration." Advances in Neural Information Processing Systems 33 (2020)

\bibitem{ramakrishnan2022poni} Ramakrishnan, Santhosh Kumar, et al. "Poni: Potential functions for objectgoal navigation with interaction-free learning." Proceedings of the IEEE/CVF Conference on Computer Vision and Pattern Recognition. 2022.

\bibitem{ramrakhya2022habitat} Ramrakhya, Ram, et al. "Habitat-web: Learning embodied object-search strategies from human demonstrations at scale." Proceedings of the IEEE/CVF Conference on Computer Vision and Pattern Recognition. 2022.

\bibitem{mousavian2019visual} Mousavian, Arsalan, et al. "Visual representations for semantic target driven navigation." 2019 International Conference on Robotics and Automation (ICRA). IEEE, 2019.

\bibitem{dhariwal2021diffusion} Dhariwal, Prafulla, and Alexander Nichol. "Diffusion models beat gans on image synthesis." Advances in neural information processing systems 34 (2021)

\bibitem{ho2022classifier} Ho, Jonathan, and Tim Salimans. "Classifier-free diffusion guidance." arXiv preprint arXiv:2207.12598 (2022).

\bibitem{nichol2021glide} Nichol, Alex, et al. "Glide: Towards photorealistic image generation and editing with text-guided diffusion models." arXiv preprint arXiv:2112.10741 (2021)

\bibitem{brooks2023instructpix2pix} Brooks, Tim, Aleksander Holynski, and Alexei A. Efros. "Instructpix2pix: Learning to follow image editing instructions." Proceedings of the IEEE/CVF Conference on Computer Vision and Pattern Recognition. 2023.

\bibitem{fu2023guiding} Fu, Tsu-Jui, et al. "Guiding instruction-based image editing via multimodal large language models." arXiv preprint arXiv:2309.17102 (2023).

\bibitem{geng2024instructdiffusion} Geng, Zigang, et al. "Instructdiffusion: A generalist modeling interface for vision tasks." Proceedings of the IEEE/CVF Conference on Computer Vision and Pattern Recognition. 2024.

\bibitem{zhou2024minedreamer} Zhou, Enshen, et al. "Minedreamer: Learning to follow instructions via chain-of-imagination for simulated-world control." arXiv preprint arXiv:2403.12037 (2024).

\bibitem{zhou2023esc} Zhou, Kaiwen, et al. "Esc: Exploration with soft commonsense constraints for zero-shot object navigation." International Conference on Machine Learning. PMLR, 2023.

\bibitem{yu2023l3mvn} Yu, Bangguo, Hamidreza Kasaei, and Ming Cao. "L3mvn: Leveraging large language models for visual target navigation." 2023 IEEE/RSJ International Conference on Intelligent Robots and Systems (IROS). IEEE, 2023.

\bibitem{gadre2023cows} Gadre, Samir Yitzhak, et al. "Cows on pasture: Baselines and benchmarks for language-driven zero-shot object navigation." Proceedings of the IEEE/CVF Conference on Computer Vision and Pattern Recognition. 2023.

\bibitem{shah2023navigation} Shah, Dhruv, et al. "Navigation with large language models: Semantic guesswork as a heuristic for planning." Conference on Robot Learning. PMLR, 2023.

\bibitem{cai2024bridging} Cai, Wenzhe, et al. "Bridging zero-shot object navigation and foundation models through pixel-guided navigation skill." 2024 IEEE International Conference on Robotics and Automation (ICRA). IEEE, 2024.

\bibitem{yu2023co} Yu, Bangguo, Hamidreza Kasaei, and Ming Cao. "Co-NavGPT: Multi-robot cooperative visual semantic navigation using large language models." arXiv preprint arXiv:2310.07937 (2023).

\bibitem{wu2024voronav} Wu, Pengying, et al. "Voronav: Voronoi-based zero-shot object navigation with large language model." arXiv preprint arXiv:2401.02695 (2024).

\bibitem{qin2023mp5} Qin, Yiran, et al. "Mp5: A multi-modal open-ended embodied system in minecraft via active perception." arXiv preprint arXiv:2312.07472 (2023).

\bibitem{du2024learning} Du, Yilun, et al. "Learning universal policies via text-guided video generation." Advances in Neural Information Processing Systems 36 (2024).

\bibitem{ajay2024compositional} Ajay, Anurag, et al. "Compositional foundation models for hierarchical planning." Advances in Neural Information Processing Systems 36 (2024).

\bibitem{liang2024skilldiffuser} Liang, Zhixuan, et al. "Skilldiffuser: Interpretable hierarchical planning via skill abstractions in diffusion-based task execution." Proceedings of the IEEE/CVF Conference on Computer Vision and Pattern Recognition. 2024.

\bibitem{heusel2017gans} Heusel, Martin, et al. "Gans trained by a two time-scale update rule converge to a local nash equilibrium." Advances in neural information processing systems 30 (2017).

\bibitem{zhang2018unreasonable} Zhang, Richard, et al. "The unreasonable effectiveness of deep features as a perceptual metric." Proceedings of the IEEE conference on computer vision and pattern recognition. 2018.

\bibitem{brown2020language} Brown, Tom B. "Language models are few-shot learners." arXiv preprint arXiv:2005.14165 (2020).

\bibitem{podell2023sdxl} Podell, Dustin, et al. "Sdxl: Improving latent diffusion models for high-resolution image synthesis." arXiv preprint arXiv:2307.01952 (2023).

\bibitem{brohan2022rt} Brohan, Anthony, et al. "Rt-1: Robotics transformer for real-world control at scale." arXiv preprint arXiv:2212.06817 (2022).

\bibitem{brohan2023rt} Brohan, Anthony, et al. "Rt-2: Vision-language-action models transfer web knowledge to robotic control." arXiv preprint arXiv:2307.15818 (2023).

\bibitem{li2024manipllm} Li, Xiaoqi, et al. "Manipllm: Embodied multimodal large language model for object-centric robotic manipulation." Proceedings of the IEEE/CVF Conference on Computer Vision and Pattern Recognition. 2024.

\bibitem{shah2023vint} Shah, Dhruv, et al. "ViNT: A foundation model for visual navigation." arXiv preprint arXiv:2306.14846 (2023).

\bibitem{liu2024visual} Liu, Haotian, et al. "Visual instruction tuning." Advances in neural information processing systems 36 (2024).

\bibitem{hu2021lora} Hu, Edward J., et al. "Lora: Low-rank adaptation of large language models." arXiv preprint arXiv:2106.09685 (2021).

\bibitem{qin2023supfusion} Qin, Yiran, et al. "SupFusion: Supervised LiDAR-camera fusion for 3D object detection." Proceedings of the IEEE/CVF International Conference on Computer Vision. 2023.

\bibitem{qin2024worldsimbench} Qin, Yiran, et al. "Worldsimbench: Towards video generation models as world simulators." arXiv preprint arXiv:2410.18072 (2024).

\bibitem{yu2025gamefactory} Yu, Jiwen, et al. "GameFactory: Creating New Games with Generative Interactive Videos." arXiv preprint arXiv:2501.08325 (2025).

\bibitem{zhou2024code} Zhou, Enshen, et al. "Code-as-Monitor: Constraint-aware Visual Programming for Reactive and Proactive Robotic Failure Detection." arXiv preprint arXiv:2412.04455 (2024).

\bibitem{zhang2024ad} Zhang, Zaibin, et al. "AD-H: Autonomous Driving with Hierarchical Agents." arXiv preprint arXiv:2406.03474 (2024).

\bibitem{wang2024toward} Wang, Chaoqun, et al. "Toward Accurate Camera-based 3D Object Detection via Cascade Depth Estimation and Calibration." arXiv preprint arXiv:2402.04883 (2024).

\bibitem{huang2024story3d} Huang, Yuzhou, et al. "Story3d-agent: Exploring 3d storytelling visualization with large language models." arXiv preprint arXiv:2408.11801 (2024).

\bibitem{savinov2018semi} Savinov, Nikolay, Alexey Dosovitskiy, and Vladlen Koltun. "Semi-parametric topological memory for navigation." arXiv preprint arXiv:1803.00653 (2018).

\bibitem{majumdar2022zson} Majumdar, Arjun, et al. "Zson: Zero-shot object-goal navigation using multimodal goal embeddings." Advances in Neural Information Processing Systems 35 (2022): 32340-32352.

\bibitem{yadav2023offline} Yadav, Karmesh, et al. "Offline visual representation learning for embodied navigation." Workshop on Reincarnating Reinforcement Learning at ICLR 2023. 2023.

\bibitem{yadav2023ovrl} Yadav, Karmesh, et al. "Ovrl-v2: A simple state-of-art baseline for imagenav and objectnav." arXiv preprint arXiv:2303.07798 (2023).

\bibitem{sun2024fgprompt} Sun, Xinyu, et al. "FGPrompt: fine-grained goal prompting for image-goal navigation." Advances in Neural Information Processing Systems 36 (2024).

\bibitem{zhu2017target} Zhu, Yuke, et al. "Target-driven visual navigation in indoor scenes using deep reinforcement learning." 2017 IEEE international conference on robotics and automation (ICRA). IEEE, 2017.

\bibitem{koh2024generating} Koh, Jing Yu, Daniel Fried, and Russ R. Salakhutdinov. "Generating images with multimodal language models." Advances in Neural Information Processing Systems 36 (2024).

\bibitem{krantz2022instance} Krantz, Jacob, et al. "Instance-specific image goal navigation: Training embodied agents to find object instances." arXiv preprint arXiv:2211.15876 (2022).

\bibitem{schulman2017proximal} Schulman, John, et al. "Proximal policy optimization algorithms." arXiv preprint arXiv:1707.06347 (2017).

\bibitem{anderson2018evaluation} Anderson, Peter, et al. "On evaluation of embodied navigation agents." arXiv preprint arXiv:1807.06757 (2018).

\bibitem{lin2024navcot} Lin, Bingqian, et al. "NavCoT: Boosting LLM-Based Vision-and-Language Navigation via Learning Disentangled Reasoning." arXiv preprint arXiv:2403.07376 (2024).

\bibitem{NavGPT} Zhou, Gengze, Yicong Hong, and Qi Wu. "Navgpt: Explicit reasoning in vision-and-language navigation with large language models." Proceedings of the AAAI Conference on Artificial Intelligence.

\bibitem{hahn2021no} Hahn, Meera, et al. "No rl, no simulation: Learning to navigate without navigating." Advances in Neural Information Processing Systems 34 (2021): 26661-26673.

\bibitem{li2025t2isafety} Li, Lijun, et al. "T2ISafety: Benchmark for Assessing Fairness, Toxicity, and Privacy in Image Generation." arXiv preprint arXiv:2501.12612 (2025).

\bibitem{an2024agfsync} An, Jingkun, et al. "AGFSync: Leveraging AI-Generated Feedback for Preference Optimization in Text-to-Image Generation." arXiv preprint arXiv:2403.13352 (2024).


\end{thebibliography}
\end{sloppypar}

\clearpage
\beginsupplement
\section*{Appendix}
\renewcommand{\thesubsection}{S\arabic{subsection}}

\subsection{\label{chap:S1}PanNuke and MoNuSAC preprocessing}
The PanNuke dataset comprises a set of 7,901 RGB patches, each with dimensions of $256 \times 256$ pixels, which we set as the standard patch size for our analysis. In contrast, the MoNuSAC dataset encompasses 294 images of heterogeneous dimensions. To standardize the MoNuSAC images with our experiments, we implement a standardization protocol. Specifically, for images exceeding the dimensions of $256 \times 256$ pixels, we segment them into equal-sized patches and apply mirror padding to the remaining portions to avoid information loss at the peripherals. Patches with dimensions less than $128 \times 128$ pixels are excluded from the dataset due to the insufficient resolution to capture relevant cellular details. For patches where either dimension falls between 128 and 256 pixels, we employ upsampling to achieve the standard patch size. As a result, we obtain a total of 2,823 RGB patches derived from the MoNuSAC dataset for subsequent analysis. For additional details on the MoNuSAC data preparation process, refer to the source code \cite{Shvetsov_2025a}.
\clearpage

\subsection{\label{chap:S2}Data usage for the methodology}

\counterwithin{figure}{subsection}
\renewcommand{\thefigure}{S\arabic{subsection}}

\begin{figure}[h!]
    \centering
    \includegraphics[width=\textwidth, height=0.85\textheight, keepaspectratio]{images/A2.pdf}
    \caption{Overview of the methodology for cross-labeling, dataset refinement, and model comparison. (1) Cross-relabeling - training and testing cell classification models, (2) Cross-relabeling - using cell classification models to create refined dataset, (3) Fine-tuning and training models for comparison, (4) Student knowledge distillation with refined dataset}
    \label{fig:S2}
\end{figure}
\clearpage

\subsection{\label{chap:S3}Confusion matrices for classification models}
\counterwithin{figure}{subsection}
\renewcommand{\thefigure}{S\arabic{subsection}.\arabic{figure}}

\begin{figure}[h!]
    \centering
    \includegraphics[width=\textwidth, height=0.4\textheight, keepaspectratio]{images/A3_1.pdf}
    \caption{Confusion matrix for PanNuke trained model}
    \label{fig:S3.1}
\end{figure}

\begin{figure}[h!]
    \centering
    \includegraphics[width=\textwidth, height=0.4\textheight, keepaspectratio]{images/A3_2.pdf}
    \caption{Confusion matrix for MoNuSAC trained model}
    \label{fig:S3.2}
\end{figure}

\clearpage

\subsection{\label{chap:S4}Datasets cell counts}

\counterwithin{table}{subsection}
\renewcommand{\thetable}{S\arabic{subsection}}

\begin{table}[h!]
\renewcommand{\arraystretch}{2.0}
\centering
\caption{\label{tab:S4}Cell counts for PanNuke, MoNuSAC and refined datasets. Numbers in parentheses indicate preprocessed cell counts for cell classifier models training and testing.}
%\adjustbox{max width=\textwidth}{%
\begin{tabular}{|l|c|c|c|}
\hline
%\rowcolor{gray!30}
Cell type & PanNuke & MoNuSAC & Refined \\
\hline
Neoplastic & 77,403 (68,031) & - & 105,451 \\
\hline
Epithelial & 26,572 (23,207) & - & 29,926 \\
\hline
Epithelial (benign and malignant) & - & 31,402 & - \\
\hline
Inflammatory & 32,276 & - & - \\
\hline
Lymphocytes & - & 37,045 (33,104) & 65,275 \\
\hline
Neutrophils & - & 1,355 (1,252) & 3,833 \\
\hline
Macrophage & - & 1,842 (1,695) & 3,410 \\
\hline
Dead & 2,908 & - & 2,908 \\
\hline
Connective & 50,585 & - & 50,585 \\
\hline
\end{tabular}
%
%}
\end{table}



\clearpage

\subsection{\label{chap:S5}Definition of validation metrics}
\counterwithin{equation}{subsection}
\renewcommand{\theequation}{\arabic{equation}}

\subsubsection{\label{chap:S5.1}R\textsuperscript{2}}
The coefficient of determination, denoted as $R^2$, is a statistical measure that represents the proportion of variance in the dependent variable that is predictable from the independent variables. In the context of cell quantification in pathology, $R^2$ is used to assess how well the predicted quantities of different cell types in a patch align with the actual quantities observed in the ground truth data, with higher values representing more accurate quantification. $R^2$ is defined as
\begin{equation*}
R^2 = 1 - \frac{\sum_{i=1}^n (y_i - \hat{y}_i)^2}{\sum_{i=1}^n (y_i - \bar{y})^2},
\end{equation*}
where $y_i$ represents the actual number of cells of a specific type in the $i$-th image, $\hat{y}_i$ represents the predicted number of cells of that type in the $i$-th image, $\bar{y}$ is the mean of the actual numbers across all images, and $n$ is the total number of images in the dataset.

The $R^2$ metric has a range of $(-\infty, 1]$. An $R^2$ of 1 indicates perfect prediction, where all predicted values exactly match the actual values. An $R^2$ of 0 suggests that the model explains none of the variability of the response data around its mean. If $R^2$ is negative, it indicates that the model performs worse than a model that simply predicts the mean of the actual values for all observations.

\subsubsection{\label{chap:S5.2}PQ}
Panoptic Quality ($PQ$) is a comprehensive metric used to evaluate the performance of segmentation models in tasks that require both instance segmentation and classification. $PQ$ provides a single score that encapsulates both the detection accuracy (i.e., how many objects were correctly identified) and the segmentation quality (i.e., how accurately the objects' boundaries were delineated). This metric is particularly useful in multiclass scenarios where each pixel is classified into distinct categories, such as different cell types in pathology images.

$PQ$ is calculated as the product of two terms: Detection Quality ($DQ$) and Segmentation Quality ($SQ$). It can be expressed as
\begin{equation*}
PQ = DQ \cdot SQ,
\end{equation*}
where
\begin{equation*}
DQ = \frac{TP}{TP + 0.5\, FP + 0.5\, FN},
\end{equation*}
\begin{equation*}
SQ = \frac{\sum_{(p, g) \in \mathcal{M}} IoU(p, g)}{TP}.
\end{equation*}
In these formulas, $TP$ denotes the number of correctly matched instances between ground truth and prediction, $FP$ denotes the predicted instances that have no corresponding ground truth, $FN$ denotes the ground truth instances that were not detected, $IoU(p, g)$ is the Intersection over Union for a pair of matched instances $p$ (prediction) and $g$ (ground truth), and $\mathcal{M}$ is the set of matched pairs.

The $PQ$ metric is calculated for each class and is averaged across classes to provide a global performance measure.

The $PQ$ score has a range of $[0, 1.0]$, where a higher score indicates better performance in both detecting and segmenting the instances correctly. A $PQ$ of 1 signifies perfect identification and segmentation of all instances, whereas a $PQ$ of 0 indicates that no instances were correctly identified and segmented.

\clearpage

\subsection{\label{chap:S6}Segmentation and Detection quality metrics for teacher and student models}

\begin{table}[h!]
\renewcommand{\arraystretch}{2.0}
\centering
\caption{Segmentation and detection quality for student and teacher models (CI 95\%)}
\label{tab:S6}
%\adjustbox{max width=\textwidth}{%
\begin{tabular}{|l|c|c|}
\hline
%\rowcolor{gray!30}
Metric & Teacher & Student \\
\hline
$SQ_{neoplastic}$ & 0.819 (0.815--0.823) & 0.824 (0.819--0.828) \\
\hline
$SQ_{lymphocyte}$ & 0.795 (0.788--0.802) & 0.790 (0.783--0.796) \\
\hline
$SQ_{connective}$ & 0.770 (0.762--0.776) & 0.780 (0.772--0.786) \\
\hline
$SQ_{dead}$ & 0.659 (0.623--0.688) & 0.657 (0.624--0.695) \\
\hline
$SQ_{epithelial}$ & 0.780 (0.770--0.790) & 0.788 (0.779--0.797) \\
\hline
$SQ_{macrophage}$ & 0.788 (0.760--0.810) & 0.757 (0.730--0.783) \\
\hline
$SQ_{neutrofil}$ & 0.782 (0.761--0.801) & 0.775 (0.759--0.792) \\
\hline
$DQ_{neoplastic}$ & 0.706 (0.692--0.719) & 0.727 (0.712--0.741) \\
\hline
$DQ_{lymphocyte}$ & 0.675 (0.656--0.698) & 0.713 (0.691--0.734) \\
\hline
$DQ_{connective}$ & 0.566 (0.546--0.584) & 0.583 (0.565--0.602) \\
\hline
$DQ_{dead}$ & 0.410 (0.361--0.465) & 0.435 (0.306--0.561) \\
\hline
$DQ_{epithelial}$ & 0.668 (0.639--0.694) & 0.673 (0.644--0.702) \\
\hline
$DQ_{macrophage}$ & 0.657 (0.583--0.727) & 0.615 (0.531--0.703) \\
\hline
$DQ_{neutrofil}$ & 0.691 (0.625--0.753) & 0.729 (0.679--0.778) \\
\hline
\end{tabular}
%
%}
\end{table}

\clearpage

\subsection{\label{chap:S7}QuPath integration method}
We adopt an integration strategy leveraging the paquo \cite{Bayer_AG} library, a Python package that enables direct interaction with QuPath’s internal API, thereby facilitating seamless data exchange without intermediate conversion steps. The data processing pipeline (\hyperref[fig:S7]{Appendix Figure S7}) begins with the acquisition of WSIs and their associated annotations from QuPath, which are represented as Shapely \cite{Gillies_Wel_etal._2024} polygons. Utilizing paquo, we directly read, create, and modify these annotations and detections within a QuPath project in the Python environment. Images are then cropped using these polygons and processed by cell segmentation and classification models employing standard vision processing toolkits such as OpenCV, pyvips, and PyTorch. Additionally, QuPath employs Groovy scripts to initiate a Python process that starts the entire pipeline from QuPath graphical interface: fetching polygons, extracting images from them, and running deep learning model inference on the cropped images. 
The results are returned to QuPath, leveraging paquo's Python bindings to manipulate QuPath data while minimizing the computational overhead typically associated with cross-environment communication.

\counterwithin{figure}{subsection}
\renewcommand{\thefigure}{S\arabic{subsection}}

\begin{figure}[h!]
    \centering
    \includegraphics[width=\textwidth]{images/A7.pdf}
    \caption{QuPath integration workflow using Python environment}
    \label{fig:S7}
\end{figure}

Compared to traditional workflows that involve exporting annotations as GeoJSON, classifying them in Python, and reimporting them into QuPath, our approach offers several advantages. We eliminate the need to switch between programming languages, providing a cohesive and streamlined development process entirely within QuPath software and removing the necessity to use other tools. Meanwhile, we avoid storing annotations as intermediate JSON files unless required for external use or archiving. By conducting the entire inference and post-processing workflow within the Python environment, we leverage the power and flexibility of Python libraries for image processing and machine learning. This approach also enables adjustments to any set of labels and models, thereby improving its applicability.

%\hfill

The distilled model and QuPath integration code are packaged into a Docker container, enabling streamlined execution with the Docker engine. Detailed integration code and deployment instructions can be found in the GitHub repository \cite{Shvetsov_2025b}.

Despite these benefits, we acknowledge that the paquo library is a proof‑of‑concept project in its early development stage and has not been tested across all versions of QuPath.

\clearpage

\subsection{\label{chap:S8}Data and code availability statement}
All datasets, models, and code used in this study are publicly available and can be obtained from the repositories listed below. 
The PanNuke \cite{Gamper_Koohbanani_etal._2019} and MoNuSAC \cite{Verma_Kumar_etal._2021} datasets are publicly accessible, and download information along with detailed descriptions can be found in their respective articles. Preprocessing scripts for PanNuke and MoNuSAC data, as well as individual cell extraction scripts, are available on GitHub \cite{Shvetsov_2025a}. The H-Optimus foundation model used in our experiments can be downloaded from the HuggingFace repository \cite{hoptimus2024}, and model information is available on GitHub \cite{Saillard_Jenatton_etal._2024}. In addition, the integration code for QuPath and the distilled model packaged in a Docker container are provided in the repository \cite{Shvetsov_2025b}, and paquo Python library is available from the authors GitHub repository \cite{Bayer_AG}.
\clearpage

\end{document}


\clearpage

\onecolumn
\appendix

\section{Flowchart for Constructing Preference Data}
\label{appendix:preference_data}
Figure~\ref{appendix:constructing_preference_data} shows the flowchart for the construction of a pair of chosen-rejected rationales for a given news article in the training data. We apply this process and construct a dataset containing preferences for all the instance in the training data. Then the preference data is used for DPO. 
\begin{figure}[!h]
  \centering
  \includegraphics[width=\textwidth]{f5}
  \caption{Flowchart for constructing preference data}
  \label{appendix:constructing_preference_data}
\end{figure}

% \clearpage

\section{Implementation Details}
\label{appendix:implementation}
We employ GPT-4o as the teacher LLM to generate supervision data and experiment with two student LLMs: Mistral-7B-v0.3 and Llama-3.1-8B. All our rationale generators and headline generators are fine-tuned on the training data for three epochs. We apply QLoRA \citep{dettmers_qlora_2023} techniques to fine-tune the student LLMs efficiently. For all rationale generators, we set the LoRA rank and LoRA alpha to 128 and 64, respectively. For all headline generators, we set these values to 64 and 32. We fine-tune Mistral-7B-v0.3 with a learning rate of 2e-4 and Llama-3.1-8B with 8e-4. To create the preference dataset of chosen and rejected rationales, we fine-tune Mistral-7B-v0.3 and Llama-3.1-8B on the training data for a single epoch, sample 15 TEN rationales from each model for every news article, remove duplicate generations, and create a pair of chosen and rejected rationales for each training sample. During DPO, we set the LoRA rank to 256, LoRA alpha to 128, and DPO beta to 0.8. We set the learning rates for Mistral-7B-v0.3 and Llama-3.1-8B to 2e-6 and 8e-6 when doing experiments on NumHG, while setting them to 5e-6 and 2e-5 on XSum.
% \clearpage

% \section{Error Analysis}
% \label{appendix:error_analysis}


\clearpage
\section{Generation of Supervision Data}
\label{appendix: five-demonstrations}

The following five examples are demonstrations for five distinct operations needed to calculate the correct numbers in the headlines. The outputs (TEN rationales) are generated by the GPT-4o via zero-shot prompting respectively. Then the five examples, including generated TEN rationales, are used as demonstrations for the GPT-4o to generate supervision data for the entire training dataset.

\begin{table}[htbp]
    \centering
    \begin{tabular}{p{0.9\textwidth}}
    \hline
    \textbf{Example 1: Copy} \\
    \hline
    \textbf{Input:} \\
    \textbf{News:} (Oct 29, 2013 8:15 AM CDT) Dax Shepard and Kristen Bell got married at the Beverly Hills courthouse, in a ceremony about as different from Kim Kardashian's last wedding extravaganza as it is possible to be. As Shepard revealed last night on Jimmy Kimmel Live, the whole thing—including the fuel it took to get to the courthouse—cost \$142. It was just Kristen and I at this lonely courthouse, he said, so friends showed up afterward with a cake reading, in icing, The World's Worst Wedding. How many people can say they threw the world's worst wedding? Shepard asked. \\
    \textbf{Headline:} Dax Shepard: Wedding to Kristen Bell Cost \$\_\_\_\_. \\
    \textbf{Correct Number:} 142 \\
    \textbf{System Instruction: } You will be given a piece of news after prefix "News: ". You will be given a masked headline of the news after prefix "Headline:". You will be provided the correct missing numeral after prefix "Correct Number:". Please emphasise what Topic the headline should focus on, list all Entities and Numbers mentioned in the news, and elaborate on the Reasoning steps for how to calculate the correct number in the headline. \\
    \hline
    \textbf{Output:} \\
    \***Topic the headline should focus on** \\
    The headline should focus on the cost of Dax Shepard and Kristen Bell's wedding. \\
    \\
    \***Entities Mentioned** \\
    1. Dax Shepard \\
    2. Kristen Bell \\
    3. Kim Kardashian \\
    4. Jimmy Kimmel \\
    \\
    \***Numbers Mentioned** \\
    1. \$142 (cost of the wedding) \\
    \\
    \***Reasoning steps** \\
    The news content directly states that the entire wedding, including the fuel to get to the courthouse, cost \$142. \\
    
    Therefore, the numeral in the headline should be 142. \\
    \hline
    \end{tabular}
  % \caption{\label{tbl:demonstration_copy}
  % }
  
\end{table}

\begin{table}[htbp]
    \centering
    \begin{tabular}{p{0.9\textwidth}}
    \hline
    \textbf{Example 2: Add} \\
    \hline
    \textbf{Input:} \\
    \textbf{News:} (Sep 16, 2017 8:26 AM CDT) Police in St. Louis used tear gas to break up protests Friday night after around 1,000 demonstrators surrounded the home of Mayor Lyda Krewson, hurling red paint and breaking a front window. Police say at least nine St. Louis officers and a Highway Patrol trooper were injured in the protests, which followed the acquittal of white former police officer Jason Stockley in the 2011 shooting death of black motorist Anthony Lamar Smith, the St. Louis Post-Dispatch reports. Police say 32 people were arrested when a demonstration that had been largely peaceful turned violent after dark and agitators began to destroy property and assault police officers. Activists have promised more protests against the acquittal of Stockley, who, during a police chase, was caught on dashcam video saying I'm going to kill this motherf*****, don't you know it, the AP reports. Less than a minute later, he shot Smith five times.....  \\
    \textbf{Headline:} \_\_\_\_ Cops Injured in St. Louis Protests. \\
    \textbf{Correct Number:} 10 \\
    \textbf{System Instruction: } You will be given a piece of news after prefix "News: ". You will be given a masked headline of the news after prefix "Headline:". You will be provided the correct missing numeral after prefix "Correct Number:". Please emphasise what Topic the headline should focus on, list all Entities and Numbers mentioned in the news, and elaborate on the Reasoning steps for how to calculate the correct number in the headline. \\
    \hline
    \textbf{Output:} \\
    \***Topic the headline should focus on** \\
    The headline should focus on the number of police officers injured during the St. Louis protests. \\
    \\
    \***Entities Mentioned:**\\
    1. Mayor Lyda Krewson \\
    2. Jason Stockley (white former police officer) \\
    3. Anthony Lamar Smith (black motorist) \\
    4. Reverend Clinton Stancil \\
    5. St. Louis officers \\
    6. Highway Patrol trooper \\
    \\
    \***Numbers Mentioned** \\
    1. 1,000 (demonstrators) \\
    2. 9 (St. Louis officers injured) \\
    3. 1 (Highway Patrol trooper injured) \\
    4. 2011 (year of the shooting) \\
    5. 32 (people arrested) \\
    6. 3 (years since Ferguson unrest) \\
    7. 18 (age of Michael Brown) \\
    8. 5 (times Smith was shot) \\
    9. 24 (age of Anthony Lamar Smith) \\
    \\
    \***Reasoning steps** \\
    To find the total number of cops injured in the St. Louis protests, we need to sum the number of St. Louis officers injured and the Highway Patrol trooper injured. \\
    
    - St. Louis officers injured: 9 \\
    - Highway Patrol trooper injured: 1 \\

    Total injured cops = 9 (St. Louis officers) + 1 (Highway Patrol trooper) = 10 \\
    
    Therefore, the numeral in the headline should be 10. \\
    \hline
    \end{tabular}
  % \caption{\label{tbl:demonstration_add}
  % }
  
\end{table}

\begin{table}[htbp]
    \centering
    \begin{tabular}{p{0.9\textwidth}}
    \hline
    \textbf{Example 3: Subtract} \\
    \hline
    \textbf{Input:} \\
    \textbf{News:} (Jun 26, 2017 9:30 AM CDT) After three decades on the lam, an Arkansas man has been quickly reacquainted with life behind bars. Steven Dishman, 60—who was five months into a seven-year sentence for burglary and property theft when he escaped from the Cummins Unit in Lincoln County, Ark., on May 28, 1985, reports KTHV—was arrested Sunday at a home in Springdale in the northwestern part of the state, report Arkansas Online and KFSM. Arkansas State Police and local cops raided the home after receiving a tip. A rep for the Arkansas Department of Correction tells Fox 16 that Dishman will now be required to serve the remainder of his sentence. Dishman may also face additional charges related to his escape, the details of which were not immediately explained. Had Dishman stayed in jail, he would have been eligible for parole in 1987 and freed by 1991.\\
    \textbf{Headline:} He Spent \_\_\_\_ Years on the Run. Then Police Received a Tip. \\
    \textbf{Correct Number:} 32 \\
    \textbf{System Instruction: } You will be given a piece of news after prefix "News: ". You will be given a masked headline of the news after prefix "Headline:". You will be provided the correct missing numeral after prefix "Correct Number:". Please emphasise what Topic the headline should focus on, list all Entities and Numbers mentioned in the news, and elaborate on the Reasoning steps for how to calculate the correct number in the headline. \\
    \hline
    \textbf{Output:} \\
    \***Topic the headline should focus on** \\
    The headline should focus on the number of years Steven Dishman spent on the run before being recaptured. \\
    \\
    \***Entities Mentioned** \\
    1. Steven Dishman \\
    2. Arkansas State Police \\
    3. Local cops \\
    4. Arkansas Department of Correction representative \\
    \\
    \***Numbers Mentioned** \\
    1. 60 (age of Steven Dishman) \\
    2. 7 (years sentence for burglary and property theft) \\
    3. 5 (months into his sentence when he escaped) \\
    4. May 28, 1985 (date of escape) \\
    5. 1987 (year he would have been eligible for parole) \\
    6. 1991 (year he would have been freed) \\
    7. 2017 (current year) \\
     \\
    \***Reasoning steps** \\
    To find the number of years Steven Dishman spent on the run, we need to calculate the difference between the year he escaped (1985) and the year he was recaptured (2017). \\
    
    - Year of escape: 1985 \\
    - Year of recapture: 2017 \\
    
    Years on the run = 2017 - 1985 = 32 \\
    
    Therefore, the numeral in the headline should be 32. \\
    \hline
    \end{tabular}
  % \caption{\label{tbl:demonstration_subtract}
  % }
  
\end{table}


\begin{table}[htbp]
    \centering
    \begin{tabular}{p{0.9\textwidth}}
    \hline
    \textbf{Example 4: Paraphrase} \\
    \hline
    \textbf{Input:} \\
    \textbf{News:} News: (Oct 7, 2014 12:40 PM CDT) As of Jan. 1, Walmart will no longer offer 30,000 of its employees health insurance. Bloomberg notes that's about 2\% of its workforce. The move comes as a reaction to the company's rising health care costs as far more of its employees and their families enrolled in its health care plans than it had expected following the ObamaCare rollout. The AP reports those costs will surge \$500 million this fiscal year, \$170 million more than had been estimated. Those affected are employees who average fewer than 30 hours of work per week; the Wall Street Journal explains they were grandfathered in when Walmart in 2012 stopped offering insurance to new hires who didn't exceed the 30-hour threshold. A benefits expert says Walmart is actually late to the game in terms of cutting insurance to some part-time workers; Target, the Home Depot, and others have already done so. Meanwhile, Walmart's full time workers will see their premiums rise in 2015. Premiums for the basic plan, which 40\% of its workforce is on, will increase 19\% to \$21.90 per pay period come Jan. 1. \\
    \textbf{Headline:} Headline: \_\_\_\_K Walmart Part-Timers to Lose Health Insurance. \\
    \textbf{Correct Number:} 30. \\
    \textbf{System Instruction: } You will be given a piece of news after prefix "News: ". You will be given a masked headline of the news after prefix "Headline:". You will be provided the correct missing numeral after prefix "Correct Number:". Please emphasise what Topic the headline should focus on, list all Entities and Numbers mentioned in the news, and elaborate on the Reasoning steps for how to calculate the correct number in the headline. \\
    \hline
    \textbf{Output:} \\
    \***Topic the headline should focus on** \\
    The headline should focus on the number of Walmart part-time employees who will lose their health insurance. \\
    \\
    \***Entitiess Mentioned** \\
    1. Walmart employees \\
    2. Walmart's full-time workers \\
    3. Target \\
    4. The Home Depot \\
    5. ObamaCare \\
    \\
    \***Numbers Mentioned** \\
    1. 30,000 (employees losing health insurance) \\
    2. 2\% (of Walmart's workforce) \\
    3. \$500 million (surge in health care costs) \\
    4. \$170 million (more than estimated) \\
    5. 30 hours (work per week threshold) \\
    6. 2012 (year when Walmart stopped offering insurance to new hires under 30 hours) \\
    7. 2015 (year when premiums will rise) \\
    8. 40\% (of workforce on the basic plan) \\
    9. 19\% (increase in premiums) \\
    10. \$21.90 (new premium per pay period) \\
    \\
    \***Reasoning steps** \\
    The news content directly states that Walmart will no longer offer health insurance to 30,000 of its employees. \\

    To convert this number into thousands (K): \\
    - 30,000 employees = 30K employees \\
    
    Therefore, the numeral in the headline should be 30. \\    \hline
    \end{tabular}
  % \caption{\label{tbl:demonstration_paraphrase}
  % }
  
\end{table}

\begin{table}[htbp]
    \centering
    \begin{tabular}{p{0.9\textwidth}}
    \hline
    \textbf{Example 5: Rounding} \\
    \hline
    \textbf{Input:} \\
    \textbf{News:} (Apr 6, 2011 10:10 AM CDT) Well that was fast. Scott Walker did an about-face yesterday and demoted the son of a major campaign donor from his \$81,500-per-year Commerce Department gig—to his former \$64,728 job. Brian Deschane will return to his previous post as a bureau director at the Department of Regulation and Licensing, the Milwaukee Journal-Sentinel reports. The 27-year-old, who has no college degree and little experience (but does have two drunk-driving convictions!), had been in that job just two months when he was promoted to head of Wisconsin's environmental and regulatory affairs. Among those Deschane beat out for the job: a former Cabinet secretary with a doctoral degree and eight years experience overseeing oil contamination cleanup, and a professional engineer who had held the job since 2003 under the last governor. Neither was even interviewed. When Governor Walker learned the details of this agency staffing decision, he directed his administration to move in another direction, a Walker spokesman says (last week, this same rep called Deschane a natural fit for the position). But the demotion didn't satisfy some Democrats. I want to make sure he was properly hired for that job in the first place, said Assembly Minority Leader Peter Barca. Click for more on the brouhaha. \\
    \textbf{Headline:}  Headline: Walker Demotes Donor's Kid— to Puny \$\_\_\_\_K Job. \\
    \textbf{Correct Number:} 65. \\
    \textbf{System Instruction: } You will be given a piece of news after prefix "News: ". You will be given a masked headline of the news after prefix "Headline:". You will be provided the correct missing numeral after prefix "Correct Number:". Please emphasise what Topic the headline should focus on, list all Entities and Numbers mentioned in the news, and elaborate on the Reasoning steps for how to calculate the correct number in the headline. \\
    \hline
    \textbf{Output:} \\
    \***Topic the headline should focus on** \\
    The headline should focus on the new, lower salary of the demoted son of a major campaign donor. \\
    \\
    \***Entities Mentioned** \\
    1. Scott Walker \\
    2. Brian Deschane \\
    3. Assembly Minority Leader Peter Barca \\
    4. Former Cabinet secretary (unnamed) \\
    5. Professional engineer (unnamed) \\
    \\
    \***Numbers Mentioned** \\
    1. \$81,500 (initial salary at Commerce Department) \\
    2. \$64,728 (new salary after demotion) \\
    3. 27 (age of Brian Deschane) \\
    4. 2 (months in the initial job before promotion) \\
    5. 2003 (year since the professional engineer held the job) \\
    6. 8 (years of experience of the former Cabinet secretary) \\
    \\
    \***Reasoning Steps** \\
    The news content states that Brian Deschane was demoted from his \$81,500-per-year job to his former \$64,728 job. \\
    
    To convert this new salary into thousands (K): \\
    $\$64,728 \approx \$65,000$ \\
    
    Therefore, the numeral in the headline should be 65. \\
    \hline
    \end{tabular}
  
\end{table}



\end{document}
