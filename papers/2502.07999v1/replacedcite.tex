\section{Related Work}
\subsection{The Routine Infrastructures of Creative Identity}

    To enact one's creative identity, or any identity for that matter, people rely on routines. Routines are recognizable patterns of action or behavior that are carried out by one or multiple actors within a specific context ____, and are assemblages of sociomaterial configurations people and artifacts (e.g., tools, procedures, technologies) ____. People have agency to adapt their routines or to create new ones as need be ____. Having the agency to adapt is key to the foundational routine of building and rebuilding a coherent and rewarding sense of identity ____. A strong sense of self-identity, which is how a person thinks about themselves socially or physically ____, can give people a deep sense of security in their everyday lives ____. This sense of security emerges when routines are continuous and predictable, a state of ontological security, emerging from the "routine project of the self" ____. The flexibility in the routine project of the self comes from people's abilities to consciously or unconsciously use inferences from the past to anticipate a future ____. Thus, identity is a routine personal and social undertaking—as one must routinely interact with the world and reflect on the impacts of those interactions in their routine project of the self. 
    
    The routines of our everyday lives are enacted on, through, and within larger societal systems. The foundations of these systems are known as infrastructures, and they support the large scale-systems that society relies upon to routinely function ____. Infrastructure can be anything from large-scale highway systems to information and communication technologies (e.g., social media platforms) in how they support routine societal function ____. Infrastructure is defined in use – as they are entwined with human social practice as relational systems that take on meaning or changes in meaning in a continually negotiated way depending on the social practice taking place and the actors involved ____. Importantly, while humans are a part of the construction of the social meaning of infrastructures, they can be infrastructures, functioning as a combination of both known and unknown entities that animate a particular system ____. In this sense infrastructures are sociotechnical, meaning that they both shape and are shaped by social practices built around and with them ____. As with any large system, infrastructures have many interconnected parts that weave themselves seamlessly into the fabric of society and go unnoticed most of the time ____. \par

    While only one small aspect of ourselves, being \textit{a creative person} is also something that is produced through routines. Creative identity is a "representational project engaging the self in dialogue with multiple others about the meaning of creativity as constructed in societal discourses" ____. Creative identity cannot be understood by the actions of individuals alone, but rather relationships and connections between the self and others as they develop a shared notion of creativity ____. To have a creative identity, a person must do creative work, present that creative work to others, and have others also deem that work to be creative. One must be flexible in the routine project of the self ____, and creative identities are no different. Drawing ontological security around creative identity requires flexibility in how the artist constructs knowledge about the world and about themselves as creative people. \par
   
    Creative identities are supported by human infrastructures. There are people that do the work required to animate the physical and digital infrastructures where creative work is shared, as well as the electric and network infrastructures that mediate the routine presentation of one's self as a creative person. Often, artists do not know who these human infrastructures are, but their work is vital to the continued ability of creatives to routinely express themselves as creative people. \par

    Yet, sometimes infrastructures do not work in the way that they are intended to, and sometimes they break down ____, which can become chronic in certain circumstances ____. For artists, infrastructural breakdowns may come from how their needs and values may not match the ways infrastructures are designed ____, as infrastructures are not value-neutral. The human actors that build, maintain, and repair infrastructures embed their values, norms, and biases into them through this routine work ____. As they are embedded into infrastructures, these values can be at the heart of the routine sources of disruptions in artists' everyday lives and routines ____. In some cases, value misalignment between infrastructure designers and users can result in incomplete infrastructure---an infrastructure that does not meet the needs of those who depend upon it to enact their routines, which is a common concern for artists ____. Infrastructural breakdowns may surface the underlying ways these infrastructures support the routine development and expression of creative identity ____ - such as how people routinely draw on online spaces like Instagram to be inspired by the creative work of others or how designers use mood boards to frame or direct their design process ____. When online spaces fail to meet artists’ needs, they can leave people at a loss of where to find the people and creative objects necessary to feel inspired ____ and to negotiate their creative identity with others ____.  \par 
    

\subsection{The Sociomaterial Foundations of Inspiration}
    The word \textit{inspiration} has its root in the Latin \textit{inspirare} - which means to breathe on or into, or to animate the soul ____. During the course of our routine encounters with the world, we will, on occasion, develop "intense object relationships" with \textbf{necessary others} that are the seeds of creative products or ideas that would not otherwise come about ____. Put another way, when we, the subject of these interactions, are inspired, we are putting ourselves in direct relationship with a necessary other, which could be anything. These relationships are contradictory in nature, as inspiration requires both "discipline and spontaneity, mindlessness and mindfulness, receptive waiting and active searching" to come into being ____. Similar to how routines are patterns of action that everyone enacts in a slightly different way than everybody else; inspirational objects and the transformational relations they produce, are not static, but rather emerge in slightly different ways each time the sociomaterial relationships and particular conditions that evoke these creative products or concepts to come into alignment ____.\par

    For the purposes of this paper, we understand sociomateriality as the entanglement of people and objects during the routine actions of individuals or collectives within organizations ____, such as in how open-concept offices often have an organization's management team sitting in corners or around the edges of a collective workspace to enforce sociomaterial control over interactions between their team and others ____, which can only emerge through an entanglement of physical space, humans, and technology objects. The intertwining of humans and objects is situational, meaning that it is produced and reproduced differently depending on the context within which the practices are taking place. Inspiration emerges from the sociomaterial relationships between people and objects that we encounter as we go about our everyday lives. But, importantly, while an object may shape the practice of an individual in a specific organizational context, it is just one thing out of many things that exists within that context. This object shapes the practice of that individual in that particular context, which, in turn, shapes the object itself ____. In a sociomaterial context, a necessary object may be inspirational to one person, but may not be inspirational to another -- yet when that one person acts on that inspiration, they in turn shape the necessary object, which may lead to it becoming inspirational to someone else. \par

    Sharon Hymer ____ points out four key relationships that are particularly generative of inspiration: relationships with the divine, with inanimate objects (e.g., music or nature), the secular (e.g., mentors, teachers), and the self. These relationships emerge differently for different people, and what is inspirational to one person may not be inspirational to another - or it may not be inspirational in the same way. For artists, these encounters with inspirational objects are happening as the result of creative work taking place increasingly online. Prior work has shown that encountering the creative work of others has led to what is known as "divergent thinking" -- the free-forming of new ideas that branch off from the original idea or concept ____ and that attempting to copy or recreate the creative work of others allows for transformation of the original idea into something new ____. Graphic designers, for example, seek out visual information--such as the creative work of others--to inform their personal development as well as to capture certain aesthetics or ideas as a part of their inspiration and ideation process ____; or will put together moodboards (usually of other people's art or photographs) that frame, direct or otherwise inspire their design process ____. This prior work demonstrates that artists are inspired by their encounters with, deliberate search for, and transformation into inspirational tools like moodboards; the creative work of others.\par

    At the heart of many of these encounters is technology, which facilitates the creative process and management of creative ideas on both an individual and collaborative level ____. Online, these spaces and encounters are more plentiful than ever before and are increasingly facilitated and impeded by platform infrastructures ____, meaning that the artist who creates and shares their work online is constantly bombarded by the potential for inspiration that stems from the creative work of others. One may be deliberate in seeking out visual information to help with the creative process ____, but one could also stumble on an inspirational object without actually meaning to look for one. How these spaces are designed and the infrastructural elements that facilitate these contradictory inspirational encounters can place the creative self - a person's \textit{creative identity} - in flux. The constant search for inspiration becomes a matter of routine engagement with one's creative identity as it relates to the creative expression and identities of others, while it also is, increasingly reliant and entangled with technical tools and their infrastructures. 

    While creative practice is a matter of routine for many artists, it is also a routine that is pulled in contradictory directions when it comes to how inspiration emerges. Inspiration emerges along three key contradictions: one must be disciplined, as well as able to be spontaneous to become inspired; one must always be mindless, but also mindful of when inspiration may strike; and one must be actively searching for inspiration, while also receptively waiting for a chance encounter with an inspirational object ____. We adopt Hymer's ____ contradictions as a lens by which to understand the role inspiration plays in the enactment and realization of creative identities. Below, we detail the contradictions, ground them in the existing literature, and provide definitions. 

\begin{itemize}

\item \textbf{Discipline and Spontaneity} are centered around creative practice itself and the inspiration that emerges from the practice of doing art. According to Hymer, "what appears to be a serendipitous thought or discovery [...] derives from both flashes of quick impressions and from the slower, more painstaking analytic work that precedes and follows such flashes" \cite[~p.30]{hymer1990inspiration}. For our purposes, discipline surrounds the routine practice of creative work that comes both before and after the spontaneous, serendipitous, moment of inspiration. Self-discipline is the routine enactment of creative practice through the discipline to do creative work even when struggling with creative blocks ____, by drawing on peer support ____ and engaging in critique and feedback sessions ____ that negotiate meaning of creativity and artist identity. Spontaneity, conversely, is the ability to capitalize on the convergence of inspirational objects and necessary others that produce new combinations of possibilities that inspire ____ and then being open and able to do that creative work ____. 

\item \textbf{Mindfulness and Mindlessness}, according to Hymer are are framed around artistic awareness and engagement with the world ____. These contradictions are embodied in the creative objects produced and the collective negotiation of creativity and creative identity that online platforms facilitate ____, as well as the tools that we use to create potential inspiration ____. Mindfulness speaks to the awareness of the inspirational elements of one’s environment and the openness to documenting inspirational objects when they are encountered for further reference. Mindlessness, in contrast, speaks to the routine, yet mindless, use of online platforms to engage with the creative work of others and to become inspired to do creative work. We note that a key element of these routine encounters is engagement with various recommender systems, which mediate the artists’ encounters with the creative work of others ____. 

\item \textbf{Active Searching and Receptive Waiting} are focused on a person's intentionality in searching for inspiration. Hymer describes this contradiction as being both aware of how one has routinely gone about inspirational practices in the past (e.g., through deliberate routine), but also the "spontaneity and receptivity to surprise elements which enter into a transformational experience" \cite[~p.31]{hymer1990inspiration}. While actively searching for inspiration is a routine practice for many creative people ____, receptive waiting is about browsing - or, “a search, hopefully serendipitous [...] which might contribute the fact or idea needed in some intellectual effort” \cite[~p.4]{morse1970browsing}. Here, an element of control that must be exercised to create the possibilities of chance encounters that could be considered “serendipitous” - thus, receptive waiting  ____.

\end{itemize}

    Using this framework, we describe how the online platforms where many artists are encountering inspirational objects are supporting - and not supporting - the articulation of their artistic identities. In the next section, we discuss our method. \par