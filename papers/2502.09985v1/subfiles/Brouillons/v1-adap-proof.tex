At the beginning of Section~\ref{sec:adap-oracle}, we mention the fact that over $\calC^{\text{adap}}_{\calF,\calS}$ the oracle should solve:
\begin{align*}
    \min_{f\in\calF,s\in\calS} &\; \EE[s(X)]  \\
     \quad s.t. \quad &\;  \IP(|Y-f(X)|\leq s(X)) \geq 1-\alpha
     \;. \nonumber
\end{align*}

In practice, we propose to use~\methodAD, however in order to obtain theoretical results similar to that of Theorem~\ref{thme:main-constant}, another possibility would be to solve, during the learning step, the empirical version of the previous oracle problem, which is complicated to solve in practice:
\begin{align*}
    \min_{f\in\calF,s\in\calS} &\; \frac{1}{n_\ell}\sum_{i\in\calD^{lrn}}s(X_i)  \\
     \quad s.t. \quad &\;  \frac{1}{n_\ell}\sum_{i\in\calD^{lrn}}\1\{|Y_i-f(X_i)|\leq s(X_i)\}\geq 1-\alpha
     \;. \nonumber
\end{align*}

Denote by $\hf$ and $\hs$ the solutions of the empirical problem, and $f^*$ and $s^*$ the solutions of the oracle one. In order to derive a result similar to that of Theorem~\ref{thme:main-constant}, we consider the following assumption.

\begin{assumption} \label{ass:complexity-ad} There exists two quantities $\phiFS<+\infty$ and $\psiS<+\infty$ such that:
    \begin{equation*}
        \IP\left(\sup_{f\in\calF,s\in\calS}\Big|\IP\left(|Y-f(X)|\leq s(X)\right) - \frac{1}{n}\sum_{i=1}^n\1\{|Y_i-f(X_i)|\leq s(X_i)\}\Big|\leq \phiFS\right)\geq 1-\delta
    \end{equation*}
    and
    \begin{equation*}
        \IP\left(\sup_{s\in\calS}\Big|\EE[s(X)] - \frac{1}{n}\sum_{i=1}^n s(X_i)\Big|\leq \psiS\right)\geq 1-\delta
    \end{equation*}
\end{assumption}

In words, this assumption generalizes Assumption~\ref{ass:complexity} to the adaptive size setting, at least for the first equation. Since in this setting we also estimate the expectation of the size, we need the second equation to make sure that its worst-case estimation error is bounded w.h.p. Closed-form expressions for $\phiFS$ and $\psiS$ can be obtained similarly as in Appendix~\ref{sec:closed-form-phi}.

Under Assumption~\ref{ass:complexity-ad}, we can derive the following lemma, which is an extension of the result obtained at the end of Step 1 in the proof of Theorem~\ref{thme:main-constant}.

\begin{lemma}
    \label{lem:scott-adap}
    Under Assumption~\ref{ass:complexity-ad}, we have with probability greater than $1-2\delta$:    
    \begin{equation}
        \label{eq:lower-scott-app-AD}
        \IP(Y\in C_{\hf,\hs}^{1-\alpha}(X)|\calD^{lrn})\geq 1-\alpha - \phiFSl
    \end{equation}
    and
    \begin{equation}
        \label{eq:lower-scott-app2-AD}
        \EE\left[\lambda\left(C_{\hat{f},\hs}^{1-\alpha}(X)\right)\Big|\calD^{lrn}\right] \leq \EE\left[\lambda\left(C_{f_{1-\alpha+\phi}^*,s_{1-\alpha+\phi}^*}^{1-\alpha + \phi}(X)\right)\right] + 4\psiSl\;,
    \end{equation}
    where $\phi \equiv \phiFSl$ and $C_{f_{1-\alpha+\phi}^*,s_{1-\alpha+\phi}^*}^{1-\alpha + \phi}(X)$ denotes the optimal oracle interval with increased coverage $1-\alpha + \phiFSl$. In other word, $f_{1-\alpha+\phi}^*$ and $s_{1-\alpha+\phi}^*$ are the solutions of $\min_{f\in\calF, S\in \calS}~ \EE[s(X)]$ s.t. $\IP(|Y-f(X)|\leq t)\geq 1-\alpha + \phiFl$.
\end{lemma}

\begin{proof}
    The proof closely follows the one of Step 1 in Appendix~\ref{app:proof-main}.
    Let:
    \begin{itemize}
        \item $\Theta_\IP=\Big\{\IP(Y\in C_{\hf,\hs}^{1-\alpha}(X)|\calD^{lrn})< 1-\alpha - \phiFSl\Big\}$
        \item $\Theta_\lambda=\Big\{\EE\left[\lambda\left(C_{\hat{f},\hs}^{1-\alpha}(X)\right)\Big|\calD^{lrn}\right] > \EE\left[\lambda\left(C_{f_{1-\alpha+\phi}^*,s_{1-\alpha+\phi}^*}^{1-\alpha + \phi}(X)\right)\right] + 4\psiSl\Big\}$
        \item $\Theta_\phi=\Big\{\sup_{f\in\calF,s\in\calS}\Big|\IP\left(|Y-f(X)|\leq s(X)\right) - \frac{1}{n_\ell}\sum_{i\in\calD^{lrn}}\1\{|Y_i-f(X_i)|\leq s(X_i)\}\Big|> \phiFSl\Big\}$
        \item $\Theta_\psi = \Big\{\sup_{s\in\calS}\Big|\EE[s(X)] - \frac{1}{n_\ell}\sum_{i\in\calD^{lrn}} s(X_i)\Big|> \psiSl\Big\}$
    \end{itemize}

The objective is to show that $(\Theta_\IP\cup\Theta_\lambda) \subset (\Theta_\phi \cup \Theta_\psi)$. Indeed, using the union bound and Assumption~\ref{ass:complexity-ad}, this would imply that $\IP(\Theta_\IP\cup\Theta_\lambda)\leq \IP(\Theta_\phi\cup \Theta_\psi)\leq 2\delta$, concluding the proof.

\underline{$\Theta_\IP\subset  (\Theta_\phi \cup \Theta_\psi)$:} Proved by showing that $\Theta_\IP\subset \Theta_\phi$ using the same arguments as in the proof of the main result.

\underline{$\Theta_\lambda\subset (\Theta_\phi \cup \Theta_\psi)$:}

Let the event $\Omega=\Big\{\frac{1}{n_\ell}\sum_{i\in\calD^{lrn}}\1\{|Y_i-f_{1-\alpha+\phi}^*(X_i)|\leq s_{1-\alpha+\phi}^*(X_i)\}<1-\alpha\Big\}$. We first show that $\Theta_\lambda\subset (\Omega \cup \Theta_\psi)$, by proving that $(\Omega^c \cap \Theta^c_\psi)\subset \Theta_\lambda^c$. Indeed, under $(\Omega^c \cap \Theta^c_\psi)$ we have:

\begin{align*}
    &\frac{1}{n_\ell}\sum_{i\in\calD^{lrn}}\1\{|Y_i-f_{1-\alpha+\phi}^*(X_i)|\leq s_{1-\alpha+\phi}^*(X_i)\} \geq 1-\alpha \\
    \Longrightarrow\hspace{0.1cm} & \frac{1}{n_\ell}\sum_{i\in\calD^{lrn}}\hs(X_i) \leq \frac{1}{n_\ell}\sum_{i\in\calD^{lrn}}s_{1-\alpha+\phi}^*(X_i) \\
    \Longrightarrow\hspace{0.1cm} & \EE[\hs(X)|\calD^{lrn}]+\frac{1}{n_\ell}  \sum_{i\in\calD^{lrn}}\hs(X_i) - \EE[\hs(X)|\calD^{lrn}] \leq \EE[s_{1-\alpha+\phi}^*(X)] + \frac{1}{n_\ell}\sum_{i\in\calD^{lrn}}s_{1-\alpha+\phi}^*(X_i) -  \EE[s_{1-\alpha+\phi}^*(X)]\\
    \Longrightarrow\hspace{0.1cm} & \EE[\hs(X)|\calD^{lrn}] \leq \EE[s_{1-\alpha+\phi}^*(X)] + 2\sup_{s\in\calS}\Big|\EE[s(X)] - \frac{1}{n_\ell}\sum_{i\in\calD^{lrn}} s(X_i)\Big| \\
    \Longrightarrow\hspace{0.1cm} & \EE[\hs(X)|\calD^{lrn}] \leq \EE[s_{1-\alpha+\phi}^*(X)] + 2\psiSl\\
    \Longrightarrow\hspace{0.1cm} & 2\EE[\hs(X)|\calD^{lrn}] \leq 2\EE[s_{1-\alpha+\phi}^*(X)] + 4\psiSl\Longrightarrow \Theta_\lambda^c
\end{align*}

It remains to prove that $\Omega \subset \Theta_\phi$. Under $\Omega$ and using the fact that $\IP\Big(|Y-f^*_{1-\alpha+\phi}(X)|\leq s^*_{1-\alpha+\phi}(X)\Big)\geq 1- \alpha + \phiFSl$: 

    \begin{align*}
        &\frac{1}{n_\ell}\sum_{i=1}^{n_\ell}\1\{|Y_i-f^*_{1-\alpha+\phi}(X_i)|\leq s^*_{1-\alpha+\phi}(X_i)\} - \IP\Big(|Y-f^*_{1-\alpha+\phi}(X)|\leq s^*_{1-\alpha+\phi}(X)\Big) < - \phiFSl \\
        \Longrightarrow \hspace{0.1cm}& \Big|\frac{1}{n_\ell}\sum_{i=1}^{n_\ell}\1\{|Y_i-f^*_{1-\alpha+\phi}(X_i)|\leq s^*_{1-\alpha+\phi}(X_i)\} - \IP\Big(|Y-f^*_{1-\alpha+\phi}(X)|\leq s^*_{1-\alpha+\phi}(X)\Big)\Big| > \phiFSl \\
        \Longrightarrow \hspace{0.1cm}& \Theta_\phi
    \end{align*}
    Hence $\Omega \subset \Theta_\phi$, i.e. $\Theta_\lambda\subset (\Omega \cup \Theta_\psi)\subset (\Theta_\phi \cup \Theta_\psi)$, which concludes the proof.
\end{proof}

Like after step 1 of~\method, with Lemma~\ref{lem:scott-adap} we have probabilistic guarantees on the coverage and on the expected size of the returned set. 
