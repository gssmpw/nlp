\documentclass{article}
\usepackage[utf8]{inputenc} % allow utf-8 input
\usepackage[T1]{fontenc}    % use 8-bit T1 fonts
\usepackage{hyperref}       % hyperlinks
\usepackage{url}            % simple URL typesetting
\usepackage{booktabs}       % professional-quality tables
\usepackage{amsfonts}       % blackboard math symbols
\usepackage{nicefrac}       % compact symbols for 1/2, etc.
\usepackage{microtype}      % microtypography
%\usepackage{xcolor}         % colors
\usepackage{amsmath}
\usepackage[normalem]{ulem}
\usepackage[dvipsnames]{xcolor}
\usepackage{multirow}
\usepackage{array}
\usepackage{lmodern}

\usepackage{algorithm}
\usepackage{algpseudocode}

\usepackage[shortlabels]{enumitem}
\usepackage[font=small,labelfont=bf]{caption}
\usepackage{subcaption}
\usepackage[percent]{overpic}
\usepackage{natbib}
\usepackage[margin=1.2in]{geometry}
\usepackage{amssymb, amsmath}
\usepackage{amsthm}
\usepackage{dsfont}
\usepackage{stmaryrd}
\usepackage{caption}
%\usepackage[colorlinks]{hyperref}

\usepackage{algorithm}
\usepackage{algpseudocode}

%%%% Needed to cure problem with todonotes
\setlength{\marginparwidth}{23mm}
%%% Put "disable" option to suppress all todonotes
\usepackage[linecolor=blue!60!,backgroundcolor=blue!10!,textwidth=2.5cm,textsize=scriptsize]{todonotes}


\usepackage{color} 

\definecolor{airforceblue}{rgb}{0.36, 0.54, 0.66}
\newcommand\boldblue[1]{\textcolor{airforceblue}{\textbf{#1}}}


\newcommand\bat[1]{\textcolor{red}{#1}}
\newcommand\pie[1]{\textcolor{blue}{{\bf Pierre:} #1}}
\newcommand\marc[1]{\textcolor{green!40!black}{{\bf Marc:} #1}}





\title{
\vspace{-3.5em}
\noindent\rule[0.5ex]{\linewidth}{1pt} On Volume Minimization in Conformal Regression  \noindent\rule[0.5ex]{\linewidth}
{1pt}\vspace{-1em}}
\author{Batiste Le Bars\textsuperscript{\textdagger} \qquad Pierre Humbert*  \\[0.5cm]
\textsuperscript{\textdagger} Univ. Lille, Inria, CNRS, Centrale Lille, UMR 9189, CRIStAL, F-59000 Lille\\[0.1cm]
* Sorbonne Université et Université Paris Cité, CNRS, Laboratoire de Probabilités, \\Statistique et
Modélisation, F-75005 Paris, France \\[0.1cm]}
\date{}


%
\setlength\unitlength{1mm}
\newcommand{\twodots}{\mathinner {\ldotp \ldotp}}
% bb font symbols
\newcommand{\Rho}{\mathrm{P}}
\newcommand{\Tau}{\mathrm{T}}

\newfont{\bbb}{msbm10 scaled 700}
\newcommand{\CCC}{\mbox{\bbb C}}

\newfont{\bb}{msbm10 scaled 1100}
\newcommand{\CC}{\mbox{\bb C}}
\newcommand{\PP}{\mbox{\bb P}}
\newcommand{\RR}{\mbox{\bb R}}
\newcommand{\QQ}{\mbox{\bb Q}}
\newcommand{\ZZ}{\mbox{\bb Z}}
\newcommand{\FF}{\mbox{\bb F}}
\newcommand{\GG}{\mbox{\bb G}}
\newcommand{\EE}{\mbox{\bb E}}
\newcommand{\NN}{\mbox{\bb N}}
\newcommand{\KK}{\mbox{\bb K}}
\newcommand{\HH}{\mbox{\bb H}}
\newcommand{\SSS}{\mbox{\bb S}}
\newcommand{\UU}{\mbox{\bb U}}
\newcommand{\VV}{\mbox{\bb V}}


\newcommand{\yy}{\mathbbm{y}}
\newcommand{\xx}{\mathbbm{x}}
\newcommand{\zz}{\mathbbm{z}}
\newcommand{\sss}{\mathbbm{s}}
\newcommand{\rr}{\mathbbm{r}}
\newcommand{\pp}{\mathbbm{p}}
\newcommand{\qq}{\mathbbm{q}}
\newcommand{\ww}{\mathbbm{w}}
\newcommand{\hh}{\mathbbm{h}}
\newcommand{\vvv}{\mathbbm{v}}

% Vectors

\newcommand{\av}{{\bf a}}
\newcommand{\bv}{{\bf b}}
\newcommand{\cv}{{\bf c}}
\newcommand{\dv}{{\bf d}}
\newcommand{\ev}{{\bf e}}
\newcommand{\fv}{{\bf f}}
\newcommand{\gv}{{\bf g}}
\newcommand{\hv}{{\bf h}}
\newcommand{\iv}{{\bf i}}
\newcommand{\jv}{{\bf j}}
\newcommand{\kv}{{\bf k}}
\newcommand{\lv}{{\bf l}}
\newcommand{\mv}{{\bf m}}
\newcommand{\nv}{{\bf n}}
\newcommand{\ov}{{\bf o}}
\newcommand{\pv}{{\bf p}}
\newcommand{\qv}{{\bf q}}
\newcommand{\rv}{{\bf r}}
\newcommand{\sv}{{\bf s}}
\newcommand{\tv}{{\bf t}}
\newcommand{\uv}{{\bf u}}
\newcommand{\wv}{{\bf w}}
\newcommand{\vv}{{\bf v}}
\newcommand{\xv}{{\bf x}}
\newcommand{\yv}{{\bf y}}
\newcommand{\zv}{{\bf z}}
\newcommand{\zerov}{{\bf 0}}
\newcommand{\onev}{{\bf 1}}

% Matrices

\newcommand{\Am}{{\bf A}}
\newcommand{\Bm}{{\bf B}}
\newcommand{\Cm}{{\bf C}}
\newcommand{\Dm}{{\bf D}}
\newcommand{\Em}{{\bf E}}
\newcommand{\Fm}{{\bf F}}
\newcommand{\Gm}{{\bf G}}
\newcommand{\Hm}{{\bf H}}
\newcommand{\Id}{{\bf I}}
\newcommand{\Jm}{{\bf J}}
\newcommand{\Km}{{\bf K}}
\newcommand{\Lm}{{\bf L}}
\newcommand{\Mm}{{\bf M}}
\newcommand{\Nm}{{\bf N}}
\newcommand{\Om}{{\bf O}}
\newcommand{\Pm}{{\bf P}}
\newcommand{\Qm}{{\bf Q}}
\newcommand{\Rm}{{\bf R}}
\newcommand{\Sm}{{\bf S}}
\newcommand{\Tm}{{\bf T}}
\newcommand{\Um}{{\bf U}}
\newcommand{\Wm}{{\bf W}}
\newcommand{\Vm}{{\bf V}}
\newcommand{\Xm}{{\bf X}}
\newcommand{\Ym}{{\bf Y}}
\newcommand{\Zm}{{\bf Z}}

% Calligraphic

\newcommand{\Ac}{{\cal A}}
\newcommand{\Bc}{{\cal B}}
\newcommand{\Cc}{{\cal C}}
\newcommand{\Dc}{{\cal D}}
\newcommand{\Ec}{{\cal E}}
\newcommand{\Fc}{{\cal F}}
\newcommand{\Gc}{{\cal G}}
\newcommand{\Hc}{{\cal H}}
\newcommand{\Ic}{{\cal I}}
\newcommand{\Jc}{{\cal J}}
\newcommand{\Kc}{{\cal K}}
\newcommand{\Lc}{{\cal L}}
\newcommand{\Mc}{{\cal M}}
\newcommand{\Nc}{{\cal N}}
\newcommand{\nc}{{\cal n}}
\newcommand{\Oc}{{\cal O}}
\newcommand{\Pc}{{\cal P}}
\newcommand{\Qc}{{\cal Q}}
\newcommand{\Rc}{{\cal R}}
\newcommand{\Sc}{{\cal S}}
\newcommand{\Tc}{{\cal T}}
\newcommand{\Uc}{{\cal U}}
\newcommand{\Wc}{{\cal W}}
\newcommand{\Vc}{{\cal V}}
\newcommand{\Xc}{{\cal X}}
\newcommand{\Yc}{{\cal Y}}
\newcommand{\Zc}{{\cal Z}}

% Bold greek letters

\newcommand{\alphav}{\hbox{\boldmath$\alpha$}}
\newcommand{\betav}{\hbox{\boldmath$\beta$}}
\newcommand{\gammav}{\hbox{\boldmath$\gamma$}}
\newcommand{\deltav}{\hbox{\boldmath$\delta$}}
\newcommand{\etav}{\hbox{\boldmath$\eta$}}
\newcommand{\lambdav}{\hbox{\boldmath$\lambda$}}
\newcommand{\epsilonv}{\hbox{\boldmath$\epsilon$}}
\newcommand{\nuv}{\hbox{\boldmath$\nu$}}
\newcommand{\muv}{\hbox{\boldmath$\mu$}}
\newcommand{\zetav}{\hbox{\boldmath$\zeta$}}
\newcommand{\phiv}{\hbox{\boldmath$\phi$}}
\newcommand{\psiv}{\hbox{\boldmath$\psi$}}
\newcommand{\thetav}{\hbox{\boldmath$\theta$}}
\newcommand{\tauv}{\hbox{\boldmath$\tau$}}
\newcommand{\omegav}{\hbox{\boldmath$\omega$}}
\newcommand{\xiv}{\hbox{\boldmath$\xi$}}
\newcommand{\sigmav}{\hbox{\boldmath$\sigma$}}
\newcommand{\piv}{\hbox{\boldmath$\pi$}}
\newcommand{\rhov}{\hbox{\boldmath$\rho$}}
\newcommand{\upsilonv}{\hbox{\boldmath$\upsilon$}}

\newcommand{\Gammam}{\hbox{\boldmath$\Gamma$}}
\newcommand{\Lambdam}{\hbox{\boldmath$\Lambda$}}
\newcommand{\Deltam}{\hbox{\boldmath$\Delta$}}
\newcommand{\Sigmam}{\hbox{\boldmath$\Sigma$}}
\newcommand{\Phim}{\hbox{\boldmath$\Phi$}}
\newcommand{\Pim}{\hbox{\boldmath$\Pi$}}
\newcommand{\Psim}{\hbox{\boldmath$\Psi$}}
\newcommand{\Thetam}{\hbox{\boldmath$\Theta$}}
\newcommand{\Omegam}{\hbox{\boldmath$\Omega$}}
\newcommand{\Xim}{\hbox{\boldmath$\Xi$}}


% Sans Serif small case

\newcommand{\Gsf}{{\sf G}}

\newcommand{\asf}{{\sf a}}
\newcommand{\bsf}{{\sf b}}
\newcommand{\csf}{{\sf c}}
\newcommand{\dsf}{{\sf d}}
\newcommand{\esf}{{\sf e}}
\newcommand{\fsf}{{\sf f}}
\newcommand{\gsf}{{\sf g}}
\newcommand{\hsf}{{\sf h}}
\newcommand{\isf}{{\sf i}}
\newcommand{\jsf}{{\sf j}}
\newcommand{\ksf}{{\sf k}}
\newcommand{\lsf}{{\sf l}}
\newcommand{\msf}{{\sf m}}
\newcommand{\nsf}{{\sf n}}
\newcommand{\osf}{{\sf o}}
\newcommand{\psf}{{\sf p}}
\newcommand{\qsf}{{\sf q}}
\newcommand{\rsf}{{\sf r}}
\newcommand{\ssf}{{\sf s}}
\newcommand{\tsf}{{\sf t}}
\newcommand{\usf}{{\sf u}}
\newcommand{\wsf}{{\sf w}}
\newcommand{\vsf}{{\sf v}}
\newcommand{\xsf}{{\sf x}}
\newcommand{\ysf}{{\sf y}}
\newcommand{\zsf}{{\sf z}}


% mixed symbols

\newcommand{\sinc}{{\hbox{sinc}}}
\newcommand{\diag}{{\hbox{diag}}}
\renewcommand{\det}{{\hbox{det}}}
\newcommand{\trace}{{\hbox{tr}}}
\newcommand{\sign}{{\hbox{sign}}}
\renewcommand{\arg}{{\hbox{arg}}}
\newcommand{\var}{{\hbox{var}}}
\newcommand{\cov}{{\hbox{cov}}}
\newcommand{\Ei}{{\rm E}_{\rm i}}
\renewcommand{\Re}{{\rm Re}}
\renewcommand{\Im}{{\rm Im}}
\newcommand{\eqdef}{\stackrel{\Delta}{=}}
\newcommand{\defines}{{\,\,\stackrel{\scriptscriptstyle \bigtriangleup}{=}\,\,}}
\newcommand{\<}{\left\langle}
\renewcommand{\>}{\right\rangle}
\newcommand{\herm}{{\sf H}}
\newcommand{\trasp}{{\sf T}}
\newcommand{\transp}{{\sf T}}
\renewcommand{\vec}{{\rm vec}}
\newcommand{\Psf}{{\sf P}}
\newcommand{\SINR}{{\sf SINR}}
\newcommand{\SNR}{{\sf SNR}}
\newcommand{\MMSE}{{\sf MMSE}}
\newcommand{\REF}{{\RED [REF]}}

% Markov chain
\usepackage{stmaryrd} % for \mkv 
\newcommand{\mkv}{-\!\!\!\!\minuso\!\!\!\!-}

% Colors

\newcommand{\RED}{\color[rgb]{1.00,0.10,0.10}}
\newcommand{\BLUE}{\color[rgb]{0,0,0.90}}
\newcommand{\GREEN}{\color[rgb]{0,0.80,0.20}}

%%%%%%%%%%%%%%%%%%%%%%%%%%%%%%%%%%%%%%%%%%
\usepackage{hyperref}
\hypersetup{
    bookmarks=true,         % show bookmarks bar?
    unicode=false,          % non-Latin characters in AcrobatÕs bookmarks
    pdftoolbar=true,        % show AcrobatÕs toolbar?
    pdfmenubar=true,        % show AcrobatÕs menu?
    pdffitwindow=false,     % window fit to page when opened
    pdfstartview={FitH},    % fits the width of the page to the window
%    pdftitle={My title},    % title
%    pdfauthor={Author},     % author
%    pdfsubject={Subject},   % subject of the document
%    pdfcreator={Creator},   % creator of the document
%    pdfproducer={Producer}, % producer of the document
%    pdfkeywords={keyword1} {key2} {key3}, % list of keywords
    pdfnewwindow=true,      % links in new window
    colorlinks=true,       % false: boxed links; true: colored links
    linkcolor=red,          % color of internal links (change box color with linkbordercolor)
    citecolor=green,        % color of links to bibliography
    filecolor=blue,      % color of file links
    urlcolor=blue           % color of external links
}
%%%%%%%%%%%%%%%%%%%%%%%%%%%%%%%%%%%%%%%%%%%




\begin{document}

\maketitle

\begin{abstract}
    We study the question of volume optimality in split conformal regression, a topic still poorly understood in comparison to coverage control. Using the fact that the calibration step can be seen as an empirical volume minimization problem, we first derive a finite-sample upper-bound on the excess volume loss of the interval returned by the classical split method. This important quantity measures the difference in length between the interval obtained with the split method and the shortest oracle prediction interval.
    Then, we introduce~\method, a methodology that modifies the learning step so that the base prediction function is selected in order to minimize the length of the returned intervals. 
    In particular, our theoretical analysis of the excess volume loss of the prediction sets produced by~\method~reveals the links between the learning and calibration steps, and notably the impact of the choice of the function class of the base predictor. We also introduce \methodAD, an extension of the previous method, which produces intervals whose size adapts to the value of the covariate. Finally, we evaluate the empirical performance and the robustness of our methodologies.
\end{abstract}



\section{Introduction}
\label{sec:introduction}
The business processes of organizations are experiencing ever-increasing complexity due to the large amount of data, high number of users, and high-tech devices involved \cite{martin2021pmopportunitieschallenges, beerepoot2023biggestbpmproblems}. This complexity may cause business processes to deviate from normal control flow due to unforeseen and disruptive anomalies \cite{adams2023proceddsriftdetection}. These control-flow anomalies manifest as unknown, skipped, and wrongly-ordered activities in the traces of event logs monitored from the execution of business processes \cite{ko2023adsystematicreview}. For the sake of clarity, let us consider an illustrative example of such anomalies. Figure \ref{FP_ANOMALIES} shows a so-called event log footprint, which captures the control flow relations of four activities of a hypothetical event log. In particular, this footprint captures the control-flow relations between activities \texttt{a}, \texttt{b}, \texttt{c} and \texttt{d}. These are the causal ($\rightarrow$) relation, concurrent ($\parallel$) relation, and other ($\#$) relations such as exclusivity or non-local dependency \cite{aalst2022pmhandbook}. In addition, on the right are six traces, of which five exhibit skipped, wrongly-ordered and unknown control-flow anomalies. For example, $\langle$\texttt{a b d}$\rangle$ has a skipped activity, which is \texttt{c}. Because of this skipped activity, the control-flow relation \texttt{b}$\,\#\,$\texttt{d} is violated, since \texttt{d} directly follows \texttt{b} in the anomalous trace.
\begin{figure}[!t]
\centering
\includegraphics[width=0.9\columnwidth]{images/FP_ANOMALIES.png}
\caption{An example event log footprint with six traces, of which five exhibit control-flow anomalies.}
\label{FP_ANOMALIES}
\end{figure}

\subsection{Control-flow anomaly detection}
Control-flow anomaly detection techniques aim to characterize the normal control flow from event logs and verify whether these deviations occur in new event logs \cite{ko2023adsystematicreview}. To develop control-flow anomaly detection techniques, \revision{process mining} has seen widespread adoption owing to process discovery and \revision{conformance checking}. On the one hand, process discovery is a set of algorithms that encode control-flow relations as a set of model elements and constraints according to a given modeling formalism \cite{aalst2022pmhandbook}; hereafter, we refer to the Petri net, a widespread modeling formalism. On the other hand, \revision{conformance checking} is an explainable set of algorithms that allows linking any deviations with the reference Petri net and providing the fitness measure, namely a measure of how much the Petri net fits the new event log \cite{aalst2022pmhandbook}. Many control-flow anomaly detection techniques based on \revision{conformance checking} (hereafter, \revision{conformance checking}-based techniques) use the fitness measure to determine whether an event log is anomalous \cite{bezerra2009pmad, bezerra2013adlogspais, myers2018icsadpm, pecchia2020applicationfailuresanalysispm}. 

The scientific literature also includes many \revision{conformance checking}-independent techniques for control-flow anomaly detection that combine specific types of trace encodings with machine/deep learning \cite{ko2023adsystematicreview, tavares2023pmtraceencoding}. Whereas these techniques are very effective, their explainability is challenging due to both the type of trace encoding employed and the machine/deep learning model used \cite{rawal2022trustworthyaiadvances,li2023explainablead}. Hence, in the following, we focus on the shortcomings of \revision{conformance checking}-based techniques to investigate whether it is possible to support the development of competitive control-flow anomaly detection techniques while maintaining the explainable nature of \revision{conformance checking}.
\begin{figure}[!t]
\centering
\includegraphics[width=\columnwidth]{images/HIGH_LEVEL_VIEW.png}
\caption{A high-level view of the proposed framework for combining \revision{process mining}-based feature extraction with dimensionality reduction for control-flow anomaly detection.}
\label{HIGH_LEVEL_VIEW}
\end{figure}

\subsection{Shortcomings of \revision{conformance checking}-based techniques}
Unfortunately, the detection effectiveness of \revision{conformance checking}-based techniques is affected by noisy data and low-quality Petri nets, which may be due to human errors in the modeling process or representational bias of process discovery algorithms \cite{bezerra2013adlogspais, pecchia2020applicationfailuresanalysispm, aalst2016pm}. Specifically, on the one hand, noisy data may introduce infrequent and deceptive control-flow relations that may result in inconsistent fitness measures, whereas, on the other hand, checking event logs against a low-quality Petri net could lead to an unreliable distribution of fitness measures. Nonetheless, such Petri nets can still be used as references to obtain insightful information for \revision{process mining}-based feature extraction, supporting the development of competitive and explainable \revision{conformance checking}-based techniques for control-flow anomaly detection despite the problems above. For example, a few works outline that token-based \revision{conformance checking} can be used for \revision{process mining}-based feature extraction to build tabular data and develop effective \revision{conformance checking}-based techniques for control-flow anomaly detection \cite{singh2022lapmsh, debenedictis2023dtadiiot}. However, to the best of our knowledge, the scientific literature lacks a structured proposal for \revision{process mining}-based feature extraction using the state-of-the-art \revision{conformance checking} variant, namely alignment-based \revision{conformance checking}.

\subsection{Contributions}
We propose a novel \revision{process mining}-based feature extraction approach with alignment-based \revision{conformance checking}. This variant aligns the deviating control flow with a reference Petri net; the resulting alignment can be inspected to extract additional statistics such as the number of times a given activity caused mismatches \cite{aalst2022pmhandbook}. We integrate this approach into a flexible and explainable framework for developing techniques for control-flow anomaly detection. The framework combines \revision{process mining}-based feature extraction and dimensionality reduction to handle high-dimensional feature sets, achieve detection effectiveness, and support explainability. Notably, in addition to our proposed \revision{process mining}-based feature extraction approach, the framework allows employing other approaches, enabling a fair comparison of multiple \revision{conformance checking}-based and \revision{conformance checking}-independent techniques for control-flow anomaly detection. Figure \ref{HIGH_LEVEL_VIEW} shows a high-level view of the framework. Business processes are monitored, and event logs obtained from the database of information systems. Subsequently, \revision{process mining}-based feature extraction is applied to these event logs and tabular data input to dimensionality reduction to identify control-flow anomalies. We apply several \revision{conformance checking}-based and \revision{conformance checking}-independent framework techniques to publicly available datasets, simulated data of a case study from railways, and real-world data of a case study from healthcare. We show that the framework techniques implementing our approach outperform the baseline \revision{conformance checking}-based techniques while maintaining the explainable nature of \revision{conformance checking}.

In summary, the contributions of this paper are as follows.
\begin{itemize}
    \item{
        A novel \revision{process mining}-based feature extraction approach to support the development of competitive and explainable \revision{conformance checking}-based techniques for control-flow anomaly detection.
    }
    \item{
        A flexible and explainable framework for developing techniques for control-flow anomaly detection using \revision{process mining}-based feature extraction and dimensionality reduction.
    }
    \item{
        Application to synthetic and real-world datasets of several \revision{conformance checking}-based and \revision{conformance checking}-independent framework techniques, evaluating their detection effectiveness and explainability.
    }
\end{itemize}

The rest of the paper is organized as follows.
\begin{itemize}
    \item Section \ref{sec:related_work} reviews the existing techniques for control-flow anomaly detection, categorizing them into \revision{conformance checking}-based and \revision{conformance checking}-independent techniques.
    \item Section \ref{sec:abccfe} provides the preliminaries of \revision{process mining} to establish the notation used throughout the paper, and delves into the details of the proposed \revision{process mining}-based feature extraction approach with alignment-based \revision{conformance checking}.
    \item Section \ref{sec:framework} describes the framework for developing \revision{conformance checking}-based and \revision{conformance checking}-independent techniques for control-flow anomaly detection that combine \revision{process mining}-based feature extraction and dimensionality reduction.
    \item Section \ref{sec:evaluation} presents the experiments conducted with multiple framework and baseline techniques using data from publicly available datasets and case studies.
    \item Section \ref{sec:conclusions} draws the conclusions and presents future work.
\end{itemize}
\section{Background}\label{sec:backgrnd}

\subsection{Cold Start Latency and Mitigation Techniques}

Traditional FaaS platforms mitigate cold starts through snapshotting, lightweight virtualization, and warm-state management. Snapshot-based methods like \textbf{REAP} and \textbf{Catalyzer} reduce initialization time by preloading or restoring container states but require significant memory and I/O resources, limiting scalability~\cite{dong_catalyzer_2020, ustiugov_benchmarking_2021}. Lightweight virtualization solutions, such as \textbf{Firecracker} microVMs, achieve fast startup times with strong isolation but depend on robust infrastructure, making them less adaptable to fluctuating workloads~\cite{agache_firecracker_2020}. Warm-state management techniques like \textbf{Faa\$T}~\cite{romero_faa_2021} and \textbf{Kraken}~\cite{vivek_kraken_2021} keep frequently invoked containers ready, balancing readiness and cost efficiency under predictable workloads but incurring overhead when demand is erratic~\cite{romero_faa_2021, vivek_kraken_2021}. While these methods perform well in resource-rich cloud environments, their resource intensity challenges applicability in edge settings.

\subsubsection{Edge FaaS Perspective}

In edge environments, cold start mitigation emphasizes lightweight designs, resource sharing, and hybrid task distribution. Lightweight execution environments like unikernels~\cite{edward_sock_2018} and \textbf{Firecracker}~\cite{agache_firecracker_2020}, as used by \textbf{TinyFaaS}~\cite{pfandzelter_tinyfaas_2020}, minimize resource usage and initialization delays but require careful orchestration to avoid resource contention. Function co-location, demonstrated by \textbf{Photons}~\cite{v_dukic_photons_2020}, reduces redundant initializations by sharing runtime resources among related functions, though this complicates isolation in multi-tenant setups~\cite{v_dukic_photons_2020}. Hybrid offloading frameworks like \textbf{GeoFaaS}~\cite{malekabbasi_geofaas_2024} balance edge-cloud workloads by offloading latency-tolerant tasks to the cloud and reserving edge resources for real-time operations, requiring reliable connectivity and efficient task management. These edge-specific strategies address cold starts effectively but introduce challenges in scalability and orchestration.

\subsection{Predictive Scaling and Caching Techniques}

Efficient resource allocation is vital for maintaining low latency and high availability in serverless platforms. Predictive scaling and caching techniques dynamically provision resources and reduce cold start latency by leveraging workload prediction and state retention.
Traditional FaaS platforms use predictive scaling and caching to optimize resources, employing techniques (OFC, FaasCache) to reduce cold starts. However, these methods rely on centralized orchestration and workload predictability, limiting their effectiveness in dynamic, resource-constrained edge environments.



\subsubsection{Edge FaaS Perspective}

Edge FaaS platforms adapt predictive scaling and caching techniques to constrain resources and heterogeneous environments. \textbf{EDGE-Cache}~\cite{kim_delay-aware_2022} uses traffic profiling to selectively retain high-priority functions, reducing memory overhead while maintaining readiness for frequent requests. Hybrid frameworks like \textbf{GeoFaaS}~\cite{malekabbasi_geofaas_2024} implement distributed caching to balance resources between edge and cloud nodes, enabling low-latency processing for critical tasks while offloading less critical workloads. Machine learning methods, such as clustering-based workload predictors~\cite{gao_machine_2020} and GRU-based models~\cite{guo_applying_2018}, enhance resource provisioning in edge systems by efficiently forecasting workload spikes. These innovations effectively address cold start challenges in edge environments, though their dependency on accurate predictions and robust orchestration poses scalability challenges.

\subsection{Decentralized Orchestration, Function Placement, and Scheduling}

Efficient orchestration in serverless platforms involves workload distribution, resource optimization, and performance assurance. While traditional FaaS platforms rely on centralized control, edge environments require decentralized and adaptive strategies to address unique challenges such as resource constraints and heterogeneous hardware.



\subsubsection{Edge FaaS Perspective}

Edge FaaS platforms adopt decentralized and adaptive orchestration frameworks to meet the demands of resource-constrained environments. Systems like \textbf{Wukong} distribute scheduling across edge nodes, enhancing data locality and scalability while reducing network latency. Lightweight frameworks such as \textbf{OpenWhisk Lite}~\cite{kravchenko_kpavelopenwhisk-light_2024} optimize resource allocation by decentralizing scheduling policies, minimizing cold starts and latency in edge setups~\cite{benjamin_wukong_2020}. Hybrid solutions like \textbf{OpenFaaS}~\cite{noauthor_openfaasfaas_2024} and \textbf{EdgeMatrix}~\cite{shen_edgematrix_2023} combine edge-cloud orchestration to balance resource utilization, retaining latency-sensitive functions at the edge while offloading non-critical workloads to the cloud. While these approaches improve flexibility, they face challenges in maintaining coordination and ensuring consistent performance across distributed nodes.


\section{Restriction to intervals with constant size}
\label{sec:constant}

In this section, we restrict the space of research in Problem \eqref{eq:formal-opt} to the class of prediction sets $\calC^{\text{const}}_{\calF} = \{C_{f,t}(\cdot) = [f(\cdot)-t, f(\cdot)+t]; f \in \calF, t\geq0\}$. This class is already quite interesting as it encapsulates the standard split CP regressor (see Example~\ref{exemple:base-predictor}.1). Notice that for simplicity of exposition, and because it does not depend on $x$, in this section the expected size of $C_{f,t}\in\calC^{\text{const}}_{\calF}$ is simply denoted $\lambda(C_{f,t})=2t$. 

\subsection{Base predictor $f\in\calF$ is given: optimality of the conformal step}
\label{sec:f-given}

We first start in the setting where the base predictor $f$ is given, meaning that we do not consider the learning phase. Over $\calC^{\text{const}}_{\calF}$, the optimization problem~\eqref{eq:formal-opt} becomes:
%
% \begin{align}
%     \min_{t \geq 0} \quad &\; 2t  \\
%      \quad \text{s.t.} \quad & \IP(|Y-f(X)|\leq t) \geq 1-\alpha\;. \nonumber
% \end{align}
\begin{equation}
    \label{eq:obj-constant}
    \min_{t \geq 0} \hspace{0.2cm}  2t  \hspace{0.2cm}\text{s.t.}\hspace{0.2cm}  \IP(|Y-f(X)|\leq t) \geq 1-\alpha\hspace{0.05cm}.
\end{equation}
%
Denoting by $S=|Y-f(X)|$ the random variable of the absolute residual, the solution of the above optimization corresponds to the quantile of order $1-\alpha$ of the random variable $S$. More formally, if we denote by $Q(\hspace{.2em} \cdot \hspace{.2em}  ; S):[0,1]\rightarrow \IR$ the quantile function of $S$, then the optimal value solving~\eqref{eq:obj-constant} is exactly $t^*=Q(1-\alpha; S)$ and the associated optimal set is %$$C^{1-\alpha}_{f,t^*} = \{y \in \calY : |y-f(x)| \leq t^*\} \; .$$
$C^{1-\alpha}_{f,t^*}(x) = [f(x) - t^*, f(x) + t^*] \; .$

Importantly, notice that the conformal step of the original split CP in fact solves an empirical version of the previous problem, but with a slightly increased coverage: %Indeed, by replacing the probability constraint with its empirical counterpart \pie{(with an inflated coverage) - bof}, we obtain:
%
\begin{align}
    \min_{t \geq 0} \quad &\; 2t \label{eq:obj-constant-emp} \\
     \quad \text{s.t.} \quad & \frac{1}{n_c}\sum_{i=1}^{n_c}\1\{|Y_i-f(X_i)|\leq t\} \geq \frac{(1-\alpha)(n_c+1)}{n_c}\; , \nonumber
\end{align}
%
with solution $\hat{t} = S_{(\lceil (n_c+1)(1-\alpha) \rceil)}$ and associated set denoted %\todo{Attention car ici on a dans la notation $\alpha$ mais en fait c'est le "inflated"} %$$C^{1-\alpha}_{f,\hat{t}} = \{y \in \calY : |y-f(x)| \leq \widehat{t}\} \; .$$
$C^{1-\alpha}_{f,\hat{t}}(x) = [f(x) - \hat{t}, f(x) + \hat{t}] \; .$
%
As mentioned above, $\hat{t}$ is the quantity computed during the calibration step of the split CP method (see Section~\ref{sec:conform-background}). It corresponds to the empirical quantile function of $S$, defined by $\widehat{Q}(\hspace{.1em} q \hspace{.1em};\{S_i\}_{i=1}^{n_c}):= \inf\{t : \frac{1}{n_c}\sum_{i=1}^{n_c}\1\{S_i\leq t\}\geq q\}$, evaluated at $(1-\alpha)(n_c+1)/n_c$ instead of $1-\alpha$ to be slightly more conservative. In other words, this means that, when $f$ is given, the calibration step in split CP outputs a conservative empirical estimator of the oracle prediction interval solution of Problem \eqref{eq:obj-constant}.

%Let us denote by $C^{1-\alpha}_{f,t^*}$ (respectively $C^{1-\alpha}_{f,\hat{t}}$) the oracle interval (respectively the estimated interval). 
From the theory of CP, we already know that $\IP(Y\in C^{1-\alpha}_{f,\hat{t}}(X))\geq 1-\alpha$ (see e.g. \citet[Theorem 2.2]{lei2018distribution}). It remains to study the excess volume loss of $C^{1-\alpha}_{f,\hat{t}}$ which is measured by the difference in length between $C^{1-\alpha}_{f,t^*}$ and $C^{1-\alpha}_{f,\hat{t}}$. To this aim, it is sufficient to study the difference between the empirical quantile $\hat{t} = \widehat{Q}((1-\alpha)\frac{n_c+1}{n_c};\{S_i\}_{i=1}^{n_c})$ and the true quantile $t^* = Q(1-\alpha; S)$,  %The following result provides a distribution-free upper bound on the length of $C^{1-\alpha}_{f,\hat{t}}$.
as done in the following proposition (proof in Appendix~\ref{app:proof-prop}).\\

\begin{proposition}
    \label{prop:upper-f-given}
    Let $\hat{t} = \widehat{Q}((1-\alpha)\frac{n_c+1}{n_c};\{S_i\}_{i=1}^{n_c})$ and $C^{1-\alpha}_{f,\hat{t}}$ the corresponding set. If the points in $\calD^{cal}$ are i.i.d., and if $(n_c+1)(1-\alpha)$ is not an integer, then with probability greater than $1-\delta$ we have:
    \begin{equation}
        \label{eq:upper-f-given}
        \lambda\Big(C^{1-\alpha}_{f,\hat{t}}\Big) \leq 2Q\Big(1-\alpha + \frac{1-\alpha}{n_c} + \sqrt{\frac{\log(2/\delta)}{2n_c}}; S\Big)\;.
    \end{equation}
\end{proposition}
% \begin{proof}
% 	The proof is given in Appendix \ref{proof_prop:upper-f-given}.
% \end{proof}

Interestingly, the right-hand side of~\eqref{eq:upper-f-given} also corresponds to the optimal length of a more conservative oracle, namely $\lambda\Big(C^{1-\alpha+\beta_{n_c}}_{f,t^*}\Big)$ with $\beta_{n_c}=\frac{1-\alpha}{n_c} + \sqrt{\frac{\log(2/\delta)}{2n_c}}$. This means that, with high probability, the empirical interval obtained with the conformal step is smaller than the smallest oracle interval with increased coverage $1-\alpha+\beta_{n_c}$, and where $\beta_{n_c}$ is tending to $0$ as $n_c$ grows.

Although interesting, the previous result does not really tell us how different is the size of the predicted interval compared with the oracle one. 
To obtain a finite-sample upper bound on this difference, we must consider some regularity assumption on the distribution of $S$, and more particularly on its quantile function.
%
\begin{assumption}\emph{(Regularity condition).}
    \label{ass:regularity}
    Let $S=|Y-f(X)|$. $\forall f\in \calF$, $\forall \alpha \in (0,1), \exists r,\gamma \in (0,1]$ and $L>0$ such that $Q(\vspace{.2em}\cdot\vspace{.2em};S)$ is locally $(\gamma,L)$-Hölder continuous, i.e. $\forall q_1,q_2 \in [1-\alpha - r, 1-\alpha + r]$: $$|Q(q_1;S) - Q(q_2;S)|\leq L|q_1-q_2|^\gamma \; .$$
%     Either one of the two following conditions is true. \bat{Bat: Il faudra en choisir une et mentionner l'autre dans le texte. La première me semble plus general au sens où la deuxieme implique la première.}
%
% \begin{enumerate}
%     \item $\forall f\in \calF$, $\forall \alpha \in (0,1), \exists r,\gamma \in (0,1]$ and $L>0$ such that $Q(\cdot;S)$ is locally $(\gamma,L)$-Hölder continuous, i.e. $\forall q_1,q_2 \in [1-\alpha, 1-\alpha + r]$, $|Q(q_1;S) - Q(q_2;S)|\leq L|q_1-q_2|^\gamma$.
%     \item The random variable $S$ admits a density $p_S$ with respect to $\lambda$. Moreover, $\exists r'\in(0,1)$ and $L'>0$ such that $\forall q\in [1-\alpha, 1-\alpha + r']$, $p_S(Q(q;S))\geq L'$.
% \end{enumerate} 
\end{assumption}
%
This type of regularity condition can notably be found in \citet{lei2013distribution,yang2024selection}, where it is used to obtain finite-sample bounds on the volume of the returned set. Given this assumption, we can derive the following corollary.

\begin{corollary}
    \label{cor:f-given}
    Let the conditions of Proposition~\ref{prop:upper-f-given} and Assumption~\ref{ass:regularity} hold. If $n_c$ is large enough so that $\frac{1-\alpha}{n_c} + \sqrt{\frac{\log(2/\delta)}{2n_c}} \leq r$, then with probability greater than $1-\delta$:
    \begin{equation}
        \label{eq:upper-f-given-final}
        \lambda\Big(C^{1-\alpha}_{f,\hat{t}}\Big) \leq \lambda\Big(C^{1-\alpha}_{f,t^*}\Big) + 2 L\Big(\frac{1}{n_c} + \sqrt{\frac{\log(2/\delta)}{2n_c}}\Big)^\gamma\;.
    \end{equation} 
    %Moreover, we have the coverage guarantee $\IP(Y\in C^{1-\alpha}_{f,\hat{t}})\geq 1-\alpha$. \pie{nescessaire ca ? on l'a deja dit.} %\bat{Under the second assumption, take $\gamma=1$, replace $r$ by $r'$ and $L$ by $1/L'$.}
\end{corollary}
\begin{proof}
    Direct application of Prop~\ref{prop:upper-f-given} with Assumption~\ref{ass:regularity} and using the fact that $1-\alpha\leq1$. %The coverage guarantee is directly obtained as a consequence of split CP (see e.g.  \citet[Theorem 2.2]{lei2018distribution}). %\bat{For the second assumption, we first apply the mean value theorem.}
\end{proof}

The previous corollary provides an excess volume upper-bound for $C^{1-\alpha}_{f,\hat{t}}$ compared to the oracle $C^{1-\alpha}_{f,t^*}$. This bound does not only confirm the asymptotic optimality of the conformal procedure when $f$ is given, but also provides a rate of convergence dominated by $\tilde{\calO}(n_c^{-\gamma})$ when we get rid of constants and log factors. Although simple to be obtained, to our knowledge this type of bound has never been shown.%is the first of its kind.

\begin{remark}
    \label{rmk:nested}
    When the base predictor is given, all the previous study can be easily extended to the general CP nested set view of \citet{gupta2022nested}. For simplicity of exposition, this analysis is deferred to Appendix~\ref{sec:nested}
\end{remark}


\subsection{Base predictor $f\in\calF$ is \emph{not} given: sub-optimality of the least-square regressor}

In the previous section we saw that, when $f$ is fixed, the calibration step of the split CP method corresponds to the minimization of the size of the interval, up to some statistical error. Now, we investigate how $f$ should be learned during the learning step to obtain a prediction interval of minimal size. Let us consider Problem~\eqref{eq:formal-opt} over $\calC^{\text{const}}_{\calF}$:
% \begin{align*}
%     \min_{f\in\calF, t \geq 0} &\; 2t \\
%      \quad s.t. \quad & \Big\{\IP(|Y-f(X)|\leq t) \geq 1-\alpha \Longleftrightarrow
% \end{align*}
%
%\noindent
%\begin{minipage}{0.45\textwidth}
%\[
%\begin{aligned}
%    \min_{f\in\calF, t \geq 0} &\; \quad 2t \\
%     \quad \text{s.t.} \quad & \IP(|Y-f(X)|\leq t) \geq 1-\alpha 
%\end{aligned}
%\] \vspace{.3em}
%\end{minipage}
%\hfill
%\(\iff\)
%\hfill
%\begin{minipage}{0.45\textwidth}
%\[
%\begin{aligned}
%    \min_{f\in\calF} &\; 2Q(1-\alpha ; |Y-f(X)|)\;, 
%\end{aligned}
%\]
%\end{minipage}
%
%where the bottom optimization problem is directly obtained by replacing $t$ by its optimal value as a function of $f$, i.e. $t^*=Q(1-\alpha ; |Y-f(X)|)$. In words, this optimization problem tells us that $f$ should minimize the $(1-\alpha)$ quantile of the scores $S = |Y-f(X)|$, referred to as the $(1-\alpha)$-QAE (Quantile Absolute Error) in the following. This is quite natural, since this quantile is the one selected to build the prediction interval, and the smaller it is, the smaller the interval will be.
%
\begin{equation}
\min_{f\in\calF, t \geq 0} \hspace{0.2cm}  2t  \hspace{0.2cm}\text{s.t.}\hspace{0.2cm}  \IP(|Y-f(X)|\leq t) \geq 1-\alpha\hspace{0.05cm}.
\end{equation}
By replacing $t$ with its optimal value as a function of $f$, i.e. $t^*=Q(1-\alpha ; |Y-f(X)|)$, we obtain what we call the $(1-\alpha)$-QAE problem (Quantile Absolute Error):
%
\begin{align}\label{eq:QAE}
\min_{f\in\calF} &\; Q(1-\alpha ; |Y-f(X)|)\;.
\end{align}
%
In words, this optimization problem tells us that $f$ should minimize the $(1-\alpha)$-quantile of the distribution of $S = |Y-f(X)|$. This is quite natural, since this quantile is the one selected to build the prediction interval, and the smaller it is, the smaller the interval will be.

What this optimization problem also tells us is that taking $f$ as the minimizer of the Mean Squared Error (MSE) $\EE[(Y-f(X))^2]$, denoted $\mu(x) = \EE[Y|X=x]$, like it is suggested in classical split CP, is not generally optimal in terms of volume minimization, and one should rather take the minimizer of the $(1-\alpha)$-QAE. Notice that, while in general the minimizer of the MSE does not match the one of the $(1-\alpha)$-QAE,
%A possible intuition behind that is that the mean is very sensitive to heavy-tailed distributions and extreme values,\todo{Finalement j'aime pas ce truc...} which could damage the base model, significantly increase the values of the scores, and thus the size of the prediction interval. However, %while in all generality the minimizer of the MSE is not optimal and one should rather take the minimizer of the $(1-\alpha)$-QAE, 
it does in some settings. For instance, in \citet[Section 3]{lei2018distribution}, the authors claim %\todo{je mettrais la ref plutot en fin et la phrase en affirmatif.} 
that if the residual distribution $Y-\mu(X)$ is independent of $X$ and admits a symmetric density with one mode at $0$, then taking $f=\mu$ is optimal, i.e. the minimizer of the MSE matches the minimizer of the $(1-\alpha)$-QAE. However, this kind of assumptions can be quite strong in practice, reason why it is preferable to keep the minimization of the $(1-\alpha)$-QAE as the main objective, since it is optimal on $\calC^{\text{const}}_{\calF}$ no matter the distribution of $(X,Y)$. \looseness = -1

% \begin{remark}
%     The minimizer of the $(1-\alpha)$-QAE should not be confused with the $(1-\alpha)$-quantile regressor of $Y|X=x$. One way to illustrate that is to recall that the first one can be the same as the MSE in some settings. \bat{supprimer si besoin}
% \end{remark}

\subsection{\texttt{EffOrt}: EFFiciency-ORienTed split conformal regression}
In this section, we propose a methodology to approach the oracle prediction set $C^{1-\alpha}_{f^*,t^*}(x) = [f^*(x) - t^*, f^*(x) + t^*]$, with $f^*$ the minimizer of the $(1-\alpha)$-QAE (Problem \eqref{eq:QAE}) and $t^* = Q(1-\alpha ; |Y-f^*(X)|)$. We place ourselves in the split conformal framework of Section~\ref{sec:conform-background}, having access to a learning data set $\calD^{lrn}$ used to learn $f$, and a calibration data set $\calD^{cal}$. With a slight abuse of notation we will write $i\in\calD^{lrn}$ or $\calD^{cal}$ to indicate $(X_i,Y_i)\in\calD^{lrn}$or $\calD^{cal}$.


The proposed methodology, referred to as \texttt{EffOrt}, consists in the following steps:
\begin{enumerate}
    \item Learn $\hat{f} \in \underset{f\in\calF}{\argmin}  \; \widehat{Q}(1-\alpha;\{|Y_i-f(X_i)|\}_{i\in\calD^{lrn}})$, i.e. minimize the empirical version of the $(1-\alpha)$-QAE
    \item Proceed to the calibration step, i.e. take $\hat{t} = \widehat{Q}\Big((1-\alpha)\frac{n_c+1}{n_c};\{|Y_i-\hat{f}(X_i)|\}_{i\in\calD^{cal}}\Big)$
    \item For any test point $X\in\calX$, output the prediction interval $C_{\hat{f},\hat{t}}^{1-\alpha}(X) = [\hat{f}(X)-\hat{t},\hat{f}(X)+\hat{t}]$
\end{enumerate}
%
In \texttt{EffOrt}, the main difficulty is in the first step, where the empirical $(1-\alpha)$-QAE must be minimized. Indeed, it does not have a closed-form solution, and if we want to use a gradient-based optimization algorithm, we must compute the gradient of the empirical $(1-\alpha)$-QAE which not trivial, or might even not be clearly defined. In the following, we present a gradient-based optimization procedure inspired by \citet{pena2020solving}.


%
\subsection{Dataset construction}

In order to learn from these task groups, we must first generate data from them. It is crucial that the data we generate are diverse which would allow the model to learn different strategies without the risk of overfitting. We accomplish this by generating a large number of trajectories at a high temperature with Min-p sampling~\citep{nguyen2024turning}. Min-p sampling works by using an adaptive threshold $p_\text{scaled} \propto p_\text{max}$, where $p_\text{max}$ is the highest probability predicted by the model on the next token, to truncate the vocabulary to tokens that have a probability larger than $p_\text{scaled}$ and sample from them --- this enables us to generate diverse yet coherent trajectories at a higher temperature.


For each task in a set of chosen tasks (e.g., uniformly sampled), we generate $n_\text{sample}$ trajectories and then construct a preference pair $(h_{w}, h_{l})$ where $h_{w}$ is the highest scoring trajectory (trajectory that succeeds and does so at the fewest number of turns) and $h_{l}$ is randomly sampled from the lower scoring (failed  or takes substantially more turns to succeed) trajectories. We choose $h_l$ randomly instead of choosing the worst one to increase the diversity of our dataset. We treat $h_w$ and $h_l$ as proxies for desirable and undesirable behaviors. A dataset $\mathcal{D} = \left\{\left(h^{w}, h^{l}\right)^{(i)}\right\}_{i=1}^N$ is a collection of such trajectory pairs.

\subsection{Optimization}
\label{sec:opt}

\paragraph{Supervised fine-tuning.} If we take the winning episodes as the expert behavior, then we can discard the losing episode and maximize the likelihood of winning episodes:

\begin{align}
    \mathcal{L}_\text{SFT}(\mathcal{D}) = \mathbb{E}_{\mathcal{D}} \left[ \frac{1}{\sum_{t=0}^{|h_w|}|a_t^w|}\sum_{t=0}^{|h_w|} \log \pi_\theta \left(a^w_t \mid h^w_{:t}\right) \right].
\end{align}
where $|a|$ is the number of tokens for the agent response (discarding the environment generation). This is akin to rejection sampling fine-tuning~\citep{gulcehre2023reinforcedselftrainingrestlanguage,dong2023raft,mukobi2023superhfsupervisediterativelearning} seen in prior work.

\paragraph{Direct preference optimization.} A popular approach for finetuning LLMs is DPO~\citep{rafailov2024direct} where one directly optimizes the Bradley-Terry model~\citep{bradley1952rank} for preferences. In our setting, each trajectory consists of multiple rounds of interactions so the original DPO objective does not apply. We instead use a multi-turn version of DPO introduced in ~\citet{rafailov2024rqlanguagemodel}:
\begin{multline}
    \mathcal{L}_\text{DPO}(\gD) = \E_{\gD}\Bigg[\log \sigma\Bigg( 
    \sum_{t=0}^{|h^w|}\beta \log\frac{\pi_\theta(a_t^w \mid h_{:t}^w)}{\pi_\text{ref}(a_t^w \mid h_{:t}^w)} \\
    - \sum_{t=0}^{|h^l|}\beta \log\frac{\pi_\theta(a_t^l \mid h_{:t}^l)}{\pi_\text{ref}(a_t^l \mid h_{:t}^l)}
    \Bigg)\Bigg]
\end{multline}

where $a_t^w$ is the action tokens generated by the model at turn $t$ in the preferred trajectory $h^w$.
$\pi_\text{ref}$ is the reference policy, for which we use the initial model.
The main difference with standard DPO here is that we only calculate the loss on the action tokens --- the log probability ratios of the environment generated tokens are not included in the loss.

We note that we use DPO because it is less compute intensive. DPO allows us to decouple the data collection and policy improvement steps and offload them on different machines. However, in principle, one could also employ online RL with more resources. Following prior work that shows the efficacy of online RL compared to offline algorithms~\citep{xu2024dposuperiorppollm,tajwar2024preferencefinetuningllmsleverage}, we expect doing \ours{} with online RL would lead to even stronger results.

\paragraph{Combining objectives.} 

Finally, prior works have noted DPO having the unintended effect of reducing the probability of preferred trajectories as well, known as unintentional unalignment~\citep{razin2024unintentionalunalignmentlikelihooddisplacement}, which can affect model performance. The RPO objective~\citep{pang2024iterativereasoningpreferenceoptimization}, by combining SFT and DPO loss, has shown promising results in mitigating this issue. Formally, the RPO loss is:

\begin{equation} \label{eq:rpo_formula}
    \mathcal{L}_{\text{RPO}}(\mathcal{D}) = \mathcal{L}_\text{DPO}(\gD) + \alpha \mathcal{L}_\text{SFT}(\gD)
\end{equation}

where $\alpha$ is a hyper-parameter. Following ~\citet{pang2024iterativereasoningpreferenceoptimization}, we set $\alpha$ to be 1.0 for the rest of this paper.




\subsection{Theoretical analysis}

In this last subsection, we theoretically analyze the performance of the prediction set output by \texttt{EffOrt}. We are interested in two types of guarantees: (i) a coverage guarantee and (ii) an excess volume loss guarantee like the one in Eq.~\eqref{eq:upper-f-given-final}. To this aim, we require the following assumption.

\begin{assumption} \label{ass:complexity} There exists $\phi(\calF,\delta,n)<+\infty$ such that with probability at least $1-\delta$:
    %\begin{equation}
    %    \IP\left(\sup_{t\geq 0, f\in\calF}\Big|\IP\left(|Y-f(X)|\leq t\right) - \frac{1}{n}\sum_{i=1}^n\1\{|Y_i-f(X_i)|\leq t\}\Big|\leq \phi(\calF,\delta,n)\right)\geq 1-\delta
    %\end{equation}
   \begin{align*}
       &\sup_{\overset{\scriptstyle t\geq 0}{f\in\calF}}\Big|\IP\left(|Y-f(X)|\leq t\right) - \frac{1}{n}\sum_{i=1}^n\1\{|Y_i-f(X_i)|\leq t\}\Big| \leq \phi(\calF,\delta,n) \; .
       %
   \end{align*}
\end{assumption}
In this assumption, $\phiF$ bounds the worst-case estimation error of $\IP(|Y-f(X)|\leq t)$ using the empirical estimate $\frac{1}{n}\sum_{i=1}^n\1\{|Y_i-f(X_i)|\leq t\}$ over the whole function class $\calF$ and for any value of $t$. Typically, $\phiF$ will decrease with an increasing number of data points $n$ and increase as the \emph{complexity} of $\calF$ gets larger. In the following proposition, we explicitly derive a closed-form expression for $\phiF$ when the function class $\calF$ is finite. 
%
\begin{proposition}\emph{(Finite class $\calF$).}
    \label{prop:phi-finite}
    If $|\calF|<\infty$, then Assumption~\ref{ass:complexity} is verified with $\phiF = \sqrt{\frac{\log(2|\calF|/\delta)}{2n}}$.
\end{proposition}
%
% \begin{proof}
% 	The proof is given in Appendix \ref{..}.
% \end{proof}
Similarly to the “classical” statistical learning framework, where it is possible to obtain generalization bounds for infinite hypothesis classes, it is possible to derive other closed-forms for $\phiF$ in the infinite case by involving complexity measures like VC dimensions or Rademacher complexities. This, along with the proof of Prop.~\ref{prop:phi-finite}, is discussed in Appendix \ref{sec:closed-form-phi}. We can now present our main theoretical result.

\begin{theorem}
    \label{thme:main-constant}
    Let $C_{\hat{f},\hat{t}}^{1-\alpha}(x)$ %= [\hat{f}(x)-\hat{t},\hat{f}(x)+\hat{t}]$ 
    be the prediction interval output by \method~. If Assumption~\ref{ass:regularity} and~\ref{ass:complexity} are satisfied, the distribution of $Y$ is atomless, $n_c$ and $n_\ell$ are large enough so that $\frac{1-\alpha}{n_c} + \sqrt{\frac{\log(2/\delta)}{2n_c}} \leq r$ and $\phiFl\leq r$, then:
    \vspace{-0.2cm}
    \begin{enumerate}[leftmargin=*]
        \item $\IP(Y\in C_{\hat{f},\hat{t}}^{1-\alpha}(X)|\calD^{lrn})\geq 1-\alpha$ a.s. %\todo{pas besoin de toutes les assumptions ici}
        \item With probability greater that $1-2\delta$:
        \begin{align}
            \label{eq:thme-const}
            &\lambda\left(C_{\hat{f},\hat{t}}^{1-\alpha}\right)\leq \lambda\left(C_{f^*,t^*}^{1-\alpha}\right) + 2L\Big(\frac{1}{n_{c}} +\sqrt{\frac{\log(2/\delta)}{2n_{c}}}\Big)^{\gamma} + 4L\phiFl^\gamma %\nonumber
        \end{align}
    \end{enumerate}
\end{theorem}

\begin{proof}[Proof sketch - Details in Appendix~\ref{app:proof-main}] The first result is classical \citep{lei2018distribution}. Let us focus on the second one, proved with the following steps.
    

\underline{Step 1:} In the first step of \method, we are actually solving the empirical objective $\min_{f\in\calF, t\geq 0}~ \{t$ s.t. $n_l^{-1}\sum_{i\in\calD^{lrn}}\1\{|Y_i-f(X_i)|\leq t\}\geq 1-\alpha\}$, with solutions denoted by $\hat{f}$ and $\hat{t}_{lrn}$. Using the theory of MVS estimation \citep{NIPS2005_d3d80b65}, we can compare this solution to the oracle one. Indeed, by adapting the proof of \citet[Theorem 1]{NIPS2005_d3d80b65}, we can show that with probability greater than $1-\delta$:
\begin{equation}
    \label{eq:lower-scott}
    \IP(Y\in C_{\hat{f},\hat{t}_{lrn}}^{1-\alpha}(X)|\calD^{lrn})\geq 1-\alpha - \phiFl
\end{equation}
and
\begin{equation}
    \label{eq:lower-scott-bis}
    \lambda\left(C_{\hat{f},\hat{t}_{lrn}}^{1-\alpha}\right) \leq \lambda\left(C_{f_{1-\alpha+\phi}^*,t_{1-\alpha+\phi}^*}^{1-\alpha + \phi}\right) \;,
\end{equation}
where $\phi \equiv \phiFl$ and $C_{f_{1-\alpha+\phi}^*,t_{1-\alpha+\phi}^*}^{1-\alpha + \phi}$ denotes the optimal oracle interval with coverage increased by $\phiFl$. This tells us that after the learning step we already have some guarantees: (i) a high probability coverage guarantee, with a looser coverage decreased by $\phiFl$, (ii) an excess volume guarantee, ensuring that the volume of the learned interval is smaller than the optimal one with coverage increased by $\phi$. Interestingly, this also means that the conformal step  allows to obtain an almost sure coverage guarantee, and to get rid of the statistical error due to $\phiFl$ in the coverage. %\todo{Mettre ce dernier commentaire en dehors du sketch pour plus d'impact?}

\underline{Step 2:}  %\todo{Il faudrait des "with high probability" un peu partout mais baleck ?}
% Since the distribution of $Y$ is atomless, with probability greater than $1-\delta$ we have  \citep{humbert2024marginal}:
% \begin{equation} 
%     \label{eq:upper-classic}
%     \IP( |Y-\hat{f}(X)| \leq \hat{t} \big|\calD)\leq 1-\alpha + \frac{1}{n_c+1}+\sqrt{\frac{\log(1/\delta)}{n_{c}+1}}\;.
% \end{equation}
From~\eqref{eq:lower-scott-bis} we have $\hat{t}_{lrn} \leq t_{1-\alpha+\phi}^*$ and therefore $\hat{t} \leq t^* + \hat{t} - \hat{t}_{lrn} + t_{1-\alpha+\phi}^* - t^*$. 

With~\eqref{eq:upper-f-given} in Prop.~\ref{prop:upper-f-given}, we have $\hat{t} \leq Q(1-\alpha + \frac{1-\alpha}{n_c}+\sqrt{\frac{\log(2/\delta)}{2n_{c}}}; |Y-\hat{f}(X)|_{|\calD^{lrn}})$. Moreover, from~\eqref{eq:lower-scott}, $\hat{t}_{lrn}\geq Q(1-\alpha - \phiFl; |Y-\hat{f}(X)|_{|\calD^{lrn}})$. Hence, thanks to Assumption~\ref{ass:regularity}, $\hat{t} - \hat{t}_{lrn} \leq L\Big(\frac{1}{n_{c}} +\sqrt{\frac{\log(2/\delta)}{2n_{c}}}\Big)^{\gamma} + L\phiFl^\gamma$.
It remains to bound $t_{1-\alpha+\phi}^* - t^*$. By definition, we have $t_{1-\alpha+\phi}^* = Q(1-\alpha + \phi; |Y-f_{1-\alpha+\phi}^*(X)|)$, and $t^* = Q(1-\alpha; |Y-f^*(X)|)$. Moreover, we notice that $t_{1-\alpha+\phi}^* \leq Q(1-\alpha + \phi; |Y-f^*(X)|)$ since by definition $f_{1-\alpha+\phi}^*$ minimizes $Q(1-\alpha + \phi; |Y-f(X)|)$ over all $f\in\calF$. Hence, $t_{1-\alpha+\phi}^* - t^* \leq L\phiFl^\gamma$, by Assumption~\ref{ass:regularity}. We conclude by combining everything. %We conclude using the fact that $\lambda(C_{\hat{f},\hat{t}}^{1-\alpha}) = 2\hat{t}$ and $\lambda(C_{f^*,t ^*}^{1-\alpha}) = 2t^*$.
\end{proof}


To the best of our knowledge, Theorem~\ref{thme:main-constant} is one of the first to provide such a finite-sample upper bound on the excess-volume loss. It explicitly reveals the impact of the two split conformal steps of \method. The two first error terms (involving $n_c$) match the bound of Corollary~\ref{cor:f-given}, and can be seen as the volume loss due to the calibration step. While the third term, with $\phiFl$, is the error due to the learning step. If we omit the dependence in $\delta$, $\phiFl$ will typically be in the form of $\sqrt{\frac{\text{Compl}(\calF)}{n_\ell}}$, where $\text{Compl}(\calF)$ measures the complexity of $\calF$ (see Prop.~\ref{prop:phi-finite} and Appendix \ref{sec:closed-form-phi}). In most settings, we have $\text{Compl}(\calF)\gg \log(1/\delta)$. Hence, the rate in Eq.~\eqref{eq:thme-const} supports the important intuition that the learning step remains more important than the conformal step, at least in the sense that more data-points are needed to reach convergence. It is thus preferable to assign more points to the learning than the calibration.
%, since in most settings we have $\text{Compl}(\calF)\gg \log(1/\delta)$.% the error due to the learning step dominates the error due to the conformal step%, and contrary to what is usually done, i.e.~taking $n_c=n_\ell=n/2$.%, our result suggests taking instead $n_c$ in the order of $\sqrt{n_\ell}$.

% \begin{remark}
%     The more complex is the class of regression function $\calF$, the more data-points are needed to reach convergence. However, it should be kept in mind that the complexity $\calF$ must also be high enough to hopefully contain the \emph{true} minimizer of the $(1-\alpha)$-QAE.  
% \end{remark}

% \todo{The more complex is the class of regression function $\calF$, the more data-points are needed to reach convergence. However, it should be kept in mind that the complexity $\calF$ must also be high enough to hopefully contain the \emph{true} minimizer of the $(1-\alpha)$-QAE.  }
\section{Extension to intervals with adaptive size}
\label{sec:adaptive}

We now consider the case of prediction intervals whose size adapts to the value of $X$. Formally, we consider the class of prediction sets $\calC^{\text{adap}}_{\calF,\calS} = \{C_{f,s}(x) = [f(x)-s(x), f(x)+s(x)] : f \in \calF, s\in\calS\}$, where $\calS$ is a class of non-negative functions. Importantly, this class of prediction sets encapsulates the Locally-Weighted Conformal Inference and the CQR methods (see Examples~\ref{exemple:base-predictor}.2 and~\ref{exemple:base-predictor}.3).

\subsection{Oracle prediction set and conditioning over $X=x$}

\label{sec:adap-oracle}

Following a similar reasoning as in Section~\ref{sec:constant}, we could first consider $f$ fixed and derive a closed-form oracle expression for $s$ by solving Problem~\eqref{eq:formal-opt} with $\calC_{\text{Borel}}$ replaced by $\calC^{\text{adap}}_{\calF,\calS}$. %This would give the problem $\min_{s\in\calS}\{\EE[s(X)] \text{ s.t. } \IP(|Y-f(X)|\leq s(X)) \geq 1-\alpha\}$. 
Unfortunately, contrary to the previous section, the solution of this problem does not have a direct expression.

For this reason, we propose to modify the problem so that $s$ admits an oracle closed-form expression which can be naturally estimated empirically. More precisely,
% instead of solving Problem~\eqref{eq:formal-opt} over $\calC^{\text{adap}}_{\calF,\calS}$,
we condition the optimization problem over the event $X=x$, where $x\in\calX$. In that case, the problem becomes:
\begin{equation}
    \label{eq:obj-adap}
    \min_{s \in \calS}  s(x) \hspace{0.1cm} \text{s.t.} \hspace{0.1cm}  \IP(|Y-f(x)|\leq s(x)|X=x) \geq 1-\alpha 
\end{equation}
This problem is more difficult than \eqref{eq:QAE} as a \emph{conditional} coverage constraint is now required, which is known to be harder to obtain in practice \citep{vovk2012conditional, lei2014distribution}. If $\calS$ is sufficiently complex, Problem~\eqref{eq:obj-adap} has an oracle close-form solution, which is given by the $(1-\alpha)$-quantile of $|Y-f(X)|$ conditioned on $X=x$, denoted by $s^*(x) := Q(1-\alpha;|Y-f(X)|_{|X=x})$. Interestingly, the function $s^*(x)$ is the quantile regression function of $|Y-f(X)|$ given $X=x$, and corresponds to the solution of $\min_{s\in\calS}\EE[\rho_{1-\alpha}(|Y-f(X)| - s(X))]$, where $\rho_{1-\alpha}$ is the pinball loss. Hence, a natural solution is to use an empirical plug-in estimator of $s^*$, i.e. minimizing an empirical version of the pinball risk, as suggested in the next section.

\begin{remark}
    Another strategy could be to directly solve an empirical version of $\min_{f\in\calF,s\in\calS}\EE[s(X)]$ s.t.  $\IP(|Y-f(X)|\leq s(X)) \geq 1-\alpha$. This would allows deriving results similar to those of the previous section (see Appendix~\ref{sec:adapt-bonus}), but solving it in practice can be challenging, notably because of the empirical coverage constraint.
    Notice that, although their objective is different from ours, \citet{baiefficient} face a similar optimization problem, where they propose a smooth and differentiable relaxation to solve it. \looseness=-1%, using the Lagrangian function and a surrogate loss to approximate the constraint. %However, they do not provide theoretical guarantees regarding their approach.    
\end{remark}

%On the other side, the second solution, namely rewriting the problem in order to have a closed-form expression for $t$, can be more interesting in terms of interpretability. 


\subsection{\texttt{Ad-EffOrt}}

We now describe our second method,~\methodAD, which extends \method~to prediction intervals with adaptive size. Like in~\method, we consider the split CP framework, having access to a learning dataset $\calD^{lrn}$ used to learn the base predictors $f$ and $s$, and a calibration data set $\calD^{cal}$. %$\calD^{cal}$ used to conformalize the learned prediction set $C_{f,s}(x)$ by increasing (or decreasing) its size to satisfy the coverage guarantee.
\methodAD~consists in the following steps:%\todo{pas tellement specifique a Effort finalement. C'est un nouveau score l'idée ici. Ce que je voulais dire ici c'est que le step 2 de \methodAD peut aussi bien etre fait avec le splitCP standard (ou meme avec un $f$ donné).}
\begin{enumerate}[leftmargin=*]
    \item $\hat{f} \in \underset{f\in\calF}{\argmin}  \; \widehat{Q}(1-\alpha;\{|Y_i-f(X_i)|\}_{i\in\calD^{lrn}})$
    \item $\hat{s} \in \underset{s\in\calS}{\argmin}\frac{1}{n_\ell}\sum_{i\in\calD^{lrn}}\rho_{1-\alpha}(|Y_i-\hat{f}(X_i)| - s(X_i))$
    \item $\hat{t} = \widehat{Q}\Big((1-\alpha)\frac{n_c+1}{n_c};\{|Y_i-\hat{f}(X_i)| - \hat{s}(X_i)\}_{i\in\calD^{cal}}\Big)$
    \item For any test point $X\in\calX$, output $C_{\hat{f}, \hat{s},\hat{t}}^{1-\alpha}(X) = [\hat{f}(X)-\hat{s}(X)-\hat{t},\hat{f}(X)+\hat{s}(X) +\hat{t}] \; .$
\end{enumerate}
\vspace{-.4em}
%
In the first two steps of \methodAD, we learn the model $f$ as in \method~and then fit the residuals using a quantile regression or order $1-\alpha$. %As usual in machine learning, regularization terms can be added on top of the empirical error terms. 
Note that, in those two steps, the same data are used to learn both the prediction model $f$ and the quantile regressor $s$, but we also might split the learning set in two. %If we were to seek theoretical guarantees, this would help to provide some consistency results. 
Then, in the third step (calibration), we take the quantile of $\{|Y_i-\hat{f}(X_i)| - \hat{s}(X_i)\}_{i\in\calD^{cal}}$. %\todo{on a pas cité l'autre du coup.} %, which differs from \method. 
This comes from the fact that the final prediction interval is in the form $[f(x)-s(x)-t,f(x)+s(x)+t]$, and, given the base predictors $(f, s)$, the smallest $t$ such that we satisfy the coverage is $Q\big((1-\alpha);|Y-f(X)| - s(X)\big)$. This claim is easily proved by following the analysis of Section~\ref{sec:f-given}. 

The main limitation of \methodAD~is the difficulty of providing a theoretical guarantee similar to that of Theorem~\ref{thme:main-constant}. This is notably due to the fact that while $s$ in learned in order to obtain conditional guarantees, $f$ is learned as in \method, i.e.~in order to obtain marginal guarantees. When $f$ is fixed, one could actually derive guarantees on $\hat{s}$ and its ability to solve~\eqref{eq:obj-adap} by providing a setting under which the quantile regressor is consistent, making~\eqref{eq:obj-adap} asymptotically verified. Last but not least, it is worth mentioning that, thanks to the calibration step, the marginal coverage guarantee is verified.%\todo{J'ai l'impression que cette derniere phrase est quand meme importante mais je pas ou la mettre mieux}

%\pie{using a strategy similar to the one of \citep{han2022split}}\todo{A placer après sans dire qu'on a suivit leur "strategy". Jsp ou le mettre... Peut-être plus haut?}

%\bat{Ajouter ici, ou dans les xps, une comapraison avec CQR et locally weighted}

\section{Experiments}
\label{sec:experiments}
The experiments are designed to address two key research questions.
First, \textbf{RQ1} evaluates whether the average $L_2$-norm of the counterfactual perturbation vectors ($\overline{||\perturb||}$) decreases as the model overfits the data, thereby providing further empirical validation for our hypothesis.
Second, \textbf{RQ2} evaluates the ability of the proposed counterfactual regularized loss, as defined in (\ref{eq:regularized_loss2}), to mitigate overfitting when compared to existing regularization techniques.

% The experiments are designed to address three key research questions. First, \textbf{RQ1} investigates whether the mean perturbation vector norm decreases as the model overfits the data, aiming to further validate our intuition. Second, \textbf{RQ2} explores whether the mean perturbation vector norm can be effectively leveraged as a regularization term during training, offering insights into its potential role in mitigating overfitting. Finally, \textbf{RQ3} examines whether our counterfactual regularizer enables the model to achieve superior performance compared to existing regularization methods, thus highlighting its practical advantage.

\subsection{Experimental Setup}
\textbf{\textit{Datasets, Models, and Tasks.}}
The experiments are conducted on three datasets: \textit{Water Potability}~\cite{kadiwal2020waterpotability}, \textit{Phomene}~\cite{phomene}, and \textit{CIFAR-10}~\cite{krizhevsky2009learning}. For \textit{Water Potability} and \textit{Phomene}, we randomly select $80\%$ of the samples for the training set, and the remaining $20\%$ for the test set, \textit{CIFAR-10} comes already split. Furthermore, we consider the following models: Logistic Regression, Multi-Layer Perceptron (MLP) with 100 and 30 neurons on each hidden layer, and PreactResNet-18~\cite{he2016cvecvv} as a Convolutional Neural Network (CNN) architecture.
We focus on binary classification tasks and leave the extension to multiclass scenarios for future work. However, for datasets that are inherently multiclass, we transform the problem into a binary classification task by selecting two classes, aligning with our assumption.

\smallskip
\noindent\textbf{\textit{Evaluation Measures.}} To characterize the degree of overfitting, we use the test loss, as it serves as a reliable indicator of the model's generalization capability to unseen data. Additionally, we evaluate the predictive performance of each model using the test accuracy.

\smallskip
\noindent\textbf{\textit{Baselines.}} We compare CF-Reg with the following regularization techniques: L1 (``Lasso''), L2 (``Ridge''), and Dropout.

\smallskip
\noindent\textbf{\textit{Configurations.}}
For each model, we adopt specific configurations as follows.
\begin{itemize}
\item \textit{Logistic Regression:} To induce overfitting in the model, we artificially increase the dimensionality of the data beyond the number of training samples by applying a polynomial feature expansion. This approach ensures that the model has enough capacity to overfit the training data, allowing us to analyze the impact of our counterfactual regularizer. The degree of the polynomial is chosen as the smallest degree that makes the number of features greater than the number of data.
\item \textit{Neural Networks (MLP and CNN):} To take advantage of the closed-form solution for computing the optimal perturbation vector as defined in (\ref{eq:opt-delta}), we use a local linear approximation of the neural network models. Hence, given an instance $\inst_i$, we consider the (optimal) counterfactual not with respect to $\model$ but with respect to:
\begin{equation}
\label{eq:taylor}
    \model^{lin}(\inst) = \model(\inst_i) + \nabla_{\inst}\model(\inst_i)(\inst - \inst_i),
\end{equation}
where $\model^{lin}$ represents the first-order Taylor approximation of $\model$ at $\inst_i$.
Note that this step is unnecessary for Logistic Regression, as it is inherently a linear model.
\end{itemize}

\smallskip
\noindent \textbf{\textit{Implementation Details.}} We run all experiments on a machine equipped with an AMD Ryzen 9 7900 12-Core Processor and an NVIDIA GeForce RTX 4090 GPU. Our implementation is based on the PyTorch Lightning framework. We use stochastic gradient descent as the optimizer with a learning rate of $\eta = 0.001$ and no weight decay. We use a batch size of $128$. The training and test steps are conducted for $6000$ epochs on the \textit{Water Potability} and \textit{Phoneme} datasets, while for the \textit{CIFAR-10} dataset, they are performed for $200$ epochs.
Finally, the contribution $w_i^{\varepsilon}$ of each training point $\inst_i$ is uniformly set as $w_i^{\varepsilon} = 1~\forall i\in \{1,\ldots,m\}$.

The source code implementation for our experiments is available at the following GitHub repository: \url{https://anonymous.4open.science/r/COCE-80B4/README.md} 

\subsection{RQ1: Counterfactual Perturbation vs. Overfitting}
To address \textbf{RQ1}, we analyze the relationship between the test loss and the average $L_2$-norm of the counterfactual perturbation vectors ($\overline{||\perturb||}$) over training epochs.

In particular, Figure~\ref{fig:delta_loss_epochs} depicts the evolution of $\overline{||\perturb||}$ alongside the test loss for an MLP trained \textit{without} regularization on the \textit{Water Potability} dataset. 
\begin{figure}[ht]
    \centering
    \includegraphics[width=0.85\linewidth]{img/delta_loss_epochs.png}
    \caption{The average counterfactual perturbation vector $\overline{||\perturb||}$ (left $y$-axis) and the cross-entropy test loss (right $y$-axis) over training epochs ($x$-axis) for an MLP trained on the \textit{Water Potability} dataset \textit{without} regularization.}
    \label{fig:delta_loss_epochs}
\end{figure}

The plot shows a clear trend as the model starts to overfit the data (evidenced by an increase in test loss). 
Notably, $\overline{||\perturb||}$ begins to decrease, which aligns with the hypothesis that the average distance to the optimal counterfactual example gets smaller as the model's decision boundary becomes increasingly adherent to the training data.

It is worth noting that this trend is heavily influenced by the choice of the counterfactual generator model. In particular, the relationship between $\overline{||\perturb||}$ and the degree of overfitting may become even more pronounced when leveraging more accurate counterfactual generators. However, these models often come at the cost of higher computational complexity, and their exploration is left to future work.

Nonetheless, we expect that $\overline{||\perturb||}$ will eventually stabilize at a plateau, as the average $L_2$-norm of the optimal counterfactual perturbations cannot vanish to zero.

% Additionally, the choice of employing the score-based counterfactual explanation framework to generate counterfactuals was driven to promote computational efficiency.

% Future enhancements to the framework may involve adopting models capable of generating more precise counterfactuals. While such approaches may yield to performance improvements, they are likely to come at the cost of increased computational complexity.


\subsection{RQ2: Counterfactual Regularization Performance}
To answer \textbf{RQ2}, we evaluate the effectiveness of the proposed counterfactual regularization (CF-Reg) by comparing its performance against existing baselines: unregularized training loss (No-Reg), L1 regularization (L1-Reg), L2 regularization (L2-Reg), and Dropout.
Specifically, for each model and dataset combination, Table~\ref{tab:regularization_comparison} presents the mean value and standard deviation of test accuracy achieved by each method across 5 random initialization. 

The table illustrates that our regularization technique consistently delivers better results than existing methods across all evaluated scenarios, except for one case -- i.e., Logistic Regression on the \textit{Phomene} dataset. 
However, this setting exhibits an unusual pattern, as the highest model accuracy is achieved without any regularization. Even in this case, CF-Reg still surpasses other regularization baselines.

From the results above, we derive the following key insights. First, CF-Reg proves to be effective across various model types, ranging from simple linear models (Logistic Regression) to deep architectures like MLPs and CNNs, and across diverse datasets, including both tabular and image data. 
Second, CF-Reg's strong performance on the \textit{Water} dataset with Logistic Regression suggests that its benefits may be more pronounced when applied to simpler models. However, the unexpected outcome on the \textit{Phoneme} dataset calls for further investigation into this phenomenon.


\begin{table*}[h!]
    \centering
    \caption{Mean value and standard deviation of test accuracy across 5 random initializations for different model, dataset, and regularization method. The best results are highlighted in \textbf{bold}.}
    \label{tab:regularization_comparison}
    \begin{tabular}{|c|c|c|c|c|c|c|}
        \hline
        \textbf{Model} & \textbf{Dataset} & \textbf{No-Reg} & \textbf{L1-Reg} & \textbf{L2-Reg} & \textbf{Dropout} & \textbf{CF-Reg (ours)} \\ \hline
        Logistic Regression   & \textit{Water}   & $0.6595 \pm 0.0038$   & $0.6729 \pm 0.0056$   & $0.6756 \pm 0.0046$  & N/A    & $\mathbf{0.6918 \pm 0.0036}$                     \\ \hline
        MLP   & \textit{Water}   & $0.6756 \pm 0.0042$   & $0.6790 \pm 0.0058$   & $0.6790 \pm 0.0023$  & $0.6750 \pm 0.0036$    & $\mathbf{0.6802 \pm 0.0046}$                    \\ \hline
%        MLP   & \textit{Adult}   & $0.8404 \pm 0.0010$   & $\mathbf{0.8495 \pm 0.0007}$   & $0.8489 \pm 0.0014$  & $\mathbf{0.8495 \pm 0.0016}$     & $0.8449 \pm 0.0019$                    \\ \hline
        Logistic Regression   & \textit{Phomene}   & $\mathbf{0.8148 \pm 0.0020}$   & $0.8041 \pm 0.0028$   & $0.7835 \pm 0.0176$  & N/A    & $0.8098 \pm 0.0055$                     \\ \hline
        MLP   & \textit{Phomene}   & $0.8677 \pm 0.0033$   & $0.8374 \pm 0.0080$   & $0.8673 \pm 0.0045$  & $0.8672 \pm 0.0042$     & $\mathbf{0.8718 \pm 0.0040}$                    \\ \hline
        CNN   & \textit{CIFAR-10} & $0.6670 \pm 0.0233$   & $0.6229 \pm 0.0850$   & $0.7348 \pm 0.0365$   & N/A    & $\mathbf{0.7427 \pm 0.0571}$                     \\ \hline
    \end{tabular}
\end{table*}

\begin{table*}[htb!]
    \centering
    \caption{Hyperparameter configurations utilized for the generation of Table \ref{tab:regularization_comparison}. For our regularization the hyperparameters are reported as $\mathbf{\alpha/\beta}$.}
    \label{tab:performance_parameters}
    \begin{tabular}{|c|c|c|c|c|c|c|}
        \hline
        \textbf{Model} & \textbf{Dataset} & \textbf{No-Reg} & \textbf{L1-Reg} & \textbf{L2-Reg} & \textbf{Dropout} & \textbf{CF-Reg (ours)} \\ \hline
        Logistic Regression   & \textit{Water}   & N/A   & $0.0093$   & $0.6927$  & N/A    & $0.3791/1.0355$                     \\ \hline
        MLP   & \textit{Water}   & N/A   & $0.0007$   & $0.0022$  & $0.0002$    & $0.2567/1.9775$                    \\ \hline
        Logistic Regression   &
        \textit{Phomene}   & N/A   & $0.0097$   & $0.7979$  & N/A    & $0.0571/1.8516$                     \\ \hline
        MLP   & \textit{Phomene}   & N/A   & $0.0007$   & $4.24\cdot10^{-5}$  & $0.0015$    & $0.0516/2.2700$                    \\ \hline
       % MLP   & \textit{Adult}   & N/A   & $0.0018$   & $0.0018$  & $0.0601$     & $0.0764/2.2068$                    \\ \hline
        CNN   & \textit{CIFAR-10} & N/A   & $0.0050$   & $0.0864$ & N/A    & $0.3018/
        2.1502$                     \\ \hline
    \end{tabular}
\end{table*}

\begin{table*}[htb!]
    \centering
    \caption{Mean value and standard deviation of training time across 5 different runs. The reported time (in seconds) corresponds to the generation of each entry in Table \ref{tab:regularization_comparison}. Times are }
    \label{tab:times}
    \begin{tabular}{|c|c|c|c|c|c|c|}
        \hline
        \textbf{Model} & \textbf{Dataset} & \textbf{No-Reg} & \textbf{L1-Reg} & \textbf{L2-Reg} & \textbf{Dropout} & \textbf{CF-Reg (ours)} \\ \hline
        Logistic Regression   & \textit{Water}   & $222.98 \pm 1.07$   & $239.94 \pm 2.59$   & $241.60 \pm 1.88$  & N/A    & $251.50 \pm 1.93$                     \\ \hline
        MLP   & \textit{Water}   & $225.71 \pm 3.85$   & $250.13 \pm 4.44$   & $255.78 \pm 2.38$  & $237.83 \pm 3.45$    & $266.48 \pm 3.46$                    \\ \hline
        Logistic Regression   & \textit{Phomene}   & $266.39 \pm 0.82$ & $367.52 \pm 6.85$   & $361.69 \pm 4.04$  & N/A   & $310.48 \pm 0.76$                    \\ \hline
        MLP   &
        \textit{Phomene} & $335.62 \pm 1.77$   & $390.86 \pm 2.11$   & $393.96 \pm 1.95$ & $363.51 \pm 5.07$    & $403.14 \pm 1.92$                     \\ \hline
       % MLP   & \textit{Adult}   & N/A   & $0.0018$   & $0.0018$  & $0.0601$     & $0.0764/2.2068$                    \\ \hline
        CNN   & \textit{CIFAR-10} & $370.09 \pm 0.18$   & $395.71 \pm 0.55$   & $401.38 \pm 0.16$ & N/A    & $1287.8 \pm 0.26$                     \\ \hline
    \end{tabular}
\end{table*}

\subsection{Feasibility of our Method}
A crucial requirement for any regularization technique is that it should impose minimal impact on the overall training process.
In this respect, CF-Reg introduces an overhead that depends on the time required to find the optimal counterfactual example for each training instance. 
As such, the more sophisticated the counterfactual generator model probed during training the higher would be the time required. However, a more advanced counterfactual generator might provide a more effective regularization. We discuss this trade-off in more details in Section~\ref{sec:discussion}.

Table~\ref{tab:times} presents the average training time ($\pm$ standard deviation) for each model and dataset combination listed in Table~\ref{tab:regularization_comparison}.
We can observe that the higher accuracy achieved by CF-Reg using the score-based counterfactual generator comes with only minimal overhead. However, when applied to deep neural networks with many hidden layers, such as \textit{PreactResNet-18}, the forward derivative computation required for the linearization of the network introduces a more noticeable computational cost, explaining the longer training times in the table.

\subsection{Hyperparameter Sensitivity Analysis}
The proposed counterfactual regularization technique relies on two key hyperparameters: $\alpha$ and $\beta$. The former is intrinsic to the loss formulation defined in (\ref{eq:cf-train}), while the latter is closely tied to the choice of the score-based counterfactual explanation method used.

Figure~\ref{fig:test_alpha_beta} illustrates how the test accuracy of an MLP trained on the \textit{Water Potability} dataset changes for different combinations of $\alpha$ and $\beta$.

\begin{figure}[ht]
    \centering
    \includegraphics[width=0.85\linewidth]{img/test_acc_alpha_beta.png}
    \caption{The test accuracy of an MLP trained on the \textit{Water Potability} dataset, evaluated while varying the weight of our counterfactual regularizer ($\alpha$) for different values of $\beta$.}
    \label{fig:test_alpha_beta}
\end{figure}

We observe that, for a fixed $\beta$, increasing the weight of our counterfactual regularizer ($\alpha$) can slightly improve test accuracy until a sudden drop is noticed for $\alpha > 0.1$.
This behavior was expected, as the impact of our penalty, like any regularization term, can be disruptive if not properly controlled.

Moreover, this finding further demonstrates that our regularization method, CF-Reg, is inherently data-driven. Therefore, it requires specific fine-tuning based on the combination of the model and dataset at hand.

\section{Conclusion}
In this work, we propose a simple yet effective approach, called SMILE, for graph few-shot learning with fewer tasks. Specifically, we introduce a novel dual-level mixup strategy, including within-task and across-task mixup, for enriching the diversity of nodes within each task and the diversity of tasks. Also, we incorporate the degree-based prior information to learn expressive node embeddings. Theoretically, we prove that SMILE effectively enhances the model's generalization performance. Empirically, we conduct extensive experiments on multiple benchmarks and the results suggest that SMILE significantly outperforms other baselines, including both in-domain and cross-domain few-shot settings.

\section*{Acknowledgements}
P. Humbert gratefully acknowledges the Emergence project MARS of Sorbonne Université.


\bibliography{biblio.bib}
\bibliographystyle{apalike}

\newpage

\appendix

\subsection{Lloyd-Max Algorithm}
\label{subsec:Lloyd-Max}
For a given quantization bitwidth $B$ and an operand $\bm{X}$, the Lloyd-Max algorithm finds $2^B$ quantization levels $\{\hat{x}_i\}_{i=1}^{2^B}$ such that quantizing $\bm{X}$ by rounding each scalar in $\bm{X}$ to the nearest quantization level minimizes the quantization MSE. 

The algorithm starts with an initial guess of quantization levels and then iteratively computes quantization thresholds $\{\tau_i\}_{i=1}^{2^B-1}$ and updates quantization levels $\{\hat{x}_i\}_{i=1}^{2^B}$. Specifically, at iteration $n$, thresholds are set to the midpoints of the previous iteration's levels:
\begin{align*}
    \tau_i^{(n)}=\frac{\hat{x}_i^{(n-1)}+\hat{x}_{i+1}^{(n-1)}}2 \text{ for } i=1\ldots 2^B-1
\end{align*}
Subsequently, the quantization levels are re-computed as conditional means of the data regions defined by the new thresholds:
\begin{align*}
    \hat{x}_i^{(n)}=\mathbb{E}\left[ \bm{X} \big| \bm{X}\in [\tau_{i-1}^{(n)},\tau_i^{(n)}] \right] \text{ for } i=1\ldots 2^B
\end{align*}
where to satisfy boundary conditions we have $\tau_0=-\infty$ and $\tau_{2^B}=\infty$. The algorithm iterates the above steps until convergence.

Figure \ref{fig:lm_quant} compares the quantization levels of a $7$-bit floating point (E3M3) quantizer (left) to a $7$-bit Lloyd-Max quantizer (right) when quantizing a layer of weights from the GPT3-126M model at a per-tensor granularity. As shown, the Lloyd-Max quantizer achieves substantially lower quantization MSE. Further, Table \ref{tab:FP7_vs_LM7} shows the superior perplexity achieved by Lloyd-Max quantizers for bitwidths of $7$, $6$ and $5$. The difference between the quantizers is clear at 5 bits, where per-tensor FP quantization incurs a drastic and unacceptable increase in perplexity, while Lloyd-Max quantization incurs a much smaller increase. Nevertheless, we note that even the optimal Lloyd-Max quantizer incurs a notable ($\sim 1.5$) increase in perplexity due to the coarse granularity of quantization. 

\begin{figure}[h]
  \centering
  \includegraphics[width=0.7\linewidth]{sections/figures/LM7_FP7.pdf}
  \caption{\small Quantization levels and the corresponding quantization MSE of Floating Point (left) vs Lloyd-Max (right) Quantizers for a layer of weights in the GPT3-126M model.}
  \label{fig:lm_quant}
\end{figure}

\begin{table}[h]\scriptsize
\begin{center}
\caption{\label{tab:FP7_vs_LM7} \small Comparing perplexity (lower is better) achieved by floating point quantizers and Lloyd-Max quantizers on a GPT3-126M model for the Wikitext-103 dataset.}
\begin{tabular}{c|cc|c}
\hline
 \multirow{2}{*}{\textbf{Bitwidth}} & \multicolumn{2}{|c|}{\textbf{Floating-Point Quantizer}} & \textbf{Lloyd-Max Quantizer} \\
 & Best Format & Wikitext-103 Perplexity & Wikitext-103 Perplexity \\
\hline
7 & E3M3 & 18.32 & 18.27 \\
6 & E3M2 & 19.07 & 18.51 \\
5 & E4M0 & 43.89 & 19.71 \\
\hline
\end{tabular}
\end{center}
\end{table}

\subsection{Proof of Local Optimality of LO-BCQ}
\label{subsec:lobcq_opt_proof}
For a given block $\bm{b}_j$, the quantization MSE during LO-BCQ can be empirically evaluated as $\frac{1}{L_b}\lVert \bm{b}_j- \bm{\hat{b}}_j\rVert^2_2$ where $\bm{\hat{b}}_j$ is computed from equation (\ref{eq:clustered_quantization_definition}) as $C_{f(\bm{b}_j)}(\bm{b}_j)$. Further, for a given block cluster $\mathcal{B}_i$, we compute the quantization MSE as $\frac{1}{|\mathcal{B}_{i}|}\sum_{\bm{b} \in \mathcal{B}_{i}} \frac{1}{L_b}\lVert \bm{b}- C_i^{(n)}(\bm{b})\rVert^2_2$. Therefore, at the end of iteration $n$, we evaluate the overall quantization MSE $J^{(n)}$ for a given operand $\bm{X}$ composed of $N_c$ block clusters as:
\begin{align*}
    \label{eq:mse_iter_n}
    J^{(n)} = \frac{1}{N_c} \sum_{i=1}^{N_c} \frac{1}{|\mathcal{B}_{i}^{(n)}|}\sum_{\bm{v} \in \mathcal{B}_{i}^{(n)}} \frac{1}{L_b}\lVert \bm{b}- B_i^{(n)}(\bm{b})\rVert^2_2
\end{align*}

At the end of iteration $n$, the codebooks are updated from $\mathcal{C}^{(n-1)}$ to $\mathcal{C}^{(n)}$. However, the mapping of a given vector $\bm{b}_j$ to quantizers $\mathcal{C}^{(n)}$ remains as  $f^{(n)}(\bm{b}_j)$. At the next iteration, during the vector clustering step, $f^{(n+1)}(\bm{b}_j)$ finds new mapping of $\bm{b}_j$ to updated codebooks $\mathcal{C}^{(n)}$ such that the quantization MSE over the candidate codebooks is minimized. Therefore, we obtain the following result for $\bm{b}_j$:
\begin{align*}
\frac{1}{L_b}\lVert \bm{b}_j - C_{f^{(n+1)}(\bm{b}_j)}^{(n)}(\bm{b}_j)\rVert^2_2 \le \frac{1}{L_b}\lVert \bm{b}_j - C_{f^{(n)}(\bm{b}_j)}^{(n)}(\bm{b}_j)\rVert^2_2
\end{align*}

That is, quantizing $\bm{b}_j$ at the end of the block clustering step of iteration $n+1$ results in lower quantization MSE compared to quantizing at the end of iteration $n$. Since this is true for all $\bm{b} \in \bm{X}$, we assert the following:
\begin{equation}
\begin{split}
\label{eq:mse_ineq_1}
    \tilde{J}^{(n+1)} &= \frac{1}{N_c} \sum_{i=1}^{N_c} \frac{1}{|\mathcal{B}_{i}^{(n+1)}|}\sum_{\bm{b} \in \mathcal{B}_{i}^{(n+1)}} \frac{1}{L_b}\lVert \bm{b} - C_i^{(n)}(b)\rVert^2_2 \le J^{(n)}
\end{split}
\end{equation}
where $\tilde{J}^{(n+1)}$ is the the quantization MSE after the vector clustering step at iteration $n+1$.

Next, during the codebook update step (\ref{eq:quantizers_update}) at iteration $n+1$, the per-cluster codebooks $\mathcal{C}^{(n)}$ are updated to $\mathcal{C}^{(n+1)}$ by invoking the Lloyd-Max algorithm \citep{Lloyd}. We know that for any given value distribution, the Lloyd-Max algorithm minimizes the quantization MSE. Therefore, for a given vector cluster $\mathcal{B}_i$ we obtain the following result:

\begin{equation}
    \frac{1}{|\mathcal{B}_{i}^{(n+1)}|}\sum_{\bm{b} \in \mathcal{B}_{i}^{(n+1)}} \frac{1}{L_b}\lVert \bm{b}- C_i^{(n+1)}(\bm{b})\rVert^2_2 \le \frac{1}{|\mathcal{B}_{i}^{(n+1)}|}\sum_{\bm{b} \in \mathcal{B}_{i}^{(n+1)}} \frac{1}{L_b}\lVert \bm{b}- C_i^{(n)}(\bm{b})\rVert^2_2
\end{equation}

The above equation states that quantizing the given block cluster $\mathcal{B}_i$ after updating the associated codebook from $C_i^{(n)}$ to $C_i^{(n+1)}$ results in lower quantization MSE. Since this is true for all the block clusters, we derive the following result: 
\begin{equation}
\begin{split}
\label{eq:mse_ineq_2}
     J^{(n+1)} &= \frac{1}{N_c} \sum_{i=1}^{N_c} \frac{1}{|\mathcal{B}_{i}^{(n+1)}|}\sum_{\bm{b} \in \mathcal{B}_{i}^{(n+1)}} \frac{1}{L_b}\lVert \bm{b}- C_i^{(n+1)}(\bm{b})\rVert^2_2  \le \tilde{J}^{(n+1)}   
\end{split}
\end{equation}

Following (\ref{eq:mse_ineq_1}) and (\ref{eq:mse_ineq_2}), we find that the quantization MSE is non-increasing for each iteration, that is, $J^{(1)} \ge J^{(2)} \ge J^{(3)} \ge \ldots \ge J^{(M)}$ where $M$ is the maximum number of iterations. 
%Therefore, we can say that if the algorithm converges, then it must be that it has converged to a local minimum. 
\hfill $\blacksquare$


\begin{figure}
    \begin{center}
    \includegraphics[width=0.5\textwidth]{sections//figures/mse_vs_iter.pdf}
    \end{center}
    \caption{\small NMSE vs iterations during LO-BCQ compared to other block quantization proposals}
    \label{fig:nmse_vs_iter}
\end{figure}

Figure \ref{fig:nmse_vs_iter} shows the empirical convergence of LO-BCQ across several block lengths and number of codebooks. Also, the MSE achieved by LO-BCQ is compared to baselines such as MXFP and VSQ. As shown, LO-BCQ converges to a lower MSE than the baselines. Further, we achieve better convergence for larger number of codebooks ($N_c$) and for a smaller block length ($L_b$), both of which increase the bitwidth of BCQ (see Eq \ref{eq:bitwidth_bcq}).


\subsection{Additional Accuracy Results}
%Table \ref{tab:lobcq_config} lists the various LOBCQ configurations and their corresponding bitwidths.
\begin{table}
\setlength{\tabcolsep}{4.75pt}
\begin{center}
\caption{\label{tab:lobcq_config} Various LO-BCQ configurations and their bitwidths.}
\begin{tabular}{|c||c|c|c|c||c|c||c|} 
\hline
 & \multicolumn{4}{|c||}{$L_b=8$} & \multicolumn{2}{|c||}{$L_b=4$} & $L_b=2$ \\
 \hline
 \backslashbox{$L_A$\kern-1em}{\kern-1em$N_c$} & 2 & 4 & 8 & 16 & 2 & 4 & 2 \\
 \hline
 64 & 4.25 & 4.375 & 4.5 & 4.625 & 4.375 & 4.625 & 4.625\\
 \hline
 32 & 4.375 & 4.5 & 4.625& 4.75 & 4.5 & 4.75 & 4.75 \\
 \hline
 16 & 4.625 & 4.75& 4.875 & 5 & 4.75 & 5 & 5 \\
 \hline
\end{tabular}
\end{center}
\end{table}

%\subsection{Perplexity achieved by various LO-BCQ configurations on Wikitext-103 dataset}

\begin{table} \centering
\begin{tabular}{|c||c|c|c|c||c|c||c|} 
\hline
 $L_b \rightarrow$& \multicolumn{4}{c||}{8} & \multicolumn{2}{c||}{4} & 2\\
 \hline
 \backslashbox{$L_A$\kern-1em}{\kern-1em$N_c$} & 2 & 4 & 8 & 16 & 2 & 4 & 2  \\
 %$N_c \rightarrow$ & 2 & 4 & 8 & 16 & 2 & 4 & 2 \\
 \hline
 \hline
 \multicolumn{8}{c}{GPT3-1.3B (FP32 PPL = 9.98)} \\ 
 \hline
 \hline
 64 & 10.40 & 10.23 & 10.17 & 10.15 &  10.28 & 10.18 & 10.19 \\
 \hline
 32 & 10.25 & 10.20 & 10.15 & 10.12 &  10.23 & 10.17 & 10.17 \\
 \hline
 16 & 10.22 & 10.16 & 10.10 & 10.09 &  10.21 & 10.14 & 10.16 \\
 \hline
  \hline
 \multicolumn{8}{c}{GPT3-8B (FP32 PPL = 7.38)} \\ 
 \hline
 \hline
 64 & 7.61 & 7.52 & 7.48 &  7.47 &  7.55 &  7.49 & 7.50 \\
 \hline
 32 & 7.52 & 7.50 & 7.46 &  7.45 &  7.52 &  7.48 & 7.48  \\
 \hline
 16 & 7.51 & 7.48 & 7.44 &  7.44 &  7.51 &  7.49 & 7.47  \\
 \hline
\end{tabular}
\caption{\label{tab:ppl_gpt3_abalation} Wikitext-103 perplexity across GPT3-1.3B and 8B models.}
\end{table}

\begin{table} \centering
\begin{tabular}{|c||c|c|c|c||} 
\hline
 $L_b \rightarrow$& \multicolumn{4}{c||}{8}\\
 \hline
 \backslashbox{$L_A$\kern-1em}{\kern-1em$N_c$} & 2 & 4 & 8 & 16 \\
 %$N_c \rightarrow$ & 2 & 4 & 8 & 16 & 2 & 4 & 2 \\
 \hline
 \hline
 \multicolumn{5}{|c|}{Llama2-7B (FP32 PPL = 5.06)} \\ 
 \hline
 \hline
 64 & 5.31 & 5.26 & 5.19 & 5.18  \\
 \hline
 32 & 5.23 & 5.25 & 5.18 & 5.15  \\
 \hline
 16 & 5.23 & 5.19 & 5.16 & 5.14  \\
 \hline
 \multicolumn{5}{|c|}{Nemotron4-15B (FP32 PPL = 5.87)} \\ 
 \hline
 \hline
 64  & 6.3 & 6.20 & 6.13 & 6.08  \\
 \hline
 32  & 6.24 & 6.12 & 6.07 & 6.03  \\
 \hline
 16  & 6.12 & 6.14 & 6.04 & 6.02  \\
 \hline
 \multicolumn{5}{|c|}{Nemotron4-340B (FP32 PPL = 3.48)} \\ 
 \hline
 \hline
 64 & 3.67 & 3.62 & 3.60 & 3.59 \\
 \hline
 32 & 3.63 & 3.61 & 3.59 & 3.56 \\
 \hline
 16 & 3.61 & 3.58 & 3.57 & 3.55 \\
 \hline
\end{tabular}
\caption{\label{tab:ppl_llama7B_nemo15B} Wikitext-103 perplexity compared to FP32 baseline in Llama2-7B and Nemotron4-15B, 340B models}
\end{table}

%\subsection{Perplexity achieved by various LO-BCQ configurations on MMLU dataset}


\begin{table} \centering
\begin{tabular}{|c||c|c|c|c||c|c|c|c|} 
\hline
 $L_b \rightarrow$& \multicolumn{4}{c||}{8} & \multicolumn{4}{c||}{8}\\
 \hline
 \backslashbox{$L_A$\kern-1em}{\kern-1em$N_c$} & 2 & 4 & 8 & 16 & 2 & 4 & 8 & 16  \\
 %$N_c \rightarrow$ & 2 & 4 & 8 & 16 & 2 & 4 & 2 \\
 \hline
 \hline
 \multicolumn{5}{|c|}{Llama2-7B (FP32 Accuracy = 45.8\%)} & \multicolumn{4}{|c|}{Llama2-70B (FP32 Accuracy = 69.12\%)} \\ 
 \hline
 \hline
 64 & 43.9 & 43.4 & 43.9 & 44.9 & 68.07 & 68.27 & 68.17 & 68.75 \\
 \hline
 32 & 44.5 & 43.8 & 44.9 & 44.5 & 68.37 & 68.51 & 68.35 & 68.27  \\
 \hline
 16 & 43.9 & 42.7 & 44.9 & 45 & 68.12 & 68.77 & 68.31 & 68.59  \\
 \hline
 \hline
 \multicolumn{5}{|c|}{GPT3-22B (FP32 Accuracy = 38.75\%)} & \multicolumn{4}{|c|}{Nemotron4-15B (FP32 Accuracy = 64.3\%)} \\ 
 \hline
 \hline
 64 & 36.71 & 38.85 & 38.13 & 38.92 & 63.17 & 62.36 & 63.72 & 64.09 \\
 \hline
 32 & 37.95 & 38.69 & 39.45 & 38.34 & 64.05 & 62.30 & 63.8 & 64.33  \\
 \hline
 16 & 38.88 & 38.80 & 38.31 & 38.92 & 63.22 & 63.51 & 63.93 & 64.43  \\
 \hline
\end{tabular}
\caption{\label{tab:mmlu_abalation} Accuracy on MMLU dataset across GPT3-22B, Llama2-7B, 70B and Nemotron4-15B models.}
\end{table}


%\subsection{Perplexity achieved by various LO-BCQ configurations on LM evaluation harness}

\begin{table} \centering
\begin{tabular}{|c||c|c|c|c||c|c|c|c|} 
\hline
 $L_b \rightarrow$& \multicolumn{4}{c||}{8} & \multicolumn{4}{c||}{8}\\
 \hline
 \backslashbox{$L_A$\kern-1em}{\kern-1em$N_c$} & 2 & 4 & 8 & 16 & 2 & 4 & 8 & 16  \\
 %$N_c \rightarrow$ & 2 & 4 & 8 & 16 & 2 & 4 & 2 \\
 \hline
 \hline
 \multicolumn{5}{|c|}{Race (FP32 Accuracy = 37.51\%)} & \multicolumn{4}{|c|}{Boolq (FP32 Accuracy = 64.62\%)} \\ 
 \hline
 \hline
 64 & 36.94 & 37.13 & 36.27 & 37.13 & 63.73 & 62.26 & 63.49 & 63.36 \\
 \hline
 32 & 37.03 & 36.36 & 36.08 & 37.03 & 62.54 & 63.51 & 63.49 & 63.55  \\
 \hline
 16 & 37.03 & 37.03 & 36.46 & 37.03 & 61.1 & 63.79 & 63.58 & 63.33  \\
 \hline
 \hline
 \multicolumn{5}{|c|}{Winogrande (FP32 Accuracy = 58.01\%)} & \multicolumn{4}{|c|}{Piqa (FP32 Accuracy = 74.21\%)} \\ 
 \hline
 \hline
 64 & 58.17 & 57.22 & 57.85 & 58.33 & 73.01 & 73.07 & 73.07 & 72.80 \\
 \hline
 32 & 59.12 & 58.09 & 57.85 & 58.41 & 73.01 & 73.94 & 72.74 & 73.18  \\
 \hline
 16 & 57.93 & 58.88 & 57.93 & 58.56 & 73.94 & 72.80 & 73.01 & 73.94  \\
 \hline
\end{tabular}
\caption{\label{tab:mmlu_abalation} Accuracy on LM evaluation harness tasks on GPT3-1.3B model.}
\end{table}

\begin{table} \centering
\begin{tabular}{|c||c|c|c|c||c|c|c|c|} 
\hline
 $L_b \rightarrow$& \multicolumn{4}{c||}{8} & \multicolumn{4}{c||}{8}\\
 \hline
 \backslashbox{$L_A$\kern-1em}{\kern-1em$N_c$} & 2 & 4 & 8 & 16 & 2 & 4 & 8 & 16  \\
 %$N_c \rightarrow$ & 2 & 4 & 8 & 16 & 2 & 4 & 2 \\
 \hline
 \hline
 \multicolumn{5}{|c|}{Race (FP32 Accuracy = 41.34\%)} & \multicolumn{4}{|c|}{Boolq (FP32 Accuracy = 68.32\%)} \\ 
 \hline
 \hline
 64 & 40.48 & 40.10 & 39.43 & 39.90 & 69.20 & 68.41 & 69.45 & 68.56 \\
 \hline
 32 & 39.52 & 39.52 & 40.77 & 39.62 & 68.32 & 67.43 & 68.17 & 69.30  \\
 \hline
 16 & 39.81 & 39.71 & 39.90 & 40.38 & 68.10 & 66.33 & 69.51 & 69.42  \\
 \hline
 \hline
 \multicolumn{5}{|c|}{Winogrande (FP32 Accuracy = 67.88\%)} & \multicolumn{4}{|c|}{Piqa (FP32 Accuracy = 78.78\%)} \\ 
 \hline
 \hline
 64 & 66.85 & 66.61 & 67.72 & 67.88 & 77.31 & 77.42 & 77.75 & 77.64 \\
 \hline
 32 & 67.25 & 67.72 & 67.72 & 67.00 & 77.31 & 77.04 & 77.80 & 77.37  \\
 \hline
 16 & 68.11 & 68.90 & 67.88 & 67.48 & 77.37 & 78.13 & 78.13 & 77.69  \\
 \hline
\end{tabular}
\caption{\label{tab:mmlu_abalation} Accuracy on LM evaluation harness tasks on GPT3-8B model.}
\end{table}

\begin{table} \centering
\begin{tabular}{|c||c|c|c|c||c|c|c|c|} 
\hline
 $L_b \rightarrow$& \multicolumn{4}{c||}{8} & \multicolumn{4}{c||}{8}\\
 \hline
 \backslashbox{$L_A$\kern-1em}{\kern-1em$N_c$} & 2 & 4 & 8 & 16 & 2 & 4 & 8 & 16  \\
 %$N_c \rightarrow$ & 2 & 4 & 8 & 16 & 2 & 4 & 2 \\
 \hline
 \hline
 \multicolumn{5}{|c|}{Race (FP32 Accuracy = 40.67\%)} & \multicolumn{4}{|c|}{Boolq (FP32 Accuracy = 76.54\%)} \\ 
 \hline
 \hline
 64 & 40.48 & 40.10 & 39.43 & 39.90 & 75.41 & 75.11 & 77.09 & 75.66 \\
 \hline
 32 & 39.52 & 39.52 & 40.77 & 39.62 & 76.02 & 76.02 & 75.96 & 75.35  \\
 \hline
 16 & 39.81 & 39.71 & 39.90 & 40.38 & 75.05 & 73.82 & 75.72 & 76.09  \\
 \hline
 \hline
 \multicolumn{5}{|c|}{Winogrande (FP32 Accuracy = 70.64\%)} & \multicolumn{4}{|c|}{Piqa (FP32 Accuracy = 79.16\%)} \\ 
 \hline
 \hline
 64 & 69.14 & 70.17 & 70.17 & 70.56 & 78.24 & 79.00 & 78.62 & 78.73 \\
 \hline
 32 & 70.96 & 69.69 & 71.27 & 69.30 & 78.56 & 79.49 & 79.16 & 78.89  \\
 \hline
 16 & 71.03 & 69.53 & 69.69 & 70.40 & 78.13 & 79.16 & 79.00 & 79.00  \\
 \hline
\end{tabular}
\caption{\label{tab:mmlu_abalation} Accuracy on LM evaluation harness tasks on GPT3-22B model.}
\end{table}

\begin{table} \centering
\begin{tabular}{|c||c|c|c|c||c|c|c|c|} 
\hline
 $L_b \rightarrow$& \multicolumn{4}{c||}{8} & \multicolumn{4}{c||}{8}\\
 \hline
 \backslashbox{$L_A$\kern-1em}{\kern-1em$N_c$} & 2 & 4 & 8 & 16 & 2 & 4 & 8 & 16  \\
 %$N_c \rightarrow$ & 2 & 4 & 8 & 16 & 2 & 4 & 2 \\
 \hline
 \hline
 \multicolumn{5}{|c|}{Race (FP32 Accuracy = 44.4\%)} & \multicolumn{4}{|c|}{Boolq (FP32 Accuracy = 79.29\%)} \\ 
 \hline
 \hline
 64 & 42.49 & 42.51 & 42.58 & 43.45 & 77.58 & 77.37 & 77.43 & 78.1 \\
 \hline
 32 & 43.35 & 42.49 & 43.64 & 43.73 & 77.86 & 75.32 & 77.28 & 77.86  \\
 \hline
 16 & 44.21 & 44.21 & 43.64 & 42.97 & 78.65 & 77 & 76.94 & 77.98  \\
 \hline
 \hline
 \multicolumn{5}{|c|}{Winogrande (FP32 Accuracy = 69.38\%)} & \multicolumn{4}{|c|}{Piqa (FP32 Accuracy = 78.07\%)} \\ 
 \hline
 \hline
 64 & 68.9 & 68.43 & 69.77 & 68.19 & 77.09 & 76.82 & 77.09 & 77.86 \\
 \hline
 32 & 69.38 & 68.51 & 68.82 & 68.90 & 78.07 & 76.71 & 78.07 & 77.86  \\
 \hline
 16 & 69.53 & 67.09 & 69.38 & 68.90 & 77.37 & 77.8 & 77.91 & 77.69  \\
 \hline
\end{tabular}
\caption{\label{tab:mmlu_abalation} Accuracy on LM evaluation harness tasks on Llama2-7B model.}
\end{table}

\begin{table} \centering
\begin{tabular}{|c||c|c|c|c||c|c|c|c|} 
\hline
 $L_b \rightarrow$& \multicolumn{4}{c||}{8} & \multicolumn{4}{c||}{8}\\
 \hline
 \backslashbox{$L_A$\kern-1em}{\kern-1em$N_c$} & 2 & 4 & 8 & 16 & 2 & 4 & 8 & 16  \\
 %$N_c \rightarrow$ & 2 & 4 & 8 & 16 & 2 & 4 & 2 \\
 \hline
 \hline
 \multicolumn{5}{|c|}{Race (FP32 Accuracy = 48.8\%)} & \multicolumn{4}{|c|}{Boolq (FP32 Accuracy = 85.23\%)} \\ 
 \hline
 \hline
 64 & 49.00 & 49.00 & 49.28 & 48.71 & 82.82 & 84.28 & 84.03 & 84.25 \\
 \hline
 32 & 49.57 & 48.52 & 48.33 & 49.28 & 83.85 & 84.46 & 84.31 & 84.93  \\
 \hline
 16 & 49.85 & 49.09 & 49.28 & 48.99 & 85.11 & 84.46 & 84.61 & 83.94  \\
 \hline
 \hline
 \multicolumn{5}{|c|}{Winogrande (FP32 Accuracy = 79.95\%)} & \multicolumn{4}{|c|}{Piqa (FP32 Accuracy = 81.56\%)} \\ 
 \hline
 \hline
 64 & 78.77 & 78.45 & 78.37 & 79.16 & 81.45 & 80.69 & 81.45 & 81.5 \\
 \hline
 32 & 78.45 & 79.01 & 78.69 & 80.66 & 81.56 & 80.58 & 81.18 & 81.34  \\
 \hline
 16 & 79.95 & 79.56 & 79.79 & 79.72 & 81.28 & 81.66 & 81.28 & 80.96  \\
 \hline
\end{tabular}
\caption{\label{tab:mmlu_abalation} Accuracy on LM evaluation harness tasks on Llama2-70B model.}
\end{table}

%\section{MSE Studies}
%\textcolor{red}{TODO}


\subsection{Number Formats and Quantization Method}
\label{subsec:numFormats_quantMethod}
\subsubsection{Integer Format}
An $n$-bit signed integer (INT) is typically represented with a 2s-complement format \citep{yao2022zeroquant,xiao2023smoothquant,dai2021vsq}, where the most significant bit denotes the sign.

\subsubsection{Floating Point Format}
An $n$-bit signed floating point (FP) number $x$ comprises of a 1-bit sign ($x_{\mathrm{sign}}$), $B_m$-bit mantissa ($x_{\mathrm{mant}}$) and $B_e$-bit exponent ($x_{\mathrm{exp}}$) such that $B_m+B_e=n-1$. The associated constant exponent bias ($E_{\mathrm{bias}}$) is computed as $(2^{{B_e}-1}-1)$. We denote this format as $E_{B_e}M_{B_m}$.  

\subsubsection{Quantization Scheme}
\label{subsec:quant_method}
A quantization scheme dictates how a given unquantized tensor is converted to its quantized representation. We consider FP formats for the purpose of illustration. Given an unquantized tensor $\bm{X}$ and an FP format $E_{B_e}M_{B_m}$, we first, we compute the quantization scale factor $s_X$ that maps the maximum absolute value of $\bm{X}$ to the maximum quantization level of the $E_{B_e}M_{B_m}$ format as follows:
\begin{align}
\label{eq:sf}
    s_X = \frac{\mathrm{max}(|\bm{X}|)}{\mathrm{max}(E_{B_e}M_{B_m})}
\end{align}
In the above equation, $|\cdot|$ denotes the absolute value function.

Next, we scale $\bm{X}$ by $s_X$ and quantize it to $\hat{\bm{X}}$ by rounding it to the nearest quantization level of $E_{B_e}M_{B_m}$ as:

\begin{align}
\label{eq:tensor_quant}
    \hat{\bm{X}} = \text{round-to-nearest}\left(\frac{\bm{X}}{s_X}, E_{B_e}M_{B_m}\right)
\end{align}

We perform dynamic max-scaled quantization \citep{wu2020integer}, where the scale factor $s$ for activations is dynamically computed during runtime.

\subsection{Vector Scaled Quantization}
\begin{wrapfigure}{r}{0.35\linewidth}
  \centering
  \includegraphics[width=\linewidth]{sections/figures/vsquant.jpg}
  \caption{\small Vectorwise decomposition for per-vector scaled quantization (VSQ \citep{dai2021vsq}).}
  \label{fig:vsquant}
\end{wrapfigure}
During VSQ \citep{dai2021vsq}, the operand tensors are decomposed into 1D vectors in a hardware friendly manner as shown in Figure \ref{fig:vsquant}. Since the decomposed tensors are used as operands in matrix multiplications during inference, it is beneficial to perform this decomposition along the reduction dimension of the multiplication. The vectorwise quantization is performed similar to tensorwise quantization described in Equations \ref{eq:sf} and \ref{eq:tensor_quant}, where a scale factor $s_v$ is required for each vector $\bm{v}$ that maps the maximum absolute value of that vector to the maximum quantization level. While smaller vector lengths can lead to larger accuracy gains, the associated memory and computational overheads due to the per-vector scale factors increases. To alleviate these overheads, VSQ \citep{dai2021vsq} proposed a second level quantization of the per-vector scale factors to unsigned integers, while MX \citep{rouhani2023shared} quantizes them to integer powers of 2 (denoted as $2^{INT}$).

\subsubsection{MX Format}
The MX format proposed in \citep{rouhani2023microscaling} introduces the concept of sub-block shifting. For every two scalar elements of $b$-bits each, there is a shared exponent bit. The value of this exponent bit is determined through an empirical analysis that targets minimizing quantization MSE. We note that the FP format $E_{1}M_{b}$ is strictly better than MX from an accuracy perspective since it allocates a dedicated exponent bit to each scalar as opposed to sharing it across two scalars. Therefore, we conservatively bound the accuracy of a $b+2$-bit signed MX format with that of a $E_{1}M_{b}$ format in our comparisons. For instance, we use E1M2 format as a proxy for MX4.

\begin{figure}
    \centering
    \includegraphics[width=1\linewidth]{sections//figures/BlockFormats.pdf}
    \caption{\small Comparing LO-BCQ to MX format.}
    \label{fig:block_formats}
\end{figure}

Figure \ref{fig:block_formats} compares our $4$-bit LO-BCQ block format to MX \citep{rouhani2023microscaling}. As shown, both LO-BCQ and MX decompose a given operand tensor into block arrays and each block array into blocks. Similar to MX, we find that per-block quantization ($L_b < L_A$) leads to better accuracy due to increased flexibility. While MX achieves this through per-block $1$-bit micro-scales, we associate a dedicated codebook to each block through a per-block codebook selector. Further, MX quantizes the per-block array scale-factor to E8M0 format without per-tensor scaling. In contrast during LO-BCQ, we find that per-tensor scaling combined with quantization of per-block array scale-factor to E4M3 format results in superior inference accuracy across models. 

\section{Detailed implementation of the empirical $(1-\alpha)$-QAE minimization}\label{sec:optim_append}

As explain in Section \ref{sec:optim_emp_QAE}, to solve Problem \eqref{eq:optim_quantile} we use a gradient descent strategy. However, because the empirical quantile is not differentiable, we replace $\widehat{Q}$ in Problem \eqref{eq:optim_quantile} by the following smooth approximation:
%
\begin{equation*}
	\widetilde{Q}_{\varepsilon}(q; (\ell(\theta;Z_i))_{i\in\calD^{lrn}}) = \inf \{t \,:\, \widetilde{F}_{\varepsilon}(t, \theta) \geq q \} \; ,
\end{equation*}
%
where $\widetilde{F}_{\varepsilon}$ is an approximation of the empirical distribution of the loss-values $(\ell(\theta;Z_i))_{i\in\calD^{lrn}}$ defined for $\varepsilon > 0$ by
\begin{align*}
	\widetilde{F}_{\varepsilon}(t, \theta) = \sum_{i\in\calD^{lrn}} \Gamma_{\varepsilon}(\ell(\theta;Z_i) - t) \; ,
\end{align*}
%
with
%
\begin{align*}
	\Gamma_{\varepsilon}(z) = \left\{
	\begin{array}{ll}
		1 & z \leq - \varepsilon \\
		\gamma_{\varepsilon}(z) & -\varepsilon < z < \varepsilon \\
		0 & z \geq \varepsilon
	\end{array}
	\right. \; ,
\end{align*}
and $\gamma_{\varepsilon} : [-\varepsilon, \varepsilon] \longrightarrow [0, 1]$ a symmetric and strictly decreasing function such that it makes $\Gamma_{\varepsilon}$ differentiable. One possible choice for $\gamma_{\varepsilon}$ is given in \citep[Eq. (2.6)]{pena2020solving}:
%
\begin{equation}\label{eq:gamma_epsi}
	\gamma_{\varepsilon}(z) = \dfrac{15}{16}\left(-\dfrac{1}{5} \left(\dfrac{z}{\varepsilon}\right)^5 + \dfrac{2}{3}\left(\dfrac{z}{\varepsilon}\right)^3 - \dfrac{z}{\varepsilon} + \dfrac{8}{15} \right) \; .
\end{equation}
%

For a given $q$ and $\varepsilon > 0$, under some assumptions on the loss (see \citet{pena2020solving}), the implicit function theorem implies that:
%
\begin{align}\label{eq:grad_epsi}
	\nabla_{\theta} [\widetilde{Q}_{\varepsilon}(q; (\ell(\theta;Z_i))_{i\in\calD^{lrn}})] &= \dfrac{\sum_{i\in\calD^{lrn}} \Gamma'_{\varepsilon}(\ell(\theta;Z_i) - \widetilde{Q}_{\varepsilon}(q; (\ell(\theta;Z_i))_{i\in\calD^{lrn}})) \cdot \nabla_{\theta} \ell(\theta;Z_i)}{\sum_{i\in\calD^{lrn}} \Gamma'_{\varepsilon}(\ell(\theta;Z_i) - \widetilde{Q}_{\varepsilon}(q; (\ell(\theta;Z_i))_{i\in\calD^{lrn}}))} \; ,
\end{align}
%
where $\nabla_{\theta}$ denotes the gradient with respect to $\theta$ and $\Gamma'$ is the differential of $\Gamma$. We can therefore use a gradient descent algorithm to solve an approximation of the QAE Problem \eqref{eq:optim_quantile} given by:
%
\begin{align*}
	&\min_{\theta} \; \widetilde{Q}_{\varepsilon}(1-\alpha;\{\ell(\theta;Z_i)\}_{i\in\calD^{lrn}}) \; .
\end{align*}
%
To this end, starting from an initial guess $\widetilde{\theta}_1$, we simply make the iterates:
%
\begin{align*}
	\widetilde{\theta}_{k+1} = \widetilde{\theta}_{k} - \eta_k \nabla_{\theta} [\widetilde{Q}_{\varepsilon}(1-\alpha; (\ell(\widetilde{\theta}_k;Z_i))_{i\in\calD^{lrn}})] \; ,
\end{align*}
where $\eta_k > 0$ is the step-size. The full procedure is summary in Algorithm \ref{alg:min_quantile} when $\gamma_{\varepsilon}$ is an in Eq. \eqref{eq:gamma_epsi}.
\begin{algorithm}
	\caption{Gradient descent to solve the QAE problem (step 1 of \method~ and \methodAD)}
	\label{alg:min_quantile}
	\begin{algorithmic}[1]
				\State \textbf{Inputs:} $\varepsilon$, $\widetilde{\theta}_1$, $n_{iter}$, $(\eta_k)_{1 \leq k \leq n_{iter}}, \alpha$
				\For{$k = 1, \ldots, n_{iter}$}
				\State $A \gets \widetilde{Q}_{\varepsilon}(1-\alpha; (\ell(\widetilde{\theta}_k;Z_i))_{i\in\calD^{lrn}}))$
				\For{$i \in \calD^{lrn}$}
				\State $B_i \gets \Gamma'_{\varepsilon}(\ell(\widetilde{\theta}_{k};Z_i) - A) = -\dfrac{15}{16}\left(\Big(\varepsilon^2 - (\ell(\widetilde{\theta}_{k};Z_i) - A)^2\Big)^2/\varepsilon^5\right) \cdot \1\{-\varepsilon < (\ell(\widetilde{\theta}_{k};Z_i) - A) < \varepsilon\}$
				\State $C_i \gets \nabla_{\theta} \ell(\theta;Z_i)$
				\EndFor
				\State $\widetilde{\theta}_{k+1} \gets \widetilde{\theta}_{k} - \eta_k \cdot \sum_{i} (B_i C_i) / \sum_{i} B_i$
				\EndFor
				\State \textbf{Output:} $\widetilde{\theta}_{n_{iter}+1}$
			\end{algorithmic}
\end{algorithm}

\begin{remark}
	In our setting, $\ell$ is not differentiable because of the absolute value function. In practice, we therefore replace the gradient by a subdifferential (this is what we do in the experiments). Another possibility could be to replace the absolute value function with a smooth approximation, such as the Huber loss \citep{huber1964}. Furthermore, as also done in \citet{luo2022empirical}, in Eq \eqref{eq:grad_epsi} we replace $\widetilde{Q}_{\varepsilon}(q; (\ell(\theta;Z_i))_{i\in\calD^{lrn}})$ by the empirical quantile for computation efficiency.
	%While in practice replacing the gradient by a subdifferential works well\todo{Dire quelque part que c'est ce qu'on a fait dans les xps?} (this is what we do in the experiments), the implicit function theorem and the following theoretical\todo{Ces resultats on disparus ? Si tu ne veux pas les mettre, mentionner que la convergence est obtenue direct car c'est une GD ?} results are only true for differentiable (and smooth) loss functions. In order to fit with the theory, a possibility is to follow a similar strategy as above, by replacing the absolute value function with a smooth approximation, such as the Huber loss \citep{huber1964}.
\end{remark}

\begin{remark}(Link with other formulations)
	Problem \eqref{eq:optim_quantile} is in fact similar to the \textit{single chance constraint problem} (see e.g. \citep{curtis2018sequential}). It can also be reformulated as the following  bi-level optimization problem: 
	%
	\begin{align*}
		\min_{\theta} &\; t(\theta)
		\quad \text{s.t.} \quad t(\theta) = \arg\min_{t} \sum_{i\in\calD^{lrn}} \rho_{1-\alpha}(\ell(\theta;Z_i) - t) \; .
	\end{align*}
	%
	where $\rho_{1-\alpha}$ is the pinball loss. Indeed, from \cite{koenker1978regression, biau2011sequential} we know that $t(\theta) = \widehat{Q}(1-\alpha;\{\ell(\theta;Z_i)\}_{i\in\calD^{lrn}})$.
\end{remark}
%\todo{Je propose de le faire pour une version arxiv par exemple car on a pas fait d'xp avec. Ou alors on en parle pour "future work" ?}

%\section{Detailed implementation of step 1 of \method}
%
%Recall the setting presented in the main paper for step 1 of \method. We assumed that $f\in\calF$ was parametrized by $\theta \in\Theta$, and for the sake of generality, we considered the minimization of $\widehat{Q}(1-\alpha;\{\ell(\theta;Z_i)\}_{i=1}^n)$ with respect to $\theta$. Here, $\ell:\Theta\times\calZ\rightarrow\IR$ is a loss function, taking as input a parameter $\theta$ and a data point $Z_i$. In the setting of \texttt{EffOrt}, $Z_i=(X_i,Y_i)$, $\ell(\theta;Z_i)=|Y_i-f_\theta(X_i)|$, and its first step is then equivalent to:
%
%\begin{align} \label{eq_appen:optim_quantile}
%	%	&\hat{\theta} = \underset{\theta \in \Theta}{\argmin}  \; \widehat{Q}(1-\alpha;\{|Y_i-f_{\theta}(X_i)|\}_{i\in\calD^{lrn}}) \; . \\
%	&\min_{\theta} \; \widehat{Q}(1-\alpha;\{\ell(\theta;Z_i)\}_{i\in\calD^{lrn}}) \; .
%\end{align}
%%
%To solve this problem, one natural idea is to use a gradient descent algorithm on the empirical quantile function $\widehat{Q}(1-\alpha;\{\ell(\theta;Z_i)\}_{i\in\calD^{lrn}})$. However, this function is not differentiable in $\theta$. We therefore follow the strategy of \citep{pena2020solving} and consider a smooth approximation of it. More precisely, given the loss-values $(\ell(\theta;Z_i))_{i\in\calD^{lrn}}$, we first approximate their empirical cdf $\widehat{F}(t, \theta) := \sum_{i\in\calD^{lrn}} \1\{ \ell(\theta;Z_i) \leq t\}$ by another function $\widetilde{F}_{\varepsilon}$ where the indicator is replaced by a smooth version of it, i.e.:
%%
%\begin{align*}
%	\widetilde{F}_{\varepsilon}(t, \theta) = \sum_{i\in\calD^{lrn}} \Gamma_{\varepsilon}(\ell(\theta;Z_i) - t) \; ,
%\end{align*}
%%
%where $\varepsilon > 0$ is a parameter of the approximation:
%%
%\begin{align*}
%	\Gamma_{\varepsilon}(y) = \left\{
%	\begin{array}{ll}
%		1 & y \leq - \varepsilon \\
%		\gamma_{\varepsilon}(y) & -\varepsilon < y < \varepsilon \\
%		0 & y \geq \varepsilon
%	\end{array}
%	\right. \; ,
%\end{align*}
%and $\gamma_{\varepsilon} : [-\varepsilon, \varepsilon] \longrightarrow [0, 1]$ is a symmetric and strictly decreasing function such that it makes $\Gamma_{\varepsilon}$ differentiable. One possible choice for $\gamma_{\varepsilon}$ is given in \citep[Eq. (2.6)]{pena2020solving}. Then, we define the smooth empirical quantile function by:
%%
%\begin{equation}\label{eq_appen:smooth_quant}
%	\psi_{\varepsilon}(\theta) := \widetilde{Q}_{\varepsilon}(q; (\ell(\theta;Z_i))_{i\in\calD^{lrn}}) = \inf \{t \,:\, \widetilde{F}(t, \theta) \geq q \} \; .
%\end{equation}
%%
%For a given $q$ and $\varepsilon > 0$, if the loss function $\ell(\cdot)$ is differentiable, the implicit function theorem \pie{(il faut Lemma 2.1 de \citep{pena2020solving}) ??} implies that: %\citep{luo2022empirical}:
%%
%\begin{align*}
%	\nabla_{\theta} [\widetilde{Q}_{\varepsilon}(q; (\ell(\theta;Z_i))_{i\in\calD^{lrn}})] &= \dfrac{\sum_{i\in\calD^{lrn}} \Gamma'_{\varepsilon}(\ell(\theta;Z_i) - \widetilde{Q}_{\varepsilon}(q; (\ell(\theta;Z_i))_{i\in\calD^{lrn}})) \cdot \nabla_{\theta} \ell(\theta;Z_i)}{\sum_{i\in\calD^{lrn}} \Gamma'_{\varepsilon}(\ell(\theta;Z_i) - \widetilde{Q}_{\varepsilon}(q; (\ell(\theta;Z_i))_{i\in\calD^{lrn}}))} \; ,
%\end{align*}
%%
%where $\nabla_{\theta}$ denotes the gradient with respect to $\theta$ and $\Gamma'$ is the differential of $\Gamma$.
%
%\begin{remark}
%	In our setting, $\ell$ is not differentiable because of the absolute value function. While in practice replacing the gradient by a subdifferential works well (this is what we do in the experiments), the implicit function theorem and the following theoretical results are only true for differentiable (and smooth) loss functions. In order to fit with the theory, a possibility is to follow a similar strategy as above, by replacing the absolute value function with a smooth approximation, such as the Huber loss \citep{huber1964}.
%\end{remark}
%
%Now that we know how to compute the gradient of the smooth quantile estimator, we can use the gradient descent algorithm to solve:% the smooth version of Problem \ref{eq:optim_quantile}: 
%%
%\begin{align} \label{eq_appen:optim_smooth_quantile}
%	&\min_{\theta} \; \widetilde{Q}_{\varepsilon}(1-\alpha;\{\ell(\theta;Z_i)\}_{i\in\calD^{lrn}}) \; .
%\end{align}
%%
%To this end, starting from an initial guess $\widetilde{\theta}_1$, we simply make the iterates:
%%
%\begin{align*}
%	\widetilde{\theta}_{k+1} = \widetilde{\theta}_{k} - \gamma_k \nabla_{\theta} [\widetilde{Q}_{\varepsilon}(1-\alpha; (\ell(\widetilde{\theta}_k;Z_i))_{i\in\calD^{lrn}})] \; ,
%\end{align*}
%where $\gamma_k > 0$ is the stepsize.
%
%%The full procedure is summary in Algorithm \ref{alg:min_quantile}.
%%\begin{algorithm}
%%	\caption{...}
%%	\label{alg:min_quantile}
%%	\begin{algorithmic}[1]
%	%		\State \textbf{Inputs:} $\varepsilon$, $\widetilde{\theta}_1$, $n_{iter}$, $(\gamma_k)_{1 \leq k \leq n_{iter}}$
%	%		\For{$k = 1, \ldots, n_{iter}$}
%	%		\State $\widetilde{\theta}_{k+1} \gets \widetilde{\theta}_{k} - \gamma_k \nabla_{\widetilde{\theta}_k} [\widetilde{Q}(1-\alpha; (\ell(\widetilde{\theta}_k;Z_i))_{i\in\calD^{lrn}})]$
%	%		\EndFor
%	%		\State \textbf{Output:} $\widetilde{\theta}_{n_{iter}+1}$
%	%	\end{algorithmic}
%%\end{algorithm}
%
%\begin{lemma}[\cite{pena2020solving}]\label{lemma:smooth_grad}
%	If $\gamma_{\varepsilon}$ is defined as in \pie{???}, $\ell(\cdot)$ has a Lipschitz continuous gradient, and $(1 - \alpha)\cdot\lvert \calD^{lrn} \rvert \notin \mathbb{Z}$, then the function $\psi_{\varepsilon} : \Theta \rightarrow \IR$ defined in Eq.~\eqref{eq:smooth_quant} has also a (bounded) Lipschitz continuous gradient.
%\end{lemma}
%
%\begin{proposition} Let us suppose that the assumptions of Lemma \ref{lemma:smooth_grad} hold and denote by $L > 0$ the Lipschitz smoothness constant of $\psi_{\varepsilon}$. After $T$ iterations of the gradient descent algorithm starting at $\widetilde{\theta}_1$ and with constant stepsize equal to $1/L$, we have:
%	\begin{align*}
%		\lVert \nabla_{\theta}\psi_{\varepsilon}(\zeta_T) \rVert \leq \sqrt{\dfrac{2 L \cdot (\psi_{\varepsilon}(\widetilde{\theta}_1) - \psi_{\varepsilon}(\widetilde{\theta}_{\varepsilon}^*)) }{T}} \; ,
%	\end{align*}
%	where $\zeta_T = \widetilde{\theta}_{T_{\min}}$ with $T_{\min} = \arg\min_{0<k\leq T+1}\lVert \nabla_{\theta}\psi_{\varepsilon}(\widetilde{\theta}_k) \rVert$ and where $\widetilde{\theta}_{\varepsilon}^*$ is one minimizer of Problem \eqref{eq:optim_smooth_quantile} \pie{existance de $\widetilde{\theta}_{\varepsilon}^*$?}. In other words, $\psi_{\varepsilon}(\zeta_T)$ is a $\calO(1/\sqrt{T})$-approximate first-order critical point.
%\end{proposition}
%\begin{proof}
%	\url{https://www.ceremade.dauphine.fr/~waldspurger/tds/22_23_s1/non_convex.pdf} \pie{Preuve dans ce pdf mais j'imagine que c'est standard.}
%\end{proof}
%
%\begin{remark}%(What happen when $\varepsilon$ goes to $0$?)
%	%By construction, we know that $\Gamma_{\varepsilon}(y) \longrightarrow \1\{y \leq 0\}$ and $\widetilde{Q}_{\varepsilon}(1-\alpha;\{\ell(\theta;Z_i)\}_{i\in\calD^{lrn}}) \longrightarrow \widehat{Q}(1-\alpha;\{\ell(\theta;Z_i)\}_{i\in\calD^{lrn}})$ as $\varepsilon \longrightarrow 0$. Furthermore, we also have that, 
%	If $\varepsilon$ tends to $0$, then $\nabla_{\theta} [\widetilde{Q}_{\varepsilon}(q; (\ell(\theta;Z_i))_{i\in\calD^{lrn}})] \longrightarrow \nabla_{\theta}\ell(\theta;Z_{i^*})$ where $i^*$ is the index such that $\ell(\theta;Z_{i^*}) = \widehat{Q}(q; (\ell(\theta;Z_i))_{i\in\calD^{lrn}})$. This means that, for this particular case, at each step of the gradient descent, instead of computing the gradient over all the loss-values $(\ell(\theta;Z_i))_{i\in\calD^{lrn}}$, we simply compute it for $\ell(\theta;Z_{i^*})$ which is the $\lceil q \cdot \lvert\calD^{lrn} \rvert \rceil$-th smallest value in $(\ell(\theta;Z_i))_{i\in\calD^{lrn}}$. %Hence, the gradient descent step boils done to Algorithm 1 in \citep{lecue2020robust}.
%	\pie{a rendre rigoureux}
%\end{remark}
%
%\begin{remark}(Link with other formulations)
%	Problem \ref{eq:optim_quantile} is in fact similar to the \textit{single chance constraint problem} (see e.g. \citep{curtis2018sequential}). It can also be reformulated as the following  bi-level optimization problem: 
%	%
%	\begin{align*}
%		\min_{\theta} &\; t(\theta) \label{eq:obj-constant} \\
%		\quad s.t. \quad & t(\theta) = \arg\min_{t} \sum_{i\in\calD^{lrn}} \rho_{1-\alpha}(\ell(\theta;Z_i) - t) \; .
%	\end{align*}
%	%
%	where $\rho_{1-\alpha}$ is the pinball loss. Indeed, from \cite{koenker1978regression, biau2011sequential} we know that $t(\theta) = \widehat{Q}(1-\alpha;\{\ell(\theta;Z_i)\}_{i\in\calD^{lrn}})$.
%\end{remark}
%
%\bat{AJOUTER MON ALGO ??}
\section{Additional results} \label{sec:add_xp}

\subsection{Synthetic data}\label{sec:add_xp_synth}
\paragraph{Experimental setup details for Section \ref{sec:xpADEffort}:}
During the learning step of \methodAD, we solve the $(1-\alpha)$-QAE Problem \eqref{eq:QAE} using the gradient descent strategy of Section \ref{sec:optim_emp_QAE}. The smoothing parameter $\varepsilon$ is set to $0.1$, $n_{iter}=1000$, and the step-size sequence is $\{(1/t)^{0.6}\}^{n_{iter}}_{t=1}$. The space of research $\calF$ is restricted to the space of linear functions. The function $\hat{s}(\cdot)$ (second step of \methodAD) and the two quantile regression functions of CQR are learned by using a Random Forest (RF) quantile regressor, implemented in the Python package sklearn-quantile\footnote{\href{https://sklearn-quantile.readthedocs.io}{https://sklearn-quantile.readthedocs.io}}. The function $\hat{\sigma}$ in LW-CP is learned using the RF regression implementation of scikit-learn \citep{scikit-learn}. Each time, the max-depth of the RF is set to $5$ and the other parameters are the default ones of the sklearn-quantile and scikit-learn packages.

\paragraph{Additional experiments:} We now present additional results on synthetic data:

\begin{itemize}
	
	\item In Figure \ref{fig:illustr_synth_coverage}, we display the coverage obtained on the scenarios of Section \ref{sec:xpEffort}. We see that, as expected, all methods return sets with average coverage of $1-\alpha=0.9$ (white circle) regardless of the distribution of the noise.
	%
	\item In figure \ref{fig:illustr_synth_NN}, we present additional results obtained when the base predictor is a Networks (NNs) and not a linear regressor as made in the main paper. We consider the model $Y = X^2 + \calE$ with $\calE$ following the same distributions as presented in Section \ref{sec:xpEffort}. In detail, we learn NNs with one hidden layer of size $10$ and with a ReLU activation function. In \method, the NN is learned using the gradient descent strategy of Section \ref{sec:optim_emp_QAE}. The smoothing parameter $\varepsilon$ is set to $0.1$, $n_{iter}=1000$ and the step-size sequence is $\{(1/t)^{0.6}\}^{n_{iter}}_{t=1}$. The gradient with respect to the NN weights involved in the gradient descent is calculated using automatic differentiation. For split CP, the NN is learned using an ADAM optimizer and the loss is either a Huber loss (robust NN) or a least squares loss. Again, in all scenarios, \method~returns marginally valid sets in general smaller than those of the split CP method. This confirms that learning a model via the $(1-\alpha)$-QAE problem is a better way of obtaining small prediction sets during the calibration step.
	
	\item In Figure \ref{fig:illustr_synth_adEffort_coverage}, we display the coverage obtained on the scenarios of Section \ref{sec:xpADEffort} when using \methodAD. We see again that, as expected, all methods return sets with average coverage of $1-\alpha=0.9$ (white circle) regardless of the distribution of the noise. Finally, Figure \ref{fig:illustr_synth_adEffort_example} shows examples of prediction sets returned by \methodAD, Locally weighted CP (LW-CP) and CQR when the noise is Gaussian.
\end{itemize}




\begin{figure*}[h!]
	\centering
	\includegraphics[width=.4\linewidth]{./img/coverage_normal_linear.pdf}
	\includegraphics[width=.4\linewidth]{./img/coverage_pareto_linear.pdf}
	\caption{Synthetic data: Boxplots of the $50$ empirical coverages obtained by evaluating \method~(see Section \ref{sec:xpEffort}). The white circle corresponds to the mean.} 
	\label{fig:illustr_synth_coverage}
\end{figure*}

\begin{figure*}[h!]
	\centering
	\includegraphics[width=.4\linewidth]{./img/size_normal_NN.pdf}
	\includegraphics[width=.4\linewidth]{./img/size_pareto_NN.pdf}
	\includegraphics[width=.4\linewidth]{./img/coverage_normal_NN.pdf}
	\includegraphics[width=.4\linewidth]{./img/coverage_pareto_NN.pdf}
	\caption{Synthetic data: %Length (top) and Coverage (bottom) obtained by evaluating \method (see Section \ref{sec:xpEffort}) when $\hat{f}$ is a neural-network. The white circle corresponds to the mean.
	Boxplots of the $50$ empirical expected lengths (top) and coverages (bottom) obtained by evaluating \method~(see Section \ref{sec:xpEffort}). The white circle corresponds to the mean.} 
	\label{fig:illustr_synth_NN}
\end{figure*}

\begin{figure*}[h!]
	\centering
	%\includegraphics[width=.48\linewidth]{./img/size_normal_adEffort.pdf}
	%\includegraphics[width=.48\linewidth]{./img/size_pareto_adEffort.pdf}
	\includegraphics[width=.4\linewidth]{./img/coverage_normal_adEffort.pdf}
	\includegraphics[width=.4\linewidth]{./img/coverage_pareto_adEffort.pdf}
	\caption{Synthetic data: Boxplots of the $50$ empirical coverages obtained by evaluating \methodAD~(see Section \ref{sec:xpADEffort}). The white circle corresponds to the mean.} 
	\label{fig:illustr_synth_adEffort_coverage}
\end{figure*}

\begin{figure*}[h!]
	\centering
	\includegraphics[width=.3\linewidth]{./img/res_adEffort.pdf}
	\includegraphics[width=.3\linewidth]{./img/res_LWCP.pdf}
	\includegraphics[width=.3\linewidth]{./img/res_CQR.pdf}
	\caption{Synthetic data: Example of sets returned by \methodAD~(left), LW-CP (middle), and CQR (right).} 
	\label{fig:illustr_synth_adEffort_example}
\end{figure*}

\subsection{Real data}
\label{app:real-data}
%
We finally compare \methodAD~with Locally Weighted CP (LW-CP) and CQR on the following public-domain real data sets also considered in e.g. \citep{romano2019conformalized}: abalone \citep{abalone_1}, boston housing (housing) \citep{harrison1978hedonic}\footnote{\href{https://www.cs.toronto.edu/~delve/data/boston/bostonDetail.html}{https://www.cs.toronto.edu/~delve/data/boston/bostonDetail.html}}, and concrete
compressive strength (concrete) \citep{yeh1998modeling}.\footnote{\href{http://archive.ics.uci.edu/dataset/165/concrete+compressive+strength}{http://archive.ics.uci.edu/dataset/165/concrete+compressive+strength}} We randomly split each data set $10$ times into a training set, a calibration set and a test set of respective "size" $40\%$, $40\%$, and $20\%$. The training and calibration sets are used to apply \methodAD, LW-CP, and CQR, and the test set to compute the coverage and length metrics. For \methodAD~and LW-CP the base prediction function $\hf$ is a Neural-Network (NN) with one hidden layer of size $10$ and a ReLU activation function. The function $\hs$ in the step 2 of \methodAD~ and the two quantile regression functions of CQR are learned with a Random Forest (RF) quantile regressor, implemented in the Python package sklearn-quantile. The function $\hat{\sigma}$ in LW-CP is learned using the RF regression implementation of scikit-learn \citep{scikit-learn}. Each time, the max-depth of the RF is set to $5$ and the other parameters are the default ones of the sklearn-quantile and scikit-learn packages. To illustrate the robustness of our approach, we finally add, in all the data sets, $5\%$ of outliers to the values to be predicted, using a Gaussian distribution whose mean is equal to 2 times the maximum value of the original data.

Figure \ref{fig:real_data} displays the length and the normalized length (i.e. the length divided by the maximal length obtained with the three methods in the $10$ splits) obtained on each data set. We can see that \methodAD~is competitive, as it generally returns marginally valid sets (see figure \ref{fig:real_data_cov} for coverage) of smaller or similar size to at least one of the other two methods. This is in line with the results obtained on synthetic data (Section \ref{sec:xps} and Appendix \ref{sec:add_xp_synth}). Note also that the variability of the coverage metric (represented by the length of the boxes in Figure \ref{fig:real_data_cov}) is much smaller for \methodAD~than LW-CP. Overall, these results show that \methodAD~is empirically competitive with the main existing CP methods, while enjoying a strong theoretical grounding. It is therefore a method of choice for all practical applications.

\begin{figure*}[h!]
	\centering
	\includegraphics[width=.4\linewidth]{./img/length_realdata.pdf}
	\includegraphics[width=.4\linewidth]{./img/normlength_realdata.pdf}
	%\includegraphics[width=.33\linewidth]{./img/coverage_realdata.pdf}
	\caption{Real data: Boxplots of the lengths (left) and normalized lengths (right) obtained with \methodAD, LW-CP, and CQR on real data sets. The white circle corresponds to the mean.} 
	\label{fig:real_data}
\end{figure*}

\begin{figure*}[h!]
	\centering
	\includegraphics[width=.4\linewidth]{./img/coverage_realdata.pdf}
	\caption{Real data: Boxplots of the coverages obtained with \methodAD, LW-CP, and CQR on real data sets. The white circle corresponds to the mean.} 
	\label{fig:real_data_cov}
\end{figure*}

%\subsection{Real data}
%
%We finally evaluate \methodAD~on the abalone data set \citep{abalone_1}. The goal of this data set is to predict the age of abalone from physical measurements. The true age of an abalone is in general determined by cutting its shell and by counting the number of rings through a microscope. Because this is a time-consuming task, predicting the age of abalone using other more accessible measurements is of huge interest. The data set is composed of 9 variables: sex, length, diameter, height, whole weight, shucked weight, viscera weight, shell weight and rings. The variable to predict is the age which is equals to (number of rings + 1.5).
%
%During the learning step of \method, we solve the $(1-\alpha)$-QAE Problem \eqref{eq:QAE} using the gradient descent strategy of Section \ref{sec:optim_emp_QAE} and with $\calF$ restricted to the space of Neural-Network (NN) with one hidden-layer of size $10$. The smoothing parameter $\varepsilon$ is set to $1$, $n_{iter}=1000$, and the step-size sequence is $\{(1/i)^{0.6}\}^{n_{iter}/10}_{i=1}$.


\end{document}
