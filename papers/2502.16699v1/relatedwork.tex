\section{Background and Related Work}
\textbf{Syntactical-based Watermarking.} Syntactical-based watermarking involves editing the generated text by manipulating the log-probabilities of a certain model. Multiple studies in the literature has investigated this type by introducing hypothesis testing. Pioneering in this area, the studies in KGW~\citep{kirchenbauer2023watermark, kirchenbauer2023reliability} and EXP~\citep{aaronson_my_2022} generate the watermark as a function of hashing the previous $k-1$-gram of a prompt to generate the next $k$ token. They introduce hypothesis testing to provide theoretical guarantees of watermark detectability. This is done by performing a statistical inference on the generated text, and calculating a score that is compared against a threshold with a very low False Positive Rate (FPR). Consequently, a barrage of similar watermark studies have been conducted to investigate the robustness and quality of KGW and EXP watermarks. In Unigram~\citet{zhao2023provable}, the authors argue the robustness of the watermark to removal attacks could be mitigated by hashing a pre-determined key (uni-gram) that is used for all generations.~\citet{kuditipudi2023robust, christ2024undetectable} introduced distortion-free watermarking in which the distribution of watermarked and unwatermarked text is the same. Similarly,~\citet{dathathri2024scalable} introduces speculative sampling to generate watermarks at scale.~\citet{lu2024entropy, lee2023wrote} investigated the role of token entropy to the watermark detectability and quality and proposes methods for generating and detecting the watermark. 

\noindent\textbf{Semantic Watermarking.} Syntactical watermarking can still be compromised by paraphrasing and back-translation watermark removal attacks. In response to this, a number of studies have been conducted to evaluate watermarking methods in such scenarios. SIR~\citep{liu2024a} introduced semantic hashing of previous context instead of tokens to combat security and robustness attacks related to k-gram methods. Consequently,~\citet{he2024can} introduced XSIR, which is an extension of SIR with cross-lingual settings in which semantically cross-lingual words are clustered together. Unlike SIR, XSIR is able to mitigate translation attacks between different languages. Most recently, the work in ~\citet{chang2024postmark} introduces the use of watermarking in a blackbox setting where users can semantically mark generated output of closed-source LLMs without accessing the log-probabilities for intellectual property reasons. Similarly,~\citet{hou2023semstamp, hou2024k} introduce sentence-level clustering to semantically mark the generated output.

\noindent\textbf{Post-hoc Detection Approaches.}
In our evaluate, we evaluate proactive detection methods in which the watermark signal is embedded in the text during generation. However, 
multiple studies have investigated passive/discriminator methods for AI-generated text detection.~\citet{tian_gptzero_2023,mitchell_detectgpt_2023,gehrmann2019gltr} use some statistical patterns in AI-generated text and trained discriminator models that differentiate human-written from AI-generated texts.~\citet{alshammari2024toward} introduces a method to detect AI-generated text in Arabic language by leveraging language-specific diacritics.~\citet{abdelnabi2021adversarial} investigates adversarial text watermarking by modifying the inner workings of a transformer model to embed a hidden watermark.