%%%%%%%% ICML 2025 EXAMPLE LATEX SUBMISSION FILE %%%%%%%%%%%%%%%%%

\documentclass{article}

% Recommended, but optional, packages for figures and better typesetting:
\usepackage{microtype}
\usepackage{graphicx}
\usepackage{subfigure}
\usepackage{booktabs} % for professional tables

% hyperref makes hyperlinks in the resulting PDF.
% If your build breaks (sometimes temporarily if a hyperlink spans a page)
% please comment out the following usepackage line and replace
% \usepackage{icml2025} with \usepackage[nohyperref]{icml2025} above.
\usepackage{hyperref}


% Attempt to make hyperref and algorithmic work together better:
\newcommand{\theHalgorithm}{\arabic{algorithm}}

% Use the following line for the initial blind version submitted for review:
% \usepackage{icml2025}

% If accepted, instead use the following line for the camera-ready submission:
\usepackage[accepted]{icml2025}

% For theorems and such
\usepackage{amsmath}
\usepackage{amssymb}
\usepackage{mathtools}
\usepackage{amsthm}

% if you use cleveref..
\usepackage[capitalize,noabbrev]{cleveref}

%%%%%%%%%%%%%%%%%%%%%%%%%%%%%%%%
% THEOREMS
%%%%%%%%%%%%%%%%%%%%%%%%%%%%%%%%
\theoremstyle{plain}
\newtheorem{theorem}{Theorem}[section]
\newtheorem{proposition}[theorem]{Proposition}
\newtheorem{lemma}[theorem]{Lemma}
\newtheorem{corollary}[theorem]{Corollary}
\theoremstyle{definition}
\newtheorem{definition}[theorem]{Definition}
\newtheorem{assumption}[theorem]{Assumption}
\theoremstyle{remark}
\newtheorem{remark}[theorem]{Remark}

% Todonotes is useful during development; simply uncomment the next line
%    and comment out the line below the next line to turn off comments
%\usepackage[disable,textsize=tiny]{todonotes}
\usepackage[textsize=tiny]{todonotes}


%%%%%%%%%%%%%%%%%%%%%%%%%%%%%%%%%%%%%%%%%%%%%%%%%%%%%%%%%%%%%%%%%%%%%%%%%%%%%%%%%
% Customized packages
\usepackage{adjustbox}
\usepackage{xspace}
\usepackage{amssymb}
\usepackage{multirow}
\usepackage[acronym]{glossaries}
\usepackage[inline]{enumitem}
\usepackage{graphicx}
\usepackage{subcaption}
\usepackage{caption}
\usepackage{booktabs}
\usepackage{setspace}
\usepackage{tabularx}
\usepackage{wrapfig}
\usepackage{afterpage} % make it easy to insert new page after a figure
\usepackage{setspace}
\usepackage{float}     % For [H] specifie
% \usepackage[inkscapelatex=false]{svg}  % For svg figure
%%%%%%%%%%%%%%%%%%%%%%%%%%%%%%%%%%%%%%%%%%%%%%%%%%%%%%%%%%%%%%%%%%%%%%%%%%%%%%%%%

\usepackage[utf8]{inputenc}
\usepackage{pgfplots}
\DeclareUnicodeCharacter{2212}{−}
\usepgfplotslibrary{groupplots,dateplot}
\usetikzlibrary{patterns,shapes.arrows}
\pgfplotsset{compat=newest}
\usetikzlibrary{arrows}
\usepackage{xcolor}


% Customized commands

% \newcommand{\todo}[1]{{\color{blue}#1}}
\newcommand{\fixme}[1]{{\color{red}#1}}
\newcommand{\update}[1]{{\color{green}#1}}
\renewcommand{\vec}[1]{{\boldsymbol{#1}}}

\newcommand{\fix}{\marginpar{FIX}}
\newcommand{\new}{\marginpar{NEW}}

%%%%%%%%%%%%%%%%%%%%%%%%%%%%%%%%%%%%%%%%%%%%%%%%%%%%%%%%%%%%%%%%%%%%%%%%%%%%%%%%%
% \renewcommand{\algorithmicrequire}{\textbf{Input:}}
% \renewcommand{\algorithmicensure}{\textbf{Output:}}
% \newcommand{\algrule}[1][.2pt]{\par\vskip.5\baselineskip\hrule height #1\par\vskip.5\baselineskip}

\usepackage{subfiles} % Best loaded last in the preamble
\usepackage{amsthm}
\usepackage[shortlabels]{enumitem}

\usepackage[thinc]{esdiff}

\newcommand{\Ex}{\mathbb{E}}
\newcommand{\E}{\mathrm{E}}
\newcommand{\I}{\mathrm{I}}
\newcommand{\W}{\mathbf{W}}
\newcommand{\U}{\mathbf{U}}
\newcommand{\h}{\mathbf{h}}
\newcommand{\x}{\mathbf{x}}
\newcommand{\X}{\mathbf{X}}
\newcommand{\F}{\mathbf{F}}
\newcommand{\Y}{\mathbf{Y}}
\newcommand{\g}{\mathbf{g}}
\newcommand{\z}{\mathbf{z}}
\newcommand{\R}{\mathbb{R}}
\newcommand{\N}{\mathbb{N}}
\newcommand{\Var}{\mathrm{Var}}

\newcommand\ddfrac[2]{\frac{\displaystyle #1}{\displaystyle #2}}


\newcommand{\vones}{\mathbbm{1}}
\newcommand{\p}{\partial}
\newcommand\w{{\mathbf{w}}}
\newcommand{\Cov}{\text{Cov}}
\newcommand{\be}{\begin{equation}}
\newcommand{\ee}{\end{equation}}
\newcommand{\s}{\subseteq}
\newcommand{\bbmat}{\begin{bmatrix}}
\newcommand{\ebmat}{\end{bmatrix}}

\def\bea#1\eea{\begin{align}#1\end{align}}

\newcommand{\Target}{{\mathbf y}}
\newcommand{\normX}{{\chi}}
\newcommand{\noise}{{\xi}}
\newcommand{\actFunc}{{f}}
\newcommand{\gain}{{g}}
\newcommand{\Gain}{{\mathbf g}}
%\newcommand{\bias}{{b}}
\newcommand{\Bias}{{\mathbf b}}
\newcommand{\Weights}{{W}}
\newcommand{\weights}{{w}}
\newcommand{\hidden}{{h}}
\newcommand{\Hidden}{{\mathbf h}}
\newcommand{\act}{{a}}
\newcommand{\Act}{{\mathbf a}}
\newcommand{\fisher}{{F}}
\newcommand{\data}{{\mathbf x}}
\newcommand{\loss}{\mathcal{L}}
\newcommand{\expectation}{\mathop{\mathbb{E}}}
%\newcommand{\kldiv}{\mathrm{D}_{\rm KL}}
%\newcommand{\klBars}{\,\|\,}
\newcommand{\entropy}{\mathcal{H}}
\newcommand{\partitionFn}{\mathcal{Z}}
\newcommand{\given}{\,|\,}

\newcommand{\kldiv}{\mathrm{D}_{\rm KL}}
\newcommand{\klBars}{\,\|\,}

\newcommand{\bdelta}{\boldsymbol{\delta}}
\newcommand{\btheta}{\boldsymbol{\theta}}
\newcommand{\balpha}{\boldsymbol{\alpha}}
\newcommand{\y}{\mathbf{y}}
\newcommand{\ba}{\mathbf{a}}
\newcommand{\bias}{{\mathbf{b}}}
\newcommand{\xxTmatrix}{\left[ 
			\begin{array}{cc}  
				\mathbf{x}\mathbf{x}^T 	& \mathbf{x}\\ 
				\mathbf{x}^T 			& 1 
			\end{array}
		\right]}

\newcommand{\titleappendix}[1]{
    \vbox{%
        \hsize\textwidth\linewidth\hsize
        \vskip 0.1in \vskip -\parskip \hrule height 1pt \vskip 0.09in \vskip 0.2in
        \centering{\LARGE\sc #1\par}
        \vskip 0.29in \vskip -\parskip \hrule height 1pt \vskip 0.09in \vskip 0.2in
    }
}

\newtheorem*{theorem*}{Theorem}



% The \icmltitle you define below is probably too long as a header.
% Therefore, a short form for the running title is supplied here:
\icmltitlerunning{Towards Fusing Point Cloud and Visual Representations for Imitation Learning}

\begin{document}

\twocolumn[
% \icmltitle{UniPerceive-IL: A Unified Approach for Point Cloud and Visual Imitation Learning}
\icmltitle{Towards Fusing Point Cloud and Visual Representations for Imitation Learning}

% It is OKAY to include author information, even for blind
% submissions: the style file will automatically remove it for you
% unless you've provided the [accepted] option to the icml2025
% package.

% List of affiliations: The first argument should be a (short)
% identifier you will use later to specify author affiliations
% Academic affiliations should list Department, University, City, Region, Country
% Industry affiliations should list Company, City, Region, Country

% You can specify symbols, otherwise they are numbered in order.
% Ideally, you should not use this facility. Affiliations will be numbered
% in order of appearance and this is the preferred way.
\icmlsetsymbol{equal}{*}

\begin{icmlauthorlist}
\icmlauthor{Atalay Donat}{equal,kit}
\icmlauthor{Xiaogang Jia}{equal,kit}
\icmlauthor{Xi Huang}{kit}
\icmlauthor{Aleksandar Taranovic}{kit}
\icmlauthor{Denis Blessing}{kit}
\icmlauthor{Ge Li}{kit}
\icmlauthor{Hongyi Zhou}{kit}
\icmlauthor{Hanyi Zhang}{uk}
\icmlauthor{Rudolf Lioutikov}{kit}
\icmlauthor{Gerhard Neumann}{kit}
%\icmlauthor{}{sch}
%\icmlauthor{}{sch}
\end{icmlauthorlist}

\icmlaffiliation{kit}{Karlsruhe Institute of Technology, Germany}
\icmlaffiliation{uk}{University of Liverpool}

\icmlcorrespondingauthor{Xiaogang Jia}{jia266163@gmail.com}
% You may provide any keywords that you
% find helpful for describing your paper; these are used to populate
% the "keywords" metadata in the PDF but will not be shown in the document
\icmlkeywords{Imitation Learning, Robot Learning}

\vskip 0.3in
]

% this must go after the closing bracket ] following \twocolumn[ ...

% This command actually creates the footnote in the first column
% listing the affiliations and the copyright notice.
% The command takes one argument, which is text to display at the start of the footnote.
% The \icmlEqualContribution command is standard text for equal contribution.
% Remove it (just {}) if you do not need this facility.

%\printAffiliationsAndNotice{}  % leave blank if no need to mention equal contribution
\printAffiliationsAndNotice{\icmlEqualContribution} % otherwise use the standard text.


% Main sections
\begin{abstract}
Retrieval-Augmented Generation (RAG) is often used with Large Language Models (LLMs) to infuse domain knowledge or user-specific information. In RAG, given a user query, a retriever extracts chunks of relevant text from a knowledge base. These chunks are sent to an LLM as part of the input prompt. Typically, any given chunk is repeatedly retrieved across user questions. However, currently, for every question, attention-layers in LLMs fully compute the key values (KVs) repeatedly for the input chunks, as state-of-the-art methods cannot reuse KV-caches when chunks appear at arbitrary locations with arbitrary contexts. Naive reuse leads to output quality degradation.  This leads to potentially redundant computations on expensive GPUs and increases latency. In this work, we propose \sys, a system for managing and reusing precomputed KVs corresponding to the text chunks (we call \textit{chunk-caches}) in RAG-based systems. We present how to identify \hl{\textit{chunk-caches} that are reusable}, how to efficiently perform a small fraction of recomputation to \textit{fix} the cache to maintain output quality, and how to efficiently store and evict \textit{chunk-caches} in the hardware for maximizing reuse while masking any overheads. With real production workloads as well as synthetic datasets, we show that \sys reduces redundant computation by \textbf{51\%} over SOTA prefix-caching and \textbf{75\%} over full recomputation.
\hl{Additionally, with continuous batching on a real production workload, we get a \textbf{1.6$\times$} speedup in throughput and a \textbf{2$\times$} reduction in end-to-end response latency over prefix-caching while maintaining quality, for both the \llama-3-8B and \llama-3-70B models. 
}
\end{abstract}





\documentclass[../main.tex]{subfiles}
\graphicspath{{../images/}}
\makeatletter
\def\input@path{{../images/}}
\makeatother
\begin{document}
\section{Introduction}
\begin{figure}
\centering
\begin{tikzpicture}
\node[inner sep=0pt] (ws) at (0, 0) {
\includegraphics[height=.4\textwidth, trim={10cm 0 10cm 0},clip]{world_space.png}};
\node[inner sep=0pt] (cs) at (6,0) {\includegraphics[height=.4\textwidth, trim={10cm 1cm 10cm 4cm},clip]{conf_space.png}};
\end{tikzpicture}
\vspace{-5pt}
\label{fig:pbrm_intro}
\caption{\textbf{Left}: Shows world space obstacles as grey spheres. Robots start and goal configuration is colored red and green, respectively. Configurations along the computed path are colored transparent blue. \textbf{Right:} Mapped world space scenario to configuration space. Obstacle region is the grey mesh. Red spheres are collision-free regions computed by the neural SCDF. The optimized shortest path in the convex corridor is the blue curve.}
\vspace{-25pt}
\end{figure}
Motion planning is the problem of finding a collision-free trajectory that connects a given start and goal configuration. The planning takes place in the configuration space of the robot. For single body robots, like mobile robots or drones, the configuration space and the world space are usually the same. This simplifies the planning, since explicit obstacle representations are available which enables geometrical tools like separating hyperplanes, smallest distance to obstacles etc., to be used when designing motion planning algorithms. For multi-body robots like manipulators, the situation is completely different. The world space obstacles are usually mapped to non-convex regions, and to make the problem even harder, the mapping is usually not known. Forming explicit representations of the obstacle region in the configuration space is usually too expensive or intractable. Despite all of this, sampling based planners are used with great success, which mainly is due to their use of implicit representations of the obstacle region. The basic idea is to construct a graph in the configuration space that covers and connects the collision-free region. From this graph, a path can be extracted that connects a given start and goal configuration. The approach is computationally expensive, since the graph is constructed with the smallest geometrical building block available, points, which represents a collision-check. Furthermore, the extracted paths from the graph are non-smooth and jagged due to the stochastic nature of the approach. This adds an additional post-processing step to the process, where the paths are shortcutted and smoothened, before the path can be used for tracking. Clearly a lot of time is invested to form this graph and produce smooth paths. Thus, if the obstacles start to move, then all of this work is done in no use, since all points that make up this graph need to be re-verified, which is simply too time consuming to be done in real time.
\\\\
In this work, we want to address the existing drawbacks of the sampling based planners. Our main contribution is an improved motion planner where each vertex in the graph covers a collision-free region in the form of a sphere instead of a point and where the edges are formed with neighboring intersecting spheres. This representation has the advantage of instead of returning piecewise linear paths, returning a sequence of overlapping spheres, i.e. a convex corridor, that connects a given start and goal configuration, illustrated in Figure \ref{fig:pbrm_intro}. This convex corridor allows us to use convex optimization to produce smooth trajectories, instead of computationally expensive post-processing methods. The representation further allows us to estimate the coverage of the collision-free space, which gives us awareness and feedback in the offline roadmap construction phase. Finally, our representation is simple to adapt to moving obstacles, simply requery for the new radii and recheck for intersections. 
\\\\
The spherical collision-free regions are formed using a signed distance function (SDF), which is a function that returns the smallest distance from an arbitrary point to the boundary of an obstacle. As the name implies, the distance is signed, thus if the point is inside the obstacle it is negative otherwise positive. If the distance is positive, a sphere with radius equal to the distance is guaranteed to cover a collision-free region. Using an SDF in motion planning is not new, but what is novel about our approach is that we express the distance in the configuration space instead of the world space and by doing so allows us to form these convex collision-free regions. We refer to the resulting SDF as a signed configuration distance function (SCDF). Computing an SCDF analytically is non-trivial, our approach is therefore to parameterize the SCDF with a deep neural network and learn the mapping by supervised learning. Our resulting neural SCDF can compute distances for different parameter values of obstacle shapes and we also show how multiple distances can be combined, thus making our approach flexible.
\section{Related work}
Motion planning algorithms can roughly be divided into three families, grid-based, sampling based and optimization based methods. Grid-based methods (GBM) discretize the planning space from which a graph is then compiled. A standard search method is A$^\star$ \citep{a_star}, which is classified as an \textit{informed} search method, since it employs a heuristic function to speed up the search. A$^\star$ guarantees to return an optimal path at the level of discretization used. GBMs usually discretize the planning space by a regular lattice and this limits the GBMs to problems with low dimensionality due to the curse of dimensionality. Thus, GBMs are usually limited to single-body robots where the degrees of freedom (DOF) are low. To overcome the inherent scaling problem with the GBMs, stochastic methods are usually used for multi-body robots. These methods are termed as sampling-based methods (SBM) and core members within this family are the rapidly-exploring random trees (RRT) \citep{rrt} and the probabilistic roadmap (PRM) \citep{prm}. RRT grows a tree from the start configuration and explores the collision-free region in a rapid way until it is able to connect to the goal region. RRT is usually improved by bi-directional planning \citep{rrt_connect}, i.e. an additional tree is grown from the goal configuration and the trees are tested for connection after any tree has been expanded. RRT is a single-query method, thus it searches for a path from scratch each time it is queried. Contrary to this, PRM is a multi-query method, which solves for multiple queries without starting from scratch. PRM does this by creating a roadmap (graph) that covers the collision-free space as an offline step. The graph is then used to solve for multiple queries. PRMs are used in cases where the environment does not change since the extra offline step is too computationally costly and needs to be re-done if the environment is changed. In our work, we address this inherent issue by using a different roadmap representation. Our vertices in the graph cover a collision-free region in the form of spheres and we form the edges by checking for intersecting spheres. If something in the environment changes, we recompute the spheres radii and recheck the intersections, without relying on collision detection. We use a trained neural network to compute the sphere radius, therefore querying for the radius can be done fast, hence our representation enables the PRM for dynamic environments.
\\\\
In the recent decades, optimization based methods (OBM) \citep{chomp, schulman, itomp, stomp} have been introduced as an alternative to SBM for multi-body robots. Like the SBM, the OBMs scale well to higher dimensional problems and produce smoother motion. It is common to use a SDF in the optimization since it is a smooth function, thus enabling gradient-based methods. However, the standard way of expressing the SDF is in world space. The distance therefore needs to be mapped to the configuration space by the forward kinematics. This mapping makes the optimization problem a non-linear program (NLP), which is computationally expensive to solve. Recently, a different approach has been proposed. In \cite{mp_gcs} motion planning is formulated as a convex optimization problem by using the graph of convex sets framework \citep{gcs}. The underlying idea is to decompose the collision-free space into intersecting convex sets from which a convex optimization problem is formulated. In cases where an explicit representation of the obstacles in the configuration space exists, like for single-body robots, creating collision-free convex regions can be done fast \citep{iris}. For multi-body robots, this is non-trivial. Existing work does this successfully \citep{iris_nlp, iris_c} by an optimization based approach, but the methods are still too time consuming to be used in the presence of moving obstacles. Our approach is instead to use deep learning to learn an SDF expressed in the configuration space. With this, we can query for shortest distances to the collision boundary, which allows us to expand spherical regions which are collision-free. Our approach is fast and therefore enables our suggested roadmap planner to be used in dynamic environments.
\\\\
Recent research has focused on learning collision detection \citep{fk_kernel_distance, diffco, graphdistnet} by predicting the signed distance between the robot links and the surrounding obstacles in the world space. The learned SDF is used in trajectory optimization but since the distance is expressed in the world space, the problem becomes an NLP and therefore takes a long time to solve. We take a novel approach and suggest to instead express the signed distance in the configuration space. This allows us to improve the PRM at the same time as it enables convex optimization for trajectory optimization, which runs faster and is more reliable than NLP solvers. In \cite{cspf} a learned signed distance function in the configuration space is proposed similar to our approach. However, their approach is restricted to point cloud representations, while we propose to represent the obstacles as parameterized geometric shapes, e.g. spheres. Furthermore, we also show how to use our learned SCDF to improve an existing roadmap planner.
\section{Problem formulation}
A robot is located in the world space, $\W \subset \R^3 $. The unique location of the robot is given by its configuration $\q \in \C$, where $\C$ is the configuration space. The set of points covered by the robots bodies at a certain configuration is expressed as $\B(\q) \subset \W$. The robot is surrounded by $\NrObst$ obstacles $\O = \bigcup_{i=1}^{\NrObst} \O_i$, where  $\O_i \subset \W$. The representation of the obstacle in the configuration space is the set $\C\O_i = \{\q \in \C \: |\: \B(\q) \cap \O_i \neq \emptyset \}$. The obstacle space is formed as $\Co = \bigcup_{i=1}^{\NrObst} \C \O_i$. The complement is referred to as the free space, $\Cf = \C \setminus \Co$. The path planning problem is a tuple, ($\Cf$, $\qStart$, $\qGoal$), where we want to connect a query pair, consisting of a start, $\qStart$, and goal configuration, $\qGoal$, with a geometric path, $\q(s): [0, 1] \mapsto \Cf$, such that $\q(0)=\qStart$ and $\q(1)=\qGoal$, or report correctly when such a path does not exist.
\end{document}



\subsection{Plasticity in Neural Networks}
In recent years, various methods have been proposed to address plasticity loss.
Several works have focused on maintaining active units \cite{abbas2023loss, elsayed2024addressing} or re-initializing dead units \cite{sokar2023dormant, dohare2024loss}.
Other studies have explored limiting deviations from the initial statistics of model parameters \cite{kumar2023maintaining, lewandowski2023curvature, elsayed2024weight}.
Additionally, some methods rely on architectural modifications \cite{nikishin2024deep, lee2024slow, lewandowski2024plastic}.  
Plasticity loss also occurs in the reinforcement learning due to its inherent non-stationary. \citet{nikishin2022primacy} proposed resetting the model, while \citet{asadi2024resetting} suggested resetting the optimizer state. 

As noted by \citet{berariu2021study}, loss of plasticity can be divided into two distinct aspects: a decreased ability of networks to minimize training loss on new data (trainability) and a decreased ability to generalize to unseen data (generalizability).
While most previous works focused on trainability, \citet{lee2024slow} addressed generalizability loss.
They demonstrated that plasticity loss also occurs under a stationary distribution, as in a warm-start learning scenario where the model is pretrained on a subset of the training data and then fine-tuned on the full dataset.

Most existing studies have focused on only one of the following challenges: trainability, generalizability, or reinforcement learning.
However, in this study, we validate our AID method across all three aspects, demonstrating its effectiveness in each scenario.



\subsection{Activation Function}
Our AID method is a stochastic approach similar to Dropout while also functioning as an activation function.
Therefore, we aim to discuss previously proposed probabilistic activation functions.
Although the field of probabilistic activation functions has not seen extensive research, two noteworthy studies exist.
The first is the Randomized ReLU (RReLU) function, introduced in the Kaggle NDSB Competition \cite{xu2015empirical}.
The original ReLU function maps all negative values to zero, whereas RReLU maps negative values linearly based on a random slope.
During testing, negative values are mapped using the mean of the slope distribution.
Their experimental results suggest that RReLU effectively prevents overfitting.
Another example of a probabilistic activation function is DropReLU \cite{liang2021drop}.
DropReLU randomly determines whether a node's activation is processed through a ReLU function or a linear function.
The authors claim that DropReLU improves the generalization performance of neural networks.
The fundamental distinction between these probabilistic activation functions and our method lies in the generality of our approach.
Unlike simple probabilistic activation functions, our method encompasses techniques such as Dropout and ReLU, providing a more comprehensive framework.

Another related approach involves activation functions designed to address plasticity loss.
\citep{abbas2023loss} proposed the Concatenated Rectified Linear Units (CReLU), which concatenates the outputs of the standard ReLU applied to the input and its negation.
This structure prevents the occurrence of dead units, thereby improving plasticity.
Additionally, trainable activation functions have also been shown to effectively mitigate plasticity loss in reinforcement learning \citep{delfosseadaptive}.
Specifically, they introduced a trainable rational activation function that prevents value overfitting and overestimation in reinforcement learning.



\begin{figure*}[ht!]
    \centering
    \includegraphics[width=0.3\textwidth]{figures/sources/mainnet_pls_acc.pdf}
    \includegraphics[width=0.3\textwidth]{figures/sources/subnet_pls_acc.pdf}
    \includegraphics[width=0.3\textwidth]{figures/sources/warm_start_dropout.pdf}
    \caption{\textbf{Left.} Random label MNIST experiment using an 8-layer MLP. Higher dropout probabilities result in significant trainability loss. 
    \textbf{Middle.} Accuracy of the subnetworks trained on random target. Each subnetworks are sampled from original network after each epoch. Subnetworks of the Dropout also experience trainability loss. \textbf{Right.} Warm-start scenario of Resnet-18 model with CIFAR100 dataset. Dropout improves generalization performance; however, the reduction in accuracy compared to the cold-start scenario is nearly identical to that of the vanilla model.}
    \label{exp_dropout}
\end{figure*}




\section{Preliminaries}\label{sec:preliminaries}



%We denote by $(\Ac(x_\Ac),\Bc(x_\Bc))(z)$ a random execution of $\pi$ with private inputs $(x_\Ac,y_\Ac)$, and common input $z$.

%\Jnote{Move to DP}
% At the end of such an execution, the protocol outputs a public transcript denoted by the random variable $\trans_\pi(x_\Ac,x_\Ac,z)$ we denotes the common as $\out(\trans_\pi(x_\Ac,x_\Ac,z)$, and each party $\Pc \in \set{\Ac,\Bc}$ obtains his view denoted $\view^\Pc_\pi(x_\Ac,x_\Bc,z)$, which may also contain a ``local output'' \Jnote{Local} $\out^\Pc(x_\Ac,x_\Bc,z)$ (if the protocol specifies such an output). \Jnote{Common output, and parties output}


\subsection{Distributions and Random Variables}\label{sec:prelim:dist}
The support of a distribution $P$ over a finite set $\cS$ is defined by $\Supp(P) \eqdef \set{x\in \cS: P(x)>0}$. For a distribution or a random variable $D$, let $d\from D$ denote that $d$ was sampled according to $D$. Similarly,  for a set $\cS$, let $x \from \cS$ denote that $x$ is drawn uniformly from $\cS$, and denote by $\cU_{\cS}$ the uniform distribution over $\cS$. For a finite set $\cX$ and a distribution $C_X$ over $\cX$, we use the capital letter $X$ to denote the random variable that takes values in $\cX$ and is sampled according to $C_X$. The {\sf statistical distance} (\aka {\sf~variation distance}) of two distributions $P$ and $Q$ over a discrete domain $\cX$ is defined by $\sdist{P}{Q} \eqdef \max_{\cS\subseteq \cX} \size{P(\cS)-Q(\cS)} = \frac{1}{2} \sum_{x \in \cS}\size{P(x)-Q(x)}$. 
For a vector $x = (x_1,\ldots,x_n)$ and index $i\in [n]$, we let $x_{-i} = (x_1,\ldots,x_{i-1},x_{i+1},\ldots,x_n)$ and $x^{(i)} = (x_1,\ldots,x_{i-1}, -x_i, x_{i+1},\ldots,x_n)$, for a set $\cS \subseteq [n]$ we let $x_{\cS} = (x_i)_{i \in \cS}$ and $x_{-\cS} = (x_i)_{i \in [n]\setminus \cS}$, and for a vector $r \in \zo^n$ we let $x_r = (x_i)_{\set{i \colon r_i = 1}}$ and $x_{-r} = (x_i)_{\set{i \colon r_i = 0}}$.

%For $n \in \N$ we let $U_n$ be the uniform distribution over $\oo^n$, and let $S_n$ be the distribution induces by the sum of $n$ i.i.d.\ random variables, each is distributed according to $U_1$. Let $\cN(0,1)$ be the standard normal distribution.
%For a distribution $\cD$ and a function $f$, we define by $f(\cD)$ the distribution that is induced by the output of $f(x)$ for $x \from \cD$. 





% \begin{theorem}[\cite{McGregorMPRTV10}]\label{thm:sv-extracotr}
% 	\Enote{Remove if not needed}
% 	There is a constant $c$ to make the following holds. Let $X$ be an $\alpha$-SV source on $\{0,1\}^n$, let $Y$ be a source on $\{0,1\}^n$ with min-entropy at least $\beta n$ (independent from $X$), and let $Z=\ip{X,Y}\mbox{mod m}$ for some $m\in\mathbb{N}$. Then for every $\delta\in[0,1]$, the random variable $(Y,Z)$ is $\delta$-close to $(Y,U)$ where $U$ is uniform on $\mathbb{Z}_m$ and independent of $Y$, provided that
% 	$$
% 	n\geq c\cdot\frac{m^2}{\alpha\beta}\cdot\log(\frac{m}{\beta})\cdot\log(\frac{m}{\delta}).
% 	$$
% \end{theorem}



\Enote{I removed the definition of DP since it already appears in the intro}
\remove{
\subsection{Differential Privacy}\label{sec:prelim:DP}
We use the following standard definition of (information theoretic) differential privacy, due to \citet{DMNS06}. For notational convenience, we focus on databases over $\oo$.
\begin{definition}[Differentially private mechanisms]\label{def:mech}
	A randomized function $f\colon\oo^n\mapsto \zs$ is an {\sf $n$-size, $(\eps,\delta)$-differentially private mechanism} (denoted $(\eps,\delta)$-\DP) if for every neighboring $w,w'\in \oo^n$ and every function $g\colon \zs\mapsto \zo$, it holds that 
	$$
	\pr{g(f(w))=1}\leq e^{\eps}\cdot \pr{g(f(w'))=1} +\delta.
	$$ 	
	If $\delta=0$, we omit it from the notation.
\end{definition}
}


\subsubsection{Computational Differential Privacy}
There are several ways for defining computational differential privacy (see \cref{sec:related-works}). We use the most relaxed version due to \cite{BNO08}. For notational convenience, we focus on databases over $\oo$.
\begin{definition}[Computational differentially private mechanisms]\label{def:ComMech}
	A randomized function ensemble $f=\set{f_\pk\colon\oo^{n(\pk)}\mapsto \zs}$ is an {\sf $n$-size, $(\eps,\delta)$-computationally differentially private} (denoted $(\eps,\delta)$-$\CDP$) if for every poly-size circuit family $\set{\Ac_\pk}_{\pk\in \N}$, the following holds for every large enough $\pk$ and every neighboring $w,w'\in\oo^{n(\pk)}$:
	$$
	\pr{\Ac_\pk(f_\pk(w))=1}\leq e^{\eps(\pk)}\cdot \pr{\Ac_\pk(f_\pk(w'))=1} +\delta(\pk).
	$$ 
	If $\delta(\pk) = \negl(\pk)$, we omit it from the notation. 
\end{definition}



\subsubsection{Two-Party Differential Privacy}\label{sec:DP}
In this section we formally define distributed differential privacy mechanism (\ie protocols). %For the ease of notation, we consider protocol with no common input.

\begin{definition}\label{def:DP}%\Nnote{fix security parameter}
	A two-party protocol $\Pi=(\Ac,\Bc)$ is {\sf $(\eps,\delta)$-differentially private}, denoted $(\eps,\delta)$-$\DP$, if the following holds for every algorithm $\Dc$: let $\V^\Pc(x,y)(\pk)$ be the view of party $\Pc$ in a random execution of $\Pi(x,y)(1^\pk)$. Then for every $\pk,n \in \N$, $x\in \oo^n$ and neighboring $y,y'\in\oo^n$:
	\begin{align*}
	\pr{\Dc(V^\Ac(x,y)(\pk))=1}\le e^{\eps(\pk)}\cdot \pr{\Dc(V^\Ac (x,y')(\pk))=1}+\delta(\pk),
	\end{align*} 
	and for every $y\in \oo^n$ and neighboring $x,x'\in\oo^{n}$:
	\begin{align*}
	\pr{\Dc(V^\Bc(x,y)(\pk))=1}\le e^{\eps(\pk)}\cdot \pr{\Dc(V^\Bc (x',y)(\pk))=1}+\delta(\pk).
	\end{align*} 	
	Protocol $\Pi$ is {\sf $(\eps,\delta)$-computational differentially private}, denoted $(\eps,\delta)$-$\CDP$, if the above inequalities only hold for a non-uniform \ppt $\Dc$ and large enough $\pk$. We omit $\delta = \negl(\pk)$ from the notation. \footnote{Note that define we give for two-party differentially private protocols is a semi-honest definition, in which we ask for the security to hold when the parties interact in an honest execution of the protocol. Since we are proving a lower bound, starting from this weaker guarantee (as opposed to security against malicious players), yields a stronger result.}
\end{definition}
%We omit $\delta$ from the notation if $\delta$ is a negligible function of $n$.

%\Enote{simulation-based}
\begin{remark}[The definition for computational differential privacy we use]\label{rem:comDPChannel} 
	An alternative, stronger definition of computational differential privacy, known as simulation-based computational differential privacy, requires that the distribution of each party’s view be computationally indistinguishable from a distribution that ensures privacy in an information-theoretic sense. \cref{def:DP} is a weaker notion in comparison. Consequently, establishing a lower bound for a protocol that satisfies this weaker guarantee (as we do in this work) yields a stronger result.%Actually, our lower bound only requires the privacy to hold against \emph{uniform} external observer.
	%\Nnote{Maybe add: When only interesting in \Dp against external observer, the two definitions can be achieve using key-agreement and (single-party) \Dp mechanism. }
\end{remark}




\subsection{Useful Claims}
\remove{
In this section, we state generic lemmas and propositions that we will use later in our proofs.

The following lemma which we prove in \cref{sec:missing-proofs:distance-I}, measures the distance between two uniform stings conditioned one a random index $i$ either being fixed to $0$ or to $1$.

\def\distanceILemma{
    Let $R \la \zo^n$. For any (randomized) function $f:\{0,1\}^n\rightarrow \{0,1\}$ and $\alpha > 0$, it holds that
    \begin{align}\label{eq:f-alpha}
        \ppr{i \la [n]}{\size{\:\ex{f(R) \mid R_i = 0}-\ex{f(R) \mid R_i = 1}\:}\geq \alpha} \leq \frac{2}{n \alpha^2},
    \end{align}
    where the expectations are taken over $R$ and the randomness of $f$.
}

\begin{lemma}\label{lem:distance-I}
    \distanceILemma
\end{lemma}
}

The following two propositions state that given the output of a differentially private function, it is not possible to predict well even a random index (even if all other indexes are leaked). The first proposition handles the information-theoretic case and the second handles the computation case. Both propositions are proven in \cref{sec:missing-proofs:hard-to-guess}. 

\def\propHardToGuessInf{
    Let $f\colon \oo^n \rightarrow \cY$ be an $(\eps,\delta)$-\DP function, let $g \colon [n] \times \oo^{n-1} \times \cY \rightarrow \set{-1,1,\bot}$ be a (randomized) function, and let $X = (X_1,\ldots,X_n) \la \oo^n$. Then the following holds for every $i \in [n]$ where $X_i^* = g(i,X_{-i},f(X_1,\ldots,X_n))$:
    \begin{align*}
        \pr{X_i^* = X_i} \leq e^{\eps}\cdot \pr{X_i^* = -X_i} + \delta.
    \end{align*}
}

\begin{proposition}\label{prop:hard-to-guess-inf}
    \propHardToGuessInf
\end{proposition}


\def\propHardToGuessComp{
    Let $f = \set{f_{\pk} \colon \oo^{n(\pk)} \rightarrow \zo^{m(\pk)}}_{\pk \in \bbN}$ be an $(\eps,\delta)$-\CDP function ensemble, and let $\set{g_{\pk}}_{\pk \in \bbN}$ be a poly-size circuit family. Then, for large enough $\pk$ and $X = (X_1,\ldots,X_{n(\pk)}) \la \oo^{n(\pk)}$, the following holds for every $i \in [n(\pk)]$ where $X_i^* = g_{\pk}(i,X_{-i},f_{\pk}(X_1,\ldots,X_n))$:
    \begin{align*}
        \pr{X_i^* = X_i} \leq e^{\eps(\pk)}\cdot \pr{X_i^* = -X_i} + \delta(\pk).
    \end{align*}
}

\begin{proposition}\label{prop:hard-to-guess-comp}
    \propHardToGuessComp
\end{proposition}





\remove{
\Enote{Chao's old statement:}
\begin{lemma}\label{lem:distance-I-old}
        Let $R \la \zo^n$. 
	For any function $f:\{0,1\}^n\rightarrow \{0,1\}$ and $\alpha<0.01$, it holds that
	$$
	\Pr_{i\la[n]}\left[\: \size{\:\mathbb{E}[f(R) \mid R_i = 0]-\mathbb{E}[f(R) \mid R_i = 1]\:}\geq \alpha\right]\leq \frac{2+2\log(\frac{1}{\alpha})}{n\alpha^2}.
	$$
\end{lemma}
\begin{proof}
	Define $S_1=\{r \in \zo^n \colon f(r)=1\}$. Then for any $i\in[n]$, we have
	$$
	\begin{array}{rl}
		\size{\mathbb{E}[f(R) \mid R_i = 0]-\mathbb{E}[f(R) \mid R_i = 1]}
		&=\size{\Pr[R\in S_1|R_i=0]-\Pr[R\in S_1|R_i=1]}\\
		&=\size{\frac{\Pr[R_i=0|R\in S_1]\cdot\Pr[R\in S_1]}{\Pr[R_i=0]}-\frac{\Pr[R_i=1|R\in S_1]\cdot\Pr[R\in S_1]}{\Pr[R_i=1]}}\\
		&=\frac{2\size{S_1}}{2^n}\size{\Pr[R_i=0|R\in S_1]-\Pr[R_i=1|R\in S_1]}
	\end{array}
	$$
	When $|S_1|\leq \alpha\cdot 2^{n-1}$, we have $\size{\mathbb{E}[f(R) \mid R_i = 0]-\mathbb{E}[f(R) \mid R_i = 1]}\leq\frac{2\size{S_1}}{2^n}\leq \alpha$ for any $i\in[n]$. Hence, in the following, we assume $|S_1|> \alpha\cdot 2^{n-1}$.

	%Define $I_{bad}=\{i|\size{\Pr[R_i=0|R\in S_1]-\Pr[R_i=1|R\in S_1]}>2\alpha\}$ and $k=\size{I_{bad}}$, then for any $i\notin I_{bad}$, we have 
    %$$
    %\begin{array}{rl}
    %    2\alpha&\geq \size{\Pr[R_i=0|R\in S_1]-\Pr[R_i=1|R\in S_1]}\\
    %    &=\size{\frac{\Pr[R\in S_1|R_i=0]\cdot\Pr[R_i=0]}{\Pr[R\in S_1]}-\frac{\Pr[R\in S_1|R_i=1]\cdot\Pr[R_i=1]}{\Pr[R\in S_1]}}\\
    %    &=\size{\Pr[R\in S_1|R_i=0]-\Pr[R\in S_1|R_i=1]}\cdot\frac{1}{2\Pr[R\in S_1]}\\
    %    &\geq \size{\mathbb{E}[f(R) \mid R_i = 0]-\mathbb{E}[f(R) \mid R_i = 1]}\cdot \frac{1}{2},
    %\end{array}
    %$$ 
    %where the last inequality is because $\Pr[R\in S_1]\leq 1$. So that $\size{\mathbb{E}}[f(R) \mid R_i = 0]-\mathbb{E}[f(R) \mid R_i = 1]\leq %4\alpha$.
    Define $I_{bad}=\{i \colon \size{\Pr[R_i=0|R\in S_1]-\Pr[R_i=1|R\in S_1]} \geq 2\alpha\}$ and $k=\size{I_{bad}}$, and denote $I_{bad}=\{i_1,\dots,i_k\}$. Define $(X_{i_1}, \ldots X_{i_k}) = (R_{i_1},\dots,R_{i_k})\mid_{R \in S_1}$. 
    Consider the min-entropy
	$$
	\begin{array}{rl}
		H_{min}(X_{i_1},\dots,X_{i_k})&\leq H(X_{i_1},\dots,X_{i_k})\\
		&\leq \sum_{j=1}^k H(X_{i_j})\\
		&\leq k\cdot \left(-(\frac{1}{2}+2\alpha)\cdot\log(\frac{1}{2}+2\alpha)-(\frac{1}{2}-2\alpha)\cdot\log(\frac{1}{2}-2\alpha)\right)\\
            &=k\cdot \left(-(\frac{1}{2}+2\alpha)\cdot(\log(1+4\alpha)-1)-(\frac{1}{2}-2\alpha)\cdot(\log(1-4\alpha)-1)\right)\\
            &=k\cdot \left(1-(\frac{1}{2}+2\alpha)\cdot\log(1+4\alpha)-(\frac{1}{2}-2\alpha)\cdot\log(1-4\alpha)\right),
		
	\end{array}
	$$
	where $H_{min}(Y)$ is the minimum entropy of $Y$ and $H(Y)$ is the Shannon entropy of $Y$.\Enote{add to preliminaries.}
        The third inequality holds since by the definition of $I_{bad}$, for every $j \in [k]$ it holds that $\size{\pr{X_{i_j} = 1}-\pr{X_{i_j} = 0}} > 2\alpha$, and therefore $H(X_{i_j}) \leq H(1/2 + 2\alpha)$\Enote{define}.
	
	Therefore, there exists $b_1,\dots,b_k\in\{0,1\}$, such that 
	
	\begin{align}\label{eq:min-entropy-result}
		\Pr\left[(R_{i_1},\ldots,R_{i_k}) = (b_1,\ldots,b_k) \mid R\in S_1\right]
		&= \pr{(X_{i_1},\ldots,X_{i_k}) = (b_1,\ldots,b_k)}\\
		&= 2^{-H_{min}(X_{i_1},\dots,X_{i_k})}\nonumber\\
		&\geq 2^{k\cdot \left(-1+(\frac{1}{2}+2\alpha)\cdot\log(1+4\alpha)+(\frac{1}{2}-2\alpha)\cdot\log(1-4\alpha)\right)}.\nonumber
	\end{align}
	
	Let $S_{bad}=\{r \in \zo^n  \colon \set{(r_{i_1},\ldots,r_{i_k}) = (b_1,\ldots,b_k)} \land \set{r\in S_1}\}$.
	It holds that
	\begin{align*}
		|S_{bad}|
		&= \size{S_1} \cdot \Pr\left[(R_{i_1},\ldots,R_{i_k}) = (b_1,\ldots,b_k) \mid R\in S_1\right]\\
		&\geq \alpha\cdot 2^{n-1}\cdot2^{k\cdot \left(-1+(\frac{1}{2}+2\alpha)\cdot\log(1+4\alpha)+(\frac{1}{2}-2\alpha)\cdot\log(1-4\alpha)\right)},
	\end{align*} 
	where the inequality holds by \cref{eq:min-entropy-result} and since $\size{S_1} \geq \alpha\cdot 2^{n-1}$.
	Notice that any string in $S_{bad}$ depends on at most $n-k$ bits. It implies that $|S_{bad}|\leq 2^{n-k}$. Therefore, we have
	$$
	\begin{array}{rl}
		&2^{n-k}\geq \alpha\cdot 2^{n-1}\cdot2^{k\cdot \left(-1+(\frac{1}{2}+2\alpha)\cdot\log(1+4\alpha)+(\frac{1}{2}-2\alpha)\cdot\log(1-4\alpha)\right)} \\
		\Rightarrow& n-k \geq \log \alpha+n-1+k\cdot \left(-1+(\frac{1}{2}+2\alpha)\cdot\log(1+4\alpha)+(\frac{1}{2}-2\alpha)\cdot\log(1-4\alpha)\right)\\
		\Rightarrow& 1-\log \alpha \geq k\cdot((\frac{1}{2}+2\alpha)\cdot\log(1+4\alpha)+(\frac{1}{2}-2\alpha)\cdot\log(1-4\alpha))\\
		\Rightarrow& 1-\log \alpha \geq k\cdot(4\alpha\cdot\log(1+4\alpha)+(\frac{1}{2}-2\alpha)\cdot\log(1-16\alpha^2))\\
        \Rightarrow& 1-\log\alpha \geq k\cdot(15.9\alpha^2-8\alpha^2+32\alpha^3)=k\cdot(7.9\alpha^2+32\alpha^3)>0.5k\alpha^2\\
		\Rightarrow& k\leq \frac{2-2\log \alpha}{\alpha^2} = \frac{2+2\log (1/\alpha)}{\alpha^2},
	\end{array}
	$$
	Where the third transition holds since 
	\begin{align*}
		\lefteqn{(\frac{1}{2}+2\alpha)\cdot\log(1+4\alpha)+(\frac{1}{2}-2\alpha)\cdot\log(1-4\alpha)}\\
		&= 4\alpha\cdot\log(1+4\alpha) + (\frac{1}{2}-2\alpha)\paren{\log(1+4\alpha)+\log(1-4\alpha)}\\
		&= 4\alpha\cdot\log(1+4\alpha)+(\frac{1}{2}-2\alpha)\cdot\log(1-16\alpha^2),
	\end{align*}
	and the forth transition holds since $4\alpha\cdot\log(1+4\alpha)+(\frac{1}{2}-2\alpha)\cdot\log(1-16\alpha^2) > 15.9\alpha^2-8\alpha^2+32\alpha^3$ for $\alpha < 0.01$.
	Thus, we conclude that 
	$$
	\Pr_{i\la[n]}\left[\size{\mathbb{E}[f(R) \mid R_i=0]-\mathbb{E}[f(R) \mid R_i = 1]}\geq \alpha\right]\leq \frac{k}{n}\leq \frac{2+2\log (1/\alpha)}{n\alpha^2}.
	$$
\end{proof}
}


\subsection{Channels and Two-Party Protocols}\label{sec:protocol}

\paragraph{Channels.}A channel is simply a distribution of a pair of tuples defined as follows. 
\begin{definition}[Channels]\label{def:channel} A {\sf channel} $C_{(X,U)(Y,V)}$ of size $\isize$ over alphabet $\Sigma$ is a probability distribution over $(\Sigma^\isize \times\zo^\ast) \times(\Sigma^\isize \times\zo^\ast)$. The ensemble $C_{(X,U)(Y,V)}= \set{C_{(X_\pk,U_\pk)(Y_\pk,V_\pk)}}_{\pk\in \N}$ is an $\isize$-size channel ensemble, if for every $\pk\in \N$, $C_{(X_\pk,U_\pk)(Y_\pk,V_\pk)}$ is an $\isize(\pk)$-size channel. %We denote a channel of size one by a \emph{single-bit} channel. 
We refer to $X$ and $Y$ as the {\sf local outputs}, and to $U$ and $V$ as the {\sf views}.	
\end{definition}

We view a  channel as the experiment in which there are two parties $\Ac$ and $\Bc$.  Party $\Ac$ receives ``output'' $X$ and ``view'' $U$, and party $\Bc$ receives ``output'' $Y$ and ``view'' $V$. Unless stated otherwise, the channels we consider are over the alphabet $\Sigma = \oo$. We naturally identify channels with the distribution that characterizes their output.








\subsubsection{Two-Party Protocols}

A two-party protocol $\Pi=(\Ac,\Bc)$ is \ppt if the running time of both parties is polynomial in their input length. We let $\Pi(x,y)(z)$ or $(\Ac(x),\Bc(y))(z)$ denote a random execution of $\Pi$ on a common input $z$, and private inputs $x,y$.%We assume \wlg that a protocol has a common output (part of its transcript).\Jnote{This is not really the case we consider in this paper..}

\begin{definition}[Oracle-aided protocols]\label{def:ChannelAidedProtocol}
	In a two-party protocol $\Pi$ with oracle access to a {\sf protocol} $\Psi$, denoted $\Pi^\Psi$, the parties make use of the \textit{next-message function} of $\Psi$.\footnote{The function that on a partial view of one of the parties, returns its next message.} In a two-party protocol $\Pi$ with oracle access to a {\sf channel} $C_{Z W}$, denoted $\Pi^C$, the parties can jointly invoke $C$ for several times. In each call, an independent pair $(z,w)$ is sampled according to $C_{Z W}$, one party gets $z$, the other gets $w$.
\end{definition}


\begin{definition}[The channel of a protocol]\label{def:ChannlOfProtocol}
	For a no-input two-party protocol $\Pi= (\Ac,\Bc)$, we associate the channel $C_\Pi$, defined by $\C_\Pi= C_{(X, U),(Y, V)}$, where $X$ and $Y$ are the local outputs of $\Ac$ and $\Bc$ (respectively) and
	$U$ and $V$ are the local views of $\Ac$ and $\Bc$ (respectively).
    
	For a two-party protocol $\Pi$ that gets a security parameter $1^\pk$ as its (only, common) input, we associate the channel ensemble $ \set{C_{\Pi(1^\pk)}}_{\pk\in \N}$. 
\end{definition}

\begin{definition}[$(\alpha,\gamma)$-Accurate channel]\label{def:accurate-func}
	A channel $C = C_{(X, U),(Y, V)}$ is {\sf $(\alpha,\gamma)$-accurate for the function $f$}, if $\ppr{C}{\size{\out(V)-f(X,Y)}\leq \alpha}\ge \gamma$, where $\out(V)$ is the designated output.
    A channel ensemble $C_{(X, U),(Y, V)}= \set{C_{(X_\pk, U_\pk),(Y_\pk, V_\pk)}}_{\pk\in \N}$ is  $(\alpha,\gamma)$-accurate for  $f$ if $C_{(X_\pk, U_\pk),(Y_\pk, V_\pk)}$ is $(\alpha(\pk),\gamma(\pk))$-accurate for $f$, for every $\pk \in \N$.
\end{definition}

\subsubsection{Differentially Private Channels}\label{sec:DPChannel}
Differentially private channels are naturally defined as follows:
\begin{definition}[Differentially private channels]\label{def:DPChannel}
	An $n$-size channel $C = C_{(X, U),(Y, V)}$ with $X, Y$ over $\oo^n$ 
	is {\sf$(\eps,\delta)$-differentially private} (denoted $(\eps,\delta)$-$\DP$) if for every $x \in \Supp(X)$ there exists an $n$-size $(\eps,\delta)$-$\DP$ mechanisms $\Mc_x$ such that $(X,Y,U) \equiv (X,Y,\Mc_X(Y))$, and for every $y \in \Supp(Y)$ there exists an $n$-size $(\eps,\delta)$-$\DP$ mechanisms $\Mc_y'$ such that $(X,Y,V) \equiv (X,Y,\Mc_Y'(X))$. In addition, we say that the channel is \emph{uniform} if $X$ and $Y$ are independent random variables uniformly distributed in $\oo^n$. 
\end{definition}

\begin{definition}[Computational differentially private channels]\label{def:CDPChannel}
	An $n$-size channel ensemble $C = \set{C_{(X_\pk, U_\pk),(Y_\pk, V_\pk)}}_{\pk\in\N}$ with $X_\pk, Y_\pk$ over $\oo^n$ 
	is {\sf$(\eps,\delta)$-computationally differentially private} (denoted $(\eps,\delta)$-$\CDP$) if for every ensemble $\set{x_\pk \in \Supp(X_\pk)}_{\pk\in\N}$ there exists an $n$-size $(\eps,\delta)$-\CDP mechanisms ensemble $\set{\Mc_{x_\pk}}_{\pk\in\N}$ such that $(X_\pk,Y_\pk,U_\pk) \equiv (X_\pk,Y_\pk,\Mc_{X_\pk}(Y_\pk))$, for every $\pk\in\N$, and for every ensemble $\set{y_\pk \in \Supp(Y_\pk)}_{\pk\in\N}$ there exists an $n$-size $(\eps,\delta)$-$\CDP$ mechanisms ensemble $\set{\Mc'_{y_\pk}}_{\pk\in\N}$ such that $(X_\pk,Y_\pk,V_\pk) \equiv (X_\pk,Y_\pk,\Mc_{Y_\pk}'(X_\pk))$ for every $\pk\in \N$. In addition, we say that the channel is \emph{uniform} if $X_\pk$ and $Y_\pk$ are independent random variables uniformly distributed in $\{\pm 1\}^n$ for all $\pk\in\N$.
\end{definition}




% \begin{lemma}~\label{lem:dp-sv-source}
% 	Let $P$ be an $\varepsilon$-DP randomized protocol. Let $X$ and $Y$ be independent random variables uniformly distributed in $\{\pm 1\}^n$ and let random variable $\Pi(X,Y)$ denote the transcript of running $P(X,y)$. Then for every $\pi\in Supp(\Pi)$, the random variables corresponding to the inputs conditioned on transcript $\pi$, $X_\pi$ and $Y_\pi$, are independent $e^{-\varepsilon}$-strong SV source.
% \end{lemma}





\subsubsection{Weak Erasure Channel (\WEC)}

\begin{definition}[\WEC]\label{def:WEC}
	A channel $((O_A,V_A), (O_B,V_B))$ with $O_A \in \set{0,1}$ and $O_B \in \set{0,1,\bot}$ is a {\sf weak erasure channel}, denoted $(\alpha,p,q)$-$\WEC$, if:
	\begin{itemize}
		%\item $O_A\in \set{-1,1}$ and $O_B\in \set{-1,1,\bot}$.
		\item Random erasure: $\pr{O_B = \perp} = 1/2$.
		
		\item Agreement: $\pr{O_A\ne O_B\mid O_B\ne \bot}\le \alpha$.
		
		\item Secrecy:
		
		\begin{enumerate}
			\item For every algorithm $\Dc$ it holds that\label{WEC:item:A}
			\begin{align*}
				%\size{\pr{\Ac(O_A,V_A) = 1 \mid O_B \neq \perp} - \pr{\Ac(O_A,V_A) = 1 \mid O_B = \perp}} \le p
				\size{\pr{\Dc(V_A) = 1 \mid O_B \neq \perp} - \pr{\Dc(V_A) = 1 \mid O_B = \perp}} \le p
			\end{align*}
			(Alice doesn't know if $O_B = \perp$.)
			
			\item For every algorithm $\Dc$ it holds that\label{WEC:item:B}
			\begin{align*}
				\pr{\Dc(V_B) = O_A \mid O_B=\bot} \leq \frac{1+q}{2}.
			\end{align*}
			(i.e., if $O_B=\bot$, Bob don't know what is the value of $O_A$).
			
			%\item $SD((O_A U|O_B=\bot),(O_A U|O_B\ne \bot))\le p$ (The sender don't know if $O_B=\bot$).
			
			%\item $SD(V O_A|O_B=\bot,V(-O_A)|O_B=\bot)\le q$ (If $O_B=\bot$, Bob don't know what the value of $O_A$).
		\end{enumerate}
	\end{itemize}
   We say that a channel ensemble $C=\set{C_\pk}_{\pk\in N}$ is a {\sf computational weak erasure channel}, denoted $(\alpha,p,q)$-\CompWEC, if for every \ppt algorithm $\Dc$ and every sufficiently large $\pk\in\N$, $C_\pk$ satisfies the properties stated in the items above, where the secrecy property holds with respect to a \ppt algorithm $\Dc$. A protocol $\Lambda$ is said to be $(\alpha,p,q)$-$\CompWEC$, if the ensemble induces by the protocol (that is, $C=\set{C_{\Lambda(\pk)}}_{\pk\in\N}$) is $(\alpha,p,q)$-$\CompWEC$.  
\end{definition}



\subsubsection{Approximate Weak Erasure Channel (\AWEC)}\label{sec:AWEC}

\begin{definition}[\AWEC]\label{def:AWEC}
	A channel $C = ((O_A,V_A), (O_B,V_B))$ over $([-n,n] \times \zo^*) \times (([-n,n] \cup \bot)  \times \zo^*)$ is an {\sf approximate weak erasure channel}, denoted $(\ell,\alpha,p,q)$-\AWEC if:
	\begin{itemize}
		
		\item Random erasure: $\pr{O_B = \perp} = 1/2$.
		
		\item Accuracy: $\pr{\size{O_A - O_B} > \ell \mid O_B \ne \bot}\le \alpha$.
		
		\item Secrecy:
		
		\begin{enumerate}
			\item For every algorithm $\Dc$ it holds that\label{AWEC:item:A}
			\begin{align*}
				%\size{\pr{\Ac(O_A,V_A) = 1 \mid O_B \neq \perp} - \pr{\Ac(O_A,V_A) = 1 \mid O_B = \perp}} \le p
				\size{\pr{\Dc(V_A) = 1 \mid O_B \neq \perp} - \pr{\Dc(V_A) = 1 \mid O_B = \perp}} \le p
			\end{align*}
			(Alice doesn't know if $O_B=\bot$).
			
			\item For every algorithm $\Dc$ it holds that\label{AWEC:item:B}
			\begin{align*}
				\pr{\size{\Dc(V_B) - O_A} \leq 1000 \ell \mid O_B=\bot} \leq q.
			\end{align*}
			(i.e., if $O_B=\bot$, Bob can't estimate the value of $O_A$ with error $\leq 1000 \ell$).
		\end{enumerate}
	\end{itemize}
     We say that a channel ensemble $C=\set{C_\pk}_{\pk\in N}$ is a {\sf computational approximate weak erasure channel}, denoted $(\ell,\alpha,p,q)$-\CompAWEC, if for every \ppt algorithm $\Dc$ and every sufficiently large $\pk\in\N$, $C_\pk$ satisfies the properties stated in the items above. A protocol $\Gamma$ is said to be $(\ell,\alpha,p,q)$-$\CompAWEC$, if the ensemble induced by the protocol (that is, $C=\set{C_{\Gamma(\pk)}}_{\pk\in\N}$) is $(\ell,\alpha,p,q)$-$\CompAWEC$.  
\end{definition}

We will make use of the following lemma, which shows that for some choices of the parameters, \AWEC implies \WEC. The lemma is proven in \cref{sec:AWEC-to-WEC}.

\begin{lemma}\label{lemma:AWEC-to-WEC}
	For every $\ell> 0$, there exists a \ppt protocol $\Lambda = (\Pc_1,\Pc_2)$ such that given an oracle access to an $(\ell,\alpha,p,q)$-\AWEC $C$, the channel $\tilde{C}$ induced by $\Lambda^C$ is $(\alpha'=\alpha+0.001,\: p' = p ,\:  q' = 1/2 + 2(q+0.01))$-\WEC.
	Furthermore, the proof is constructive in a black-box manner:
	\begin{enumerate}
		\item There exists an oracle-aided \ppt algorithm $\Ec_1$ such that for every channel $C = ((\OA,\VA), (\OB,\VB))$ and algorithm $\Dc$ violating the \WEC secrecy property~\ref{WEC:item:A} of $\tilde{C}$, algorithm $\Ec_1^{\Dc}$ violates the \AWEC secrecy property~\ref{AWEC:item:A} of $C$.
		
		\item There exists an oracle-aided \ppt algorithm $\Ec_2$ such that for every channel $C = ((\OA,\VA), (\OB,\VB))$ and algorithm $\Dc$ violating the \WEC secrecy property~\ref{WEC:item:B} of $\tilde{C}$, algorithm $\Ec_2^{\Dc}$ violates the \AWEC secrecy property~\ref{AWEC:item:B} of $C$.
	\end{enumerate}
\end{lemma}

Since \cref{lemma:AWEC-to-WEC} is constructive, the following is an immediate corollary.
\begin{corollary}\label{cor:CompAWEC to CompWEC}
There exists an oracle aided \ppt protocol $\Lambda$, such that given a protocol $\Gamma$ that induces $(\ell,\alpha,p,q)$-\CompAWEC, it holds that $\Lambda^\Gamma$ is $(\alpha'=\alpha+0.001,\: p' = p ,\:  q' = 1/2 + 2(q+0.01))$-\CompWEC.  
\end{corollary}
\begin{proof}[Proof of \ref{cor:CompAWEC to CompWEC}]
Let $\Lambda$ be the \ppt algorithm guaranteed  by Lemma \ref{lemma:AWEC-to-WEC}. Given an $(\ell,\alpha,p,q)$-\CompAWEC protocol $\Gamma$, we define $\Lambda(\pk)={\Lambda^{\Gamma(\pk)}(\pk)}$. Assume towards a contradiction that $\Lambda$ is not a $(\alpha',p',q')$-\CompWEC. It follows that there exists a \ppt $\Dc$ that for infinity many $\pk\in\N$ contradicts one of the \WEC secrecy properties of channel ensemble $\set{C_{\Lambda(\pk)}}_{\pk\in\N}$. Fix $\pk\in\N$ for which this holds. By Lemma \ref{lemma:AWEC-to-WEC}, there exists a \ppt $\Ec^\Dc$ that for every such $\pk$  contradicts one of the secrecy properties of the channel $C_{\Gamma(\pk)}$. This implies that for infinity many $\pk\in\N$, $\Ec^\Dc$  contradict the secrecy of the channel ensemble $\set{C_{\Gamma(\pk)}}_{\pk\in\N}$, which is a contradiction since this would means that $\Gamma$ is not a $(\ell,\alpha,p,q)$-\CompAWEC.       
\end{proof}



\subsection{Oblivious Transfer (\OT)}

\paragraph{Secure Computation.}
We use the standard notion of securely computing a functionality, \cf  \cite{Goldreich04}.
\begin{definition}[Secure computation]\label{def:SFE}
	A two-party protocol {\sf securely computes a functionality $f$}, if it does so according to the real/ideal paradigm.   We add the term perfectly/statistically/computationally/non-uniform computationally, if the simulator's output is  perfect/statistical/computationally indistinguishable/  non-uniformly indistinguishable from  the real distribution.  The protocol have the above notions of security {\sf against semi-honest  adversaries}, if its security only  guaranteed to holds against an adversary that follows the prescribed protocol.   Finally, for the case of perfectly secure computation, we naturally apply the above notion also to the non-asymptotic case: the protocol with no security parameter perfectly  compute a functionality $f$.
	
	A two-party protocol {\sf securely computes a functionality ensemble $f$ with oracle to a channel $C$}, if it does so according to the above definition when the parties have access to a trusted party computing $C$. All the above adjectives naturally extend to this setting.
\end{definition}

\paragraph{Oblivious Transfer.}
The (one-out-of-two) oblivious transfer functionality is defined as follows.
\begin{definition}[oblivious transfer functionality $f_{\OT}$]\label{def:OTfunc}
	The oblivious transfer functionality over $\zo \times (\zs)^2$ is defined by  $f_{\OT} (i,(\sigma_0,\sigma_1)) = (\perp,\sigma_i)$.
\end{definition}
A protocol is $\ast$ secure OT,   for \\$\ast\in \set{\text{semi-honest statistically/computationally/computationally non-uniform}}$, if it  compute the $f_{\OT}$  functionality with $\ast$ security.





% \begin{definition}[Computational oblivious transfer, semi-honest model]
% A protocol $\Pi=(\Ac,\Bc)$ is a semi-honest 1-out-of-2 computational oblivious transfer (comp-OT) protocol if the following holds. Given a common input $1^{\pk}$, the parties $\Ac$ and $\Bc$ run the protocol $\Pi(1^\pk)$ (in an honest manner) and    
% $\Ac$ outputs $X=(m_1,m_2)\in \zo\times\zo$ and has a view $U$ and $\Bc$ outputs $Y=(i,\hat{m})\in\zo\times\zo$ and has a view $V$, and the following properties are satisfied:
% \begin{enumerate}
%     \item \textbf{Correctness:} 
%     $\pr{\hat{m}\neq m_i}<\negl(\pk).$ 
    
%     \item \textbf{A's Privacy:} For every \ppt $\Dc$ and every sufficiently large $\pk$:
%     $\pr{\Dc(V)=m_{i-1}}<(1+\negl(\pk))/2$
    
%     \item \textbf{B's Privacy:} For every \ppt $\Dc$ and every sufficiently large $\pk$:
%     $\pr{\Dc(U)=i}<(1+\negl(\pk))/2$  
% \end{enumerate}
% \end{definition}

We make use of the following useful results by Wullschleger on oblivious transfer amplification from weak channels.
\begin{theorem}[\cite{Wullschleger09}, from \WEC to statistically secure \OT]\label{thm:WEC TO OT IT}
    There exists an oracle aided protocol $\Pi$ such that the following holds: Given a $(\alpha,p,q)$-\WEC $C$, if $44(\alpha+p)\le 1-q$ then $\Pi^{C}(1^\pk)$ is a semi-honest statistically secure \OT.
\end{theorem}

The following computational version of \cref{thm:WEC TO OT IT} is implicit in \cite{Wullschleger09} and is based on the computational proof explicitly stated in \cite{Wul07} (see Section 6 in \cite{Wullschleger09} for discussion).   

\begin{theorem}[\cite{Wullschleger09,   Wul07}, from \CompWEC to computinally secure \OT]\label{thm:WEC TO OT Comp}
    There exists an oracle aided protocol $\Pi$ such that the following holds: Given a $(\alpha,p,q)$-\CompWEC protocol $\Lambda$, if $44(\alpha+p)\le 1-q$ then $\Pi^{\Lambda}$ is a semi-honest computational secure \OT.
\end{theorem}



% \begin{definition}[Computational 1-out-of-2 Oblivious Transfer, semi-honest model]
% A protocol $\Pi=(\Ac,\Bc)$ is a semi-honest 1-out-of-2 $(\eps,\alpha,\beta)$-oblivious transfer (OT) protocol if the following holds. 

% The parties $\Ac$ and $\Bc$ run the protocol (in an honest manner) and    
% $\Ac$ outputs $X=(m_1,m_2)\in \zo\times\zo$ and has a view $U$ and $\Bc$ outputs $Y=(i,\hat{m})\in\zo\times\zo$ and has a view $V$, and following properties are satisfied:
% \begin{enumerate}
%     \item \textbf{Correctness:} 
%     $\pr{\hat{m}\neq m_i}<\eps.$ 
    
%     \item \textbf{A's Privacy:} For every adversary $\Dc$:
%     $\pr{\Dc(V)=m_{i-1}}<(1+\alpha)/2$
    
%     \item \textbf{B's Privacy:} For every adversary $\Dc$: $\pr{\Dc(U)=i}<(1+\beta)/2$  
% \end{enumerate}
% \end{definition}
To illustrate equilibria and dynamics of performative prediction games, we focus on a scenario in which a \emph{duopoly} of mortgage companies, i.e. banks, compete to sell loans to customers.

\paragraph{Customer Model:} In our game, each bank is trying to attract customers from a given population $\mathcal{P}$. We model this population as comprised of individuals with a single-dimensional type: we denote individual $j$'s type as $y_j \in [0,1]$. For simplicity, we assume that \(y\) represents the customer’s probability of repaying the loan\footnote{In practice, a customer's (normalized) credit score can be interpreted as a noisy observation of $y_j$. This also corresponds to credit scores being \emph{calibrated}.}, i.e., $y_j := \P[Y_j = 1]$, where $Y_j$ is a random variable such that $Y_j = 0$ means that $j$ defaults on their loan, and $Y_j = 1$ means they repay their loan. Customer types in the population are drawn from a known distribution $D_y$ supported on $[0,1]$. 

\paragraph{Game between Banks:} Each Bank \(i \in \{1, 2\}\) selects two parameters \( (\tau_i, \gamma_i) := \theta_i\), where:
\begin{itemize}
    \item \(\tau_i \in \{\tau_l,\tau_h\}\) is the credit score threshold for approving a customer\footnote{We restrict the bank to only pick between two thresholds, $\tau_l$ and $\tau_h$. However, we highlight how our results are affected when we expand the strategy space to $n > 2$ actions in our experiments of Appendix \ref{app:3gamma}.}. Specifically, a customer $j$ with credit score \(y_j\) is approved by Bank $i$ if and only if \(y_j \geq \tau_i\);
    \item \(\gamma_i \in \{\gamma_l, \gamma_h\}\) is the interest rate offered to approved customers.
\end{itemize}
We denote as shorthand the space of allowable thresholds by $\Gamma := [0,1]$ and allowable interests rates by $\Lambda := [0,1]$. %The latter is set without loss of generality---we simply normalize the rates to be at most $1$. 
% {\color{red} Vidya: just thinking about this but is it natural to restrict interest rate to $1$? I don't think it would affect the equilibrium structure of the game but theoretically I think the interest rate could be anything in $[0,\infty)$.} {\color{green} Guanghui: Could we say something like this is without loss of generality} \gua{changed.}\juba{I think we repeated this twice, the next sentence already had this}
The loan amount is normalized to $1$ in the entire paper, without loss of generality; in this case, if a customer chooses Bank $i$, and the customer is approved by the bank at an interest rate of $\gamma_i$, the expected utility for the bank is equal to
\[
(1+\gamma_i)\cdot \P[Y_i = 1]-\P[Y_i = 0] = (1+\gamma_i)y_i-(1-y_i).
\]


%In practice, the credit score \(y\) serves as a noisy observation of the true likelihood of the customer's repayment. 

\paragraph{Banks' Utilities:} For given parameter choices \(\theta_1 = (\tau_1, \gamma_1)\) by Bank 1 and \(\theta_2 = (\tau_2, \gamma_2)\) by Bank 2, a \emph{rational} customer with credit score $y$ acts as follows:

\begin{enumerate}
    \item \textbf{Qualified for a single bank}: 
        \begin{itemize}
        \item If \(\tau_1 \leq y < \tau_2\), the customer goes to Bank 1, as the score qualifies for Bank 1 but not Bank 2. Conversely, if \(\tau_2 \leq y < \tau_1\), the customer chooses Bank 2.
    \end{itemize}
    \item \textbf{Qualified for both banks}:
     \begin{itemize}
        \item If \(\tau_1, \tau_2 \leq y\) and \(\gamma_1 < \gamma_2\), the customer selects Bank 1 for its lower interest rate. Conversely, if \(\gamma_1 > \gamma_2\), the customer chooses Bank 2.
        \item If \(\gamma_1 = \gamma_2\), the customer picks each bank with probability $1/2$. 
    \end{itemize}
    \item \textbf{Unqualified for both banks}:
    \begin{itemize}
        \item If \(y < \tau_1\) and \(y < \tau_2\), the customer is rejected by both banks.
    \end{itemize}
\end{enumerate}

The expected reward for Bank 1, denoted as \(u_1(\theta_1, \theta_2)\), can then be expressed as:
\begin{align}\label{eq:utility}
    u_1(\theta_1, \theta_2) 
    &=  \mathbb{E}_{y \sim D_y} \left[ \mathbb{I}\{\underbrace{\tau_1 \leq y < \tau_2 \ \cup \ (\tau_1, \tau_2 \leq y \ \cap \ \gamma_1 < \gamma_2)}_{\text{accepted by Bank 1}}\} \cdot \big((1+\gamma_1)y - (1-y)\big) \right] \nonumber\\
    & + \frac{1}{2} \mathbb{E}_{y \sim D_y} \left[ \mathbb{I}\{\underbrace{\tau_1, \tau_2 \leq y \ \cap \ \gamma_1 = \gamma_2}_{\text{accepted by both Banks}}\} \cdot \big((1+\gamma_1)y - (1-y)\big) \right].
\end{align}
Note that the problem is \emph{symmetric}, i.e., the utility function for Bank 2 can be derived by swapping the roles of \(\theta_1\) and \(\theta_2\). I.e., $u_2(\theta_1, \theta_2) = u_1(\theta_2, \theta_1)$. 

% If a bank only attracts customers between thresholds $\tau_a$ and $\tau_b$, for $\tau_a<\tau_b$, we call $[\tau_a,\tau_b]$ the \emph{threshold} range for that bank. For example, if Bank $1$ sets a threshold of $\tau_1$, Bank $2$ a threshold of $\tau_2 > \tau_1$, and $\gamma_1 > \gamma_2$, then Bank 1 has a threshold range of $[\tau_1,\tau_2]$, while bank $2$ has a threshold range of $[\tau_2,1]$.
% Note that the parameters set by \emph{both} banks, i.e. $(\theta_1,\theta_2)$ both influence the threshold range for each of Bank 1 and 2.  If $\tau_1>\tau_2$, $\gamma_1>\gamma_2$, then $\tau_a>\tau_b$, and the bank does not attract any customers. 
% {\color{red} is it possible for $\tau_a > \tau_b$, leading to the bank never attracting customers?} \gua{if $\gamma_1>\gamma_2$, $\tau_1>\tau_2$, then it gets no customer. I think it also makes sense.}\juba{I think we said we wanted to delete the discussion of the threshold range, no?}

% \noindent \textbf{Discrete Model}   
% We now present the discrete version of our model, where the interest rates and thresholds are selected from finite sets \(\Gamma\) and \(\Lambda\), respectively, with $\tau\in[0,1], \gamma\in[0,1]$,  for all $\tau\in\Lambda$ and $\gamma\in\Gamma$, \(|\Gamma| = n\) and \(|\Lambda| = m\). Let \(p_1, p_2 \in \Delta(\Gamma \times \Lambda)\) represent the mixed strategies of the two banks, where \(\Delta(\Gamma \times \Lambda)\) denotes the set of probability distributions over the discrete decision space \(\Gamma \times \Lambda\).


% \begin{Remark}
%    Note that our proposed problem can be reformulated as a standard multi-player performative prediction problem \citep{narang2023multiplayer}. However, in our problem, the data distribution faced by each learner breaks the Lipschitzness assumption of previous work~\citep{hardt2023performative,narang2023multiplayer}. A small modification in one of the learner's thresholds can completely change how demand is allocated across both learners, as is often the case in Bertrand-style games. 
% \end{Remark} 

% \gua{I made some changes to Remark 1, please have a look}
\begin{Remark}
   Previous works in multi-learner performative prediction~\citep{narang2023multiplayer} resort to an insensitivity assumption, i.e., the data distribution faced by each player can only changes slightly when the parameters also change slightly; formally, the data distribution faced by each player is Lipschitz in their decisions. This is immediately not true in our setting: the bank slightly changing its parameters can completely changes the demand distribution of customers it faces. Intuitively, this is because of Bertrand-competition-style effects, where if two banks have similar rates, one bank that lowers their rate by a small amount suddenly captures the entire customer demand that is eligible for that rate.%\juba{made further light edits adding intuition}
   
   In Appendix \ref{Appendix:refumulation}, we discuss this problem more carefully by reformulating our problem in the standard multi-learner performative prediction form given by~\citep{narang2023multiplayer}. We show the distribution is not Lipschitz with respect to the parameters, and thus does not satisfy the insensitivity assumption. 
%Prior work~\citep{hardt2023performative,narang2023multiplayer} showed that, for a general multi-agent performative prediction framework to work, insensitivity assumptions are needed: in the \textbf{worst case}, they can construct settings where the insensitivity assumption does not hold and simple dynamics do not converge anymore. We add nuance to this picture. We will show that our dynamics often converge, even absent insensitivity assumptions, highlighting that while the impossibility results of previous work hold in the worst case, they may not hold in the ``average case'' and especially not in problems motivated by applications. In particular, we will show convergence to a variety of equilibria of our game, and often to symmetric Nash equilibria where insensitivity is immediately violated.
     
\end{Remark}



% \paragraph{Relationship to Performative Prediction} A central point of our work is to highlight that \textcolor{red}{needs writing from intro}. We highlight how our work specifically ties to ``Performative Prediction'' below:


%\textcolor{red}{needs a definition environment}



%Here, \(\E_{\theta_1, \theta_2}\) represents the expected utility of the banks over their respective strategies \((\theta_1, \theta_2)\). These inequalities ensure that neither bank can unilaterally improve its expected utility by deviating from its mixed strategy in the equilibrium.



%and  for all $\tau\in\Gamma$, we have $\tau\in\Lambda$, $(\tau,\gamma)\in[0,1]^2$. Let $\Gamma\times\Lambda$
%In this paper, we focus on the most fundamental case, where there are two choices for each parameter: $0\leq\tau_{\ell}<\tau_{h}\leq 1$, and $0\leq \gamma_{\ell}< \gamma_{h}\leq 1$. In this case, the utility for each pair of decisions forms a $4\times4$ matrix (given in Table \ref{tab:my-table}). We consider the canonical case where $\tau_{\ell}=\frac{1}{2+\gamma_{h}}$, and $\tau_{h}=\frac{1}{2+\gamma_{\ell}}.$ Note that these are natural choices for the thresholds, in the sense that, if there is only one bank and the interest rate is set to be $\gamma$, then $\frac{1}{2+\gamma}$ is the optimal threshold corresponding to the fixed $\gamma$.


%and the thresholds are chosen in $\Lambda=\{\tau^{(1)},\dots,\tau^{(m)}\}$. Here, we only assume that, for each $\gamma\in\Gamma$, there at least exist one $\tau\in\Lambda$ such that $f(\gamma,\tau,1)>0$. Note that this is a very minor assumption, in the sense that, if for a $\gamma$ such that $f(\gamma,\tau,1)<0$ for all $\tau\in\Lambda$, then adopting this decision will lead to negative utility regardless of the opponent's decision, and thus is not an interesting case. 

%\textcolor{red}{The model section is missing the dynamic version of the game. We should clearly define the one-shot and the dynamic game}
% we only considered one-shot case in our paper



\section{Experiments}\label{sec_exp}
%\hp{Accelerating IM simulation~\cite{tang2015influence}}

% \begin{itemize}
%     \item 6.1. Problem setting of three COPs, including the general model and three specific CO problems 
%     \item 6.2. Experiment Setting (hyperparameters, details of training, evaluation, and test) 写在appendix里吧
%     \item 6.3. Performance analysis 这个要占半页
% \end{itemize}

%\hp{need to think of a way to compress these tables / visuals.} 

%\hp{\cancel{Baselines}; hyperparamters; \cancel{metrics}; etc.}

With theoretical guarantees on the existence and convergence of NE for ACCES games, we are also interested in how our proposed algorithm CCDO-RL works empirically. To evaluate this, we conduct experiments of CCDO-RL on three distinct ACCES game instances introduced in Section \ref{sub_exp_ins} and analyze the performance of CCDO-RL in Section \ref{sub_train_eval}. Section 6.2.1 aims to empirically demonstrate the convergence (Figures \ref{fig_exploit_20} and \ref{fig_exploit_50}) of the algorithm CCDO-RL over realistic CO problems, and show its consistency with Theorem \ref{CCDOA}. Section 6.2.2 intends to show the average reward (to seen training graphs) as well as the generalizability (to unseen test graphs) of the combinatorial player in real-world ACCES games (shown in Tables \ref{tab_aver}, and \ref{tab_gene}).

\subsection{Three Instances of ACCES Games} \label{sub_exp_ins}
% \hp{This para does not make much sense. Need to follow the framework in the Preliminaries section.}
% For combinatorial optimization problems in real-world applications, situations are more complicated and intractable due to changeable environmental or physical parameters. The form of parameter sets is very crucial because different types have different solvability and computation complexity. Forms of parameter sets mainly contain discrete sets, interval sets \cite{buchheim2018robust} like polyhedral and ellipsoid, probability distributions \cite{carlsson2018wasserstein}, and variable functions \cite{krause2008robust}.

% In reality, these parameters are often impacted by some common factors, such as conditions of weather, transportation, and individual personalities. \cite{kalimeris2019robust} proposed an assumption that real instances (e.g. demands in CVRP, coverages in CSP) 
%Considering affected or attacked COPs, the real instance $\{\theta_{i}\}$ always relied on the estimated value $\{\hat{\theta}_{i}$\} and the variation determined by independent factors $\{g_{i}\}$ and environment/physical parameters/attacker actions $\{\eta\}$. The concrete parameter influence model is stated as follows:

We consider a certain COP which is parameterized with $\{\theta_{i}\}$, where $i$ is the index of nodes (such as a target in security games) -- e.g., such parameters can be interpreted as attack probability of targets.
%coverage radius, customer's demands, or attack probability of targets. 
In real-world applications, we often need to estimate such parameters before solving the COPs. Unfortunately, the estimation $\{\hat{\theta}_{i}\}$ often bears a gap to the true value $\{\theta_{i}\}$, which derives from e.g. environment (aleatoric) uncertainty, model (epistemic) uncertainty, or an attacker trying to manipulate the defender's utility. We use a generic model to formulate this gap:
\begin{equation}\label{linrob}
    \theta_{i} = \hat{\theta}_{i} + y \cdot \tau_{i},
\end{equation}
where $y$ represents the strategy of the nature/attacker, $\tau_{i}$ is the environment factors like weather and transportation conditions, or human subjective factors like the preference of the attacker. 
Such abstraction can represent a wide range of ACCES games, such as facility location covering problems \cite{an2020battery, TIRKOLAEE2020340}, CVRP \cite{vehiclerouting.ch8,dinh2018exact, FLORIO20231081}, security patrolling (OP) \citep{xu2021robust}, and influence maximization problem \cite{kalimeris2019robust}. We describe three instances of ACCES games based on the model (\ref{linrob}).%Based on this model (\ref{linrob}), we focus on three combinatorial optimization problems with attacks or environmental/physical influence.

% \hp{Hard to follow. We should point out what are the two players, what are X, Y, u etc}

\textbf{Adversarial Covering Salesman Problem (ACSP):} In a map of cities, every city $i$ has a coverage $\theta_{i}$. A salesman finds the shortest path such that all cities are visited or covered, with $\theta_{i}$ influenced by physical factors $\tau_i$ and transportation parameters $y$ based on Eq.(\ref{linrob}). The salesman is Player 1 where $X$ consists of the feasible paths of the salesman. Nature is Player 2 with $Y$ = $[0, 1]^K \ni y, K \in \mathbb{N}$. The utility function of Player 1 $u$ is the opposite of the total traveling distance.

\textbf{Adversarial Capacitated Vehicle Routing Problem (ACVRP):} A vehicle with a constrained capacity of goods finds the shortest path under the worst case with the $i_{th}$ customer's demand $\theta_i$ changed by environmental factors $\tau_i$ and weather parameter $y$ on Eq.(\ref{linrob}). The vehicle is Player 1 where $X$ is the set of the feasible path $x$. Nature is Player 2 where $Y$ is $[0, 1]^K \ni y, K \in \mathbb{N}$. The utility function of Player 1  $u$ is the opposite of total delivery distance satisfying all the demands of customers.


\textbf{Patrolling Game (PG):} The patrolling game is described in the introduction.

For all the problem instances, we run our algorithm on two problem sizes: 20 nodes and 50 nodes. The detailed description and problem parameters of the three game instances are in Appendix \ref{app_ex_para_set}.

% Similarly, in the vehicle route problem (VRP), conditions with correlated parameters arouse broad attention from scholars \cite{vehiclerouting.ch8,dinh2018exact,FLORIO20231081}. \cite{dinh2018exact} considered the demand correlation by geographical proximity of nodes, described by some independent random variables in the fractional form. \cite{FLORIO20231081} utilized 'external factors' to stand for unknown covariates affecting all demands and presented a Bayesian model to learn correlations. Further more, about IM problems, \cite{kalimeris2019robust} combined node features and uncertain hyperparameters to fit the influence probability on each edge.

% \subsection{Training CCDO-RL}

% For all the problems, CCDO-RL adopts the REINFORCE algorithm with an attention-based encoder-decoder framework \cite{kool2018attention} (used as an inductive graph representation component) to learn a (generalizable) COP solver for one player (protagonist), and PPO \cite{schulman2017proximal} to train a policy for the other player (adversary) whose strategy space is continuous. CCDO-RL is trained with 50 epochs on a set of 10,000 graphs (with 20 or 50 nodes). The hyperparameters of CCDO-RL are specified in Appendix \ref{app_ex_para_set} (Table \ref{tab_hyper_ccdorl}). Our code is included as supplementary material for ease of reproduction. 
% % \hp{need to specify hyperparas}

\subsection{Performance of CCDO-RL}\label{sub_train_eval}

Two aspects are evaluated for the performance of CCDO-RL, i.e., i) Convergence to NE (Section \ref{sub_per_conver}) exploring whether CCDO-RL can compute the NE, and ii) Protagonist policy's average reward and generalizability (Section \ref{sub_per_rob}). Generalizability refers to the ability of RL models trained on previously seen graphs (problem instances), to perform well on a new set of unseen test graphs. The model’s usability is enhanced by generalizability, rather than focusing solely on the average reward, which is a critical motivation in the literature on RL for COPs \citep{khalil2017learning, kool2018attention}.

For all the problems, CCDO-RL adopts the REINFORCE algorithm with an attention-based encoder-decoder framework \citep{kool2018attention} (used as an inductive graph representation component) to learn a generalizable COP solver for Player 1 (protagonist), and PPO to train a policy for Player 2 (adversary) whose strategy space is continuous. CCDO-RL is trained on a set of 10,000 graphs (with 20 or 50 nodes). The hyperparameters of CCDO-RL are specified in Appendix \ref{app_ex_para_set} (Table \ref{tab_hyper_ccdorl}). Our code is included as supplementary material and will be open-sourced for ease of reproduction. 

% \textbf{Training.} For all the problems, CCDO-RL adopts the REINFORCE algorithm with attention-based encoder-decoder framework \cite{kool2018attention} (used as an inductive graph representation component) to learn a (generalizable) COP solver for one player (protagonist), and PPO \cite{schulman2017proximal} to train a policy for the other player (adversary) whose strategy space is continuous. CCDO-RL is trained with 50 epochs on a set of 10,000 graphs (with 20 or 50 nodes). 

% \hp{We should first present results about convergence as it is mostly aligned with the theory.}

\subsubsection{Convergence to NE} \label{sub_per_conver}

Exploitability is a common metric to describe the closeness to true NE by calculating the sum of performance distances between each new best response and subgame NE, i.e. $\sum_{i=1,2} U(\pi_{i,k}^{br}, \sigma_{-i,k}) - U(\sigma)$ in the general two-player game. Since our game is zero-sum, the calculation is as follows:
\begin{equation*}
   \text{Exploitability}(\sigma) = \max_{\pi_1 \in \Sigma_1} U(\pi_1, \sigma_{2}) - \min_{\pi_2 \in \Sigma_2} U(\sigma_1, \pi_2).
\end{equation*}
From Figure \ref{fig_exploit_20}, we can see that CCDO-RL can converge to approximate NE in 25 iterations or less (in the PG setting), reaching 0.05 in ACSP, 0.10 in ACVRP, and 0.03 in PG with 20 nodes. Similar results are observed in problems with 50 nodes (see Figure \ref{fig_exploit_50} in Appendix \ref{app_exp}). These results validate the effectiveness of CCDO-RL in finding the NE for various types of games.

%Similarly, the exploitability of three COPs in 50 nodes is provided in the appendix \ref{app_exp}.
\vspace{-\baselineskip}
\begin{figure}[htbp]
	\centering
    \subfigure[ACSP20]{
    \label{csp20_nashconv}
    \includegraphics[scale=0.20]{Figures/nashconv_log_csp20_sm_7.eps}
    }
    \subfigure[ACVRP20]{
    \label{cvrp20_nashconv}%文中引用该图片代号
    \includegraphics[scale=0.20]{Figures/nashconv_log_svrp20_sm_7.eps}
    }
    \subfigure[PG20]{
    \label{opsa20_nashconv}
    \includegraphics[scale=0.20]{Figures/nashconv_log_pg20_sm_7.eps}
    }
    \caption{Exploitability curve of CCDO-RL on three games of 20 nodes}
    \label{fig_exploit_20}
\end{figure}
\vspace{-\baselineskip}
\subsubsection{Average reward and Generalizability of Combinatorial player} \label{sub_per_rob}
% \subsubsection{Robustness and Generalizability of Protagonist Policy} \label{sub_per_rob}
%\hp{CCDO-RL being better in these following metrics is only kind of a by-product.}

% \textbf{Evaluation.} The learned policies are then tested on 200 graphs, where 100 of them are randomly selected from the 10,000 training graphs, and the other 100 are unseen graphs. 
% We use two metrics to evaluate the performance of different policies for the protagonist player: \textbf{Average proportional loss} $R-$ describes the policy overfitting degree \citep{lanctot2017unified}; \textbf{Reward} evaluates the performance of the protagonist with the adversary under three COPs.  
% \begin{eqnarray}
%         &R- = (\hat{D} - \hat{O}) / \hat{D}.
% \end{eqnarray}
% in which $\hat{D}$ is the mean value of the diagonals and $\hat{O}$ is the mean value of the off-diagonals in the payoff matrix provided in the Appendix \ref{app_exp}.

% Because the protagonist policy is trained against a powerful adversary under our ACCES game setting, the obtained policy is naturally robust against adversarial perturbations. This subsection sheds a bit of light on this perspective and quantifies the extent of robustness of CCDO-RL as well as the ability of RL to generalize to unseen test graphs.

\textbf{Evaluation.} The learned policies are tested on 200 graphs, with 100 being randomly selected from the 10,000 training graphs (to show the average reward), and the other 100 being unseen graphs (to test policy generalization). We evaluate the performance of the protagonist with the adversary under three COPs. For each COP, the performance is considered both on the 20-node and 50-node map.
% We use two metrics to evaluate the performance of different policies for the protagonist player: \textbf{Average proportional loss} $R-$ describes the policy overfitting degree \citep{lanctot2017unified}; \textbf{Reward} evaluates the performance of the protagonist with the adversary under three COPs.

\textbf{Baselines.} There are heuristic algorithms for each game instance (Heuristic in Table \ref{tab_aver} and \ref{tab_gene}) and a single-player RL algorithm. For ACVRP, we adopt the Tabu Search algorithm (Tabu) \citep{li2020improved} as the heuristic algorithm, which is widely applied in the routing problem. For ACSP, the common benchmark local search algorithm, LS2 \citep{golden2012generalized}, is used. For PG, we choose the greedy algorithm as the baseline. The "RL against Stoc" algorithm in Tables \ref{tab_aver} and \ref{tab_gene} is identical to the protagonist model in CCDO-RL but trained in environments with stochastic adversarial perturbations.

% \textbf{Baselines.} There are a heuristic algorithms for each game instance {\color{red} (Heuristic mentioned in the Table \ref{tab_aver} and \ref{tab_gene})} and a single-player RL algorithm. For ACVRP, we adopt the Clarke-Wright (CW) algorithm \citep{pichpibul2013heuristic} and the Tabu Search algorithm (Tabu) \citep{li2020improved} as heuristics, which are applied widely in the routing problem. For ACSP, two common benchmark local search algorithms, LS1 and LS2 \citep{golden2012generalized}, are used. For PG, we choose a local search algorithm \citep{vansteenwegen2009iterated} and the greedy algorithm as the heuristic baselines. {\color{red} The "RL  against Stoc" algorithm referred to Tables \ref{tab_aver} and \ref{tab_gene}} is identical to the protagonist model in CCDO-RL {\color{red} but trained on environments with stochastic adversarial perturbations.} 

\textbf{Average Reward.}  As illustrated in Table \ref{tab_aver}, our algorithm achieves a better average reward than baselines (10.08\% improvement on average of all settings against two baselines), regardless of CO instance or problem size, when confronting the adversary trained by CCDO-RL. In the setting of CSP-20 nodes, the average reward is improved by 46.98\% compared to the heuristic and by 7.14\% compared with the RL against Stoc. For the 50-node setting, the improvements are 45.91\% and 5.28\% respectively. Similarly, the improvements in contrast to Heuristic and RL against Stoc are as follows: 1.72\% and 3.01\%  for CVRP-20 nodes, 0.75\% and 4.46\% for CVRP-50 nodes, 4.17\% and 1.48\% for PG-20 nodes, and 10.60\% and 4.38\% for PG-50 nodes.

\textbf{Generalizability.} From Table \ref{tab_gene}, CCDO-RL continues to achieve a better average reward when facing the adversary, demonstrating that the learned RL policies generalize well to unseen graphs. Even though the non-RL baselines do have access to the graph structures and other problem information of the unseen problem instances, CCDO-RL can obtain comparable performances without re-training on the new problem instances. The improvements versus Heuristic and RL against Stoc are 46.61\% and 7.02\% for CSP-20 nodes, 42.24\% and 3.94\% for CSP-50 nodes, 1.12\% and 1.56\% for CVRP-20 nodes, 0.90\% and 5.05\% for CVRP-50 nodes, 5.35\% and 2.40\% for PG-20 nodes, and 12.17\% and 10.33\% for PG-50 nodes. Even when confronting the stochastic adversary, CCDO shows superior generalizability compared to two baselines across three COPs, with average improvements of 6.31\%, 3.42\%, and 3.95\% respectively. Detailed results are provided in Appendix \ref{app_exp} (Tables \ref{tab_csp_full_20} - \ref{tab_op_full_50}). 
% The model’s usability is enhanced by the ability to generalize rather than focusing solely on the average reward, which is a critical motivation of the RL for combinatorial optimization literature \citep{khalil2017learning, kool2018attention}.  

\begin{remark}
    In CO problems (or more broadly, operations research and economics), it is known that achieving solution quality improvements against strong baselines (e.g., the RL methods trained with a stochastic adversary) is very challenging, and the margins are usually small \citep{kool2018attention}, sometimes even less than 1\%. However, these “tiny” marginal improvements in profits keep small business owners in the real world alive. Last, the improvement depends a lot on the problem settings, and we show that sometimes the improvement can be much more significant.
\end{remark}
\vspace{-\baselineskip}
% \textbf{Performance analysis.} The robustness results of CCDO-RL for ACSP are shown in Table \ref{tab_csp}. We have the following observations: 1) On both of the 100 seen/unseen graphs, single-player RL performs better than heuristic algorithms no matter whether attacked or not. (2) When confronting the adversary trained by CCDO-RL, CCDO-RL exceeds RL by 0.25 and 0.24 on the training set, and by 0.25 and 0.18 on the test set, respectively under the 20-node and 50-node graphs. This demonstrates the robustness of CCDO-RL. 3) Compared to the performance of the training set with that of the test set, we can see that RL and CCDO-RL both maintain a certain degree of generalization. Similar results for ACVRP (Table \ref{tab_cvrp}) and SPG (Table \ref{tab_op}) are provided in Appendix \ref{app_exp}. 

\begin{table}[ht]
  \caption{Average reward against CCDO-RL's adversary (on seen graphs)}
  \vspace{\baselineskip}
  \label{tab_aver}
  \centering
  \small
  \begin{tabular}{lllllll}
    \toprule
    \multirow{2}{*}{method} & \multicolumn{2}{c}{ACSP (Mean$\pm$Std)} & \multicolumn{2}{c}{ACVRP (Mean$\pm$Std)} & \multicolumn{2}{c}{PG (Mean$\pm$Std)} \\
    \cmidrule(r){2-3} \cmidrule{4-5} \cmidrule(r){6-7}
                            & 20 nodes & 50 nodes & 20 nodes & 50 nodes & 20 nodes & 50 nodes\\
    \midrule
    Heuristic & 6.13$\pm$1.20 & 7.55$\pm$1.42 & 7.65$\pm$1.23  & 13.38$\pm$1.70 & 2.64$\pm$1.03 & 4.53$\pm$1.84   \\
    RL against Stoc    & 3.50$\pm$0.47  & 4.55$\pm$0.62  & 7.55$\pm$1.16  & 13.90$\pm$1.63 & 2.71$\pm$0.90 & 4.80$\pm$2.18   \\
    CCDO-RL   & $\pmb{3.25}$$\pm$0.42 & $\pmb{4.31}$$\pm$0.51  & $\pmb{7.42}$$\pm$1.21  & $\pmb{13.28}$$\pm$1.52 &  $\pmb{2.75}$$\pm$0.87 & $\pmb{5.01}$$\pm$1.91  \\
    \bottomrule
  \end{tabular}
\end{table}
\vspace{-\baselineskip}

\begin{table}[htp]
  \caption{Generalizability against CCDO-RL's adversary (on unseen graphs)}
  \vspace{\baselineskip}
  \label{tab_gene}
  \centering
  \small
  \begin{threeparttable}
  \begin{tabular}{lllllll}
    \toprule
    \multirow{2}{*}{method} & \multicolumn{2}{c}{ACSP (Mean$\pm$Std)} & \multicolumn{2}{c}{ACVRP (Mean$\pm$Std)} & \multicolumn{2}{c}{PG (Mean$\pm$Std)} \\
    \cmidrule(r){2-3} \cmidrule{4-5} \cmidrule(r){6-7}
                            & 20 nodes & 50 nodes & 20 nodes & 50 nodes & 20 nodes & 50 nodes\\
    \midrule
    Heuristic & 6.20$\pm$1.33 & 7.60$\pm$1.37   & 7.64$\pm$1.30  & 13.27$\pm$1.87 & 2.43$\pm$0.98 & 4.19$\pm$1.69    \\
    RL against Stoc  & 3.56$\pm$0.37  & 4.57$\pm$0.58  & 7.67$\pm$1.30  & 13.85$\pm$1.53 &  2.50$\pm$0.95 & 4.26$\pm$2.17 \\
    CCDO-RL   & $\pmb{3.31}$$\pm$0.35 & $\pmb{4.39}$$\pm$0.52  & $\pmb{7.55}$$\pm$1.28  & $\pmb{13.15}$$\pm$1.59 & $\pmb{2.56}$$\pm$0.92 & $\pmb{4.70}$$\pm$1.94\\

    \bottomrule
  \end{tabular}
  \begin{tablenotes}
      \footnotesize
      \item[1] For the average reward of ACSP and ACVRP, smaller is better while for that of PG larger is better.
  \end{tablenotes}
  \end{threeparttable}
\end{table}
\vspace{-\baselineskip}
% two heuristics and one RL
% \begin{table}[ht]
%   \caption{{\color{red} Average reward of CCDO-RL (on seen graphs). For the value of CSP and CVRP, larger is better while for that of PG smaller is better.}}
%   \label{tab_aver}
%   \centering
%   \small
%   \begin{tabular}{lllllll}
%     \toprule
%     \multirow{2}{*}{method} & \multicolumn{2}{c}{CSP (Mean$\pm$Std)} & \multicolumn{2}{c}{CVRP (Mean$\pm$Std)} & \multicolumn{2}{c}{PG (Mean$\pm$Std)} \\
%     \cmidrule(r){2-3} \cmidrule{4-5} \cmidrule(r){6-7}
%                             & 20 nodes & 50 nodes & 20 nodes & 50 nodes & 20 nodes & 50 nodes\\
%     \midrule
%     Baseline 1 & 4.52$\pm$0.71  & 5.98$\pm$0.94 & 7.64$\pm$1.56  & 13.49$\pm$2.10 & 2.71$\pm$1.10 & 1.82$\pm$1.40   \\
%     Baseline 2 & 6.13$\pm$1.20 & 7.55$\pm$1.42   & 7.65$\pm$1.23  & 13.38$\pm$1.70 & 2.64$\pm$1.03 & 1.47$\pm$0.99  \\
%     RL {\color{red}against Stoc}    & 3.50$\pm$0.47  & 4.55$\pm$0.62  & 7.55$\pm$1.16  & 13.90$\pm$1.63 & 2.71$\pm$0.90 & 1.54$\pm$1.03   \\
%     CCDO-RL   & $\pmb{3.25}$$\pm$0.42 & $\pmb{4.31}$$\pm$0.51  & $\pmb{7.42}$$\pm$1.21  & $\pmb{13.28}$$\pm$1.52 &  $\pmb{2.75}$$\pm$0.87 & $\pmb{1.87}$$\pm$1.22  \\
%     \bottomrule
%   \end{tabular}
% \end{table}


% \begin{table}[htp]
%   \caption{{\color{red}Generalizability of CCDO-RL (on unseen graphs)}}
%   \label{tab_gene}
%   \centering
%   \small
%   \begin{threeparttable}
%   \begin{tabular}{lllllll}
%     \toprule
%     \multirow{2}{*}{method} & \multicolumn{2}{c}{CSP (Mean$\pm$Std)} & \multicolumn{2}{c}{CVRP (Mean$\pm$Std)} & \multicolumn{2}{c}{PG (Mean$\pm$Std)} \\
%     \cmidrule(r){2-3} \cmidrule{4-5} \cmidrule(r){6-7}
%                             & 20 nodes & 50 nodes & 20 nodes & 50 nodes & 20 nodes & 50 nodes\\
%     \midrule
%     Baseline 1 & 4.53$\pm$0.79  & 5.95$\pm$0.96 & 7.55$\pm$1.39  & 13.35$\pm$2.04 & 2.52$\pm$1.08 & $\pmb{1.86}$$\pm$1.44  \\
%     Baseline 2 & 6.20$\pm$1.33 & 7.60$\pm$1.37   & 7.64$\pm$1.3  & 13.27$\pm$1.87 & 2.43$\pm$0.98 & 1.52$\pm$1.20    \\
%     RL {\color{red}against Stoc}  & 3.56$\pm$0.37  & 4.57$\pm$0.58  & 7.67$\pm$1.30  & 13.85$\pm$1.53 &  2.50$\pm$0.95 & 1.03$\pm$5.05 \\
%     CCDO-RL   & $\pmb{3.31}$$\pm$0.35 & $\pmb{4.39}$$\pm$0.52  & $\pmb{7.55}$$\pm$1.28  & $\pmb{13.15}$$\pm$1.59 & $\pmb{2.56}$$\pm$0.92 & 1.35$\pm$5.09\\

%     \bottomrule
%   \end{tabular}
%   \begin{tablenotes}
%       \footnotesize
%       \item[1] For the value of CSP and CVRP, larger is better while for that of PG smaller is better.
%   \end{tablenotes}
%   \end{threeparttable}
% \end{table}

\section*{Conclusion}
This paper aims to enhance our understanding of the computational complexity of computing various Shapley value variants. We found that for various ML models --- including decision trees, regression tree ensembles, weighted automata, and linear regression --- both local and global interventional and baseline SHAP can be computed in polynomial time under HMM modeled distributions. This extends popular algorithms, such as TreeSHAP, beyond their empirical distributional scope. We also establish strict complexity gaps between the various SHAP variants (baseline, interventional, and conditional) and prove the intractability of computing SHAP for tree ensembles and neural networks in simplified scenarios. Overall, we present SHAP as a versatile framework whose complexity depends on four key factors: \begin{inparaenum}[(i)] \item model type, \item SHAP variant, \item distribution modeling approach, \item and local vs. global explanations\end{inparaenum}. We believe this perspective provides deeper insight into the computational complexity of SHAP, paving the way for future work.




%We believe that our framework provides a more intricate understanding of SHAP computation complexity across different models, distributions, and variants, paving the way for further research.

Our work opens promising directions for future research. First, expanding our computational analysis to other SHAP-related metrics, such as asymmetric SHAP~\citep{frye20} and SAGE~\citep{covert2020understanding}, would be valuable. Additionally, we aim to explore more expressive distribution classes and relaxed assumptions beyond those in Section \ref{sec:tractable} while maintaining tractable SHAP computation. Finally, when exact computation is intractable (Section \ref{sec:intractable}), investigating the approximability of SHAP metrics through approximation and parameterized complexity theory~\citep{downey2012parameterized} is an important direction.

%Our work opens several promising avenues for future research on the computational properties of explainable AI methods, with a particular focus on SHAP. First, it would be interesting to broaden the computational analysis conducted in this work to include other popular SHAP-related metrics in the literature, such as asymmetric SHAP \cite{frye20} and SAGE \cite{covert2020understanding}. Also, in the future, we aim to explore more expressive distribution classes and relaxed distributional assumptions—extending beyond those examined in Section \ref{sec:tractable} —that still yield tractable SHAP computation. Finally, when exact computation proves intractable (Section \ref{sec:intractable}), it is worthwhile to theoretically investigate the question of the approximability of computing the SHAP metrics across various configurations, through the lens of approximation and parametrized complexity theory \cite{arora2009computational}.

%This paper aims to deepen our understanding of the computational complexity involved in obtaining different Shapley value variants. We found that for a variety of ML models, including decision trees, tree ensembles for regression, weighted automata, and linear regression models — computing both local and global interventional and baseline SHAP can be done in polynomial time when distributions are modeled by HMMs. This extends the distributional scope of popular algorithms like TreeSHAP, which is limited to empirical distributions. Additionally, we demonstrate a strict complexity gap between SHAP variants, showing that interventional and baseline SHAP can be strictly easier to compute than conditional SHAP. Despite these positive results, we uncovered intractability for various SHAP variants in neural networks and tree ensembles. Finally, we provided generalized complexity relations across SHAP variants. We believe that our framework offers a deeper understanding of the complexity involved in computing SHAP across various variants, models, distributions, as well as in both local and global computations, laying the groundwork for future research.
% \newpage

% \input{main_content/__impact_statement__}

% \newpage
\bibliography{icml2025}
\bibliographystyle{icml2025}


%%%%%%%%%%%%%%%%%%%%%%%%%%%%%%%%%%%%
%%%%%%%%%%%%%%%%%%%%%%%%%%%%%%%%%%%%
% APPENDIX
%%%%%%%%%%%%%%%%%%%%%%%%%%%%%%%%%%%%
%%%%%%%%%%%%%%%%%%%%%%%%%%%%%%%%%%%%
% \newpage
\appendix
\onecolumn

\section{Experiment Settings}
\label{appendix:exp_settings}

RoboCasa is a state-of-the-art simulation framework developed to advance the training of generalist robots in diverse and realistic household settings, particularly in kitchen environments. It comprises 120 meticulously modeled kitchen layouts, over 2,500 high-quality 3D objects spanning 150 categories, and 25 foundational atomic tasks that are the building blocks for robot learning. These atomic tasks encompass essential sensorimotor skills, including pick-and-place, opening and closing doors or drawers, twisting knobs, turning levers, pressing buttons, performing insertions, and navigating kitchen spaces. In our work, we evaluated our model in 24 of these tasks, except for the navigation. A list of these tasks evaluated in our work is given in Table \ref{tab:robo_casa}.

The benchmark is particularly challenging due to its unparalleled diversity and realism. Each scenario includes unique configurations and employs just a single demonstration, significantly raising the bar for generalization. For example, in pick-and-place tasks, the objects vary extensively between scenarios, with no repetitions, forcing models to adapt to new instances without direct prior exposure. Furthermore, the training and evaluation environments are entirely distinct, compelling robotic agents to exhibit robust transfer learning capabilities across unseen kitchens and objects.

These features create a demanding benchmark, testing models on their ability to understand and generalize robotic behavior in highly diverse, real-world-inspired scenarios. RoboCasa's emphasis on realistic physics, photorealistic rendering, and the integration of generative AI tools for diverse asset creation ensures it sets a new standard for evaluating robotic learning methodologies. Its extensive task variability and high fidelity make it one of the most rigorous and comprehensive platforms for advancing generalist robot capabilities in everyday household environments.




\begin{table}[t!]
\renewcommand{\arraystretch}{1.3}
\setlength{\aboverulesep}{0pt}
\setlength{\belowrulesep}{0pt}
\small
    \centering
    \begin{tabularx}{\textwidth}{l l X}
        \toprule
        \textbf{Task}              & \textbf{Skill Family} & \textbf{Description} \\ \hline
        PickPlaceCounterToCabinet   & Pick and place & Pick an object from the counter and place it inside the cabinet. The cabinet is already open. \\ 
        PickPlaceCabinetToCounter   & Pick and place & Pick an object from the cabinet and place it on the counter. The cabinet is already open. \\ 
        PickPlaceCounterToSink      & Pick and place & Pick an object from the counter and place it in the sink. \\\\ 
        PickPlaceSinkToCounter      & Pick and place & Pick an object from the sink and place it on the counter area next to the sink. \\ 
        PickPlaceCounterToMicrowave & Pick and place & Pick an object from the counter and place it inside the microwave. The microwave door is already open. \\ 
        PickPlaceMicrowaveToCounter & Pick and place & Pick an object from inside the microwave and place it on the counter. The microwave door is already open. \\ 
        PickPlaceCounterToStove     & Pick and place & Pick an object from the counter and place it in a pan or pot on the stove. \\ 
        PickPlaceStoveToCounter     & Pick and place & Pick an object from the stove (via a pan or pot) and place it on (the plate on) the counter. \\ \hline
        \\
        OpenSingleDoor              & Opening and closing doors & Open a microwave door or a cabinet with a single door. \\\\
        CloseSingleDoor             & Opening and closing doors & Close a microwave door or a cabinet with a single door. \\\\
        OpenDoubleDoor              & Opening and closing doors & Open a cabinet with two opposite-facing doors. \\ \\
        CloseDoubleDoor             & Opening and closing doors & Close a cabinet with two opposite-facing doors. \\ \\
        OpenDrawer                  & Opening and closing drawers & Open a drawer. \\ \\
        CloseDrawer                 & Opening and closing drawers & Close a drawer. \\\hline
        \\
        TurnOnStove                 & Twisting knobs & Turn on a specified stove burner by twisting the respective stove knob. \\
        TurnOffStove                & Twisting knobs & Turn off a specified stove burner by twisting the respective stove knob. \\ \hline \\
        TurnOnSinkFaucet            & Turning levers & Turn on the sink faucet to begin the flow of water. \\ \\
        TurnOffSinkFaucet           & Turning levers & Turn off the sink faucet to stop the flow of water. \\ \\
        TurnSinkSpout               & Turning levers & Turn the sink spout. \\ \hline \\
        CoffeePressButton           & Pressing buttons & Press the button on the coffee machine to pour coffee into the mug. \\ 
        TurnOnMicrowave             & Pressing buttons & Turn on the microwave by pressing the start button. \\ 
        TurnOffMicrowave            & Pressing buttons & Turn off the microwave by pressing the stop button. \\ \hline \\
        CoffeeSetupMug              & Insertion & Pick the mug from the counter and insert it onto the coffee machine mug holder area. \\
        CoffeeServeMug              & Insertion & Remove the mug from the coffee machine mug holder and place it on the counter. \\
        \bottomrule
    \end{tabularx}
    \caption{The 24 manipulation tasks of RoboCasa used in our work. This list was originally reported as Fig. 11 in \citet{robocasa2024}.}
    \label{tab:robo_casa}
\end{table}

\begin{longtable}{|p{3cm}|p{4cm}|p{8cm}|}
\hline

\end{longtable}
\begin{table}
    \begin{center}    
    \begin{sc}
    \centering
    \setlength{\aboverulesep}{0pt}
    \setlength{\belowrulesep}{0pt}
    \renewcommand{\arraystretch}{1.3}  % Increase row height for readability
    \setlength{\tabcolsep}{6pt}  % Adjust column spacing    
    \resizebox{0.7\textwidth}{!}{  % Set table width to half the page
        \begin{tabular}{l l l l}
            \toprule
            \textbf{PnP1} & \textbf{PnP2} & \textbf{Doors} & \textbf{Drawer} \\
            \midrule
            PnPCounterToCab & PnPCounterToMicrowave & OpenSingleDoor & CloseDrawer \\
            PnPCabToCounter & PnPMicrowaveToCounter & CloseSingleDoor & OpenDrawer \\
            PnPCounterToSink & PnPStoveToCounter & OpenDoubleDoor &  \\
            PnPSinkToCounter & PnPCounterToStove & CloseDoubleDoor &  \\
            \midrule
            \textbf{Stove} & \textbf{Sink} & \textbf{Buttons} & \textbf{Coffee} \\
            \midrule
            TurnOnStove & TurnOnSinkFaucet & CoffeePressButton & CoffeeSetupMug \\
            TurnOffStove & TurnOffSinkFaucet & TurnOffMicrowave & CoffeeServeMug \\
             & TurnSinkSpout & TurnOnMicrowave &  \\
            \bottomrule
        \end{tabular}
    }
    \end{sc}
    \end{center}
    \caption{Task groups used for training the models.}
    \label{tab:task_groups}
\end{table}
\begin{table*}[h]
    \centering
    %\small % Apply small font size to the entire table
    \begin{tabular}{c c c c c| c c c c }
    \toprule
        \multicolumn{5}{c}{Expriment Settings} & $\alpha$ & $\lambda_1$ &$\lambda_2$ & \# Training Steps \\
        Attack & Rt. & Rr. & Gen. & Task\\
        \midrule
        LPA-Rt & CLIP & - & - & MMQA&0.005&-&-&50 \\
        LPA-Rt & CLIP & - & - & WebQA&0.005&-&-&50 \\
        GPA-Rt & CLIP & - & - &MMQA&0.01&-&-& 500\\
        GPA-Rt & CLIP & - & - &WebQA&0.01&-&-&500 \\
        GPA-RtRrGen& CLIP& Llava &Llava &MMQA & 0.01&0.2&0.3&2000\\
                GPA-RtRrGen& CLIP& Qwen &Qwen &MMQA & 0.005&0.2&0.3&2500 \\
                GPA-RtRrGen& CLIP& Llava &Qwen &MMQA & 0.01&0.08&0.9&2500\\
        GPA-RtRrGen& CLIP& Llava &Llava &WebQA & 0.01&0.2&0.3&2000\\
                GPA-RtRrGen& CLIP& Qwen &Qwen &WebQA & 0.01&0.3&0.3&1000\\
                GPA-RtRrGen& CLIP& Llava &Qwen &WebQA & 0.01&0.1&0.8&3000\\
         \bottomrule
    \end{tabular}%
    \caption{Hyper-parameters for training adversarial images.}
    \vspace{-0.1in}
    \label{tab:hyper_parameters}
\end{table*}

\section{Further Experiments}

We conduct further experiments trying out different hyperparameters in the models which conditioned on local ResNet features. The results can be seen in Figure \ref{table:appendix_table}. The models used are as follows:

\textbf{MLP}
uses the MLP point encoder and 4x4 feature map from ResNet. The diffusion policy uses an embedding dimension of 128. 

\textbf{MLP256}
is similar to MLP but the diffusion policy has an embedding dimension of 256. 

\textbf{SUGAR}
uses the point cloud encoder from the SUGAR pretrained model and 4x4 feature map from ResNet. The point cloud encoder is frozen. The diffusion policy uses an embedding dimension of 128.

\textbf{SUGAR-FT2}
is similar to SUGAR but the last two layers are finetuned while keeping the other layers frozen.

\textbf{SUGAR256-FT2}
is similar to SUGAR-FT2 but the diffusion policy uses an embedding dimension of 256.

\textbf{MLP8x8}
uses the MLP point encoder and 4x4 feature map from ResNet. The transformer used to get the condition vector from the ResNet features has an embedding dimension of 256. The diffusion policy uses an embedding dimension of 128. 

\textbf{MLP8x8-L512}
is similar to MLP8x8 but the transformer used to get the condition vector from the ResNet features has an embedding dimension of 512.

% \textbf{SUGAR8x8}
% is similar to SUGAR but uses 8x8 feature map from ResNet.
\section{Adaptive LayerNorm conditioning}
\label{sec:adaln}

A visualization of the adaptive layer norm is given in Figure \ref{fig:dit_block}. We use the point cloud and language as primary modality in this visualization. In a Diffusion Transformer (DiT) block visualized in Figure \ref{fig:dit_block}, the most significant difference to a vanilla transformer block is scaling and shifting operations conditioned on the image CLS token. The scaling factors $\alpha$, $\gamma$ and the shifting factor $\beta$ are applied to self-attention and feed-forward part of the DiT block. The expression $\text{AdaLN}(z_t^P, z_t^L | z_t^I)$ indicates that image embedding is used as condition and mapped to factors $\alpha$, $\gamma$ and $\beta$, while the point cloud and language embeddings go through the self-attention and feed-forward blocks with additional scaling and shifting operations by these factors.


%%%%%%%%%%%%%%%%%%%%%%%%%%%%%%%%%%%%%%%%%%%%%%%%%%%%%%%%%%%%%%%%%%%%%%%%%%%%%%%%%%%%%%%%
\begin{table*}[t!]
% \caption{\textbf{VI Results.} $ \log Z$ values for a different number of steps $K$.}
\begin{center}
\begin{sc}
\resizebox{0.9\textwidth}{!}{%
\setlength{\aboverulesep}{0pt}
\setlength{\belowrulesep}{0pt}
\renewcommand{\arraystretch}{1.5}
\begin{tabular}{l|cccccccc}
\toprule
\textbf{Task} & MLP & MLP256 & SUGAR & SUGAR-FT2 & SUGAR256-FT2 & MLP8x8 & MLP8x8-L512 \\
\midrule
%%%%%%%%%%%%%%%%%%%%%%%%%%%%%%%%%%%%%%%%%%%%%%%%%%%%%%%%%%%%%%%%%%%%%%%%%%%%%%%%%%%%%%%%
PnPCabToCounter
& $0.16$
& $0.10$
& $0.04$
& $0.08$
& $0.10$
& $0.10$
& $0.16$
\\
%%%%%%%%%%%%%%%%%%%%%%%%%%%%%%%%%%%%%%%%%%%%%%%%%%%%%%%%%%%%%%%%%%%%%%%%%%%%%%%%%%%%%%%%

PnPCounterToCab
& $0.08$
& $0.08$
& $0.04$
& $0.02$
& $0.14$
& $0.22$
& $0.08$
\\
%%%%%%%%%%%%%%%%%%%%%%%%%%%%%%%%%%%%%%%%%%%%%%%%%%%%%%%%%%%%%%%%%%%%%%%%%%%%%%%%%%%%%%%%

PnPCounterToMicrowave
& $0.22$
& $0.20$
& $0.04$
& $0.08$
& $0.10$
& $0.18$
& $0.26$
\\
%%%%%%%%%%%%%%%%%%%%%%%%%%%%%%%%%%%%%%%%%%%%%%%%%%%%%%%%%%%%%%%%%%%%%%%%%%%%%%%%%%%%%%%%

PnPCounterToSink
& $0.08$
& $0.08$
& $0.00$
& $0.00$
& $0.08$
& $0.06$
& $0.06$
\\
%%%%%%%%%%%%%%%%%%%%%%%%%%%%%%%%%%%%%%%%%%%%%%%%%%%%%%%%%%%%%%%%%%%%%%%%%%%%%%%%%%%%%%%%

PnPCounterToStove
& $0.02$
& $0.06$
& $0.00$
& $0.02$
& $0.04$
& $0.04$
& $0.06$
\\
%%%%%%%%%%%%%%%%%%%%%%%%%%%%%%%%%%%%%%%%%%%%%%%%%%%%%%%%%%%%%%%%%%%%%%%%%%%%%%%%%%%%%%%%

PnPMicrowaveToCounter
& $0.04$
& $0.08$
& $0.02$
& $0.06$
& $0.12$
& $0.10$
& $0.08$
\\
%%%%%%%%%%%%%%%%%%%%%%%%%%%%%%%%%%%%%%%%%%%%%%%%%%%%%%%%%%%%%%%%%%%%%%%%%%%%%%%%%%%%%%%%

PnPSinkToCounter
& $0.24$
& $0.26$
& $0.08$
& $0.08$
& $0.30$
& $0.20$
& $0.22$
\\
%%%%%%%%%%%%%%%%%%%%%%%%%%%%%%%%%%%%%%%%%%%%%%%%%%%%%%%%%%%%%%%%%%%%%%%%%%%%%%%%%%%%%%%%
PnPStoveToCounter
& $0.26$
& $0.28$
& $0.02$
& $0.04$
& $0.26$
& $0.18$
& $0.20$
\\
%%%%%%%%%%%%%%%%%%%%%%%%%%%%%%%%%%%%%%%%%%%%%%%%%%%%%%%%%%%%%%%%%%%%%%%%%%%%%%%%%%%%%%%%
\midrule
OpenSingleDoor
& $0.62$
& $0.58$
& $0.52$
& $0.44$
& $0.74$
& $0.64$
& $0.68$
\\

OpenDoubleDoor
& $0.88$
& $0.94$
& $0.74$
& $0.70$
& $0.92$
& $0.90$
& $0.94$
\\
CloseDoubleDoor
& $0.84$
& $0.82$
& $0.56$
& $0.76$
& $0.78$
& $0.70$
& $0.82$
\\
CloseSingleDoor
& $0.80$
& $0.84$
& $0.68$
& $0.84$
& $0.84$
& $0.86$
& $0.86$
\\
%%%%%%%%%%%%%%%%%%%%%%%%%%%%%%%%%%%%%%%%%%%%%%%%%%%%%%%%%%%%%%%%%%%%%%%%%%%%%%%%%%%%%%%%
\midrule
OpenDrawer
& $0.66$
& $0.68$
& $0.76$
& $0.84$
& $0.72$
& $0.60$
& $0.62$
\\

CloseDrawer
& $0.90$
& $0.96$
& $0.96$
& $0.96$
& $0.94$
& $0.96$
& $0.90$
\\
%%%%%%%%%%%%%%%%%%%%%%%%%%%%%%%%%%%%%%%%%%%%%%%%%%%%%%%%%%%%%%%%%%%%%%%%%%%%%%%%%%%%%%%%
\midrule
TurnOnStove
& $0.56$
& $0.46$
& $0.62$
& $0.54$
& $0.66$
& $0.48$
& $0.46$
\\

TurnOffStove
& $0.14$
& $0.16$
& $0.22$
& $0.14$
& $0.20$
& $0.12$
& $0.12$
\\
%%%%%%%%%%%%%%%%%%%%%%%%%%%%%%%%%%%%%%%%%%%%%%%%%%%%%%%%%%%%%%%%%%%%%%%%%%%%%%%%%%%%%%%%
\midrule
TurnOnSinkFaucet
& $0.40$
& $0.60$
& $0.68$
& $0.58$
& $0.70$
& $0.68$
& $0.68$
\\

TurnOffSinkFaucet
& $0.50$
& $0.80$
& $0.68$
& $0.82$
& $0.78$
& $0.76$
& $0.82$
\\

TurnSinkSpout
& $0.50$
& $0.52$
& $0.58$
& $0.60$
& $0.52$
& $0.60$
& $0.54$
\\
%%%%%%%%%%%%%%%%%%%%%%%%%%%%%%%%%%%%%%%%%%%%%%%%%%%%%%%%%%%%%%%%%%%%%%%%%%%%%%%%%%%%%%%%
\midrule
CoffeePressButton
& $0.92$
& $0.90$
& $0.84$
& $0.92$
& $0.90$
& $0.84$
& $0.86$
\\

TurnOnMicrowave
& $0.76$
& $0.26$
& $0.62$
& $0.68$
& $0.68$
& $0.60$
& $0.74$
\\

TurnOffMicrowave
& $0.92$
& $0.68$
& $0.90$
& $0.82$
& $0.96$
& $0.82$
& $0.86$
\\
%%%%%%%%%%%%%%%%%%%%%%%%%%%%%%%%%%%%%%%%%%%%%%%%%%%%%%%%%%%%%%%%%%%%%%%%%%%%%%%%%%%%%%%%
\midrule
CoffeeServeMug
& $0.50$
& $0.56$
& $0.56$
& $0.60$
& $0.48$
& $0.56$
& $0.62$
\\

CoffeeSetupMug
& $0.18$
& $0.14$
& $0.14$
& $0.14$
& $0.16$
& $0.20$
& $0.22$
\\
%%%%%%%%%%%%%%%%%%%%%%%%%%%%%%%%%%%%%%%%%%%%%%%%%%%%%%%%%%%%%%%%%%%%%%%%%%%%%%%%%%%%%%%%
\midrule
\textbf{Average Success Rate}
& $0.4658$
& $0.4600$
& $0.4292$
& $0.4483$
& $0.5050$
& $0.4750$
& $0.4942$
\\
%%%%%%%%%%%%%%%%%%%%%%%%%%%%%%%%%%%%%%%%%%%%%%%%%%%%%%%%%%%%%%%%%%%%%%%%%%%%%%%%%%%%%%%%
\bottomrule
\end{tabular}
}
\end{sc}
\end{center}
% \vskip -0.15in
    \caption{Further results for RoboCasa with 50 Human Demonstrations conditioning on local ResNet features \\
    }
    % \vspace{-0.2cm}
\label{table:appendix_table}
\end{table*}
%%%%%%%%%%%%%%%%%%%%%%%%%%%%%%%%%%%%%%%%%%%%%%%%%%%%%%%%%%%%%%%%%%%%%%%%%%%%%


\end{document}