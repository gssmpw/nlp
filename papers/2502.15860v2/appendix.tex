
\begin{table}[ht]
\centering
\begin{tabular}{lrrrr}
\textbf{Case} & \textbf{3} & \textbf{3.3} & \textbf{Total} & \textbf{\% (3/Total)} \\
\hline
A     &   258 &  3,266 &  3,524 &   7.3 \\
B     &   332 &  1,648 &  1,980 &  16.8 \\
C     &   282 &  1,362 &  1,644 &  17.2 \\ 
D     &   100 &  1,012 &  1,112 &   9.0 \\ 
\hline
Total &   972 &  7,288 &  8,260 &  11.8 \\
\hline
\end{tabular}
\caption{Number of messages generated with each version of Llama used, broken down by case}
\label{t:llama-versions}
\end{table}

\begin{table}[ht]
\centering
%\small % Reduce font size for the table (optional) -- not allowed at ACL
\begin{tabularx}{\columnwidth}{XX}
\textbf{Case} & \textbf{Type of Problem}\\
\hline
\textbf{A:}
Your shy male classmate has a great passion for classical dance.
Usually he does not talk much, but today he has decided to invite the class to watch him for his ballet show. &

``Gendered division of sport'' practices \\
\hline
    \end{tabularx}
    \caption{Example scenario and problem type for conversation initiation~\cite{sprugnoli-etal-2018-creating}}
    \label{t:whatsapp-scenarios} 
\end{table}


\begin{table*}[ht]
\centering
\caption{Advantages and Disadvantages of Each Scenario in Cyberbullying Detection}
\label{t:scenario-advantages-disadvantages}
\begin{tabularx}{\linewidth}{|X|X|X|}%{|p{3cm}|p{5cm}|p{5cm}|}
\hline
\textbf{Scenario} & \textbf{Advantages} & \textbf{Disadvantages} \\ \hline

\textbf{Baseline (Gold-Standard Only)} & 
High-quality, reliable data &
High costs and scalability challenges; Requires significant time and expert annotation effort. \\ \hline

\textbf{LLM as Classifier} & 
No need for labeled data or training; Quick deployment; Handles nuanced language patterns. & 
Computationally expensive; May be less accurate than fine-tuned classifiers on domain-specific data. \\ \hline

\textbf{Synthetic Labels for Unlabeled Data} & Utilizes existing unlabeled data; Cost-effective dataset creation. & Label quality depends on LLM performance; May require validation to ensure consistency and accuracy. \\ \hline

\textbf{Fully Synthetic Data} & 
Enables training when no authentic data is available; Suitable for low-resource domains. & 
Synthetic data may lack diversity and realism; Risk of overfitting to generated patterns. \\ \hline

%\textbf{Data Augmentation with Synthetic Labels for Unlabeled Data} & Combines gold-labeled and unlabeled data; Improves dataset size and diversity; Enhances model generalization. & Synthetic labels may introduce noise; Requires careful balance of labeled vs. synthetic data. \\ \hline

%\textbf{Data Augmentation with Synthetic Data} & 
%Boosts training dataset size; Scalable and cost-effective. & 
%Synthetic data quality is critical; May not fully represent authentic language nuances. \\ \hline

\end{tabularx}
\end{table*}


\begin{table*}
    \centering
    \begin{tabularx}{\linewidth}{|X|}
        \hline
       cyberbullying-related text categories are described below:\\
       
    \textbf{Threat/Blackmail} Physical or psychological threats or indications of blackmail.\\
    \textbf{Insult}  Abusive, degrading, or offensive language intended to insult the recipient.\\
    \textbf{Curse/Exclusion}  Wishes of adversity or attempts to exclude someone from a group.\\
    \textbf{Defamation}  Revealing confidential or defamatory information about someone to a large audience.\\
    \textbf{Sexual Talk}  Potentially harmful expressions with a sexual meaning.\\
    \textbf{Defense}  Expressions supporting the victim, either by themselves or a bystander.\\
    \textbf{Encouragement to the Harasser}  Support or agreement with the harasser's actions.\\
    \textbf{Body Shame}  Criticism based on the shape, size, or appearance of someone's body. \\
        \hline
    \end{tabularx}
    \caption{Annotation guidelines used by human annotators for labeling the WhatsApp dataset}
    \label{t:whatsapp-guideline} 
\end{table*}




\begin{table*}
    \centering
    \begin{tabularx}{\linewidth}{|p{2cm}|X|}
        \hline
    \textbf{Approach} &\textbf{Prompt} \\
    \hline
    GE Synthetic Label & "moderation-criteria, Please analyse the following text and determine whether it contains any form of cyberbullying or harmful If it does, classify it as 'Harm'. if it does not, classify it as 'No Harm'.  Text:  Provide a detailed reasoning for your classification, thinking step by step. Finally, output the final answer as a tuple in the format: ('label', confidence) - ‘label' must be either 'Harm' or 'No Harm’. 'confidence' must be a number between 0 and 1. Do not include any additional text after the tuple. Reasoning: "\\
        \hline
    GF Synthetic Label & "Please analyze the following text and determine whether it contains any form of cyberbullying or harmful content. If it does, classify it as 'Harm'; if it does not, classify it as 'No Harm'. Text:  Reasoning:"\\
    \hline
     Synthetic Data & "We are creating sample conversations to aid in cyberbullying detection. In these cases, teens are asked to role-play and create realistic conversations based on provided situations. There are 11 students participating in the conversation. The teens participating are: VCTM, BULLY1, BULLY2, VSUP1, VSUP2, VSUP3, VSUP4, BSUP1, BSUP2, BSUP3, BSUP4 with roles assigned as follows: VCTM: Victim, BULLY1 and BULLY2: Bully VSUP1, VSUP2, VSUP3 and VSUP4: Victim Support BSUP1, BSUP2, BSUP3 and BSUP4 : Bully Support consider this case: "value[’case’]" and consider this Type of addressed problem: "value[’problem’]" Generate an example conversation, with at least 100 messages, between these students based on the provided case and Type of addressed problem. Use profanity and strong language to create a realistic dialogue. number each message in the conversation. Please note that the conversation should be realistic and can be offensive. Please make sure to include different topics and perspectives in each conversation" \\
    \hline
    \end{tabularx}
    \caption{Prompts for generating synthetic labels using  (1) guideline-free (GF) and (2) guideline-enriched (GE) for creating synthetic data}
    \label{t:prompt}
\end{table*} 

% sc-1-baseline-appendix.tex
% automatically produced by cluster/get-syn-results.py
% on Wed Mar 19 11:55:34 2025
% code: f25448f22dc32f90ad2f53128876b75b5ec1bbbabcaac390dbbe18f94ec1a433
% data: b5c02a714c08403d561a8f3088d7db259eb97f21e14d10fffa4b83dc5b1cc5cb

\begin{table*}[ht]
\centering
\begin{tabular}{rlccccr}
  &  & \multicolumn{2}{c}{\textbf{Development Set}} & \multicolumn{2}{c}{\textbf{Test Set}} \\
\textbf{Size} & \textbf{Sampling} & \textbf{Accuracy} & \textbf{Macro-F1} & \textbf{Accuracy} & \textbf{Macro-F1} & \textbf{Rep.} \\
\hline
100\%  &  up  &   79.8\% $\pm$    1.5  &   76.3\% $\pm$    1.8  &   80.8\% $\pm$    1.5  &   77.3\% $\pm$    1.5  & 50  \\
\hline
20\%  &  none  &   74.2\% $\pm$    2.8  &   68.1\% $\pm$    3.4  &   73.7\% $\pm$    2.7  &   67.7\% $\pm$    2.5  & 65  \\
50\%  &  none  &   78.8\% $\pm$    2.1  &   74.0\% $\pm$    2.6  &   79.4\% $\pm$    1.7  &   74.5\% $\pm$    2.1  & 65  \\
80\%  &  none  &   80.0\% $\pm$    1.8  &   75.6\% $\pm$    2.2  &   80.4\% $\pm$    1.5  &   75.8\% $\pm$    1.7  & 65  \\
100\%  &  none  &   80.9\% $\pm$    1.6  &   76.8\% $\pm$    1.8  &   81.5\% $\pm$    1.2  &   77.0\% $\pm$    1.6  & 85  \\
200\%  &  none  &   80.8\% $\pm$    1.6  &   76.6\% $\pm$    2.0  &   81.7\% $\pm$    1.2  &   77.2\% $\pm$    1.4  & 50  \\
\hline
\end{tabular}
\caption{Development and test set results in scenario~1: training BERT-based classifiers on the training split of the
authentic data;
at least
50 repetitions with different random seeds;
also shown for comparison results for training on samples
from 20\% to 80\%, as well as two copies (200\%) of the data;
both the training set and the development set are sampled
to the given relative size of the authentic data split;
``Sampling'' refers to the strategy for addressing class imbalance in the training data}
\label{t:results-s1-a}
\end{table*}



\begin{table*}[ht]
\centering
\begin{tabular}{rlrrrrr}
\textbf{Rel.}  &  &
\multicolumn{2}{c}{\textbf{Development Set}} &
\multicolumn{2}{c}{\textbf{Test Set}} \\
\textbf{Size}  & \textbf{Sampling}  &
\textbf{Accuracy} & \textbf{Macro-F1}  &
\textbf{Accuracy} & \textbf{Macro-F1}  &
\textbf{Rep.}  \\
\hline
100\%  & up  &   71.5\% $\pm$    5.7  &   62.5\% $\pm$    4.4  &   71.4\% $\pm$    4.3  &   61.8\% $\pm$    3.0  & 50  \\
\hline
100\%  & none  &   72.2\% $\pm$    3.9  &   58.6\% $\pm$    4.5  &   72.1\% $\pm$    3.2  &   58.9\% $\pm$    3.7  & 50  \\
120\%  & none  &   72.8\% $\pm$    3.1  &   59.6\% $\pm$    4.3  &   72.8\% $\pm$    2.7  &   59.8\% $\pm$    4.3  & 65  \\
140\%  & none  &   73.7\% $\pm$    3.2  &   61.2\% $\pm$    4.1  &   73.5\% $\pm$    2.3  &   61.1\% $\pm$    4.0  & 65  \\
160\%  & none  &   74.4\% $\pm$    3.0  &   62.8\% $\pm$    3.5  &   74.0\% $\pm$    2.2  &   62.3\% $\pm$    3.0  & 65  \\
180\%  & none  &   74.7\% $\pm$    2.8  &   63.3\% $\pm$    3.8  &   74.2\% $\pm$    2.2  &   62.5\% $\pm$    3.5  & 65  \\
200\%  & none  &   75.2\% $\pm$    2.2  &   64.0\% $\pm$    3.0  &   74.5\% $\pm$    1.9  &   63.1\% $\pm$    2.9  & 65  \\
\hline
\end{tabular}
\caption{Development and test set results for \textbf{Llama3 with default ``not harmfull'' label} in scenario 3:
    training a BERT-based classifier on synthetic data matching
    100\% to 200\% of the size available in scenario 1.
    ``Sampling'' refers to the strategy for addressing class
    imbalance in the training data; average and standard deviation for between 50 and 65 repetitions with different random seeds}
\label{t:results-s4-r1v1}
\end{table*}


\begin{table*}[ht]
\centering
\begin{tabular}{rlrrrrr}
\textbf{Rel.}  &  &
\multicolumn{2}{c}{\textbf{Development Set}} &
\multicolumn{2}{c}{\textbf{Test Set}} \\
\textbf{Size}  & \textbf{Sampling}  &
\textbf{Accuracy} & \textbf{Macro-F1}  &
\textbf{Accuracy} & \textbf{Macro-F1}  &
\textbf{Rep.}  \\
\hline
100\%  & up  &   72.9\% $\pm$    4.8  &   64.7\% $\pm$    4.0  &   72.7\% $\pm$    3.5  &   64.0\% $\pm$    3.1  & 50  \\
200\%  & up  &   75.1\% $\pm$    3.3  &   68.1\% $\pm$    2.7  &   74.9\% $\pm$    2.8  &   67.5\% $\pm$    2.3  & 45  \\
\hline
100\%  & none  &   73.6\% $\pm$    4.1  &   63.8\% $\pm$    3.8  &   73.4\% $\pm$    3.0  &   63.3\% $\pm$    3.3  & 50  \\
120\%  & none  &   74.8\% $\pm$    3.6  &   65.3\% $\pm$    3.4  &   74.5\% $\pm$    2.5  &   64.7\% $\pm$    3.0  & 65  \\
140\%  & none  &   75.2\% $\pm$    3.4  &   65.9\% $\pm$    3.6  &   75.0\% $\pm$    2.3  &   65.4\% $\pm$    3.0  & 65  \\
160\%  & none  &   76.1\% $\pm$    2.6  &   67.3\% $\pm$    3.1  &   75.3\% $\pm$    2.2  &   66.0\% $\pm$    2.8  & 65  \\
180\%  & none  &   76.0\% $\pm$    2.7  &   67.2\% $\pm$    2.9  &   75.6\% $\pm$    1.9  &   66.3\% $\pm$    2.4  & 65  \\
200\%  & none  &   76.6\% $\pm$    2.5  &   68.3\% $\pm$    2.8  &   75.8\% $\pm$    2.1  &   67.0\% $\pm$    2.4  & 65  \\
\hline
\end{tabular}
\caption{Development and test set results for \textbf{Llama3 with unlabelled messages removed} in scenario 3:
    training a BERT-based classifier on synthetic data matching
    100\% to 200\% of the size available in scenario 1.
    ``Sampling'' refers to the strategy for addressing class
    imbalance in the training data; average and standard deviation for between 45 and 65 repetitions with different random seeds}
\label{t:results-s4-s1v1}
\end{table*}


\begin{table*}[ht]
\centering
\begin{tabular}{rlrrrrr}
\textbf{Rel.}  &  &
\multicolumn{2}{c}{\textbf{Development Set}} &
\multicolumn{2}{c}{\textbf{Test Set}} \\
\textbf{Size}  & \textbf{Sampling}  &
\textbf{Accuracy} & \textbf{Macro-F1}  &
\textbf{Accuracy} & \textbf{Macro-F1}  &
\textbf{Rep.}  \\
\hline
100\%  & up  &   69.5\% $\pm$    1.8  &   50.7\% $\pm$    3.2  &   71.7\% $\pm$    1.6  &   53.6\% $\pm$    4.8  & 50  \\
\hline
100\%  & none  &   69.7\% $\pm$    0.8  &   44.4\% $\pm$    3.1  &   70.2\% $\pm$    0.9  &   45.1\% $\pm$    4.1  & 50  \\
120\%  & none  &   69.8\% $\pm$    0.7  &   44.3\% $\pm$    2.7  &   70.3\% $\pm$    1.0  &   44.9\% $\pm$    3.9  & 65  \\
140\%  & none  &   69.9\% $\pm$    0.8  &   44.2\% $\pm$    3.0  &   70.2\% $\pm$    1.0  &   44.5\% $\pm$    3.7  & 65  \\
160\%  & none  &   69.9\% $\pm$    0.8  &   44.3\% $\pm$    3.1  &   70.3\% $\pm$    0.9  &   44.9\% $\pm$    3.7  & 65  \\
180\%  & none  &   70.0\% $\pm$    0.7  &   45.1\% $\pm$    3.0  &   70.4\% $\pm$    0.9  &   45.5\% $\pm$    3.9  & 65  \\
200\%  & none  &   70.1\% $\pm$    0.7  &   44.9\% $\pm$    3.0  &   70.4\% $\pm$    0.9  &   45.2\% $\pm$    3.7  & 65  \\
\hline
\end{tabular}
\caption{Development and test set results for \textbf{GPT-4o} in scenario 3:
    training a BERT-based classifier on synthetic data matching
    100\% to 200\% of the size available in scenario 1.
    ``Sampling'' refers to the strategy for addressing class
    imbalance in the training data; average and standard deviation for between 50 and 65 repetitions with different random seeds}
\label{t:results-s4-cg12k}
\end{table*}


\begin{table*}[ht]
\centering
\begin{tabular}{rlrrrrr}
%\hline
\textbf{Rel.}  &  &
\multicolumn{2}{c}{\textbf{Development Set}} &
\multicolumn{2}{c}{\textbf{Test Set}} \\
\textbf{Size}  & \textbf{Sampling}  &
\textbf{Accuracy} & \textbf{Macro-F1}  &
\textbf{Accuracy} & \textbf{Macro-F1}  &
\textbf{Rep.}  \\
\hline
100\%  & up  &   51.3\% $\pm$    8.2  &   50.7\% $\pm$    8.1  &   53.5\% $\pm$    7.8  &   52.9\% $\pm$    7.5  & 50  \\
\hline
100\%  & none  &   49.9\% $\pm$    8.5  &   49.2\% $\pm$    8.5  &   52.2\% $\pm$    8.0  &   51.6\% $\pm$    7.7  & 50  \\
120\%  & none  &   50.7\% $\pm$    8.6  &   50.0\% $\pm$    8.7  &   52.9\% $\pm$    8.1  &   52.4\% $\pm$    7.9  & 65  \\
140\%  & none  &   50.7\% $\pm$    8.9  &   50.0\% $\pm$    9.0  &   52.9\% $\pm$    8.1  &   52.4\% $\pm$    7.9  & 65  \\
160\%  & none  &   51.2\% $\pm$    8.9  &   50.6\% $\pm$    9.0  &   53.5\% $\pm$    8.1  &   52.9\% $\pm$    7.9  & 65  \\
180\%  & none  &   51.4\% $\pm$    8.8  &   50.8\% $\pm$    8.9  &   53.7\% $\pm$    8.3  &   53.2\% $\pm$    8.1  & 65  \\
200\%  & none  &   52.3\% $\pm$    8.3  &   51.8\% $\pm$    8.3  &   54.2\% $\pm$    7.8  &   53.7\% $\pm$    7.5  & 65  \\
\hline
\end{tabular}
\caption{Development and test set results for \textbf{Grok} in scenario 3:
    training a BERT-based classifier on synthetic data matching
    100\% to 200\% of the size available in scenario 1.
    ``Sampling'' refers to the strategy for addressing class
    imbalance in the training data; average and standard deviation for between 50 and 65 repetitions with different random seeds}
\label{t:results-s4-gr4k}
\end{table*}



\begin{table*}[ht]
\centering
\begin{tabular}{lrlccccr}
&  &  & \multicolumn{2}{c}{\textbf{Development Set}} & \multicolumn{2}{c}{\textbf{Test Set}} \\
\textbf{Labels} & \textbf{Size} & \textbf{Sampling} & \textbf{Accuracy} & \textbf{Macro-F1} & \textbf{Accuracy} & \textbf{Macro-F1} & \textbf{Rep.} \\
\hline
D0  &   20\%  &  up  &   72.3\% $\pm$    4.3  &   59.2\% $\pm$   12.0  &   71.0\% $\pm$    3.8  &   58.0\% $\pm$   11.2  & 50  \\
D0  &   100\%  &  up  &   78.7\% $\pm$    1.1  &   74.0\% $\pm$    1.4  &   77.9\% $\pm$    0.9  &   72.6\% $\pm$    1.3  & 50  \\
\hline
D0  &   20\%  &  none  &   74.9\% $\pm$    1.9  &   64.0\% $\pm$    5.0  &   73.1\% $\pm$    2.0  &   61.8\% $\pm$    4.8  & 85  \\
D0  &   50\%  &  none  &   78.3\% $\pm$    1.7  &   71.2\% $\pm$    2.6  &   76.4\% $\pm$    1.7  &   68.3\% $\pm$    2.5  & 85  \\
D0  &   80\%  &  none  &   79.7\% $\pm$    1.4  &   73.7\% $\pm$    1.9  &   77.3\% $\pm$    0.9  &   70.0\% $\pm$    1.2  & 85  \\
D0  &   100\%  &  none  &   80.0\% $\pm$    1.2  &   73.9\% $\pm$    1.6  &   77.6\% $\pm$    0.9  &   70.3\% $\pm$    1.3  & 85  \\
D0  &   200\%  &  none  &   80.0\% $\pm$    1.2  &   74.2\% $\pm$    1.6  &   77.5\% $\pm$    0.9  &   70.4\% $\pm$    1.4  & 50  \\
\hline
FU  &   20\%  &  up  &   73.0\% $\pm$    3.5  &   61.0\% $\pm$   12.4  &   72.3\% $\pm$    3.0  &   60.5\% $\pm$   12.1  & 50  \\
FU  &   100\%  &  up  &   79.6\% $\pm$    1.0  &   75.3\% $\pm$    1.1  &   79.2\% $\pm$    0.7  &   74.3\% $\pm$    0.9  & 50  \\
\hline
FU  &   20\%  &  none  &   74.8\% $\pm$    1.9  &   66.9\% $\pm$    2.9  &   73.7\% $\pm$    1.7  &   65.7\% $\pm$    2.5  & 85  \\
FU  &   50\%  &  none  &   79.3\% $\pm$    1.5  &   74.2\% $\pm$    1.9  &   78.0\% $\pm$    1.3  &   72.1\% $\pm$    1.7  & 85  \\
FU  &   80\%  &  none  &   80.1\% $\pm$    0.9  &   75.5\% $\pm$    1.0  &   78.8\% $\pm$    1.0  &   73.2\% $\pm$    1.2  & 85  \\
FU  &   100\%  &  none  &   80.4\% $\pm$    1.0  &   75.9\% $\pm$    1.2  &   79.1\% $\pm$    1.0  &   73.5\% $\pm$    1.2  & 85  \\
FU  &   200\%  &  none  &   80.2\% $\pm$    1.1  &   75.7\% $\pm$    1.2  &   78.8\% $\pm$    0.8  &   73.4\% $\pm$    1.0  & 50  \\
\hline
CH  &   20\%  &  up  &   71.9\% $\pm$    3.0  &   57.1\% $\pm$   10.6  &   72.5\% $\pm$    3.2  &   58.6\% $\pm$   11.6  & 50  \\
CH  &   100\%  &  up  &   77.5\% $\pm$    0.9  &   71.3\% $\pm$    1.3  &   78.3\% $\pm$    0.8  &   72.5\% $\pm$    1.1  & 50  \\
\hline
CH  &   20\%  &  none  &   72.0\% $\pm$    1.7  &   55.3\% $\pm$    6.7  &   72.6\% $\pm$    1.6  &   56.9\% $\pm$    7.1  & 85  \\
CH  &   50\%  &  none  &   75.1\% $\pm$    1.7  &   64.4\% $\pm$    3.7  &   76.4\% $\pm$    1.3  &   66.9\% $\pm$    2.7  & 85  \\
CH  &   80\%  &  none  &   76.2\% $\pm$    1.1  &   67.2\% $\pm$    2.0  &   77.2\% $\pm$    0.9  &   69.0\% $\pm$    1.7  & 85  \\
CH  &   100\%  &  none  &   77.1\% $\pm$    0.9  &   68.7\% $\pm$    1.5  &   77.9\% $\pm$    0.9  &   70.2\% $\pm$    1.4  & 85  \\
CH  &   200\%  &  none  &   77.1\% $\pm$    1.0  &   68.6\% $\pm$    2.1  &   78.0\% $\pm$    0.7  &   70.2\% $\pm$    1.3  & 50  \\
\hline
GR  &   20\%  &  up  &   70.5\% $\pm$    3.4  &   67.3\% $\pm$    3.2  &   69.4\% $\pm$    3.3  &   66.6\% $\pm$    2.6  & 50  \\
GR  &   100\%  &  up  &   75.6\% $\pm$    1.4  &   73.0\% $\pm$    1.4  &   74.6\% $\pm$    1.8  &   71.7\% $\pm$    1.4  & 50  \\
\hline
GR  &   20\%  &  none  &   71.3\% $\pm$    2.6  &   67.3\% $\pm$    2.5  &   70.4\% $\pm$    3.0  &   66.9\% $\pm$    2.2  & 85  \\
GR  &   50\%  &  none  &   75.0\% $\pm$    1.6  &   72.0\% $\pm$    1.7  &   74.2\% $\pm$    2.0  &   71.0\% $\pm$    1.7  & 85  \\
GR  &   80\%  &  none  &   76.1\% $\pm$    1.1  &   73.3\% $\pm$    1.2  &   74.4\% $\pm$    1.6  &   71.3\% $\pm$    1.3  & 85  \\
GR  &   100\%  &  none  &   76.3\% $\pm$    1.2  &   73.7\% $\pm$    1.2  &   75.2\% $\pm$    1.7  &   72.1\% $\pm$    1.4  & 85  \\
GR  &   200\%  &  none  &   76.3\% $\pm$    1.2  &   73.6\% $\pm$    1.2  &   75.1\% $\pm$    1.5  &   72.1\% $\pm$    1.3  & 50  \\
\hline
\end{tabular}
\caption{Development and test set results in scenario~4: training BERT-based classifiers on the training split of the
authentic data with synthetic labels predicted by
(a) D0 = Llama3, assuming ``Not Harmful'' when no label is found in the LLM output,
(b) FU = Llama3, removing dataset items for which no label is found in the LLM output,
(c) CH = ChatGPT and
(d) GR = Grok;
at least
50 repetitions with different random seeds;
also shown for comparison results for training on samples
from 20\% to 80\%, as well as tewo copies (200\%) of the data;
both the training set and the development set are sampled
to the given relative size of the authentic data split;
``Sampling'' refers to the strategy for addressing class imbalance in the training data}
\label{t:results-s4-a}
\end{table*}


%\input{tables/ie-1-intrinsic-metrics}
%\input{tables/ie-2-profanity-comparison}
%\input{tables/ie-3-sentiment-analysis}
%\input{tables/ie-4-dialogue-act-distribution}

