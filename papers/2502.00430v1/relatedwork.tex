\section{Related Work}
\label{related_works}

Numerous works in literature propose solutions and improvements for OFDMA
performance optimization in Wi-Fi networks. Many papers~\cite{1_Wang, 1_Bankov,
1_Inamullah, 1_Qadri, noh2024joint, tan2024deep} consider resource allocation
algorithms, while BSR collection is not considered as a potential bottleneck for
UL OFDMA performance. In~\cite{1_Wang}, the resource allocation problem was
formulated as an optimization problem and addressed using a sub-optimal
divide-and-conquer recursive algorithm. In~\cite{1_Bankov}, the authors propose
an adaptation of three well-known scheduling algorithms --- MaxRate, Proportional
Fair and Shortest Remaining Processing Time --- to Wi-Fi. In~\cite{1_Inamullah},
the authors propose a scheduling algorithm for deadline-constrained traffic. To
estimate the delay the packet will spend in the queue before the transmission,
the authors apply queuing theory using the buffer statuses reported by the
STAs. In~\cite{1_Qadri}, a heuristic algorithm is designed to prioritize users
based on the reported buffer sizes for each access category (AC). In more recent
papers~\cite{noh2024joint, tan2024deep}, a deep reinforcement learning approach
is applied. For most of the solutions proposed in these papers, the knowledge
of buffer statuses plays a crucial role, but the overhead related to the BSR
collection procedure is left out of scope.

A common approach considered in literature for BSR is to utilize Uplink
OFDMA-based Random Access (UORA) defined in the IEEE 802.11 standard.
%According to UORA, the AP allocates a subset of the available RUs to random
%access procedure. All STAs are allowed to transmit data in these RUs using
%random access procedure, thus some of the transmission will lead to collisions
%and unsuccessful data delivery.
Since it is based on random access, collisions are possible. After a collision,
the STA will have to retransmit lost packets, that is why UORA is often
considered as a solution only for BSRs, because they are generally significantly
shorter than data packets. There are multiple papers in literature evaluating
performance of UORA and proposing enhancements. In~\cite{1_Naik,
1_Bhattarai}, the metric BSR delivery rate is introduced.
%corresponding to the average number of BSRs delivered to the AP in a TF cycle. 
The performance evaluation shows that BSR delivery is of crucial importance for
the performance of the scheduling algorithm. Besides, the authors propose a resource
allocation algorithm that varies the number of random access RUs depending on
the number of STAs with non-zero BSRs. However, the proposed algorithm will not
allocate any RUs to UORA if the number of STAs with non-zero BSRs is higher or
equal than the number of available RUs, thus significantly favoring the STAs
with already known BSR. In~\cite{1_Avdotin, 2_Avdotin, 3_Avdotin}, the authors
propose different schemes for collision resolution in UORA to reduce latency
related to retransmissions of the lost packets. In this paper, we leave UORA out
of scope for several reasons. First, we argue that UORA sacrifices the part of
available bandwidth leading to its underutilization when there are not many
STAs with new data. Second, UORA is not a mandatory feature in the IEEE 802.11ax
standard. Third, there is no verified and publicly available implementation of
UORA in the available network simulators. However, it will be considered in our
future works.

Recently, a few papers proposed solutions that do not rely on UORA.
In~\cite{1_Schneider}, the authors consider the problem of providing fully
deterministic channel access in Wi-Fi. For that, they propose to completely
disable random access procedures, and let the AP orchestrate all UL
transmissions. Although such an approach is very powerful when the traffic
pattern is known, e.g., when it is periodic, we show in this paper that it will
lead to significant performance degradation in scenarios with many devices with
unpredictable on-off traffic.
In~\cite{shao2024access}, the BSR collection is proposed based on multiple
rounds of BSRP TF and BSR exchanges to collect BSRs from all STAs before
scheduling resources for data transmission. This approach can improve the
fairness by giving more information to the scheduling algorithm before it makes
its decision, but does not solve the problem of BSR collection overhead.
In~\cite{goncalves2023access}, client-side access manipulation is proposed.
According to this manipulation, the STA switches between two channel access
schemes, EDCA and OFDMA, depending on its latest buffer status. In the OFDMA
state, it fully disables EDCA, while in the EDCA state it competes for the
channel with other STAs and the AP. However, according to the standard it is
the AP who controls these two states via the MU EDCA Parameter Set. Moreover,
excluding the AP from the decision process may lead to inconsistent behavior
and resource waste, as the AP does not know the STA's state.

In this paper, we propose a fully standard compliant A2P algorithm that
significantly improves the latency compared to the state of the art. This is
verified in a challenging teleconference scenario with a high number of STAs.