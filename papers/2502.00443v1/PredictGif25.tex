%===============================================================================
% ifacconf.tex 2022-02-11 jpuente  
% 2022-11-11 jpuente change length of abstract
% Template for IFAC meeting papers
% Copyright (c) 2022 International Federation of Automatic Control
%===============================================================================
\documentclass{ifacconf}

\usepackage{graphicx}      % include this line if your document contains figures
\usepackage{natbib}        % required for bibliography
%===============================================================================
% Ajout CJ
\usepackage{epsfig}
\usepackage{graphicx}
%\usepackage[hang]{caption2}
\usepackage[normalsize,it,hang]{subfigure}
\usepackage{amsfonts}
\usepackage{amssymb,hhline}
\usepackage{graphics}
%\usepackage[latin1]{inputenc}   % Neccessaire pour paragraphe (Harald)
%\usepackage{fancyhdr}
\usepackage{latexsym}
\usepackage{makeidx}
%\newcommand{\substyle}[0]{\itshape}
\usepackage{amsmath}
\usepackage{wasysym}
\usepackage{amscd}
\usepackage{color}
\usepackage{rotating}
\usepackage{array}

\newtheorem{remark}{Remark}[section]

%
\begin{document}
\begin{frontmatter}

\title{Model-Free Predictive Control: \\ Introductory Algebraic Calculations, \\ and a Comparison with HEOL and ANNs}
%\thanksref{footnoteinfo}} 
%Title, preferably not more than 10 words.

%\thanks[footnoteinfo]{Sponsor and financial support acknowledgment
%goes here. Paper titles should be written in uppercase and lowercase
%letters, not all uppercase.}

\author[1 ,4]{Cédric Join}
\author[2]{Emmanuel Delaleau} 
\author[3,4]{Michel Fliess} 

\address[1 ]{CRAN (CNRS, UMR 7039), Universit\'{e} de Lorraine, BP 239, 54506 Vand{\oe}uvre-l\`{e}s-Nancy, France (e-mail: {cedric.join@univ-lorraine.fr})}
\address[2]{ENI Brest, UMR CNRS 6027, IRDL, 29200 Brest, France\\ (e-mail: emmanuel.delaleau@enib.fr)}
\address[3]{LIX (CNRS, UMR 7161), \'Ecole polytechnique, 91128 Palaiseau, France (e-mail: {michel.fliess@polytechnique.edu})}
\address[4]{AL.I.E.N., 7 rue Maurice Barr\`{e}s, 54330 V\'{e}zelise, France\\ (e-mail: \{{michel.fliess,cedric.join}\}{@alien-sas.com})}

\vspace{3cm}

\begin{abstract}  % Abstract of 50--100 words 
Model predictive control (MPC) is a popular control engineering practice, but requires a sound knowledge of the model. Model-free predictive control (MFPC), a burning issue today, also related to reinforcement learning (RL) in AI, is reformulated here via a linear differential equation with constant coefficients, thanks to a new perspective on optimal control combined with recent advances in the field of model-free control. It is replacing Dynamic Programming, the Hamilton-Jacobi-Bellman equation, and Pontryagin's Maximum Principle. The computing burden is low. The implementation is straightforward.
Two nonlinear examples, a chemical reactor and a two tank system, are illustrating our approach. A comparison with the HEOL setting, where some expertise of the process model is needed, shows only a slight superiority of the later. A recent identification of the two tank system via a complex ANN architecture might indicate that a full modeling and the corresponding machine learning mechanism are not always necessary neither in control, nor, more generally, in AI. 
%The computing burden is quite low and the implementation is straightforward.
% From the mid-80s, the introduction of differential algebra in automatic control had remarkable repercussions: formulation and resolution of the inversion problem; intrinsic formulation of the notion of controllability, introduction of new classes of feedback,s and resolution of main synthesis problems. In the early 90s, a new class of systems was invented: differentially flat systems. This renewed a large part of nonlinear control with notable industrial repercussions. In the early 2000s, other algebraic tools led to new estimators allowing the denoising of measured signals, the reconstruction of their derivatives as well as providing new identification methods. This also made it possible to develop model-free control. Very recently, still with algebraic approach, HEOL control appeared, which combines flatness and model-free control. This paper addresses the very popular model-based predictive control which is often cited as the greatest advance after PI, both academically and industrially. This topic has also been the subject of renewed attention thanks to methods derived from AI. The algebraic approach now makes it possible to rethink optimal control and model predictive control. The point of view defended here brings predictive control closer to its foundations: the notion of predicted trajectory and the tracking of these trajectories. The article also offers a comparison in favor of HEOL.
% Recall that differential geometry had a notable impact in the 70s on the development of control nonlinear systems by allowing to generalize results of the linear state variable approach. Differential algebra then took precedence over geometric methods in the middle of the 80s. It made it possible to consider systems intrinsically and introduce new concepts: differential flatness, model-free control, and their combination with the HEOL control.
% This paper addresses the very popular model-based predictive control which is often cited as the greatest advance after PI, both academically and industrially. The algebraic approach now makes it possible to rethink optimal control and model predictive control. The point of view defended here brings predictive control closer to its foundations: the notion of predicted trajectory and the tracking of these trajectories. This is done \textcolor{red}{ajouter HEOL}
% The paper is illustrated by an example of a chemical reactor often used in the predictive control literature. Comparaison de HEOL et superiorité de celle-ci.
\end{abstract}

%\begin{keyword}
%Predictive control, optimal control, %model-free control, flatness-based %control. %Five to %ten keywords, %preferably chosen from the IFAC %keyword list.
%\end{keyword}

\end{frontmatter}
%===============================================================================
\onecolumn
\section{Introduction}
Recent advances, with a strong algebraic flavor, in signal processing and time series analysis, have been applied to energy forecasting by \cite{solar} and \cite{energy}. Those forecasts are easier to implement than the current machine-learning approaches which are prominent now in AI. A similar goal is pursued here with {\em model-free predictive control} ({\em MFPC}), which is a natural extension of \emph{model predictive control} ({\em MPC}), one of the most popular control engineering practice, where the mathematical modeling should be well known (see, e.g., \cite{rawlings}). Our approach to this hot topic may be summarized as follows:
\begin{enumerate}
    \item We use {\em model-free control} ({\em MFC}) and especially the corresponding {\em ultra-local model} in the sense of Fliess and Join (2013, 2022). Those tools have already been exploited several times in investigations about MFPC: \cite{feng}, \cite{hegendus}, \cite{huo}, \cite{khal}, \cite{lammouchi}, \cite{liu}, \cite{long},
\cite{sun1}, \cite{sun2}, \cite{xu}, \cite{xu2}, \cite{wang}, \cite{yin}, \cite{yuan}, \cite{zhang1}, \cite{zhang2}, \cite{zhang3}, \cite{zhou}.
\item Thanks to a new understanding of optimal control (\cite{join2}) via an algebraic interpretation of controllability (\cite{fliess90}, \cite{flat}), we greatly simplify the connection between predictive control and optimization techniques:
\begin{itemize}
    \item On a small time lapse, where a time-varying quantity may be approximated by a constant, the Euler-Lagrange equation, i.e., the fundamental equation of the calculus of variations (see, e.g., \cite{gelfand}), becomes an elementary linear ordinary differential equation of order $2$, with constant coefficients. 
    \item The computing burden is low. The computer implementation is straightforward. It might lead to a dramatic simplification of the connection between predictive control and {\em reinforcement learning} (\emph{RL}), which is today a mainstay (see, e.g., \cite{recht}, \cite{adhau}, \cite{bertsekas}). It confirms the viewpoint expressed by \cite{lecun}:  {\it I do favor MPC over RL. I've been making that point since at least 2016. RL requires ridiculously large numbers of trials to learn any new task}.
\end{itemize}
\end{enumerate}

If the key condition of a good model knowledge is not met, many studies use various artificial neural network (ANN) architectures to apply efficiently MPC techniques, without the need for MFPC. See, e.g., \cite{aa}, \cite{nubert}, \cite{ren}, \cite{salzmann}, \cite{adhau}, and references therein. The same two tank example used by \cite{adhau} via a \emph{recurrent neural network} (see, e.g., \cite{ag}) for its modeling is treated here with our ultra-local model of order $1$ (\cite{mfc1}), and yields, therefore, a trivial implementation. Is this an indication that ANNs do not always provide the best solution?


 
% Cite also \cite{adhau}, \cite{gros}, \cite{shi}, \cite{low}, and \cite{wei} for other %approaches. %and \cite{berberich} for some possible %connections with the data-driven %viewpoint. 
%Furthermore, it follows from latest analyses (see, e.g., \cite{recht}, \cite{gros}) %that a better understanding of MFPC might contribute to a more efficient grasp of %\emph{reinforcement learning} (\emph{RL}), a fundamental theme in today's AI.

The other example studied is a chemical reactor (\cite{panno}, \cite{rawlings}), which was introduced in order to study the robustness of MPC with respect to disturbances. In those references, a time-invariant linearized modeling around a setpoint is used. It is therefore assumed that the reactor is always near the setpoint. This most stringent condition is lifted in our approach.

In both cases, we are comparing our results with the {\em HEOL} approach (\cite{heol}), already used by \cite{delaleau}, which is combining flatness-based and model-free controls. Although HEOL requires more process knowledge, its performance is not really superior.
 

%The above MFPC is compared with the HEOL setting recently introduced by \cite{heol}, %where an approximate (\emph{differentially}) \emph{flat} (\cite{flat}) modeling is %available. The linearization around an appropriate reference trajectory, which is %easily obtained thanks to the flatness property, yields a linear system, which is time-%varying in general. It leads to a \emph{homeostat}, which is analogous to the ultra-%local model for MFC. Local stability around trajectories and robustness are ensured by %an \emph{intelligent} controller, which was introduced by \cite{mfc1}. 




Our paper is organized as follows. MFPC is introduced in Sect.~\ref{mfpc} and HEOL in Sect.~\ref{setting}. Sect.~\ref{chem} considers the chemical reactor. Sect.~\ref{tank} depicts the two tank system. See some concluding remarks in Sect.~\ref{conclu} and some reflections on modeling in AI. The mathematical modelings of the chemical reactor and of the two tank system are given in Appendices A and B for digital simulation purposes.

\section{Model-free predictive control}\label{mfpc}

\subsection{Ultra-local model}

With \cite{mfc1} consider for simplicity's sake the SISO (single-input single output) \emph{ultra-local model} or order $1$, which is replacing the poorly known plant and disturbance description  
\begin{equation}
\dot{y} = \mathcal{F} + \alpha u
\label{1}
\end{equation}
The control and output variables are respectively $u$ and $y$.
The derivation order of $y$ is $1$ like in many concrete situations.
$\mathcal{F}$ subsumes not only the unknown structure of the system, which most of the time is nonlinear, but also
any external disturbance. 
The constant $\alpha \in \mathbb{R}$ is chosen by the practitioner such that $\alpha u$ and $\dot{y}$ are of the same magnitude. Therefore $\alpha$ does not need to be precisely estimated.

A data-driven estimation of $\mathcal{F}$ in Eq. \eqref{1}, which is obtained via algebraic manipulations (\cite{garnier}), reads according to \cite{mfc1}:%\footnotesize
\begin{equation}\label{integral}
\begin{aligned}
 \mathcal{F}_{\text{est}}(t)  =-\frac{6}{T^3}\int_{0}^T& \left( (T -2\sigma)y(t-T+\sigma)\right.\\&\left.+\alpha(T-\sigma)\sigma u(t-T+\sigma) \right) d\sigma 
\end{aligned}
\end{equation}
\begin{remark}
Model-free control and the associated ultra-local model enjoy already many successes: See numerous references in Fliess and Join (2013, 2022), \cite{heol}, and, e.g., \cite{michel} for a benchmark).  
\end{remark}
\subsection{Optimization via the Euler-Lagrange equation}\label{calcul}
\subsubsection{Preliminary calculations.}
Assume that $F = a$ is a constant in Eq. \eqref{1}, which corresponds now to an elementary flat system, where $y$ is a flat output. Introduce the Lagrangian, or cost function, 
\begin{equation}\label{lag}
\mathcal{L} = (y - y_{\rm setpoint})^2 + u^2 
\end{equation}
where $y_{\rm setpoint}$ denotes a given setpoint. Eq. \eqref{1} yields $$\mathcal{L} = (y - y_{\rm setpoint})^2 + \left(\frac{\dot{y} - a}{\alpha}\right)^2$$
For the criterion
$
    J = \int_{t_i}^{t_f} \mathcal{L}dt 
$
the Euler-Lagrange equation (see, e.g., \cite{gelfand}) 
$
\frac{\partial \mathcal{L}}{\partial y} - \frac{d}{dt} \frac{\partial \mathcal{L}}{\partial \dot{y}}  = 0
$
corresponds to a non-homogeneous linear ordinary diffrential equation of $2$nd order
\begin{equation}\label{ele}
    \boxed{\ddot{y} - \alpha^2 y = y_{\rm setpoint}}
\end{equation}
Any solution  reads 
\begin{equation}\label{sol}
    y^\star (t)= y_{\rm setpoint} + c_1\exp(\alpha t) + c_2\exp(-\alpha t), \quad c_1, c_2 \in \mathbb{R}
\end{equation}of Eq. \eqref{ele}, which is independent of $a$, is said to be optimal.
The constants $c_1$, $c_2$ are obtained via the two-point boundary conditions $y(t_i) = y_i$, $y(t_f) = y_{\rm setpoint}$: $$c_1 =\frac{y_i\exp(-\alpha t_f) - y_{\rm setpoint}\exp(-\alpha t_f)}{\exp(\alpha t_i)\exp(-\alpha t_f) - \exp(-\alpha t_i)\exp(\alpha t_f)}$$
$$c_2 =-\frac{\exp(\alpha t_f)(y_i - y_{\rm setpoint})}{\exp(\alpha t_i)\exp(-\alpha t_f) - \exp(-\alpha t_i)\exp(\alpha t_f)}$$
\begin{remark}\label{infinite}
    If $\alpha > 0$, and $t_f = +\infty$ (infinite time horizon), Eq. \eqref{sol} becomes 
$y^\star (t)= y_{\rm setpoint} + c_2\exp(- \alpha t)$.
\end{remark}

\subsubsection{Application to MFPC.}
%\textcolor{red}{\texttt{[Previous version:]}
%Consider the time interval $0 \leqslant t \leqslant t_f$. %Subdivide it $0 < t_{\nu_\iota} \leq t_f$. At time %%$t_{\nu_\iota}$, replace $\mathcal{F}$ in Eq. \eqref{1} by the %constant $F_{\text{est}}(t_{\nu_\iota}) = a_{\nu_\iota}$. Reproduce the above optimization calculations on the horizon $t_{\nu_\iota} \leqslant t \leqslant t_f$. Iterate it on the interval $t_{\nu_\iota + 1} \leqslant t \leqslant t_f$.}
% \textcolor{green}{\texttt{[Suggestion:]}
% Subdivide the time interval $[t_i,t_f[$, $t_i< \cdots < t_k < t_{k+1} < \cdots < t_f$. On each time interval $[t_k,t_{k+1}[$ replace $\mathcal{F}$ of Eq.~\eqref{1} by a constant and reproduce the above optimization procedure on the time horizon $[t_{k+1},t_f[$.
% }
Consider the subdivision $0< \cdots < t_k < t_{k+1} < \cdots < t_f$. On each time lapse $[t_k,t_{k+1}]$ replace $\mathcal{F}$ in Eq.~\eqref{1} by the constant $\mathcal{F}_\text{est}(t_k)$. Start the above optimization procedure again on the time horizon $[t_{k+1},t_f]$. The criterion and the horizon may be modified if necessary.
%\begin{remark}
%It is difficult, if not impossible, to examine the stability without any further %assumptions on the domains where $y$ and $u$ should live.
%\end{remark}

\section{The HEOL setting}\label{setting}
\subsection{Tangent linear system}
The calculations may be summarized as follows%\textcolor{red}{ici il faut un double indice à $\nu_1$ et $\nu_2$}
$$
\mathcal{A}_i (u_i, y_1, \dots, y_{1}^{(\nu_{i,1})}, y_2, \dots, y_{2}^{(\nu_{i,2})}) = 0, \quad i = 1, 2
$$
where $\mathcal{A}_i$ is an an analytic function of its arguments; $y_1$, $y_2$ are \emph{flat outputs} of a flat system (\cite{flat}) with control variables $u_1$, $u_2$.
Differentiate $\mathcal{A}_i$:
\begin{equation*}\label{diff}
    \frac{\partial \mathcal{A}_i}{\partial u_i}du_i + \sum_{\iota = 1, 2}\frac{\partial \mathcal{A}_i}{\partial y_\iota} dy_\iota + \dots + \frac{\partial \mathcal{A}_i}{\partial y_{\iota}^{(\nu_\iota)}} d y_{\iota}^{(\nu_\iota)}  = 0
\end{equation*}
where $d y_{\iota}^{({\mu_\iota)}} = \frac{d^{\mu_\iota}}{dt^{\mu_\iota}} dy_\iota$. 
Associate to it the \emph{tangent}, or \emph{variational}, linear system, which is time-varying in general,
\begin{equation}\label{tang}
    \frac{\partial \mathcal{A}_i}{\partial u_i}\Delta u_i + \sum_{\iota = 1, 2}\left(\frac{\partial \mathcal{A}_i}{\partial y_\iota} + \dots + \frac{\partial \mathcal{A}_i}{\partial y_{\iota}^{(\nu_\iota)}} \frac{d^{\nu_\iota}}{dt^{\nu_\iota}}\right) \Delta y_{\iota}  = 0
\end{equation}
$\Delta u_\iota$ (resp. $\Delta y_\iota$) is a control (resp. output) variable.
\subsection{Homeostat}\label{homeos}
Assume that $\frac{\partial \mathcal{A}_\iota}{\partial \dot{y}_\iota} \neq 0$, $\iota = 1, 2$, in Eq. \eqref{tang}. Then write with \cite{heol} the \emph{homeostat}
\begin{equation}\label{hom}
    \frac{d}{dt} (\Delta y_\iota) = \mathfrak{F}_\iota + \alpha_\iota \Delta u_\iota, \quad \iota = 1, 2
\end{equation}
where
\begin{itemize}
    \item $\alpha_\iota = - \frac{\frac{\partial \mathcal{A}_i}{\partial u_i}}{\frac{\partial \mathcal{A}_i}{\partial \dot{y}_\iota}}$ is not constant in general;
    \item $\mathfrak{F}_\iota$ is not given by the missing terms in Eq. \eqref{tang}, but is data-driven, i.e., it corresponds to model mismatches and disturbances.
\end{itemize}
Write with \cite{heol} the following estimate $\mathfrak{F}_{\iota, \rm est}$, which is easily deduced from Eq. \eqref{integral}
{\small
\begin{equation}\label{integralbis}
\mathfrak{F}_{\iota, \rm est} = - \frac{6}{T^3} \int_{0}^{T} \left( (T - 2 \sigma)\Delta \tilde{y_\iota}(\sigma) {+} \sigma (T - \sigma)\tilde\alpha(\sigma)\Delta \tilde {u_\iota}(\sigma)\right)d\sigma
\end{equation}}
where 
\begin{itemize}
\item the time lapse $T > 0$  is ``small.''
\item $\Delta\tilde{y_\iota}(\sigma)=\Delta y_\iota(\sigma+t-T)$, $\tilde{\alpha}_\iota(\sigma)\Delta \tilde u(\sigma)= \alpha_\iota (\sigma+t-T)\Delta u_\iota (\sigma+t-T)$.
\end{itemize}
\subsection{iP controller}
Mimicking \cite{mfc1}, associate, with \cite{heol}, to the homeostat \eqref{hom} the \emph{intelligent proportional} (\emph{iP}) controller
\begin{equation}\label{ip}
 \Delta u_\iota = - \frac{\mathfrak{F}_{\iota, \rm est} + K_{\iota, P} \Delta y_\iota}{\alpha_\iota}, \quad \iota = 1, 2   
\end{equation}
where $K_{\iota, P} \in \mathbb{R}$ is the gain. Combine Eqs. \eqref{hom} and \eqref{ip}:
$$ \left(\frac{d}{dt} + K_{\iota, P}\right) \Delta y_\iota  = 0$$
Thus $\lim_{t \to \infty} \Delta y_\iota \approx 0$ if, and only if, $K_{\iota, P} >0$.

\section{Computer experiments}\label{exp}
\subsection{Chemical reactor}\label{chem}
\subsubsection{Ultra-local model}
There are two control variables in the chemical reactor of Appendix \ref{A}. We are therefore using twice Eq. \eqref{1}. For the first (resp. second) ultra-local model, $u = T_c$ (resp. $u = F$), $y = c$ (resp. $y = h$). 

\subsubsection{Homeostat for HEOL}
Eqs. \eqref{hom} and \eqref{chemical} yield%\footnote{See \cite{delaleau} for another %example.}

$$\frac{d}{dt} {\Delta c} = \mathfrak{F}_c - \frac{2{EUk_0}\exp(-\frac{E}{RT})c}{RC_pr\rho T^2} \Delta u_c$$
and 
$$\frac{d}{dt} {\Delta h} = \mathfrak{F}_h - \frac{1}{\pi r^2} \Delta u_h$$

\subsubsection{Simulations}
We follow \cite{panno} and \cite{rawlings}. For the initial operating point, $c(0) = 0.878$\,kmol/m$^3$, $T(0) = 324.5$\,K, $h(0) = 0.659$\,m, $T_c(0) = 300$\,K, $F(0) = 0.1$\,m$^3$/min. The sampling time for the control variables is $1$\,min. The total duration is $50$\,min. On the other hand, to implement Formulae \eqref{integral} and \eqref{integralbis}, the sampling time for measurements $0.1$\,min.

For MFPC, set for the first (resp. second) ultra-local model \eqref{1} $\alpha = -0.05$ (resp. $\alpha = - 5$). Following Sect. \ref{calcul}, a new optimal trajectory is calculated every minute.

For HEOL, set $K = 1$ for the gain in the iP \eqref{ip}. 
Reference trajectories are deduced from classic B\'{e}zier curves (see, \textit{e.g.}, \cite{rogers}).
% \begin{equation*}
%     \begin{cases}
%         x^\star(t)=&x^\star(t_0)\text{ if } 0\leq t \leq t_0\\
%         x^\star(t)=&x^\star(t_0)+(x^\star(t_f)-x^\star(t_0))T^4\\
%         &(70-224T+280T^2-160T^3+35T^4) \\ 
%         & \text{if } t_0 \leq t \leq t_f \\
%         x^\star(t)=&x^\star(t_f)\text{ if }t \geq t_f
%     \end{cases}
% \end{equation*}
%where $0 \leq t_0 < t_f$, $T=(t-t_0)/(t_f-t_0)$. 
% Note that $\dot{x}^\star(t)= \ddot{x}^\star(t)= \dddot{x}^\star(t)=0$ for $t \leq t_0$ and $t\geq t_f$: our 
%B\'ezier 
These curves ensure continuity up to the third order derivative.

% Two scenarios are examined.
% \subsubsection{Scenario 1.}
% While maintaining a constant flow rate $F$, we seek to reduce the level $h$ in the tank by $20\,\%$ while maintaining the concentration $c$: See Figs. \ref{H1} and \ref{M1}.%, and \ref{I1}.
%Dans ce qui suit et pour se conformer aux articles [REF], on considère que la commande %est appliquée toutes les minutes pour une simulation d'une durée de $50min$. En %revanche, on suppose disposer d'une mesure échantillonnée toutes les Dixième de minute, %i.e. $T_e=\frac{1}{10}min$.\\
%Le point de fonctionnemt initial est également celui des références citées plus haut %$(c(0),T(0),h(0)=(0.878 324.5 0.659)$ avec $(Tc(0),F(0))=(300,0.1)$
%\\
%Deux scénarios sont évalués avec les commandes HEOL et MFPC : 
% scenario 1 : tout en gardant un débit constant $F$ et on cherche à réaliser une %diminution du niveau $h$ dans la cuve de $20$ tout en maintenant la concentration $c$ % (voir Figures \ref{H1}, \ref{M1} et \ref{I1}).
 %\subsubsection{Scenario 2.} 
 Introduce the following disturbance: An unmeasured increase of the flow F of $10\,\%$. We seek to
to maintain the initial operating point for $c$ and $h$: See Figs. \ref{H2} and \ref{M2}.%, and \ref{I2}.

The figures above show a certain superiority of HEOL over MFPC. It should be added, however, that MFPC's results are satisfactory and, above all, require no knowledge of the model or theoretical understanding of its structural properties.
 
% une élévation du débit $F$ non mesuré de $10\%$ perturbe le bon fonctionnement et nous %cherchons à conserver le point de fonctionnement initial pour $c$ et $h$ (voir Figures %\ref{H2}, \ref{M2} et \ref{I2}).



% \begin{figure*}[!ht]
% \centering%
% \subfigure[\footnotesize $T_c$ (-- blue) and nominal control (- - red)]
% {\epsfig{figure=H1Tc.eps,width=0.265\textwidth}}%\hspace{.5cm}
% %
% \subfigure[\footnotesize $F$ (-- blue) and nominal control (- - red)]
% {\epsfig{figure=H1F.eps,width=0.265\textwidth}}%\hspace{.5cm}
% %
% \\
% \subfigure[\footnotesize $c$ (-- black) and its reference trajectory (- - red)]
% {\epsfig{figure=H1c.eps,width=0.265\textwidth}}%\hspace{.5cm}
% %
% \subfigure[\footnotesize $T$]
% {\epsfig{figure=H1T.eps,width=0.265\textwidth}}%\hspace{.5cm}
% %
% \subfigure[\footnotesize $h$ (-- black) and its reference trajectory (- - red)]
% {\epsfig{figure=H1h.eps,width=0.265\textwidth}}%\hspace{.5cm}
% %
% \caption{HEOL: setpoint change}\label{H1}
% \end{figure*}
%%
\begin{figure*}[!ht]
\centering%
\subfigure[\footnotesize $T_c$ (-- blue) and nominal control (- - red)]
{\epsfig{figure=H2Tc.eps,width=0.265\textwidth}}%\hspace{.5cm}
%
\subfigure[\footnotesize $F$ (-- blue) and nominal control (- - red)]
{\epsfig{figure=H2F.eps,width=0.265\textwidth}}%\hspace{.5cm}
%
\\
\subfigure[\footnotesize $c$ (-- black) and its reference trajectory (- - red)]
{\epsfig{figure=H2c.eps,width=0.265\textwidth}}%\hspace{.5cm}
%
\subfigure[\footnotesize $T$]
{\epsfig{figure=H2T.eps,width=0.265\textwidth}}%\hspace{.5cm}
%
\subfigure[\footnotesize $h$ (-- black) and its reference trajectory (- - red)]
{\epsfig{figure=H2h.eps,width=0.265\textwidth}}%\hspace{.5cm}
%
\caption{Chemical reactor: HEOL}\label{H2}
\end{figure*}
%%%%%%
% \begin{figure*}[!ht]
% \centering%
% \subfigure[\footnotesize $T_c$]
% {\epsfig{figure=M1Tc.eps,width=0.265\textwidth}}%\hspace{.5cm}
% %
% \subfigure[\footnotesize $F$]
% {\epsfig{figure=M1F.eps,width=0.265\textwidth}}%\hspace{.5cm}
% %
% \\
% \subfigure[\footnotesize $c$ (-- black) and its set-point (- - red)]
% {\epsfig{figure=M1c.eps,width=0.265\textwidth}}%\hspace{.5cm}
% %
% \subfigure[\footnotesize $T$]
% {\epsfig{figure=M1T.eps,width=0.265\textwidth}}%\hspace{.5cm}
% %
% \subfigure[\footnotesize $h$ (-- black) and its set-point (- - red)]
% {\epsfig{figure=M1h.eps,width=0.265\textwidth}}%\hspace{.5cm}
% %
% \caption{MFPC: setpoint change}\label{M1}
% \end{figure*}
%%
\begin{figure*}[!ht]
\centering%
\subfigure[\footnotesize $T_c$]
{\epsfig{figure=M2Tc.eps,width=0.265\textwidth}}%\hspace{.5cm}
%
\subfigure[\footnotesize $F$]
{\epsfig{figure=M2F.eps,width=0.265\textwidth}}%\hspace{.5cm}
%
\\
\subfigure[\footnotesize $c$ (-- black) and its set-point (- - red)]
{\epsfig{figure=M2c.eps,width=0.265\textwidth}}%\hspace{.5cm}
%
\subfigure[\footnotesize $T$]
{\epsfig{figure=M2T.eps,width=0.265\textwidth}}%\hspace{.5cm}
%
\subfigure[\footnotesize $h$ (-- black) and its set-point (- - red)]
{\epsfig{figure=M2h.eps,width=0.265\textwidth}}%\hspace{.5cm}
%
\caption{Chemical reactor: MFPC}\label{M2}
\end{figure*}
%%%%%
% \begin{figure*}[!ht]
% \centering%
% \subfigure[\footnotesize Input 1]
% {\epsfig{figure=I1Tc.eps,width=0.45\textwidth}}%\hspace{.5cm}
% %
% \subfigure[\footnotesize Input 2]
% {\epsfig{figure=I1F.eps,width=0.45\textwidth}}%\hspace{.5cm}
% %
% \\
% \subfigure[\footnotesize Measured output 1]
% {\epsfig{figure=I1c.eps,width=0.265\textwidth}}%\hspace{.5cm}
% %
% \subfigure[\footnotesize Output 2]
% {\epsfig{figure=I1T.eps,width=0.265\textwidth}}%\hspace{.5cm}
% %
% \subfigure[\footnotesize Measured output 3]
% {\epsfig{figure=I1h.eps,width=0.265\textwidth}}%\hspace{.5cm}
% %
% \caption{MFPC$\infty$ : setpoint change}\label{I1}
% \end{figure*}
% %%
% \begin{figure*}[!ht]
% \centering%
% \subfigure[\footnotesize Input 1]
% {\epsfig{figure=I2Tc.eps,width=0.45\textwidth}}%\hspace{.5cm}
% %
% \subfigure[\footnotesize Input 2]
% {\epsfig{figure=I2F.eps,width=0.45\textwidth}}%\hspace{.5cm}
% %
% \\
% \subfigure[\footnotesize Measured output 1]
% {\epsfig{figure=I2c.eps,width=0.265\textwidth}}%\hspace{.5cm}
% %
% \subfigure[\footnotesize Output 2]
% {\epsfig{figure=I2T.eps,width=0.265\textwidth}}%\hspace{.5cm}
% %
% \subfigure[\footnotesize Measured output 3]
% {\epsfig{figure=I2h.eps,width=0.265\textwidth}}%\hspace{.5cm}
% %
% \caption{MFPC$\infty$ : perturbation rejection}\label{I2}
% \end{figure*}

\subsection{Two tank system}\label{tank}
\subsubsection{Ultra-local system}
Use Eq. \eqref{1}, where, according to Appendix B, $y = h_2$.

%The system consists of two tanks with cross-sections $s_1$ and $s_2$. A controlled %inlet flow $u>0$ feeds the first tank. The free outlet flow $q_{12}>0$ feeds the %second tank, which also has a free outlet flow $q_2>0$.

%Le système est constitué de deux cuves de section $s_1$ et $s_2$. Un débit commandé %d'entrée $u>0$ alimente la première cuve. Le débit de sortie, libre, noté $q_{12}>0$ %alimente la seconde cuve qui possède également un débit de sortie, libre, noté %$q_2>0$.\\
%En utilisant les équations de conservation, on écrit 
%\begin{equation}
%\begin{cases}
%s_1 \frac{d}{dt} h_1=u-q_{12}\\
%s_2 \frac{d}{dt} h_2= q_{12}-q_2
%\end{cases}
%\end{equation}

%où $10>h_1>0$ et $10>h_2>0$ sont les hauteurs d'eau dans les cuves.\\
%Il est habituel de modéliser un débit de sortie libre en utilisant les lois de %Bernoulli soit $q_{12}=k_1'\sqrt{h_1}$ et  $q_{2}=k_3'\sqrt{h_2}$ où $k_1'$ et $k_2'$ %sont liés à la viscosité du fluide. Ceci permet alors l'écriture 
%\begin{equation}\label{model1}
%\begin{cases}
%s_1 \dot h_1=u-k_1'\sqrt{h_1}\\
%s_2 \dot h_2= k_1'\sqrt{h_1}-k_3'\sqrt{h_2}
%\end{cases}
%\end{equation}
%le mesure $y=h_2$ est également sortie plate, notée $y^\star$.\\
%En divisant par $s_1$ et $s_2$ les deux équations différentielles du modèle %\eqref{model1} et en posant $k_1=k_1'/s_1$, $k_4=1/s_1$, $k_2=k_1'/s_2$ et %$k_3=k_3'/s_2$ on retrouve les constantes du benchmark "Cascaded Tanks System with %Overflow" décrit https://www.nonlinearbenchmark.org.

\subsubsection{Homeostat}
Eq. \eqref{bernoulli} yields by differentiation:
$$d \dot h_2=\frac{1}{s_2}\left( -\frac{k_2}{2\sqrt{h_2}}dh_2-s_1d \dot h_1+d u \right)$$
%où $\dot h_1$ est indépendant de $du$.\\
The corresponding homeostat reads :
$$\frac{d}{dt} {\Delta h_2} = \mathfrak{F} + \frac{1}{s_2} \Delta u$$

Thus $\alpha=\frac{1}{s_2}$ is constant.

% Avec la sortie plate $y^\star$ et donc la trajectoire de référence $h_2^\star$, on détermine 
% $$h_1^\star=\left(\frac{s_2 \dot y^\star+k_3'\sqrt{y^\star}}{k_1'} \right)^2$$
% et 
% $$u^\star=s_1\dot h_1^\star+k_1'\sqrt{h_1^\star}$$
% La somme des deux équations différentielles donne 
% $$s_1 \dot h_1+s_2 \dot h_2=u-k_3'\sqrt{h_2}$$
% soit
% $$s_2 \dot h_2=u-k_3'\sqrt{h_2}-s_1 \dot h_1$$
% Il est alors évident que $\alpha=\frac{1}{s_2}$ est constant et que $\nu=1$ en référence à la forme du modèle ultra-local \eqref{1}. Cela mène à $\mathcal{C}(\Delta y)= K_p \Delta y$.

%Pour MFPC, on applique scrupuleusement la démarche expliquée plus haut en privilégiant %la solution économe.

\subsubsection{Simulations}
An additive white Gaussian noise $\mathcal{N}(0,0.1)$ corrupts the measurement.
Simulations last $400$s with a sampling period of $0.1$s.
For the iP controller associated to the homeostat command set $K=0.1$. For the MFPC the time horizon is $2$s. Our experiments mimic \cite{adhau}.

Introduce an uncertainty $\nabla=0.2$: The parameters are now $k_1(1+\nabla)$ and $k_2(1-\nabla)$. Without changing the control law parameters, the tracking remains excellent. As shown in Fig. \ref{HU}, the control variable differs nevertheless from the nominal one. Using this uncertain model, the MFPC is evaluated, According to Fig. \ref{MU}, the  setpoints are well reached. Moreover the constraints in Appendix B on the control variable and the water levels are respected.

% Un bruit additif ($\mathcal{N}(0,0.1)$) corrompt la mesure.
%Les simulations durent $400s$ avec une période d'échantillonnage de $0.1s$.
%Pour la commande HEOL, nous fixons $K=0.1$ et pour la MFPC, $T=2s$.
%\\
%Nous avons reproduit la même évolution du setpoint que celle de l'article [Gros 24].\\
% En l'absence d'incertitude HEOL corrige légèrement la commande nominale pour faire face à l'erreur initiale de $0.2$ sur la mesure. Comme le montre la Figure \ref{HN}, la poursuite de trajectoire est parfaite avec $u^\ast$ (resp. $y^\ast$) et $u$ (resp. $y$) pratiquement confondues.\\\\
%L'introduction d'une incertitude $\nabla=0.2$ modifie les paramètres $k_1(1+\nabla)$ %et $k_2(1-\nabla)$. Sans changer le paramètrage de la loi de commande, la poursuite %reste parfaite. Comme on peut le remarquer, Figure \ref{HU}, la commande appliquée %diffère de $u^\ast$ calculée sur le modèle nominale.\\
%\\
%En utilisant ce modèle incertain, la MFPC est évaluée, Figure \ref{MU}. Les Setpoint %sont atteints sans défaut et sans aucune connaissance modèle.


% \begin{figure*}[!ht]
% \centering
% \subfigure[\footnotesize $u^\star$ (- -) and $u$ (--)]
% {\epsfig{figure=HU.eps,width=0.265\textwidth}}%\hspace{.5cm}
% %
% \subfigure[\footnotesize $y^\star$ (- -) and $y$ (--)]
% {\epsfig{figure=HY.eps,width=0.265\textwidth}}%\hspace{.5cm}
% %
% \subfigure[\footnotesize Tank levels $h_1$  and $h_2$]
% {\epsfig{figure=HH.eps,width=0.265\textwidth}}%\hspace{.5cm}
% %
% \caption{HEOL : nominal case}\label{HN}
% \end{figure*}

\begin{figure*}[!ht]
\centering
\subfigure[\footnotesize $u^\star$ (- -) and $u$ (--)]
{\epsfig{figure=HIU.eps,width=0.265\textwidth}}%\hspace{.5cm}
%
\subfigure[\footnotesize $y^\star$ (- -) and $y$ (--)]
{\epsfig{figure=HIY.eps,width=0.265\textwidth}}%\hspace{.5cm}
%
\subfigure[\footnotesize Tank levels $h_1$  and $h_2$]
{\epsfig{figure=HIH.eps,width=0.265\textwidth}}%\hspace{.5cm}
%
\caption{Two tanks: HEOL}\label{HU}
\end{figure*}

\begin{figure*}[!ht]
\centering
\subfigure[\footnotesize $u$ (--)]
{\epsfig{figure=MU.eps,width=0.265\textwidth}}%\hspace{.5cm}
%
\subfigure[\footnotesize Setpoint (- -) and $y$ (--)]
{\epsfig{figure=MY.eps,width=0.265\textwidth}}%\hspace{.5cm}
%
\subfigure[\footnotesize Tank levels $h_1$  and $h_2$]
{\epsfig{figure=MH.eps,width=0.265\textwidth}}%\hspace{.5cm}
%
\caption{Two tanks: MFPC}\label{MU}
\end{figure*}


\section{Conclusion}\label{conclu}
Although the above results on MFPC look promising, most important issues on constraints and stability are missing in this preliminary conference communication. They are now being investigated.

% The stunning successes of modern AI is often based on models as complete as possible and on the associated machine-learning mechanism. The now classic terminology {\em large language model} ({\em LLM}) bears witness to this, as well as to the colossal energy resources required for its development (see, e.g., \cite{llm}). The results reported here may cast some doubt on the universality of this point of view, and, perhaps, pave the way for less greedy techniques.

% \textcolor{red}{The remarkable advances in modern AI often rely on models striving for comprehensiveness, along with their associated machine learning mechanisms. The term 'large language model' (LLM) exemplifies this trend towards comprehensive modeling, reflecting the significant energy resources required for their development (see, e.g., \cite{llm}). The results presented here suggest a complementary perspective to this view, and may open the way for exploring more resource-efficient techniques.}

% \textcolor{green}{
The remarkable advances in modern AI often rely on models striving for comprehensiveness, along with their associated machine learning mechanisms. While outside the realm of control theory, the term {\em large language model} ({\em LLM}) exemplifies this trend towards comprehensive modeling, reflecting the significant energy resources required for their development (see, e.g., \cite{llm}). The results presented here suggest a complementary perspective to this view, and may open the way for exploring more resource-efficient techniques. Why not starting with systematic 
investigations in control engineering?


%Let us emphasize once again the strong algebraic flavor of our tools:
%flatness-based control (\cite{flat}), MFC (Fliess and Join (2013, 2022)), HEOL %%%    %(\cite{heol}), optimal control (\cite{join2}). 


%Most of our tools have an algebraic origin:
%\begin{itemize}
%    \item The Euler-Lagrange equation in optimal control was introduced (\cite{join2}) %via the module-theoretic interpretation of linear controllability (\cite{fliess90}).
%    \item Key parts of MFC have a deep algebraic flavor, for instance Formulae %\eqref{integral} and \eqref{integralbis} for data-driven estimations of key quantities %by employing algebraic estimation techniques (\cite{garnier}). 
%    \item Flatness (\cite{flat}) and the HEOL setting (\cite{heol}) were defined via %the language of differential algebra. 
%\end{itemize}



%\begin{ack}
%Place acknowledgments here.
%\end{ack}

%\bibliography{ifacconf}             % bib file to produce the bibliography
 % with bibtex (preferred)

\newpage
 
                                         \appendix
\section{A chemical reactor}\label{A}  
%\subsection{System description}
Fig. \ref{Process}-(a) depicts a well-stirred chemical reactor, which was introduced by \cite{panno} and \cite{rawlings} for investigating the robustness of linear time-invariant MPC with respect to disturbances. The following nonlinear model is used:
\begin{equation}\label{chemical}
\begin{cases}
  \dot{c}=\frac{F_0(c_0-c)}{\pi r^2h}-k_0\exp{\left(-\frac{E}{RT}\right)}c\\
  \dot T=\frac{F_0(T_0-T)}{\pi r^2h}-\frac{\Delta H}{\rho C_p}k_0\exp{\left(-\frac{E}{RT}\right)}c+\frac{2U}{r\rho C_p}(T_c-T)\\
  \dot h=\frac{F_0-F}{\pi r^2}
  \end{cases}
\end{equation}
where $c$ (resp. $h$) is the fluid concentration (resp. height). Both quantities are measured. The control variables are $F$ and $T_c$. %\footnote{In the above references, a linearized version of Eq. \eqref{chemical},around an equilibrium point, is employed.} 
In the computer experiments $F_0=0.1$, $T_0=350$, $c_0=1$, $r=0.219$, $k_0=7.2.10^{10}$, $\frac{E}{R}=8750$, $U=54.94$, $\rho=1000$, $C_p=0.239$, $\Delta H=-5.10^4$.

%\begin{figure}[!ht]
%\centering
%\subfigure[\footnotesize Well-stirred reactor]
%{\epsfig{figure=React.jpg,width=0.22\textwidth}}%\hspace{.5cm}
%
%\subfigure[\footnotesize Two tanks]
%{\epsfig{figure=2tank,width=0.22\textwidth}}%\hspace{.5cm}
%
%\caption{Process schemes}\label{Process}
%\end{figure}

% \begin{figure}[!ht]
% \centering%
% %\subfigure[\footnotesize Measured output 2 (-- blue) and reference trajectory (- - red)]
% {\epsfig{figure=React.jpg,width=0.265\textwidth}}%\hspace{.5cm}
% %
% \caption{Well-stirred reactor}\label{React}
% \end{figure}

%\subsection{Flatness}\label{appflat}
Eq. \eqref{chemical} defines a \emph{flat} system (\cite{flat}). It is straightforward to check that $c$ and $h$ are \emph{flat outputs}: %All other system variables may be %expressed as \emph{differential functions} of $c$ and $h$, i.e., as functions of $c$, %$h$ and their derivatives up to some finite order: 
$F=-(\dot h\pi r^2-F_0)$, and $T_c=\frac{(\dot T-F_0\frac{T_0-T}{\pi r^2 h}+\frac{\Delta H}{\rho C_p}k_0\exp(-\frac{E}{R T})c)(r\rho C_p)}{2U}+T$, where $T=\frac{-R}{E \log(\dot c+F_0 \frac{c_0-c}{\pi r^2 h k_0 c})}$.


\begin{figure}[!ht]
\centering
\subfigure[\footnotesize Well-stirred reactor]
{\epsfig{figure=React.jpg,width=0.20\textwidth}}%\hspace{.5cm}
%
\subfigure[\footnotesize Two tanks]
{\epsfig{figure=2tank,width=0.20\textwidth}}%\hspace{.5cm}
%
\caption{Process schemes}\label{Process}
\end{figure}

\section{A two tank system}\label{B} 
Consider with \cite{adhau} the two tanks system described by \cite{schoukens}:\footnote{Fig. \ref{Process}-(b) is borrowed from \cite{easy}, where this system is examined via algebraic estimation techniques.}
\begin{equation}\label{bernoulli}
\begin{cases}
s_1\dot{h}_1=u-k_1\sqrt{h_1}\\
s_2\dot{h}_2= k_1\sqrt{h_1}-k_2\sqrt{h_2}
\end{cases}
\end{equation}
where $u \geq 0$ is the control variable, $h_\iota$, $\iota = 1, 2$, $0 \leq h_\iota \leq 10$, is the water level, $k_\iota$ and $s_\iota$ are constant parameters. Eq. \eqref{bernoulli} defines a flat system, where $h_2$ is a flat output. 
%We obtain
%$$h_1=\left(\frac{s_2 \dot h_2+k_2\sqrt{h_2}}{k_1} \right)^2$$
%and 
%$$u=s_1\dot h_1+k_1\sqrt{h_1}$$
In the computer experiments $s_1, s_2 =1$, $k_1=0.6$, $k_2=0.5$.

\newpage
          
\begin{thebibliography}{xx}  % you can also add the bibliography by hand

%\bibitem[Berberich et~al.(2021a)]{berberich1}
%J. Berberich, J. K\"{o}hler, M.A. M\"{u}ller, and F. Allg\"{o}wer.
%\newblock Data-driven model predictive control: closed-loop guarantees and experimental %results.
%\newblock \emph{at - Automatisierungstech.}, 69:\penalty0 608--618, 2021.%

%\bibitem[Berberich et~al.(2021b)]{berberich2}
%J. Berberich, J. K\"{o}hler, M.A. M\"{u}ller, and F. Allg\"{o}wer. 
%Data-Driven Model Predictive Control With Stability and Robustness Guarantees, \emph{IEEE %Trans. Automat. Contr.}, 66:\penalty0 1702--1717, 2021.
\bibitem[Adhau et~al.(2024)]{adhau}
S. Adhau, S. Gros, and S. Skogestad (2024).
\newblock Reinforcement learning based MPC with neural dynamical models.
\newblock \emph{Europ. J. Control}, 80:\penalty0 101048.

\bibitem[Aggarwal(2018)]{ag}
C.C. Aggarwal (2018).
\newblock \emph{Neural Networks and Deep Learning: A Textbook}. Springer, 2018.

\bibitem[{\AA}kesson and Toivonen(2006)]{aa}
B.M. Åkesson, and H.T. Toivonen (2006)
\newblock A neural network model predictive controller.
{\em J. Process Contr.}, 16:\penalty0 937-946.

\bibitem[Bertsekas(2024)]{bertsekas}
J. Berbertsekas (2024).
\newblock Model predictive control and reinforcement learning: A unified framework based on dynamic programming.
\newblock {\em IFAC-PapersOnLine}, 58:\penalty0 363-383



\bibitem[Delaleau et~al.(2025)]{delaleau}
E. Delaleau, C. Join, and M. Fliess (2025).
\newblock Synchronization of Kuramoto oscillators via HEOL, and a  discussion on AI.
\newblock \emph{MATHMOD}, Vienna. 
{\tt 
https://doi.org/10.48550/arXiv.2501.07948} 

%\bibitem[Faulwasser et~al.(2021)]{faul}
%T. Faulwasser, M. Mehrez, K. Worthmann (2021). 
%\newblock Predictive path following control without terminal constraints.  
%\newblock T. Faulwasser, M.A. M\"{u}ller, K. Worthmann, (eds): {\em Recent Advances in %Model Predictive Control}. Lect. Notes Contr. Informat. Sci., vol 485. Springer.

\bibitem[Feng et~al.(2024)]{feng}
Y. Feng, C. Zhang, S. Huang, S. Zhang, J. Qu, Z. Li, and Z. Hue (2024). 
\newblock An improved model-free predictive current control for PMSM drives based on current circle tracking under low-speed conditions. \newblock \emph{IEEE Access}, 12:\penalty0 57767--57779. 

\bibitem[Fliess(1990)]{fliess90}
M. Fliess (1990). 
\newblock Some basic structural properties of generalized linear systems.
\emph{Syst. Contr. Lett.}, 15:\penalty0 391--396.

\bibitem[Fliess and Join(2013)]{mfc1}
M. Fliess, and C. Join (2013). Model-free control. {\em Int. J. Contr.}, 86:\penalty0 2228--2252.

\bibitem[Fliess and Join (2022)]{mfc2}
M. Fliess, and C. Join (2022). An alternative to proportional-integral and proportional-integral-derivative regulators: Intelligent proportional-derivative regulators. {\em Int. J. Robust Nonlin. Contr.}, 32:\penalty0 9512--9524.

\bibitem[Fliess et~al.(2008a)]{easy}
M. Fliess, C. Join, and H. Sira-Rami\`{\i}rez (2008a). Non-linear estimation is easy. 
{\em Int. J. Model. Ident. Contr.}, 4:\penalty0 12-27. 

\bibitem[Fliess et~al.(2018)]{solar}
M. Fliess, C. Join, and C. Voyant (2018).
\newblock Prediction bands for solar energy: New short-term time series forecasting
techniques.
\newblock \emph{Solar Energ.}, 166:\penalty0 519-528.

\bibitem[Fliess et~al.(1995)]{flat}
M. Fliess, J. L{\'e}vine, P. Martin, and P. Rouchon (1995). 
\newblock Flatness and defect of non-linear systems: introductory theory and examples.
\newblock {\em Int. J. Contr.}, vol. 61:\penalty0 1327--1361.

\bibitem[Fliess and Sira-Ram\'{\i}rez(2008b)]{garnier}
M. Fliess, and H. Sira-Ram\'{\i}rez (2008b). 
\newblock Closed-loop parametric identification for continuous-time
linear systems via new algebraic techniques. 
\newblock H. Garnier, and L. Wang (eds):
\emph{Identification of Continuous-time Models from Sampled Data}.
pages 362--391, Springer.

\bibitem[Gelfand and Fomin(1963)]{gelfand}
I.M. Gelfand, and S.V. Fomin (1963).
\newblock \emph{Calculus of Variations} (translated from the Russian).
\newblock Prentice-Hall.

\bibitem[Hegendus et~al.(2023)]{hegendus}
T. Hegedus, D. Fenyes, V. Van Tan, and P. Gaspar (2023).
\newblock Lateral control for automated vehicles based on model predictive control and error-based ultra-local model.
\newblock \emph{Proc. 20th Internat. Conf. Informat. Contr. Automat. Robot.}, Rome.

\bibitem[Huo and Li(2025)]{huo}
X. Huo, P. Li (2025).
\newblock Parameter-free ultralocal model-based predictive current control method for grid-tied inverters using extremum seeking control.
\newblock \emph{Contr. Engin. Pract.},
157:\penalty0 106253.

\bibitem[Join et~al.(2024a)]{heol}
C. Join, E. Delaleau, and M. Fliess (2024a).
\newblock Flatness-based control revisited: The HEOL setting.
\newblock \emph{C.R. Math.}, 362:\penalty0 1693--1706.

\bibitem[Join et~al.(2024b)]{join2}
C. Join, E. Delaleau, and M. Fliess (2024b).
\newblock The Euler-Lagrange equation and optimal control: Preliminary results.
\newblock \emph{12th Internat. Conf. Syst. Contr.}, Batna. 
{\tt https://doi.org/10.48550/arXiv.2410.07040}

\bibitem[Khalilzadeh et~al.(2021)]{khal}
M. Khalilzadeh, S. Vaez-Zadeh, J. Rodriguez, and R. Heydari (2021). 
\newblock Model-free predictive control of motor drives and power converters: A review. 
\newblock {\em IEEE Access} 9:\penalty0 105733-105747.

\bibitem[Lammouchi et~al.(2024)]{lammouchi}
Z. Lammouchi, C. Labiod, K. Srairi, M. Benbouzid, A. Khechekhouche, F. Albalawi, S.S.M. Ghoneim, E. Ali, and A.B.A. Sharaf (2024). 
\newblock Experimental validation of model-free predictive control based on the active vector execution time for grid-connected system.
\newblock {\em Sci. Rep.}, 14:\penalty0 30326. %(2024). https://doi.org/10.1038/s41598-%024m-77846-0

\bibitem[LeCun(2024)]{lecun}
Y. LeCun (2024). 
\newblock {\em X (Twitter)}, 25 August 2024.

\bibitem[Liu et~al.(2024)]{liu}
X. Liu, L. Qiu, Y. Fang, K. Wang, Y. Li, and J. Rodríguez (2024) 
\newblock A two-step event-triggered-based data-driven predictive control for power converters.
\newblock \emph{IEEE Trans. Indust. Electron.},719:\penalty0 13545--13555.

\bibitem[Long et~al.(2023)]{long}
B. Long, J. Zhang, D. Shen, J. Rodríguez, J.M. Guerrero, and K. t. Chong (2023).
\newblock Ultralocal model-free predictive control of T-type grid-connected converters based on extended sliding-mode disturbance observer.
\newblock \emph{IEEE Trans. Power Electron.}, 38:\penalty0 15494--15508.

\bibitem[Michel et~al.(2025)]{michel}
L. Michel1, C. Braud, J.-P. Barbot, F. Plestan, D. Peaucelle, and
X. Boucher (2025).
\newblock Comparison of different feedback controllers on an airfoil benchmark.
\newblock {\em Wind Energ. Sci.}, 10:\penalty0 177–191.

\bibitem[Nubert et~al.(2020)]{nubert}
J. Nubert, J. K\"{o}hler, V. Berenz, F. Allg\"{o}wer, and S. Trimpe (2020). 
\newblock Safe and fast tracking on a robot manipulator: Robust MPC and neural network control.
\newblock {\em IEEE Robot. Automat. Lett.}, 5:\penalty0 3050-3057.

%\bibitem[Nauman and Shireen(2022)]{nauman}
%M. Nauman, W. Shireen, and A. Hussain (2022).
%\newblock Model-free predictive control and its applications. 
%\newblock \emph{Energies}, 15:\penalty0 5131. 

\bibitem[Pannochia and Rawlings(2003)]{panno}
G. Pannochia, and J.B. Rawlings (2003).
\newblock Disturbance models for offset-free model-predictive control.
\newblock \emph{AIChE J.}, 49:\penalty0 426--439.

\bibitem[Rawlings et~al.(2022)]{rawlings}
J.B. Rawlings, D.Q. Mayne, and M.M. Diehl (2022).
\newblock \emph{Model-Predictive Control: Theory, Computation, and Design} (2nd ed.).
\newblock Nob Hill Publishing.

\bibitem[Recht(2019)]{recht}
B. Recht (2019).
\newblock A tour of reinforcement learning: The view from continuous control.
\newblock \emph{Annual Rev. Contr. Robot. Auton. Syst.},
2:\penalty0 253-279.

\bibitem[Ren et~al.(2022)]{ren}
Y.M. Ren, M.S. Alhajeri, J. Luo, S. Chen, F. Abdullah, Z. Wu, and P.D. Christofides (2022).
\newblock A tutorial review of neural network modeling approaches for model predictive control.
\newblock {\em Comput. Chem. Engin.}, 165:\penalty0 107956.

\bibitem[Rogers(2001)]{rogers}
D.F. Rogers (2001).
\newblock \emph{An Introduction to NURBS with Historical Perspective}. \newblock Morgan Kaufmann.

\bibitem[Salzmann et~al.(2023)]{salzmann}
T. Salzmann, E. Kaufmann, J. Arrizabalaga, M. Pavone, D. Scaramuzza, and M. Ryll (2003). 
\newblock Real-time neural MPC: Deep learning model predictive control for quadrotors and agile robotic platforms. 
\newblock {\em IEEE Robot. Automat. Lett.}, 8:\penalty0 2397-2404.

\bibitem[Schoukens and No\"{e}l(2017)]{schoukens}
J. Schoukens, and J.-P. No\"{e}l (2017).
\newblock Nonlinear system identification: A user-oriented road map.
\newblock \emph{IFAC-PapersOnLine}, 5:\penalty0 446-451.

%\bibitem[Shi et~al.(2023)]{shi}
%G. Shi, M. Ma, D. Li, Y. Ding, and %K.Y. Lee (2023).
%\newblock A process-model-free method
%for model predictive control via a
%reference model-based proportional-%integral-derivative controller with
%application to a thermal power plant.
%\newblock \emph{Front. Control. %Eng.}, 4:\penalty0 1185502.


\bibitem[Sun et~al.(2023a)]{sun1}
Z. Sun, Y. Deng, J. Wang. H. Li, and H. Cao (2023a).
\newblock Improved cascaded model-free predictive speed control for PMSM speed ripple minimization based on ultra-local model.
\newblock \emph{ISA Trans.}, 143:\penalty0 666--677.

\bibitem[Sun et~al.(2023b)]{sun2}
Z. Sun, Y. Deng, J. Wang, T. Yang, Z. Wei, and H. Cao (2023b).
\newblock Finite control set model-free predictive current control of PMSM with two voltage vectors based on ultralocal model. 
\newblock \emph{IEEE Trans. Power Electr.}, 38:\penalty0 776-788.

\bibitem[Xu et~al.(2021)]{xu}
L. Xu, G. Chen,  and Q. Li (2021).
\newblock Cascaded speed and current model of PMSM with ultra-local
model-free predictive control.
\newblock \emph{IEEE Electr. Power Appl.}, 15:\penalty0 1424-1437.

\bibitem[Xu and Poo(2023)]{llm}
B. Xu, and M. Poo (2023).
\newblock Large language models and brain-inspired general intelligence.
\newblock {\em Natl. Sci. Rev.}, 10:\penalty0 267.

\bibitem[Xu et~al.(2024)]{xu2}
B. Xu, Q. Wu, J. Ma, X. Liu, and Y. Fang (2024).
\newblock Improved non-cascaded continuous-set model-free predictive
control scheme for PMSM drives.
\newblock \emph{IET Power Electron.}, {\tt https://doi.org/10.1049/pel2.12830}

\bibitem[Wang et~al.(2025)]{energy}
Z. Wang, G. Xu, R. Sun, A. Zhou, A. Cook, and Y. Chen (2025).  
\newblock Online energy consumption forecast for battery electric buses using a learning-free algebraic method. 
\newblock \emph{Sci. Rep.}, 15:\penalty0 1931.

%\bibitem[Wei et~al.(2025)]{wei}
%Y. Wei, H. Young, S. Fang, D. Ke, H. Xie; F. Wang, J. Rodrigez (2025). 
%\newblock Low prediction error model-free predictive control on PMSM drives with %ordinary kriging time-shift. 
%\newblock {\em IEEE Trans. Transport. Electrif.}, {\tt doi: 10.1109/TTE.2025.3526791}

%\bibitem[Wang et~al.(2025)]{low}
%F. Wang, Y. Wei, H. Young, D. Ke, X. %Yu, and J. Rodríguez (2025). 
%\newblock Low-stagnation model-free %predictive current control of PMSM %drives. 
%\newblock \emph{IEEE Trans. Industr. %Electron.}, 2024, {\tt doi: %10.1109/TIE.2024.3508070} 

\bibitem[Wang and Wang(2020)]{wang}
Z. Wang, and J. Wang (2020).
\newblock Ultra-local model predictive control: A model-free approach and its application on automated vehicle trajectory tracking.
\newblock \emph{Contr. Engin. Pract.}, 181:\penalty0 104482.

%\bibitem[Wei et~al.(2023)]{wei}
%Y. Wei, H. Young, F. Wang, and J. %Rodríguez (2023). 
%\newblock Generalized data-driven %model-free predictive control for %electrical drive systems.
%\newblock \emph{IEEE Trans. Industr. %Electron.}, 70:\penalty0  7642--7652. 

%\bibitem[Wei et~al.(2025)]{wei2}
%Y. Wei et al., "Low Prediction Error %Model-Free Predictive Control on PMSM %Drives with Ordinary Kriging Time-%Shift," in IEEE Transactions on %Transportation Electrification, doi: %10.1109/TTE.2025.3526791.


\bibitem[Yin et~al.(2024b)]{yin}
Z. Yin, X. Wang, X. Su, Y. Shen, D. Xiao, and H. Zhao (2024b). 
\newblock A switched ultra-local model-free predictive controller for PMSMs. 
\newblock \emph{IEEE Trans. Power Electron.}, 39:\penalty0 10665--10669. 

\bibitem[Yuan et~al.(2024)]{yuan}
L. Yuan, J. Mei, A. Xu, and X. Li (2024).
\newblock Model-free predictive control with super- twisting observer of single-phase neutral-clamped H-bridge cascade inverters with fixed switching frequency. 
\newblock {\em IEEE J. Emerg. Select. Topics Power Electron.}, 2024 {\tt doi: 10.1109/JESTPE.2024.3508749}

\bibitem[Zhang et~al.(2020a)]{zhang1}
Y. Zhang, T. Jiang, and J. Jiao (2020a). 
\newblock Model-free predictive current control of a DFIG using an ultra-local model for grid synchronization and power regulation. 
\emph{IEEE Trans. Energy Convers.}, 35:\penalty0 2269--2280. 

\bibitem[Zhang et~al.(2020b)]{zhang2}
Y. Zhang, X. Liu, J. Liu, J. Rodriguez, and C. Garcia (2020b). 
\newblock Model-free predictive current control of power converters based on ultra-local model. 
\newblock \emph{IEEE Internat. Conf. Indust. Techno.}, pp. 1089-1093, Buenos Aires.

\bibitem[Zhang et~al.(2025)]{zhang3}
G. Zhang, X. Yao, L. Peretti, J. Bai, X. Gao, and Z. Li (2025) 
\newblock Smooth nonlinear ESO-based model-free predictive current control with an extended control set for SPMSM drives. 
\newblock {\em IEEE J. Emerg. Select. Topics Power Electr.}, {\tt doi: 10.1109/JESTPE.2025.3528761}

\bibitem[Zhou et~al.(2022)]{zhou}
Y. Zhou, Z. Wang, X. Zhou, H. Shen, J. Ahn, and J. Wang (2022).
\newblock Extremum-seeking-based ultra-local model predictive control and its application to electric motor speed regulation.
\newblock \emph{~IFAC-PapersOnLine}, 55-37:\penalty0 56-61.

%\bibitem[Able(1956)]{Abl:56}
%B.C. Able.
%\newblock Nucleic acid content of microscope.
%\newblock \emph{Nature}, 135:\penalty0 7--9, 1956.

%\bibitem[Able et~al.(1954)Able, Tagg, and Rush]{AbTaRu:54}
%B.C. Able, R.A. Tagg, and M.~Rush.
%\newblock Enzyme-catalyzed cellular transanimations.
%\newblock In A.F. Round, editor, \emph{Advances in Enzymology}, %volume~2, pages
% 125--247. Academic Press, New York, 3rd edition, 1954.

%\bibitem[Keohane(1958)]{Keo:58}
%R.~Keohane.
%\newblock \emph{Power and Interdependence: World Politics in %Transitions}.
%\newblock Little, Brown \& Co., Boston, 1958.

%\bibitem[Powers(1985)]{Pow:85}
%T.~Powers.
%\newblock Is there a way out?
%\newblock \emph{Harpers}, pages 35--47, June 1985.

%\bibitem[Soukhanov(1992)]{Heritage:92}
%A.~H. Soukhanov, editor.
%\newblock \emph{{The American Heritage. Dictionary of the American %Language}}.
%\newblock Houghton Mifflin Company, 1992.

\end{thebibliography}


                                                                         % in the appendices.
\end{document}
Pour la commande HEOL, nous avons choisi les gain proportionnels $K_c=K_h=1$.\\
Pour la commande MFPC, nous avons choisi de calculer une nouvelle trajectoire toutes les minutes, $\alpha_c=-0.05$ et $\alpha_h=-5$.\\
Pour la commande MFPC$\infty$ nous avons choisi de calculer une nouvelle trajectoire toutes les minutes, $\alpha_c=-0.05$ et $\alpha_h=-5$. De plus, comme précédemment expliqué, $c_1=0$.
\\
La commande HEOL montre une certaine supériorité dans les deux scénarios mais exploite une connaissance plus poussée du modèle. La commande MFPC$\infty$ est en retrait concernant la maitrise de la concentration. Exclusivement dépendante de $\alpha_h$ (par ailleurs plus ou moins imposé par le système) la convergence semble maitrisée mais très lente.



%où $\Delta c=c-c^\star$, $\Delta h=h-h^\star$, $F_c$ et $F_h$ deux fonctions estimées à %chaque échantillon de mesure, $\Delta u_c=T_c-T_c^\star$ et $\Delta u_h=F-F^\star$. Les %valeurs des paramètres $\alpha_c=-\frac{2{EUk_0}\exp(-\frac{E}{RT^\star})c^\star}%   {RC_pr\rho (T^\star)^2}$ et $\alpha_h=-\frac{1}{\pi r^2}$ sont obtenues en difféenciant %respectivement $\ddot c$ et $\dot h$  autour des trajectoires nominales $c^\star$ et %$h^\star$.\\
\\
%Les commandes, en boucle fermée, sont donc définies par $T_c=\Delta u_c+T_c^\star$ et %$F=\Delta u_h+F^\star$ avec
%$$\Delta u_c=\frac{1}{\alpha_c }(-F_c-K_c \Delta c)$$
et
$$\Delta u_h=\frac{1}{\alpha_h }(-F_h-K_h \Delta h)$$
avec $K_c>0$ et $K_h>0$ pour assurer une convergence de l'erreur de poursuite vers $0$. Ici, nous avons choisi les gains proportionnels $K_c=K_h=1$.\\


Puisque le traitement des deux sortie $c$ et $h$ est indifférencié, on considère le modèle ultra-local générique
\begin{equation}\label{ml}
\dot y=F+\alpha u
\end{equation} où $\alpha$ est réglé par l'utilisateur pour rendre comparable les trois quantités de l'équation précédente.\\
$y$ constitue $c$ dans un cas puis $h$ dans l'autre. Quant à $u$, il s'agit de $T_c$ puis $F$.\\
De l'équation précédente, en notant $\hat F$ l'estimation de $F$, on obtient la relation $u=\frac{\dot y+\hat F}{\alpha}$.\\
\\

Avec la connaissance du modèle local, on cherche à minimiser le critère 
\begin{equation}\label{jy}J(y)=\int_{t_0}^{t_f} \mathcal{L} dt=\int_{t_0}^{t_f}\left( (y-y_{ref})^2+u^2 \right) dt=\int_{t_0}^{t_f}\left( (y-y_{ref})^2+\left(\frac{\dot y+\hat F}{\alpha}\right)^2 \right) dt\end{equation}
En considérant $\hat F$ indépendant de $y$ et de ses dérivées, la solution optimale satisfait l'équation différentielle linéaire suivante :
$$y - y_{ref} - \frac{\dot{\hat F} + \ddot y}{\alpha^2} = 0$$
En ajoutant que $\hat F$ peut être considérée constante, notée $\bar F$, durant un intervalle de temps relativement court, cette équation se simplifie encore comme suit 
\begin{equation}\label{yed} y - y_{ref} - \frac{ \ddot y}{\alpha^2} = 0\end{equation}
La solution, $y^\star$, que nous qualifions de trajectoire optimale s'écrit
$$y^\star= y_{ref} + c_1\exp(\alpha t) + c_2\exp(-\alpha t)$$
Les deux constantes $c_1$ et $c_2$ sont déterminées avec les conditions aux bords $y^\star(t_0)=y(t_0)$ et $y^\star(t_f)=y(t_f)=y_{ref}$.\\
$$c_1 =\frac{y(t_0)\exp(-\alpha t_f) - y_{ref}\exp(-\alpha t_f)}{\exp(\alpha t_0)\exp(-\alpha t_f) - \exp(-\alpha t_0)\exp(\alpha t_f)}$$
$$c_2 =-\frac{\exp(\alpha t_f)(y_0 - y_{ref})}{\exp(\alpha t_0)\exp(-\alpha t_f) - \exp(-\alpha t_0)\exp(\alpha t_f)}$$
On connait d'ailleurs également la forme analytique de $\dot y^\star$.\\
\\
Une autre solution consiste à travailler de manière asymptotique (ou en temps infini) qui se traduit par imposer $c_1=0$ et à exprimer $c_2$ de manière à assurer la continuité en $t_0$ soit
$$c_2=\left(y(t_0)-y_{ref}\right)\exp(\alpha t_0)$$
\\
C'est la reproduction de ces calculs, à intervalle de temps régulier, qui joue le rôle de boucle fermée
\begin{equation}\label{u}u=\frac{\dot y^\star-\hat F}{\alpha}\end{equation}


\subsection{System description}
We consider a well-stirred chemical reactor depicted in Figure \ref{React}, as
in Pannocchia and Rawlings (2003). An irreversible, first-order reac-
tion A -→ B occurs in the liquid phase and the reactor temperature is
regulated with external cooling. Mass and energy balances lead to the
following nonlinear state space model
% dc
% dt =
% F0(c0 − c)
% πr 2h
% − k0 exp 
% −
% E
% RT 
% c

% dT
% dt =
% F0(T0 − T )
% πr 2h
% +
% −∆H
% ρCp
% k0 exp 
% −
% E
% RT 
% c +
% 2UcPuisque le traitement des deux sortie $c$ et $h$ est indifférencié, on considère le modèle ultra-local générique
\begin{equation}\label{ml}
\dot y=F+\alpha u
\end{equation} où $\alpha$ est réglé par l'utilisateur pour rendre comparable les trois quantités de l'équation précédente.\\
$y$ constitue $c$ dans un cas puis $h$ dans l'autre. Quant à $u$, il s'agit de $T_c$ puis $F$.\\
De l'équation précédente, en notant $\hat F$ l'estimation de $F$, on obtient la relation $u=\frac{\dot y+\hat F}{\alpha}$.\\
\\

Avec la connaissance du modèle local, on cherche à minimiser le critère 
\begin{equation}\label{jy}J(y)=\int_{t_0}^{t_f} \mathcal{L} dt=\int_{t_0}^{t_f}\left( (y-y_{ref})^2+u^2 \right) dt=\int_{t_0}^{t_f}\left( (y-y_{ref})^2+\left(\frac{\dot y+\hat F}{\alpha}\right)^2 \right) dt\end{equation}
En considérant $\hat F$ indépendant de $y$ et de ses dérivées, la solution optimale satisfait l'équation différentielle linéaire suivante :
$$y - y_{ref} - \frac{\dot{\hat F} + \ddot y}{\alpha^2} = 0$$
En ajoutant que $\hat F$ peut être considérée constante, notée $\bar F$, durant un intervalle de temps relativement court, cette équation se simplifie encore comme suit 
\begin{equation}\label{yed} y - y_{ref} - \frac{ \ddot y}{\alpha^2} = 0\end{equation}
La solution, $y^\star$, que nous qualifions de trajectoire optimale s'écrit
$$y^\star= y_{ref} + c_1\exp(\alpha t) + c_2\exp(-\alpha t)$$
Les deux constantes $c_1$ et $c_2$ sont déterminées avec les conditions aux bords $y^\star(t_0)=y(t_0)$ et $y^\star(t_f)=y(t_f)=y_{ref}$.\\
$$c_1 =\frac{y(t_0)\exp(-\alpha t_f) - y_{ref}\exp(-\alpha t_f)}{\exp(\alpha t_0)\exp(-\alpha t_f) - \exp(-\alpha t_0)\exp(\alpha t_f)}$$
$$c_2 =-\frac{\exp(\alpha t_f)(y_0 - y_{ref})}{\exp(\alpha t_0)\exp(-\alpha t_f) - \exp(-\alpha t_0)\exp(\alpha t_f)}$$
On connait d'ailleurs également la forme analytique de $\dot y^\star$.\\
\\
Une autre solution consiste à travailler de manière asymptotique (ou en temps infini) qui se traduit par imposer $c_1=0$ et à exprimer $c_2$ de manière à assurer la continuité en $t_0$ soit
$$c_2=\left(y(t_0)-y_{ref}\right)\exp(\alpha t_0)$$
\\
C'est la reproduction de ces calculs, à intervalle de temps régulier, qui joue le rôle de boucle fermée
\begin{equation}\label{u}u=\frac{\dot y^\star-\hat F}{\alpha}\end{equation}

\subsection{Numerical experiments}
Dans ce qui suit et pour se conformer aux articles [REF], on considère que la commande est appliquée toutes les minutes pour une simulation d'une durée de $50min$. En revanche, on suppose disposer d'une mesure échantillonnée toutes les Dixième de minute, i.e. $T_e=\frac{1}{10}min$.\\
Le point de fonctionnemt initial est également celui des références citées plus haut $(c(0),T(0),h(0)=(0.878 324.5 0.659)$ avec $(Tc(0),F(0))=(300,0.1)$
\\
Deux scénarios sont évalués avec les commandes HEOL et MFPC : 
\begin{itemize}
\item scenario 1 : tout en gardant un débit constant $F$ et on cherche à réaliser une diminution du niveau $h$ dans la cuve de $20\%$ tout en maintenant la concentration $c$ (voir Figures \ref{H1}, \ref{M1} et \ref{I1}).
\item scenario 2 : une élévation du débit $F$ non mesuré de $10\%$ perturbe le bon fonctionnement et nous cherchons à conserver le point de fonctionnement initial pour $c$ et $h$ (voir Figures \ref{H2}, \ref{M2} et \ref{I2}).
\end{itemize}
Pour la commande HEOL, nous avons choisi les gain proportionnels $K_c=K_h=1$.\\
Pour la commande MFPC, nous avons choisi de calculer une nouvelle trajectoire toutes les minutes, $\alpha_c=-0.05$ et $\alpha_h=-5$.\\
Pour la commande MFPC$\infty$ nous avons choisi de calculer une nouvelle trajectoire toutes les minutes, $\alpha_c=-0.05$ et $\alpha_h=-5$. De plus, comme précédemment expliqué, $c_1=0$.
\\
La commande HEOL montre une certaine supériorité dans les deux scénarios mais exploite une connaissance plus poussée du modèle. La commande MFPC$\infty$ est en retrait concernant la maitrise de la concentration. Exclusivement dépendante de $\alpha_h$ (par ailleurs plus ou moins imposé par le système) la convergence semble maitrisée mais très lente.




This document is a template for \LaTeXe. If you are reading a paper or
PDF version of this document, please download the electronic file
\texttt{ifacconf.tex}. You will also need the class file
\texttt{ifacconf.cls}. Both files are available on the IFAC web site.

Please stick to the format defined by the \texttt{ifacconf} class, and
do not change the margins or the general layout of the paper. It
is especially important that you do not put any running header/footer
or page number in the submitted paper.\footnote{
This is the default for the provided class file.}
Use \emph{italics} for emphasis; do not underline.

Page limits may vary from conference to conference. Please observe the 
page limits of the event for which your paper is intended.


\section{Procedure for Paper Submission}

Next we see a few subsections.

\subsection{Review Stage}

For submission guidelines, follow instructions on paper submission
system as well as the event website.

Note that conferences impose strict page limits, so it will be better
for you to prepare your initial submission in the camera ready layout
so that you will have a good estimate for the paper
length. Additionally, the effort required for final submission will be
minimal.

\subsection{Equations}

Some words might be appropriate describing equation~(\ref{eq:sample}), if 
we had but time and space enough. 

\begin{equation} \label{eq:sample}
{{\partial F}\over {\partial t}} = D{{\partial^2 F}\over {\partial x^2}}.
\end{equation}

See \cite{Abl:56}, \cite{AbTaRu:54}, \cite{Keo:58} and \cite{Pow:85}.

\subsubsection{Example.} This equation goes far beyond the
celebrated theorem ascribed to the great Pythagoras by his followers.

\begin{thm}   % use the thm environment for theorems
The square of the length of the hypotenuse of a right triangle equals
the sum of the squares of the lengths of the other two sides.
\end{thm}

\begin{pf}    % and the pf environment for proofs
The square of the length of the hypotenuse of a right triangle equals the sum of the squares 
of the lengths of the other two sides.
\end{pf}

%% There are a number of predefined theorem-like environments in
%% ifacconf.cls:
%%
%% \begin{thm} ... \end{thm}            % Theorem
%% \begin{lem} ... \end{lem}            % Lemma
%% \begin{claim} ... \end{claim}        % Claim
%% \begin{conj} ... \end{conj}          % Conjecture
%% \begin{cor} ... \end{cor}            % Corollary
%% \begin{fact} ... \end{fact}          % Fact
%% \begin{hypo} ... \end{hypo}          % Hypothesis
%% \begin{prop} ... \end{prop}          % Proposition
%% \begin{crit} ... \end{crit}          % Criterion

Of course LaTeX manages equations through built-in macros. You may
wish to use the \texttt{amstex} package for enhanced math
capabilities.

\subsection{Figures}

To insert figures, use the \texttt{graphicx} package. Although other
graphics packages can also be used, \texttt{graphicx} is simpler to
use. See  Fig.~\ref{fig:bifurcation} for an example.

\begin{figure}
\begin{center}
\includegraphics[width=8.4cm]{bifurcation}    % The printed column width is 8.4 cm.
\caption{Bifurcation: Plot of local maxima of $x$ with damping $a$ decreasing} 
\label{fig:bifurcation}
\end{center}
\end{figure}

Figures must be centered, and have a caption at the bottom. 

\subsection{Tables}
Tables must be centered and have a caption above them, numbered with
Arabic numerals. See table~\ref{tb:margins} for an example.

\begin{table}[hb]
\begin{center}
\caption{Margin settings}\label{tb:margins}
\begin{tabular}{cccc}
Page & Top & Bottom & Left/Right \\\hline
First & 3.5 & 2.5 & 1.5 \\
Rest & 2.5 & 2.5 & 1.5 \\ \hline
\end{tabular}
\end{center}
\end{table}

\subsection{Final Stage}

Authors are expected to mind the margins diligently.  Papers need to
be stamped with event data and paginated for inclusion in the
proceedings. If your manuscript bleeds into margins, you will be
required to resubmit and delay the proceedings preparation in the
process.

\subsubsection{Page margins.} See table~\ref{tb:margins} for the
page margins specification. All dimensions are in \emph{centimeters}.


\subsection{PDF Creation}

All fonts must be embedded/subsetted in the PDF file. Use one of the
following tools to produce a good quality PDF file:

\subsubsection{PDFLaTeX} is a special version of LaTeX by Han The
Thanh which produces PDF output directly using Type-1 fonts instead of
the standard \texttt{dvi} file. It accepts figures in JPEG, PNG, and PDF
formats, but not PostScript. Encapsulated PostScript figures can be
converted to PDF with the \texttt{epstopdf} tool or with Adobe Acrobat
Distiller.

\subsubsection{Generating PDF from PostScript} is the classical way of
producing PDF files from LaTeX. The steps are:

\begin{enumerate}
  \item Produce a \texttt{dvi} file by running \texttt{latex} twice.
  \item Produce a PostScript (\texttt{ps}) file with \texttt{dvips}.
  \item Produce a PDF file with \texttt{ps2pdf} or Adobe Acrobat
  Distiller.
\end{enumerate}

\subsection{Copyright Form}

IFAC will put in place an electronic copyright transfer system in due
course. Please \emph{do not} send copyright forms by mail or fax. More
information on this will be made available on IFAC website.


\section{Units}

Use SI as primary units. Other units may be used as secondary units
(in parentheses). This applies to papers in data storage. For example,
write ``$15\,\mathrm{Gb}/\mathrm{cm}^2$ ($100\,\mathrm{Gb}/\mathrm{in}^2$)''. 
An exception is when
English units are used as identifiers in trade, such as ``3.5 in
disk drive''. Avoid combining SI and other units, such as current in
amperes and magnetic field in oersteds. This often leads to confusion
because equations do not balance dimensionally. If you must use mixed
units, clearly state the units for each quantity in an equation.  The
SI unit for magnetic field strength $\mathbf{H}$ is $\mathrm{A}/\mathrm{m}$. However, if you wish to
use units of $\mathrm{T}$, either refer to magnetic flux density $\mathbf{B}$ or
magnetic field strength symbolized as $\mu_0\,\mathbf{H}$. Use the center dot to
separate compound units, e.g., ``$\mathrm{A} \cdot \mathrm{m}^2$''.

\section{Helpful Hints}

\subsection{Figures and Tables}

Figure axis labels are often a source of confusion. Use words rather
than symbols. As an example, write the quantity ``Magnetization'', or
``Magnetization M'', not just ``M''. Put units in parentheses. Do not
label axes only with units.  For example, write ``Magnetization
($\mathrm{A}/\mathrm{m}$)'' or ``Magnetization ($\mathrm{A} \mathrm{m}^{-1}$)'', not just
 ``$\mathrm{A}/\mathrm{m}$''. Do not
label axes with a ratio of quantities and units. For example, write
``Temperature ($\mathrm{K}$)'', not ``$\mbox{Temperature}/\mathrm{K}$''.

Multipliers can be especially confusing. Write ``Magnetization
($\mathrm{kA}/\mathrm{m}$)'' or ``Magnetization ($10^3 \mathrm{A}/\mathrm{m}$)''. Do not write
``Magnetization $(\mathrm{A}/\mathrm{m}) \times 1000$'' because the reader would not know
whether the axis label means $16000\,\mathrm{A}/\mathrm{m}$ or $0.016\,\mathrm{A}/\mathrm{m}$.

\subsection{References}

Use Harvard style references (see at the end of this document). With
\LaTeX, you can process an external bibliography database 
using \texttt{bibtex},\footnote{In this case you will also need the \texttt{ifacconf.bst}
file, which is part of the \texttt{ifaconf} package.}
or insert it directly into the reference section. Footnotes should be avoided as
far as possible.  Please note that the references at the end of this
document are in the preferred referencing style. Papers that have not
been published should be cited as ``unpublished''.  Capitalize only the
first word in a paper title, except for proper nouns and element
symbols.

\subsection{Abbreviations and Acronyms}

Define abbreviations and acronyms the first time they are used in the
text, even after they have already been defined in the
abstract. Abbreviations such as IFAC, SI, ac, and dc do not have to be
defined. Abbreviations that incorporate periods should not have
spaces: write ``C.N.R.S.'', not ``C. N. R. S.'' Do not use abbreviations
in the title unless they are unavoidable (for example, ``IFAC'' in the
title of this article).

\subsection{Equations}

Number equations consecutively with equation numbers in parentheses
flush with the right margin, as in (\ref{eq:sample}).  To make your equations more
compact, you may use the solidus ($/$), the $\exp$ function, or
appropriate exponents. Use parentheses to avoid ambiguities in
denominators. Punctuate equations when they are part of a sentence, as
in

\begin{equation} \label{eq:sample2}
\begin{array}{ll}
\int_0^{r_2} & F (r, \varphi ) dr d\varphi = [\sigma r_2 / (2 \mu_0 )] \\
& \cdot \int_0^{\inf} exp(-\lambda |z_j - z_i |) \lambda^{-1} J_1 (\lambda  r_2 ) J_0 (\lambda r_i ) d\lambda 
\end{array}
\end{equation}

Be sure that the symbols in your equation have been defined before the
equation appears or immediately following. Italicize symbols ($T$
might refer to temperature, but T is the unit tesla). Refer to
``(\ref{eq:sample})'', not ``Eq. (\ref{eq:sample})'' or ``equation
(\ref{eq:sample})'', except at the beginning of a sentence: ``Equation
(\ref{eq:sample}) is \ldots''.

\subsection{Other Recommendations}

Use one space after periods and colons. Hyphenate complex modifiers:
``zero-field-cooled magnetization''. Avoid dangling participles, such
as, ``Using (1), the potential was calculated'' (it is not clear who or
what used (1)). Write instead: ``The potential was calculated by using
(1)'', or ``Using (1), we calculated the potential''.

A parenthetical statement at the end of a sentence is punctuated
outside of the closing parenthesis (like this). (A parenthetical
sentence is punctuated within the parentheses.) Avoid contractions;
for example, write ``do not'' instead of ``don' t''. The serial comma
is preferred: ``A, B, and C'' instead of ``A, B and C''.


\subsection{System description}
We consider a well-stirred chemical reactor depicted in Figure \ref{React}, as
in Pannocchia and Rawlings (2003). An irreversible, first-order reac-
tion A -→ B occurs in the liquid phase and the reactor temperature is
regulated with external cooling. Mass and energy balances lead to the
following nonlinear state space model
% dc
% dt =
% F0(c0 − c)
% πr 2h
% − k0 exp 
% −
% E
% RT 
% c

% dT
% dt =
% F0(T0 − T )
% πr 2h
% +
% −∆H
% ρCp
% k0 exp 
% −
% E
% RT 
% c +
% 2U
% r ρCp
% (Tc − T )

% dh
% dt =
% F0 − F
% πr 2
\begin{figure}[!ht]
\centering%
%\subfigure[\footnotesize Measured output 2 (-- blue) and reference trajectory (- - red)]
{\epsfig{figure=React.jpg,width=0.4\textwidth}}%\hspace{.5cm}
%
\caption{Well-stirred reactor}\label{React}
\end{figure}
The controlled variables are h, the level of the tank, and c, the molar

concentration of species A. The additional state variable is T, the re-
actor temperature; while the manipulated variables are Tc, the coolant

liquid temperature, and F, the outlet flowrate. Moreover, it is assumed
that the inlet flowrate acts as an unmeasured disturbance. The model

parameters in nominal conditions are reported in Table 1.1. The open-
loop stable steady-state operating conditions are the following

On considère le réacteur chimique décrit par les équations suivantes 
\begin{equation}
\begin{cases}
  \dot{c}=\frac{F_0(c_0-c)}{\pi r^2h}-k_0\exp{\left(-\frac{E}{RT}\right)}c\\
  \dot T=\frac{F_0(T_0-T)}{\pi r^2h}-\frac{\Delta H}{\rho C_p}k_0\exp{\left(-\frac{E}{RT}\right)}c+\frac{2U}{r\rho C_p}(T_c-T)\\
  \dot h=\frac{F_0-F}{\pi r^2}
  \end{cases}
\end{equation}
où $c$ et $h$ correspondent respectivement à la concentration et à la hauteur de fluide et sont mesurés. $F$ et $T_c$ sont les deux commandes.\\
\\
Les valeurs numériques utilisées sont $F_0=0.1$, $T_0=350$, $c_0=1$, $r=0.219$, $k_0=7.2.10^{10}$, $\frac{E}{R}=8750$, $U=54.94$, $\rho=1000$, $C_p=0.239$ et $\Delta H=-5.10^4$.\\
\\
Ce système est plat et possède pour sorties plates $c$ et $h$.
