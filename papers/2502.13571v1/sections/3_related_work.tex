

\section{Related work}

\subsection{Influence Maximization}

The IM problem has been studied for decades~\cite{zareie2023influence_survey, lihui2023_influence_survey, TKDE18_li2018influence_survey}.
Traditional IM methods can be broadly divided into three categories: 
simulation-based~\cite{kempe2003im, leskovec2007CELF, wang2010-simu}, 
heuristic~\cite{chen2009DegreeDiscountIC, chen2010MIA, chen2010LDAG, goyal2011simpath}, 
and sampling-based~\cite{tang2015IMM, wang2016BKRIS, NIPS2024_Rui, WWW2024-Chen}.
Sampling-based methods, such as IMM~\cite{tang2015IMM}, achieve solutions close to theoretical guarantees while maintaining computational efficiency. However, most rely on fixed diffusion models, assuming the model is fully known and using it to select seed nodes. This limits their generalization ability, making them difficult to apply to real-world IM problems.

Recently, numerous studies have applied various learning techniques to solve problems, offering new insights into IM research~\cite{TKDD23_li2023survey}.
Data-driven methods can bypass propagation model dependency and directly extract complex propagation features from influence data~\cite{du2014learn_dif_model, WWW2024-Huang}.
Among these, graph learning and reinforcement learning (RL) are the most popular~\cite{hevapathige2024_DeepSN, tang2024graph, panagopoulos2024learning, feng2024influence, chowdhury2024deep}.
For instance, PIANO~\cite{li2022piano} and ToupleGDD~\cite{chen2023ToupleGDD} combine both to solve the IM problem. 
Yet, the learning processes of RL still depend on prior knowledge of diffusion models to define rewards and incur high computational costs.
In fact, graph learning techniques can independently be leveraged to design solutions.
Kumar et al.~\cite{kumar2022gnn} developed a GNN-based regression model that uses the estimated influence spread of each node as the label. 
Similarly, GCNM \cite{zhang2022GCNM}, based on GCN, learns node representations using node degree as labels and selects seed users via the Mahalanobis distance.
Very recently, Ling et al. \cite{ling2023icml} proposed an end-to-end model, DeepIM, which initially uses an autoencoder to infer seed users with high influence and then applies a GNN-based model to deduce the correlation between the seed collection and its corresponding influence spread.
Despite their effectiveness, most methods still suffer from severe scalability issues and fail to capture intrinsic social influence features, especially hierarchical structures, resulting in limited performance.
% Despite their effectiveness, these methods suffer from severe scalability issues and struggle to construct representations that capture the intrinsic features of social influence, especially hierarchical structures, resulting in limited performance.
% Despite their effectiveness, these methods either fail to overcome assumptions on diffusion models or struggle to construct representations that capture the intrinsic features of social influence, especially hierarchical structures, resulting in limited performance.

% Most studies operate under a predefined diffusion model or aim to learn one. In contrast, we rely solely on observed cascades to identify the most influential seed users.

\subsection{Hyperbolic Representation Learning}

% Hyperbolic Representation Learning
Recent studies have shown that hyperbolic spaces can effectively model data with inherent complex hierarchical structures~\cite{sarkar2011low, sala2018representation, chami2020trees}.
Learning methods based on hyperbolic geometry have been widely studied.
For instance, some studies \cite{ganea2018hdnn, gulcehre2018hdnn, liu2019hdnn, chami2019hdnn} extended classic deep neural networks into hyperbolic space, demonstrating outstanding performance in graph-related tasks such as node classification~\cite{WWW2023_fu2023hyperbolic} and link prediction~\cite{AAAI2019_wang2019hyperbolic}. 
Besides, other studies~\cite{ACL20_chami2020low, feng2022role} utilized rotation operations in hyperbolic spaces to depict complex relations and enhance the expressive power of representations for various scenarios.
Indeed, Methods leveraging hyperbolic representation learning have achieved great success in various domains, including recommender systems~\cite{yang2022hicf, wang2023hdnr}, information diffusion prediction~\cite{feng2022h-diffu, CIKM23_qiao}, knowledge graph completion~\cite{ACL20_chami2020low, wang2023mixed}, and others~\cite{xu2023decoupled, song2023hisum, WWW2024_GraphHAM}.
To the best of our knowledge, no prior research has investigated the potential of hyperbolic representation learning for the IM problem. 
We state that the intrinsic hierarchical structure in social data can be effectively leveraged to capture influence spread characteristics, which existing studies have overlooked.

% To this end, unlike existing studies, we utilize hyperbolic representations to depict complex social influence features.
% Based on that, we propose a novel diffusion model agnostic method to efficiently and flexibly estimate influence spread for seed node selection.
% Unlike existing studies, we leverage the properties of hyperbolic geometry to estimate users' social influence and propose a new adaptive selection strategy for selecting seed nodes based on distance-related scores.