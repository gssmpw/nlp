\section{Introduction}

As an important problem in social network analysis, the Influence Maximization (IM)~\cite{kempe2003im, TKDE18_li2018influence_survey, lihui2023_influence_survey, TKDD23_li2023survey} aims to identify a small set of social users as the seed users in a given social network to achieve maximum influence spread, i.e., the expected number of users would be influenced by the seed users. 
Addressing IM problems is of great significance for understanding complex influence propagation patterns of real-world social networks, which is beneficial to many applications, such as viral marketing \cite{nguyen2016SSA, AAAI2018_tang}, outbreak detection \cite{leskovec2007CELF, AAAI2024_neophytou}, and disinformation monitoring \cite{budak2011limiting, sharma2021network}.


% Background Traditional Method
% Classic 
The IM problem has been widely studied for decades. 
Traditional methods~\cite{kempe2003im, leskovec2007CELF, tang2015IMM, wang2016BKRIS} solve the IM problem under the assumption that diffusion parameters are known and fixed.
High computational complexity and strong propagation model assumptions significantly limit their real-world applicability.
Later, machine learning techniques have been applied to IM research, aiming to remove the assumption on diffusion parameters.
Notably, graph representation learning methods have gained significant attention for the IM problem due to their ability to simplify the processing of graph data while preserving its intricate features.
Building on this foundation, some methods~\cite{kumar2022gnn, zhang2022GCNM, li2022piano, chen2023ToupleGDD, ling2023icml, hevapathige2024_DeepSN} have been proposed and shown improved performance. 
% Despite their effectiveness, these methods still face several key limitations.
% Most methods fail to overcome diffusion model constraints, as they still rely on a diffusion model with known parameters~\cite{li2022piano, chen2023ToupleGDD}.
% Meanwhile, most graph learning-based methods~\cite{panagopoulos2020IMINFECTOR, kumar2022gnn, zhang2022GCNM, ling2023icml, hevapathige2024_DeepSN} fail to capture influence characteristics effectively and suffer from severe scalability issues.
% These constraints prevent these solutions from being applied to complex real-world scenarios.
However, these methods are typically based on Euclidean spaces, which fail to capture the complex patterns of social data, especially the hidden hierarchical structures in social influence distribution. Real-world social networks inherently possess hierarchical structures~\cite{barabasi1999emergence}.
Moreover, influence spread of users follows a power-law distribution, where only a small number of users hold significant influence~\cite{verbeek2014metric}.
%All these methods cannot effectively capture all these features.
Existing methods cannot effectively capture these inherent characteristics. Moreover, most graph-learning methods still suffer from severe scalability issues, which is crucial for real-world large-scale datasets.

% To address these limitations, we aim to develop a diffusion model agnostic method that captures influence spread patterns directly from social influence data without relying on known diffusion parameters.


% Problem 1
% Since the ultimate goal of the IM problem is to find the most influential seed users to maximize the influence spread, the first problem we consider is whether the IM can be solved without relying on any specific diffusion models. 
% Observing influence propagation in the real world provides new insights.
% Real-world influence propagation facts show that the spread of similar products or related content offers valuable insights for new promotions.
% This suggests that knowing the exact diffusion model may be unnecessary. 
% Instead, influence activations, i.e., the history of influence propagation among users, provide rich information for uncovering social influence spread patterns.
% Influence activations can be collected and utilized for learning tasks.
% We state that leveraging these spread patterns enables approaches to effectively identify influential users, a direction that has not been explored before.
% We can utilize the ability of proper machine learning methods to uncover complex patterns in these influence data to capture influence spread patterns and evaluate users' social influence.
% Therefore, instead of finding an exact diffusion model, we aim to infer propagation trends directly from influence data to guide the selection of highly influential users.


% % Problem 2
% Although knowing that spread patterns can measure users' social influence, we still lack a specific method for the IM problem. The second problem arises: How can we learn from social influence data to reflect users' influence ability?
% The inherent nature of the vanilla IM problem makes it hard to define effective labels for supervised learning.
% Several IM studies~\cite{panagopoulos2020IMINFECTOR, kumar2022gnn, zhang2022GCNM} define pseudo-labels to learn spread patterns, but their performance is limited and still constrained by the diffusion model.
% Therefore, we consider whether labels can be avoided.
% Instead, studies~\cite{KDD2016_grover_node2vec, KDD2016_wang_SDNE, KDD2017_ribeiro_struc2vec} have shown that network embedding can be performed in an unsupervised manner to learn meaningful node representations by capturing their structural roles and relational dependencies.
% However, directly embedding user nodes based on network structure into Euclidean space may not effectively capture user influence for IM.
% We observe that influence activations during the propagation process can be collected as edges for network embedding. 
% This approach enables direct learning from propagation history without relying on any assumptions of diffusion models.
% Even so, the limited expressive power of Euclidean embeddings fails to capture users' social influence, leading to limited performance for IM. 

% Motivation
In this work, we demonstrate that hyperbolic representations can effectively model social influence patterns, thereby providing novel and efficient solutions to the IM problem with unknown diffusion parameters.
We found that hyperbolic space is highly suitable for modeling user social influence due to its exceptional capabilities.  
Recent studies have shown that the hierarchical features in information graphs can be effectively captured in hyperbolic spaces~\cite{nickel2017poincare, nickel2018_hype_geo, ICML2023_Yang}.
Points near the origin in hyperbolic space are considered high-hierarchical, while those farther away have lower hierarchy levels but grow in number exponentially~\cite{TPAMI2021_peng_survey}.
The distribution of nodes in hyperbolic space aligns with the scale-free pattern of user influence distribution.
Inspired by this, we aim to leverage hyperbolic representations to extract influence information from social data, including social networks and propagation instances.
% 
This approach allows us to effectively estimate user influence strength from learned hyperbolic representations without relying on diffusion model parameters, making it more practical for real-world applications.
% We do not need exact influence values but should distinguish high- and low-influence users properly.
% Apart from removing assumptions on diffusion models
We identify two key benefits of learning hyperbolic user representations for IM: (1) Hyperbolic representations provide effective differentiation of users' influence patterns by preserving the hierarchical structure inherent in social data. (2) The influence spread of user nodes can be efficiently approximated by their positions within the hyperbolic embedding space.

% Challenge
Motivated by the above, we learn user representations in hyperbolic space from network structure and influence activations to estimate social influence spread patterns.
Instead of assuming any specific diffusion models, we capture potential influence spread patterns directly based on social data that includes social network and propagation history.
This provides a novel solution framework for high-influential user selection in the IM problem, which has not been previously explored.
Yet, this also is a non-trivial task due to several key challenges.
First, we need a proper learning strategy to capture user influence characteristics from social influence data, ensuring that the learned representations effectively capture the potential spread patterns of users.
Second, after encoding influence patterns into user representations, influence strength should be properly quantified and assessed, facilitating effective seed user selection.
Moreover, the proposed solution should not only produce high-quality seed sets with significant influence spread but also be efficient enough to be applied to large-scale social networks.

% Method
To address the above challenges, we develop a novel \underline{\textbf{H}}yperbolic \underline{\textbf{I}}nfluence \underline{\textbf{M}}aximization (HIM) method, where we fully utilize the geometric property of hyperbolic representations to estimate the social influence and then select the influential user set.
HIM mainly consists of two parts: (1) \emph{Hyperbolic Influence Representation} extracts the hierarchical structure information of influence distribution in the social network and influence propagation instances to construct hyperbolic user representations.
(2) \emph{Adaptive Seed Selection} incrementally selects seed users based on the squared Lorentzian distances between their representations and the space origin. It dynamically updates user scores to explore more high influential candidates, resulting in higher influence spread.
%Extensive experiments on five real-world social networks of varying scales demonstrate the effectiveness and efficiency of HIM.

Our main contributions can be summarized as follows:
\begin{itemize}
    % \item We explore how to infer influence spread from partially observed propagation history in hyperbolic space to identify influential seed nodes and maximize influence spread without knowing the exact diffusion model parameters. To our knowledge, this is the first work to study hyperbolic representation learning for the IM problem.
     \item We explore the influence maximization task from a novel perspective. Instead of developing sophisticated models, we investigate how to leverage the geometry of hyperbolic space to select influential users based on partially observed propagation history, without knowing the exact diffusion model parameters. To our knowledge, this is the first work to study hyperbolic representation learning for IM.
    \item We propose a novel diffusion model agnostic method HIM for the IM problem. HIM first extracts influence information from social networks and propagation instances into hyperbolic representations.
    Then, it flexibly identifies the seed set based on the learned representations via an adaptive selection algorithm.
    \item Extensive experiments show that HIM significantly outperforms baseline methods in influence spread across five real-world datasets. Compared to state-of-the-art learning-based approaches, our method not only achieves greater influence spread but also demonstrates superior scalability. 
    % The source code and datasets are anonymously available at~\url{https://anonymous.4open.science/r/anonymous-238/}.
\end{itemize}