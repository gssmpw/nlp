

\section{Preliminary}
\subsection{Problem Definition and Assumptions}
\label{sec:assume}
A social network can be defined as a graph $G=(V, E)$, where $V$ is the set of users (nodes) and $E$ is the set of relationships or connections (edges) between them.
Given a seed set $S \subseteq V$, influence propagation is the process where influence spreads from 
$S$ to other users in the social network, with the final spread outcome governed by a diffusion model $\mathcal{M}$.
In IM, we focus on the total number of people influenced at the end of the propagation process.
Specifically, the IM problem aims to maximize the number of influenced nodes in $G$ by selecting an optimal seed node set $S^{*} \subseteq V$. 
Let $\sigma(\cdot)$ be the spread function representing the total number of users ultimately influenced by a given seed set. Given a social network $G$, an diffusion model $\mathcal{M}$ with unknown diffusion parameters and a positive integer $k$, the IM problem can be formulated as:
\begin{equation}
\begin{aligned}
    \text{argmax}_{S^{*} \subseteq V} \sigma(S^{*}), \: \text{ s.t. } |S^{*}| = k.
    \label{eq:im_def}
\end{aligned}
\end{equation}

\textbf{
Assumption 1: Diffusion parameters of the diffusion model are unknown, and only propagation history is available.
} 

% It is worth noting that there are two key points assuming the diffusion parameters are unknown.
It is worth noting that real-world diffusion parameters are often unknown.
First, the actual propagation mechanism is highly complex and cannot be accurately characterized. 
Second, previous studies commonly assume that diffusion parameters can be fully accessed ~\cite{tang2015IMM, guo2020-RIS, li2022piano, chen2023ToupleGDD}, which limits their applicability to real-world scenarios.
In contrast, our work assumes that although the diffusion model exists and remains unknown parameters, the influence spread of given seed users can be directly estimated from historical influence propagation processes, i.e., the partially observed propagation instances.

An \textbf{influence propagation instance} records the entire process of influence spread from a given seed set in the network, capturing all influence activations.
An \textbf{influence activation} between two users can be represented as a directed edge $(u \rightarrow v)$, indicating that user $u$ successfully influences user $v$ in the process.
Accordingly, an influence propagation instance can be defined as a directed graph $G_D^i = (V_D^i, E_D^i)$, where $V_D^i$ represents the set of users and $E_D^i$ represents all influence activations. Note that $V_D^i \subseteq V$.
The $M$ propagation instances can be represented as a graph set $\mathcal{G}_D = \{ G^1_D, G^2_D, \dots, G^M_D \}$. Besides, we name a $G_D^i$ as a propagation graph.

We assume a diffusion model has unknown diffusion parameters, and only $M$ propagation instances are available. 
We aim to infer user influence from these instances to solve the IM problem in a given social network, alleviating the constraint of diffusion models. 
This is practical, as real-world product promotion often depends on the propagation history of similar products.


\textbf{Assumption 2: The influence spread characteristics of influential users can be effectively captured through hyperbolic representations.}

Hyperbolic space itself is an embedding space with inherent hierarchical properties~\cite{TPAMI2021_peng_survey}. 
Nodes closer to the origin are considered to have higher hierarchy levels, while those farther away exhibit lower hierarchy levels. 
We found that this structural characteristic aligns well with the distribution of social influence in the real-world propagation process on social networks, where most users have low influence, and only a small fraction are highly influential.
If users can be embedded into appropriate positions in hyperbolic space based on their influence levels, the properties of the mapping representations can be leveraged to efficiently select seed users.

Therefore, in this work, we leverage hyperbolic representation learning to capture influence characteristics in both the social network and propagation instances. 
This enables us to effectively estimate influence spread without relying on an explicit diffusion model.
We introduce the technical details in Section~\ref{sec:method} and demonstrate its effectiveness through extensive experiments in Section~\ref{sec:experiments}. Before that, we provide a brief introduction to hyperbolic geometry.
\subsection{Hyperbolic Geometry}
We chose the Lorentz model as the embedding space due to its computational efficacy and stability~\cite{nickel2018_hype_geo}.
An $n$-dimensional Lorentz model $\mathbb{L}^{n}_{\gamma}$ is defined as the Riemannian manifold 
with constant negative curvature $-1/\gamma$, where $\gamma>0$ is the curvature parameter.
When $\gamma = 1$, the Lorentz model can be viewed as a unit hyperboloid model.
The points in the Lorentz model satisfy $\mathbb{L}^n_{\gamma} = \{\mathbf{x} \in \mathbb{R}^{n+1}: \langle  \mathbf{x}, \mathbf{x} \rangle_{\mathcal{L}} = -\gamma \}$, 
where $\mathbf{x} = (x_0, x_1, \cdots, x_n ) \in \mathbb{R}^{n+1}$ with $x_0 = \sqrt{\gamma+ \sum^{n}_{i=1}x_i^2} > 0$, and the origin of the space is $\mathbf{o}_{\mathcal{L}} = (\sqrt{\gamma}, 0, \cdots, 0)$. 
$\langle  \cdot, \cdot \rangle_{\mathcal{L}}$ represents the Lorentzian scalar product which is defined as:
$\langle  \mathbf{x}, \mathbf{y} \rangle_{\mathcal{L}} = -x_0 \cdot y_0 + \sum^{n}_{i=1}x_i \cdot y_i$.
For any two points $\mathbf{x}, \mathbf{y} \in \mathbb{L}^n_{\gamma}$, there is $\langle  \mathbf{x}, \mathbf{y} \rangle_{\mathcal{L}} \le -\gamma$. The squared Lorentzian distance is defined as:
$ d^2_{\mathcal{L}}(\mathbf{x}, \mathbf{y}) =-2 \gamma-2\left\langle\mathbf{x}, \mathbf{y} \right\rangle_{\mathcal{L}}$.
Besides, rotations are hyperbolic isometries (distance-preserving) and can be applied to hyperbolic vectors~\cite{ACL20_chami2020low, feng2022role}. A block-diagonal matrix can describe the rotation operation as: $\mathbf{Rot}_{\Theta}= \text{diag} \left(\mathbf{R}\left(\theta_{r, 1}\right), \mathbf{R}\left(\theta_{r, 2}\right), \cdots, \mathbf{R}\left(\theta_{r, n / 2}\right) \right)$, where $\Theta = \{ \theta_{r,1}, \cdots, \theta_{r, n/2} \}$ are the rotation parameters. The $2 \times 2$ block $\mathbf{R}(\theta_{r, i})$ is defined as:
\begin{equation}
\footnotesize
\begin{array}{cc} \mathbf{R}(\theta_{r,i}) = \begin{bmatrix}
    \cos(\theta_{r,i}) & -\sin(\theta_{r,i}) \\
    \sin(\theta_{r,i}) & \cos(\theta_{r,i}) \end{bmatrix}
\end{array}.
\end{equation}
In this work, all rotation matrix parameters are learnable.
We use $\mathbf{Rot}(\mathbf{x})$ to denote the rotation operation applied to a given vector $\mathbf{x}$.


% Specifically, a propagation graph $G^i_D$ depicts the propagation process for a given seed set $S_i$.
% If $u, v \in V$ and user $u$ performed an influence action to user $v$, then there is a directed edge $(u \rightarrow v)$ in the propagation graph $G_D^i$.
% %
% In general, we can employ the influence spread $\delta(S_i)$ as the metric to assess the quality of a given seed set $S_i$, which refers to the total number of users ultimately influenced by the seed set.
% Afterward, we define the classic influence maximization problem and suggest a new perspective to address it. 
% \begin{definition}[\textbf{Influence Maximization (IM) Problem}]
%     Given a social network $G=(V,E)$, a diffusion model $M$ and a positive integer $k$, the purpose is to select a set $S^{*}$ of $k$ users from $V$ as the seed users to maximize the influence spread $\delta(\cdot)$.
% \end{definition}

% Note that the classic IM problem is NP-hard under general diffusion models and is extremely difficult to obtain the optimal solutions~\cite{TKDE18_li2018influence_survey,TKDD23_li2023survey}. Various traditional algorithms and learning algorithms have been proposed, aiming at achieving high influence spread and being efficient at the same time.    
% Different from all previous studies, we address the IM problem via hyperbolic spread representation learning.
% Before we introduce the details of our method, we first provide a brief introduction to hyperbolic space.

