\section{Experiments}
\label{sec:experiments}
\begin{table}[!tb]
\centering
\caption{Statistics of datasets.}
% \vspace{-1em}
\resizebox{\columnwidth}{!} {
\begin{tabular}{ccccccc}
\toprule
    & Cora-ML 
    & Power Grid 
    & Facebook 
    & GitHub 
    & YouTube
    % & Flixster 
    \\
\midrule
\#Nodes  
& 2,810   
& 4,941    
& 5,908    
& 37,700    
& 1,134,890  
% & 2,523,386 
\\
\#Edges  
& 7,981    
& 6,594    
& 41,729   
& 289,003   
& 2,987,624  
% & 7,918,801 
\\
$d_{\text{avg}}$ 
& 5.7     
& 2.7      
& 14.1     
& 15.3      
& 5.3        
% & 6.3 
\\
\bottomrule
\end{tabular} }
% \vspace{-1em}
\label{table:datasets}
\end{table}

\begin{table*}[!ht]
\begin{center}
% \vspace{-1em}
\centering
\caption{The overall performance on four regular networks under the IC model. We report average spread ratios (\%) with standard deviations. The best scores are in bold, and the second-best scores are underlined.}
% \vspace{-1em}
\resizebox{\textwidth}{!} {
\begin{tabular}{c ccc ccc ccc ccc } 
\toprule
\multirow{2}{*}{Method} 
        & \multicolumn{3}{c}{Cora-ML} 
        & \multicolumn{3}{c}{Power Grid} 
        & \multicolumn{3}{c}{Facebook}
        & \multicolumn{3}{c}{GitHub} \\ 
        \cmidrule{2-13}

    & 1\% & 5\% & 10\% 
    & 1\% & 5\% & 10\%
    & 1\% & 5\% & 10\%
    & 1\% & 5\% & 10\%  \\
\midrule  

IMM$_{p}$
    & \textbf{27.71}$_{\pm1.79}$ & \textbf{34.32}$_{\pm0.68}$ & \textbf{39.02}$_{\pm0.82}$ 
    & 03.24$_{\pm0.36}$ & \textbf{11.45}$_{\pm0.49}$ & \underline{18.68}$_{\pm0.65}$ 
    & 20.05$_{\pm2.51}$ & \textbf{28.46}$_{\pm0.83}$ & \textbf{34.85}$_{\pm0.53}$ 
    & \textbf{28.12}$_{\pm0.21}$ & \textbf{31.63}$_{\pm0.18}$ & \textbf{36.38}$_{\pm0.21}$\\

\midrule  

IMM
    & 24.34$_{\pm2.51}$ & 31.64$_{\pm1.38}$ & 36.65$_{\pm1.23}$
    & 02.33$_{\pm0.21}$ & 09.25$_{\pm0.37}$ & 16.86$_{\pm0.31}$ 
    & 16.99$_{\pm4.51}$ & 23.47$_{\pm1.43}$ & 28.72$_{\pm1.76}$ 
    & 27.12$_{\pm0.28}$ & 30.66$_{\pm0.23}$ & 33.22$_{\pm0.16}$ \\

SubSIM
    & 21.89$_{\pm4.51}$ & 31.61$_{\pm2.19}$ & 36.06$_{\pm3.10}$ 
    & 02.85$_{\pm0.31}$ & 09.65$_{\pm0.38}$ & 17.17$_{\pm0.32}$ 
    & 16.67$_{\pm3.06}$ & 23.40$_{\pm1.81}$ & 28.95$_{\pm1.83}$ 
    & 26.68$_{\pm0.13}$ & 30.58$_{\pm025}$ & 33.21$_{\pm0.20}$ \\

\midrule

PIANO
    & 25.21$_{\pm1.24}$ & 30.14$_{\pm1.19}$ & 35.78$_{\pm0.74}$ 
    & 02.84$_{\pm0.33}$ & 10.23$_{\pm0.39}$ & 17.43$_{\pm0.21}$ 
    & 18.77$_{\pm0.62}$ & 24.78$_{\pm1.01}$ & 29.07$_{\pm0.37}$ 
    & 26.13$_{\pm0.24}$ & 28.51$_{\pm0.25}$ & 35.01$_{\pm0.20}$ \\
    
ToupleGDD
    & 26.74$_{\pm1.39}$ & 30.02$_{\pm2.72}$ & 36.32$_{\pm0.83}$ 
    & 02.90$_{\pm0.31}$ & 10.74$_{\pm0.44}$ & 17.74$_{\pm0.42}$ 
    & 19.32$_{\pm0.90}$ & 25.64$_{\pm0.65}$ & 30.26$_{\pm0.47}$ 
    & 26.62$_{\pm0.29}$ & 29.27$_{\pm0.24}$ & 34.82$_{\pm0.26}$ \\

\midrule

IMINFECTOR
    & 25.80$_{\pm1.37}$ & 31.66$_{\pm1.07}$ & 37.42$_{\pm0.89}$ 
    & 01.65$_{\pm0.12}$ & 07.96$_{\pm0.41}$ & 15.73$_{\pm0.46}$ 
    & 18.43$_{\pm2.07}$ & 23.76$_{\pm0.77}$ & 27.94$_{\pm0.73}$ 
    & 26.43$_{\pm0.12}$ & 30.76$_{\pm0.28}$ & 34.88$_{\pm0.27}$ \\

SGNN
    & 25.95$_{\pm1.58}$ & 31.17$_{\pm1.20}$ & 36.57$_{\pm1.31}$ 
    & 01.88$_{\pm0.15}$ & 10.39$_{\pm0.34}$ & 17.89$_{\pm0.41}$ 
    & 18.23$_{\pm1.44}$ & 22.62$_{\pm1.34}$ & 27.63$_{\pm1.06}$ 
    & 26.62$_{\pm0.23}$ & 29.57$_{\pm0.16}$ & 35.01$_{\pm0.15}$ \\

GCNM
    & 25.50$_{\pm1.73}$ & 30.81$_{\pm1.35}$ & 34.16$_{\pm1.04}$ 
    & 01.94$_{\pm0.28}$ & 08.43$_{\pm0.35}$ & 15.09$_{\pm0.43}$ 
    & 18.57$_{\pm0.93}$ & 22.84$_{\pm0.75}$ & 28.43$_{\pm0.85}$ 
    & 25.72$_{\pm0.17}$ & 27.57$_{\pm0.32}$ & 34.38$_{\pm0.19}$ \\
    
DeepIM
    & 26.00$_{\pm1.28}$ & 30.90$_{\pm1.26}$ & 36.65$_{\pm0.89}$ 
    & 03.54$_{\pm0.25}$ & 11.19$_{\pm0.41}$ & 18.32$_{\pm0.67}$ 
    & 19.67$_{\pm0.90}$ & 24.77$_{\pm0.95}$ & 28.63$_{\pm0.86}$ 
    & 26.75$_{\pm0.24}$ & 27.47$_{\pm0.23}$ & 29.51$_{\pm0.19}$ \\

DeepSN
    & 23.68$_{\pm4.33}$ & 32.19$_{\pm0.90}$ & 37.54$_{\pm1.24}$ 
    & 02.20$_{\pm0.21}$ & 08.67$_{\pm0.40}$ & 16.51$_{\pm0.77}$ 
    & 19.63$_{\pm1.42}$ & 25.78$_{\pm0.84}$ & 32.42$_{\pm0.76}$ 
    & 27.37$_{\pm0.24}$ & 30.81$_{\pm0.23}$ & 34.81$_{\pm0.21}$ \\
    
\midrule
HIM$_{MD}$
    & 27.22$_{\pm1.21}$ & 32.98$_{\pm1.04}$ & 38.19$_{\pm0.76}$ 
    & \underline{03.37}$_{\pm0.18}$ & 11.28$_{\pm0.42}$ & 18.66$_{\pm0.39}$ 
    & \underline{20.45}$_{\pm0.90}$ & 26.02$_{\pm0.56}$ & 30.16$_{\pm0.61}$ 
    & 27.69$_{\pm0.18}$ & 31.41$_{\pm0.22}$ & 36.12$_{\pm0.15}$ \\  

HIM
    & \underline{27.65}$_{\pm1.27}$ & \underline{33.41}$_{\pm0.94}$ & \underline{38.87}$_{\pm0.96}$
    & \textbf{03.46}$_{\pm0.27}$ & \textbf{11.45}$_{\pm0.46}$ & \textbf{19.13}$_{\pm0.50}$ 
    & \textbf{21.33}$_{\pm1.08}$ & \underline{26.99}$_{\pm0.64}$ & \underline{32.69}$_{\pm0.59}$ 
    & \underline{27.74}$_{\pm0.21}$ & \underline{31.60}$_{\pm0.31}$ & \underline{36.27}$_{\pm0.22}$ \\

\bottomrule
\end{tabular}}
% \vspace{-1em}
\label{table:result-ic}
\end{center}
\end{table*}
\begin{table*}[!ht]
\begin{center}
\centering
\caption{The overall performance on four regular networks under the WLT model. We report average spread ratios (\%) with standard deviations. The best scores are in bold, and the second-best scores are underlined.}
% \vspace{-1em}
\resizebox{\textwidth}{!} {
\begin{tabular}{c ccc ccc ccc ccc } 
\toprule
\multirow{2}{*}{Method} 
        & \multicolumn{3}{c}{Cora-ML} 
        & \multicolumn{3}{c}{Power Grid} 
        & \multicolumn{3}{c}{Facebook}
        & \multicolumn{3}{c}{GitHub} \\ 
        \cmidrule{2-13}

    & 1\% & 5\% & 10\% 
    & 1\% & 5\% & 10\%
    & 1\% & 5\% & 10\%
    & 1\% & 5\% & 10\%  \\
\midrule  

IMM
    & 08.00$_{\pm0.12}$ & 21.65$_{\pm0.44}$ & 37.83$_{\pm0.53}$
    & \textbf{04.09}$_{\pm0.11}$ & 15.30$_{\pm0.19}$ & 28.58$_{\pm0.33}$ 
    & 04.21$_{\pm0.09}$ & 15.05$_{\pm0.19}$ & 28.78$_{\pm0.28}$ 
    & 17.89$_{\pm0.09}$ & 36.84$_{\pm0.17}$ & 79.34$_{\pm1.64}$\\

SubSIM
    & 07.92$_{\pm0.12}$ & 20.78$_{\pm0.39}$ & 37.09$_{\pm0.34}$ 
    & \underline{03.99}$_{\pm0.11}$ & 15.08$_{\pm0.15}$ & 28.71$_{\pm0.39}$ 
    & 04.28$_{\pm0.13}$ & 15.64$_{\pm0.20}$ & 28.41$_{\pm0.45}$ 
    & 17.45$_{\pm0.08}$ & 36.82$_{\pm0.14}$ & 78.70$_{\pm2.29}$\\

\midrule

PIANO
    & 02.43$_{\pm0.08}$ & 09.87$_{\pm0.32}$ & 17.36$_{\pm0.18}$ 
    & 02.76$_{\pm0.06}$ & 11.39$_{\pm0.23}$ & 20.58$_{\pm0.19}$ 
    & 03.85$_{\pm0.09}$ & 11.48$_{\pm0.18}$ & 22.48$_{\pm0.23}$ 
    & 16.93$_{\pm0.09}$ & 32.67$_{\pm0.11}$ & 41.82$_{\pm0.25}$\\
    
ToupleGDD
    & 02.39$_{\pm0.12}$ & 10.43$_{\pm0.38}$ & 18.65$_{\pm0.20}$ 
    & 02.83$_{\pm0.07}$ & 13.03$_{\pm0.10}$ & 25.05$_{\pm0.21}$ 
    & 03.89$_{\pm0.09}$ & 12.81$_{\pm0.15}$ & 24.19$_{\pm0.33}$ 
    & 17.02$_{\pm0.09}$ & 36.75$_{\pm0.25}$ & 61.69$_{\pm2.02}$\\

\midrule

IMINFECTOR
    & 01.77$_{\pm0.08}$ & 06.11$_{\pm0.12}$ & 12.58$_{\pm0.12}$
    & 01.69$_{\pm0.04}$ & 07.18$_{\pm0.11}$ & 15.87$_{\pm0.15}$ 
    & 01.32$_{\pm0.02}$ & 05.94$_{\pm0.06}$ & 12.49$_{\pm0.15}$ 
    & 01.21$_{\pm0.07}$ & 07.84$_{\pm0.03}$ & 15.79$_{\pm0.04}$\\

SGNN
    & 01.32$_{\pm0.05}$ & 07.14$_{\pm0.08}$ & 14.14$_{\pm0.23}$
    & 01.58$_{\pm0.05}$ & 07.66$_{\pm0.06}$ & 15.52$_{\pm0.14}$ 
    & 01.27$_{\pm0.04}$ & 06.48$_{\pm0.04}$ & 12.58$_{\pm0.06}$ 
    & 01.36$_{\pm0.04}$ & 08.16$_{\pm0.12}$ & 14.70$_{\pm0.03}$\\

GCNM
    & 01.10$_{\pm0.02}$ & 05.13$_{\pm0.03}$ & 10.90$_{\pm0.07}$
    & 01.29$_{\pm0.03}$ & 06.48$_{\pm0.08}$ & 12.43$_{\pm0.13}$ 
    & 01.12$_{\pm0.05}$ & 06.83$_{\pm0.73}$ & 12.10$_{\pm0.64}$ 
    & 01.27$_{\pm0.02}$ & 07.31$_{\pm0.06}$ & 13.50$_{\pm0.14}$\\
    
DeepIM
    & 07.66$_{\pm0.24}$ & 22.00$_{\pm0.41}$ & 38.80$_{\pm0.57}$
    & 03.78$_{\pm0.12}$ & 15.95$_{\pm0.27}$ & 28.98$_{\pm0.29}$ 
    & 02.63$_{\pm0.64}$ & 11.32$_{\pm0.56}$ & 25.49$_{\pm1.01}$ 
    & 16.92$_{\pm0.05}$ & 55.36$_{\pm0.81}$ & 76.11$_{\pm0.33}$\\

DeepSN
    & 01.42$_{\pm0.04}$ & 08.30$_{\pm0.10}$ & 14.69$_{\pm0.13}$
    & 01.84$_{\pm0.05}$ & 07.99$_{\pm0.12}$ & 16.22$_{\pm0.21}$ 
    & 01.55$_{\pm0.07}$ & 06.98$_{\pm0.22}$ & 15.95$_{\pm1.31}$ 
    & 01.45$_{\pm0.03}$ & 07.85$_{\pm0.02}$ & 16.72$_{\pm0.19}$\\
    
\midrule
HIM$_{MD}$
    & \underline{08.12}$_{\pm0.13}$ & \underline{23.82}$_{\pm0.61}$ & \underline{39.52}$_{\pm0.53}$ 
    & 03.62$_{\pm0.21}$ & \underline{16.26}$_{\pm0.20}$ & \underline{29.29}$_{\pm0.27}$ 
    & \underline{05.91}$_{\pm0.10}$ & \underline{18.37}$_{\pm0.28}$ & \underline{34.12}$_{\pm0.55}$ 
    & \underline{17.74}$_{\pm0.10}$ & \underline{57.02}$_{\pm0.59}$ & \underline{80.58}$_{\pm0.97}$\\  

HIM
    & \textbf{08.27}$_{\pm0.18}$ & \textbf{24.22}$_{\pm0.63}$ & \textbf{40.43}$_{\pm0.83}$
    & 03.89$_{\pm0.11}$ & \textbf{17.17}$_{\pm0.19}$ & \textbf{30.76}$_{\pm0.33}$ 
    & \textbf{06.14}$_{\pm0.10}$ & \textbf{18.79}$_{\pm0.30}$ & \textbf{34.51}$_{\pm0.63}$ 
    & \textbf{18.09}$_{\pm0.04}$ & \textbf{57.91}$_{\pm0.64}$ & \textbf{81.61}$_{\pm1.43}$\\ 

\bottomrule
\end{tabular}}
% \vspace{-1.5em}
\label{table:result-lt}
\end{center}
\end{table*}
\subsection{Experimental setups}
We conduct extensive experiments to verify whether our method can effectively and efficiently identify the most influential seed users under unknown diffusion model parameters.
We validate our methods on 
four normal-size networks and a large-scale network, including:
Cora-ML, and Power Grid~\cite{ling2023icml}, 
Facebook~\cite{AAAI2015_nr_data},
GitHub~\cite{rozemberczki2019multiscale},
and Youtube~\cite{yang2012defining}.
We evaluate our method and baselines under two diffusion models, IC and WLT, using spread ratio $\delta(S^{*})/|V|$ (in percentage \%) as the primary evaluation metric. 
WLT is a stochastic variant of LT, which will be introduced later.
For each dataset, similar to work~\cite{ling2023icml, hevapathige2024_DeepSN}, we set the target number of seed nodes $k$ based on the proportions (1\%, 5\%, 10\%) of the total number of nodes in each dataset.
The datasets statistics are presented in Table \ref{table:datasets}.
More detailed settings can be found in Appendix.


\subsubsection{Diffusion Models}
We take classic IC and LT models (a detailed description can be found in~\cite{kempe2003im, li2022piano}) as the underlying diffusion models for experiments.
We set the diffusion parameters as unknown to all methods.
Particularly, the traditional LT model assumes that different users exert the same influence weight on a given user, which may be naive and unrealistic in practice.
Thus, we assign random influence weights between users, resulting in a practical variant diffusion model called WLT.
In detail, we set:

\textbf{IC}: Given $G=(V,E)$, each node $u$ has a fixed probability of activating its neighbors $v$ with probability $p_{u,v}$ in range $[0, 0.2]$. If user $u$ activates user $v$, an influence activation is collected as $(u \rightarrow v)$. Each node has only one chance to activate other nodes.

\textbf{WLT}: The classic LT model assigns each node a threshold, representing the total influence needed from neighbors to activate. It assumes all neighbors have equal influence on a target node. To better reflect real-world complexity, we refer to work~\cite{kempe2003im} and introduce uncertainty by assigning different influence weights to each neighbor while ensuring their sum remains 1, obtaining the weighted LT model (WLT).
Influence activation is defined as the actual participation of all users activating a given user in the diffusion process.
For example, if users $u$ and $v$ simultaneously activate $c$, we obtain the influence activations $(u \rightarrow c)$ and $(v \rightarrow c)$. The threshold is uniformly sampled from the range $[0.5, 0.9]$.

For each dataset with a diffusion model, we sample a diffusion model instance to fix its diffusion parameters as the true propagation model, which remains unknown to all methods. All methods are evaluated on this same diffusion model.
Our goal is to investigate if influence spread can be maximized using only observed propagation instances.
Based on these two diffusion models, we simulate $M=30$ propagation instances for each experiment with a given seed ratio to evaluate whether our method and other learning-based methods can learn propagation spread trends in IM problems with unknown diffusion parameters.

Note that, for our method, various diffusion models only affect how we collect propagation instances.
Therefore, our model can be extended to any diffusion model if the influence activation between users can be properly defined.
For instance, point-to-point diffusion models like SIS  and SIR~\cite{1994_allen_SIS_SIR} can be seen as variants of the IC model. Due to space limitations, we omit experiments on these models.

\subsubsection{Baselines}
We compare HIM with various types of state-of-the-art baseline methods:
(1) \textbf{IMM} \cite{tang2015IMM}  
and \textbf{SubSIM} \cite{guo2020-RIS} are two remarkable traditional methods based on RR set and approximation methods.
We further set an IMM variant~\textbf{IMM$_{p}$} that can access full knowledge of the diffusion model as a reference\footnote{
Both IMM and SubSIM assume known propagation probabilities when sampling reverse reachable sets under the IC model. 
They were originally designed for settings where sampling and validation use the same diffusion model. 
Thus, since propagation probabilities are unknown in our work, we set IMM$_p$ as a baseline for further investigation.
For the LT model, both methods assume equal influence weights and use uniform sampling,
which indeed makes the parameters unknown in the WLT model. 
Therefore, we do not set IMM$_p$ as a baseline under the WLT model.
}.
(2) \textbf{PIANO} \cite{li2022piano} and \textbf{ToupleGDD} \cite{chen2023ToupleGDD} are learning-based methods based on the reinforcement learning associated with graph representation learning.
(3) \textbf{IMINFECTOR}~\cite{panagopoulos2020IMINFECTOR}, \textbf{SGNN}~\cite{kumar2022gnn}, \textbf{GCNM}~\cite{zhang2022GCNM}, \textbf{DeepIM}~\cite{ling2023icml},
and \textbf{DeepSN}~\cite{hevapathige2024_DeepSN} are learning-based methods that are mainly based on graph representation learning. 
In addition, we implement \textbf{HIM$_{MD}$} as a variant of HIM that removes adaptive seed selection and instead selects the top $k$ users with the lowest $LDO$s as the seed set.
Note that HIM$_{MD}$ aligns with our representation learning objective, where highly influential users are embedded closer to the origin.

\subsection{Performance on Normal-size Networks}
\subsubsection{Results under IC Model}
The overall experimental results on four normal-size networks under the IC model are presented in Table~\ref{table:result-ic}.
Obviously, IM problems under unknown diffusion model parameters are more challenging. 
When the diffusion model is fully known, IMM$_p$ utilizes prior knowledge of propagation probabilities to sample reverse reachable sets, achieving effective influence spread in most cases.
However, the situation changes when uncertainty is introduced.
Without knowing propagation probabilities, both IMM and SubSIM fail to sample reverse reachable sets that accurately reflect influence spread, leading to a significant performance drop.
PIANO and ToupleGDD also require a fully known diffusion model to design rewards during reinforcement learning, leading to limited performance when the diffusion model parameters are unknown.
Performance varies significantly among graph learning-based baselines. 
IMINFECTOR and SGNN use diffusion cascades of individuals as pseudo-labels, while GCNM designs them based on degree information. Their limited performance shows that these methods fail to effectively capture complex influence spread information.
DeepIM and DeepSN are black-box models that learn from seed-user and final-activation pairs without considering the diffusion process. 
Both methods struggle to achieve consistently effective performance as they fail to capture propagation patterns.
Instead, our methods consistently outperform all baselines when the diffusion parameters are unknown.
Compared to graph representation learning-based methods, our methods achieve significant performance improvements.
This suggests that our methods that learn representations in hyperbolic space can more effectively capture users' social influence information and approximate their social influence strength.
Compared with HIM$_{MD}$, the performance of HIM is further enhanced. 
This indicates that the proposed adaptive seed selection can mitigate the issue of spread overlapping by exploring more potential users.
Overall, the experimental results demonstrate that our methods can effectively identify high-influential users in complex scenarios in one-to-one propagation mode.

\subsubsection{Results under WLT Model}
The results on four normal-size networks under the WLT model are shown in Table~\ref{table:result-lt}.
Unlike the IC model, the WLT model represents a more complex scenario as it follows a many-to-one propagation mode.
Moreover, identifying target seed users becomes more challenging due to the uncertainty in user influence weights.
Most baselines, including IMINFECTOR, SGNN, GCNM, and DeepSN, are affected by the propagation mode and even fail to work under the WLT model.
These methods cannot generalize well to complex diffusion scenarios.
IMM and SubSIM have limited performance as they cannot perceive different users' influence weights when sampling reverse reachable sets.
For the same reason, PIANO and ToupleGDD fail to design effective rewards, resulting in even worse performance.
DeepIM directly learns from seed and influence spread pairs, showing competitive performance in some cases.
In comparison, our methods consistently outperform all baselines in nearly all cases.
Results show that \HIMMD~outperforms all other baselines in most cases.
This indicates that under the many-to-one diffusion mode, our methods can effectively encode spread patterns into hyperbolic user representations and place highly influential users near the space origin.
Moreover, HIM refines seed selection using distance information between users, achieving a larger influence spread than \HIMMD. 
Overall, our methods demonstrate superior performance across four datasets under the WLT model, validating their effectiveness in complex diffusion scenarios.
\subsection{Performance on Large-scale Network}
\begin{figure}[!ht]
% \vspace{-1.5em}
\centering
    \subfigure{\includegraphics[width=0.44\columnwidth]{figures/large_scale/Youtube_IC_all.pdf}}
    % \hspace{1em}
    \subfigure{\includegraphics[width=0.44\columnwidth]{figures/large_scale/Youtube_PLT_all.pdf}}
    % \vspace{-1em}
    \caption{Results on Youtube under IC (left) and WLT (right).}
    \label{fig:large}
    % \vspace{-1em}
\end{figure}
We investigate the scalability of \HIMMD~and HIM on a large-scale network YouTube that has millions of nodes.
The results are shown in Figure~\ref{fig:large}.
We compare our methods with IMM, SubSIM, and ToupleGDD, as other baselines failed to produce valid results in our experimental environment due to out-of-memory issues, showing limited scalability.
Even in large-scale networks, our method consistently outperforms IMM, SubSIM, and ToupleGDD in all cases, demonstrating its effectiveness and scalability.
IMM and SubSIM fail to produce valid results within 24 hours at seed ratio 10\% under the IC model.
As $k$ increases, the unknown diffusion parameters affect reverse reachable set sampling, leading to high computational costs for IMM and SubSIM.
Both methods require over 8 and 18 hours, respectively, to produce results at 1\% and 5\% seed ratios under the IC model.
ToupleGDD requires a minimum selection time of more than 2 hours at a 1\% seed ratio.
Instead, the maximum selection time for \HIMMD~and HIM is 2 seconds and 20 minutes, respectively, across all cases.
Besides, it is worth noting that the training time for our method is around 20 minutes, significantly shorter than that of ToupleGDD, which exceeds 15 hours.
The results demonstrate that when the diffusion model is not fully known, hyperbolic network embedding based on social network structure and observed propagation instances can effectively and efficiently identify high-influential seed users in large-scale networks. 
This further exhibits the potential of our methods for real-world applications.

\subsection{Ablation Study}
We validate the effectiveness of our method design through a comprehensive ablation study.
The ablation results on four datasets under two diffusion models with seed ratio $10\%$ are shown in Table~\ref{table:ablation}.
(1) \textbf{w/o Hyp} represents the implementation of the entire HIM model in Euclidean space. 
The significant decrease in influence ratio across all cases indicates that, compared to Euclidean space, hyperbolic space more effectively captures influence spread patterns.
This also validates the effectiveness of hyperbolic network embedding for IM, leveraging the hierarchical structure of hyperbolic space to distinguish highly influential users.
(2) \textbf{w/o Rot} removes rotation operations from Eq. (\ref{eq:relation-score}) and Eq. (\ref{eq:propagation_score}).
Instead, we directly calculate the squared Lorentzian distance between $\mathbf{x}_u$ and $\mathbf{x}_v$.
The performance drop indicates that rotation operations assist in capturing influence information in user representations when learning from social influence data.
(3) \textbf{w/o NSL} removes the network structure learning module, and (4) \textbf{w/o IPL} removes the influence propagation learning module.
The results (3) and (4) show that both network structure information and historical spread patterns in propagation instances are crucial for accurately quantifying influence spread. In comparison, propagation instances play a more significant role in measuring potential influence trends.
% \begin{table}[!tb]
% \footnotesize
% \centering
%     \caption{The results of ablation study in average spread ratio (\%) in 100 simulations on four datasets with seed ratio 10\%.}
%     \vspace{-1em}
%     \resizebox{\columnwidth}{!} {
%     \begin{tabular}{ccccc} 
%         \toprule
%         Setting & Cora-ML & Power Grid & Facebook & GitHub \\ 
%         \midrule
%         & \multicolumn{4}{c}{Ablation Results under IC Model} \\
%         \midrule
%         (1) w/o Hyp & 37.15$_{\pm0.39}$ & 18.28$_{\pm0.44}$ & 30.78$_{\pm0.33}$ & 32.33$_{\pm1.09}$ \\ 
%         (2) w/o Rot & 38.52$_{\pm0.67}$ & 18.52$_{\pm0.32}$ & 32.12$_{\pm0.52}$ & 35.17$_{\pm0.34}$ \\
%         (3) w/o NSL & 38.02$_{\pm0.58}$ & 18.19$_{\pm0.43}$ & 31.51$_{\pm0.34}$ & 35.23$_{\pm0.74}$ \\
%         (4) w/o IPL & 35.20$_{\pm0.83}$ & 17.43$_{\pm0.57}$ & 27.89$_{\pm0.58}$ & 28.59$_{\pm0.44}$ \\
%         \midrule
%         HIM & \textbf{38.87}$_{\pm0.96}$ & \textbf{19.13}$_{\pm0.50}$ & \textbf{32.69}$_{\pm0.59}$ & \textbf{36.27}$_{\pm0.22}$ \\
%         \midrule
%         & \multicolumn{4}{c}{Ablation Results under PLT Model} \\
%         \midrule
%         (1) w/o Hyp & 38.15$_{\pm0.75}$ & 28.91$_{\pm0.42}$ & 32.02$_{\pm0.43}$ & 76.35$_{\pm0.55}$ \\ 
%         (2) w/o Rot & 40.19$_{\pm0.91}$ & 30.51$_{\pm0.47}$ & 33.85$_{\pm0.80}$ & 80.34$_{\pm0.69}$ \\
%         (3) w/o NSL & 39.27$_{\pm0.61}$ & 30.35$_{\pm0.28}$ & 33.02$_{\pm0.50}$ & 79.82$_{\pm1.54}$ \\
%         (4) w/o IPL & 37.23$_{\pm0.56}$ & 27.77$_{\pm0.24}$ & 31.88$_{\pm0.47}$ & 73.48$_{\pm2.17}$ \\
%         \midrule
%         HIM & \textbf{40.43}$_{\pm0.83}$ & \textbf{30.76}$_{\pm0.33}$ & \textbf{34.51}$_{\pm0.63}$ & \textbf{81.61}$_{\pm1.43}$ \\ 
%         \bottomrule
%     \end{tabular} 
%     }
% \label{table:ablation} 
% \end{table}

% Please add the following required packages to your document preamble:
% \usepackage{multirow}

\begin{table}[!tb]
\caption{The results of ablation study in average spread ratio (\%) on four datasets with seed ratio 10\%.}
% \vspace{-1em}
\resizebox{\columnwidth}{!} {
\begin{tabular}{l|lcccc}
\toprule
\multicolumn{1}{c}{~} & \multicolumn{1}{c}{Setting}         
  & Cora-ML                                    
  & Power Grid           
  & Facebook           
  & GitHub     
\\ 
\midrule
\multirow{5}{*}{IC}   
& (1) w/o Hyp & 37.15$_{\pm0.39}$         
& 18.28$_{\pm0.44}$              
& 30.78$_{\pm0.33}$                 
& 32.33$_{\pm1.09}$              
\\
       
& (2) w/o Rot & 38.52$_{\pm0.67}$       
& 18.52$_{\pm0.32}$                  
& 32.12$_{\pm0.52}$                
& 35.17$_{\pm0.34}$              
\\
        
& (3) w/o NSL & 38.02$_{\pm0.58}$       
& 18.19$_{\pm0.43}$                
& 31.51$_{\pm0.34}$               
& 35.23$_{\pm0.74}$             
\\
       
& (4) w/o IPL & 35.20$_{\pm0.83}$         
& 17.43$_{\pm0.57}$                     
& 27.89$_{\pm0.58}$                   
& 28.59$_{\pm0.44}$                    
\\ \cline{2-6} 
           
& \multicolumn{1}{c}{HIM}      
& \textbf{38.87}$_{\pm0.96}$ 
& \textbf{19.13}$_{\pm0.50}$ 
& \textbf{32.69}$_{\pm0.59}$ 
& \textbf{36.27}$_{\pm0.22}$ 
\\ 
\midrule
\multirow{5}{*}{WLT}                    
& (1) w/o Hyp & 38.15$_{\pm0.75}$           
& 28.91$_{\pm0.42}$                       
& 32.02$_{\pm0.43}$                     
& 76.35$_{\pm0.55}$                
\\
& (2) w/o Rot & 40.19$_{\pm0.91}$        
& 30.51$_{\pm0.47}$                      
& 33.85$_{\pm0.80}$                      
& 80.34$_{\pm0.69}$                   
\\
& (3) w/o NSL & 39.27$_{\pm0.61}$  
& 30.35$_{\pm0.28}$                 
& 33.02$_{\pm0.50}$                    
& 79.82$_{\pm1.54}$                    
\\
& (4) w/o IPL & 37.23$_{\pm0.56}$   
& 27.77$_{\pm0.24}$                  
& 31.88$_{\pm0.47}$                 
& 73.48$_{\pm2.17}$               
\\ \cline{2-6} 
& \multicolumn{1}{c}{HIM}      
& \textbf{40.43}$_{\pm0.83}$
& \textbf{30.76}$_{\pm0.33}$ 
& \textbf{34.51}$_{\pm0.63}$ 
& \textbf{81.61}$_{\pm1.43}$ 
\\ 
\bottomrule
\end{tabular} }
\label{table:ablation} 
% \vspace{-2em}
\end{table}


\subsection{Visualization}
\begin{figure}[!ht]
% \vspace{-1em}
\centering
    \subfigure{\includegraphics[width=0.44\columnwidth]{figures/hyper_inf.pdf}}
    \subfigure{\includegraphics[width=0.44\columnwidth]{figures/euc_inf.pdf}}
    % \vspace{-1.5em}
    \caption{Distance-influence relation on Power Grid in hyperbolic space (left) and Euclidean space (right).}
    \label{fig:dis_inf}
    \subfigure{\includegraphics[width=0.44\columnwidth]{figures/power_grid_LDO_degree.pdf}}
    % \vspace{-1.5em}
    \subfigure{\includegraphics[width=0.44\columnwidth]{figures/cora_ml_LDO_degree.pdf}}
    \caption{LDO-degree relations on Power Grid (left) and Cora-ML (right) datasets.}
    \label{fig:ldo_degree}
    % \vspace{-3em}
\end{figure}
We investigate the interpretability of our method through visualization on Power Grid and Cora-ML datasets under the IC model. Figure~\ref{fig:dis_inf} shows the relationship between the distances of each user's representations in different spaces, i.e., $LDO$ in hyperbolic space (left) and L2-norm in Euclidean space (right), and their influence strength. For clarity, 100 points were randomly sampled for plotting. The $LDO$ of hyperbolic representations demonstrates a clear inverse relationship with individual user influence, unlike the Euclidean representations. This indicates that hyperbolic representations effectively reflect the magnitude of user influence through $LDO$. 
Moreover, Figure~\ref{fig:ldo_degree} shows the relationship between the hyperbolic distance, i.e., $LDO$, and the degree of nodes on Cora-ML and Power Grid datasets, respectively. 
Intuitively, the degree serves as a rough measure of user influence, assuming that nodes with larger degrees usually tend to be more influential. 
Overall, both figures reveal an inverse relationship between degree and $LDO$, which indicates that the nodes with smaller $LDO$s are more likely to have more connections. In addition, the nodes with the same degree may have different $LDO$s, suggesting that hyperbolic representation can help differentiate nodes with similar degrees. 
Similar trends are also observed in other datasets, but we omit them here due to space limitations.
Overall, the results show that our method effectively captures user influence spread with hyperbolic representations.