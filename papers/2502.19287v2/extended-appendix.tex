\section{Proofs of Section~\ref{sec:permutation-fix-nominal-sets}}\label{app:permutation-fix-nominal-sets}

\generated*

  \begin{proof}
    \begin{enumerate}
        \item 
   
     If $\pi$ is a product of disjoint cycles, then write $\pi = \mu_1\circ\ldots\circ\mu_n$. Note that $\eta$ is one of these $\mu_{i}$'s, say $\eta = \mu_{i_0}$. Then $a\in \dom{\mu_{i_0}} - \supp{}{x}$. Suppose, by contradiction, that there is an $b\in \dom{\mu_{i_0}}\cap \supp{}{x}$. Then  $b\in\supp{}{x}$ implies $\mu_{n}(b)\in\supp{}{\mu_n\act x}$ and since $\mu_n(b) = b$ this leads to $b\in\supp{}{\mu_n\act x}$, which in turn yields $\mu_{n-1}(b)\in\supp{}{(\mu_{n-1}\mu_n)\act x}$, that is equivalent to $b\in\supp{}{(\mu_{n-1}\mu_n)\act x}$ because $\mu_{n-1}(b) = b$. Proceeding this way, eventually we will reach $b\in\supp{}{(\mu_{i_0+1}\ldots\mu_{n-1}\mu_n)\act x}$. Applying $\mu_i$ on both sides yields
     \(
          \mu_{i_0}(b)\in\supp{}{(\mu_{i_0}\mu_{i_0+1}\ldots\mu_{n-1}\mu_n)\act x}.
     \)
     
     Since $\mu_{i_0-1}$ and $\mu_{i_0}$ are disjoint, we have $\mu_{i_0-1}(\mu_{i_0}(b)) = \mu_{i_0}(b)$ and so we get
     \(
          \mu_{i_0}(b)\in\supp{}{(\mu_{i_0-1}\mu_{i_0}\mu_{i_0+1}\ldots\mu_n)\act x}.
     \)
     Therefore, applying the rest of the cycles, we obtain $\mu_{i_0}(b)\in\supp{}{\pi\act x}$ and, consequently, $\mu_{i_0}(b)\in\supp{}{x}$ because $\pi\act x = x$. Since there is a $l>0$ such that $\mu_{i_0}^l(b) = a$, by repeating this argument as many times as necessary, we obtain $a = \mu_{i_0}^l(b) \in \supp{}{x}$, reaching a contradiction. 

     \item First, note that $\eta_i\act x = x$ iff $\eta_i^{-1}\act x = x$. Since $\rho\in \pair{\eta_1,\ldots,\eta_n}$, there is a $r\in \mathbb{N}$ such that $\rho = \zeta_1\zeta_2\ldots\zeta_r$ where $\zeta_i\in \{\eta_1,\ldots,\eta_n\}$ or $\zeta_i\in \{\eta_1^{-1},\ldots,\eta_n^{-1}\}$. Thus, $\zeta_i\act x = x$ for all $i=1,\ldots,r$. Therefore,
     \[
        (\zeta_1\zeta_2\ldots\zeta_r)\act x = (\zeta_1\zeta_2\ldots\zeta_{r-1})\act(\zeta_r\act x) =  (\zeta_1\zeta_2\ldots\zeta_{r-1})\act x = \ldots =  \zeta_1\act x = x.
     \]
    \end{enumerate}
 \end{proof}

\pitts*

\begin{proof}
    \begin{enumerate}
        \item Suppose $\newc{c}{}.\pi\act x = x$. By definition, $D_1=\{\catnew{c} \mid \pi\act x  = x\}$ is cofinite. Since the set $D_2 = \{\catnew{c} \mid \catnew{c}\cap\supp{}{x} = \emptyset\}$ is cofinite, it follows that $D_1\cap D_2$ is still cofinite. Take $\catnew{c}\in D_1\cap D_2$. Then $\pi\act x = x$ and $\catnew{c}\cap \supp{}{x} = \emptyset$ hold simultaneously. Since $\pi\act x = x$ and each cycle of $\pi_{\catnew{c}}$ mention atoms from $\catnew{c}$, it follows by Lemma~\ref{lemma:generated-group} that $\dom{\pi_{\catnew{c}}}\cap \supp{}{x} = \emptyset$ and, by the definition of support, $\pi_{\catnew{c}}\act  x = x$.

        Now, it remains to prove that $\pi_{\neg\catnew{c}}\act x = x$. Using the fact that $\pi_{\catnew{c}}$ and $\pi_{\catnew{c}}$ commute because they are disjoint, we get
        \begin{align*}
            \pi\act x = x &\Longrightarrow (\pi_{\catnew{c}}\circ\pi_{\neg\catnew{c}})\act x = x\\
            &\Longrightarrow (\pi_{\neg\catnew{c}}\circ \pi_{\catnew{c}})\act x = x\\
            &\Longrightarrow \pi_{\neg\catnew{c}}\act(\pi_{\catnew{c}}\act x) = x\\
            &\Longrightarrow \pi_{\neg\catnew{c}}\act x = x.
        \end{align*}
        Thus, the sets $\{\catnew{c} \mid \pi_{\catnew{c}}\act x = x\}$ and $\{\catnew{c} \mid \pi_{\neg\catnew{c}}\act  x = x\}$ are cofinite and the result follows.

        Conversely, suppose $\newc{c}{}.\pi_{\catnew{c}}\act x = x$ and $\newc{c}{}.\pi_{\neg\catnew{c}}\act x = x$. This implies that the sets $D_1 = \{\catnew{c} \mid \pi_{\catnew{c}}\act  x = x\}$ and $D_2 = \{\catnew{c} \mid \pi_{\neg\catnew{c}}\act  x = x\}$ are cofinite. By assumption, both $D_1$ and $D_2$ are cofinite. Consequently, their intersection $D_1 \cap D_2$ is also cofinite. For all $\catnew{c} \in D_1 \cap D_2$, $\pi_{\catnew{c}}\act x = x$ and $\pi_{\neg\catnew{c}}\act x = x$ hold simultaneously. Applying $\pi_{\neg\catnew{c}}$ to both sides of $\pi_{\catnew{c}}\act x = x$, we obtain $\pi\act x = x$. Therefore, $\{\catnew{c} \mid \pi\act  x = x\}$ is cofinite and so $\newc{c}{}.\pi\act x = x$.

        \item Suppose, by contradiction, that there is a $\catnew{c}$ satisfying $\catnew{c}\cap \supp{}{x} = \emptyset$ such that $\dom{\pi_{\catnew{c}}}\cap \supp{}{x} \neq \emptyset$. Then there is an atom $a$ (that cannot be in $\catnew{c}$) such that $a\in \dom{\pi_{\catnew{c}}}$ and $a\in \supp{}{x}$. Thus, for some $l\geq 1$, we can write
    \[
        \pi_{\catnew{c}} = (a\  \pi_{\catnew{c}}(a) \ \ldots \  \pi_{\catnew{c}}^l(a))\circ\rho
    \]
    where $\rho$ is some permutation such that $\dom{\rho}\cap \{a, \pi_{\catnew{c}}(a),\ldots,\pi_{\catnew{c}}^l(a)\} = \emptyset$. Since each cycle of $\pi_{\catnew{c}}$ mention at least one atom from $\catnew{c}$, it follows that there is some $1\leq k \leq l$ such that $\pi_{\catnew{c}}^k(a) \in \catnew{c}$. As a consequence, $\pi_{\catnew{c}}^k(a)\notin\supp{}{x}$. By Lemma~\ref{lemma:generated-group}, it follows that $a\notin \supp{}{x}$, a contradiction.

     Conversely, suppose that $\dom{\pi_{\catnew{c}}}\cap \supp{}{x} = \emptyset$ for all $\catnew{c}\cap\supp{}{x} = \emptyset$. By the definition of support, this is equivalent to say that $\pi_{\catnew{c}}\act x = x$ for all $\catnew{c}\cap\supp{}{x} = \emptyset$. In other words, $\{\catnew{c} \mid \catnew{c}\cap \supp{}{x} = \emptyset\} \subseteq \{\catnew{c} \mid \pi_{\catnew{c}}\act  x = x\}$. Therefore, $\{\catnew{c} \mid \pi_{\catnew{c}}\act  x = x\}$ is cofinite and so $\newc{c}{}.  \pi_{\catnew{c}}\act  x = x$.

     \item Assume $\newc{c}{}.\pi_{\neg\catnew{c}}\act x = x$. By definition, this is equivalent to say that $\{\catnew{c} \mid \pi_{\neg\catnew{c}}\act x \neq x\}$ is finite. Suppose, by contradiction, that $\pi_{\neg \atnew{\pvec{c}'}}\act x \neq x$ for some $\atnew{\pvec{c}'} = \atnew{c_1'},\ldots,\atnew{c_n'}$ such that $\atnew{\pvec{c}'}\cap\supp{}{x} = \emptyset$. Take another list $\atnew{\pvec{c}''} = \atnew{c_1''},\ldots,\atnew{c_n''}$ such that $\atnew{\pvec{c}''}\cap(\supp{}{x}\cup\dom{\pi}\cup\atnew{\pvec{c}'}) = \emptyset$ (there are a cofinite amount of them). Then $\pi_{\neg\atnew{\pvec{c}'}} = \pi_{\neg\atnew{\pvec{c}''}}$ and so $\pi_{\neg\atnew{\pvec{c}''}}\act x \neq x$, proving that the set $\{\catnew{c} \mid \pi_{\neg\catnew{c}}\act x\neq x\}$ is cofinite, a contradiction. The converse follows by the definition of $\new$.

    \end{enumerate}
\end{proof}

\section{Proofs of Section~\ref{sec:properties} - Properties}\label{app:properties}

\miscellaneous*

\begin{proof}
\begin{enumerate}
    \item We want to prove that the conclusion of each of these rules implies its respective premise. But this follows directly from the fact that each rule corresponds to a unique class of terms. For example, if $\Upsilon_{\catnew{c}} \vdash  \tf{f^C}(t_0,t_1) \faeq{C} \tf{f^C}(t_1,t_0)$, then there is a proof $\Pi$ of this judgment. However, the only possible rule applicable as a last step is $(\frule{\faeq{C}}{\tf{f^C}})$, so we either have $\Upsilon_{\catnew{c}}\vdash t_0 \aeq{C} t_0$ and $\Upsilon_{\catnew{c}} \vdash t_1 \aeq{C} t_1$, or $\Upsilon_{\catnew{c}} \vdash t_0 \aeq{C} t_1$ and $\Upsilon_{\catnew{c}} \vdash  t_1 \aeq{\C} t_0$. The same reasoning applies to all the other rules.

    \item It's sufficient to prove only the left-to-right case. The proof is by induction on the last rule applied.
        \begin{itemize}
            \item The last rule is $(\frule{\faeq{C}}{a})$. In this case, $\Upsilon_{\catnew{c}} \vdash  a \aeq{C}a$. By $(\frule{\faeq{C}}{a})$, we have $\Upsilon_{\catnew{c}} \vdash  \rho\act a \aeq{C}\rho\act a$.

            \item The last rule is $(\frule{\faeq{C}}{var})$. In this case, $\Upsilon_{\catnew{c}} \vdash \pi_1\act X \aeq{C} \pi_2\act X$. By Inversion (Theorem~\ref{thm:miscellaneous}(\ref{thm:inversion})), $\pi_2^{-1}\circ\pi_1\in \PN{}{\Upsilon_{\catnew{c}}|_X}$. Since $(\rho\circ\pi_2)^{-1}\circ(\rho\circ\pi_1) = \pi_2^{-1}\circ\pi_1$, the result follows by an application of rule $(\frule{\faeq{C}}{var})$.

            \item The rules $(\frule{\faeq{C}}{\tf{f}}),(\frule{\faeq{C}}{\tf{f^C}}), (\frule{\faeq{C}}{[a]})$ and $(\frule{\faeq{C}}{ab})$ follow by induction. Here we prove only the case of the rule $(\frule{\faeq{C}}{ab})$. In this case, we have $\Upsilon_{\catnew{c}} \vdash [a]t \aeq{C} [b]s$. By Inversion (Theorem~\ref{thm:miscellaneous}(\ref{thm:inversion})), we obtain $\Upsilon_{\catnew{c},\atnew{c_1}} \vdash  \newswap{a}{c_1}\act t \aeq{C} \newswap{b}{c_1}\act s$ where $\atnew{c_1}$ is taken such that it doesn't occur in $\Upsilon_{\catnew{c}},a,b,t,s,\rho$. By induction, we have $\Upsilon_{\catnew{c},\atnew{c_1}} \vdash  \rho\act(\newswap{a}{c_1}\act t) \aeq{C} \rho\act(\newswap{b}{c_1}\act s)$, which is equivalent to $\Upsilon_{\catnew{c},\atnew{c_1}}  \vdash \newswap{\rho(a)}{c_1}\act (\rho\act t) \aeq{C} \newswap{\rho(b)}{c_1}\act (\rho\act s)$. Applying the $(\frule{\faeq{C}}{ab})$ rule, we obtain $\Upsilon_{\catnew{c}}  \vdash [\rho(a)]\rho\act t \aeq{C} [\rho(b)]\rho\act s$, which is equivalent to $\Upsilon_{\catnew{c}}  \vdash \rho\act[a]t \aeq{C} \rho\act[b]s$.

            \item  We will prove only the interesting cases.
    \begin{itemize}
        \item {\it Reflexivity.} The variable case is trivial because $\pi^{-1}\circ\pi = \id$ and $\id \in \PN{}{\Upsilon_{\catnew{c}}|_X}$ always holds.

        \item {\it Symmetry.} Suppose $\Upsilon_{\catnew{c}} \vdash \pi_1\act X \aeq{C} \pi_2\act X$. By Inversion (Theorem~\ref{thm:miscellaneous}(\ref{thm:inversion})), we obtain $\pi_2^{-1}\circ\pi_1 \in \PN{}{\Upsilon_{\catnew{c}}|_X}$. Hence $\pi_1^{-1}\circ \pi_2 = (\pi_2^{-1}\circ\pi_1)^{-1} \in  \PN{}{\Upsilon_{\catnew{c}}|_X}$. Therefore, the result follows by rule $(\frule{\faeq{C}}{var})$.

        \item {\it Transitivity.}

        Given nominal terms $s,t,u$ and derivations $\Upsilon_{\catnew{c}} \vdash s \aeq{C} u$ and $\Upsilon_{\atnew{\pvec{c}'}}\vdash u \aeq{C} t$, we will show that $\Upsilon_{\catnew{c}\cup\atnew{\pvec{c}'}} \vdash s \aeq{C} t$. We establish the result by induction on $s$, whose structure influences the other terms. We present only the suspension case here, as it is the most interesting. For $s \equiv \pi_1\act X$, we have $\Upsilon_{\catnew{c}} \vdash \pi_1\act X \aeq{C} u$. This forces $u \equiv \pi_2\act X$, which, in turn, forces $t\equiv \pi_3\act X$. Consequently, $\Upsilon_{\catnew{c}} \vdash \pi_1\act X \aeq{C} \pi_2\act X$ and $\Upsilon_{\catnew{c}} \vdash \pi_2\act X \aeq{C} \pi_3\act X$. By Inversion (Theorem~\ref{thm:miscellaneous}(\ref{thm:inversion})), $\pi_2^{-1}\circ\pi_1 \in \PN{}{\Upsilon_{\catnew{c}}|_X}$ and $\pi_3^{-1}\circ\pi_2 \in \PN{}{\Upsilon_{\atnew{\pvec{c}'}}|_X}$. Since both $\PN{}{\Upsilon_{\catnew{c}}|_X}$ and $\PN{}{\Upsilon_{\atnew{\pvec{c}'}}|_X}$ are subsets of $\PN{}{\Upsilon_{\catnew{c}\cup\atnew{\pvec{c}'}}|_X}$ we have that $\pi_2^{-1}\circ\pi_1$ and
            $\pi_3^{-1}\circ\pi_2$ are in $\PN{}{\Upsilon_{\catnew{c}\cup\atnew{\pvec{c}'}}|_X}$.
            Therefore, $\pi_3^{-1}\circ\pi_1 = (\pi_3^{-1}\circ\pi_2)\circ(\pi_2^{-1}\circ\pi_1)$ is also in $\PN{}{\Upsilon_{\catnew{c}\cup\atnew{\pvec{c}'}}|_X}$
            and thus the result follows by rule $(\frule{\faeq{C}}{var})$.
    \end{itemize}
        \end{itemize}

        \item Direct consequence Equivariance and Equivalence.

        \item The proof is by induction on the structure of $s$. Here we show only the interesting cases.
       \begin{itemize}

        \item $s \equiv \pi_1 \act Y$. This forces $t \equiv \pi_2 \act Y$. Then we have $(\Upsilon\uplus\{\pi\fix{C} X\})_{\catnew{c}} \vdash \pi_1 \act Y \aeq{C} \pi_2 \act Y$. By Inversion (Theorem~\ref{thm:miscellaneous}(\ref{thm:inversion})), it follows that $\pi_2^{-1}\circ\pi_1\in \PN{}{(\Upsilon\uplus\{\pi\fix{C} X\})_{\catnew{c}}|_Y}$.

        \begin{itemize}
            \item $Y\not\equiv X$.

            Then $\PN{}{(\Upsilon\uplus\{\pi\fix{C} X\})_{\catnew{c}}|_Y} = \PN{}{\Upsilon_{\catnew{c}}|_Y}$ and so the result follows by rule $(\frule{\faeq{C}}{var})$.

            \item $Y\equiv X$.

            In this case, the condition $(\dom{\pi}\setminus\catnew{c})\cap\atm{s,t} = \emptyset$ is the same as
            \[
                (\dom{\pi}\setminus\catnew{c})\cap(\dom{\pi_1}\cup\dom{\pi_2}) = \emptyset.
            \]
            This implies $\dom{\pi_2^{-1}\circ\pi_1}\cap (\dom{\pi}\setminus\catnew{c}) = \emptyset$. Consequently, we have that $\pi_2^{-1}\circ\pi_1\in\PN{}{\Upsilon_{\catnew{c}}|_X}$ and the result follows by an application of rule $(\frule{\faeq{C}}{var})$.
        \end{itemize}

        \item $s \equiv [a]s'$. We consider two cases for $t$:

            \begin{itemize}
                \item Case 1: $t \equiv [a]t'$.

                In this situation, we have $(\Upsilon\uplus\{\pi\fix{C} X\})_{\catnew{c}} \vdash [a]s' \aeq{C} [a]t'$. By Inversion (Theorem~\ref{thm:miscellaneous}(\ref{thm:inversion})), we obtain $(\Upsilon\uplus\{\pi\fix{C} X\})_{\catnew{c}} \vdash s' \aeq{C} t'$ and since $(\dom{\pi}\setminus\catnew{c})\cap \atm{s,t} = \emptyset$ implies $(\dom{\pi}\setminus\catnew{c})\cap \atm{s',t'} = \emptyset$, the inductive hypothesis gives us $\Upsilon_{\catnew{c}} \vdash s'\aeq{C} t'$, and the result follows by applying rule $(\frule{\faeq{C}}{ab})$.


                \item Case 2: $t \equiv [b]t'$.

                    Here, we have $(\Upsilon\uplus\{\pi\fix{C} X\})_{\catnew{c}} \vdash [a]s' \aeq{C} [b]t'$. By Inversion (Theorem~\ref{thm:miscellaneous}(\ref{thm:inversion})), we obtain $(\Upsilon\uplus\{\pi\fix{C} X\})_{\catnew{c},\atnew{c_1}} \vdash \newswap{a}{c_1} \act s' \aeq{C} \newswap{b}{c_1} \act t'$ for some $\atnew{c_1}\notin \atm{\Upsilon_{\catnew{c}},a,b,s',t',\pi}$.  By the choice of $\atnew{c_1}$, we have that $\dom{\pi}\setminus\catnew{c} = \dom{\pi}\setminus\catnew{c},\atnew{c_1}$ and so $(\dom{\pi}\setminus\catnew{c})\cap\atm{s,t} = \emptyset$ implies $(\dom{\pi}\setminus\catnew{c},\atnew{c_1})\cap\atm{s,t} = \emptyset$. We claim that
                    \[
                        (\dom{\pi}\setminus\catnew{c},\atnew{c_1})\cap\atm{\newswap{a}{c_1}\act s',\newswap{b}{c_1}\act t'}) = \emptyset.
                    \]
                    Indeed, suppose, by contradiction, that there is an atom, say $d$, such that $d\in \dom{\pi}\setminus\catnew{c},\atnew{c_1}$ and $d\in \atm{\newswap{a}{c_1}\act s',\newswap{b}{c_1}\act t'})$. Then $\newswap{a}{c_1}(d)\in \atm{s'}$. Note that $d$ cannot be $a$ or $\atnew{c_1}$. Thus $\newswap{a}{c_1}(d) = d$ and hence $d\in \atm{s'}$, contradicting $(\dom{\pi}\setminus\catnew{c},\atnew{c_1})\cap\atm{s,t} = \emptyset$. 

                    Now, by induction, we have $\Upsilon_{\catnew{c},\atnew{c_1}} \vdash \newswap{a}{c_1} \act s' \aeq{C} \newswap{b}{c_1} \act t'$ and the result follows by applying the rule $(\frule{\faeq{C}}{ab})$. 
            \end{itemize}
        \end{itemize}
\end{enumerate}

\end{proof}

% \begin{remark}
%     Suppose $\Upsilon_{\catnew{c}} \vdash \pi\act(\rho_1\act X) \aeq{C} \rho_2\act X$ (this includes the case where $\rho_1 = \rho_2$.). We can always assume, w.l.o.g., that the names in $\catnew{c}$ do not occur in $\rho_1$ and $\rho_2$. Indeed, we know that $\rho_1 = \pnew{\rho_1}{\catnew{c}}\circ\npnew{\rho_1}{\catnew{c}}$ and $\rho_2 = \pnew{\rho_2}{\catnew{c}}\circ\npnew{\rho_2}{\catnew{c}}$. Substituting this into the judgment, we get $\Upsilon_{\catnew{c}} \vdash \pi\act((\pnew{\rho_1}{\catnew{c}}\circ\npnew{\rho_1}{\catnew{c}})\act X) \aeq{C} (\pnew{\rho_2}{\catnew{c}}\circ\npnew{\rho_2}{\catnew{c}})\act X$. By Equivariance (Theorem~\ref{thm:miscellaneous}(\ref{thm:object-equivariance}), we multiply both sides by $\pnew{\rho_2}{\catnew{c}}^{-1}$ and obtain $\Upsilon_{\catnew{c}} \vdash (\pnew{\rho_2}{\catnew{c}}^{-1}\circ\pi\circ\pnew{\rho_1}{\catnew{c}})\act(\npnew{\rho_1}{\catnew{c}}\act X) \aeq{C} \npnew{\rho_2}{\catnew{c}}\act X$. Define $\pi' := \pnew{\rho_2}{\catnew{c}}^{-1}\circ\pi\circ\pnew{\rho_1}{\catnew{c}}, \rho_1' := \npnew{\rho_1}{\catnew{c}}$ and $\rho_2' := \npnew{\rho_2}{\catnew{c}}$. Then the judgment  $\Upsilon_{\catnew{c}} \vdash \pi\act(\rho_1\act X) \aeq{C} \rho_2\act X$ is equivalent to  $\Upsilon_{\catnew{c}} \vdash \pi'\act(\rho_1'\act X) \aeq{C} \rho_2'\act X$. 
% \end{remark}

\section{Proofs of Section~\ref{sec:soundness-completeness} - Soundness and Completeness}\label{app:soundness-completeness}

\soundcomplete*

Since the proof of Completeness is much more elaborate, we will separate the proofs in different sections.

\subsection{Soundness}\label{ssec:soundness}

\begin{proof}[of Soundness]
     Let $\nalg{A}$ be a model of $\C$. We must show for any valuation $\varsigma$ that if $\Upsilon_{\catnew{c}} \vdash t\aeq{C} u$ then $\Int{\Upsilon_{\catnew{c}} \vdash t\aeq{C} u}{\nalg{A}}{\varsigma}$ is valid.

     The proof proceeds by induction on the last rule applied in the derivation of the judgment. To illustrate, we show the cases where the last rule applied is...
     \begin{itemize}
         \item $(\frule{\faeq{C}}{var})$. In this case, we have
            \begin{prooftree}
                \AxiomC{$\rho^{-1}\circ \pi\in \PN{}{\Upsilon_{\catnew{c}}|_X}$}
                \RightLabel{$(\frule{\faeq{C}}{var})$}
                \UnaryInfC{$\Upsilon_{\catnew{c}} \vdash \pi\act X \aeq{C} \rho\act X$}
            \end{prooftree}
%
            Suppose $\Int{\Upsilon_{\catnew{c}}}{\nalg{A}}{\varsigma}$ is valid. Then:
            \(
                \newc{c}{}. \eta\act \Int{Y}{\nalg{A}}{\varsigma} =  \Int{Y}{\nalg{A}}{\varsigma} \text{ for all $\eta\fix{C} Y\in \Upsilon$.}
            \)
            In particular,
            \[
                \newc{c}{}. \eta\act \Int{X}{\nalg{A}}{\varsigma} =  \Int{X}{\nalg{A}}{\varsigma} \text{ for all $\eta\fix{C} X\in \Upsilon|_X$.}
            \]
            By Pitts' equivalence (Lemma~\ref{lemma:pitts-eq-generalized}) it follows, for any $\eta\fix{C} Y\in \Upsilon$, that for all $\catnew{c}\cap \supp{}{\Int{X}{\nalg{A}}{\varsigma}} = \emptyset$,
            (i) $\dom{\eta_{\catnew{c}}}\cap\supp{}{\Int{X}{\nalg{A}}{\varsigma}}=\emptyset$,
                and
                (ii) $\eta_{\neg\catnew{c}}\act \Int{X}{\nalg{A}}{\varsigma} =  \Int{X}{\nalg{A}}{\varsigma}$. From the condition $\rho^{-1}\circ\pi \in \PN{}{\Upsilon_{\catnew{c}}|_X}$ and Lemma~\ref{lemma:generated-group}, it follows that $(\rho^{-1}\circ\pi) \act \Int{X}{\nalg{A}}{\varsigma} =  \Int{X}{\nalg{A}}{\varsigma}$ holds.

            \item $(\frule{\faeq{C}}{ab})$. In this case, we have
           \begin{prooftree}
               \AxiomC{$\Upsilon_{\catnew{c},\atnew{c_1}} \vdash \newswap{a}{c_1}\act t' \aeq{C} \newswap{b}{c_1}\act u'$}
          \RightLabel{$(\frule{\faeq{C}}{ab})$}
          \UnaryInfC{$\Upsilon_{\catnew{c}} \vdash  [a]t' \aeq{C} [b]u'$}
           \end{prooftree}

           Suppose $\Int{\Upsilon_{\catnew{c},\atnew{c_1}}}{\nalg{A}}{\varsigma}$ is valid. We aim to show that $\Int{[a]t'}{\nalg{A}}{\varsigma} = \Int{[b]u'}{\nalg{A}}{\varsigma}$, which is equivalent to:
            \[
                \abs^{\nalg{A}}(a,\Int{t'}{\nalg{A}}{\varsigma}) = \abs^{\nalg{A}}(b,\Int{u'}{\nalg{A}}{\varsigma}).
            \]
            From the validity of $\Int{\Upsilon_{\catnew{c},\atnew{c_1}}}{\nalg{A}}{\varsigma}$, it follows that:
            \[
                \newc{c}{},\atnew{c_1}.\pi\act \Int{X}{\nalg{A}}{\varsigma} = \Int{X}{\nalg{A}}{\varsigma} \text{ for all } \pi\fix{C} X\in\Upsilon.
            \]
            Since $\atnew{c_1}$ is taken such that $\atnew{c_1}\notin\atm{\Upsilon_{\catnew{c}}}$ we have that $\atnew{c_1}\notin \dom{\pi}$ for all $\pi\fix{C} X\in \Upsilon$. Furthermore, we can assume w.l.o.g. that $\atnew{c_1}$ is taken fresh for $\Int{t'}{\nalg{A}}{\varsigma}$, $\Int{u'}{\nalg{A}}{\varsigma}$  and $\Int{X}{\nalg{A}}{\varsigma}$ for all $X\in \var{\Upsilon_{\catnew{c}}}$. Then,
           \[
                \newc{c}{}.\pi\act \Int{X}{\nalg{A}}{\varsigma} = \Int{X}{\nalg{A}}{\varsigma} \text{ holds for all } \pi\fix{C} X\in\Upsilon.
            \]
            Thus, $\Int{\Upsilon_{\catnew{c}}}{\nalg{A}}{\varsigma}$ is valid. By induction, we get
            \[
                \Int{\newswap{a}{c_1}\act t'}{\nalg{A}}{\varsigma} = \Int{\newswap{b}{c_1}\act u'}{\nalg{A}}{\varsigma}.
            \]

            \begin{claim}[1]
                We claim that $a\notin\supp{}{\Int{u'}{\nalg{A}}{\varsigma}}$ and $b\notin\supp{}{\Int{t'}{\nalg{A}}{\varsigma}}$.  We will prove only $a\notin\supp{}{\Int{u'}{\nalg{A}}{\varsigma}}$, because the proof for $b \notin\supp{}{\Int{t'}{\nalg{A}}{\varsigma}}$ follows by a similar argument. Suppose, by contradiction, that $a \in \supp{}{\Int{u'}{\nalg{A}}{\varsigma}}$. Under this assumption, we have $a \in \supp{}{\newswap{b}{c_1} \act \Int{u'}{\nalg{A}}{\varsigma}}$. Consequently, it follows that $a \in \supp{}{\newswap{a}{c_1} \act \Int{t'}{\nalg{A}}{\varsigma}}$. This implication leads to $\atnew{c_1} \in \supp{}{\Int{t'}{\nalg{A}}{\varsigma}}$. However, this result contradicts the assumption that $\atnew{c_1}$ is fresh for $\Int{t'}{\nalg{A}}{\varsigma}$. 
            \end{claim}

            Now, let's use this information to proceed with the proof. Using that $a\notin\supp{}{\Int{u'}{\nalg{A}}{\varsigma}}$, we infer $a\notin\supp{}{\abs^{\nalg{A}}(b, \Int{u'}{\nalg{A}}{\varsigma})}$. Moreover, by definition, we have $\new \atnew{c}. \newswap{b}{c}\act \abs^{\nalg{A}}(b, \Int{u'}{\nalg{A}}{\varsigma}) = \abs^{\nalg{A}}(b, \Int{u'}{\nalg{A}}{\varsigma})$, which by Pitts' equivalence means that $b \notin\supp{}{\abs^{\nalg{A}}(b, \Int{u'}{\nalg{A}}{\varsigma})}$. Therefore, we obtain $(a \ b) \act \abs^{\nalg{A}}(b, \Int{u'}{\nalg{A}}{\varsigma}) = \abs^{\nalg{A}}(b, \Int{u'}{\nalg{A}}{\varsigma})$. Hence, we can rewrite it  as follows:
            \begin{align*}
                \abs^{\nalg{A}}(b,\Int{u'}{\nalg{A}}{\varsigma}) &= (a \ b)\act\abs^{\nalg{A}}(b,\Int{u'}{\nalg{A}}{\varsigma}) \\
                &= \abs^{\nalg{A}}(a,(a \ b)\act \Int{u'}{\nalg{A}}{\varsigma}).
            \end{align*}
            To complete the proof, note that all we need to do is to show that $(a \ b) \act \Int{u'}{\nalg{A}}{\varsigma} = \Int{t'}{\nalg{A}}{\varsigma}$. Then let's prove this final step.

            \begin{claim}[2]
                We claim that $(a \ b)\act \Int{u'}{\nalg{A}}{\varsigma} = \Int{t'}{\nalg{A}}{\varsigma}$. On one hand, by Claim (1), we know that $b\notin\supp{}{\Int{t'}{\nalg{A}}{\varsigma}}$ and since $\atnew{c_1}\notin\supp{}{\Int{t'}{\nalg{A}}{\varsigma}}$, this implies $\newswap{b}{c_1}\act \Int{t'}{\nalg{A}}{\varsigma} = \Int{t}{\nalg{A}}{\varsigma}$. On the other hand, we already know that $\newswap{a}{c_1}\act\Int{t'}{\nalg{A}}{\varsigma} = \newswap{b}{c_1}\act\Int{u'}{\nalg{A}}{\varsigma}$ holds by induction. Thus,
                \begin{align*}
                    & \newswap{a}{c_1}\act\Int{t'}{\nalg{A}}{\varsigma} =  \newswap{b}{c_1}\act\Int{u'}{\nalg{A}}{\varsigma} \\
                    \Longleftrightarrow{} & \newswap{b}{c_1}\act(\newswap{a}{c_1}\act\Int{t'}{\nalg{A}}{\varsigma}) =  \Int{u'}{\nalg{A}}{\varsigma}\\
                    \Longleftrightarrow{}& ((a \ b)\circ\newswap{b}{c_1}\circ\newswap{a}{c_1})\act\Int{t'}{\nalg{A}}{\varsigma} =  (a \ b)\act\Int{u'}{\nalg{A}}{\varsigma}\\
                    \Longleftrightarrow{}& \newswap{b}{c_1}\act\Int{t'}{\nalg{A}}{\varsigma} =  (a \ b)\act\Int{u'}{\nalg{A}}{\varsigma}\\
                    \Longleftrightarrow{}& \Int{t'}{\nalg{A}}{\varsigma} =  (a \ b)\act\Int{u'}{\nalg{A}}{\varsigma}.
                \end{align*}
            \end{claim}

             This proves completes the proof that $\abs^{\nalg{A}}(b,\Int{u'}{\nalg{A}}{\varsigma}) =  \abs^{\nalg{A}}(a,\Int{t'}{\nalg{A}}{\varsigma})$.

                \end{itemize}
\end{proof}

\subsection{Completeness}\label{app:completeness}

The objective of this section is to show that the calculus is complete w.r.t. the nominal set semantics. The construction is long and complex, therefore, we will just present the roadmap of the constructions and results needed to prove Completeness, and only give the complete proof of the most challenging result (Lemma\ref{alemma:valid-judge-pi-generated}). We refer the reader interested in the details to the extended version of this paper~\cite{arxiv/cairessantos2025}.

% The goal is to prove following theorem:

% %Specifically, we will prove that the following statement holds:

% \begin{theorem}[Completeness]\label{athm:completeness-fix}
%     For any $\Upsilon_{\catnew{c}},t,u$: If $\Upsilon_{\catnew{c}} \vDash t\aeq{C} u$, then $\Upsilon_{\catnew{c}} \vdash t\aeq{C} u$.
% \end{theorem}

\subsubsection*{Free Term Models.}\label{app:free-models}
The set of nominal terms $\F(\Sigma, \V)$ cannot form a model of commutative ($\C$) because it is not a nominal set under the usual permutation action. Specifically, the support of a suspension, such as $\rho \act X$, cannot be precisely determined, as $X$ represents an unknown term. The standard approach is to work with the {\em free term models}, consisting of ground terms $\F(\Sigma\cup \D)$, over a signature $\Sigma$ and a set of term-formers $\D$ disjoint from $\Sigma$.  Here we assume that $\Sigma$ contains commutative function symbols, but $\D$ doesn't. The set of {\em free terms} is defined by the following grammar:
\[ g,g' ::=  a \mid [a]g \mid {\tf f} (g_1,\ldots, g_n) \mid {\tf d}(a_1,\ldots, a_k)\]
Here $\tf{f}:n$ ranges over $\Sigma$ and $\tf{d}:k$ range over elements of $\D$.

As expected, to obtain nominal models for a theory $\C$ we will work with the set of free terms quotient by provable equality modulo $\C$. 
%Note that instead of adding constants to the language (as it is usually done in first-order logic), we added a set of term-formers that are applied to atoms to guarantee that resulting nominal terms have support.
%
This quotient set, defined next, effectively incorporates both $\alpha$-equivalence and commutative behaviour, enabling us to construct a model that satisfies the desired properties in terms of nominal set semantics.

\begin{definition}[Free terms up to $\C$]
Write $\lin{g}$ for the set of free terms $g'$ such that $\emptyset\vdash g \aeq{C} g'$.
The {\em set of free terms $\F(\Sigma\cup \D)$ up to $\C$}, denoted as  $\lin{\F}(\Sigma\cup \D)$, is the set of equivalence classes (modulo $\C$) of free terms, i.e.,  $\lin{\F}(\Sigma\cup \D)=\{\lin{g} \mid g \text{ is a free term}\}$.
\end{definition}

In the following, we will abbreviate $\lin{\F}(\Sigma\cup \D)$ as  $\cF$. The next lemma presents a collection of diverse properties of free terms modulo $\C$. Notably, item \ref{alemma:ground-algebra-nominal-set} establishes that $\lin{\F}_{\C}$ is a nominal set, while item \ref{athm:supp-free-names} defines the support of the equivalence class of a free term modulo $\C$.

We will use the following elementary result from nominal set theory. 
%The proof can be found in Pitts~\cite{book/Pitts}:
\begin{lemma}[Pitts~\cite{book/Pitts}]\label{alemma:quotient-nominal}
    If $\nom{X}$ is a nominal set and $\sim$ is an equivariant equivalence relation on $\nom{X}$ and $x \in \nom{X}/_{\sim}$, then $\supp{}{x} = \bigcap\{\supp{}{g}\mid g\in x\}$.
\end{lemma}

\begin{lemma}[Properties of $\cF$]\label{alemma:ground-algebra-properties} Let $g,g'\in \F(\Sigma\cup \D)$. The following hold: \hfill

   \begin{enumerate}
    \item \label{alemma:ground-algebra-nominal-set} The set $\cF$ with the action $\pi\act\lin{g} = \lin{\pi\act g}$ is a nominal set and $\supp{}{\lin{g}} = \bigcap\{\supp{}{g'}\mid g'\in \lin{g}\}$.
    \item  \label{alemma:free-names-equivariant} The map $\tf{fn}(-):\F(\Sigma\cup \D) \to \pow{\tf{fin}}{\A}$ is equivariant.
    \item \label{alemma:free-names-preserve} If $g' \in \lin{g}$  then $\tf{fn}(g) = \tf{fn}(g')$.
    \item  \label{athm:supp-free-names} $\supp{}{\lin{g}} = \tf{fn}(g)$.
    \end{enumerate}
\end{lemma}

\begin{proof}
\begin{enumerate}
    \item Direct consequence of Lemma~\ref{alemma:quotient-nominal}.

    \item For all $g\in \lin{\F}(\Sigma\cup\D)$ and all permutation $\pi$, we must prove that $\tf{fn}(\pi\act g) = \pi\act\tf{fn}(g)$. This is done by induction on the ground term $g$. As an illustrative case, let $g\equiv [a]g'$. By induction $\tf{fn}(\pi\act g') = \pi\act \tf{fn}(g')$. Thus,
    \begin{align*}
         \tf{fn}(\pi\act[a]g') = \tf{fn}([\pi(a)]\pi\act g') = \tf{fn}(\pi\act g') \setminus \{\pi(a)\}
         &= \pi\act \tf{fn}(g') \setminus \pi\act \{a\} \\
         &= \pi\act(\tf{fn}(g')\setminus\{a\})\\
         &= \pi\act \tf{fn}([a]g').
    \end{align*}

    \item Suppose $g \sim g'$. Then by definition there is a derivation $~\vdash g \aeq{C} g'$. The proof follows by induction on the last rule applied to obtain  $~\vdash g \aeq{C} g'$. The only non-trivial case is when the last rule applied is $(\frule{\faeq{C}}{ab})$. In this case, $g \equiv [a]g_1$ and $g' \equiv [b]g_1'$ and $~\vdash [a]g_1 \aeq{C} [b]g_1'$. Then by Inversion (Theorem~\ref{thm:miscellaneous}(\ref{thm:inversion})), we have that $~\vdash \newswap{a}{c_1}\act g_1 \aeq{C} \newswap{b}{c_1}\act g_1'$ where $\atnew{c_1}\notin \atm{\catnew{c},a,b,g_1,g_1'}$. Thus $\newswap{a}{c_1}\act g_1 \sim \newswap{b}{c_1}\act g_1'$. By induction, we have $\tf{fn}(\newswap{a}{c_1}\act g_1) = \tf{fn}(\newswap{b}{c_1}\act g_1')$.
    
    \begin{claim}[1]
         We claim that $a\notin \tf{fn}(g_1')$. Suppose, by contradiction, that $a\in \tf{fn}(g_1')$. Then
        \begin{equation*}
            \begin{tabular}{@{}l@{ }ll@{}}
               & $a \in\tf{fn}(g_1')$ &  \\
              $\Longrightarrow$ & $a \in\tf{fn}(\newswap{b}{c_1} \act g_1')$ & (item~\ref{alemma:free-names-equivariant})\\
              $\Longrightarrow$ & $a \in\tf{fn}(\newswap{a}{c_1} \act g_1)$ & \\
           $\Longrightarrow$ & $\atnew{c_1} \in\tf{fn}(g_1)$ & (item~\ref{alemma:free-names-equivariant})
            \end{tabular}
        \end{equation*}
        This leads to a contradiction, because we took $\atnew{c_1}$ such that $\atnew{c_1}\notin \atm{g_1}$ and $\tf{fn}(g_1) \subseteq \atm{g_1}$. This proves the claim.
    \end{claim}

    \begin{claim}[2]
         We also claim that $\tf{fn}(g_1) = (a \ b)\act \tf{fn}(g_1')$. Indeed, from $a,\atnew{c_1} \notin \tf{fn}(g_1')$ we conclude $\newswap{a}{c_1}\act \tf{fn}(g_1') = \tf{fn}(g_1')$. Thus,
        \begin{equation*}
            \begin{tabular}{@{}l@{ }l@{}l@{}}
               & $\tf{fn}(\newswap{a}{c_1}\act g_1) = \tf{fn}(\newswap{b}{c_1}\act g_1')$ \\
               $\Longrightarrow$ & $\newswap{a}{c_1}\act \tf{fn}(g_1) = \newswap{b}{c_1}\act\tf{fn}(g_1')$ & (item~\ref{alemma:free-names-equivariant})\\
               $\Longrightarrow$ &  $\tf{fn}(g_1) = ((\atnew{c_1} \ a)\circ\newswap{b}{c_1})\act\tf{fn}(g_1')$ & \\
               $\Longrightarrow$ &  $\tf{fn}(g_1) = ((b \ a)\circ(\atnew{c_1} \ a))\act\tf{fn}(g_1')$ & \\
               $\Longrightarrow$ &   $\tf{fn}(g_1) = (a \ b)\act(\newswap{a}{c_1}\act\tf{fn}(g_1'))$ & \\
               $\Longrightarrow$ &  $\tf{fn}(g_1) = (a \ b)\act \tf{fn}(g_1')$ &
            \end{tabular}
        \end{equation*}
        This finish the prove of the second claim.
    \end{claim}
    
   Now, let's prove $\tf{fn}(g_1) \setminus\{a\} = \tf{fn}(g_1')\setminus\{b\}$. To show this, first observe that from $a\notin \tf{fn}(g_1')$, we get $a\notin \tf{fn}(g_1')\setminus\{b\}$ and, since $b\notin \tf{fn}(g_1')\setminus\{b\}$, this yields $(a \ b)\act(\tf{fn}(g_1')\setminus\{b\}) = \tf{fn}(g_1')\setminus\{b\}$. Therefore,
        \begin{equation*}
            \begin{tabular}{@{}l@{ }lll@{}}
                $\tf{fn}(g_1)\setminus\{a\}$ & $=$ &  $((a \ b)\act\tf{fn}(g_1'))\setminus((a \ b)\act\{b\})$ & by Claim (2) \\
                 & $=$ & $(a \ b)\act(\tf{fn}(g_1')\setminus\{b\})$ \\
                 & $=$ & $\tf{fn}(g_1')\setminus\{b\}$ 
            \end{tabular}
        \end{equation*}
        This completes the proof that $\tf{fn}(g) = \tf{fn}(g')$.

    \item Take an arbitrary but fixed $g\in\F(\Sigma\cup\D)$.

    \paragraph*{Step 1: Show that $\tf{fn}(g)$ supports $[g]$.}  We will do this by showing, using induction on the structure of $g$, that for any permutation $\pi$, the following holds:
    \[
        \pi\in\Fix{\tf{fn}(g)} \implies \pi\act \lin{g} = \lin{g}.
    \]
    The only interesting case is when $g \equiv [a]g_1$. This implies $\tf{fn}([a]g_1) = \tf{fn}(g_1)\setminus\{a\}$. Then $\pi\in\Fix{\tf{fn}([a]g_1)}$ implies $\pi(b) = b$ for all $b \in  \tf{fn}(g_1)\setminus\{a\}$.

            \begin{itemize}
                \item If $a\notin \tf{fn}(g_1)$, then $ \tf{fn}(g_1) \setminus \{a\} =  \tf{fn}(g_1)$ and then $\pi\in\Fix{\tf{fn}(g_1)}$ which, by induction, implies that $\pi\act \lin{g_1} = \lin{g_1}$. This means that $~\vdash \pi\act g_1 \aeq{C} g_1$, that is, $\pi\act g_1 \sim g_1$.

                \begin{enumerate}
                    \item $\pi(a) = a$: In this case,
                    \begin{prooftree}
                        \AxiomC{$\vdash \pi\act g_1 \aeq{C} g_1$}
                        \UnaryInfC{$\vdash [\pi(a)]\pi\act g_1\aeq{C} [a]g_1$}
                    \end{prooftree}
                    which implies that $\pi\act \lin{[a]g_1} = \lin{[a]g_1}$.

                    \item $\pi(a) \neq a$: In this case, the condition $a\notin\tf{fn}(g_1)$ implies $\pi(a)\notin \tf{fn}(g_1)$ by item~\ref{alemma:free-names-preserve}. Take $\atnew{c_1}\notin \atm{g_1}\cup\{a,\pi(a)\}\cup\dom{\pi}$. Then, $\newswap{a}{c_1}$ and $\newswap{\pi(a)}{c_1}$ are in $\Fix{\tf{fn}(g_1)}$ which, by induction, implies $\newswap{a}{c_1}\act \lin{g_1} = \lin{g_1}$ and $\newswap{\pi(a)}{c_1}\act \lin{g_1} = \lin{g_1}$, that is, $\newswap{\pi(a)}{c_1}\act g_1 \sim g_1$ and $g_1 \sim \newswap{a}{c_1}\act g_1$. Hence
                    \begin{equation*}
                        \begin{tabular}{@{}r@{ }lll@{}}
                             & $g_1 \sim \newswap{a}{c_1}\act g_1$ &  & \\
                             $\Longrightarrow$ & $\pi\act g_1 \sim \pi\act (\newswap{a}{c_1}\act g_1)$ &\\
                            $\Longrightarrow$ & $g_1 \sim \newswap{\pi(a)}{c_1}\act (\pi\act g_1)$ & \\
                            $\Longrightarrow$ & $\newswap{a}{c_1}\act g_1 \sim \newswap{\pi(a)}{c_1}\act (\pi\act g_1)$ & \\
                            $\Longrightarrow$ & $\newswap{\pi(a)}{c_1}\act (\pi\act g_1) \sim \newswap{a}{c_1}\act g_1$ &
                        \end{tabular}
                    \end{equation*}
                    This implies $~\vdash \newswap{\pi(a)}{c_1}\act (\pi\act g_1)  \aeq{C} \newswap{a}{c_1}\act g_1$. By an application of rule $(\frule{\faeq{C}}{ab})$, we obtain $\vdash [\pi(a)]\pi\act g_1 \aeq{C} [a]g_1$ and so $\pi\act \lin{[a]g_1} = \lin{[a]g_1}$.
                \end{enumerate}

                \item If $a\in \tf{fn}(g_1)$, then
                \begin{enumerate}
                    \item $\pi(a) = a$: In this case, $\pi\in\Fix{\tf{fn}(g_1)}$. By induction, $\pi\act\lin{g_1} = \lin{g_1}$, i.e., $~\vdash \pi\act g_1 \aeq{C} g_1.$ The result follows by an application of rule $(\frule{\faeq{C}}{[a]})$.

                    \item $\pi(a) \neq a$: In this case, we claim $a\notin\tf{fn}(\pi\act g_1)$.  Suppose, by contradiction, that $a\in\tf{fn}(\pi\act g_1)$. Then $a\in \tf{fn}(\pi\act g_1)\setminus\{\pi(a)\}$ because $\pi(a)\neq a$. This is equivalent to $a\in \pi\act(\tf{fn}(g_1)\setminus\{a\})$, which in turn is equivalent to $\pi^{-1}(a) \in \tf{fn}(g_1)\setminus\{a\}$. From the hypothesis that $\pi\in\Fix{\tf{fn}([a]g_1)} = \Fix{\tf{fn}(g_1)\setminus\{a\}}$, we get that $\pi(\pi^{-1}(a)) = \pi^{-1}(a)$, i.e., $\pi^{-1}(a) = a$. Applying $\pi$ on both sides yields $\pi(a) = a$, a contradiction.

                    Moreover, we also claim $\pi\act g_1 \sim (\pi(a) \ a)\act g_1$.  By the induction hypothesis, it is sufficient to prove that $(a \ \pi(a))\circ\pi\in\Fix{\tf{fn}(g_1)}$. Since $a\in \tf{fn}(g_1)$, let's see how $(a \ \pi(a))\circ\pi$ interacts with $a$ before analysing other atoms in $\tf{fn}(g_1)$. Then
                            \[
                                ((a \ \pi(a))\circ\pi)(a) = (a \ \pi(a))(\pi(a)) = a
                            \]
                            Now, take $b\in\tf{fn}(g_1)$. Then
                            \begin{align*}
                                ((a \ \pi(a))\circ\pi)(b) &= (a \ \pi(a))(\pi(b)) \\
                                &= \pi(b)\\
                                &= b
                            \end{align*}
                        because $\pi\in\Fix{\tf{fn}(g_1)\setminus\{a\}}$. Now, take $c_1\notin \atm{g_1}\cup\{a,\pi(a)\}\cup\dom{\pi}$. Then, $\newswap{a}{c_1}\in \Fix{\tf{fn}(\pi\act g_1)}$ because $a\notin \tf{fn}(\pi\act g_1)$ and $\atnew{c_1}\notin \tf{fn}(\pi\act g_1)$, the latter being a consequence of the fact the $\tf{fn}(\pi\act g_1)\subseteq \atm{\pi\act g_1}$. By induction, we have $\newswap{a}{c_1}\act\lin{\pi\act g_1} = \lin{\pi\act g_1}$, that is, $\pi\act g_1 \sim \newswap{a}{c_1}\act (\pi\act g_1)$. Hence,
                   \begin{equation*}
                        \begin{tabular}{@{}l@{ }l@{}l@{}}
                            & $\pi\act g_1 \sim (\pi(a) \ a)\act g_1$ &  \\
                            $\Longrightarrow$ & $\pi\act g_1 \sim \newswap{\pi(a)}{c_1}^{(\atnew{c_1} \ a)}\act g_1$ & \\
                            $\Longrightarrow$ & $\newswap{a}{c_1}\act (\pi\act g_1) \sim \newswap{\pi(a)}{c_1}\act (\newswap{a}{c_1}\act g_1)$ & \\
                            $\Longrightarrow$ & $\pi\act g_1 \sim \newswap{\pi(a)}{c_1}\act (\newswap{a}{c_1}\act g_1)$ & \\
                             $\Longrightarrow$ & $(\atnew{c_1} \ \pi(a))\act (\pi\act g_1) \sim \newswap{a}{c_1}\act g_1$ & \\
                            $\Longrightarrow$ & $\newswap{\pi(a)}{c_1}\act (\pi\act g_1) \sim \newswap{a}{c_1}\act g_1$ &
                        \end{tabular}
                    \end{equation*}
                    This implies $\vdash \newswap{\pi(a)}{c_1}\act (\pi\act g_1)  \aeq{C} \newswap{a}{c_1}\act g_1$. By an application of rule $(\frule{\faeq{C}}{ab})$, we obtain $\vdash [\pi(a)]\pi\act g_1 \aeq{C} [a]g_1$ and so $\pi\act \lin{[a]g_1} = \lin{[a]g_1}$.
                \end{enumerate}
            \end{itemize}

        
        \paragraph*{Step 2: Show that $\supp{}{g} = \tf{fn}(g)$.} Since $\tf{fn}(g)$ supports $[g]$. By the minimality of the support, we obtain $\supp{}{g} \subseteq \tf{fn}(g)$. It remains to prove the other inclusion: $\tf{fn}(g) \subseteq \supp{}{\lin{g}}$. Suppose, by contraction, that there exists $a\in \tf{fn}(g)$ such that $a\notin \supp{}{\lin{g}}$. Choose $b\notin \tf{fn}(g)$. Then $b\notin \supp{}{\lin{g}}$ because we proved that $\supp{}{\lin{g}} \subseteq \tf{fn}(g)$. Since $a,b \notin\supp{}{\lin{g}}$, we have $(a \ b)\act \lin{g} = \lin{g}$. Thus, $(a \ b)\act g \sim g$. By item~\ref{alemma:free-names-preserve}, it follows that $\tf{fn}((a \ b)\act g) = \tf{fn}(g)$. Since $b\notin \tf{fn}(g)$, it follows by item~\ref{alemma:free-names-equivariant} that $a\notin \tf{fn}((a \ b)\act g) = \tf{fn}(g)$, a contradiction. Therefore, this proves the claim that $\supp{}{[g]} = \tf{fn}(g)$.

        \paragraph*{Step 3: Completing the proof.}

        As a direct consequence of item~\ref{alemma:free-names-preserve}, we conclude that $\supp{}{\lin{g}}$ does not  depend on the representative of the class $\lin{g}$, and this completes the proof.
\end{enumerate}
\end{proof}

Now, we construct a nominal commutative $\Sigma$-algebra  $\nalg{F}$ as follows:

\begin{itemize}
    \item The nominal set $\nom{F}=(|\nom{F}|, \cdot )$ with domain $|\nom{F}|=\cF$.
     \item Define the injective equivariant map $\atom^{\nalg{F}}(a) = \lin{a}$, for all $a\in \A$;
        \item Define the equivariant map $\abs^{\nalg{F}}(a,\lin{g}) = \lin{[a]g}$ for all $a\in\A$ and all $g\in \F(\Sigma\cup \D)$;
        \item Define the equivariant map
        $f^{\nalg{F}}(\lin{g_1},\ldots,\lin{g_n}) = \lin{\tf{f}(g_1,\ldots,g_n)}$,  for $g_1,\ldots,g_n \in \F(\Sigma\cup \D)$ and each $\tf{f}:n$ in $\Sigma$.
\end{itemize}

\begin{theorem}\label{alemma:ground-algebra-model}
    $\nalg{F}$ is a nominal model of $\C$.
\end{theorem}

\begin{proof} The proof is standard and can be found in~\cite{arxiv/cairessantos2025}. 
     \begin{itemize}
         \item $\lin{\F}_\C$ is a nominal set by Lemma~\ref{alemma:ground-algebra-properties}(\ref{alemma:ground-algebra-nominal-set}).

         \item The equivariance of each map follows by the fact that $\pi\act \lin{g} = \lin{\pi\act g}$.

         \item It's easy to see that $\atom^{\nalg{F}}$ is injective because $\lin{a} = \{a\}$ for all $a\in\A$.

         \item The following condition is satisfied: $\new\atnew{c}. \newswap{a}{c} \act\abs^{\nalg{F}}(a,\lin{g}) = \abs^{\nalg{F}}(a,\lin{g})$ for all $a\in \A$ and all $\lin{g}\in\lin{\F}(\Sigma\cup\D)$. Indeed, suppose, by contradiction, that $\new \atnew{c}.\newswap{a}{c} \act \abs^{\nalg{F}}(a,\lin{g}) = \abs^{\nalg{F}}(a,\lin{g})$ does not  hold, that is, the set $D_1 = \{\atnew{c} \mid \newswap{a}{c} \act \abs^{\nalg{F}}(a,\lin{g}) =\abs^{\nalg{F}}(a,\lin{g})\}$ is finite. By Theorem~\ref{athm:supp-free-names}, we have $a\notin \supp{}{\abs^{\nalg{F}}(a,\lin{g})}$. Define $D_2 = \{a\}\cup\supp{}{\abs^{\nalg{F}}(a,\lin{g})}$.  Then, for any atom $\atnew{c_1}\notin D_2$, it follows that $\newswap{a}{c_1}\act \abs^{\nalg{F}}(a,\lin{g}) = \abs^{\nalg{F}}(a,\lin{g})$. Therefore, $\A\setminus D_2 \subseteq D_1$, which is a contradiction with the fact that $D_1$ is finite.

         \item  Given $g_1,g_2\in \F(\Sigma\cup\D)$. For each $\tf{f^C}$ in $\Sigma$, note that we have
         \begin{prooftree}
             \AxiomC{ }
             \RightLabel{(refl)}
             \UnaryInfC{$~\vdash g_1 \aeq{C} g_1$}
             \AxiomC{ }
             \RightLabel{(refl)}
             \UnaryInfC{$~\vdash g_2 \aeq{C} g_2$}
             \RightLabel{$(\frule{\faeq{C}}{\tf{f^C}})$}
             \BinaryInfC{$~\vdash \tf{f^C}(g_1,g_2) \aeq{C} \tf{f^C}(g_2,g_1)$}
         \end{prooftree}
         Thus, $\tf{f^C}(g_1,g_2) \sim \tf{f^C}(g_2,g_1)$ and hence $\lin{\tf{f^C}(g_1,g_2)} = \lin{\tf{f^C}(g_2,g_1)}$. Consequently, $f^{\C,\nalg{F}}(\lin{g_1},\lin{g_2}) = f^{\C,\nalg{F}}(\lin{g_2},\lin{g_1})$, proving the result.
    \end{itemize}
\end{proof}


$\nalg{F}$ will be called the {\em free-term model of $\C$}.

\subsubsection*{Useful auxiliary results.}
Below, we present some results from nominal set theory that will be useful:

\begin{itemize}
    \item If $\nom{X}$ is a nominal set, then the powerset $\pow{}{\nom{X}}$ with the action $\pi\act S = \{\pi\act x \mid x \in S\}$, is not necessarily a nominal set. However, the restriction of $\pow{}{\nom{X}}$,
    \[
         \pow{\tf{fs}}{\nom{X}} = \{S\subseteq \nom{X} \mid \text{$S$ is finitely supported}\},
    \]
    equipped with the same action, is a nominal set.

    \item Suppose $S$ is a set, all of whose elements have finite support. If $\bigcup \{\supp{}{x} \mid x\in S\}$ is finite then $\supp{}{S}$ exists and
    \(
        \supp{}{S} = \bigcup \{\supp{}{x} \mid x\in S\}.
    \)
    The proof can be found in~\cite{DBLP:journals/bsl/Gabbay11}.
\end{itemize}

\begin{lemma}\label{lemma:power-set-algebra}
    For all natural $n\geq 1$, the nominal set $\pow{\tf{fs}}{\A^n}$ is a model of $\C$ when equipped with the following structure:
\begin{enumerate}
    \item $\atom(a) = \{(a,\ldots,a)\}$ for all $a\in\A$

    \item $ \abs(a,S) = S\setminus\{x\in S\mid a\in\supp{}{x}\}$ for all $a\in\A$ and all $S\in \pow{\tf{fs}}{\A^n}$

    \item Every $\tf{f}:n$ is mapped to $f(S_1,\ldots,S_n) = S_1\cup\ldots\cup S_n$ for all $S_i\in  \pow{\tf{fs}}{\A^n}$.
\end{enumerate}
\end{lemma}

\begin{proof}
    It's not difficult to see that:
   \begin{itemize}
       \item all mappings are equivariant.
       \item $\atom$ is injective.
       \item $\abs$ satisfies $\new\atnew{c}.\newswap{a}{c}\act\abs(a,S) = \abs(a,S)$ for all $a\in\A$ and $S\in  \pow{\tf{fs}}{\A^n}$. This is a consequence of Pitts' equivalence and $a\notin \supp{}{\abs(a,S)}$ by construction.
       \item Each $\tf{f^C}:n$ in the signature is mapped to $f(S_1,S_2) = S_1\cup S_2$ and $f(S_1,S_2) = f(S_2,S_1)$  for all $S_1,S_2\in  \pow{\tf{fs}}{\A^n}$. Thus, it is indeed a model of $\C$.
   \end{itemize}
\end{proof}

\begin{lemma} \label{alemma:valid-fix-point-judgment}
The following hold:
\begin{enumerate}
     \item \label{alemma:domain-preservation} If $\Upsilon_{\catnew{c}} \vDash \pi\act X \aeq{C} X$, then $\dom{\pi}\subseteq \atm{\Upsilon_{\catnew{c}}|_X}$.
     \item \label{alemma:valid-pi-c-in-generated} If $\Upsilon_{\catnew{c}} \vDash \pi_{\catnew{c}}\act X \aeq{C} X$, then $\pi_{\catnew{c}}\in \Perm{\atm{(\Upsilon_{\catnew{c}}|_X)_{\fresh}}}$.
\end{enumerate}
\end{lemma}

\begin{proof}
     \begin{enumerate}
         \item If $\pi = \id$, then the result follows trivially. Assume that $\pi\neq \id$. Suppose, by contradiction, that there is an atom $a\in \dom{\pi}$ such that $a\notin\atm{\Upsilon_{\catnew{c}}|_X}$. Consider the algebra $\nalg{A} = \pow{\tf{fin}}{\A}$ and the valuation $\varsigma$ given by $\varsigma(X) = \{a\}$ and $\varsigma(Y) = \{b\}$ for all $Y\not\equiv X$, where $b$ is fresh. Then, it is not difficult to see that $\Int{\Upsilon_{\catnew{c}}}{\nalg{A}}{\varsigma}$ is valid. By hypothesis, this means that $\pi\act \Int{X}{\nalg{A}}{\varsigma} = \Int{X}{\nalg{A}}{\varsigma}$, i.e., $\pi(a) = a$, which in turn implies that $a\notin \dom{\pi}$, a contradiction.

        \item Let's start proving the following claims

        \begin{claim}[1]
            We claim that $\dom{\pi_{\catnew{c}}}\cap \supp{}{\Int{X}{\nalg{A}}{\varsigma}} = \emptyset$ for all $\catnew{c}\cap \supp{}{\Int{X}{\nalg{A}}{\varsigma}} = \emptyset$. In fact, $\Upsilon_{\catnew{c}} \vDash \pi_{\catnew{c}}\act X \aeq{C} X$ means that for all models $\nalg{A}$ and all valuations $\varsigma$,
        \begin{equation}\label{eq:pi-c-in-generated}
            \text{If $\Int{\Upsilon_{\catnew{c}}}{\nalg{A}}{\varsigma}$ is valid, then $\pi_{\catnew{c}}\act \Int{X}{\nalg{A}}{\varsigma} = \Int{X}{\nalg{A}}{\varsigma}$.}
        \end{equation}
        $\Int{\Upsilon_{\catnew{c}}}{\nalg{A}}{\varsigma}$ being valid, by definition, means that $\newc{c}{}. \pi'\act \Int{Y}{\nalg{A}}{\varsigma} = \Int{Y}{\nalg{A}}{\varsigma}$ holds for all $\pi'\fix{C} Y\in (\Upsilon_{\catnew{c}})_{\fresh}$. By generalized Pitts's equivalence (Lemma~\ref{lemma:pitts-eq-generalized}) we have, for all $\pi'\fix{C} Y\in (\Upsilon_{\catnew{c}})_{\fresh}$, that
        \begin{itemize}
            \item $\dom{\pnew{\pi'}{\catnew{c}}}\cap \supp{}{\Int{Y}{\nalg{A}}{\varsigma}} = \emptyset$ for all $\catnew{c}$ such that $\catnew{c}\cap(\bigcup\supp{}{\Int{Y}{\nalg{A}}{\varsigma}}) = \emptyset$. (note that $\pi' = \pnew{\pi'}{\catnew{c}}$ for all $\pi'\fix{C} Y\in (\Upsilon_{\catnew{c}})_{\fresh}$.)
        \end{itemize}
        By (\ref{eq:pi-c-in-generated}), we have $\pi_{\catnew{c}}\act \Int{X}{\nalg{A}}{\varsigma} = \Int{X}{\nalg{A}}{\varsigma}$ holds for all $\catnew{c}\cap\supp{}{\Int{X}{\nalg{A}}{\varsigma}} = \emptyset$. As a consequence of Lemma~\ref{lemma:generated-group}, we have $\dom{\pi_{\catnew{c}}}\cap \supp{}{\Int{X}{\nalg{A}}{\varsigma}} = \emptyset$.
        \end{claim}

        \begin{claim}[2]
            We claim that  $\dom{\pi_{\catnew{c}}}\subseteq \atm{(\Upsilon_{\catnew{c}}|_X)_{\fresh}}$.  Suppose, by contradiction, that there is an atom $a\in \dom{\pi_{\catnew{c}}}$ such that $a\notin \atm{(\Upsilon_{\catnew{c}}|_X)_{\fresh}}$. Then $a\in \atm{(\Upsilon_{\catnew{c}}|_X)_{\fix{C}}}\setminus{\catnew{c}}$ by item~\ref{alemma:domain-preservation}. Consider the valuation $\varsigma^*$ given by $\varsigma^*(X) = \atm{(\Upsilon_{\catnew{c}}|_X)_{\fix{C}}}\setminus{\catnew{c}}$ and $\varsigma^*(Y) = \{b\}$ for all $Y\not\equiv X$, where $b$ is fresh. Thus, $\supp{}{\Int{X}{\nalg{A}}{\varsigma}} = \atm{(\Upsilon_{\catnew{c}}|_X)_{\fix{C}}}\setminus{\catnew{c}}$ and $\supp{}{\Int{Y}{\nalg{A}}{\varsigma}} = \{b\}$ for all $Y\not\equiv X$. Moreover, it's not hard to see that $\Int{\Upsilon_{\catnew{c}}}{\pow{\tf{fin}}{\A}}{\varsigma^*}$ is valid. Therefore, $a \in \dom{\pi_{\catnew{c}}}\cap \supp{}{\Int{X}{\nalg{A}}{\varsigma}}$, which is a contradiction with Claim (1).
        \end{claim}


        As a consequence of Claim (2), we have that $\pi_{\catnew{c}}\in \Perm{\atm{(\Upsilon_{\catnew{c}}|_X)_{\fresh}}}$.
     \end{enumerate}
\end{proof}

The next lemma is the most challenging as it establishes a characterisation of the permutations on semantic judgements of the form $\Upsilon_{\catnew{c}} \vDash \pi\act X\aeq{C} X$, which by definition, are the judgements that are valid in every model and under every valuation: It must be the case that the permutation is generated by the permutations in the context.

\begin{lemma}[Characterisation of valid fixed-points]\label{alemma:valid-judge-pi-generated} \hfill

    If $\Upsilon_{\catnew{c}} \vDash \pi\act X\aeq{C} X$, then $\pi\in \PN{}{\Upsilon_{\catnew{c}}|_X}$.
\end{lemma}

\begin{proof}
If $\pi = \id$, then the result follows trivially. Assume that $\pi\neq \id$. Here are some useful information:

\begin{itemize}
    \item By generalized Pitts' equivalence (Lemma~\ref{lemma:pitts-eq-generalized}), from $\Upsilon_{\catnew{c}} \vDash \pi\act X\aeq{C} X$, we have $ \Upsilon_{\catnew{c}} \vDash \pi_{\catnew{c}}\act X\aeq{C} X$ and $\Upsilon_{\catnew{c}} \vDash \pi_{\neg\catnew{c}}\act X\aeq{C} X$.

    \item By Lemma~\ref{alemma:valid-fix-point-judgment}(\ref{alemma:valid-pi-c-in-generated}), we have $\pi_{\catnew{c}}\in \Perm{\atm{(\Upsilon_{\catnew{c}}|_X)_{\fresh}}}$.

    \item By Lemma~\ref{alemma:valid-fix-point-judgment}(\ref{alemma:domain-preservation}),  $\dom{\pi_{\neg\catnew{c}}} \subseteq\atm{(\Upsilon_{\catnew{c}}|_X)_{\fix{C}}}\setminus{\catnew{c}}$.

    \item To simplify the proof we abbreviate $\PN{}{\Upsilon_{\catnew{c}}|_X} = \Perm{(\Upsilon_{\catnew{c}}|_X)_{\fresh}}\pair{(\Upsilon_{\catnew{c}}|_X)_{\fix{C}}}$.
\end{itemize}

Then, it is sufficient to prove that $\pi_{\neg\catnew{c}} \in \pair{(\Upsilon_{\catnew{c}}|_X)_{\fix{C}}}$. Let $(a_1,\ldots,a_n)$ be a list enumerating all atoms being mentioned in all permutations of $(\Upsilon_{\catnew{c}}|_X)_{\fix{C}}$. Then $(a_1,\ldots,a_n) \in \A^n$ where $\A^n$ is the $n$-ary Cartesian power of the nominal set $\A$. The set $\A^n$ is a nominal set when equipped with the usual pointwise action. The orbit of $(a_1,\ldots,a_n)$ over $\pair{(\Upsilon_{\catnew{c}}|_X)_{\fix{C}}}$, is the set ${\cal O} = \{\pi\act (a_1,\ldots,a_n) \mid \pi\in \pair{(\Upsilon_{\catnew{c}}|_X)_{\fix{C}}}\}$ .
Since every element of ${\cal O}$ has finite support and $\bigcup_{x\in {\cal O}} \supp{}{x}  = \{a_1,\ldots,a_n\}$, it follows that ${\cal O}$ itself has finite support and $\supp{}{{\cal O}} = \bigcup_{x\in {\cal O}} \supp{}{x}$. This proves that ${\cal O}\in \pow{\tf{fs}}{\A^n}$. Take the model $\nalg{A}$ as $\pow{\tf{fs}}{\A^n}$ and the valuation
\[
    \varsigma(Y) = \left\{\begin{array}{lc}
        {\cal O}, & Y\equiv X \\
        \{(d_Y,\ldots,d_Y)\} & Y\not\equiv X
    \end{array}\right.
\]
where $d_Y$ is a fresh atom for $\Upsilon_{\catnew{c}}|_Y$.

\begin{itemize}
    \item For every $\rho \fix{C} Y \in \Upsilon_{\catnew{c}}$ such that $Y\not\equiv X$, it holds that $\dom{\rho} \cap \supp{}{\varsigma(Y)} = \emptyset$. Consequently, by the definition of support, we have $\rho \act \varsigma(Y) = \varsigma(Y)$ and so $\newc{c}{}. \rho \act \varsigma(Y) = \varsigma(Y).$

    For  those $\rho\fix{C} Y\in (\Upsilon_{\catnew{c}})_{\fresh}$, we have that $\rho_{\catnew{c}} = \rho$, and given that $\rho \act \varsigma(Y) = \varsigma(Y)$ is true for all $\catnew{c}$ satisfying $\catnew{c} \cap \bigcup_{Y} \supp{}{\varsigma(Y)} = \emptyset$, it follows, by the generalized Pitts' equivalence (Lemma~\ref{lemma:pitts-eq-generalized}), that $\newc{c}{}.\rho \act \varsigma(Y) = \varsigma(Y)$.

    \item Now, for those $\rho\fix{C} X\in \Upsilon_{\fresh}$, we have that $\dom{\rho}\cap \supp{}{\cal O} = \emptyset$ and hence by the same reason, we have $\newc{c}{}.\rho\act \varsigma(X) = \varsigma(X)$.

    \item By the definition of ${\cal O}$, we have that $\rho \act {\cal O} = {\cal O}$ holds for all $(\Upsilon_{\catnew{c}}|_X)_{\fix{C}}$. In particular, $\rho \act {\cal O} = {\cal O}$ for all permutation $\rho$ such that $\rho\fix{C} X\in (\Upsilon_{\catnew{c}}|_X)_{\fix{C}}$. So $\newc{c}{}.\rho\act \varsigma(X) = \varsigma(X)$.
\end{itemize}

Consequently, $\Int{\Upsilon_{\catnew{c}}}{\nalg{A}}{\varsigma}$ is valid.

From $\Upsilon \vDash \pi_{\neg\catnew{c}}\act X \aeq{C} X$, we have $\pi_{\neg\catnew{c}}\act {\cal O} = {\cal O}$. Then for all element $x\in{\cal O}$ there is an element $y\in{\cal O}$ such that $\pi_{\neg\catnew{c}}\act x = y$. By the definition of ${\cal O}$, $x = \pi_x\act (a_1,\ldots,a_n)$ and $y = \pi_y\act (a_1,\ldots,a_n)$ where $\pi_x,\pi_y\in \pair{(\Upsilon_{\catnew{c}}|_X)_{\fix{C}}}$. Hence $ \pi_{\neg\catnew{c}}\act x = y$ implies $\pi_{\neg\catnew{c}}\act (\pi_x\act (a_1,\ldots,a_n)) = \pi_y\act (a_1,\ldots,a_n)$, leading to
\[
    (\pi_y^{-1}\circ\pi_{\neg\catnew{c}}\circ\pi_x)\act (a_1,\ldots,a_n) = (a_1,\ldots,a_n).
\]
This implies $(\pi_y^{-1}\circ\pi_{\neg\catnew{c}}\circ\pi_x)(a_i) = a_i$ for all $i=1,\ldots,n$ and thus $\dom{\pi_y^{-1}\circ\pi_{\neg\catnew{c}}\circ\pi_x}\cap\{a_1,\ldots,a_n\} = \emptyset.$

As observed at the beginning, we have $\dom{\pi_{\neg\catnew{c}}}\subseteq \atm{(\Upsilon_{\catnew{c}}|_X)_{\fix{C}}}\setminus\catnew{c}$. Then the only possibility is $\pi_y^{-1}\circ\pi_{\neg\catnew{c}}\circ\pi_x = \id\in \pair{(\Upsilon_{\catnew{c}}|_X)_{\fix{C}}}$. Therefore,
\[
    \pi_{\neg\catnew{c}} = \pi_y\circ(\pi_y^{-1}\circ\pi_{\neg\catnew{c}}\circ\pi_x)\circ \pi_x^{-1} \in \pair{(\Upsilon_{\catnew{c}}|_X)_{\fix{C}}},
\]
Since $\pi = \pi_{\catnew{c}}\circ\pi_{\neg\catnew{c}}$, we obtain $\pi\in \Perm{(\Upsilon_{\catnew{c}}|_X)_{\fresh}}\pair{(\Upsilon_{\catnew{c}}|_X)_{\fix{C}}}$.
\end{proof}

\begin{lemma}\label{alemma:var-judge-valid}
    If $\Upsilon_{\catnew{c}} \vDash t \aeq{C} u$ and $t$ is a suspension, i.e., $t\equiv \pi_1\act X$, then $u$ must be a suspension, i.e.,  $u\equiv \pi_2\act X$, for some $\pi_2$.
\end{lemma}

\begin{proof}
        Just observe that for each $u \not\equiv \pi_2\act X$, it is possible to find an algebra $\nalg{A}$ and a valuation $\varsigma$ that validate the context, but $\Int{\pi_1\act X}{\nalg{A}}{\varsigma} \neq \Int{u}{\nalg{A}}{\varsigma}$. Here, we will analyse only the case where $u \equiv \pi_2\act Y$.

        Take $a_1,a_2,a_3\notin \atm{\Upsilon_{\catnew{c}},\pi_1,\pi_2}$. Consider the algebra $\nalg{A} = \pow{\tf{fin}}{\A}$ with the valuation $\varsigma$ defined by $\varsigma(X) = \{a_1,a_2\}, \varsigma(Y) = \{a_1,a_3\}$, and $\varsigma(Z) = \emptyset$ for all $Z\not\equiv  X,Y$. Then $\Int{\Upsilon_{\catnew{c}}}{\nalg{A}}{\varsigma}$ is clearly valid but
        \[
           \pi_1\act \Int{X}{\nalg{A}}{\varsigma} = \{a_1,a_2\} \neq \{a_1,a_3\} =  \pi_2\act \Int{Y}{\nalg{A}}{\varsigma}.
       \]
\end{proof}

\subsubsection{Proof of Completeness.}

The rest of the section will be based on the assumption that $\Upsilon_{\catnew{c}}\vDash t \aeq{C} u$.

\begin{definition}
    Let $\mathcal{A}$ be the atoms in $\atm{\Upsilon_{\catnew{c}},t,u}$. Let $\mathcal{X}$ be the variables mentioned anywhere in $\Upsilon_{\catnew{c}}, t$ or $u$.
    \begin{enumerate}
        \item For each $X\in\mathcal{X}$ pick the following data:
    \begin{itemize}
        \item let ${\cal A}_X$ be the set of atoms $a\in{\cal A}\setminus\catnew{c}$ such that there is no $\pi\fix{C} X\in(\Upsilon_{\catnew{c}}|_X)_{\fresh}$ with $a\in \dom{\pi}$.
        \item let $a_{X_1},\ldots,a_{X_{k_X}}$ be the atoms in ${\cal A}_X$ in some arbitrary but fixed order.
        \item let $\tf{d}_X:k_X$ be a fresh term-former.
    \end{itemize}

    \item For each $X \notin \mathcal{X}$, let $\tf{d}_X : 0$ be a term-former. Let $\sf{D}$ be the set of all $\tf{d}$'s (so for each $X\in\V$ we have a $\tf{d}_X\in \sf{D}$).
    \end{enumerate}
\end{definition}

\begin{definition}
    Let $\sigma$ be the following substitution:
    \[
        X\sigma \equiv \left\{\begin{array}{ll}
            \tf{d}_X(a_{X_1},\ldots,a_{X_{k_X}}) &  (X\in\mathcal{X})\\
            \tf{d}_X() & (X\notin\mathcal{X})
        \end{array}\right.
    \]
\end{definition}

Thus, $\sigma$ maps all variables mentioned in $t$ or $u$ to an appropriate ground term such that the support we know.

Let $\Sigma^+ = \Sigma\cup\sf{D}$ and consider $\F(\Sigma^+)$ the set of free (nominal) terms. For each $X\in{\cal X}$, define $R_X$ as the set of all identities of the following form:
\[
    \tf{d}_X(a_{X_1},\ldots,a_{X_{k_X}}) \approx \pi\act \tf{d}_X(a_{X_1},\ldots,a_{X_{k_X}}),
\]
for all $\pi\fix{C} X\in (\Upsilon_{\catnew{c}}|_X)_{\fix{C}}$. Let $R := \bigcup_{X\in{\cal X}} R_X$. Denote by ${\cal R}$ the equivariant equivalence closure of $R$. Define the relation ${\cal E}$ by
\[
    (g_1,g_2)\in {\cal E} \text{ iff } \vdash g_1\aeq{C} g_2 \text{ or } (g_1,g_2)\in {\cal R}.
\]

\begin{lemma}
    ${\cal E}$ is an equivariant equivalence relation on $\lin{\F}(\Sigma^+)$.
\end{lemma}

\begin{proof}
    \begin{enumerate}
        \item {\em Reflexivity}, {\em Symmetry} and {\em Equivariance} follows because both relations $\aeq{C}$ and ${\cal R}$ are reflexive, symmetric and equivariant.

        \item {\em Transitivity}:  Suppose
        $(g_1,g_2)\in {\cal E}$ and $(g_2,g_3)\in {\cal E}$.
        \begin{enumerate}
            \item If$~\vdash g_1\aeq{C} g_2$, then we have two cases:

            \begin{itemize}
                \item Case$~\vdash g_2\aeq{C} g_3$. In this case,$~\vdash g_1\aeq{C} g_3$ follows by the fact that $\aeq{C}$ is transitive and so $(g_1,g_3)\in {\cal E}$.

                \item Case $(g_2,g_3)\in {\cal R}$. In this case, $g_2$ is of the form
                \[
                    g_2 \equiv \tf{d}_X(a_{X_1}',\ldots,a_{X_{k_X}}')
                \]
                which, by$~\vdash g_1\aeq{C} g_2$ forces $g_1 \equiv g_2$. Thus, $(g_1,g_3)\in {\cal E}$.
            \end{itemize}

            \item  If $(g_1,g_2)\in {\cal R}$, then we have two cases:\begin{itemize}
                \item Case$~\vdash g_2\aeq{C} g_3$. This case is similar to the second bullet of the previous case (a).

                \item Case $(g_2,g_3)\in {\cal R}$. In this case, $(g_1,g_3)\in{\cal R}$ follows by the fact that ${\cal R}$ is transitive and hence $(g_1,g_3)\in {\cal E}$.
            \end{itemize}
        \end{enumerate}
    \end{enumerate}
\end{proof}

As a consequence, the set $\F(\Sigma^+)/{{\cal E}}$ forms a nominal set, which we denote by $\lin{\F}_{\cal E}$. This set satisfies properties analogous to those stated in Lemma~\ref{alemma:ground-algebra-properties}. The proof follows a very similar structure, with the only difference lying in item~\ref{alemma:free-names-preserve}.

\begin{lemma}
    If $g'\in \lin{g}$, then $\tf{fn}(g) = \tf{fn}(g')$.
\end{lemma}

\begin{proof}
    Suppose  $g'\in \lin{g}$. Then $(g,g')\in{\cal E}$, so$~\vdash g\aeq{C} g'$ or $(g,g')\in {\cal R}$. If$~\vdash g\aeq{C} g'$, then the proof is the same as the one in Lemma~\ref{alemma:ground-algebra-properties}(\ref{alemma:free-names-preserve}). If $(g,g')\in {\cal R}$, then we have two possibilities:

    \begin{itemize}
        \item $g \equiv  \tf{d}_X(a_{X_1},\ldots,a_{X_{k_X}})$.

        In this case, $g' \equiv g$ or $g' \equiv \pi \act \tf{d}_X(a_{X_1}, \ldots, a_{X_{k_X}})$ for some permutation $\pi$ such that $\pi\fix{C} X \in (\Upsilon_{\catnew{c}}|_X)_{\fix{C}}$. In both scenarios, $\tf{fn}(g) = \tf{fn}(g')$, as $\pi$ merely rearranges some of the atoms in $a_{X_1}, \ldots, a_{X_{k_X}}$.

        \item  $g \equiv \tf{d}_X(a_{X_1}',\ldots,a_{X_{k_X}}')$,

        In this case,
        \[
            g \equiv \mu\act \tf{d}_X(a_{X_1},\ldots,a_{X_{k_X}})
        \]
        where $\mu = (a_{X_1} \ a_{X_1}')\ldots(a_{X_{k_X}} \ a_{X_{k_X}}')$. Then $g' \equiv g$ or 
        \[
            g'\equiv \mu\act(\pi\act \tf{d}_X(a_{X_1},\ldots,a_{X_{k_X}}))
        \]
        for some permutation $\pi$ such that $\pi\fix{C} X \in (\Upsilon_{\catnew{c}}|_X)_{\fix{C}}$. Then
        \begin{align*}
            \tf{fn}(g) &= \tf{fn}(\mu\act \tf{d}_X(a_{X_1},\ldots,a_{X_{k_X}}))\\
            &= \mu\act\tf{fn}( \tf{d}_X(a_{X_1},\ldots,a_{X_{k_X}})\\
            &= \mu\act\tf{fn}(\pi\act \tf{d}_X(a_{X_1},\ldots,a_{X_{k_X}}))\\
            &= \tf{fn}(\mu\act(\pi\act \tf{d}_X(a_{X_1},\ldots,a_{X_{k_X}})))\\
            &= \tf{fn}(g').
        \end{align*}
    \end{itemize}
\end{proof}


Additionally, the set $\lin{\F}_{\cal E}$, equipped with the same equivariant maps defined in Subsection~\ref{app:free-models}, forms an algebra that is a model of $\C$. We denote this algebra by $\nalg{F}_{\cal E}$.

Define $\varsigma^*$ on $\nalg{F}_{\cal E}$ by $\varsigma^*(X) =\lin{X\sigma}$ for every $X\in\V$.

\begin{lemma}\label{alemma:interpretation}
The following hold:
\begin{enumerate}
\item \label{alemma:substitution-interpretation}
    $\Int{t}{\nalg{F}_{\cal E}}{\varsigma^*} = \lin{t\sigma}$ and $\Int{u}{\nalg{F}_{\cal E}}{\varsigma^*} = \lin{u\sigma}$.
\item \label{alemma:context-valid}
    $\Int{\Upsilon_{\catnew{c}}}{\nalg{F}_{\cal E}}{\varsigma^*}$ is valid.
\end{enumerate}
\end{lemma}


\begin{proof}
    \begin{enumerate}
        \item Direct by induction on the structure of $t$ and $u$. To illustrate, consider the case $t \equiv \pi\act X$.
        \begin{align*}
            \Int{\pi\act X}{\nalg{F}_{\cal E}}{\varsigma^*} = \pi\act \Int{X}{\nalg{F}_{\cal E}}{\varsigma^*} = \pi\act\varsigma(X) = \pi\act \lin{X\sigma} = \lin{\pi\act(X\sigma)} = \lin{(\pi\act X)\sigma}.
        \end{align*}

        \item Let $\pi\fix{C} X\in \Upsilon_{\catnew{c}}$. Then $X\in {\cal X}$.
    \begin{itemize}
        \item Suppose $\pi\fix{C} X\in (\Upsilon_{\catnew{c}})_{\fresh}$. Let $\catnew{c}$ be such that  $\catnew{c}\cap\supp{}{ \Int{X}{\nalg{F}_{\cal E}}{\varsigma^*}} = \emptyset$. By construction $\dom{\pi}\cap{\cal A}_X = \emptyset$   so $\dom{\pi}\cap \supp{}{\varsigma^*(X)} = \emptyset$ which, by the definition of support, gives us $\pi\act \varsigma^*(X) \equiv \varsigma^*(X)$. Consequently. we have $\pi\act \lin{\varsigma^*(X)} = \lin{\varsigma^*(X)}$, that is, $\pi\act \Int{X}{\nalg{F}_{\cal E}}{\varsigma^*} = \Int{X}{\nalg{F}_{\cal E}}{\varsigma^*}$. This proves that the set $\{\catnew{c} \mid  \pi\act \Int{X}{\nalg{F}_{\cal E}}{\varsigma^*} = \Int{X}{\nalg{F}_{\cal E}}{\varsigma^*}\}$ is cofinite. Thus, $\newc{c}{}.\pi\act \Int{X}{\nalg{F}_{\cal E}}{\varsigma^*} = \Int{X}{\nalg{F}_{\cal E}}{\varsigma^*}$.

        \item Suppose $\pi\fix{C} X\in (\Upsilon_{\catnew{c}})_{\fix{C}}$.  By construction,
        \begin{align*}
             \lin{\tf{d}_X(a_{X_1},\ldots,a_{X_{k_X}})} &= \lin{\pi\act\tf{d}_X(a_{X_1},\ldots,a_{X_{k_X}})}\\
             &= \pi\act \lin{\tf{d}_X(a_{X_1},\ldots,a_{X_{k_X}})},
        \end{align*}
        so $\pi\act \Int{X}{\nalg{F}_{\cal E}}{\varsigma^*} = \Int{X}{\nalg{F}_{\cal E}}{\varsigma^*}$ holds. Since this holds for all $\catnew{c}$ such that  $\catnew{c}\cap\supp{}{ \Int{X}{\nalg{F}_{\cal E}}{\varsigma^*}} = \emptyset$, the result follows.
    \end{itemize}
    \end{enumerate}
\end{proof}

Let $\Pi$ be a proof of the judgment$~\vdash t\sigma \aeq{C} u\sigma$. By the definition of $\sigma$, the terms $t\sigma$ and $u\sigma$ belong to $\F(\Sigma^+)$, and the subterm $X\sigma$ occurs in the position where $X$ occurs in $t$ or $u$.  Now, we aim to find a method to reverse $t\sigma$ or $u\sigma$ into terms in $\Sigma$, in such a way that it allows us to recover a derivation of $\Upsilon_{\catnew{c}} \vdash t\aeq{C} u$.

\begin{definition}
    Let $\mathcal{A}^+$ be $\mathcal{A}$ extended with:
    \begin{itemize}
        \item a set $\mathcal{C}$ of atoms mentioned anywhere in $\Pi$ (that were not already in $\mathcal{A}$).

        \item a set $\mathcal{B}$ of fresh atoms, in bijection with $\mathcal{A}$ — for convenience, we fix a bijection and write $b_{X_i}$ for the atom corresponding under that bijection with $a_{X_i}$ — and


        \item one fresh atom $e$ (so $e$ does not occur in $\mathcal{A}, \mathcal{C}$ or $\mathcal{B}$).
    \end{itemize}
    \[
        {\cal A}^+ = {\cal A\cup C\cup B}\cup\{e\}.
    \]
\end{definition}

Let ${\cal B}_X$ be the set of $b_{X_i}$'s corresponding with $a_{X_i}$'s. Denote by $({\cal A}_X \ {\cal B}_X)$ the permutation $(a_{X_1} \ b_{X_1})\circ \ldots\circ (a_{X_{k_X}} \ b_{X_{k_X}})$.


\begin{definition}
    Define $\Upsilon^+_{\atnew{\pvec{c}'}}$ to be $\Upsilon_{\catnew{c}}$ extended with the following constraints:
    \begin{itemize}
        \item $\atnew{\pvec{c}'} = \catnew{c}\cup\{c^*\}$ where $\atnew{c^*}$ is a fresh name.

        \item $(\atnew{c_1^*} \ \atnew{c_2^*} \ \ldots \ \atnew{c_n^*} \ e \ \atnew{c^*})\fix{C} X$ for all $X\in{\cal X}$, where ${\cal C} = \{\atnew{c_1^*},\ldots,\atnew{c_n^*}\}$.

        \item $\pi^{({\cal A}_X \ {\cal B}_X)}\fix{C} Y$ for all $\pi\fix{C} Y\in \Upsilon_{\fix{C}}$.
    \end{itemize}
\end{definition}

\begin{definition}
    For the rest of this subsection let $g$ and $h$ range over ground terms in $\Sigma^+$ that mention only atoms from $\mathcal{A}^+\setminus\mathcal{B}\cup\{e\}$. Define an \emph{inverse mapping} $(-)^{-1}$ from such ground terms to terms in $\Sigma$ inductively as follows:
    $$
    \begin{array}{rcl}
    a^{-1}&\equiv& a \\
    \tf{d}_X()^{-1} &\equiv & e\\
    \tf{f}(g_1,\ldots,g_n)^{-1}&\equiv &\tf{f}(g_1^{-1},\ldots,g_n^{-1}) \\
    ([a]g)^{-1}&\equiv& [a]g^{-1}\\
    \tf{d}_X(a_{X_1}', \ldots, a_{X_{k_{\scalebox{.4}{$X$}}}}')^{-1} &\equiv& \pi_X(a_{X_1}', \ldots, a_{X_{k_{\scalebox{.4}{$X$}}}}')\act X
    \end{array}
    $$
    where $\pi_X(a_{X_1}', \ldots, a_{X_{k_{\scalebox{.4}{$X$}}}}') = (a_{X_1}' \ b_{X_1})\circ \ldots\circ (a_{X_{k_{\scalebox{.4}{$X$}}}}' \ b_{X_{k_{\scalebox{.4}{$X$}}}})\circ (b_{X_1} \ a_{X_1})\circ \ldots\circ (b_{X_{k_{\scalebox{.4}{$X$}}}} \ a_{X_{k_{\scalebox{.4}{$X$}}}}).$
\end{definition}

\begin{lemma}\label{alemma:d-X}
    $\tf{d}_{X}(a_{X_1}, \ldots, a_{X_{k_X}})^{-1} \equiv X$.
\end{lemma}

\begin{proof}
    Consequence of $\pi_X(a_{X_1}, \ldots, a_{X_{k_X}}) = \id$.
\end{proof}

The inverse mapping is equivariant (for the terms we care about):

\begin{lemma}\label{alemma:equivariance-for-completeness}
    \sloppy{If $\pi\in \bigcap_{X\in\var{g^{-1}}}\Perm{\atm{(\Upsilon_{\atnew{\pvec{c}'}}|_X)_{\fresh}}\cup{\cal C}}\alert{\circ}\pair{\perm{}{(\Upsilon_{\atnew{\pvec{c}'}}|_X)_{\fix{C}}}}$, then  $\Upsilon^+_{\atnew{\pvec{c}'}} \vdash  (\pi\act g)^{-1} \aeq{C} \pi\act g^{-1}$.}
\end{lemma}

\begin{proof}
The proof is by induction on the structure of $g$. The only non-trivial case is when $g \equiv \tf{d}_X(a_{X_1}',\ldots,a_{X_{k_X}}')$. To prove $\Upsilon^+_{\atnew{\pvec{c}'}} \vdash  (\pi\act g)^{-1} \aeq{C} \pi\act g^{-1}$, we must show that $(\pi^{-1})^{({\cal A}_X \ {\cal B}_X)^{-1}} \in \PN{}{\Upsilon^+_{\atnew{\pvec{c}'}}|_X} $. This is equivalent to prove that $\pi^{({\cal A}_X \ {\cal B}_X)} \in \PN{}{\Upsilon^+_{\atnew{\pvec{c}'}}|_X}$ because $({\cal A}_X \ {\cal B}_X)^{-1} = ({\cal A}_X \ {\cal B}_X)$. In this case, $\var{g^{-1}} = \{X\}$ and so $\pi\in \Perm{\atm{(\Upsilon_{\atnew{\pvec{c}'}}|_X)_{\fresh}}\cup{\cal C}}\alert{\circ}\pair{\perm{}{(\Upsilon_{\atnew{\pvec{c}'}}|_X)_{\fix{C}}}}$. To simply notation, let's call it just $\Perm{(\Upsilon_{\atnew{\pvec{c}'}}|_X)_{\fresh}\cup{\cal C}}\alert{\circ}\pair{(\Upsilon_{\atnew{\pvec{c}'}}|_X)_{\fix{C}}}$. Applying $({\cal A}_X \ {\cal B}_X)$ on both sides, we get $\pi^{({\cal A}_X \ {\cal B}_X)}\in  \Perm{(\Upsilon_{\atnew{\pvec{c}'}}|_X)_{\fresh}\cup{\cal C}}\alert{\circ}\pair{(\Upsilon_{\atnew{\pvec{c}'}}^{({\cal A}_X \ {\cal B}_X)}|_X)_{\fix{C}}}$. Since
\begin{itemize}
    \item $\PN{}{\Upsilon^+_{\atnew{\pvec{c}'}}|_X} =  \Perm{(\Upsilon^+_{\atnew{\pvec{c}'}}|_X)_{\fresh}\cup{\cal C}}\alert{\circ}\pair{(\Upsilon_{\atnew{\pvec{c}'}}^+|_X)_{\fix{C}}}$
    \item $\Perm{(\Upsilon_{\atnew{\pvec{c}'}}|_X)_{\fresh}\cup{\cal C}} \subseteq \Perm{(\Upsilon^+_{\atnew{\pvec{c}'}}|_X)_{\fresh}\cup{\cal C}}$
    \item $\pair{(\Upsilon_{\atnew{\pvec{c}'}}^{({\cal A}_X \ {\cal B}_X)}|_X)_{\fix{C}}} \subseteq \pair{(\Upsilon_{\atnew{\pvec{c}'}}^+|_X)_{\fix{C}}}$
\end{itemize}
it follows that $\pi^{({\cal A}_X \ {\cal B}_X)} \in  \PN{}{\Upsilon^+_{\atnew{\pvec{c}'}}|_X}$.
\end{proof}


\begin{lemma}\label{alemma:inversion-properties}
The following hold:
\begin{enumerate}
\item \label{alemma:substitution-inverse-fixed-point}
    $\Upsilon^+_{\atnew{\pvec{c}'}}\vdash (t\sigma)^{-1} \aeq{C} t$ and $\Upsilon^+_{\atnew{\pvec{c}'}}\vdash (u\sigma)^{-1} \aeq{C} u$.
\item \label{alemma:ground-derivation-implies-derivation}
    If $~\vdash t\sigma \aeq{C} u\sigma$, then $\Upsilon_{\catnew{c}} \vdash t \aeq{C} u$.
\end{enumerate}
\end{lemma}

\begin{proof}
    \begin{enumerate}
        \item The proof is by induction on $t$. The only interesting case is when $t\equiv \pi_1\act X$. First note that

\begin{itemize}
    \item $X\sigma \in \F(\Sigma^+)$, and $X\sigma$ only involves atoms from $\mathcal{A}^+ \setminus \mathcal{B} \cup \{e\}$.

    \item By Lemma~\ref{alemma:var-judge-valid}, $t \equiv \pi_2 \act X$. Since $\Upsilon_{\catnew{c}} \vDash \pi_1 \act X \aeq{C} \pi_2 \act X$ if and only if $\Upsilon_{\catnew{c}} \vDash (\pi_2^{-1} \circ \pi_1) \act X \aeq{C} \pi_2 \act X$, we may assume w.l.o.g. that $s \equiv \pi \act X$ and $t \equiv X$.

    \item By Lemma~\ref{alemma:d-X}, $(X\sigma)^{-1} \equiv X$ and so $\var{(X\sigma)^{-1}} = \{X\}$.

    \item By Lemma~\ref{alemma:valid-judge-pi-generated}, we get
    \[
        \pi \in\PN{}{\Upsilon_{\catnew{c}}|_X} \subseteq \Perm{\atm{(\Upsilon_{\atnew{\pvec{c}'}}|_X)_{\fresh}}\cup {\cal C}}\alert{\circ}\pair{\perm{}{(\Upsilon_{\atnew{\pvec{c}'}}|_X)_{\fix{C}}}}
    \]
\end{itemize}

Applying Lemma~\ref{alemma:equivariance-for-completeness}, we conclude that $\Upsilon^+_{\atnew{\pvec{c}'}} \vdash (\pi \act (X\sigma))^{-1} \aeq{C} \pi \act (X\sigma)^{-1}$. Finally, by applying Lemma~\ref{alemma:d-X} once more, and using that $(\pi\act X)\sigma \equiv \pi\act(X\sigma)$, we derive $\Upsilon^+_{\atnew{\pvec{c}'}} \vdash ((\pi \act X)\sigma)^{-1} \aeq{C} \pi \act X$.

\item Firstly, let's prove the following claim: If $~\vdash s\sigma \aeq{C} t\sigma$ then $\Upsilon^+_{\atnew{\pvec{c}'}} \vdash (s\sigma)^{-1} \aeq{C} (t\sigma)^{-1}$.


     The proof is by induction on the last rule used to obtain the derivation $~\vdash s\sigma \aeq{C} t\sigma$.

         \begin{itemize}
             \item If the last rule applied is $(\frule{\faeq{C}}{a})$, then by Inversion (Theorem~\ref{thm:miscellaneous}(\ref{thm:inversion})), $s\sigma \equiv a \equiv t\sigma$. Consequently, $(a\sigma)^{-1} \equiv a \equiv (t\sigma)^{-1}$. Therefore, the derivation $\Upsilon^+_{\atnew{\pvec{c}'}} \vdash (a\sigma)^{-1} \aeq{C} (a\sigma)^{-1}$ follows by rule $(\frule{\faeq{C}}{a})$.

             \item If the last rule applied is $(\frule{\faeq{C}}{\tf{f}})$, then by Inversion (Theorem~\ref{thm:miscellaneous}(\ref{thm:inversion})), $s\sigma \equiv \tf{f}(g_1,\ldots,g_n)$, $t\sigma \equiv \tf{f}(g_1',\ldots,g_n')$,
              and $\vdash g_i \aeq{C} g_i'$ for all $i = 1,\ldots,n$.
              By the construction of $\sigma$, it forces $s \equiv \tf{f}(s_1,\ldots,s_n)$ and $t \equiv \tf{f}(t_1,\ldots,t_n)$ because the substitution does not  change the structure of the term $s$ and $t$. Thus, $s_i\sigma \equiv g_i$ and $t_i\sigma \equiv g_i'$ for all $i=1,\ldots,n$. Then, for each $i=1,\ldots,n$, the derivation $\vdash g_i \aeq{C} g_i'$ become $\vdash s_i\sigma \aeq{C} t_i\sigma$. By induction, we have that  $\Upsilon^+_{\atnew{\pvec{c}'}} \vdash (s_i\sigma)^{-1} \aeq{C} (t_i\sigma)^{-1}$. Then, by an application of the rule $(\frule{\faeq{C}}{\tf{f}})$, we obtain $\Upsilon^+_{\atnew{\pvec{c}'}} \vdash (\tf{f}(s_1,\ldots,s_n)\sigma)^{-1} \aeq{C} (\tf{f}(t_1,\ldots,t_n)\sigma)^{-1}$.

              \item If the last rule applied is $(\frule{\faeq{C}}{\tf{f^C}})$, then the it follows similarly to the previous case.

               In this case, by Inversion (Theorem~\ref{thm:miscellaneous}(\ref{thm:inversion})), $s\sigma \equiv \tf{f^C}(g_0,g_1)$ and $t\sigma \equiv \tf{f^C}(g_0',g_1')$, and
               \begin{itemize}
            \item  either $\vdash g_0 \aeq{C} g_0'$ and $\vdash g_1 \aeq{C} g_1'$

            This case is analogous to the case of the rule $(\frule{\faeq{C}}{\tf{f}})$.

            \item  or $\vdash g_0 \aeq{C} g_1'$ and $\vdash g_1 \aeq{C} g_0'$.

            This case is also analogous to the case of the rule $(\frule{\faeq{C}}{\tf{f}})$, but we will prove it anyway.

            By the construction of $\sigma$, we conclude that $s\equiv \tf{f^C}(s_0,s_1)$ and $t\equiv\tf{f^C}(t_0,t_1)$ and, for $i=0,1$, $s_i\sigma \equiv g_i$ and $t_i\sigma \equiv g_i'$. Thus, $\vdash g_0 \aeq{C} g_1'$ and and $\vdash g_1 \aeq{C} g_0'$ becomes, respectively, $\vdash s_0\sigma \aeq{C} t_1\sigma$ and $\vdash s_1\sigma \aeq{C} t_0\sigma$. The result follows by rule $(\frule{\faeq{C}}{\tf{f^C}})$.
         \end{itemize}


          \item If the last rule applied is $(\frule{\faeq{C}}{[a]})$, then by Inversion (Theorem~\ref{thm:miscellaneous}(\ref{thm:inversion})), $s\sigma \equiv [a]g_1, t\sigma \equiv [a]g_2$, and $~\vdash g_1 \aeq{C} g_2$. Again, the definition of $\sigma$ forces $s \equiv [a]s'$ and $t\equiv [a]t'$, which in turn yields $s'\sigma \equiv g_1$ and $t'\sigma \equiv g_2$. Thus, $~\vdash g_1 \aeq{C} g_2$ becomes $~\vdash s'\sigma \aeq{C} t'\sigma$. By induction, $\Upsilon^+_{\atnew{\pvec{c}'}}\vdash s'\aeq{C} t'$ and the result follows by rule $(\frule{\faeq{C}}{[a]})$.

         \item If the last rule applied is $(\frule{\faeq{C}}{ab})$, then by Inversion (Theorem~\ref{thm:miscellaneous}(\ref{thm:inversion})), $ s\sigma \equiv [a]g_1, t\sigma \equiv [b]g_2$, and $~\vdash \newswap{a}{c_1}\act g_1 \aeq{C} \newswap{b}{c_1}\act g_2$ where $\atnew{c_1}\notin a,b,g_1,g_2,\atnew{\pvec{c}'}$.

         The definition of $\sigma$ forces $s \equiv [a]s'$ and $t\equiv [b]t'$, which in turn yields $s'\sigma \equiv g_1$ and $t'\sigma\equiv g_2$. Thus,
              \begin{prooftree}
                 \AxiomC{$\vdash \newswap{a}{c_1}\act g_1 \aeq{C} \newswap{b}{c_1}\act g_2$}
                 \dashedLine
                  \UnaryInfC{$\vdash \newswap{a}{c_1}\act (s'\sigma) \aeq{C} \newswap{b}{c_1}\act (t'\sigma)$}
                  \dashedLine
                  \UnaryInfC{$\vdash (\newswap{a}{c_1}\act s')\sigma \aeq{C} (\newswap{b}{c_1}\act t')\sigma$}
              \end{prooftree}
              By induction, we get
             \begin{prooftree}
                 \AxiomC{$\Upsilon^+_{\atnew{\pvec{c}',c_1}}\vdash ((\newswap{a}{c_1}\act s')\sigma)^{-1} \aeq{C} (\newswap{b}{c_1}\act t')\sigma)^{-1}$}
                 \dashedLine
                 \UnaryInfC{$\Upsilon^+_{\atnew{\pvec{c}',c_1}}\vdash (\newswap{a}{c_1}\act (s'\sigma))^{-1} \aeq{C} (\newswap{b}{c_1}\act (t'\sigma))^{-1}$}
                 \dashedLine
                 \UnaryInfC{$\Upsilon^+_{\atnew{\pvec{c}',c_1}}\vdash (\newswap{a}{c_1}\act g_1)^{-1} \aeq{C} (\newswap{b}{c_1}\act g_2)^{-1}$}
             \end{prooftree}

             Observe that $\var{g_1^{-1}}= \var{g_2^{-1}}$. If  $\var{g_1^{-1}} \neq \var{g_2^{-1}}$, let's say $Z\in \var{g_1^{-1}}$ and $Z\notin\var{g_2^{-1}}$. By the definition of $\sigma$ and $-^{-1}$, this means that $\tf{d}_Z(a_{Z_1},\ldots,a_{Z_{k_Z}})$ occurs in $g_1$ at the same position that $Z$ occurs in $g_1^{-1}$. However, since $~\vdash \newswap{a}{c_1}\act g_1 \aeq{C} \newswap{b}{c_1}\act g_2$, this means that $\tf{d}_Z(a_{Z_1},\ldots,a_{Z_{k_Z}})$ must occur in $g_2$ at the same position that it occurs in $g_1$. This implies that $Z\in\var{g_2^{-1}}$, which is a contradiction.

             We claim that $a\notin {\cal A}_X$ for every $X\in\var{g_1^{-1}}$. Suppose, by contradiction, that $a\in {\cal A}_X$ for some $X\in\var{g_1^{-1}}$.

             From $~\vdash \newswap{a}{c_1}\act g_1 \aeq{C} \newswap{b}{c_1}\act g_2$, we have $\tf{fn}(\newswap{a}{c_1}\act g_1) = \tf{fn}(\newswap{b}{c_1}\act g_2)$. This implies $a\notin \tf{fn}(g_2)$.

             Since $\var{g_2^{-1}} = \var{g_1^{-1}}$, it follows that $X\in \var{g_2^{-1}}$ and hence $a\in \tf{d}_X(a_{X_1},\ldots,a_{X_{k_X}})$ and $\tf{d}_X(a_{X_1},\ldots,a_{X_{k_X}})$ occurs in $g_2$. Then $a\in \tf{fn}(g_2)$,  a contradiction.


             Since $a\notin {\cal A}_X$ for every $X\in\var{g_1^{-1}}$. By definition this means that for every $X\in\var{g_1^{-1}}$ there is some $\pi\fix{C} X\in\Upsilon_{\fresh}|_X$ such that $a\in \dom{\pi}$. Thus
              \[
                     \newswap{a}{c_1}\in \bigcap_{X\in\var{g_1^{-1}}}\Perm{\atm{\Upsilon_{\fresh}|_{X}}\cup\atnew{\pvec{c}',c_1}\cup{\cal C}}\pair{\Upsilon_{\fix{C}}|_{X}}
              \]
              By Lemma~\ref{alemma:equivariance-for-completeness},  we obtain the derivation $\Upsilon^+_{\atnew{\pvec{c}',c_1}}\vdash (\newswap{a}{c_1}\act g_1)^{-1} \aeq{C} \newswap{a}{c_1}\act g_1^{-1}$.

              Similarly, we have $b\notin {\cal A}_X$ for every $X\in\var{g_2^{-1}}$ and consequently $\Upsilon^+_{\atnew{\pvec{c}',c_1}}\vdash (\newswap{b}{c_1}\act g_2)^{-1} \aeq{C} \newswap{b}{c_1}\act g_2^{-1}$. Therefore, $\Upsilon^+_{\atnew{\pvec{c}',c_1}}\vdash (a \ \atnew{c_1})\act g_1^{-1} \aeq{C} (b \ \atnew{c_1})\act g_2^{-1}$ follows by transitivity. Therefore, the derivation $ \Upsilon^+_{\atnew{\pvec{c}'}} \vdash [a]g_1^{-1} \aeq{C} [b]g_2^{-1}$ follows by rule $(\frule{\faeq{C}}{ab}).$
         \end{itemize}

        \paragraph*{Completing the proof} Now that we proved the claim, let's proceed with the proof of the lemma: by Lemma~\ref{alemma:substitution-inverse-fixed-point}, symmetry and transitivity of $\aeq{C}$, we deduce $\Upsilon^+_{\atnew{\pvec{c}'}}\vdash s\aeq{C} t$ and by Strengthening (Theorem~\ref{thm:miscellaneous}(\ref{thm:strengthening})) and Lemma~\ref{alemma:vacuous-quantification}, we obtain $\Upsilon_{\catnew{c}}\vdash s\aeq{C} t$.
    \end{enumerate}
\end{proof}

With all of this established, we can finally prove Completeness.

\begin{proof}[of Completeness]
    Suppose $\Upsilon_{\catnew{c}} \vDash t\aeq{C} u$, so $\Int{\Upsilon_{\catnew{c}} \vdash t\aeq{C} u}{\nom{F}}{\varsigma^*}$ is valid. We want to show that there is a derivation of $\Upsilon_{\catnew{c}} \vdash t\aeq{C} u$. In fact, by Lemma~\ref{alemma:interpretation}(\ref{alemma:context-valid}) we have $\Int{t}{\nalg{F}}{\varsigma^*} = \Int{u}{\nalg{F}}{\varsigma^*}$. By Lemma~\ref{alemma:inversion-properties}(\ref{alemma:substitution-interpretation}), $\Int{t}{\nalg{F}}{\varsigma^*} = \lin{t\sigma}$ and $\Int{u}{\nalg{F}}{\varsigma^*} = \lin{u\sigma}$. Therefore $\lin{t\sigma} = \lin{u\sigma}$, that is, $~\vdash t\sigma \aeq{C} u\sigma$. It follows by Lemma~\ref{alemma:inversion-properties}(\ref{alemma:ground-derivation-implies-derivation}) that $\Upsilon_{\catnew{c}} \vdash t\aeq{C} u$, as desired.
\end{proof}



\section{Proof of Preservation by substitution}\label{app:preservation-substitutiton}

The next results aim to prove the property of Preservation by substitution. We begin establishing by establishing a lemma stating that fix-point constraints are closed under composition and inverses.

\begin{lemma}\label{alemma:fix-point-composition-inverse}
    \begin{enumerate}
        \item \label{alemma:fix-point-composition} If $\Upsilon_{\catnew{c}} \vdash \pi_1 \fix{C} t$  and $\Upsilon_{\catnew{c}} \vdash \pi_2 \fix{C} t$, then $\Upsilon_{\catnew{c}} \vdash (\pi_1\circ \pi_2) \fix{C} t$.
        \item \label{alemma:fix-point-inverse} If $\Upsilon_{\catnew{c}} \vdash \pi\fix{C} t$, then  $\Upsilon_{\catnew{c}} \vdash \pi^{-1}\fix{C} t$.
    \end{enumerate}
\end{lemma}

\begin{proof}
Direct consequence of Equivariance and Equivalence in Theorem~\ref{thm:miscellaneous}.
\end{proof}

The following lemma formalizes in the calculus the valid result in nominal set theory, which states that any swapping $(a \ b)$ of two fresh names $a$ and $b$ for an element $x$ fixes the element $x$.

\begin{lemma}\label{alemma:two-fresh-names-fix}
    Let $\atnew{c_1},\atnew{c_2}\notin \atm{\Upsilon_{\catnew{c}},t}$. Then  $\Upsilon_{\catnew{c},\atnew{c_1},\atnew{c_2}} \vdash (\atnew{c_1} \ \atnew{c_2})\act t \aeq{C} t$.
\end{lemma}


\begin{proof}
The proof is direct by induction on the structure of $t$. Here we show only the case when $t\equiv \pi\act X$. We aim to prove $\Upsilon_{\catnew{c},\atnew{c_1},\atnew{c_2}} \vdash (\atnew{c_1} \ \atnew{c_2}) \act (\pi\act X) \aeq{C} (\pi\act X)$. Given that $\atnew{c_1},\atnew{c_2} \notin \dom{\pi}$, we have
     \[
         \pi^{-1}\circ(\atnew{c_1} \ \atnew{c_2})\circ\pi = (\atnew{c_1} \ \atnew{c_2})\in \PN{}{\Upsilon_{\catnew{c},\atnew{c_1},\atnew{c_2}}|_X}.
     \]
     Therefore, the result follows by rule $(\frule{\faeq{C}}{var})$.
\end{proof}


The next lemma provides a method for handling vacuous quantifications in a derivation.

\begin{lemma}\label{alemma:vacuous-quantification}
     Suppose $\atnew{c_1}\notin \atm{\Upsilon_{\catnew{c}}}$. If $\Upsilon_{\catnew{c}} \vdash t\aeq{C} u$ then $\Upsilon_{\catnew{c},\atnew{c_1}}\vdash t\aeq{C} u$. The converse holds if $\atnew{c_1}\notin \atm{\Upsilon_{\catnew{c}},t,u}$.
\end{lemma}

\begin{proof}
 By induction on the last rule applied to obtain $\Upsilon_{\catnew{c}} \vdash t\aeq{C} u$. For instance, if the last rule is $(\frule{\faeq}{var})$, then we obtain $\Upsilon_{\catnew{c}} \vdash \pi_1\act X\aeq{C} \pi_2\act X$. By Inversion (Theorem~\ref{thm:miscellaneous}(\ref{thm:inversion})), it follows that $\pi_2^{-1}\circ\pi_1\in \PN{}{\Upsilon_{\catnew{c}}|_X}$. Since $\atnew{c_1}\notin \atm{\Upsilon_{\catnew{c}}}$, we  have $\PN{}{\Upsilon_{\catnew{c}}|_X}\subseteq\PN{}{\Upsilon_{\catnew{c},\atnew{c_1}}|_X} $. Then $\pi_2^{-1}\circ\pi_1\in \PN{}{\Upsilon_{\catnew{c},\atnew{c_1}}|_X}$ and so the result follows by rule $(\frule{\faeq{C}}{var})$.

\end{proof}

The following lemma formalizes a fundamental property of nominal sets, stating that if a permutation's domain contains only fresh names for an element $x$, then $x$ is fixed by that permutation.

\begin{lemma}\label{alemma:fix-point-formed-by-fresh-names}
    Let $I$ be a non-empty, finite set of indices, and let $\{\pi_i \mid i \in I\}$ be a set of permutations. Suppose $\Upsilon_{\catnew{c}, \atnew{c_1}} \vdash \newswap{a}{c_1}\fix{C} t$ for all $a \in \left(\bigcup_{i \in I} \dom{\pnew{\pi_i}{\catnew{c}}}\right) \cup \catnew{c}$, where $\atnew{c_1} \notin \atm{\Upsilon_{\catnew{c}},t, \{\pi_i \mid i \in I\}}$. Then, $\Upsilon_{\catnew{c}} \vdash \pi'\fix{C} t$ for all $\pi'$ such that $\dom{\pi'} \subseteq \left(\bigcup_{i \in I} \dom{\pnew{\pi_i}{\catnew{c}}}\right) \cup \catnew{c}$.
\end{lemma}

\begin{proof}
    We proceed by induction on the number of disjoint cycles in the decomposition of $\pi'$.

    \paragraph*{Base case.}

    Suppose $\pi'$ is a single cycle, say $\pi' = (a_1 \ a_2 \ \ldots \ a_m)$, which can be expressed as product of swappings:
     \[
        \pi' = (a_1 \ a_m)\circ (a_1 \ a_{m-1})\circ(a_1 \ a_3)\circ(a_1 \ a_2)
    \]
    For each $k\in\{2, \ldots, m\}$: $(a_1 \ a_k) = \newswap{a_1}{c_1}\circ \newswap{a_k}{c_1} \circ \newswap{a_1}{c_1}.$ By the assumption in the statement, we have $\Upsilon_{\catnew{c},\atnew{c_1}} \vdash \newswap{a_k}{c_1}\fix{C} t$ for all $k\in \{2, \ldots, m\}$. By Lemma~\ref{alemma:fix-point-composition-inverse}(\ref{alemma:fix-point-composition}), we have
    \[
        \Upsilon_{\catnew{c},\atnew{c_1}} \vdash \newswap{a_1}{c_1}\circ \newswap{a_k}{c_1} \circ \newswap{a_1}{c_1}\fix{C} t
    \]
    which implies $\Upsilon_{\catnew{c},\atnew{c_1}} \vdash (a_1 \ a_k)\fix{C} t$. By applying Lemma~\ref{alemma:fix-point-composition-inverse}(\ref{alemma:fix-point-composition}) again, we obtain:
    \[
        \Upsilon_{\catnew{c},\atnew{c_1}} \vdash (a_1 \ a_m)\circ (a_1 \ a_{m-1})\circ(a_1 \ a_3)\circ(a_1 \ a_2)\fix{C} t.
    \]
    which is the same as $\Upsilon_{\catnew{c},\atnew{c_1}} \vdash (a_1 \ a_2 \ \ldots \ a_m)\fix{C} t$. By Lemma~\ref{alemma:vacuous-quantification}, we conclude that $\Upsilon_{\catnew{c}} \vdash \pi'\fix{C} t$.

    \paragraph*{Inductive step:} Now, assume the result holds for all permutations $\pi'$ (under the conditions of the statement) that decompose into $n-1$ disjoint cycles. Consider a permutation $\pi'$ such that $\dom{\pi'} \subseteq \left(\bigcup_{i \in I} \dom{\pnew{\pi_i}{\catnew{c}}}\right) \cup \catnew{c}$ and consisting of $n$ disjoint cycles: $\pi' = \eta_1\circ\ldots\circ\eta_n.$ Since $\dom{\eta_1 \circ \ldots \circ \eta_{n-1}} \subseteq \dom{\pi'}$, it follows:
    \[
        \dom{\eta_1\circ\ldots\circ\eta_{n-1}} \subseteq \left(\bigcup_{i\in I}\dom{\pnew{\pi_i}{\catnew{c}}}\right)\cup\catnew{c}.
    \]
    By the induction hypothesis, we have $\Upsilon_{\catnew{c},\atnew{c_1}} \vdash (\eta_1\circ\ldots\circ\eta_{n-1})\fix{C} t.$ Using a similar argument to the base case, we show that $\Upsilon_{\catnew{c},\atnew{c_1}} \vdash \eta_n\fix{C} t$. By  Lemma~\ref{alemma:fix-point-composition-inverse}(\ref{alemma:fix-point-composition}), it follows that $\Upsilon_{\catnew{c},\atnew{c_1}} \vdash \pi'\fix{C} t$ which, by Lemma~\ref{alemma:vacuous-quantification}, yields $\Upsilon_{\catnew{c}} \vdash \pi'\fix{C} t$.
\end{proof}

The following lemma describes a particular derivation case, which will be useful for establishing the next result.

\begin{lemma}\label{alemma:characterization-fix-commutative}
    Suppose $\atnew{c_1}\notin \atm{\Upsilon_{\catnew{c}},a,\tf{f^C}(t_0,t_1)}$. Then $\Upsilon_{\catnew{c},\atnew{c_1}} \vdash \newswap{a}{c_1}\fix{C} \tf{f^C}(t_0,t_1)$ iff $\Upsilon_{\catnew{c},\atnew{c_1}}\vdash \newswap{a}{c_1}\fix{C} t_0$ and $\Upsilon_{\catnew{c},\atnew{c_1}}\vdash \newswap{a}{c_1}\fix{C} t_1$.
\end{lemma}

\begin{proof}
\begin{description}
    \item[$(\Leftarrow)$] This case follows by applying the rule $(\frule{\faeq{C}}{\tf{f^C}})$.

    \item[$(\Rightarrow)$]

    The proof follows by an analysis on every possible pair of terms $(t_0,t_1)$. Since $\Upsilon_{\catnew{c},\atnew{c_1}} \vdash \newswap{a}{c_1}\fix{C} \tf{f^C}(t_0,t_1)$ iff $\Upsilon_{\catnew{c},\atnew{c_1}} \vdash \newswap{a}{c_1}\fix{C} \tf{f^C}(t_1,t_0)$ holds, it's sufficient to analyse only half of the possibilities. The only non-trivial cases are the cases where $t_0$ and $t_1$ are either both suspensions or both abstractions.

    \begin{enumerate}
        \item $t_0 \equiv \pi\act X$ and $t_1 \equiv \pi'\act Y$.

        In this case, $\Upsilon_{\catnew{c},\atnew{c_1}} \vdash \newswap{a}{c_1}\fix{C} \tf{f^C}(\pi\act X,\pi'\act Y)$ implies $ \Upsilon_{\catnew{c},\atnew{c_1}} \vdash \tf{f^C}(\newswap{a}{c_1}\act(\pi\act X),\newswap{a}{c_1}\act(\pi'\act Y))\aeq{C} \tf{f^C}(\pi\act X,\pi'\act Y).$


        \begin{enumerate}
            \item $Y\not\equiv X$.

            By Inversion (Theorem~\ref{thm:miscellaneous}(\ref{thm:inversion})), the valid branch is: $\Upsilon_{\catnew{c},\atnew{c_1}} \vdash \newswap{a}{c_1}\act(\pi\act X) \aeq{C}\pi\act X$ and $\Upsilon_{\catnew{c},\atnew{c_1}} \vdash \newswap{a}{c_1}\act(\pi'\act Y)\aeq{C}\pi'\act Y$, which are precisely the derivations we wanted.

            \item $Y\equiv X$.

            By Inversion (Theorem~\ref{thm:miscellaneous}(\ref{thm:inversion})), there are two possible branches:
            \begin{itemize}
               \item $\Upsilon_{\catnew{c},\atnew{c_1}}\vdash \newswap{a}{c_1}\act(\pi\act X) \aeq{C}\pi\act X$ and $\Upsilon_{\catnew{c},\atnew{c_1}}\vdash \newswap{a}{c_1}\act(\pi'\act X)\aeq{C}\pi'\act X$, and the result follows trivially.


            \item $\Upsilon_{\catnew{c},\atnew{c_1}}\vdash \newswap{a}{c_1}\act(\pi\act X)\aeq{C}\pi'\act X$ and $\Upsilon_{\catnew{c},\atnew{c_1}}\vdash \newswap{a}{c_1}\act(\pi'\act X)\aeq{C}\pi\act X$.

            Note that, by the Equivariance (Theorem~\ref{thm:miscellaneous}(\ref{thm:object-equivariance})) and the symmetry of $\aeq{C}$, these two derivations are equivalent, so it's enough to focus on just one. We will work with the first one. Applying Inversion (Theorem~\ref{thm:miscellaneous}(\ref{thm:inversion})), we obtain: $\pi'^{-1} \circ\newswap{a}{c_1}\circ \pi\in \PN{}{\Upsilon_{\catnew{c},\atnew{c_1}}|_X}$. To simplify the proof, let's call $\gamma := \pi'^{-1} \circ\newswap{a}{c_1}\circ \pi$

            \paragraph*{Case  $\gamma = \id$.} In this case we have
            \[
               \pi'^{-1}(a)  = \gamma(\atnew{c_1}) = \id(\atnew{c_1}) = \atnew{c_1}.
            \]
            and
            \[
                \atnew{c_1}  = \gamma(\pi^{-1}(a)) = \id(\pi^{-1}(a)) = \pi^{-1}(a).
            \]
            Therefore, $\newswap{a}{c_1}^{\pi'^{-1}} =\newswap{\pi'^{-1}(a)}{c_1} = \id \in \PN{}{\Upsilon_{\catnew{c},\atnew{c_1}}|_X}$. Similarly, $\newswap{a}{c_1}^{\pi^{-1}} =\newswap{\pi^{-1}(a)}{c_1} = \id \in \PN{}{\Upsilon_{\catnew{c},\atnew{c_1}}|_X}$. Thus, the result follows by rule $(\frule{\fix{C}}{var})$.

            \paragraph*{Case $\gamma \neq \id$.}
            As observed $\gamma(\pi^{-1}(a)) = \atnew{c_1}$ and $\gamma(\atnew{c_1}) = \pi'^{-1}(a)$. Since $\dom{\gamma}$ is finite, there is a $m\geq 3$ such that $\gamma^m(\pi^{-1}(a)) = \pi^{-1}(a)$. Then
            \[
                \gamma = \underbrace{(\pi^{-1}(a) \ \atnew{c_1} \ \pi'^{-1}(a) \ \gamma^4(\pi^{-1}(a)) \ \ldots \ \gamma^{m-1}(\pi^{-1}(a)))}_{\gamma'}\circ \rho'
            \]
            where $\rho'$ is disjoint from $\gamma'$. Then $\gamma' \in \Perm{\atm{(\Upsilon_{\catnew{c},\atnew{c_1}}|_X)_{\fresh}}}$ and consequently $\pi^{-1}(a),\pi'^{-1}(a) \in \atm{(\Upsilon_{\catnew{c},\atnew{c_1}}|_X)_{\fresh}}$. So it follows that $\newswap{a}{c_1}^{\pi'^{-1}} = \newswap{\pi'^{-1}(a)}{c_1} \in \PN{}{\Upsilon_{\catnew{c},\atnew{c_1}}|_X}$ and  $\newswap{a}{c_1}^{\pi^{-1}} = \newswap{\pi^{-1}(a)}{c_1} \in \PN{}{\Upsilon_{\catnew{c},\atnew{c_1}}|_X}$ and the result follows by $(\frule{\faeq{C}}{var})$.
         \end{itemize}
        \end{enumerate}

        \item $t_0$ and $t_1$ are abstractions.
        \begin{enumerate}
            \item $t_0 \equiv [a]t_0'$ and $t_1 \equiv [a]t_1'$.

            Then $\Upsilon_{\catnew{c},\atnew{c_1}} \vdash \newswap{a}{c_1}\fix{C} \tf{f^C}([a]t_0',[a]t_1')$ implies $ \Upsilon_{\catnew{c},\atnew{c_1}} \vdash \tf{f^C}(\newswap{a}{c_1}\act[a]t_0',\newswap{a}{c_1}\act[a]t_1')\aeq{C} \tf{f^C}([a]t_0',[a]t_1')$. By Inversion (Theorem~\ref{thm:miscellaneous}(\ref{thm:inversion})), we have two possible branches:

        \begin{itemize}
            \item  $\Upsilon_{\catnew{c},\atnew{c_1}} \vdash \newswap{a}{c_1}\act [a]t_0'\aeq{C} [a]t_0'$ and $\Upsilon_{\catnew{c},\atnew{c_1}} \vdash \atnew{c_1}\act [a]t_1'\aeq{C} [a]t_1'$.

            Then the result follows.

            \item $\Upsilon_{\catnew{c},\atnew{c_1}} \vdash \newswap{a}{c_1}\act [a]t_0'\aeq{C} [a]t_1'$ and $\Upsilon_{\catnew{c},\atnew{c_1}} \vdash \newswap{a}{c_1}\act [a]t_1'\aeq{C} [a]t_0'$.

            Note that, by Equivariance (Theorem~\ref{thm:miscellaneous}(\ref{thm:object-equivariance})) and Equivalence (Theorem~\ref{thm:miscellaneous}(\ref{thm:alpha-equivalence})), these two derivations are equivalent, so it's sufficient with just one. We will work with the first one.

            We claim that $\Upsilon_{\catnew{c},\atnew{c_1}} \vdash [a]t_0' \aeq{C} [a]t_1'$. In fact, observe that the derivation $\Upsilon_{\catnew{c},\atnew{c_1}} \vdash \newswap{a}{c_1} \act [a]t_0' \aeq{C} [a]t_1'$ is equivalent to $\Upsilon_{\catnew{c},\atnew{c_1}} \vdash [\atnew{c_1}]\newswap{a}{c_1} \act t_0' \aeq{C} [a]t_1'$. By applying Inversion (Theorem~\ref{thm:miscellaneous}(\ref{thm:inversion})), we derive $\Upsilon_{\catnew{c},\atnew{c_1},\atnew{c_2}} \vdash ((\atnew{c_1} \ \atnew{c_2}) \circ \newswap{a}{c_1}) \act t_0' \aeq{C} \newswap{a}{c_2} \act t_1'$ where $\atnew{c_2}\notin \atm{\Upsilon_{\catnew{c},\atnew{c_1}},a,t_0',t_1'}$. By Equivariance  (Theorem~\ref{thm:miscellaneous}(\ref{thm:object-equivariance}), this results in $\Upsilon_{\catnew{c},\atnew{c_1},\atnew{c_2}} \vdash (\newswap{a}{c_2}^{-1} \circ (\atnew{c_1} \ \atnew{c_2}) \circ \newswap{a}{c_1}) \act t_0' \aeq{C} t_1'$. Since $(\newswap{a}{c_2}^{-1} \circ (\atnew{c_1} \ \atnew{c_2}) \circ \newswap{a}{c_1}) = (\atnew{c_1} \ \atnew{c_2})$, we conclude that $\Upsilon_{\catnew{c},\atnew{c_1},\atnew{c_2}} \vdash (\atnew{c_1} \ \atnew{c_2}) \act t_0' \aeq{C} t_1'$. On the other hand, because $\atnew{c_1},\atnew{c_2}\notin \atm{\Upsilon_{\catnew{c}},a,t_0',t_1'}$, Lemma~\ref{alemma:two-fresh-names-fix} gives us $\Upsilon_{\catnew{c},\atnew{c_1},\atnew{c_2}} \vdash (\atnew{c_1} \ \atnew{c_2})\act t_0' \aeq{C} t_0'$. This combined with Equivalence (Theorem~\ref{thm:miscellaneous}(\ref{thm:alpha-equivalence})) provides us $\Upsilon_{\catnew{c},\atnew{c_1},\atnew{c_2}} \vdash t_0' \aeq{C} t_1'$. Thus, by Lemma~\ref{alemma:vacuous-quantification}, we obtain $\Upsilon_{\catnew{c},\atnew{c_1}} \vdash  t_0' \aeq{C} t_1'$, which, by rule $(\frule{\faeq{C}}{[a]})$, leads to $\Upsilon_{\catnew{c},\atnew{c_1}} \vdash [a]t_0' \aeq{C} [a]t_1'$.

            Now, using the claim we just proved and Equivalence (Theorem~\ref{thm:miscellaneous}(\ref{thm:alpha-equivalence})), we obtain $\Upsilon_{\catnew{c},\atnew{c_1}} \vdash \newswap{a}{c_1}\act [a]t_0'\aeq{C} [a]t_0'$ and $\Upsilon_{\catnew{c},\atnew{c_1}} \vdash \newswap{a}{c_1}\act [a]t_1'\aeq{C} [a]t_1'$ and so the result follows.
            \end{itemize}

            \item $t_0\equiv [a]t_0'$ and $t_1 \equiv [d']t_1'$.

            Then $\Upsilon_{\catnew{c},\atnew{c_1}}\vdash \newswap{a}{c_1}\fix{C} \tf{f^C}([a]t_0',[d']t_1')$ implies $\Upsilon_{\catnew{c},\atnew{c_1}}\vdash \tf{f^C}( \newswap{a}{c_1}\act[a]t_0', \newswap{a}{c_1}\act  [d']t_1')\aeq{C} \tf{f^C}([a]t_0',[d']t_1')$. By Inversion (Theorem~\ref{thm:miscellaneous}(\ref{thm:inversion})), there are two branches to consider:
        \begin{itemize}
             \item $\Upsilon_{\catnew{c},\atnew{c_1}}\vdash  \newswap{a}{c_1}\act[a]t_0'\aeq{C} [a]t_0'$ and $\Upsilon_{\catnew{c},\atnew{c_1}}\vdash  \newswap{a}{c_1}\act[d']t_1'\aeq{C} [d']t_1'$.

             Then the result follows.

            \item $\Upsilon_{\catnew{c},\atnew{c_1}}\vdash  \newswap{a}{c_1}\act[a]t_0'\aeq{C} [d']t_1'$ and $\Upsilon_{\catnew{c},\atnew{c_1}}\vdash  \newswap{a}{c_1}\act[d']t_1'\aeq{C} [a]t_0'$.

            Note that, by Equivariance (Theorem~\ref{thm:miscellaneous}(\ref{thm:object-equivariance})) and Equivalence (Theorem~\ref{thm:miscellaneous}(\ref{thm:alpha-equivalence})), these two derivations are equivalent, so it's sufficient to work with just one. We will work with the first one.

            We claim that $\Upsilon_{\catnew{c},\atnew{c_1}} \vdash  [d']t_1' \aeq{C} [a]t_0'$. Indeed, observe that the derivation $\Upsilon_{\catnew{c},\atnew{c_1}} \vdash \newswap{a}{c_1} \act [a]t_0' \aeq{C} [d']t_1'$ can be rewritten as $\Upsilon_{\catnew{c},\atnew{c_1}} \vdash [\atnew{c_1}]\newswap{a}{c_1} \act t_0' \aeq{C} [d']t_1'$. By applying Inversion (Theorem~\ref{thm:miscellaneous}(\ref{thm:inversion})), we derive $\Upsilon_{\catnew{c},\atnew{c_1},\atnew{c_2}} \vdash ((\atnew{c_1} \ \atnew{c_2}) \circ \newswap{a}{c_1}) \act t_0' \aeq{C} \newswap{d'}{c_2} \act t_1'$ where $\atnew{c_2}\notin \atm{\Upsilon_{\catnew{c}},a,d',t_0',t_1'}$.

            By Equivariance (Theorem~\ref{thm:miscellaneous}(\ref{thm:object-equivariance})), we have $\Upsilon_{\catnew{c},\atnew{c_1},\atnew{c_2}} \vdash (\newswap{d'}{c_2}^{-1} \circ (\atnew{c_1} \ \atnew{c_2}) \circ \newswap{a}{c_1}) \act t_0' \aeq{C} t_1'$, which is the same as $\Upsilon_{\catnew{c},\atnew{c_2},\atnew{c_1}} \vdash (a \ d' \ \atnew{c_1} \ \atnew{c_2}) \act t_0' \aeq{C} t_1'$.

            Now, by applying Lemma~\ref{alemma:two-fresh-names-fix}, we have $\Upsilon_{\catnew{c},\atnew{c_1},\atnew{c_2}} \vdash (\atnew{c_1} \ \atnew{c_2}) \act t_0' \aeq{C} t_0'$. Since $((a \ d' \ \atnew{c_2})^{-1} \circ (a \ d' \ \atnew{c_2} \ \atnew{c_1})) = (\atnew{c_1} \ \atnew{c_2})$, we conclude that $\Upsilon_{\catnew{c},\atnew{c_1},\atnew{c_2}} \vdash ((a \ d' \ \atnew{c_2})^{-1} \circ (a \ d' \ \atnew{c_2} \ \atnew{c_1})) \act t_0' \aeq{C} t_0'$.

            By Equivariance (Theorem~\ref{thm:miscellaneous}(\ref{thm:object-equivariance})), we get $\Upsilon_{\catnew{c},\atnew{c_1},\atnew{c_2}} \vdash (a \ d' \ \atnew{c_2} \ \atnew{c_1}) \act t_0' \aeq{C} (a \ d' \ \atnew{c_2}) \act t_0'$. This, together with the derivation $\Upsilon_{\catnew{c},\atnew{c_1},\atnew{c_2}} \vdash (a \ d' \ \atnew{c_2} \ \atnew{c_1}) \act t_0' \aeq{C} t_1'$, and Equivalence (Theorem~\ref{thm:miscellaneous}(\ref{thm:alpha-equivalence}))), yields $\Upsilon_{\catnew{c},\atnew{c_1},\atnew{c_2}} \vdash (a \ d' \ \atnew{c_2}) \act t_0' \aeq{C} t_1'$.

            Since $(a \ d' \ \atnew{c_2}) = \newswap{d'}{c_2} \circ \newswap{a}{c_2}$, we get $\Upsilon_{\catnew{c},\atnew{c_1},\atnew{c_2}} \vdash (\newswap{d'}{c_2} \circ \newswap{a}{c_2}) \act t_0' \aeq{C} t_1'$, which, through Equivariance (Theorem~\ref{thm:miscellaneous}(\ref{thm:object-equivariance})) combined with Equivalence (Theorem~\ref{thm:miscellaneous}(\ref{thm:alpha-equivalence}))), gives us $\Upsilon_{\catnew{c},\atnew{c_1},\atnew{c_2}} \vdash \newswap{d'}{c_2} \act t_1' \aeq{C} \newswap{a}{c_2} \act t_0'$. Finally, by applying rule $(\frule{\faeq{C}}{ab})$, we conclude that $\Upsilon_{\catnew{c},\atnew{c_1}} \vdash [d']t_1' \aeq{C} [a]t_0'$.

           Using the claim we just proved combined with Equivalence (Theorem~\ref{thm:miscellaneous}(\ref{thm:alpha-equivalence})), the derivations $\Upsilon_{\catnew{c},\atnew{c_1}} \vdash\newswap{a}{c_2}\act[a]t_0'\aeq{C} [d']t_1'$ and $\Upsilon_{\catnew{c},\atnew{c_1}} \vdash \newswap{a}{c_2}\act[d']t_1'\aeq{C} [a]t_0'$ yield
            \[
                \Upsilon_{\catnew{c},\atnew{c_1}} \vdash \newswap{a}{c_2}\act[a]t_0'\aeq{C} [a]t_0' \text{ and } \Upsilon_{\catnew{c},\atnew{c_1}} \vdash \newswap{a}{c_2}\act[d']t_1'\aeq{C} [d']t_1',
            \]
            and the result follows.

            \end{itemize}

        \item For the following cases:
        \begin{itemize}
            \item $t_0\equiv [d']t_0'$ and $t_1 \equiv [a]t_1'$.
            \item $t_0 \equiv [d']t_0'$ and $t_1 \equiv [d']t_1'$
            \item $t_0 \equiv [d_1]t_0'$ and $t_1 \equiv [d_2]t_1'$
            \item $t_0 \equiv [d_2]t_0'$ and $t_1 \equiv [d_1]t_1'$
        \end{itemize}
        The arguments for these cases are very similar to the previous ones, so we will omit them.

            \end{enumerate}
        \end{enumerate}

\end{description}
\end{proof}

\begin{lemma}\label{alemma:characterization-fix-c-general}
    $\Upsilon_{\catnew{c}} \vdash \pi_{\catnew{c}} \fix{C} t$ iff $\Upsilon_{\catnew{c},\atnew{c_1}} \vdash \newswap{a}{\atnew{c_1}}\fix{C} t$ for all $a\in \dom{\pi_{\catnew{c}}}\cup\catnew{c}$ where $\atnew{c_1}\notin \atm{\Upsilon_{\catnew{c}},\pi_{\catnew{c}},t}$.
\end{lemma}

\begin{proof}
    Proof by induction on structure of the term $t$. The non-trivial cases are presented below.
    \begin{itemize}
        \item $t \equiv \pi'\act X$.

        \begin{description}
            \item[$(\Rightarrow)$] In this case, we have $\Upsilon_{\catnew{c}} \vdash \pi_{\catnew{c}} \fix{C} \pi'\act X$. By Inversion (Theorem~\ref{thm:miscellaneous}(\ref{thm:inversion}), it follows that $\pi_{\catnew{c}}^{\pi'^{-1}} \in \PN{}{\Upsilon_{\catnew{c}}|_X}$. In particular, $\pi_{\catnew{c}}^{\pi'^{-1}} \in \PN{}{\Upsilon_{\catnew{c},\atnew{c_1}}|_X}$. Since $\pi_{\catnew{c}}^{\pi'^{-1}} = (\pi^{\pi'^{-1}})_{\pi'^{-1}(\catnew{c})}$ and $\pi'$ does not mention names from $\catnew{c}$, we have $ \pi_{\catnew{c}}^{\pi'^{-1}} = \pnew{\pi^{\pi'^{-1}}}{\catnew{c}}$, obtaining $\pnew{\pi^{\pi^{-1}}}{\catnew{c}} \in  \PN{}{\Upsilon_{\catnew{c},\atnew{c_1}}|_X}$.

            We aim to prove that $\Upsilon_{\catnew{c},\atnew{c_1}} \vdash \newswap{a}{c_1} \fix{C} \pi'\act X$ for all $a \in \dom{\pi_{\catnew{c}}} \cup \catnew{c}$. To achieve this, it suffices to show
            \[
                \newswap{\pi'^{-1}(a)}{c_1} = \newswap{a}{c_1}^{\pi'^{-1}} \in \PN{}{\Upsilon_{\catnew{c},\atnew{c_1}}|_X}.
            \]

            \begin{itemize}
                \item First, let's analyse the cases where $a$ is an atom from $\catnew{c}$. Notice that $\catnew{c},\atnew{c_1} \subseteq \atm{(\Upsilon_{\catnew{c},\atnew{c_1}}|_X)_{\fresh}}$. Thus, for every $\atnew{c'} \in \catnew{c}$, we have $(\atnew{c'} \ \atnew{c_1}) \in \PN{}{\Upsilon_{\catnew{c},\atnew{c_1}}|_X}$, implying $(\atnew{c'} \ \atnew{c_1}) \in \PN{}{\Upsilon_{\catnew{c},\atnew{c_1}}|_X}$. Since $\pi'^{-1}(\atnew{c'}) = \atnew{c'}$ for all $\atnew{c'} \in \catnew{c}$, this ensures that $\newswap{\pi'^{-1}(a)}{c_1} \in  \PN{}{\Upsilon_{\catnew{c},\atnew{c_1}}|_X}$ for all $a \in \catnew{c}$.

                \item  Now, suppose $a \in \dom{\pi_{\catnew{c}}}$ but $a \notin \catnew{c}$. Then $\pi'^{-1}(a) \in \dom{\pnew{\pi^{\pi'^{-1}}}{\catnew{c}}}$. From $\pnew{\pi^{\pi^{-1}}}{\catnew{c}} \in  \PN{}{\Upsilon_{\catnew{c},\atnew{c_1}}|_X}$, we get $\pnew{\pi^{\pi^{-1}}}{\catnew{c}} \in \Perm{\atm{(\Upsilon_{\catnew{c},\atnew{c_1}}|_X)_{\fresh}}}$. Thus $\pi'^{-1}(a) \in \atm{(\Upsilon_{\catnew{c},\atnew{c_1}}|_X)_{\fresh}}$ and so $\newswap{a}{c_1}^{\pi'^{-1}} \in \PN{}{\Upsilon_{\catnew{c},\atnew{c_1}}|_X}$. Consequently, the result follows by applying rule $(\frule{\faeq{C}}{var})$.
            \end{itemize}


            \item[$(\Leftarrow)$] Suppose $\Upsilon_{\catnew{c},\atnew{c_1}} \vdash \newswap{a}{c_1}\fix{C} \pi'\act X$ for all $a \in \dom{\pi_{\catnew{c}}} \cup \catnew{c}$, where $\atnew{c_1} \notin \atm{\Upsilon_{\catnew{c}},\pi_{\catnew{c}},\pi'}$. By Inversion (Theorem~\ref{thm:miscellaneous}(\ref{thm:inversion}), it follows that
            \[
                \newswap{\pi'^{-1}(a)}{c_1} = \newswap{a}{c_1}^{\pi'^{-1}} \in \PN{}{\Upsilon_{\catnew{c},\atnew{c_1}}|_X}.
            \]
            Since $\atnew{c_1} \notin \dom{\pi'}$, we have $\pi'^{-1}(a) \neq \atnew{c_1}$. Moreover, $\newswap{\pi'^{-1}(a)}{c_1}\in \Perm{\atm{(\Upsilon_{\catnew{c},\atnew{c_1}})_{\fresh}}}$, which implies $\pi'^{-1}(a) \in \atm{(\Upsilon_{\catnew{c},\atnew{c_1}})_{\fresh}}$.

            Equivalently, this means that $a \in \pi'\act\atm{(\Upsilon_{\catnew{c},\atnew{c_1}})_{\fresh}}$ for all $a \in \dom{\pi_{\catnew{c}}} \cup \catnew{c}$. As a result, we have $\pi_{\catnew{c}} \in \Perm{\pi'\act\atm{(\Upsilon_{\catnew{c},\atnew{c_1}})_{\fresh}}}$. Consequently,
            \[
                \pi_{\catnew{c}}^{\pi'^{-1}} \in \Perm{\atm{(\Upsilon_{\catnew{c},\atnew{c_1}})_{\fresh}}} \subseteq \PN{}{\Upsilon_{\catnew{c}}|_X}.
            \]
            Finally, the result follows by an application of rule $(\frule{\faeq{C}}{var})$.
        \end{description}

        \item $t \equiv \tf{f^C}(t_0,t_1)$.

        \begin{description}
            \item[$(\Rightarrow)$] In this case, we have $\Upsilon_{\catnew{c}} \vdash \pi_{\catnew{c}} \fix{C} \tf{f^C}(t_0,t_1)$ which is the same as $\Upsilon_{\catnew{c}} \vdash\tf{f^C}( \pi_{\catnew{c}}\act t_0, \pi_{\catnew{c}}\act t_1) \aeq{C} \tf{f^C}(t_0,t_1)$. By Inversion (Theorem~\ref{thm:miscellaneous}(\ref{thm:inversion}),

            \begin{itemize}
                \item either $\Upsilon_{\catnew{c}} \vdash \pi_{\catnew{c}}\act t_0 \aeq{C} t_0$ and $\Upsilon_{\catnew{c}} \vdash \pi_{\catnew{c}}\act t_1 \aeq{C} t_1$.

                Then by induction, it follows that $\Upsilon_{\catnew{c},\atnew{c_1}} \vdash \newswap{a}{c_1}\act t_0\aeq{C} t_0$ and $\Upsilon_{\catnew{c},\atnew{c_1}} \vdash \newswap{a}{c_1}\act t_1\aeq{C} t_1$ for all $a \in \dom{\pi_{\catnew{c}}} \cup \catnew{c}$, where $\atnew{c_1} \notin \atm{\Upsilon_{\catnew{c}},\pi_{\catnew{c}}, t}$. By rule $(\frule{\faeq{C}}{\tf{f^C}})$, we obtain $\Upsilon_{\catnew{c},\atnew{c_1}} \vdash \newswap{a}{c_1}\fix{C} \tf{f^C}(t_0,t_1) $, for all $a \in \dom{\pi_{\catnew{c}}} \cup \catnew{c}$.


                \item or $\Upsilon_{\catnew{c}} \vdash \pi_{\catnew{c}}\act t_0 \aeq{C} t_1$ and $\Upsilon_{\catnew{c}} \vdash \pi_{\catnew{c}}\act t_0 \aeq{C} t_1$.

                By Equivariance and Equivalence of Theorem~\ref{thm:miscellaneous}, we have $\Upsilon_{\catnew{c}} \vdash \pi_{\catnew{c}}^2\act t_0 \aeq{C} t_0$ and  $\Upsilon_{\catnew{c}} \vdash \pi_{\catnew{c}}^2\act t_1 \aeq{C} t_1$. Proceeding similarly to the analysis in the previous case, we conclude that $\Upsilon_{\catnew{c},\atnew{c_1}} \vdash\newswap{a}{c_1}\fix{C} \tf{f^C}(t_0,t_1)$ for all $a\in \dom{\pi_{\catnew{c}}^2}\cup\catnew{c}$.

                Since $\dom{\pi_{\catnew{c}}^2} \subseteq \dom{\pi_{\catnew{c}}}$, it remains to show that $\Upsilon_{\catnew{c},\atnew{c_1}} \vdash \newswap{a}{c_1}\fix{C} \tf{f^C}(t_0,t_1)$ for all $a \in \dom{\pi_{\catnew{c}}} \setminus (\dom{\pi_{\catnew{c}}^2} \cup \catnew{c})$. Assume, by contradiction, that there exists some $a \in \dom{\pi_{\catnew{c}}} \setminus (\dom{\pi_{\catnew{c}}^2} \cup \catnew{c})$ such that $\Upsilon_{\catnew{c},\atnew{c_1}} \vdash \newswap{a}{c_1}\act \tf{f^C}(t_0,t_1)\naeq{C} \tf{f^C}(t_0,t_1)$. Observe that, since $a \notin \dom{\pi_{\catnew{c}}^2}$, it follows that $\pi_{\catnew{c}}^2(a) = a$. Therefore, $\pi_{\catnew{c}}$ can be expressed as $(a \ \pi_{\catnew{c}}(a)) \circ \rho$, where $\rho$ is a  disjoint from $(a \ \pi_{\catnew{c}}(a))$. We are assuming $a \notin \catnew{c}$, so we analyse $\pi_{\catnew{c}}(a)$:
                \begin{itemize}
                    \item If $\pi_{\catnew{c}}(a) \notin \catnew{c}$, then $\pi_{\catnew{c}}$ would contain the disjoint cycle $(a \ \pi_{\catnew{c}}(a))$ that does not mention any names from $\catnew{c}$, which contradicts the structure of $\pi_{\catnew{c}}$.

                    \item Hence, $\pi_{\catnew{c}}(a)$ must be an atom from $\catnew{c}$, say $\pi_{\catnew{c}}(a) \in \catnew{c}$.
                \end{itemize}
                Since we proved that $\Upsilon_{\catnew{c},\atnew{c_1}} \vdash \newswap{a}{c_1}\fix{C} \tf{f^C}(t_0,t_1)$ for all $a\in \dom{\pi_{\catnew{c}}^2}\cup\catnew{c}$.  In particular it holds for all $a\in \catnew{c}$. Then $ \Upsilon_{\catnew{c},\atnew{c_1}} \vdash (\pi_{\catnew{c}}(a) \ \atnew{c_1})\fix{C}\tf{f^C}(t_0,t_1)$, which is the same as $\Upsilon_{\catnew{c},\atnew{c_1}} \vdash (a \ \atnew{c_1})^{\pi_{\catnew{c}}}\fix{C}\tf{f^C}(t_0,t_1)$. By Equivariance (Theorem~\ref{thm:miscellaneous}(\ref{thm:object-equivariance})), this implies $\Upsilon_{\catnew{c},\atnew{c_1}} \vdash (a \ \atnew{c_1})\fix{C}\pi_{\catnew{c}}\act\tf{f^C}(t_0,t_1)$ and so the result follows by Equivalence of Theorem~\ref{thm:miscellaneous}.
            \end{itemize}

            \item[$(\Leftarrow)$]  Suppose $\Upsilon_{\catnew{c},\atnew{c_1}} \vdash \newswap{a}{c_1}\fix{C} \tf{f^C}(t_0,t_1)$ for all $a\in \dom{\pi_{\catnew{c}}}\cup\catnew{c}$ where $\atnew{c_1}\notin \atm{\Upsilon_{\catnew{c}},\pi_{\catnew{c}},t}$. By Lemma~\ref{alemma:characterization-fix-commutative}, we have $\Upsilon_{\catnew{c},\atnew{c_1}} \vdash \newswap{a}{c_1}\fix{C} t_0$ and $\Upsilon_{\catnew{c},\atnew{c_1}} \vdash \newswap{a}{c_1}\fix{C} t_1$ for all $a\in \dom{\pi_{\catnew{c}}}\cup\catnew{c}$. By induction, we obtain $\Upsilon_{\catnew{c}} \vdash \pi_{\catnew{c}} \fix{C} t_0$ and $\Upsilon_{\catnew{c}} \vdash \pi_{\catnew{c}} \fix{C} t_1$ and thus by rule $(\frule{\faeq{C}}{\tf{f^C}})$ we get $\Upsilon_{\catnew{c}} \vdash \pi_{\catnew{c}}\fix{C} \tf{f^C}(t_0,t_1)$.
        \end{description}
    \end{itemize}
\end{proof}

\begin{lemma}\label{alemma:fix-point-split}
    $\Upsilon_{\catnew{c}} \vdash \pi \fix{C} t$ iff $\Upsilon_{\catnew{c}} \vdash \pi_{\catnew{c}} \fix{C} t$ and $\Upsilon_{\catnew{c}} \vdash \pi_{\neg\catnew{c}} \fix{C} t$.
\end{lemma}

\begin{proof}
    The left-to-right case follows by Lemma~\ref{alemma:fix-point-composition-inverse}(\ref{alemma:fix-point-composition}). For the right-to-left case, assume that  $\Upsilon_{\catnew{c}} \vdash \pi \fix{C} t$. By Soundness (Theorem~\ref{thm:soundness-completeness-fix}(\ref{thm:soundness-fix})), we have $\Upsilon_{\catnew{c}} \vDash \pi \fix{C} t$. As a consequence of Pitts' equivalence (Lemma~\ref{lemma:pitts-eq-generalized}), we have $\Upsilon_{\catnew{c}} \vDash \pi_{\catnew{c}} \fix{C} t$ and $\Upsilon_{\catnew{c}} \vDash \pi_{\neg\catnew{c}} \fix{C} t$. Therefore, the result follows by Completeness (Theorem~\ref{thm:soundness-completeness-fix}(\ref{thm:completeness-fix})).
\end{proof}

 \begin{lemma}\label{alemma:perm_dec}
     Let $I$ be a non-empty, finite set of indices. Suppose, for each $i\in I$, that $\Upsilon_{\catnew{c}} \vdash \pi_i\fix{C}t$. Let $\pi$ be a permutation and $\atnew{\pvec{c}'}\subseteq \catnew{c}$ such that
     \begin{equation}
         \pi \in \Perm{\left(\bigcup_{i\in I} \dom{\pnew{\pi_i}{\atnew{\pvec{c}'}}}\right)\cup \atnew{\pvec{c}'}}\alert{\circ}\pair{\{\npnew{\pi_i}{\atnew{\pvec{c}'}} \mid i\in I\}}.
     \end{equation}
     Then $\Upsilon_{\catnew{c}} \vdash \pi\fix{C}t$.
\end{lemma}

 \begin{proof}
     By Lemma~\ref{alemma:fix-point-split}, we have: $\Upsilon_{\catnew{c}}\vdash \pnew{\pi_i}{\atnew{\pvec{c}'}}\fix{C} t$ and $\Upsilon_{\catnew{c}}\vdash \npnew{\pi_i}{\atnew{\pvec{c}'}}\fix{C} t$, for all $i\in I$. Write $\pi$ as $\pi = \eta_1\circ \eta_2$ where $\eta_1 \in \Perm{\left(\bigcup_{i\in I} \dom{\pnew{\pi_i}{\atnew{\pvec{c}'}}}\right)\cup \atnew{\pvec{c}'}}$ and $\eta_2 \in\pair{\{\npnew{\pi_i}{\atnew{\pvec{c}'}} \mid i\in I\}}$. Thus, on one hand the former combined with Lemmas~\ref{alemma:characterization-fix-c-general} and~\ref{alemma:fix-point-formed-by-fresh-names} yields $\Upsilon_{\catnew{c}} \vdash \eta_1 \fix{C} t$. On the other hand, the latter combined with Lemma~\ref{alemma:fix-point-composition} yields $\Upsilon_{\catnew{c}}\vdash \eta_2\fix{C} t$. Then the result follows by Lemma~\ref{alemma:fix-point-composition}(\ref{alemma:fix-point-composition}).

 \end{proof}


Given contexts $\Upsilon_{\catnew{c}}$ and $\Upsilon'_{\atnew{\pvec{c}'}}$ and $\sigma_{\atnew{\pvec{c}''}}$ a $\new$-substitution. We introduce some notations:
\begin{itemize}
    \item $\Upsilon_{\catnew{c}}\vdash \Upsilon'_{\atnew{\pvec{c}'}}\sigma_{\atnew{\pvec{c}''}}$ for $\Upsilon_{\catnew{c}\cup\atnew{\pvec{c}'}\cup\atnew{\pvec{c}''}}\vdash \Upsilon'\sigma$.
    \item $\Upsilon_{\catnew{c}}\vdash s\sigma_{\atnew{\pvec{c}''}} \aeq{C} t\sigma_{\atnew{\pvec{c}''}}$ for $\Upsilon_{\catnew{c}\cup\atnew{\pvec{c}''}}\vdash s\sigma \aeq{C} t\sigma$.

    \item $\Upsilon_{\catnew{c}}\vdash \pi\fix{C} t\sigma_{\atnew{\pvec{c}''}}$ for $\Upsilon_{\catnew{c}\cup\atnew{\pvec{c}''}}\vdash \pi \fix{C} t\sigma$.
\end{itemize}

 \begin{lemma}[Preservation by substitution]\label{alemma:preservation-by-subs}
     Suppose $\Upsilon_{\catnew{c}}$ and $\Upsilon'_{\atnew{\pvec{c}'}}$ are contexts and let $\sigma_{\atnew{\pvec{c}''}}$ be a substitution such that $\Upsilon_{\catnew{c}}\vdash \Upsilon'_{\atnew{\pvec{c}'}}\sigma_{\atnew{\pvec{c}''}}$. If $\Upsilon'_{\atnew{\pvec{c}'}}\vdash s\aeq{C}t$ then $\Upsilon_{\catnew{c}\cup\atnew{\pvec{c}'}}\vdash s\sigma_{\atnew{\pvec{c}''}}\aeq{C}t\sigma_{\atnew{\pvec{c}''}}$.
 \end{lemma}

 \begin{proof}
    Proof by induction on the last rule applied. The only non-trivial case is the rule $(\frule{\faeq{C}}{var})$. In this case, $s \equiv \pi_1\act X$ and $t \equiv \pi_2\act X$. Hence, $\Upsilon'_{\atnew{\pvec{c}'}}\vdash \pi_1\act X\aeq{C} \pi_2\act X$. By Inversion (Theorem~\ref{thm:miscellaneous}(\ref{thm:inversion})), we have $\pi_2^{-1}\circ\pi_1\in \PN{}{\Upsilon'_{\atnew{\pvec{c}'}}|_X} = \Perm{\atm{(\Upsilon_{\atnew{\pvec{c}'}}|_X)_{\fresh}}}\alert{\circ}\pair{\perm{}{(\Upsilon_{\atnew{\pvec{c}'}}|_X)_{\fix{C}}}}$. By hypothesis, $\Upsilon_{\catnew{c}\cup\atnew{\pvec{c}'}\cup\atnew{\pvec{c}''}}\vdash \pi\fix{C} Y\sigma$ for every $\pi\fix{C} Y\in \Upsilon'$. In particular, $\Upsilon_{\catnew{c}\cup\atnew{\pvec{c}'}\cup\atnew{\pvec{c}''}} \vdash \pi\fix{C} X\sigma$ for every $\pi\fix{C} X\in \Upsilon'|_X$. By Lemma~\ref{alemma:perm_dec}, we obtain $\Upsilon_{\catnew{c}\cup\atnew{\pvec{c}'}\cup\atnew{\pvec{c}''}} \vdash (\pi_2^{-1}\circ\pi_1)\fix{C} X\sigma$, which is the same as  $\Upsilon_{\catnew{c}\cup\atnew{\pvec{c}'}\cup\atnew{\pvec{c}''}} \vdash (\pi_2^{-1}\circ\pi_1)\act X\sigma \aeq{C} X\sigma$. By Equivariance (Theorem~\ref{thm:miscellaneous}(\ref{thm:object-equivariance})), this is equivalent to $\Upsilon_{\catnew{c}\cup\atnew{\pvec{c}'}\cup\atnew{\pvec{c}''}} \vdash \pi_1\act X\sigma \aeq{C} \pi_2\act X\sigma$ Therefore, $\Upsilon_{\catnew{c}\cup\atnew{\pvec{c}'}} \vdash \pi\act X\sigma_{\atnew{\pvec{c}''}} \aeq{C} \pi_2\act X\sigma_{\atnew{\pvec{c}''}}$.
\end{proof}


\section{Proofs of Section~\ref{sec:nominal-c-unification} - Nominal \texorpdfstring{$\C$}{C}-Unification}\label{app:nominal-c-unification}

For the rest of this section, for a solvable problem $\probc = \newc{c}{}.Pr$ and a solution $\npair{\Psi,\sigma}{\atnew{\pvec{c}'}}\in \U{\probc}$, we will write $\Psi_{\atnew{\pvec{c}'}} \vdash (Pr)\sigma$ to abbreviate that $\Psi_{\atnew{\pvec{c}'}} \vdash s\sigma\aeq{C} t\sigma$ for all $s\aeq{C}^? t\in Pr$.

\begin{lemma}\label{alemma:quasiorder}
    The relation $\ins{}$ defines a quasiorder (i.e. reflexive and transitive) in $\mathcal{U}(\probc)$.
\end{lemma}

\begin{proof}
\begin{itemize}
    \item \emph{Reflexivity.}  Just note that for every pair $\npair{\Psi,\sigma}{\catnew{c}}\in \mathcal{U}(\probc)$, we have $\npair{\Psi,\sigma}{\catnew{c}} \ins{\id} \npair{\Psi,\sigma}{\catnew{c}}$.

    \item \emph{Transitivity.} Suppose $ \npair{\Psi_1,\sigma_1}{\atnew{\vec{c}_1}} ,\npair{\Psi_2,\sigma_2}{\atnew{\vec{c}_2}},\npair{\Psi_3,\sigma_3}{\atnew{\vec{c}_3}}\in \mathcal{U}(\probc)$ are such that $\npair{\Psi_1,\sigma_1}{\atnew{\vec{c}_1}} \ins{\delta_1} \npair{\Psi_2,\sigma_2}{\atnew{\vec{c}_2}}$ and $\npair{\Psi_2,\sigma_2}{\atnew{\vec{c}_2}} \ins{\delta_2} \npair{\Psi_3,\sigma_3}{\atnew{\vec{c}_3}}$. Then, by definition, we have $\atnew{\vec{c}_1} \subseteq \atnew{\vec{c}_2} \subseteq \atnew{\vec{c}_3}$ and
          \begin{enumerate}
              \item $(\Psi_2)_{\atnew{\vec{c}_2}}\vdash \Psi_1\delta_1$  and  $(\Psi_2)_{\atnew{\pvec{c}_2}}\vdash X\sigma_2\aeq{C}X\sigma_1\delta_1$ for all $X\in\V$.
              \item  $(\Psi_3)_{\atnew{\vec{c}_3}}\vdash \Psi_2\delta_2$  and  $(\Psi_3)_{\atnew{\vec{c}_3}}\vdash X\sigma_3\aeq{C}X\sigma_2\delta_2$  for all $X\in\V$.
          \end{enumerate}
          Take $\delta = \delta_1\delta_2$. We know $(\Psi_3)_{\atnew{\vec{c}_3}}\vdash \Psi_2\delta_2$. Moreover, we have $(\Psi_2)_{\atnew{\vec{c}_2}}\vdash \Psi_1\delta_1$. Applying Lemma~\ref{alemma:preservation-by-subs} yields $(\Psi_3)_{\atnew{\vec{c}_3}}\vdash \Psi_1\delta_1\delta_2$ because $\atnew{\vec{c}_2}\subseteq\atnew{\vec{c}_3}$.

          For all $X\in \V$, we have that $(\Psi_2)_{\atnew{\vec{c}_2}}\vdash X\sigma_2\aeq{C}X\sigma_1\delta_1$. By $(\Psi_3)_{\atnew{\vec{c}_3}}\vdash \Psi_2\delta_2$, Lemma~\ref{alemma:preservation-by-subs} and $\atnew{\vec{c}_2}\subseteq\atnew{\vec{c}_3}$, we obtain  $(\Psi_3)_{\atnew{\vec{c}_3}}\vdash X\sigma_2\delta_2\aeq{C}X\sigma_1\delta_1\delta_2$ for all  $X\in\V$. Moreover, we know that $(\Psi_3)_{\atnew{\vec{c}_3}}\vdash X\sigma_3\aeq{C}X\sigma_2\delta_2$ for all $X\in\V$. Therefore, by transitivity of $\aeq{C}$, it follows that  $(\Psi_3)_{\atnew{\vec{c}_3}}\vdash X\sigma_3\aeq{C}X\sigma_1\delta_1\delta_2$ for all $X\in \V$.
\end{itemize}

\end{proof}

\begin{lemma}[Closure by Instantiation]\label{alemma:closure-by-instantiation}
    Suppose $\npair{\Psi_1,\sigma_1}{\atnew{\vec{c}_1}}\in \mathcal{U}(\probc)$ and $\npair{\Psi_1,\sigma_1}{\atnew{\vec{c}_1}} \ins{} \npair{\Psi_2,\sigma_2}{\atnew{\vec{c}_2}}$. Then $ \npair{\Psi_2,\sigma_2}{\atnew{\vec{c}_2}} \in \mathcal{U}(\probc)$.
\end{lemma}

\begin{proof}
     Suppose $\probc = \newc{c}{}.Pr$. Let $\npair{\Psi_1,\sigma_1}{\atnew{\vec{c}_1}}\in \mathcal{U}(\probc)$. Then by definition, we have $(\Psi_1)_{\atnew{\vec{c}_1}} \vdash (Pr)\sigma_1$ and $(\Psi_1)_{\atnew{\pvec{c}_1}} \vdash X\sigma_1 \aeq{C} X\sigma_1\sigma_1$ for all $X\in \V$. From  $\npair{\Psi_1,\sigma_1}{\atnew{\vec{c}_1}} \ins{} \npair{\Psi_2,\sigma_2}{\atnew{\vec{c}_2}}$, we know that there exist a substitution $\delta$ such that:

     \begin{enumerate}
         \item $(\Psi_2)_{\atnew{\vec{c}_2}}\vdash \Psi_1\delta$.

         \item $(\Psi_2)_{\atnew{\vec{c}_2}}\vdash X\sigma_2\aeq{C}X\sigma_1\delta$, for all $X\in \V$;
     \end{enumerate}

     \begin{claim}[1]
          We claim that $(\Psi_2)_{\atnew{\vec{c}_2}}\vdash t\sigma_2\aeq{C}t\sigma_1\delta$ for all nominal term $t$. The proof proceeds by straightforward induction on the structure of the nominal term $t$ utilizing item 2 as a key component of the argument.
     \end{claim}

     \begin{claim}[2]
          We claim that $(\Psi_2)_{\atnew{\vec{c}_2}} \vdash (Pr)\sigma_2$. For each $s\aeq{C}^?  t\in Pr$, we have $(\Psi_1)_{\atnew{\vec{c}_1}} \vdash s\sigma_1 \aeq{C} t\sigma_1$. By Lemma~\ref{alemma:preservation-by-subs}, it follows that $(\Psi_2)_{\atnew{\vec{c}_2}} \vdash s\sigma_1\delta \aeq{C} t\sigma_1\delta$. By Claim 1 and transitivity of $\aeq{C}$, it follows that $(\Psi_2)_{\atnew{\vec{c}_2}} \vdash s\sigma_2 \aeq{C} t\sigma_2$. 
     \end{claim}
\end{proof}

For the  termination proof, we define the \emph{size of a nominal term} $t$, denoted by $|t|$, inductively by: $|a| := 1, |\pi\act X| := 1, |\tf{f}(\tilde{t})_n| := 1+|t_1|+\ldots+|t_n|$ for $\tf{f}\in\Sigma$, and $|[a]t| := 1+|t|$. 
The \emph{size of a constraint} is defined by $|s\aeq{C}^? t| := |s|+|t|$. As a consequence, for all term $t$ and permutation $\pi$, we have that $|\pi\act t| = |t|$.

\termination*

\begin{proof}
 Define a measure for the size of a problem $\probc = \newc{c}{}.Pr$ as $[\probc] = (n,M)$, where $n$ is the number of distinct variables occurring in the constraints of $Pr$ and $M$ is the multiset of the sizes of the equality constraints that are not of the form $\pi\act X\aeq{C}^? X$ that are occurring in $Pr$. To compare $[\probc]$ and $[\probc']$ we use the lexicographic order $>_{lex}$ that is defined upon the product of the usual order on natural numbers, $>$, and its multiset extension $>_{mul}$, that is,  $>_{lex} \; \triangleq \; > \times >_{mul}$. 
\end{proof}

\begin{lemma}\label{alemma:preservation-solution}
     Let $\probc$ be a problem such that $\probc\overset{*}{\Longrightarrow} \exprob$ without using instantiating rules. Assume $\probc \overset{*}{\Longrightarrow} \probc'$ where $\probc'\in \exprob$. Then
     \begin{enumerate}
         \item  $\U{\probc}=\U{\probc'}$. 

         \item If $\probc'$ contains inconsistent constraints, then $\U{\probc} = \emptyset$.
     \end{enumerate}
 \end{lemma}

\begin{proof}
    For the first item the proof follows by induction on the length of the derivation $\probc\overset{*}{\Longrightarrow} \probc'$.

    \paragraph{(Base).} The base of the induction corresponds to the case where $\probc\overset{0}{\Longrightarrow} \probc'$, which implies that $\probc = \probc'$ and hence  $\U{\probc} = \U{\probc'}$.

    \paragraph{(Inductive step).} Now, we are going to prove that the result holds for reductions of length $n>0$, that is, $\probc\overset{n}{\Longrightarrow} \probc'$. Our induction hypothesis is
           \begin{equation}\label{eq:induction-hyp-1}
                \text{If ${\probc}_1\overset{n-1}{\Longrightarrow} {\probc}_2$, then $\U{{\probc}_1}=\U{{\probc}_2}$ }\tag{I.H.}
            \end{equation}
            Note that we can rewrite  $\probc\overset{n}{\Longrightarrow} \probc'$ as  $\probc\overset{n-1}{\Longrightarrow} \probc''\Longrightarrow \probc'$. By (\ref{eq:induction-hyp-1}), it follows that $\U{\probc} = \U{\probc''}$. It remains to analyse the last step of the reduction, specifically $\probc''\Longrightarrow \probc'$. We present a illustrative case: The last rule is $(var)$. In this case, $\probc'' = \newc{c}{}.Pr_0\uplus\{\pi\act X \aeq{C}^? \pi'\act X\} \Longrightarrow \newc{c}{}.Pr_0 \cup\{\pi'^{-1}\circ\pi\act X\aeq{C}^? X\} = \probc'$, where $\pi'\neq \id$. If $\npair{\Psi,\sigma}{\atnew{\pvec{c}'}}\in\U{\probc''}$, then
                $\Psi_{\atnew{\pvec{c}'}} \vdash (\pi\act X)\sigma \aeq{C} \pi'\act X\sigma$. Since $(\rho\act X)\sigma \equiv \rho\act X\sigma$, it follows that $\Psi_{\atnew{\pvec{c}'}} \vdash \pi\act (X\sigma) \aeq{C} \pi'\act (X\sigma)$. By Equivariance (Theorem~\ref{thm:miscellaneous}(\ref{thm:object-equivariance})), we have $\Psi_{\atnew{\pvec{c}'}} \vdash (\pi'^{-1}\circ\pi)\act (X\sigma) \aeq{C} X\sigma$. Therefore, $\npair{\Psi,\sigma}{\atnew{\pvec{c}'}}\in\U{\prob'}$ and hence $\U{\probc''} \subseteq \U{\probc'}$. For the other inclusion the argument is the same.

                
        The second item of the lemma follows directly from the fact that inconsistent constraints are not derivable, therefore have no solutions.
 \end{proof}

\correctness*

\begin{proof}
  By induction on the number of steps in the reduction $\probc \overset{*}{\Longrightarrow} \nf{\probc}$.

       \paragraph{(Base).}  This corresponds to the case where $\probc\overset{0}{\Longrightarrow} \nf{\probc}$. In this case, $\nf{\probc} = \probc = \newc{c}{}.Pr$ and so $Pr$ contains only consistent constraints. Then $Pr = \{\pi_i\act X_i\aeq{C}^? X_i \mid i=1,\ldots,n\}$. By construction, the pair $\sol{\probc} = \{\npair{\Psi,Id}{\catnew{c}}\}$, where $\Psi_{\catnew{c}} = \newc{c}{}. \{\pi_i\fix{C} X_i \mid i=1,\ldots,n\}$, satisfies $\Psi_{\catnew{c}} \vdash \pi_i\act X_iId \aeq{C} X_iId$  for all $i=1,\ldots,n$. Therefore, $\sol{\probc} \subseteq \U{\probc}$.

           For the second item, for any other $\npair{\Phi,\tau}{\atnew{\pvec{c}''}} \in \U{\probc}$, by definition, $\tau$ is such that $\Phi_{\atnew{\pvec{c}''}} \vdash (Pr)\tau$ which is the same as $\Phi_{\atnew{\pvec{c}''}} \vdash \Psi\tau$. Moreover,  $\Phi_{\atnew{\pvec{c}''}} \vdash X\tau \aeq{C} XId\tau$ follows for all $X\in \V$ by reflexivity. Therefore, this proves that $\npair{\Psi,Id}{\catnew{c}} \ins{} \npair{\Phi,\tau}{\atnew{\pvec{c}''}}$ and the second item follows.

        \paragraph*{(Inductive step).}  Now, we are going to prove that the result holds for reductions of length $n>0$, that is, $\probc\overset{n}{\Longrightarrow} \nf{\probc}$. We rewrite it as follows: $\probc \Longrightarrow \probc''\overset{n-1}{\Longrightarrow} \nf{\probc}$.

           \begin{itemize}
                \item Suppose $\probc \Longrightarrow \probc''$ by some non-instantiating simplification. Then using Lemma~\ref{alemma:preservation-solution}, we know that $\U{\probc} = \U{\probc''}$.

                By induction, we have $\sol{\probc''} \subseteq \U{\probc''}$. Consequently, $\sol{\probc''} \subseteq \U{\probc}$. Since no instantiation rule was used, by the construction of $\sol{-}$ we have that $\sol{\probc} = \sol{\probc''}$ and the first item follows.

                For the second item, for any other $\npair{\Phi,\tau}{\atnew{\pvec{c}''}} \in \U{\probc}$, by induction, there is some $\npair{\Psi,\sigma}{\atnew{\pvec{c}'}} \in \sol{\probc''}$ such that $\npair{\Psi,\sigma}{\atnew{\pvec{c}'}} \ins{}\npair{\Phi,\tau}{\atnew{\pvec{c}''}}$ and the result follows because $\sol{\probc} = \sol{\probc''}$.
                

                \item Suppose $\probc \overset{\theta}{\Longrightarrow} \probc''$  by an instantiating rule, say $(inst_1)$. So $\probc = \newc{c}{}.Pr\uplus\{\pi\act X \aeq{C}^? t\} \overset{\theta}{\Longrightarrow} \newc{c}{}.Pr\theta = \probc''$ where $\theta = [X\mapsto \pi^{-1}\act t]$ and $X\notin \var{t}$. 
                
                Let $\npair{\Psi,\sigma}{\atnew{\pvec{c}'}} \in \sol{\probc}$, so by construction $\sigma = \theta\sigma'$ and $\npair{\Psi,\sigma'}{\atnew{\pvec{c}'}}\in \sol{\probc''}$. By induction, $\sol{\probc''} \subseteq \U{\probc''}$ which implies $\Psi_{\atnew{\pvec{c}'}} \vdash (Pr\theta)\sigma'$, that is the same as  $\Psi_{\atnew{\pvec{c}'}} \vdash (Pr)(\theta\sigma')$ and thus $\npair{\Psi,\sigma}{\atnew{\pvec{c}'}} \in \U{\probc}$ which proves the first item.

                It remains to prove that $\sol{\probc}$ is a complete set of solutions of $\probc$. So, take $\npair{\Phi,\tau}{\atnew{\pvec{c}''}} \in \U{\probc}$. By definition, $\Phi_{\atnew{\pvec{c}''}} \vdash (Pr\uplus\{\pi\act X\aeq{C}^? t\})\tau$. In particular, $\Phi'_{\atnew{\pvec{c}''}} \vdash (\pi\act X)\tau \aeq{C} t\tau$ which, by Equivariance (Theorem~\ref{thm:miscellaneous}(\ref{thm:object-equivariance})) and the fact that permutations and substitutions commute, results in $\Phi_{\atnew{\pvec{c}''}} \vdash X\tau \aeq{C} (\pi^{-1}\act t)\tau$ that is the same as $\Phi_{\atnew{\pvec{c}''}} \vdash X\tau \aeq{C} X\theta\tau$.
        
                Let $\tau'$ be the substitution that acts just like $\tau$, only $X\tau \equiv X$ (remember that $X\tau \not\equiv X$). Then since $X\notin \var{X\theta}$, we have $X\theta\tau \equiv X\theta\tau'$ and hence $\Phi_{\atnew{\pvec{c}''}} \vdash X\tau \aeq{C} X\theta\tau'$. For all other variable $Y$, we have $\Phi_{\atnew{\pvec{c}''}} \vdash Y\tau \aeq{C} Y\tau'$. Since $Y\theta \equiv Y$, we conclude that  $\Phi_{\atnew{\pvec{c}''}} \vdash Z\tau \aeq{C} Z\theta\tau'$ holds for all variable $Z\in \V$. Then $\Phi_{\atnew{\pvec{c}''}} \vdash s\tau \aeq{C} s\theta\tau'$ for all nominal term $s$. As a consequence, because $\Phi_{\atnew{\pvec{c}''}} \vdash (Pr)\tau$ we conclude $\Phi_{\atnew{\pvec{c}''}} \vdash (Pr)(\theta\tau')$. This is the same as $\Phi_{\atnew{\pvec{c}''}} \vdash (Pr\theta)\tau'$. So $\npair{\Phi,\tau'}{\atnew{\pvec{c}''}}\in \U{\probc''}$ and by inductive hypothesis there is some $\npair{\Psi,\sigma}{\atnew{\pvec{c}'}}\in\sol{\probc''}$ such that  $\npair{\Psi,\sigma}{\atnew{\pvec{c}'}}\ins{} \npair{\Phi,\tau'}{\atnew{\pvec{c}''}}$. By construction, $\npair{\Psi,\theta\sigma}{\atnew{\pvec{c}'}}\in\sol{\probc}$.

                \begin{claim}[1]
                    \sloppy{We claim that $\npair{\Psi,\theta\sigma}{\atnew{\pvec{c}'}} \ins{} \npair{\Phi,\theta\tau'}{\atnew{\pvec{c}''}}$. Since $\npair{\Psi,\sigma}{\atnew{\pvec{c}'}}\ins{} \npair{\Phi,\tau'}{\atnew{\pvec{c}''}}$ we know that there is some $\delta$ such that $\Phi_{\atnew{\pvec{c}''}} \vdash \Psi\delta$ and $\Phi_{\atnew{\pvec{c}''}} \vdash Z\tau' \aeq{C} Z\sigma\delta$ for all $Z\in \V$. Then, $\Phi_{\atnew{\pvec{c}''}} \vdash u\tau' \aeq{C} u\sigma\delta$ holds for all terms $u$, in particular, for every $u\equiv X\theta$ where $X\in V$. So $\Phi_{\atnew{\pvec{c}''}} \vdash X\theta\tau' \aeq{C} X\theta\sigma\delta$ for all $X\in \V$, proving the claim.}
                \end{claim}

                \begin{claim}[2]
                    We claim that $\npair{\Phi,\theta\tau'}{\atnew{\pvec{c}''}} \ins{} \npair{\Phi,\tau}{\atnew{\pvec{c}''}}$. In fact, take $\delta = Id$. Then $\Phi_{\atnew{\pvec{c}''}} \vdash \Phi Id$ trivially. Furthermore, we proved that $\Phi_{\atnew{\pvec{c}''}} \vdash Z\tau \aeq{C}^? Z\theta\tau'Id$ for all variable $Z\in\V$. 
                \end{claim}

        The result follows by combining Claims (1) and (2) and using Lemma~\ref{alemma:quasiorder}, concluding the proof of Correctness.

        \end{itemize}
\end{proof}