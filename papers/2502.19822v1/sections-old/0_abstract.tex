\begin{abstract}
\begin{abstract}
% Social service practitioners face multiple obstacles in their work, including administrative burdens and challenges in decision-making. Generative AI has the potential to streamline these tasks and improve client outcomes. However, these systems exacerbate existing concerns about AI such as overreliance, erosion of worker skill, and algorithmic bias, and risk undermining the professional discretion essential to social service workers. In this paper, we explore these issues through a participatory design study with 51 social service practitioners. We present an empirical investigation into the interaction between practitioners and a co-designed prototype AI tool, which assists in client assessment and intervention planning. Through thematic analysis, we propose a framework describing the interplay between the core ethos of practitioners, practitioner attitudes towards AI, and the resulting considerations for designing AI systems. We also propose a framework for generative AI in human-AI collaboration, and conclude with novel design guidelines for AI systems in the sector.

In social service, administrative burdens and decision-making challenges hinder practitioners from performing effective casework. Generative AI can potentially streamline or automate these tasks, but also exacerbates existing concerns about AI such as overreliance, algorithmic bias, and loss of identity within the profession. We explore these issues through a participatory design study \textit{(n=51)}, investigating the interactions between practitioners and a co-designed AI tool for client assessment and intervention planning. We discover novel use cases for AI in this sector, and discuss how the interplay between the core ethos of practitioners, their attitudes towards AI, and the resulting considerations for the effective design of such systems. We propose an alternative framing of AI in the context of human-AI collaboration, and conclude with new design guidelines for AI systems in the social service sector.
% and algorithmic bias, and risks replacing the professional discretion at the heart of social work and threatening the profession itself. 


\end{abstract}