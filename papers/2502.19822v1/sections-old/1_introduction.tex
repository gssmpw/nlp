\section{Introduction}

The social service sector is crucial for a just and humane society, addressing various social injustices and supporting the well-being of individuals and communities \cite{difranks2008social}. Social service agencies render critical services such as client home visits, case analysis, and intervention planning, aiming to aid vulnerable clients in the best way possible. Social service practitioners (SSPs) must consider the myriad interacting factors in clients' social networks, physical environment, and intrapersonal cognitive and emotional systems, in context of clients' beliefs, perceptions, and desires, and all while respecting individual worth and dignity \cite{rooney2017direct}. This requires extensive training and hands-on experience, which may be challenging given the endlessly varying situations a worker may encounter \cite{rooney2017direct}.
% social work is hard -> needs experiences

These demands put a strain on the limited pool of social service manpower \cite{cho2017determinants}. SSPs grapple with time-consuming data processing \cite{singer_ai_2023, tiah_can_2024} and administrative \cite{meilvang_working_2023} tasks, compounded by the psychologically stressful nature of the job \cite{kalliath2012work}. Newer workers may experience challenging, unfamiliar client interactions, while experienced workers also face the added burden of mentoring junior workers. Current artificial intelligence (AI) systems aim to alleviate this through \textit{decision-support tools} \cite{gambrill2001need, de2020case, brown2019toward} which provide statistically validated assessments of client conditions \cite{gillingham2019can, van2017predicting}, thereby raising service quality and consistency. However, attempts to integrate AI systems in social service have faced a myriad of difficulties: mistrust in and uncertainty about technology \cite{gambrill2001need, brown2019toward}, worker confusion due to unclear organisational direction \cite{kawakami2022improving}, AI systems' inability to include local and contextual knowledge, and fear of AI potentially replacing jobs. As such, many attempts have been met with failure and inadequacies \cite{saxena2021framework, kawakami2022improving}.

% Recent, more promising systems powered by Large Language Models (LLMs) \cite{socialworkmagic2024, caseworthy2024} provide more flexible and detailed analyzes and recommendations to clients. Their ability to efficiently process vast amount of heterogeneous case data and provide a wide range of useful outputs to users significantly increases their utility to SSPs.

% MH: social work is hard -> manpower is not enough -> ai can help human judgment to lighten workload -> llm is better
% i feel a bit disconnected when i read new workers and experienced workers, why this message is important? i know that we will talk about the supervision stuffs in results, but the motivation seems can be strengthened? Like manpower is not enough is because new -> experienced is not easy..?

% MH: why talked about human judgment here? deciding the intervention protocol mentioned in the first paragraph? i feel decide intervention scheme and provide care need to be aligned? like 'provide the consistent, high-quality, and personalized care' = manpower, but 'Deciding how to help a given client is an intricate task' = manpower? i guess this can be clarified clearer. I feel manpower = labor, but decision-making is not very similar with this word
% \cite{socialworkmagic2024, caseworthy2024} they are LLM system for social work already? or just LLMs? how about change their position here 'These systems hold vast potential in significantly advancing the state of the social service sector \cite{socialworkmagic2024, caseworthy2024}.'

% Digital systems have long served to alleviate some of the burden, by both deputising for workers in client-facing care \cite{ruggiano2023examining, li2023systematic} 


Recent technological advancements, in particular generative AI, have the potential to play a greater, transformative role in the social service sector. Generative AI systems are capable of a wider range of tasks, including analysing written case recordings, performing qualitative risk assessments, providing crisis assistance, and aiding prevention efforts \cite{reamer_artificial_2023}.  
% which can generate novel outputs ranging from text reports to case assessments from large amounts of training data, are qualitatively different from most historical examples of automation technologies \cite{noy2023experimental}, and offer significant potential to transform the social service sector, with its ability to handle a myriad of social service tasks like analyzing case data, performing risk assessments, provide crisis assistance and prevention efforts, enhance social service outcomes and more \cite{reamer_artificial_2023}. This is especially crucial in a time-pressed sector like social service that suffers from resource constraints and highly manual processes, thereby helping to reduce the burdens faced by SSPs and freeing up their time to better serve clients, organizations and communities \cite{tiah_can_2024, fernando_integration_2023}. These systems hold vast potential in significantly advancing the state of the social service sector. Generative AI  
However, despite their advantages, these systems may exacerbate existing concerns with AI, risk overreliance \cite{van2023chatgpt} and potentially erode the human skills and values \cite{littlechild2008child, oak2016minority} core to the social service profession. 
% \cite{kawakami2022improving, de2020case} and adjacent healthcare \cite{devaraj2014barriers, elwyn2013many} sectors investigates systems that are vastly more primitive and limited in scope. 
As a novel technology, the efficacy of generative AI systems are still largely under-explored and untested in extant research \cite{gambrill2001need, de2020case, brown2019toward}, thus necessitating deeper scrutiny of AI technologies in social service. 

% Such systems, however, pale in comparison to the opportunities afforded by recent Large Language Models (LLMs), which demonstrate excellent performance in aggregating, understanding, and analysing large amounts of information. This lends them the ability to decision-making assistants in complex use cases. The core of social work is in humanising and thoroughly analysing each client,  The social work field offers many instances where information about a client must be organised, analysed, and weighed, in order to come up with a suitable and tailored intervention plan. This task is highly complex, since information can be missed or misinterpreted and theories can be wrongly applied by fallible humans. Modern LLMs like GPT-4 \cite{achiam2023gpt} seem well-placed to reduce human error and workload, with their ability to process large quantities of information almost instantly. 

% Uniquely, social work also presents the concern of responsibility attribution. Unlike, say, a travel planner that recommends holiday destinations or an ecommerce algorithm that suggests products, a social work assistant tool produces outputs with weighty moral implications. As a case worker's recommendations to and therapeutic work with a client can significantly affect their life trajectory, adopting an AI's suggestions leads to the natural question of who should be blamed for unfavourable outcomes. Blindly following AI guidance is clearly inappropriate. The resulting questions are therefore the degree to which AI guidance should be followed, and how AI tools should be designed and implemented to provide the right amount of guidance.

% These questions cannot be answered solely by system designers. The insight of users, i.e. social work practitioners, is invaluable in effective co-design of social work assistant tools.

% To fully investigate the risks, unknowns, and opportunities in AI-powered social service systems, 
To address this gap, this study therefore co-designs and pilots a novel generative AI tool with 51 SSPs, to better understand practitioners' perspectives on implementing AI in the social service sector. Uniquely, our study includes a diverse range of roles spanning the full range of social service practice, including many caseworkers who serve the general population, thus offering a broader perspective than past studies with more specialised caseworkers.
% we conduct a participatory design (PD) study to integrate the unique expertise and perspectives of SSPs with traditional computer system design. We conduct the study in two stages. 
We first run preliminary workshops with 27 SSPs to examine the place of AI within social service. From the insights gathered, we then build a prototype AI tool and conduct user testing and focus group discussions (FGDs) with 24 additional SSPs. Through thematic analysis of practitioners' responses, we develop a framework synthesising the critical values and attitudes of SSPs towards AI use. Finally, we present a set of guidelines for the effective conceptualisation and development of AI tools. 

% MH: I guess need to say why voice from social work practitioners is important can be elaborated, in the previous paragraph
% so here I still didn't get why mention new and experienced at first? I feel we may need at least one sentence here to echo

Overall, our study offers an empirically-grounded understanding of how workers use and perceive AI, shedding new light on the challenges and opportunities of generative AI in the social service sector and beyond. This paper makes the following contributions:
\begin{itemize}
    \item We conduct an empirical study to uncover new ways AI can complement social service practice;
    \item We develop a framework for understanding the role and place of AI in the social service sector;
    \item We propose a framework for designers that extends the use of AI \textit{beyond decision support} to inform the design of future AI tools in social service and broader sectors.
\end{itemize}

% This paper makes the following contributions:
% \begin{itemize}
%     \item We conduct empirical studies on the opportunities and requirements of AI tools in the social sector;
%     \item We develop a foundational framework for understanding the role and place of AI in the social service sector;
%     \item We develop a foundational framework for approaching the design of modern AI tools for social service and related sectors.
% \end{itemize}