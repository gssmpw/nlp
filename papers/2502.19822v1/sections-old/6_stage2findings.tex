\section{Findings}
\label{sec:stage2findings}

% Based on the codebook and themes, a final framework was developed, summarizing the Values, Requirements, and Attitudes that social workers have towards the application of AI tools in the social work practice.

\begin{figure}
    \centering
    \includegraphics[scale=0.15]{images/ai in ss.png}
    \caption{Framework Situating AI in the Social Service Sector}
    \label{fig:situatingAI}
    \Description{An image showing our framework situating AI in the social service sector. There are three columns of subthemes: Ethos, comprising Ethics Experience, and Expertise; Attitudes, comprising Apprehension, Ambivalence, and Acceptance; and Considerations, comprising Care, Context, Control, and Collaboration.}
\end{figure}

We propose a framework encapsulating the fundamental place of AI in the social sector (Figure \ref{fig:situatingAI}). We term this the \textit{\textbf{ACE}} framework, comprising \textit{\textbf{A}ttitudes} of SSPs towards AI, \textit{\textbf{C}onsiderations} for systems congruent with SSPs' core identities, and the \textit{\textbf{E}thos} of SSPs and what matters to their profession in the \textit{context of AI systems}.

Each of the three themes consist of 3-4 subthemes. Under \textbf{E}thos, we discuss SSPs' \textit{\textbf{e}thics} (core values and beliefs), \textit{\textbf{e}xperience} (human touch and instinct), and \textit{\textbf{e}xpertise} (knowledge and skills), which are fundamental to the profession of social work. These in turn inform SSPs' \textbf{A}ttitudes towards AI, which are characterised by a combination of \textit{\textbf{a}cceptance}, openness and amazement towards AI, \textit{\textbf{a}pprehension} and caution about AI, and \textit{\textbf{a}mbivalence} and uncertainty towards AI. Finally, these ethos and attitudes contribute towards key \textbf{C}onsiderations for AI systems in the social service sector, emphasising \textit{\textbf{c}are} for safety and privacy, \textit{\textbf{c}ontext} of case and clients, \textit{\textbf{c}ontrol} by users, and \textit{\textbf{c}ollaboration} between humans and AI. In the following subsections below, we will expand upon each of the themes and subthemes in greater detail.

\subsection{Ethos: What matters to social service practitioners?}
\label{subsec:ethos}

We first consider the principles core to SSPs and their profession. SSPs espouse a set of core values and competencies \cite{rooney2017direct}; any use of AI in the field must therefore respect these values to achieve acceptance and ethical usage. Here, we describe three main subthemes: \textbf{E}thics, \textbf{E}xperience, and \textbf{E}xpertise.

\subsubsection{Ethics: Values \& Beliefs}
\label{subsubsec:ethics}

A foremost concern amongst participants was aligning AI use with the key ethical principles of the profession \cite{difranks2008social}. Two main areas of concern emerged: safety and privacy.

\begin{quote}
\textit{"At Agency A, we think a lot about the safety and risk concerns first."} (W2)
\end{quote}

The first was the primacy of the \textit{safety and risk of clients}. 
% Safety is a crucial dimension given the sensitive nature social work cases. This leads 
SSPs were "very particular about... risks," (S9), treating it as their top priority (W1, S10) and "at the front" for all client interventions and risk assessments (S2, S9); preventing imminent physical and psychological harms such as incarceration, abuse, and addiction, were cited as "non-negotiables" and primary risk factors (W2). Participants emphasised that in any AI system, "risk and safety... is basically something that should be there for all kinds of output... in every single case" (S2). 
% Ensuring client safety and mitigating risks is therefore a core requirement in all facets of social service work, including AI systems.

% underpinning all AI systems in social work, both in terms of AI-generated assessments and outputs, and the necessity of AI tools in being able to prioritize and safeguard the safety and well-being of clients.

% while auditors also pay close "attention to much of it" (S9).

\begin{quote}
\textit{"It's in our code of ethics... to protect as much [client information] as we can."} (D1)
\end{quote}

SSPs also expressed concerns about \textit{data privacy and confidentiality}. 
% Social workers discussed their dilemma with 
These are cornerstones of the social service profession \cite{dickson1998confidentiality}, but are also gray areas fraught with ethical ambiguity and uncertainty (D1). Social service agencies accumulate vast amounts of sensitive personal information about clients over years of interactions.
% from years of session notes and case reports. These contain highly personal information about clients' mental health, socioeconomic status, interpersonal relationships and more. 
Participants worried about leaking this information, such as by "accidentally putting a name inside the system" (W8), and questioned whether it was "safe enough for us to put a name" (W8) or sensitive topics like suicide and rape (D1) into the AI system directly, and whether there was a "need to change the [client's] name" (W5).
% , reflecting their concerns on client privacy. 

\subsubsection{Experience: Human Touch and Instinct}
\label{subsubsec:experience}


    % \textit{"I think we still need human skills... maybe we just need a human connection."} (D1) 
    % \textit{"We still need the human factor."} [W8]
% \end{quote}

\label{sec:perception}
Our second theme relates to the innately human element of social work. Real-life work often deviates from textbook procedures; SSPs must thus be able to perceive a client's situation through instinctive deductions empowered by years of experience and training. Participants repeatedly emphasised the "human skills" (D1) and "human factor" (W8) required to talk to, connect with, and understand a client (S7).
% This is perhaps the one area in which AI systems might never fully replace humans (D1).   
% An obvious skill here is the actual ability to talk to clients during sessions (S7), which ultimately determines the effectiveness of a social worker in connecting with a client. Beyond that, however, being able to understand a client's situation and reactions comprises many more aspects.

% The theme of adaptability related to the core social work value of treating every client as a unique individual with distinct circumstances and needs. 

\begin{quote}
    \textit{"This family, this mother is really, wow, you know... [this workshop] will not work."} (W5)
\end{quote}

The instinct to \textit{recognise when and why to choose certain interventions} for a client is key. Interventions that are sound in theory may turn out to be unsuitable, such as in the above quote by W5 where he recognised the reluctance of a client to embrace outside help, making a parenting workshop inappropriate, or when W11 judged the lower intellectual capacity of their client to make CBT unsuitable. Furthermore, given the dynamic nature of social service practice, the instinct acquired by SSPs through experience is crucial for tailoring assessments dynamically based on individual client needs, as highlighted by the quote by S10.
% to coming up with case assessments and interventions, which require an intrinsic human touch and dynamism to tailor and personalize to clients' individual needs. Such human experience and ability is therefore difficult to be easily replicated by an AI system, which is generally more rigid and adheres to a certain set of rules and training data.


\begin{quote}
    % \textit{"Assessment is very dynamic...[AI can provide] guidelines but you also need to customize it based on where your client is at and where you are at in terms of your experience and confidence."} (S10)
    \textit{"Sometimes information may suggest a symptom of a different problem." (C1)}
\end{quote}

Experience also allows SSPs to make \textit{instinctual deductions about a client}. C1 raised the example that a complaint of joint pain would normally be associated with arthritis, but if the client were a teenager (an age group not often inflicted with joint damage), they would more likely be experiencing psychosomatic pain caused by mental stress, leading to a vastly different treatment plan. While such information could be found in a medical handbook, the ability to instinctively recall the requisite knowledge and and quickly synthesise it with information about a client sets SSPs apart.

\begin{quote}
    \textit{"Many of us who have been in the sector before, probably were like,} aiyah, \textit{this won't work."} (W5)

\end{quote}

Finally, \textit{knowledge of local and cultural factors} helps SSPs make better decisions. As a seasoned industry worker, S6 knew that "there are a few parenting workshops available [here in our country] and how they are structured," and that they would not "fit into the family" under her care. Such experience can only be gained through years of on-the-ground work and effectively utilized by appropriate human decision-making.

% This experience is also irreplaceable by AI: no system will be able to capture the "structure" of local parenting workshops and subsequently interpret their suitability for a client, since there is no way of representing such knowledge in digital form. 
 % This last example in particular illustrates one challenge of AI-assisted social work processes: it is not possible for any AI system to produce such an insight; the worker (W5) himself would not have noted down the relevant information that would have led to this assessment, and could therefore not possibly have entered it into any system for analysis. \textit{[Cite more here...]} 

\subsubsection{Expertise: Knowledge and Skills}
\label{sec:proficiency}

The third principle relates to the \textit{theoretical competencies} of SSPs, and how they use their knowledge and skills to most effectively help their clients.
% We analogize the theoretical knowledge of SSPs to a vast toolbox with tools for different purposes. 

\begin{quote}
    \textit{"The beauty of social work is [in] the dynamism of the interactions."} (S9)
\end{quote}

SSPs highlighted the need to \textit{react and adapt to client circumstances}. The "dynamics of the [client's] ecological system" (S9), such as a client's behaviour or living situation, mean that the resulting interventions "may change according to the need and time" (S4). S10 also referenced how "assessment is very dynamic", requiring SSPs to constantly asses the situation and adapt their intervention plans accordingly.

\begin{quote}
    \textit{"It's not one-size-fits-all." } (W13) 
        
\end{quote}

SSPs also require the \textit{ability to tailor each case to meet the unique needs}. C1 mentioned, "It really depends on.. the case... we will consider different models based on the information that we receive...", listing examples of violence, homelessness, and marital issues that she would use different assessment frameworks for, even "mix[ing] and match[ing]" theories as appropriate. W10 concurred, saying that from a given set of theories, they had to "see what fits... according to the key details of the case" and decide on an optimal course of action.

% Case conceptualization is a flexible, iterative, and dynamic process of selecting the right tools and combining them in the right order to create a pipeline that turns raw data about a client into a specialized and accurate assessment. W10 commented: "I think, for social workers, not all of us stick to just one type of modality". C1 mentioned: "It really depends on.. the case... we will consider different models based on the information that we receive... is this, say, a violence issue? Is this more of a homelessness issue? Is this more of a marital issue? So if it's a marital issue, I will select a certain different assessment. If it's a child behaviour issue, then I may want to choose maybe more of the, you know, certain developmental frameworks. I may use different- so  I mix and match to the case, to the type of case. For children, you may even use attachment theory and then you want to see what comes out." W10 followed up: "So [the theories are] there, but then you will choose according to the key details of the case. So you see what fits." Evidently, each practitioner follows a unique set of theories and approaches (W10) in conceptualizing a case, due to individual preference and differing competencies. 
% W10 noted that the intervention modalities they choose "might be slightly different based on the worker. So okay, some workers might prefer to try [intervention modalities like] SFBT approach or CBT approach… also there are workers who do [assessment techniques] like strengths, challenges [as a starting point]… because for CBT and SFBT there's also these concepts in these intervention style. So a lot of times when they do their assessments it's also based on these concepts." \textit{[Cite more quotes of participants talking about their individual preferences/competencies.]} 




\subsection{Considerations: What do social service practitioners want from AI?}
From SSPs' core values, we next discuss how these principles give rise to four key considerations of adopting AI in social service: \textbf{C}are, \textbf{C}ontext, \textbf{C}ontrol, and \textbf{C}ollaboration.

\subsubsection{Care: Privacy and Safety}

Given their emphasis on ethical treatment of clients (Section \ref{subsubsec:ethics}), participants were notably fastidious on both issues: the \textit{data confidentiality} and \textit{physical safety} of clients. 

\begin{quote}
    \textit{"There's a concern... that [the AI] doesn't send home [to its servers] a lot of these personal data, right?"} (D1)
\end{quote}

With \textit{confidentiality}, a recurring concern was how much client information could be entered into the AI tool (D1, W5, W8), given the unclear ethical and legal gray areas (D1) surrounding these nascent technology. This points to multiple considerations for AI system designers: to provide \textbf{ways to obsfucate or anonymise personal information}, \textbf{clarify data storage and usage policies}, and \textbf{involve users early in the design process to ensure alignment} on confidentiality.
% The quote by D1 underscores SSPs' care and concern for the personal data protection of their clients, describing that in their use of an AI tool, they would intentionally delete and obfuscate certain sensitive or personal information so it would not be captured by the AI system. Participants were reassured that our system would not store user inputs (though OpenAI stores all requests to the GPT API for 30 days for usage monitoring and abuse prevention \cite{openai2023how}\footnote{OpenAI offers Zero Data Retention upon request, but we could not get this request approved.}), yet still felt the need to err on the side of caution and "play [it] safe" (D1). W8 felt that "name[s] shouldn't be inside" AI systems, while D1 suggested using generic initials instead. This highlights the need for systems to \textbf{clarify data storage and usage policies}, while \textbf{involving users early in the design process to ensure alignment} on confidentiality.
% These various precautions and pre-processing steps therefore highlight the care for client safety and privacy as a key consideration that SSPs had towards AI, pointing to a clear need for systems that \textbf{clearly show how any information entered will be stored and used}, but also that potential users of the tool should be \textbf{involved in discussions early and often during the design of the system} to ensure that all users are on the same page with regard to data storage and confidentiality.

\begin{quote}
    \textit{"The [risk of] violence is not highlighted. Where is the violence?"} (W1)
\end{quote}

Prioritising \textit{risks to client safety} was something else participants stressed. They expressed concern when the tool's assessments failed to emphasise (S2) or entirely omitted (W1) safety risks,
% During trials of the prototype when SSPs fed in sample case notes involving situations of domestic violence or child abuse, they were especially concerned about AI's omission of certain safety aspects, commenting that 
% "The violence is not highlighted. Where is the violence?" (W1) and that "risk isn't exactly highlighted" (S2) in AI-generated assessments. 
% They therefore feedbacked that "the risk is basically something that should be there for all kinds of output" (S2), and suggested that in future iterations of the AI tool, a useful feature would be to include "what are the non-negotiables, like no abuse, no violence, et cetera...[and] safety planning" (W2). \textit{[cite more examples here]} 
and suggested that risk indicators (S2, W4) were non-negotiables.
% We note here that while GPT-4 is generally capable at producing a great deal of relevant information about a client, it has little notion of which information is the most salient and critical, particularly in accordance to specific procedures or conventions adopted by individual organisations. 
These concerns underscore the importance of \textbf{AI tools prioritizing client safety} to avoid guiding workers toward suboptimal courses of action (S9).

% As the above suggests, the word "care" here serves double duty, for both clients and system designers. On one hand, it aptly sums up the profession as a whole: above all, SSPs are driven by care for the clients they develop relationships with. 
% Care for well- being, safety (W2), and privacy (D1) are all fundamental requirements of the profession. 
% On the other, "care" is therefore something that must be liberally applied when leveraging technologies that possess immense power and limitless use cases. Failure to do so may result in systems and user workflows that compromise core tenets of social work.

\subsubsection{Context: Client Background and History}

Experience (Section \ref{sec:perception}) strongly relates to contextual analysis in formulating cases. Numerous small details that SSPs observe interacting with clients, like personality (W5), cognition (W11), psychological state (S4), and multiple small behavioural events (D1), collectively form the contextual background that informs the assessment of a case. Context can be drawn from two sources: \textit{written} records and \textit{unrecorded} insights.

\begin{quote}
    \textit{"Cases [can be] really an accumulation of years of case recordings... [so the machine] really needs context.} (S4)
\end{quote}

\textit{Written context} is often pieced together from months or years of case notes (S4), or sometimes the "verbal diarrhea" - large amounts of unorganised information - of a new client's intake session (S6). Cleaning and analysing such raw data is time-consuming (W3) and error-prone (D1). AI systems should therefore be designed to \textbf{accept large amount of text or documents} while \textbf{thoroughly analysing all given facts and information} contained in the input, to effectively streamline processes (W1) while ensuring no critical information is missed (S6). 

% While formulating a clear view of a client's situation from these notes is imperative for effective interventions, cleaning up such overwhelming amounts of raw data and formulating them into assessments this process can be difficult and "time-consuming" (W3). This is one area where LLMs and AI, with its superior data processing and generative potential, can "definitely help" with in terms of helping social workers to "streamline [their processes] and save...time" (W1). 


% One tangible way in which this can be overcome is by designing tools capable of \textit{taking in large amounts of information}.   The ability to "plug in" (S6) tens of past reports is important to ensuring all relevant information is captured. On the other hand, access to context in past information may sometimes still be insufficient. As noted previously, social work cases involve many intangible factors that cannot be captured by AI systems. It is therefore crucial that AI system are \textit{clear and transparent on their limitations}, in order to avoid misleading users.

\begin{quote}
    \textit{"[This is your direct assessment of the family?] Yes. [But you won't necessarily detail it in the case report?] No." (W5)}
\end{quote}

With unrecorded insights, social work cases often involve intangible factors that cannot be captured by AI systems. One example is W5's quote from Section \ref{subsubsec:experience}, where the worker did not record his feeling of the client as being unreceptive to outside help. Such information, naturally, is inaccessible by the AI for its decision making. It is therefore crucial that AI system are \textbf{clear and transparent on their limitations}, in order to avoid misleading users and further encourage users to critically analyse their cases.


% The irreplaceability of the human touch in social work (Section \ref{sec:perception}) suggested the use of AI in instead developing human skills.

\subsubsection{Control: Flexibility to Customize Use}
\label{subsubsec:control}



% \begin{quote}
%     \textit{"Our assessment depends on what style we adopt... 
%     % some workers might prefer [an] SFBT approach or CBT approach... 
%     [we] take in a modality as a guide or framework that we perceive things with.
%     % .. Let's say I want to choose strength assessment but I want the SFBT interventions. Then we can look at the assessment and see, these are the strengths, so this is the proposed solutions based on SFBT approach. Then you can see the interlink between all...Also, [regarding] efficiency - you don't [have to go] back and forth on my computer; it's just one generated output report."
%     } (W9)
% \end{quote}

AI systems should complement the theoretical competencies of SSPs (Section \ref{sec:proficiency}) by offering customisable workflows instead of one-size-fits-all user flows (W13), allowing different ways of using and combining the theoretical frameworks in social service practice. 

\begin{quote}
    \textit{"Maybe put risks and strengths [side-by-side], then maybe the DIAP and the 5Ps too... then we know which [parts] we can contrast, and also get a more rounded intervention process"} (W11)
\end{quote}

Firstly, many participants suggested allowing \textit{multiple theoretical modalities in a single output}. As W9 highlighted above, SSPs have different styles and preferences in formulating cases and interventions. Therefore, participants suggested that the AI could be in-built with functions to offer such greater control and customizability, allowing users the flexibility to select and include different frameworks and modalities in the AI-generated output. This could include generating an intervention integrated with an assessment or incorporating multiple modalities within a single output (e.g., SFBT, CBT) (W9) and combining risks and strengths to assess the two in tandem (S7). Participants also proposed side-by-side outputs to compare and contrast frameworks (W11), noting that frameworks are often used together and inextricably (S10). 
% Given that many of these modalities and frameworks used by SSPs are not used in isolation of one another, but often used side-by-side to complement each other, S10 noted such combinations could be useful since certain assessment frameworks are closely or inextricably linked with others. 
% An AI system with affordances to provide higher degrees of control and customization would enable users to 
Enabling practitioners to "see the interlink between" different modalities (W9) provides a "a deeper, richer perspective" where different modalities "feed back into each other" (S7). This gives a more holistic view of each case, while improving efficiency by consolidating information in one view (S9). 
% Furthermore, a practical benefit of this is that it would allow users to access all the essential case information in a single view, without having to switch between multiple screens or files (S9), thus enhancing efficiency and saving time (which is especially useful for time-pressed SSPs, for whom every second counts).

% W9 said, "I was just thinking, would it be possible if we could do for example like an assessment and an intervention and then we generate at one shot." S7 wanted to combine the outputs for risks and strengths, to assess the two in tandem and see how one fed into the other. W10 wanted to combine "a strength assessment with an SFBT intervention to see the interlink between all because SFBT is always about building on the person's strengths". \textit{[Cite more here...]} Other participants requested parallel, side-by-side outputs. W11 suggested showing "maybe risk[s] and strengths [side-by-side], then maybe the DIAP and the 5Ps [side-by-side]... we know which [parts] we can contrast, and also get a more rounded intervention process". \textit{[Cite more here...]}. Participants showed that, while our system had a range of useful tools, participants did not want to be restricted to a single option at a time, and instead have the flexibility to pick and choose different theoretical frameworks to use.

\begin{quote}
    \textit{"If you are able to choose the people in the family, whose strength we want to look at... [that would be good]." (W5)}
\end{quote}

Secondly, participants desired \textit{control over the focus of the system's output}. Our system tended to give generic, safe answers according to its pre-programmed theoretical frameworks; participants, however, demanded more. W2 considered a "complex family with multiple issues within the family unit", and wanted to customize the tool to get intervention plans for different family members. S6 and W5 mentioned they had to "choose what they wanted" from the complete system output, and preferred to specify their needs to begin with. 
% On a more administrative level, being able to generate outputs for different purposes and audiences was also important. For example, 
Meanwhile, W13 noted the importance of tailoring social reports for different audiences, such as housing agencies or citizenship applications.
% W13 mentioned that when producing social reports, "we might categorise it as employment, financial, and so on... it depends on who we are writing the referral to. If it's for the [local housing agency], we might talk more about what the client has tried for their housing situation, whether it's for citizenship... then it's different from if I was writing to a different agency." 
These show the varied uses of AI in social services, where SSPs require control to meet specific goals, whether for interventions or documentation.
% that that SSPs may utilize AI for multiple different functions, and therefore require different degrees of control over the output in accordance to their varying needs and aims, whether for purposes of generating client interventions or for reporting purposes to different external stakeholders and target audiences. Given the multifaceted and idiosyncratic nature of the social sector, it is therefore essential that any AI systems designed for social service must be able to afford its users with control to fit their various needs and purposes. 

% Given the multi-faceted nature of social work cases, workers often need to consider multiple perspective for the same theoretical approach. W2 said, "if it's a situation of a complex family with multiple issues within the family unit itself... we have to plan not just for the client, but also for other main family members as well... to help the mother manage her issues, would we have to remove everything else and just input the mother's portion? And the biological father has a pressing issue of alcohol addiction, we have to get him help as well." 
% This indicated a desire to be able to choose multiple individuals from whose perspective the system should analyse the case. 

% The overall finding here is that social workers do not approach case conceptualization as a linear, step-by-step process. It is rather a much more dynamic approach where workers draw on various theoretical tools, almost experimenting with different ideas until they gain a sufficient understanding of a case. AI systems are perfectly suited to help with "experimenting", being able to rapidly generate content and ideas for consideration. However, their design must facilitate this experimentation by allowing sufficient flexibility in usage. We may view flexibility of input UIs as a spectrum: on one end, a single text box connected to the unlimited powers of an LLM, and on the other, a series of preset buttons and checkboxes that rigidly control the type of output. Our key finding here is that, while having certain preset options is necessary for affordance, it is also crucial that these options can be mixed, combined, and modified according to user desires. This gives social workers the level of control they need to fully harness the power of AI systems, without being hamstrung by restrictive UI options.

% Overall, greater control and flexibility improves system usefulness, while respecting the autonomy and agency of SSPs as experts in their field,  enhancing their competencies rather than reducing them to mere system operators.

% not reducing them to mere operators of a system, but instead enhancing their existing competencies workers by allowing them full control over how the tool helps them.

% In sum, context matters greatly to system designers in two ways: the context in which the system will be used by certain workers,

\subsubsection{Collaboration: Between Humans and AI}
\label{subsubsec:collaboration}

% We view collaboration through two lenses: an idea that brings great benefits when harnessed properly, but also one whose careless omission leads to many potential risks. 

% Comment from Kai Xin: I agree with Prof EJ that the "Collaboration" theme currently covers two distinct points, which may make it too wide. I revised the section to focus it purely on human-AI collaboration and cooperation, and less so on the risks (which I feel is less applicable in terms of collaboration). The original text is still retained (with % added) below.

Finally, we highlight opportunities for HAI collaboration in social service. AI can enhance casework in many ways: by picking up details workers might miss (D1), training workers' skills and knowledge (S6), and providing pointers for supervision (C1, W4).

\begin{quote}
    \textit{"I'm hoping the system as a whole can be smarter in terms of things that we might miss out [and] capture [what] people might not be aware of.
    % So for example, a client is crying all the time. Sleeping stuck and not sleeping well. So a series of these factors putting it together can raise a red flag.
    "} (D1)
\end{quote}

Firstly, AI can provide augmented intelligence in client assessments and diagnoses. D1 mentioned how AI could pick up on subtle signs that SSPs might miss due to time constraints or high caseloads: AI tools can identify patterns or trends, such as excessive crying or sleep difficulties, that indicate mental health risks. This allows practitioners to make better-informed decisions. In this sense, AI augments human intelligence, improving the accuracy and comprehensiveness of client assessments.

\begin{quote}
    % \textit{"I can bring this [question] to my supervisee... [and get] five more questions back that I can ask."} (C1)

    \textit{"Not just you train the model, the model also trains them."} (S6)
\end{quote}

HAI collaboration also creates \textit{learning} opportunities for SSPs, especially junior personnel. S4 said the thorough examples generated by the model were "really helpful in... guiding the worker" to develop interventions and "expand their perspectives". S6 felt that AI could "train" workers as they used it, "polishing" them as caseworkers.

\begin{quote}
    % \textit{"I can bring this [question] to my supervisee... [and get] five more questions back that I can ask."} (C1)

    \textit{"And this can also be used in supervision! You know!"} (C1)
\end{quote}

Finally, we discovered the unexpected role of \textit{supervision} that AI can assist with. C1 pointing out how many of the AI's outputs, while not designed for that purpose, provided useful starting points for discussion and reflection with supervisees. S4 and S7 agreed, pointing out that the AI's ability to quickly "concretise theoretical models" to build on during sessions. W4 and W5 said the AI provided useful intervention suggestions for supervisees to bring into sessions. W9 even cited AI as a "readily-available supervisor". 

% AI provided useful prompts for discussion plays a valuable role in human-human-AI collaboration, especially in supervisory contexts. C1, a senior counsellor, explained how AI can assist in supervision sessions by providing a foundation for discussion and reflection. While AI-generated outputs might not be immediately applicable to experienced practitioners, they can serve as useful tools to guide less experienced workers. Supervisors can leverage AI outputs to facilitate discussions with their supervisees about potential intervention plans, creating a more robust and collaborative learning environment.
% These findings highlight the dynamic and multifaceted nature of human-AI collaboration in social work, where AI tools not only augment human capabilities but also foster new ways of working and learning in a collaborative environment.

% Another useful application of AI was its ability to be used in \textbf{human-human-AI settings} such as supervision. Extending the above quote, workers with differing levels of experience and positions may benefit from AI differently. C1, a senior counsellor, said: "at my level [this output] is not helpful [for my own work] because I know what to do already", but for supervision, "this I can bring... to see if my supervisee has used these questions [to ask the client], and if they haven't, I may want to introduce them". AI here provides a foundation on which C1 could build upon to more effectively facilitate a supervision session. Junior workers also expressed similar sentiments: when S4 brought up the tool's ability to quickly generate many different intervention plans, W4 said they "might not have thought of [some of these] interventions", and while he might not adopt them directly, he could discuss these with his supervisor on which ones to try out. This is an interesting sentiment, which surfaces the potential of not just human-to-AI collaboration, but also more complex dynamics involving more than two parties like human-human-AI collaboration and cooperation \cite{munyaka2023decision}. In the context of social service and the examples raised by our participants, we can see that AI can be employed by supervisors and junior workers as a facilitation tool for supervision sessions, allowing supervisor and supervisee to work collaboratively not just with one another, but also with AI, to develop and discuss different ideas such as intervention plans. Most human activities today are done collaboratively, and social service is no exception, where traditionally, senior SSPs serve as mentors to supervise and guide junior workers. When we add AI into the mix, AI tools can therefore serve as a de facto supervisor to similarly guide workers (especially those in more junior roles) and assist with and augment the human supervisor's role.

% Participants highlighted the potential of working with AI to facilitate client assessments.
% There has been emerging interest in HAI collaboration \cite{wang_human-human_2020}, a term that refers to the process in which teams, comprising of human and AI agents, "work together in a synergistic manner to achieve shared goals, e.g., with the AI providing recommendations or insights, and humans guiding and refining the AI-generated outputs" \cite{hemmer2024complementarity}. 
% A key way in which HAI collaboration could help in social service is to provide \textbf{HAI augmented intelligence in client assessments and diagnosis}. As the above quote by D1 reveals, SSPs may sometimes miss out on certain symptoms or signs that clients display (an understandable issue, given the lack of time and multiple demands that SSPs face). AI can help by recognising patterns or trends (e.g., excessive crying, difficulty sleeping) as potential mental health risks, thus bringing them to the attention of the SSP who can then make a better, more informed diagnosis of the case. In this sense, AI can augment the intelligence of humans by picking up on non-obvious patterns that social workers may miss out on or overlook, thereby increasing the accuracy of client assessments and subsequent interventions. This thus gives rise to a form of "augmented intelligence" \cite{jarrahi_artificial_2018}, whereby humans and AI can work together to enhance one another's intelligence and abilities beyond what an AI tool or a human worker alone could achieve on their own.

% Participants also cited the \textbf{benefits of HAI symbiosis in mutually learning from and training one another}. S6, a senior supervisor, noted that for junior workers who were new to the field and using AI to generate ideas, 
% \begin{quote}
%     \textit{"when they look at it, they will know that, oh, so this is how, the way that I should actually do it. And in terms, the more that when they do with the model helping them, and of course they will have to also put in their own effort to, you know get supervision from the supervisors and all these things, their knowledge, I think it will polish them as well. Not just you train the model, the model also train them. I think it works both ways."}
% \end{quote}
% This quote showcases the concept of HAI augmentation, whereby at the same time as the human worker trains and improves the AI, the AI simultaneously also helps to train and enhance the worker's skill, highlighting the duality and dynamic nature of HAI collaboration in social service.



% Our final theme highlights the importance of viewing AI through the dual lenses of \textit{collaboration}. Human-AI (HAI) collaboration has been extensively studied \textit{[Cite]} as new ways of improving human productivity and work processes; simultaneously, overlooking the need for collaboration when using AI tools can lead to many potential risks. We examine our findings from both of these angles. 

% First, we discuss the risks of a lack of collaboration. This is most salient in applications of instructions and guidance. The use of AI for ideation and session planning, despite its flaws, has been an omnipresent suggestion (e.g. S4, S6, W7, W11, W12, D1). However, its effectiveness here depends on the worker's individual competencies: a newer worker might be unable to compensate for the tool's shortcomings.
% In particular, the tension between the extensive theoretical underpinnings of social work and the realities of its dynamic and fast-changing nature create unique challenges.
%S9 used the example of a Task-Centered intervention suggested by the tool: 
%"So the step one, two, three means what? It means that I need to do it in the sequence, isn't it? So if you don't tell me and I'm a newbie, I’m thinking also, I need to do step one first and then step two and then step three and step four.

%"For our interventions, the safety plan is always the thing to prioritize. But it seems like [safety here is] at step four... and our interventions are not so linear, sometimes you make two steps forward, then you backtrack one time. So it depends on your target group. If newbies see this, I need to follow one, two, three, four steps, die \textit{liao}\footnote{A local slang term expressing resolution and finality.}."

%In essence, the intervention plan generated by the tool followed a theoretically-grounded sequence of steps, 
% These steps provide a useful reference for more experienced workers, but S9 pointed out that a newer worker would interpret it more 
%that when interpreted literally and rigidly produce a poor outcome. \textit{[Cite more here...]} The output, then, is a useful reference, but without either \textbf{human input by the worker to complement the system's guidance or more experienced workers to lend their expertise}, users can easily be led astray in the execution of their work.

%On the flip side, the same procedural, systematic nature of AI can also be a boon for human-AI collaboration. 

% Move this to discussion section?: What differentiates the social work field in how it has to be treated differently is...
% 

\subsection{Attitudes}

The values in Section \ref{subsec:ethos} give rise to three attitude themes along a spectrum of valence: \textbf{A}cceptance, \textbf{A}mbivalence, and \textbf{A}pprehension. These capture the widely varying feelings that SSPs hold toward AI. 
% \textbf{Apprehension} captures the negative or wary feelings of social workers towards AI, a largely unfamiliar and still-untested technology. \textbf{Acceptance} encapsulates the openness and optimism that social workers feel towards embracing and acceptance AI, along with its limitless possibilities. \textbf{Ambivalence} falls in the middle, with our participants demonstrating varying degrees of understanding towards AI and expressing a range of opinions about the extent to which AI can truly help in their work.

\subsubsection{Acceptance: Openness and Amazement}

Throughout our sessions, participants displayed a range of positive emotions towards AI.

\begin{quote}
    \textit{"This model is not perfect... But for it to continue to evolve further, there must be a willingness to try and to start moving things along... If it's here to stay, how can we make the best out of it?"} (D1)
\end{quote}

Participants expressed \textit{openness and curiosity} towards AI and its possibilities. D1 was "excited" by the potential benefits AI could bring to the sector, acknowledging its limitations but embracing the opportunity to incorporate it into practice. W5 noted, “There’s no harm in using something that makes your life easier,” and S4  highlighted the benefit of "expanding [workers'] perspectives".

% Extending from this sense of openness and acceptance, participants' \textit{amazement towards AI's capabilities} was also a contributory factor towards their positive sentiments towards it. There was a feeling of great astonishment, wonder, and even awe as participants discovered the algorithmic system's ability to summarize years' worth of case notes at a touch of a button (S6), infer client identities from shorthand notation (W3), and recognize acronyms for local organizations (S2). 

\begin{quote}
    \textit{"Let's see if it can pick up the strengths... oh, wow! Impressed."} (W5)
\end{quote}

Participants were also \textit{amazed by AI’s capabilities}, such as its ability to summarize years of case notes, infer client identities from shorthand, and recognize local acronyms. W3, for example, marveled at how AI inferred “DG” as “daughter” without context: “Oh my god... they know that A is the daughter.” Similarly, S2 was impressed by AI’s inference of “SYC” as Syariah Court, automatically generating relevant legal advice. [add more here and a conclusion, and we're good!]

% Participants were impressed by how the system could infer names (W3) and institutions (S2) from shorthand or initialisms.
% In one instance, W3, a SSP who was testing out the tool by inputting his own case notes from a session with a client, remarked in amazement, 

% \begin{quote}
%     \textit{"Oh my god, I didn't even mention that it's a daughter, right? They know that DG is daughter. I put DG, then I put A...Then they [the AI] know that A is the daughter. 'Ensuring her daughter (A).'"}
% \end{quote}

% His exclamation of surprise reveals his astonishment at AI's ability to intelligently infer and accurately guess the identity and relationship between his client and daughter A, without any context or explanation of the abbreviation "DG" that he used in place of the word "daughter". Similarly, on another occasion, S2 also found himself amazed by the capability of AI to make intelligent inferences from his raw notes:

% \begin{quote}
%     \textit{"They [AI] can actually pull out data because I indicated SYC [Syariah Court], so they already know how to link it up with legal advice. It's quite good. SYC is for divorce. They were able to retrieve it and put it in the component for legal advice, which is relevant."}
% \end{quote}

% Despite him not explicitly referring to the acronym "SYC" as "Syariah Court" (a Muslim institution in charge of settling disputes between divorcing parties), S2 was impressed by the AI tool's ability to infer the organisation and automatically generate appropriate recommendations for legal advice for the client, which he felt was "relevant" and "good". Therefore, overall, from participants' responses, we can see that SSPs in general were impressed and amazed by AI's abilities, and open-minded in trying out different ways in which they could utilise the algorithmic system enhance their existing work practices, such as in summarising case notes and generating client recommendations with greater ease.

% Some participants frequently expressed amazement at the AI's capabilities. "Oh my god," said S6 at the possibility of summarizing years' worth of case notes with the click of a button. W3 was similarly impressed by the LLM's intelligence: "Oh my god, I didn't even mention that it's a daughter, right? They know that DG is daughter. I put DG, then I put A. Someone is like DG, then bracket A. Then they know that A is the daughter. No, you see? Ensuring her daughter, bracket A." S2 was amazed at how "they can actually pull out data, because I indicated SYC [Syariah Court; an acronym for a local organisation], so they already know how to link it up with legal advice," liking the system's ability to understand local terms and subsequently draw inferences from it.

% Notably, many participants were willing to embrace AI despite its flaws. For example, when the system gave an overly verbose output, S2 and S3 adapted by "just... sift[ing] through the info" and pick[ing] out what is relevant", and W2 said that it still reduced their mental effort in formulating the case. In another case, W1 had the system misinterpret an individual mentioned in their input, but said, "Yeah, but I mean I think... I would just [use it], then amend if I need to amend".

% Also include amazement >> product of their proficiencies, what's lacking there, how this leads to amazement

\subsubsection{Apprehension: Distrust of AI}
\label{subsubsec:apprehension}

Despite their positive attitude towards AI, SSPs also expressed exhibited apprehension regarding the risks and limitations of algorithmic systems. 
% Here, we can see how the ethos and principles of SSPs (Section \ref{subsubsec:ethics}) inform and contribute towards their attitudes about AI. 

\begin{quote}
    \textit{"I would say right now, we are prepared to be more cautious."} (D1)
\end{quote}

A major concern was the \textit{tension between the highly confidential nature of social work data and the opacity of third-party AI systems}, coupled with the nascent nature of generative AI. D1 considered it a "very gray" area in which he was "prepared to... be more cautious". Echoing their ethos of protecting client privacy
% Firstly, participants were notably \textit{concerned about the black box nature and nascence of AI systems}. AI has often been criticised for its opacity and lack of transparency, essentially serving as a "black box" \cite{castelvecchi2016can} that can be difficult for people, especially non-technical experts, to understand. The black box nature of AI systems developed by multinational corporations (which themselves often have a lack of transparency on data management policies and protocols), the highly confidential and sensitive nature of social service work, and the nascence and novelty of AI and LLMs in the field therefore give rise to significant worries in adopting AI in the social sector and make it a "very gray" area for SSPs to navigate (D1). In particular, echoing their ethos of protecting client privacy 
(Section \ref{subsubsec:ethics}), many workers were concerned about entering (W5, W8) and storing (S1) client information into an external server, and even about the security of the AI system itself (S7). Both third-party data storage and the process of feeding client data to an AI model (i.e. an LLM) were causes for concern.
% These fears and concerns about the lack of transparency and inability of AI to properly safeguard client safety and privacy therefore contribute towards a sense of apprehension, distrust and hesitance towards fully embracing AI in the social work practice.

% ?Therefore, we can see how in spite of their aforementioned positivity and acknowledgement of AI's beneficial capabilities, SSPs nonetheless still have their reservations and apprehension towards AI, and are hesitant towards fully embracing it in social service due to privacy and safety risk concerns. This is an important element for designers to remember as they design AI systems for social service (or any field in general): a technology can have many irrefutable benefits, but designers always have to keep in mind the attitudes and core principles of users when designing a tool in order to overcome their fears and reservations. Specifically, in the case of social service, due to the importance of ethical principles and sensitivity of the work of SSPs, \textbf{it is essential for designers to develop AI tools that can safeguard the safety and privacy of clients and offer greater transparency}, so as to truly convince and convert practitioners to adopt and utilise AI.
% W5 talked about needing to create and "set a common protocol".

% The center director D1 said: "It's in our code of ethics..., so we want to protect as much as we can... Right now if you ask me, we are prepared to say, to be more cautious." Yet, in requesting participants to only "delete \textit{some} of this private information", it is clear how the use of AI is largely a "very grey" area that warrants significant consideration.

% Another concern was \textit{the inability of AI to handle the nuances} of social work. The irreplaceable experience of SSPs (Section \ref{subsubsec:experience}) led to sentiments of AI inadequacy in certain areas, 

% S4 expressed confusion over the AI's output: "I don't know why [the AI] gave me this." W5 was more circumspect, saying, "how can the model [do this]? I think that one is hard." 
% reveals, the experience and expertise of SSPs, which imparts them with a human touch and ability to understand the dynamic context of their clients' cases and circumstances, are important and irreplaceable parts of the social service practice. In implementing an AI tool, participants were therefore naturally concerned and apprehensive about the machine's ability to adequately replicate the human experience and grasp these subtle nuances. In one example, S4 had some reservations about the algorithmically generated output when testing the AI tool to come up with recommendations for an actual case he had involving a father and son:
% SSPs questioned AI’s ability to grasp the nuances of social work. Social service practice often requires a "human touch" to evaluate complex client contexts, something participants felt AI struggled with. 
% \begin{quote}
%     \textit{"This is SFBT...SFBT wasn't used [with the actual client] because simply the person wasn't ready to engage. So there is no subsequent moving on for this case. Like some examples may not be contextual, like 'considering your strong support'. I don't know why they [the AI] gave me this. [Or] 'How have you and your friends helped you cope with the marital distance and parenting challenges?' I guess for the dad, there wasn't any really real issues for his coping. So I think we would have covered that, really, in terms of how his friends helped him.}
% \end{quote}
% S4 mentioned how the AI system suggested using SFBT, but missed how "the person wasn't ready to engage. So there is no subsequent moving on for this case."
% As the quote reveals, AI can have certain limitations in grasping the nuances and contexts of a case, which human SSPs can more readily pick up on. 
% In the above case, the AI system recommended a SFBT intervention for the client, which S4 judged as inappropriate as "the person wasn't ready to engage", showing how unlike the AI, the SSP was able to evaluate the suitability of intervention modes and approaches based on the client's state of mind and circumstances. This is an important dimension of social work practice, which is so highly individualistic and specific to clients' differing and dynamic situations that it requires the human expertise of a human worker to judge and recommend measures that are tailored to the client's needs, which an algorithm may not be able to accomplish. Similarly, the AI generated a question asking the client about how his friends helped with coping with marital and parenting challenges, which S4 again judged as inappropriate and irrelevant, given that the client did not face any coping issues. Therefore, while the AI may have superior information and data processing capabilities, we can see here that the human expertise and experience in evaluating the client's individual needs are ultimately still superior to the machine. 
% W5 raised questions about cases that were "already open for quite a while... I don't know how the model can actually help that... how can the model actually also list down the progress? I think that one is hard." 

% Therefore. the inadequacy of AI to accurately capture and comprehend the complex nuances involved in social service contribute towards SSPs' reservations and apprehension towards fully utilising them in their practice. 

% This emphasises to us that even while we try to implement AI in social service, there is still an irrefutable human dimension that cannot be easily outsourced or replaced, making it vital for us to \textit{complement the capabilities of machine and man in social practice}.

% Principles > distrust that AI can really d handle their issues

% This links to the issue of human perception and the core role it plays in 

% Perception > Distrust: Wary that AI cannot capture some capabilities of humans

\begin{quote}
    \textit{"One of our concerns is... [will using this AI] actually disable our ability to make assessment?"} (D1)
\end{quote}

There were also \textit{concerns over overuse and overreliance on AI}.
% worrying that the overuse and outsourcing of tasks to AI may lead to over-dependence and even degradation of ability for SSPs to perform tasks on their own. In the quote highlighted at the top of this section, the center director, 
These centered around the possibility of degradation of SSP's skills and judgement (D1). 
% D1, raised their primary concerns on the adoption of AI, citing worries that it may weaken human workers' ability to exercise their own judgment in decision-making and assessments. In another session, D1 also noted that: 
% \begin{quote}
%     \textit{"Some of our colleagues have...raised a concern about [AI] that then if we move forward with this, then [it may] impair the social workers' ability to make assessment. Because...if you find that it is useful, then the more you use it, then the more you rely on it, then you stop using your own."}
% \end{quote}
S3 agreed about the risk of overrliance by "spoonfeeding" junior workers, and S9 noted the need for workers to "still use their brain... otherwise... they just rely on this."
S9 also worried about how junior workers might fail to recognise suboptimal outputs by AI, and be misled as a result.
% His sentiments are mirrored by S3, a senior supervisor, who agreed that, "there will be a reliance on it if it is given spoon-fed to them...they will be reliant on this." The above quotes reveal the double-edged nature of AI's efficacy. While on one hand, it can offer clear benefits to helping SSPs to improve the speed and accuracy of their assessments and reporting. However, on the other hand, the usefulness of AI can also make it easy for it to become a handicap and a crutch, whereby human workers end up using it more and more, at the expense of their own decision-making skills. As supervisor S9 highlighted, "I think the expectation need[s] to be very clearly communicated. This [AI] is to help you with documentation. But your case conceptualization still need to use your brain. Right? Otherwise, I think our new colleagues just slowly rely on this." There is thus a very valid fear that the overuse and overdependence on AI over time can lead to a deterioration of workers' skill and threaten their agency and autonomy in making decisions for themselves.

\subsubsection{Ambivalence: Uncertainty and Conflicting Sentiments}

Finally, we present the intersection between embracing AI and rejecting it: the \textit{ambivalance} of intra- and inter-personal conflict in viewpoints and feelings towards AI.
% because participants not only displayed an awareness of the limitations of technology, but were also willing to find ways to work around and with it to harness its benefits while respecting its limits.
% For instance, some participants discussed the tendency of the system to output generic or overly verbose text. 
% For instance, many participants S6 analyzed the system's output: "It also depends on how accurate the case notes we put in are, because I put in a very generic one, then it just analyzes it accordingly and then you just give me a very generic form of interventions... So that means if you want to have an accurate kind of very precise intervention, then the way that we actually write it, you have to feed it as much info as possible." 
% S3 discussed how they would use a report generated by the system: "You cannot just copy blindly. You must choose, pick and choose which one is relevant to your case, and then put it up accordingly." Similarly, S2 commented that "You just have to sift through the info and pick out what is relevant in different areas." W2 said, "I feel that sometimes it's that they will give... I think it's more like the \textit{duh}\footnote{A local slang term meaning obvious and not worth much further thought.} information like you mentioned that sometimes I have to read and then I don't think that is as relevant. So I will just not use it. But I think it's okay. The time that it takes is more like I'm just reading.
% I'm not actually having to go and think so much, \textit{How do I want to formulate this}? So I just need to read and then if it's helpful, then I will just put it in." Another participant tried a set of case notes where the client colloquially referred to her boss as "uncle", causing the system to misinterpret the individuals involved in the case: "Yeah, but I mean this I think is just... I have to correct it. I would just copy and paste on my assessment. Then I amend if I need to amend." These participants all noted how the AI system, given limited input or context to work with, failed to give the most concise or suitable outputs; yet at the same time, they demonstrated a willingness to adapt their work patterns to harness the benefits it provided.

\begin{quote}
    \textit{"Efficiency-wise, I'm really still on the fence, because I'm spending time giving prompts to the [AI]."} (S4)
\end{quote}

% A few participants were unsure about the benefits of AI.

% \begin{quote}
%     \textit{"All seem relevant, but I'm not sure which one to specifically pick up?"} (W11)
% \end{quote}

Participants were sometimes \textit{unsure if they would benefit from an AI system}, weighing if the benefits of a mostly-accurate AI system outweighed the effort of using it. Examples of such outputs included case recordings that might miss out on specific details (S2, W11), sample interventions that lacked specificity (S4), and reports that might turn out too lengthy and require trimming (S5). A related sentiment came up discussing potential future ideas, specifically with use cases that required a large model context window: S6 and W5 "didn't know" how useful a summarisation tool for years' of past reports would be. 
% This possibly hinted at an incoherence between their mental models of technological capabilities and the state of modern LLMs.


\begin{quote}
    \textit{"It's not very nice when, you know, people talk to me... [and I'm like] keep tap tap tap [on the laptop].} (S6)
\end{quote}


\textit{Usage of technology in the presence of a client} was also a contentious topic, given the discussions on using AI systems during client sessions. S2 was "not comfortable" using a laptop while attending directly to a client, but W1 was.
% D1 was happy to audio-record sessions for real-time processing by the AI. 
S6 and W5 disliked live note-taking on a laptop, but were happy to replace it by audio recording the sessions.

% S10 discussed an assessment produced by the system: "So because this one, I mean AI also has certain limitations... So I think that what is for us to actually emphasize with the supervisee that these are just guidelines, but you also need to customize it based on where your client is at and where you are at in terms of your experience and confidence. So that's my thought." This demonstrated an additional element of involving not just a single user, but multiple stakeholders in adapting to the introduction of AI technologies in their daily work.


\begin{quote}
    \textit{"It's the supervisor's role to keep emphasising to the supervisee: to not be married to this assessment."} (S10)
\end{quote}

Finally, participants were divided over the 
% potential of overreliance 
\textit{potential negative effects on worker skills} of AI. S7 felt face-to-face client interactions were still the litmus test of a worker's skill: "it will eventually be clear that this was the recommendation of [the tool] but then you went and did something else... and then you'll see that it's just a mess in that way." S3 agreed: "When they do their work, it is mostly real-time. You don't go back to pen and paper and start to develop [a solution], but it's more about what is presented to them immediately [and how they react]."
% S5 felt it unlikely that workers would "bring the[ir] laptops in" and directly read off the screen to their clients. 
% These participants indicated that the human nature of the social work profession itself guarded against overreliance on computer systems. 
On the other hand, others still expressed caution. S9 felt "the expectation [for use] needs to be very clearly communicated", and S10 wanted "to actually emphasize with the supervisee that these are just guidelines" rather than a gospel to be followed. S7 mentioned the "integrity of their work", hinting at a need for workers to maintain an awareness that the system was only an aid and not a replacement for their skills.

% The above examples showcase some key points of uncertainty about how AI and technology fit into the sector; undoubtedly, these are some of the most important questions going forward. 

% mention quote on precedence?

% Yet, the use of AI here presents an ethical conundrum. Participants mentioned multiple times how risk and safety concerns (i.e. any potential threats to the physical safety of the client)
% Yet, herein lies the ethical conundrum of utilizing AI in social work: it can help social workers to reduce the mental and manual burdens of data cleaning and processing, but before doing so, they need to first and foremost ensure that it is in line with "

% These statements show that social workers recognize and value the tremendous time-saving benefits that AI can offer, but are hesitant and cautious about using them due to potential privacy concerns (e.g., whether the AI tool would retain clients' information, or whether the data would be used to train the AI model). , and with the technological advancements and accelerated data processing powers brought about by big data and AI technologies, "[t]oday's social workers face issues involving values and ethics that their predecessors in the profession could not possibly have imagined", including concerns of monitoring, surveillance, data privacy and confidentiality \cite{reamer2018social}. Our interviews therefore underscore the importance of data privacy as a core value of social workers and the exigency of designing and developing ethical AI systems that can safeguard and protect the confidentiality and personal information of clients in the context of social work.


% Proficiency >> accepting how AI can "replace" some part of their work
% i just briefly go through these sections and feel the quotes and illustrations are good. But how can i connect them with 5.2.1 and 5.2.2? I feel 5.2.1/2 is the main question that we want to answer, so we may need to correspond our findings to them explicitly?