\section{Stage 1: Preliminary Interviews}
\label{sec:stage1design}


% \begin{table}
%   \caption{Workshop Participants from Agency A and Agency B}
%   \label{tab:workshopParticipants}
%   \begin{tabular}{cc|cc}
%     \toprule
%     \multicolumn{2}{c|}{\textbf{Agency A}}&\multicolumn{2}{c}{\textbf{Agency B}} \\
%     \midrule
%     Code&Role&Code&Role\\
%     \midrule
%     \multicolumn{2}{c|}{Group 1}& \multicolumn{2}{c}{Group 1} \\
%     \midrule
%     TS1 & Senior Social Worker & CD1 & Director\\
%     TS4 & Social Worker & CY1 & Youth Work Services\\
%     TS5 & Social Worker & CY2 & Youth Work Services\\ 
%     TS6 & Social Worker & CY3 & Youth Work Services\\
%     TS7 & Social Worker & CP1 & Youth Projects\\
%     && CP2 & Youth Projects\\
%     && CP3 & Youth Projects\\
%     \midrule
%     \multicolumn{2}{c|}{Group 2}&\multicolumn{2}{c}{Group 2} \\
%     \midrule
%     TD1 & Executive Director & CC1 & Counsellor\\
%     TD2 & Senior Director & CC2 & Counsellor\\
%     TS2 & Social Worker & CC3 & Counsellor\\
%     TS3 & Social Worker & CC4 & Counsellor\\
%     TP1 & Psychologist & CP4 & Youth Projects\\
%   \bottomrule
% \end{tabular}
% \end{table}

% \begin{table}
%   \caption{Other Participants from Agency A and Agency B}
%   \label{tab:otherParticipants}
%   \begin{tabular}{ccl}
%     \toprule
%     Code&Role\\
%     \midrule
%     \multicolumn{2}{c}{\textbf{Agency A}} \\
%     \midrule
%     TS8 & Social Worker\\
%     TS9 & Social Worker\\
%     TS10 & Social Worker\\
%     \midrule
%     \multicolumn{2}{c}{\textbf{Agency B}} \\
%     \midrule
%     CC5 & Counsellor\\
%   \bottomrule
% \end{tabular}
% \end{table}

% \begin{figure}
% \begin{minipage}{.5\textwidth}
\begin{figure}
    \centering
    \includegraphics[scale=0.25]{images/masked participants + postits.png}
    \caption{(Left) Workshop participants (faces masked for review); (Right) Notes generated by participants}
    \Description{(Left) A picture of workshop participants with faces masked for review; (Right) Post-It notes generated by participants.}
    \label{fig:workshops}
\end{figure}
% \end{minipage}%
% \begin{minipage}{.5\textwidth}
\begin{table}
  \caption{Workshop Participants from Agency A and Agency B}
  \label{tab:workshopParticipants}
  \begin{tabular}{cc|cc}
    \toprule
    \multicolumn{2}{c|}{\textbf{Agency A}}&\multicolumn{2}{c}{\textbf{Agency B}} \\
    \midrule
    Code&Role&Code&Role\\
    \midrule
    TS1 & Snr Social Worker & CD1 & Director\\
    TS2 & Social Worker & CY1 & Youth Work \\
    TS3 & Social Worker & CY2 & Youth Work \\ 
    TS4 & Social Worker & CY3 & Youth Work \\
    TS5 & Social Worker & CP1 & Youth Projects\\
    TS6 & Social Worker & CP2 & Youth Projects\\
    TS7 & Social Worker & CP3 & Youth Projects\\
    TS8 & Social Worker & CP4 & Youth Projects \\
    TS9 & Social Worker & CC1 & Counsellor \\
    TS10 & Social Worker & CC2 & Counsellor \\
    TD1 & Exec. Director & CC3 & Counsellor\\
    TD2 & Snr Director & CC4 & Counsellor\\
    TP1 &  Psychologist & CC3 & Counsellor\\
    && CC4 & Counsellor\\
    \Description{Table describing workshop Participants from Agency A and Agency B. There are 13 participants from Agency A, mostly social workers, and 14 participants from Agency B, a mix of youth workers, youth project workers, and counsellors.}
  \bottomrule
\end{tabular}
\end{table}
% \end{minipage}
% \end{figure}


We adopt a two-stage PD process for a comprehensive and structured understanding of the role of social work in AI. In Stage 1, we conduct formative PD workshops to shape our understanding of the interplay between AI and social work. The knowledge gained here sets the foundation for a more empirical, in-depth investigation in Stage 2 (Section \ref{sec:stage2design}).

We began with two workshops (Fig. \ref{fig:workshops}) with two social service agencies, A and B, in November 2023. This provided a high-level understanding of the social service sector pertaining to possible AI solutions and possible associated implementation challenges. A total of 27 SSPs were involved, hailing from different roles and experience levels (Table \ref{tab:workshopParticipants}),
% came from a variety of roles within their agencies (Table \ref{tab:workshopParticipants}) and worked with different groups of clients (e.g. youths, families, and adult clients). 
giving us a wide range of perspectives to draw upon. 

Each workshop lasted around 90 minutes, comprising a  briefing and discussion session. In the briefing, we introduced LLMs in the form of ChatGPT\footnote{At the time of the workshops (October 2023), ChatGPT was the main LLM widely accessible to the public.}. We demonstrated AI's strengths (e.g., brainstorming, ideation \cite{gero2023social, shaer2024ai}, summarisation \cite{wang2023chatgpt}, formatting and rewriting of text \cite{bhattaru2024revolutionizing}) and weaknesses, including hallucination \cite{maynez2020faithfulness} and limited context windows \cite{liu2024lost}. This ensured that all participants, including those with only a cursory understanding or awareness of ChatGPT\footnote{ChatGPT had been available for around 10 months at this point. Some workers had tried using it, even experimenting with it for their work, while others had never used it themselves.} had a baseline understanding of LLMs.

% just to clarify, so here is like before conducting the interview, there is a session to inform participants about the potential use of LLM? I wonder if this will give a preconception to them when answering the questions regarding LLM's pros/cons? As I remembered when I was coding there was a code called ideation/summarization/formatting haha. Will it be affected by this preconception? I wonder if it would be better to be a subsection..?

Next, we split participants into tables of 4-6 people each for small-group discussions. Here, we sought to encourage participants to freely express their thoughts and build on one anothers' ideas \cite{bonnardel2020brainstorming}. Participants were given stacks of Post-Its and asked a series of questions. For each question, they spent around five minutes writing their responses on individual Post-Its. Then, they pasted their Post-Its on a board and clustered similar ideas together, allowing common themes to emerge. To kickstart the discussion with more familiar topics, we asked the participants about their work: \textit{"What does a day at work look like for you?"}, \textit{"What difficulties have you recently faced in your work?"}. We then brought AI into the picture:\textit{ "How can AI help you in your work?"}, \textit{"What expectations do you have for an AI-powered social service tool?"}. Having elicited some ideas, we then introduced two rounds of 6-8-5 sketching where participants quickly drew out concepts of possible AI tool that could help them. Finally, we asked, \textit{"What are some concerns you might have with using AI in your work?"}.  % just to confirm, the procedure of stage 1 is like: basic knowledge about LLM -> participants are given some predefined questions -> start interview -> merge the answers directly WITH the participants? Looks like it can be unpacked more later

After the workshops, the researchers transferred the physical Post-Its to a Miro board for further analysis, and deduced themes from the clustering done during the workshops. The main findings are presented in the next section.

% Due to the iterative nature of our research process, these later sessions served to both further ideate and critique potential solutions, and also test the concrete prototypes developed at that stage. We include the former as part of our discussion here, and leave the latter for a later part of the paper.

% We start by summarising some common difficulties that plague social workers in their work. We then present how participants felt that AI could help address these issues. Finally, we discuss the potential for AI to negatively affect organisational stability in a social work agency, and ways to begin understanding this complex issue.

