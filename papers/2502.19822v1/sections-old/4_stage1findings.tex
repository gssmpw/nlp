\section{Preliminary Findings}
\label{sec:preliminaryfindings}

We categorise our findings into three main themes: Difficulties faced by SSPs in their daily work, possible AI-assisted solutions, and potential risks of using AI systems.

% Comment: Below are the themes that discovered WITH participants right? If it will be better to write an overview at first?

\subsection{Difficulties in Social Service Practice}
\label{sec:stage1difficulties}

% The most common frustration that emerged was documentation. Across the social work workflow, the constant need to document "anything and everything" (TS3) is a major timesink. Reports must be written in a systematic manner in many different categories like bio-psychological scales (TS2) or numerous risk factors (TS3). Further, the same content may need to be repackaged for different stakeholders (TD2, TS3) like colleagues or other agencies, creating tedious duplicate work. Beyond repackaging and reformatting information, documentation also involves analysis of client profiles. Workers may need to synthesise information from many pages of case notes to formulate assessments and justify a certain judgement (TS3), a cognitively challenging task. 

% Beyond being time-consuming, documentation is also not a task that is in the innate skillset of a social worker. Senior personnel desired greater incorporation of theoretical concepts in their staff's reports (TD1, TD2), observing them "knowing what to do" in practice but being unable to translate that well into writing (TD1). This is especially relevant when writing assessments of a client, where workers ideally select from a "toolkit" of frameworks to produce a justified and sound assessment (TS3).

% Aside from documentation, case formulation is another key step integral to case work. Piecing together disparate information from across many pages of case notes is time-consuming and prone to workers inadvertently missing certain key information (TD1). This is particularly tricky during a live session with the client, where the non-linear process of information discovery (CC2) means workers are constantly trying to organise their "all over the place" (CC5) thoughts to understand a case holistically before proceeding to ask the client the most pertinent questions. This may inadvertently cause them to miss certain key insights or red flags (TS2, TD1) only evident from looking at the gathered information as a whole.

Echoing past sentiments \cite{singer_ai_2023, tiah_can_2024}, participants cited the \textit{need to document "anything and everything"} (TS3) as a major pain point. SSPs have to write systematic reports in different structured formats (e.g. bio-psychological scales (TS2), risk factor assessments (TS3)) and repack the same content for different stakeholders (TD2, TS3) like colleagues or other agencies, creating tedious duplicate work. Additionally, some SSPs struggle with putting their ideas into text, "knowing what to do" in practice but being unable to translate that well into writing (TD1). This results in a lack of ability to incorporate theoretical concepts in reports, something noted particularly by senior personnel (TD1, TD2). 
% This is especially relevant when writing assessments of a client, where workers should ideally select from a "toolkit" of frameworks to produce a justified and sound assessment (TS3). 
% MH: difficulty 1: doc - interesting! just tried to summarize the mainpoints: 1) reduce time; 2) help junior staffs to proofread the report writing? Will it be better to seperated to two paragraphs..?

Participants also noted the \textit{challenges in case formulation}, the synthesising of information to craft assessments and justify a certain judgement (TS3). Piecing together disparate information across many pages of case notes is cognitively challenging and time-consuming (TD1). This is especially tricky during direct interaction with a client, where the non-linear process of information discovery (CC2) leaves workers constantly trying to organise their "all over the place" (CC5) thoughts to understand a case holistically before proceeding to ask the client the most pertinent questions. This may inadvertently cause them to miss certain key insights or red flags (TS2, TD1) only evident from looking at the gathered information as a whole.

\subsection{Possible Solutions}
\label{sec:stage1solutions}

Having elucidated these difficulties, participants floated numerous ideas for how AI could help with these issues. Many participants expressed a desire for a tool to help with \textit{manual labour}- for example, turning point-form notes into formal reports (CP1, CP2, CC1, TS2). TD1 shared how SSPs frequently used the "5Ps" framework\footnote{Presenting problem, Predisposing factors, Precipitating factors, Perpetuating factors, Protective factors} to organise case information into different, logically linked categories to aid subsequent analysis. This is one possible, largely menial and procedural task that an LLM could perform with high accuracy. Other forms of documentation mentioned, like the Data, Intervention, Assessment, Plan (DIAP) or Bio-Psychosocial-Spiritual (BPSS) formats, were also common report structures that could be automatically generated. 

More cognitively-intensive, \textit{mental labor} tasks were also considered, such as the generation of assessments (TS3, TS4, TS6, TS7). This requires fitting a client's information into theoretical frameworks to produce well-grounded assessments and analyses. AI tools could improve both the speed at which relevant information is synthesised and categorised, and the quality of the output through use of technical, industry-standard terms, something that is often lacking in many SSPs (TD1, TD2). Another critical task is intervention planning, where workers follow established models like Cognitive Behaviour Therapy (CBT) or Solution-Focused Brief Therapy (SFBT) and craft a plan for their clients based on the guiding steps in each model. An AI tool could allow rapid generation of numerous possible interventions (CP1), leaving the user to pick and choose from the suggestions offered (TD1).


\subsection{Areas of Risk}
\label{sec:stage1risks}
In line with past work on the dangers of AI adoption [cite], participants echoed certain concerns about the use of AI tools. Privacy of client data was uniterally mentioned by all, a universal concern core to the social sector. Storage of personal client information (TS3, TP1) was a particular worry, and TS1 noted the possibility of workers entering sensitive information into a system by mistake. On the staff competency front, a common concern was the consequences of AI taking on an increasing part of the worker's job scopes. Specifically for analytical or ideative tasks, some seniors (TD1, TD2, TS1) were concerned about the loss in critical thinking skills of junior workers who might become overreliant on the tool to perform their work for them. Multiple participants also raised the possibility of inaccurate outputs from an AI system, particularly risky when less experienced workers fail to tell when the AI's output might be suboptimal and proceed to adopt its suggestions anyway. 
% 1) data privacy; 2) overreliance; 3) inaccurate output (participant identifiers are missed for 3 (?)


% This doubtless was influenced by our intial presentation on generative AI capabilities, where we showed an example of an LLM summarising and reformatting case notes. 

% \subsection{Research Questions and Testing Goals}

% From the above, we derive the following research questions.

% First, while many workers suggested they would be interested in AI automation tools, care needs to be taken to design such tools in a manner that is useful to them. Developing a tool that does not fit in their workflow or give outputs that are what they are looking for is pointless.


% \textbf{\textit{RQ1}}: What kind of AI tools do social workers find useful in their daily lives? How can they use these in their daily work?


% Besides non-useful outputs (possibly a system designer flaw), AI tools may also make more micro-level mistakes, such as by making topically relevant, yet incorrect or suboptimal suggestions. The ability of LLMs and other AI systems to handle social work-related queries is, to our knowledge, almost entirely unexplored.


% \textbf{\textit{RQ2}}: How accurately can modern AI tools handle these tasks?


% Given a tool that provides relevant, high-quality aid to social workers, the question shifts to how these tools should be most effectively implemented in and throughout an agency, maximising benefits while minimising harm.


% \textbf{\textit{RQ3}}: How do staff in different positions in the organisation use or think about the tool differently? What different user requirements might they have?


% \textbf{\textit{RQ4}}: How should staff in different positions in the organisation use or be given access to the tool differently? How can the tool serve to complement while not replacing human social workers?