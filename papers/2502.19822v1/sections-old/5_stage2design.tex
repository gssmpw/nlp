\section{Stage 2: Prototype and User Testing}
\label{sec:stage2design}

The above workshops gave us an understanding of what a relevant and useful AI system to aid SSPs might look like. To empirically evaluate how well the ideas raised by participants would translate to real use, we built a prototype tool incorporating many of the proposed solutions (Section \ref{sec:stage2systemdesign}), and tested these out with another group of SSPs (Section \ref{sec:stage2fgddesign}). 

\subsection{System Design}
\label{sec:stage2systemdesign}

\begin{figure}
    \centering
    \includegraphics[scale=0.25]{images/prototypetool.png}
    \caption{Prototype AI Tool}
    \Description{A screenshot of our prototype AI tool. It has a text box for entry at the top, a list of radio buttons to select output modalities, a text box for extra instructions to the model a user might want to input, and a "generate" button.}
    \label{fig:tool}
\end{figure}

Based on the opportunities and concerns identified above, we created a prototype assistant tool (Figure \ref{fig:tool}) and tested it out in a series of focus group discussions (FGDs). Our design approach was to explore the full range of possible ways in which workers could use AI for help. We positioned the prototype system as a \textit{general purpose AI assistant for social service practitioners}, where users could present the system with as little or as much information about a client as they wished, before asking it to generate various outputs according to what they needed. % that's pretty clear! Will there be an interface of it? Also a bit not sure - so the function modules of this prototype is implemented based on stage 1 right? Will it be necessary to unpack how we correspond the participant's envision to the function module?

The tool comprised an input text box for users to enter details about their client and a number of different types of output options. Users could select an output option based on a desired use case, then click a "generate" button to produce an LLM-written\footnote{OpenAI GPT-4 Turbo, gpt-4-1025-preview} response. These output options addressed various difficulties in social work, such as manual labor and mental labor, (Section \ref{sec:stage1difficulties}) and simulated potential uses addressing these difficulties (Section \ref{sec:stage1solutions}). In response to complaints about manual work, the tool offered options to rewrite workers' rough notes into various organization-wide formats (e.g. BPSS\footnote{Bio-Psychological-Spiritual - a recognised standard for client screening}, DIAP\footnote{Data, Intervention, Assessment, Plan - a standard post-client meeting report format}, Ecological, 5Ps). Users also raised points about assessments, case conceptualization, and ideation. We included options for strength, risk, and challenge assessments, following common social service practices \cite{rooney2017direct}. Finally, we added options to generate client intervention plans according to three common theoretical models: CBT\footnote{Cognitive Behavioural Therapy}, SFBT\footnote{Solution-Focused Brief Therapy}, and Task-Centered interventions. 

An important consideration was ensuring the generalisability of our findings, making sure our system and subsequent discussions adequately represented the vast possibilities of AI use cases, and not being limited to exploring a particular LLM in a specific configuration. We address this by noting the difference between areas that other or future AI systems may improve in, such as reasoning and knowledge, and those that they are unlikely to, such as being able to account for a user's (or client's) contextual factors. It is the latter that we focus our investigation on: Given an AI system is ultimately constrained by its training dataset, what are the possibilities with such systems and what are the potential downsides?

\subsection{Focus Group Discussion Design}
\label{sec:stage2fgddesign}

We tested the prototype system in a round of Focus Group Discussions (FGD), where we simulated a contextual inquiry process \cite{holtzblatt2017contextual} by asking participants to walk us through their use of our tool and explain their thought processes along the way. We also encouraged them to point out weaknesses or flaws in the system and suggest potential improvements. We structured the FGDs for two main purposes: 1) to understand the \textit{opportunities} of AI use in social work (Section \ref{sec:stage2designuses}), based on the opportunities identified in Section \ref{sec:stage1solutions}; and 2) to identify the \textit{challenges} of using AI in this sector (Section \ref{sec:stage2designrisks}), based on the concerns raised in Section \ref{sec:stage1risks}.

\begin{table}
  \caption{Focus Group Participants from Agency A}
  \label{tab:thkMayParticipants}
  \begin{tabular}{cc|cc|cc|cc|cc}
    \toprule
    \multicolumn{2}{c|}{\textbf{Centre 1}}&\multicolumn{2}{c|}{\textbf{Center 2}}&\multicolumn{2}{c|}{\textbf{Center 3}}&\multicolumn{2}{c|}{\textbf{Center 4}}&\multicolumn{2}{c}{\textbf{Management}}\\
    \midrule
    Code&Role&Code&Role&Code&Role&Code&Role&Code&Role\\
    \midrule
    \multicolumn{2}{c|}{Group 1}& \multicolumn{2}{c|}{Group 3}& \multicolumn{2}{c|}{Group 4}& \multicolumn{2}{c|}{Group 7}&\multicolumn{2}{c}{-}\\
    \midrule
    W1 & Social Worker & S4 & Supervisor & S6 & Senior SW & S8 & Senior SW & D1 & Director*\\
    W2 & Social Worker & S5 & Supervisor & W5 & Social Worker & S9 & Senior SW & \multicolumn{2}{c}{\textit{*Observer in}}\\
    W3 & Social Worker & W4 & Social Worker &&& S10 & Senior SW & \multicolumn{2}{c}{\textit{multiple sessions}}\\
    \midrule
    \multicolumn{2}{c|}{Group 2}&\multicolumn{2}{c|}{}&\multicolumn{2}{c|}{Group 5}&\multicolumn{2}{c|}{Group 8} \\
    \midrule
    S1 & Supervisor &&& S7 & Senior SW & W11 & Social Worker\\
    S2 & Supervisor &&& W7 & Social Worker & W12 & Social Worker\\
    S3 & Supervisor &&& W8 & Social Worker & W13 & Social Worker\\
    \midrule
    &&&& \multicolumn{2}{c|}{Group 6}\\
    \midrule
    &&&& W9 & Social Worker\\
    &&&& W10 & Social Worker\\
    &&&& C1 & Snr Counsellor\\
  \bottomrule
  \Description{Table describing Focus Group Participants from Agency A. There are 8 groups from 4 different centres, plus a director classified under "Management".}
\end{tabular}
\end{table}

We conducted the FGDs in groups of 2-4 SSPs from Agency A.
%\footnote{Agency B opted to explore a different LLM-based social service solution. We discuss this work in another paper (forthcoming).}. 
We sought a mix of workers of varying seniorities (Table \ref{tab:thkMayParticipants}), to gain perspectives across different organizational levels. We held a total of 8 sessions totaling 24 participants. Each session averaged 45-60 minutes in length and was audio-recorded for transcription and analysis. 



\subsubsection{Understanding Opportunities with AI}
\label{sec:stage2designuses}

The main goal of the FGDs was to simulate real-life usage of our system. We asked participants to bring along real case notes from recent clients they had worked with, anonymizing them beforehand to remove any personal information. We explained the various functions of the system, then told participants to imagine themselves using it to help with the cases they had on hand. As participants brought along different types of case files (e.g., rough short-hand notes, complete reports, intake files), this allowed us to explore a range of different use cases, from generating reports for documentation to planning future sessions or interventions for the client.

We also learned early on that case supervision by senior workers was a key means of addressing the difficulties faced by junior workers and encouraging worker development. We therefore sought to understand how AI could play a collaborative role in the supervision process. In sessions with supervisors present, we asked these more senior workers how they felt the various functions of the system could aid them in their discussion of cases with their supervisees. 

% just have a general question emerged when i read here: why two stage? anywhere explained this? I feel the workshop is also used to understand what is expected by SW? What's the main difference between FGD and workshop? I feel maybe need to say 1-2 sentence somewhere? Is that because something will be found ONLY AFTER a real prototype is there?

% With state-of-the-art LLMs as a platform, we sought to elucidate the specific areas in which AI systems fall short in providing the required level of specificity in social work cases. 
% oh maybe a silly question is: I haven't read a paper that use FGD as its main research method, but I wonder will it require describing the FGD protocol? Like specifically what questions are asked?

% is then offered two sets of analysis, roughly corresponding to the user needs of documentation and ideation. The four options in the first row - Ecological, Strengths, Risks, 5 Ps - are some of the most common ways of analysing case information that came up throughout our discussions. These can be used either during a session with a client, to quickly organise their notes and take stock of their current progress, or after a session when writing a report in a certain format. The three options below - Task-Centered, CBT, SFBT - are three major intervention modalities \cite{rooney2017direct} that collectively cover a wide range of client types. These can help social workers review their options for selecting a suitable plan for their client, or easily selecting and combining pieces of each potential plan to form a new, hybrid one.

\subsubsection{Understanding Challenges with AI}
\label{sec:stage2designrisks}

On the other hand, we also sought to understand the possible challenges of using AI tools in social work. In testing the tool, we asked participants to review the quality of the outputs, compare them to their own, and identify areas where they would fall short of expectations or fail to be useful in their daily work. We note that while our platform is based on GPT-4, a general purpose model not tuned for social work analysis\footnote{Specialised AI systems for social work case management do exist (e.g. \cite{socialworkmagic2024, caseworthy2024}, but these are focused on user experience and do not provide greater insight than vanilla GPT; furthermore, \cite{socialworkmagic2024} advises users to take its assessments and suggestions as only a starting point, which aligns with how we position our tool.}, our aim was not to discover the specific weaknessses of GPT-4 or any LLM in particular. Rather, it was to understand what participants perceived to be good outputs from an AI system, and in the process understand how they might be affected by any possible shortcomings of such a system. % I actually feel the footnote here is pretty important at the first glimpse haha

We also explored some of the wider effects of AI use. Participants were asked about potential overreliance on the tool and the possibility of becoming overly trusting of the system's output. Senior workers in particular were asked about how they perceived junior workers new to the sector having such a system to help them with their daily work.
% so the novelties lied in: test vanilla GPT and analyze the need/requirement/concern from users, rather than design a new non-generic tool to address these concern?

% This interface was meant to mimic what a actual assistant tool used by social workers might look like. A large, free-text text box at the top of the screen allows users to put large sets of case notes in. 

% \subsection{Remote Testing}

% In addition to the above, we tested the same tool on a wider scale within THK. These staff members were given access to a limited set of functionalities of the tool, specifically those that had been identified as having more reliable outputs, and told to use them in their daily work. From this we collected more data on the tool's performance and its users behaviours.

\subsection{Data Analysis}
\label{sec:stage2analysis}

To analyze the session transcripts, we performed qualitative thematic analysis \cite{braun2006using}. Qualitative thematic analysis is a method for identifying and analyzing key patterns and themes within interview and focus group data that affords a flexible and iterative coding process \cite{braun2006using}, making it particularly suited to the iterative nature of PD studies like this one. Due to the nascence of AI in social service sector and the paucity of existing theoretical frameworks specific to the context of our study, we adopted an inductive, bottom-up approach of coding, which allowed us to uncover emergent themes in a data-driven manner and surface the latent thoughts, feelings and attitudes of SSPs in their own voice, rather than attempting to fit the data within a preexisting framework or being influenced by researchers' prior analytic preconceptions. 

Coding of the transcripts was carried out sequentially and iteratively. First, four researchers independently coded two transcripts, and met to discuss and share the codes. Subsequently, two researchers then independently developed a codebook based on identified codes, and met to combine the two into a unified codebook of themes and subthemes. The codebook was shared with the other two researchers alongside explanatory memos, and used to code the remaining six transcripts, which were equally divided among all four researchers, with at least two researchers coding each transcript. The researchers then met two additional times to discuss the codes and resolve disagreements, until consensus was met. Finally, two researchers met in multiple rounds to collaboratively revise and refine the codes, as new themes were discovered and old ones were combined or retired. These themes form the final framework described below. % do we need to spell out the coding procedure here? Like how many coders, open discussions, etc