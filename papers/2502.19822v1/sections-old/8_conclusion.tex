\section{Conclusion}

In this study, we conducted a series of on-the-ground workshops and focus group discussions to understand the use of generative AI in the social sector. We find that social service practitioners are generally welcoming to AI, and discover that it can potentially be used not just for mundane administrative tasks but also in a range of collaborative discussion scenarios. We elucidate how such systems can be effectively designed through our proposed ACE framework (Attitudes, Considerations, Ethos). We discuss our findings in the context of HAI collaboration and suggest an alternative framing of AI as a (meta-)facilitator - a role that creates and then facilitates new work processes and systems. We then propose key considerations for future system designers, both in social service and beyond, to best harness the potential of generative AI.