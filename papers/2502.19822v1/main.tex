%%
%% This is file `sample-manuscript.tex',
%% generated with the docstrip utility.
%%
%% The original source files were:
%%
%% samples.dtx  (with options: `manuscript')
%% 
%% IMPORTANT NOTICE:
%% 
%% For the copyright see the source file.
%% 
%% Any modified versions of this file must be renamed
%% with new filenames distinct from sample-manuscript.tex.
%% 
%% For distribution of the original source see the terms
%% for copying and modification in the file samples.dtx.
%% 
%% This generated file may be distributed as long as the
%% original source files, as listed above, are part of the
%% same distribution. (The sources need not necessarily be
%% in the same archive or directory.)
%%
%% Commands for TeXCount
%TC:macro \cite [option:text,text]
%TC:macro \citep [option:text,text]
%TC:macro \citet [option:text,text]
%TC:envir table 0 1
%TC:envir table* 0 1
%TC:envir tabular [ignore] word
%TC:envir displaymath 0 word
%TC:envir math 0 word
%TC:envir comment 0 0
%%
%%
%% The first command in your LaTeX source must be the \documentclass command.
%%%% Small single column format, used for CIE, CSUR, DTRAP, JACM, JDIQ, JEA, JERIC, JETC, PACMCGIT, TAAS, TACCESS, TACO, TALG, TALLIP (formerly TALIP), TCPS, TDSCI, TEAC, TECS, TELO, THRI, TIIS, TIOT, TISSEC, TIST, TKDD, TMIS, TOCE, TOCHI, TOCL, TOCS, TOCT, TODAES, TODS, TOIS, TOIT, TOMACS, TOMM (formerly TOMCCAP), TOMPECS, TOMS, TOPC, TOPLAS, TOPS, TOS, TOSEM, TOSN, TQC, TRETS, TSAS, TSC, TSLP, TWEB.
% \documentclass[acmsmall]{acmart}

%%%% Large single column format, used for IMWUT, JOCCH, PACMPL, POMACS, TAP, PACMHCI
% \documentclass[acmlarge,screen]{acmart}

%%%% Large double column format, used for TOG
% \documentclass[acmtog, authorversion]{acmart}

%%%% Generic manuscript mode, required for submission
%%%% and peer review
\documentclass[sigconf]{acmart}
%% Fonts used in the template cannot be substituted; margin 
%% adjustments are not allowed.
%%
%% \BibTeX command to typeset BibTeX logo in the docs
\AtBeginDocument{%
  \providecommand\BibTeX{{%
    \normalfont B\kern-0.5em{\scshape i\kern-0.25em b}\kern-0.8em\TeX}}}

%% Rights management information.  This information is sent to you
%% when you complete the rights form.  These commands have SAMPLE
%% values in them; it is your responsibility as an author to replace
%% the commands and values with those provided to you when you
%% complete the rights form.
\copyrightyear{2025}
\acmYear{2025}
\setcopyright{rightsretained}
\acmConference[CHI EA '25]{Extended Abstracts of the CHI Conference on
Human Factors in Computing Systems}{April 26-May 1, 2025}{Yokohama, Japan}
\acmBooktitle{Extended Abstracts of the CHI Conference on Human Factors in
Computing Systems (CHI EA '25), April 26-May 1, 2025, Yokohama,
Japan}\acmDOI{10.1145/3706599.3719736}
%
%  Uncomment \acmBooktitle if th title of the proceedings is different
%  from ``Proceedings of ...''!
%
% \acmBooktitle{Woodstock '18: ACM Symposium on Neural Gaze Detection,
%  June 03--05, 2018, Woodstock, NY} 
% \acmPrice{15.00}
% \acmISBN{978-1-4503-XXXX-X/18/06}


%%
%% Submission ID.
%% Use this when submitting an article to a sponsored event. You'll
%% receive a unique submission ID from the organizers
%% of the event, and this ID should be used as the parameter to this command.
%%\acmSubmissionID{123-A56-BU3}

%%
%% For managing citations, it is recommended to use bibliography
%% files in BibTeX format.
%%
%% You can then either use BibTeX with the ACM-Reference-Format style,
%% or BibLaTeX with the acmnumeric or acmauthoryear sytles, that include
%% support for advanced citation of software artefact from the
%% biblatex-software package, also separately available on CTAN.
%%
%% Look at the sample-*-biblatex.tex files for templates showcasing
%% the biblatex styles.
%%

%%
%% The majority of ACM publications use numbered citations and
%% references.  The command \citestyle{authoryear} switches to the
%% "author year" style.
%%
%% If you are preparing content for an event
%% sponsored by ACM SIGGRAPH, you must use the "author year" style of
%% citations and references.
%% Uncommenting
%% the next command will enable that style.
%%\citestyle{acmauthoryear}

%%
%% end of the preamble, start of the body of the document source.
\begin{document}

%%
%% The "title" command has an optional parameter,
%% allowing the author to define a "short title" to be used in page headers.
% \title{Beyond Decision Support - Exploring? New? Designing? Understanding Modern HAI Collaboration? Systems? in the Social Sector?}

% Designing AI for Social Services: Insights from Social Workers on Automation and Human Judgement

% AI in Social Services: Balancing Automation with Professional Judgement

% Navigating AI Integration in Social Work: A Framework for Complementing Human Expertise

\title{Empowering Social Service with AI: Insights from a Participatory Design Study with Practitioners}

% Navigating Trust and Discretion: Generative AI in Social Work Practice
% Human-AI Collaboration in Social Services: Balancing Efficiency and Expertise
% Empowering or Undermining? Exploring Generative AI’s Role in Social Work
% From Support to Collaboration: Designing Generative AI for Social Service Workers
% Ethics and Efficiency: Integrating Generative AI into Social Service Practice
% Rethinking AI in Social Work: A Framework for Human-AI Collaboration
% Generative AI in Social Services: Addressing Overreliance, Bias, and Professional Discretion
% Co-Designing AI Tools for Social Work: Balancing Innovation with Practitioner Values

%%
%% The "author" command and its associated commands are used to define
%% the authors and their affiliations.
%% Of note is the shared affiliation of the first two authors, and the
%% "authornote" and "authornotemark" commands
%% used to denote shared contribution to the research.
% \author{Anonymous Authors}
\author{Yugin Tan}
\email{tan.yugin@u.nus.edu}
\affiliation{
  \institution{School of Computing}
  \institution{National University of Singapore}
  \country{Singapore}
  % \streetaddress{P.O. Box 1212}
  % \city{Dublin}
  % \state{Ohio}
  % \country{USA}
  % \postcode{43017-6221}
}

\author{Soh Kai Xin}
\email{kaisoh2030@u.northwestern.edu}
\affiliation{
  \institution{School of Communications}
  \institution{Northwestern University}
  \country{United States of America}
}

\author{Zhang Renwen}
\email{r.zhang@nus.edu.sg}
\affiliation{
  \institution{Department of Communications and New Media}
  \institution{National University of Singapore}
  \country{Singapore}
}
\author{Lee Jungup}
\email{swklj@nus.edu.sg}
\affiliation{
  \institution{Department of Social Work}
  \institution{National University of Singapore}
  \country{Singapore}
}

\author{Meng Han}
\email{han.meng@u.nus.edu}
\affiliation{
  \institution{School of Computing}
  \institution{National University of Singapore}
  \country{Singapore}
}

\author{Biswadeep Sen}
\email{e0989386@u.nus.edu}
\affiliation{
  \institution{School of Computing}
  \institution{National University of Singapore}
  \country{Singapore}
}

\author{Lee Yi-Chieh}
\email{yclee@nus.edu.sg}
\affiliation{
  \institution{School of Computing}
  \institution{National University of Singapore}
  \country{Singapore}
}
% \email{trovato@corporation.com}
% \orcid{1234-5678-9012}
% \author{G.K.M. Tobin}
% \authornotemark[1]
% \email{webmaster@marysville-ohio.com}
% \affiliation{%
%   \institution{Institute for Clarity in Documentation}
%   \streetaddress{P.O. Box 1212}
%   \city{Dublin}
%   \state{Ohio}
%   \country{USA}
%   \postcode{43017-6221}
% }

%%
%% By default, the full list of authors will be used in the page
%% headers. Often, this list is too long, and will overlap
%% other information printed in the page headers. This command allows
%% the author to define a more concise list
%% of authors' names for this purpose.
\renewcommand{\shortauthors}{Tan et al.}

%%
%% The abstract is a short summary of the work to be presented in the
%% article.
\begin{abstract}
In social service, administrative burdens and decision-making challenges often hinder practitioners from performing effective casework. Generative AI (GenAI) offers significant potential to streamline these tasks, yet exacerbates concerns about overreliance, algorithmic bias, and loss of identity within the profession. We explore these issues through a two-stage participatory design study. We conducted formative co-design workshops (\textit{n=27}) to create a prototype GenAI tool, followed by contextual inquiry sessions with practitioners (\textit{n=24}) using the tool with real case data. We reveal opportunities for AI integration in documentation, assessment, and worker supervision, while highlighting risks related to GenAI limitations, skill retention, and client safety. Drawing comparisons with GenAI tools in other fields, we discuss design and usage guidelines for such tools in social service practice.
\end{abstract}

%%
%% The code below is generated by the tool at http://dl.acm.org/ccs.cfm.
%% Please copy and paste the code instead of the example below.
%%
\begin{CCSXML}
<ccs2012>
   <concept>
       <concept_id>10003120.10003121.10011748</concept_id>
       <concept_desc>Human-centered computing~Empirical studies in HCI</concept_desc>
       <concept_significance>500</concept_significance>
       </concept>
   <concept>
       <concept_id>10003120.10003130.10003131.10003570</concept_id>
       <concept_desc>Human-centered computing~Computer supported cooperative work</concept_desc>
       <concept_significance>500</concept_significance>
       </concept>
 </ccs2012>
\end{CCSXML}

\ccsdesc[500]{Human-centered computing~Empirical studies in HCI}
\ccsdesc[500]{Human-centered computing~Computer supported cooperative work}
%%
%% Keywords. The author(s) should pick words that accurately describe
%% the work being presented. Separate the keywords with commas.
\keywords{AI Decision-Making, Human-AI collaboration, LLM, Social Service}

%% A "teaser" image appears between the author and affiliation
%% information and the body of the document, and typically spans the
%% page.
% \begin{teaserfigure}
%   % \includegraphics[width=\textwidth]{sample-franklin.png}
%   \caption{Seattle Mariners at Spring Training, 2010.}
%   \Description{Enjoying the baseball game from the third-base
%   seats. Ichiro Suzuki preparing to bat.}
%   \label{fig:teaser}
% \end{teaserfigure}

% \received{?}
% \received[revised]{?}
% \received[accepted]{?}

%%
%% This command processes the author and affiliation and title
%% information and builds the first part of the formatted document.
\maketitle


\section{Introduction}\label{sec:intro}

In computational finance, Monte Carlo simulations are used extensively to estimate the expected value of financial payoffs based on the solution of stochastic differential equations (SDEs) which model the evolution of stock prices, interest rates, exchange rates and other quantities \cite{glasserman04}.  Monte Carlo methods are very general and flexible, but for high accuracy it requires generating a large number of costly SDE path approximations, which has motivated research into a number of variance reduction or, equivalently, cost reduction techniques. One such method is
Multilevel Monte Carlo (MLMC), which was proposed in \cite{GILES2008} and was adapted for various applications that are summarised in \cite{Giles_overview17} and successfully combined with other methods such as quasi-Monte Carlo methods. The main idea of MLMC is to approximate the payoff using different time stepping resolutions when numerically solving the underlying SDE and to generate an optimal number of samples on each level, such that the overall computational cost is minimised subject to the desired bound on the variance. %, such that the total computational cost is minimised. 
The computational savings come from the fact that most samples are computed on the coarser levels and hence are less expensive while only a few samples from the finest levels are required \cite{GILES2008}.


Among the directions in which the computational cost 
of MLMC methods could further be reduced, an important avenue is the use of lower precision calculations, especially for the first Monte Carlo levels where the targeted accuracy is relatively low. 
 An overview of the research on mixed precision for the standard Monte Carlo (MC) framework is provided in \cite{ChowMixedPrecisionStandardMC} but only a few references study the potential of low precision computation in the MLMC framework \cite{Rounding_error_oliver}. To the best of our knowledge, the only MLMC framework with customised precision in the literature is \cite{brugger2014mixed}, but they use a uniform precision for all operations on each Monte Carlo level instead of optimising 
 the precision of each intermediary variable to reduce as much as possible the cost of path generation.
 
An important motivation for an MLMC framework with variable precision would be performing the low precision computations on reconfigurable hardware devices such as Field Programmable Gate Arrays (FPGAs). FPGAs contain customizable logic blocks and connectors that make it easy to adapt the digital circuit architecture for a specific application, leading to a highly parallel and optimised implementation. Therefore they are successfully exploited in applications that require high speed and have high computational workload, such as signal processing \cite{woods2008fpga}, and real time applications like high frequency trading \cite{HFT1,HFT2}. That is why a number of previous works in hardware architecture design implemented the MLMC algorithm to price financial options using FPGAs as accelerators, which resulted in improved speed and power efficiency compared to full CPU architectures \cite{Schryver2013AMM}. The paper \cite{lindsey2016domain} also proposed 
a Domain Specific Language to automate the configuration of FPGAs for this specific application. However, only \cite{brugger2014mixed} proposed a heuristic to reduce the precision in calculations.

In addition, all aforementioned works considered that the random number generation (RNG) is performed in single or double precision. Yet in most cases an important portion of the workload in the overall MLMC simulation comes from the RNG and in \cite{brugger2014mixed} this limited the total computational savings.
To reduce the cost of MLMC simulations in particular those based on the Geometric Brownian Motion (GBM), \cite{approximateICDF_Oliver, NestedOliver} have proposed to use approximate random numbers that are generated by applying an approximation of the inverse CDF to uniform random numbers. In \cite{NestedOliver}, the authors proposed a way to integrate these lower precision random variables into a \textit{nested} MLMC framework and completed a numerical analysis to bound the resulting error at each MC level by a product of the time step and the error in the random number approximation. The same authors show in \cite{approximateICDF_Oliver} that using approximate random variables reduces the cost of path generation by a factor 7.


In this paper we propose a nested MLMC framework that combines the use of approximate random normal variables and lower precision calculations to reduce the computational cost of MLMC even further than \cite{brugger2014mixed,NestedOliver}. We illustrate the efficiency of our framework in Matlab, after making several assumptions on the cost of operations and size of the errors that we carefully justify. We focus on the case of GBM and use the approximate RNG methods presented in \cite{approximateICDF_Oliver} as well as a new slightly modified method that combines CDF inversion and the central limit theorem. To choose the precision of the variables in the low precision path generation, we introduce a novel method to optimise the bit-widths. This optimisation is performed before the main path generation loop is executed and is based on a linear model of the payoff error  
due to rounding when computing in low precision. The error model relies on algorithmic differentiation in a similar manner to \cite{unifying-bwoptim,bitwidth-AD,ADAPT}. The bit-width optimisation procedure can be performed off-line, so this stage can be excluded from the on-line time complexity of our framework. The user specified desired accuracy is then enforced by calculating on-line the number of samples that need to be generated.

In terms of hardware design, we suggest implementing the low precision path generation on FPGAs and the full-precision ones on a CPU or GPU. 
The FPGA offers enough flexibility to define a separate bit-width for every variable in the low precision path generation, and can be reconfigured periodically to update the bit-widths when the market parameters have changed considerably. 


The paper is organized as follows : \Cref{sec:MLMC} introduces MLMC and nested MLMC to make clear the estimator that is implemented in our framework. Then in \Cref{sec:RNG} we detail the methods that could be used to obtain approximate random normally distributed numbers very cheaply for the low precision path generation. In \Cref{sec:error_model} and \Cref{sec:costModel} we propose an error model and a cost model (resp.) that we then use to formulate the optimisation problem that is solved to obtain the optimal bit-widths of fixed point variables in \Cref{sec:optimisation}. Finally we summarise our results and future directions in \Cref{sec:conclusion}.



\section{Related Work}
\label{sec:relatedwork}

\subsection{Current AI Tools for Social Service Practitioners}
\label{subsec:relatedtools}
% the title I feel is quite broad

Artificial Intelligence (AI) has long been used for risk assessments, decision-making, and workload management in sectors like child protection services and mental health treatment \cite{fluke_artificial_1989, patterson_application_1999}. 
Recent applications in clinical social work include risk assessments \cite{gillingham2019can, jacobi_functions_2023, liedgren_use_2016, molala_social_2023}, public health initiatives \cite{rice_piloting_2018}, and education and training for practitioners \cite{asakura_call_2020, tambe_artificial_2018}. Present studies on case management focus mainly on decision support tools \cite{james_algorithmic_2023, kawakami2022improving}, especially predictive risk models (PRMs) used to predict social service risks and outcomes \cite{gillingham2019can, van2017predicting}. A prominent example is the Allegheny Family Screening Tool (AFST), which assesses child abuse risk using data from US public systems \cite{chouldechova_case_2018, vaithianathan2017developing}. Elsewhere, researchers have also piloted AI systems to predict social service needs for the homeless using Medicaid data \cite{erickson_automatic_2018, pourat_easy_2023} or promote health interventions like HIV testing among at-risk populations \cite{rice_piloting_2018, yadav_maximizing_2017}.

Beyond such tools, however, the sector also stands to benefit immensely from newer forms of AI, such as GenAI. SSPs work in time-poor environments \cite{tiah_can_2024}, being often overwhelmed with tedious administrative work \cite{meilvang_working_2023} and large amounts of paperwork and data processing \cite{singer_ai_2023, tiah_can_2024}. GenAI is well placed to streamline and automate tasks such as the formatting of case notes, the formulation of treatment plans, and the writing of progress reports, allowing valuable time to be spent on more meaningful work, such as client engagement and the improvement of service quality \cite{fernando_integration_2023, perron_generative_2023, tiah_can_2024, thesocialworkaimentor_ai_nodate}. There is, however, scant research on GenAI in the social service sector \cite{wykman_artificial_2023}.

% This study therefore seeks to fill this critical gap by exploring how SSPs use and interact with a novel GenAI tool, helping to expand our understanding of the new opportunities that HAI collaboration can bring to the social service sector.

% AI has also been employed for mental health support and therapeutic interventions, with conversational agents serving as on-demand virtual counsellors to provide clinical care and support \cite{lisetti_i_2013, reamer_artificial_2023}.

% The recent rise of GenAI is poised to further advance social service practice, facilitating the automation of administrative tasks, streamlining of paperwork and documentation, optimisation of resource allocation, data analysis, and enhancing client support and interventions \cite{fernando_integration_2023, perron_generative_2023}.


% commercial solutions include Woebot, which simulates therapeutic conversation, and Wysa, an “emotionally intelligent” AI coach, powered by evidenced-based clinical techniques \cite{reamer_artificial_2023}. 
% Non-clinical AI agents like Replika and companion robots can also provide social support and reduce loneliness amongst individuals \cite{ahmed_humanrobot_2024, chaturvedi_social_2023, pani_can_2024, ta_user_2020}.


\subsection{Challenges in AI Use in Social Service}
\label{subsec:relatedworkaiuse}

% Despite the immense potential of AI systems to augment social work practice, there are multiple challenges with integrating such systems into real-life practice. 
Despite its evident benefits, multiple challenges plague the integration of AI and its vast potential into real-life social service practice.
% Numerous studies have investigated the use of PRMs to help practitioners decide on a course of action for their clients. 
When employing algorithmic decision-making systems, practitioners often experience tension in weighing AI suggestions against their own judgement \cite{kawakami2022improving, saxena2021framework}, being uncertain of how far they should rely on the machine. 
% Despite often being instructed to use the tool as part of evaluating a client, 
Workers are often reluctant to fully embrace AI assessments due to its inability to adequately account for the full context of a case \cite{kawakami2022improving, gambrill2001need}, and lack of clarity and transparency on AI systems and limitations \cite{kawakami2022improving}. Brown et al. \cite{brown2019toward} conducted workshops using hypothetical algorithmic tools 
% to understand service providers' comfort levels with using such tools in their work,
and found similar issues with mistrust and perceived unreliability. Furthermore, introducing AI tools can create new problems of its own, causing confusion and distrust amongst workers \cite{kawakami2022improving}. Such factors are critical barriers to the acceptance and effective use of AI in the sector.

\citeauthor{meilvang_working_2023} (2023) cites the concept of \textit{boundary work}, which explores the delineation between "monotonous" administrative labour and "professional", "knowledge-based" work drawing on core competencies of SSPs. While computers have long been used for bureaucratic tasks such as client registration, the introduction of decision support systems like PRMs stirred debate over AI "threatening professional discretion and, as such, the profession itself" \cite{meilvang_working_2023}. Such latent concerns arguably drive the resistance to technology adoption described above. GenAI is only set to further push this boundary, 
% these concerns are only set to grow in tandem with the vast capabilities of generative and other modern AI systems. Compared to the relatively primitive AI systems in past years, perceived as statistical algorithms \cite{brown2019toward} turning preset inputs like client age and behavioural symptoms \cite{vaithianathan2017developing} into simple numerical outputs indicating various risk scores, modern AI systems are vastly more capable: LLMs 
with its ability to formulate detailed reports and assessments that encroach upon the "core" work of SSPs.
% accept unrestricted and unstructured inputs and return a range of verbose and detailed evaluations according to the user's instructions. 
Introducing these systems exacerbates previously-raised issues such as understanding the limitations and possibilities of AI systems \cite{kawakami2022improving} and risk of overreliance on AI \cite{van2023chatgpt}, and requires a re-examination of where users fall on the algorithmic aversion-bias scale \cite{brown2019toward} and how they detect and react to algorithmic failings \cite{de2020case}. We address these critical issues through an empirical, on-the-ground study that to our knowledge is the first of its kind since the new wave of GenAI.

% W 

% Yet, to date, we have limited knowledge on the real-world impacts and implications of human-AI collaboration, and few studies have investigated practitioners’ experiences working with and using such AI systems in practice, especially within the social work context \cite{kawakami2022improving}. A small number of studies have explored practitioner perspectives on the use of AI in social work, including Kawakami et al. \cite{kawakami2022improving}, who interviewed social workers on their experiences using the AFST; Stapleton et al. \cite{stapleton_imagining_2022}, who conducted design workshops with caseworkers on the use of PRMs in child welfare; and Wassal et al. \cite{wassal_reimagining_2024}, who interviewed UK social work professionals on the use of AI. A common thread from all these studies was a general disregard for the context and users, with many practitioners criticising the failure of past AI tools arising from the lack of participation and involvement of social workers and actual users of such systems in the design and development of algorithmic systems \cite{wassal_reimagining_2024}. Similarly, in a scoping review done on decision-support algorithms in social work, Jacobi \& Christensen \cite{jacobi_functions_2023} reported that the majority of studies reveal limited bottom-up involvement and interaction between social workers, researchers and developers, and that algorithms were rarely developed with consideration of the perspective of social workers.
% so the \cite{yang_unremarkable_2019} and \cite{holten_moller_shifting_2020} are not real-world impacts? real-world means to hear practitioner's voice? I feel this is quite important but i didnt get this point in intro!

% why mentioning 'which have largely focused on existing ADS tools (e.g., AFST)'? i can see our strength is more localized, but without basic knowledge of social work i didnt get what's the 'departure' here orz
% the paragraph is great! do we need to also add one in line 20 21?

% \subsection{Designing AI for Social Service through Participatory Design}
% \label{subsec:relatedworkpd}
% % i think it's important! but maybe not a whole subsection? but i feel the strong connection with practitioners is indeed one of our novelties and need to highlight it, also in intro maybe
% % Participatory design (PD) has long been used extensively in HCI \cite{muller1993participatory}, to both design effective solutions for a specific community and gain a deep understanding of that community. Of particular interest here is the rich body of literature on PD in the field of healthcare \cite{donetto2015experience}, which in this regard shares many similarities and concerns with social work. PD has created effective health improvement apps \cite{ryu2017impact}, 

% % PD offers researchers the chance to gather detailed user requirements \cite{ryu2017impact}...

% Participatory design (PD) is a staple of HCI research \cite{muller1993participatory}, facilitating the design of effective solutions for a specific community while gaining a deep understanding of its stakeholders. The focus in PD of valuing the opinions and perspectives of users as experts \cite{schuler_participatory_1993} 
% % In recent years, the tech and social work sectors have awakened to the importance of involving real users in designing and implementing digital technologies, developing human-centred design processes to iteratively design products or technologies through user feedback 
% has gained importance in recent years \cite{storer2023reimagining}. Responding to criticisms and failures of past AI tools that have been implemented without adequate involvement and input from actual users, HCI scholars have adopted PD approaches to design predictive tools to better support human decision-making \cite{lehtiniemi_contextual_2023}.
% % ; accordingly, in social service, a line of research has begun studying and designing for human-AI collaboration with real-world users (e.g. \cite{holten_moller_shifting_2020, kawakami2022improving, yang_unremarkable_2019}).
% Section \ref{subsec:relatedworkaiuse} shows a clear need to better understand SSP perspectives when designing and implementing AI tools in the social sector. 
% Yet, PD research in this area has been limited. \citeauthor{yang2019unremarkable} (2019), through field evaluation with clinicians, investigated reasons behind the failure of previous AI-powered decision support tools, allowing them to design a new-and-improved AI decision-support tool that was better aligned with healthcare workers’ workflows. Similarly, \citeauthor{holten_moller_shifting_2020} (2020) ran PD workshops with caseworkers, data scientists and developers in public service systems to identify the expectations and needs that different stakeholders had in using ADS tools.

% % Indeed, it is as Wise \cite{wise_intelligent_1998} noted so many years ago on the rise of intelligent agents: “it is perhaps when technologies are new, when their (and our) movements, habits and attitudes seem most awkward and therefore still at the forefront of our thoughts that they are easiest to analyse” (p. 411). 
% Building upon this existing body of work, we thus conduct a study to co-design an AI tool \textit{for} and \textit{with} SSPs through participatory workshops and focus group discussions. In the process, we revisit many of the issues mentioned in Section \ref{subsec:relatedworkaiuse}, but in the context of novel GenAI systems, which are fundamentally different from most historical examples of automation technologies \cite{noy2023experimental}. This valuable empirical inquiry occurs at an opportune time when varied expectations about this nascent technology abound \cite{lehtiniemi_contextual_2023}, allowing us to understand how SSPs incorporate AI into their practice, and what AI can (or cannot) do for them. In doing so, we aim to uncover new theoretical and practical insights on what AI can bring to the social service sector, and formulate design implications for developing AI technologies that SSPs find truly meaningful and useful.

\section{Study Design}
\label{sec:stage1design}


% \begin{table}
%   \caption{Workshop Participants from Agency A and Agency B}
%   \label{tab:workshopParticipants}
%   \begin{tabular}{cc|cc}
%     \toprule
%     \multicolumn{2}{c|}{\textbf{Agency A}}&\multicolumn{2}{c}{\textbf{Agency B}} \\
%     \midrule
%     Code&Role&Code&Role\\
%     \midrule
%     \multicolumn{2}{c|}{Group 1}& \multicolumn{2}{c}{Group 1} \\
%     \midrule
%     TS1 & Senior Social Worker & CD1 & Director\\
%     TS4 & Social Worker & CY1 & Youth Work Services\\
%     TS5 & Social Worker & CY2 & Youth Work Services\\ 
%     TS6 & Social Worker & CY3 & Youth Work Services\\
%     TS7 & Social Worker & CP1 & Youth Projects\\
%     && CP2 & Youth Projects\\
%     && CP3 & Youth Projects\\
%     \midrule
%     \multicolumn{2}{c|}{Group 2}&\multicolumn{2}{c}{Group 2} \\
%     \midrule
%     TD1 & Executive Director & CC1 & Counsellor\\
%     TD2 & Senior Director & CC2 & Counsellor\\
%     TS2 & Social Worker & CC3 & Counsellor\\
%     TS3 & Social Worker & CC4 & Counsellor\\
%     TP1 & Psychologist & CP4 & Youth Projects\\
%   \bottomrule
% \end{tabular}
% \end{table}

% \begin{table}
%   \caption{Other Participants from Agency A and Agency B}
%   \label{tab:otherParticipants}
%   \begin{tabular}{ccl}
%     \toprule
%     Code&Role\\
%     \midrule
%     \multicolumn{2}{c}{\textbf{Agency A}} \\
%     \midrule
%     TS8 & Social Worker\\
%     TS9 & Social Worker\\
%     TS10 & Social Worker\\
%     \midrule
%     \multicolumn{2}{c}{\textbf{Agency B}} \\
%     \midrule
%     CC5 & Counsellor\\
%   \bottomrule
% \end{tabular}
% \end{table}

% \begin{figure}
% \begin{minipage}{.5\textwidth}

% \end{minipage}
% \end{figure}

We engaged in a two-stage study: a formative, participatory design study to understand the opportunities and challenges perceived by our participants and to co-design a GenAI tool, followed by an evaluative contextual inquiry to assess the effectiveness of the resulting tool. We partnered with two local, government-funded social service agencies (SSAs) in a Southeast Asian country that had expressed interest in adopting GenAI. Agency A focused on family-oriented casework, handling mostly walk-in clients and taking them through the full process \cite{rooney2017direct} of exploration, assessment, implementation, and eventually termination. Agency B worked more with schools and youths, partnering with education institutes to render assistance to children or teenagers in need. Both agencies used English as a working language and for all official documentation. In a small proportion of cases, Agency A's client interactions took place in a different language that was more comfortable for the client; in these instances, workers would either record their notes in English or manually translate them back into English before taking them back to the agency for further work. All participant interactions in this study were also conducted in English.

% \subsection{Workshops}

In \textbf{stage 1, the co-design phase}, we conducted two 90-minute long workshops with agencies A and B in November 2023. A total of 27 SSPs were involved, hailing from different roles and experience levels\footnote{Pictures of the workshops are in Appendix \ref{appendix:workshops}.}. We aimed to understand 1) the nature of the day-to-day work that our participants performed and what opportunities they perceived for using AI to help with it, and 2) the perceived risks and challenges of AI use to shape the design of the second phase of our study. In each workshop, we briefly introduced LLMs in the form of ChatGPT\footnote{At the time of the workshops (October 2023), ChatGPT (GPT-3.5) was the most well-known LLM.}, then conducted brainstorming and sketching sessions in small groups of 4-6. Full details of the workshops are in Appendix \ref{appendix:workshops}.

Based on the opportunities and concerns identified in the workshops (see Section \ref{findings:workshop} or Appendix \ref{appendix:workshopFindings}), we created a prototype assistant tool (Fig. \ref{fig:participantsAndTool}, right). The tool comprised an input text box for users to enter details about their client and a number of different types of output options. Users could select an output option based on a desired use case, then click a "generate" button to produce an LLM-written\footnote{OpenAI GPT-4 Turbo, gpt-4-1025-preview.} response. These output options addressed various manual and mental labour difficulties in social work, and simulated potential uses addressing these difficulties. In response to complaints about manual work, the tool offered options to rewrite workers' rough notes into various organization-wide formats (e.g. BPSS, DIAP, Ecological, 5Ps). Users also raised points about assessments, case conceptualization, and ideation. We included options for strength, risk, and challenge assessments, following common social service practices \cite{rooney2017direct}. Finally, we added options to generate client intervention plans according to three common theoretical models: CBT, SFBT\footnote{CBT: Cognitive Behavioural Therapy; SFBT: Solution-Focused Brief Therapy.}, and Task-Centred Interventions.

In \textbf{stage 2, we conducted focus group discussions} (FGDs), where we simulated a contextual inquiry process \cite{holtzblatt2017contextual} by asking participants to walk us through their use of our tool and explain their thought processes along the way. We also encouraged them to point out weaknesses or flaws in the system and suggest potential improvements. We conducted the FGDs in groups of 2-4 SSPs from Agency A (the family service centre agency), with a mix of workers of varying seniority levels. We held a total of 8 sessions totalling 24 participants (Fig. \ref{fig:participantsAndTool}, left). Each session averaged 45-60 minutes in length and was audio-recorded for transcription and analysis. These sessions were conducted in May 2024.

In these sessions, we sought to understand the \textit{opportunities} of AI use in social work. Participants brought along different types of anonymised case files (e.g., rough short-hand notes, complete reports, intake files) from recent clients they had worked with. We explained the various functions of the system, then told participants to imagine themselves using it to help with the cases they had on hand, exploring a range of use cases from generating documentation reports to planning future sessions or interventions for the client. We also identified early on that case supervision by senior workers was a key means of addressing the difficulties faced by junior workers and encouraging worker development. We therefore sought to understand how AI could play a collaborative role in the supervision process. In sessions with supervisors present, we asked these more senior workers how they felt the various functions of the system could assist them in their discussion of cases with their supervisees. 
% We then discussed how the tool's outputs could help in areas such as In sessions with supervisors present,  We also learned early on that case supervision by senior workers was a key means of addressing the difficulties faced by junior workers and encouraging worker development. We therefore sought to understand how AI could play a collaborative role in the supervision process. , we asked these more senior workers how they felt the various functions of the system could aid them in their discussion of cases with their supervisees. 

We also aimed to understand the possible \textit{challenges} of using AI tools in social work. We asked participants to review the quality of the tool's outputs, compare them to their own, and identify areas where they would fall short of expectations or fail to be useful in their daily work. We note that while our platform is based on GPT-4, a general purpose model not tuned for social work analysis\footnote{Specialised AI systems for social work case management do exist (e.g. \cite{socialworkmagic2024, caseworthy2024}, but these are focused on user experience and do not provide greater insight than vanilla GPT; furthermore, \cite{socialworkmagic2024} advises users to take its assessments and suggestions as only a starting point, which aligns with how we position our tool.}, our aim was not to discover the specific weaknesses of GPT-4 or any LLM in particular. Rather, it was to understand what participants perceived to be good outputs from an AI system, and in the process understand how they might be affected by any possible shortcomings of such a system. % I actually feel the footnote here is pretty important at the first glimpse haha

Finally, we explored some of the longer-term effects of AI use, such as potential overreliance on the tool and the possibility of becoming overly trusting of the system's output. Senior workers in particular were asked about how they perceived junior workers new to the sector having such a system to help them with their daily work.

We performed qualitative thematic analysis \cite{braun2006using} on the transcripts, adopting a bottom-up, inductive approach to data coding. This process is detailed in Appendix \ref{appendix:analysis}, with the findings presented in the next section.


% \begin{table}[H]
% \centering
% \caption{Focus Group Participants from Agency A}
%   \label{tab:thkMayParticipants}
%   \begin{tabular}{cc|cc}
%     \toprule
%     \multicolumn{2}{c|}{\textbf{Centre 1}}&\multicolumn{2}{c|}{\textbf{Center 2}} \\
%     Code&Role&Code&Role\\
%     \midrule
%     W1 & Social Worker & S4 & Supervisor\\
%     W2 & Social Worker & S5 & Supervisor \\
%     W3 & Social Worker & W4 & Social Worker\\
%     \midrule
%     S1 & Supervisor \\
%     S2 & Supervisor \\
%     S3 & Supervisor \\
%     \midrule
%     &&&& S6 & Senior SW\\
%     &&&& W5 & Social Worker\\
%     \toprule
%     Code&Role&Code&Role\\
%     \multicolumn{2}{c|}{\textbf{Center 3}}&\multicolumn{2}{c|}{\textbf{Center 4}}\\
%      W9 & Social Worker & S8 & Senior SW\\
%      W10 & Social Worker & S9 & Senior SW \\
%      C1 & Snr Counsellor & S10 & Senior SW \\
%      \midrule
%      S7 & Senior SW & W11 & Social Worker \\
%      W7 & Social Worker & W12 & Social Worker\\
%      W8 & Social Worker & W13 & Social Worker \\
%     \toprule
%     Code&Role\\
%     \multicolumn{2}{c}{\textbf{Management}}\\
%     D1 & Director*\\
%     \multicolumn{2}{c}{\textit{*Observer in}}\\
%     \midrule
%   \bottomrule
%   \Description{Table describing Focus Group Participants from Agency A. There are 8 groups from 4 different centres, plus a director classified under "Management".}
% \end{tabular}
% \label{tab:workshopParticipants}
% \Description{Table describing workshop Participants from Agency A and Agency B. There are 13 participants from Agency A, mostly social workers, and 14 participants from Agency B, a mix of youth workers, youth project workers, and counsellors.}

% \caption{Workshop Participants from Agency A and Agency B}
% \end{table}


% \begin{minipage}{0.5\linewidth}
% \begin{table}[H]
% \centering
% \begin{tabular}{cc|cc}
%     \toprule
%     \multicolumn{2}{c|}{\textbf{Agency A}}&\multicolumn{2}{c}{\textbf{Agency B}} \\
%     \midrule
%     Code&Role&Code&Role\\
%     \midrule
%     TS1 & Snr Social Worker & CD1 & Director\\
%     TS2 & Social Worker & CY1 & Youth Work \\
%     TS3 & Social Worker & CY2 & Youth Work \\ 
%     TS4 & Social Worker & CY3 & Youth Work \\
%     TS5 & Social Worker & CP1 & Youth Projects\\
%     TS6 & Social Worker & CP2 & Youth Projects\\
%     TS7 & Social Worker & CP3 & Youth Projects\\
%     TS8 & Social Worker & CP4 & Youth Projects \\
%     TS9 & Social Worker & CC1 & Counsellor \\
%     TS10 & Social Worker & CC2 & Counsellor \\
%     TD1 & Exec. Director & CC3 & Counsellor\\
%     TD2 & Snr Director & CC4 & Counsellor\\
%     TP1 &  Psychologist & CC3 & Counsellor\\
%     && CC4 & Counsellor\\
% \bottomrule   
% \end{tabular}
% \label{tab:workshopParticipants}
% \Description{Table describing workshop Participants from Agency A and Agency B. There are 13 participants from Agency A, mostly social workers, and 14 participants from Agency B, a mix of youth workers, youth project workers, and counsellors.}

% \caption{Workshop Participants from Agency A and Agency B}
% \end{table}
% \end{minipage}
% \begin{minipage}[HT]{0.5\linewidth}
% \begin{figure}[H]
% \centering
% \includegraphics[width=3in]{images/prototypetool.png}
% \caption{Prototype AI Tool}
% \Description{A screenshot of our prototype AI tool. It has a text box for entry at the top, a list of radio buttons to select output modalities, a text box for extra instructions to the model a user might want to input, and a "generate" button.}
% \label{fig:tool}
% \end{figure}
% \end{minipage}

% \begin{table}
%   \caption{Focus Group Participants from Agency A}
%   \label{tab:thkMayParticipants}
%   \begin{tabular}{cc|cc|cc|cc|cc}
%     \toprule
%     \multicolumn{2}{c|}{\textbf{Centre 1}}&\multicolumn{2}{c|}{\textbf{Center 2}}&\multicolumn{2}{c|}{\textbf{Center 3}}&\multicolumn{2}{c|}{\textbf{Center 4}}&\multicolumn{2}{c}{\textbf{Management}}\\
%     \midrule
%     Code&Role&Code&Role&Code&Role&Code&Role&Code&Role\\
%     \midrule
%     W1 & Social Worker & S4 & Supervisor & W9 & Social Worker & S8 & Senior SW & D1 & Director*\\
%     W2 & Social Worker & S5 & Supervisor & W10 & Social Worker & S9 & Senior SW & \multicolumn{2}{c}{\textit{*Observer in}}\\
%     W3 & Social Worker & W4 & Social Worker & C1 & Snr Counsellor & S10 & Senior SW & \multicolumn{2}{c}{\textit{multiple sessions}}\\
%     \midrule
%     S1 & Supervisor &&& S7 & Senior SW & W11 & Social Worker\\
%     S2 & Supervisor &&& W7 & Social Worker & W12 & Social Worker\\
%     S3 & Supervisor &&& W8 & Social Worker & W13 & Social Worker\\
%     \midrule
%     &&&& S6 & Senior SW\\
%     &&&& W5 & Social Worker\\
%   \bottomrule
%   \Description{Table describing Focus Group Participants from Agency A. There are 8 groups from 4 different centres, plus a director classified under "Management".}
% \end{tabular}
% \end{table}




 

% An important consideration was ensuring the generalisability of our findings, making sure our system and subsequent discussions adequately represented the vast possibilities of AI use cases, and not being limited to exploring a particular LLM in a specific configuration. We address this by noting the difference between areas that other or future AI systems may improve in, such as reasoning and knowledge, and those that they are unlikely to, such as being able to account for a user's (or client's) contextual factors. It is the latter that we focus our investigation on: Given an AI system is ultimately constrained by its training dataset, what are the possibilities with such systems and what are the potential downsides?

% 1) data privacy; 2) overreliance; 3) inaccurate output (participant identifiers are missed for 3 (?)


% This doubtless was influenced by our intial presentation on GenAI capabilities, where we showed an example of an LLM summarising and reformatting case notes. 

% \subsection{Research Questions and Testing Goals}

% From the above, we derive the following research questions.

% First, while many workers suggested they would be interested in AI automation tools, care needs to be taken to design such tools in a manner that is useful to them. Developing a tool that does not fit in their workflow or give outputs that are what they are looking for is pointless.


% \textbf{\textit{RQ1}}: What kind of AI tools do social workers find useful in their daily lives? How can they use these in their daily work?


% Besides non-useful outputs (possibly a system designer flaw), AI tools may also make more micro-level mistakes, such as by making topically relevant, yet incorrect or suboptimal suggestions. The ability of LLMs and other AI systems to handle social work-related queries is, to our knowledge, almost entirely unexplored.


% \textbf{\textit{RQ2}}: How accurately can modern AI tools handle these tasks?


% Given a tool that provides relevant, high-quality aid to social workers, the question shifts to how these tools should be most effectively implemented in and throughout an agency, maximising benefits while minimising harm.


% \textbf{\textit{RQ3}}: How do staff in different positions in the organisation use or think about the tool differently? What different user requirements might they have?


% \textbf{\textit{RQ4}}: How should staff in different positions in the organisation use or be given access to the tool differently? How can the tool serve to complement while not replacing human social workers?

% Due to the iterative nature of our research process, these later sessions served to both further ideate and critique potential solutions, and also test the concrete prototypes developed at that stage. We include the former as part of our discussion here, and leave the latter for a later part of the paper.

% We start by summarising some common difficulties that plague social workers in their work. We then present how participants felt that AI could help address these issues. Finally, we discuss the potential for AI to negatively affect organisational stability in a social work agency, and ways to begin understanding this complex issue.


\section{Findings}
\label{sec:stage2findings}

\subsection{Findings from Stage 1: Workshop and Co-Design}
\label{findings:workshop}
% Based on the codebook and themes, a final framework was developed, summarizing the Values, Requirements, and Attitudes that social workers have towards the application of AI tools in the social work practice.

Through the workshops, participants revealed a few major aspects of their work that could be assisted by AI. For brevity, detailed findings are in Appendix \ref{appendix:workshopFindings}. In brief, the main findings are below.

Documentation was a major pain point, with our participants needing to document "anything and everything", writing systematic reports in different structured formats (e.g. bio-psychological scales, risk factor assessments) or repacking the same content for different stakeholders like colleagues or other agencies, creating tedious duplicate work. Participants hence expressed a desire for a tool to help with manual labour, like turning point-form notes into formal reports, in various formats such as the 5Ps, DIAP, or BPSS\footnote{5Ps: Presenting problem, Predisposing factors, Precipitating factors, Perpetuating factors, Protective factors; DIAP: Data, Intervention, Assessment, Plan; BPSS: Bio-Psycho-Social-Spiritual}. Some senior personnel also suggested that AI could help workers incorporate theoretical concepts into their written work. 

% \begin{minipage}{0.5\linewidth}
% \begin{table}[H]
% \centering
% \label{tab:agencyA}

% % --- Row 1: Centres 1 & 2 ---
% \begin{tabular}{ll|ll}
% \toprule
% \multicolumn{2}{c|}{\textbf{Centre 1}} & \multicolumn{2}{c}{\textbf{Centre 2}} \\
% \midrule
% Code & Role        & Code & Role        \\
% \midrule
% W1   & Social Worker & W9   & Social Worker  \\
% W2   & Social Worker & W10  & Social Worker  \\
% W3   & Social Worker & C1   & Snr Counsellor \\
% \midrule
% S1   & Supervisor    & S7   & Senior SW      \\
% S2   & Supervisor    & W7   & Social Worker  \\
% S3   & Supervisor    & W8   & Social Worker  \\
% \midrule
%       &              & S6   & Senior SW      \\
%       &              & W5   & Social Worker  \\
% \bottomrule
% \multicolumn{2}{c|}{\textbf{Centre 3}} & \multicolumn{2}{c}{\textbf{Centre 4}} \\
% \midrule
% Code & Role        & Code & Role         \\
% \midrule
% S4   & Supervisor  & S8   & Senior SW    \\
% S5   & Supervisor  & S9   & Senior SW    \\
% W4   & Social Worker & S10  & Senior SW   \\
% \midrule
%       &             & W11  & Social Worker \\
%       &             & W12  & Social Worker \\
%       &             & W13  & Social Worker \\
% \bottomrule
% \end{tabular}

% \caption{Focus Group Participants from Agency A}
% \end{table}
% \end{minipage}
% \begin{minipage}[HT]{0.5\linewidth}
% \begin{figure}[H]
% \centering
% \includegraphics[width=2.15in]{images/prototypetool.png}
% \caption{Prototype AI Tool}
% \Description{A screenshot of our prototype AI tool. It has a text box for entry at the top, a list of radio buttons to select output modalities, a text box for extra instructions to the model a user might want to input, and a "generate" button.}
% \label{fig:participantsAndTool}
% \end{figure}
% \end{minipage}

\begin{table}
\centering
\label{tab:agencyA}

% --- Row 1: Centres 1 & 2 ---
\begin{tabular}{ll|ll}
\toprule
\multicolumn{2}{c|}{\textbf{Centre 1}} & \multicolumn{2}{c}{\textbf{Centre 2}} \\
\midrule
Code & Role        & Code & Role        \\
\midrule
W1   & Social Worker & W9   & Social Worker  \\
W2   & Social Worker & W10  & Social Worker  \\
W3   & Social Worker & C1   & Snr Counsellor \\
\midrule
S1   & Supervisor    & S7   & Senior SW      \\
S2   & Supervisor    & W7   & Social Worker  \\
S3   & Supervisor    & W8   & Social Worker  \\
\midrule
      &              & S6   & Senior SW      \\
      &              & W5   & Social Worker  \\
\bottomrule
\multicolumn{2}{c|}{\textbf{Centre 3}} & \multicolumn{2}{c}{\textbf{Centre 4}} \\
\midrule
Code & Role        & Code & Role         \\
\midrule
S4   & Supervisor  & S8   & Senior SW    \\
S5   & Supervisor  & S9   & Senior SW    \\
W4   & Social Worker & S10  & Senior SW   \\
\midrule
      &             & W11  & Social Worker \\
      &             & W12  & Social Worker \\
      &             & W13  & Social Worker \\
\bottomrule
\end{tabular}

\caption{Focus Group Participants from Agency A}
\end{table}

\begin{figure}
\centering
\includegraphics[width=2.15in]{images/prototypetool.png}
\caption{Prototype AI Tool}
\Description{A screenshot of our prototype AI tool. It has a text box for entry at the top, a list of radio buttons to select output modalities, a text box for extra instructions to the model a user might want to input, and a "generate" button.}
\label{fig:participantsAndTool}
\end{figure}



Participants also noted challenges with case formulation, in which piecing together disparate information across many pages of case notes to craft and justify an assessment is cognitively challenging and time-consuming, particularly during direct interaction with a client. This may inadvertently cause them to miss certain key insights or red flags only evident from looking at the gathered information as a whole. Participants therefore also suggested systems for generating case assessments, to improve output quality and adherence to industry-standard terms, and intervention planning, to generate possible plans for helping the client that the practitioner could consider and choose from.

% Another critical task is intervention planning, where workers follow established models like Cognitive Behaviour Therapy (CBT) or Solution-Focused Brief Therapy (SFBT) and craft a plan for their clients based on the guiding steps in each model. An AI tool could allow rapid generation of numerous possible interventions (CP1), leaving the user to pick and choose from the suggestions offered (TD1).

% This requires fitting a client's information into theoretical frameworks to produce well-grounded assessments and analyses. AI tools could improve both the speed at which relevant information is synthesised and categorised, and the quality of the output through use of technical, industry-standard terms, something that is often lacking in many SSPs

% In line with past work on the dangers of AI adoption [cite], participants echoed certain concerns about the use of AI tools.

Finally, participants also noted a few potential areas of risk. Privacy of client data was unilaterally mentioned by all, a universal concern core to the social sector. The storage of personal client information was a significant concern, particularly regarding the possibility of workers mistakenly entering sensitive information into the system. Regarding staff competency, a common concern was the impact of AI taking on an increasing part of the worker's job scopes. Specifically for analytical or ideative tasks, some seniors were concerned about the loss of critical thinking skills of junior workers who might become overreliant on the tool to perform their work for them. Multiple participants also raised the possibility of inaccurate outputs from an AI system, particularly risky when less experienced workers fail to tell when the AI's output might be suboptimal and proceed to adopt its suggestions anyway. 

\subsection{Findings from Stage 2: Focus Group Discussions}

\subsubsection{Documentation} 
\label{subsubsec:discussionuses}

Participants found many applications of GenAI in helping with multiple writing-focused tasks in the social service sector, such as summarising intake information, formatting case recordings, and writing reports. Participants generally were happy to embrace AI for such purposes; for documentation tasks such as writing case reports, the tool's outputs were largely in line with what they required, allowing them to simply "copy and paste" (W1, W2, W13) the outputs for direct use. This was, in a large way, down to how much of our participants' regular daily work focused on consistently structured, fixed-format reports. One strength evident in GPT-4 and our prototype was its ability to consistently follow instructions to produce outputs in a desired format, such as the 5Ps format (S1, S3, W1). W8, for instance, quoted, "\textit{being new, it really helps in categorising these items... I like the fact that it segregates all the [different categories]}".

Even when the output was imperfect, participants often expressed a willingness to work around these errors and make manual corrections where needed. For instance, W2 suggested they would "sift through and pick out" the more relevant parts of an overly lengthy report, while W1 would "\textit{copy and paste, then amend if I need to amend}" when the tool misinterpreted a nickname given by the worker to the client.


% S9: "I like how it is to have a sub-titles (sub-headings) and things like that"

\subsubsection{Brainstorming}

Workers felt that the structured and detailed outputs of the tool encouraged and facilitated their cognitive processes in analysing a case. W8 felt the way the tool categorised the issues in their client's case pushed them to "think more" about how they viewed it. In a similar vein, looking at a generated CBT assessment prompted W5 to consider "\textit{certain things also that I should probably look into}", which they might otherwise have overlooked. W11 commented on how the "very in-depth" explanations given by the tool helped them "expand on what they already have". Even a senior worker, S4, called the tool's analysis of a case "\textit{really helpful - it's expanding my perspective already}".

Junior workers in particular appreciated the guidance from the tool, especially with tasks they were less familiar with. For instance, W7 had not attended formal CBT training, and thus felt the tool gave them "some idea where to start" in formulating a CBT intervention plan for their client. W4 also liked having sample interventions from the tool; being a newer worker, they were uncertain of which plan of action to take, so the tool gave them a "better understanding" of the different ways to move forward with the case. Meanwhile, more generally, the tool helped with guiding workers towards formulating a course of action, such as by suggesting interventions they might not have thought of (W4) and thereby prompting newer staff with a "direction" to work towards (W5), or by being "very useful" in helping new workers prepare for sessions with clients (W9).

Supervisors, too, agreed that the tool was useful for junior workers. S4 reiterated that it could "expand the worker's perspective", and C1 called it "a good start [to help] staff think about" the case if they "got stuck" with something. S6 cited the example of how an SFBT output provided a "really good foundation" for questions for workers to ask their clients. S6 also felt the tool provided a good framework and guideline, suggesting it as a way to "polish" newer workers' skills: "\textit{For all those who are really new and do not really know how to formulate interventions or theory support and all that. I think it's quite useful... or, if they are really lost, then they can probably try the different things that are written here}." 

\subsubsection{Supervision}

The use of AI to aid in supervision emerged as a key theme. Supervision sessions consist of junior workers discussing cases with their supervisors, to refine and improve their assessment of the client. Given this, the ability of AI to quickly generate lists of ideas provided useful starting points for discussion and reflection with supervisees (C1). Many (W4, W5) suggested that the tool provided useful intervention suggestions so that workers could "ask their supervisors like, maybe, you know, maybe I can try this" (S6). From the supervisors' perspectives, the tool helped to improve and expedite the supervision process by prompting them with questions to ask supervisees: "I can bring [this list of exception questions\footnote{Asking "exception questions" to clients is a technique used in Solution-Focused Brief Therapy.}] to supervision to see whether my supervisee has used these questions... I can ask my supervisee, okay, if you have to ask this exception question to the client, how comfortable do you feel? So we can have that discussion" (C1). S4 and S7, meanwhile, highlighted the AI's ability to quickly "concretise theoretical models" to build on during sessions. 

Some even suggested that the tool could itself serve as a supervisor to help newer workers. W9 called the tool a "readily-available supervisor to get us thinking," referencing how their supervisors frequently prompted them with questions to think about their case more. W4 meanwhile felt the tool could help them move forward with a case "without having to consult with their supervisor", in instances where they were uncertain of how to proceed.
% One unexpected use of AI that emerged prominently was as a supervisory aid. Our participants mainly described supervision sessions as senior and junior workers discussing how to formulate and proceed with a case further. With this, the suggested interventions or possible solutions generated by our AI system served as good discussion points. 


\subsubsection{Concerns and Issues}

Participants raised a few issues with using AI in their work. Agency A's focus on providing family service meant that they prioritised addressing safety and risk concerns (W2), such as possible self-harm, suicide, or harm towards others. Participants were "very particular about... risks" (S9), treating it as their top priority (W1, S10) and "at the front" for all client interventions and risk assessments (S2, S9). Preventing imminent physical and psychological harms such as incarceration, abuse, and addiction were therefore cited as "non-negotiables" (W4) and primary risk factors (W2). Thus, they expressed concern when the tool "didn't exactly highlight" (W2) or entirely omitted (W1) safety risks in its output, such as in assessing a case of intra-family conflict: "\textit{The [risk of] violence is not highlighted. Where is the violence?}" (W1).

There were also worries over overuse and overreliance on AI. Participants were divided on this issue. The centre director (D1) quoted, "\textit{One of our concerns is... [will using this AI] actually disable our ability to make assessments?}" S3 agreed about the risk of over-reliance by "spoonfeeding" junior workers, and S9 noted the need for workers to "\textit{still use their brain... otherwise... they just rely on this}." S9 also worried about how junior workers might fail to recognise suboptimal AI outputs and be misled as a result. As a result, some emphasised the need for a balance between AI use and human intervention. S10 remarked, \textit{"It's the supervisor's role to keep emphasising to the supervisee, to not be married to this assessment"}, and that the AI's outputs were often "just guidelines" rather than a gospel to be followed. S9 agreed that "\textit{the expectation [for use] needs to be very clearly communicated}", highlighting the need for careful and judicious implementation of any AI system. 

However, others felt this to not be a major problem, due to the inherent focus of the profession on face-to-face client interactions. On practitioners blindly following AI recommendations, S7 commented, "\textit{it will eventually be clear that this was the recommendation of [the tool] but then you went and did something else... and then you'll see that it's just a mess in that way.}" S3 agreed: "\textit{When they do their work, it is mostly real-time. You don't go back to pen and paper and start to develop [a solution], but it's more about what is presented to them immediately [and how they react].}"

Finally, there were multiple instances where participants found the AI's output to be inadequate. These often centred around the tool failing to recognise more subtle factors in a given case that would affect the preferred course of action. For example, when the system recommended a family therapy workshop, W5 recognised that the reluctance of their client to embrace outside help made such an intervention unsuitable. In another case, when the tool recommended CBT to aid a client, W11 judged that their client lacked the intellectual capacity for such therapy to help. Knowledge of local contextual factors was another common point. S6, for instance, commented on how "\textit{there are a few parenting workshops available [here in our country] and [we know] how they are structured}," so they would be able to assess which options were suitable for their client in a way that the AI, lacking local knowledge, would not. W5 also commented on the importance of practitioners' instincts and experience: \textit{"Many of us who have been in the sector before, probably were like, this won't work."}
\section{Discussion}
\label{section:discussion}


\subsection{Practical Implications for Feedforward Prompting}

Of course, prompting an LLM continuously before the user submits their prompt is significantly most costly over submitting the prompt just once, once the user is ready.

% But user might not be ready, and the cognitive costs is pretty heavy.


\subsection{}


% Does this work well with Chain of Thought actually?
% Maybe this approach will actually incentivize self-prompt-chaining???
% What are the implications of this?


% A benefit of this is certainly more transparency in the LLM
% LLM is so flexible that adding this kind of structure is still okay for the LLM



% What's more costly, entering a prompt, then responding and saying, no i want this, or typing a prompt, and tuning the prompt/expected output to reduce message exchanges?

% Learning to become a better prompter. One is by trial and error experience. Perhaps another is through this feedforward that tells you what you might be able to anticipate.

%%
%% The acknowledgments section is defined using the "acks" environment
%% (and NOT an unnumbered section). This ensures the proper
%% identification of the section in the article metadata, and the
%% consistent spelling of the heading.
\begin{acks}
The authors would like to thank the directors and workers at Agency A for our close working relationship over the past 18 months. 

We also acknowledge the use of GPT-4o for generating ideas for the \textit{paper title}, giving suggestions for the \textit{abstract} after the rest of the paper was completed, and \textit{shortening} parts of the paper after they were written.

This work was supported by the National University of Singapore's Centre for Computational Social Science and Humanities (CSSH), under fund number A-8002954-01-00.
\end{acks}

%%
%% The next two lines define the bibliography style to be used, and
%% the bibliography file.
\bibliographystyle{ACM-Reference-Format}
\bibliography{references}

%%
%% If your work has an appendix, this is the place to put it.
\appendix

\section{Metric}
\label{sec:metric}

\textbf{Mean Squared Error (MSE)} Mean Squared Error (MSE) is a common statistical metric used to assess the difference between predicted and actual values. The formula is:
\begin{equation}
    MSE = \frac{1}{n} \sum_{i=1}^{n} (y_i - \hat{y}_i)^2
\end{equation}
where $ n $ is the number of samples, $ y_i $ is the actual value, and $ \hat{y}_i $ is the predicted value.

\textbf{Relative L2 Error} Relative L2 error measures the relative difference between predicted and actual values, commonly used in time series prediction. The formula is:
\begin{equation}
    \text{Relative L2 Error} = \frac{\| Y_{\text{pred}} - Y_{\text{true}} \|_2}{\| Y_{\text{true}} \|_2}
\end{equation}
where $ Y_{\text{pred}} $ is the predicted value and $ Y_{\text{true}} $ is the actual value.

\textbf{Structural Similarity Index Measure (SSIM)} The Structural Similarity Index (SSIM) measures the similarity between two images in terms of luminance, contrast, and structure. The formula is:
\begin{equation}
    SSIM(x, y) = \frac{(2\mu_x \mu_y + C_1)(2\sigma_{xy} + C_2)}{(\mu_x^2 + \mu_y^2 + C_1)(\sigma_x^2 + \sigma_y^2 + C_2)}
\end{equation}
where $ \mu_x $ and $ \mu_y $ are the mean values, $ \sigma_x $ and $ \sigma_y $ are the standard deviations, $ \sigma_{xy} $ is the covariance.

\section{Related Work}
\subsection{Deep Learning based Weather Forecasting}
\textbf{Global Weather Forecasting.} Global weather forecasting has seen significant progress with deep learning models. FourCastNet, based on Fourier neural operators, provides global forecasts comparable to traditional numerical methods like IFS, but at much higher speeds~\cite{pathak2022fourcastnet}. Pangu, utilizing the Swin Transformer, exceeds NWP methods, incorporating earth-specific location embeddings for better performance~\cite{bi2023accurate}. The Spherical Fourier Neural Operator (SFNO) extends Fourier methods using spherical harmonics, offering more stable long-term predictions~\cite{bonev2023spherical}. FuXi focuses on long-term forecasting, achieving a 15-day forecasts comparable to ECMWF~\cite{chen2023fuxi}. GraphCast leverages message-passing networks to improve efficiency and forecasting accuracy~\cite{lam2023learning}, and GenCast builds on this to enhance ensemble forecasting~\cite{price2023gencast}. Further, diffusion models like those in~\cite{li2024generative} generate probabilistic ensembles by sampling, while NeuralGCM~\cite{kochkov2024neural} focuses on atmospheric circulation with a dynamic core, offering climate simulation capabilities but at higher training and inference costs. 

\textbf{Regional Weather Forecasting.} The goal of regional weather forecasting is to enhance local prediction accuracy with high-resolution models. CorrDiff~\cite{mardani2023generative} combines U-Net and diffusion models to improve local forecasts. MetaWeather~\cite{kim2024metaweather} adapts global forecasts to regional contexts using meta-learning. GNNs are also widely applied in regional forecasting, with Graphcast~\cite{lam2023learning} enhancing accuracy by modeling complex spatial dependencies. MetNet-3~\cite{espeholt2022deep} offers high-accuracy forecasts for weather variables, such as precipitation, temperature, and wind speed, at 2-minute intervals and 1–4 km resolution, outperforming traditional models like HRRR. NowcastNet~\cite{zhang2023skilful} and DGMR~\cite{ravuri2021skilful} excel in short-term extreme precipitation forecasts using deep generative models and radar data. In spatiotemporal prediction, NMO~\cite{wu2024neural} models the evolution of physical dynamics, providing new insights for local weather forecasting. Similarly, SimVP~\cite{gao2022simvp} and PastNet~\cite{wu2024pastnet} achieve good results in forecasting local precipitation evolution using spatiotemporal convolution methods.
    
% Despite these advances, none of these methods effectively address the challenge of balancing global and regional high-resolution forecasts or capturing the fine-grained, dynamic interactions important for extreme event prediction.
    
\subsection{Numerical analysis methods}
Multigrid methods~\cite{mccormick1987multigrid,wesseling1995introduction,hackbusch2013multi,bramble2019multigrid,hiptmair1998multigrid,brandt1983multigrid,borzi2009multigrid} and nested grid strategies~\cite{miyakoda1977one,zhang2012nested,sullivan1996grid} are widely used to solve PDEs and handle multi-scale problems~\cite{debreu2008two,xue2000advanced}. Multigrid methods use grids of different resolutions to transfer information and accelerate iterations. They efficiently solve large-scale problems and improve computational accuracy. By eliminating low-frequency errors on coarse grids and high-frequency errors on fine grids, multigrid methods effectively handle error convergence at different scales~\cite{he2019mgnet,he2023mgno,shao2022fast}. Nested grid strategies embed higher-resolution fine grids into regions of interest based on a global coarse grid to capture local complex physical phenomena in detail. In weather forecasting, this method provides large-scale background fields on a global scale while refining the grid for target regions to accurately simulate the evolution of local weather systems and the occurrence of extreme events~\cite{bacon2000dynamically}. 

% Our proposed neural nested grid method helps address challenges like boundary information loss in regional forecasting and multi-scale feature capture.

\section{Additional Results}
%
We present more additional results in Figure \ref{fig_0.25-day}, \ref{fig_0.5-day}, \ref{fig_1.0-day} \ref{fig_1.5-day}, \ref{fig_2.0-day}, \ref{fig_2.5-day}, \ref{fig_3.0-day}, \ref{fig_3.5-day}, \ref{fig_4.0-day}, \ref{fig_4.5-day}, \ref{fig_5.0-day}, \ref{fig_5.5-day}, \ref{fig_6.0-day}, \ref{fig_6.5-day}, \ref{fig_7.0-day}, \ref{fig_7.5-day},
\ref{fig_8.0-day}, \ref{fig_8.5-day}, \ref{fig_9.0-day}, \ref{fig_9.5-day},
\ref{fig_10.0-day}, including 18 variables that are importmant to weather forecasting, each with results ranging from 6 hours to 10 days. These additional results further demonstrate the effectiveness of OneForecast. Same as the Figure \ref{fig:visual_results}
, the initial conditions is 00:00 UTC, 1 January 2020.


\begin{figure*}[h]
\centering
\includegraphics[width=1\linewidth]{figures/fig_0.25-day.jpg}
\vspace{-20pt}
\caption{6-hour forecast results of different models.}
\label{fig_0.25-day}
\end{figure*}

\begin{figure*}[h]
\centering
\includegraphics[width=1\linewidth]{figures/fig_0.5-day.jpg}
\vspace{-20pt}
\caption{0.5-day forecast results of different models.}
\label{fig_0.5-day}
\end{figure*}

\begin{figure*}[h]
\centering
\includegraphics[width=1\linewidth]{figures/fig_1.0-day.jpg}
\vspace{-20pt}
\caption{1-day forecast results of different models.}
\label{fig_1.0-day}
\end{figure*}

\begin{figure*}[h]
\centering
\includegraphics[width=1\linewidth]{figures/fig_1.5-day.jpg}
\vspace{-20pt}
\caption{1.5-day forecast results of different models.}
\label{fig_1.5-day}
\end{figure*}

\begin{figure*}[h]
\centering
\includegraphics[width=1\linewidth]{figures/fig_2.0-day.jpg}
\vspace{-20pt}
\caption{2-day forecast results of different models.}
\label{fig_2.0-day}
\end{figure*}


\begin{figure*}[h]
\centering
\includegraphics[width=1\linewidth]{figures/fig_2.5-day.jpg}
\vspace{-20pt}
\caption{2.5-day forecast results of different models.}
\label{fig_2.5-day}
\end{figure*}

\begin{figure*}[h]
\centering
\includegraphics[width=1\linewidth]{figures/fig_3.0-day.jpg}
\vspace{-20pt}
\caption{3-day forecast results of different models.}
\label{fig_3.0-day}
\end{figure*}

\begin{figure*}[h]
\centering
\includegraphics[width=1\linewidth]{figures/fig_3.5-day.jpg}
\vspace{-20pt}
\caption{3.5-day forecast results of different models.}
\label{fig_3.5-day}
\end{figure*}

\begin{figure*}[h]
\centering
\includegraphics[width=1\linewidth]{figures/fig_4.0-day.jpg}
\vspace{-20pt}
\caption{4-day forecast results of different models.}
\label{fig_4.0-day}
\end{figure*}

\begin{figure*}[h]
\centering
\includegraphics[width=1\linewidth]{figures/fig_4.5-day.jpg}
\vspace{-20pt}
\caption{4.5-day forecast results of different models.}
\label{fig_4.5-day}
\end{figure*}


\begin{figure*}[h]
\centering
\includegraphics[width=1\linewidth]{figures/fig_5.0-day.jpg}
\vspace{-20pt}
\caption{5.0-day forecast results of different models.}
\label{fig_5.0-day}
\end{figure*}

\begin{figure*}[h]
\centering
\includegraphics[width=1\linewidth]{figures/fig_5.5-day.jpg}
\vspace{-20pt}
\caption{5.5-day forecast results of different models.}
\label{fig_5.5-day}
\end{figure*}

\begin{figure*}[h]
\centering
\includegraphics[width=1\linewidth]{figures/fig_6.0-day.jpg}
\vspace{-20pt}
\caption{6.0-day forecast results of different models.}
\label{fig_6.0-day}
\end{figure*}

\begin{figure*}[h]
\centering
\includegraphics[width=1\linewidth]{figures/fig_6.5-day.jpg}
\vspace{-20pt}
\caption{6.5-day forecast results of different models.}
\label{fig_6.5-day}
\end{figure*}

\begin{figure*}[h]
\centering
\includegraphics[width=1\linewidth]{figures/fig_7.0-day.jpg}
\vspace{-20pt}
\caption{7.0-day forecast results of different models.}
\label{fig_7.0-day}
\end{figure*}

\begin{figure*}[h]
\centering
\includegraphics[width=1\linewidth]{figures/fig_7.5-day.jpg}
\vspace{-20pt}
\caption{7.5-day forecast results of different models.}
\label{fig_7.5-day}
\end{figure*}

\begin{figure*}[h]
\centering
\includegraphics[width=1\linewidth]{figures/fig_8.0-day.jpg}
\vspace{-20pt}
\caption{8.0-day forecast results of different models.}
\label{fig_8.0-day}
\end{figure*}

\begin{figure*}[h]
\centering
\includegraphics[width=1\linewidth]{figures/fig_8.5-day.jpg}
\vspace{-20pt}
\caption{8.5-day forecast results of different models.}
\label{fig_8.5-day}
\end{figure*}

\begin{figure*}[h]
\centering
\includegraphics[width=1\linewidth]{figures/fig_9.0-day.jpg}
\vspace{-20pt}
\caption{9.0-day forecast results of different models.}
\label{fig_9.0-day}
\end{figure*}

\begin{figure*}[h]
\centering
\includegraphics[width=1\linewidth]{figures/fig_9.5-day.jpg}
\vspace{-20pt}
\caption{9.5-day forecast results of different models.}
\label{fig_9.5-day}
\end{figure*}

\begin{figure*}[h]
\centering
\includegraphics[width=1\linewidth]{figures/fig_10.0-day.jpg}
\vspace{-20pt}
\caption{10.0-day forecast results of different models.}
\label{fig_10.0-day}
\end{figure*}


\section{Detailed Mathematical Proof}
\label{sec:proof}
\textbf{Proof of Theorem 1}

Now we have N augmented data and we need to select the best from them. We consider both the quality and the diversity of these data and get the sampling strategy from an optimization problem.

We model the sampling strategy as a multinomial distribution supported on all the augmented data $S = \{\mathbf{X}_j\}_{j=1}^N$, which means that the sampling strategy $\pi=(\pi_1,...,\pi_N)^\top$ is the corresponding probabilities of selecting $\mathbf{X}_1,...,\mathbf{X}_N$, then we can model the expectation of the similarity as:
$$\begin{aligned}
 & \mathbb{E}_{Y_x,Y_{x^{\prime}}\in\mathcal{C}}\{g(x,x^{\prime})\mid S\} \\
 & =\quad\int g(\mathbf{x},\mathbf{x}^{\prime})\boldsymbol{\pi}(\mathbf{x})\mathrm{Pr}_{S}(Y_{x}\in\mathcal{C}\mid\boldsymbol{x}=\mathbf{x})\boldsymbol{\pi}(\mathbf{x}^{\prime})\mathrm{Pr}_{S}(Y_{x}\in\mathcal{C}\mid\boldsymbol{x}=\mathbf{x}^{\prime})d\mathbf{x}d\mathbf{x}^{\prime} \\
 & =\quad\sum_{i,j=1}^Ng(\mathbf{X}_i,\mathbf{X}_j)\pi_i\pi_j\mathrm{Pr}_{S}(Y_x\in\mathcal{C}\mid\boldsymbol{x}=\mathbf{X}_i)\mathrm{Pr}_{S}(Y_x\in\mathcal{C}\mid\boldsymbol{x}=\mathbf{X}_j),
\end{aligned}$$
where the set $\mathcal{C}$ denotes the criterion of selection we are using, the function $g$ can be chosen as any similarity metric function and $x$ means a random variable.

The core to solving the above optimization problem is to use predictive inference to approximate the conditional probability of $\{Y_x\in\mathcal{C}\}$ given $x = \mathbf{X}$
Let $\mu ( \mathbf{x} ) : = \mathbb{E} ( Y\mid \mathbf{X} = \mathbf{x} )$ be the oracle associated with $( \mathbf{X} , Y) .$ Denote $\theta_j=\mathbb{I}\{Y_j\in\mathcal{C}\}$. As the augmented data
$\mathbf{X}_1,...,\mathbf{X}_N$ are independently identically distributed, $\theta_1,...,\theta_N$ can be regarded as independent Bernoulli($q)$ variables with $q=\Pr(Y_j\in\mathcal{C}).$ The probability distribution of the predicted result $W_j$ for $j=1,...,N$ is
$$\Pr(W_j\mid\theta_j)=(1-\theta_j)f_0+\theta_jf_1,\quad$$
where $f_0$ and $f_1$ are the conditional distributions of $W_j$ on $Y_j \in \mathcal{C}$ or not.

Denote $T(w) = \frac{(1-q)f_0(W_j)}{f(W_j)}$, we can rewrite the expectation of the similarity as
$$\mathbb{E}_{Y_x,Y_{x^{\prime}}\in\mathcal{C}}\{g(x,x^{\prime})|S\}=\sum_{i,j=1}^Ng(\mathbf{X}_i,\mathbf{X}_j)\pi_i\pi_j(1-T_i)(1-T_j)=\boldsymbol{\pi}^\top A_\mathbb{T}\boldsymbol{\pi},$$

Next, we use the expectation to control the quality of the data.
$$\mathbb{E}\{\mathbb{I}(Y_x\not\in\mathcal{C})\mid S\}=\sum_{i=1}^N\Pr(Y_i\not\in\mathcal{C}\mid\mathbf{X}_i)\pi_i=\sum_{i=1}^N\pi_iT_i\leq\alpha,$$

In all, the optimization problem can be modeled as 
\begin{align}
    & \arg\min_{\boldsymbol{\pi}}\quad h(\boldsymbol{\pi},\mathbb{T}):=\boldsymbol{\pi}^\top A_\mathbb{T}\boldsymbol{\pi}, \\
    & \text{subject to} \quad
        \begin{cases}
            \sum_{i = 1}^N\pi_iT_i\leq\alpha, \\
            \sum_{i = 1}^N\pi_i = 1, \\
            0\leq\pi_i\leq m^{-1}, \quad 1\leq i\leq N.
        \end{cases}
\end{align}

where $m$ is used to control the maximum selection.

The best selection of K is determined by the strategy $\pi$ which serves as the solution to the above optimization problem.

\section{Additional Experiments}
\label{sec:more_experiments}
\subsection{Long-term forecasting experiment expansion}

In the long-term forecasting experiments, we compare the performance of different backbone models on the SWE benchmark, evaluating the relative L2 error for three variables (U, V, and H). Our setup inputs 5 frames and predicts 50 frames. For the SimVP-v2 model, using \method{} reduces the relative L2 error for SWE (u) from 0.0187 to 0.0154, SWE (v) from 0.0387 to 0.0342, and SWE (h) from 0.0443 to 0.0397. We visualize SWE (h) in 3D as shown in Figure~\ref{fig:case} [\textcolor{red}{I}]. For the ConvLSTM model, applying \method{} reduces the relative L2 error for SWE (u) from 0.0487 to 0.0321, SWE (v) from 0.0673 to 0.0351, and SWE (h) from 0.0762 to 0.0432. For the FNO model, using \method{} reduces the relative L2 error for SWE (u) from 0.0571 to 0.0502, SWE (v) from 0.0832 to 0.0653, and SWE (h) from 0.0981 to 0.0911. Overall, \method{} significantly improves the long-term forecasting accuracy of different backbone models.

\begin{figure*}[h]
    \centering
    \includegraphics[width=\textwidth]{image/casestudy.pdf}
    \caption{
    \textcolor{red}{I.} 3D visualization of the SWE(h), showing Ground-truth, SimVP-V2+BeamVQ predictions, and Error at T=1, 10, 20, 30, 40, 50. The first row shows Ground-truth, the second SimVP-V2+BeamVQ predictions, and the third Error. \textcolor{red}{II.} A case study. Building fire simulation with ventilation settings added to Wu's Prometheus~\cite{wu2024prometheus}. (a) Layout and HRR growth. (b) Comparison of physical metrics for different methods. (c) Ground-truth, ResNet+BeamVQ, and ResNet predictions.
    }
    \label{fig:case} 
\end{figure*}


\subsection{Experiment Statistical Significance}
\label{sec:significance}
To measure the statistical significance of our main experiment results, we choose three backbones to train on two datasets to run 5 times. 
Table~\ref{tab:significance} records the average and standard deviation of the test MSE loss.
The results prove that our method is statistically significant to outperform the baselines
because our confidence interval is always upper than the confidence interval of the baselines. 
Due to limited computation resources, we do not cover all ten backbones and five datasets, 
but we believe these results have shown that our method has consistent advantages.


\begin{table}[h]
\label{tab:significance}
\centering
\begin{scriptsize}
    \begin{sc}
    \caption{ The average and standard deviation of MSE in 5 runs}
    \label{tab:significance}
    \centering
        \renewcommand{\multirowsetup}{\centering}
        \setlength{\tabcolsep}{10pt}
        \begin{tabular}{l|cc|cc}
            \toprule
            
            \multirow{4}{*}{Model} & \multicolumn{4}{c}{Benchmarks}  \\
            \cmidrule(lr){2-5}
            & \multicolumn{2}{c}{NSE} &   \multicolumn{2}{c}{SEVIR}   \\
            \cmidrule(lr){2-5}
           & Ori & + BeamVQ & Ori & + BeamVQ  \\
            \midrule
            ConvLSTM &0.4092$\pm$0.0002 &\textbf{0.1277$\pm$0.0001}  & 0.1762 0.0007  & \textbf{0.1279$\pm$0.0009}  \\
            FNO &  0.2227$\pm$0.0003 &\textbf{0.1007 $\pm$0.0002}& 0.0787$\pm$0.0012 & \textbf{ 0.0437$\pm$0.0013} \\
            CNO & 0.2192 $\pm$0.0008 &\textbf{ 0.1492$\pm$0.0011}& 0.0057$\pm$0.0005 & \textbf{ 0.0053$\pm$0.0006} \\
            \bottomrule
        \end{tabular}
    \end{sc}

\end{scriptsize}
\end{table}



\end{document}
\endinput
%%
%% End of file `sample-authordraft.tex'.
