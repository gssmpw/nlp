%%
%% This is file `sample-manuscript.tex',
%% generated with the docstrip utility.
%%
%% The original source files were:
%%
%% samples.dtx  (with options: `manuscript')
%% 
%% IMPORTANT NOTICE:
%% 
%% For the copyright see the source file.
%% 
%% Any modified versions of this file must be renamed
%% with new filenames distinct from sample-manuscript.tex.
%% 
%% For distribution of the original source see the terms
%% for copying and modification in the file samples.dtx.
%% 
%% This generated file may be distributed as long as the
%% original source files, as listed above, are part of the
%% same distribution. (The sources need not necessarily be
%% in the same archive or directory.)
%%
%% Commands for TeXCount
%TC:macro \cite [option:text,text]
%TC:macro \citep [option:text,text]
%TC:macro \citet [option:text,text]
%TC:envir table 0 1
%TC:envir table* 0 1
%TC:envir tabular [ignore] word
%TC:envir displaymath 0 word
%TC:envir math 0 word
%TC:envir comment 0 0
%%
%%
%% The first command in your LaTeX source must be the \documentclass command.
%%%% Small single column format, used for CIE, CSUR, DTRAP, JACM, JDIQ, JEA, JERIC, JETC, PACMCGIT, TAAS, TACCESS, TACO, TALG, TALLIP (formerly TALIP), TCPS, TDSCI, TEAC, TECS, TELO, THRI, TIIS, TIOT, TISSEC, TIST, TKDD, TMIS, TOCE, TOCHI, TOCL, TOCS, TOCT, TODAES, TODS, TOIS, TOIT, TOMACS, TOMM (formerly TOMCCAP), TOMPECS, TOMS, TOPC, TOPLAS, TOPS, TOS, TOSEM, TOSN, TQC, TRETS, TSAS, TSC, TSLP, TWEB.
% \documentclass[acmsmall]{acmart}

%%%% Large single column format, used for IMWUT, JOCCH, PACMPL, POMACS, TAP, PACMHCI
% \documentclass[acmlarge,screen]{acmart}

%%%% Large double column format, used for TOG
% \documentclass[acmtog, authorversion]{acmart}

%%%% Generic manuscript mode, required for submission
%%%% and peer review
\documentclass[sigconf]{acmart}
%% Fonts used in the template cannot be substituted; margin 
%% adjustments are not allowed.
%%
%% \BibTeX command to typeset BibTeX logo in the docs
\AtBeginDocument{%
  \providecommand\BibTeX{{%
    \normalfont B\kern-0.5em{\scshape i\kern-0.25em b}\kern-0.8em\TeX}}}

%% Rights management information.  This information is sent to you
%% when you complete the rights form.  These commands have SAMPLE
%% values in them; it is your responsibility as an author to replace
%% the commands and values with those provided to you when you
%% complete the rights form.
\copyrightyear{2025}
\acmYear{2025}
\setcopyright{rightsretained}
\acmConference[CHI EA '25]{Extended Abstracts of the CHI Conference on
Human Factors in Computing Systems}{April 26-May 1, 2025}{Yokohama, Japan}
\acmBooktitle{Extended Abstracts of the CHI Conference on Human Factors in
Computing Systems (CHI EA '25), April 26-May 1, 2025, Yokohama,
Japan}\acmDOI{10.1145/3706599.3719736}
%
%  Uncomment \acmBooktitle if th title of the proceedings is different
%  from ``Proceedings of ...''!
%
% \acmBooktitle{Woodstock '18: ACM Symposium on Neural Gaze Detection,
%  June 03--05, 2018, Woodstock, NY} 
% \acmPrice{15.00}
% \acmISBN{978-1-4503-XXXX-X/18/06}


%%
%% Submission ID.
%% Use this when submitting an article to a sponsored event. You'll
%% receive a unique submission ID from the organizers
%% of the event, and this ID should be used as the parameter to this command.
%%\acmSubmissionID{123-A56-BU3}

%%
%% For managing citations, it is recommended to use bibliography
%% files in BibTeX format.
%%
%% You can then either use BibTeX with the ACM-Reference-Format style,
%% or BibLaTeX with the acmnumeric or acmauthoryear sytles, that include
%% support for advanced citation of software artefact from the
%% biblatex-software package, also separately available on CTAN.
%%
%% Look at the sample-*-biblatex.tex files for templates showcasing
%% the biblatex styles.
%%

%%
%% The majority of ACM publications use numbered citations and
%% references.  The command \citestyle{authoryear} switches to the
%% "author year" style.
%%
%% If you are preparing content for an event
%% sponsored by ACM SIGGRAPH, you must use the "author year" style of
%% citations and references.
%% Uncommenting
%% the next command will enable that style.
%%\citestyle{acmauthoryear}

%%
%% end of the preamble, start of the body of the document source.
\begin{document}

%%
%% The "title" command has an optional parameter,
%% allowing the author to define a "short title" to be used in page headers.
% \title{Beyond Decision Support - Exploring? New? Designing? Understanding Modern HAI Collaboration? Systems? in the Social Sector?}

% Designing AI for Social Services: Insights from Social Workers on Automation and Human Judgement

% AI in Social Services: Balancing Automation with Professional Judgement

% Navigating AI Integration in Social Work: A Framework for Complementing Human Expertise

\title{Empowering Social Service with AI: Insights from a Participatory Design Study with Practitioners}

% Navigating Trust and Discretion: Generative AI in Social Work Practice
% Human-AI Collaboration in Social Services: Balancing Efficiency and Expertise
% Empowering or Undermining? Exploring Generative AI’s Role in Social Work
% From Support to Collaboration: Designing Generative AI for Social Service Workers
% Ethics and Efficiency: Integrating Generative AI into Social Service Practice
% Rethinking AI in Social Work: A Framework for Human-AI Collaboration
% Generative AI in Social Services: Addressing Overreliance, Bias, and Professional Discretion
% Co-Designing AI Tools for Social Work: Balancing Innovation with Practitioner Values

%%
%% The "author" command and its associated commands are used to define
%% the authors and their affiliations.
%% Of note is the shared affiliation of the first two authors, and the
%% "authornote" and "authornotemark" commands
%% used to denote shared contribution to the research.
% \author{Anonymous Authors}
\author{Yugin Tan}
\email{tan.yugin@u.nus.edu}
\affiliation{
  \institution{School of Computing}
  \institution{National University of Singapore}
  \country{Singapore}
  % \streetaddress{P.O. Box 1212}
  % \city{Dublin}
  % \state{Ohio}
  % \country{USA}
  % \postcode{43017-6221}
}

\author{Soh Kai Xin}
\email{kaisoh2030@u.northwestern.edu}
\affiliation{
  \institution{School of Communications}
  \institution{Northwestern University}
  \country{United States of America}
}

\author{Zhang Renwen}
\email{r.zhang@nus.edu.sg}
\affiliation{
  \institution{Department of Communications and New Media}
  \institution{National University of Singapore}
  \country{Singapore}
}
\author{Lee Jungup}
\email{swklj@nus.edu.sg}
\affiliation{
  \institution{Department of Social Work}
  \institution{National University of Singapore}
  \country{Singapore}
}

\author{Meng Han}
\email{han.meng@u.nus.edu}
\affiliation{
  \institution{School of Computing}
  \institution{National University of Singapore}
  \country{Singapore}
}

\author{Biswadeep Sen}
\email{e0989386@u.nus.edu}
\affiliation{
  \institution{School of Computing}
  \institution{National University of Singapore}
  \country{Singapore}
}

\author{Lee Yi-Chieh}
\email{yclee@nus.edu.sg}
\affiliation{
  \institution{School of Computing}
  \institution{National University of Singapore}
  \country{Singapore}
}
% \email{trovato@corporation.com}
% \orcid{1234-5678-9012}
% \author{G.K.M. Tobin}
% \authornotemark[1]
% \email{webmaster@marysville-ohio.com}
% \affiliation{%
%   \institution{Institute for Clarity in Documentation}
%   \streetaddress{P.O. Box 1212}
%   \city{Dublin}
%   \state{Ohio}
%   \country{USA}
%   \postcode{43017-6221}
% }

%%
%% By default, the full list of authors will be used in the page
%% headers. Often, this list is too long, and will overlap
%% other information printed in the page headers. This command allows
%% the author to define a more concise list
%% of authors' names for this purpose.
\renewcommand{\shortauthors}{Tan et al.}

%%
%% The abstract is a short summary of the work to be presented in the
%% article.
\begin{abstract}
In social service, administrative burdens and decision-making challenges often hinder practitioners from performing effective casework. Generative AI (GenAI) offers significant potential to streamline these tasks, yet exacerbates concerns about overreliance, algorithmic bias, and loss of identity within the profession. We explore these issues through a two-stage participatory design study. We conducted formative co-design workshops (\textit{n=27}) to create a prototype GenAI tool, followed by contextual inquiry sessions with practitioners (\textit{n=24}) using the tool with real case data. We reveal opportunities for AI integration in documentation, assessment, and worker supervision, while highlighting risks related to GenAI limitations, skill retention, and client safety. Drawing comparisons with GenAI tools in other fields, we discuss design and usage guidelines for such tools in social service practice.
\end{abstract}

%%
%% The code below is generated by the tool at http://dl.acm.org/ccs.cfm.
%% Please copy and paste the code instead of the example below.
%%
\begin{CCSXML}
<ccs2012>
   <concept>
       <concept_id>10003120.10003121.10011748</concept_id>
       <concept_desc>Human-centered computing~Empirical studies in HCI</concept_desc>
       <concept_significance>500</concept_significance>
       </concept>
   <concept>
       <concept_id>10003120.10003130.10003131.10003570</concept_id>
       <concept_desc>Human-centered computing~Computer supported cooperative work</concept_desc>
       <concept_significance>500</concept_significance>
       </concept>
 </ccs2012>
\end{CCSXML}

\ccsdesc[500]{Human-centered computing~Empirical studies in HCI}
\ccsdesc[500]{Human-centered computing~Computer supported cooperative work}
%%
%% Keywords. The author(s) should pick words that accurately describe
%% the work being presented. Separate the keywords with commas.
\keywords{AI Decision-Making, Human-AI collaboration, LLM, Social Service}

%% A "teaser" image appears between the author and affiliation
%% information and the body of the document, and typically spans the
%% page.
% \begin{teaserfigure}
%   % \includegraphics[width=\textwidth]{sample-franklin.png}
%   \caption{Seattle Mariners at Spring Training, 2010.}
%   \Description{Enjoying the baseball game from the third-base
%   seats. Ichiro Suzuki preparing to bat.}
%   \label{fig:teaser}
% \end{teaserfigure}

% \received{?}
% \received[revised]{?}
% \received[accepted]{?}

%%
%% This command processes the author and affiliation and title
%% information and builds the first part of the formatted document.
\maketitle


\section{Introduction}

Video generation has garnered significant attention owing to its transformative potential across a wide range of applications, such media content creation~\citep{polyak2024movie}, advertising~\citep{zhang2024virbo,bacher2021advert}, video games~\citep{yang2024playable,valevski2024diffusion, oasis2024}, and world model simulators~\citep{ha2018world, videoworldsimulators2024, agarwal2025cosmos}. Benefiting from advanced generative algorithms~\citep{goodfellow2014generative, ho2020denoising, liu2023flow, lipman2023flow}, scalable model architectures~\citep{vaswani2017attention, peebles2023scalable}, vast amounts of internet-sourced data~\citep{chen2024panda, nan2024openvid, ju2024miradata}, and ongoing expansion of computing capabilities~\citep{nvidia2022h100, nvidia2023dgxgh200, nvidia2024h200nvl}, remarkable advancements have been achieved in the field of video generation~\citep{ho2022video, ho2022imagen, singer2023makeavideo, blattmann2023align, videoworldsimulators2024, kuaishou2024klingai, yang2024cogvideox, jin2024pyramidal, polyak2024movie, kong2024hunyuanvideo, ji2024prompt}.


In this work, we present \textbf{\ours}, a family of rectified flow~\citep{lipman2023flow, liu2023flow} transformer models designed for joint image and video generation, establishing a pathway toward industry-grade performance. This report centers on four key components: data curation, model architecture design, flow formulation, and training infrastructure optimization—each rigorously refined to meet the demands of high-quality, large-scale video generation.


\begin{figure}[ht]
    \centering
    \begin{subfigure}[b]{0.82\linewidth}
        \centering
        \includegraphics[width=\linewidth]{figures/t2i_1024.pdf}
        \caption{Text-to-Image Samples}\label{fig:main-demo-t2i}
    \end{subfigure}
    \vfill
    \begin{subfigure}[b]{0.82\linewidth}
        \centering
        \includegraphics[width=\linewidth]{figures/t2v_samples.pdf}
        \caption{Text-to-Video Samples}\label{fig:main-demo-t2v}
    \end{subfigure}
\caption{\textbf{Generated samples from \ours.} Key components are highlighted in \textcolor{red}{\textbf{RED}}.}\label{fig:main-demo}
\end{figure}


First, we present a comprehensive data processing pipeline designed to construct large-scale, high-quality image and video-text datasets. The pipeline integrates multiple advanced techniques, including video and image filtering based on aesthetic scores, OCR-driven content analysis, and subjective evaluations, to ensure exceptional visual and contextual quality. Furthermore, we employ multimodal large language models~(MLLMs)~\citep{yuan2025tarsier2} to generate dense and contextually aligned captions, which are subsequently refined using an additional large language model~(LLM)~\citep{yang2024qwen2} to enhance their accuracy, fluency, and descriptive richness. As a result, we have curated a robust training dataset comprising approximately 36M video-text pairs and 160M image-text pairs, which are proven sufficient for training industry-level generative models.

Secondly, we take a pioneering step by applying rectified flow formulation~\citep{lipman2023flow} for joint image and video generation, implemented through the \ours model family, which comprises Transformer architectures with 2B and 8B parameters. At its core, the \ours framework employs a 3D joint image-video variational autoencoder (VAE) to compress image and video inputs into a shared latent space, facilitating unified representation. This shared latent space is coupled with a full-attention~\citep{vaswani2017attention} mechanism, enabling seamless joint training of image and video. This architecture delivers high-quality, coherent outputs across both images and videos, establishing a unified framework for visual generation tasks.


Furthermore, to support the training of \ours at scale, we have developed a robust infrastructure tailored for large-scale model training. Our approach incorporates advanced parallelism strategies~\citep{jacobs2023deepspeed, pytorch_fsdp} to manage memory efficiently during long-context training. Additionally, we employ ByteCheckpoint~\citep{wan2024bytecheckpoint} for high-performance checkpointing and integrate fault-tolerant mechanisms from MegaScale~\citep{jiang2024megascale} to ensure stability and scalability across large GPU clusters. These optimizations enable \ours to handle the computational and data challenges of generative modeling with exceptional efficiency and reliability.


We evaluate \ours on both text-to-image and text-to-video benchmarks to highlight its competitive advantages. For text-to-image generation, \ours-T2I demonstrates strong performance across multiple benchmarks, including T2I-CompBench~\citep{huang2023t2i-compbench}, GenEval~\citep{ghosh2024geneval}, and DPG-Bench~\citep{hu2024ella_dbgbench}, excelling in both visual quality and text-image alignment. In text-to-video benchmarks, \ours-T2V achieves state-of-the-art performance on the UCF-101~\citep{ucf101} zero-shot generation task. Additionally, \ours-T2V attains an impressive score of \textbf{84.85} on VBench~\citep{huang2024vbench}, securing the top position on the leaderboard (as of 2025-01-25) and surpassing several leading commercial text-to-video models. Qualitative results, illustrated in \Cref{fig:main-demo}, further demonstrate the superior quality of the generated media samples. These findings underscore \ours's effectiveness in multi-modal generation and its potential as a high-performing solution for both research and commercial applications.
\section{Related Work}
\label{sec:relatedwork}

\subsection{Current AI Tools for Social Service Practitioners}
\label{subsec:relatedtools}
% the title I feel is quite broad

Artificial Intelligence (AI) has long been used for risk assessments, decision-making, and workload management in sectors like child protection services and mental health treatment \cite{fluke_artificial_1989, patterson_application_1999}. 
Recent applications in clinical social work include risk assessments \cite{gillingham2019can, jacobi_functions_2023, liedgren_use_2016, molala_social_2023}, public health initiatives \cite{rice_piloting_2018}, and education and training for practitioners \cite{asakura_call_2020, tambe_artificial_2018}. Present studies on case management focus mainly on decision support tools \cite{james_algorithmic_2023, kawakami2022improving}, especially predictive risk models (PRMs) used to predict social service risks and outcomes \cite{gillingham2019can, van2017predicting}. A prominent example is the Allegheny Family Screening Tool (AFST), which assesses child abuse risk using data from US public systems \cite{chouldechova_case_2018, vaithianathan2017developing}. Elsewhere, researchers have also piloted AI systems to predict social service needs for the homeless using Medicaid data \cite{erickson_automatic_2018, pourat_easy_2023} or promote health interventions like HIV testing among at-risk populations \cite{rice_piloting_2018, yadav_maximizing_2017}.

Beyond such tools, however, the sector also stands to benefit immensely from newer forms of AI, such as GenAI. SSPs work in time-poor environments \cite{tiah_can_2024}, being often overwhelmed with tedious administrative work \cite{meilvang_working_2023} and large amounts of paperwork and data processing \cite{singer_ai_2023, tiah_can_2024}. GenAI is well placed to streamline and automate tasks such as the formatting of case notes, the formulation of treatment plans, and the writing of progress reports, allowing valuable time to be spent on more meaningful work, such as client engagement and the improvement of service quality \cite{fernando_integration_2023, perron_generative_2023, tiah_can_2024, thesocialworkaimentor_ai_nodate}. There is, however, scant research on GenAI in the social service sector \cite{wykman_artificial_2023}.

% This study therefore seeks to fill this critical gap by exploring how SSPs use and interact with a novel GenAI tool, helping to expand our understanding of the new opportunities that HAI collaboration can bring to the social service sector.

% AI has also been employed for mental health support and therapeutic interventions, with conversational agents serving as on-demand virtual counsellors to provide clinical care and support \cite{lisetti_i_2013, reamer_artificial_2023}.

% The recent rise of GenAI is poised to further advance social service practice, facilitating the automation of administrative tasks, streamlining of paperwork and documentation, optimisation of resource allocation, data analysis, and enhancing client support and interventions \cite{fernando_integration_2023, perron_generative_2023}.


% commercial solutions include Woebot, which simulates therapeutic conversation, and Wysa, an “emotionally intelligent” AI coach, powered by evidenced-based clinical techniques \cite{reamer_artificial_2023}. 
% Non-clinical AI agents like Replika and companion robots can also provide social support and reduce loneliness amongst individuals \cite{ahmed_humanrobot_2024, chaturvedi_social_2023, pani_can_2024, ta_user_2020}.


\subsection{Challenges in AI Use in Social Service}
\label{subsec:relatedworkaiuse}

% Despite the immense potential of AI systems to augment social work practice, there are multiple challenges with integrating such systems into real-life practice. 
Despite its evident benefits, multiple challenges plague the integration of AI and its vast potential into real-life social service practice.
% Numerous studies have investigated the use of PRMs to help practitioners decide on a course of action for their clients. 
When employing algorithmic decision-making systems, practitioners often experience tension in weighing AI suggestions against their own judgement \cite{kawakami2022improving, saxena2021framework}, being uncertain of how far they should rely on the machine. 
% Despite often being instructed to use the tool as part of evaluating a client, 
Workers are often reluctant to fully embrace AI assessments due to its inability to adequately account for the full context of a case \cite{kawakami2022improving, gambrill2001need}, and lack of clarity and transparency on AI systems and limitations \cite{kawakami2022improving}. Brown et al. \cite{brown2019toward} conducted workshops using hypothetical algorithmic tools 
% to understand service providers' comfort levels with using such tools in their work,
and found similar issues with mistrust and perceived unreliability. Furthermore, introducing AI tools can create new problems of its own, causing confusion and distrust amongst workers \cite{kawakami2022improving}. Such factors are critical barriers to the acceptance and effective use of AI in the sector.

\citeauthor{meilvang_working_2023} (2023) cites the concept of \textit{boundary work}, which explores the delineation between "monotonous" administrative labour and "professional", "knowledge-based" work drawing on core competencies of SSPs. While computers have long been used for bureaucratic tasks such as client registration, the introduction of decision support systems like PRMs stirred debate over AI "threatening professional discretion and, as such, the profession itself" \cite{meilvang_working_2023}. Such latent concerns arguably drive the resistance to technology adoption described above. GenAI is only set to further push this boundary, 
% these concerns are only set to grow in tandem with the vast capabilities of generative and other modern AI systems. Compared to the relatively primitive AI systems in past years, perceived as statistical algorithms \cite{brown2019toward} turning preset inputs like client age and behavioural symptoms \cite{vaithianathan2017developing} into simple numerical outputs indicating various risk scores, modern AI systems are vastly more capable: LLMs 
with its ability to formulate detailed reports and assessments that encroach upon the "core" work of SSPs.
% accept unrestricted and unstructured inputs and return a range of verbose and detailed evaluations according to the user's instructions. 
Introducing these systems exacerbates previously-raised issues such as understanding the limitations and possibilities of AI systems \cite{kawakami2022improving} and risk of overreliance on AI \cite{van2023chatgpt}, and requires a re-examination of where users fall on the algorithmic aversion-bias scale \cite{brown2019toward} and how they detect and react to algorithmic failings \cite{de2020case}. We address these critical issues through an empirical, on-the-ground study that to our knowledge is the first of its kind since the new wave of GenAI.

% W 

% Yet, to date, we have limited knowledge on the real-world impacts and implications of human-AI collaboration, and few studies have investigated practitioners’ experiences working with and using such AI systems in practice, especially within the social work context \cite{kawakami2022improving}. A small number of studies have explored practitioner perspectives on the use of AI in social work, including Kawakami et al. \cite{kawakami2022improving}, who interviewed social workers on their experiences using the AFST; Stapleton et al. \cite{stapleton_imagining_2022}, who conducted design workshops with caseworkers on the use of PRMs in child welfare; and Wassal et al. \cite{wassal_reimagining_2024}, who interviewed UK social work professionals on the use of AI. A common thread from all these studies was a general disregard for the context and users, with many practitioners criticising the failure of past AI tools arising from the lack of participation and involvement of social workers and actual users of such systems in the design and development of algorithmic systems \cite{wassal_reimagining_2024}. Similarly, in a scoping review done on decision-support algorithms in social work, Jacobi \& Christensen \cite{jacobi_functions_2023} reported that the majority of studies reveal limited bottom-up involvement and interaction between social workers, researchers and developers, and that algorithms were rarely developed with consideration of the perspective of social workers.
% so the \cite{yang_unremarkable_2019} and \cite{holten_moller_shifting_2020} are not real-world impacts? real-world means to hear practitioner's voice? I feel this is quite important but i didnt get this point in intro!

% why mentioning 'which have largely focused on existing ADS tools (e.g., AFST)'? i can see our strength is more localized, but without basic knowledge of social work i didnt get what's the 'departure' here orz
% the paragraph is great! do we need to also add one in line 20 21?

% \subsection{Designing AI for Social Service through Participatory Design}
% \label{subsec:relatedworkpd}
% % i think it's important! but maybe not a whole subsection? but i feel the strong connection with practitioners is indeed one of our novelties and need to highlight it, also in intro maybe
% % Participatory design (PD) has long been used extensively in HCI \cite{muller1993participatory}, to both design effective solutions for a specific community and gain a deep understanding of that community. Of particular interest here is the rich body of literature on PD in the field of healthcare \cite{donetto2015experience}, which in this regard shares many similarities and concerns with social work. PD has created effective health improvement apps \cite{ryu2017impact}, 

% % PD offers researchers the chance to gather detailed user requirements \cite{ryu2017impact}...

% Participatory design (PD) is a staple of HCI research \cite{muller1993participatory}, facilitating the design of effective solutions for a specific community while gaining a deep understanding of its stakeholders. The focus in PD of valuing the opinions and perspectives of users as experts \cite{schuler_participatory_1993} 
% % In recent years, the tech and social work sectors have awakened to the importance of involving real users in designing and implementing digital technologies, developing human-centred design processes to iteratively design products or technologies through user feedback 
% has gained importance in recent years \cite{storer2023reimagining}. Responding to criticisms and failures of past AI tools that have been implemented without adequate involvement and input from actual users, HCI scholars have adopted PD approaches to design predictive tools to better support human decision-making \cite{lehtiniemi_contextual_2023}.
% % ; accordingly, in social service, a line of research has begun studying and designing for human-AI collaboration with real-world users (e.g. \cite{holten_moller_shifting_2020, kawakami2022improving, yang_unremarkable_2019}).
% Section \ref{subsec:relatedworkaiuse} shows a clear need to better understand SSP perspectives when designing and implementing AI tools in the social sector. 
% Yet, PD research in this area has been limited. \citeauthor{yang2019unremarkable} (2019), through field evaluation with clinicians, investigated reasons behind the failure of previous AI-powered decision support tools, allowing them to design a new-and-improved AI decision-support tool that was better aligned with healthcare workers’ workflows. Similarly, \citeauthor{holten_moller_shifting_2020} (2020) ran PD workshops with caseworkers, data scientists and developers in public service systems to identify the expectations and needs that different stakeholders had in using ADS tools.

% % Indeed, it is as Wise \cite{wise_intelligent_1998} noted so many years ago on the rise of intelligent agents: “it is perhaps when technologies are new, when their (and our) movements, habits and attitudes seem most awkward and therefore still at the forefront of our thoughts that they are easiest to analyse” (p. 411). 
% Building upon this existing body of work, we thus conduct a study to co-design an AI tool \textit{for} and \textit{with} SSPs through participatory workshops and focus group discussions. In the process, we revisit many of the issues mentioned in Section \ref{subsec:relatedworkaiuse}, but in the context of novel GenAI systems, which are fundamentally different from most historical examples of automation technologies \cite{noy2023experimental}. This valuable empirical inquiry occurs at an opportune time when varied expectations about this nascent technology abound \cite{lehtiniemi_contextual_2023}, allowing us to understand how SSPs incorporate AI into their practice, and what AI can (or cannot) do for them. In doing so, we aim to uncover new theoretical and practical insights on what AI can bring to the social service sector, and formulate design implications for developing AI technologies that SSPs find truly meaningful and useful.

\section{Study Design}
\label{sec:stage1design}


% \begin{table}
%   \caption{Workshop Participants from Agency A and Agency B}
%   \label{tab:workshopParticipants}
%   \begin{tabular}{cc|cc}
%     \toprule
%     \multicolumn{2}{c|}{\textbf{Agency A}}&\multicolumn{2}{c}{\textbf{Agency B}} \\
%     \midrule
%     Code&Role&Code&Role\\
%     \midrule
%     \multicolumn{2}{c|}{Group 1}& \multicolumn{2}{c}{Group 1} \\
%     \midrule
%     TS1 & Senior Social Worker & CD1 & Director\\
%     TS4 & Social Worker & CY1 & Youth Work Services\\
%     TS5 & Social Worker & CY2 & Youth Work Services\\ 
%     TS6 & Social Worker & CY3 & Youth Work Services\\
%     TS7 & Social Worker & CP1 & Youth Projects\\
%     && CP2 & Youth Projects\\
%     && CP3 & Youth Projects\\
%     \midrule
%     \multicolumn{2}{c|}{Group 2}&\multicolumn{2}{c}{Group 2} \\
%     \midrule
%     TD1 & Executive Director & CC1 & Counsellor\\
%     TD2 & Senior Director & CC2 & Counsellor\\
%     TS2 & Social Worker & CC3 & Counsellor\\
%     TS3 & Social Worker & CC4 & Counsellor\\
%     TP1 & Psychologist & CP4 & Youth Projects\\
%   \bottomrule
% \end{tabular}
% \end{table}

% \begin{table}
%   \caption{Other Participants from Agency A and Agency B}
%   \label{tab:otherParticipants}
%   \begin{tabular}{ccl}
%     \toprule
%     Code&Role\\
%     \midrule
%     \multicolumn{2}{c}{\textbf{Agency A}} \\
%     \midrule
%     TS8 & Social Worker\\
%     TS9 & Social Worker\\
%     TS10 & Social Worker\\
%     \midrule
%     \multicolumn{2}{c}{\textbf{Agency B}} \\
%     \midrule
%     CC5 & Counsellor\\
%   \bottomrule
% \end{tabular}
% \end{table}

% \begin{figure}
% \begin{minipage}{.5\textwidth}

% \end{minipage}
% \end{figure}

We engaged in a two-stage study: a formative, participatory design study to understand the opportunities and challenges perceived by our participants and to co-design a GenAI tool, followed by an evaluative contextual inquiry to assess the effectiveness of the resulting tool. We partnered with two local, government-funded social service agencies (SSAs) in a Southeast Asian country that had expressed interest in adopting GenAI. Agency A focused on family-oriented casework, handling mostly walk-in clients and taking them through the full process \cite{rooney2017direct} of exploration, assessment, implementation, and eventually termination. Agency B worked more with schools and youths, partnering with education institutes to render assistance to children or teenagers in need. Both agencies used English as a working language and for all official documentation. In a small proportion of cases, Agency A's client interactions took place in a different language that was more comfortable for the client; in these instances, workers would either record their notes in English or manually translate them back into English before taking them back to the agency for further work. All participant interactions in this study were also conducted in English.

% \subsection{Workshops}

In \textbf{stage 1, the co-design phase}, we conducted two 90-minute long workshops with agencies A and B in November 2023. A total of 27 SSPs were involved, hailing from different roles and experience levels\footnote{Pictures of the workshops are in Appendix \ref{appendix:workshops}.}. We aimed to understand 1) the nature of the day-to-day work that our participants performed and what opportunities they perceived for using AI to help with it, and 2) the perceived risks and challenges of AI use to shape the design of the second phase of our study. In each workshop, we briefly introduced LLMs in the form of ChatGPT\footnote{At the time of the workshops (October 2023), ChatGPT (GPT-3.5) was the most well-known LLM.}, then conducted brainstorming and sketching sessions in small groups of 4-6. Full details of the workshops are in Appendix \ref{appendix:workshops}.

Based on the opportunities and concerns identified in the workshops (see Section \ref{findings:workshop} or Appendix \ref{appendix:workshopFindings}), we created a prototype assistant tool (Fig. \ref{fig:participantsAndTool}, right). The tool comprised an input text box for users to enter details about their client and a number of different types of output options. Users could select an output option based on a desired use case, then click a "generate" button to produce an LLM-written\footnote{OpenAI GPT-4 Turbo, gpt-4-1025-preview.} response. These output options addressed various manual and mental labour difficulties in social work, and simulated potential uses addressing these difficulties. In response to complaints about manual work, the tool offered options to rewrite workers' rough notes into various organization-wide formats (e.g. BPSS, DIAP, Ecological, 5Ps). Users also raised points about assessments, case conceptualization, and ideation. We included options for strength, risk, and challenge assessments, following common social service practices \cite{rooney2017direct}. Finally, we added options to generate client intervention plans according to three common theoretical models: CBT, SFBT\footnote{CBT: Cognitive Behavioural Therapy; SFBT: Solution-Focused Brief Therapy.}, and Task-Centred Interventions.

In \textbf{stage 2, we conducted focus group discussions} (FGDs), where we simulated a contextual inquiry process \cite{holtzblatt2017contextual} by asking participants to walk us through their use of our tool and explain their thought processes along the way. We also encouraged them to point out weaknesses or flaws in the system and suggest potential improvements. We conducted the FGDs in groups of 2-4 SSPs from Agency A (the family service centre agency), with a mix of workers of varying seniority levels. We held a total of 8 sessions totalling 24 participants (Fig. \ref{fig:participantsAndTool}, left). Each session averaged 45-60 minutes in length and was audio-recorded for transcription and analysis. These sessions were conducted in May 2024.

In these sessions, we sought to understand the \textit{opportunities} of AI use in social work. Participants brought along different types of anonymised case files (e.g., rough short-hand notes, complete reports, intake files) from recent clients they had worked with. We explained the various functions of the system, then told participants to imagine themselves using it to help with the cases they had on hand, exploring a range of use cases from generating documentation reports to planning future sessions or interventions for the client. We also identified early on that case supervision by senior workers was a key means of addressing the difficulties faced by junior workers and encouraging worker development. We therefore sought to understand how AI could play a collaborative role in the supervision process. In sessions with supervisors present, we asked these more senior workers how they felt the various functions of the system could assist them in their discussion of cases with their supervisees. 
% We then discussed how the tool's outputs could help in areas such as In sessions with supervisors present,  We also learned early on that case supervision by senior workers was a key means of addressing the difficulties faced by junior workers and encouraging worker development. We therefore sought to understand how AI could play a collaborative role in the supervision process. , we asked these more senior workers how they felt the various functions of the system could aid them in their discussion of cases with their supervisees. 

We also aimed to understand the possible \textit{challenges} of using AI tools in social work. We asked participants to review the quality of the tool's outputs, compare them to their own, and identify areas where they would fall short of expectations or fail to be useful in their daily work. We note that while our platform is based on GPT-4, a general purpose model not tuned for social work analysis\footnote{Specialised AI systems for social work case management do exist (e.g. \cite{socialworkmagic2024, caseworthy2024}, but these are focused on user experience and do not provide greater insight than vanilla GPT; furthermore, \cite{socialworkmagic2024} advises users to take its assessments and suggestions as only a starting point, which aligns with how we position our tool.}, our aim was not to discover the specific weaknesses of GPT-4 or any LLM in particular. Rather, it was to understand what participants perceived to be good outputs from an AI system, and in the process understand how they might be affected by any possible shortcomings of such a system. % I actually feel the footnote here is pretty important at the first glimpse haha

Finally, we explored some of the longer-term effects of AI use, such as potential overreliance on the tool and the possibility of becoming overly trusting of the system's output. Senior workers in particular were asked about how they perceived junior workers new to the sector having such a system to help them with their daily work.

We performed qualitative thematic analysis \cite{braun2006using} on the transcripts, adopting a bottom-up, inductive approach to data coding. This process is detailed in Appendix \ref{appendix:analysis}, with the findings presented in the next section.


% \begin{table}[H]
% \centering
% \caption{Focus Group Participants from Agency A}
%   \label{tab:thkMayParticipants}
%   \begin{tabular}{cc|cc}
%     \toprule
%     \multicolumn{2}{c|}{\textbf{Centre 1}}&\multicolumn{2}{c|}{\textbf{Center 2}} \\
%     Code&Role&Code&Role\\
%     \midrule
%     W1 & Social Worker & S4 & Supervisor\\
%     W2 & Social Worker & S5 & Supervisor \\
%     W3 & Social Worker & W4 & Social Worker\\
%     \midrule
%     S1 & Supervisor \\
%     S2 & Supervisor \\
%     S3 & Supervisor \\
%     \midrule
%     &&&& S6 & Senior SW\\
%     &&&& W5 & Social Worker\\
%     \toprule
%     Code&Role&Code&Role\\
%     \multicolumn{2}{c|}{\textbf{Center 3}}&\multicolumn{2}{c|}{\textbf{Center 4}}\\
%      W9 & Social Worker & S8 & Senior SW\\
%      W10 & Social Worker & S9 & Senior SW \\
%      C1 & Snr Counsellor & S10 & Senior SW \\
%      \midrule
%      S7 & Senior SW & W11 & Social Worker \\
%      W7 & Social Worker & W12 & Social Worker\\
%      W8 & Social Worker & W13 & Social Worker \\
%     \toprule
%     Code&Role\\
%     \multicolumn{2}{c}{\textbf{Management}}\\
%     D1 & Director*\\
%     \multicolumn{2}{c}{\textit{*Observer in}}\\
%     \midrule
%   \bottomrule
%   \Description{Table describing Focus Group Participants from Agency A. There are 8 groups from 4 different centres, plus a director classified under "Management".}
% \end{tabular}
% \label{tab:workshopParticipants}
% \Description{Table describing workshop Participants from Agency A and Agency B. There are 13 participants from Agency A, mostly social workers, and 14 participants from Agency B, a mix of youth workers, youth project workers, and counsellors.}

% \caption{Workshop Participants from Agency A and Agency B}
% \end{table}


% \begin{minipage}{0.5\linewidth}
% \begin{table}[H]
% \centering
% \begin{tabular}{cc|cc}
%     \toprule
%     \multicolumn{2}{c|}{\textbf{Agency A}}&\multicolumn{2}{c}{\textbf{Agency B}} \\
%     \midrule
%     Code&Role&Code&Role\\
%     \midrule
%     TS1 & Snr Social Worker & CD1 & Director\\
%     TS2 & Social Worker & CY1 & Youth Work \\
%     TS3 & Social Worker & CY2 & Youth Work \\ 
%     TS4 & Social Worker & CY3 & Youth Work \\
%     TS5 & Social Worker & CP1 & Youth Projects\\
%     TS6 & Social Worker & CP2 & Youth Projects\\
%     TS7 & Social Worker & CP3 & Youth Projects\\
%     TS8 & Social Worker & CP4 & Youth Projects \\
%     TS9 & Social Worker & CC1 & Counsellor \\
%     TS10 & Social Worker & CC2 & Counsellor \\
%     TD1 & Exec. Director & CC3 & Counsellor\\
%     TD2 & Snr Director & CC4 & Counsellor\\
%     TP1 &  Psychologist & CC3 & Counsellor\\
%     && CC4 & Counsellor\\
% \bottomrule   
% \end{tabular}
% \label{tab:workshopParticipants}
% \Description{Table describing workshop Participants from Agency A and Agency B. There are 13 participants from Agency A, mostly social workers, and 14 participants from Agency B, a mix of youth workers, youth project workers, and counsellors.}

% \caption{Workshop Participants from Agency A and Agency B}
% \end{table}
% \end{minipage}
% \begin{minipage}[HT]{0.5\linewidth}
% \begin{figure}[H]
% \centering
% \includegraphics[width=3in]{images/prototypetool.png}
% \caption{Prototype AI Tool}
% \Description{A screenshot of our prototype AI tool. It has a text box for entry at the top, a list of radio buttons to select output modalities, a text box for extra instructions to the model a user might want to input, and a "generate" button.}
% \label{fig:tool}
% \end{figure}
% \end{minipage}

% \begin{table}
%   \caption{Focus Group Participants from Agency A}
%   \label{tab:thkMayParticipants}
%   \begin{tabular}{cc|cc|cc|cc|cc}
%     \toprule
%     \multicolumn{2}{c|}{\textbf{Centre 1}}&\multicolumn{2}{c|}{\textbf{Center 2}}&\multicolumn{2}{c|}{\textbf{Center 3}}&\multicolumn{2}{c|}{\textbf{Center 4}}&\multicolumn{2}{c}{\textbf{Management}}\\
%     \midrule
%     Code&Role&Code&Role&Code&Role&Code&Role&Code&Role\\
%     \midrule
%     W1 & Social Worker & S4 & Supervisor & W9 & Social Worker & S8 & Senior SW & D1 & Director*\\
%     W2 & Social Worker & S5 & Supervisor & W10 & Social Worker & S9 & Senior SW & \multicolumn{2}{c}{\textit{*Observer in}}\\
%     W3 & Social Worker & W4 & Social Worker & C1 & Snr Counsellor & S10 & Senior SW & \multicolumn{2}{c}{\textit{multiple sessions}}\\
%     \midrule
%     S1 & Supervisor &&& S7 & Senior SW & W11 & Social Worker\\
%     S2 & Supervisor &&& W7 & Social Worker & W12 & Social Worker\\
%     S3 & Supervisor &&& W8 & Social Worker & W13 & Social Worker\\
%     \midrule
%     &&&& S6 & Senior SW\\
%     &&&& W5 & Social Worker\\
%   \bottomrule
%   \Description{Table describing Focus Group Participants from Agency A. There are 8 groups from 4 different centres, plus a director classified under "Management".}
% \end{tabular}
% \end{table}




 

% An important consideration was ensuring the generalisability of our findings, making sure our system and subsequent discussions adequately represented the vast possibilities of AI use cases, and not being limited to exploring a particular LLM in a specific configuration. We address this by noting the difference between areas that other or future AI systems may improve in, such as reasoning and knowledge, and those that they are unlikely to, such as being able to account for a user's (or client's) contextual factors. It is the latter that we focus our investigation on: Given an AI system is ultimately constrained by its training dataset, what are the possibilities with such systems and what are the potential downsides?

% 1) data privacy; 2) overreliance; 3) inaccurate output (participant identifiers are missed for 3 (?)


% This doubtless was influenced by our intial presentation on GenAI capabilities, where we showed an example of an LLM summarising and reformatting case notes. 

% \subsection{Research Questions and Testing Goals}

% From the above, we derive the following research questions.

% First, while many workers suggested they would be interested in AI automation tools, care needs to be taken to design such tools in a manner that is useful to them. Developing a tool that does not fit in their workflow or give outputs that are what they are looking for is pointless.


% \textbf{\textit{RQ1}}: What kind of AI tools do social workers find useful in their daily lives? How can they use these in their daily work?


% Besides non-useful outputs (possibly a system designer flaw), AI tools may also make more micro-level mistakes, such as by making topically relevant, yet incorrect or suboptimal suggestions. The ability of LLMs and other AI systems to handle social work-related queries is, to our knowledge, almost entirely unexplored.


% \textbf{\textit{RQ2}}: How accurately can modern AI tools handle these tasks?


% Given a tool that provides relevant, high-quality aid to social workers, the question shifts to how these tools should be most effectively implemented in and throughout an agency, maximising benefits while minimising harm.


% \textbf{\textit{RQ3}}: How do staff in different positions in the organisation use or think about the tool differently? What different user requirements might they have?


% \textbf{\textit{RQ4}}: How should staff in different positions in the organisation use or be given access to the tool differently? How can the tool serve to complement while not replacing human social workers?

% Due to the iterative nature of our research process, these later sessions served to both further ideate and critique potential solutions, and also test the concrete prototypes developed at that stage. We include the former as part of our discussion here, and leave the latter for a later part of the paper.

% We start by summarising some common difficulties that plague social workers in their work. We then present how participants felt that AI could help address these issues. Finally, we discuss the potential for AI to negatively affect organisational stability in a social work agency, and ways to begin understanding this complex issue.


\section{Findings}
\label{sec:stage2findings}

\subsection{Findings from Stage 1: Workshop and Co-Design}
\label{findings:workshop}
% Based on the codebook and themes, a final framework was developed, summarizing the Values, Requirements, and Attitudes that social workers have towards the application of AI tools in the social work practice.

Through the workshops, participants revealed a few major aspects of their work that could be assisted by AI. For brevity, detailed findings are in Appendix \ref{appendix:workshopFindings}. In brief, the main findings are below.

Documentation was a major pain point, with our participants needing to document "anything and everything", writing systematic reports in different structured formats (e.g. bio-psychological scales, risk factor assessments) or repacking the same content for different stakeholders like colleagues or other agencies, creating tedious duplicate work. Participants hence expressed a desire for a tool to help with manual labour, like turning point-form notes into formal reports, in various formats such as the 5Ps, DIAP, or BPSS\footnote{5Ps: Presenting problem, Predisposing factors, Precipitating factors, Perpetuating factors, Protective factors; DIAP: Data, Intervention, Assessment, Plan; BPSS: Bio-Psycho-Social-Spiritual}. Some senior personnel also suggested that AI could help workers incorporate theoretical concepts into their written work. 

% \begin{minipage}{0.5\linewidth}
% \begin{table}[H]
% \centering
% \label{tab:agencyA}

% % --- Row 1: Centres 1 & 2 ---
% \begin{tabular}{ll|ll}
% \toprule
% \multicolumn{2}{c|}{\textbf{Centre 1}} & \multicolumn{2}{c}{\textbf{Centre 2}} \\
% \midrule
% Code & Role        & Code & Role        \\
% \midrule
% W1   & Social Worker & W9   & Social Worker  \\
% W2   & Social Worker & W10  & Social Worker  \\
% W3   & Social Worker & C1   & Snr Counsellor \\
% \midrule
% S1   & Supervisor    & S7   & Senior SW      \\
% S2   & Supervisor    & W7   & Social Worker  \\
% S3   & Supervisor    & W8   & Social Worker  \\
% \midrule
%       &              & S6   & Senior SW      \\
%       &              & W5   & Social Worker  \\
% \bottomrule
% \multicolumn{2}{c|}{\textbf{Centre 3}} & \multicolumn{2}{c}{\textbf{Centre 4}} \\
% \midrule
% Code & Role        & Code & Role         \\
% \midrule
% S4   & Supervisor  & S8   & Senior SW    \\
% S5   & Supervisor  & S9   & Senior SW    \\
% W4   & Social Worker & S10  & Senior SW   \\
% \midrule
%       &             & W11  & Social Worker \\
%       &             & W12  & Social Worker \\
%       &             & W13  & Social Worker \\
% \bottomrule
% \end{tabular}

% \caption{Focus Group Participants from Agency A}
% \end{table}
% \end{minipage}
% \begin{minipage}[HT]{0.5\linewidth}
% \begin{figure}[H]
% \centering
% \includegraphics[width=2.15in]{images/prototypetool.png}
% \caption{Prototype AI Tool}
% \Description{A screenshot of our prototype AI tool. It has a text box for entry at the top, a list of radio buttons to select output modalities, a text box for extra instructions to the model a user might want to input, and a "generate" button.}
% \label{fig:participantsAndTool}
% \end{figure}
% \end{minipage}

\begin{table}
\centering
\label{tab:agencyA}

% --- Row 1: Centres 1 & 2 ---
\begin{tabular}{ll|ll}
\toprule
\multicolumn{2}{c|}{\textbf{Centre 1}} & \multicolumn{2}{c}{\textbf{Centre 2}} \\
\midrule
Code & Role        & Code & Role        \\
\midrule
W1   & Social Worker & W9   & Social Worker  \\
W2   & Social Worker & W10  & Social Worker  \\
W3   & Social Worker & C1   & Snr Counsellor \\
\midrule
S1   & Supervisor    & S7   & Senior SW      \\
S2   & Supervisor    & W7   & Social Worker  \\
S3   & Supervisor    & W8   & Social Worker  \\
\midrule
      &              & S6   & Senior SW      \\
      &              & W5   & Social Worker  \\
\bottomrule
\multicolumn{2}{c|}{\textbf{Centre 3}} & \multicolumn{2}{c}{\textbf{Centre 4}} \\
\midrule
Code & Role        & Code & Role         \\
\midrule
S4   & Supervisor  & S8   & Senior SW    \\
S5   & Supervisor  & S9   & Senior SW    \\
W4   & Social Worker & S10  & Senior SW   \\
\midrule
      &             & W11  & Social Worker \\
      &             & W12  & Social Worker \\
      &             & W13  & Social Worker \\
\bottomrule
\end{tabular}

\caption{Focus Group Participants from Agency A}
\end{table}

\begin{figure}
\centering
\includegraphics[width=2.15in]{images/prototypetool.png}
\caption{Prototype AI Tool}
\Description{A screenshot of our prototype AI tool. It has a text box for entry at the top, a list of radio buttons to select output modalities, a text box for extra instructions to the model a user might want to input, and a "generate" button.}
\label{fig:participantsAndTool}
\end{figure}



Participants also noted challenges with case formulation, in which piecing together disparate information across many pages of case notes to craft and justify an assessment is cognitively challenging and time-consuming, particularly during direct interaction with a client. This may inadvertently cause them to miss certain key insights or red flags only evident from looking at the gathered information as a whole. Participants therefore also suggested systems for generating case assessments, to improve output quality and adherence to industry-standard terms, and intervention planning, to generate possible plans for helping the client that the practitioner could consider and choose from.

% Another critical task is intervention planning, where workers follow established models like Cognitive Behaviour Therapy (CBT) or Solution-Focused Brief Therapy (SFBT) and craft a plan for their clients based on the guiding steps in each model. An AI tool could allow rapid generation of numerous possible interventions (CP1), leaving the user to pick and choose from the suggestions offered (TD1).

% This requires fitting a client's information into theoretical frameworks to produce well-grounded assessments and analyses. AI tools could improve both the speed at which relevant information is synthesised and categorised, and the quality of the output through use of technical, industry-standard terms, something that is often lacking in many SSPs

% In line with past work on the dangers of AI adoption [cite], participants echoed certain concerns about the use of AI tools.

Finally, participants also noted a few potential areas of risk. Privacy of client data was unilaterally mentioned by all, a universal concern core to the social sector. The storage of personal client information was a significant concern, particularly regarding the possibility of workers mistakenly entering sensitive information into the system. Regarding staff competency, a common concern was the impact of AI taking on an increasing part of the worker's job scopes. Specifically for analytical or ideative tasks, some seniors were concerned about the loss of critical thinking skills of junior workers who might become overreliant on the tool to perform their work for them. Multiple participants also raised the possibility of inaccurate outputs from an AI system, particularly risky when less experienced workers fail to tell when the AI's output might be suboptimal and proceed to adopt its suggestions anyway. 

\subsection{Findings from Stage 2: Focus Group Discussions}

\subsubsection{Documentation} 
\label{subsubsec:discussionuses}

Participants found many applications of GenAI in helping with multiple writing-focused tasks in the social service sector, such as summarising intake information, formatting case recordings, and writing reports. Participants generally were happy to embrace AI for such purposes; for documentation tasks such as writing case reports, the tool's outputs were largely in line with what they required, allowing them to simply "copy and paste" (W1, W2, W13) the outputs for direct use. This was, in a large way, down to how much of our participants' regular daily work focused on consistently structured, fixed-format reports. One strength evident in GPT-4 and our prototype was its ability to consistently follow instructions to produce outputs in a desired format, such as the 5Ps format (S1, S3, W1). W8, for instance, quoted, "\textit{being new, it really helps in categorising these items... I like the fact that it segregates all the [different categories]}".

Even when the output was imperfect, participants often expressed a willingness to work around these errors and make manual corrections where needed. For instance, W2 suggested they would "sift through and pick out" the more relevant parts of an overly lengthy report, while W1 would "\textit{copy and paste, then amend if I need to amend}" when the tool misinterpreted a nickname given by the worker to the client.


% S9: "I like how it is to have a sub-titles (sub-headings) and things like that"

\subsubsection{Brainstorming}

Workers felt that the structured and detailed outputs of the tool encouraged and facilitated their cognitive processes in analysing a case. W8 felt the way the tool categorised the issues in their client's case pushed them to "think more" about how they viewed it. In a similar vein, looking at a generated CBT assessment prompted W5 to consider "\textit{certain things also that I should probably look into}", which they might otherwise have overlooked. W11 commented on how the "very in-depth" explanations given by the tool helped them "expand on what they already have". Even a senior worker, S4, called the tool's analysis of a case "\textit{really helpful - it's expanding my perspective already}".

Junior workers in particular appreciated the guidance from the tool, especially with tasks they were less familiar with. For instance, W7 had not attended formal CBT training, and thus felt the tool gave them "some idea where to start" in formulating a CBT intervention plan for their client. W4 also liked having sample interventions from the tool; being a newer worker, they were uncertain of which plan of action to take, so the tool gave them a "better understanding" of the different ways to move forward with the case. Meanwhile, more generally, the tool helped with guiding workers towards formulating a course of action, such as by suggesting interventions they might not have thought of (W4) and thereby prompting newer staff with a "direction" to work towards (W5), or by being "very useful" in helping new workers prepare for sessions with clients (W9).

Supervisors, too, agreed that the tool was useful for junior workers. S4 reiterated that it could "expand the worker's perspective", and C1 called it "a good start [to help] staff think about" the case if they "got stuck" with something. S6 cited the example of how an SFBT output provided a "really good foundation" for questions for workers to ask their clients. S6 also felt the tool provided a good framework and guideline, suggesting it as a way to "polish" newer workers' skills: "\textit{For all those who are really new and do not really know how to formulate interventions or theory support and all that. I think it's quite useful... or, if they are really lost, then they can probably try the different things that are written here}." 

\subsubsection{Supervision}

The use of AI to aid in supervision emerged as a key theme. Supervision sessions consist of junior workers discussing cases with their supervisors, to refine and improve their assessment of the client. Given this, the ability of AI to quickly generate lists of ideas provided useful starting points for discussion and reflection with supervisees (C1). Many (W4, W5) suggested that the tool provided useful intervention suggestions so that workers could "ask their supervisors like, maybe, you know, maybe I can try this" (S6). From the supervisors' perspectives, the tool helped to improve and expedite the supervision process by prompting them with questions to ask supervisees: "I can bring [this list of exception questions\footnote{Asking "exception questions" to clients is a technique used in Solution-Focused Brief Therapy.}] to supervision to see whether my supervisee has used these questions... I can ask my supervisee, okay, if you have to ask this exception question to the client, how comfortable do you feel? So we can have that discussion" (C1). S4 and S7, meanwhile, highlighted the AI's ability to quickly "concretise theoretical models" to build on during sessions. 

Some even suggested that the tool could itself serve as a supervisor to help newer workers. W9 called the tool a "readily-available supervisor to get us thinking," referencing how their supervisors frequently prompted them with questions to think about their case more. W4 meanwhile felt the tool could help them move forward with a case "without having to consult with their supervisor", in instances where they were uncertain of how to proceed.
% One unexpected use of AI that emerged prominently was as a supervisory aid. Our participants mainly described supervision sessions as senior and junior workers discussing how to formulate and proceed with a case further. With this, the suggested interventions or possible solutions generated by our AI system served as good discussion points. 


\subsubsection{Concerns and Issues}

Participants raised a few issues with using AI in their work. Agency A's focus on providing family service meant that they prioritised addressing safety and risk concerns (W2), such as possible self-harm, suicide, or harm towards others. Participants were "very particular about... risks" (S9), treating it as their top priority (W1, S10) and "at the front" for all client interventions and risk assessments (S2, S9). Preventing imminent physical and psychological harms such as incarceration, abuse, and addiction were therefore cited as "non-negotiables" (W4) and primary risk factors (W2). Thus, they expressed concern when the tool "didn't exactly highlight" (W2) or entirely omitted (W1) safety risks in its output, such as in assessing a case of intra-family conflict: "\textit{The [risk of] violence is not highlighted. Where is the violence?}" (W1).

There were also worries over overuse and overreliance on AI. Participants were divided on this issue. The centre director (D1) quoted, "\textit{One of our concerns is... [will using this AI] actually disable our ability to make assessments?}" S3 agreed about the risk of over-reliance by "spoonfeeding" junior workers, and S9 noted the need for workers to "\textit{still use their brain... otherwise... they just rely on this}." S9 also worried about how junior workers might fail to recognise suboptimal AI outputs and be misled as a result. As a result, some emphasised the need for a balance between AI use and human intervention. S10 remarked, \textit{"It's the supervisor's role to keep emphasising to the supervisee, to not be married to this assessment"}, and that the AI's outputs were often "just guidelines" rather than a gospel to be followed. S9 agreed that "\textit{the expectation [for use] needs to be very clearly communicated}", highlighting the need for careful and judicious implementation of any AI system. 

However, others felt this to not be a major problem, due to the inherent focus of the profession on face-to-face client interactions. On practitioners blindly following AI recommendations, S7 commented, "\textit{it will eventually be clear that this was the recommendation of [the tool] but then you went and did something else... and then you'll see that it's just a mess in that way.}" S3 agreed: "\textit{When they do their work, it is mostly real-time. You don't go back to pen and paper and start to develop [a solution], but it's more about what is presented to them immediately [and how they react].}"

Finally, there were multiple instances where participants found the AI's output to be inadequate. These often centred around the tool failing to recognise more subtle factors in a given case that would affect the preferred course of action. For example, when the system recommended a family therapy workshop, W5 recognised that the reluctance of their client to embrace outside help made such an intervention unsuitable. In another case, when the tool recommended CBT to aid a client, W11 judged that their client lacked the intellectual capacity for such therapy to help. Knowledge of local contextual factors was another common point. S6, for instance, commented on how "\textit{there are a few parenting workshops available [here in our country] and [we know] how they are structured}," so they would be able to assess which options were suitable for their client in a way that the AI, lacking local knowledge, would not. W5 also commented on the importance of practitioners' instincts and experience: \textit{"Many of us who have been in the sector before, probably were like, this won't work."}

\section{Discussion and Conclusion}

% \begin{quote}
% \textit{"We believe it is unethical for social workers not to learn... about technology-mediated social work."} (\citeauthor{singer_ai_2023}, 2023)
% \end{quote}

In this study, we uncovered multiple ways in which GenAI can be used in social service practice. While some concerns did arise, practitioners by and large seemed optimistic about the possibilities of such tools, and that these issues could be overcome. We note that while most participants found the tool useful, it was far from perfect in its outputs. This is not surprising, since it was powered by a generic LLM rather than one fine-tuned for social service case management. However, despite these inadequacies, our participants still found many uses for most of the tool's outputs. Many flaws pointed out by our participants related to highly contextualised, local knowledge. To tune an AI system for this would require large amounts of case files as training data; given the privacy concerns associated with using client data, this seems unlikely to happen in the near future. What our study shows, however, is that GenAI systems need not aim to be perfect to be useful to social service practitioners, and can instead serve as a complement to the critical "human touch" in social service.

We draw both inspiration and comparisons with prior work on AI in other settings. Studies on creative writing tools showed how the "uncertainty" \cite{wan2024felt} and "randomness" \cite{clark2018creative} of AI outputs aid creativity. Given the promise that our tool shows in aiding brainstorming and discussion, future social service studies could consider AI tools explicitly geared towards creativity - for instance, providing side-by-side displays of how a given case would fit into different theoretical frameworks, prompting users to compare, contrast, and adopt the best of each framework; or allowing users to play around with combining different intervention modalities to generate eclectic (i.e. multi-modal) interventions.

At the same time, the concept of supervision creates a different interaction paradigm to other uses of AI in brainstorming. Past work (e.g. \cite{shaer2024ai}) has explored the use of GenAI for ideation during brainstorming sessions, wherein all users present discuss the ideas generated by the system. With supervision in social service practice, however, there is a marked information and role asymmetry: supervisors may not have had the time to fully read up on their supervisee's case beforehand, yet have to provide guidance and help to the latter. We suggest that GenAI can serve a dual purpose of bringing supervisors up to speed quickly by summarising their supervisee's case data, while simultaneously generating a list of discussion and talking points that can improve the quality of supervision. Generalising, this interaction paradigm has promise in many other areas: senior doctors reviewing medical procedures with newer ones \cite{snowdon2017does} could use GenAI to generate questions about critical parts of a procedure to ask the latter, confirming they have been correctly understood or executed; game studio directors could quickly summarise key developmental pipeline concerns to raise at meetings and ensure the team is on track; even in academia, advisors involved in rather too many projects to keep track of could quickly summarise each graduate student's projects and identify potential concerns to address at their next meeting.

In closing, we are optimistic about the potential for GenAI to significantly enhance social service practice and the quality of care to clients. Future studies could focus on 1) longitudinal investigations into the long-term impact of GenAI on practitioner skills, client outcomes, and organisational workflows, and 2) optimising workflows to best integrate GenAI into casework and supervision, understanding where best to harness the speed and creativity of such systems in harmony with the experience and skills of practitioners at all levels.

% GenAI here thus serves as a tool that supervisors can use before rather then using the session, taking just a few minutes of their time to generate a list of discussion points with their supervisees.

% Traditional brainstorming comes up with new things that users discuss. In supervision, supervisors can use AI to more efficiently generate talking points with their supervisees. These are generally not novel ideas, since an experienced worker would be able to come up with these on their own. However, the interesting and novel use of AI here is in its use as a preparation tool, efficiently generating talking and discussion points, saving supervisors' time in preparing for a session, while still serving as a brainstorming tool during the session itself.

% The idea of embracing imperfect AI echoes the findings of \citeauthor{bossen2023batman} (2023) in a clinical decision setting, which examined the successful implementation of an "error-prone but useful AI tool". This study frames human-AI collaboration as "Batman and Robin", where AI is a useful but ultimately less skilled sidekick that plays second fiddle to Batman. This is similar to \citeauthor{yang2019unremarkable}'s (2019) idea of "unremarkable AI", systems designed to be unobtrusive and only visible to the user when they add some value. As compared to \citeauthor{bossen2023batman}, however, we see fewer instances of our AI system producing errors, and more examples of it providing learning and collaborative opportunities and other new use cases. We build on the idea of "complementary performance" \cite{bansal2021does}, which discusses how the unique expertise of AI enhances human decision-making performance beyond what humans can achieve alone. Beyond decision-making, GenAI can now enable "complementary work patterns", where the nature of its outputs enables humans to carry out their work in entirely new ways. Our study suggests that rather being a sidekick - Robin - AI is growing into the role of a "second Batman" or "AI-Batman": an entity with distinct abilities and expertise from humans, and that contributes in its own unique way. There is certainly still a time and place for unremarkable AI, but exploring uses beyond that paradigm uncovers entirely new areas of system design.

% % \cite{gero2022sparks} found AI to be useful for science writers to translate ideas already in their head into words, and to provide new perspectives to spark further inspiration. \textit{But how is ours different from theirs?}

% \subsection{New Avenues of Human-AI Collaboration}
% \label{subsubsec:discussionhaicollaboration}

% Past HCI literature in other areas \cite{nah2023generative} has suggested that GenAI represents a "leap" \cite{singh2023hide} in human-AI collaboration, 
% % Even when an AI system sometimes produces irrelevant outputs, it can still provide users 
% % Such systems have been proposed as ways to 
% helping users discover new viewpoints \cite{singh2023hide}, scour existing literature to suggest new hypotheses 
% \cite{cascella2023evaluating} and answer questions \cite{biswas2023role}, stimulate their cognitive processes \cite{memmert2023towards}, and overcome "writer's block" \cite{singh2023hide, cooper2023examining} (particularly relevant to SSPs and the vast amount of writing required of them). Our study finds promise for AI to help SSPs in all of these areas. By nature of being more verbose and capable of generating large amounts of content, GenAI seems to create a new way in which AI can complement human work and expertise. Our system, as LLMs tend to do, produced a lot of "bullshit" (S6) \cite{frankfurt2005bullshit} - superficially true statements that were often only "tangentially related" and "devoid of meaning" \cite{halloran2023ai}. Yet, many participants cited the page-long analyses and detailed multi-step intervention plans generated by the AI system to be a good starting point for further discussion, both to better conceptualize a particular case and to facilitate general worker growth and development. Almost like throwing mud at a wall to see what sticks, GenAI can quickly produce a long list of ideas or information, before the worker glances through it and quickly identifies the more interesting points to discuss. Playing the proposed role as a "scaffold" for further work \cite{cooper2023examining}, GenAI, literally, generates new opportunities for novel and more effective processes and perspectives that previous systems (e.g., PRMs) could not. This represents an entirely new mode of human-AI collaboration.
% % This represents a new mode of collaboration not possible with the largely quantitative AI models (like PRMs) of the past.

% Our work therefore supports and extends prior research that have postulated the the potential of AI's shifting roles from decision-maker to human-supporter \cite{wang_human-human_2020}. \citeauthor{siemon2022elaborating} (2022) suggests the role of AI as a "creator" or "coordinator", rather than merely providing "process guidance" \cite{memmert2023towards} that does not contribute to brainstorming. Similarly, \citeauthor{memmert2023towards} (2023) propose GenAI as a step forward from providing meta-level process guidance (i.e. facilitating user tasks) to actively contributing content and aiding brainstorming. We suggest that beyond content-support, AI can even create new work processes that were not possible without GenAI. In this sense, AI has come full circle, becoming a "meta-facilitator".

% % --- WIP BELOW ---

% % Our work echoes and extends previous research on HAI collaboration in tasks requiring a human touch. \cite{gero2023social} found AI to be a safe space for creative writers to bounce ideas off of and document their inner thoughts. \cite{dhillon2024shaping} reference the idea of appropriate scaffolding in argumentative writing, where the user is providing with guidance appropriate for their competency level, and also warns of decreased satisfaction and ownership from AI use. 

% Separately, we draw parallels with the field of creative writing, where HAI collaboration has been extensively researched. Writers note the "irreducibly human" aspect of creativity in writing \cite{gero2023social}, similar to the "human touch" core to social service practice (D1); both groups therefore expressed few concerns about AI taking over core aspects of their jobs. Another interesting parallel was how writers often appreciated the "uncertainty" \cite{wan2024felt} and "randomness" \cite{clark2018creative} of AI systems, which served as a source of inspiration. This echoes the idea of "imperfect AI" "expanding [the] perspective[s]" (S4) of our participants when they simply skimmed through what the AI produced. \cite{wan2024felt} cited how the "duality of uncertainty in the creativity process advances the exploration of the imperfection of GenAI models". While social service work is not typically regarded as "creative", practitioners nonetheless go through processes of ideation and iteration while formulating a case. Our study showed hints of how AI can help with various forms of ideation, but, drawing inspiration from creative writing tools, future studies could consider designs more explicitly geared towards creativity - for instance, by attempting to fit a given case into a number of different theories or modalities, and displaying them together for the user to consider. While many of these assessments may be imperfect or even downnright nonsensical, they may contain valuable ideas and new angles on viewing the case that the practitioner can integrate into their own assessment.

% % \cite{foong2024designing}, describing the design of caregiver-facing values elicitation tools, cites the "twin scenarios" that caregivers face - private use, where they might use a tool to discover their patient's values, and collaborative use, where they discuss the resulting values with other parties close to the patient. This closely mirrors how SSPs in our study reference both individual and collaborative uses of our tool. Unlike in \cite{foong2024designing}, however, we do not see a resulting need to design a "staged approach" with distinct interface features for both stages.

% % --- END OF WIP ---

% Having mentioned algorithm aversion previously, we also make a quick point here on the other end of the spectrum - automation bias, or blind trust in an automated system \cite{brown2019toward}. LLMs risk being perceived as an "ultimate epistemic authority" \cite{cooper2023examining} due to their detailed, life-like outputs. While automation bias has been studied in many contexts, including in the social sector or adjacent areas, we suggest that the very nature of GenAI systems fundamentally inhibits automation bias. The tendency of GenAI to produce verbose, lengthy explanations prompts users to read and think through the machine's judgement before accepting it, bringing up opportunities to disagree with the machine's opinion. This guards against blind acceptance of the system's recommendations, particularly in the culture of a social work agency where constant dialogue - including discussing AI-produced work - is the norm.


% % : Perception of AI in Social Service Work ??

% \subsection{Redefining the Boundary}
% \label{subsubsec:discussiontheoretical}

% As \citeauthor{meilvang_working_2023} (2023) describes, the social service profession has sought to distance itself from comprising mostly "administrative work" \cite{abbott2016boundaries}, and workers have long tried to tried to reduce their considerable time \cite{socialraadgiverforening2010notat} spent on such tasks in favour of actual casework with clients \cite{toren1972social}. Our study, however, suggests a blurring of the line between "manual" administrative tasks and "mental" casework that draws on practitioner expertise. Many tasks our participants cited involve elements of both: for instance, documenting a case recording requires selecting only the relevant information to include, and planning an intervention can be an iterative process of drafting a plan and discussing it with colleagues and superiors. This all stems from the fact that GenAI can produce virtually any document required by the user, but this document almost always requires revision under a watchful human eye.

% \citeauthor{meilvang_working_2023} (2023) also describes a more recent shift in the perceived accepted boundary of AI interventions in social service work. From "defending [the] jurisdiction [of social service work] against artificial intelligence" in the early days of PRM and other statistical assessment tools, the community has started to embrace AI as a "decision-support ... element in the assessment process". Our study concurs and frames GenAI as a source of information that can be used to support and qualify the assessments of SSPs \cite{meilvang_working_2023}, but suggests that we can take a step further: AI can be viewed as a \textit{facilitator} rather than just a supporter. GenAI can facilitate a wide range of discussions that promote efficiency, encourage worker learning and growth, and ultimately enhance client outcomes. This entails a much larger scope of AI use, where practitioners use the information provided by AI in a range of new scenarios. 

% Taken together, these suggest a new focus for boundary work and, more broadly, HCI research. GAI can play a role not just in menial documentation or decision-support, but can be deeply ingrained into every facet of the social service workflow to open new opportunities for worker growth, workflow optimisation, and ultimately improved client outcomes. Future research can therefore investigate the deeper, organisational-level effects of these new uses of AI, and their resulting impact on the role of profession discretion in effective social service work.

% % MH: oh i feel this paragraph is quite new to me! Could we elaborate this more, and truncate the first two paragraphs a bit to adjust the word propotion?


% % Our study extensively documents this for the first time in social service practice, and in the process reveals new insights about how AI can play such a role.



% \subsection{Design Implications}

% % Add link from ACE diagram?

% % EJ: it would be interesting to discuss how LLMs could help "hands-on experience" in the discussion section

% Addressing the struggle of integrating AI amidst the tension between machine assessment and expert judgement, we reframe AI as an \textit{facilitator} rather than an algorithm or decision-support tool, alleviating many concerns about trust and explainablity. We now present a high-level framework (Figure \ref{fig:hai-collaboration}) on human-AI collaboration, presenting a new perspective on designing effective AI systems that can be applied to both the social service sector and beyond.

% \begin{figure}
%     \centering
%     \includegraphics[scale=0.15]{images/designframework.png}
%     \caption{Framework for Human-AI Collaboration}
%     \label{fig:hai-collaboration}
%     \Description{An image showing our framework for Human-AI Collaboration. It shows that as stakeholder level increases from junior to senior, the directness of use shifts from co-creation to provision.}
% \end{figure}
% % MH: so this paradigm is proposed by us? I wonder if this could a part of results as well..?

% % \subsubsection{From Creation to Provision}

% In Section \ref{sec:stage2findings}, we uncovered the different ways in which SSPs of varying seniorities use, evaluate, and suggest uses of AI. These are intrinsically tied to the perspectives and levels of expertise that each stakeholder possesses. We therefore position the role of AI along the scale of \textit{creation} to \textit{provision}. 

% With junior workers, we recommend \textbf{designing tools for co-creation}: systems that aid the least experienced workers in creating the required deliverables for their work. Rather than \textit{telling} workers what to do - a difficult task in any case given the complexity of social work solutions - AI systems should instead \textit{co-create} deliverables required of these workers. These encompass the multitude of use cases that junior workers found useful: creating reports, suggesting perspectives from which to formulate a case, and providing a starting template for possible intervention plans. Notably, since AI outputs are not perfect, we emphasise the "co" in "co-creation": AI should only be a part of the workflow that also includes active engagement on the part of the SSPs and proactive discussion with supervisors. 

% For more experienced SSPs, we recommend \textbf{designing tools for provision}. Again, this is not the mere provision of recommendations or courses of action with clients, but rather that of resources which complement the needs of workers with greater responsibilities. This notably includes supplying materials to aid with supervision, a novel use case that to our knowledge has not surfaced in previous literature. In addition, senior workers also benefit greatly from manual tasks such as routine report writing and data processing. Since these workers are more experienced and can better spot inaccuracies in AI output, we suggest that AI can "provide" a more finished product that requires less vetting and corrections, and which can be used more directly as part of required deliverables.

% % MH: can we seperate here? above is about the guidance to paradigm, below is the practical roadmap for implementation
% In terms of concrete design features, given the constant focus on discussing AI outputs between colleagues in our FGDs, we recommend that AI tools, particularly those for junior workers, \textbf{include collaborative features} that facilitate feedback and idea sharing between users. We also suggest that designers work closely with domain experts (i.e. social work practitioners and agencies) to identify areas where the given AI model tends to make more mistakes, and to build in features that \textbf{highlight potential mistakes or inadequacies} in the AI's output to facilitate further discussion and avoid workers adopting suboptimal suggestions. 

% We also point out a fundamental difference between GenAI systems and previous systems: that GenAI can now play an important role in aiding users \textit{regardless of its flaws}. The nature of GenAI means that it promotes discussion and opens up new workflows by nature of its verbose and potentially incomplete outputs. Rather than working towards more accurate or explainable outcomes, which may in any case have minimal improvement on worker outcomes \cite{li2024advanced}, designers can also focus on \textbf{understanding how GenAI outputs can augment existing user flows and create new ones}.

% % for more senior workers...

% % how to differentiate levels of workers?

% % \subsubsection{Provider}

% % The most basic and obvious role of modern AI that we identify leverages the main strength of LLMs. They have the ability to produce high-quality writing from short, point-form, or otherwise messy and disjoint case notes that user often have \textit{[cite participant here]}. 

% Finally, given the limited expertise of many workers at using AI, it is important that systems \textbf{explicitly guide users to the features they need}, rather than simply relying on the ability of GAI to understand complex user instructions. For example, in the case of flexibility in use cases (Section \ref{subsubsec:control}), systems should include user flows that help combine multiple intervention and assessment modalities in order to directly meet the needs of workers.

% \subsection{Limitations and Future Work}

% While we attempt to mimic a contextual inquiry and work environment in our study design, there is no substitute for real data from actual system deployment. The use of an AI system in day-to-day work could reveal a different set of insights. Future studies could in particular study how the longitudinal context of how user attitudes, behaviours, preferences, and work outputs change with extended use of AI. 

% While we tried to include practitioners from different agencies, roles, and seniorities, social service practice may differ culturally or procedurally in other agencies or countries. Future studies could investigate different kinds of social service agencies and in different cultures to see if AI is similarly useful there.

% As the study was conducted in a country with relatively high technology literacy, participants naturally had a higher baseline understanding and acceptance of AI and other computer systems. However, we emphasise that our findings are not contingent on this - rather, we suggest that our proposed lens of viewing AI in the social sector is a means for engaging in relevant stakeholders and ensuring the effective design and implementation of AI in the social sector, regardless of how participants feel about AI to begin with. 



% % \subsection{Notes}

% % 1) safety and risks and 2) privacy - what does the emphasis on this say about a) design recommendations and b) approach to designing/PD of such systems?


% % W9 was presented with "Strengths" and "SFBT" output options. They commented, "solution focus is always building on the person's strengths". W9 therefore requested being able to output strengths and SFBT at the same time. But this would suggest that the SFBT output does not currently emphasise strengths strongly enough. However, W9 did not specifically evaluate that, and only made this comment because they saw the "strengths" option available, and in their head, strengths are key to SFBT.
% % What does this say about system design and UI in relation to user mental models?

%%
%% The acknowledgments section is defined using the "acks" environment
%% (and NOT an unnumbered section). This ensures the proper
%% identification of the section in the article metadata, and the
%% consistent spelling of the heading.
\begin{acks}
The authors would like to thank the directors and workers at Agency A for our close working relationship over the past 18 months. 

We also acknowledge the use of GPT-4o for generating ideas for the \textit{paper title}, giving suggestions for the \textit{abstract} after the rest of the paper was completed, and \textit{shortening} parts of the paper after they were written.

This work was supported by the National University of Singapore's Centre for Computational Social Science and Humanities (CSSH), under fund number A-8002954-01-00.
\end{acks}

%%
%% The next two lines define the bibliography style to be used, and
%% the bibliography file.
\bibliographystyle{ACM-Reference-Format}
\bibliography{references}

%%
%% If your work has an appendix, this is the place to put it.
\appendix
\section{Secure Token Pruning Protocols}
\label{app:a}
We detail the encrypted token pruning protocols $\Pi_{prune}$ in Figure \ref{fig:protocol-prune} and $\Pi_{mask}$ in Figure \ref{fig:protocol-mask} in this section.

%Optionally include supplemental material (complete proofs, additional experiments and plots) in appendix.
%All such materials \textbf{SHOULD be included in the main submission.}
\begin{figure}[h]
%vspace{-0.2in}
\begin{protocolbox}
\noindent
\textbf{Parties:} Server $P_0$, Client $P_1$.

\textbf{Input:} $P_0$ and $P_1$ holds $\{ \left \langle Att \right \rangle_{0}^{h}, \left \langle Att \right \rangle_{1}^{h}\}_{h=0}^{H-1} \in \mathbb{Z}_{2^{\ell}}^{n\times n}$ and $\left \langle x \right \rangle_{0}, \left \langle x \right \rangle_{1} \in \mathbb{Z}_{2^{\ell}}^{n\times D}$ respectively, where H is the number of heads, n is the number of input tokens and D is the embedding dimension of tokens. Additionally, $P_1$ holds a threshold $\theta \in \mathbb{Z}_{2^{\ell}}$.

\textbf{Output:} $P_0$ and $P_1$ get $\left \langle y \right \rangle_{0}, \left \langle y \right \rangle_{1} \in \mathbb{Z}_{2^{\ell}}^{n'\times D}$, respectively, where $y=\mathsf{Prune}(x)$ and $n'$ is the number of remaining tokens.

\noindent\rule{13.2cm}{0.1pt} % This creates the horizontal line
\textbf{Protocol:}
\begin{enumerate}[label=\arabic*:, leftmargin=*]
    \item For $h \in [H]$, $P_0$ and $P_1$ compute locally with input $\left \langle Att \right \rangle^{h}$, and learn the importance score in each head $\left \langle s \right \rangle^{h} \in \mathbb{Z}_{2^{\ell}}^{n} $, where $\left \langle s \right \rangle^{h}[j] = \frac{1}{n} \sum_{i=0}^{n-1} \left \langle Att \right \rangle^{h}[i,j]$.
    \item $P_0$ and $P_1$ compute locally with input $\{ \left \langle s \right \rangle^{i} \in \mathbb{Z}_{2^{\ell}}^{n}  \}_{i=0}^{H-1}$, and learn the final importance score $\left \langle S \right \rangle \in \mathbb{Z}_{2^{\ell}}^{n}$ for each token, where  $\left \langle S \right \rangle[i] = \frac{1}{H} \sum_{h=0}^{H-1} \left \langle s \right \rangle^{h}[i]$.
    \item  For $i \in [n]$, $P_0$ and $P_1$ invoke $\Pi_{CMP}$ with inputs  $\left \langle S \right \rangle$ and $ \theta $, and learn  $\left \langle M \right \rangle \in \mathbb{Z}_{2^{\ell}}^{n}$, such that$\left \langle M \right \rangle[i] = \Pi_{CMP}(\left \langle S \right \rangle[i] - \theta) $, where: \\
    $M[i] = \begin{cases}
        1  &\text{if}\ S[i] > \theta, \\
        0  &\text{otherwise}.
            \end{cases} $
    % \item If the pruning location is insensitive, $P_0$ and $P_1$ learn real mask $M$ instead of shares $\left \langle M \right \rangle$. $P_0$ and $P_1$ compute $\left \langle y \right \rangle$ with input $\left \langle x \right \rangle$ and $M$, where  $\left \langle x \right \rangle[i]$ is pruned if $M[i]$ is $0$.
    \item $P_0$ and $P_1$ invoke $\Pi_{mask}$ with inputs  $\left \langle x \right \rangle$ and pruning mask $\left \langle M \right \rangle$, and set outputs as $\left \langle y \right \rangle$.
\end{enumerate}
\end{protocolbox}
\setlength{\abovecaptionskip}{-1pt} % Reduces space above the caption
\caption{Secure Token Pruning Protocol $\Pi_{prune}$.}
\label{fig:protocol-prune}
\end{figure}




\begin{figure}[h]
\begin{protocolbox}
\noindent
\textbf{Parties:} Server $P_0$, Client $P_1$.

\textbf{Input:} $P_0$ and $P_1$ hold $\left \langle x \right \rangle_{0}, \left \langle x \right \rangle_{1} \in \mathbb{Z}_{2^{\ell}}^{n\times D}$ and  $\left \langle M \right \rangle_{0}, \left \langle M \right \rangle_{1} \in \mathbb{Z}_{2^{\ell}}^{n}$, respectively, where n is the number of input tokens and D is the embedding dimension of tokens.

\textbf{Output:} $P_0$ and $P_1$ get $\left \langle y \right \rangle_{0}, \left \langle y \right \rangle_{1} \in \mathbb{Z}_{2^{\ell}}^{n'\times D}$, respectively, where $y=\mathsf{Prune}(x)$ and $n'$ is the number of remaining tokens.

\noindent\rule{13.2cm}{0.1pt} % This creates the horizontal line
\textbf{Protocol:}
\begin{enumerate}[label=\arabic*:, leftmargin=*]
    \item For $i \in [n]$, $P_0$ and $P_1$ set $\left \langle M \right \rangle$ to the MSB of $\left \langle x \right \rangle$ and learn the masked tokens $\left \langle \Bar{x} \right \rangle \in Z_{2^{\ell}}^{n\times D}$, where
    $\left \langle \Bar{x}[i] \right \rangle = \left \langle x[i] \right \rangle + (\left \langle M[i] \right \rangle << f)$ and $f$ is the fixed-point precision.
    \item $P_0$ and $P_1$ compute the sum of $\{\Pi_{B2A}(\left \langle M \right \rangle[i]) \}_{i=0}^{n-1}$, and learn the number of remaining tokens $n'$ and the number of tokens to be pruned $m$, where $m = n-n'$.
    \item For $k\in[m]$, for $i\in[n-k-1]$, $P_0$ and $P_1$ invoke $\Pi_{msb}$ to learn the highest bit of $\left \langle \Bar{x}[i] \right \rangle$, where $b=\mathsf{MSB}(\Bar{x}[i])$. With the highest bit of $\Bar{x}[i]$, $P_0$ and $P_1$ perform a oblivious swap between $\Bar{x}[i]$ and $\Bar{x}[i+1]$:
    $\begin{cases}
        \Tilde{x}[i] = b\cdot \Bar{x}[i] + (1-b)\cdot \Bar{x}[i+1] \\
        \Tilde{x}[i+1] = b\cdot \Bar{x}[i+1] + (1-b)\cdot \Bar{x}[i]
    \end{cases} $ \\
    $P_0$ and $P_1$ learn the swapped token sequence $\left \langle \Tilde{x} \right \rangle$.
    \item $P_0$ and $P_1$ truncate $\left \langle \Tilde{x} \right \rangle$ locally by keeping the first $n'$ tokens, clear current MSB (all remaining token has $1$ on the MSB), and set outputs as $\left \langle y \right \rangle$.
\end{enumerate}
\end{protocolbox}
\setlength{\abovecaptionskip}{-1pt} % Reduces space above the caption
\caption{Secure Mask Protocol $\Pi_{mask}$.}
\label{fig:protocol-mask}
%\vspace{-0.2in}
\end{figure}

% \begin{wrapfigure}{r}{0.35\textwidth}  % 'r' for right, and the width of the figure area
%   \centering
%   \includegraphics[width=0.35\textwidth]{figures/msb.pdf}
%   \caption{Runtime of $\Pi_{prune}$ and $\Pi_{mask}$ in different layers. We compare different secure pruning strategies based on the BERT Base model.}
%   \label{fig:msb}
%   \vspace{-0.1in}
% \end{wrapfigure}

% \begin{figure}[h]  % 'r' for right, and the width of the figure area
%   \centering
%   \includegraphics[width=0.4\textwidth]{figures/msb.pdf}
%   \caption{Runtime of $\Pi_{prune}$ and $\Pi_{mask}$ in different layers. We compare different secure pruning strategies based on the BERT Base model.}
%   \label{fig:msb}
%   % \vspace{-0.1in}
% \end{figure}

\textbf{Complexity of $\Pi_{mask}$.} The complexity of the proposed $\Pi_{mask}$ mainly depends on the number of oblivious swaps. To prune $m$ tokens out of $n$ input tokens, $O(mn)$ swaps are needed. Since token pruning is performed progressively, only a small number of tokens are pruned at each layer, which makes $\Pi_{mask}$ efficient during runtime. Specifically, for a BERT base model with 128 input tokens, the pruning protocol only takes $\sim0.9$s on average in each layer. An alternative approach is to invoke an oblivious sort algorithm~\citep{bogdanov2014swap2,pang2023bolt} on $\left \langle \Bar{x} \right \rangle$. However, this approach is less efficient because it blindly sort the whole token sequence without considering $m$. That is, even if only $1$ token needs to be pruned, $O(nlog^{2}n)\sim O(n^2)$ oblivious swaps are needed, where as the proposed $\Pi_{mask}$ only need $O(n)$ swaps. More generally, for an $\ell$-layer Transformer with a total of $m$ tokens pruned, the overall time complexity using the sort strategy would be $O(\ell n^2)$ while using the swap strategy remains an overall complexity of $O(mn).$ Specifically, using the sort strategy to prune tokens in one BERT Base model layer can take up to $3.8\sim4.5$ s depending on the sorting algorithm used. In contrast, using the swap strategy only needs $0.5$ s. Moreover, alternative to our MSB strategy, one can also swap the encrypted mask along with the encrypted token sequence. However, we find that this doubles the number of swaps needed, and thus is less efficient the our MSB strategy, as is shown in Figure \ref{fig:msb}.

\section{Existing Protocols}
\label{app:protocol}
\noindent\textbf{Existing Protocols Used in Our Private Inference.}  In our private inference framework, we reuse several existing cryptographic protocols for basic computations. $\Pi_{MatMul}$ \citep{pang2023bolt} processes two ASS matrices and outputs their product in SS form. For non-linear computations, protocols $\Pi_{SoftMax}, \Pi_{GELU}$, and $\Pi_{LayerNorm}$\citep{lu2023bumblebee, pang2023bolt} take a secret shared tensor and return the result of non-linear functions in ASS. Basic protocols from~\citep{rathee2020cryptflow2, rathee2021sirnn} are also utilized. $\Pi_{CMP}$\citep{EzPC}, for example, inputs ASS values and outputs a secret shared comparison result, while $\Pi_{B2A}$\citep{EzPC} converts secret shared Boolean values into their corresponding arithmetic values.

\section{Polynomial Reduction for Non-linear Functions}
\label{app:b}
The $\mathsf{SoftMax}$ and $\mathsf{GELU}$ functions can be approximated with polynomials. High-degree polynomials~\citep{lu2023bumblebee, pang2023bolt} can achieve the same accuracy as the LUT-based methods~\cite{hao2022iron-iron}. While these polynomial approximations are more efficient than look-up tables, they can still incur considerable overheads. Reducing the high-degree polynomials to the low-degree ones for the less important tokens can imporve efficiency without compromising accuracy. The $\mathsf{SoftMax}$ function is applied to each row of an attention map. If a token is to be reduced, the corresponding row will be computed using the low-degree polynomial approximations. Otherwise, the corresponding row will be computed accurately via a high-degree one. That is if $M_{\beta}'[i] = 1$, $P_0$ and $P_1$ uses high-degree polynomials to compute the $\mathsf{SoftMax}$ function on token $x[i]$:
\begin{equation}
\mathsf{SoftMax}_{i}(x) = \frac{e^{x_i}}{\sum_{j\in [d]}e^{x_j}}
\end{equation}
where $x$ is a input vector of length $d$ and the exponential function is computed via a polynomial approximation. For the $\mathsf{SoftMax}$ protocol, we adopt a similar strategy as~\citep{kim2021ibert, hao2022iron-iron}, where we evaluate on the normalized inputs $\mathsf{SoftMax}(x-max_{i\in [d]}x_i)$. Different from~\citep{hao2022iron-iron}, we did not used the binary tree to find max value in the given vector. Instead, we traverse through the vector to find the max value. This is because each attention map is computed independently and the binary tree cannot be re-used. If $M_{\beta}[i] = 0$, $P_0$ and $P_1$ will approximate the $\mathsf{SoftMax}$ function with low-degree polynomial approximations. We detail how $\mathsf{SoftMax}$ can be approximated as follows:
\begin{equation}
\label{eq:app softmax}
\mathsf{ApproxSoftMax}_{i}(x) = \frac{\mathsf{ApproxExp}(x_i)}{\sum_{j\in [d]}\mathsf{ApproxExp}(x_j)}
\end{equation}
\begin{equation}
\mathsf{ApproxExp}(x)=\begin{cases}
    0  &\text{if}\ x \leq T \\
    (1+ \frac{x}{2^n})^{2^n} &\text{if}\ x \in [T,0]\\
\end{cases}
\end{equation}
where the $2^n$-degree Taylor series is used to approximate the exponential function and $T$ is the clipping boundary. The value $n$ and $T$ determines the accuracy of above approximation. With $n=6$ and $T=-13$, the approximation can achieve an average error within $2^{-10}$~\citep{lu2023bumblebee}. For low-degree polynomial approximation, $n=3$ is used in the Taylor series.

Similarly, $P_0$ or $P_1$ can decide whether or not to approximate the $\mathsf{GELU}$ function for each token. If $M_{\beta}[i] = 1$, $P_0$ and $P_1$ use high-degree polynomials~\citep{lu2023bumblebee} to compute the $\mathsf{GELU}$ function on token $x[i]$ with high-degree polynomial:
% \begin{equation}
% \mathsf{GELU}(x) = 0.5x(1+\mathsf{Tanh}(\sqrt{2/\pi}(x+0.044715x^3)))
% \end{equation}
% where the $\mathsf{Tanh}$ and square root function are computed via a OT-based lookup-table.

\begin{equation}
\label{eq:app gelu}
\mathsf{ApproxGELU}(x)=\begin{cases}
    0  &\text{if}\ x \leq -5 \\
    P^3(x), &\text{if}\ -5 < x \leq -1.97 \\
    P^6(x), &\text{if}\ -1.97 < x \leq 3  \\
    x, &\text{if}\ x >3 \\
\end{cases}
\end{equation}
where $P^3(x)$ and $P^6(x)$ are degree-3 and degree-6 polynomials respectively. The detailed coefficient for the polynomial is: 
\begin{equation*}
    P^3(x) = -0.50540312 -  0.42226581x - 0.11807613x^2 - 0.01103413x^3
\end{equation*}
, and
\begin{equation*}
    P^6(x) = 0.00852632 + 0.5x + 0.36032927x^2 - 0.03768820x^4 + 0.00180675x^6
\end{equation*}

For BOLT baseline, we use another high-degree polynomial to compute the $\mathsf{GELU}$ function.

\begin{equation}
\label{eq:app gelu}
\mathsf{ApproxGELU}(x)=\begin{cases}
    0  &\text{if}\ x < -2.7 \\
    P^4(x), &\text{if}\   |x| \leq 2.7 \\
    x, &\text{if}\ x >2.7 \\
\end{cases}
\end{equation}
We use the same coefficients for $P^4(x)$ as BOLT~\citep{pang2023bolt}.

\begin{figure}[h]
 % \vspace{-0.1in}
    \centering
    \includegraphics[width=1\linewidth]{figures/bumble.pdf}
    % \captionsetup{skip=2pt}
    % \vspace{-0.1in}
    \caption{Comparison with prior works on the BERT model. The input has 128 tokens.}
    \label{fig:bumble}
\end{figure}

If $M_{\beta}'[i] = 0$, $P_0$ and $P_1$ will use low-degree 
polynomial approximation to compute the $\mathsf{GELU}$ function instead. Encrypted polynomial reduction leverages low-degree polynomials to compute non-linear functions for less important tokens. For the $\mathsf{GELU}$ function, the following degree-$2$ polynomial~\cite{kim2021ibert} is used:
\begin{equation*}
\mathsf{ApproxGELU}(x)=\begin{cases}
    0  &\text{if}\ x <  -1.7626 \\
    0.5x+0.28367x^2, &\text{if}\ x \leq |1.7626| \\
    x, &\text{if}\ x > 1.7626\\
\end{cases}
\end{equation*}


\section{Comparison with More Related Works.}
\label{app:c}
\textbf{Other 2PC frameworks.} The primary focus of CipherPrune is to accelerate the private Transformer inference in the 2PC setting. As shown in Figure \ref{fig:bumble}, CipherPrune can be easily extended to other 2PC private inference frameworks like BumbleBee~\citep{lu2023bumblebee}. We compare CipherPrune with BumbleBee and IRON on BERT models. We test the performance in the same LAN setting as BumbleBee with 1 Gbps bandwidth and 0.5 ms of ping time. CipherPrune achieves more than $\sim 60 \times$ speed up over BOLT and $4.3\times$ speed up over BumbleBee.

\begin{figure}[t]
 % \vspace{-0.1in}
    \centering
    \includegraphics[width=1\linewidth]{figures/pumab.pdf}
    % \captionsetup{skip=2pt}
    % \vspace{-0.1in}
    \caption{Comparison with MPCFormer and PUMA on the BERT models. The input has 128 tokens.}
    \label{fig:pumab}
\end{figure}

\begin{figure}[h]
 % \vspace{-0.1in}
    \centering
    \includegraphics[width=1\linewidth]{figures/pumag.pdf}
    % \captionsetup{skip=2pt}
    % \vspace{-0.1in}
    \caption{Comparison with MPCFormer and PUMA on the GPT2 models. The input has 128 tokens. The polynomial reduction is not used.}
    \label{fig:pumag}
\end{figure}

\textbf{Extension to 3PC frameworks.} Additionally, we highlight that CipherPrune can be also extended to the 3PC frameworks like MPCFormer~\citep{li2022mpcformer} and PUMA~\citep{dong2023puma}. This is because CipherPrune is built upon basic primitives like comparison and Boolean-to-Arithmetic conversion. We compare CipherPrune with MPCFormer and PUMA on both the BERT and GPT2 models. CipherPrune has a $6.6\sim9.4\times$ speed up over MPCFormer and $2.8\sim4.6\times$ speed up over PUMA on the BERT-Large and GPT2-Large models.


\section{Communication Reduction in SoftMax and GELU.}
\label{app:e}

\begin{figure}[h]
    \centering
    \includegraphics[width=0.9\linewidth]{figures/layerwise.pdf}
    \caption{Toy example of two successive Transformer layers. In layer$_i$, the SoftMax and Prune protocol have $n$ input tokens. The number of input tokens is reduced to $n'$ for the Linear layers, LayerNorm and GELU in layer$_i$ and SoftMax in layer$_{i+1}$.}
    \label{fig:layer}
\end{figure}

\begin{table*}[h]
\captionsetup{skip=2pt}
\centering
\scriptsize
\caption{Communication cost (in MB) of the SoftMax and GELU protocol in each Transformer layer.}
\begin{tblr}{
    colspec = {c |c c c c c c c c c c c c},
    row{1} = {font=\bfseries},
    row{2-Z} = {rowsep=1pt},
    % row{4} = {bg=LightBlue},
    colsep = 2.5pt,
    }
\hline
\textbf{Layer Index} & \textbf{0}  & \textbf{1}  & \textbf{2} & \textbf{3} & \textbf{4} & \textbf{5} & \textbf{6} & \textbf{7} & \textbf{8} & \textbf{9} & \textbf{10} & \textbf{11} \\
\hline
Softmax & 642.19 & 642.19 & 642.19 & 642.19 & 642.19 & 642.19 & 642.19 & 642.19 & 642.19 & 642.19 & 642.19 & 642.19 \\
Pruned Softmax & 642.19 & 129.58 & 127.89 & 119.73 & 97.04 & 71.52 & 43.92 & 21.50 & 10.67 & 6.16 & 4.65 & 4.03 \\
\hline
GELU & 698.84 & 698.84 & 698.84 & 698.84 & 698.84 & 698.84 & 698.84 & 698.84 & 698.84 & 698.84 & 698.84 & 698.84\\
Pruned GELU  & 325.10 & 317.18 & 313.43 & 275.94 & 236.95 & 191.96 & 135.02 & 88.34 & 46.68 & 16.50 & 5.58 & 5.58\\
\hline
\end{tblr}
\label{tab:layer}
\end{table*}

{
In Figure \ref{fig:layer}, we illustrate why CipherPrune can reduce the communication overhead of both  SoftMax and GELU. Suppose there are $n$ tokens in $layer_i$. Then, the SoftMax protocol in the attention module has a complexity of $O(n^2)$. CipherPrune's token pruning protocol is invoked to select $n'$ tokens out of all $n$ tokens, where $m=n-n'$ is the number of tokens that are removed. The overhead of the GELU function in $layer_i$, i.e., the current layer, has only $O(n')$ complexity (which should be $O(n)$ without token pruning). The complexity of the SoftMax function in $layer_{i+1}$, i.e., the following layer, is reduced to $O(n'^2)$ (which should be $O(n^2)$ without token pruning). The SoftMax protocol has quadratic complexity with respect to the token number and the GELU protocol has linear complexity. Therefore, CipherPrune can reduce the overhead of both the GELU protocol and the SoftMax protocols by reducing the number of tokens. In Table \ref{tab:layer}, we provide detailed layer-wise communication cost of the GELU and the SoftMax protocol. Compared to the unpruned baseline, CipherPrune can effectively reduce the overhead of the GELU and the SoftMax protocols layer by layer.
}

\section{Analysis on Layer-wise redundancy.}
\label{app:f}

\begin{figure}[h]
    \centering
    \includegraphics[width=0.9\linewidth]{figures/layertime0.pdf}
    \caption{The number of pruned tokens and pruning protocol runtime in different layers in the BERT Base model. The results are averaged across 128 QNLI samples.}
    \label{fig:layertime}
\end{figure}

{
In Figure \ref{fig:layertime}, we present the number of pruned tokens and the runtime of the pruning protocol for each layer in the BERT Base model. The number of pruned tokens per layer was averaged across 128 QNLI samples, while the pruning protocol runtime was measured over 10 independent runs. The mean token count for the QNLI samples is 48.5. During inference with BERT Base, input sequences with fewer tokens are padded to 128 tokens using padding tokens. Consistent with prior token pruning methods in plaintext~\citep{goyal2020power}, a significant number of padding tokens are removed at layer 0.  At layer 0, the number of pruned tokens is primarily influenced by the number of padding tokens rather than token-level redundancy.
%In Figure \ref{fig:layertime}, we demonstrate the number of pruned tokens and the pruning protocol runtime in each layer in the BERT Base model. We averaged the number of pruned tokens in each layer across 128 QNLI samples and then tested the pruning protocol runtime in 10 independent runs. The mean token number of the QNLI samples is 48.5. During inference with BERT Base, input sequences with small token number are padded to 128 tokens with padding tokens. Similar to prior token pruning methods in the plaintext~\citep{goyal2020power}, a large number of padding tokens can be removed at layer 0. We remark that token-level redundancy builds progressively throughout inference~\citep{goyal2020power, kim2022LTP}. The number of pruned tokens in layer 0 mostly depends on the number of padding tokens instead of token-level redundancy.
}

{
%As shown in Figure \ref{fig:layertime}, more tokens are removed in the intermediate layers, e.g., layer $4$ to layer $7$. This suggests there is more redundant information in these intermediate layers. 
In CipherPrune, tokens are removed progressively, and once removed, they are excluded from computations in subsequent layers. Consequently, token pruning in earlier layers affects computations in later layers, whereas token pruning in later layers does not impact earlier layers. As a result, even if layers 4 and 7 remove the same number of tokens, layer 7 processes fewer tokens overall, as illustrated in Figure \ref{fig:layertime}. Specifically, 8 tokens are removed in both layer $4$ and layer $7$, but the runtime of the pruning protocol in layer $4$ is $\sim2.4\times$ longer than that in  layer $7$.
}

\section{Related Works}
\label{app:g}

{
In response to the success of Transformers and the need to safeguard data privacy, various private Transformer Inferences~\citep{chen2022thex,zheng2023primer,hao2022iron-iron,li2022mpcformer, lu2023bumblebee, luo2024secformer, pang2023bolt}  are proposed. To efficiently run private Transformer inferences, multiple cryptographic primitives are used in a popular hybrid HE/MPC method IRON~\citep{hao2022iron-iron}, i.e., in a Transformer, HE and SS are used for linear layers, and SS and OT are adopted for nonlinear layers. IRON and BumbleBee~\citep{lu2023bumblebee} focus on optimizing linear general matrix multiplications; SecFormer~\cite{luo2024secformer} improves the non-linear operations like the exponential function with polynomial approximation; BOLT~\citep{pang2023bolt} introduces the baby-step giant-step (BSGS) algorithm to reduce the number of HE rotations, proposes a word elimination (W.E.) technique, and uses polynomial approximation for non-linear operations, ultimately achieving state-of-the-art (SOTA) performance.
}

{Other than above hybrid HE/MPC methods, there are also works exploring privacy-preserving Transformer inference using only HE~\citep{zimerman2023converting, zhang2024nonin}. The first HE-based private Transformer inference work~\citep{zimerman2023converting} replaces \mysoftmax function with a scaled-ReLU function. Since the scaled-ReLU function can be approximated with low-degree polynomials more easily, it can be computed more efficiently using only HE operations. A range-loss term is needed during training to reduce the polynomial degree while maintaining high accuracy. A training-free HE-based private Transformer inference was proposed~\citep{zhang2024nonin}, where non-linear operations are approximated by high-degree polynomials. The HE-based methods need frequent bootstrapping, especially when using high-degree polynomials, thus often incurring higher overhead than the hybrid HE/MPC methods in practice.
}



\end{document}
\endinput
%%
%% End of file `sample-authordraft.tex'.
