\newpage

\section{Workshop Images}
\label{appendix:workshops}
\begin{figure}[H]
    \centering
    \includegraphics[scale=0.45]{images/participants_masked.png}
    \caption{Workshop participants}
    \label{fig:workshopParticipants}
    \Description{A picture of workshop participants.}
\end{figure}

\begin{figure}
    \centering
    \includegraphics[scale=0.25]{images/postits.png}
    \caption{Notes generated by participants}
    \label{fig:workshopPostIts}
    \Description{Post-It notes generated by participants.}
\end{figure}
% \end{minipage}%
% \begin{minipage}{.5\textwidth}

\section{Workshop Design}
% \label{appendix:workshops}

We began with two workshops (Fig. \ref{fig:workshops}) with two social service agencies, A and B, in November 2023. This provided a high-level understanding of the social service sector pertaining to possible AI solutions and possible associated implementation challenges. A total of 27 SSPs were involved, hailing from different roles and experience levels (Table \ref{tab:workshopParticipants}),
giving us a wide range of perspectives to draw upon. 

Each workshop lasted around 90 minutes, comprising a briefing and discussion session. In the briefing, we introduced LLMs in the form of ChatGPT\footnote{At the time of the workshops (October 2023), ChatGPT was the main LLM widely accessible to the public.}. We demonstrated AI's strengths (e.g., brainstorming, ideation \cite{gero2023social, shaer2024ai}, summarisation \cite{wang2023chatgpt}, formatting and rewriting of text \cite{bhattaru2024revolutionizing}) and weaknesses, including hallucination \cite{maynez2020faithfulness} and limited context windows \cite{liu2024lost}. This ensured that all participants, including those with only a cursory understanding or awareness of ChatGPT\footnote{ChatGPT had been available for around 10 months at this point. Some workers had tried using it, even experimenting with it for their work, while others had never used it themselves.} had a baseline understanding of LLMs.

Next, we split participants into tables of 4-6 people each for small-group discussions. Here, we sought to encourage participants to freely express their thoughts and build on one another's ideas \cite{bonnardel2020brainstorming}. Participants were given stacks of Post-Its and asked a series of questions. For each question, they spent around five minutes writing their responses on individual Post-Its. Then, they pasted their Post-Its on a board and clustered similar ideas together, allowing common themes to emerge. To kickstart the discussion with more familiar topics, we asked the participants about their work: \textit{"What does a day at work look like for you?"}, \textit{"What difficulties have you recently faced in your work?"}. We then brought AI into the picture:\textit{ "How can AI help you in your work?"}, \textit{"What expectations do you have for an AI-powered social service tool?"}. Having elicited some ideas, we then introduced two rounds of 6-8-5 sketching where participants quickly drew out concepts of possible AI tools that could help them. Finally, we asked, \textit{"What are some concerns you might have with using AI in your work?"}.

After the workshops, the researchers transferred the physical Post-Its to a Miro board for further analysis and deduced themes from the clustering done during the workshops.

\begin{table}[H]
\centering
\begin{tabular}{cc|cc}
    \toprule
    \multicolumn{2}{c|}{\textbf{Agency A}}&\multicolumn{2}{c}{\textbf{Agency B}} \\
    \midrule
    Code&Role&Code&Role\\
    \midrule
    TS1 & Snr Social Worker & CD1 & Director\\
    TS2 & Social Worker & CY1 & Youth Work \\
    TS3 & Social Worker & CY2 & Youth Work \\ 
    TS4 & Social Worker & CY3 & Youth Work \\
    TS5 & Social Worker & CP1 & Youth Projects\\
    TS6 & Social Worker & CP2 & Youth Projects\\
    TS7 & Social Worker & CP3 & Youth Projects\\
    TS8 & Social Worker & CP4 & Youth Projects \\
    TS9 & Social Worker & CC1 & Counsellor \\
    TS10 & Social Worker & CC2 & Counsellor \\
    TD1 & Exec. Director & CC3 & Counsellor\\
    TD2 & Snr Director & CC4 & Counsellor\\
    TP1 &  Psychologist & CC3 & Counsellor\\
    && CC4 & Counsellor\\
\bottomrule   
\end{tabular}
\caption{Workshop Participants from Agency A and Agency B}
\label{tab:workshopParticipants}
\Description{Table describing workshop Participants from Agency A and Agency B. There are 13 participants from Agency A, mostly social workers, and 14 participants from Agency B, a mix of youth workers, youth project workers, and counsellors.}
\end{table}

\section{Workshop Findings}
\label{appendix:workshopFindings}

We categorise our findings into three main themes: Difficulties faced by SSPs in their daily work, possible AI-assisted solutions, and potential risks of using AI systems.

% Comment: Below are the themes that discovered WITH participants right? If it will be better to write an overview at first?

\subsection{Difficulties in Social Service Practice}
\label{sec:stage1difficulties}

% The most common frustration that emerged was documentation. Across the social work workflow, the constant need to document "anything and everything" (TS3) is a major timesink. Reports must be written in a systematic manner in many different categories like bio-psychological scales (TS2) or numerous risk factors (TS3). Further, the same content may need to be repackaged for different stakeholders (TD2, TS3) like colleagues or other agencies, creating tedious duplicate work. Beyond repackaging and reformatting information, documentation also involves analysis of client profiles. Workers may need to synthesise information from many pages of case notes to formulate assessments and justify a certain judgement (TS3), a cognitively challenging task. 

% Beyond being time-consuming, documentation is also not a task that is in the innate skillset of a social worker. Senior personnel desired greater incorporation of theoretical concepts in their staff's reports (TD1, TD2), observing them "knowing what to do" in practice but being unable to translate that well into writing (TD1). This is especially relevant when writing assessments of a client, where workers ideally select from a "toolkit" of frameworks to produce a justified and sound assessment (TS3).

% Aside from documentation, case formulation is another key step integral to case work. Piecing together disparate information from across many pages of case notes is time-consuming and prone to workers inadvertently missing certain key information (TD1). This is particularly tricky during a live session with the client, where the non-linear process of information discovery (CC2) means workers are constantly trying to organise their "all over the place" (CC5) thoughts to understand a case holistically before proceeding to ask the client the most pertinent questions. This may inadvertently cause them to miss certain key insights or red flags (TS2, TD1) only evident from looking at the gathered information as a whole.

Echoing past sentiments \cite{singer_ai_2023, tiah_can_2024}, participants cited the \textit{need to document "anything and everything"} (TS3) as a major pain point. SSPs have to write systematic reports in different structured formats (e.g. bio-psychological scales (TS2), risk factor assessments (TS3)) and repack the same content for different stakeholders (TD2, TS3) like colleagues or other agencies, creating tedious duplicate work. Additionally, some SSPs struggle with putting their ideas into text, "knowing what to do" in practice but being unable to translate that well into writing (TD1). This results in a lack of ability to incorporate theoretical concepts in reports, something noted particularly by senior personnel (TD1, TD2). 
% This is especially relevant when writing assessments of a client, where workers should ideally select from a "toolkit" of frameworks to produce a justified and sound assessment (TS3). 
% MH: difficulty 1: doc - interesting! just tried to summarize the mainpoints: 1) reduce time; 2) help junior staffs to proofread the report writing? Will it be better to seperated to two paragraphs..?

Participants also noted the \textit{challenges in case formulation}, the synthesising of information to craft assessments and justify a certain judgement (TS3). Piecing together disparate information across many pages of case notes is cognitively challenging and time-consuming (TD1). This is especially tricky during direct interaction with a client, where the non-linear process of information discovery (CC2) leaves workers constantly trying to organise their "all over the place" (CC5) thoughts to understand a case holistically before proceeding to ask the client the most pertinent questions. This may inadvertently cause them to miss certain key insights or red flags (TS2, TD1) only evident from looking at the gathered information as a whole.

\subsection{Possible Solutions}
\label{sec:stage1solutions}

Having elucidated these difficulties, participants floated numerous ideas for how AI could help with these issues. Many participants expressed a desire for a tool to help with \textit{manual labour}- for example, turning point-form notes into formal reports (CP1, CP2, CC1, TS2). TD1 shared how SSPs frequently used the "5Ps" framework\footnote{Presenting problem, Predisposing factors, Precipitating factors, Perpetuating factors, Protective factors} to organise case information into different, logically linked categories to aid subsequent analysis. This is one possible, largely menial and procedural task that an LLM could perform with high accuracy. Other forms of documentation mentioned, like the Data, Intervention, Assessment, Plan (DIAP), or Bio-Psychosocial-Spiritual (BPSS) formats, were also common report structures that could be automatically generated. 

More cognitively-intensive, \textit{mental labour} tasks were also considered, such as the generation of assessments (TS3, TS4, TS6, TS7). This requires fitting a client's information into theoretical frameworks to produce well-grounded assessments and analyses. AI tools could improve both the speed at which relevant information is synthesised and categorised, and the quality of the output through use of technical, industry-standard terms, something that is often lacking in many SSPs (TD1, TD2). Another critical task is intervention planning, where workers follow established models like Cognitive Behaviour Therapy (CBT) or Solution-Focused Brief Therapy (SFBT) and craft a plan for their clients based on the guiding steps in each model. An AI tool could allow the rapid generation of numerous possible interventions (CP1), leaving the user to pick and choose from the suggestions offered (TD1).


\subsection{Areas of Risk}
\label{sec:stage1risks}
In line with past work on the dangers of AI adoption, participants echoed certain concerns about the use of AI tools. Privacy of client data was unilaterally mentioned by all, a universal concern core to the social sector. Storage of personal client information (TS3, TP1) was a particular worry, and TS1 noted the possibility of workers entering sensitive information into a system by mistake. On the staff competency front, a common concern was the consequences of AI taking on an increasing part of the worker's job scopes. Specifically for analytical or ideative tasks, some seniors (TD1, TD2, TS1) were concerned about the loss in critical thinking skills of junior workers who might become overreliant on the tool to perform their work for them. Multiple participants also raised the possibility of inaccurate outputs from an AI system, particularly risky when less experienced workers fail to tell when the AI's output might be suboptimal and proceed to adopt its suggestions anyway. 

\section{Thematic Analysis}
\label{appendix:analysis}

To analyze the session transcripts, we performed qualitative thematic analysis \cite{braun2006using}. Qualitative thematic analysis is a method for identifying and analyzing key patterns and themes within interview and focus group data that affords a flexible and iterative coding process \cite{braun2006using}, making it particularly suited to the iterative nature of PD studies like this one. Due to the nascence of AI in the social service sector and the paucity of existing theoretical frameworks specific to the context of our study, we adopted an inductive, bottom-up approach of coding, which allowed us to uncover emergent themes in a data-driven manner and surface the latent thoughts, feelings and attitudes of SSPs in their own voice, rather than attempting to fit the data within a preexisting framework or being influenced by researchers' prior analytic preconceptions. 

Coding of the transcripts was carried out sequentially and iteratively. First, four researchers independently coded two transcripts and met to discuss and share the codes. Subsequently, two researchers then independently developed a codebook based on identified codes and met to combine the two into a unified codebook of themes and subthemes. The codebook was shared with the other two researchers alongside explanatory memos and used to code the remaining six transcripts, which were equally divided among all four researchers, with at least two researchers coding each transcript. The researchers then met two additional times to discuss the codes and resolve disagreements, until consensus was met. Finally, two researchers met in multiple rounds to collaboratively revise and refine the codes, as new themes were discovered and old ones were combined or retired. These themes form the final framework that informs our findings.