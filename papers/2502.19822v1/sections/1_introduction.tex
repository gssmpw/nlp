\section{Introduction}

The social service sector is crucial for a just and humane society, addressing various social injustices and supporting the well-being of individuals and communities \cite{difranks2008social}. Social service agencies render critical services such as client home visits, case analysis, and intervention planning, aiming to aid vulnerable clients in the best way possible. Social service practitioners (SSPs) must consider the myriad interacting factors in clients' social networks, physical environment, and intrapersonal cognitive and emotional systems, in the context of clients' beliefs, perceptions, and desires, all while respecting individual worth and dignity \cite{rooney2017direct}. This requires extensive training and hands-on experience, which may be challenging given the endlessly varying situations a worker may encounter \cite{rooney2017direct}.
% social work is hard -> needs experiences

These demands put a strain on the limited pool of social service manpower \cite{cho2017determinants}. SSPs grapple with time-consuming data processing \cite{singer_ai_2023, tiah_can_2024} and administrative \cite{meilvang_working_2023} tasks, compounded by the psychologically stressful nature of the job \cite{kalliath2012work}. Newer workers may experience challenging, unfamiliar client interactions, while experienced workers also face the added burden of mentoring junior workers. Current artificial intelligence (AI) systems aim to alleviate this through \textit{decision-support tools} \cite{gambrill2001need, de2020case, brown2019toward} which provide statistically validated assessments of client conditions \cite{gillingham2019can, van2017predicting}, thereby raising service quality and consistency. However, attempts to integrate AI systems in social service have faced a myriad of difficulties: mistrust in and uncertainty about technology \cite{gambrill2001need, brown2019toward}, worker confusion due to unclear organisational direction \cite{kawakami2022improving}, AI systems' inability to include local and contextual knowledge, and fear of AI potentially replacing jobs. As such, many attempts have been met with failure and inadequacies \cite{saxena2021framework, kawakami2022improving}.

Recent technological advancements, in particular generative AI (GenAI), have the potential to play a greater, transformative role in the social service sector. GenAI systems are capable of a wider range of tasks, including analysing written case recordings, performing qualitative risk assessments, providing crisis assistance, and aiding prevention efforts \cite{reamer_artificial_2023}.  However, despite their advantages, these systems may exacerbate existing concerns with AI, risk overreliance \cite{van2023chatgpt} and potentially erode the human skills and values \cite{littlechild2008child, oak2016minority} core to the social service profession. 
% \cite{kawakami2022improving, de2020case} and adjacent healthcare \cite{devaraj2014barriers, elwyn2013many} sectors investigates systems that are vastly more primitive and limited in scope. 
As a novel technology, the efficacy of GenAI systems are still largely under-explored and untested in extant research \cite{gambrill2001need, de2020case, brown2019toward}, thus necessitating deeper scrutiny of AI technologies in social service. We therefore ask, \textbf{RQ1:} How can GenAI help SSPs in their daily work, and \textbf{RQ2:} How can GenAI help SSPs at an organisational level?


To address this gap, we conducted a two-stage study with 51 SSPs, to better understand practitioners' perspectives on implementing AI in the social service sector. 
We first ran preliminary workshops with 27 SSPs to co-design a prototype GenAI tool, then conducted user testing and focus group discussions (FGDs) with 24 additional SSPs. Through this study, we offer an empirically-grounded understanding of how workers use and perceive AI, shedding new light on the challenges and opportunities of GenAI in the social service sector and beyond. We conclude with suggestions for social service organisations and design and usage guidelines for AI tools in the sector.

% The social service sector plays a critical role in promoting social justice and well-being. Practitioners face complex demands, including understanding clients’ multifaceted social, emotional, and environmental contexts while adhering to principles of dignity and respect. However, limited resources, time-consuming administrative tasks, and high psychological strain significantly impact their ability to provide consistent and effective support. These challenges are further magnified in settings with less experienced staff who require guidance from senior practitioners.

% GenAI, a promising technological advancement, has the potential to alleviate these constraints by automating labor-intensive tasks like documentation and enabling data-driven decision-making. Despite its potential, the adoption of AI tools in social service has been met with skepticism due to concerns about trust, transparency, and ethical considerations, such as the risk of eroding critical human skills and values central to the profession. Additionally, existing AI systems often struggle to incorporate local and contextual knowledge, further limiting their applicability in dynamic social service environments.

% This study explores how GenAI can be responsibly integrated into social service practices, focusing on both individual and organizational levels. Through a two-stage participatory design approach, we engaged 51 SSPs to co-design and evaluate an AI tool tailored to their needs. By highlighting the interplay between human expertise, professional ethos, and AI capabilities, this work aims to bridge the gap between technological possibilities and the nuanced realities of social work practice.

