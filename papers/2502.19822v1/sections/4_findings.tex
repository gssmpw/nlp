\section{Findings}
\label{sec:stage2findings}

\subsection{Findings from Stage 1: Workshop and Co-Design}
\label{findings:workshop}
% Based on the codebook and themes, a final framework was developed, summarizing the Values, Requirements, and Attitudes that social workers have towards the application of AI tools in the social work practice.

Through the workshops, participants revealed a few major aspects of their work that could be assisted by AI. For brevity, detailed findings are in Appendix \ref{appendix:workshopFindings}. In brief, the main findings are below.

Documentation was a major pain point, with our participants needing to document "anything and everything", writing systematic reports in different structured formats (e.g. bio-psychological scales, risk factor assessments) or repacking the same content for different stakeholders like colleagues or other agencies, creating tedious duplicate work. Participants hence expressed a desire for a tool to help with manual labour, like turning point-form notes into formal reports, in various formats such as the 5Ps, DIAP, or BPSS\footnote{5Ps: Presenting problem, Predisposing factors, Precipitating factors, Perpetuating factors, Protective factors; DIAP: Data, Intervention, Assessment, Plan; BPSS: Bio-Psycho-Social-Spiritual}. Some senior personnel also suggested that AI could help workers incorporate theoretical concepts into their written work. 

% \begin{minipage}{0.5\linewidth}
% \begin{table}[H]
% \centering
% \label{tab:agencyA}

% % --- Row 1: Centres 1 & 2 ---
% \begin{tabular}{ll|ll}
% \toprule
% \multicolumn{2}{c|}{\textbf{Centre 1}} & \multicolumn{2}{c}{\textbf{Centre 2}} \\
% \midrule
% Code & Role        & Code & Role        \\
% \midrule
% W1   & Social Worker & W9   & Social Worker  \\
% W2   & Social Worker & W10  & Social Worker  \\
% W3   & Social Worker & C1   & Snr Counsellor \\
% \midrule
% S1   & Supervisor    & S7   & Senior SW      \\
% S2   & Supervisor    & W7   & Social Worker  \\
% S3   & Supervisor    & W8   & Social Worker  \\
% \midrule
%       &              & S6   & Senior SW      \\
%       &              & W5   & Social Worker  \\
% \bottomrule
% \multicolumn{2}{c|}{\textbf{Centre 3}} & \multicolumn{2}{c}{\textbf{Centre 4}} \\
% \midrule
% Code & Role        & Code & Role         \\
% \midrule
% S4   & Supervisor  & S8   & Senior SW    \\
% S5   & Supervisor  & S9   & Senior SW    \\
% W4   & Social Worker & S10  & Senior SW   \\
% \midrule
%       &             & W11  & Social Worker \\
%       &             & W12  & Social Worker \\
%       &             & W13  & Social Worker \\
% \bottomrule
% \end{tabular}

% \caption{Focus Group Participants from Agency A}
% \end{table}
% \end{minipage}
% \begin{minipage}[HT]{0.5\linewidth}
% \begin{figure}[H]
% \centering
% \includegraphics[width=2.15in]{images/prototypetool.png}
% \caption{Prototype AI Tool}
% \Description{A screenshot of our prototype AI tool. It has a text box for entry at the top, a list of radio buttons to select output modalities, a text box for extra instructions to the model a user might want to input, and a "generate" button.}
% \label{fig:participantsAndTool}
% \end{figure}
% \end{minipage}

\begin{table}
\centering
\label{tab:agencyA}

% --- Row 1: Centres 1 & 2 ---
\begin{tabular}{ll|ll}
\toprule
\multicolumn{2}{c|}{\textbf{Centre 1}} & \multicolumn{2}{c}{\textbf{Centre 2}} \\
\midrule
Code & Role        & Code & Role        \\
\midrule
W1   & Social Worker & W9   & Social Worker  \\
W2   & Social Worker & W10  & Social Worker  \\
W3   & Social Worker & C1   & Snr Counsellor \\
\midrule
S1   & Supervisor    & S7   & Senior SW      \\
S2   & Supervisor    & W7   & Social Worker  \\
S3   & Supervisor    & W8   & Social Worker  \\
\midrule
      &              & S6   & Senior SW      \\
      &              & W5   & Social Worker  \\
\bottomrule
\multicolumn{2}{c|}{\textbf{Centre 3}} & \multicolumn{2}{c}{\textbf{Centre 4}} \\
\midrule
Code & Role        & Code & Role         \\
\midrule
S4   & Supervisor  & S8   & Senior SW    \\
S5   & Supervisor  & S9   & Senior SW    \\
W4   & Social Worker & S10  & Senior SW   \\
\midrule
      &             & W11  & Social Worker \\
      &             & W12  & Social Worker \\
      &             & W13  & Social Worker \\
\bottomrule
\end{tabular}

\caption{Focus Group Participants from Agency A}
\end{table}

\begin{figure}
\centering
\includegraphics[width=2.15in]{images/prototypetool.png}
\caption{Prototype AI Tool}
\Description{A screenshot of our prototype AI tool. It has a text box for entry at the top, a list of radio buttons to select output modalities, a text box for extra instructions to the model a user might want to input, and a "generate" button.}
\label{fig:participantsAndTool}
\end{figure}



Participants also noted challenges with case formulation, in which piecing together disparate information across many pages of case notes to craft and justify an assessment is cognitively challenging and time-consuming, particularly during direct interaction with a client. This may inadvertently cause them to miss certain key insights or red flags only evident from looking at the gathered information as a whole. Participants therefore also suggested systems for generating case assessments, to improve output quality and adherence to industry-standard terms, and intervention planning, to generate possible plans for helping the client that the practitioner could consider and choose from.

% Another critical task is intervention planning, where workers follow established models like Cognitive Behaviour Therapy (CBT) or Solution-Focused Brief Therapy (SFBT) and craft a plan for their clients based on the guiding steps in each model. An AI tool could allow rapid generation of numerous possible interventions (CP1), leaving the user to pick and choose from the suggestions offered (TD1).

% This requires fitting a client's information into theoretical frameworks to produce well-grounded assessments and analyses. AI tools could improve both the speed at which relevant information is synthesised and categorised, and the quality of the output through use of technical, industry-standard terms, something that is often lacking in many SSPs

% In line with past work on the dangers of AI adoption [cite], participants echoed certain concerns about the use of AI tools.

Finally, participants also noted a few potential areas of risk. Privacy of client data was unilaterally mentioned by all, a universal concern core to the social sector. The storage of personal client information was a significant concern, particularly regarding the possibility of workers mistakenly entering sensitive information into the system. Regarding staff competency, a common concern was the impact of AI taking on an increasing part of the worker's job scopes. Specifically for analytical or ideative tasks, some seniors were concerned about the loss of critical thinking skills of junior workers who might become overreliant on the tool to perform their work for them. Multiple participants also raised the possibility of inaccurate outputs from an AI system, particularly risky when less experienced workers fail to tell when the AI's output might be suboptimal and proceed to adopt its suggestions anyway. 

\subsection{Findings from Stage 2: Focus Group Discussions}

\subsubsection{Documentation} 
\label{subsubsec:discussionuses}

Participants found many applications of GenAI in helping with multiple writing-focused tasks in the social service sector, such as summarising intake information, formatting case recordings, and writing reports. Participants generally were happy to embrace AI for such purposes; for documentation tasks such as writing case reports, the tool's outputs were largely in line with what they required, allowing them to simply "copy and paste" (W1, W2, W13) the outputs for direct use. This was, in a large way, down to how much of our participants' regular daily work focused on consistently structured, fixed-format reports. One strength evident in GPT-4 and our prototype was its ability to consistently follow instructions to produce outputs in a desired format, such as the 5Ps format (S1, S3, W1). W8, for instance, quoted, "\textit{being new, it really helps in categorising these items... I like the fact that it segregates all the [different categories]}".

Even when the output was imperfect, participants often expressed a willingness to work around these errors and make manual corrections where needed. For instance, W2 suggested they would "sift through and pick out" the more relevant parts of an overly lengthy report, while W1 would "\textit{copy and paste, then amend if I need to amend}" when the tool misinterpreted a nickname given by the worker to the client.


% S9: "I like how it is to have a sub-titles (sub-headings) and things like that"

\subsubsection{Brainstorming}

Workers felt that the structured and detailed outputs of the tool encouraged and facilitated their cognitive processes in analysing a case. W8 felt the way the tool categorised the issues in their client's case pushed them to "think more" about how they viewed it. In a similar vein, looking at a generated CBT assessment prompted W5 to consider "\textit{certain things also that I should probably look into}", which they might otherwise have overlooked. W11 commented on how the "very in-depth" explanations given by the tool helped them "expand on what they already have". Even a senior worker, S4, called the tool's analysis of a case "\textit{really helpful - it's expanding my perspective already}".

Junior workers in particular appreciated the guidance from the tool, especially with tasks they were less familiar with. For instance, W7 had not attended formal CBT training, and thus felt the tool gave them "some idea where to start" in formulating a CBT intervention plan for their client. W4 also liked having sample interventions from the tool; being a newer worker, they were uncertain of which plan of action to take, so the tool gave them a "better understanding" of the different ways to move forward with the case. Meanwhile, more generally, the tool helped with guiding workers towards formulating a course of action, such as by suggesting interventions they might not have thought of (W4) and thereby prompting newer staff with a "direction" to work towards (W5), or by being "very useful" in helping new workers prepare for sessions with clients (W9).

Supervisors, too, agreed that the tool was useful for junior workers. S4 reiterated that it could "expand the worker's perspective", and C1 called it "a good start [to help] staff think about" the case if they "got stuck" with something. S6 cited the example of how an SFBT output provided a "really good foundation" for questions for workers to ask their clients. S6 also felt the tool provided a good framework and guideline, suggesting it as a way to "polish" newer workers' skills: "\textit{For all those who are really new and do not really know how to formulate interventions or theory support and all that. I think it's quite useful... or, if they are really lost, then they can probably try the different things that are written here}." 

\subsubsection{Supervision}

The use of AI to aid in supervision emerged as a key theme. Supervision sessions consist of junior workers discussing cases with their supervisors, to refine and improve their assessment of the client. Given this, the ability of AI to quickly generate lists of ideas provided useful starting points for discussion and reflection with supervisees (C1). Many (W4, W5) suggested that the tool provided useful intervention suggestions so that workers could "ask their supervisors like, maybe, you know, maybe I can try this" (S6). From the supervisors' perspectives, the tool helped to improve and expedite the supervision process by prompting them with questions to ask supervisees: "I can bring [this list of exception questions\footnote{Asking "exception questions" to clients is a technique used in Solution-Focused Brief Therapy.}] to supervision to see whether my supervisee has used these questions... I can ask my supervisee, okay, if you have to ask this exception question to the client, how comfortable do you feel? So we can have that discussion" (C1). S4 and S7, meanwhile, highlighted the AI's ability to quickly "concretise theoretical models" to build on during sessions. 

Some even suggested that the tool could itself serve as a supervisor to help newer workers. W9 called the tool a "readily-available supervisor to get us thinking," referencing how their supervisors frequently prompted them with questions to think about their case more. W4 meanwhile felt the tool could help them move forward with a case "without having to consult with their supervisor", in instances where they were uncertain of how to proceed.
% One unexpected use of AI that emerged prominently was as a supervisory aid. Our participants mainly described supervision sessions as senior and junior workers discussing how to formulate and proceed with a case further. With this, the suggested interventions or possible solutions generated by our AI system served as good discussion points. 


\subsubsection{Concerns and Issues}

Participants raised a few issues with using AI in their work. Agency A's focus on providing family service meant that they prioritised addressing safety and risk concerns (W2), such as possible self-harm, suicide, or harm towards others. Participants were "very particular about... risks" (S9), treating it as their top priority (W1, S10) and "at the front" for all client interventions and risk assessments (S2, S9). Preventing imminent physical and psychological harms such as incarceration, abuse, and addiction were therefore cited as "non-negotiables" (W4) and primary risk factors (W2). Thus, they expressed concern when the tool "didn't exactly highlight" (W2) or entirely omitted (W1) safety risks in its output, such as in assessing a case of intra-family conflict: "\textit{The [risk of] violence is not highlighted. Where is the violence?}" (W1).

There were also worries over overuse and overreliance on AI. Participants were divided on this issue. The centre director (D1) quoted, "\textit{One of our concerns is... [will using this AI] actually disable our ability to make assessments?}" S3 agreed about the risk of over-reliance by "spoonfeeding" junior workers, and S9 noted the need for workers to "\textit{still use their brain... otherwise... they just rely on this}." S9 also worried about how junior workers might fail to recognise suboptimal AI outputs and be misled as a result. As a result, some emphasised the need for a balance between AI use and human intervention. S10 remarked, \textit{"It's the supervisor's role to keep emphasising to the supervisee, to not be married to this assessment"}, and that the AI's outputs were often "just guidelines" rather than a gospel to be followed. S9 agreed that "\textit{the expectation [for use] needs to be very clearly communicated}", highlighting the need for careful and judicious implementation of any AI system. 

However, others felt this to not be a major problem, due to the inherent focus of the profession on face-to-face client interactions. On practitioners blindly following AI recommendations, S7 commented, "\textit{it will eventually be clear that this was the recommendation of [the tool] but then you went and did something else... and then you'll see that it's just a mess in that way.}" S3 agreed: "\textit{When they do their work, it is mostly real-time. You don't go back to pen and paper and start to develop [a solution], but it's more about what is presented to them immediately [and how they react].}"

Finally, there were multiple instances where participants found the AI's output to be inadequate. These often centred around the tool failing to recognise more subtle factors in a given case that would affect the preferred course of action. For example, when the system recommended a family therapy workshop, W5 recognised that the reluctance of their client to embrace outside help made such an intervention unsuitable. In another case, when the tool recommended CBT to aid a client, W11 judged that their client lacked the intellectual capacity for such therapy to help. Knowledge of local contextual factors was another common point. S6, for instance, commented on how "\textit{there are a few parenting workshops available [here in our country] and [we know] how they are structured}," so they would be able to assess which options were suitable for their client in a way that the AI, lacking local knowledge, would not. W5 also commented on the importance of practitioners' instincts and experience: \textit{"Many of us who have been in the sector before, probably were like, this won't work."}