\begin{abstract}
In social service, administrative burdens and decision-making challenges often hinder practitioners from performing effective casework. Generative AI (GenAI) offers significant potential to streamline these tasks, yet exacerbates concerns about overreliance, algorithmic bias, and loss of identity within the profession. We explore these issues through a two-stage participatory design study. We conducted formative co-design workshops (\textit{n=27}) to create a prototype GenAI tool, followed by contextual inquiry sessions with practitioners (\textit{n=24}) using the tool with real case data. We reveal opportunities for AI integration in documentation, assessment, and worker supervision, while highlighting risks related to GenAI limitations, skill retention, and client safety. Drawing comparisons with GenAI tools in other fields, we discuss design and usage guidelines for such tools in social service practice.
\end{abstract}