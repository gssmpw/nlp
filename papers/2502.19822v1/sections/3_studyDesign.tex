\section{Study Design}
\label{sec:stage1design}


% \begin{table}
%   \caption{Workshop Participants from Agency A and Agency B}
%   \label{tab:workshopParticipants}
%   \begin{tabular}{cc|cc}
%     \toprule
%     \multicolumn{2}{c|}{\textbf{Agency A}}&\multicolumn{2}{c}{\textbf{Agency B}} \\
%     \midrule
%     Code&Role&Code&Role\\
%     \midrule
%     \multicolumn{2}{c|}{Group 1}& \multicolumn{2}{c}{Group 1} \\
%     \midrule
%     TS1 & Senior Social Worker & CD1 & Director\\
%     TS4 & Social Worker & CY1 & Youth Work Services\\
%     TS5 & Social Worker & CY2 & Youth Work Services\\ 
%     TS6 & Social Worker & CY3 & Youth Work Services\\
%     TS7 & Social Worker & CP1 & Youth Projects\\
%     && CP2 & Youth Projects\\
%     && CP3 & Youth Projects\\
%     \midrule
%     \multicolumn{2}{c|}{Group 2}&\multicolumn{2}{c}{Group 2} \\
%     \midrule
%     TD1 & Executive Director & CC1 & Counsellor\\
%     TD2 & Senior Director & CC2 & Counsellor\\
%     TS2 & Social Worker & CC3 & Counsellor\\
%     TS3 & Social Worker & CC4 & Counsellor\\
%     TP1 & Psychologist & CP4 & Youth Projects\\
%   \bottomrule
% \end{tabular}
% \end{table}

% \begin{table}
%   \caption{Other Participants from Agency A and Agency B}
%   \label{tab:otherParticipants}
%   \begin{tabular}{ccl}
%     \toprule
%     Code&Role\\
%     \midrule
%     \multicolumn{2}{c}{\textbf{Agency A}} \\
%     \midrule
%     TS8 & Social Worker\\
%     TS9 & Social Worker\\
%     TS10 & Social Worker\\
%     \midrule
%     \multicolumn{2}{c}{\textbf{Agency B}} \\
%     \midrule
%     CC5 & Counsellor\\
%   \bottomrule
% \end{tabular}
% \end{table}

% \begin{figure}
% \begin{minipage}{.5\textwidth}

% \end{minipage}
% \end{figure}

We engaged in a two-stage study: a formative, participatory design study to understand the opportunities and challenges perceived by our participants and to co-design a GenAI tool, followed by an evaluative contextual inquiry to assess the effectiveness of the resulting tool. We partnered with two local, government-funded social service agencies (SSAs) in a Southeast Asian country that had expressed interest in adopting GenAI. Agency A focused on family-oriented casework, handling mostly walk-in clients and taking them through the full process \cite{rooney2017direct} of exploration, assessment, implementation, and eventually termination. Agency B worked more with schools and youths, partnering with education institutes to render assistance to children or teenagers in need. Both agencies used English as a working language and for all official documentation. In a small proportion of cases, Agency A's client interactions took place in a different language that was more comfortable for the client; in these instances, workers would either record their notes in English or manually translate them back into English before taking them back to the agency for further work. All participant interactions in this study were also conducted in English.

% \subsection{Workshops}

In \textbf{stage 1, the co-design phase}, we conducted two 90-minute long workshops with agencies A and B in November 2023. A total of 27 SSPs were involved, hailing from different roles and experience levels\footnote{Pictures of the workshops are in Appendix \ref{appendix:workshops}.}. We aimed to understand 1) the nature of the day-to-day work that our participants performed and what opportunities they perceived for using AI to help with it, and 2) the perceived risks and challenges of AI use to shape the design of the second phase of our study. In each workshop, we briefly introduced LLMs in the form of ChatGPT\footnote{At the time of the workshops (October 2023), ChatGPT (GPT-3.5) was the most well-known LLM.}, then conducted brainstorming and sketching sessions in small groups of 4-6. Full details of the workshops are in Appendix \ref{appendix:workshops}.

Based on the opportunities and concerns identified in the workshops (see Section \ref{findings:workshop} or Appendix \ref{appendix:workshopFindings}), we created a prototype assistant tool (Fig. \ref{fig:participantsAndTool}, right). The tool comprised an input text box for users to enter details about their client and a number of different types of output options. Users could select an output option based on a desired use case, then click a "generate" button to produce an LLM-written\footnote{OpenAI GPT-4 Turbo, gpt-4-1025-preview.} response. These output options addressed various manual and mental labour difficulties in social work, and simulated potential uses addressing these difficulties. In response to complaints about manual work, the tool offered options to rewrite workers' rough notes into various organization-wide formats (e.g. BPSS, DIAP, Ecological, 5Ps). Users also raised points about assessments, case conceptualization, and ideation. We included options for strength, risk, and challenge assessments, following common social service practices \cite{rooney2017direct}. Finally, we added options to generate client intervention plans according to three common theoretical models: CBT, SFBT\footnote{CBT: Cognitive Behavioural Therapy; SFBT: Solution-Focused Brief Therapy.}, and Task-Centred Interventions.

In \textbf{stage 2, we conducted focus group discussions} (FGDs), where we simulated a contextual inquiry process \cite{holtzblatt2017contextual} by asking participants to walk us through their use of our tool and explain their thought processes along the way. We also encouraged them to point out weaknesses or flaws in the system and suggest potential improvements. We conducted the FGDs in groups of 2-4 SSPs from Agency A (the family service centre agency), with a mix of workers of varying seniority levels. We held a total of 8 sessions totalling 24 participants (Fig. \ref{fig:participantsAndTool}, left). Each session averaged 45-60 minutes in length and was audio-recorded for transcription and analysis. These sessions were conducted in May 2024.

In these sessions, we sought to understand the \textit{opportunities} of AI use in social work. Participants brought along different types of anonymised case files (e.g., rough short-hand notes, complete reports, intake files) from recent clients they had worked with. We explained the various functions of the system, then told participants to imagine themselves using it to help with the cases they had on hand, exploring a range of use cases from generating documentation reports to planning future sessions or interventions for the client. We also identified early on that case supervision by senior workers was a key means of addressing the difficulties faced by junior workers and encouraging worker development. We therefore sought to understand how AI could play a collaborative role in the supervision process. In sessions with supervisors present, we asked these more senior workers how they felt the various functions of the system could assist them in their discussion of cases with their supervisees. 
% We then discussed how the tool's outputs could help in areas such as In sessions with supervisors present,  We also learned early on that case supervision by senior workers was a key means of addressing the difficulties faced by junior workers and encouraging worker development. We therefore sought to understand how AI could play a collaborative role in the supervision process. , we asked these more senior workers how they felt the various functions of the system could aid them in their discussion of cases with their supervisees. 

We also aimed to understand the possible \textit{challenges} of using AI tools in social work. We asked participants to review the quality of the tool's outputs, compare them to their own, and identify areas where they would fall short of expectations or fail to be useful in their daily work. We note that while our platform is based on GPT-4, a general purpose model not tuned for social work analysis\footnote{Specialised AI systems for social work case management do exist (e.g. \cite{socialworkmagic2024, caseworthy2024}, but these are focused on user experience and do not provide greater insight than vanilla GPT; furthermore, \cite{socialworkmagic2024} advises users to take its assessments and suggestions as only a starting point, which aligns with how we position our tool.}, our aim was not to discover the specific weaknesses of GPT-4 or any LLM in particular. Rather, it was to understand what participants perceived to be good outputs from an AI system, and in the process understand how they might be affected by any possible shortcomings of such a system. % I actually feel the footnote here is pretty important at the first glimpse haha

Finally, we explored some of the longer-term effects of AI use, such as potential overreliance on the tool and the possibility of becoming overly trusting of the system's output. Senior workers in particular were asked about how they perceived junior workers new to the sector having such a system to help them with their daily work.

We performed qualitative thematic analysis \cite{braun2006using} on the transcripts, adopting a bottom-up, inductive approach to data coding. This process is detailed in Appendix \ref{appendix:analysis}, with the findings presented in the next section.


% \begin{table}[H]
% \centering
% \caption{Focus Group Participants from Agency A}
%   \label{tab:thkMayParticipants}
%   \begin{tabular}{cc|cc}
%     \toprule
%     \multicolumn{2}{c|}{\textbf{Centre 1}}&\multicolumn{2}{c|}{\textbf{Center 2}} \\
%     Code&Role&Code&Role\\
%     \midrule
%     W1 & Social Worker & S4 & Supervisor\\
%     W2 & Social Worker & S5 & Supervisor \\
%     W3 & Social Worker & W4 & Social Worker\\
%     \midrule
%     S1 & Supervisor \\
%     S2 & Supervisor \\
%     S3 & Supervisor \\
%     \midrule
%     &&&& S6 & Senior SW\\
%     &&&& W5 & Social Worker\\
%     \toprule
%     Code&Role&Code&Role\\
%     \multicolumn{2}{c|}{\textbf{Center 3}}&\multicolumn{2}{c|}{\textbf{Center 4}}\\
%      W9 & Social Worker & S8 & Senior SW\\
%      W10 & Social Worker & S9 & Senior SW \\
%      C1 & Snr Counsellor & S10 & Senior SW \\
%      \midrule
%      S7 & Senior SW & W11 & Social Worker \\
%      W7 & Social Worker & W12 & Social Worker\\
%      W8 & Social Worker & W13 & Social Worker \\
%     \toprule
%     Code&Role\\
%     \multicolumn{2}{c}{\textbf{Management}}\\
%     D1 & Director*\\
%     \multicolumn{2}{c}{\textit{*Observer in}}\\
%     \midrule
%   \bottomrule
%   \Description{Table describing Focus Group Participants from Agency A. There are 8 groups from 4 different centres, plus a director classified under "Management".}
% \end{tabular}
% \label{tab:workshopParticipants}
% \Description{Table describing workshop Participants from Agency A and Agency B. There are 13 participants from Agency A, mostly social workers, and 14 participants from Agency B, a mix of youth workers, youth project workers, and counsellors.}

% \caption{Workshop Participants from Agency A and Agency B}
% \end{table}


% \begin{minipage}{0.5\linewidth}
% \begin{table}[H]
% \centering
% \begin{tabular}{cc|cc}
%     \toprule
%     \multicolumn{2}{c|}{\textbf{Agency A}}&\multicolumn{2}{c}{\textbf{Agency B}} \\
%     \midrule
%     Code&Role&Code&Role\\
%     \midrule
%     TS1 & Snr Social Worker & CD1 & Director\\
%     TS2 & Social Worker & CY1 & Youth Work \\
%     TS3 & Social Worker & CY2 & Youth Work \\ 
%     TS4 & Social Worker & CY3 & Youth Work \\
%     TS5 & Social Worker & CP1 & Youth Projects\\
%     TS6 & Social Worker & CP2 & Youth Projects\\
%     TS7 & Social Worker & CP3 & Youth Projects\\
%     TS8 & Social Worker & CP4 & Youth Projects \\
%     TS9 & Social Worker & CC1 & Counsellor \\
%     TS10 & Social Worker & CC2 & Counsellor \\
%     TD1 & Exec. Director & CC3 & Counsellor\\
%     TD2 & Snr Director & CC4 & Counsellor\\
%     TP1 &  Psychologist & CC3 & Counsellor\\
%     && CC4 & Counsellor\\
% \bottomrule   
% \end{tabular}
% \label{tab:workshopParticipants}
% \Description{Table describing workshop Participants from Agency A and Agency B. There are 13 participants from Agency A, mostly social workers, and 14 participants from Agency B, a mix of youth workers, youth project workers, and counsellors.}

% \caption{Workshop Participants from Agency A and Agency B}
% \end{table}
% \end{minipage}
% \begin{minipage}[HT]{0.5\linewidth}
% \begin{figure}[H]
% \centering
% \includegraphics[width=3in]{images/prototypetool.png}
% \caption{Prototype AI Tool}
% \Description{A screenshot of our prototype AI tool. It has a text box for entry at the top, a list of radio buttons to select output modalities, a text box for extra instructions to the model a user might want to input, and a "generate" button.}
% \label{fig:tool}
% \end{figure}
% \end{minipage}

% \begin{table}
%   \caption{Focus Group Participants from Agency A}
%   \label{tab:thkMayParticipants}
%   \begin{tabular}{cc|cc|cc|cc|cc}
%     \toprule
%     \multicolumn{2}{c|}{\textbf{Centre 1}}&\multicolumn{2}{c|}{\textbf{Center 2}}&\multicolumn{2}{c|}{\textbf{Center 3}}&\multicolumn{2}{c|}{\textbf{Center 4}}&\multicolumn{2}{c}{\textbf{Management}}\\
%     \midrule
%     Code&Role&Code&Role&Code&Role&Code&Role&Code&Role\\
%     \midrule
%     W1 & Social Worker & S4 & Supervisor & W9 & Social Worker & S8 & Senior SW & D1 & Director*\\
%     W2 & Social Worker & S5 & Supervisor & W10 & Social Worker & S9 & Senior SW & \multicolumn{2}{c}{\textit{*Observer in}}\\
%     W3 & Social Worker & W4 & Social Worker & C1 & Snr Counsellor & S10 & Senior SW & \multicolumn{2}{c}{\textit{multiple sessions}}\\
%     \midrule
%     S1 & Supervisor &&& S7 & Senior SW & W11 & Social Worker\\
%     S2 & Supervisor &&& W7 & Social Worker & W12 & Social Worker\\
%     S3 & Supervisor &&& W8 & Social Worker & W13 & Social Worker\\
%     \midrule
%     &&&& S6 & Senior SW\\
%     &&&& W5 & Social Worker\\
%   \bottomrule
%   \Description{Table describing Focus Group Participants from Agency A. There are 8 groups from 4 different centres, plus a director classified under "Management".}
% \end{tabular}
% \end{table}




 

% An important consideration was ensuring the generalisability of our findings, making sure our system and subsequent discussions adequately represented the vast possibilities of AI use cases, and not being limited to exploring a particular LLM in a specific configuration. We address this by noting the difference between areas that other or future AI systems may improve in, such as reasoning and knowledge, and those that they are unlikely to, such as being able to account for a user's (or client's) contextual factors. It is the latter that we focus our investigation on: Given an AI system is ultimately constrained by its training dataset, what are the possibilities with such systems and what are the potential downsides?

% 1) data privacy; 2) overreliance; 3) inaccurate output (participant identifiers are missed for 3 (?)


% This doubtless was influenced by our intial presentation on GenAI capabilities, where we showed an example of an LLM summarising and reformatting case notes. 

% \subsection{Research Questions and Testing Goals}

% From the above, we derive the following research questions.

% First, while many workers suggested they would be interested in AI automation tools, care needs to be taken to design such tools in a manner that is useful to them. Developing a tool that does not fit in their workflow or give outputs that are what they are looking for is pointless.


% \textbf{\textit{RQ1}}: What kind of AI tools do social workers find useful in their daily lives? How can they use these in their daily work?


% Besides non-useful outputs (possibly a system designer flaw), AI tools may also make more micro-level mistakes, such as by making topically relevant, yet incorrect or suboptimal suggestions. The ability of LLMs and other AI systems to handle social work-related queries is, to our knowledge, almost entirely unexplored.


% \textbf{\textit{RQ2}}: How accurately can modern AI tools handle these tasks?


% Given a tool that provides relevant, high-quality aid to social workers, the question shifts to how these tools should be most effectively implemented in and throughout an agency, maximising benefits while minimising harm.


% \textbf{\textit{RQ3}}: How do staff in different positions in the organisation use or think about the tool differently? What different user requirements might they have?


% \textbf{\textit{RQ4}}: How should staff in different positions in the organisation use or be given access to the tool differently? How can the tool serve to complement while not replacing human social workers?

% Due to the iterative nature of our research process, these later sessions served to both further ideate and critique potential solutions, and also test the concrete prototypes developed at that stage. We include the former as part of our discussion here, and leave the latter for a later part of the paper.

% We start by summarising some common difficulties that plague social workers in their work. We then present how participants felt that AI could help address these issues. Finally, we discuss the potential for AI to negatively affect organisational stability in a social work agency, and ways to begin understanding this complex issue.

