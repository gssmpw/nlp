%% PRELIMINARIES

%% (2) background and preliminaries
\subsection{Preliminaries}
%
Following the notation in \citep{vaswaniAttentionAllYou2017, radfordImprovingLanguageUnderstanding2018}, we define a Transformer architecture as,
%
%
\begin{definition}
\label{def-transformer-model}
%
(\textbf{\emph{Transformer architecture}}).
%
%
Let $\bm{U} \in \mathbb{R}^{N,V}$ be a matrix representing the sequence of $N$ one-hot encoded tokens of dimension $V$.
%
A Transformer architecture consists of $L$ stacked attention blocks, each one composed of a self-attention layer and a feedforward layer, as follows,
%
\begin{equation}
\label{eq:def:transformer-model}
\begin{cases}
%
\bm{X}_0(\bm{U}) = \bm{U} \bm{W_e} + \bm{W}_p \\[5pt]
\vspace{5pt}
\begin{aligned}
\Bigg\{
\begin{aligned}
&\hat{\bm{X}}_l = \bm{X}_{l-1} + a_l(\bm{X}_{l-1}; \bm{W}_q^l, \bm{W}_k^l, \bm{W}_v^l) \\[2pt]
&\bm{X}_l = \hat{\bm{X}}_l + m_l(\hat{\bm{X}}_l ; \bm{W}_1^l, \bm{W}_2^l)
\end{aligned} & \,\,\, \forall l \in [1, L] 
\end{aligned} \\
\sigma\big(\bm{Z}\big) = \sigma\big(\bm{X}_L \bm{W_u}\big) \,,
%
\end{cases}
\end{equation}
%
where $\bm{W_e}\in \mathbb{R}^{V,d}$ represents the linear transformation from the vocabulary space to the embedding space of dimension $d$, $\bm{W_p}\in \mathbb{R}^{V,d}$ represents the positional encoding,  $\bm{X}_0\in \mathbb{R}^{N,d}$ is the initial embedding of the sequence, 
$a_l(\cdot)$ is a self-attention function given by 
%
\begin{equation}
    a(\bm{X}_{l-1}) = \bm{A}^l(\bm{X}_{l-1})\bm{V}^l(\bm{X}_{l-1})
\end{equation}
%
where the matrix of attention scores $\bm{A}^l(\bm{X}_{l-1})$ is given by 
%
\begin{equation}
\label{eq:results:self-attention-transformers}
\begin{cases}
%
\bm{Q}^l(\bm{X}_{l-1}) = \bm{X}_{l-1}\bm{W}^l_q \\
\bm{K}^l(\bm{X}_{l-1}) = \bm{X}_{l-1}\bm{W}^l_k \\
\bm{A}^l(\bm{X}_{l-1}) = \sigma\left(\frac{1}{\sqrt{d}}\,\bm{Q}^l {\bm{K}^l}^T\right) \,,\\
%
\end{cases}
\end{equation}
%
where $1/\sqrt{d}$ is a constant normalization factor, and $\bm{W}_q^l\in \mathbb{R}^{d,d}$ and $\bm{W}^l_k\in \mathbb{R}^{d,d}$ represent linear transformations within the embedding space,
$\bm{V}^l(\bm{X}_{l-1}) = \bm{X}_{l-1}\bm{W}^l_v$ represents a linear transformation within the embedding space, $m_l(\cdot)$ is a position-wise feedforward layer with hidden dimension $d_f$ and learnable matrices $\bm{W}^l_1 \in \mathbf{R}^{d,d_f}$ and $\bm{W}^l_2\in \mathbf{R}^{d_f,d}$, $\bm{W_u}\in \mathbb{R}^{d, V}$ represents the linear transformation from the embedding space back to the vocabulary space, $\sigma(\cdot)$ is the row-wise softmax function,  and $\sigma\big(\bm{Z}\big)\in \mathbb{R}^{N,V}$ is the estimated probability distribution over the vocabulary.
%
We omit layer normalization and biases for simplicity (see also \citep{elhageMathematicalFrameworkTransformer2021}).
%
\end{definition}
%
%
Furthermore, we use the following definition of a bilinear form,
\begin{definition}
\label{def-bilinear-form}
(\textbf{\emph{Bilinear form}}).
%
A  bilinear form on a vector space $V$ over a field $F$ is a map $M: V \times V \to F$ that is linear in each argument separately, that is, 
%
\begin{equation}
\begin{split}
    & M(a\bm{x} + b\bm{y}, \bm{z}) = aM(\bm{x}, \bm{z}) + bM(\bm{y}, \bm{z}) \\
    & M( \bm{x}, a\bm{y} + b\bm{z}) = aM(\bm{x}, \bm{y}) + bM(\bm{x}, \bm{z}) \,,\\
\end{split}
\end{equation}
%%
for all \(\bm{x}, \bm{y}, \bm{z} \in V \) and \( a, b \in F \).
%
Let $\{\bm{e}_1,\dots,\bm{e}_d\}$ be a basis for the vector space $V$.
%
The matrix $\bm{M}$ such that $[\bm{M}]_{ij} = M(\bm{e}_i,\bm{e}_j)$ is the matrix of the bilinear form on this basis, and it follows
%
\begin{equation}
    M(\bm{x},\bm{y}) = \bm{x}^\top\bm{M}\bm{y}
\end{equation}
%
\end{definition}
%
Finally, we provide the following definition of autoregressive and bidirectional training objectives,
%
\begin{definition}
\label{def-objective-functions}
%
(\textbf{Autoregressive and bidirectional objectives})
%
Let $U = \{t_1, \dots, t_N\}$ a sequence of tokens.
%
The joint probability of $U$ is factorized autoregressively as follows,
%
\begin{equation}
    \Pr[U] = \Pr[t_1,\dots,t_N] = \Pi_{i=1}^N \Pr[t_i|t_1,\dots,t_{i-1}]\,.
\end{equation}
%
During autoregressive training, a model with set of parameters $\mathcal{W}$ is optimized to minimize the following negative log-likelihood
%
\begin{equation}
\mathcal{L}(U; \mathcal{W}) = - \sum_{i=1}^{N} \log p(t_i \,|\, \{t_j\} : j < i\,; \mathcal{W})\,,
\end{equation}
%
where the conditional probabilities are modeled with learnable parameters $\mathcal{W}$.
%
The join probability of $U$ can also be factorized bidirectionally as follows,
%
\begin{equation}
    \Pr[U] = \Pr[t_1,\dots,t_N] = \Pi_{i=1}^N \Pr[t_i|t_1,\dots,t_{i-1}, t_{i+1}, \dots, t_N]\,.
\end{equation}
%
During bidirectional training, a model with set of parameters $\mathcal{W}$ is optimized to minimize the following negative log-likelihood
%
\begin{equation}
\mathcal{L}(U; \mathcal{W}) = - \sum_{i=1}^{N} \log p(t_i \,|\, \{t_j\} : j \neq i\,; \mathcal{W})\,.
\end{equation}
%
In practice, only a subset $M\in [1,\dots,N]$ of the tokens are predicted as in  Masked Language Modelling (MLM) \cite[see ][]{devlinBERTPretrainingDeep2019,warnerSmarterBetterFaster2024}, leading to the following negative log-likelihood
%
\begin{equation}
\mathcal{L}(U; \mathcal{W}) = - \sum_{i\in M} \log p(t_i \,|\, \{t_j\} : j \notin M\,; \mathcal{W})\,.
\end{equation}
%
\end{definition}





%% PROOF PROPOSITION SELF-ATTENTION AND BILINEAR FORM
\subsection{Proof of Proposition \ref{prop-self-attention-bilinear-form} and related remarks}
\label{supp-math-self-attention-bilinear-form}
%
\begin{proof}
%
Let $\bm{A}$ be the matrix of attention scores in a self-attention layer
as in Definition \ref{def-transformer-model}.
%
It follows that,
%
\begin{equation}
    \hat{\bm{X}}(\bm{X}) = \bm{X} + \bm{A}(\bm{X})\bm{V}(\bm{X}) = \bm{X} + \sigma\left(\frac{1}{\sqrt{d}}\,\bm{X} \bm{W}_{qk}\bm{X}^T\right)\bm{X}\bm{W}_v \,,
\end{equation}
%
where $\bm{W}_{qk} = \bm{W}_q \bm{W}_k^\top$, and where we refer to a general layer in a Transformer model, omitting the subscript $l$.
%
It follows that the entry $\alpha_{ij} = [\bm{A}]_{ij}$ of the attention score matrix is given by, 
%
\begin{equation}
    \alpha_{ij} = \frac{\exp{(\bm{x}_i^\top \bm{W}_{qk} \bm{x}_j)}}{\sum_j \exp{(\bm{x}_i^\top \bm{W}_{qk} \bm{x}_j)}} = \frac{\exp{(\hat{\alpha}_{ij})}}{\mathcal{N}_i} \,,
\label{eq-proof-self-attention-coefficients-convex-hull}
\end{equation}
%
where we neglect the scaling factor $1 / \sqrt{d}$ for simplicity, every $\hat{\alpha}_{ij}$ is a coefficient given by $\hat{\alpha}_{ij} = \bm{x}_i^\top \bm{W}_{qk} \bm{x}_j$, and $\mathcal{N}_i$ is a normalization factor defined by the row-wise softmax function $\sigma(\cdot)$ and shared across all entries in the $i$-th row of $\bm{A}$,
%
\begin{equation}
    \mathcal{N}_i(\bm{X}; \bm{W}_{qk}) = \sum_j \exp{(\bm{x}_i^\top \bm{W}_{qk} \bm{x}_j)} = \sum_j \exp{(\hat{\alpha}_{ij})} \,.
\end{equation}
%
The coefficients $\{\hat{\alpha}_{ij}\}$ in the second term represent the projection of $\bm{x}_i$ onto $\text{span}\{\bm{X}\}$ (the subspace spanned by $\bm{X} = \{\bm{x}_0^\top, \bm{x}_1^\top, \dots, \bm{x}_N^\top\})$ in the transformed embedding space defined by $\bm{W}_{qk}$,
%
\begin{equation}
    \sum_j\hat{\alpha}_{ij}\, \bm{x}_j = \sum_j \bm{x}_i^\top \bm{W}_{qk} \bm{x}_j\,\bm{x}_j = \sum_j \langle \bm{x}_i, \bm{W}_{qk}\bm{x}_j \rangle \,\bm{x}_j \,.
\label{eq-proof-self-attention-bilinear-form}
\end{equation}
%
It follows from the monotonically increasing property of the soft-max function that the coefficients $\{\alpha_{ij}\}$ defined in Equation \eqref{eq-proof-self-attention-coefficients-convex-hull} preserve the order of the coefficients $\{\hat{\alpha}_{ij}\}$ while exponentially suppressing negative and near-zero values,
%
\begin{equation}
     \alpha_{ij} < \alpha_{ij'} \,  \Leftrightarrow \,\hat{\alpha}_{ij} < \hat{\alpha}_{ij'}\quad \forall i,j, j' \,.
\end{equation}
%
Therefore, the coefficients $\{\alpha_{ij}\}$ specify a convex combination that restricts the resulting vector in the convex hull $\text{Conv}(\bm{X}) \subset \text{span}\{\bm{X}\}$, where the $i$-th row of $\bm{X}$ corresponding to the $i$-th token in the sequence is transformed into the row vector $\hat{\bm{x}}_i$ as follows,
%
\begin{equation}
\hat{\bm{x}}^\top_i = \bm{x}^\top_i + \sum_j\alpha_{ij}\,\bm{x}_j\bm{W}_v\,,
\end{equation}
%
thus concluding the proof.
% 
\end{proof}
\begin{remark}
%
Each element of the sum in Equation \eqref{eq-proof-self-attention-bilinear-form} implicitly defines an operator in the subspace spanned by $\bm{X} = \{\bm{x}_0^\top, \bm{x}_1^\top, \dots, \bm{x}_N^\top\}$ given the transformed embedding space defined by $\bm{W}_{qk}$,
%
\begin{equation}
     \sum_j \alpha_{ij}(\bm{W}_{qk})\, \bm{x}_j = \sum_j \bm{x}_i^\top \bm{W}_{qk} \bm{x}_j\,\bm{x}_j = \sum_j P_{\bm{W}_{qk}}(\bm{x}_i ,\bm{x}_j)\,,
\end{equation}
%
where $P_{\bm{W}_{qk}}(\,\cdot\,, \bm{x}_j)$ are operators over the subset $\bm{X}$.
%
The set $\bm{X}$ is in general linearly dependent, and the number of tokens $N$ in the sequence differs from the embedding space dimension $d$.
%
Furthermore, $\bm{W}_{qk}$ represents a general bilinear form (see Definition \ref{def-bilinear-form}), which may not satisfy all the defining axioms of a formal inner product — namely, linearity, conjugate symmetry, and positive definiteness.
%
Finally, the operators $P_{\bm{W}_{qk}}(\cdot \bm{x}_j)$ are not formal projection operators since they are not nilpotent ($P_{\bm{W}_{qk}}(\,\cdot\, \bm{x}_j) \circ P_{\bm{W}_{qk}}(\,\cdot\, \bm{x}_j) \neq P_{\bm{W}_{qk}}(\,\cdot\, \bm{x}_j)$).
%
Nonetheless, the bilinear map $\bm{W}_{qk}: \mathbb{R}^d \times \mathbb{R}^d \rightarrow \mathbb{R}$ still associates any pair of vectors with a scalar value quantifying their alignment as determined by the geometric relations encoded in $\bm{W}_{qk}$.
%.
% Nevertheless, the sum in Equation \ref{eq:math:linear-decomposition-subspace-tokens} represents a decomposition of $\bm{x}_i$ on the subspace $\text{span}\{\bm{X}\}$ where each coefficient $\alpha_{ij}(\bm{W}_{qk})$ quantifies the aligment of $\bm{x}_i$ with the corresponding element of $\bm{X}$ following the bilinear form $\bm{W}_{qk}$. 
% %%
Therefore, self-attention computes a generalized decomposition of $\bm{x}_i$ on $\text{Conv}(\bm{X})$ in the transformed embedding space defined by $\bm{W}_{qk}$.
%
A convex combination ensures that the resulting vector remains within the region enclosed by the basis vectors $\bm{X}$.
%  
\end{remark}
\begin{remark}
%
Following Definition \ref{def-transformer-model}, multi-head attention consists of parallelizing the self-attention operation across $H$ different heads with an embedding space $d_h < d$, 
%
\begin{equation}
    \hat{\bm{X}}(\bm{X}) = \bm{X} + \text{concat}\big(\bm{A}_1\bm{V}_1, \bm{A}_2\bm{V}_2, \dots, \bm{A}_h\bm{V}_h\big)\bm{W}_o
\end{equation}
%
where $\bm{A}_h = \sigma(d^{-1/2}\,\,\bm{X} \,\bm{W}_{q,h}\,\bm{W}^\top_{k,h}\,\bm{X}^T)$ is the self-attention of the $h$-th head, $\bm{W}_{q,h}\in \mathbb{R}^{d,d_h}$, $\bm{W}_{k,h}\in \mathbb{R}^{d,d_h}$ and $\bm{W}_{v,h}\in \mathbb{R}^{d,d_h}$ are the query, key, and value matrices of the $h$-th attention head, respectively, and $\bm{W}_o \in \mathbb{R}^{d,d}$ is a linear transformation \citep{vaswaniAttentionAllYou2017}.
%
Operationally, the self-attention computation is performed in parallel by factorizing the $\bm{W}_q$ and $\bm{W}_k$ matrices into $H$ rectangular blocks, as follows,
%
\begin{equation}
\begin{split}
    & \bm{W}_q = \big[\bm{W}_{q,1} \big| \bm{W}_{q,2} \big| \dots \big| \bm{W}_{q,H}\big] \\
    & \bm{W}_k = \big[\bm{W}_{k,1} \big| \bm{W}_{k,2} \big| \dots \big| \bm{W}_{k,H}\big] \,,
\end{split}
\end{equation}
%
and performing the matrix multiplication $\bm{W}_{q,h}\bm{W}_{k,h}^\top$ per every $h$-th head independently in one step.
%
It follows that the full $\bm{W}_{qk}$ matrix is given by the sum of the bilinear forms $\bm{W}_{qk,h}$ of every head, as follows,
%
\begin{equation}
    \bm{W}_{qk} = \bm{W}_q\bm{W}_k^\top = \sum_h \bm{W}_{q,h}\bm{W}^\top_{k,h} = \sum_h \bm{W}_{qk,h}
\end{equation}
%
where each $\bm{W}_{qk,h} \in \mathbb{R}^{d,d}$ is a square matrix with $\text{rank}(\bm{W}_{qk,h}) \leq d_h$.
%
Therefore, each head perform independent projections onto $\text{Conv}(\bm{X})$ that are then summed together, as follows,
%
\begin{equation}
    \sum_j\hat{\alpha}_{ij}\, \bm{x}_j = \sum_j \bm{x}_i^\top \bm{W}_{qk} \bm{x}_j\,\bm{x}_j = 
    \sum_j \bm{x}_i^\top \Big(\sum_h\bm{W}_{qk,h}\Big) \bm{x}_j\,\bm{x}_j = \sum_j \sum_h \langle \bm{x}_i, \bm{W}_{qk,h}\bm{x}_j \rangle \,\bm{x}_j \,,
\end{equation}
%
thus performing the same operations as in Equation \eqref{eq-proof-self-attention-bilinear-form}.
%
\end{remark}
%



%% PROOF PROPOSITION GRADIENTS AND RANK-1 MATRIX
\subsection{Proof of Proposition \ref{prop-gradients-self-attention} and related remarks}
\label{supp-math-gradients-self-attention}
%
\begin{proof}
%
Let $U = \{t_1, \dots, t_N\}$ be a  sequence of $N$ tokens from a vocabulary of dimension $V$.
%
Let $\mathcal{L}(U)$ be the negative log-likelihood of each token $t_i$, expressed as
%
\begin{equation}
\label{eq:results:self-supervised-pretraining}
    \mathcal{L}(U) = \sum_i \mathcal{L}(t_i)=  - \sum_i \log p(t_i \,|\, \{t_j : j \in  \mathcal{C}_i\}) \,,
\end{equation}
%
where $\mathcal{C}_i \subset [0,1,\dots, N]$ is the set of indices defining the set of tokens $\{t_j\}$ of the conditional probability distribution.
%
Let $\bm{U} = [\bm{t}_0, \bm{t}_1, \dots, \bm{t}_N]$ be the sequence of $N$ one-hot encoded tokens $\bm{t}_i \in \mathbb{R}^V$ associated with $U$, where $V$ is the dimension of the vocabulary.
%
Let $\mathcal{L}(\bm{t}_i)$ be the cross-entropy of the one-hot encoded token $\bm{t}_i$  and the estimated probability distribution $\sigma(\bm{z}_i) \in \mathbb{R}^V$, as follows,
%
\begin{equation}
 \mathcal{L}(\bm{U}) = \sum_{i=1}^N \mathcal{L}(\bm{t}_i) = \sum_{i=1}^N \bm{t}_i\log(\sigma(\bm{z}_i))\,,
\end{equation}
%
where we let $\bm{z}_i$ be the prediction of the $i$-th token $\bm{t}_i$ from the representations in the last layer of a Transformer model, following Definition \ref{def-transformer-model},
%
%Let the self-attention function in Definition~\ref{def-transformer-model} be linear, that is,  
%
% \begin{equation}
%     \bm{A}^l(\bm{X}_{l-1}) = \,\bm{Q}^l {\bm{K}^l}^T
% \end{equation}
%
% where we neglect the scaling factor $\sqrt{d}$ for simplicity.
% %
% The self-attention function thus maps the token embeddings $\{\bm{x}_i^{l-1}\}$ from layer $l-1$ to the embeddings $\{\hat{\bm{x}}^l_i\}$ as follows,
% %
% \begin{equation}
% {\bm{\hat{x}}^{l-1}_i}^\top = {\bm{x}^{l-1}_i}^\top + \sum_{j\in C_i} 
% \alpha^l_{ij}\,
% {\bm{x}^{l-1}_j}^\top \bm{W}^l_v \,,
% \end{equation}
% %
% where $C_i \subset \bm{U}$ is the subset of tokens that defines the conditional probability distribution.
%
\begin{equation}
\begin{cases}
%
{\bm{x}^0_i}^\top = \bm{t_i}^\top \bm{W_e} + \bm{W}_p \\[5pt]
\vspace{5pt}
{\bm{x}_i^l}^\top = \mathcal{F}_l(\bm{x}_i^{l-1})\quad \forall l \in [1, L] \\
\sigma\big(\bm{z}_i\big)  = \sigma\big({\bm{x}_i^L}^\top \bm{W_u}\big)\,,
%
\end{cases}
\label{eq-proof-gradient-bilinear-form-transformer-model}
\end{equation}
%
where ${\bm{x}^l_i}^\top = \mathcal{F}_l(\bm{x}_i^{l-1})$ is a short notation for the self-attention and multi-layered perception transformation of the $l$-th layer,
%
\begin{equation}
\mathcal{F}_l(\bm{x}_i^{l-1}) =
\left\{
\begin{aligned}
& \hat{\bm{x}}_i^{l^{\top}} = \hat{\bm{x}}_i^{{l-1}^{\top}} + a_l(\bm{x}_i^{l-1}) \\
& \bm{x}_i^{l^{\top}} = \hat{\bm{x}}_i^{l^{\top}} + m_l(\hat{\bm{x}}^l_i)
\end{aligned}
\right.\,,
\end{equation}
%
where the self-attention function is given by
%
\begin{equation}
    a_l(\bm{x}_i^{l-1}) = \sum_{j\in C_i} 
\alpha^l_{ij}(w^l)\,
{\bm{x}^{l-1}_j}^\top \bm{W}^l_v \,.
\end{equation}
%
Let attention coefficients $\alpha^l_{ij} \equiv \alpha^l_{ij} (w^l)$ of the $l$-th layer be parameterized with a general parameter $w^l$.
%
% We divide the derivation of the gradient of $\mathcal{L}(\bm{t}_i)$ w.r.t. the self-attention function into two steps.
%
Let the gradient of $\mathcal{L}(\bm{t}_i)$ w.r.t. $w_l$ (the parameterization of the attention scores) be factorized as follows,
%
\begin{equation}
    \nabla_{w^l} \mathcal{L}(\bm{t}_i) = \frac{\partial \mathcal{L}(\bm{t}_i)}{\partial \alpha^l_{ij}} \frac{\partial \alpha^l_{ij}}{\partial w^l}\,.
\end{equation}
%
It follows that,
%
\begin{equation}
\begin{split}
    \frac{\partial \mathcal{L}(\bm{t}_i)}{\partial \alpha^l_{ij}} 
    & = \frac{\partial \mathcal{L}(\bm{t}_i)}{\partial \bm{z}_i} \, \frac{\partial \bm{z}_i}{\partial \bm{x}^L_i} \, \frac{\partial \bm{x}^L_i}{\partial \hat{\bm{x}}^l_i}\,\frac{\partial \hat{\bm{x}}^l_i} {\partial \alpha^l_{ij}} 
    \\
    & = (\bm{t}_i - \sigma(\bm{z}_i))^\top \bm{W}_u^\top \, \frac{\partial \bm{x}^L_i}{\partial \hat{\bm{x}}^l_i}\,{\bm{W}^l_v}^\top \sum_{j\in C_i} \bm{x}^{l-1}_j \,,
\end{split}
\end{equation}
%
where the term $\partial \bm{x}_i^L / \partial \hat{\bm{x}}_i^l$ includes the set of partial derivatives that define the gradient of the representation $\bm{x}_i^L $ at the last layer w.r.t. the self-attention representation $\hat{\bm{x}}_i^l$ at the $l$-layer, as follows,
%
\begin{equation}
    \frac{\partial \bm{x}_i^L}{\partial \hat{\bm{x}}_i^l} = \left(1 + \sum_{m = l}^{L-1}\mathcal{F}_l'(\bm{x}_i^m)\right) \left(1 + m_l'(\hat{\bm{x}}^l_i)\right)\,,
\end{equation}
%
where 
%
\begin{equation}
    \mathcal{F}_l'(\bm{x}_i^m) = \frac{\partial}{\partial \bm{x}_i^l}\mathcal{F}_l(\bm{x}_i^m) \quad ; \quad m_l'(\hat{\bm{x}}^l_i) = \frac{\partial}{\partial \bm{\hat{x}}_i^l} m_l(\hat{\bm{x}}^l_i) \,.
\end{equation}
%
Let $\bm{\delta}^l_i$ be the error at the last layer propagated to the self-attention function at the $l$-th layer, as follows, 
%
\begin{equation}
    {\bm{\delta}^l_i}^\top = (\bm{t}_i - \sigma(\bm{z}_i))^\top \bm{W}_u^\top \, \frac{\partial \bm{x}^L_i}{\partial \hat{\bm{x}}^l_i} \,
     {\bm{W}^l_v}^\top \,,
\end{equation}
%
thus obtaining the following equation for the gradient,
%
\begin{equation}
     \nabla_{w^l} \mathcal{L}(\bm{t}_i) = {\bm{\delta}^l_i}^\top  \sum_{j\in C_i} \bm{x}^{l-1}_j  \frac{\partial \alpha^l_{ij}}{\partial w^l} \,.
\end{equation}
%
Let the attention scores be computed without the row-wise softmax operation and explicitly with the bilinear form $\bm{W}_{qk}$, such that the following expression gives the score between the $i$-th and $j$th token,
%
\begin{equation}
    \alpha^l_{ij} \equiv \alpha^l_{ij}(\bm{W}_{qk}^l) = {\bm{x}^{l-1}_i}^\top \bm{W}_{qk}^l \bm{x}^{l-1}_j \,,
\end{equation}
%
from which we obtain
%
\begin{equation}
    \frac{\partial \alpha^l_{ij}}{\partial \bm{W}_{qk}^l} = \bm{x}^{l-1}_i{\bm{x}^{l-1}_j}^\top = \bm{K}^{l-1}_{ij}\, ,
\end{equation}
%
where $\bm{K}^{l-1}_{ij} \in \mathbb{M}_n$ is a square rank-1 matrix given by the outer product between the $i$-th and $j$-th token from the $l-1$-th layer.
%
It follows that the total gradient of $\mathcal{L}(\bm{t}_i)$ is given by
%
\begin{equation}
    \nabla_{\bm{W}_{qk}^l} \mathcal{L}(\bm{t}_i) = {\bm{\delta}^l_i}^\top \sum_{j\in C_i} \bm{x}^{l-1}_j \bm{K}^{l-1}_{ij} \,,
\end{equation}
%
where we notice that, for every $j$, the term ${\bm{\delta}^l_i}^\top \bm{x}^{l-1}_j$ is a scalar quantity that we define as $\beta^l_{ij}$, thus obtaining,
%
\begin{equation}
\label{eq:math:gradient-linear-combination-rank-1-matrices}
    \nabla_{\bm{W}_{qk}^l} \mathcal{L}(\bm{t}_i) = \sum_{j\in C_i} \beta^l_{ij} \bm{K}^{l-1}_{ij} \,,
\end{equation}
%
and therefore,
%
\begin{equation}
    \nabla_{\bm{W}_{qk}^l} \mathcal{L}(U) = \sum_i \sum_{j\in C_i} \beta^l_{ij} \bm{K}^{l-1}_{ij} \quad \forall\, \bm{W}_{qk}^l, \,l \in [1,L]\,.
\end{equation}
%
Equivalently, we can rewrite the double summation as follows,
%
\begin{equation}
    \nabla_{\bm{W}_{qk}^l} \mathcal{L}(U) = \sum_{i \in P_j} \sum_{j} \beta^l_{ij} \bm{K}^{l-1}_{ij} \quad \forall\, \bm{W}_{qk}^l, \,l \in [1,L]\,,
\end{equation}
%
where $P_j \subset [0,1,\dots, N]$ is the set of indices defining the set of tokens $\{t_i\}$ that are predicted by $t_j$, thus concluding the proof.
%
\end{proof}

%
In standard Transformer models, the bilinear form $\bm{W}_{qk}$ is not directly computed, and as such, it is not explicitly updated through gradient descent.
%
Nonetheless, $\bm{W}_{qk}$ is implicitly updated with a combination of the weight updates of $\bm{W}_q$ and $\bm{W}_k$ having the same form as in Proposition \ref{prop-gradients-self-attention}, see the following.
%
\begin{remark}
%
Let $\mathcal{L}(\bm{U})$ be the negative log-likelihood of a given sequence of one-hot encoded tokens $\bm{U}$.
%
Let $\bm{W}^l_q$ and $\bm{W}^l_k$ be the query and key transformation matrices of the $l$-th layer of Transformer models, see Definition \ref{def-transformer-model}.
%
Let $\bm{W}^l_q$ and $\bm{W}^l_k$ be updated via gradient descent, that is, $\bm{W}^l_q \rightarrow \bm{W}^l_q +  \eta\nabla_{\bm{W}_q^l} \mathcal{L}(\bm{U})$ and $\bm{W}^l_k \rightarrow \bm{W}^l_k +  \eta\nabla_{\bm{W}_k^l} \mathcal{L}(\bm{U})$, where $\eta$ is the learning rate.
%
It follows that the matrix $\bm{W}_{qk}^l = \bm{W}^l_q{\bm{W}^l_k}^\top$ is implicitly updated following,
%
% \begin{equation}
%     \bm{W}_{qk}^l = \bm{W}^l_q{\bm{W}^l_k}^\top 
%     \rightarrow
%     \,\big(\, \bm{W}^l_q + \eta\nabla_{\bm{W}_q^l} \mathcal{L}(U) \,\big)\big(\,\bm{W}^l_k + \eta\nabla_{\bm{W}_k^l} \mathcal{L}(U) \,\big)^\top
% \end{equation}

% which we rewrite explicitly as,
%
\begin{equation}
\begin{split}
    \bm{W}_{qk}^l \rightarrow & 
    \,\big(\, \bm{W}^l_q + \eta\nabla_{\bm{W}_q^l} \mathcal{L}(U) \,\big)\big(\,\bm{W}^l_k + \eta\nabla_{\bm{W}_k^l} \mathcal{L}(U) \,\big)^\top \\
    & = \bm{W}^l_q{\bm{W}^l_k}^\top + \eta\, \big(\, \bm{W}^l_q\,{\nabla_{\bm{W}_k^l} \mathcal{L}}^\top (U) + \nabla_{\bm{W}_q^l} \mathcal{L}(U){\bm{W}^l_k}^\top \,\big) + o(\eta^2) \\
    & = \bm{W}_{qk}^l + \eta \sum_i \sum_{j\in C_i} \beta_{ij}^l\big(\bm{W}_q^l{\bm{W}_q^l}^\top\, \bm{K}_{ij}^{l-1} + \bm{K}_{ij}^{l-1}\,\bm{W}_k^l{\bm{W}_k^l}^\top \big) 
    + o(\eta^2) \\
    & = \bm{W}_{qk}^l + \Delta \bm{W}_{qk}^l + o(\eta^2) \simeq \bm{W}_{qk}^l + \Delta \bm{W}_{qk}^l \,,
\end{split}
\end{equation}
%
assuming that the learning rate $\eta$ is small. 
%
Therefore, the implicit weight update of $\bm{W}_{qk}^l$ following gradient descent is given by
%
\begin{equation}
    \Delta \bm{W}_{qk}^l    = \eta \sum_i \sum_{j\in C_i} \beta_{ij}^l
    \left[\big(\bm{W}_q^l{\bm{W}_q^l}^\top\, \bm{K}_{ij}^{l-1} \big)
    + 
    \big(\bm{K}_{ij}^{l-1}\,\bm{W}_k^l{\bm{W}_k^l}^\top \big)\right] 
    \propto \sum_i \sum_{j\in C_i} \beta_{ij}^l\, \bm{K}_{ij}^{l-1} \,,
\end{equation}
%
where both $\bm{W}_q^l{\bm{W}_q^l}^\top\, \bm{K}_{ij}^{l-1}$ and $\bm{K}_{ij}^{l-1}\,\bm{W}_k^l{\bm{W}_k^l}^\top$ are rank-1 matrices, 
%
\begin{equation}
\begin{split}
    & \bm{W}_q^l{\bm{W}_q^l}^\top\, \bm{K}_{ij}^{l-1} = \Big( \bm{W}_q^l{\bm{W}_q^l} \,\bm{x}^{l-1}_i\Big){\bm{x}^{l-1}_j}^\top = \bm{\bar{x}}^{l-1}_i {\bm{x}^{l-1}_j}^\top  \\
    & \bm{K}_{ij}^{l-1}\,\bm{W}_k^l{\bm{W}_k^l}^\top = \bm{x}^{l-1}_i\Big({\bm{x}^{l-1}_j}^\top\,\bm{W}_k^l{\bm{W}_k^l}^\top \Big) = \bm{x}^{l-1}_i \bm{\bar{x}}_j^{{l-1}^\top} \,.
\end{split}
\end{equation}
%
%
\end{remark}



\subsection{Formal proofs for the connection between objective functions, directionality, and symmetry}
\label{supp-math-directionality-symmetry}
%
In this section, we provide a formal mathematical derivation of the informal Theorems \ref{theo-informal-directionality} and \ref{theo-informal-symmetry} introduced in the main text.
%
To do so, we provide a formal description and proof of the Theorems for the directionality and symmetric structures during autoregressive and bidirectional training, respectively with Theorem \ref{theo-gradient-directionality} and Theorem \ref{theo-gradients-symmetry}.
%
We also introduce a series of related Propositions and Lemmas.
%
%

%
%% (1) PROPOSITION COLUMN AND ROWS
%
\subsubsection{Different implicit update of rows and columns}
\label{supp-math-prop-gradient-column-rows}
%
First, we show that a token $t^*$ contributes differently to the updates of $\bm{W}_{qk}$ depending on whether it serves as context for predicting other tokens or is itself predicted.
%
When $t^*$ is used to predict a set of tokens:
%
\begin{itemize}[noitemsep,nolistsep]
%
    \item All predicted tokens contribute to the column space of $\bm{W}_{qk}$ with the update of the $k$-th column scaled by the associated entry $[\bm{x}_{t^*}]_k$.
    \item Only the token $t^*$ contributes to the row space, with the update to the $m$-th row scaled a linear combination of  $[\bm{x}_i]_m$ for all predicted tokens.
%
\end{itemize}
%
On the other hand, when $t^*$ is predicted by a set of tokens:
%
\begin{itemize}[noitemsep,nolistsep]
%
    \item Only the token $t^*$ contributes to the column space, with the update to the $k$-th row scaled a linear combination of  $[\bm{x}_i]_k$ for all predicted tokens.
    \item All context tokens contribute to the row space with the update of the $m$-th column scaled by the associated entry $[\bm{x}_{t^*}]_m$.
%
\end{itemize}
%
We formalize this in the following Proposition.
%
\begin{proposition}
\label{prop-gradient-columns-rows}
%
(\textbf{Different implicit updates for context and prediction}).
%
%
Let $U = [t_0, \dots, t_N]$ be a sequence of tokens and let $\bm{x}_i \in \mathbb{R}^d$ be the embedding of the $i$-th token. 
%
Let $\Delta \bm{W}_{qk}$ be the weight update from Proposition \ref{prop-gradients-self-attention}.
%
Let $t^*$ be a given token in $U$.
%
When using $t^*$ as context, the $k$-th column of $\Delta \bm{W}_{qk}$ is given by
%
\begin{equation}
    \Delta \bm{w}_{\cdot, k} \bigg|_{t_j = t^*}= [\bm{x}_{t^*}]_k \left(\sum_{i \in P_{t^*}}\beta_{ij}\bm{x}_i\right)\,,
\end{equation}
%
while the $m$-th row of $\Delta \bm{W}_{qk}$ is given by
%
\begin{equation}
    \Delta \bm{w}_{m, \cdot} \bigg|_{t_j = t^*}= \left(\sum_{i \in P_{t^*}}\beta_{ij}[\bm{x}_i]_m\right)\bm{x}_{t^*} \,,
\end{equation}
%
where $P_{t^*}$ is the set of tokens predicted by $t^*$.
%
When predicting $t^*$, the $k$-th column of $\Delta \bm{W}_{qk}$ is given by
%
\begin{equation}
    \Delta \bm{w}_{k, \cdot} \bigg|_{t_i = t^*} = \left(\sum_{j \in C_{t^*}}\beta_{t^*j}[\bm{x}_j]_k\right)\bm{x}_{t^*} \,,
\end{equation}
%
while the $m$-th row of $\Delta \bm{W}_{qk}$ is given by
%
\begin{equation}
    \Delta \bm{w}_{m, \cdot} \bigg|_{t_i = t^*} = [\bm{x}_{t^*}]_m \left(\sum_{j \in C_{t^*}}\beta_{t^*j}\bm{x}_j^\top\right)\,.
\end{equation}
%
where $C_{t^*}$ is the set of context tokens for $t^*$.
%
\end{proposition}
%
% First we consider the gradient of a column for a given predictor-predicted pair $ij$, and derive the gradient associated to it
% \begin{equation}
%     \Delta \bm{w}_{\cdot, k}(i,j) = \Delta \bm{w}_{\cdot, k}\Big|_{\bm{t}_i \leftarrow \bm{t}_j}
%     = 
%      (\beta_{ij}[\bm{x}_j]_k)\,\bm{x}_i \,,
% \end{equation}
% where clearly the direction is given by the $\bm{x}_i$
% \begin{equation}
%     \Delta \bm{w}_{\cdot, k}(i,j) \mathbin{\|}  \bm{x}_i.
% \end{equation}
% Second, we note that the norm is proportional to the error ($\bm{t}_i - \sigma(\bm{z}_i)$ and is also proportional to the occurrence of $j$ before $i$. Assuming that most words have low predictive power, we would have that for each $i$ there are few $j$ with high gradients. 

% Now we look at the norm of the gradient for one column for the rest (for one of the high-gradient $j$)
% \begin{equation}
%     \dfrac{\|\Delta \bm{w}_{\cdot, k}(i,j)\|}{\sum_k \|\Delta \bm{w}_{\cdot, c}(i,j)\|} 
% \end{equation}
% and since $\beta_{ij}$, is the same scalar for all columns and the $\bm{x}_i$ is the same vector for all columns,
% \begin{equation}
%     \dfrac{\|\Delta \bm{w}_{\cdot, k}(i,j)\|}{\sum_c \|\Delta \bm{w}_{\cdot, k}(i,j)\|} 
%     = \dfrac{|[\bm{x}_j]_k|}{\sum_c |[\bm{x}_j]_k|}.
% \end{equation}
% Thus, for each pair $i,j$ the gradient would grow proportionally to $\bm{x}_j$ in the direction of $\bm{x}_i$.

% %Stop point 1: We assume that there is a low-rank structure in the data: few j that are predicting i. Then 
% Now we study how does the gradient change the attention, which is quantified
\begin{proof}
%
Let $\bm{U} = [t_0, \dots, t_N]$ a sequence of tokens with the embedding of every $i$-th token be given by $\bm{x}_i \in \mathbb{R}^d$.
%
The implicit weight update of Proposition \ref{prop-gradients-self-attention} can be decomposed with two equivalent regrouping of the double summations, as follows,
%
\begin{equation}
\Delta \bm{W}_{qk} = \sum_{(i,j) \in \bm{U}} \beta_{ij}\bm{x}_i\bm{x}_j^T 
= \sum_j \left(\sum_{i\in P_j} \beta_{ij}\bm{x}_i\right)\bm{x}^\top_j
= \sum_i \bm{x}_i\left(\sum_{i\in C_i} \beta_{ij}\bm{x}^\top_j\right)\,,
\end{equation}
%
where $P_{i} \subset U$ is the set of tokens predicted by a given token $t_i$, while $C_{j} \subset U$ is the set of tokens that predict a given token $t_j$.
%
%
We neglect any constant of proportionality - such as a learning rate - for simplicity.
%
%
For simplicity, we do not assume any specific structure on $P_i$ and $C_j$ (autoregressive training, bidirectional training, or others).
%
First, the contribution of $t^* \in U$ to the weight update when $t^*$ is used as context to predict a set of tokens $P_{t^*} \subset U$ is
%%
\begin{equation}
\Delta \bm{W}_{qk} \bigg|_{t_j = t^*}
= \left(\sum_{i\in P_{t^*}} \beta_{it^*}\bm{x}_i\right)\bm{x}^\top_{t^*}\,.
\end{equation}
%
The associated weight update of the $k$-th column is then given by
%
\begin{equation}
    \Delta \bm{w}_{\cdot, k} \bigg|_{t_j = t^*} = 
    \left(\sum_{i\in P_{t^*}} \beta_{it^*}\bm{x}_i\right)[\bm{x}_{t^*}]_k
    =
    [\bm{x}_{t^*}]_k \left(\sum_{i \in P_{t^*}}\beta_{it^*}\bm{x}_i\right)\,,
\end{equation}
%
while the update of the $m$-th row is
%
\begin{equation}
    \Delta \bm{w}_{m, \cdot} \bigg|_{t_j = t^*} = \left(\sum_{i \in P_{t^*}}\beta_{it^*}[\bm{x}_i]_m\right)\bm{x}_{t^*} \,.
\end{equation}
%
A complementary argument can be made when predicting a given token $t^*$ from a set of tokens $C_{t^*}$.
%
Indeed, the contribution to the weight update when $t^*$ is predicted by set of tokens $C_{t^*} \subset U$ is given by 
%
\begin{equation}
\Delta \bm{W}_{qk} \bigg|_{t_i = t^*}
= \left(\sum_{j\in C_{t^*}} \beta_{t^*j}\bm{x}_{t^*}\right)\bm{x}_j^\top\,.
\end{equation}
%
The associated weight update of the $k$-th column is then given by
%
\begin{equation}
    \Delta \bm{w}_{k, \cdot} \bigg|_{t_i = t^*} = \left(\sum_{j \in C_{t^*}}\beta_{t^*j}[\bm{x}_j]_k\right)\bm{x}_{t^*} \,,
\end{equation}
%
while the update of the $m$-th row is
%
\begin{equation}
    \Delta \bm{w}_{m, \cdot} \bigg|_{t_i = t^*} =  \left(\sum_{j \in C_{t^*}}\beta_{t^*j}\bm{x}_j^\top\right) [\bm{x}_{t^*}]_m
    = [\bm{x}_{t^*}]_m \left(\sum_{j \in C_{t^*}}\beta_{t^*j}\bm{x}_j^\top\right)\,.
\end{equation}
%
Therefore, the weight update of each column (row) when a set of tokens predicts $t^*$ is equivalent to the weight update of each row (column) when $t^*$ is used to predict a set of tokens.
%
This concludes the proof.
%
\end{proof}
%

%





%%
%% (2) 
\subsubsection{Asymmetric growth of rows and columns during weight update}
\label{supp-math-prop-gradient-asymmetric-growth-rows-columns}
%
Next, we demonstrate that under reasonable assumptions about the statistical distribution of token embeddings, the expected norm of column updates exceeds that of row updates when using the token $t^*$ to predict other tokens. 
%
Conversely, when $t^*$ is being predicted by other tokens, the row updates become dominant.
%
We begin by assuming that the token embeddings $\bm{x}_i$ are independent and identically distributed (i.i.d.) random vectors drawn from a probability distribution $\mathcal{D}$ with zero mean, $\mathbb{E}[\bm{x}_i] = \mathbf{0}$, and covariance matrix $\text{Cov}(\bm{x}_i) = \Sigma$.
%
This assumption holds true at initialization for any Transformer model with learnable embeddings. 
%
Additionally, we assume that the covariance matrix satisfies $\Sigma \neq \sigma^2 \mathbb{I}$,.
%
More specifically, we posit that there is partial alignment between the embeddings $\bm{x}_i$ due to the semantic and predictive relationships between tokens, which typically emerge in the vector embeddings during training.
%
Similar to Proposition \ref{prop-gradient-columns-rows}, the scenarios where $t^*$ is used to predict other tokens and where $t^*$ is being predicted by other tokens are complementary. 
%
In the following Proposition, we focus solely on the case where $t^*$ serves as context to predict a set of tokens. 
%
A formal derivation for the opposite case where $t^*$ is predicted by other tokens is provided in a subsequent Corollary.

\begin{proposition}
\label{prop-gradient-asymmetric-growth-rows-columns}
%
(\textbf{Asymmetric growth of columns and rows for context}).
%
Let $U = [t_0, \dots, t_N]$ a sequence of tokens, and let $t^*$ be a token representing the context of every token $t_i \in U$.
%
Let $\{\bm{x}_1 \dots, \bm{x}_N\}$ be the token embedding associated with $U$ such that each $\bm{x}_i \sim \mathcal{D}$ is a i.i.d. random vector drawn from a probability distribution $\mathcal{D}$ with zero mean and non-isotropic covariance $\text{Cov}(\bm{x}_i)  = \Sigma$.
%
Let $\bm{y}$ be the independent random embedding of $t^*$ with zero mean and covariance $\text{Cov}(\bm{y})  = \sigma_y^2\mathbb{I}$.
%
Let $\Delta \bm{W}_{qk}$ be the weight update from Proposition \ref{prop-gradients-self-attention}.
%
Then the squared norm of the $m$-th row and $k$-th column of $\bm{W}_{qk}$ satisfies
%
\begin{equation}
\frac{\mathbb{E}\left[ ||\Delta \bm{w}_{\cdot, k}||^2 \right]}{\mathbb{E}\left[ ||\Delta \bm{w}_{m, \cdot}||^2 \right]} > 1 \quad \forall k \in \{1,\dots,d\}, \forall m \,\,\text{s.t.} \,\, \Sigma_{m,m} < \frac{\text{Tr}(\Sigma)}{d}
\end{equation}
%
\end{proposition}
\begin{proof}
%
%
Let $U = [t_0, \dots, t_N]$ a sequence of tokens, and let $t^*$ be the context of every token $t_i \in U$. 
%
Let $\{\bm{x}_i\}$ be the set of token embeddings associated with $U$ and let $\bm{y}$ be the token embedding of $t^*$.
%
It follows from Proposition \ref{prop-gradient-columns-rows} that the squared norm of the weight update of the $k$-th column is given by 
%
%
\begin{equation}
\Delta \bm{w}_{\cdot, k} = \sum_{j \in \{t^*\}}\sum_{i\in P_j} \beta_{ij}\bm{x}_i[\bm{x_j}]_k  = \sum_{i=1}^N \beta_{it^*}\bm{x}_i[\bm{y}]_k 
%
\,\,\Rightarrow \,\,
%
||\Delta \bm{w}_{\cdot, k}||^2 = [\bm{y}]^2_k || \sum_{i=1}^N \beta_{it^*}\bm{x}_i||^2 \,,
\end{equation}
%
while the squared norm of the weight update of the $m$-th row by
%
%
\begin{equation}
\Delta \bm{w}_{m, \cdot} = \sum_{j \in \{t^*\}}\sum_{i\in P_j} \beta_{it^*}[\bm{x}_i]_m\bm{x}_j= \sum_{i=1}^N \beta_{it^*}[\bm{x}_i]_m\bm{y} 
%
\,\,\Rightarrow \,\,
%
||\Delta \bm{w}_{m, \cdot}||^2 = \big(\sum_{i=1}^N\beta_{it^*}[\bm{x}_i]_m\big)^2 ||  \bm{y}||^2 \,.
\end{equation}
%
Let $\{\bm{x}_i\} \sim \mathcal{D}$ be a set of i.i.d. random vectors from a probability distribution $\mathcal{D}$ such that $\mathbb{E}[\bm{x}_i] = 0$ and $\text{Cov}(\bm{x}_i) = \Sigma \in \mathbb{R}^{d,d}$.
%
Therefore, each $k$-th coordinate $[\bm{x}_i]_k$ is such that  $\mathbb{E}[[\bm{x}_i]_k] = 0$ and $\text{Var}([\bm{x}_i]_k) = \Sigma_{k,k}$. 
%
We also assume that $\bm{y}$ is statistically independent from $\bm{x}_i \, \forall i$ with $\mathbb{E}[\bm{y}] = 0$ and covariance $\text{Cov}(\bm{y})  = \sigma_y^2\mathbb{I}$.
%
It follows that the expected value of $||\Delta \bm{w}_{\cdot, k}||^2$ over  $\mathcal{D}$ is given by
%
\begin{equation}
\mathbb{E}\left[ ||\Delta \bm{w}_{\cdot, k}||^2 \right] 
= 
\mathbb{E}\left[ [\bm{y}]^2_k ||\sum_{i=1}^N \beta_{it^*}\bm{x}_i||^2 \right]
= 
\mathbb{E}\left[ [\bm{y}]^2_k \right] \mathbb{E}\left[ || \sum_{i=1}^N \beta_{it^*}\bm{x}_i||^2 \right] \,.
\end{equation}
%
Given the statistical independence between the entries of $\bm{x}_i$, the second term is equal to
%
\begin{equation}
\begin{split}
\mathbb{E}\left[ || \sum_{i=1}^N \beta_{it^*}\bm{x}_i||^2 \right]
& = \sum_{i,i'}\mathbb{E}\left[ \beta_{it^*}\beta_{i't^*}\bm{x}^\top_{i}\bm{x}_{i'}\right]
\\
&= \sum_i\beta^2_{it^*}\mathbb{E}\left[\bm{x}^\top_{i}\bm{x}_{i}\right] +
\sum_{i'\neq i}\beta_{it^*}\beta_{i't^*}\mathbb{E}\left[\bm{x}^\top_{i}\bm{x}_{i'}\right]
\\
& =
\sum_i\beta^2_{it^*}\text{Tr}\big(\mathbb{E}\left[\bm{x}_{i}\bm{x}_{i}^\top\right]\big) +
\sum_{i'\neq i}\beta_{it^*}\beta_{i't^*}\mathbb{E}\left[\bm{x}^\top_{i}\bm{x}_{i'}\right] 
\\
& = \text{Tr}(\Sigma)\sum_i\beta^2_{it^*} \,,
\end{split}
\end{equation}
%
and therefore
%
\begin{equation}
    \mathbb{E}\left[ ||\Delta \bm{w}_{\cdot, k}||^2 \right] = \Gamma_{k,k}\text{Tr} (\Sigma)\sum_i\beta^2_{it^*} \,.
\end{equation}
%
Similarly, the expected value of $||\Delta \bm{w}_{m, \cdot}||^2$ is given by
%
\begin{equation}
\mathbb{E}\left[ ||\Delta \bm{w}_{m, \cdot}||^2 \right] 
= 
\mathbb{E}\left[\big(\sum_{i=1}^N\beta_{it^*}[\bm{x}_i]_m\big)^2 ||  \bm{y}||^2
\right]
= 
\mathbb{E}\left[ \big(\sum_{i=1}^N\beta_{it^*}[\bm{x}_i]_m\big)^2 \right] \mathbb{E}\left[ ||  \bm{y}||^2\right]\,.
\end{equation}
%
The first term can be decomposed as
%
\begin{equation}
\mathbb{E}\left[ \big(\sum_{i=1}^n\beta_{it^*}[\bm{x}_i]_m\big)^2 \right]  =
\sum_{i=1}^n \beta^2_{it^*}\mathbb{E}\left[[\bm{x}_i]_m^2 \right]  + \sum_{i' \neq i}\beta_{it^*}\beta_{i't^*}\mathbb{E}\left[ [\bm{x}_{i'}]_m [\bm{x}_{i}]_m \right] =  \Sigma_{m,m}\sum_{i=1}^n\beta^2_{it^*} \,,
\end{equation}
%
and therefore
%
\begin{equation}
\mathbb{E}\left[ ||\Delta \bm{w}_{m, \cdot}||^2 \right] = \Sigma_{m,m}\text{Tr}(\Gamma)\sum_i\beta^2_{it^*}\,.
\end{equation}
%
The ratio of the expected value of these squared norms is
%
\begin{equation}
\frac{\mathbb{E}\left[ ||\Delta \bm{w}_{\cdot, k}||^2 \right]}{\mathbb{E}\left[ ||\Delta \bm{w}_{m, \cdot}||^2 \right]} = \frac{ \Gamma_{k,k}\text{Tr}(\Sigma)}{ \Sigma_{m,m}\text{Tr}(\Gamma)} = \frac{\Gamma_{k,k}}{\text{Tr}(\Gamma)}\frac{\text{Tr}(\Sigma)}{\Sigma_{m,m}} = \frac{1}{d}\frac{\text{Tr}(\Sigma)}{\Sigma_{m,m}} \,.
\end{equation}
%
We assume a non-isotropic covariance structure in $\Sigma$, that is, the average variance per dimension is lower than the total variance across all dimensions.
%
This implies $\text{Tr}(\Sigma) > d\,\Sigma_{m,m}$ for some $m$.
%
It follows that,
%
\begin{equation}
\frac{\mathbb{E}\left[ ||\Delta \bm{w}_{\cdot, k}||^2 \right]}{\mathbb{E}\left[ ||\Delta \bm{w}_{m, \cdot}||^2 \right]} = \frac{1}{d}\frac{\text{Tr}(\Sigma)}{\Sigma_{m,m}}  > \frac{1}{d}\frac{d \,\Sigma_{m,m}}{\Sigma_{m,m}} = 1 \quad \forall m \,\,\text{s.t.}\,\,\Sigma_{m,m} < \frac{\text{Tr}(\Sigma)}{d}\,,
\end{equation}
%
thus concluding the proof.
%
\end{proof}
%

%
Again, a complementary argument on asymmetric weight update can be made when a set of tokens $U = \{t_1, \dots, t_N\}$ is used to predict a given token $t^*$, that is, $C_i = \{t^*\} \,\,\forall t_i \in U$. We formalize this in the following Corollary.
%
\begin{corollary}
\label{corollary-gradient-asymmetric-growth-rows-columns}
%
(\textbf{Asymmetric growth of columns and rows for prediction}).
%
%
Let $U = [t_0, \dots, t_N]$ a sequence of tokens, and let every token $t_i \in U$ be the context of a given token $t^*$.
%
Let the associated embedding be drawn i.i.d. from a probability distribution $\mathcal{D}$ as in Proposition \ref{prop-gradient-asymmetric-growth-rows-columns}.
%
Let $\Delta \bm{W}_{qk}$ be the weight update from Proposition \ref{prop-gradients-self-attention}.
%
Then the squared norm of the $m$-th row and $k$-th column of $\bm{W}_{qk}$ satisfies
%
\begin{equation}
\frac{\mathbb{E}\left[ ||\Delta \bm{w}_{\cdot, k}||^2 \right]}{\mathbb{E}\left[ ||\Delta \bm{w}_{m, \cdot}||^2 \right]} < 1 \quad \forall m \in \{1,\dots,d\}, \forall k \,\,\text{s.t.} \,\, \Sigma_{k,k} < \frac{\text{Tr}(\Sigma)}{d}
\end{equation}
%
\end{corollary}
\begin{proof}
%
Let $U = [t_0, \dots, t_N]$ a sequence of tokens, and let $t^*$ be predicted by every token $t_i \in U$. 
%
Let $\{\bm{x}_i\}$ be the set of token embeddings associated with $U$ and let $\bm{y}$ be the token embedding of $t^*$.
%
It follows from Proposition \ref{prop-gradient-columns-rows} that the squared norm of the weight update of the $k$-th column is given by 
%
\begin{equation}
\Delta \bm{w}_{\cdot, k} = \sum_{i = \{t^*\}}\sum_{j\in C_i} [\bm{x}_j]_k\bm{x}_i= \sum_{j=1}^N [\bm{x}_j]_k\bm{y} 
%
\,\,\Rightarrow \,\,
%
||\Delta \bm{w}_{\cdot, k}||^2 = \big(\sum_{i=1}^N[\bm{x}_j]_k\big)^2 ||  \bm{y}||^2 \,,
\end{equation}
%
while the weight update of the $m$-th row follows
%
\begin{equation}
\Delta \bm{w}_{m, \cdot} = \sum_{i = \{t^*\}}\sum_{j\in C_i} \bm{x}_j[\bm{x_i}]_m  = \sum_{j=1} ^N\bm{x}_j[\bm{y}]_m 
\,\,\Rightarrow \,\,
||\Delta \bm{w}_{m, \cdot}||^2 = [\bm{y}]^2_m || \sum_{i=1}^N \bm{x}_j||^2 \,.
\end{equation}
%
Therefore, following the same arguments as in Proposition \ref{supp-math-prop-gradient-asymmetric-growth-rows-columns}, the ratio of the expected value of the square norms is given by
%
\begin{equation}
\frac{\mathbb{E}\left[ ||\Delta \bm{w}_{\cdot, k}||^2 \right]}{\mathbb{E}\left[ ||\Delta \bm{w}_{m, \cdot}||^2 \right]} = \frac{ \Sigma_{k,k}\text{Tr}(\Gamma)}{ \Gamma_{m,m}\text{Tr}(\Sigma)} = d\frac{\Sigma_{k,k}}{\text{Tr}(\Sigma)}  < \frac{d\,\Sigma_{k,k}}{d\,\Sigma_{k,k}} = 1 \quad \forall k \,\,\text{s.t.}\,\,\Sigma_{k,k} < \frac{\text{Tr}(\Sigma)}{d} \,,
\end{equation}
%
thus concluding the proof.
%
%
\end{proof}






%
%% (3) PROPOSITION COUNTING + PROOF
\subsubsection{Average contribution of a given token for context and prediction}
\label{supp-math-prop-counting-prediction-context}
%%
%%
%% PROVIDE ASSUMPTIONS ON MASKING AND BIDIRECTIONAL TRAINING
%
Additionally, we demonstrate that during autoregressive training, the expected number of tokens predicted by a specific token $t^*$ can differ from the expected number of tokens that predict $t^*$.
%
More specifically, in autoregressive training, the ratio between these two expected numbers is influenced by the statistical correlations between tokens in the dataset.
%
In contrast, for bidirectional training, the expected number of tokens predicted by $t^*$ and the number of tokens that predict $t^*$ are always equal. 
%
This equality holds true regardless of how tokens are correlated within the corpus, resulting in a ratio of 1.
%
We formalize this in the following Proposition.
%
\begin{proposition}
\label{prop-counting-prediction-context}
%
Let $\mathcal{V} = [t_0, \dots, t_V]$ be a set of tokens.
%
Let $\mathcal{U}$ be the sample space of all possible sequences of $N$ tokens, and let $U\in\mathcal{U}$ be a sequence $U = [t_1, \dots, t_N]$.
%
Let $\Pr[t_j = t^*]$ be the probability that the token at index $j$ in is given by $t^* \in \mathcal{V}$.
%
Let $\mathbb{E}[\mu_c(t^*)]$ be the expected number of tokens that are predicted by a given token $t^*$, and $\mathbb{E}[\mu_p(t^*)]$ the expected number of tokens that predict a given token $t^*$.
%
For autoregressive training, the ratio
%
\begin{equation}
    \frac{\mathbb{E}[\mu_c(t^*)]}{\mathbb{E}[\mu_p(t^*)]} = \frac{\sum_{k=1}^N (N-k)\Pr[t_k = t^*]}{\sum_{k=1}^N(k-1)\Pr[t_k = t^*]} \,,
\end{equation}
%
depends on $\Pr[t_j = t^*]$, while for bidirectional training the same ratio is given by 
%
\begin{equation}
     \frac{\mathbb{E}[\mu_c(t^*)]}{\mathbb{E}[\mu_p(t^*)]} = 1 \,.
\end{equation}
%    
\end{proposition}
\begin{proof}
%
Let $\mathcal{V} = [t_0, \dots, t_V]$ be a set of tokens.
%
Let $\mathcal{U}$ be the sample space of all possible sequences of $N$ tokens, and let $U$ be a sequence $U = [t_1, \dots, t_N] \in\mathcal{U}$.
%
Let $E = \{U \in \mathcal{U}: t_k = t^*\}$ the event of all sequences where the $k$-th token is $t^*$. 
%
The indicator function $\mathds{1}\{E\}(U)$ is a random variable defined as
%
\begin{equation}
\mathds{1}\{E\}(U) =
\begin{cases}
    1 \,\,\text{if} \,\,t_k = t^*\\
    0 \,\, \text{otherwise} \,,
\end{cases}
\end{equation}
%
and as such its expected value over $\mathcal{U}$ is
%
\begin{equation}
\mathbb{E}\left[\mathds{1}\{t_k = t^*\}(U)\right] = \sum_{U \in \,\mathcal{U}} \mathds{1}\{t_k = t^*\}(U)\Pr[U] = \sum_{U \in \mathcal{U} \,:\, w_j = t^*} \Pr[U] = \Pr[t_k = t^*] \,.
\end{equation}
%  
Let $\mu_c(t^*)$ be the random variable quantifying the number of tokens predicted by $t^*$, while $\mu_p(t^*)$ is the random variable quantifying the number of tokens predicted by $t^*$.
%
We analyzed the case of autoregressive and bidirectional training separately.
%

During autoregressive training, each time the token $t^*$ appears at position $k$ it is used as context to predict $N-k$ tokens, and it is predicted by $k-1$ tokens.
%
It follows that $\mu_c(t^*)$ is given by 
%
\begin{equation}
\mu_c(t^*) = \sum_{l=2}^N\sum_{k=1}^{l-1} \mathds{1}\{t_k = t^*\} = \sum_{k=1}^N (N-k)\mathds{1}\{t_k = t^*\}\,,
\end{equation}
%
while $\mu_p(t^*)$ is given by 
%
\begin{equation}
\mu_p(t^*) = \sum_{l=1}^{N-1}\sum_{k=1}^{l-1} \mathds{1}\{t_l = t^*\} = \sum_{k=1}^N (k-1)\mathds{1}\{t_k = t^*\}\,.
\end{equation}
%
Therefore, the expected value of $\mu_c(t^*)$ over all possible sequences is
%
\begin{equation}
\mathbb{E}[\mu_c(t^*)] = \mathbb{E}\left[\sum_{k=0}^N (N-k)\mathds{1}\{t_k = t^*\}\right] =   \sum_{k=0}^N(N-k)\Pr[t_k = t^*]\,,
\end{equation}
%
the expected value of $\mu_p(t^*)$ is
%
\begin{equation}
\mathbb{E}[\mu_p(t^*)] = \mathbb{E}\left[\sum_{k=0}^N (k-1)\mathds{1}\{t_k = t^*\}\right] =   \sum_{k=0}^N(k-1)\Pr[t_k = t^*]\,,
\end{equation}
%
and their ratio is given by
%
\begin{equation}
    \frac{\mathbb{E}[\mu_c(t^*)]}{\mathbb{E}[\mu_p(t^*)]} = \frac{\sum_{k=1}^N (N-k)\Pr[t_k = t^*]}{\sum_{k=1}^N(k-1)\Pr[t_k = t^*]} \,.
\end{equation}
%

During bidirectional training, each time the token $t^*$ is masked at position $k$, it is predicted by $N-1$ tokens.
%
We assume that the token $t^*$ significantly contributes to the context of a masked token only if it is itself not masked.
%
During bidirectional training, whenever a token $t^*$ at position 
$k$ is masked, it is predicted using the other $N-1$ tokens in the sequence.
%
We assume that $t^*$ significantly contributes to the context for predicting the masked token only if
$t^*$ itself is not masked.
%
Additionally, we assume that the probability $\rho$ of masking any given position $k$ is the same for all positions and does not depend on the specific token $t_k$ at that position.
%
It follows that $\mu_c(t^*)$ is given by 
%
\begin{equation}
\mu_c(t^*) = \sum_{k=1}^N\mathds{1}\{t_k = t^*\} \mathds{1}\{\text{mask}(k) = 0\}\sum_{k' \neq k} \mathds{1}\{\text{mask}(k') = 1\}  \,,
\end{equation}
%
while $\mu_p(t^*)$ is given by 
%
\begin{equation}
\mu_p(t^*) = \sum_{k=1}^N\mathds{1}\{t_k = t^*\} \mathds{1}\{\text{mask}(k) = 1\}\sum_{k' \neq k} \mathds{1}\{\text{mask}(k') = 0\}\,.
\end{equation}
%
Therefore, their expected values are given by 
%
\begin{equation}
\begin{split}
\mathbb{E}[\mu_c(t^*)] 
& =  \mathbb{E} \left[\sum_{k=1}^N\mathds{1}\{t_k = t^*\} \mathds{1}\{\text{mask}(k) = 0\}\sum_{k' \neq k} \mathds{1}\{\text{mask}(k') = 1\} \right] \\
& = \sum_{k=1}^N \mathbb{E} \left[\mathds{1}\{t_k = t^*\} \mathds{1}\{\text{mask}(k) = 0\}\sum_{k' \neq k} \mathds{1}\{\text{mask}(k') = 1\} \right]\\
& = \sum_{k=1}^N \mathbb{E} \left[\mathds{1}\{t_k = t^*\} \mathds{1}\{\text{mask}(k) = 0\}\right]\sum_{k' \neq k} \left[\mathds{1}\{\text{mask}(k') = 1\} \right]\\
& = \sum_{k=1}^N \text{Pr}(t_k = t^*) \text{Pr}(\text{mask}(k) = 0)\sum_{k' \neq k} \text{Pr}(\text{mask}(k') = 1) \\
& = N\rho(N-1)(1 - \rho) \sum_{k=1}^N \text{Pr}(t_k = t^*)\,,
\end{split}
\end{equation}
%
and
%
\begin{equation}
\begin{split}
\mathbb{E}[\mu_p(t^*)] 
& =  \mathbb{E} \left[\sum_{k=1}^N\mathds{1}\{t_k = t^*\} \mathds{1}\{\text{mask}(k) = 1\}\sum_{k' \neq k} \mathds{1}\{\text{mask}(k') = 0\} \right] \\
& = N\rho(N-1)(1 - \rho)\sum_{k=1}^N \text{Pr}(t_k = t^*)  \,,
\end{split}
\end{equation}
%
and their ratio is
\begin{equation}
     \frac{\mathbb{E}[\mu_c(t^*)]}{\mathbb{E}[\mu_p(t^*)]} = 1 \,,
\end{equation}
%
thus concluding the proof.
%











% Assuming that $\text{Pr}[t_j = t^*] = p(t^*)$.
% %
% For autoregressive training, 
% %
% \begin{equation}
% \mathbb{E}[\mu_c(t^*)] = \mathbb{E}\left[\sum_j (N-j)\mathds{1}\{t_j = t^*\}\right] =   \sum_{j=0}^N(N-j)\Pr[t_j = t^*] =  \left(\sum_{j=0}^N(N-j)\right)p(t^*) = \frac{N(N-1)}{2}p(t^*)\,.
% \end{equation}
% %
% \begin{equation}
% \mathbb{E}[\mu_p(t^*)] = \mathbb{E}\left[\sum_j (j-1)\mathds{1}\{t_j = t^*\}\right] = \sum_{j=0}^N(j-1)\Pr[t_j = t^*] =  \left(\sum_{j=0}^N(j-1)\right)p(t^*)= \frac{N(N-1)}{2}p(t^*)\,.
% \end{equation}
% %
% \begin{equation}
%      \frac{\mathbb{E}[\mu_c(t^*)]}{\mathbb{E}[\mu_p(t^*)]} = 1 \,.
% \end{equation}
% %    
% %
% For bidirectional training, 
% %
% \begin{equation}
% \begin{split}
% \mathbb{E}[\mu_c(t^*)] 
% & =  \mathbb{E} \left[\sum_{j=1}^N\mathds{1}\{t_j = t^*\} \mathds{1}\{\text{mask}(j) = 0\}\sum_{j' \neq j} \mathds{1}\{\text{mask}(j') = 1\} \right] \\
% & = \sum_{j=1}^N \mathbb{E} \left[\mathds{1}\{t_j = t^*\} \mathds{1}\{\text{mask}(j) = 0\}\sum_{j' \neq j} \mathds{1}\{\text{mask}(j') = 1\} \right]\\
% & = \sum_{j=1}^N \mathbb{E} \left[\mathds{1}\{t_j = t^*\} \mathds{1}\{\text{mask}(j) = 0\}\right]\sum_{j' \neq j} \left[\mathds{1}\{\text{mask}(j') = 1\} \right]\\
% & = \sum_{j=1}^N \text{Pr}(t_j = t^*) \text{Pr}(\text{mask}(j) = 0)\sum_{j' \neq j} \text{Pr}(\text{mask}(j') = 1) \\
% & = p(t^*)N\rho(N-1)(1 - \rho)
% \end{split}
% \end{equation}
% %
% \begin{equation}
% \begin{split}
% \mathbb{E}[\mu_c(t^*)] 
% & =  \mathbb{E} \left[\sum_{j=1}^N\mathds{1}\{t_j = t^*\} \mathds{1}\{\text{mask}(j) = 1\}\sum_{j' \neq j} \mathds{1}\{\text{mask}(j') = 0\} \right] \\
% & = p(t^*)N\rho(N-1)(1 - \rho)
% \end{split}
% \end{equation}
% %
% \begin{equation}
%      \frac{\mathbb{E}[\mu_c(t^*)]}{\mathbb{E}[\mu_p(t^*)]} = 1 \,.
% \end{equation}
% %

% Relaxing the assumption on $\text{Pr}[t_j = t^*] = p(t^*)$.
% %
% For autoregressive training,
% %
% \begin{equation}
%     \frac{\mathbb{E}[\mu_c(t^*)]}{\mathbb{E}[\mu_p(t^*)]} = \frac{\sum_{j=1}^N (N-j)\Pr[t_j = t^*]}{\sum_{j=1}^N(j-1)\Pr[t_j = t^*]} \,,
% \end{equation}
% %
% For bidirectional training
% %
% \begin{equation}
%     \frac{\mathbb{E}[\mu_c(t^*)]}{\mathbb{E}[\mu_p(t^*)]} =  \frac{\rho(N-1)(1 - \rho)\sum_{j=1}^N\Pr[t_j = t^*]}{\rho(N-1)(1 - \rho)\sum_{j=1}^N\Pr[t_j = t^*]} = 1 \,.
% \end{equation}
% %
% thus concluding the proof.



% %
% % Let $\lambda_p(t^*)$ and $\lambda_c(t^*)$ be the number of times $t^*$ is predicted and used as context in a given sequence, respectively.
% % %
% % First, we derive the expected value of $\lambda_p(t^*)$ and $\lambda_c(t^*)$ for the case in which there is no correlation structure between tokens, $\Pr[t_j = t^*] = p(t^*) \,\, \forall t^* \in V$. 
% % %
% % %and as such $\Pr[t_1 = t^*_1, \dots, t_N = t^*_N] = \Pi_i p(t^*_i)$.
% % %
% % During autoregressive training, a token $t^*$ is predicted every time it appears in a given sequence, and it is used as context to predict every subsequent token.
% % %
% % The expected value of $\lambda_p(t^*)$ is thus given by,
% % %
% % \begin{equation}
% %     \langle \lambda_p(t^*) \rangle = \mathbb{E} \left[\sum_{j=0}^N \mathds{1}\{t_j = t^*\} \right] = \sum_{j=0}^N \Pr[t_j = t^*] = N p(t^*)\,,
% % \end{equation}
% % %
% % where $\mathds{1}\{t_j = t^*\}$ be the indicator function for the event $t_j = t^*$.
% % %
% % The expected value of $\lambda_c(t^*)$ is,
% % %
% % \begin{equation}
% %     \langle \lambda_c(t^*) \rangle = \mathbb{E} \left[\sum_{j'=1}^N \sum_{j=0}^{j'}\mathds{1}\{t_j = t^*\} \right] = \sum_{j=0}^N(N-j)p(t^*) = \frac{N(N-1)}{2}p(t^*)\,.
% % \end{equation}
% % %
% % The ratio of $\langle \lambda_c(t^*) \rangle$ and $\langle \lambda_p(t^*) \rangle$ is thus given by
% % %
% % \begin{equation}
% %     \frac{\langle \lambda_c(t^*) \rangle}{\langle \lambda_p(t^*) \rangle} = \frac{N-1}{2} \,,
% % \end{equation}
% % %
% % which does not depend on $p(t^*)$.
% % %
% % During bidirectional training, a token $t^*$ is predicted every time it appears in a position masked with probability $\rho$, and it is used as context to predict every masked token in the sequence.
% % %
% % The expected $\langle \lambda_p(t^*) \rangle$ is thus given by,
% % %
% % \begin{equation}
% %     \langle \lambda_p(t^*) \rangle = 
% %     \mathbb{E} \left[\sum_{j=1}^N\mathds{1}\{t_j = t^*\} \mathds{1}\{\text{mask}(j) = 1\} \right] = \sum_{j=1}^N\Pr[t_j = t^*]\Pr[\text{mask}(j) = 1] = \rho N p(t^*)\,
% % \end{equation}
% % %
% % while $\langle \lambda_c(t^*) \rangle$ is given by
% % %
% % \begin{equation}
% % \begin{split}
% %     \langle \lambda_c(t^*) \rangle 
% %     & = 
% %     \mathbb{E} \left[\sum_{j=1}^N\mathds{1}\{t_j = t^*\} \mathds{1}\{\text{mask}(j) = 0\}\sum_{j' \neq j} \mathds{1}\{\text{mask}(j') = 1\} \right] 
% %     \\
% %     & = (1-\rho)Np(t^*)(N-1)\rho \,,
% % \end{split}
% % \end{equation}
% % %
% % where we assume independence between random masking (the presence or absence of a mask at $j$ is  independent of masks at the other positions $\{1,\dots,N\} \ j$), thus obtaining
% % %
% % \begin{equation}
% %     \frac{\langle \lambda_c(t^*) \rangle}{\langle \lambda_p(t^*) \rangle} = (N-1)(1 - \rho)\,,
% % \end{equation}
% % %
% % which also does not depend on $p(t^*)$.
% % %
% % We now generalize this to the case where there are correlation structures between tokens, $\Pr[t_j = t^*] \neq p(t^*)$.
% % %
% % This is the case when (1) there are correlations between tokens such as bigrams, trigrams, etc (these correlations can cause certain tokens to appear in contexts that precede more tokens or fewer tokens), (2) if data is chunked into sequences of different lengths, or if some tokens appear more often in the shorter and longer segment, and (3) there are correlations in higher-level structures such as paragraphs, sections, and documents.
% % %
% % For example, 
% % %
% % For unidirectional training, $\langle \lambda_p(t^*) \rangle$ is given by
% % %
% % \begin{equation}
% %     \langle \lambda_p(t^*) \rangle = \sum_{j=0}^N \Pr[t_j = t^*]\,,
% % \end{equation}
% % %
% % while $\langle \lambda_c(t^*) \rangle$ is given by 
% % %
% % \begin{equation}
% %     \langle \lambda_c(t^*) \rangle = \sum_{j=0}^N (N-j)\Pr[t_j = t^*]\,.
% % \end{equation}
% % %
% % The ratio is thus given by 
% % %
% % \begin{equation}
% %     \frac{\langle \lambda_c(t^*) \rangle}{\langle \lambda_p(t^*) \rangle} = \frac{\sum_{j=0}^N (N-j)\Pr[t_j = t^*]}{\sum_{j=0}^N \Pr[t_j = t^*]} \,.
% % \end{equation}
% % %
% % On the other hand, bidirectional training leads to
% % %
% % \begin{equation}
% %     \langle \lambda_p(t^*) \rangle = \rho \sum_{j=0}^N\Pr[t_j = t^*]\,,
% % \end{equation}
% % %
% % and
% % %
% % \begin{equation}
% %     \langle \lambda_c(t^*) \rangle = (N-1)(1 - \rho)\rho\sum_{j=0}^N \Pr[t_j = t^*]\,,
% % \end{equation}
% % %
% % where the ratio is given by
% % %
% % \begin{equation}
% %     \frac{\langle \lambda_c(t^*) \rangle}{\langle \lambda_p(t^*) \rangle} = \frac{(N-1)(1 - \rho)\rho\sum_{j=0}^N \Pr[t_j = t^*]}{\rho \sum_{j=0}^N\Pr[t_j = t^*]} = (1 - \rho)(N-1)\,,
% % \end{equation}
% % %
% % which is independent of $\Pr[t_j = t^*]$, thus concluding the proof.
% % %
\end{proof}
%

Let us assume that there is no statistical correlation between the tokens in $U$, we can factorize the joint probability $\Pr[U]$ as follows
%
\begin{equation}
    \Pr[U] = \Pr[t_1,\dots,t_N] = \Pi_{i=1}^N\Pr[t_i]\,.
\end{equation}
%
Additionally, let us assume that each token is identically distributed independently of the position.
%
Therefore, during autoregressive training, the expected number  $\mathbb{E}[\mu_c(t^*)]$ can be simplified as
%
\begin{equation}
\mathbb{E}[\mu_c(t^*)] = \sum_{k=0}^N(N-k)\Pr[t_k = t^*] = \sum_{k=0}^N(N-k)\Pr[t^*] = \Pr[t^*]\frac{N(N-1)}{2} \,,
\end{equation}
%
and the same is true for $\mathbb{E}[\mu_p(t^*)]$
%
\begin{equation}
\mathbb{E}[\mu_p(t^*)] = \sum_{k=0}^N(k-1)\Pr[t_k = t^*] = \sum_{k=0}^N(k-1)\Pr[t^*] = \Pr[t^*]\frac{N(N-1)}{2} \,.
\end{equation}
%
It follows that, in the case of statistical independence between tokens, the expected number of tokens predicted by $t^*$ and the expected number of tokens predicting $t^*$.
%
Under these assumptions, there is no difference between autoregressive and bidirectional training.

On the other hand, let us assume a general joint probability between tokens, $\Pr[U] = \Pr[t_1,\dots,t_N]$.
%
When conditioning the probabilities over the future tokens, we obtain a factorization of the joint probability as follows,
%
%
\begin{equation}
    \Pr[U] = \prod_{i=1}^{N} \Pr[t_i \mid t_{i+1}, \dots, t_N]\,.
\end{equation}
%
The probability of the $k$-th token being $t^*$ across the possible sequences is thus given by
%
\begin{equation}
\begin{split}
\Pr[t_k = t^*] & = \sum_{t_1, \dots,t_N} \mathds{1}\{t_k = t^*\}\Pr[t_1,\dots,t_n] \\
&= \sum_{\substack{t_1, \dots, t_{k-1}, t_{k+1}, \dots, t_N}} \left( \prod_{\substack{l=1 \\ l \neq k}}^{T} \Pr[t_l | t_{l+1}, \dots, t_N] \right) \cdot \Pr[t_k = t | t_{k+1}, \dots, t_N] \,,
\end{split}
\end{equation}
%
leading to the ratio
%
\begin{equation}
    \frac{\mathbb{E}[\mu_c(t^*)]}{\mathbb{E}[\mu_p(t^*)]} = \frac{\sum_{k=1}^N (N-k)\sum_{\substack{t_1, \dots, t_{k-1}, t_{k+1}, \dots, t_N}} \left( \prod_{\substack{l=1 \\ l \neq k}}^{T} \Pr[t_l | t_{l+1}, \dots, t_N] \right) \cdot \Pr[t_k = t | t_{k+1}, \dots, t_N]}{\sum_{k=1}^N(k-1)\sum_{\substack{t_1, \dots, t_{k-1}, t_{k+1}, \dots, t_N}} \left( \prod_{\substack{l=1 \\ l \neq k}}^{T} \Pr[t_l | t_{l+1}, \dots, t_N] \right) \cdot \Pr[t_k = t | t_{k+1}, \dots, t_N]} \,.
\end{equation}
%
Therefore, when $t^*$ has a high probability of occurring before a given set of tokens, or when $t^*$ is likely to occur at the beginning of sentences and specific pieces of text, this ratio is likely to be higher than 1.






%% (5) THEOREM DIRECTIONALITY
\subsubsection{Theorem on the emergence of directionality}
\label{supp-math-theo-gradient-directionality}
%
Here, we show that, under the same assumptions as in Proposition \ref{prop-gradient-asymmetric-growth-rows-columns}, the weight updates of $\bm{W}_{qk}$ induce column dominance during autoregressive training. 
%
Specifically, there are on average more columns with high norms than rows with high norms.
%
We assume that, at initialization, the entries of $\bm{W}_{qk}$ are drawn from a probability distribution $\mathcal{P}$ with finite mean $\mu$ and variance $\sigma^2$. 
%
This assumption holds for any standard machine learning initialization scheme.
%
This implies that each $k$-th column $\bm{w}_{\cdot,k}$ and $m$-th row $\bm{w}_{m, \cdot}$ have the same mean $\mu$ and variance $\sigma^2/\sqrt{n}$.
%
Furthermore, the squared norm of both rows and columns have equal mean $n(\sigma^2 + \mu^2)$ and variance $n(2\sigma^4 + 4\mu^2)$.
%


%% LEMMA OUTLIERS
%
To demonstrate the column dominance, we first show that if one probability distribution has a higher variance than another, it is more likely to produce samples with higher values.
%
We formalize this in the following Lemma.
%
%
\begin{lemma}
\label{lemma-same_mean_different_variance_bound} 
%
Let two probability distributions $a,b$ have the same mean $\mu$ and different variance $\sigma_a^2>\sigma_b^2$. 
%
The probability of a random sample of each distribution being larger than a value $z$ is higher for samples from $a$ than from $b$ for $z>\sqrt{\sigma_a \sigma_b} - \mu$,
%
\begin{equation}
    \text{Pr}\left[X_a>z\right]\geq \text{Pr}\left[X_b>z\right]\quad \forall z> \sqrt{\sigma_a \sigma_b} - \mu
\end{equation}
%
\end{lemma}
%
\begin{proof}
%
Our goal is to find a $z$ such that
\begin{equation}
    \text{Pr}\left[X_a>z\right]\geq \text{Pr}\left[X_b>z\right]%\quad \forall z> \sqrt{\sigma_a \sigma_b} - \mu
\end{equation}
for which we need the following steps: A lower bound for $\text{Pr}\left[X_a>z\right]$, an upper bound for $\text{Pr}\left[X_b>z\right]$, and a value of $z$ that makes the upper bound lower than the lower bound.

We start by using Chebyshev's inequality, to derive an upper bound for $\text{Pr}\left[X_b>z\right]$,
\begin{equation}
    \text{Pr}\left[X_b>z\right]= 
     \text{Pr}\left[X_b - \mu >q\sigma_b\right]
     \leq  \dfrac{1}{q^2} = \left(\dfrac{\sigma_b}{z-\mu}\right)^2.
\end{equation}

Now compute a lower bound for $\text{Pr}\left[X_b>z\right]$ through the second moment method (which is very similar to the previous one, but inverted),
\begin{equation}
    \text{Pr}\left[X_a>z\right] = \text{Pr}\left[X_a-z>0\right]
    \geq  \dfrac{\left(\mathbb{E}\left[X_a- z\right]\right)^2}
    {\mathbb{E}\left[\left(X_a-z \right)^2 \right]}
    = \left(\dfrac{\mu - z}{\sigma_a}\right)^2
\end{equation}
Thus, we now only need to find $z$ satisfying
\begin{equation}
    \left(\dfrac{\sigma_b}{z-\mu}\right)^2 <  \left(\dfrac{z-\mu}{\sigma_a}\right)^2
\end{equation}
for the case $z>\mu$, the inequality is fulfilled when
\begin{equation}
    z>\sqrt{\sigma_a \sigma_b }-\mu.
\end{equation}
%
\end{proof}



%

We now formalize the connection between this result and the asymmetric growth of columns in the following Theorem.
%
\begin{theorem}
\label{theo-gradient-directionality}
%
(\textbf{Autoregressive training induces directionality})
%
Let $\mathcal{V} = [t_0, \dots, t_V]$ be a set of tokens.
%
Let $U$ be an ordered sequence of $N$ tokens $U = [t_1, \dots, t_N]$ where $\Pr[t_j = t^*]$ is the probability that the token at index $j$ in $U$ is given by $t^* \in \mathcal{V}$.
%
Let $\{\bm{x}_1 \dots, \bm{x}_n\}$ be the embedding associated with $\mathcal{V}$ such that each $\bm{x}_i \sim \mathcal{D}$ is drawn i.i.d. from a probability distribution $\mathcal{D}$ with zero mean and semipositive covariance $\text{Cov}(\bm{x}_i)  \neq \sigma_x^2\mathbb{I}$.
%
Let $\Delta \bm{W}_{qk}$ be the weight update of $\bm{W}_{qk}$ from Proposition \ref{prop-gradients-self-attention}.
%
Let $\Delta \bm{W}_{qk}$ be derived from an autoregressive objective function as in Definition \ref{def-objective-functions}.
%
Then, there exists a value $\gamma \in \mathbb{R}$ such that,
%
\begin{equation}
    \text{Pr}[||\bm{w}_{\cdot, k}|| > w] > \text{Pr}[|| \bm{w}_{m, \cdot}|| > w] \,\,\ \forall \,w > \gamma
\end{equation}
%
% Then, for autoregressive training, the squared norm of the $m$-th row and $k$-th column satifies
% %
% \begin{equation}
% \frac{\mathbb{E}\left[ ||\Delta \bm{w}_{\cdot, k}||^2 \right]}{\mathbb{E}\left[ ||\Delta \bm{w}_{m, \cdot}||^2 \right]} > 1 \quad \forall k,m \in \{1,\dots,d\}\,,
% \end{equation}
% %
% and the number of columns with high-norm is higher than the number of rows with high-norm.
%
\end{theorem}
\begin{proof}
%


Let $\mathbb{E}[\mu_c(t^*)]$ be the expected number of tokens that are predicted by a given token $t^*$, and $\mathbb{E}[\mu_p(t^*)]$ the expected number of tokens that predict a given token $t^*$.
%
When assuming autoregressive training, it follows from Proposition \ref{prop-counting-prediction-context} that
%
\begin{equation}
    \frac{\mathbb{E}[\mu_c(t^*)]}{\mathbb{E}[\mu_p(t^*)]} = \frac{\sum_{j=1}^N (N-j)\Pr[t_j = t^*]}{\sum_{j=1}^N(j-1)\Pr[t_j = t^*]} \,.
\end{equation}
%
Let the statistical correlation between tokens in $U$,
%
\begin{equation}
\frac{\mathbb{E}[\mu_c(t^*)]}{\mathbb{E}[\mu_p(t^*)]} > 1\,, 
\end{equation}
%
for most $t^* \in \mathcal{V}$.
%
It follows that the average number of tokens that are predicted by $t^*$ is smaller than the average number of tokens that predict it.
%
Therefore, the number of times $t^*$ contributes to the weight update of $\bm{W}_{qk}$ as predicted token is on average smaller than the number of times it contributes as context token.
%
It follows from Proposition \ref{prop-gradient-asymmetric-growth-rows-columns} that the number of times the following is true,
%
\begin{equation}
\frac{\mathbb{E}\left[ ||\Delta \bm{w}_{\cdot, k}||^2 \right]}{\mathbb{E}\left[ ||\Delta \bm{w}_{m, \cdot}||^2 \right]} > 1 \,,
\end{equation}
%
is on average higher than the number of times the following is true,
%
\begin{equation}
\label{eq:varianceGrowthColsVsRows}
\frac{\mathbb{E}\left[ ||\Delta \bm{w}_{\cdot, k}||^2 \right]}{\mathbb{E}\left[ ||\Delta \bm{w}_{m, \cdot}||^2 \right]} < 1 \,.
\end{equation}
%
Therefore, the net increase of the norm of the columns is bigger than the net increase of the norm of the rows. 
%
We now prove that the number of columns with a high norm exceeds the number of rows with a high norm.
%
At initialization, let each $(k,m)$ entry of $\bm{W}_{qk}$ be an i.i.d. random variable drawn from a probability distribution with mean $\mu$ and variance $\sigma^2$.
%
It follows that the mean norm of rows and columns are equal at initialization, as follows,
%
\begin{equation}
\mathbb{E} \left[|\bm{w}_{m, \cdot}|\right]\bigg|_{\text{init}} = \mathbb{E} \left[|\bm{w}_{\cdot, k}|\right]\bigg|_{\text{init}} = \mu \,\, \forall k,m\,.
\end{equation}
%
Furthermore, the variance of the norms of the rows $\sigma_\text{rows}^2$ and of the columns $\sigma_\text{cols}^2$ are equal at initialization, 
%
\begin{equation}
    \text{Var}\left(|\bm{w}_{m, \cdot}|\right)\bigg|_{\text{init}} = \text{Var}\left(|\bm{w}_{\cdot, k, \cdot}|\right)\bigg|_{\text{init}} =\sigma^2 \,\, \forall k,m\,.
\end{equation}
%
Therefore, the following inequality holds at the end of training,
%
\begin{equation}
    \text{Var}\left(|\bm{w}_{m, \cdot}|\right) > \text{Var}\left(|\bm{w}_{ \cdot, k}|\right) \,\, \forall k,m \,.
\end{equation}
%
Finally, we obtain from Lemma~\ref{lemma-same_mean_different_variance_bound} that the following inequality holds,
%
\begin{equation}
    \text{Pr}[||\bm{w}_{\cdot, k}|| > w] > \text{Pr}[|| \bm{w}_{m, \cdot}|| > w] \,\,\ \forall \,w > \gamma
\end{equation}
%
where $\gamma = \sqrt{\text{Var}\left(|\bm{w}_{m, \cdot}|\right)\text{Var}\left(|\bm{w}_{\cdot, k}|\right)}-\mu $, thus concluding the proof.
%
%
% It follows that the mean absolute value of both rows $\mu_\text{rows}$ and columns $\mu_\text{cols}$ are equal, that is, 
% \begin{equation}
%     \mu_{\text{cols}} = \mathbb{E}_r\left[\mathbb{E}_c\left[ |\bm{w}_{c,r}| \right]\right]
%     =
%     \mathbb{E}_c\left[\mathbb{E}_r\left[ |\bm{w}_{c,r}| \right]\right] = \mu_{\text{rows}}\,,
% \end{equation}
% where $r,c$ indicate rows and columns, respectively.
%
\end{proof}
%







%% THEOREM SYMMETRY
\subsubsection{Theorem on the emergence of symmetry}
\label{supp-math-theo-symmetry-gradients}
%
Finally, we prove that the weight updates of $\bm{W}_{qk}$ induce symmetry during bidirectional training.
%
Importantly, the column dominance is present only during autoregressive training.
%
Indeed, it follows from Proposition \ref{prop-counting-prediction-context} that, during bidirectional training, the net increase of the norm of the columns is equal to the net increase of the norm of the rows.
%
We formalize this in the following Corollary.
%
\begin{corollary}
%
(\textbf{Bidirectional training induces rows and columns with equal norm})
%
Let $\mathcal{V} = [t_0, \dots, t_V]$ be a set of tokens.
%
Let $U$ be an ordered sequence of $N$ tokens $U = [t_1, \dots, t_N]$ where $\Pr[t_j = t^*]$ is the probability that the token at index $j$ in $U$ is given by $t^* \in \mathcal{V}$.
%
Let $\{\bm{x}_1 \dots, \bm{x}_n\}$ be the random embedding of the tokens as in Theorem \ref{theo-gradient-directionality}.
%
Let $\Delta \bm{W}_{qk}$ be the weight update of $\bm{W}_{qk}$ from Proposition \ref{prop-gradients-self-attention}.
%
Let $\Delta \bm{W}_{qk}$ be derived from a bidirectional objective function as in Definition \ref{def-objective-functions}.
%
Then, the variance of the norm of the rows and the columns at the end of training is equal,
%
%
\begin{equation}
    \text{Var}\left(|\bm{w}_{m, \cdot}|\right) =\text{Var}\left(|\bm{w}_{ \cdot, k}|\right) \,\, \forall k,m \,.
\end{equation}
%
\end{corollary}
\begin{proof}
%
It follows from Proposition \ref{prop-counting-prediction-context} that the number of times the following is true,
%
\begin{equation}
\frac{\mathbb{E}\left[ ||\Delta \bm{w}_{\cdot, k}||^2 \right]}{\mathbb{E}\left[ ||\Delta \bm{w}_{m, \cdot}||^2 \right]} > 1 \,,
\end{equation}
%
is on average equal to the number of times the following is true,
%
\begin{equation}
\frac{\mathbb{E}\left[ ||\Delta \bm{w}_{\cdot, k}||^2 \right]}{\mathbb{E}\left[ ||\Delta \bm{w}_{m, \cdot}||^2 \right]} < 1 \,.
\end{equation}
%
Therefore, the net increase of the norm of the columns is equal to the net increase of the norm of the rows.
%
Following the same initialization assumptions as in Theorem \ref{theo-gradient-directionality}, it follows that at the end of training,
%
\begin{equation}
    \text{Var}\left(|\bm{w}_{m, \cdot}|\right) =\text{Var}\left(|\bm{w}_{ \cdot, k}|\right) \,\, \forall k,m \,.
\end{equation}
%
thus concluding the proof.
%
\end{proof}
%
%

Now, we show that the bidirectional nature of the training objective is such that every pair of tokens $(i,j)$ in a given sequence contributes to a term that is approximately symmetric.
%
Here, we assume that predicting the $i$-th token from the $j$-th token is correlated to predicting the $j$-th token from the $i$-th token.
%
In other words, predicting one token from the other gives similar predictions, that is, the term $\beta_{ij}$ and $\beta_{ji}$ from Proposition \ref{prop-gradients-self-attention} are correlated.
%
We formalize this in the following Theorem.
%
\begin{theorem}
\label{theo-gradients-symmetry}
%
(\textbf{Bidirectional training induces symmetry})
%
Let $U$ be an ordered sequence of $N$ tokens $U = [t_1, \dots, t_N]$.
%
Let $\Delta \bm{W}_{qk}$ be derived from a bidirectional objective function as in Definition \ref{def-objective-functions}.
%
It follows that the weight update of $\bm{W}_{qk}^l$ is given by
%
\begin{equation}
    \Delta \bm{W}_{qk}^l = \sum_{i = 1}^N \sum_{j = 1}^N \beta^l_{ij} \bm{K}^{l-1}_{ij} \,,
\end{equation}
%
and that every pair $(i,j)$ with $i \neq j$ contributes to the weight update with a term
%
\begin{equation}
    \Delta \bm{W}_{qk}^l\big|_{\bm{t}_i \leftrightarrow \bm{t}_j} = \beta_{ij}^l\bm{K}^{l-1}_{ij} + \beta_{ji}^l{\bm{K}^{l-1}_{ij}}^T \,,
\end{equation}
%
that is approximately symmetric,
%
\begin{equation}
    \Delta \bm{W}_{qk}^l\big|_{\bm{t}_i \leftrightarrow \bm{t}_j} \approx \Delta \bm{W}_{qk}^l\big|^\top_{\bm{t}_i \leftrightarrow \bm{t}_j} \,.
\end{equation}
% %
% \begin{equation}
%     \Delta \bm{W}_{qk}^l\Big|_{\bm{t}_i \leftarrow \bm{t}_j} + \Delta \bm{W}_{qk}^l\Big|_{\bm{t}_j \leftarrow \bm{t}_i} \, \simeq \, [(1+\alpha)\beta_{ij} + \epsilon_{ij}]\bm{S}^{l-1}_{ij} \, , 
% \end{equation}
% %
% where $\alpha$ is a constant factor proportional to the correlation between $\beta_{ij}$ and $\beta_{ji}$, $\epsilon_{ij}$ is random variable uncorrelated with $\beta_{ij}$, and where $\bm{S}^{l-1}_{ij}$ is the symmetric component of  $\bm{K}^{l-1}_{ij}$.
%
\end{theorem}
\begin{proof}
%
%
Let $U$ be an ordered sequence of $N$ tokens $U = [t_1, \dots, t_N]$.
%
It follows from Proposition \ref{prop-gradients-self-attention} that the weight update for $\bm{W}_{qk}^l$ following the gradient of $\mathcal{L}(\bm{U})$ w.r.t. $\bm{W}_{qk}^l$ is given by
%
\begin{equation}
\label{eq:gradient-self-attention-summations}
    \Delta \bm{W}_{qk}^l = \sum_{i =1}^N \sum_{j=1}^N \beta^l_{ij} \bm{K}^{l-1}_{ij} \,,
\end{equation}
%
where we neglect any constant of proportionality - such as a learning rate - for simplicity.
%
The double summation in Equation \eqref{eq:gradient-self-attention-summations} contains $N^2$ elements.
%
We can rewrite the double summation as follows,
%
\begin{equation}
\sum_{i = 1}^N \sum_{j =1}^N \beta^l_{ij} \bm{K}^{l-1}_{ij} =
\sum_{i =1}^N \beta^l_{ii} \bm{K}^{l-1}_{ii} +
\sum_{\substack{i,j = 1 \\ i < j}}^N(\beta^l_{ij} \bm{K}^{l-1}_{ij} + \beta^l_{ji} \bm{K}^{l-1}_{ji})
\end{equation}
%
where the first term includes the diagonal terms, and the second includes the contributions of every pair $(i,j)$ with $i,j \in [0, \dots, N]$.
%
The second term can be written as,
%
\begin{equation}
 \sum_{\substack{i,j = 1 \\ i < j}}^N (\beta^l_{ij} \bm{K}^{l-1}_{ij} + \beta^l_{ji} \bm{K}^{l-1}_{ji}) = \sum_{\substack{i,j = 1 \\ i < j}}^N(\beta^l_{ij} \bm{K}^{l-1}_{ij} + \beta^l_{ji} {\bm{K}^{l-1}_{ij}}^\top) \,, 
\end{equation}
%
and by decomposing $\bm{K}^{l-1}_{ij}$ in its symmetric and skew-symmetric parts, such that $\bm{K}^{l-1}_{ij} = \bm{S}^{l-1}_{ij} + \bm{N}^{l-1}_{ij}$, we obtain,
%
\begin{equation}
\label{eq:gradients-symmetric-terms}
\begin{split}
    \sum_{\substack{i,j = 1 \\ i < j}}^N (\beta^l_{ij} \bm{K}^{l-1}_{ij} + \beta^l_{ji} {\bm{K}^{l-1}_{ij}}^\top)
    & = \sum_{\substack{i,j = 1 \\ i < j}}^N \big[ \beta_{ij}^l(\bm{S}^{l-1}_{ij} + \bm{N}^{l-1}_{ij}) + \beta_{ji}^l(\bm{S}^{l-1}_{ij} + \bm{N}^{l-1}_{ij})^T \big] \\
    & = \sum_{\substack{i,j = 1 \\ i < j}}^N \big[(\beta^l_{ij} + \beta^l_{ji})\bm{S}^{l-1}_{ij} + (\beta^l_{ij} - \beta^l_{ji})\bm{N}^{l-1}_{ij}\big]\,.
\end{split}
\end{equation}
%
Let $\Delta \bm{W}_{qk}^l\big|_{\bm{t}_i \leftrightarrow \bm{t}_j} = \beta_{ij}^l\bm{K}^{l-1}_{ij} + \beta_{ji}^l{\bm{K}^{l-1}_{ij}}^T$,
and let $\beta^l_{ij}$ and $\beta^l_{ji}$ be such that $\text{sign}(\beta^l_{ij}) = \text{sign}(\beta^l_{ji})$ and $|\beta^l_{ij}| \approx $$|\beta^l_{ji}|$.
%
It follows that
%
\begin{equation}
 \Delta \bm{W}_{qk}^l\big|_{\bm{t}_i \leftrightarrow \bm{t}_j} \approx \sum_{\substack{i,j = 1 \\ i < j}}^N \beta^l_{ij}\bm{S}^{l-1}_{ij} = \sum_{\substack{i,j = 1 \\ i < j}}^N \beta^l_{ij}{\bm{S}^{l-1}}^\top_{ij} = \Delta {\bm{W}_{qk}^l}^\top\big|_{\bm{t}_i \leftrightarrow \bm{t}_j} \,
\end{equation}
%
thus concluding the proof.
%
%
%
%
% Let $\bm{U} = [\bm{t}_0, \bm{t}_1, \dots, \bm{t}_N]$ be a sequence of $N$ one-hot encoded tokens $\bm{t}_i \in \mathbb{R}^V$ where $V$ is the dimension of the vocabulary.
% %
% Following Equation \eqref{eq:results:self-supervised-pretraining}, let $\mathcal{L}(\bm{t}_i)$ be the cross-entropy of the one-hot encoded token $\bm{t}_i$  and let  $\sigma(\bm{z}_i) \in \mathbb{R}^V$ be the estimated probability distribution, as follows,
% %
% \begin{equation}
%  \mathcal{L}(\bm{U}) = \sum_{i=1}^N \mathcal{L}(\bm{t}_i) = \sum_{i=1}^N \bm{t}_i\log(\sigma(\bm{z}_i))\,.
% \end{equation}
% %
% Following Definition \ref{def-transformer-model}, let $\sigma(\bm{z}_i)$ the prediction of the $i$-th from the last layer of a Transformer models with $L$ layers,
% %
% \begin{equation}
% \begin{cases}
% %
% {\bm{x}^0_i}^\top = \bm{t_i}^\top \bm{W_e} + \bm{W}_p \\[5pt]
% \vspace{5pt}
% {\bm{x}_i^l}^\top = \mathcal{F}_l(\bm{x}_i^{l-1})\quad \forall l \in [1, L] \\
% \sigma\big(\bm{z}_i\big)  = \sigma\big({\bm{x}_i^L}^\top \bm{W_u}\big)\,,
% %
% \end{cases}
% \end{equation}
% %
% where ${\bm{x}^l_i}^\top = \mathcal{F}_l(\bm{x}_i^{l-1})$ is a short notation for the self-attention and multi-layered perception transformation of the $l$-th layer,
% %
% \begin{equation}
% \mathcal{F}_l(\bm{x}_i^{l-1}) =
% \left\{
% \begin{aligned}
% &{\hat{\bm{x}}^l_i}^\top = {\bm{x}^{l-1}_i}^\top + a_l(\bm{x}^{l-1}_i) \\
% &{\bm{x}^l_i}^\top = {\hat{\bm{x}}^l_i}^\top + m_l(\hat{\bm{x}}^l_i)
% \end{aligned}
% \right.\,.
% \end{equation}
% %
%
%
%
%
% We first obtain a relation between $\beta_{ij}$ and $\beta_{ji}$ in the case of $L = 1$ (1-layer self-attention).
% %
% It follows from Proposition \ref{prop-gradients-self-attention} that $\beta_{ij}$ is given by,
% %
% \begin{equation}
% \begin{split}
%     \beta_{ij} 
%     & = {\bm{\delta}_i}^\top \bm{x}_j 
%     \\
%     & = {\bm{\delta}_i}^\top \Big(\bm{x}_j^0 + a(\bm{x}_j^0) + m(\bm{\hat{x}}_j) \Big)
%     \\
%     & = {\bm{\delta}_i}^\top \Big(\bm{W_e}^\top\bm{t}_j + \bm{W}^\top_p +a(\bm{x}_j^0) + m(\bm{\hat{x}}_j) \Big)
%     \\
%     & = {\bm{\delta}_i}^\top\,\bm{W_e}^\top\bm{t}_j + \bm{\delta}_i^\top \Big(\bm{W}_p^\top + a(\bm{x}_j^0) + m(\bm{\hat{x}}_j) \Big)
%     \\
%     & = (\bm{t}_i - \sigma(\bm{z}_i))^\top \bm{W}_u^\top {\bm{W}_v}^\top\,\bm{W_e}^\top\bm{t}_j+ \bm{\delta}_i^\top \Big(\bm{W}_p^\top + a(\bm{x}_j^0) + m(\bm{\hat{x}}_j) \Big)
%     \\
%     & = \bm{t}_i^\top \bm{W}_u^\top {\bm{W}_v}^\top\,\bm{W}_e^\top \bm{t}_j- \sigma(\bm{z}_i)^\top \bm{W}_u^\top {\bm{W}_v}^\top\,\bm{W}_e^\top \bm{t}_j + \bm{\delta}_i^\top \Big(\bm{W}_p^\top + a(\bm{x}_j^0) + m(\bm{\hat{x}}_j) \Big) \,,
% \end{split}
% \end{equation}
% %
% while $\beta_{ji}$ is given by,
% %
% \begin{equation}
%     \beta_{ji} = \bm{t}_j^\top \bm{W}_u^\top {\bm{W}_v}^\top\,\bm{W}_e^\top \bm{t}_i- \sigma(\bm{z}_j)^\top \bm{W}_u^\top {\bm{W}_v}^\top\,\bm{W}_e^\top \bm{t}_i + \bm{\delta}_j^\top \Big(\bm{W}_p^\top + a(\bm{x}_i^0) + m(\bm{\hat{x}}_i) \Big) \,.
% \end{equation}
% %
% Therefore, the first terms in $\beta_{ij}$ and $\beta_{ji}$ are correlated, while the other terms are not.
% %
% Next, we generalize the relation between $\beta_{ij}$ and $\beta_{ji}$ for the case $L>1$.
% %
% In this case, $\beta^l_{ij}$ ath the $l$-th layer is given by,
% %
% \begin{equation}
% \begin{split}
%     \beta_{ij}^l 
%     & = {\bm{\delta}^l_i}^\top \bm{x}^{l-1}_j  
%     \\
%     & = {\bm{\delta}_i}^\top \Big(\bm{x}_j^0 + \sum_{l' =1}^l \mathcal{F}_l(\bm{x}^l_j)\Big) 
%     \\
%     & = {\bm{\delta}_i}^\top \Big(\bm{W_e}^\top\bm{t}_j + \bm{W}^\top_p + \sum_{l' =1}^l \mathcal{F}_l(\bm{x}^l_j)\Big) 
%     \\
%     & = {\bm{\delta}_i}^\top\,\bm{W_e}^\top\bm{t}_j + \bm{\delta}_i^\top \Big(\bm{W}_p^\top + \sum_{l' =1}^l \mathcal{F}_l(\bm{x}^l_j) \Big)
%     \\
%     & = (\bm{t}_i - \sigma(\bm{z}_i))^\top \bm{W}_u^\top
%     \left[
%     \left(1 + \sum_{l' = l}^{L-1}\mathcal{F}_l'(\bm{x}_i^{l'})\right) \left(1 +  m_l'(\hat{\bm{x}}^l_i)\right)
%     \right]
%     {\bm{W}_v}^\top\,\bm{W_e}^\top\bm{t}_j
%     + 
%     \bm{\delta}_i^\top \Big(\bm{W}_p^\top + \sum_{l' =1}^l \mathcal{F}_l(\bm{x}^l_j) \Big) \,,
%     \\
% \end{split}
% \end{equation}
% %
% which can be written as,
% %
% \begin{equation}
% \begin{split}
%     \beta_{ij}^l = 
%     & 
%     \,\, \bm{t}_i^\top \bm{W}_u^\top {\bm{W}_v}^\top\,\bm{W}_e^\top \bm{t}_j + \bm{t}_i^\top \bm{W}_u^\top \left(
%     \sum_{l' = l}^{L-1}\mathcal{F}_l'(\bm{x}_i^{l'}) +  m_l'(\hat{\bm{x}}^l_i) + m_l'(\hat{\bm{x}}^l_i)\,\sum_{l' = l}^{L-1}\mathcal{F}_l'(\bm{x}_i^{l'})
%     \right){\bm{W}_v}^\top\,\bm{W}_e^\top \bm{t}_j
%     \\
%     &
%     + \bm{\delta}_i^\top \Big(\bm{W}_p^\top + \sum_{l' =1}^l \mathcal{F}_l(\bm{x}^l_j) \Big)\,.
% \end{split}
% \end{equation}
% %
% Similarly, $\beta^l_{ji}$ is given by,
% %
% \begin{equation}
% \begin{split}
%     \beta_{ji}^l = 
%     & 
%     \,\, \bm{t}_j^\top \bm{W}_u^\top {\bm{W}_v}^\top\,\bm{W}_e^\top \bm{t}_i + \bm{t}_j^\top \bm{W}_u^\top \left(
%     \sum_{l' = l}^{L-1}\mathcal{F}_l'(\bm{x}_j^{l'}) +  m_l'(\hat{\bm{x}}^l_j) + m_l'(\hat{\bm{x}}^l_j)\,\sum_{l' = l}^{L-1}\mathcal{F}_l'(\bm{x}_j^{l'}) 
%     \right){\bm{W}_v}^\top\,\bm{W}_e^\top \bm{t}_i
%     \\
%     &
%     + \bm{\delta}_j^\top \Big(\bm{W}_p^\top + \sum_{l' =1}^l \mathcal{F}_l(\bm{x}^l_i) \Big)\,.
% \end{split}
% \end{equation}
% %
% The first terms in $\beta^l_{ij}$ and $\beta^l_{ji}$ are the same as for the case $L=1$, therefore the same correlation argument holds for the case $L>1$.
% %
% %
\end{proof}
%

Encoder-only models are typically not trained to predict every token in a sequence, but rather a random subset of tokens, and the model can attend to tokens bidirectionally.
%
This is usually called Masked Language Modeling (MLM) \citep{devlinBERTPretrainingDeep2019, liuRoBERTaRobustlyOptimized2019, lanALBERTLiteBERT2020, warnerSmarterBetterFaster2024}.
%
Therefore, only a subset of terms in the double summation of Equation \eqref{eq:gradients-symmetric-terms} has the symmetric properties described above. 
%
We generalize the proof to this case in the following.
%
%
\begin{remark}
%
Let $\mathcal{C}_i = [0,1,\dots, N]$ and let the summation indexed by $i$ to run over a random subset of tokens $ M \subset [0,1,\dots, N]$.
%
The weight update of $\bm{W}_{qk}$ is then given by
%
\begin{equation}
    \Delta \bm{W}_{qk}^l = \sum_{i\in M} \sum_{j=1}^N \beta^l_{ij} \bm{K}^{l-1}_{ij} \,.
\end{equation}
%
The double summation contains $N|M|$ elements, where $|M|$ is the cardinality of the subset $M$. 
%
We can rewrite the double summation as follows,
%
\begin{equation}
\label{eq:factorization-summation}
\sum_{i \in M} \sum_{j =1}^N \beta^l_{ij} \bm{K}^{l-1}_{ij} =
\sum_{i \in M} \beta^l_{ii} \bm{K}^{l-1}_{ii} +
\sum_{\substack{i,j \in M \\ i < j}} (\beta^l_{ij} \bm{K}^{l-1}_{ij} + \beta^l_{ji} \bm{K}^{l-1}_{ji}) +
\sum_{i \in M} \sum_{\substack{j \in \bar{M}}} \beta^l_{ij} \bm{K}^{l-1}_{ij},
\end{equation}
%
where the first term includes the diagonal terms, the second includes the contributions of the pairs $(i,j)$ with $i,j \in M$, and the third includes the remaining terms with $\bar{M} = [1,\dots,N] \setminus M$.
%
The second term can be written as,
%
\begin{equation}
 \sum_{\substack{i,j \in M \\ i < j}} (\beta^l_{ij} \bm{K}^{l-1}_{ij} + \beta^l_{ji} \bm{K}^{l-1}_{ji}) = \sum_{\substack{i,j \in M \\ i < j}} (\beta^l_{ij} \bm{K}^{l-1}_{ij} + \beta^l_{ji} {\bm{K}^{l-1}_{ij}}^\top) \,, 
\end{equation}
%
and by decomposing $\bm{K}^{l-1}_{ij}$ in its symmetric and skew-symmetric parts, such that $\bm{K}^{l-1}_{ij} = \bm{S}^{l-1}_{ij} + \bm{N}^{l-1}_{ij}$, we obtain,
%
\begin{equation}
\begin{split}
    \sum_{\substack{i,j \in M \\ i < j}} (\beta^l_{ij} \bm{K}^{l-1}_{ij} + \beta^l_{ji} {\bm{K}^{l-1}_{ij}}^\top)
    & = \sum_{\substack{i,j \in M \\ i < j}} \big[ \beta_{ij}^l(\bm{S}^{l-1}_{ij} + \bm{N}^{l-1}_{ij}) + \beta_{ji}^l(\bm{S}^{l-1}_{ij} + \bm{N}^{l-1}_{ij})^T \big] \\
    & = \sum_{\substack{i,j \in M \\ i < j}} \big[(\beta^l_{ij} + \beta^l_{ji})\bm{S}^{l-1}_{ij} + (\beta^l_{ij} - \beta^l_{ji})\bm{N}^{l-1}_{ij}\big]\,,
\end{split}
\end{equation}
%
with a similar structure as in Equation \eqref{eq:gradients-symmetric-terms}.
%
\end{remark}
%
Let $|M| = pN$ with $0<p<1$ being the percentage of tokens to be predicted during bidirectional training.
%
The total number of pairs in the second term of Equation \eqref{eq:factorization-summation} is given by a binomial coefficient, thus the total number of elements in the summation is $pN(pN - 1)$.
%
The total number of elements in the third term is instead the product $pN(N - pN)$.
%
Therefore, the percentage of symmetric weight updates from the second term over the total number of updates in the third term is given by
%
\begin{equation}
    \frac{pN(pN - 1)}{pN(N - pN)} \approx \, \frac{pN}{(N - pN)}\,,
\end{equation}
%
in the limit of large $N$.
%
In practice, $p$ is set to be around 15\%-30\% \citep{devlinBERTPretrainingDeep2019, liuRoBERTaRobustlyOptimized2019,lanALBERTLiteBERT2020,warnerSmarterBetterFaster2024}, leading to $ \approx 25\%$ of symmetric weight updates on average.
%





%% PROOF PROPERTIES OF SYMMETRY SCORE
\subsection{Properties of the symmetry score in Definition \ref{def-symmetry-score} and related proofs}
\label{supp-math-symmetry-score}
%
The score $s$ we introduce in Section~\ref{sec-results-pretrained-models} indicates the \emph{degree} of symmetry of a matrix $\bm{M}$ by quantifying the contribution to the Frobenious norm of its symmetric and skew-symmetric parts. 
%
In particular, $s$ equals 1 and -1 for a fully symmetric and skew-symmetric matrix.
%
Accordingly, positive (negative) values of $s$ indicate the presence of symmetric (skew-symmetric) structures.
%
Here, we provide a proof for these properties. 
%
First, we show that the Frobenious norm of any square matrix $\bm{M}$ can be decomposed in the sum of the Frobenious norm of its symmetric and skew-symmetric components, as in the following Lemma,
%
%
\begin{lemma}
\label{lemma-norm-symmetric-skew-symmetric} 
%
For any square matrix $\bm{M} \in \mathbb{M}_n$ the following equivalence holds
%
\begin{equation}
    ||\bm{M}||_F^2 = ||\bm{M}_s||_F^2 + ||\bm{M}_n||_F^2 \,.
\end{equation}
%
\end{lemma}
%
\begin{proof}
%
    The Frobenius norm of a matrix $\bm{M}$ can be defined as $|| \bm{M} ||_F = \sqrt{\text{Tr}(\bm{M} \bm{M}^\top)}$, and as such we observe that for any square matrix $\bm{M}$ we get
    %
    \begin{equation}
    \begin{split}
    %
        ||\bm{M}||_F & = \sqrt{\text{Tr}\big(\bm{M} \bm{M}^\top \big)} = \sqrt{\text{Tr}\big((\bm{M}_s + \bm{M}_n)(\bm{M}_s + \bm{M}_n)^\top\big)} = \\
    %
        & = \sqrt{\text{Tr}(\bm{M}_s \bm{M}_s^\top) + \text{Tr}(\bm{M}_s \bm{M}_n^\top) + \text{Tr}(\bm{M}_n \bm{M}_s^\top) + \text{Tr}(\bm{M}_n \bm{M}_n^\top)}\,. \\
    %
    \end{split}
    \end{equation}
    %
    It follows from the cyclic property of the trace operator that the mixing terms cancel out as follows,
    %
    \begin{equation}
        \text{Tr}(\bm{M}_s \bm{M}_n^\top) + \text{Tr}(\bm{M}_n \bm{M}_s^\top) = - \text{Tr}(\bm{M}_s \bm{M}_n) + \text{Tr}(\bm{M}_s \bm{M}_n) = 0 \,,
    \end{equation}
    %
    resulting in
    %
    %
    \begin{equation}
    %
        ||\bm{M}||_F  = \sqrt{\text{Tr}(\bm{M}_s \bm{M}_s^\top) + \text{Tr}(\bm{M}_n \bm{M}_n^\top)}\,.
    \end{equation}
    %
    Therefore, as both terms on the right-hand side are semi-positive definite, we conclude the proof as follows,
    %
    \begin{equation}
        ||\bm{M}||_F^2 = \text{Tr}(\bm{M}_s \bm{M}_s^\top) + \text{Tr}(\bm{M}_n \bm{M}_n^\top) = ||\bm{M}_s||_F^2 + ||\bm{M}_n||_F^2 \,.
    \end{equation}
%
\end{proof}
%
%
Next, we formulate the properties of the symmetry score as follows,
%
\begin{proposition}
\label{prop-symmetry-score}
%
The symmetry score $s$ quantifies the degree of symmetry or skew-symmetry of a given square matrix $\bm{M}$. In particular, \\
%
1) The symmetry score $s$ is a scalar value bounded in the range $[-1, 1]$. \\
%
2) A symmetry score $s = \pm 1$ indicates a fully symmetric or skew-symmetric matrix, respectively. 
\\
3) The symmetry score of a random matrix $\bm{M} \in \mathbb{M}_n$ with entries $\bm{M}_{ij} \sim p(0,\sigma)$ from a probability distribution with zero mean and finite variance tends to zero as $8/n$ in the limit $n\rightarrow\infty$.
%
\end{proposition}
\begin{proof}
%
To prove the points (1) and (2), we first show that it follows from Lemma~\ref{lemma-norm-symmetric-skew-symmetric} that the squared Frobenious norm of $\bm{M}_s$ and $\bm{M}_n$ are in an orthogonal relation
\begin{equation}
%
    ||\bm{M}||_F  = \sqrt{||\bm{M}_s||^2_F + ||\bm{M}_n||^2_F}\,.
%
\end{equation}
%
Therefore, for any given $\bm{M}$, the norms $||\bm{M}_s||^2_2$ and $||\bm{M}_n||^2_2$ are such that a higher value of the first leads to to a lower value of the second, and vice versa.
%
In particular, it is straightforward to observe that $||\bm{M}_s||_2 = ||\bm{M}||_2$ and $||\bm{M}_n||_2 = 0$ if $\bm{M}$ is symmetric.
%
Next, we derive a decomposition of the squared Frobenious norm of the symmetric and skew-symmetric part of $\bm{M}$.
%
From the definition of $\bm{M}_s$ we obtain that
%
\begin{equation}
\begin{split}
%
    ||\bm{M}_s||^2_F & = \text{Tr}\big(\bm{M}_s\bm{M}^\top_s\big) = \frac14 \, \text{Tr}\big[(\bm{M} + \bm{M}^\top)(\bm{M}^\top + \bm{M})\big]\\
%
    & = \frac14 \, \big[ \text{Tr}(\bm{M}\bm{M}^\top) + \text{Tr}(\bm{M}\bm{M}) + \text{Tr}(\bm{M}^\top\bm{M}^\top) + \text{Tr}(\bm{M}^\top\bm{M})\big] \\
    %
    & = \frac12 \, \big[\text{Tr}(\bm{M}\bm{M}^\top) + \text{Tr}(\bm{M}\bm{M})\big] \\
    %
    & = \frac12 \, \big[ ||\bm{M}||_F^2 + \text{Tr}(\bm{M}\bm{M})\big] \,.
%
\end{split}
\end{equation}
%
Since the upper bound for $ ||\bm{M}_s||^2_F$ is given by $ ||\bm{M}||^2_F$, the second term on the left-hand side has an upper bound given by,
%
\begin{equation}
    \text{Tr}(\bm{M}\bm{M}) \leq \frac12 ||\bm{M}||_F^2 \,,
\end{equation}
%
A complementary relation holds for the skew-symmetric component of $\bm{M}$,
%
\begin{equation}
    ||\bm{M_n}||^2_F = \frac14 \, \text{Tr}\big[(\bm{M} - \bm{M}^\top)(\bm{M}^\top - \bm{M})\big] = \frac12 \, \big[ ||\bm{M}||_F^2 - \text{Tr}(\bm{M}\bm{M})\big] \,,
\end{equation}
%
which, following the same logic, defines a lower-bound for $\text{Tr}(\bm{M}\bm{M})$ as follows,
%
\begin{equation}
    -\frac12 ||\bm{M}||_F^2  \leq \text{Tr}(\bm{M}\bm{M}) \,.
\end{equation}
%
Given Definition~\ref{def-symmetry-score} we can write,
%
\begin{equation}
    s = \frac{||\bm{M}||_F^2 + \text{Tr}(\bm{M}\bm{M}) - ||\bm{M}||_F^2 + \text{Tr}(\bm{M}\bm{M})}{||\bm{M}||_F^2 } = 2 \frac{\text{Tr}(\bm{M}\bm{M})}{||\bm{M}||_F^2 } 
\end{equation}
%
and by combining the bounds derived previously we obtain,
%
\begin{equation}
    -1 \leq s \leq 1    
\end{equation}
%
with 
%
\begin{equation}
\begin{cases}
& s = 1 \quad \text{if} \quad \bm{M} = \bm{M}^\top \\ 
& s = -1 \quad \text{if} \quad \bm{M} = - \bm{M}^\top \\ 
\end{cases}
\end{equation}
%

To prove the point (3), let each entry $m_{ij} = [\bm{M}]_{ij}$ be an independent, identically distributed sample from a random distribution with mean zero and a finite variance $\sigma^2$. 
%
We compute the Frobenius norm of the symmetric and skew-symmetric parts as follows,
%
\begin{equation}
\begin{split}
    \|\bm{M}_s\|^2_F & = \sum_{i\neq j}\left(\bm{M}_{ij} + \bm{M}_{ji} \right)^2 + \sum_{i}\left(2\bm{M}_{ii}\right)^2 \\
    \|\bm{M}_n\|^2_F & = \sum_{i\neq j}\left(\bm{M}_{ij} - \bm{M}_{ji} \right)^2\, .
\end{split}
\end{equation}
%
Here, the skew-symmetric part has a zero diagonal term (because of the subtraction), and the symmetric part has twice the diagonal of the original matrix $\bm{M}$ (because of the addition).
%
Since the entries are independent, $\bm{M}_{ij}$ is independent of $\bm{M}_{ji}$ for all $j\neq i$, and thus we can treat the off-diagonal entries of the $\bm{M}_s$ and $\bm{M}_n$ terms as a sum and difference of two independent random samples having mean zero and the same variance.
%
It follows that the resulting distribution has a mean zero and a variance of $2\sigma^2$ in both cases,
%
\begin{equation}
    \sum_{i\neq j}\left(\bm{M}_{ij} \pm \bm{M}_{ji} \right)^2 = 2\sum_{i=1}^n\sum_{j=i+1}^n \left(\bm{M}_{ij} \pm \bm{M}_{ji} \right)^2 
    \underset{n\rightarrow\infty}{\approx} n(n-1) \text{Var}\left[\bm{M}_{ij} \pm \bm{M}_{ji}\right]
    = n(n-1) 2 \sigma^2 \,,
\end{equation}
%
where the approximation is due to the central limit theorem.
%
Applying a similar logic to the second term on the symmetric norm, each entry is the double of a random i.i.d. distribution with 
\begin{equation}
    \sum_{i=1}^N\left(2\bm{M}_{ii}\right)^2 \underset{N\rightarrow\infty}{\approx}  n\text{Var}\left[\bm{M}_{ij}\right]= n4\sigma^2 \,.
\end{equation}
%
Finally, we take the Frobenius norm of the random matrix itself and apply the same logic, where there are $n^2$ entries with a variance of $\sigma^2$,
%
\begin{equation}
    \|\bm{M}\|_F^2 \underset{n\rightarrow\infty}{\approx} n^2\sigma^2\,.
\end{equation}
%
It follows that the symmetry score is given by
%
\begin{equation}
    s = 2\,\dfrac{\|\bm{M}_s\|^2_F - \|\bm{M}_n\|^2_F}{\|\bm{M}\|^2_F} \underset{n\rightarrow\infty}{\approx} \dfrac{8\sigma^2n}{\sigma^2n^2} = \dfrac{8}{n}\,,
\end{equation}
%
where the symmetry score is zero in the limit $n \rightarrow\infty$ with convergence from the positive side.
%
\end{proof}
%



%% PROOF PROPERTIES OF DIRECTIONALITY SCORE
\subsection{Properties of the directionality score in Definition \ref{def-directionality-score} and related proofs}
\label{supp-math-directionality-score}
%
The score $d$ we introduce in Section~\ref{sec-results-pretrained-models} quantifies the directional bias of a square matrix $\bm{M}$ by comparing the total norm of the "outliers" rows and columns, that is, that are higher than $\gamma$ times the standard deviations of the norms.
%
A directionality score $d$ of 1 indicates the presence of rows with high ``outlier'' norms and the absence of outliers in the distribution of the column norms.  
%
The opposite is true for a directionality score $d$ of -1.
%
Accordingly, positive (negative) values of $d$ indicate the presence of row (column) dominance in the matrix.
%
Here, we provide a proof for these properties.
%
\begin{proposition}
\label{prop-directionality-score}
%
The symmetry score $d$ provides a quantitative measure of the degree of directional bias in a given square matrix $\bm{M}$. \\
%
1) The directionality score $d$ is a scalar value that lies within the range  $[-1, 1]$. \\
%
2) For any given $\gamma > 0$, a directionality score $d = \pm 1$ indicates that vectors satisfying the condition defined by $\gamma$
are exclusively present in the rows or the column distribution, respectively. 
\\
3) The directionality score of a random matrix $\bm{M} \in \mathbb{M}_n$ with entries $\bm{M}_{ij} \sim p(0,\sigma)$ from a probability distribution with zero mean and a variance that scales as $O(n^{-1}) $ tends to zero in the limit $n\rightarrow\infty$.
%
\end{proposition}
\begin{proof}
%
To prove that the directionality score is bounded in the interval $[-1,1]$, note that $\bar{c}_{\mathbf{M}}, \bar{r}_{\mathbf{M}}>0$ simply because they are sums of norms. As both are positive,
\begin{equation}
    |\bar{c}_{\mathbf{M}}- \bar{r}_{\mathbf{M}}|
    < \bar{c}_{\mathbf{M}}
    < \bar{c}_{\mathbf{M}}+ \bar{r}_{\mathbf{M}}
\end{equation}
and thus 
\begin{equation}
    \dfrac{\bar{c}_{\mathbf{M}}- \bar{r}_{\mathbf{M}}}{\bar{c}_{\mathbf{M}}+ \bar{r}_{\mathbf{M}}} <1
\end{equation}
and taking the negative sign for the absolute value,
\begin{equation}
   - \dfrac{\bar{r}_{\mathbf{M}}- \bar{c}_{\mathbf{M}}}{\bar{c}_{\mathbf{M}}+ \bar{r}_{\mathbf{M}}} > -1
\Rightarrow   \dfrac{\bar{c}_{\mathbf{M}}- \bar{r}_{\mathbf{M}}}{\bar{c}_{\mathbf{M}}+ \bar{r}_{\mathbf{M}}} > -1.
\end{equation}

In the extremes $d=\pm1$, the numerator must be equal to the denominator in absolute value, implying that either $\bar{r}_{\mathbf{M}}$ or $\bar{c}_{\mathbf{M}}$ are zero and the other is positive. For completeness, we define the score as zero if both are zero.

Finally, we study the case of a random matrix. We start by noting that the values of $\bar{r}_{\mathbf{M}},\ \bar{c}_{\mathbf{M}}$ are interchanged when we take the transpose, hence
\begin{equation}\label{eq:rowColTranspose_direction}
    \bar{r}_{\mathbf{M}} =  \bar{c}_{\mathbf{M}^\top}
\end{equation}
Regardless of the scaling of the matrix and the value $\gamma$, the key property of a random matrix is that all entries are drawn from the same distribution. Hence,
 \begin{equation}
     \text{Pr}\left[\bm{M}_{ij} = x\right] = 
     \text{Pr}\left[\bm{M}_{ji} = x\right]
     \Rightarrow
      \text{Pr}\left[\bm{M} = \bm{X}\right] 
      = \text{Pr}\left[\bm{M} = \bm{X}^\top\right] 
 \end{equation}
 for $x$ and $\bm{X}$ being any arbitrary value or matrix. As a consequence,
 \begin{equation}
     \text{Pr}\left[\bar{c}_{\mathbf{M}} = x\right]
     = \text{Pr}\left[\bar{c}_{\mathbf{M}^\top} = x\right] 
     = \text{Pr}\left[\bar{r}_{\mathbf{M}} = x\right]
 \end{equation}
where the last equality comes from Eq.~\ref{eq:rowColTranspose_direction}. The main point here is that the probability distribution of both rows and columns is the same. Pushing this forward, the expected value of $\bar{r}_{\mathbf{M}}- \bar{c}_{\mathbf{M}}$ is
\begin{align}
    \text{E}\left[\bar{r}_{\mathbf{M}}- \bar{c}_{\mathbf{M}}\right]
   & = \text{E}\left[\bar{r}_{\mathbf{M}}\right]-\text{E}\left[\bar{c}_{\mathbf{M}}\right]
    = \int \bar{c}_{\mathbf{M}} \text{Pr}\left[\bm{M}\right] d\bm{M}
    - \int \bar{r}_{\mathbf{M}} \text{Pr}\left[\bm{M}\right] d\bm{M}\\
     &= \int \bar{r}_{\mathbf{M}} \text{Pr}\left[\bm{M}^\top\right] d\bm{M}^\top
    - \int \bar{r}_{\mathbf{M}} \text{Pr}\left[\bm{M}\right] d\bm{M}
    = 0
\end{align}
Furthermore, the expected value of $\bar{r}_{\mathbf{M}}+ \bar{c}_{\mathbf{M}}$ is strictly positive, since both values are positive. Thus, their ratio, the directionality score of a random matrix, is zero. 

Notice that to be thorough we must show that their variance is bounded scales down. Since weight initialization has been extensively studied, we will just make a general reference to it here. In machine learning, all weights are initialized with zero mean and variances that scale as $O(n^{-1})$. As $\bm{M}$ is a product of two matrices with such scaling, each entry would consist of the sum of $n$ random variables, where each one has a scaling of $O(n^{-2})$ since it is the product of two random variables with an $O(n^{-1})$ scaling. Thus, the entries of  $\bm{M}$ also have a scaling of  $O(n^{-1})$. Applying the mean value theorem gives us the desired result.


\end{proof}