% % 
% In this section, we train Transformer models from scratch and perform a series of experiments to analyze how symmetric and directional structures develop during training across layers.
% %
% Next, we provide a first endeavor in leveraging these symmetric structures to achieve more efficient training of Transformer models on standard datasets.
% %
In this final section, we test if using structural priors based on our previous results can improve the pretraining of Transformer models. 
%
To do so, we train Transformer models from scratch and perform a series of experiments to analyze how symmetric and directional structures develop during training across layers.
%





\subsection{Evolution of symmetric and directional structures during learning}
\label{sec-evolution-structures}
%
%
\begin{figure*}[ht]
\begin{center}
\includegraphics[width=1.\linewidth]{ICML2025/figures/figures-pdf/figure-training-layers.pdf}
\end{center}
%
\caption{
%
\textbf{a})
%
Evolution of symmetry score during training.
%
Shown are the median and the interquartile range.
%
Models were trained on the Wikipedia dataset \citep{wikidump}.
%
Encoder-only and decoder-only models are color-coded in blue and purple, respectively (see legend).
%
\textbf{b})
%
Same as in panel \textbf{a}
for the median directionality score.
%
\textbf{c})
%
Evolution of the median symmetry score across layers of the encoder-only (left) and decoder-only (right) models.
%
Each layer is color-coded as shown on the legend.
%
\textbf{d})
%
Same as panel $\textbf{c}$ for the median directionality score.
%
}
%
\label{fig-training-layers}
%
\end{figure*}
%
To test the applicability of our result, we first train 12-layer transformer models in both encoder and decoder modes and quantify the median symmetry and directionality scores across epochs.
%
At initialization, the symmetry and directionality score of the matrix $\bm{W}_{qk}$ at any layer is zero (see  Definition \ref{def-symmetry-score} and Definition \ref{def-directionality-score} and related Appendix  \ref{supp-math-symmetry-score} and \ref{supp-math-directionality-score}).
%
The incremental update of $\bm{W}_{qk}$ we described in the previous sections predicts that decoder-only models develop high-norm columns incrementally during training (see Theorem \ref{theo-informal-directionality}).
%
Likewise, as symmetric weight updates are added to $\bm{W}_{qk}$ in encoder-only models, Theorem \ref{theo-informal-symmetry} predicts that symmetric structures emerge incrementally during training.
%

% It follows from Definition \ref{def-symmetry-score} and Definition \ref{def-directionality-score} that the bilinear form at any $l$-th layer $\bm{W}^l_{qk}$  has both a symmetry score and a directionality score of zero at initialization (see Appendix \ref{supp-math-symmetry-score} and \ref{supp-math-directionality-score}).
%
% Theorem \ref{theo-unidirectional-structures}
% predicts that decoder-only models develop high-norm columns incrementally during training.
% %
% Likewise, as symmetric weight updates are added to $\bm{W}_{qk}$ in encoder-only models, Theorem \ref{theo-symmetric-structures} predicts that symmetric structures emerge incrementally during training. 
% %
% To test these predictions, we train 12-layer transformer models in both encoder and decoder modes and quantify the median symmetry and directionality scores across epochs (Figure \ref{fig-training-layers}a-b).
% %

Consistent with our results on pre-trained models, encoder-only models show a higher degree of symmetry than decoder-only models (Figure \ref{fig-training-layers}a). 
%
In contrast, decoder-only models have a higher directionality score (Figure \ref{fig-training-layers}b).
%
We observe this difference on all datasets we tested (Jigsaw \citep{jigsaw_challenge}, Wikipedia \citep{wikidump}, Red Pajama \citep{together2023redpajama}, see Figure \ref{suppfig-training-custom-models}).
%
Furthermore, late layers of encoder-only models are more symmetric and converge faster than early layers when training bidirectionally. At the same time, decoder-only models learn almost non-symmetric matrices with strong skew-symmetric matrices in the middle layers (Figure \ref{fig-training-layers}c).
% 
When training unidirectionally, both encoder and decoder models show a higher degree of directionality for late layers, which is remarkably stronger for decoder-only models (Figure \ref{fig-training-layers}d).
%
We observe similar differences across layers with all the datasets we tested (Figure \ref{suppfig-training-custom-models-layers}), despite these models having
less significant differences in directionality scores.
%
See Appendix~\ref{sec-experimental-details} for a detailed description of the experiments.
% 







\subsection{Enforcing symmetry at initialization improves the training of encoder-only models}
\label{sec-symmetric-initialization}
%
\begin{table}[t]
\caption{The final loss at the end of training and the speed-up for the 4 and 12-layer models trained on the Jigsaw dataset \citep{jigsaw_challenge}, Wikipedia \citep{wikidump}, and Red Pajama \citep{together2023redpajama}, with and without symmetry initialization (see Appendix~\ref{sec:exp_bertmodels_models}).
%
Speed-up (\%) is calculated by subtracting the epoch at which the symmetrically initialized model reaches the non-symmetric model’s final loss from the total number of epochs, and then dividing by the total number of epochs.
%
For example, a 50\% speed-up means that the model with symmetric initialization achieves the final loss of the non-symmetric model in half the number of training epochs.
%
}
\label{table:symmetric-initialization}
\vskip 0.15in
\begin{center}
\begin{small}
\begin{sc}
\begin{tabular}{lcccr}
\toprule
Model & Loss & Speed-up \\
\midrule\midrule
4-layer model & & \\
\midrule\midrule
Jigsaw & 2.782 & \\
Jigsaw (+ symm) & \textbf{2.758} & 26 \% \\
\midrule
Wikipedia & 0.984& \\
Wikipedia (+ symm) & \textbf{0.812} & 73 \%\\
\midrule
Red Pajama & 1.106 &\\
Red Pajama (+ symm) & \textbf{0.907} & 69 \%\\
\midrule\midrule
12-layer model & & \\
\midrule\midrule
Jigsaw & 1.419 & \\
Jigsaw (+ symm) & 1.430 & 0 \%\\
\midrule
Wikipedia & 0.256 & \\
Wikipedia (+ symm) & \textbf{0.247} & 20 \%\\
\midrule
Red Pajama & 0.297&\\
Red Pajama (+ symm) & \textbf{0.274} & 35 \%\\
\bottomrule
\end{tabular}
\end{sc}
\end{small}
\end{center}
\vskip -0.1in
\end{table}
%
The previous section showed that symmetric structures incrementally emerge during training in the $\bm{W}_{qk}$ matrices of encoder-only models.
%
Here, we first provide evidence that these findings can be exploited to speed up training using symmetry as an inductive bias.
%
Specifically, we explore how symmetric initialization influences the training dynamics of the model and whether it enhances learning efficiency and overall performance.
%

We train 4-layer and 12-layer encoder-only models, comparing two initialization strategies: initialize the self-attention matrices independently versus initializing the $\bm{W}_q$ and $\bm{W}_k$ in each self-attention layer to ensure that $\bm{W}_{qk}$ is symmetric (see Appendix \ref{sec-experimental-details}).
%
We report the results of our experiments in Table \ref{table:symmetric-initialization}.
%
We observe that enforcing symmetry at initialization leads to lower loss values at the end of training for most of the models.
%
Importantly, symmetric initialization significantly accelerates convergence, reaching the final loss value faster than those with random initialization (up to $~75\%$ faster for 4-layer models and $~35\%$ faster for 12-layer models, see also Figure \ref{suppfig-symmetric-initialization}a).
%
Moreover, we observe that self-attention matrices initialized symmetrically lose symmetry during training but converge to higher symmetry levels than random initialization (Figure \ref{suppfig-symmetric-initialization}b)
%
This symmetric initialization decreases the gap in symmetric scores between layers compared to random initialization (Figure \ref{suppfig-symmetric-initialization}c).
%
% Interestingly, we also observe that enforcing symmetry at initialization improves accuracy on small, linear self-attention models (Figure \ref{suppfig:linear-attention-configs-symmetric-data}). 
%
These results highlight that embedding symmetry as an inductive bias across all Transformer layers can enhance training efficiency and model performance.
%