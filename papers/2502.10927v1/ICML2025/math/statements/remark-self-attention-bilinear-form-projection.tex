\begin{remark}
%
Each element of the sum in Equation \eqref{eq-proof-self-attention-bilinear-form} implicitly defines an operator in the subspace spanned by $\bm{X} = \{\bm{x}_0^\top, \bm{x}_1^\top, \dots, \bm{x}_N^\top\}$ given the transformed embedding space defined by $\bm{W}_{qk}$,
%
\begin{equation}
     \sum_j \alpha_{ij}(\bm{W}_{qk})\, \bm{x}_j = \sum_j \bm{x}_i^\top \bm{W}_{qk} \bm{x}_j\,\bm{x}_j = \sum_j P_{\bm{W}_{qk}}(\bm{x}_i ,\bm{x}_j)\,,
\end{equation}
%
where $P_{\bm{W}_{qk}}(\,\cdot\,, \bm{x}_j)$ are operators over the subset $\bm{X}$.
%
The set $\bm{X}$ is in general linearly dependent, and the number of tokens $N$ in the sequence differs from the embedding space dimension $d$.
%
Furthermore, $\bm{W}_{qk}$ represents a general bilinear form (see Definition \ref{def-bilinear-form}), which may not satisfy all the defining axioms of a formal inner product — namely, linearity, conjugate symmetry, and positive definiteness.
%
Finally, the operators $P_{\bm{W}_{qk}}(\cdot \bm{x}_j)$ are not formal projection operators since they are not nilpotent ($P_{\bm{W}_{qk}}(\,\cdot\, \bm{x}_j) \circ P_{\bm{W}_{qk}}(\,\cdot\, \bm{x}_j) \neq P_{\bm{W}_{qk}}(\,\cdot\, \bm{x}_j)$).
%
Nonetheless, the bilinear map $\bm{W}_{qk}: \mathbb{R}^d \times \mathbb{R}^d \rightarrow \mathbb{R}$ still associates any pair of vectors with a scalar value quantifying their alignment as determined by the geometric relations encoded in $\bm{W}_{qk}$.
%.
% Nevertheless, the sum in Equation \ref{eq:math:linear-decomposition-subspace-tokens} represents a decomposition of $\bm{x}_i$ on the subspace $\text{span}\{\bm{X}\}$ where each coefficient $\alpha_{ij}(\bm{W}_{qk})$ quantifies the aligment of $\bm{x}_i$ with the corresponding element of $\bm{X}$ following the bilinear form $\bm{W}_{qk}$. 
% %%
Therefore, self-attention computes a generalized decomposition of $\bm{x}_i$ on $\text{Conv}(\bm{X})$ in the transformed embedding space defined by $\bm{W}_{qk}$.
%
A convex combination ensures that the resulting vector remains within the region enclosed by the basis vectors $\bm{X}$.
%  
\end{remark}