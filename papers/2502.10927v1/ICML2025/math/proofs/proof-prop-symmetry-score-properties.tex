\begin{proof}
%
To prove the points (1) and (2), we first show that it follows from Lemma~\ref{lemma-norm-symmetric-skew-symmetric} that the squared Frobenious norm of $\bm{M}_s$ and $\bm{M}_n$ are in an orthogonal relation
\begin{equation}
%
    ||\bm{M}||_F  = \sqrt{||\bm{M}_s||^2_F + ||\bm{M}_n||^2_F}\,.
%
\end{equation}
%
Therefore, for any given $\bm{M}$, the norms $||\bm{M}_s||^2_2$ and $||\bm{M}_n||^2_2$ are such that a higher value of the first leads to to a lower value of the second, and vice versa.
%
In particular, it is straightforward to observe that $||\bm{M}_s||_2 = ||\bm{M}||_2$ and $||\bm{M}_n||_2 = 0$ if $\bm{M}$ is symmetric.
%
Next, we derive a decomposition of the squared Frobenious norm of the symmetric and skew-symmetric part of $\bm{M}$.
%
From the definition of $\bm{M}_s$ we obtain that
%
\begin{equation}
\begin{split}
%
    ||\bm{M}_s||^2_F & = \text{Tr}\big(\bm{M}_s\bm{M}^\top_s\big) = \frac14 \, \text{Tr}\big[(\bm{M} + \bm{M}^\top)(\bm{M}^\top + \bm{M})\big]\\
%
    & = \frac14 \, \big[ \text{Tr}(\bm{M}\bm{M}^\top) + \text{Tr}(\bm{M}\bm{M}) + \text{Tr}(\bm{M}^\top\bm{M}^\top) + \text{Tr}(\bm{M}^\top\bm{M})\big] \\
    %
    & = \frac12 \, \big[\text{Tr}(\bm{M}\bm{M}^\top) + \text{Tr}(\bm{M}\bm{M})\big] \\
    %
    & = \frac12 \, \big[ ||\bm{M}||_F^2 + \text{Tr}(\bm{M}\bm{M})\big] \,.
%
\end{split}
\end{equation}
%
Since the upper bound for $ ||\bm{M}_s||^2_F$ is given by $ ||\bm{M}||^2_F$, the second term on the left-hand side has an upper bound given by,
%
\begin{equation}
    \text{Tr}(\bm{M}\bm{M}) \leq \frac12 ||\bm{M}||_F^2 \,,
\end{equation}
%
A complementary relation holds for the skew-symmetric component of $\bm{M}$,
%
\begin{equation}
    ||\bm{M_n}||^2_F = \frac14 \, \text{Tr}\big[(\bm{M} - \bm{M}^\top)(\bm{M}^\top - \bm{M})\big] = \frac12 \, \big[ ||\bm{M}||_F^2 - \text{Tr}(\bm{M}\bm{M})\big] \,,
\end{equation}
%
which, following the same logic, defines a lower-bound for $\text{Tr}(\bm{M}\bm{M})$ as follows,
%
\begin{equation}
    -\frac12 ||\bm{M}||_F^2  \leq \text{Tr}(\bm{M}\bm{M}) \,.
\end{equation}
%
Given Definition~\ref{def-symmetry-score} we can write,
%
\begin{equation}
    s = \frac{||\bm{M}||_F^2 + \text{Tr}(\bm{M}\bm{M}) - ||\bm{M}||_F^2 + \text{Tr}(\bm{M}\bm{M})}{||\bm{M}||_F^2 } = 2 \frac{\text{Tr}(\bm{M}\bm{M})}{||\bm{M}||_F^2 } 
\end{equation}
%
and by combining the bounds derived previously we obtain,
%
\begin{equation}
    -1 \leq s \leq 1    
\end{equation}
%
with 
%
\begin{equation}
\begin{cases}
& s = 1 \quad \text{if} \quad \bm{M} = \bm{M}^\top \\ 
& s = -1 \quad \text{if} \quad \bm{M} = - \bm{M}^\top \\ 
\end{cases}
\end{equation}
%

To prove the point (3), let each entry $m_{ij} = [\bm{M}]_{ij}$ be an independent, identically distributed sample from a random distribution with mean zero and a finite variance $\sigma^2$. 
%
We compute the Frobenius norm of the symmetric and skew-symmetric parts as follows,
%
\begin{equation}
\begin{split}
    \|\bm{M}_s\|^2_F & = \sum_{i\neq j}\left(\bm{M}_{ij} + \bm{M}_{ji} \right)^2 + \sum_{i}\left(2\bm{M}_{ii}\right)^2 \\
    \|\bm{M}_n\|^2_F & = \sum_{i\neq j}\left(\bm{M}_{ij} - \bm{M}_{ji} \right)^2\, .
\end{split}
\end{equation}
%
Here, the skew-symmetric part has a zero diagonal term (because of the subtraction), and the symmetric part has twice the diagonal of the original matrix $\bm{M}$ (because of the addition).
%
Since the entries are independent, $\bm{M}_{ij}$ is independent of $\bm{M}_{ji}$ for all $j\neq i$, and thus we can treat the off-diagonal entries of the $\bm{M}_s$ and $\bm{M}_n$ terms as a sum and difference of two independent random samples having mean zero and the same variance.
%
It follows that the resulting distribution has a mean zero and a variance of $2\sigma^2$ in both cases,
%
\begin{equation}
    \sum_{i\neq j}\left(\bm{M}_{ij} \pm \bm{M}_{ji} \right)^2 = 2\sum_{i=1}^n\sum_{j=i+1}^n \left(\bm{M}_{ij} \pm \bm{M}_{ji} \right)^2 
    \underset{n\rightarrow\infty}{\approx} n(n-1) \text{Var}\left[\bm{M}_{ij} \pm \bm{M}_{ji}\right]
    = n(n-1) 2 \sigma^2 \,,
\end{equation}
%
where the approximation is due to the central limit theorem.
%
Applying a similar logic to the second term on the symmetric norm, each entry is the double of a random i.i.d. distribution with 
\begin{equation}
    \sum_{i=1}^N\left(2\bm{M}_{ii}\right)^2 \underset{N\rightarrow\infty}{\approx}  n\text{Var}\left[\bm{M}_{ij}\right]= n4\sigma^2 \,.
\end{equation}
%
Finally, we take the Frobenius norm of the random matrix itself and apply the same logic, where there are $n^2$ entries with a variance of $\sigma^2$,
%
\begin{equation}
    \|\bm{M}\|_F^2 \underset{n\rightarrow\infty}{\approx} n^2\sigma^2\,.
\end{equation}
%
It follows that the symmetry score is given by
%
\begin{equation}
    s = 2\,\dfrac{\|\bm{M}_s\|^2_F - \|\bm{M}_n\|^2_F}{\|\bm{M}\|^2_F} \underset{n\rightarrow\infty}{\approx} \dfrac{8\sigma^2n}{\sigma^2n^2} = \dfrac{8}{n}\,,
\end{equation}
%
where the symmetry score is zero in the limit $n \rightarrow\infty$ with convergence from the positive side.
%
\end{proof}