\section{Case Study}
This section demonstrates how LAMD overcomes the limitations of current LLMs in Android malware detection. Since LLMs cannot process APKs directly, we use JADX\footnote{https://github.com/skylot/jadx} to decompile them, concatenating all pesudo source codes for input~\cite{llm4codeanalysis}. Besides GPT-4o-mini, we select Gemini 1.5 pro~\cite{gemini1.5} as another comparison, which claims their longest context windows and for its malware detection capabilities.

% In this section, we present a case study on leveraging LLMs for Android malware detection to show the limitations of current LLMs and how our LAMD helps enhance them. Since neither LLM natively processes APKs, We use JADX\footnote{https://github.com/skylot/jadx} to decompile APKs to get pseudo source code. Following prior work suggesting this enhances LLM comprehension~\cite{llm4codeanalysis}, we concatenate all decompiled files into a single input for each LLM as the comparison of our framework. Besides GPT-4o-mini, we select Gemini 1.5 pro~\cite{gemini1.5} as another comparison, which claims their longest context windows and for its malware detection capabilities.

Figure~\ref{fig:motivation} highlights key failure cases. In the first example, we decompile a randomly selected sample\footnote{MD5: c37e223e3388b31b323ad39af45180fc}, which contains 1,547,806 lines of code, even after restricting the scope to the ``source/com'' folder containing critical files. Gemini 1.5 Pro fails to process it, exceeding its 20,971,520-byte context limit, while LAMD enables GPT-4o-mini (with 10× lower token capacity) to generate an accurate prediction, demonstrating its ability to efficiently analyze large-scale Android applications within constrained LLM contexts.

% The Gemini 1.5 Pro fails to process the APK, citing an input size exceeding its context limit (20,971,520 bytes). In contrast, LAMD successfully enables GPT-4o-mini, which has a 10$x$ lower token limit (128,000 vs. 2M tokens in Gemini 1.5 Pro), to process the entire input and generate an accurate prediction. This demonstrates LAMD's effectiveness in handling large-scale Android applications within constrained LLM contexts.

In the second case, we verify the detection ability of LAMD. Analyzing an \verb|SMSReg| malware sample\footnote{MD5: 2be97287c6af70f2074686b1a9021c06}, Gemini misclassifies it as ``BENIGN'', erroneously identifying it as part of the ``\textit{xUtils}” library due to excessive benign classes disguising malicious behavior. With LAMD, both GPT-4o-mini and Gemini 1.5 pro correctly classify it as malware and identify key behaviors. The \verb|SMSReg| malware family usually harvests device data, sends unauthorized SMS, and registers users for premium services to evading detection. Detection explanations enhanced by LAMD align with these behaviors, showing unauthorized access to \verb|getDeviceId()|, \verb|getSubscriberId()|, and \verb|sendTextMessage()|, reflection-based evasion, insecure SSL handling, and excessive location tracking—all indicative of SMS fraud and unauthorized transactions.


% We analyze a malware sample\footnote{MD5: 2be97287c6af70f2074686b1a9021c06} from the \verb|SMSReg| family, which is decompiled into 2,165 Java files. Gemini misclassifies the sample as ``BENIGN”, erroneously identifying it as part of the ``\textit{xUtils}” library due to a large amount of benign classes indicating the normal behavior it uses for camouflage. We then execute the LAMD pipeline and compare generated explanations by Gemini and GPT-4o-mini on the same APK. Both models correctly classify it as malware and describe specific behaviors of \verb|SMSReg| malware family, i.e., harvest device data, send unauthorized SMS, and register users for premium services to evading detection. The following texts show detection results of two LLMs, where summaries enhanced by LAMD align with these behaviors, showing unauthorized access to \verb|getDeviceId()|, \verb|getSubscriberId()|, and \verb|sendTextMessage()|, reflection-based evasion, insecure SSL handling, and excessive location tracking—all indicative of SMS fraud and unauthorized transactions.

\begin{tcolorbox}[fontupper=\small, fontlower=\small, 
left=2pt, % Adds 10pt space on the left
right=2pt, % Adds 10pt space on the right
top=3pt, % Adds 5pt space above the box
bottom=3pt, % Adds 5pt space below the box
% before upper=\raggedright, 
before upper=\raggedright, 
after upper={\vspace{-4pt}},
before lower=\raggedright]
\textbf{GPT-4o-mini Output with LAMD}:
\tcblower
\textbf{Final Prediction}: MALWARE \\
\textbf{Key Findings}: \\
* Sensitive API Misuse: Uses ``getDeviceId()'', ``getSubscriberId()'', and ``sendTextMessage()'' via reflection to bypass access controls for data harvesting.\\
* SSL Vulnerability: Unchecked ``SslErrorHandler.proceed()'' and ``onReceivedSslError'' expose users to MITM attacks.\\
* Location Tracking: Collects ``getLastKnownLocation'' and ``requestLocationUpdates'' without user consent.
\end{tcolorbox}


\begin{tcolorbox}[fontupper=\small, fontlower=\small, 
left=2pt, % Adds 10pt space on the left
right=2pt, % Adds 10pt space on the right
top=3pt, % Adds 5pt space above the box
bottom=3pt, % Adds 5pt space below the box
before upper=\raggedright, 
after upper={\vspace{-4pt}},
before lower=\raggedright]
\textbf{Gemini Output with LAMD}:
\tcblower
\textbf{Final Prediction}: MALWARE \\
\textbf{Key Findings}: \\
* Extensive use of sensitive APIs (e.g., ``getDeviceId()'', ``getSubscriberId()'', ``sendTextMessage()'') strongly suggesting data exfiltration and unauthorized actions. \\
* Unconditional ``SslErrorHandler.proceed()'' bypass risks data security and enables MITM attacks. \\
* Collects and transmits data to remote servers with obfuscation and poor error handling.\\
\end{tcolorbox}


% generate similar analytical reports that accurately reflect the real behavior of this malware family. Specifically, the \verb|SMSReg| malware family harvests device data, sends unauthorized SMS, and registers users for premium services to evading detection. 


% \begin{tcolorbox}[fontupper=\small, fontlower=\small, before upper=\raggedright, before lower=\raggedright]
% \textbf{Gemini Output without LAMD}: 
% \tcblower
% \textbf{Final Prediction}: BENIGN\\
% The code belongs to the ``xUtils'' library for Android simplifies view injection, resource loading, database operations, and network requests.
% \end{tcolorbox}

% We then execute the LAMD pipeline and perform a comparative analysis using Gemini and GPT-4o-mini on the same APK. Both models correctly classify it as malware and generate similar analytical reports that accurately reflect the real behavior of this malware family. Specifically, the \verb|SMSReg| malware family harvests device data, sends unauthorized SMS, and registers users for premium services to evading detection. 
% Android application summaries provided by LLMs 
% Summaries enhanced by LAMD align with these behaviors, showing unauthorized access to \verb|getDeviceId()|, \verb|getSubscriberId()|, and \verb|sendTextMessage()|, reflection-based evasion, insecure SSL handling, and excessive location tracking—all indicative of SMS fraud and unauthorized transactions.




% While different variants have slightly different flows, the crucial step for SMSreg to perform its malicious actions typically revolves around gaining SMS-related permissions. Specifically:
% \begin{itemize}
%     \item Step 1: 
% \end{itemize}



