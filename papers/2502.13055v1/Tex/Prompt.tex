\section{Data Dependency used for Factual Consistency Verification}
\label{datadependency}
\noindent To check factual consistency, we leverage data dependencies that capture relationships between variables and APIs. We focus on five dependencies (see Table~\ref{tab:data_dependency}) as they represent fundamental program relationships critical for malware reasoning and detection. Specifically:

\vspace{0.5em}
\noindent \hspace{0.5em} \textbf{(1) Variable-to-API Dependencies.} These dependencies determine whether the execution of an API is influenced by specific variables.

% Establish whether an API's execution is influenced by specific variables.

\begin{itemize}
    \item Direct dependencies: ensure that variables explicitly control API calls, providing strong evidence of intended execution.
    \item Transitive dependencies: track how variables propagate through function calls.
    \item Conditional dependencies: account for control-flow influences (e.g., if statements), identifying cases where malicious logic may be context-dependent.
\end{itemize}

\noindent \hspace{0.5em} \textbf{(2) Inter-variable Dependencies.} These capture relationships between variables that may affect security-sensitive operations.

\begin{itemize}
    \item Parallel dependencies: detect multiple variables jointly contributing to a computation, highlighting complex conditions leading to API execution.
    \item Derived dependencies: reveal computations that transform one variable into another, helping detect disguised or obfuscated malicious behaviors.
\end{itemize}




% \textbf{Variable-to-API Dependencies}: establish whether an API's execution is influenced by specific variables.
% \begin{itemize}
%     \item Direct dependencies: ensure that variables explicitly control API calls, providing strong evidence of intended execution.
%     \item Transitive dependencies: track how variables propagate through function calls.
%     \item Conditional dependencies: account for control-flow influences (e.g., if statements), identifying cases where malicious logic may be context-dependent.
% \end{itemize}

% \textbf{Inter-variable Dependencies}: capture relationships between variables that may affect security-sensitive operations.
% \begin{itemize}
%     \item Parallel dependencies: detect multiple variables jointly contributing to a computation, highlighting complex conditions leading to API execution.
%     \item Derived dependencies: reveal computations that transform one variable into another, helping detect disguised or obfuscated malicious behaviors.
% \end{itemize}

\noindent By incorporating these dependencies into factual consistency verification, LAMD ensures that summarized malicious behaviors are not based on hallucinatioms but are grounded in actual program logic, improving both precision and interpretability in Android malware detection.

\begin{table*}
\centering
\caption{Data dependencies are used for factual consistency verification.}
\label{tab:data_dependency}
\begin{tblr}{
  cell{2}{1} = {r=3}{},
  cell{5}{1} = {r=2}{},
  vline{2-3} = {1-4,5-6}{},
  vline{3} = {3-4,6}{},
  hline{1-2,5,7} = {-}{},
  hline{3-4,6} = {2-3}{},
}
\textbf{Category}        & \textbf{Type}        & \textbf{Description}                                                             \\
Variable-to-API & Direct      & A variable directly determines API execution.                          \\
                & Transitive  & A variable is propagated through a call chain to an API.               \\
                & Conditional & A variable influences API execution via control flow (e.g., if/else ). \\
Inter-variable  & Parallel & Two variables jointly contribute to computing another variable.        \\
                & Derived     & A variable is derived from another through computation.                
\end{tblr}
\end{table*}

