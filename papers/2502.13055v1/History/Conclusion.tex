\section{Conclusion}
Android malware detectors face performance degradation due to distribution shifts from malware evolution. Our findings reveal invariant patterns among malware samples with similar intents, creating a stable feature space over time. However, ERM-based detectors, limited by i.i.d. assumptions, struggle to capture these invariances due to disrupted malware release sequences. To address this, we propose a temporal invariant training framework that organizes samples into temporally ordered subsets, exposing evolutionary patterns to help arbitrary detectors capture cross-subset invariances and enhance performance on temporally distant samples. Our framework combines multi-proxy contrastive learning with invariant gradient alignment, allowing the model to focus on stable, discriminative features within complex malware distributions. Systematic evaluations across feature spaces and drift scenarios show that this approach significantly enhances robustness, enabling the model to learn enduring, discriminative patterns over time.
% Android malware detectors suffer performance degradation due to distribution shifts from malware evolution. Our findings reveal invariant patterns among malware samples with similar intents, forming a stable and effective feature space over time. However, ERM-based detectors, constrained by i.i.d. assumptions, struggle to capture these invariances due to the disrupted sequence of malware releases. To address this, we propose a temporal invariant training framework for malware detection, organizing samples into temporally ordered subsets that expose evolutionary patterns during training, enabling the model to capture cross-subset invariances and improve performance on temporally distant test samples. Our framework integrates multi-proxy contrastive learning with invariant gradient alignment to model complex malware distributions, allowing the detector to focus on stable, discriminative features. Systematic evaluations across feature spaces and drift scenarios demonstrate that this approach significantly enhances robustness, enabling the model to learn enduring, discriminative patterns over time.