\section{Methodology}

\subsection{Preliminary}
\subsubsection{Notations}
We use uppercase and lowercase letters to denote matrices and vectors, respectively; $|\mathcal{D}|$ represents the total number of elements in set $\mathcal{D}$. 

\begin{figure*}
    \centering
    % \setlength{\abovecaptionskip}{0cm}
    \includegraphics[width=1.0\linewidth]{Figure/model_architecture.pdf}
    \caption{The proposed invariant training framework and its core components}
    \label{fig:model_architecture}
\end{figure*}

\subsubsection{Problem setting}
In the problem of Android malware detection, we divide the samples $x \in \mathbb{R}^{d}$ into two parts: the first part is the labeled current moment training data $\mathcal{D}_{tr} = {(x_i^{tr},y_i^{tr})}^{|\mathcal{D}_{tr }|}_{i=1}$, where $y_{i}^{tr} \in \{0, 1\}$ represents the labels for benign and malicious applications. All training data are from a specific period $\mathcal{T}_{tr}$, and each sample $x_i^{ts}$ contains its corresponding release timestamp $t_i^{tr}$. The second part is the upcoming unlabeled test data $\mathcal{D}_{ts}$, the timestamps of whose samples $t_i^{ts}$ are strictly from future moments compared to $\mathcal{T}_{tr}$ and they appear progressively over time. We assume that the training and test data share the same label space and have unknown semantic similarity, but show a certain degree of difference in data distribution. These distribution differences lead to degradation of the performance of the currently deployed malware detectors on the test data. The model is represented as $\mathcal{M} = c \circ \phi$, where $c$ denotes the classifier and $\phi$ is the feature encoder. This paper aims to help the encoder in arbitrary detectors capture invariant semantic information shared with unseen test data based on training data. 




\subsection{Overall Architecture}
Section~\ref{motivation: failure} illustrates the importance of learning temporally stable and discriminative features to improve robustness against distribution drift. Based on this, we propose a temporal invariant training method, which can be integrated into arbitrary malware detectors. This method enables the encoder to capture rich discriminative information from the training data, and then suppress the unstable cross-environment factors through invariant risk minimization, enhancing robustness in drift scenarios. The training process relies on two core components: \textbf{Multi-Proxy Contrastive Learning (MPC)}: This component models the diversity in the representations of benign samples and malware using dynamically updated proxies. It specifically encodes the complex semantic information of multi-family malware into the embedding space, learning more compact and faithful embeddings that fully expose the latent discriminative information in the training data. \textbf{Invariant Gradient Alignment (IGA)}: This component uses invariant risk minimization (IRM) to reduce gradient variance across different environments for samples of the same class, encouraging the encoder to learn a unified representation for all environments (i.e., time intervals).

Figure~\ref{fig:model_architecture} illustrates the overall framework of the proposed method. We first divide Android applications into multiple non-overlapping environments based on their release time. Since the prerequisite for learning invariant features is the model’s ability to construct rich representations of potentially useful features, the invariant training process is divided into two stages: 
\begin{itemize}
    \item Discriminative Information Amplification Stage: In this stage, MPC is independently applied within each environment while minimizing empirical risk, allowing the shared encoder to integrate discriminative information from each environment.
    \item Unstable Information Compression Stage: Building on the model trained in the previous stage, MPC is applied across all environments to align feature representations. Following this, the IGA fine-tunes the encoder using IRM through the dummy classifiers to suppress unstable information.
\end{itemize}

% Section 3.2 illustrates the importance of learning stable and discriminative features for drift robustness enhancement, centred on the ability of the encoder to adequately capture the rich discriminative information in the training set, and later compress cross-environmental instabilities through invariant learning. Based on this, we propose a temporal invariant training method that can be integrated into an arbitrary malware detector to help it learn stable and discriminative information to enhance its robustness in drifting scenarios. The training process split into two key phases: \textbf{Discriminative information amplification}: We introduce multi-proxy contrastive learning to encode the complex semantic information of multi-family malware into the embedding space. Using multiple dynamically updated proxies, we model the natural diversity within benign and malicious samples, thereby learning more compact and faithful embeddings that fully expose the potential discriminative information in the training data. \textbf{Unstable information suppression}: This module suppresses unstable transient features through invariant risk minimization (IRM), reducing the variance of similar samples across different environments and encouraging the encoder to learn temporal-invariant feature representations.

% Motivated by the above considerations, we propose a temporal invariant training method that can learn stable and discriminative information from the training set in an end-to-end manner. This approach can be integrated into any malware detector to enhance its robustness in drift scenarios.  

% As shown in Figure.\ref{fig:model_architecture}, the overall architecture first divides the training samples into multiple disjoint environments. Discriminative information within each environment is amplified through multi-proxy contrastive learning to motivate the encoder to learn the rich discriminative information in each environment. Afterwards, unstable transient features are suppressed through invariant risk minimization, ensuring that the final embeddings primarily capture stable and discriminative features.

\subsection{Training Environments Segmentation}
\label{environment_split}
Segmenting the training environment requires exposing unstable information. Malware distribution drift arises from multiple factors, such as application market demands or Android version updates. Dividing the training environment based on application release times effectively captures this mix of unknown causes. For a dataset with timestamps from $T_{\min}$ to $T_{\max}$, and a chosen time granularity $\Delta$, samples are divided into $t$ time windows, each representing an environment $e \in \mathcal{E}$, where $|\mathcal{E}| = t$. The resulting training set is an ordered multiset, $\mathcal{D} = \{D_1,D_2,...,D_t\}$, with each environment $D_t$ containing $|D_t|$ samples. A sample $x_i$ with timestamp $t_i$ is assigned to an environment using:
\begin{equation}
\mathcal{E}(x_i)=\left\lfloor\frac{t_i-T_{\min }}{\Delta}\right\rfloor.
\end{equation}
As the environment index increases, the corresponding timestamps move closer to the present. The choice of time granularity depends heavily on dataset sparsity and involves a trade-off: finer granularity reveals detailed distribution patterns but risks insufficient sample size for effective representation learning, while coarser granularity may overlook distribution shifts. A balanced granularity can mitigate these issues but requires careful consideration of label distribution within each environment to prevent misalignment in learned representations. Section~\ref{env seg} examines the impact of different granularity choices.

\subsection{Multi-Proxy Contrastive Learning}
\label{multi-proxy contrastive learning}
Malware families have distinct distributions and imbalanced sample sizes, complicating malware category modeling in the embedding space. Benign samples may also display complex feature distributions due to factors like geographic location or user behavior. Treating all relationships within a category equally can exacerbate imbalances, while overly homogenizing samples overlooks valuable information. Thus, effective discrimination requires balancing the complex pairwise relationships within each category to ensure high-quality representations that capture sample diversity.

To address this, we design a multi-proxy contrastive learning method, where each proxy represents a subset of samples with similar behaviors or structures. This approach enables the encoder to capture finer-grained features within each class, rather than treating all benign or malicious samples as homogeneous groups.



% Modeling the malware category with different kinds of malware families which have diverse distributions and imbalanced sample sizes is challenging. Therefore, it is crucial to capture this kind of diversity within each class to better align the complex positive pair relationships among samples from the same category. To this end, we propose a multi-proxy contrastive learning method, where multiple proxies are dynamically obtained within each class. Each proxy represents a subset of samples with similar behavior or structural characteristics. This allows the encoder to capture more fine-grained features for each category at this stage, rather than naively treating all benign/malicious samples as a homogeneous group.

For a given batch of samples $\mathbf{X}_c \in \mathbb{R}^{|\mathcal{B}_c| \times d}$ from category $c$ , where $|\mathcal{B}_c|$ is the batch size and $d$ is the feature dimension, we randomly initialize $K$ proxies for both goodware and malware categories. Each sample is then assigned to relevant proxies. In this multi-proxy setup, avoiding dominance by any single proxy helps mitigate noise. Inspired by the Sinkhorn~\cite{sinkhorn} algorithm, we compute a probability matrix $\mathbf{W}_c$ for proxies within category $c$, assigning a weight $w_{ij}$ to proxy $j$ for each sample $x_i$ as follows:
\begin{equation}
w_{i j}^{(c)}=\frac{u_i \cdot \exp \left(\frac{sim(\mathbf{x}_i, \mathbf{P}_c^{(j)})}{\epsilon}\right) \cdot v_j}{\sum_{i=1}^{|\mathcal{B}_c|} \sum_{j=1}^K u_i \cdot \exp \left(\frac{sim(\mathbf{x}_i, \mathbf{P}_c^{(j)})}{\epsilon}\right) \cdot v_j},
\end{equation}
where $\epsilon$ is a temperature parameter controlling the smoothness of the assignment. $sim(\mathbf{x}_i, \mathbf{P}_c^{(j)})$ represents the dot-product similarity between the sample and the proxy. $u_i$ and $v_j$ are auxiliary normalization factors to ensure that all assigned weights satisfy a doubly-normalized constraint, adjusted iteratively through row and column scaling.

To reduce the complexity, the top-N proxies with larger weights are selected here to participate in the computation of the proxy loss. Thus, for the sample over the selected proxies, the proxy alignment loss is denoted as:
\begin{equation}
\mathcal{L}_{pal}=-\frac{1}{|\mathcal{B}_c|} \sum_{c=1}^C \sum_{i=1}^{|\mathcal{B}_c|}\left(\sum_{j=1}^N \mathbf{W}_c^{(i, j)} \cdot \log \left(p_{i j}^c\right)\right),
\end{equation}
where $p_{i j}^c$ is the softmax probability calculation for sample $i$ and proxy $j$ within each class $c$, represent as follow:
\begin{equation}
p_{i j}^c=\frac{\exp \left(sim(\mathbf{x}_i, \mathbf{P}_c^{(j)}) / \tau\right)}{\sum_{k=1}^N \exp \left(sim(\mathbf{x}_i, \mathbf{P}_c^{(k)}) / \tau\right)},
\end{equation}
where a temperature parameter $\tau$ is applied to scale the similarities.

The distribution of proxies determines the diversity of distributions the embedding can represent. To enhance the diversity, we introduce a proxy contrastive loss that disperses proxies within the same class. Each proxy serves as a positive sample for itself, while other proxies in the class are negatives. To distinguish proxy-to-proxy similarities from sample-proxy similarities, we define the former similarity as $sim_p(\mathbf{P}_c^{(i)}, \mathbf{P}_c^{(i)})$ and the proxy contrastive loss is given as:
\begin{equation}
\mathcal{L}_{pcl}=-\frac{1}{K} \sum_{c=1}^C \sum_{i=1}^K\left(\log \left(\tilde{p}_{ij}^c\right)\right)
\end{equation}
Similarly, $\tilde{p}_{ij}^c$ denotes the computation of softmax probability between proxy $i$ and proxy $j$ in category $c$, i.e:
\begin{equation}
\tilde{p}_{ij}^c = \frac{\exp \left(sim_p(\mathbf{P}_c^{(i)}, \mathbf{P}_c^{(i)}) / \tau\right)}{\sum_{j=1}^K \exp \left(sim_p(\mathbf{P}_c^{(i)}, \mathbf{P}_c^{(k)}) / \tau\right)}.
\end{equation}

During training, proxies in each category are progressively updated to adapt to the evolving distribution of applications. Updates are performed using a momentum strategy, where momentum is derived from the weighted sum of the previously obtained weight matrix and the samples associated with each proxy. The updated proxies $\mathbf{P}_c^{\prime}$ of class $c$ are computed as follows:
\begin{equation}
\mathbf{P}_c^{\prime}=\gamma \mathbf{P}_c+(1-\gamma) \mathbf{W}_c^T \mathbf{X}_c,
\end{equation}
where $\gamma$ is the momentum coefficient controlling the update rate. Therefore, we can obtain the multi-proxy contrastive loss as the training objective in this module, which can be formalized as:
\begin{equation}
\mathcal{L}_{MPC} = \mathcal{L}_{pal} + \lambda \cdot \mathcal{L}_{pcl}
\end{equation}
where $\lambda$ balance the weight of these two loss functions.




\subsection{Invariant Gradient Alignment}
Section~\ref{learn_invariant_feature} discusses the prerequisites for learning invariant features, specifically, the need for environment segmentation to expose unstable information and for rich feature representations. Therefore, based on the definition of Invariant Risk Minimization (IRM) in Section~\ref{invariant_learning}, our goal is to encourage encoders that can learn rich feature representations to further focus on the stable part of them. So here it is necessary to ensure that the representations generated by the encoder for the same class of samples in different environments produce similar gradients in the classifier. To achieve this, we construct an invariant gradient alignment module based on IRMv1, as proposed by Arjovsky \textit{et al.}~\cite{IRM_training}, which is used to strengthen the model's focus on invariant features. The objective function is described as Eq.\ref{grad_align}:
\begin{equation}
\label{grad_align}
\mathcal{L}_{IGA} = \frac{1}{|\mathcal{E}|} \sum_{e \in \mathcal{E}} \left\|\nabla_{s_{e} \mid s_{e}=1.0} R^{e}(s_e \circ \phi)\right\|^2
\end{equation}
Here, $s_{e}$ acts as a scalar, serving the role of a dummy classifier, set to 1.0 and updates through gradient backpropagation. $\mathcal{L}_{grad}$ evaluates how adjusting $s_e$ minimizes the empirical risk in environment $e$. $R^e(s_e \circ \phi)$ is represented as Eq.~\ref{risk_minimization}:
\begin{equation}
\label{risk_minimization}
R^{e}(s_{e} \circ \phi)=\mathbb{E}^{e}\left[\mathcal{L}_{CLS} \left(s_{e}\left(\phi\right(x\left)\right), y\right)\right],
\end{equation}
where $\mathcal{L}_{CLS}$ denotes the classification loss function used in the current environment, such as binary cross-entropy for malware detection. This loss is computed following standard ERM training. The term $\phi(x)$ represents the output of the shared encoder for samples $x, y \sim p(x, y|e)$ from environment $e$. The gradient penalty term encourages uniform feature representation by aligning gradients of classifiers across environments, thereby promoting consistent model performance.

\subsection{Invariant Training Framework}
\label{invariant training}
Gradient adjustment for invariant learning classifiers relies heavily on the encoder’s ability to learn rich representations~\cite{rich}. Starting classifier training from random initialization can cause IRM to converge to local optima. To address this, we propose a two-stage training strategy to capture a broad set of useful features before applying invariant learning.

\subsubsection{Discriminative Information Amplification}
In the first stage, we perform ERM training independently for each environment as the encoder's initialization. The multi-proxy contrastive learning module is applied to each environment respectively to fully exploit the discriminative features within. The optimization objective in this stage is defined as Eq.\ref{erm loss}:
\begin{equation}
\label{erm loss}
\mathcal{L}_{ERM} = \frac{1}{|\mathcal{E}|} \sum_{e \in \mathcal{E}} (\mathcal{L}_{CLS}^e + \alpha \cdot \mathcal{L}_{MPC}^e),
\end{equation}
where $\mathcal{L}_{CLS}^e$ and $\mathcal{L}_{MPC}^e$ denote the classification loss and multi-proxy contrastive loss for environment $e$, respectively. Jointly minimizing the empirical risk across all environments in this stage enables the encoder to learn diverse feature representations, creating a robust foundation for subsequent invariant training.

\subsubsection{Unstable Information Suppression}
To mitigate overfitting to environment-specific features from the earlier stage, we reset the optimizer’s parameters before the second training phase. This reset allows the model to refocus on the objectives of invariant learning. In this phase, we first apply a multi-proxy contrastive loss across all samples to enhance class representation learning. Next, invariant gradient alignment is used to harmonize classification gradients across environments. The updated optimization objective is defined in Eq.~\ref{irm loss}:
\begin{equation}
\label{irm loss}
\mathcal{L}_{IRM} = \mathcal{L}_{CLS} + \alpha \cdot \mathcal{L}_{MPC} + \beta \cdot \mathcal{L}_{IGA},
\end{equation}
In training, the hyperparameters $\alpha$ and $\beta$ balance the contributions of each loss term. This two-stage approach enables the model to first capture a broad feature set, then refine it for cross-environment invariance, enhancing generalization under distribution shifts. The pseudo-code of the training process is shown in Appendix~\ref{alg1}.

% \begin{algorithm}[htb]
\caption{Algorithm of Invariant Training}
\label{alg1}
    \begin{algorithmic}[1]
    \REQUIRE ~training dataset containing $|\mathcal{E}|$ environments $\mathcal{D}_{tr}$ and each sample $x$ has a binary classification label $y_{c}, c \in [0, 1]$ and an environment label $y_e, e \in \mathcal{E}$, batch size $|\mathcal{B}|$, predefined number of epochs for stage 1 training $N$, hyperparameters $\alpha$ and $\beta$ for loss weighting, model $f$ with encoder network $\phi$ and predictor $h$. 
    \ENSURE 
    \STATE \textbf{Stage 1: Discriminative Information Amplification}
    \FOR{$\text{epoch} = 1$ to $N$}
        \FOR{$i = 1, \ldots, |\mathcal{B}|$}
            \FOR{$e \in \mathcal{E}$}
            \STATE select mini-batch of samples ${x_i^{e}} \subseteq \mathcal{B}$ \\
            \STATE calculate classification loss $\mathcal{L}^{e}_{CLS}$\\
            \STATE calculate multi-proxy contrastive loss $\mathcal{L}^{e}_{MPC}$ \\
            \ENDFOR
        \STATE obtain loss for empirical risk minimization $\mathcal{L}_{ERM} = \frac{1}{|\mathcal{E}|} \sum_{e \in \mathcal{E}} \mathcal{L}^{e}_{CLS} + \alpha \cdot \mathcal{L}^{e}_{MPC}$
        \STATE update $f$ by backpropagating $\mathcal{L}_{ERM}$
        \ENDFOR
    \ENDFOR
    \STATE \textbf{Stage 2: Unstable Information Suppression}
    \STATE Reset optimizer parameters
    \STATE Continue with model parameters from Stage 1\\
    \FOR{$\text{epoch} = N+1$ to $\text{TotalEpochs}$}
        \FOR{$i = 1, \ldots, |\mathcal{B}|$}
            \FOR{$e \in \mathcal{E}$}
            \STATE initialize dummy classifier $\{s_{e} = 1.0\}$ \\
            \ENDFOR
            \STATE calculate invariant gradient alignment loss $\mathcal{L}_{IGA}$ \\
            \STATE obtain classification loss $\mathcal{L}_{CLS}$ \\
            \STATE obtain multi-proxy contrastive loss $\mathcal{L}_{MPC}$ \\
            \STATE $\mathcal{L}_{IRM} \gets \mathcal{L}_{CLS} + \alpha \cdot \mathcal{L}_{MPC} + \beta \cdot \mathcal{L}_{IGA}$ \\
            \STATE update $f$ by backpropagating $\mathcal{L}_{IRM}$
        \ENDFOR
    \ENDFOR
    \end{algorithmic}
\end{algorithm}







