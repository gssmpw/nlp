\section{Background}
In this section, we review the development of malware detectors robust to drift and the key components of invariant learning, laying the groundwork for proposing an temporal invariant representation learning solution tailored for Android malware detection.


% \subsection{Concept Drift Caused by Malware Evolution}
% \label{concept_drift_in_malware}
% Concept drift refers to the changes in the statistical properties of data over time, which can render models trained under the assumption of independent and identically distributed (i.i.d.) data ineffective. This causes new data distributions to gradually deviate from the original decision boundaries, significantly impairing malware detection performance. Researchers typically identify two main causes of concept drift in Android malware: 

% \begin{enumerate}
%     \item Technological evolution of malware: Even malware from the same family can evolve due to market updates, framework iterations, and new obfuscation techniques, altering the program architecture that implements the same semantics~\cite{malware_evolution_obf}. These changes lead to a shift in the feature distribution, causing the classification boundaries established by the original detectors to gradually become obsolete~\cite{malware_evolution_update}.
%     \item Emergence of unknown malware families: Learning-based classification models are generally trained on datasets with fixed categories. However, the emergence of new malware families and changes in trends can introduce previously unseen malware families during model inference~\cite{new_family}. These samples may not resemble any known malware from the training dataset, leading to unreliable confidence scores and incorrect classifications.
% \end{enumerate}

\subsection{Drift-robust Malware Detectors}
Existing approaches to enhancing model robustness against drift focus primarily on refining feature representations. One line of work aims to identify features that remain robust to distribution changes, either through manual or automated feature engineering, allowing models to learn patterns less influenced by malware evolution~\cite{apigraph}. Another approach anticipates potential future distribution patterns by employing specific training strategies. Lu~et al.~\cite{DOMR} constructs pseudo-unknown families in the training set and uses episode learning to help the model learn robust representations that can generalize to open environments. Episode-based training strategy is introduced in the few-shot learning that divides the training process into a large number of episodes~\cite{episode_learning_1, episode_learning_2}, each simulating a task with limited samples, to foster representation learning. Here, the unknown families are treated as low-sample tasks. This method primarily addresses drift caused by the emergence of new families and requires managing an excessive number of episodes due to the need to cover all families in the training set. Yang et al.~\cite{scrr} propose the use of feature de-entanglement to keep the features as independent as possible during the model training process, minimizing their spurious correlations to preserve invariant information. However, the correlation between malware features is often critical, as certain sensitive API calls or intent combinations may be directly indicative of malicious behavior, and simply reducing feature correlation may be suboptimal. Angioni~et al.~\cite{svm_ce} introduce a feature stability metric for linear classifiers in the Drebin feature space, contributing insights for drift robustness in linear models but with limited applicability to complex feature relationships and non-linear models.

These prior methods seek to enhance invariant feature information by mitigating instabilities, which aligns with our approach. However, due to the diverse evolution and complexity of malware, pinpointing the precise cause of drift remains challenging. Strategies that rely on assumptions, such as drift being driven by new malware families, are limited. Instead, recognizing that distribution drift inevitably occurs over time motivates our exploration of temporally invariant information.

\subsection{Invariant Learning}
\label{invariant_learning}
\subsubsection{Notations}
Assume that the training data $\mathcal{D}_{tr}$ is collected from multiple environments $e \in \mathcal{E}$, i.e., $\mathcal{D}_{tr}=\{D^e_{tr}\}_{e \in \mathcal{E}}$. Let the input space be $x \in \mathcal{X}$ and the target space be $y \in \mathcal{Y}$, and for the sample observations from each environment denoted $x,y \sim p(x,y|e)$, the samples within the environments obey an independent and identical distribution. Suppose a classification model $\mathcal{M}$, denoted as a composite function $\mathcal{M}=c \circ \phi$. where $\phi:\mathcal{X} \rightarrow \mathcal{H}$ denotes a feature encoder that maps the input samples into a feature representation space $\mathcal{H}$, and $\phi(x) \in \mathcal{H}$ is the ``representation'' of sample $x$. The $c:\mathcal{H} \rightarrow \mathcal{Y}$ denotes a classifier that maps the feature representation to the logits space of $\mathcal{Y}$. 


\subsubsection{Learning Invariant Representation}
In test environments where distribution drift exists, the test data $\mathcal{D}_{ts}$ may come from a distribution $p(x,y|e_{ts})$ that does not appear in the training set, i.e. $e_{ts} \notin \mathcal{E}$. Robustness to drift yields lower error rates on unknown test data distributions.

Invariant learning enhances the generalization ability of a model to unknown distributions by learning label distributions that are invariant across training environments. Its goal is to develop a classifier $c(\cdot)$ that satisfies the environmental invariance constraint (EIC)~\cite{EIC}:

\begin{equation}
\mathbb{E}\left[y \mid \phi(x), e\right] = \mathbb{E}\left[y \mid \phi(x), e^{\prime}\right], \quad \forall e, e^{\prime} \in \mathcal{E},
\end{equation}

where $e$ and $e'$ denote different environments to which the samples belong. This constraint is integrated into the training objective through a penalty term. Thus, the goal can be formalized as:

\begin{equation}
\min _f\sum_{e \in \mathcal{E}} R^e_{erm}(f)+\lambda \cdot \operatorname{penalty}\left(\left\{S^e(f)\right\}_{e \in \mathcal{E}}\right),
\end{equation}

where $R^e_{erm}(f) = \mathbb{E}_{p(x,y|e)}[\ell(f(x),y)]$ represents the expected loss on environment $e$. Empirical risk minimization (ERM) is to minimize this expected loss within each environment. $S^e(f)$ is some statistic of the model in $e$ (see next), and the penalty is to constrain the change in this statistic to control the degree of deviation from the EIC. Optimizing this objective prevents mapping all $x$ to the same value to satisfy environmental invariance, as it encourages the predictive utility of $\phi$ by minimizing expected loss. In addition, the form of the penalty term is variable to achieve constraints for different objectives. Krueger et al.~\cite{v-rex} proposed V-Rex such that $S^e(f)=R^e_{erm}(f)$, to minimize the variance of $S^e(f)$ in different environments. In CLOvE~\cite{causal_ir}, the penalty is defined as the sum of the calibration errors of the model in each environment. One widely used scheme is Invariant Risk Minimization (IRM) and its practical variant IRMv1 proposed by Arjovsky et al.~\cite{IRM_training}. The penalty term is the sum of $S^e(f)=\left\|\nabla_w R^e_{erm}(w \circ \phi)\right\|^2$. $w$ is a constant scalar multiplier of 1.0 for each output dimension, forcing the shared classifier to be optimal in all environments. 

\subsubsection{Split Environments for Invariant Learning}
Invariant learning relies on the premise that environments should be segmented to maximize inter-environment differences~\cite{IR_intro}. Earlier solutions where environment labels are known a prior knowledge~\cite{bottleneck_ir, empirical_ir}. However, in practice, such a clear and accurate environment index of the environment is always unavailable~\cite{environment_label,inaccessible_label_2}. Nowadays, solutions turn to studying invariant learning without environment labels. Creager et al.~\cite{EIC} propose environment inference for invariant learning (EIIL), where the environment labels of the training data are estimated employing prior clustering, after which invariant learning is performed. Teney et al.~\cite{unshuffling} explore environment segmentation in visual question-answering, using two approaches: one based on provided question-type labels, and another using unsupervised clustering to create environments. Their results indicate that natural segmentation outperforms unsupervised clustering. Regardless of whether environment segmentation is learned through additional optimization or based on dataset-provided information, it aims to expose and ignore unstable information across environments~\cite{IR_intro}. 