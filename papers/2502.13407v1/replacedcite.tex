\section{RELATED WORKS}
\label{B}
\subsection{Traditional and ML-Based CD}  

Traditional CD methods have been extensively studied in remote sensing, with early approaches relying on simple algebraic operations such as image differencing ____ and image ratioing ____. 
These techniques compute pixel-level differences or ratios between two images and apply a threshold to identify change regions. 
Subsequent advancements introduced improved thresholding strategies, such as the Otsu method ____ and the normalized difference vegetation index (NDVI) algorithm ____, to enhance detection accuracy. 
Transform-based methods, such as PCA ____ and MAD ____, were later adopted, leveraging statistical properties of images. 
However, these methods are heavily dependent on image statistics, which limits their scalability and precision in large-scale, high-resolution CD applications.  
The advent of machine learning has significantly enhanced the ability to extract useful change features. 
For instance, Bovolo et al. ____ proposed an unsupervised change detection framework that leverages a semi-supervised SVM initialized with pseudo-training data, effectively addressing the complexity of multi-temporal spectral feature analysis. 
Wessels et al. ____ developed an automated system for land cover update mapping, integrating iteratively reweighted MAD (IRMAD) for change mask generation and RF classifiers for robust land cover classification, achieving notable accuracy in operational settings.  
Despite these advancements, traditional and early ML-based approaches often rely on manually designed features, which perform well in straightforward scenarios but exhibit limited generalization capability for complex and diverse change types. 

\subsection{DL-Based CD}
In recent years, deep learning has experienced rapid advancements, achieving remarkable success in remote sensing image CD. 
As illustrated in Fig. \ref{fig:survey}, we present a timeline of the development of mainstream DL-based CD algorithms. 
Based on the differences in neural network architectures and training paradigms, DL-based CD methods can be classified into three main categories

\paragraph{CNN-Based CD}  
CNNs serve as the foundation of many early DL-based CD methods and remain widely utilized today.
Daudt et al. ____ proposed three fully convolutional neural network architectures, including two Siamese network extensions, which achieved significant improvements in accuracy and efficiency for CD tasks on multiple datasets.
Zhang et al. ____ introduced the Image Fusion Network (IFN), which employs a deeply supervised two-stream architecture for high-resolution remote sensing CD, achieving SOTA performance with superior boundary completeness and compactness in change maps.
Chen et al. ____ proposed the Siamese-based Spatial-Temporal Attention Network (STANet), incorporating a novel CD self-attention module to model spatial-temporal dependencies at various scales, significantly improving F1-scores on benchmark datasets.
Fang et al. ____ designed SNUNet-CD, a densely connected Siamese network that preserves localization information and employs an Ensemble Channel Attention Module (ECAM) for deep supervision, achieving better trade-offs between accuracy and computational cost.
Zheng et al. ____ proposed ChangeStar, a model leveraging single-temporal supervised learning with ChangeMixin modules to train CD models using unpaired images.
Han et al. ____ introduced HANet, a hierarchical attention network with progressive foreground-balanced sampling and a lightweight self-attention mechanism, effectively addressing class imbalance in CD tasks and achieving superior results on highly imbalanced datasets.
The Change Guiding Network (CGNet) introduced by Han et al. ____ utilizes a self-attention mechanism to improve edge detection and internal consistency in change maps, demonstrating robust performance across multiple CD datasets.
Some studies have focused on designing lightweight and fast CD models.
Codegoni et al. ____ presented TinyCD, a lightweight and efficient CD model using a Siamese U-Net architecture and the Mix and Attention Mask Block (MAMB), outperforming SOTA models while being significantly smaller and faster.
Xing et al. ____ proposed LightCDNet, a lightweight CD model with an early fusion backbone and pyramid decoder.

% Figure: Dataset
\begin{figure*}[!t]
    \centering
    \includegraphics[width=7in]{pic/dataset-class.png}
    \caption{Sample images from the JL1-CD dataset. Each row, from top to bottom, represents: the image at time 1, the image at time 2, and the ground truth label. Each column corresponds to different change types: (a) Decrease in woodland; (b) Building changes; (c) Conversion of cropland to greenhouses; (d) Road changes; (e) Waterbody changes; and (f) Surface hardening (central region).}
    \label{fig:dataset-class}
\end{figure*}

\paragraph{Transformer-Based CD}  
Transformer-based methods have emerged as a promising approach for CD. 
Chen et al. ____ introduced the bitemporal image transformer (BIT), combining a transformer encoder with a ResNet backbone to model spatial-temporal contexts efficiently. 
Bandara et al. ____ proposed ChangeFormer, a fully transformer-based Siamese network for CD, which unifies a hierarchical transformer encoder with a multi-layer perceptron (MLP) decoder.
Fang et al. ____ introduced the Changer series framework, a novel architecture for CD that incorporates alternative interaction layers between bi-temporal features. 
This framework is applicable to both CNN-based and Transformer-based models, significantly enhancing the performance of the original models.

\paragraph{FM-Based CD}  
Recently, foundation models have become a new training paradigm, and some works have applied them to CD tasks.
Li et al. ____ proposed the Bi-Temporal Adapter Network (BAN), a universal FM-based framework for CD, which enhances existing models with minimal additional parameters and achieves significant performance improvements.
Chen et al. ____ introduced Time Travelling Pixels (TTP), a method that integrates latent knowledge from the SAM model into CD, overcoming domain shifts and spatio-temporal complexities, demonstrating SOTA results on the LEVIR-CD____ dataset.
Zheng et al. ____ developed AnyChange, a zero-shot CD model built on the SAM that utilizes bitemporal latent matching for training-free adaptation, setting a new SOTA on the SECOND____ benchmark and achieving significant improvements in both unsupervised and supervised CD tasks.

\subsection{Knowledge Distillation in CD}

Knowledge distillation (KD), introduced by Hinton et al. ____, aims to transfer the representational knowledge of a teacher network to a smaller student network. 
In recent years, as the complexity of DL models in remote sensing tasks has increased, researchers have explored how to transfer knowledge from large, complex teacher models to smaller, more efficient student models through KD, thereby improving performance____.

Yan et al. ____ proposed a novel self-supervised learning approach for unsupervised CD by fusing domain knowledge of remote sensing indices during both training and inference. By calculating cosine similarity, they selected high-similarity feature vectors from both the teacher and student networks to implement a hard negative sampling strategy, effectively improving CD performance.
Wang et al. ____ addressed remote sensing semantic CD (SCD), which focuses on detecting changes in land cover and land use over time. The authors introduced a dual-dimension feature interaction network (DFINet) that enhances intraclass and interclass feature differentiation by incorporating a temporal difference feature enhancement (TDFE) module, which captures temporal features comprehensively. 
Wang et al. ____ proposed a KD-based method for CD (CDKD), designed to overcome the challenges of deploying large deep learning models with high computational and storage requirements on resource-constrained spaceborne edge devices. 
Although these methods have successfully utilized KD to enhance the performance of various student models, they are tailored to specific models and do not provide a generalized distillation framework applicable to various CD models. 
Furthermore, there is a lack of open-source KD-based code for remote sensing image CD tasks.

In contrast, the proposed MTKD framework significantly improves the performance of CD models with various architectures and parameter sizes, and we commit to open-sourcing all the code and models.


\begin{table}[!t]
\centering
\caption{Information of Open-Source CD Datasets and the Proposed JL1-CD Dataset \label{table:dataset_information}}
\renewcommand{\arraystretch}{1.25}
\begin{tabular}{lcccc}
\hline
Dataset & Class & Image Pairs & Image Size & Resolution \\ \hline
SZTAKI____ & 1  & 13  & \begin{tabular}[c]{@{}c@{}}$952\times640$\\ $1,048\times724$\end{tabular} & 1.5 \\
DSIFN____     & 1     & 394         & 512 × 512  & 2  \\
SECOND____    & 6     & 4,662   & 512 × 512 & 0.5-3 \\
WHU-CD____    & 1     & 1    & 32,20 × 15,354 & 0.2    \\
LEVIR-CD____  & 1     & 637  & 1,024 × 1,024   & 0.3   \\
S2Looking____ & 1  & 5,000 & 1,024 × 1,024 & 0.5-0.8 \\
CDD____       & 1     & 16,000      & 256 × 256  & 0.03-1 \\
SYSU-CD____   & 1     & 20,000  & 256 × 256 & 0.5  \\
JL1-CD  & 1  & 5,000   & 512 × 512  & 0.5-0.75     \\ \hline
\end{tabular}
\end{table}

% Figure: Diagram
\begin{figure*}[!t]
    \centering
    \includegraphics[width=7in]{pic/pipeline.png}
    \caption{Overview of the training (green boxes) and testing (pink boxes) pipelines of the proposed Origin-Partition (O-P) strategy and Multi-Teacher Knowledge Distillation (MTKD) framework.}
    \label{fig:pipeline}
\end{figure*}