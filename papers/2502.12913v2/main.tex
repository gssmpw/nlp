% This must be in the first 5 lines to tell arXiv to use pdfLaTeX, which is strongly recommended.
\pdfoutput=1
% In particular, the hyperref package requires pdfLaTeX in order to break URLs across lines.

\documentclass[11pt]{article}

% Change "review" to "final" to generate the final (sometimes called camera-ready) version.
% Change to "preprint" to generate a non-anonymous version with page numbers.
% \usepackage[review]{acl}
\usepackage[final]{acl}

% Standard package includes
\usepackage{times}
\usepackage{latexsym}

% For proper rendering and hyphenation of words containing Latin characters (including in bib files)
\usepackage[T1]{fontenc}
% For Vietnamese characters
% \usepackage[T5]{fontenc}
% See https://www.latex-project.org/help/documentation/encguide.pdf for other character sets

% This assumes your files are encoded as UTF8
\usepackage[utf8]{inputenc}

% This is not strictly necessary, and may be commented out,
% but it will improve the layout of the manuscript,
% and will typically save some space.
\usepackage{microtype}

% This is also not strictly necessary, and may be commented out.
% However, it will improve the aesthetics of text in
% the typewriter font.
\usepackage{inconsolata}

%Including images in your LaTeX document requires adding
%additional package(s)
\usepackage{microtype}
\usepackage{graphicx}
\usepackage{subfigure}
\usepackage{booktabs} % for professional tables
% hyperref makes hyperlinks in the resulting PDF.
% If your build breaks (sometimes temporarily if a hyperlink spans a page)
% please comment out the following usepackage line and replace
% \usepackage{icml2025} with \usepackage[nohyperref]{icml2025} above.
\usepackage{hyperref}
% Attempt to make hyperref and algorithmic work together better:
\newcommand{\theHalgorithm}{\arabic{algorithm}}
% For theorems and such
\usepackage{amsmath}
\usepackage{mathrsfs}
\usepackage{amssymb}
\usepackage{mathtools}
\usepackage{amsthm}
\usepackage{url}            % simple URL typesetting
\usepackage{bm}
\usepackage{booktabs}       % professional-quality tables
\usepackage{amsfonts}       % blackboard math symbols
\usepackage{nicefrac}       % compact symbols for 1/2, etc.
\usepackage{microtype}      % microtypography
\usepackage{adjustbox}
\usepackage{xcolor}         % colors
\usepackage{blindtext}
\usepackage{enumitem}
\usepackage{listings}
\usepackage{pifont}
\usepackage{epsfig} % for postscript graphics files
\usepackage{times}
\usepackage{color}
% \usepackage[table]{xcolor}
\usepackage{multirow}
\usepackage{multicol}
\usepackage{makecell}
\usepackage{algorithm}
\usepackage{rotating}
\usepackage{subfigure}
\usepackage[T1]{fontenc}
\usepackage{algorithm}
\usepackage{algorithmic}
% \usepackage[noend]{algpseudocode}
\usepackage{colortbl}
\usepackage{color, soul}
\usepackage{arydshln} % For dashline
\usepackage{tcolorbox}
\definecolor{babyblue}{rgb}{0.54, 0.81, 0.94}
\definecolor{bisque}{rgb}{1.0, 0.89, 0.77}
\definecolor{bshade}{rgb}{0.55,0.75,0.95}
\def\eg{{\it e.g.}\xspace}
\def\ie{{\it i.e.}\xspace}
\definecolor{mygray}{gray}{.6}
\definecolor{myblue}{RGB}{89,158,254}
\definecolor{mygreen1}{RGB}{81,150,111}
\definecolor{mygreen2}{RGB}{93,174,86}
\definecolor{myred}{RGB}{160,0,0}
\definecolor{myyellow}{RGB}{227,207,87}
% For pictures
\usepackage{caption}
\usepackage[capitalize,noabbrev]{cleveref}
\usepackage{threeparttable}
\usepackage{wrapfig}

%%%%%%%%%%%%%%%%%%%%%%%%%%%%%%%%
% THEOREMS
%%%%%%%%%%%%%%%%%%%%%%%%%%%%%%%%
\theoremstyle{plain}
\newtheorem{theorem}{Theorem}[section]
\newtheorem{proposition}[theorem]{Proposition}
\newtheorem{lemma}[theorem]{Lemma}
\newtheorem{corollary}[theorem]{Corollary}
\theoremstyle{definition}
\newtheorem{definition}[theorem]{Definition}
\newtheorem{assumption}[theorem]{Assumption}
\theoremstyle{remark}
\newtheorem{remark}[theorem]{Remark}
\definecolor{babyblue}{rgb}{0.54, 0.81, 0.94}
\definecolor{bisque}{rgb}{1.0, 0.89, 0.77}
\definecolor{bshade}{rgb}{0.55,0.75,0.95}
\def\eg{{\it e.g.}\xspace}
\def\ie{{\it i.e.}\xspace}
\definecolor{mygray}{gray}{.6}
\definecolor{myblue}{RGB}{89,158,254}
\definecolor{mygreen1}{RGB}{81,150,111}
\definecolor{mygreen2}{RGB}{93,174,86}
\definecolor{myred}{RGB}{160,0,0}
\definecolor{myyellow}{RGB}{227,207,87}
\usepackage{bbding}
\newcommand{\cmark}{\ding{52}}%
\newcommand{\xmark}{\ding{55}}%
\usepackage{enumitem}
\let\oldding\ding% Store old \ding in \oldding
\renewcommand{\ding}[2][1]{\scalebox{#1}{\oldding{#2}}}
% Todonotes is useful during development; simply uncomment the next line
%    and comment out the line below the next line to turn off comments
%\usepackage[disable,textsize=tiny]{todonotes}
\usepackage[textsize=tiny]{todonotes}
\newcommand{\ceil}[1]{\left\lceil #1 \right\rceil}
\newcommand{\floor}[1]{\left\lfloor #1 \right\rfloor}
\newcommand{\round}[1]{\left\lfloor #1 \right\rceil}
\newcommand{\sign}{\mathrm{sign}}
\newcommand{\tofloat}[1]{\mbox{float}\left(#1\right)}
\newcommand{\toint}[2]{\mbox{int}_{#1}\left(#2\right)}


\newcommand{\methodname}{GSQ-Tuning}
\newcommand{\dataname}{GSE}
\newcommand{\syz}[1]{\textcolor{teal}{SYZ:#1}}
% SamJ's commands
%!TEX root=main.tex
\newif\ifspacehack
%\spacehacktrue
\usepackage{natbib}
\hypersetup{
    colorlinks = blue,
    breaklinks,
    linkcolor = blue,
    citecolor = blue,
    urlcolor  = blue,
}
\usepackage{url} 
\usepackage{graphicx}
\usepackage{mathtools}
\usepackage{footnote}
\usepackage{float}
\usepackage{xspace}
\usepackage{multirow}
\usepackage{xcolor}
\usepackage{wrapfig}
\usepackage{framed}
\usepackage{bbm}
\usepackage[most]{tcolorbox}

\usepackage{footnote}
\usepackage{nicefrac}
\usepackage{makecell}
\usepackage[ruled,vlined]{algorithm2e}
\usepackage{amssymb}
\usepackage{bm}
\makesavenoteenv{tabular}
\makesavenoteenv{table}

\newcommand{\hl}[1]{{\color{red}[HL: #1]}}
\newcommand{\mznote}[1]{{\color{blue}[MZ: #1]}}

% macros@Peng
\newcommand\innerp[2]{\langle #1, #2 \rangle}
\renewcommand{\tilde}{\widetilde}
\renewcommand{\hat}{\widehat}


\newcommand{\TVD}[1]{\norm{#1}_\text{TV}}
\newcommand{\corral}{\textsc{Corral}\xspace}
\newcommand{\expthree}{\textsc{Exp3}\xspace}
\newcommand{\expfour}{\ensuremath{\mathsf{Exp4}}\xspace}
\newcommand{\expthreeP}{\textsc{Exp3.P}\xspace}
\newcommand{\scrible}{\textsc{SCRiBLe}\xspace}

\def \R {\mathbb{R}}
\newcommand{\eps}{\epsilon}
\newcommand{\vecc}{\mathrm{vec}}
\newcommand{\LS}{\mathrm{LS}}
\newcommand{\FG}{\mathrm{FG}}
\newcommand{\DL}{\Delta \ellhat}
\newcommand{\calA}{{\mathcal{A}}}
\newcommand{\smax}{{\mathrm{smax}}}
\newcommand{\calB}{{\mathcal{B}}}
\newcommand{\calX}{{\mathcal{X}}}
\newcommand{\calS}{{\mathcal{S}}}
\newcommand{\calF}{{\mathcal{F}}}
\newcommand{\calI}{{\mathcal{I}}}
\newcommand{\calJ}{{\mathcal{J}}}
\newcommand{\calK}{{\mathcal{K}}}
\newcommand{\calH}{{\mathcal{H}}}
\newcommand{\calD}{{\mathcal{D}}}
\newcommand{\calE}{{\mathcal{E}}}
\newcommand{\calG}{{\mathcal{G}}}
\newcommand{\calU}{{\mathcal{U}}}
\newcommand{\calR}{{\mathcal{R}}}
\newcommand{\calT}{{\mathcal{T}}}
\newcommand{\calP}{{\mathcal{P}}}
\newcommand{\calQ}{{\mathcal{Q}}}
\newcommand{\calZ}{{\mathcal{Z}}}
\newcommand{\calM}{{\mathcal{M}}}
\newcommand{\calN}{{\mathcal{N}}}
\newcommand{\calW}{{\mathcal{W}}}
\newcommand{\calY}{{\mathcal{Y}}}
\newcommand{\cD}{{\mathcal{D}_{\mathcal{X}}}}
\newcommand{\mcD}{{\mathcal{D}}}
\newcommand{\cF}{{\mathcal{F}}}
\newcommand{\cA}{{\mathcal{A}}}
\newcommand{\cX}{{\mathcal{X}}}
\newcommand{\cE}{{\mathcal{E}}}
\newcommand{\cV}{{\mathcal{V}}}
\newcommand{\cR}{{\mathcal{R}}}
\newcommand{\wcR}{\widehat{\mathcal{R}}}
\newcommand{\Reg}{{\mathrm{Reg}}}
\newcommand{\Alg}{{\mathsf{Alg}}}
\newcommand{\wReg}{\widehat{\mathrm{Reg}}}
\newcommand{\cB}{\mathcal{B}}
\newcommand{\cP}{\mathcal{P}}
\newcommand{\nctx}{\text{n-ctx}}
\newcommand{\ctx}{\text{ctx}}
\newcommand{\E}{{\mathbb{E}}}
\newcommand{\V}{\mathbb{V}}
\newcommand{\Prob}{\mathbb{P}}
\newcommand{\1}{\mathbb{I}}
\newcommand{\N}{\mathbb{N}}
\newcommand{\tup}[1]{t^{(#1)}}
\newcommand{\gup}[1]{g^{(#1)}}
\newcommand{\hatfm}{\widehat{f}_m}
\newcommand{\haty}{\widehat{y}}
\newcommand{\hatx}{\widehat{x}}
\newcommand{\yhat}{\widehat{y}}
\newcommand{\xhat}{\widehat{x}}
\newcommand{\fhat}{\widehat{f}}
\newcommand{\ghat}{\widehat{g}}

\newcommand{\inner}[1]{ \left\langle {#1} \right\rangle }
\newcommand{\ind}{\mathbb{I}}
\newcommand{\diag}{\textrm{diag}}
\newcommand{\Nout}{N_{\textrm{out}}}
\newcommand{\nout}{N_{\textrm{out}}}
\newcommand{\Nin}{{\textrm{Nin}}}
\newcommand{\nin}{{\textrm{Nin}}}
\newcommand{\order}{\mathcal{O}}


\newcommand{\Acal}{\mathcal{A}}
\newcommand{\Bcal}{\mathcal{B}}
\newcommand{\Ccal}{\mathcal{C}}
\newcommand{\Dcal}{\mathcal{D}}
\newcommand{\Ecal}{\mathcal{E}}
\newcommand{\Fcal}{\mathcal{F}}
\newcommand{\Gcal}{\mathcal{G}}
\newcommand{\Hcal}{\mathcal{H}}
\newcommand{\Ical}{\mathcal{I}}
\newcommand{\Jcal}{\mathcal{J}}
\newcommand{\Kcal}{\mathcal{K}}
\newcommand{\Lcal}{\mathcal{L}}
\newcommand{\Mcal}{\mathcal{M}}
\newcommand{\Ncal}{\mathcal{N}}
\newcommand{\Ocal}{\mathcal{O}}
\newcommand{\Pcal}{\mathcal{P}}
\newcommand{\Qcal}{\mathcal{Q}}
\newcommand{\Rcal}{\mathcal{R}}
\newcommand{\Scal}{\mathcal{S}}
\newcommand{\Tcal}{\mathcal{T}}
\newcommand{\Ucal}{\mathcal{U}}
\newcommand{\Vcal}{\mathcal{V}}
\newcommand{\Wcal}{\mathcal{W}}
\newcommand{\Xcal}{\mathcal{X}}
\newcommand{\Ycal}{\mathcal{Y}}
\newcommand{\Zcal}{\mathcal{Z}}
\newcommand{\wkdn}{d}


\newcommand{\avgR}{\wh{\cal{R}}}
%\newcommand{\ips}{\wh{r}}
\newcommand{\whpi}{\wh{\pi}}
\newcommand{\whE}{\wh{\E}}
\newcommand{\whV}{\wh{V}}

\newcommand{\whReg}{\wh{\text{\rm Reg}}}
\newcommand{\flg}{\text{\rm flag}}
\newcommand{\one}{\boldsymbol{1}}
\newcommand{\var}{\Delta}
\newcommand{\Var}{\mathrm{Var}}
\newcommand{\bvar}{\bar{\Delta}}
\newcommand{\p}{\prime}
\newcommand{\evt}{\textsc{Event}}
\newcommand{\unif}{\text{\rm Unif}}
\newcommand{\KL}{\text{\rm KL}}
\newcommand{\Lstar}{{L^\star}}
\newcommand{\istar}{{i^\star}}
\newcommand{\dynreg}{\text{Dyn-Reg}}
\newcommand{\tildedynreg}{\widetilde{\text{Dyn-Reg}}}
\newcommand{\Bstar}{{B^\star}}
\newcommand{\Ustar}{\rho}
\newcommand{\Aconst}{a}
\newcommand{\dplus}[1]{\bm{#1}}
\newcommand{\lambdamax}{\lambda_\text{\rm max}}
\newcommand{\biasone}{\textsc{Deviation}\xspace}
\newcommand{\bias}{\textsc{Bias-1}\xspace}
\newcommand{\biastwo}{\textsc{Bias-2}\xspace}
\newcommand{\biasthree}{\textsc{Bias-3}\xspace}
\newcommand{\term}[1]{\texttt{Term} ~(#1)\xspace}
\newcommand{\x}{\mathbf{x}}
\newcommand{\errorterm}{\textsc{Error}\xspace}
\newcommand{\Err}[1]{\textsc{Err-Term}(#1)\xspace}
\newcommand{\regnctx}{\textsc{Reg-NCTX}\xspace}
\newcommand{\regterm}{\textsc{Reg-Term}\xspace}
\newcommand{\LTtilde}{\wt{L}_T}
\newcommand{\Bomega}{B_{\Omega}}
\newcommand{\UOB}{UOB-REPS\xspace}
\newcommand{\Holder}{{H{\"o}lder}\xspace}
\newcommand{\dpg}{\dplus{g}}
\newcommand{\dpM}{\dplus{M}}
\newcommand{\dpf}{\dplus{f}}
\newcommand{\dpX}{\dplus{\calX}}
\newcommand{\dpw}{\dplus{w}}
\newcommand{\dpF}{\dplus{F}}
\newcommand{\dpu}{\dplus{u}}
\newcommand{\dpwtilde}{\dplus{\wtilde}}
\newcommand{\dps}{\dplus{s}}
\newcommand{\dpe}{\dplus{e}}
\newcommand{\dpx}{\dplus{x}}
\newcommand{\dpy}{\dplus{y}}
\newcommand{\dpH}{\dplus{H}}
\newcommand{\dpOmega}{\dplus{\Omega}}
\newcommand{\dpellhat}{\dplus{\ellhat}}
\newcommand{\dpell}{\dplus{\ell}}
\newcommand{\dpr}{\dplus{r}}
\newcommand{\dpxi}{\dplus{\xi}}
\newcommand{\dpv}{\dplus{v}}
\newcommand{\dpI}{\dplus{I}}
\newcommand{\dpA}{\dplus{A}}
\newcommand{\dph}{\dplus{h}}
\newcommand{\cprob}{6}
\newcommand{\sigmoid}{\ensuremath{\mathsf{Sigmoid}}\xspace}
\newcommand{\relu}{\ensuremath{\mathsf{ReLU}}\xspace}

\DeclareMathOperator*{\argmin}{argmin}
\DeclareMathOperator*{\argmax}{argmax}
\DeclareMathOperator*{\argsmax}{argsmax}
%\DeclareMathOperator*{\range}{range}
%\DeclareMathOperator*{\mydet}{det_{+}}
%\DeclarePairedDelimiter\abs{\lvert}{\rvert}
%\DeclarePairedDelimiter\bigabs{\big\lvert}{\big\rvert}
\DeclarePairedDelimiter\ceil{\lceil}{\rceil}
%\DeclarePairedDelimiter\floor{\lfloor}{\rfloor}
%\DeclarePairedDelimiter\bigceil{\big\lceil}{\big\rceil}
%\DeclarePairedDelimiter\bigfloor{\big\lfloor}{\big\rfloor}

\newcommand{\field}[1]{\mathbb{#1}}
\newcommand{\fY}{\field{Y}}
\newcommand{\fX}{\field{X}}
\newcommand{\fH}{\field{H}}
\newcommand{\fR}{\field{R}}
\newcommand{\fN}{\field{N}}
\newcommand{\fS}{\field{S}}
\newcommand{\UCB}{{\operatorname{UCB}}}
\newcommand{\LCB}{{\operatorname{LCB}}}
\newcommand{\testblock}{\textsc{EndofBlockTest}\xspace}
\newcommand{\testreplay}{\textsc{EndofReplayTest}\xspace}

\newcommand{\theset}[2]{ \left\{ {#1} \,:\, {#2} \right\} }
% \newcommand{\inner}[1]{ \langle {#1} \rangle }
\newcommand{\inn}[1]{ \langle {#1} \rangle }
\newcommand{\Ind}[1]{ \field{I}_{\{{#1}\}} }
\newcommand{\eye}[1]{ \boldsymbol{I}_{#1} }
\newcommand{\norm}[1]{\left\|{#1}\right\|}
%\newcommand{\trace}[1]{\text{tr}\left({#1}\right)}
\newcommand{\trace}[1]{\textsc{tr}({#1})}


\newcommand{\defeq}{\stackrel{\rm def}{=}}
\newcommand{\sgn}{\mbox{\sc sgn}}
\newcommand{\scI}{\mathcal{I}}
\newcommand{\scO}{\mathcal{O}}
\newcommand{\scN}{\mathcal{N}}
\newcommand{\msmwc}{\textsc{MsMwC}}

\newcommand{\dt}{\displaystyle}
\renewcommand{\ss}{\subseteq}
\newcommand{\wh}{\widehat}
\newcommand{\wt}{\widetilde}
\newcommand{\wb}{\overline}
\newcommand{\ve}{\varepsilon}
\newcommand{\hlambda}{\wh{\lambda}}

\newcommand{\Jd}{J}
\newcommand{\ellhat}{\wh{\ell}}
\newcommand{\rhat}{\wh{r}}
\newcommand{\elltilde}{\wt{\ell}}
\newcommand{\wtilde}{\wt{w}}
\newcommand{\what}{\wh{w}}

\DeclareMathOperator{\conv}{conv}
\newcommand{\ellprime}{\ellhat^\prime}

\newcommand{\upconf}{\phi}

%\newcommand{\Ltilde}{\wt{L}}

\newcommand{\hDelta}{\wh{\Delta}}
\newcommand{\hdelta}{\wh{\delta}}
\newcommand{\spin}{\{-1,+1\}}

\newcommand{\ep}[1]{\E\!\left[#1\right]}
\newcommand{\LT}{L_T}
\newcommand{\LTbar}{\overline{L}_T}
\newcommand{\LTbarep}{\mathring{L}_T}
\newcommand{\circxhat}{\mathring{\xh}}
\newcommand{\circx}{\mathring{x}}
\newcommand{\circu}{\mathring{u}}
\newcommand{\circcalX}{\mathring{\calX}}
\newcommand{\circg}{\mathring{g}}
\newcommand{\Lubar}{\overline{L}_{u}}
%\newcommand{\Lustarbar}{\overline{L}_{u^\star}}

\newcommand{\Lyr}{J}
\newcommand{\QQ}{{w}}
\newcommand{\Qt}{{\QQ_t}}
\newcommand{\Qstar}{{u}}
\newcommand{\Qpistar}{{\Qstar^{\star}}}
\newcommand{\Qhat}{\wh{\QQ}}
\newcommand{\Ut}{{\upconf_t}}
\newcommand{\intO}{\mathrm{int}(\Omega)}
\newcommand{\intK}{\mathrm{int}(K)}

\newcommand{\squareCB}{\ensuremath{\mathsf{SquareCB}}\xspace}
\newcommand{\feelgood}{\ensuremath{\mathsf{FGTS}}\xspace}
\newcommand{\graphCB}{\ensuremath{\mathsf{SquareCB.G}}\xspace}
\newcommand{\squareCBAuc}{\ensuremath{\mathsf{SquareCB.A}}\xspace}
\newcommand{\AlgSq}{\ensuremath{\mathsf{AlgSq}}\xspace}
\newcommand{\AlgLog}{\ensuremath{\mathsf{AlgLog}}\xspace}
\newcommand{\ips}{\ensuremath{\mathsf{(IPS)}}\xspace}
\newcommand{\optsq}{\ensuremath{\mathsf{(OptSq)}}\xspace}
\newcommand{\sq}{\ensuremath{\mathsf{(Sq)}}\xspace}
\newcommand{\dec}{\ensuremath{\mathsf{dec}_\gamma}\xspace}
\newcommand{\dectwo}{\ensuremath{\mathsf{dec}_{\gamma_1,\gamma_2}}\xspace}
%\newcommand{\theHalgorithm}{\arabic{algorithm}}
\newtheorem{cor}[theorem]{Corollary}
\newcommand{\context}{\text{ctx}}
\newcommand{\noncontext}{\text{n-ctx}}
%\newtheorem{remark}{Remark}
%\newtheorem{prop}{Proposition}
%\newtheorem{definition}{Definition}
%\newtheorem{assumption}{Assumption}
\newtheorem{event}{Event}
%\newtheorem*{main}{Main Result}
%\newtheorem{fact}[theorem]{Fact}

\newcommand{\paren}[1]{\left({#1}\right)}
\newcommand{\brackets}[1]{\left[{#1}\right]}
\newcommand{\braces}[1]{\left\{{#1}\right\}}

\newcommand{\normt}[1]{\norm{#1}_{t}}
\newcommand{\dualnormt}[1]{\norm{#1}_{t,*}}

\newcommand{\otil}{\ensuremath{\tilde{\mathcal{O}}}}

\newcommand{\dist}{\calP}

%%%%  brackets
\newcommand{\rbr}[1]{\left(#1\right)}
\newcommand{\sbr}[1]{\left[#1\right]}
\newcommand{\cbr}[1]{\left\{#1\right\}}
\newcommand{\nbr}[1]{\left\|#1\right\|}
\newcommand{\abr}[1]{\left|#1\right|}

\usepackage{lipsum,booktabs}
\usepackage{amsmath,mathrsfs,amssymb,amsfonts,bm,enumitem}
\usepackage{rotating}
\usepackage{pdflscape}
\usepackage{hyperref,url}
\hypersetup{
    colorlinks,
    breaklinks,
    linkcolor = blue,
    citecolor = blue,
    urlcolor  = blue,
}
\allowdisplaybreaks
\usepackage{appendix}
\usepackage{multirow,makecell}

%\usepackage{algorithmic,algorithm}
%\renewcommand{\algorithmicrequire}{ \textbf{Input:}}
%\renewcommand{\algorithmicensure}{ \textbf{Output:}}

\renewcommand{\tilde}{\widetilde}
\renewcommand{\hat}{\widehat}
\newcommand{\obs}{O}
\newcommand{\unobs}{E}
\newcommand{\unbiasSize}{c}
\newcommand{\unbias}{C}
\newcommand{\cnt}{k}

% define some macros
\def \A {\mathcal{A}}

\def \B {\mathbb{B}}
\def \B {\mathcal{B}}
\def \C {\mathcal{C}}
\def \D {\mathcal{D}}
\def \E {\mathbb{E}}
\def \F {\mathcal{F}}
\def \G {\mathcal{G}}
\def \H {\mathcal{H}}
\def \I {\mathcal{I}}
\def \J {\mathcal{J}}
\def \K {\mathcal{K}}
\def \L {\mathcal{L}}
\def \M {\mathcal{M}}
\def \N {\mathcal{N}}
\def \O {\mathcal{O}}
\def \P {\mathcal{P}}
\def \Q {\mathcal{Q}}
\def \R {\mathbb{R}}
\def \S {\mathcal{S}}
% \def \T {\mathrm{T}}
\def \T {\top}
\def \U {\mathcal{U}}
\def \V {\mathcal{V}}
\def \W {\mathcal{W}}
\def \X {\mathcal{X}}
\def \Y {\mathcal{Y}}
\def \Z {\mathcal{Z}}

\def \a {\mathbf{a}}
\def \b {\mathbf{b}}
\def \c {\mathbf{c}}
\def \d {\mathbf{d}}
\def \e {\mathbf{e}}
\def \f {\mathbf{f}}
\def \g {\mathbf{g}}
\def \h {\mathbf{h}}
\def \m {\mathbf{m}}
\def \p {\mathbf{p}}
\def \q {\mathbf{q}}
\def \u {\mathbf{u}}
\def \w {\mathbf{w}}
\def \s {\mathbf{s}}
\def \t {\mathbf{t}}
\def \v {\mathbf{v}}
\def \x {\mathbf{x}}
\def \y {y}
\def \z {\mathbf{z}}

\def \ph {\hat{p}}

\def \fh {\hat{f}}
\def \fb {\bar{f}}
\def \ft{\tilde{f}}

\def \gh {\hat{\g}}
\def \gb {\bar{\g}}
\def \gt {\tilde{g}}

\def \uh {\hat{\u}}
\def \ub {\bar{\u}}
\def \ut{\tilde{\u}}

\def \vh {\hat{\v}}
\def \vb {\bar{\v}}
\def \vt{\tilde{\v}}

\def \xh {\hat{x}}
\def \xb {\bar{\x}}
\def \xt {\tilde{\x}}

\def \zh {\hat{\z}}
\def \zb {\bar{\z}}
\def \zt {\tilde{\z}}

\def \Ecal {\mathcal{E}}
\def \Rcal {\mathcal{R}}
\def \Ot {\tilde{\O}}
\def \indicator {\mathds{1}}
\def \regret {\mbox{Regret}}
\def \proj {\mbox{Proj}}
\def \Pr {\mathsf{Pr}}
\def \ellb {\boldsymbol{\ell}}
\def \thetah {\hat{\theta}}

\newcommand{\RegSq}{\ensuremath{\mathrm{\mathbf{Reg}}_{\mathsf{Sq}}}\xspace}
\newcommand{\RegCB}{\ensuremath{\mathrm{\mathbf{Reg}}_{\mathsf{CB}}}\xspace}
\newcommand{\RegDyn}{\ensuremath{\mathrm{\mathbf{Reg}}_{\mathsf{Dyn}}}\xspace}
\usepackage{mathtools}
\let\oldnorm\norm   % <-- Store original \norm as \oldnorm
\let\norm\undefined % <-- "Undefine" \norm
\DeclarePairedDelimiter\norm{\lVert}{\rVert}
\DeclarePairedDelimiter\abs{\lvert}{\rvert}
%\newcommand\inner[2]{\langle #1, #2 \rangle}
\newcommand*\diff{\mathop{}\!\mathrm{d}}
\newcommand*\Diff[1]{\mathop{}\!\mathrm{d^#1}}

%\DeclareMathOperator*{\Reg}{Regret}
\DeclareMathOperator*{\AReg}{A-Regret}
\DeclareMathOperator*{\WAReg}{WA-Regret}
\DeclareMathOperator*{\SAReg}{SA-Regret}
\DeclareMathOperator*{\DReg}{\mbox{D-Regret}}
\DeclareMathOperator*{\poly}{poly}
%\DeclareMathOperator*{\argmax}{arg\,max}
%\DeclareMathOperator*{\argmin}{arg\,min}

% define new theorem environments
% \let\proof\relax
% \let\endproof\relax
% \newenvironment{proof}{\par\noindent{\bf Proof\ }}{\hfill\BlackBox\\[2mm]}
% \renewcommand\qedsymbol{$\blacksquare$}
\newtheorem{myThm}{Theorem}
\newtheorem{myFact}{Fact}
\newtheorem{myClaim}{Claim}
\newtheorem{myLemma}[myThm]{Lemma}
\newtheorem{myObservation}{Observation}
\newtheorem{myProp}[myThm]{Proposition}
\newtheorem{myProperty}{Property}

% Define a custom environment for prompts
\newtcolorbox{promptbox}[1][]{
  colback=blue!5!white, colframe=blue!75!black,
  fonttitle=\bfseries, title=Prompt,
  left=2mm, right=2mm, top=2mm, bottom=2mm,
  boxrule=0.5mm,  % Thickness of the frame
  coltitle=black, % Color of the title text
  colbacktitle=blue!15!white, % Background color of the title
  breakable,      % Allows the box to break across pages
  #1
}
\newtheorem{myAssum}{Assumption}
\newtheorem{myConj}{Conjecture}
\newtheorem{myCor}{Corollary}
\newtheorem{myDef}{Definition}
\newtheorem{myExample}{Example}
\newtheorem{myNote}{Note}
\newtheorem{myProblem}{Problem}

\newtheorem{myRemark}{Remark}

% add comments
\usepackage{graphicx,color} % more modern
\newcommand{\red}{\color{red}}
\newcommand{\blue}{\color{blue}}
\definecolor{wine_red}{RGB}{228,48,64}
\definecolor{DSgray}{cmyk}{0,1,0,0}
%\newcommand{\Authornote}[2]{{\small\textcolor{NavyBlue}{\sf$<<<${  #1: #2 }$>>>$}}}
% \newcommand{\Authormarginnote}[2]{\marginpar{\parbox{2cm}{\raggedright\tiny \textcolor{DSgray}{#1: #2}}}}
% \newcommand{\pnote}[1]{{\Authornote{Peng}{#1}}}
% \newcommand{\pmarginnote}[1]{{\Authormarginnote{Peng}{#1}}}

\usepackage{prettyref}
\newcommand{\pref}[1]{\prettyref{#1}}
\newcommand{\pfref}[1]{Proof of \prettyref{#1}}
\newcommand{\savehyperref}[2]{\texorpdfstring{\hyperref[#1]{#2}}{#2}}
\newrefformat{eq}{\savehyperref{#1}{Eq. \textup{(\ref*{#1})}}}
\newrefformat{eqn}{\savehyperref{#1}{Eq.~(\ref*{#1})}}
\newrefformat{lem}{\savehyperref{#1}{Lemma~\ref*{#1}}}
\newrefformat{event}{\savehyperref{#1}{Event~\ref*{#1}}}
\newrefformat{def}{\savehyperref{#1}{Definition~\ref*{#1}}}
\newrefformat{line}{\savehyperref{#1}{Line~\ref*{#1}}}
\newrefformat{thm}{\savehyperref{#1}{Theorem~\ref*{#1}}}
\newrefformat{tab}{\savehyperref{#1}{Table~\ref*{#1}}}
\newrefformat{corr}{\savehyperref{#1}{Corollary~\ref*{#1}}}
\newrefformat{cor}{\savehyperref{#1}{Corollary~\ref*{#1}}}
\newrefformat{sec}{\savehyperref{#1}{Section~\ref*{#1}}}
\newrefformat{app}{\savehyperref{#1}{Appendix~\ref*{#1}}}
\newrefformat{assum}{\savehyperref{#1}{Assumption~\ref*{#1}}}
\newrefformat{asm}{\savehyperref{#1}{Assumption~\ref*{#1}}}
\newrefformat{ex}{\savehyperref{#1}{Example~\ref*{#1}}}
\newrefformat{fig}{\savehyperref{#1}{Figure~\ref*{#1}}}
\newrefformat{alg}{\savehyperref{#1}{Algorithm~\ref*{#1}}}
\newrefformat{rem}{\savehyperref{#1}{Remark~\ref*{#1}}}
\newrefformat{conj}{\savehyperref{#1}{Conjecture~\ref*{#1}}}
\newrefformat{prop}{\savehyperref{#1}{Proposition~\ref*{#1}}}
\newrefformat{proto}{\savehyperref{#1}{Protocol~\ref*{#1}}}
\newrefformat{prob}{\savehyperref{#1}{Problem~\ref*{#1}}}
\newrefformat{claim}{\savehyperref{#1}{Claim~\ref*{#1}}}
\newrefformat{que}{\savehyperref{#1}{Question~\ref*{#1}}}
\newrefformat{op}{\savehyperref{#1}{Open Problem~\ref*{#1}}}
\newrefformat{fn}{\savehyperref{#1}{Footnote~\ref*{#1}}}

\def \p {\boldsymbol{p}}
\def \s {\boldsymbol{s}}
\def \m {\boldsymbol{m}}
\def \epsilon {\varepsilon}

% \def \base {\mathtt{base}\mbox{-}\mathtt{regret}}
% \def \meta {\mathtt{meta}\mbox{-}\mathtt{regret}}
\def \base {\textsc{base-regret}}
\def \meta {\textsc{meta-regret}}
\def \xref {\x_{\text{ref}}}
\def \fb {\bar{f}}
\def \interior {\text{int}}
\def \yh {\hat{\y}}
\def \RegLog {\Reg_{\log}^G}
\newcommand{\bra}[1]{\left[#1\right]}
\newcommand{\pa}[1]{\left(#1\right)}
\newcommand{\hhat}{\wh{h}}
\newcommand{\epsn}{\epsilon_N}
\newcommand{\rad}{\mathsf{rad}}
\newcommand{\hatr}{\wh{r}}
\newcommand{\fl}{\underline{f}^\star}


% If the title and author information does not fit in the area allocated, uncomment the following
%
%\setlength\titlebox{<dim>}
%
% and set <dim> to something 5cm or larger.

\title{GSQ-Tuning: \underline{G}roup-\underline{S}hared Exponents Integer in \\ Fully \underline{Q}uantized Training for LLMs On-Device Fine-tuning}


\author{
 \textbf{Sifan Zhou\textsuperscript{1,2$\dagger$$\ddagger$}},
 \textbf{Shuo Wang\textsuperscript{2$\dagger$}},
 \textbf{Zhihang Yuan\textsuperscript{2$\dagger$}},
 \textbf{Mingjia Shi\textsuperscript{2}},
 \textbf{Yuzhang Shang\textsuperscript{3}},
 \textbf{Dawei Yang\textsuperscript{2}}
\\
\\
\\
 \textsuperscript{1}Southeast University,
 \textsuperscript{2}Houmo AI,
 \textsuperscript{3}Illinois Institute of Technology
\\
 % \small{
   % \textbf{Correspondence:} \href{mailto:email@domain}{email@domain}
 % }
}

\begin{document}
\maketitle
\def\thefootnote{$\dagger$}\footnotetext{Equal contribution}
\def\thefootnote{$\ddagger$}\footnotetext{Work down as an intern at Houmo AI}
% \def\thefootnote{$\Diamond$}\footnotetext{Corresponding author}
\begin{abstract}
Building a virtual cell capable of accurately simulating cellular behaviors in silico has long been a dream in computational biology. We introduce \emph{CellFlow}, an image-generative model that simulates cellular morphology changes induced by chemical and genetic perturbations using flow matching. Unlike prior methods, \emph{CellFlow} models distribution-wise transformations from unperturbed to perturbed cell states, effectively distinguishing actual perturbation effects from experimental artifacts such as batch effects—a major challenge in biological data. Evaluated on chemical (BBBC021), genetic (RxRx1), and combined perturbation (JUMP) datasets, \emph{CellFlow} generates biologically meaningful cell images that faithfully capture perturbation-specific morphological changes, achieving a 35\% improvement in FID scores and a 12\% increase in mode-of-action prediction accuracy over existing methods. Additionally, \emph{CellFlow} enables continuous interpolation between cellular states, providing a potential tool for studying perturbation dynamics. These capabilities mark a significant step toward realizing virtual cell modeling for biomedical research.
\end{abstract}

\section{Introduction}
Backdoor attacks pose a concealed yet profound security risk to machine learning (ML) models, for which the adversaries can inject a stealth backdoor into the model during training, enabling them to illicitly control the model's output upon encountering predefined inputs. These attacks can even occur without the knowledge of developers or end-users, thereby undermining the trust in ML systems. As ML becomes more deeply embedded in critical sectors like finance, healthcare, and autonomous driving \citep{he2016deep, liu2020computing, tournier2019mrtrix3, adjabi2020past}, the potential damage from backdoor attacks grows, underscoring the emergency for developing robust defense mechanisms against backdoor attacks.

To address the threat of backdoor attacks, researchers have developed a variety of strategies \cite{liu2018fine,wu2021adversarial,wang2019neural,zeng2022adversarial,zhu2023neural,Zhu_2023_ICCV, wei2024shared,wei2024d3}, aimed at purifying backdoors within victim models. These methods are designed to integrate with current deployment workflows seamlessly and have demonstrated significant success in mitigating the effects of backdoor triggers \cite{wubackdoorbench, wu2023defenses, wu2024backdoorbench,dunnett2024countering}.  However, most state-of-the-art (SOTA) backdoor purification methods operate under the assumption that a small clean dataset, often referred to as \textbf{auxiliary dataset}, is available for purification. Such an assumption poses practical challenges, especially in scenarios where data is scarce. To tackle this challenge, efforts have been made to reduce the size of the required auxiliary dataset~\cite{chai2022oneshot,li2023reconstructive, Zhu_2023_ICCV} and even explore dataset-free purification techniques~\cite{zheng2022data,hong2023revisiting,lin2024fusing}. Although these approaches offer some improvements, recent evaluations \cite{dunnett2024countering, wu2024backdoorbench} continue to highlight the importance of sufficient auxiliary data for achieving robust defenses against backdoor attacks.

While significant progress has been made in reducing the size of auxiliary datasets, an equally critical yet underexplored question remains: \emph{how does the nature of the auxiliary dataset affect purification effectiveness?} In  real-world  applications, auxiliary datasets can vary widely, encompassing in-distribution data, synthetic data, or external data from different sources. Understanding how each type of auxiliary dataset influences the purification effectiveness is vital for selecting or constructing the most suitable auxiliary dataset and the corresponding technique. For instance, when multiple datasets are available, understanding how different datasets contribute to purification can guide defenders in selecting or crafting the most appropriate dataset. Conversely, when only limited auxiliary data is accessible, knowing which purification technique works best under those constraints is critical. Therefore, there is an urgent need for a thorough investigation into the impact of auxiliary datasets on purification effectiveness to guide defenders in  enhancing the security of ML systems. 

In this paper, we systematically investigate the critical role of auxiliary datasets in backdoor purification, aiming to bridge the gap between idealized and practical purification scenarios.  Specifically, we first construct a diverse set of auxiliary datasets to emulate real-world conditions, as summarized in Table~\ref{overall}. These datasets include in-distribution data, synthetic data, and external data from other sources. Through an evaluation of SOTA backdoor purification methods across these datasets, we uncover several critical insights: \textbf{1)} In-distribution datasets, particularly those carefully filtered from the original training data of the victim model, effectively preserve the model’s utility for its intended tasks but may fall short in eliminating backdoors. \textbf{2)} Incorporating OOD datasets can help the model forget backdoors but also bring the risk of forgetting critical learned knowledge, significantly degrading its overall performance. Building on these findings, we propose Guided Input Calibration (GIC), a novel technique that enhances backdoor purification by adaptively transforming auxiliary data to better align with the victim model’s learned representations. By leveraging the victim model itself to guide this transformation, GIC optimizes the purification process, striking a balance between preserving model utility and mitigating backdoor threats. Extensive experiments demonstrate that GIC significantly improves the effectiveness of backdoor purification across diverse auxiliary datasets, providing a practical and robust defense solution.

Our main contributions are threefold:
\textbf{1) Impact analysis of auxiliary datasets:} We take the \textbf{first step}  in systematically investigating how different types of auxiliary datasets influence backdoor purification effectiveness. Our findings provide novel insights and serve as a foundation for future research on optimizing dataset selection and construction for enhanced backdoor defense.
%
\textbf{2) Compilation and evaluation of diverse auxiliary datasets:}  We have compiled and rigorously evaluated a diverse set of auxiliary datasets using SOTA purification methods, making our datasets and code publicly available to facilitate and support future research on practical backdoor defense strategies.
%
\textbf{3) Introduction of GIC:} We introduce GIC, the \textbf{first} dedicated solution designed to align auxiliary datasets with the model’s learned representations, significantly enhancing backdoor mitigation across various dataset types. Our approach sets a new benchmark for practical and effective backdoor defense.



\section{Study Design}
% robot: aliengo 
% We used the Unitree AlienGo quadruped robot. 
% See Appendix 1 in AlienGo Software Guide PDF
% Weight = 25kg, size (L,W,H) = (0.55, 0.35, 06) m when standing, (0.55, 0.35, 0.31) m when walking
% Handle is 0.4 m or 0.5 m. I'll need to check it to see which type it is.
We gathered input from primary stakeholders of the robot dog guide, divided into three subgroups: BVI individuals who have owned a dog guide, BVI individuals who were not dog guide owners, and sighted individuals with generally low degrees of familiarity with dog guides. While the main focus of this study was on the BVI participants, we elected to include survey responses from sighted participants given the importance of social acceptance of the robot by the general public, which could reflect upon the BVI users themselves and affect their interactions with the general population \cite{kayukawa2022perceive}. 

The need-finding processes consisted of two stages. During Stage 1, we conducted in-depth interviews with BVI participants, querying their experiences in using conventional assistive technologies and dog guides. During Stage 2, a large-scale survey was distributed to both BVI and sighted participants. 

This study was approved by the University’s Institutional Review Board (IRB), and all processes were conducted after obtaining the participants' consent.

\subsection{Stage 1: Interviews}
We recruited nine BVI participants (\textbf{Table}~\ref{tab:bvi-info}) for in-depth interviews, which lasted 45-90 minutes for current or former dog guide owners (DO) and 30-60 minutes for participants without dog guides (NDO). Group DO consisted of five participants, while Group NDO consisted of four participants.
% The interview participants were divided into two groups. Group DO (Dog guide Owner) consisted of five participants who were current or former dog guide owners and Group NDO (Non Dog guide Owner) consisted of three participants who were not dog guide owners. 
All participants were familiar with using white canes as a mobility aid. 

We recruited participants in both groups, DO and NDO, to gather data from those with substantial experience with dog guides, offering potentially more practical insights, and from those without prior experience, providing a perspective that may be less constrained and more open to novel approaches. 

We asked about the participants' overall impressions of a robot dog guide, expectations regarding its potential benefits and challenges compared to a conventional dog guide, their desired methods of giving commands and communicating with the robot dog guide, essential functionalities that the robot dog guide should offer, and their preferences for various aspects of the robot dog guide's form factors. 
For Group DO, we also included questions that asked about the participants' experiences with conventional dog guides. 

% We obtained permission to record the conversations for our records while simultaneously taking notes during the interviews. The interviews lasted 30-60 minutes for NDO participants and 45-90 minutes for DO participants. 

\subsection{Stage 2: Large-Scale Surveys} 
After gathering sufficient initial results from the interviews, we created an online survey for distributing to a larger pool of participants. The survey platform used was Qualtrics. 

\subsubsection{Survey Participants}
The survey had 100 participants divided into two primary groups. Group BVI consisted of 42 blind or visually impaired participants, and Group ST consisted of 58 sighted participants. \textbf{Table}~\ref{tab:survey-demographics} shows the demographic information of the survey participants. 

\subsubsection{Question Differentiation} 
Based on their responses to initial qualifying questions, survey participants were sorted into three subgroups: DO, NDO, and ST. Each participant was assigned one of three different versions of the survey. The surveys for BVI participants mirrored the interview categories (overall impressions, communication methods, functionalities, and form factors), but with a more quantitative approach rather than the open-ended questions used in interviews. The DO version included additional questions pertaining to their prior experience with dog guides. The ST version revolved around the participants' prior interactions with and feelings toward dog guides and dogs in general, their thoughts on a robot dog guide, and broad opinions on the aesthetic component of the robot's design. 

\section{Experiments}
\label{sec:exp}
\subsection{Experimental settings}

\noindent\textbf{Benchmark.}  We conduct experiments on two established 3D occupancy benchmarks: (i) nuScenes~\cite{nuScenes}, which provides instance-level annotations with manually labeled 3D bounding boxes (position/size/orientation) for dynamic objects, and (ii) Occ3D~\cite{Occ3D}, which generates voxel-level occupancy labels (0.4m resolution) through automated LiDAR point cloud aggregation and mesh reconstruction, including occlusion states. Both benchmarks share identical scene configurations of 1,050 driving scenes, each containing up to 40 timestamped frames. Every frame includes six synchronized camera views (front, front-left, front-right, back, back-left, back-right) at 1600$\times$900 resolution. In our experiments, we extend single-frame baselines~\cite{MonoScene,surroundOcc,viewformer} by aggregating features from $N$ historical keyframes. Additionally, we extract unlabeled intermediate frames from the ``sweeps'' folder~\cite{nuScenes} to provide implicit motion cues, enabling self-supervised temporal consistency learning.

\noindent\textbf{Implementation details.} For the nuScenes benchmark~\cite{nuScenes}, we follow the parameter settings of SurroundOcc~\cite{surroundOcc}, using $Cam=6$, $p=6$, $v=32$,$L=116$, and $W=200$. For the Occ3D benchmark~\cite{Occ3D}, we adopt ViewFormer's~\cite{viewformer} standard setup with $Cam=6$, $p=6$, $v=32$, $L=32$, and $W=88$. The output of the occupancy result on both benchmarks is formatted into a vector with dimensions $[200, 200, 16]$. In this vector, the first two dimensions (200 and 200) represent the length and width, while the third (16) indicates the height. The occupancy result covers a range from -50 meters to 50 meters in both width and length, and the vertical height varies from -5 meters to 3 meters. Each voxel corresponds to a cube measuring 0.5 meters on each side. Occupied voxels are categorized into one of 17~\cite{nuScenes,surroundOcc} and 18~\cite{Occ3D} semantic classes.
More details on implementation can be found in the supplementary material.

\subsection{Evaluation Metrics}

To validate the temporal consistency and occupancy accuracy of moving and static objects, objects are divided into two general classes~\cite{Cam4docc}: General Moving Objects (GMO) and General Static Objects (GSO). Detailed classification classes are introduced in the supplementary material.

\noindent\textbf{Occupancy Accuracy Metric.} To ensure rigorous evaluation across different benchmarks, we employ both Intersection over Union (IoU) and Mean Intersection over Union (mIoU) metrics. These metrics are widely adopted in 3D semantic occupancy prediction tasks~\cite{PASCAL, Microsoft_COCO, Cityscapes_dataset, Mask_R_CNN}. The mIoU are calculated separately for three category groups: All classes, GMO classes, and GSO classes.

\noindent\textbf{Temporal Consistency Metric.} \label{para:consistency_metric} To evaluate the effect achieved by integrating \ours\ with baseline models, we propose a temporal consistency metric. We aim to detect and measure changes in a scene from one frame to the next. This metric reflects the stability of prediction results, which directly impacts the user's visual experience. Let $\sigma_{i,n}^{(x,y,z)}$ denote the semantic label of the $n$-th voxel point (with coordinates $(x,y,z)$) in frame $i$, and define the indicator function $\delta(e_1,e_2) = \mathbb{I}(e_1 \neq e_2)$. 

In the occupancy results of frames $i$ and $j$, voxels at corresponding positions may undergo changes, which are categorized into two types: ``Static Object Change"~(SOC) and ``Moving Object Change"~(MOC). The definitions of these changes are as table \ref{tab:moc-soc}.

\begin{wrapfigure}[7]{l}{80mm}
\centering
% \setlength{\tabcolsep}{8pt}
\captionsetup{type=table}
    \begin{tabular}{c|c}
    \toprule
    \textbf{Type} & \textbf{Condition} \\  
    \midrule
    MOC & $\sigma_{i,n}^{(x,y,z)} \in \text{GMO} \lor \sigma_{j,n}^{(x,y,z)} \in \text{GMO}$ \\
    % \midrule
    SOC & $\sigma_{i,n}^{(x,y,z)} \wedge \sigma_{j,n}^{(x,y,z)} \in \text{GSO}$ \\  
    \bottomrule
    \end{tabular}
    \vspace{8pt}
    \caption{Definition of MOC and SOC. $N_{mc}$/$N_{sc}$ denote the number of MOC/SOC voxels, respectively.}
    \label{tab:moc-soc}
    % \vspace{-10pt}
\end{wrapfigure}

Based on these definitions, we can define disparity metrics~($\Delta_{m}$/$\Delta_{s}$) to quantify temporal inconsistencies across frames~($i$ and $j$). The process is defined as:
\begin{equation}
\begin{dcases}
\Delta_{m}(i,j) = \dfrac{1}{N_{mc}} \sum\limits_{n=1}^{N_{mc}} \delta\left(\sigma_{i,n}^{(x,y,z)}, \sigma_{j,n}^{(x,y,z)}\right) \\
\Delta_{s}(i,j) = \dfrac{1}{N_{sc}} \sum\limits_{n=1}^{N_{sc}} \delta\left(\sigma_{i,n}^{(x,y,z)}, \sigma_{j,n}^{(x,y,z)}\right).
\end{dcases}
\label{eq:disparity}
\end{equation}

The temporal consistency metrics -- $S_m$ (moving) and $S_s$ (static) -- are derived through aggregation of $\Delta_{m}$ and $\Delta_{s}$ across sequential frames. Formally, we have:
\begin{equation}
S_{m/s} = 1 - \dfrac{1}{M-1} \sum\limits_{k=1}^{M-1} \Delta_{m/s}(k,k+1),
\label{eq:consistency_scores}
\end{equation}
where $M$ is the scene's total frame count. Final metrics $\overline{S_m}$/$\overline{S_s}$ average across all scenes. A higher temporal consistency score indicates that the predictions within the scene are smoother and more consistent over time.

\begin{table}[t]
    % \centering
    \footnotesize 
    \begin{minipage}[t]{0.48\textwidth}
        \centering
        % \captionsetup{justification=centering, singlelinecheck=false}
        \setlength{\tabcolsep}{2pt}
        \renewcommand{\arraystretch}{1.25}
        \begin{tabular}{r|cccc|cc}
            \toprule
            \multicolumn{1}{r|}{\multirow{2}{*}[-0.4em]{Method}} & \multicolumn{1}{c|}{\multirow{2}{*}[-0.4em]{IoU~$\uparrow$}} & \multicolumn{3}{c|}{mIoU~$\uparrow$} & \multicolumn{1}{c}{\multirow{2}{*}[-0.4em]{$\overline{S_m}\uparrow$}} & \multicolumn{1}{c}{\multirow{2}{*}[-0.4em]{$\overline{S_s}\uparrow$}} \\ \cmidrule(lr){3-5}
            \multicolumn{1}{c|}{} & \multicolumn{1}{c|}{} & All & GMO & GSO & \multicolumn{1}{c}{} & \multicolumn{1}{c}{} \\ 
            \midrule        
            Atlas~\cite{Atlas} & 28.66 & 15.00 & 12.64 & 17.35 & \text{--} & \text{--}  \\
            BEVFormer{~\cite{BEVFormer}} & 30.50 & 16.75 & 14.17 & 19.33 & \text{--} & \text{--}  \\
            TPVFormer~\cite{TPVFormer} & 30.86 & 17.10 & 14.04 & 20.15 & \text{--} & \text{--} \\
            BEVDet4D-Occ~\cite{bevdet4d} & 24.26 & 14.22 & 11.10 & 17.34 & \text{--} & \text{--} \\
            MonoScene~\cite{MonoScene} & 10.04 & 1.15 & 0.24 & 2.07 & 46.53 & 81.77 \\
            % Cam4DOcc~\cite{Cam4docc} & 23.92 & 7.12 & 4.71 & 10.17 & 60.34 & 91.15 \\
            SurroundOcc~\cite{surroundOcc} & 31.49 & 20.30  & \cellcolor{gray!20}18.39 & 22.20 & 58.33 & 91.71 \\
            \midrule
            \makecell[r]{MonoScene \\ \textbf{+\ours}} & \makecell{13.10\\\textbf{+3.06}} & \makecell{1.69\\\textbf{+0.54}}  & \makecell{0.34\\\textbf{+0.10}} & \makecell{3.04\\\textbf{+0.98}} & \makecell{54.21\\\textbf{+7.68}} & \makecell{83.84\\\textbf{+2.07}} \\
            \midrule
            \makecell[r]{SurroundOcc \\ \textbf{+\ours}} & \cellcolor{gray!20}\makecell{33.12 \\ \textbf{+1.63}} & \cellcolor{gray!20}\makecell{20.67\\\textbf{+0.37}}  & \makecell{18.26\\-0.13} & \cellcolor{gray!20}\makecell{23.08\\\textbf{+0.88}} & \cellcolor{gray!20}\makecell{60.64\\\textbf{+2.31}} & \cellcolor{gray!20}\makecell{92.54\\\textbf{+0.83}} \\
            \bottomrule
        \end{tabular}
        \vspace{2mm}
        \caption{Occupancy prediction accuracy on \textbf{nuScenes benchmark~\cite{nuScenes}}. For a fair comparison, we ensure that all models have uniform input data. The best performance is highlighted in gray.}
        \label{tab:main-res-a}
    \end{minipage}\hfill
    \begin{minipage}[t]{0.48\textwidth}
        \centering
        % \captionsetup{justification=centering, singlelinecheck=false}
        \setlength{\tabcolsep}{2pt}
        \begin{tabular}{r|cccc|cc}
            \toprule
            \multicolumn{1}{r|}{\multirow{2}{*}[-0.4em]{Method}} & \multicolumn{1}{c|}{\multirow{2}{*}[-0.4em]{IoU~$\uparrow$}} & \multicolumn{3}{c|}{mIoU~$\uparrow$} & \multicolumn{1}{c}{\multirow{2}{*}[-0.4em]{$\overline{S_m}\uparrow$}} & \multicolumn{1}{c}{\multirow{2}{*}[-0.4em]{$\overline{S_s}\uparrow$}} \\ \cmidrule(lr){3-5}
            \multicolumn{1}{c|}{} & \multicolumn{1}{c|}{} & All & GMO & GSO & \multicolumn{1}{c}{} & \multicolumn{1}{c}{} \\ 
            \midrule    
            MonoScene~\cite{MonoScene} & \text{--} & 6.06 & 5.36 & 6.68 & \text{--} & \text{--} \\
            OccFormer~\cite{OccFormer} & \text{--} & 21.93 & 21.78 & 22.06 & \text{--} & \text{--} \\
            % CTF-Occ~\cite{Occ3D} & 28.53 & 27.42 & 29.52 & \text{--} & \text{--} \\
            FB-OCC~\cite{fb_occ} & \text{--} & 39.11  & 33.74 & 43.88 & \text{--} & \text{--} \\
            SparseOcc~\cite{SparseOcc_Liu} & \text{--} & 30.10  & \text{--} & \text{--} & \text{--} & \text{--} \\
            BEVDet4D-Occ~\cite{bevdet4d} & \text{--} & 39.30  & 29.09 & 42.16 & \text{--} & \text{--} \\ 
            OPUS-L~\cite{opus} & \text{--} & 36.20  & 31.25 & 40.44 & \text{--} & \text{--} \\      
            SurroundOcc~\cite{surroundOcc} & 51.89 & 7.24  & 0.36 & 13.35 & 65.35 & 89.54 \\
            ViewFormer~\cite{viewformer} & 70.39 & 40.46  & 33.73 & 46.45 & 67.26 & 86.06 \\
            \midrule
            \makecell[r]{SurroundOcc \\ \textbf{+\ours}} & \makecell{52.13\\\textbf{+0.24}}& \makecell{10.33\\\textbf{+3.09}}  & \makecell{1.98\\\textbf{+1.62}} & \makecell{17.76\\\textbf{+4.41}} & \makecell{69.60\\\textbf{+4.25}} & \cellcolor{gray!20}\makecell{90.91\\\textbf{+1.37}} \\
            \midrule
            \makecell[r]{ViewFormer \\ \textbf{+\ours}} & \cellcolor{gray!20}\makecell{70.63\\\textbf{+0.24}} & \cellcolor{gray!20}\makecell{41.30\\\textbf{+0.84}}  & \cellcolor{gray!20}\makecell{34.33\\\textbf{+0.60}} & \cellcolor{gray!20}\makecell{47.50\\\textbf{+1.05}} & \cellcolor{gray!20}\makecell{70.13\\\textbf{+2.87}} & \makecell{87.10\\\textbf{+1.04}} \\
            \bottomrule
        \end{tabular}
        \vspace{2mm}
        \caption{Occupancy prediction accuracy on \textbf{Occ3D benchmark~\cite{Occ3D}}. For a fair comparison, we ensure that all models have uniform input data. The best performance is highlighted in gray.}
        \label{tab:main-res-b}
    \end{minipage}
    \vspace{-3mm}
    % \caption{Occupancy prediction accuracy on two benchmarks. The best performance is highlighted in gray.}
    \label{tab:main-res}
    \vspace{-5mm}
\end{table}

\subsection{Comparison Results}

\noindent\textbf{Occupancy accuracy on nuScenes.} We compare our method against several SOTA models, including Atlas~\cite{Atlas}, BEVFormer~\cite{BEVFormer}, TPVFormer~\cite{TPVFormer}, MonoScene~\cite{MonoScene}, and SurroundOcc~\cite{surroundOcc}. For a fair comparison, all methods are trained on the same ground truth and follow the same training procedure. By combining methods such as MonoScene~\cite{MonoScene} and SurroundOcc~\cite{surroundOcc} with \ours, we evaluate the effect of \ours\ in performance enhancement. The results presented in \cref{tab:main-res-a} show that our performance improvement is significant. Notably, the incorporation of \ours\ into SurroundOcc~\cite{surroundOcc} has led to improved metrics that surpass those of all other models listed in this table. The results are improved by 1.63\% and 0.37\% compared with SurroundOcc~\cite{surroundOcc} in IoU and mIoU~(All), respectively.

\noindent\textbf{Occupancy accuracy on Occ3D.} We also conduct experiments on Occ3D~\cite{Occ3D} in \cref{tab:main-res-b}. To validate \ours, we conducted two sets of experiments: First, integrating \ours\ with the 3D VONs~\cite{surroundOcc,viewformer} improved one of the original models'~\cite{viewformer} performance by 0.24\% in IoU and 0.84\% in mIoU. Second, \ours\ consistently outperforms existing history-aware VONs~\cite{opus,bevdet,SparseOcc_Liu,fb_occ} by over 2\% mIoU, demonstrating the efficacy of the \ours.


\noindent\textbf{Temporal Consistency.} The results of $\overline{S_m}$ and $\overline{S_s}$ shown in \cref{tab:main-res} indicate that the integration of \ours\ improved the temporal consistency of occupancy across all frames in all scenes for all models, demonstrating \ours' effectiveness. This enhancement can be attributed to the incorporation of previous keyframes from the dataset~\cite{nuScenes,Occ3D}, along with the addition of intermediate frames from the ``sweeps''~\cite{nuScenes} directory for the SFE and MFE modules. These elements provide critical historical information and motion clues for the model.

\subsection{Ablation study}

Our ablation experiments are all conducted on the nuScenes benchmark~\cite{nuScenes}. The results are presented in~\cref{tab:ablation-studies}.

\begin{wrapfigure}{l}{90mm}
\centering
\captionsetup{type=table}
    \begin{subtable}[t]{0.48\textwidth}
    \centering
    \footnotesize
    \begin{tabular}{c|ccc|cccc}
    \toprule 
    Idx. & Pre & Cur & Mid & IoU$\uparrow$ & mIoU$\uparrow$ & $\overline{S_m}\uparrow$ & $\overline{S_s}\uparrow$ \\
    \midrule
    \textbf{M0} & \ding{55} & \ding{55} & \ding{55} & 31.49 & 20.30 & 58.33 & 91.71 \\
    \textbf{M1} & \ding{55} & \ding{51} & \ding{51} & 33.04 & 20.04 & 60.59 & 92.25 \\
    \textbf{M2} & \ding{51} & \ding{55} & \ding{51} & 33.05 & 19.98 & 60.09 & 92.44 \\
    \textbf{M3} & \ding{51} & \ding{51} & \ding{55} & 32.88 & 20.10 & 60.24 & 92.24 \\
    \textbf{M4} & \ding{51} & \ding{51} & \ding{51} & 31.97 & 20.11 & 60.19 & 92.01 \\
    \textbf{M5} & \ding{51} & \ding{51} & \ding{51} & \textbf{33.12} & \textbf{20.67} & \textbf{60.64} & \textbf{92.54} \\
    \bottomrule
    \end{tabular}
    \vspace{1mm}
    \caption{Ablation study of \ours. \textbf{Cur}, \textbf{Pre} and \textbf{Mid} represent the $f_{cur}$, $f_{pre}$ and $f_{mid}$ input, respectively, in the MSI.}
    \label{tab:ablation-modules}
    \end{subtable}
% \hfill
    \vspace{2mm}
    
    \begin{subtable}[t]{0.48\textwidth}
    \centering
    \footnotesize
    \setlength{\tabcolsep}{7pt}
    \begin{tabular}{c|c|cccc}
    \toprule 
    Idx. & \makecell{Type of\\motion info.} & IoU~$\uparrow$ & mIoU~$\uparrow$ & $\overline{S_m}\uparrow$ & $\overline{S_s}\uparrow$ \\
    \midrule
    \textbf{I0} & -  & 31.49 & 20.30 & 58.33 & 91.71 \\
    \textbf{I1} & Raw Image  & 32.39 & 19.45 & 59.01 & 91.15 \\
    \textbf{I2} & Optical Flow  & 32.80 & 20.27 & 60.53 & 92.13 \\
    \textbf{I3} & Frame Diff.  & \textbf{33.12} & \textbf{20.67}  & \textbf{60.64} & \textbf{92.54} \\
    \bottomrule
    \end{tabular}
    \vspace{1mm}
    \caption{Effect of different types of motion information.}
    \label{tab:ablation-motion}
    \end{subtable}
    % \vspace{-2mm}
\caption{Ablation studies on \ours\ modules and motion information. Best results are \textbf{bolded}.}
\label{tab:ablation-studies}
\end{wrapfigure}

\noindent\textbf{Different combinations of \ours.} \label{para:aba-comb}\cref{tab:ablation-modules} presents the performance results of different combination of \ours's components for $N$=$1$. In \cref{tab:ablation-modules}, there are 6 different combinations: \textbf{M0} shows results from SurroundOcc~\cite{surroundOcc}, which represents the basic model without our method. \textbf{M1} means the model variant in which the part responsible for processing previous keyframes is removed, thereby excluding the input data \(F_{pre}^{1}\). \textbf{M2} refers to the model variant that omits the current feature \(F_{cur}\). \textbf{M3} indicates the model configuration that has \(F_{mid}^{1}\) removed. \textbf{M4} indicates that $I_{cur}$ is used to compute MHAM's query, while $I_{pre}$ and $I_{mid}$ are utilized to compute MHAM's key and value, which differs from the standard design. \textbf{M5} represents the full model with all components included.
% 验证了我们三种输入的必要性
\cref{tab:ablation-modules} clearly demonstrates that the removal of any single input from \ours\ module significantly reduces performance both in prediction accuracy and in temporal consistency. This validates the necessity of the three inputs. Furthermore, the comparison between \textbf{M4} and \textbf{M5} confirms that the cues provided by the previous keyframes and the intermediate frames are crucial for occupancy prediction.

\noindent\textbf{Impact of different types of motion information.} \label{para:motion-extracting} This experiment was conducted on the MFE module to investigate the effects of various types of motion information for $N=1$. The results are presented in \cref{tab:ablation-motion}. Specifically, \textbf{I0} served as the base model~\cite{surroundOcc} without using any motion information. \textbf{I1} employed raw intermediate frames as the input for the MFE. \textbf{I2} used optical flow~\cite{OpticalFlow} as the motion information input. \textbf{I3} used frame difference~\cite{Frame_difference} to capture motion information. It is clear that \textbf{I1} surpasses \textbf{I0} in terms of IoU metrics; however, it exhibits the lowest performance in mIoU, $\overline{S_m}$, and $\overline{S_s}$ metrics compared with \textbf{I1}, \textbf{I2}, and \textbf{I3}. This discrepancy is mainly because of the substantial amount of irrelevant information in the raw, intermediate frames, which complicates the extraction of motion features by the MFE. In addition, the results show that \textbf{I3} significantly outperforms \textbf{I2} in both IoU and mIoU metrics and slightly improves in $\overline{S_m}$ and $\overline{S_s}$ metrics. This indicates that frame difference more effectively captures sudden changes in a scene, such as the abrupt appearance of pedestrians or vehicles exiting intersections, while optical flow may experience delays in processing these sudden events. Furthermore, given the lightweight design of \ours, the frame difference method~\cite{Frame_difference} reduces data processing complexity by only processing simple differential data, thereby contributing to computing speed.

\begin{wrapfigure}[22]{l}{90mm}
    \centering
    \includegraphics[width=90mm]{assets/case_single_light.png}
    \caption{
    Several challenging scenarios are presented: pedestrians are partially occluded by vehicles in the first and second columns, and the road boundary appears visually obscure in the third column. Ours achieves more accurate predictions, while SOTA methods display significant artifacts.
    }
    \label{fig:case-extra}
\end{wrapfigure}

\noindent\textbf{Impact of different numbers of previous keyframes.}\label{para:ntrack} We conduct ablation experiments on $N$ to explore the performance of the model when $N$=$0$, $N$=$1$ and $N$=$2$. $N$=$0$ represents SurroundOcc~\cite{surroundOcc}, which does not use any previous keyframes. Detailed experiment results are documented in the supplementary material.

\begin{figure}[!t]
\centering
% \fbox{\rule{0pt}{2in} \rule{0.9\linewidth}{0pt}}
\includegraphics[width=\linewidth]{assets/case_big_light.png}
% \vspace{-7mm}
\caption{
Comparison under a T-junction scenario, where a pedestrian is partially and dynamically occluded in certain frames. Ours showcases robust predictions, with the pedestrian being consistently tracked, while SOTA methods show a flickering phenomenon.
}
\label{fig:case-study-big}
\vspace{-20pt}
\end{figure}

\subsection{Case analysis}
To visually evaluate the effectiveness of our method~(SurroundOcc+\ours), we compare it with the SOTA 3D VONs~\cite{surroundOcc} and the SOTA history-aware VONs~\cite{bevdet4d}.

\noindent\textbf{Temporal visualization case.} As shown in~\cref{fig:case-study-big} (Scene 277, Frames \#7-\#11), a pedestrian traversing the sidewalk parallel to the ego-motion trajectory is intermittently occluded by roadside vegetation. SurroundOcc~\cite{surroundOcc} exhibits severe instability in predictions (missing in Frames \#7/\#9), revealing fundamental limitations in temporal modeling. BEVDet4D-Occ~\cite{bevdet4d} alleviates this issue through data fusion but still suffers from occasional inconsistencies, such as detection dropout in Frame \#8. In contrast, our method completely eliminates flickering artifacts and maintains consistent detection across all occlusion states.

\noindent\textbf{Extra single frame visualization case.} ~\cref{fig:case-extra} highlights challenging scenarios: 
(i) Vehicle-pedestrian occlusion (Scene-0911 Frame \#15, Scene-0928 Frame \#14): Both SurroundOcc~\cite{surroundOcc} and BEVDet4D-Occ~\cite{bevdet4d} fail to recover the occluded pedestrian’s occupancy, while our method successfully localizes the target with precise geometry.
(ii) Curved road prediction (Scene-0923 Frame \#28): Our approach correctly anticipates the right-turn road geometry where baselines produce fragmented or erroneous occupancy, achieving superior shape consistency with real-world conditions.




\subsection{Overhead analysis}

For a fair comparison, all overhead analysis experiments are performed on a single NVIDIA L20 GPU.

\begin{figure}[htbp]
    \centering
    \begin{minipage}{0.48\textwidth}
        \footnotesize
        \begin{tabular}{r|ccc}
            \toprule
            Model & mIoU~$\uparrow$ & \makecell{Memory (MB)\\ Train~/~Test}~$\downarrow$ & Latency~$\downarrow$ \\
            \midrule
            FB-Occ~\cite{fb_occ} & 39.11 & 32,915~/~5,933 & 0.09s \\
            % SparseOcc~\cite{SparseOcc_Liu} & 30.10 & $>$49,140~/~7,147 & \cellcolor{Gray} 0.05s \\
            OPUS-L~\cite{opus} & 36.20 & OOM~/~10,579 & 0.16s \\
            OPUS-T~\cite{opus} & 33.20 & 48,532~/~6,711 & \textbf{0.03s} \\
            BEVDet4D-Occ~\cite{bevdet4d} & 39.30 & 22,833~/~4,689 & 0.26s \\
            \midrule
            ViewFormer+Ours & \textbf{41.30} & \textbf{16,619~/~4,687} & 0.12s \\
            \bottomrule
        \end{tabular}
        \vspace{2mm}
        \caption{Comparison of computational overhead. All models are benchmarked with ResNet-50 backbones. Our result (ViewFormer+\ours) in this table is measured for $N = 1$. OOM indicates out of CUDA memory. Best results are \textbf{bolded}.}
        \label{tab:efficiency}
    \end{minipage}\hfill
    \begin{minipage}{0.48\textwidth}
        \centering
        \includegraphics[width=\linewidth]{assets/bubble_0304.png}  % 替换成你的图片
        % \vspace{-8mm}
        \caption{Comparison of memory and latency overheads. Lower-left positions indicate superior performance with reduced memory consumption and faster inference. Large circles indicate better mIoU quality.}
        \label{fig:bubble}
    \end{minipage}
\end{figure}



As illustrated in \cref{tab:efficiency} and \cref{fig:bubble}, we conducted a comparative study to evaluate the computational overhead of our model against existing temporal methods~\cite{bevdet4d,opus,fb_occ}. The analysis focuses on GPU memory consumption during the training/testing phases and per-sample inference latency. The result shows that our method establishes an optimal accuracy-memory balance, achieving state-of-the-art mIoU while maintaining minimal GPU memory consumption alongside sustained computational efficiency that avoids runtime bottlenecks. For quantitative benchmarking, we compare two baseline frameworks:
\begin{itemize}
    \item ViewFormer on Occ3D: (i) Training memory: ViewFormer+\ours\ requires 16 GB of GPU memory, with the \ours\ module consuming only 0.22 GB, accounting for \textbf{1.4\%} of total usage; (ii) Inference latency: Full sample processing takes 0.1218s, where \ours\ contributes merely 0.0043s, accounting for \textbf{3.5\%} of total computation.
    \item SurroundOcc on nuScenes: (i) Training memory: SurroundOcc+\ours\ consumes 39 GB of GPU memory, with \ours\ occupying only 0.69 GB, which is \textbf{1.8\%} of total memory; (ii) Inference latency: Complete sample inference requires 0.9200s, while \ours\ takes 0.0065s, contributing to \textbf{0.7\%} of total latency.
\end{itemize}

These measurements confirm that our architecture introduces negligible computational overhead while delivering competitive performance.
\section{Related Work}
\label{sec:related-works}
\subsection{Novel View Synthesis}
Novel view synthesis is a foundational task in the computer vision and graphics, which aims to generate unseen views of a scene from a given set of images.
% Many methods have been designed to solve this problem by posing it as 3D geometry based rendering, where point clouds~\cite{point_differentiable,point_nfs}, mesh~\cite{worldsheet,FVS,SVS}, planes~\cite{automatci_photo_pop_up,tour_into_the_picture} and multi-plane images~\cite{MINE,single_view_mpi,stereo_magnification}, \etal
Numerous methods have been developed to address this problem by approaching it as 3D geometry-based rendering, such as using meshes~\cite{worldsheet,FVS,SVS}, MPI~\cite{MINE,single_view_mpi,stereo_magnification}, point clouds~\cite{point_differentiable,point_nfs}, etc.
% planes~\cite{automatci_photo_pop_up,tour_into_the_picture}, 


\begin{figure*}[!t]
    \centering
    \includegraphics[width=1.0\linewidth]{figures/overview-v7.png}
    %\caption{\textbf{Overview.} Given a set of images, our method obtains both camera intrinsics and extrinsics, as well as a 3DGS model. First, we obtain the initial camera parameters, global track points from image correspondences and monodepth with reprojection loss. Then we incorporate the global track information and select Gaussian kernels associated with track points. We jointly optimize the parameters $K$, $T_{cw}$, 3DGS through multi-view geometric consistency $L_{t2d}$, $L_{t3d}$, $L_{scale}$ and photometric consistency $L_1$, $L_{D-SSIM}$.}
    \caption{\textbf{Overview.} Given a set of images, our method obtains both camera intrinsics and extrinsics, as well as a 3DGS model. During the initialization, we extract the global tracks, and initialize camera parameters and Gaussians from image correspondences and monodepth with reprojection loss. We determine Gaussian kernels with recovered 3D track points, and then jointly optimize the parameters $K$, $T_{cw}$, 3DGS through the proposed global track constraints (i.e., $L_{t2d}$, $L_{t3d}$, and $L_{scale}$) and original photometric losses (i.e., $L_1$ and $L_{D-SSIM}$).}
    \label{fig:overview}
\end{figure*}

Recently, Neural Radiance Fields (NeRF)~\cite{2020NeRF} provide a novel solution to this problem by representing scenes as implicit radiance fields using neural networks, achieving photo-realistic rendering quality. Although having some works in improving efficiency~\cite{instant_nerf2022, lin2022enerf}, the time-consuming training and rendering still limit its practicality.
Alternatively, 3D Gaussian Splatting (3DGS)~\cite{3DGS2023} models the scene as explicit Gaussian kernels, with differentiable splatting for rendering. Its improved real-time rendering performance, lower storage and efficiency, quickly attract more attentions.
% Different from NeRF-based methods which need MLPs to model the scene and huge computational cost for rendering, 3DGS has stronger real-time performance, higher storage and computational efficiency, benefits from its explicit representation and gradient backpropagation.

\subsection{Optimizing Camera Poses in NeRFs and 3DGS}
Although NeRF and 3DGS can provide impressive scene representation, these methods all need accurate camera parameters (both intrinsic and extrinsic) as additional inputs, which are mostly obtained by COLMAP~\cite{colmap2016}.
% This strong reliance on COLMAP significantly limits their use in real-world applications, so optimizing the camera parameters during the scene training becomes crucial.
When the prior is inaccurate or unknown, accurately estimating camera parameters and scene representations becomes crucial.

% In early works, only photometric constraints are used for scene training and camera pose estimation. 
% iNeRF~\cite{iNerf2021} optimizes the camera poses based on a pre-trained NeRF model.
% NeRFmm~\cite{wang2021nerfmm} introduce a joint optimization process, which estimates the camera poses and trains NeRF model jointly.
% BARF~\cite{barf2021} and GARF~\cite{2022GARF} provide new positional encoding strategy to handle with the gradient inconsistency issue of positional embedding and yield promising results.
% However, they achieve satisfactory optimization results when only the pose initialization is quite closed to the ground-truth, as the photometric constrains can only improve the quality of camera estimation within a small range.
% Later, more prior information of geometry and correspondence, \ie monocular depth and feature matching, are introduced into joint optimisation to enhance the capability of camera poses estimation.
% SC-NeRF~\cite{SCNeRF2021} minimizes a projected ray distance loss based on correspondence of adjacent frames.
% NoPe-NeRF~\cite{bian2022nopenerf} chooses monocular depth maps as geometric priors, and defines undistorted depth loss and relative pose constraints for joint optimization.
In earlier studies, scene training and camera pose estimation relied solely on photometric constraints. iNeRF~\cite{iNerf2021} refines the camera poses using a pre-trained NeRF model. NeRFmm~\cite{wang2021nerfmm} introduces a joint optimization approach that simultaneously estimates camera poses and trains the NeRF model. BARF~\cite{barf2021} and GARF~\cite{2022GARF} propose a new positional encoding strategy to address the gradient inconsistency issues in positional embedding, achieving promising results. However, these methods only yield satisfactory optimization when the initial pose is very close to the ground truth, as photometric constraints alone can only enhance camera estimation quality within a limited range. Subsequently, 
% additional prior information on geometry and correspondence, such as monocular depth and feature matching, has been incorporated into joint optimization to improve the accuracy of camera pose estimation. 
SC-NeRF~\cite{SCNeRF2021} minimizes a projected ray distance loss based on correspondence between adjacent frames. NoPe-NeRF~\cite{bian2022nopenerf} utilizes monocular depth maps as geometric priors and defines undistorted depth loss and relative pose constraints.

% With regard to 3D Gaussian Splatting, CF-3DGS~\cite{CF-3DGS-2024} also leverages mono-depth information to constrain the optimization of local 3DGS for relative pose estimation and later learn a global 3DGS progressively in a sequential manner.
% InstantSplat~\cite{fan2024instantsplat} focus on sparse view scenes, first use DUSt3R~\cite{dust3r2024cvpr} to generate a set of densely covered and pixel-aligned points for 3D Gaussian initialization, then introduce a parallel grid partitioning strategy in joint optimization to speed up.
% % Jiang et al.~\cite{Jiang_2024sig} proposed to build the scene continuously and progressively, to next unregistered frame, they use registration and adjustment to adjust the previous registered camera poses and align unregistered monocular depths, later refine the joint model by matching detected correspondences in screen-space coordinates.
% \gjh{Jiang et al.~\cite{Jiang_2024sig} also implemented an incremental approach for reconstructing camera poses and scenes. Initially, they perform feature matching between the current image and the image rendered by a differentiable surface renderer. They then construct matching point errors, depth errors, and photometric errors to achieve the registration and adjustment of the current image. Finally, based on the depth map, the pixels of the current image are projected as new 3D Gaussians. However, this method still exhibits limitations when dealing with complex scenes and unordered images.}
% % CG-3DGS~\cite{sun2024correspondenceguidedsfmfree3dgaussian} follows CF-3DGS, first construct a coarse point cloud from mono-depth maps to train a 3DGS model, then progressively estimate camera poses based on this pre-trained model by constraining the correspondences between rendering view and ground-truth.
% \gjh{Similarly, CG-3DGS~\cite{sun2024correspondenceguidedsfmfree3dgaussian} first utilizes monocular depth estimation and the camera parameters from the first frame to initialize a set of 3D Gaussians. It then progressively estimates camera poses based on this pre-trained model by constraining the correspondences between the rendered views and the ground truth.}
% % Free-SurGS~\cite{freesurgs2024} matches the projection flow derived from 3D Gaussians with optical flow to estimate the poses, to compensate for the limitations of photometric loss.
% \gjh{Free-SurGS~\cite{freesurgs2024} introduces the first SfM-free 3DGS approach for surgical scene reconstruction. Due to the challenges posed by weak textures and photometric inconsistencies in surgical scenes, Free-SurGS achieves pose estimation by minimizing the flow loss between the projection flow and the optical flow. Subsequently, it keeps the camera pose fixed and optimizes the scene representation by minimizing the photometric loss, depth loss and flow loss.}
% \gjh{However, most current works assume camera intrinsics are known and primarily focus on optimizing camera poses. Additionally, these methods typically rely on sequentially ordered image inputs and incrementally optimize camera parameters and scene representation. This inevitably leads to drift errors, preventing the achievement of globally consistent results. Our work aims to address these issues.}

Regarding 3D Gaussian Splatting, CF-3DGS~\cite{CF-3DGS-2024} utilizes mono-depth information to refine the optimization of local 3DGS for relative pose estimation and subsequently learns a global 3DGS in a sequential manner. InstantSplat~\cite{fan2024instantsplat} targets sparse view scenes, initially employing DUSt3R~\cite{dust3r2024cvpr} to create a densely covered, pixel-aligned point set for initializing 3D Gaussian models, and then implements a parallel grid partitioning strategy to accelerate joint optimization. Jiang \etal~\cite{Jiang_2024sig} develops an incremental method for reconstructing camera poses and scenes, but it struggles with complex scenes and unordered images. 
% Similarly, CG-3DGS~\cite{sun2024correspondenceguidedsfmfree3dgaussian} progressively estimates camera poses using a pre-trained model by aligning the correspondences between rendered views and actual scenes. Free-SurGS~\cite{freesurgs2024} pioneers an SfM-free 3DGS method for reconstructing surgical scenes, overcoming challenges such as weak textures and photometric inconsistencies by minimizing the discrepancy between projection flow and optical flow.
%\pb{SF-3DGS-HT~\cite{ji2024sfmfree3dgaussiansplatting} introduced VFI into training as additional photometric constraints. They separated the whole scene into several local 3DGS models and then merged them hierarchically, which leads to a significant improvement on simple and dense view scenes.}
HT-3DGS~\cite{ji2024sfmfree3dgaussiansplatting} interpolates frames for training and splits the scene into local clips, using a hierarchical strategy to build 3DGS model. It works well for simple scenes, but fails with dramatic motions due to unstable interpolation and low efficiency.
% {While effective for simple scenes, it struggles with dramatic motion due to unstable view interpolation and suffers from low computational efficiency.}

However, most existing methods generally depend on sequentially ordered image inputs and incrementally optimize camera parameters and 3DGS, which often leads to drift errors and hinders achieving globally consistent results. Our work seeks to overcome these limitations.

\section{Conclusion}
% This study demonstrates that classic GNNs, when enhanced with our GNN$^+$ framework, can match and even surpass GTs on graph-level tasks. Across 14 benchmark datasets, these upgraded GNNs consistently rank in the top three, achieving first place in eight while also exhibiting greater efficiency. Our findings challenge the prevailing assumption that GTs inherently outperform GNNs and reaffirm the potential of well-structured GNNs as a powerful model. 
% We hope that our findings encourage more rigorous empirical evaluations in the field of graph machine learning.

%This study highlights the often-overlooked potential of classic GNNs. By integrating six widely used techniques into a unified GNN$^+$ framework, we enhance 3 classic GNNs for graph-level tasks. Evaluations on 14 benchmark datasets show that, these enhanced GNNs consistently rank among the top three and secure first place on eight, while also exhibiting greater efficiency. These findings challenge the prevailing belief that GTs are inherently superior, reaffirming that well-designed GNNs remain highly competitive.

This study highlights the often-overlooked potential of classic GNNs in tacking graph-level tasks. By integrating six widely used techniques into a unified GNN$^+$ framework, we enhance three classic GNNs for graph-level tasks. Evaluations on 14 benchmark datasets reveal that, these enhanced GNNs match or outperform GTs, while also demonstrating greater efficiency. These findings challenge the prevailing belief that GTs are inherently superior, reaffirming the capability of simple GNN structures as powerful models.

% \newpage
\section{Limitations and Future Work}

\paragraph{Limitations} While our GSQ-Tuning significantly advances on-device LLM adaptation through integer-focused optimization and parameter-efficient quantization, two key limitations warrant discussion:

\paragraph{Non-linear Operator Precision.} Our current implementation maintains non-linear operations (e.g., LayerNorm, Softmax) in 16-bit to preserve numerical stability. This introduces partial precision conversion overhead during computation. However, non-linear operations do not contain additional learnable parameters and thus do not consume memory. Moreover, these non-linear operations are generally computation-light, making their computational burden negligible. Future work could explore fully integer implementations for non-linear layers. \textbf{Bit-Width Range Constraints.}
The current framework operates effectively in 5-8bit configurations but didn't present the performance at extreme low bit ($\leq$ 4bit) precision. This stems from gradient direction distortion under extreme quantization—a challenge requiring new error compensation mechanisms. We plan to investigate two directions: (1) 4bit stochastic rounding with gradient-aware scaling, and (2) mixed-precision adapters allocating higher bits to critical gradient dimensions.

\paragraph{Future Work} Furthermore, future work could explore (1) full integer fine-tuning, (2) extreme low-bit quantized fine-tuning and (3) co-design with emerging integer-optimized AI accelerators. Our code will be publicly available to advance edge LLMs research.
% \section*{Acknowledgments}
\clearpage
\bibliography{custom}

\appendix

\section{Appendix}
\label{sec:appendix}
\subsection{Differences with Quantization-aware training (QAT): }
\label{qat_fqt}
Quantization-aware training (QAT)~\cite{choi2018pact,Zhang_2018_ECCV,zhou2017incremental,jacob2018quantization,dong2019hawq,dong2019hawqv2,shen2019q,zafrir2019q8bert,shen2020QBERT,tang2022mkq,zhang2020ternarybert,bai2020binarybert,foret2020sharpness,wang2022squat} is an \emph{inference acceleration} technique which trains networks with quantizers inserted in the forward propagation graph, so the trained network can perform efficiently during inference. 
QAT can compress activation/weights to extremely low precision (e.g. 1-2 bits). 
It is tempting to think that directly applying a quantizer for QAT to FQT can lead to similar low activation/weights bit-width. However, even only quantizing the forward propagation for FQT is much more challenging than QAT because:  \raisebox{-0.5pt}{\ding[1.1]{182\relax}} QAT requires a converged full-precision model as initialization~\cite{esser2019learned} and/or as a teacher model for knowledge distillation~\cite{bai2020binarybert}; \raisebox{-0.5pt}{\ding[1.1]{184\relax}} QAT may approximate the discrete quantizer with continuous functions during training~\cite{gong2019differentiable}, which cannot be implemented with integer arithmetic. Due to these challenges, it is still an open problem to do FQT with low-bit activations/weights. 

\begin{table*}[h]
\renewcommand\arraystretch{1.0}
\centering
\caption{$0$-shot commonsense QA accuracy (\%) across different bits and rank on llama2-7B.}
\label{tab:llama2-7b}
\setlength{\tabcolsep}{1.2mm}
{\resizebox{0.98\textwidth}{!}{
\begin{tabular}{lcccccccccccccc|c}
\noalign{\vspace{0.3em}}
\toprule
\noalign{\vspace{0.1em}}
\textbf{Method} & \textbf{rank}& LLMs branch & low-rank branch &\textbf{Avg.} & \textbf{ARC-c} & \textbf{ARC-e} & \textbf{BoolQ} & \textbf{HellaS.} & \textbf{OBQA} & \textbf{PIQA} & \textbf{SCIQ.} & \textbf{WinoG.} & \textbf{Mem. (G)} \\
\midrule
\noalign{\vspace{0.1em}}
 LLaMA2-7B    & &16-16-16 &w/o & 64.13 &46.25 &74.62 &77.68 &76.01 &44.20 &79.11 &46.11 &69.06 &13.2\\
 \noalign{\vspace{0.1em}}\hdashline[0.8pt/1pt]\noalign{\vspace{0.1em}}
 \textit{w/ QLoRA} & 16 & 4-16-16 &16-16-16 & 65.05 &47.53 &75.17 &78.59 &76.09 &44.00 &79.54 &49.44 &70.09&10.18\\
\noalign{\vspace{0.1em}}\hdashline[0.8pt/1pt]\noalign{\vspace{0.1em}}
\multirow{4}{*}{w/ GSQ-Tuning}  &\multirow{4}{*}{16} & 4-8-8 &8-8-8  & \default{\textbf{65.10}}& 47.53 &74.71 &78.35 &75.99 &45.00 &79.65 &49.28 &70.32&6.73\\
& & 4-7-7 &7-7-7  & \default{\textbf{64.96}} &47.18 &75.21 &78.10 &75.98 &44.80 &79.27 &49.95 &69.38 &5.98\\
& & 4-6-6 &6-6-6  & \default{\textbf{64.87}} &46.84 &73.78 &78.07 &75.88 &45.80 &79.22 &49.39 &70.01 &5.32\\
  & & 4-5-5 &5-5-5 & \default{63.97} &46.76 &72.64 &75.78 &74.95 &45.20 &79.05 &48.62 &68.75 &5.27\\
\midrule
\noalign{\vspace{0.1em}}
 \textit{w/ QLoRA} & 32 & 4-16-16 &16-16-16 & 65.44 &47.27 &75.04 &78.87 &76.11 &44.60 &79.76 &49.95 &70.32 &10.37\\
\noalign{\vspace{0.1em}}\hdashline[0.8pt/1pt]\noalign{\vspace{0.1em}}
\multirow{4}{*}{w/ GSQ-Tuning}  &\multirow{4}{*}{32} & 4-8-8 &8-8-8  & \default{\textbf{65.45}} &48.12 &74.71 &78.38 &76.14 &46.00 &79.71 &49.64 &70.96 &6.92\\
& & 4-7-7 &7-7-7  & \default{\textbf{65.43}} &47.35 &74.20 &78.99 &75.84 &46.00 &79.92 &49.59 &71.59 &6.16\\
& & 4-6-6 &6-6-5  & \default{\textbf{65.01}} &47.44 &74.62 &78.65 &76.03 &44.00 &79.60 &50.05 &69.69 &5.55\\
  & &4-5-5 &5-5-5 & \default{64.00} &44.97 &73.32 &75.29 &74.95 &44.60 &79.27 &48.93 &70.24&5.45\\
\midrule
 \textit{w/ QLoRA} & 64 & 4-16-16 &16-16-16 & 65.69 &47.14 &74.75 &79.50 &76.46 &45.50 &79.63 &50.26 &71.32 &10.73\\
\noalign{\vspace{0.1em}}\hdashline[0.8pt/1pt]\noalign{\vspace{0.1em}}
\multirow{4}{*}{w/ GSQ-Tuning}  &\multirow{4}{*}{64} & 4-8-8 &8-8-8  & \default{\textbf{65.60}} &48.12 &74.24 &79.72 &76.00 &45.80 &79.60 &49.69 &71.67 &7.28\\
& & 4-7-7 &7-7-7  & \default{\textbf{65.47}} &47.78 &74.71 &79.51 &76.09 &45.80 &79.60 &49.80 &70.48 &6.52\\
& & 4-6-6 &6-6-6  & \default{\textbf{65.39}} &47.70 &74.58 &79.24 &76.05 &44.60 &79.60 &50.41 &70.96 &5.97\\
  & & 4-5-5 &5-5-5 & \default{64.18} &45.14 &72.69 &75.20 &75.27 &46.40 &79.65 &48.62 &70.48 &5.81\\
\midrule
 \textit{w/ QLoRA} & 128 & 4-16-16 &16-16-16& 65.84 &48.24 &74.91 &79.78 &76.27 &45.52 &79.77 &50.48 &71.79 &11.46\\
\noalign{\vspace{0.1em}}\hdashline[0.8pt/1pt]\noalign{\vspace{0.1em}}
\multirow{4}{*}{w/ GSQ-Tuning}  &\multirow{4}{*}{128} & 4-8-8 &8-8-8  & \default{\textbf{65.79}} &48.12 &74.83 &80.28 &75.96 &45.80 &79.54 &50.61 &71.19 &8.02\\
& & 4-7-7 &7-7-7  & \default{\textbf{65.69}} &48.04 &74.87 &79.79 &76.08 &45.00 &79.49 &50.61 &71.67 &7.26\\
& & 4-6-6 &6-6-6  & \default{\textbf{65.58}} &47.87 &74.54 &80.09 &76.05 &45.40 &79.38 &50.10 &71.27 &6.10\\
  & &4-5-5 &5-5-5 & \default{64.46} &46.50 &72.77 &75.99 &75.31 &46.60 &79.00 &48.98 &70.56 &6.14\\
\midrule
 \textit{w/ QLoRA} & 256 & 4-16-16 &16-16-16 & 66.12 &48.33 &75.00 &80.94 &76.37 &45.61 &79.97 &51.13 &71.64 &12.93\\
\noalign{\vspace{0.1em}}\hdashline[0.8pt/1pt]\noalign{\vspace{0.1em}}
\multirow{4}{*}{w/ GSQ-Tuning}  &\multirow{4}{*}{256} & 4-8-8 &8-8-8  & \default{\textbf{66.19}} &48.55 &75.13 &80.76 &76.14 &47.00 &79.38 &50.72 &71.82 &9.47\\
& & 4-7-7 &7-7-7  & \default{\textbf{65.96}} &48.46 &75.08 &80.43 &76.04 &45.60 &79.76 &50.72 &71.59 &8.42\\
& & 4-6-6 &6-6-6  & \default{\textbf{65.90}} &48.38 &74.16 &79.94 &75.81 &46.80 &79.43 &50.87 &71.82 &7.66\\
  & & 4-5-5 &5-5-5 & \default{64.59} &46.33 &72.60 &76.51 &75.57 &46.40 &79.60 &49.39 &70.32 &6.75\\
\midrule
 \textit{w/ QLoRA} & 512 & 4-16-16 &16-16-16& 66.59 &49.26 &75.20 &81.99 &76.06 &46.74 &79.49 &51.71 &72.27 &15.85\\
\noalign{\vspace{0.1em}}\hdashline[0.8pt/1pt]\noalign{\vspace{0.1em}}
\multirow{4}{*}{w/ GSQ-Tuning}  &\multirow{4}{*}{512} & 4-8-8 &8-8-8  & \default{\textbf{66.52}} &49.49 &74.92 &81.28 &75.89 &47.60 &79.49 &51.59 &71.90 &11.40\\
& & 4-7-7 &7-7-7  & \default{\textbf{66.33}} &48.89 &74.75 &81.41 &76.06 &47.00 &79.54 &51.74 &71.27 &9.95\\
& & 4-6-6 &6-6-6  & \default{\textbf{66.31}} &48.55 &75.51 &80.80 &76.42 &46.00 &79.60 &51.64 &71.98&9.19\\
  & & 4-5-5 &5-5-5 & \default{64.86} &47.44 &73.15 &76.85 &75.62 &47.00 &79.33 &49.18 &70.32 &8.25\\
\bottomrule
\end{tabular}}}
\end{table*}
\begin{table*}[!t]
\renewcommand\arraystretch{1.0}
\centering
\caption{$0$-shot commonsense QA accuracy (\%) across different bits and rank on llama2-13B.}
\label{tab:llama2-13b}
\setlength{\tabcolsep}{1.2mm}
{\resizebox{0.98\textwidth}{!}{
\begin{tabular}{lcccccccccccccc|c}
\noalign{\vspace{0.3em}}
\toprule
\noalign{\vspace{0.1em}}
\textbf{Method} & \textbf{rank}& LLMs branch & low-rank branch &\textbf{Avg.} & \textbf{ARC-c} & \textbf{ARC-e} & \textbf{BoolQ} & \textbf{HellaS.} & \textbf{OBQA} & \textbf{PIQA} & \textbf{SCIQ.} & \textbf{WinoG.} & \textbf{Mem. (G)} \\
\midrule
\noalign{\vspace{0.1em}}
 LLaMA2-13B    &-     & 16-16-16 & w/o&66.65 &48.81 &76.47 &82.45 &79.67 &44.80 &80.36 &48.31 &72.38 &25.70\\
 \noalign{\vspace{0.1em}}\hdashline[0.8pt/1pt]\noalign{\vspace{0.1em}}
 \textit{w/ QLoRA} & 16 &4-16-16 & 16-16-16 & 67.32 &49.74 &76.98 &82.94 &78.85 &46.00 &80.52 &50.36 &73.16 &16.56\\
\noalign{\vspace{0.1em}}\hdashline[0.8pt/1pt]\noalign{\vspace{0.1em}}
\multirow{4}{*}{w/ GSQ-Tuning}  &\multirow{4}{*}{16} &4-8-8 & 8-8-8  & \default{\textbf{67.35}} &49.83 &77.06 &83.09 &78.89 &46.00 &80.47 &50.31 &73.16 &11.13\\
& & 4-7-7 & 7-7-7   & \default{\textbf{67.29}} &49.91 &76.94 &83.03 &78.90 &45.40 &80.58 &50.61 &73.01 &10.58\\
& & 4-6-6 & 6-6-6   & \default{\textbf{67.23}} &49.66 &76.98 &82.75 &78.79 &46.00 &80.47 &50.05 &73.16 &10.03\\
  & & 4-5-5 & 5-5-5  & \default{66.57} &49.57 &76.43 &81.62 &77.98 &45.40 &80.09 &49.39 &72.06 &9.47\\
\midrule
\noalign{\vspace{0.1em}}
 \textit{w/ QLoRA} & 32 &4-16-16 & 16-16-16 & 67.47 &49.83 &77.02 &83.24 &78.92 &46.20 &80.58 &50.77 &73.24 &16.85\\
\noalign{\vspace{0.1em}}\hdashline[0.8pt/1pt]\noalign{\vspace{0.1em}}
\multirow{4}{*}{w/ GSQ-Tuning}  &\multirow{4}{*}{32} & 4-8-8 & 8-8-8  & \default{\textbf{67.49}} &49.83 &76.98 &83.15 &78.94 &45.60 &80.79 &51.07 &73.56 &11.42\\
& & 4-7-7 & 7-7-7  & \default{\textbf{67.38}} &50.17 &77.06 &82.81 &78.99 &45.40 &80.79 &50.46 &73.40  &10.87\\
& & 4-6-6 & 6-6-6  & \default{\textbf{67.35}} &49.83 &77.06 &83.09 &78.89 &46.00 &80.47 &50.31 &73.16 &10.31\\
  & & 4-5-5 & 5-5-5 & \default{66.65} &48.38 &76.18 &82.08 &78.07 &45.60 &80.36 &49.74 &72.77 &9.76 \\
\midrule
 \textit{w/ QLoRA} & 64 &4-16-16 & 16-16-16 & 67.61 &49.66 &77.23 &83.30 &78.95 &45.40 &80.74 &51.59 &73.24 &17.42\\
\noalign{\vspace{0.1em}}\hdashline[0.8pt/1pt]\noalign{\vspace{0.1em}}
\multirow{4}{*}{w/ GSQ-Tuning}  &\multirow{4}{*}{64} & 4-8-8 & 8-8-8   & \default{\textbf{67.48}} &49.57 &77.40 &82.87 &78.88 &46.20 &80.90 &50.72 &73.32 &11.99\\
& & 4-7-7 & 7-7-7   & \default{\textbf{67.43}} &49.74 &77.27 &82.91 &78.89 &46.00 &80.90 &50.61 &73.09 &11.44\\
& & 4-6-6 & 6-6-6   & \default{\textbf{67.35}} &49.66 &77.27 &82.75 &78.66 &79.05 &80.90 &50.97 &73.16 &10.89\\
  & & 4-5-5 & 5-5-5 & \default{66.97} &49.91 &76.60 &81.87 &78.15 &46.20 &80.41 &49.54 &73.09 &10.33\\
\midrule
 \textit{w/ QLoRA} & 128&4-16-16 & 16-16-16& 67.61 &50.34 &77.40 &83.55 &78.89 &46.00 &80.85 &50.92 &72.93&18.56\\
\noalign{\vspace{0.1em}}\hdashline[0.8pt/1pt]\noalign{\vspace{0.1em}}
\multirow{4}{*}{w/ GSQ-Tuning}  &\multirow{4}{*}{128} & 4-8-8 & 8-8-8   & \default{\textbf{67.62}} &50.34 &77.06 &83.18 &78.96 &46.40 &80.69 &50.92 &73.40&13.14\\
& & 4-7-7 & 7-7-7   & \default{\textbf{67.57}} &50.43 &77.36 &83.06 &79.05 &45.60 &80.85 &51.28 &72.93 &12.58\\
& & 4-6-6 & 6-6-6   & \default{\textbf{67.53}} &50.43 &77.31 &83.15 &78.81 &45.80 &80.58 &50.97 &73.16 &12.03\\
  & & 4-5-5 & 5-5-5  & \default{67.10} &49.49 &76.81 &82.08 &78.22 &46.40 &80.03 &50.56 &73.24 &11.48\\
\midrule
 \textit{w/ QLoRA} & 256 &4-16-16 & 16-16-16 & 67.91 &50.77 &77.36 &83.64 &78.88 &46.60 &80.74 &51.69 &73.64&20.85\\
\noalign{\vspace{0.1em}}\hdashline[0.8pt/1pt]\noalign{\vspace{0.1em}}
\multirow{4}{*}{w/ GSQ-Tuning}  &\multirow{4}{*}{256} & 4-8-8 & 8-8-8   & \default{\textbf{67.84}} &51.11 &77.06 &83.82 &78.80 &46.40 &80.69 &52.00 &72.85 &15.42\\
& &4-7-7 & 7-7-7   & \default{\textbf{67.74}} &50.77 &77.31 &83.79 &78.84 &46.00 &80.63 &51.89 &72.69 &14.87\\
& & 4-6-6 & 6-6-6   & \default{\textbf{67.68}} &50.77 &77.19 &83.49 &78.82 &46.00 &80.58 &51.38 &73.24 &14.32\\
  & & 4-5-5 & 5-5-5  & \default{67.22} &50.85 &75.84 &82.11 &78.21 &46.00 &80.36 &50.92 &73.48 &13.76\\
\midrule
 \textit{w/ QLoRA} & 512 &4-16-16 & 16-16-16 & 67.94 &50.60 &77.48 &83.88 &79.00 &46.40 &80.74 &52.05 &73.40&25.43\\
\noalign{\vspace{0.1em}}\hdashline[0.8pt/1pt]\noalign{\vspace{0.1em}}
\multirow{4}{*}{w/ GSQ-Tuning}  &\multirow{4}{*}{512} & 4-8-8 & 8-8-8   & \default{\textbf{67.92}} &51.02 &77.27 &83.27 &79.04 &46.40 &81.01 &51.79 &73.56 &20.00\\
& & 4-7-7 & 7-7-7   & \default{\textbf{67.90}} &51.19 &77.15 &83.79 &78.82 &46.80 &80.69 &51.79 &73.01 &19.45\\
& & 4-6-6 & 6-6-6   & \default{\textbf{67.82}} &51.02 &77.02 &83.85 &78.93 &46.20 &80.90 &51.54 &73.09 &18.89\\
  & & 4-5-5 & 5-5-5  & \default{67.39} &50.94 &76.68 &82.29 &78.39 &46.20 &80.41 &51.69 &72.53 &18.34\\
\bottomrule
\end{tabular}}}
\end{table*}
\begin{table*}[!t]
\renewcommand\arraystretch{1.0}
\centering
\caption{$0$-shot commonsense QA accuracy (\%) across different bits and rank on llama2-70B.}
\label{tab:llama2-70b}
\setlength{\tabcolsep}{1.2mm}
{\resizebox{0.98\textwidth}{!}{
\begin{tabular}{lcccccccccccccc|c}
\noalign{\vspace{0.3em}}
\toprule
\noalign{\vspace{0.1em}}
\textbf{Method} & \textbf{rank}& LLMs branch & low-rank branch &\textbf{Avg.} & \textbf{ARC-c} & \textbf{ARC-e} & \textbf{BoolQ} & \textbf{HellaS.} & \textbf{OBQA} & \textbf{PIQA} & \textbf{SCIQ.} & \textbf{WinoG.} & \textbf{Mem. (G)} \\
\midrule
\noalign{\vspace{0.1em}}
 LLaMA2-70B    &-     &  16-16-16 &w/o & 70.68 &56.91 &80.05 &85.78 &83.59 &48.60 &82.48 &48.67 &79.40 &137.42\\
 \noalign{\vspace{0.1em}}\hdashline[0.8pt/1pt]\noalign{\vspace{0.1em}}
 \textit{w/ QLoRA} & 16 & 4-16-16 &16-16-16 & 71.72 &58.62 &81.44 &86.39 &83.92 &49.80 &83.03 &50.46 &80.11 &63.90\\
\noalign{\vspace{0.1em}}\hdashline[0.8pt/1pt]\noalign{\vspace{0.1em}}
\multirow{4}{*}{w/ GSQ-Tuning}  &\multirow{4}{*}{16} & 4-8-8&8-8-8  & \default{\textbf{71.65}} &58.62 &81.23 &86.36 &83.87 &49.60 &83.19 &50.41 &79.95 &49.17\\
& & 4-7-7&7-7-7  & \default{\textbf{71.63}} &58.87 &81.57 &86.24 &83.89 &49.20 &83.19 &50.46 &79.64 &47.44\\
& & 4-6-6&6-6-6  & \default{\textbf{71.58}} &58.62 &81.36 &86.15 &83.84 &49.60 &82.97 &50.41 &79.64 &45.72\\
  & & 4-5-5&5-5-5 & \default{71.02} &57.34 &80.56 &85.93 &83.75 &49.00 &82.59 &49.33 &79.64 &43.99\\
\midrule
\noalign{\vspace{0.1em}}
 \textit{w/ QLoRA} & 32 &4-16-16 &16-16-16 & 71.84 &59.13 &81.82 &86.27 &83.88 &49.20 &83.03 &51.02 &80.35 &64.87\\
\noalign{\vspace{0.1em}}\hdashline[0.8pt/1pt]\noalign{\vspace{0.1em}}
\multirow{4}{*}{w/ GSQ-Tuning}  &\multirow{4}{*}{32} & 4-8-8&8-8-8  & \default{\textbf{71.78}} &59.04 &81.90 &86.33 &83.89 &49.00 &83.19 &51.07 &79.79 &50.17\\
& & 4-7-7&7-7-7  & \default{\textbf{71.76}} &59.30 &81.61 &86.18 &83.98 &49.00 &83.19 &51.02 &79.79 &48.44\\
& & 4-6-6&6-6-6  & \default{\textbf{71.60}} &58.96 &81.36 &86.15 &83.87 &48.80 &83.03 &51.02 &79.64 &46.72\\
  & & 4-5-5&5-5-5 & \default{71.26} &57.59 &80.85 &86.15 &83.93 &49.00 &83.13 &50.00 &79.40 &44.99\\
\midrule
 \textit{w/ QLoRA} & 64 & 4-16-16 &16-16-16 & 72.22 &59.81 &82.20 &86.51 &83.89 &50.40 &83.13 &51.48 &80.35 &66.82\\
\noalign{\vspace{0.1em}}\hdashline[0.8pt/1pt]\noalign{\vspace{0.1em}}
\multirow{4}{*}{w/ GSQ-Tuning}  &\multirow{4}{*}{64} & 4-8-8&8-8-8  & \default{\textbf{72.20}} &59.90 &82.32 &86.51 &83.90 &50.20 &83.08 &51.59 &80.11 &52.17\\
& & 4-7-7&7-7-7  & \default{\textbf{72.18}} &59.81 &82.28 &86.39 &83.88 &50.20 &83.13 &51.54 &80.19 &50.44\\
& & 4-6-6&6-6-6  & \default{\textbf{72.10}}  &59.39 &82.15 &86.51 &83.94 &50.00 &83.30 &50.92 &80.58 &48.71\\
  & & 4-5-5&5-5-5 & \default{71.70} &58.87 &81.48 &85.90 &83.91 &49.60 &82.81 &50.67 &80.43 &46.98\\
\midrule
 \textit{w/ QLoRA} & 128 & 4-16-16 &16-16-16& 72.39& 60.67 &82.37 &86.88 &84.05 &49.20 &83.19 &52.15 &80.66&70.96\\
\noalign{\vspace{0.1em}}\hdashline[0.8pt/1pt]\noalign{\vspace{0.1em}}
\multirow{4}{*}{w/ GSQ-Tuning}  &\multirow{4}{*}{128} & 4-8-8&8-8-8  &\default{\textbf{72.37}} &60.75 &82.49 &87.00 &83.94 &49.40 &83.08 &52.15 &80.19 &56.16\\
& & 4-7-7&7-7-7  & \default{\textbf{72.32}} &60.41 &82.45 &86.94 &83.94 &49.00 &83.08 &52.15 & 80.58 &54.43\\
& & 4-6-6&6-6-6  & \default{\textbf{72.28}}  &59.81 &82.45 &86.91 &83.99 &49.60 &83.35 &51.89 & 80.27 &52.70\\
  & & 4-5-5&5-5-5 & \default{71.85} &59.47 &81.90 &86.48 &83.82 &48.20 &83.08 &51.02 & 80.82&50.97\\
\bottomrule
\end{tabular}}}
\end{table*}
\begin{table*}[!t]
\renewcommand\arraystretch{1.0}
\centering
\caption{$0$-shot commonsense QA accuracy (\%) across different bits and rank on llama3-3B.}
\label{tab:llama3-3b}
\setlength{\tabcolsep}{1.2mm}
{\resizebox{0.98\textwidth}{!}{
\begin{tabular}{lcccccccccccccc|c}
\noalign{\vspace{0.3em}}
\toprule
\noalign{\vspace{0.1em}}
\textbf{Method} & \textbf{rank}& LLMs branch &low-rank branch &\textbf{Avg.} & \textbf{ARC-c} & \textbf{ARC-e} & \textbf{BoolQ} & \textbf{HellaS.} & \textbf{OBQA} & \textbf{PIQA} & \textbf{SCIQ.} & \textbf{WinoG.} & \textbf{Mem. (G)} \\
\midrule
\noalign{\vspace{0.1em}}
 LLaMA3-3B    &-     &  16-16-16 & w/o & 64.13 &46.25 &74.62 &77.68 &76.01 &44.20 &79.11 &46.11 &69.06 &6.42\\
 \noalign{\vspace{0.1em}}\hdashline[0.8pt/1pt]\noalign{\vspace{0.1em}}
 \textit{w/ QLoRA} & 16 & 4-16-16 & 16-16-16 & 65.05 &47.53 &75.17 &78.59 &76.09 &44.00 &79.54 &49.44 &70.09 &6.42\\
\noalign{\vspace{0.1em}}\hdashline[0.8pt/1pt]\noalign{\vspace{0.1em}}
\multirow{4}{*}{w/ GSQ-Tuning}  &\multirow{4}{*}{16} & 4-8-8 & 8-8-8  & \default{\textbf{65.10}}& 47.53 &74.71 &78.35 &75.99 &45.00 &79.65 &49.28 &70.32 &3.57\\
& & 4-7-7 & 7-7-7  & \default{\textbf{64.96}} &47.18 &75.21 &78.10 &75.98 &44.80 &79.27 &49.95 &69.38 &3.34\\
& & 4-6-6 & 6-6-6  & \default{\textbf{64.87}} &46.84 &73.78 &78.07 &75.88 &45.80 &79.22 &49.39 &70.01 &3.11\\
  & & 4-5-5 & 5-5-5 & \default{63.97} &46.76 &72.64 &75.78 &74.95 &45.20 &79.05 &48.62 &68.75 &2.88\\
\midrule
\noalign{\vspace{0.1em}}
 \textit{w/ QLoRA} & 32 &4-16-16 & 16-16-16 & 65.24 &47.27 &75.04 &78.87 &76.11 &44.60 &79.76 &49.95 &70.32 &6.54\\
\noalign{\vspace{0.1em}}\hdashline[0.8pt/1pt]\noalign{\vspace{0.1em}}
\multirow{4}{*}{w/ GSQ-Tuning}  &\multirow{4}{*}{32} & 4-8-8 & 8-8-8 & \default{\textbf{65.45}} &48.12 &74.71 &78.38 &76.14 &46.00 &79.71 &49.64 &70.96 &3.69\\
& & 4-7-7 & 7-7-7  & \default{\textbf{65.43}} &47.35 &74.20 &78.99 &75.84 &46.00 &79.92 &49.59 &71.59 &3.46\\
& & 4-6-6 & 6-6-6 & \default{\textbf{65.01}} &47.44 &74.62 &78.65 &76.03 &44.00 &79.60 &50.05 &69.69 &3.23\\
  & & 4-5-5 & 5-5-5 & \default{64.00} &44.97 &73.32 &75.29 &74.95 &44.60 &79.27 &48.93 &70.24 &3.00\\
\midrule
 \textit{w/ QLoRA} & 64 & 4-16-16 & 16-16-16 & 65.69 &47.14 &74.75 &79.50 &76.46 &45.50 &79.63 &50.26 &71.32 &6.78\\
\noalign{\vspace{0.1em}}\hdashline[0.8pt/1pt]\noalign{\vspace{0.1em}}
\multirow{4}{*}{w/ GSQ-Tuning}  &\multirow{4}{*}{64} & 4-8-8 & 8-8-8  & \default{\textbf{65.60}} &48.12 &74.24 &79.72 &76.00 &45.80 &79.60 &49.69 &71.67 &3.93\\
& & 4-7-7 & 7-7-7  & \default{\textbf{65.47}} &47.78 &74.71 &79.51 &76.09 &45.80 &79.60 &49.80 &70.48 &3.70\\
& & 4-6-6 & 6-6-6  & \default{\textbf{65.39}} &47.70 &74.58 &79.24 &76.05 &44.60 &79.60 &50.41 &70.96 &3.47\\
  & &4-5-5 & 5-5-5 & \default{64.18} &45.14 &72.69 &75.20 &75.27 &46.40 &79.65 &48.62 &70.48 &3.24\\
\midrule
 \textit{w/ QLoRA} & 128 & 4-16-16 & 16-16-16& 65.84 &48.24 &74.91 &79.78 &76.27 &45.52 &79.77 &50.48 &71.79 &6.76\\
\noalign{\vspace{0.1em}}\hdashline[0.8pt/1pt]\noalign{\vspace{0.1em}}
\multirow{4}{*}{w/ GSQ-Tuning}  &\multirow{4}{*}{128} & 4-8-8 & 8-8-8  & \default{\textbf{65.79}} &48.12 &74.83 &80.28 &75.96 &45.80 &79.54 &50.61 &71.19 &4.41\\
& & 4-7-7 & 7-7-7  & \default{\textbf{65.69}} &48.04 &74.87 &79.79 &76.08 &45.00 &79.49 &50.61 &71.67 &4.18\\
& & 4-6-6 & 6-6-6  & \default{\textbf{65.58}} &47.87 &74.54 &80.09 &76.05 &45.40 &79.38 &50.10 &71.27 &3.95\\
  & & 4-5-5 & 5-5-5 & \default{64.46} &46.50 &72.77 &75.99 &75.31 &46.60 &79.00 &48.98 &70.56 &3.72\\
\midrule
 \textit{w/ QLoRA} & 256 &4-16-16 & 16-16-16 & 66.12 &48.33 &75.00 &80.94 &76.37 &45.61 &79.97 &51.13 &71.64 &7.61\\
\noalign{\vspace{0.1em}}\hdashline[0.8pt/1pt]\noalign{\vspace{0.1em}}
\multirow{4}{*}{w/ GSQ-Tuning}  &\multirow{4}{*}{256} & 4-8-8 & 8-8-8  & \default{\textbf{66.19}} &48.55 &75.13 &80.76 &76.14 &47.00 &79.38 &50.72 &71.82 &5.37\\
& & 4-7-7 & 7-7-7  & \default{\textbf{65.96}} &48.46 &75.08 &80.43 &76.04 &45.60 &79.76 &50.72 &71.59 &5.13\\
& & 4-6-6 & 6-6-6  & \default{\textbf{65.90}} &48.38 &74.16 &79.94 &75.81 &46.80 &79.43 &50.87 &71.82 &4.90\\
  & & 4-5-5 & 5-5-5 & \default{64.59} &46.33 &72.60 &76.51 &75.57 &46.40 &79.60 &49.39 &70.32 &4.67\\
\midrule
 \textit{w/ QLoRA} & 512 & 4-16-16 & 16-16-16 & 66.59 &49.26 &75.20 &81.99 &76.06 &46.74 &79.49 &51.71 &72.27&9.73\\
\noalign{\vspace{0.1em}}\hdashline[0.8pt/1pt]\noalign{\vspace{0.1em}}
\multirow{4}{*}{w/ GSQ-Tuning}  &\multirow{4}{*}{512} & 4-8-8 & 8-8-8  & \default{\textbf{66.52}} &49.49 &74.92 &81.28 &75.89 &47.60 &79.49 &51.59 &71.90&7.28\\
& & 4-7-7 & 7-7-7  & \default{\textbf{66.33}} &48.89 &74.75 &81.41 &76.06 &47.00 &79.54 &51.74 &71.27&7.05\\
& & 4-6-6 & 6-6-6 & \default{\textbf{66.31}} &48.55 &75.51 &80.80 &76.42 &46.00 &79.60 &51.64 &71.98 &6.82\\
  & & 4-5-5 & 5-5-5 & \default{64.86} &47.44 &73.15 &76.85 &75.62 &47.00 &79.33 &49.18 &70.32 &6.59\\
\bottomrule
\end{tabular}}}
\end{table*}
\begin{table*}[!t]
\renewcommand\arraystretch{1.0}
\centering
\caption{$0$-shot commonsense QA accuracy (\%) across different bits and rank on llama3-8B.}
\label{tab:llama3-8b}
\setlength{\tabcolsep}{1.2mm}
{\resizebox{0.98\textwidth}{!}{
\begin{tabular}{lcccccccccccccc|c}
\noalign{\vspace{0.3em}}
\toprule
\noalign{\vspace{0.1em}}
\textbf{Method} & \textbf{rank}& LLMs branch &low-rank branch &\textbf{Avg.} & \textbf{ARC-c} & \textbf{ARC-e} & \textbf{BoolQ} & \textbf{HellaS.} & \textbf{OBQA} & \textbf{PIQA} & \textbf{SCIQ.} & \textbf{WinoG.} & \textbf{Mem. (G)} \\
\midrule
\noalign{\vspace{0.1em}}
 LLaMA3-8B    &-     &  16-16-16 & w/o & 67.18 &53.50 &77.74 &81.13 &79.20 &45.00 &80.63 &47.03 &73.24 &15.01\\
 \noalign{\vspace{0.1em}}\hdashline[0.8pt/1pt]\noalign{\vspace{0.1em}}
 \textit{w/ QLoRA} & 16 & 4-16-16 &16-16-16 & 68.14 &54.52 &79.50 &83.43 &78.66 &44.80 &80.85 &50.00 &73.32 &10.71\\
\noalign{\vspace{0.1em}}\hdashline[0.8pt/1pt]\noalign{\vspace{0.1em}}
\multirow{4}{*}{w/ GSQ-Tuning}  &\multirow{4}{*}{16} &  4-8-8 & 8-8-8  & \default{\textbf{68.16}} &54.61 &79.84 &83.70 &78.58 &44.80 &80.79 &49.85 &73.16 &7.03\\
& & 4-7-7 & 7-7-7  & \default{\textbf{68.00}} &54.01 &79.29 &83.46 &78.65 &45.00 &80.85 &49.80 &73.01 &6.65\\
& & 4-6-6 & 6-6-6  & \default{\textbf{67.74}} &54.01 &78.70 &83.09 &78.49 &44.00 &80.90 &49.44 &73.32 &6.26\\
  & & 4-5-5 & 5-5-5 & \default{66.51} &51.54 &77.27 &81.99 &77.00 &44.40 &78.84 &48.46 &72.61 &5.87\\
\midrule
\noalign{\vspace{0.1em}}
 \textit{w/ QLoRA} & 32 &4-16-16 &16-16-16 & 68.31 &55.55 &80.39 &83.36 &78.65 &44.60 &81.28 &50.05 &72.61 &11.02\\
\noalign{\vspace{0.1em}}\hdashline[0.8pt/1pt]\noalign{\vspace{0.1em}}
\multirow{4}{*}{w/ GSQ-Tuning}  &\multirow{4}{*}{32} & 4-8-8 & 8-8-8  & \default{\textbf{68.45}} &55.72 &80.22 &83.43 &78.60 &45.00  &81.18 &50.20 &73.32 &7.23\\
& & 4-7-7 & 7-7-7  & \default{\textbf{68.29}} &54.95 &80.13 &83.36 &78.53 &44.80 &81.01 &50.20 &73.32 &6.84\\
& & 4-6-6 & 6-6-6  & \default{\textbf{68.08}} &55.29 &79.29 &83.55 &78.28 &45.80 &81.07 &49.39 &71.98 &6.46\\
  & & 4-5-5 & 5-5-5 & \default{66.48} &51.71 &77.69 &82.11 &76.91 &44.20 &79.43 &48.16 &71.67 &6.07\\
\midrule
 \textit{w/ QLoRA} & 64 & 4-16-16 &16-16-16 & 68.45 &55.63 &80.13 &83.67 &78.78 &44.80 &81.28 &50.41 &72.93 &11.64\\
\noalign{\vspace{0.1em}}\hdashline[0.8pt/1pt]\noalign{\vspace{0.1em}}
\multirow{4}{*}{w/ GSQ-Tuning}  &\multirow{4}{*}{64} & 4-8-8 & 8-8-8  & \default{\textbf{68.61}} &55.97 &80.22 &83.61 &78.68 &45.20 &81.50 &50.41 &73.32 &7.63\\
& & 4-7-7 & 7-7-7  & \default{\textbf{68.57}}  &55.97 &80.68 &83.73 &78.84 &45.20 &81.01 &50.26 &72.85 &7.24\\
& & 4-6-6 & 6-6-6  & \default{\textbf{68.22}}  &55.55 &79.29 &83.67 &78.47 &44.80 &80.90 &50.05 &73.09 &6.86\\
  & & 4-5-5 & 5-5-5 & \default{66.69} &54.10 &77.99 &81.65 &77.12 &43.80 &79.54 &47.90 &71.43 &6.47\\
\midrule
 \textit{w/ QLoRA} & 128 & 4-16-16 &16-16-16 & 68.77 &56.14 &80.56 &83.98 &79.03 &45.60 &81.34 &50.56 &72.93 &12.13\\
\noalign{\vspace{0.1em}}\hdashline[0.8pt/1pt]\noalign{\vspace{0.1em}}
\multirow{4}{*}{w/ GSQ-Tuning}  &\multirow{4}{*}{128} & 4-8-8 & 8-8-8  & \default{\textbf{68.72}} &56.57 &80.22 &83.82 &78.80 &45.40 &81.23 &50.41 &73.32 &8.43\\
& & 4-7-7 & 7-7-7  & \default{\textbf{68.71}} &56.48 &80.18 &83.88 &78.78 &45.80 &81.34 &50.36 &72.93 &8.04\\
& & 4-6-6 & 6-6-6  & \default{\textbf{68.67}} &56.91 &79.50 &83.79 &78.71 &46.60 &80.52 &50.36 &73.01 &7.66\\
  & &4-5-5 & 5-5-5 & \default{66.92} &52.47 &78.45 &82.63 &77.22 &44.60 &79.49 &48.52 &71.98&7.27\\
\midrule
 \textit{w/ QLoRA} & 256 & 4-16-16 &16-16-16 & 69.09 &56.74 &80.35 &84.56 &79.02 &45.20 &81.83 &50.92 &74.11&13.81\\
\noalign{\vspace{0.1em}}\hdashline[0.8pt/1pt]\noalign{\vspace{0.1em}}
\multirow{4}{*}{w/ GSQ-Tuning}  &\multirow{4}{*}{256} & 4-8-8 & 8-8-8  & \default{\textbf{69.04}} &56.57 &80.85 &84.07 &78.97 &45.40 &81.45 &51.28 &73.72 &10.03 \\
& & 4-7-7 & 7-7-7  & \default{\textbf{69.00}} &56.83 &80.89 &84.25 &78.96 &45.60 &81.50 &50.46 &73.56 &9.64\\
& & 4-6-6 & 6-6-6  & \default{\textbf{68.84}} &56.74 &79.80 &83.98 &78.84 &46.40 &81.12 &50.77 &73.09 &9.26\\
  & & 4-5-5 & 5-5-5 & \default{67.54} &53.33 &78.49 &83.21 &77.38 &44.60 &79.98 &48.93 &73.64 &8.87\\
\midrule
 \textit{w/ QLoRA} & 512 &4-16-16 &16-16-16 & 69.18 &57.17 &80.30 &84.65 &79.28 &46.40 &81.07 &50.36 &74.27 &16.81\\
\noalign{\vspace{0.1em}}\hdashline[0.8pt/1pt]\noalign{\vspace{0.1em}}
\multirow{4}{*}{w/ GSQ-Tuning}  &\multirow{4}{*}{512} & 4-8-8 & 8-8-8  & \default{\textbf{69.24}} &56.48 &80.47 &85.35 &79.13 &45.40 &81.56 &51.54 &74.03 &13.23\\
& & 4-7-7 & 7-7-7 & \default{\textbf{69.16}} &56.40 &80.68 &85.26 &79.10 &45.80 &81.28 &51.13 &73.64 &12.84\\
& & 4-6-6 & 6-6-6  & \default{\textbf{69.01}} &56.57 &80.01 &84.56 &78.84 &45.80 &81.23 &51.64 &73.48 &12.45\\
  & & 4-5-5 & 5-5-5 & \default{67.90} &54.38 &78.60 &83.97 &78.08 &45.30 &80.24 &50.00 &72.76 &12.07\\
\bottomrule
\end{tabular}}}
\end{table*}
\subsection{Detailed results on different rank setting:}
\label{sec:detailed_results}
Here, we also report the results of our GSQ-Tuning on different LlaMA model, including LlaMA2-7B (Tab.\ref{tab:llama2-7b}), LlaMA2-13B (Tab.\ref{tab:llama2-13b}), LlaMA2-70B(Tab.\ref{tab:llama2-70b}), LlaMA3-3B(Tab.\ref{tab:llama3-3b}), and LlaMA3-8B(Tab.\ref{tab:llama3-8b}). The results consistently demonstrated the effectiveness and efficiency of GSQ-Tuning.

% \begin{table*}[!t]
\renewcommand\arraystretch{1.0}
\centering
\caption{$0$-shot commonsense QA accuracy (\%) with respect to different quantization bits with 512 rank.}
\label{tab:common}
\setlength{\tabcolsep}{1.2mm}
{\resizebox{0.98\textwidth}{!}{
\begin{tabular}{lcccccccccccc|c}
\noalign{\vspace{0.3em}}
\toprule
\noalign{\vspace{0.1em}}
\textbf{Method} & \textbf{\#Bits} &\textbf{Avg.} & \textbf{ARC-c} & \textbf{ARC-e} & \textbf{BoolQ} & \textbf{HellaS.} & \textbf{OBQA} & \textbf{PIQA} & \textbf{SCIQ.} & \textbf{WinoG.} & \textbf{Mem.} \\
\midrule
\noalign{\vspace{0.1em}}
 LLaMA2-7B         &  \multirow{2}{*}{W4A16G16} & &56.3 & 78.2 & 67.1 & 67.3 & 38.2 & 72.9 & 28.4 & 58.3 \\
 \textit{+QLoRA} &  & & \textit{61.8} & \textit{78.1} & \textit{68.4} & \textit{75.8} & \textit{43.6} & \textit{73.7} & \textit{32.8} & \textit{62.0} \\
\noalign{\vspace{0.1em}}\hdashline[0.8pt/1pt]\noalign{\vspace{0.1em}}
\multirow{4}{*}{+GSQ-Tuning}  & W7A7G7  & & 54.5 & 76.5 & 66.9 & 66.1 & 36.9 & 70.9 & 27.4 & 57.0 \\
& W6A6G6  & & 54.5 & 76.5 & 66.9 & 66.1 & 36.9 & 70.9 & 27.4 & 57.0 \\
      & W5A5G5 & & 57.4 & 77.6 & 66.2 & 70.9 & 41.8 & 73.5 & 31.2 & 59.8 \\
\midrule
\noalign{\vspace{0.1em}}
 LLaMA2-13B         & \multirow{2}{*}{W4A16G16} & &56.3 & 78.2 & 67.1 & 67.3 & 38.2 & 72.9 & 28.4 & 58.3 \\
 \textit{+QLoRA} &  & & \textit{61.8} & \textit{78.1} & \textit{68.4} & \textit{75.8} & \textit{43.6} & \textit{73.7} & \textit{32.8} & \textit{62.0} \\
\noalign{\vspace{0.1em}}\hdashline[0.8pt/1pt]\noalign{\vspace{0.1em}}
 \multirow{4}{*}{W/ GSQ-Tuning}  & W7A7G7  & & 54.5 & 76.5 & 66.9 & 66.1 & 36.9 & 70.9 & 27.4 & 57.0 \\
 & W6A6G6 & & 57.4 & 77.6 & 66.2 & 70.9 & 41.8 & 73.5 & 31.2 & 59.8 \\
      & W5A5G5 & & 57.4 & 77.6 & 66.2 & 70.9 & 41.8 & 73.5 & 31.2 & 59.8 \\
\midrule
 LLaMA2-70B         & \multirow{2}{*}{W4A16G16} & &56.3 & 78.2 & 67.1 & 67.3 & 38.2 & 72.9 & 28.4 & 58.3 \\
 \textit{+QLoRA} &  & & \textit{61.8} & \textit{78.1} & \textit{68.4} & \textit{75.8} & \textit{43.6} & \textit{73.7} & \textit{32.8} & \textit{62.0} \\
\noalign{\vspace{0.1em}}\hdashline[0.8pt/1pt]\noalign{\vspace{0.1em}}
 \multirow{4}{*}{W/ GSQ-Tuning} & W7A7G7  & & 54.5 & 76.5 & 66.9 & 66.1 & 36.9 & 70.9 & 27.4 & 57.0 \\
 & W6A6G6 & & 57.4 & 77.6 & 66.2 & 70.9 & 41.8 & 73.5 & 31.2 & 59.8 \\
& W5A5G5 & & 57.4 & 77.6 & 66.2 & 70.9 & 41.8 & 73.5 & 31.2 & 59.8 \\
\midrule
 LLaMA3-3B         & \multirow{2}{*}{W4A16G16}& &56.3 & 78.2 & 67.1 & 67.3 & 38.2 & 72.9 & 28.4 & 58.3 \\
 \textit{+QLoRA} &  & & \textit{61.8} & \textit{78.1} & \textit{68.4} & \textit{75.8} & \textit{43.6} & \textit{73.7} & \textit{32.8} & \textit{62.0} \\
\noalign{\vspace{0.1em}}\hdashline[0.8pt/1pt]\noalign{\vspace{0.1em}}
 \multirow{4}{*}{W/ GSQ-Tuning} & W7A7G7  & & 54.5 & 76.5 & 66.9 & 66.1 & 36.9 & 70.9 & 27.4 & 57.0 \\
 & W6A6G6 & & 57.4 & 77.6 & 66.2 & 70.9 & 41.8 & 73.5 & 31.2 & 59.8 \\
& W5A5G5 & & 57.4 & 77.6 & 66.2 & 70.9 & 41.8 & 73.5 & 31.2 & 59.8 \\
\midrule
 LLaMA3-8B         & \multirow{2}{*}{W4A16G16} & &56.3 & 78.2 & 67.1 & 67.3 & 38.2 & 72.9 & 28.4 & 58.3 \\
 \textit{+QLoRA} &  & & \textit{61.8} & \textit{78.1} & \textit{68.4} & \textit{75.8} & \textit{43.6} & \textit{73.7} & \textit{32.8} & \textit{62.0} \\
\noalign{\vspace{0.1em}}\hdashline[0.8pt/1pt]\noalign{\vspace{0.1em}}
 \multirow{4}{*}{W/ GSQ-Tuning} & W7A7G7  & & 54.5 & 76.5 & 66.9 & 66.1 & 36.9 & 70.9 & 27.4 & 57.0 \\
 & W6A6G6 & & 57.4 & 77.6 & 66.2 & 70.9 & 41.8 & 73.5 & 31.2 & 59.8 \\
& W5A5G5 & & 57.4 & 77.6 & 66.2 & 70.9 & 41.8 & 73.5 & 31.2 & 59.8 \\
\bottomrule
\end{tabular}}}
\end{table*}

\subsection{Comparison with FP8 with 64 rank}
Here, we compare the designed GSE data format with FP8 in fully quantized fine-tuning framework with 32 rank setting. As shown in Tab.~\ref{tab:copare_fp8_r64}, the results still demonstrate that the designed GSE implemented in our GSQ-Tuning method achieves superior fine-tuning performance compared to FP8 while significantly reducing computation efficiency. Even under 5-bit settings, GSQ-Tuning maintains fine-tuning performance on par with FP8, validating its effectiveness.
\begin{table*}[!t]
\renewcommand\arraystretch{1.0}
\centering
\caption{$0$-shot accuracy comparison with FP8 in different quantization bits in 64 rank setting.}
\label{tab:copare_fp8_r64}
\setlength{\tabcolsep}{1.2mm}
{\resizebox{0.98\textwidth}{!}{
\begin{tabular}{lccccccccccccc|c}
\noalign{\vspace{0.3em}}
\toprule
\noalign{\vspace{0.1em}}
\textbf{Method} & LLMs branch &low-rank branch  &\textbf{Avg.} & \textbf{ARC-c} & \textbf{ARC-e} & \textbf{BoolQ} & \textbf{HellaS.} & \textbf{OBQA} & \textbf{PIQA} & \textbf{SCIQ.} & \textbf{WinoG.} & \textbf{Mem. (G)} \\
\midrule
\noalign{\vspace{0.1em}}
 LLaMA2-7B         &  16-16-16 & w/o & 64.13 &46.25 &74.62 &77.68 &76.01 &44.20 &79.11 &46.11 &69.06 &13.20\\
 \textit{w/ QLoRA} &  4-16-16 & 16-16-16 & 65.69 &47.14 &74.75 &79.50 &76.46 &45.50 &79.63 &50.26 &71.32 &9.73\\
\noalign{\vspace{0.1em}}\hdashline[0.8pt/1pt]\noalign{\vspace{0.1em}}
w/ FP8 & 4-8-8 & 8-8-8 & 64.46 &46.84 &73.61 &77.83 &76.03 &44.60 &79.65 &47.80 &69.38 &6.88\\
\noalign{\vspace{0.1em}}\hdashline[0.8pt/1pt]\noalign{\vspace{0.1em}}
\multirow{3}{*}{w/ GSQ-Tuning}  &4-8-8 & 8-8-8  & 65.60 &48.12 &74.24 &79.72 &76.00 &45.80 &79.60 &49.69 &71.67 &6.88\\
& 4-6-6 & 6-6-6  & 65.39 &47.70 &74.58 &79.24 &76.05 &44.60 &79.60 &50.41 &70.96 &6.17\\
  & 4-5-5 & 5-5-5 & 64.18 &45.14 &72.69 &75.20 &75.27 &46.40 &79.65 &48.62 &70.48 &5.81\\
\midrule
% \midrule
 LLaMA3-8B         & 16-16-16 & w/o & 67.18 &53.50 &77.74 &81.13 &79.20 &45.00 &80.63 &47.03 &73.24 &15.01\\
 \textit{w/ QLoRA} & 4-16-16 & 16-16-16 & 68.45 &55.63 &80.13 &83.67 &78.78 &44.80 &81.28 &50.41 &72.93 &11.71\\
\noalign{\vspace{0.1em}}\hdashline[0.8pt/1pt]\noalign{\vspace{0.1em}}
w/ FP8  & 4-8-8 & 8-8-8  & 66.46 &50.77 &76.39 &81.38 &78.19 &43.40 &79.92 &47.29 &74.35 &7.63\\
\noalign{\vspace{0.1em}}\hdashline[0.8pt/1pt]\noalign{\vspace{0.1em}}
 \multirow{3}{*}{w/ GSQ-Tuning} & 4-8-8 & 8-8-8  & 68.61 &55.97 &80.22 &83.61 &78.68 &45.20 &81.50 &50.41 &73.32 &7.63\\
 & 4-6-6 & 6-6-6 & 68.22 &55.55 &79.29 &83.67 &78.47 &44.80 &80.90 &50.05 &73.09 &6.86\\
& 4-5-5 & 5-5-5 & 66.69 &54.10 &77.99 &81.65 &77.12 &43.80 &79.54 &47.90 &71.43 &6.47\\
\bottomrule
\end{tabular}}}
\end{table*}

\end{document}
