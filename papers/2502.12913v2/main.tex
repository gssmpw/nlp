% This must be in the first 5 lines to tell arXiv to use pdfLaTeX, which is strongly recommended.
\pdfoutput=1
% In particular, the hyperref package requires pdfLaTeX in order to break URLs across lines.

\documentclass[11pt]{article}

% Change "review" to "final" to generate the final (sometimes called camera-ready) version.
% Change to "preprint" to generate a non-anonymous version with page numbers.
% \usepackage[review]{acl}
\usepackage[final]{acl}

% Standard package includes
\usepackage{times}
\usepackage{latexsym}

% For proper rendering and hyphenation of words containing Latin characters (including in bib files)
\usepackage[T1]{fontenc}
% For Vietnamese characters
% \usepackage[T5]{fontenc}
% See https://www.latex-project.org/help/documentation/encguide.pdf for other character sets

% This assumes your files are encoded as UTF8
\usepackage[utf8]{inputenc}

% This is not strictly necessary, and may be commented out,
% but it will improve the layout of the manuscript,
% and will typically save some space.
\usepackage{microtype}

% This is also not strictly necessary, and may be commented out.
% However, it will improve the aesthetics of text in
% the typewriter font.
\usepackage{inconsolata}

%Including images in your LaTeX document requires adding
%additional package(s)
\usepackage{microtype}
\usepackage{graphicx}
\usepackage{subfigure}
\usepackage{booktabs} % for professional tables
% hyperref makes hyperlinks in the resulting PDF.
% If your build breaks (sometimes temporarily if a hyperlink spans a page)
% please comment out the following usepackage line and replace
% \usepackage{icml2025} with \usepackage[nohyperref]{icml2025} above.
\usepackage{hyperref}
% Attempt to make hyperref and algorithmic work together better:
\newcommand{\theHalgorithm}{\arabic{algorithm}}
% For theorems and such
\usepackage{amsmath}
\usepackage{mathrsfs}
\usepackage{amssymb}
\usepackage{mathtools}
\usepackage{amsthm}
\usepackage{url}            % simple URL typesetting
\usepackage{bm}
\usepackage{booktabs}       % professional-quality tables
\usepackage{amsfonts}       % blackboard math symbols
\usepackage{nicefrac}       % compact symbols for 1/2, etc.
\usepackage{microtype}      % microtypography
\usepackage{adjustbox}
\usepackage{xcolor}         % colors
\usepackage{blindtext}
\usepackage{enumitem}
\usepackage{listings}
\usepackage{pifont}
\usepackage{epsfig} % for postscript graphics files
\usepackage{times}
\usepackage{color}
% \usepackage[table]{xcolor}
\usepackage{multirow}
\usepackage{multicol}
\usepackage{makecell}
\usepackage{algorithm}
\usepackage{rotating}
\usepackage{subfigure}
\usepackage[T1]{fontenc}
\usepackage{algorithm}
\usepackage{algorithmic}
% \usepackage[noend]{algpseudocode}
\usepackage{colortbl}
\usepackage{color, soul}
\usepackage{arydshln} % For dashline
\usepackage{tcolorbox}
\definecolor{babyblue}{rgb}{0.54, 0.81, 0.94}
\definecolor{bisque}{rgb}{1.0, 0.89, 0.77}
\definecolor{bshade}{rgb}{0.55,0.75,0.95}
\def\eg{{\it e.g.}\xspace}
\def\ie{{\it i.e.}\xspace}
\definecolor{mygray}{gray}{.6}
\definecolor{myblue}{RGB}{89,158,254}
\definecolor{mygreen1}{RGB}{81,150,111}
\definecolor{mygreen2}{RGB}{93,174,86}
\definecolor{myred}{RGB}{160,0,0}
\definecolor{myyellow}{RGB}{227,207,87}
% For pictures
\usepackage{caption}
\usepackage[capitalize,noabbrev]{cleveref}
\usepackage{threeparttable}
\usepackage{wrapfig}

%%%%%%%%%%%%%%%%%%%%%%%%%%%%%%%%
% THEOREMS
%%%%%%%%%%%%%%%%%%%%%%%%%%%%%%%%
\theoremstyle{plain}
\newtheorem{theorem}{Theorem}[section]
\newtheorem{proposition}[theorem]{Proposition}
\newtheorem{lemma}[theorem]{Lemma}
\newtheorem{corollary}[theorem]{Corollary}
\theoremstyle{definition}
\newtheorem{definition}[theorem]{Definition}
\newtheorem{assumption}[theorem]{Assumption}
\theoremstyle{remark}
\newtheorem{remark}[theorem]{Remark}
\definecolor{babyblue}{rgb}{0.54, 0.81, 0.94}
\definecolor{bisque}{rgb}{1.0, 0.89, 0.77}
\definecolor{bshade}{rgb}{0.55,0.75,0.95}
\def\eg{{\it e.g.}\xspace}
\def\ie{{\it i.e.}\xspace}
\definecolor{mygray}{gray}{.6}
\definecolor{myblue}{RGB}{89,158,254}
\definecolor{mygreen1}{RGB}{81,150,111}
\definecolor{mygreen2}{RGB}{93,174,86}
\definecolor{myred}{RGB}{160,0,0}
\definecolor{myyellow}{RGB}{227,207,87}
\usepackage{bbding}
\newcommand{\cmark}{\ding{52}}%
\newcommand{\xmark}{\ding{55}}%
\usepackage{enumitem}
\let\oldding\ding% Store old \ding in \oldding
\renewcommand{\ding}[2][1]{\scalebox{#1}{\oldding{#2}}}
% Todonotes is useful during development; simply uncomment the next line
%    and comment out the line below the next line to turn off comments
%\usepackage[disable,textsize=tiny]{todonotes}
\usepackage[textsize=tiny]{todonotes}
\newcommand{\ceil}[1]{\left\lceil #1 \right\rceil}
\newcommand{\floor}[1]{\left\lfloor #1 \right\rfloor}
\newcommand{\round}[1]{\left\lfloor #1 \right\rceil}
\newcommand{\sign}{\mathrm{sign}}
\newcommand{\tofloat}[1]{\mbox{float}\left(#1\right)}
\newcommand{\toint}[2]{\mbox{int}_{#1}\left(#2\right)}


\newcommand{\methodname}{GSQ-Tuning}
\newcommand{\dataname}{GSE}
\newcommand{\syz}[1]{\textcolor{teal}{SYZ:#1}}
% SamJ's commands
% \usepackage{xargs} % Used for new commands with optional arguments
\usepackage{soul}  % Used for custom comments
\usepackage{color} % Used for custom colors in comments
% \usepackage{comment}
\usepackage{xspace}
\usepackage{listings}
\usepackage{enumitem}
% \usepackage{ulem}
%% Note: Some commands for spacing Latin letters/abbreviations
\newcommand{\eg}{{\it e.g.,\ }}
\newcommand{\etal}{{\it et al.\ }}
\newcommand{\etc}{{\it etc.}}
\newcommand{\ie}{{\it i.e.,\ }}
\newcommand{\cf}{{c.f.}\xspace}
\newcommand{\aka}{{a.k.a.}\xspace}

%%%%%%%%%%%%%%%%%%%%%%%%%%%%%%%%%%%%%%%%%%%%%%%%
%% Commands for adding comments to the paper. %%
%%%%%%%%%%%%%%%%%%%%%%%%%%%%%%%%%%%%%%%%%%%%%%%%

\usepackage{booktabs}
\definecolor{oxfordblue}{rgb}{0.0, 0.13, 0.28}
\definecolor{harvardcrimson}{rgb}{0.79, 0.0, 0.09}
\definecolor{dartmouthgreen}{rgb}{0.05, 0.5, 0.06}
\definecolor{princetonorange}{rgb}{1.0, 0.56, 0.0}
\definecolor{yaleblue}{rgb}{0.06, 0.3, 0.57}
\definecolor{usccardinal}{rgb}{0.6, 0.0, 0.0}
\definecolor{uclablue}{rgb}{0.33, 0.41, 0.58}
\definecolor{msugreen}{rgb}{0.09, 0.27, 0.23}
\definecolor{cornellred}{rgb}{0.7, 0.11, 0.11}
\definecolor{pomegranate}{RGB}{192, 57, 43}
\definecolor{anti-pomegranate}{RGB}{43,178,192}
\definecolor{alizarin}{RGB}{231, 76, 60}
\definecolor{anti-belize}{RGB}{185, 41, 56}
\definecolor{belize}{RGB}{41, 128, 185}
\definecolor{peter}{RGB}{52, 152, 219}
\definecolor{green}{RGB}{22, 160, 133}
\definecolor{anti-green}{RGB}{160,22,118}
\definecolor{turquoise}{RGB}{26, 188, 156}
\definecolor{pumpkin}{RGB}{211, 84, 0}
\definecolor{anti-pumpkin}{RGB}{0,22,211}
\definecolor{carrot}{RGB}{230, 126, 34}
\definecolor{wisteria}{RGB}{142, 68, 173}
\definecolor{anti-wisteria}{RGB}{99,173,68}
\definecolor{amethyst}{RGB}{155, 89, 182}
\definecolor{nephritis}{RGB}{39, 174, 96}
\definecolor{anti-nephritis}{RGB}{174,39,117}

% \newcommand{\pzh}[1]{{#1}}
% \newcommand{\peng}[1]{{\color{red} #1}}
% \newcommand{\xingbo}[1]{{\textcolor{black}{#1}}}
% \newcommand{\wxb}[1]{{\textcolor{orange}{#1}}}
% \newcommand{\chen}[1]{{\color{green} #1}}

\newcommand{\penguin}[1]{{#1}}
\newcommand{\pzh}[1]{{#1}}
\newcommand{\peng}[1]{{#1}}
\newcommand{\zhenhui}[1]{{#1}}
\newcommand{\haoxiang}[1]{{#1}}
\newcommand{\yh}[1]{{#1}}
\newcommand{\fhx}[1]{{{#1}}}

% \newcommand{\penguin}[1]{{\color{red} #1}}
% \newcommand{\fhx}[1]{{\textcolor{orange}{#1}}}
% \newcommand{\zhenhui}[1]{{\color{carrot} #1}}

% \newcommand{\penguin}[1]{{\color{blue} #1}}
% \newcommand{\fhx}[1]{{\color{blue}{#1}}}
% \newcommand{\zhenhui}[1]{{\color{blue} #1}}

% \newcommand{\fanhx}[1]{{\color{blue}{#1}}}
\newcommand{\fanhx}[1]{{{#1}}}

%% Note: Comment this in to see all comments and unfinished text.
\newcommand{\todo}[1]{\textcolor{red}{[TODO] \emph{#1}}}
\newcommand{\cut}[1]{\textcolor{red}{\st{#1}}}
\newcommand{\sout}[1]{\cut{#1}}
\newcommand{\gray}[1]{\textcolor{gray}{#1}}


\newcommand{\systemname}{{\textit{SystemName}}}
\newcommand{\name}{{\textit{LitLinker}}}
    
% Capitalizing the first letter for section autorefs
\renewcommand{\sectionautorefname}{Section}
\renewcommand{\subsectionautorefname}{Section}
\renewcommand{\subsubsectionautorefname}{Section}




% If the title and author information does not fit in the area allocated, uncomment the following
%
%\setlength\titlebox{<dim>}
%
% and set <dim> to something 5cm or larger.

\title{GSQ-Tuning: \underline{G}roup-\underline{S}hared Exponents Integer in \\ Fully \underline{Q}uantized Training for LLMs On-Device Fine-tuning}


\author{
 \textbf{Sifan Zhou\textsuperscript{1,2$\dagger$$\ddagger$}},
 \textbf{Shuo Wang\textsuperscript{2$\dagger$}},
 \textbf{Zhihang Yuan\textsuperscript{2$\dagger$}},
 \textbf{Mingjia Shi\textsuperscript{2}},
 \textbf{Yuzhang Shang\textsuperscript{3}},
 \textbf{Dawei Yang\textsuperscript{2}}
\\
\\
\\
 \textsuperscript{1}Southeast University,
 \textsuperscript{2}Houmo AI,
 \textsuperscript{3}Illinois Institute of Technology
\\
 % \small{
   % \textbf{Correspondence:} \href{mailto:email@domain}{email@domain}
 % }
}

\begin{document}
\maketitle
\def\thefootnote{$\dagger$}\footnotetext{Equal contribution}
\def\thefootnote{$\ddagger$}\footnotetext{Work down as an intern at Houmo AI}
% \def\thefootnote{$\Diamond$}\footnotetext{Corresponding author}
\begin{abstract}

In this work, we tackle the challenge of disambiguating queries in retrieval-augmented generation (RAG) to diverse yet answerable interpretations.
State-of-the-arts follow a Diversify-then-Verify (DtV) pipeline, where diverse interpretations are generated by an LLM,
later used as search queries to retrieve supporting passages.
Such a process
may introduce noise in either interpretations or retrieval,
particularly in enterprise settings, where LLMs---trained on static data---may struggle with domain-specific disambiguations.
Thus, a post-hoc verification phase is introduced to prune noises.
Our distinction is \textbf{to unify diversification with verification} by incorporating feedback from retriever and generator early on.
This joint approach improves both efficiency and robustness by reducing reliance on multiple retrieval and inference steps, which are susceptible to cascading errors.
We validate the efficiency and effectiveness of our method, \ourslong (\ours), on the widely adopted ASQA benchmark
to achieve diverse yet verifiable interpretations.
Empirical results show that \ours improves grounding-aware $\textrm{F}_1$ score by an average of 23\% over the strongest baseline across different backbone LLMs.
\end{abstract}

\section{Introduction}
\label{sec:intro}
% Image editing methods in diffusion models depend on user-defined control directions - users can unlock their creativity using these methods by specifying the desired manipulation through prompts~\cite{gandikota2023concept}, reference images~\cite{ruiz2022dreambooth, kumari2022customdiffusion, gal2022image, chen2024trainingfreeregionalpromptingdiffusion}, or attribute vectors~\cite{parmar2023zero,hertz2022prompt}. In this work, we ask a fundamentally different question: \emph{Can we automatically discover the underlying visual structure of a concept within diffusion model's knowledge?} %Rather than requiring user-specified controls, we aim to decompose the model's internal knowledge into meaningful directions.

% This question touches on a fundamental limitation in how we interact with diffusion models. Current control methods ~\cite{zhang2023addingconditionalcontroltexttoimage, gandikota2023concept, ye2023ipadaptertextcompatibleimage,ye2023ipadaptertextcompatibleimage, hertz2024stylealignedimagegeneration, li2023photomaker, shi2024instantbooth, chen2024trainingfreeregionalpromptingdiffusion} require users to specify their desired manipulations in advance, limiting interactive creativity. This contrasts with natural human artistic workflows, where creators dynamically explore creative ideas while jointly refining them toward meaningful artistic outcomes~\cite{hoffmann2016modeling}. This synergy between specification and exploration is not new to generative models. Early GAN architectures naturally developed disentangled latent spaces that enabled continuous\cite{harkonen2020ganspace,radford2015unsupervised, wu2021stylespace, shen2020interfacegan}, compositional control over generated images. Users could explore these spaces to discover interesting variations that would be difficult to describe in words~\cite{wu2021stylespace}, then combine them to achieve their creative goals~\cite{grabe2022towards}. 


% While diffusion models have largely superseded GANs in conditional image synthesis~\cite{dhariwal2021diffusion},  their underlying structure remains less understood. Diffusion models achieve remarkable diversity through high-dimensional latents, unlike GANs' compact latent spaces.  With a single prompt, diffusion models can generate radically different variations through different random initializations of input noise. We ask - Is it possible to discover interpretable structure within this vast space of variations?

Text-to-image diffusion models are capable of generating remarkable visual variations from a single prompt through different random initializations. However, this vast creative potential remains largely opaque to users---while we can generate diverse images, we lack understanding of the underlying structure of these variations. This presents a fundamental challenge: how can we discover and expose the latent visual capabilities encoded within these models?

\let\thefootnote\relax \footnote{$^{*}$Correspondence to \texttt{gandikota.ro@northeastern.edu}}

The challenge touches on a key limitation in how we interact with diffusion models today. Current control methods require users to explicitly specify their desired edits in advance through prompts~\cite{gandikota2023concept}, reference images~\cite{zhang2023addingconditionalcontroltexttoimage, chen2024trainingfreeregionalpromptingdiffusion, ruiz2022dreambooth,kumari2022customdiffusion, Ryu_lora, hu2021lora}, or attribute vectors~\cite{ye2023ipadaptertextcompatibleimage, hertz2024stylealignedimagegeneration, li2023photomaker, shi2024instantbooth,parmar2023zero,hertz2022prompt}. That contrasts sharply with natural human creative workflows, where artists dynamically explore creative ideas and jointly refine them toward meaningful artistic outcomes~\cite{hoffmann2016modeling}. The need for pre-specified controls creates a barrier between users and the full creative potential of these models.

Interestingly, earlier generative models like GANs~\cite{gans,karras2019style,brock2018large} naturally developed more interpretable internal structures. Their compact latent spaces often exhibited emergent disentanglement~\cite{harkonen2020ganspace,radford2015unsupervised, wu2021stylespace, shen2020interfacegan}, enabling continuous and compositional control over generated images. Users could explore these spaces to discover interesting variations that would be difficult to describe in words~\cite{wu2021stylespace}, then combine them to achieve their creative goals~\cite{grabe2022towards}.

Diffusion models have largely superseded GANs in conditional image synthesis~\cite{dhariwal2021diffusion}, achieving greater diversity through much higher-dimensional latents. And yet an understanding of the underlying structure of these larger latent spaces has remained elusive. In this work, we ask a fundamental question: \emph{Can we automatically discover the visual structure within a diffusion model's knowledge of a concept?} Rather than requiring user-specified controls, we aim to decompose the model's internal representations into expressive directions that users can explore and combine.

To address these needs, we present \textbf{SliderSpace}, a framework that brings systematic explorability to diffusion models. Given just a text prompt, SliderSpace discovers a canonical set of meaningful, diverse, and controllable directions within the model's knowledge of that concept. Each direction is implemented as a low-rank adapter~\cite{hu2021lora} that can be scaled and composed with others, allowing users to explore and smoothly combine different aspects of variation, as shown in Figure~\ref{fig:intro}.

We ground SliderSpace discovery in three key requirements for meaningful decomposition of a diffusion model's visual manifold: 
\begin{enumerate}
    \item \textbf{Unsupervised Discovery:} The decomposition process should emerge from the intrinsic structure of the model's learned representation, rather than being guided by predefined attributes. This ensures we capture the true topology of the model's knowledge space rather than projecting our assumptions onto it.
    
    \item \textbf{Semantic Orthogonality:} Each discovered control must represent a distinct semantic direction. This is enforced in a semantic feature space, like CLIP, where every slider has an orthogonal effect in embeddings. This prevents discovering multiple controls that create similar semantic effects, making the system more efficient and easier.
    
    \item \textbf{Distribution Consistency:} Directions must induce consistent transformations across both random seeds and prompt variations. 
\end{enumerate}

These requirements naturally lead to our proposed framework, which we formalize in Section~\ref{sec:method}. As we show in our experiments, SliderSpace is architecture-agnostic, working with both conventional U-Net based models like Stable Diffusion~\cite{rombach2022high, rombach2022sd20, podell2023sdxl, turbo, dmd} and recent transformer-based architectures like Flux~\cite{flux}.

We demonstrate the expressiveness of SliderSpace through three applications: First, we show how SliderSpace can decompose high-level concepts into diverse and expressive components, revealing the natural axes of variation in the model's understanding. Second, we explore artistic style variation, where SliderSpace discovers directions that match or exceed the diversity of manually curated artist lists while being judged more useful by human evaluators. Finally, we show how SliderSpace can help reverse the mode collapse commonly observed in distilled diffusion models, restoring diversity while maintaining generation speed.

Beyond providing practical creative control, SliderSpace opens new avenues for understanding and utilizing the latent capabilities of diffusion models. By mapping these models' visual potential into intuitive, composable directions, we take a step toward making their creative possibilities more accessible and interpretable to users.

% Image editing methods in diffusion models unlock the creativity of users. In this work we ask an alternate question: \emph{Can we organize and expose what of the diffusion model is already capable of?}.
% Existing methods for controlling image generation typically require users to manually specify edit directions for desired changes. This process is time-consuming, requires technical expertise, and limits the spontaneity of the creative process. For instance, if a user wants to adjust the smile of a generated person, they must explicitly request this edit, often through imprecise prompt engineering or model fine-tuning. This approach of predefined controls or manual specifications restricts users from fully exploring the latent capabilities of the model. There may be interesting stylistic variations or attributes that the model can generate, but users have no easy way to discover or utilize these.

% Natural visual disentanglement was an emergent property in the latent space of Generative Adversarial Models (GANs) \cite{harkonen2020ganspace,radford2015unsupervised, wu2021stylespace, shen2020interfacegan}. In particular, it has been observed that StyleGAN~\cite{karras2019style} stylespace neurons offer detailed control over many meaningful aspects of images that would be difficult to describe in words~\cite{wu2021stylespace}. However, diffusion models do not share such a compact latent space~\cite{park2023unsupervised}; and efforts to uncover such a space in the semantic embeddings of the text conditioning have met with limited success \nik{Nick - is there a specific citation you were thinking about?}.

% In this work we introduce \textbf{SliderSpace}, which takes a step towards uncovering an analogous low dimensional representation of diffusion models' visual breadth; in essence treating the diffusion model as many generators sharing parameters, where a particular generator is defined by a specific prompt. For a given prompt we sample many random seeds (and optionally prompt expansions using an LLM), generate the corresponding images, and apply an off the shelf feature extractor (in this work CLIP, but our method can be applied to any differentiable feature extractor). We use PCA to analyze these features, and for each of the leading $k$ principal components we train a LoRA \cite{} which causes the diffusion model to produces images which increase the feature magnitude along that component when passed back through the same feature extractor. This leads to a 'Slider' for each principal component, because each LoRA can be scaled and applied to the original diffusion model, continuously varying those visual features in the generated results (as measured, in our case, by CLIP).

% There are many other works that enhance the controllability of diffusion models. One common approach is enabling users to add spatial constraints to a generation either manually, or via a reference image \cite{zhang2023addingconditionalcontroltexttoimage, chen2024trainingfreeregionalpromptingdiffusion}, a second is leveraging more abstract embeddings (e.g. identity, style) extracted from a reference image \cite{ye2023ipadaptertextcompatibleimage, hertz2024stylealignedimagegeneration, li2023photomaker, shi2024instantbooth}, a third is finetuning a foundation model to better generate a concept important to the user \cite{ruiz2022dreambooth, kumari2022customdiffusion, Ryu_lora, hu2021lora}, and a fourth (most relevant to this work) is finding low-rank adaptors of the model based on a prompt or small training set which can be scaled to provide continous control over one aspect of generated image (e.g. night vs day, basic vs luxury, etc.) \cite{gandikota2023concept}. SliderSpace is complementary to all of these methods and offers something distinct. All of the other methods we are aware require the user (and / or model designer) to know in advance what type of control they want. In contrast SliderSpace assists users in discovering and controlling hidden capabilities present in the diffusion model's distribution of possible generations.

%We propose that truly intuitive creative control in a text-to-image model should meet three key criteria: \emph{discoverability}, \emph{intuitiveness}, and \emph{specificity}. The model should reveal controllable attributes that may not be immediately obvious, offer controls that are easy to understand and manipulate, and ensure each control affects a distinct attribute of the generated image.

% We demonstrate the utility and power of SliderSpace using three applications built on top of SDXL-DMD \cite{dmd}, because its fast generation speed lends itself well to the continuous control offered by SliderSpace.

% First, we study concept decomposition (Section \ref{sec:concept_exp}), where we learn sliders for a specific concept (e.g. 'monster', 'waterfall', 'car'). Through quantitative metrics of diversity and text alignment we demonstrate that the learned sliders dramatically boost the diversity of generations when randomly applied without harming text alignment; we also ask humans to qualitatively judge these results in a user study where they find the SliderSpace results to be more 'Diverse', 'Useful', and 'Creative' than our baselines.

% Second, we attempt to compare the automatic discoveries of SliderSpace to a large scale manual study of artistic styles (Section \ref{sec:art_exp}), open-sourced by ParrotZone \cite{parrotzone}. In this study SDXL was prompted with over 4300 artist names,  and based on visual inspection the cases of successful stylistic mimicry recorded. Quantitatively SliderSpace more closely matches the distribution of artistic variation discovered by ParrotZone than other baselines, and in our user studies was judged to be significantly more 'Diverse' and 'Useful' than the baselines. To our surprise humans even judged SliderSpace results to be slightly more 'Diverse' than the results generated by the manually discovered artist names of \cite{parrotzone}.

% Third, we attempt to use SliderSpace to reverse the mode collapse commonly observed in distilled few-step diffusion models relative to the original teacher model (Section \ref{sec:diverse_exp}). We quantitatively demonstrate that applying SliderSpace to SDXL-DMD leads to more closely matching the distribution of images by the original teacher, SDXL.

%Through extensive experiments on various state-of-the-art text-to-image models, we demonstrate that SliderSpace significantly enhances user control and creative expression in AI-assisted image generation tasks. Our method enables a range of applications, including concept decomposition and control, diversity improvement in generated images, customization dissection and edits, and the exploration of artistic styles inherent in the model.

% SliderSpace goes beyond providing a practical tool for enhanced creative control. By mapping the visual potential of diffusion models it can open new avenues for generative creativity and deepens our understanding of each model's hidden potential.


\section{Methodology}
\paragraph{Preliminaries.}
We primarily focus on the homologous model merging, in which $\boldsymbol{\theta}_i$ all come from the same base model $\boldsymbol{\theta}_{\rm{base}}$. Given $K$ tasks $\{T_1,T_2,\cdots,T_K\}$ and $K$ corresponding fine-tuned models with parameters $\{\boldsymbol{\theta}_1,\boldsymbol{\theta}_2,\cdots,\boldsymbol{\theta}_K\}$, model merging aims to combine $K$ fine-tuned models into one single model simultaneously performing on $\{T_1,T_2,\cdots,T_K\}$ without post-training~\cite{method_p1_1,method_p1_2}.
Task vector~\cite{ilharco2023editing,yang2024adamerging} is a key element in merging method which could enhances the base model‘s ability or enable the model to handle other tasks. Specifically, for task $T_i$, the task vector $\boldsymbol\tau_i\in \mathbb{R}^D$ is defined as the vector obtained by subtracting the SFT weights $\boldsymbol{\theta}_i$ from the base model weight
$\boldsymbol{\theta}_{\rm{base}}$, \emph{i.e.}, $\boldsymbol\tau_i=\boldsymbol{\theta}_i-\boldsymbol{\theta}_{\rm{base}}$. The merged model could be denoted as $\boldsymbol{\theta}_m=\boldsymbol{\theta}_{\rm{base}}+\sum_i \lambda_i\boldsymbol{\tau}_i$, which $\lambda_i$ is the scaling factor measuring the importance of task vector. For clarification, we also denote the neuron set in $\boldsymbol{\theta}_i$ as $\mathcal{N}_i$, the neuron set in $\boldsymbol{\tau}_i$ as $\mathcal{T}_i$.



\begin{algorithm}[!ht]
    \caption{LED-Merging}
    \label{alg1}
    \begin{algorithmic}[1]
        \REQUIRE  base model $\boldsymbol{\theta}_{\rm{base}}$, SFT models $\{\boldsymbol{\theta}_{i}\mid i\in [K]\}$, mask ratios \{$r_{i} \mid i\in [K]\}$, scaling factors $\{\lambda_i\mid i\in[K]\}$, location datasets $\{\mathcal{X}_{i}\mid i\in[K]\}$
        \ENSURE merged parameter $\boldsymbol{\theta}_{m}$
        \STATE $\mathcal{M}\leftarrow\phi$
        \STATE $\boldsymbol{\theta}_{m}\leftarrow \boldsymbol{\theta}_{\rm{base}}$
        \FOR{$i\in [K]$}
        \STATE $I(\boldsymbol{\theta}_i)=\mathbb{E}_{x\sim \mathcal{X}_i}|\boldsymbol{\theta}_{i}\odot \nabla_{\boldsymbol{\theta}_i}\mathcal{L}(x)|$
        \STATE $I(\boldsymbol{\theta}_{\rm{base}})=\mathbb{E}_{x\sim \mathcal{X}_i}|\boldsymbol{\theta}_{\rm{base}}\odot \nabla_{\boldsymbol{\theta}_{\rm{base}}}\mathcal{L}(x)|$
        
        \STATE calculate $\mathcal{T}^{r_i}_{i}$ following Equation \ref{vote}
        \STATE  $\mathcal{M}\leftarrow \mathcal{M}\cup\{\mathcal{T}^{r_i}_i\}$
       
        
   
        
        
        \ENDFOR  
        \FOR{$i\in [K]$}
        
        \STATE calculate $\text{Disjoint}(\mathcal{T}_i^{r_i})$ use Equation~\ref{disjoint_safety}
        \STATE $\boldsymbol{m}_i \leftarrow \boldsymbol{0}$
        \FOR{$d\in \mathcal{T}_i^{r_i}$}
        \STATE $\boldsymbol{m}_{i,d}=1$
        \ENDFOR
        \STATE $\boldsymbol{\theta}_{m}\leftarrow \boldsymbol{\theta}_{m}+\lambda_i \boldsymbol{\tau}_i\odot \boldsymbol{m}_{i}$
        \ENDFOR
    \end{algorithmic}
\end{algorithm}
    %\vspace{-5pt}
\begin{figure*}[h!]
    \centering
    \includegraphics[width=\linewidth]{figs/pipeline_v2.pdf}
    \vspace{-40mm}
    \caption{Overview of our two-stage training pipeline {\ours}.}
    \label{fig:pipeline}
\end{figure*}


\paragraph{LED-Merging: Location, Election, and Disjoint Merging}
To address the neuron misidentification and interference issues in existing model merging methods, we propose LED-Merging (Location, Election, and Disjoint Merging). Specifically, previous studies \cite{modelstock, ilharco2023editing, tiesmerging} fail to accurately identify safety-related neurons in task vectors with a single magnitude score, namely \textit{neuron misidentification}. Meanwhile, there exists an interference between safety-related and utility-related task vector neurons during the merging process, namely \textit{neuron interference}. To address neuron misidentification, we first locate important neurons both in the base and fine-tuned models and then elect neurons from the task vector considering these two scores together. Subsequently, to mitigate the interference, we introduce a disjoint step, isolating these important neurons so that they influence different base neurons. The whole process is illustrated in Figure~\ref{fig:method}. 




In the location and election step, we consider the importance score from base and fine-tuned models simultaneously to locate task-specific neurons. In this way, it is more accurate than relying on the magnitude score alone because task-specific neurons with high importance score in the fine-tuned model may not necessarily score high in the base model, and vice versa.

{\textbf{Location}}.  We first calculate importance scores for each neuron in a base/fine-tuned model. Given a location dataset $\mathcal{X}_i=\{(x,y)_k\}$, where $x$ is the question and $y$ is the answer, we calculate the importance scores for the weight $\boldsymbol{\theta}_i\in\mathbb{R}^D$ in any  layer as follows~\cite{snip,spareseGPT,sun2024a}:
\begin{equation}
    I(\boldsymbol{\theta}_i)=\mathbb{E}_{x\sim \mathcal{X}_i}[\boldsymbol{\theta}_i\odot \nabla _{\boldsymbol{\theta}_i}\mathcal{L}(x)],
    \label{location}
\end{equation}
which $\mathcal{L}(x)=-\log p(y\mid x)$ is the conditional negative log-likelihood loss. We choose the SNIP score~\cite{snip} because it balances computational efficiency and performance~\cite{cq}. Please refer to Sec.~\ref{sec:ablation} for the comparison between different location methods. After computing importance scores, we choose top-$r_i$ neurons as the important neuron subset $\mathcal{N}_{i}^{r_i}$ from $I(\boldsymbol{\theta}_i)$.
 
 % After computing locating scores, we select the neurons scoring both high in base and fine-tuned models as important neurons in task vectors. Then in the disjoint step,  with preventing  polysemantic neurons  from receiving gradient updates towards different directions,
 % we use set difference to isolate the safety   and utility-related neurons  and construct corresponding masks for merging process,

{\textbf{Election}}. A natural question is how to select important neurons in the task vector $\boldsymbol{\tau}_i$ based on $I(\boldsymbol{\theta}_{\rm{base}})$ and $I(\boldsymbol{\theta}_{i})$. The important neurons in the base model may be different from neurons in the fine-tuned model. Therefore, we introduce the following election strategy to select neurons with high scores in both base and fine-tuned models:
\begin{equation}
    \mathcal{T}_i^{r_i}=\mathcal{N}_i^{r_i}\cap \mathcal{N}_{\rm{base}}^{r_i}.
    \label{vote}
\end{equation}
\emph{Remark}. We compare different choosing methods, including scoring low or high in base or fine-tuned model in Section~\ref{sec:ablation} and find that Equation \ref{vote} achieves the best performance.





{\textbf{Disjoint}}. As important neurons from different task vectors may conflict with each other at the same position, we use the set difference to disjoint the neurons from others to prevent interference:
\begin{equation}
    \text{Disjoint}(\mathcal{T}^{r_i}_{i})=\mathcal{T}^{r_i}_{i}-\mathop{\cup}\limits_{{J}\subsetneqq [K],|J|\geq 2}\mathop{\cap}\limits_{j\in {J}}\mathcal{T}^{r_j}_{j}.
    \label{disjoint_safety}
\end{equation}

Next, we construct a mask $\boldsymbol{m}_i\in\mathbb{R}^D$ to implement disjoint in the merging process. Specifically, this mask $\boldsymbol{m}_i$ is used to select neurons from $\mathcal{T}_i$. The mask ratio is $r_i$, where $r\in(0,1]$. The mask $\boldsymbol{m}_i$ can be derived from:
\begin{equation}
    \boldsymbol{m}_{i,d}=\begin{aligned} &\left\{ \begin{array}{ll} 1, & \text{if } d\in \text{Disjoint}(\mathcal{T}_{i}^{r_i}), \\ 0, & \text{otherwise}. \end{array} \right. \end{aligned}
    \label{mask_safety}
\end{equation}


% \subsection{Merging Models with Masks}
{\textbf{Merging}}. The final
merged task vector $\boldsymbol{\tau}_m$ is as follows:
\begin{equation}
    \boldsymbol{\tau}_m= \sum_i \lambda_i\boldsymbol{\tau}_{i}\odot\boldsymbol{m}_i.
    \label{merged_task_vector}
\end{equation}
We summarize the workflow in Algorithm \ref{alg1}.



\section{Experiments}
\subsection{Implementation Details}
\textbf{Datasets.} We train our model on the CelebV-HQ \cite{zhu2022celebv} and VFHQ \cite{xie2022vfhq} datasets. Since the backbone of SVD \cite{blattmann2023stable} is sensitive to video quality, we first evaluate each video in two datasets with the video quality assessment method FasterVQA \cite{wu2023neighbourhood}, and remove videos with scores lower than 0.6. In the end, 37,644 videos remain for training. To ensure a fair comparison in experiments, we evaluate our method on the portrait video dataset HDTF \cite{zhang2021flow} and FFHQ \cite{karras2019style}.

\noindent\textbf{Training Details.} During the training phase, for the temporal attention layers of the SVD, we sample 16-frame video sequences to establish temporal consistency, with each frame at a resolution of $512\times512$. Unlike methods such as \cite{hu2024animate,ma2024follow}, which require two separate training stages, we update all the weights of both the SVD and two adapters simultaneously. The model is trained for 30,000 steps with a batch size of 8 using gradient accumulation, optimized by 8bit-Adam \cite{kingma2014adam} with a learning rate of $1\times10^{-5}$. 
\subsection{Metrics and Comparisons}
\textbf{Evaluation Metrics.} To evaluate the performance of our method, following \cite{cai2024real}, we relight the first 100 frames of each video in the HDTF dataset. Each video is rendered with four distinct lighting conditions derived from four different lighting-effect reference faces, resulting in a total of 44,000 frames for comprehensive comparison. Following \cite{nerffacelighting}, we use an off-the-shelf estimator \cite{feng2021learning} to calculate the Lighting Error (LE). Arcface \cite{deng2019arcface} is used to measure Identity Preservation (ID) between the relit results and the original images. To assess temporal consistency, we compute LPIPS \cite{zhang2018perceptual} between adjacent frames. We further employ an image quality assessment model \cite{pyiqa} and a video quality assessment model \cite{wu2023neighbourhood} to evaluate Image Quality (IQ) and Video Quality (VQ), respectively. Additionally, Fréchet Inception Distance (FID) \cite{heusel2017gans} and Fréchet Video Distance (FVD) \cite{skorokhodov2022stylegan} are used to measure video fidelity. In addition to objective evaluation, we conduct a user study in which 17 participants rate the videos based on three criteria: Lighting Accuracy (LA-User), Identity Similarity (ID-User), and Video Quality (VQ-User). Each criterion is rated on a scale of 1 to 5: poor, fair, average, good, and excellent. Finally, we calculate the average score for each criterion across participants.

\noindent\textbf{Comparative Methods. }For the portrait relighting task, we conduct a comparative analysis between LCVD and five state-of-the-art portrait relighting methods: DPR \cite{zhou2019deep}, SMFR \cite{hou2021towards}, NFL \cite{nerffacelighting}, StyleFlow \cite{10.1145/3447648}, and DiFaReli \cite{ponglertnapakorn2023difareli}, evaluating performance on both the HDTF and FFHQ datasets. For the portrait animation task, we compare LCVD with three state-of-the-art portrait animation methods: DaGAN \cite{hong2022depth}, StyleHEAT \cite{yin2022styleheat}, and AnimateAnyone \cite{hu2024animate}, using the HDTF dataset for evaluation.
\begin{table*}[t!]
    \centering
    \caption{Quantitative comparison of portrait relighting with DPR, SMFR, NFL, StyleFlow, and DiFaReli based on objective evaluation and user study on the HDTF video dataset. The best scores are highlighted in bold, and the second-best are underlined.}
    \vspace{-2mm}
    \label{tab:compare}
    \scalebox{1.0}
    {\begin{tabular}{cccccccc||ccc}
        \hline
        &\multicolumn{7}{c}{Objective Evaluation}&\multicolumn{3}{c}{User Study} \\
        \cmidrule(r){2-8} \cmidrule(r){9-11}
        Methods & LE$\downarrow$ & ID$\uparrow$ & LPIPS$\downarrow$ & IQ$\uparrow$ & VQ$\uparrow$ & FID$\downarrow$ & FVD$\downarrow$ & LA-User$\uparrow$ & ID-User$\uparrow$ & VQ-User$\uparrow$\\
        \hline\hline
        %\multicolumn{11}{c}{\textbf{HDTF} \cite{zhang2021flow}} \\ \hline
        DPR \cite{zhou2019deep} & 0.768 & \textbf{0.730} & \underline{0.0295} & \underline{2.646} & 0.734 & \underline{44.57} & \underline{403.0} & \underline{3.423} & \underline{3.462} &\underline{3.125}\\
        SMFR \cite{hou2021towards} & \underline{0.747} & \underline{0.601} & 0.0333 & 1.057 & 0.588 & 60.50 & 551.6 & 3.047 & 2.877 & 2.604\\
        NFL \cite{nerffacelighting} & 0.784 & 0.199 & 0.0823 & 2.586 & \underline{0.766} & 96.17 & 819.3 & 2.894 & 2.553 & 2.398\\
        StyleFlow \cite{10.1145/3447648} & 0.932 & 0.474 & 0.1088 & 2.614 & 0.746 & 161.3 & 900.6 & 2.103 & 1.929 &1.563\\
        DiFaReli \cite{ponglertnapakorn2023difareli} & 0.783 & 0.531 & 0.1152 & 1.103 & 0.458 & 57.49 & 743.2 & 3.141 & 2.592 & 2.284\\
        \hline
        Ours & \textbf{0.738} & 0.585 & \textbf{0.0282} & \textbf{3.034} & \textbf{0.775} & \textbf{37.46} & \textbf{273.3} & \textbf{3.534} & \textbf{4.000} & \textbf{3.398}\\
        \hline\hline
    \end{tabular}}
    \vspace{-4mm}
\end{table*}
%\iffalse
%\begin{table}
%    \centering
%    \caption{Quantitative comparison with DPR and SMFR on the synthesized portrait videos, which are generated using our method (Ours w/o Rel.) with the lighting from the reference image itself. The best scores are highlighted in bold, and the second-best scores are underlined.}
%    \vspace{-2mm}
%    \scalebox{0.95}
%    {\begin{tabular}{cccccc}
%        \hline
%        Methods & LE\downarrow & ID\uparrow & LPIPS\downarrow & IQ\uparrow & FID\downarrow\\
%        \hline\hline
%        DPR\cite{zhou2019deep} & 0.770 & \underline{0.695} & 0.032 & 2.63 & 48.2 \\
%        SMFR\cite{hou2021towards} & \underline{0.750} & 0.581 & 0.039 & 1.05 & 63.9 \\
%        \hline
%        Ours w/o Reli. & -- & \textbf{0.837} & \textbf{0.027} & \textbf{3.04} & \textbf{34.8} \\
%        Ours & \textbf{0.738} & 0.585 & \underline{0.028} & \underline{3.03} & \underline{37.5} \\
%        \hline
%    \end{tabular}}
%    \vspace{-4mm}
%    \label{tab:self}
%\end{table}
%\fi

\begin{table}
    \centering
    \caption{Quantitative comparison of portrait relighting with NFL, StyleFlow and DiFaReli on the FFHQ dataset. The best scores are highlighted in bold, and the second-best are underlined.}
    \vspace{-2mm}
    \begin{tabular}{ccccc}
        \hline
        Methods & LE$\downarrow$ & ID$\uparrow$ & IQ$\uparrow$ & FID$\downarrow$\\
        \hline\hline
        NFL\cite{nerffacelighting} & \underline{0.892} & 0.253 & 3.020 & 118.9\\
        StyleFlow\cite{10.1145/3447648} & 1.042 & 0.485 & \underline{3.846} & 102.7\\
        DiFaReli\cite{ponglertnapakorn2023difareli} & \textbf{0.749} & \underline{0.687} & 1.591 & \textbf{25.98}\\
        \hline
        Ours & 0.938 & \textbf{0.765} & \textbf{4.465} & \underline{26.71}\\
        \hline
    \end{tabular}
    \vspace{-4mm}
    \label{tab:ffhq}
\end{table}
\begin{table}
    \centering
    \caption{Quantitative comparison of cross-identity portrait animation with DaGAN, StyleHEAT, and AnimateAnyone on the HDTF dataset. The best scores are highlighted in bold, and the second-best scores are underlined.}
    \vspace{-2mm}
    \scalebox{0.95}
    {\begin{tabular}{cccccc}
        \hline
        Methods & ID$\uparrow$ & POSE$\downarrow$ & IQ$\uparrow$ & VQ$\uparrow$ & FID$\downarrow$\\
        \hline\hline
        DaGAN\cite{hong2022depth} & 0.645 & \underline{3.935} & 1.005 & 0.528 & 107.4 \\
        StyHE.\cite{yin2022styleheat} & 0.201 & 34.58 & 1.554 & 0.612 & 149.9 \\
        AniAny.\cite{hu2024animate} & \underline{0.806} & 5.086 & \underline{2.744} & \underline{0.706} & \underline{69.85} \\
        \hline
        Ours & \textbf{0.876} & \textbf{3.805} & \textbf{3.021} & \textbf{0.717} & \textbf{49.11} \\
        \hline
    \end{tabular}}
    \vspace{-4mm}
    \label{tab:animate}
\end{table}
\subsection{Quantitative Evaluation}
In portrait video relighting, Table \ref{tab:compare} shows that our method outperforms other state-of-the-art methods in all metrics except for ID. Specifically, it improves video fidelity (FVD) by 32\%, image fidelity (FID) by 16\%, and image quality (IQ) by 14.6\% compared to the second-best method, demonstrating excellent video quality. While our method does not achieve the highest ID performance, this is because relighting in our method is applied during portrait animation, where ID information is derived only from the reference, unlike other methods that relight each frame individually. However, our method achieves the best ID performance in the user study, likely due to its higher-quality, more stable video synthesis, which visually aligns with better ID preservation. This also proves that the ID loss in our method is within an acceptable range for human perception.

Since NFL \cite{nerffacelighting}, StyleFlow \cite{10.1145/3447648}, and DiFaReli \cite{ponglertnapakorn2023difareli} are trained on the aligned FFHQ facial dataset, we compare our method on 500 FFHQ images for a fair evaluation. As shown in Table \ref{tab:ffhq}, our method outperforms the second-best method in identity preservation (ID) by 11.4\% and image quality (IQ) by 16.1\%. However, it does not achieve the best performance in lighting error (LE) and image fidelity (FID) because these methods are trained on FFHQ, while our model is trained on different video datasets, resulting in slightly lower lighting and fidelity performance. Notably, since our method is designed for video sequences and FFHQ is an image dataset, we replicate each image 16 times to form a video sequence in order to adapt the method for image testing.

In addition to portrait relighting, we use the lighting and shape from the reference image and the pose from the driving image to render shading hints, guiding our model to achieve cross-identity portrait animation, which we then evaluate. Beyond the previously mentioned metrics, we incorporate a POSE metric to assess the pose accuracy of the animated portraits, ensuring alignment with the poses in the driving video. The POSE evaluation method follows that of \cite{siarohin2021motion}, using a facial landmark detection model \cite{bulat2017far} to measure the pose error between the animated portraits and the driving portraits based on facial keypoints. As shown in Table \ref{tab:animate}, our method outperforms the other methods in all metrics, particularly achieving a 29.7\% improvement in image fidelity (FID), a 10.1\% improvement in image quality (IQ), and an 8.7\% improvement in identity preservation (ID) compared to the second-best method.
\begin{figure*}[!htbp]
	\centering
	\includegraphics[width=0.85\textwidth]{resources/exp_compare_V2.pdf}
	\caption{Qualitative comparisons with DPR \cite{zhou2019deep}, SMFR \cite{hou2021towards}, StyleFlow \cite{10.1145/3447648}, NFL \cite{nerffacelighting}, and DiFaReli \cite{ponglertnapakorn2023difareli}. The first column shows the input video frames, and the remaining columns present relighted results under various lighting conditions. Our method demonstrates more realistic performance, particularly in challenging cases such as side lighting.}
    \label{fig:compare_hdtf}
    \vspace{-0.18in}
\end{figure*}
\begin{figure}[!htbp]
	\centering
	\includegraphics[width=0.45\textwidth]{resources/exp_compare_ffhq_V2.pdf}
	\caption{Qualitative comparison of portrait relighting with NFL \cite{nerffacelighting}, StyleFlow \cite{10.1145/3447648}, and DiFaReli \cite{ponglertnapakorn2023difareli} on the FFHQ dataset \cite{karras2019style}. The first column shows the input FFHQ portrait images, and the remaining column display the relighted results under various lighting conditions. Our method demonstrates more realistic results.}
    \label{fig:compare_ffhq}
    \vspace{-0.15in}
\end{figure}
\begin{figure}[!htbp]
	\centering
	\includegraphics[width=0.45\textwidth]{resources/exp_animate.pdf}
	\caption{Qualitative comparison of cross-identity portrait animation with DaGAN \cite{hong2022depth}, StyleHEAT \cite{yin2022styleheat} and AnimateAnyone \cite{hu2024animate} on the HDTF dataset. Our method demonstrates more lifelike results.}
    \label{fig:compare_animate}
    \vspace{-0.18in}
\end{figure}

\begin{figure}[!htbp]
	\centering
	\includegraphics[width=0.45\textwidth]{resources/exp_ablation_module.pdf}
	\caption{Ablation study comparing the performance of our model in portrait generation under different adapter combinations. $F_s$ represents using only the shading adapter, $F_r$ represents using only the reference adapter, and $F_s + F_r$ represents using both adapters together.}
    \label{fig:ablation_module}
    \vspace{-0.25in}
\end{figure}
\begin{figure}[!htbp]
	\centering
	\includegraphics[width=0.45\textwidth]{resources/ablationstudy_w_V2.pdf}
	\caption{Ablation study comparing our model with varying strengths of multi-condition classifier-free guidance $\omega$. As $\omega$ increases, the relighting effect increasingly aligns with the target lighting; however, this comes at the cost of some loss of identity information and a decrease in image quality.}
    \label{fig:ablation}
    \vspace{-0.18in}
\end{figure}
\subsection{Qualitative Evaluation}
We compare our approach with previous portrait relighting methods on the HDTF dataset, including state-of-the-art face alignment-based approaches such as StyleFlow \cite{10.1145/3447648}, NFL \cite{nerffacelighting}, and DiFaReli \cite{ponglertnapakorn2023difareli}. Additionally, we compare our method with face alignment-free methods like DPR \cite{zhou2019deep} and SMFR \cite{hou2021towards}. The results are shown in Fig. \ref{fig:compare_hdtf}. We find that face alignment-based methods easily suffer from background detail loss and identity degradation, especially in pre-trained StyleGAN-based \cite{karras2020analyzing} methods like StyleFlow and NFL (e.g., see the results in the fourth and fifth columns, where the background details are completely lost, and the facial identity is inconsistent with the input). On the other hand, DiFaReli, based on a pre-trained diffusion model \cite{preechakul2021diffusion}, benefits from the DDIM inverse \cite{song2020denoising} method, which successfully reconstructs background details and preserves identity; however, it introduces noticeable artifacts on the face.

Although face alignment-free methods like DPR and SMFR achieve relighting without losing background and facial identity, the trade-off is a significant reduction in image quality, with the lighting appearing unnatural, as if a shadow has been cast over the image (e.g., in the first and second rows of the third column for SMFR). In contrast, our method in the final column greatly outperforms others in both image quality and the realism of the lighting effects. Notably, our approach accurately renders specular reflections on the face and eyes, as well as realistic shadows cast by facial muscles, while keeping identity loss within acceptable limits. The background details are also largely preserved. Overall, our approach demonstrates superior capability.

Since NFL, StyleFlow, and DiFaReli are trained on the aligned FFHQ dataset, we visualize the relighting results on FFHQ for a fair comparison. As shown in Fig. \ref{fig:compare_ffhq}, NFL and StyleFlow lose background details and alter the portrait identity. DiFaReli preserves background details but introduces facial artifacts, lowering image quality. In contrast, our method maintains background details and identity consistency, achieving optimal image quality.

Additionally, we compare our method with DaGAN, StyleHEAT, and AnimateAnyone for portrait animation. As shown in Fig. \ref{fig:compare_animate}, while DaGAN preserves the pose from the driving frame, the portrait identity differs significantly from the reference, and the image quality is low. StyleHEAT introduces distortions in cross-identity portrait animation, and although AnimateAnyone, a diffusion model guided by a reference-net, generates higher image quality, it still suffers from identity loss and occasional facial artifacts.
\subsection{Ablation Study}
\textbf{Effectiveness of Adapters.} Our method constructs intrinsic and extrinsic feature subspaces using the reference and shading adapters, respectively, enabling relightable portrait animation by merging these subspaces. We conduct an ablation study with different adapter combinations. First, when retaining only the shading adapter as shown in Fig. \ref{fig:ablation_module}, the column labeled $F_s$  illustrates that the generated portrait’s pose and lighting align with the shading hints, indicating that only the extrinsic features are transferred. When only the reference adapter is used, the column labeled $F_r$ shows that the generated portrait closely resembles the reference with only minor variations, such as blinking, indicating intrinsic feature preservation. When both adapters are used, the column labeled $F_s + F_r$ demonstrates that the generated portrait not only matches the pose and lighting of the shading hints but also maintains the identity and appearance of the reference.

\noindent\textbf{Effectiveness of Guidance Strength.} In Fig. \ref{fig:ablation}, we visualize the relighting results for different $\omega$ values. When $\omega = 2$, the lighting effect is minimal, with only small differences from the input image, resulting in good identity retention. In contrast, when $\omega = 8$, the lighting effect closely aligns with the target lighting, but this also leads to reduced image quality and some loss of identity retention. The primary reason for this phenomenon is that as $\omega$ increases, the proportion of extrinsic features grows, while the proportion of intrinsic features diminishes, resulting in a degradation of identity information from the reference image. Consequently, higher values of $\omega$ enhance lighting effects but lead to greater identity loss.


\section{Related Work}

\paragraph{LLMs for Agent tasks.}

Our research is related to deploying large language models (LLMs) as agents for decision-making tasks in interactive environments~\citep{liu2023agentbench,zhou2023webarena,shridhar2020alfred,toyama2021androidenv}. Earlier works, such as~\citep{yao2023webshopscalablerealworldweb}, fine-tuned models like BERT~\citep{devlin2019bertpretrainingdeepbidirectional} for decision-making in simplified environments, such as online shopping or mobile phone manipulation. With the advent of large language models~\citep{brown2020languagemodelsfewshotlearners,openai2024gpt4technicalreport}, it became feasible to perform decision-making tasks through zero-shot or few-shot in-context learning. To better assess the capabilities of LLMs as agents, several models have been developed~\citep{deng2024mind2web,xiong2024watch,hong2023cogagent,yan2023gpt}. Most approaches~\citep{zheng2024seeact,deng2024mind2web} provide the agent with observation and action history, and the language model predicts the next action via in-context learning. Additionally, some methods~\citep{zhang2023building,li2023camel,song2024trial} attempt to distill trajectories from state-of-the-art language models to train more effective policy models. In contrast, our paper introduces a novel framework that automatically learns a reward model from LLM agent navigation, using it to guide the agents in making more effective plans.

\textbf{LLM Planning.} Our paper is also related to planning with large language models. Early researchers~\citep{brown2020languagemodelsfewshotlearners} often prompted large language models to directly perform agent tasks. Later, \citet{yao2022react} proposed ReAct, which combined LLMs for action prediction with chain-of-thought prompting~\citep{wei2022chain}. Several other works~\citep{yao2023treethoughtsdeliberateproblem,hao2023reasoning,zhao2023large,qiao2024agentplanningworldknowledge} have focused on enhancing multi-step reasoning capabilities by integrating LLMs with tree search methods. Our model differs from these previous studies in several significant ways. First, rather than solely focusing on text generation tasks, our pipeline addresses multi-step action planning tasks in interactive environments, where we must consider not only historical input but also multimodal feedback from the environment. Additionally, our pipeline involves automatic learning of the reward model from the environment without relying on human-annotated data, whereas previous works rely on prompting-based frameworks that require large commercial LLMs like GPT-4~\citep{openai2024gpt4technicalreport} to learn action prediction. Furthermore, \Model supports a variety of planning algorithms beyond tree search.

\textbf{Learning from AI Feedback.} In contrast to prior work on LLM planning, our approach also draws on recent advances in learning from AI feedback~\citep{bai2022constitutional,lee2023rlaif,yuan2024self,sharma2024critical,pan2024autonomous,koh2024tree}. These studies initially prompt state-of-the-art large language models to generate text responses that adhere to predefined principles and then potentially fine-tune the LLMs with reinforcement learning. Like previous studies, we also prompt large language models to generate synthetic data. However, unlike them, we focus not on fine-tuning a better generative model but on developing a classification model that evaluates how well action trajectories fulfill the intended instructions. This approach is simpler, requires no reliance on state-of-the-art LLMs, and is more efficient. We also demonstrate that our learned reward model can integrate with various LLMs and planning algorithms, consistently improving their performance.

\textbf{Inference-Time Scaling.} ~\citet{snell2024scaling} validates the efficacy of inference-time scaling for language models. Based on inference-time scaling, various methods have been proposed, such as random sampling~\citep{wang2022self} and tree-search methods~\citep{hao2023reasoning, zhang2024accessing, guan2025rstar}. Concurrently, several works have also leveraged inference-time scaling to improve the performance of agentic tasks. ~\citet{koh2024tree} adopts a training-free approach, employing MCTS to enhance policy model performance during inference and prompting the LLM to return the reward. ~\citet{gu2024your} introduces a novel speculative reasoning approach to bypass irreversible actions by leveraging LLMs or VLMs. It also employs tree search to improve performance and prompts an LLM to output rewards. ~\citet{yu2024exact} proposes Reflective-MCTS to perform tree search and fine-tune the GPT model, leading to improvements in ~\citet{koh2024visualwebarena}. ~\citet{putta2024agent} also utilizes MCTS to enhance performance on web-based tasks such as ~\citet{yao2023webshopscalablerealworldweb} and real-world booking environments. ~\cite{lin2025qlass} utilizes the stepwise reward to give effective intermediate guidance across different agentic tasks. Our work differs from previous efforts in two key aspects: (1) Broader Application Domain. Unlike prior studies that primarily focus on tasks from a single domain, our method demonstrates strong generalizability across web agents, mathematical reasoning, and scientific discovery domains, further proving its effectiveness. (2) Flexible and Effective Reward Modeling. Instead of simply prompting an LLM as a reward model, we finetune a small scale VLM~\citep{lin2023vila} to evaluate input trajectories. %Our reward scores range continuously between 0 and 1, in contrast to existing methods that rely on discrete scoring (e.g., 0 and 1, or 0, 0.5, and 1) through direct LLM prompting.

% Concurrently, several works have also leveraged inference-time scaling to improve the performance of agentic tasks. ~\citet{pan2024autonomous} demonstrates that LLMs and VLMs, such as the GPT series, can function as evaluators or reward models to provide guidance for fine-tuning or reflection, thereby enhancing digital agents. This lays the groundwork for subsequent studies that directly prompt LLMs as reward models. ~\citet{koh2024tree} adopts a training-free approach, employing MCTS to enhance policy model performance during inference. However, it is limited to web environments~\citep{koh2024visualwebarena}. Moreover, its value function relies on prompting an LLM, which is less effective than our proposed method. We validate our approach through ablation studies, demonstrating that our fine-tuned reward model is more effective. ~\citet{gu2024your} introduces a novel speculative reasoning approach to bypass irreversible actions, such as purchasing a product, by leveraging LLMs or VLMs. It also employs tree search to improve performance, but it remains restricted to the web domain~\citep{koh2024visualwebarena, deng2024mind2web}. Additionally, it lacks reward modeling and instead prompts an LLM to output rewards. ~\citet{yu2024exact} proposes Reflective-MCTS to perform tree search and fine-tune the GPT model, leading to improvements in ~\citep{koh2024visualwebarena}. However, this work focuses solely on a single web agent task, and its reward modeling is derived from multi-agent debate, differing from our more effective and efficient reward modeling approach. ~\citet{putta2024agent} also utilizes MCTS to enhance performance, but it is limited to web-based tasks such as ~\citep{yao2023webshopscalablerealworldweb} and real-world booking environments.
\section{Conclusion}
% This study demonstrates that classic GNNs, when enhanced with our GNN$^+$ framework, can match and even surpass GTs on graph-level tasks. Across 14 benchmark datasets, these upgraded GNNs consistently rank in the top three, achieving first place in eight while also exhibiting greater efficiency. Our findings challenge the prevailing assumption that GTs inherently outperform GNNs and reaffirm the potential of well-structured GNNs as a powerful model. 
% We hope that our findings encourage more rigorous empirical evaluations in the field of graph machine learning.

%This study highlights the often-overlooked potential of classic GNNs. By integrating six widely used techniques into a unified GNN$^+$ framework, we enhance 3 classic GNNs for graph-level tasks. Evaluations on 14 benchmark datasets show that, these enhanced GNNs consistently rank among the top three and secure first place on eight, while also exhibiting greater efficiency. These findings challenge the prevailing belief that GTs are inherently superior, reaffirming that well-designed GNNs remain highly competitive.

This study highlights the often-overlooked potential of classic GNNs in tacking graph-level tasks. By integrating six widely used techniques into a unified GNN$^+$ framework, we enhance three classic GNNs for graph-level tasks. Evaluations on 14 benchmark datasets reveal that, these enhanced GNNs match or outperform GTs, while also demonstrating greater efficiency. These findings challenge the prevailing belief that GTs are inherently superior, reaffirming the capability of simple GNN structures as powerful models.

% \newpage
\section{Limitations and Future Work}

\paragraph{Limitations} While our GSQ-Tuning significantly advances on-device LLM adaptation through integer-focused optimization and parameter-efficient quantization, two key limitations warrant discussion:

\paragraph{Non-linear Operator Precision.} Our current implementation maintains non-linear operations (e.g., LayerNorm, Softmax) in 16-bit to preserve numerical stability. This introduces partial precision conversion overhead during computation. However, non-linear operations do not contain additional learnable parameters and thus do not consume memory. Moreover, these non-linear operations are generally computation-light, making their computational burden negligible. Future work could explore fully integer implementations for non-linear layers. \textbf{Bit-Width Range Constraints.}
The current framework operates effectively in 5-8bit configurations but didn't present the performance at extreme low bit ($\leq$ 4bit) precision. This stems from gradient direction distortion under extreme quantization—a challenge requiring new error compensation mechanisms. We plan to investigate two directions: (1) 4bit stochastic rounding with gradient-aware scaling, and (2) mixed-precision adapters allocating higher bits to critical gradient dimensions.

\paragraph{Future Work} Furthermore, future work could explore (1) full integer fine-tuning, (2) extreme low-bit quantized fine-tuning and (3) co-design with emerging integer-optimized AI accelerators. Our code will be publicly available to advance edge LLMs research.
% \section*{Acknowledgments}
\clearpage
\bibliography{custom}

\appendix

\section{Appendix}
\label{sec:appendix}
\subsection{Differences with Quantization-aware training (QAT): }
\label{qat_fqt}
Quantization-aware training (QAT)~\cite{choi2018pact,Zhang_2018_ECCV,zhou2017incremental,jacob2018quantization,dong2019hawq,dong2019hawqv2,shen2019q,zafrir2019q8bert,shen2020QBERT,tang2022mkq,zhang2020ternarybert,bai2020binarybert,foret2020sharpness,wang2022squat} is an \emph{inference acceleration} technique which trains networks with quantizers inserted in the forward propagation graph, so the trained network can perform efficiently during inference. 
QAT can compress activation/weights to extremely low precision (e.g. 1-2 bits). 
It is tempting to think that directly applying a quantizer for QAT to FQT can lead to similar low activation/weights bit-width. However, even only quantizing the forward propagation for FQT is much more challenging than QAT because:  \raisebox{-0.5pt}{\ding[1.1]{182\relax}} QAT requires a converged full-precision model as initialization~\cite{esser2019learned} and/or as a teacher model for knowledge distillation~\cite{bai2020binarybert}; \raisebox{-0.5pt}{\ding[1.1]{184\relax}} QAT may approximate the discrete quantizer with continuous functions during training~\cite{gong2019differentiable}, which cannot be implemented with integer arithmetic. Due to these challenges, it is still an open problem to do FQT with low-bit activations/weights. 

\input{text/tables/main_0shot_llama2-7b}
\begin{table*}[!t]
\renewcommand\arraystretch{1.0}
\centering
\caption{$0$-shot commonsense QA accuracy (\%) across different bits and rank on llama2-13B.}
\label{tab:llama2-13b}
\setlength{\tabcolsep}{1.2mm}
{\resizebox{0.98\textwidth}{!}{
\begin{tabular}{lcccccccccccccc|c}
\noalign{\vspace{0.3em}}
\toprule
\noalign{\vspace{0.1em}}
\textbf{Method} & \textbf{rank}& LLMs branch & low-rank branch &\textbf{Avg.} & \textbf{ARC-c} & \textbf{ARC-e} & \textbf{BoolQ} & \textbf{HellaS.} & \textbf{OBQA} & \textbf{PIQA} & \textbf{SCIQ.} & \textbf{WinoG.} & \textbf{Mem. (G)} \\
\midrule
\noalign{\vspace{0.1em}}
 LLaMA2-13B    &-     & 16-16-16 & w/o&66.65 &48.81 &76.47 &82.45 &79.67 &44.80 &80.36 &48.31 &72.38 &25.70\\
 \noalign{\vspace{0.1em}}\hdashline[0.8pt/1pt]\noalign{\vspace{0.1em}}
 \textit{w/ QLoRA} & 16 &4-16-16 & 16-16-16 & 67.32 &49.74 &76.98 &82.94 &78.85 &46.00 &80.52 &50.36 &73.16 &16.56\\
\noalign{\vspace{0.1em}}\hdashline[0.8pt/1pt]\noalign{\vspace{0.1em}}
\multirow{4}{*}{w/ GSQ-Tuning}  &\multirow{4}{*}{16} &4-8-8 & 8-8-8  & \default{\textbf{67.35}} &49.83 &77.06 &83.09 &78.89 &46.00 &80.47 &50.31 &73.16 &11.13\\
& & 4-7-7 & 7-7-7   & \default{\textbf{67.29}} &49.91 &76.94 &83.03 &78.90 &45.40 &80.58 &50.61 &73.01 &10.58\\
& & 4-6-6 & 6-6-6   & \default{\textbf{67.23}} &49.66 &76.98 &82.75 &78.79 &46.00 &80.47 &50.05 &73.16 &10.03\\
  & & 4-5-5 & 5-5-5  & \default{66.57} &49.57 &76.43 &81.62 &77.98 &45.40 &80.09 &49.39 &72.06 &9.47\\
\midrule
\noalign{\vspace{0.1em}}
 \textit{w/ QLoRA} & 32 &4-16-16 & 16-16-16 & 67.47 &49.83 &77.02 &83.24 &78.92 &46.20 &80.58 &50.77 &73.24 &16.85\\
\noalign{\vspace{0.1em}}\hdashline[0.8pt/1pt]\noalign{\vspace{0.1em}}
\multirow{4}{*}{w/ GSQ-Tuning}  &\multirow{4}{*}{32} & 4-8-8 & 8-8-8  & \default{\textbf{67.49}} &49.83 &76.98 &83.15 &78.94 &45.60 &80.79 &51.07 &73.56 &11.42\\
& & 4-7-7 & 7-7-7  & \default{\textbf{67.38}} &50.17 &77.06 &82.81 &78.99 &45.40 &80.79 &50.46 &73.40  &10.87\\
& & 4-6-6 & 6-6-6  & \default{\textbf{67.35}} &49.83 &77.06 &83.09 &78.89 &46.00 &80.47 &50.31 &73.16 &10.31\\
  & & 4-5-5 & 5-5-5 & \default{66.65} &48.38 &76.18 &82.08 &78.07 &45.60 &80.36 &49.74 &72.77 &9.76 \\
\midrule
 \textit{w/ QLoRA} & 64 &4-16-16 & 16-16-16 & 67.61 &49.66 &77.23 &83.30 &78.95 &45.40 &80.74 &51.59 &73.24 &17.42\\
\noalign{\vspace{0.1em}}\hdashline[0.8pt/1pt]\noalign{\vspace{0.1em}}
\multirow{4}{*}{w/ GSQ-Tuning}  &\multirow{4}{*}{64} & 4-8-8 & 8-8-8   & \default{\textbf{67.48}} &49.57 &77.40 &82.87 &78.88 &46.20 &80.90 &50.72 &73.32 &11.99\\
& & 4-7-7 & 7-7-7   & \default{\textbf{67.43}} &49.74 &77.27 &82.91 &78.89 &46.00 &80.90 &50.61 &73.09 &11.44\\
& & 4-6-6 & 6-6-6   & \default{\textbf{67.35}} &49.66 &77.27 &82.75 &78.66 &79.05 &80.90 &50.97 &73.16 &10.89\\
  & & 4-5-5 & 5-5-5 & \default{66.97} &49.91 &76.60 &81.87 &78.15 &46.20 &80.41 &49.54 &73.09 &10.33\\
\midrule
 \textit{w/ QLoRA} & 128&4-16-16 & 16-16-16& 67.61 &50.34 &77.40 &83.55 &78.89 &46.00 &80.85 &50.92 &72.93&18.56\\
\noalign{\vspace{0.1em}}\hdashline[0.8pt/1pt]\noalign{\vspace{0.1em}}
\multirow{4}{*}{w/ GSQ-Tuning}  &\multirow{4}{*}{128} & 4-8-8 & 8-8-8   & \default{\textbf{67.62}} &50.34 &77.06 &83.18 &78.96 &46.40 &80.69 &50.92 &73.40&13.14\\
& & 4-7-7 & 7-7-7   & \default{\textbf{67.57}} &50.43 &77.36 &83.06 &79.05 &45.60 &80.85 &51.28 &72.93 &12.58\\
& & 4-6-6 & 6-6-6   & \default{\textbf{67.53}} &50.43 &77.31 &83.15 &78.81 &45.80 &80.58 &50.97 &73.16 &12.03\\
  & & 4-5-5 & 5-5-5  & \default{67.10} &49.49 &76.81 &82.08 &78.22 &46.40 &80.03 &50.56 &73.24 &11.48\\
\midrule
 \textit{w/ QLoRA} & 256 &4-16-16 & 16-16-16 & 67.91 &50.77 &77.36 &83.64 &78.88 &46.60 &80.74 &51.69 &73.64&20.85\\
\noalign{\vspace{0.1em}}\hdashline[0.8pt/1pt]\noalign{\vspace{0.1em}}
\multirow{4}{*}{w/ GSQ-Tuning}  &\multirow{4}{*}{256} & 4-8-8 & 8-8-8   & \default{\textbf{67.84}} &51.11 &77.06 &83.82 &78.80 &46.40 &80.69 &52.00 &72.85 &15.42\\
& &4-7-7 & 7-7-7   & \default{\textbf{67.74}} &50.77 &77.31 &83.79 &78.84 &46.00 &80.63 &51.89 &72.69 &14.87\\
& & 4-6-6 & 6-6-6   & \default{\textbf{67.68}} &50.77 &77.19 &83.49 &78.82 &46.00 &80.58 &51.38 &73.24 &14.32\\
  & & 4-5-5 & 5-5-5  & \default{67.22} &50.85 &75.84 &82.11 &78.21 &46.00 &80.36 &50.92 &73.48 &13.76\\
\midrule
 \textit{w/ QLoRA} & 512 &4-16-16 & 16-16-16 & 67.94 &50.60 &77.48 &83.88 &79.00 &46.40 &80.74 &52.05 &73.40&25.43\\
\noalign{\vspace{0.1em}}\hdashline[0.8pt/1pt]\noalign{\vspace{0.1em}}
\multirow{4}{*}{w/ GSQ-Tuning}  &\multirow{4}{*}{512} & 4-8-8 & 8-8-8   & \default{\textbf{67.92}} &51.02 &77.27 &83.27 &79.04 &46.40 &81.01 &51.79 &73.56 &20.00\\
& & 4-7-7 & 7-7-7   & \default{\textbf{67.90}} &51.19 &77.15 &83.79 &78.82 &46.80 &80.69 &51.79 &73.01 &19.45\\
& & 4-6-6 & 6-6-6   & \default{\textbf{67.82}} &51.02 &77.02 &83.85 &78.93 &46.20 &80.90 &51.54 &73.09 &18.89\\
  & & 4-5-5 & 5-5-5  & \default{67.39} &50.94 &76.68 &82.29 &78.39 &46.20 &80.41 &51.69 &72.53 &18.34\\
\bottomrule
\end{tabular}}}
\end{table*}
\begin{table*}[!t]
\renewcommand\arraystretch{1.0}
\centering
\caption{$0$-shot commonsense QA accuracy (\%) across different bits and rank on llama2-70B.}
\label{tab:llama2-70b}
\setlength{\tabcolsep}{1.2mm}
{\resizebox{0.98\textwidth}{!}{
\begin{tabular}{lcccccccccccccc|c}
\noalign{\vspace{0.3em}}
\toprule
\noalign{\vspace{0.1em}}
\textbf{Method} & \textbf{rank}& LLMs branch & low-rank branch &\textbf{Avg.} & \textbf{ARC-c} & \textbf{ARC-e} & \textbf{BoolQ} & \textbf{HellaS.} & \textbf{OBQA} & \textbf{PIQA} & \textbf{SCIQ.} & \textbf{WinoG.} & \textbf{Mem. (G)} \\
\midrule
\noalign{\vspace{0.1em}}
 LLaMA2-70B    &-     &  16-16-16 &w/o & 70.68 &56.91 &80.05 &85.78 &83.59 &48.60 &82.48 &48.67 &79.40 &137.42\\
 \noalign{\vspace{0.1em}}\hdashline[0.8pt/1pt]\noalign{\vspace{0.1em}}
 \textit{w/ QLoRA} & 16 & 4-16-16 &16-16-16 & 71.72 &58.62 &81.44 &86.39 &83.92 &49.80 &83.03 &50.46 &80.11 &63.90\\
\noalign{\vspace{0.1em}}\hdashline[0.8pt/1pt]\noalign{\vspace{0.1em}}
\multirow{4}{*}{w/ GSQ-Tuning}  &\multirow{4}{*}{16} & 4-8-8&8-8-8  & \default{\textbf{71.65}} &58.62 &81.23 &86.36 &83.87 &49.60 &83.19 &50.41 &79.95 &49.17\\
& & 4-7-7&7-7-7  & \default{\textbf{71.63}} &58.87 &81.57 &86.24 &83.89 &49.20 &83.19 &50.46 &79.64 &47.44\\
& & 4-6-6&6-6-6  & \default{\textbf{71.58}} &58.62 &81.36 &86.15 &83.84 &49.60 &82.97 &50.41 &79.64 &45.72\\
  & & 4-5-5&5-5-5 & \default{71.02} &57.34 &80.56 &85.93 &83.75 &49.00 &82.59 &49.33 &79.64 &43.99\\
\midrule
\noalign{\vspace{0.1em}}
 \textit{w/ QLoRA} & 32 &4-16-16 &16-16-16 & 71.84 &59.13 &81.82 &86.27 &83.88 &49.20 &83.03 &51.02 &80.35 &64.87\\
\noalign{\vspace{0.1em}}\hdashline[0.8pt/1pt]\noalign{\vspace{0.1em}}
\multirow{4}{*}{w/ GSQ-Tuning}  &\multirow{4}{*}{32} & 4-8-8&8-8-8  & \default{\textbf{71.78}} &59.04 &81.90 &86.33 &83.89 &49.00 &83.19 &51.07 &79.79 &50.17\\
& & 4-7-7&7-7-7  & \default{\textbf{71.76}} &59.30 &81.61 &86.18 &83.98 &49.00 &83.19 &51.02 &79.79 &48.44\\
& & 4-6-6&6-6-6  & \default{\textbf{71.60}} &58.96 &81.36 &86.15 &83.87 &48.80 &83.03 &51.02 &79.64 &46.72\\
  & & 4-5-5&5-5-5 & \default{71.26} &57.59 &80.85 &86.15 &83.93 &49.00 &83.13 &50.00 &79.40 &44.99\\
\midrule
 \textit{w/ QLoRA} & 64 & 4-16-16 &16-16-16 & 72.22 &59.81 &82.20 &86.51 &83.89 &50.40 &83.13 &51.48 &80.35 &66.82\\
\noalign{\vspace{0.1em}}\hdashline[0.8pt/1pt]\noalign{\vspace{0.1em}}
\multirow{4}{*}{w/ GSQ-Tuning}  &\multirow{4}{*}{64} & 4-8-8&8-8-8  & \default{\textbf{72.20}} &59.90 &82.32 &86.51 &83.90 &50.20 &83.08 &51.59 &80.11 &52.17\\
& & 4-7-7&7-7-7  & \default{\textbf{72.18}} &59.81 &82.28 &86.39 &83.88 &50.20 &83.13 &51.54 &80.19 &50.44\\
& & 4-6-6&6-6-6  & \default{\textbf{72.10}}  &59.39 &82.15 &86.51 &83.94 &50.00 &83.30 &50.92 &80.58 &48.71\\
  & & 4-5-5&5-5-5 & \default{71.70} &58.87 &81.48 &85.90 &83.91 &49.60 &82.81 &50.67 &80.43 &46.98\\
\midrule
 \textit{w/ QLoRA} & 128 & 4-16-16 &16-16-16& 72.39& 60.67 &82.37 &86.88 &84.05 &49.20 &83.19 &52.15 &80.66&70.96\\
\noalign{\vspace{0.1em}}\hdashline[0.8pt/1pt]\noalign{\vspace{0.1em}}
\multirow{4}{*}{w/ GSQ-Tuning}  &\multirow{4}{*}{128} & 4-8-8&8-8-8  &\default{\textbf{72.37}} &60.75 &82.49 &87.00 &83.94 &49.40 &83.08 &52.15 &80.19 &56.16\\
& & 4-7-7&7-7-7  & \default{\textbf{72.32}} &60.41 &82.45 &86.94 &83.94 &49.00 &83.08 &52.15 & 80.58 &54.43\\
& & 4-6-6&6-6-6  & \default{\textbf{72.28}}  &59.81 &82.45 &86.91 &83.99 &49.60 &83.35 &51.89 & 80.27 &52.70\\
  & & 4-5-5&5-5-5 & \default{71.85} &59.47 &81.90 &86.48 &83.82 &48.20 &83.08 &51.02 & 80.82&50.97\\
\bottomrule
\end{tabular}}}
\end{table*}
\begin{table*}[!t]
\renewcommand\arraystretch{1.0}
\centering
\caption{$0$-shot commonsense QA accuracy (\%) across different bits and rank on llama3-3B.}
\label{tab:llama3-3b}
\setlength{\tabcolsep}{1.2mm}
{\resizebox{0.98\textwidth}{!}{
\begin{tabular}{lcccccccccccccc|c}
\noalign{\vspace{0.3em}}
\toprule
\noalign{\vspace{0.1em}}
\textbf{Method} & \textbf{rank}& LLMs branch &low-rank branch &\textbf{Avg.} & \textbf{ARC-c} & \textbf{ARC-e} & \textbf{BoolQ} & \textbf{HellaS.} & \textbf{OBQA} & \textbf{PIQA} & \textbf{SCIQ.} & \textbf{WinoG.} & \textbf{Mem. (G)} \\
\midrule
\noalign{\vspace{0.1em}}
 LLaMA3-3B    &-     &  16-16-16 & w/o & 64.13 &46.25 &74.62 &77.68 &76.01 &44.20 &79.11 &46.11 &69.06 &6.42\\
 \noalign{\vspace{0.1em}}\hdashline[0.8pt/1pt]\noalign{\vspace{0.1em}}
 \textit{w/ QLoRA} & 16 & 4-16-16 & 16-16-16 & 65.05 &47.53 &75.17 &78.59 &76.09 &44.00 &79.54 &49.44 &70.09 &6.42\\
\noalign{\vspace{0.1em}}\hdashline[0.8pt/1pt]\noalign{\vspace{0.1em}}
\multirow{4}{*}{w/ GSQ-Tuning}  &\multirow{4}{*}{16} & 4-8-8 & 8-8-8  & \default{\textbf{65.10}}& 47.53 &74.71 &78.35 &75.99 &45.00 &79.65 &49.28 &70.32 &3.57\\
& & 4-7-7 & 7-7-7  & \default{\textbf{64.96}} &47.18 &75.21 &78.10 &75.98 &44.80 &79.27 &49.95 &69.38 &3.34\\
& & 4-6-6 & 6-6-6  & \default{\textbf{64.87}} &46.84 &73.78 &78.07 &75.88 &45.80 &79.22 &49.39 &70.01 &3.11\\
  & & 4-5-5 & 5-5-5 & \default{63.97} &46.76 &72.64 &75.78 &74.95 &45.20 &79.05 &48.62 &68.75 &2.88\\
\midrule
\noalign{\vspace{0.1em}}
 \textit{w/ QLoRA} & 32 &4-16-16 & 16-16-16 & 65.24 &47.27 &75.04 &78.87 &76.11 &44.60 &79.76 &49.95 &70.32 &6.54\\
\noalign{\vspace{0.1em}}\hdashline[0.8pt/1pt]\noalign{\vspace{0.1em}}
\multirow{4}{*}{w/ GSQ-Tuning}  &\multirow{4}{*}{32} & 4-8-8 & 8-8-8 & \default{\textbf{65.45}} &48.12 &74.71 &78.38 &76.14 &46.00 &79.71 &49.64 &70.96 &3.69\\
& & 4-7-7 & 7-7-7  & \default{\textbf{65.43}} &47.35 &74.20 &78.99 &75.84 &46.00 &79.92 &49.59 &71.59 &3.46\\
& & 4-6-6 & 6-6-6 & \default{\textbf{65.01}} &47.44 &74.62 &78.65 &76.03 &44.00 &79.60 &50.05 &69.69 &3.23\\
  & & 4-5-5 & 5-5-5 & \default{64.00} &44.97 &73.32 &75.29 &74.95 &44.60 &79.27 &48.93 &70.24 &3.00\\
\midrule
 \textit{w/ QLoRA} & 64 & 4-16-16 & 16-16-16 & 65.69 &47.14 &74.75 &79.50 &76.46 &45.50 &79.63 &50.26 &71.32 &6.78\\
\noalign{\vspace{0.1em}}\hdashline[0.8pt/1pt]\noalign{\vspace{0.1em}}
\multirow{4}{*}{w/ GSQ-Tuning}  &\multirow{4}{*}{64} & 4-8-8 & 8-8-8  & \default{\textbf{65.60}} &48.12 &74.24 &79.72 &76.00 &45.80 &79.60 &49.69 &71.67 &3.93\\
& & 4-7-7 & 7-7-7  & \default{\textbf{65.47}} &47.78 &74.71 &79.51 &76.09 &45.80 &79.60 &49.80 &70.48 &3.70\\
& & 4-6-6 & 6-6-6  & \default{\textbf{65.39}} &47.70 &74.58 &79.24 &76.05 &44.60 &79.60 &50.41 &70.96 &3.47\\
  & &4-5-5 & 5-5-5 & \default{64.18} &45.14 &72.69 &75.20 &75.27 &46.40 &79.65 &48.62 &70.48 &3.24\\
\midrule
 \textit{w/ QLoRA} & 128 & 4-16-16 & 16-16-16& 65.84 &48.24 &74.91 &79.78 &76.27 &45.52 &79.77 &50.48 &71.79 &6.76\\
\noalign{\vspace{0.1em}}\hdashline[0.8pt/1pt]\noalign{\vspace{0.1em}}
\multirow{4}{*}{w/ GSQ-Tuning}  &\multirow{4}{*}{128} & 4-8-8 & 8-8-8  & \default{\textbf{65.79}} &48.12 &74.83 &80.28 &75.96 &45.80 &79.54 &50.61 &71.19 &4.41\\
& & 4-7-7 & 7-7-7  & \default{\textbf{65.69}} &48.04 &74.87 &79.79 &76.08 &45.00 &79.49 &50.61 &71.67 &4.18\\
& & 4-6-6 & 6-6-6  & \default{\textbf{65.58}} &47.87 &74.54 &80.09 &76.05 &45.40 &79.38 &50.10 &71.27 &3.95\\
  & & 4-5-5 & 5-5-5 & \default{64.46} &46.50 &72.77 &75.99 &75.31 &46.60 &79.00 &48.98 &70.56 &3.72\\
\midrule
 \textit{w/ QLoRA} & 256 &4-16-16 & 16-16-16 & 66.12 &48.33 &75.00 &80.94 &76.37 &45.61 &79.97 &51.13 &71.64 &7.61\\
\noalign{\vspace{0.1em}}\hdashline[0.8pt/1pt]\noalign{\vspace{0.1em}}
\multirow{4}{*}{w/ GSQ-Tuning}  &\multirow{4}{*}{256} & 4-8-8 & 8-8-8  & \default{\textbf{66.19}} &48.55 &75.13 &80.76 &76.14 &47.00 &79.38 &50.72 &71.82 &5.37\\
& & 4-7-7 & 7-7-7  & \default{\textbf{65.96}} &48.46 &75.08 &80.43 &76.04 &45.60 &79.76 &50.72 &71.59 &5.13\\
& & 4-6-6 & 6-6-6  & \default{\textbf{65.90}} &48.38 &74.16 &79.94 &75.81 &46.80 &79.43 &50.87 &71.82 &4.90\\
  & & 4-5-5 & 5-5-5 & \default{64.59} &46.33 &72.60 &76.51 &75.57 &46.40 &79.60 &49.39 &70.32 &4.67\\
\midrule
 \textit{w/ QLoRA} & 512 & 4-16-16 & 16-16-16 & 66.59 &49.26 &75.20 &81.99 &76.06 &46.74 &79.49 &51.71 &72.27&9.73\\
\noalign{\vspace{0.1em}}\hdashline[0.8pt/1pt]\noalign{\vspace{0.1em}}
\multirow{4}{*}{w/ GSQ-Tuning}  &\multirow{4}{*}{512} & 4-8-8 & 8-8-8  & \default{\textbf{66.52}} &49.49 &74.92 &81.28 &75.89 &47.60 &79.49 &51.59 &71.90&7.28\\
& & 4-7-7 & 7-7-7  & \default{\textbf{66.33}} &48.89 &74.75 &81.41 &76.06 &47.00 &79.54 &51.74 &71.27&7.05\\
& & 4-6-6 & 6-6-6 & \default{\textbf{66.31}} &48.55 &75.51 &80.80 &76.42 &46.00 &79.60 &51.64 &71.98 &6.82\\
  & & 4-5-5 & 5-5-5 & \default{64.86} &47.44 &73.15 &76.85 &75.62 &47.00 &79.33 &49.18 &70.32 &6.59\\
\bottomrule
\end{tabular}}}
\end{table*}
\begin{table*}[!t]
\renewcommand\arraystretch{1.0}
\centering
\caption{$0$-shot commonsense QA accuracy (\%) across different bits and rank on llama3-8B.}
\label{tab:llama3-8b}
\setlength{\tabcolsep}{1.2mm}
{\resizebox{0.98\textwidth}{!}{
\begin{tabular}{lcccccccccccccc|c}
\noalign{\vspace{0.3em}}
\toprule
\noalign{\vspace{0.1em}}
\textbf{Method} & \textbf{rank}& LLMs branch &low-rank branch &\textbf{Avg.} & \textbf{ARC-c} & \textbf{ARC-e} & \textbf{BoolQ} & \textbf{HellaS.} & \textbf{OBQA} & \textbf{PIQA} & \textbf{SCIQ.} & \textbf{WinoG.} & \textbf{Mem. (G)} \\
\midrule
\noalign{\vspace{0.1em}}
 LLaMA3-8B    &-     &  16-16-16 & w/o & 67.18 &53.50 &77.74 &81.13 &79.20 &45.00 &80.63 &47.03 &73.24 &15.01\\
 \noalign{\vspace{0.1em}}\hdashline[0.8pt/1pt]\noalign{\vspace{0.1em}}
 \textit{w/ QLoRA} & 16 & 4-16-16 &16-16-16 & 68.14 &54.52 &79.50 &83.43 &78.66 &44.80 &80.85 &50.00 &73.32 &10.71\\
\noalign{\vspace{0.1em}}\hdashline[0.8pt/1pt]\noalign{\vspace{0.1em}}
\multirow{4}{*}{w/ GSQ-Tuning}  &\multirow{4}{*}{16} &  4-8-8 & 8-8-8  & \default{\textbf{68.16}} &54.61 &79.84 &83.70 &78.58 &44.80 &80.79 &49.85 &73.16 &7.03\\
& & 4-7-7 & 7-7-7  & \default{\textbf{68.00}} &54.01 &79.29 &83.46 &78.65 &45.00 &80.85 &49.80 &73.01 &6.65\\
& & 4-6-6 & 6-6-6  & \default{\textbf{67.74}} &54.01 &78.70 &83.09 &78.49 &44.00 &80.90 &49.44 &73.32 &6.26\\
  & & 4-5-5 & 5-5-5 & \default{66.51} &51.54 &77.27 &81.99 &77.00 &44.40 &78.84 &48.46 &72.61 &5.87\\
\midrule
\noalign{\vspace{0.1em}}
 \textit{w/ QLoRA} & 32 &4-16-16 &16-16-16 & 68.31 &55.55 &80.39 &83.36 &78.65 &44.60 &81.28 &50.05 &72.61 &11.02\\
\noalign{\vspace{0.1em}}\hdashline[0.8pt/1pt]\noalign{\vspace{0.1em}}
\multirow{4}{*}{w/ GSQ-Tuning}  &\multirow{4}{*}{32} & 4-8-8 & 8-8-8  & \default{\textbf{68.45}} &55.72 &80.22 &83.43 &78.60 &45.00  &81.18 &50.20 &73.32 &7.23\\
& & 4-7-7 & 7-7-7  & \default{\textbf{68.29}} &54.95 &80.13 &83.36 &78.53 &44.80 &81.01 &50.20 &73.32 &6.84\\
& & 4-6-6 & 6-6-6  & \default{\textbf{68.08}} &55.29 &79.29 &83.55 &78.28 &45.80 &81.07 &49.39 &71.98 &6.46\\
  & & 4-5-5 & 5-5-5 & \default{66.48} &51.71 &77.69 &82.11 &76.91 &44.20 &79.43 &48.16 &71.67 &6.07\\
\midrule
 \textit{w/ QLoRA} & 64 & 4-16-16 &16-16-16 & 68.45 &55.63 &80.13 &83.67 &78.78 &44.80 &81.28 &50.41 &72.93 &11.64\\
\noalign{\vspace{0.1em}}\hdashline[0.8pt/1pt]\noalign{\vspace{0.1em}}
\multirow{4}{*}{w/ GSQ-Tuning}  &\multirow{4}{*}{64} & 4-8-8 & 8-8-8  & \default{\textbf{68.61}} &55.97 &80.22 &83.61 &78.68 &45.20 &81.50 &50.41 &73.32 &7.63\\
& & 4-7-7 & 7-7-7  & \default{\textbf{68.57}}  &55.97 &80.68 &83.73 &78.84 &45.20 &81.01 &50.26 &72.85 &7.24\\
& & 4-6-6 & 6-6-6  & \default{\textbf{68.22}}  &55.55 &79.29 &83.67 &78.47 &44.80 &80.90 &50.05 &73.09 &6.86\\
  & & 4-5-5 & 5-5-5 & \default{66.69} &54.10 &77.99 &81.65 &77.12 &43.80 &79.54 &47.90 &71.43 &6.47\\
\midrule
 \textit{w/ QLoRA} & 128 & 4-16-16 &16-16-16 & 68.77 &56.14 &80.56 &83.98 &79.03 &45.60 &81.34 &50.56 &72.93 &12.13\\
\noalign{\vspace{0.1em}}\hdashline[0.8pt/1pt]\noalign{\vspace{0.1em}}
\multirow{4}{*}{w/ GSQ-Tuning}  &\multirow{4}{*}{128} & 4-8-8 & 8-8-8  & \default{\textbf{68.72}} &56.57 &80.22 &83.82 &78.80 &45.40 &81.23 &50.41 &73.32 &8.43\\
& & 4-7-7 & 7-7-7  & \default{\textbf{68.71}} &56.48 &80.18 &83.88 &78.78 &45.80 &81.34 &50.36 &72.93 &8.04\\
& & 4-6-6 & 6-6-6  & \default{\textbf{68.67}} &56.91 &79.50 &83.79 &78.71 &46.60 &80.52 &50.36 &73.01 &7.66\\
  & &4-5-5 & 5-5-5 & \default{66.92} &52.47 &78.45 &82.63 &77.22 &44.60 &79.49 &48.52 &71.98&7.27\\
\midrule
 \textit{w/ QLoRA} & 256 & 4-16-16 &16-16-16 & 69.09 &56.74 &80.35 &84.56 &79.02 &45.20 &81.83 &50.92 &74.11&13.81\\
\noalign{\vspace{0.1em}}\hdashline[0.8pt/1pt]\noalign{\vspace{0.1em}}
\multirow{4}{*}{w/ GSQ-Tuning}  &\multirow{4}{*}{256} & 4-8-8 & 8-8-8  & \default{\textbf{69.04}} &56.57 &80.85 &84.07 &78.97 &45.40 &81.45 &51.28 &73.72 &10.03 \\
& & 4-7-7 & 7-7-7  & \default{\textbf{69.00}} &56.83 &80.89 &84.25 &78.96 &45.60 &81.50 &50.46 &73.56 &9.64\\
& & 4-6-6 & 6-6-6  & \default{\textbf{68.84}} &56.74 &79.80 &83.98 &78.84 &46.40 &81.12 &50.77 &73.09 &9.26\\
  & & 4-5-5 & 5-5-5 & \default{67.54} &53.33 &78.49 &83.21 &77.38 &44.60 &79.98 &48.93 &73.64 &8.87\\
\midrule
 \textit{w/ QLoRA} & 512 &4-16-16 &16-16-16 & 69.18 &57.17 &80.30 &84.65 &79.28 &46.40 &81.07 &50.36 &74.27 &16.81\\
\noalign{\vspace{0.1em}}\hdashline[0.8pt/1pt]\noalign{\vspace{0.1em}}
\multirow{4}{*}{w/ GSQ-Tuning}  &\multirow{4}{*}{512} & 4-8-8 & 8-8-8  & \default{\textbf{69.24}} &56.48 &80.47 &85.35 &79.13 &45.40 &81.56 &51.54 &74.03 &13.23\\
& & 4-7-7 & 7-7-7 & \default{\textbf{69.16}} &56.40 &80.68 &85.26 &79.10 &45.80 &81.28 &51.13 &73.64 &12.84\\
& & 4-6-6 & 6-6-6  & \default{\textbf{69.01}} &56.57 &80.01 &84.56 &78.84 &45.80 &81.23 &51.64 &73.48 &12.45\\
  & & 4-5-5 & 5-5-5 & \default{67.90} &54.38 &78.60 &83.97 &78.08 &45.30 &80.24 &50.00 &72.76 &12.07\\
\bottomrule
\end{tabular}}}
\end{table*}
\subsection{Detailed results on different rank setting:}
\label{sec:detailed_results}
Here, we also report the results of our GSQ-Tuning on different LlaMA model, including LlaMA2-7B (Tab.\ref{tab:llama2-7b}), LlaMA2-13B (Tab.\ref{tab:llama2-13b}), LlaMA2-70B(Tab.\ref{tab:llama2-70b}), LlaMA3-3B(Tab.\ref{tab:llama3-3b}), and LlaMA3-8B(Tab.\ref{tab:llama3-8b}). The results consistently demonstrated the effectiveness and efficiency of GSQ-Tuning.

% \begin{table*}[!t]
\renewcommand\arraystretch{1.0}
\centering
\caption{$0$-shot commonsense QA accuracy (\%) with respect to different quantization bits with 512 rank.}
\label{tab:common}
\setlength{\tabcolsep}{1.2mm}
{\resizebox{0.98\textwidth}{!}{
\begin{tabular}{lcccccccccccc|c}
\noalign{\vspace{0.3em}}
\toprule
\noalign{\vspace{0.1em}}
\textbf{Method} & \textbf{\#Bits} &\textbf{Avg.} & \textbf{ARC-c} & \textbf{ARC-e} & \textbf{BoolQ} & \textbf{HellaS.} & \textbf{OBQA} & \textbf{PIQA} & \textbf{SCIQ.} & \textbf{WinoG.} & \textbf{Mem.} \\
\midrule
\noalign{\vspace{0.1em}}
 LLaMA2-7B         &  \multirow{2}{*}{W4A16G16} & &56.3 & 78.2 & 67.1 & 67.3 & 38.2 & 72.9 & 28.4 & 58.3 \\
 \textit{+QLoRA} &  & & \textit{61.8} & \textit{78.1} & \textit{68.4} & \textit{75.8} & \textit{43.6} & \textit{73.7} & \textit{32.8} & \textit{62.0} \\
\noalign{\vspace{0.1em}}\hdashline[0.8pt/1pt]\noalign{\vspace{0.1em}}
\multirow{4}{*}{+GSQ-Tuning}  & W7A7G7  & & 54.5 & 76.5 & 66.9 & 66.1 & 36.9 & 70.9 & 27.4 & 57.0 \\
& W6A6G6  & & 54.5 & 76.5 & 66.9 & 66.1 & 36.9 & 70.9 & 27.4 & 57.0 \\
      & W5A5G5 & & 57.4 & 77.6 & 66.2 & 70.9 & 41.8 & 73.5 & 31.2 & 59.8 \\
\midrule
\noalign{\vspace{0.1em}}
 LLaMA2-13B         & \multirow{2}{*}{W4A16G16} & &56.3 & 78.2 & 67.1 & 67.3 & 38.2 & 72.9 & 28.4 & 58.3 \\
 \textit{+QLoRA} &  & & \textit{61.8} & \textit{78.1} & \textit{68.4} & \textit{75.8} & \textit{43.6} & \textit{73.7} & \textit{32.8} & \textit{62.0} \\
\noalign{\vspace{0.1em}}\hdashline[0.8pt/1pt]\noalign{\vspace{0.1em}}
 \multirow{4}{*}{W/ GSQ-Tuning}  & W7A7G7  & & 54.5 & 76.5 & 66.9 & 66.1 & 36.9 & 70.9 & 27.4 & 57.0 \\
 & W6A6G6 & & 57.4 & 77.6 & 66.2 & 70.9 & 41.8 & 73.5 & 31.2 & 59.8 \\
      & W5A5G5 & & 57.4 & 77.6 & 66.2 & 70.9 & 41.8 & 73.5 & 31.2 & 59.8 \\
\midrule
 LLaMA2-70B         & \multirow{2}{*}{W4A16G16} & &56.3 & 78.2 & 67.1 & 67.3 & 38.2 & 72.9 & 28.4 & 58.3 \\
 \textit{+QLoRA} &  & & \textit{61.8} & \textit{78.1} & \textit{68.4} & \textit{75.8} & \textit{43.6} & \textit{73.7} & \textit{32.8} & \textit{62.0} \\
\noalign{\vspace{0.1em}}\hdashline[0.8pt/1pt]\noalign{\vspace{0.1em}}
 \multirow{4}{*}{W/ GSQ-Tuning} & W7A7G7  & & 54.5 & 76.5 & 66.9 & 66.1 & 36.9 & 70.9 & 27.4 & 57.0 \\
 & W6A6G6 & & 57.4 & 77.6 & 66.2 & 70.9 & 41.8 & 73.5 & 31.2 & 59.8 \\
& W5A5G5 & & 57.4 & 77.6 & 66.2 & 70.9 & 41.8 & 73.5 & 31.2 & 59.8 \\
\midrule
 LLaMA3-3B         & \multirow{2}{*}{W4A16G16}& &56.3 & 78.2 & 67.1 & 67.3 & 38.2 & 72.9 & 28.4 & 58.3 \\
 \textit{+QLoRA} &  & & \textit{61.8} & \textit{78.1} & \textit{68.4} & \textit{75.8} & \textit{43.6} & \textit{73.7} & \textit{32.8} & \textit{62.0} \\
\noalign{\vspace{0.1em}}\hdashline[0.8pt/1pt]\noalign{\vspace{0.1em}}
 \multirow{4}{*}{W/ GSQ-Tuning} & W7A7G7  & & 54.5 & 76.5 & 66.9 & 66.1 & 36.9 & 70.9 & 27.4 & 57.0 \\
 & W6A6G6 & & 57.4 & 77.6 & 66.2 & 70.9 & 41.8 & 73.5 & 31.2 & 59.8 \\
& W5A5G5 & & 57.4 & 77.6 & 66.2 & 70.9 & 41.8 & 73.5 & 31.2 & 59.8 \\
\midrule
 LLaMA3-8B         & \multirow{2}{*}{W4A16G16} & &56.3 & 78.2 & 67.1 & 67.3 & 38.2 & 72.9 & 28.4 & 58.3 \\
 \textit{+QLoRA} &  & & \textit{61.8} & \textit{78.1} & \textit{68.4} & \textit{75.8} & \textit{43.6} & \textit{73.7} & \textit{32.8} & \textit{62.0} \\
\noalign{\vspace{0.1em}}\hdashline[0.8pt/1pt]\noalign{\vspace{0.1em}}
 \multirow{4}{*}{W/ GSQ-Tuning} & W7A7G7  & & 54.5 & 76.5 & 66.9 & 66.1 & 36.9 & 70.9 & 27.4 & 57.0 \\
 & W6A6G6 & & 57.4 & 77.6 & 66.2 & 70.9 & 41.8 & 73.5 & 31.2 & 59.8 \\
& W5A5G5 & & 57.4 & 77.6 & 66.2 & 70.9 & 41.8 & 73.5 & 31.2 & 59.8 \\
\bottomrule
\end{tabular}}}
\end{table*}

\subsection{Comparison with FP8 with 64 rank}
Here, we compare the designed GSE data format with FP8 in fully quantized fine-tuning framework with 32 rank setting. As shown in Tab.~\ref{tab:copare_fp8_r64}, the results still demonstrate that the designed GSE implemented in our GSQ-Tuning method achieves superior fine-tuning performance compared to FP8 while significantly reducing computation efficiency. Even under 5-bit settings, GSQ-Tuning maintains fine-tuning performance on par with FP8, validating its effectiveness.
\begin{table*}[!t]
\renewcommand\arraystretch{1.0}
\centering
\caption{$0$-shot accuracy comparison with FP8 in different quantization bits in 64 rank setting.}
\label{tab:copare_fp8_r64}
\setlength{\tabcolsep}{1.2mm}
{\resizebox{0.98\textwidth}{!}{
\begin{tabular}{lccccccccccccc|c}
\noalign{\vspace{0.3em}}
\toprule
\noalign{\vspace{0.1em}}
\textbf{Method} & LLMs branch &low-rank branch  &\textbf{Avg.} & \textbf{ARC-c} & \textbf{ARC-e} & \textbf{BoolQ} & \textbf{HellaS.} & \textbf{OBQA} & \textbf{PIQA} & \textbf{SCIQ.} & \textbf{WinoG.} & \textbf{Mem. (G)} \\
\midrule
\noalign{\vspace{0.1em}}
 LLaMA2-7B         &  16-16-16 & w/o & 64.13 &46.25 &74.62 &77.68 &76.01 &44.20 &79.11 &46.11 &69.06 &13.20\\
 \textit{w/ QLoRA} &  4-16-16 & 16-16-16 & 65.69 &47.14 &74.75 &79.50 &76.46 &45.50 &79.63 &50.26 &71.32 &9.73\\
\noalign{\vspace{0.1em}}\hdashline[0.8pt/1pt]\noalign{\vspace{0.1em}}
w/ FP8 & 4-8-8 & 8-8-8 & 64.46 &46.84 &73.61 &77.83 &76.03 &44.60 &79.65 &47.80 &69.38 &6.88\\
\noalign{\vspace{0.1em}}\hdashline[0.8pt/1pt]\noalign{\vspace{0.1em}}
\multirow{3}{*}{w/ GSQ-Tuning}  &4-8-8 & 8-8-8  & 65.60 &48.12 &74.24 &79.72 &76.00 &45.80 &79.60 &49.69 &71.67 &6.88\\
& 4-6-6 & 6-6-6  & 65.39 &47.70 &74.58 &79.24 &76.05 &44.60 &79.60 &50.41 &70.96 &6.17\\
  & 4-5-5 & 5-5-5 & 64.18 &45.14 &72.69 &75.20 &75.27 &46.40 &79.65 &48.62 &70.48 &5.81\\
\midrule
% \midrule
 LLaMA3-8B         & 16-16-16 & w/o & 67.18 &53.50 &77.74 &81.13 &79.20 &45.00 &80.63 &47.03 &73.24 &15.01\\
 \textit{w/ QLoRA} & 4-16-16 & 16-16-16 & 68.45 &55.63 &80.13 &83.67 &78.78 &44.80 &81.28 &50.41 &72.93 &11.71\\
\noalign{\vspace{0.1em}}\hdashline[0.8pt/1pt]\noalign{\vspace{0.1em}}
w/ FP8  & 4-8-8 & 8-8-8  & 66.46 &50.77 &76.39 &81.38 &78.19 &43.40 &79.92 &47.29 &74.35 &7.63\\
\noalign{\vspace{0.1em}}\hdashline[0.8pt/1pt]\noalign{\vspace{0.1em}}
 \multirow{3}{*}{w/ GSQ-Tuning} & 4-8-8 & 8-8-8  & 68.61 &55.97 &80.22 &83.61 &78.68 &45.20 &81.50 &50.41 &73.32 &7.63\\
 & 4-6-6 & 6-6-6 & 68.22 &55.55 &79.29 &83.67 &78.47 &44.80 &80.90 &50.05 &73.09 &6.86\\
& 4-5-5 & 5-5-5 & 66.69 &54.10 &77.99 &81.65 &77.12 &43.80 &79.54 &47.90 &71.43 &6.47\\
\bottomrule
\end{tabular}}}
\end{table*}

\end{document}
