\appendix

\section{Appendix}
\label{sec:appendix}
\subsection{Differences with Quantization-aware training (QAT): }
\label{qat_fqt}
Quantization-aware training (QAT)~\cite{choi2018pact,Zhang_2018_ECCV,zhou2017incremental,jacob2018quantization,dong2019hawq,dong2019hawqv2,shen2019q,zafrir2019q8bert,shen2020QBERT,tang2022mkq,zhang2020ternarybert,bai2020binarybert,foret2020sharpness,wang2022squat} is an \emph{inference acceleration} technique which trains networks with quantizers inserted in the forward propagation graph, so the trained network can perform efficiently during inference. 
QAT can compress activation/weights to extremely low precision (e.g. 1-2 bits). 
It is tempting to think that directly applying a quantizer for QAT to FQT can lead to similar low activation/weights bit-width. However, even only quantizing the forward propagation for FQT is much more challenging than QAT because:  \raisebox{-0.5pt}{\ding[1.1]{182\relax}} QAT requires a converged full-precision model as initialization~\cite{esser2019learned} and/or as a teacher model for knowledge distillation~\cite{bai2020binarybert}; \raisebox{-0.5pt}{\ding[1.1]{184\relax}} QAT may approximate the discrete quantizer with continuous functions during training~\cite{gong2019differentiable}, which cannot be implemented with integer arithmetic. Due to these challenges, it is still an open problem to do FQT with low-bit activations/weights. 

\begin{table*}[h]
\renewcommand\arraystretch{1.0}
\centering
\caption{$0$-shot commonsense QA accuracy (\%) across different bits and rank on llama2-7B.}
\label{tab:llama2-7b}
\setlength{\tabcolsep}{1.2mm}
{\resizebox{0.98\textwidth}{!}{
\begin{tabular}{lcccccccccccccc|c}
\noalign{\vspace{0.3em}}
\toprule
\noalign{\vspace{0.1em}}
\textbf{Method} & \textbf{rank}& LLMs branch & low-rank branch &\textbf{Avg.} & \textbf{ARC-c} & \textbf{ARC-e} & \textbf{BoolQ} & \textbf{HellaS.} & \textbf{OBQA} & \textbf{PIQA} & \textbf{SCIQ.} & \textbf{WinoG.} & \textbf{Mem. (G)} \\
\midrule
\noalign{\vspace{0.1em}}
 LLaMA2-7B    & &16-16-16 &w/o & 64.13 &46.25 &74.62 &77.68 &76.01 &44.20 &79.11 &46.11 &69.06 &13.2\\
 \noalign{\vspace{0.1em}}\hdashline[0.8pt/1pt]\noalign{\vspace{0.1em}}
 \textit{w/ QLoRA} & 16 & 4-16-16 &16-16-16 & 65.05 &47.53 &75.17 &78.59 &76.09 &44.00 &79.54 &49.44 &70.09&10.18\\
\noalign{\vspace{0.1em}}\hdashline[0.8pt/1pt]\noalign{\vspace{0.1em}}
\multirow{4}{*}{w/ GSQ-Tuning}  &\multirow{4}{*}{16} & 4-8-8 &8-8-8  & \default{\textbf{65.10}}& 47.53 &74.71 &78.35 &75.99 &45.00 &79.65 &49.28 &70.32&6.73\\
& & 4-7-7 &7-7-7  & \default{\textbf{64.96}} &47.18 &75.21 &78.10 &75.98 &44.80 &79.27 &49.95 &69.38 &5.98\\
& & 4-6-6 &6-6-6  & \default{\textbf{64.87}} &46.84 &73.78 &78.07 &75.88 &45.80 &79.22 &49.39 &70.01 &5.32\\
  & & 4-5-5 &5-5-5 & \default{63.97} &46.76 &72.64 &75.78 &74.95 &45.20 &79.05 &48.62 &68.75 &5.27\\
\midrule
\noalign{\vspace{0.1em}}
 \textit{w/ QLoRA} & 32 & 4-16-16 &16-16-16 & 65.44 &47.27 &75.04 &78.87 &76.11 &44.60 &79.76 &49.95 &70.32 &10.37\\
\noalign{\vspace{0.1em}}\hdashline[0.8pt/1pt]\noalign{\vspace{0.1em}}
\multirow{4}{*}{w/ GSQ-Tuning}  &\multirow{4}{*}{32} & 4-8-8 &8-8-8  & \default{\textbf{65.45}} &48.12 &74.71 &78.38 &76.14 &46.00 &79.71 &49.64 &70.96 &6.92\\
& & 4-7-7 &7-7-7  & \default{\textbf{65.43}} &47.35 &74.20 &78.99 &75.84 &46.00 &79.92 &49.59 &71.59 &6.16\\
& & 4-6-6 &6-6-5  & \default{\textbf{65.01}} &47.44 &74.62 &78.65 &76.03 &44.00 &79.60 &50.05 &69.69 &5.55\\
  & &4-5-5 &5-5-5 & \default{64.00} &44.97 &73.32 &75.29 &74.95 &44.60 &79.27 &48.93 &70.24&5.45\\
\midrule
 \textit{w/ QLoRA} & 64 & 4-16-16 &16-16-16 & 65.69 &47.14 &74.75 &79.50 &76.46 &45.50 &79.63 &50.26 &71.32 &10.73\\
\noalign{\vspace{0.1em}}\hdashline[0.8pt/1pt]\noalign{\vspace{0.1em}}
\multirow{4}{*}{w/ GSQ-Tuning}  &\multirow{4}{*}{64} & 4-8-8 &8-8-8  & \default{\textbf{65.60}} &48.12 &74.24 &79.72 &76.00 &45.80 &79.60 &49.69 &71.67 &7.28\\
& & 4-7-7 &7-7-7  & \default{\textbf{65.47}} &47.78 &74.71 &79.51 &76.09 &45.80 &79.60 &49.80 &70.48 &6.52\\
& & 4-6-6 &6-6-6  & \default{\textbf{65.39}} &47.70 &74.58 &79.24 &76.05 &44.60 &79.60 &50.41 &70.96 &5.97\\
  & & 4-5-5 &5-5-5 & \default{64.18} &45.14 &72.69 &75.20 &75.27 &46.40 &79.65 &48.62 &70.48 &5.81\\
\midrule
 \textit{w/ QLoRA} & 128 & 4-16-16 &16-16-16& 65.84 &48.24 &74.91 &79.78 &76.27 &45.52 &79.77 &50.48 &71.79 &11.46\\
\noalign{\vspace{0.1em}}\hdashline[0.8pt/1pt]\noalign{\vspace{0.1em}}
\multirow{4}{*}{w/ GSQ-Tuning}  &\multirow{4}{*}{128} & 4-8-8 &8-8-8  & \default{\textbf{65.79}} &48.12 &74.83 &80.28 &75.96 &45.80 &79.54 &50.61 &71.19 &8.02\\
& & 4-7-7 &7-7-7  & \default{\textbf{65.69}} &48.04 &74.87 &79.79 &76.08 &45.00 &79.49 &50.61 &71.67 &7.26\\
& & 4-6-6 &6-6-6  & \default{\textbf{65.58}} &47.87 &74.54 &80.09 &76.05 &45.40 &79.38 &50.10 &71.27 &6.10\\
  & &4-5-5 &5-5-5 & \default{64.46} &46.50 &72.77 &75.99 &75.31 &46.60 &79.00 &48.98 &70.56 &6.14\\
\midrule
 \textit{w/ QLoRA} & 256 & 4-16-16 &16-16-16 & 66.12 &48.33 &75.00 &80.94 &76.37 &45.61 &79.97 &51.13 &71.64 &12.93\\
\noalign{\vspace{0.1em}}\hdashline[0.8pt/1pt]\noalign{\vspace{0.1em}}
\multirow{4}{*}{w/ GSQ-Tuning}  &\multirow{4}{*}{256} & 4-8-8 &8-8-8  & \default{\textbf{66.19}} &48.55 &75.13 &80.76 &76.14 &47.00 &79.38 &50.72 &71.82 &9.47\\
& & 4-7-7 &7-7-7  & \default{\textbf{65.96}} &48.46 &75.08 &80.43 &76.04 &45.60 &79.76 &50.72 &71.59 &8.42\\
& & 4-6-6 &6-6-6  & \default{\textbf{65.90}} &48.38 &74.16 &79.94 &75.81 &46.80 &79.43 &50.87 &71.82 &7.66\\
  & & 4-5-5 &5-5-5 & \default{64.59} &46.33 &72.60 &76.51 &75.57 &46.40 &79.60 &49.39 &70.32 &6.75\\
\midrule
 \textit{w/ QLoRA} & 512 & 4-16-16 &16-16-16& 66.59 &49.26 &75.20 &81.99 &76.06 &46.74 &79.49 &51.71 &72.27 &15.85\\
\noalign{\vspace{0.1em}}\hdashline[0.8pt/1pt]\noalign{\vspace{0.1em}}
\multirow{4}{*}{w/ GSQ-Tuning}  &\multirow{4}{*}{512} & 4-8-8 &8-8-8  & \default{\textbf{66.52}} &49.49 &74.92 &81.28 &75.89 &47.60 &79.49 &51.59 &71.90 &11.40\\
& & 4-7-7 &7-7-7  & \default{\textbf{66.33}} &48.89 &74.75 &81.41 &76.06 &47.00 &79.54 &51.74 &71.27 &9.95\\
& & 4-6-6 &6-6-6  & \default{\textbf{66.31}} &48.55 &75.51 &80.80 &76.42 &46.00 &79.60 &51.64 &71.98&9.19\\
  & & 4-5-5 &5-5-5 & \default{64.86} &47.44 &73.15 &76.85 &75.62 &47.00 &79.33 &49.18 &70.32 &8.25\\
\bottomrule
\end{tabular}}}
\end{table*}
\begin{table*}[!t]
\renewcommand\arraystretch{1.0}
\centering
\caption{$0$-shot commonsense QA accuracy (\%) across different bits and rank on llama2-13B.}
\label{tab:llama2-13b}
\setlength{\tabcolsep}{1.2mm}
{\resizebox{0.98\textwidth}{!}{
\begin{tabular}{lcccccccccccccc|c}
\noalign{\vspace{0.3em}}
\toprule
\noalign{\vspace{0.1em}}
\textbf{Method} & \textbf{rank}& LLMs branch & low-rank branch &\textbf{Avg.} & \textbf{ARC-c} & \textbf{ARC-e} & \textbf{BoolQ} & \textbf{HellaS.} & \textbf{OBQA} & \textbf{PIQA} & \textbf{SCIQ.} & \textbf{WinoG.} & \textbf{Mem. (G)} \\
\midrule
\noalign{\vspace{0.1em}}
 LLaMA2-13B    &-     & 16-16-16 & w/o&66.65 &48.81 &76.47 &82.45 &79.67 &44.80 &80.36 &48.31 &72.38 &25.70\\
 \noalign{\vspace{0.1em}}\hdashline[0.8pt/1pt]\noalign{\vspace{0.1em}}
 \textit{w/ QLoRA} & 16 &4-16-16 & 16-16-16 & 67.32 &49.74 &76.98 &82.94 &78.85 &46.00 &80.52 &50.36 &73.16 &16.56\\
\noalign{\vspace{0.1em}}\hdashline[0.8pt/1pt]\noalign{\vspace{0.1em}}
\multirow{4}{*}{w/ GSQ-Tuning}  &\multirow{4}{*}{16} &4-8-8 & 8-8-8  & \default{\textbf{67.35}} &49.83 &77.06 &83.09 &78.89 &46.00 &80.47 &50.31 &73.16 &11.13\\
& & 4-7-7 & 7-7-7   & \default{\textbf{67.29}} &49.91 &76.94 &83.03 &78.90 &45.40 &80.58 &50.61 &73.01 &10.58\\
& & 4-6-6 & 6-6-6   & \default{\textbf{67.23}} &49.66 &76.98 &82.75 &78.79 &46.00 &80.47 &50.05 &73.16 &10.03\\
  & & 4-5-5 & 5-5-5  & \default{66.57} &49.57 &76.43 &81.62 &77.98 &45.40 &80.09 &49.39 &72.06 &9.47\\
\midrule
\noalign{\vspace{0.1em}}
 \textit{w/ QLoRA} & 32 &4-16-16 & 16-16-16 & 67.47 &49.83 &77.02 &83.24 &78.92 &46.20 &80.58 &50.77 &73.24 &16.85\\
\noalign{\vspace{0.1em}}\hdashline[0.8pt/1pt]\noalign{\vspace{0.1em}}
\multirow{4}{*}{w/ GSQ-Tuning}  &\multirow{4}{*}{32} & 4-8-8 & 8-8-8  & \default{\textbf{67.49}} &49.83 &76.98 &83.15 &78.94 &45.60 &80.79 &51.07 &73.56 &11.42\\
& & 4-7-7 & 7-7-7  & \default{\textbf{67.38}} &50.17 &77.06 &82.81 &78.99 &45.40 &80.79 &50.46 &73.40  &10.87\\
& & 4-6-6 & 6-6-6  & \default{\textbf{67.35}} &49.83 &77.06 &83.09 &78.89 &46.00 &80.47 &50.31 &73.16 &10.31\\
  & & 4-5-5 & 5-5-5 & \default{66.65} &48.38 &76.18 &82.08 &78.07 &45.60 &80.36 &49.74 &72.77 &9.76 \\
\midrule
 \textit{w/ QLoRA} & 64 &4-16-16 & 16-16-16 & 67.61 &49.66 &77.23 &83.30 &78.95 &45.40 &80.74 &51.59 &73.24 &17.42\\
\noalign{\vspace{0.1em}}\hdashline[0.8pt/1pt]\noalign{\vspace{0.1em}}
\multirow{4}{*}{w/ GSQ-Tuning}  &\multirow{4}{*}{64} & 4-8-8 & 8-8-8   & \default{\textbf{67.48}} &49.57 &77.40 &82.87 &78.88 &46.20 &80.90 &50.72 &73.32 &11.99\\
& & 4-7-7 & 7-7-7   & \default{\textbf{67.43}} &49.74 &77.27 &82.91 &78.89 &46.00 &80.90 &50.61 &73.09 &11.44\\
& & 4-6-6 & 6-6-6   & \default{\textbf{67.35}} &49.66 &77.27 &82.75 &78.66 &79.05 &80.90 &50.97 &73.16 &10.89\\
  & & 4-5-5 & 5-5-5 & \default{66.97} &49.91 &76.60 &81.87 &78.15 &46.20 &80.41 &49.54 &73.09 &10.33\\
\midrule
 \textit{w/ QLoRA} & 128&4-16-16 & 16-16-16& 67.61 &50.34 &77.40 &83.55 &78.89 &46.00 &80.85 &50.92 &72.93&18.56\\
\noalign{\vspace{0.1em}}\hdashline[0.8pt/1pt]\noalign{\vspace{0.1em}}
\multirow{4}{*}{w/ GSQ-Tuning}  &\multirow{4}{*}{128} & 4-8-8 & 8-8-8   & \default{\textbf{67.62}} &50.34 &77.06 &83.18 &78.96 &46.40 &80.69 &50.92 &73.40&13.14\\
& & 4-7-7 & 7-7-7   & \default{\textbf{67.57}} &50.43 &77.36 &83.06 &79.05 &45.60 &80.85 &51.28 &72.93 &12.58\\
& & 4-6-6 & 6-6-6   & \default{\textbf{67.53}} &50.43 &77.31 &83.15 &78.81 &45.80 &80.58 &50.97 &73.16 &12.03\\
  & & 4-5-5 & 5-5-5  & \default{67.10} &49.49 &76.81 &82.08 &78.22 &46.40 &80.03 &50.56 &73.24 &11.48\\
\midrule
 \textit{w/ QLoRA} & 256 &4-16-16 & 16-16-16 & 67.91 &50.77 &77.36 &83.64 &78.88 &46.60 &80.74 &51.69 &73.64&20.85\\
\noalign{\vspace{0.1em}}\hdashline[0.8pt/1pt]\noalign{\vspace{0.1em}}
\multirow{4}{*}{w/ GSQ-Tuning}  &\multirow{4}{*}{256} & 4-8-8 & 8-8-8   & \default{\textbf{67.84}} &51.11 &77.06 &83.82 &78.80 &46.40 &80.69 &52.00 &72.85 &15.42\\
& &4-7-7 & 7-7-7   & \default{\textbf{67.74}} &50.77 &77.31 &83.79 &78.84 &46.00 &80.63 &51.89 &72.69 &14.87\\
& & 4-6-6 & 6-6-6   & \default{\textbf{67.68}} &50.77 &77.19 &83.49 &78.82 &46.00 &80.58 &51.38 &73.24 &14.32\\
  & & 4-5-5 & 5-5-5  & \default{67.22} &50.85 &75.84 &82.11 &78.21 &46.00 &80.36 &50.92 &73.48 &13.76\\
\midrule
 \textit{w/ QLoRA} & 512 &4-16-16 & 16-16-16 & 67.94 &50.60 &77.48 &83.88 &79.00 &46.40 &80.74 &52.05 &73.40&25.43\\
\noalign{\vspace{0.1em}}\hdashline[0.8pt/1pt]\noalign{\vspace{0.1em}}
\multirow{4}{*}{w/ GSQ-Tuning}  &\multirow{4}{*}{512} & 4-8-8 & 8-8-8   & \default{\textbf{67.92}} &51.02 &77.27 &83.27 &79.04 &46.40 &81.01 &51.79 &73.56 &20.00\\
& & 4-7-7 & 7-7-7   & \default{\textbf{67.90}} &51.19 &77.15 &83.79 &78.82 &46.80 &80.69 &51.79 &73.01 &19.45\\
& & 4-6-6 & 6-6-6   & \default{\textbf{67.82}} &51.02 &77.02 &83.85 &78.93 &46.20 &80.90 &51.54 &73.09 &18.89\\
  & & 4-5-5 & 5-5-5  & \default{67.39} &50.94 &76.68 &82.29 &78.39 &46.20 &80.41 &51.69 &72.53 &18.34\\
\bottomrule
\end{tabular}}}
\end{table*}
\begin{table*}[!t]
\renewcommand\arraystretch{1.0}
\centering
\caption{$0$-shot commonsense QA accuracy (\%) across different bits and rank on llama2-70B.}
\label{tab:llama2-70b}
\setlength{\tabcolsep}{1.2mm}
{\resizebox{0.98\textwidth}{!}{
\begin{tabular}{lcccccccccccccc|c}
\noalign{\vspace{0.3em}}
\toprule
\noalign{\vspace{0.1em}}
\textbf{Method} & \textbf{rank}& LLMs branch & low-rank branch &\textbf{Avg.} & \textbf{ARC-c} & \textbf{ARC-e} & \textbf{BoolQ} & \textbf{HellaS.} & \textbf{OBQA} & \textbf{PIQA} & \textbf{SCIQ.} & \textbf{WinoG.} & \textbf{Mem. (G)} \\
\midrule
\noalign{\vspace{0.1em}}
 LLaMA2-70B    &-     &  16-16-16 &w/o & 70.68 &56.91 &80.05 &85.78 &83.59 &48.60 &82.48 &48.67 &79.40 &137.42\\
 \noalign{\vspace{0.1em}}\hdashline[0.8pt/1pt]\noalign{\vspace{0.1em}}
 \textit{w/ QLoRA} & 16 & 4-16-16 &16-16-16 & 71.72 &58.62 &81.44 &86.39 &83.92 &49.80 &83.03 &50.46 &80.11 &63.90\\
\noalign{\vspace{0.1em}}\hdashline[0.8pt/1pt]\noalign{\vspace{0.1em}}
\multirow{4}{*}{w/ GSQ-Tuning}  &\multirow{4}{*}{16} & 4-8-8&8-8-8  & \default{\textbf{71.65}} &58.62 &81.23 &86.36 &83.87 &49.60 &83.19 &50.41 &79.95 &49.17\\
& & 4-7-7&7-7-7  & \default{\textbf{71.63}} &58.87 &81.57 &86.24 &83.89 &49.20 &83.19 &50.46 &79.64 &47.44\\
& & 4-6-6&6-6-6  & \default{\textbf{71.58}} &58.62 &81.36 &86.15 &83.84 &49.60 &82.97 &50.41 &79.64 &45.72\\
  & & 4-5-5&5-5-5 & \default{71.02} &57.34 &80.56 &85.93 &83.75 &49.00 &82.59 &49.33 &79.64 &43.99\\
\midrule
\noalign{\vspace{0.1em}}
 \textit{w/ QLoRA} & 32 &4-16-16 &16-16-16 & 71.84 &59.13 &81.82 &86.27 &83.88 &49.20 &83.03 &51.02 &80.35 &64.87\\
\noalign{\vspace{0.1em}}\hdashline[0.8pt/1pt]\noalign{\vspace{0.1em}}
\multirow{4}{*}{w/ GSQ-Tuning}  &\multirow{4}{*}{32} & 4-8-8&8-8-8  & \default{\textbf{71.78}} &59.04 &81.90 &86.33 &83.89 &49.00 &83.19 &51.07 &79.79 &50.17\\
& & 4-7-7&7-7-7  & \default{\textbf{71.76}} &59.30 &81.61 &86.18 &83.98 &49.00 &83.19 &51.02 &79.79 &48.44\\
& & 4-6-6&6-6-6  & \default{\textbf{71.60}} &58.96 &81.36 &86.15 &83.87 &48.80 &83.03 &51.02 &79.64 &46.72\\
  & & 4-5-5&5-5-5 & \default{71.26} &57.59 &80.85 &86.15 &83.93 &49.00 &83.13 &50.00 &79.40 &44.99\\
\midrule
 \textit{w/ QLoRA} & 64 & 4-16-16 &16-16-16 & 72.22 &59.81 &82.20 &86.51 &83.89 &50.40 &83.13 &51.48 &80.35 &66.82\\
\noalign{\vspace{0.1em}}\hdashline[0.8pt/1pt]\noalign{\vspace{0.1em}}
\multirow{4}{*}{w/ GSQ-Tuning}  &\multirow{4}{*}{64} & 4-8-8&8-8-8  & \default{\textbf{72.20}} &59.90 &82.32 &86.51 &83.90 &50.20 &83.08 &51.59 &80.11 &52.17\\
& & 4-7-7&7-7-7  & \default{\textbf{72.18}} &59.81 &82.28 &86.39 &83.88 &50.20 &83.13 &51.54 &80.19 &50.44\\
& & 4-6-6&6-6-6  & \default{\textbf{72.10}}  &59.39 &82.15 &86.51 &83.94 &50.00 &83.30 &50.92 &80.58 &48.71\\
  & & 4-5-5&5-5-5 & \default{71.70} &58.87 &81.48 &85.90 &83.91 &49.60 &82.81 &50.67 &80.43 &46.98\\
\midrule
 \textit{w/ QLoRA} & 128 & 4-16-16 &16-16-16& 72.39& 60.67 &82.37 &86.88 &84.05 &49.20 &83.19 &52.15 &80.66&70.96\\
\noalign{\vspace{0.1em}}\hdashline[0.8pt/1pt]\noalign{\vspace{0.1em}}
\multirow{4}{*}{w/ GSQ-Tuning}  &\multirow{4}{*}{128} & 4-8-8&8-8-8  &\default{\textbf{72.37}} &60.75 &82.49 &87.00 &83.94 &49.40 &83.08 &52.15 &80.19 &56.16\\
& & 4-7-7&7-7-7  & \default{\textbf{72.32}} &60.41 &82.45 &86.94 &83.94 &49.00 &83.08 &52.15 & 80.58 &54.43\\
& & 4-6-6&6-6-6  & \default{\textbf{72.28}}  &59.81 &82.45 &86.91 &83.99 &49.60 &83.35 &51.89 & 80.27 &52.70\\
  & & 4-5-5&5-5-5 & \default{71.85} &59.47 &81.90 &86.48 &83.82 &48.20 &83.08 &51.02 & 80.82&50.97\\
\bottomrule
\end{tabular}}}
\end{table*}
\begin{table*}[!t]
\renewcommand\arraystretch{1.0}
\centering
\caption{$0$-shot commonsense QA accuracy (\%) across different bits and rank on llama3-3B.}
\label{tab:llama3-3b}
\setlength{\tabcolsep}{1.2mm}
{\resizebox{0.98\textwidth}{!}{
\begin{tabular}{lcccccccccccccc|c}
\noalign{\vspace{0.3em}}
\toprule
\noalign{\vspace{0.1em}}
\textbf{Method} & \textbf{rank}& LLMs branch &low-rank branch &\textbf{Avg.} & \textbf{ARC-c} & \textbf{ARC-e} & \textbf{BoolQ} & \textbf{HellaS.} & \textbf{OBQA} & \textbf{PIQA} & \textbf{SCIQ.} & \textbf{WinoG.} & \textbf{Mem. (G)} \\
\midrule
\noalign{\vspace{0.1em}}
 LLaMA3-3B    &-     &  16-16-16 & w/o & 64.13 &46.25 &74.62 &77.68 &76.01 &44.20 &79.11 &46.11 &69.06 &6.42\\
 \noalign{\vspace{0.1em}}\hdashline[0.8pt/1pt]\noalign{\vspace{0.1em}}
 \textit{w/ QLoRA} & 16 & 4-16-16 & 16-16-16 & 65.05 &47.53 &75.17 &78.59 &76.09 &44.00 &79.54 &49.44 &70.09 &6.42\\
\noalign{\vspace{0.1em}}\hdashline[0.8pt/1pt]\noalign{\vspace{0.1em}}
\multirow{4}{*}{w/ GSQ-Tuning}  &\multirow{4}{*}{16} & 4-8-8 & 8-8-8  & \default{\textbf{65.10}}& 47.53 &74.71 &78.35 &75.99 &45.00 &79.65 &49.28 &70.32 &3.57\\
& & 4-7-7 & 7-7-7  & \default{\textbf{64.96}} &47.18 &75.21 &78.10 &75.98 &44.80 &79.27 &49.95 &69.38 &3.34\\
& & 4-6-6 & 6-6-6  & \default{\textbf{64.87}} &46.84 &73.78 &78.07 &75.88 &45.80 &79.22 &49.39 &70.01 &3.11\\
  & & 4-5-5 & 5-5-5 & \default{63.97} &46.76 &72.64 &75.78 &74.95 &45.20 &79.05 &48.62 &68.75 &2.88\\
\midrule
\noalign{\vspace{0.1em}}
 \textit{w/ QLoRA} & 32 &4-16-16 & 16-16-16 & 65.24 &47.27 &75.04 &78.87 &76.11 &44.60 &79.76 &49.95 &70.32 &6.54\\
\noalign{\vspace{0.1em}}\hdashline[0.8pt/1pt]\noalign{\vspace{0.1em}}
\multirow{4}{*}{w/ GSQ-Tuning}  &\multirow{4}{*}{32} & 4-8-8 & 8-8-8 & \default{\textbf{65.45}} &48.12 &74.71 &78.38 &76.14 &46.00 &79.71 &49.64 &70.96 &3.69\\
& & 4-7-7 & 7-7-7  & \default{\textbf{65.43}} &47.35 &74.20 &78.99 &75.84 &46.00 &79.92 &49.59 &71.59 &3.46\\
& & 4-6-6 & 6-6-6 & \default{\textbf{65.01}} &47.44 &74.62 &78.65 &76.03 &44.00 &79.60 &50.05 &69.69 &3.23\\
  & & 4-5-5 & 5-5-5 & \default{64.00} &44.97 &73.32 &75.29 &74.95 &44.60 &79.27 &48.93 &70.24 &3.00\\
\midrule
 \textit{w/ QLoRA} & 64 & 4-16-16 & 16-16-16 & 65.69 &47.14 &74.75 &79.50 &76.46 &45.50 &79.63 &50.26 &71.32 &6.78\\
\noalign{\vspace{0.1em}}\hdashline[0.8pt/1pt]\noalign{\vspace{0.1em}}
\multirow{4}{*}{w/ GSQ-Tuning}  &\multirow{4}{*}{64} & 4-8-8 & 8-8-8  & \default{\textbf{65.60}} &48.12 &74.24 &79.72 &76.00 &45.80 &79.60 &49.69 &71.67 &3.93\\
& & 4-7-7 & 7-7-7  & \default{\textbf{65.47}} &47.78 &74.71 &79.51 &76.09 &45.80 &79.60 &49.80 &70.48 &3.70\\
& & 4-6-6 & 6-6-6  & \default{\textbf{65.39}} &47.70 &74.58 &79.24 &76.05 &44.60 &79.60 &50.41 &70.96 &3.47\\
  & &4-5-5 & 5-5-5 & \default{64.18} &45.14 &72.69 &75.20 &75.27 &46.40 &79.65 &48.62 &70.48 &3.24\\
\midrule
 \textit{w/ QLoRA} & 128 & 4-16-16 & 16-16-16& 65.84 &48.24 &74.91 &79.78 &76.27 &45.52 &79.77 &50.48 &71.79 &6.76\\
\noalign{\vspace{0.1em}}\hdashline[0.8pt/1pt]\noalign{\vspace{0.1em}}
\multirow{4}{*}{w/ GSQ-Tuning}  &\multirow{4}{*}{128} & 4-8-8 & 8-8-8  & \default{\textbf{65.79}} &48.12 &74.83 &80.28 &75.96 &45.80 &79.54 &50.61 &71.19 &4.41\\
& & 4-7-7 & 7-7-7  & \default{\textbf{65.69}} &48.04 &74.87 &79.79 &76.08 &45.00 &79.49 &50.61 &71.67 &4.18\\
& & 4-6-6 & 6-6-6  & \default{\textbf{65.58}} &47.87 &74.54 &80.09 &76.05 &45.40 &79.38 &50.10 &71.27 &3.95\\
  & & 4-5-5 & 5-5-5 & \default{64.46} &46.50 &72.77 &75.99 &75.31 &46.60 &79.00 &48.98 &70.56 &3.72\\
\midrule
 \textit{w/ QLoRA} & 256 &4-16-16 & 16-16-16 & 66.12 &48.33 &75.00 &80.94 &76.37 &45.61 &79.97 &51.13 &71.64 &7.61\\
\noalign{\vspace{0.1em}}\hdashline[0.8pt/1pt]\noalign{\vspace{0.1em}}
\multirow{4}{*}{w/ GSQ-Tuning}  &\multirow{4}{*}{256} & 4-8-8 & 8-8-8  & \default{\textbf{66.19}} &48.55 &75.13 &80.76 &76.14 &47.00 &79.38 &50.72 &71.82 &5.37\\
& & 4-7-7 & 7-7-7  & \default{\textbf{65.96}} &48.46 &75.08 &80.43 &76.04 &45.60 &79.76 &50.72 &71.59 &5.13\\
& & 4-6-6 & 6-6-6  & \default{\textbf{65.90}} &48.38 &74.16 &79.94 &75.81 &46.80 &79.43 &50.87 &71.82 &4.90\\
  & & 4-5-5 & 5-5-5 & \default{64.59} &46.33 &72.60 &76.51 &75.57 &46.40 &79.60 &49.39 &70.32 &4.67\\
\midrule
 \textit{w/ QLoRA} & 512 & 4-16-16 & 16-16-16 & 66.59 &49.26 &75.20 &81.99 &76.06 &46.74 &79.49 &51.71 &72.27&9.73\\
\noalign{\vspace{0.1em}}\hdashline[0.8pt/1pt]\noalign{\vspace{0.1em}}
\multirow{4}{*}{w/ GSQ-Tuning}  &\multirow{4}{*}{512} & 4-8-8 & 8-8-8  & \default{\textbf{66.52}} &49.49 &74.92 &81.28 &75.89 &47.60 &79.49 &51.59 &71.90&7.28\\
& & 4-7-7 & 7-7-7  & \default{\textbf{66.33}} &48.89 &74.75 &81.41 &76.06 &47.00 &79.54 &51.74 &71.27&7.05\\
& & 4-6-6 & 6-6-6 & \default{\textbf{66.31}} &48.55 &75.51 &80.80 &76.42 &46.00 &79.60 &51.64 &71.98 &6.82\\
  & & 4-5-5 & 5-5-5 & \default{64.86} &47.44 &73.15 &76.85 &75.62 &47.00 &79.33 &49.18 &70.32 &6.59\\
\bottomrule
\end{tabular}}}
\end{table*}
\begin{table*}[!t]
\renewcommand\arraystretch{1.0}
\centering
\caption{$0$-shot commonsense QA accuracy (\%) across different bits and rank on llama3-8B.}
\label{tab:llama3-8b}
\setlength{\tabcolsep}{1.2mm}
{\resizebox{0.98\textwidth}{!}{
\begin{tabular}{lcccccccccccccc|c}
\noalign{\vspace{0.3em}}
\toprule
\noalign{\vspace{0.1em}}
\textbf{Method} & \textbf{rank}& LLMs branch &low-rank branch &\textbf{Avg.} & \textbf{ARC-c} & \textbf{ARC-e} & \textbf{BoolQ} & \textbf{HellaS.} & \textbf{OBQA} & \textbf{PIQA} & \textbf{SCIQ.} & \textbf{WinoG.} & \textbf{Mem. (G)} \\
\midrule
\noalign{\vspace{0.1em}}
 LLaMA3-8B    &-     &  16-16-16 & w/o & 67.18 &53.50 &77.74 &81.13 &79.20 &45.00 &80.63 &47.03 &73.24 &15.01\\
 \noalign{\vspace{0.1em}}\hdashline[0.8pt/1pt]\noalign{\vspace{0.1em}}
 \textit{w/ QLoRA} & 16 & 4-16-16 &16-16-16 & 68.14 &54.52 &79.50 &83.43 &78.66 &44.80 &80.85 &50.00 &73.32 &10.71\\
\noalign{\vspace{0.1em}}\hdashline[0.8pt/1pt]\noalign{\vspace{0.1em}}
\multirow{4}{*}{w/ GSQ-Tuning}  &\multirow{4}{*}{16} &  4-8-8 & 8-8-8  & \default{\textbf{68.16}} &54.61 &79.84 &83.70 &78.58 &44.80 &80.79 &49.85 &73.16 &7.03\\
& & 4-7-7 & 7-7-7  & \default{\textbf{68.00}} &54.01 &79.29 &83.46 &78.65 &45.00 &80.85 &49.80 &73.01 &6.65\\
& & 4-6-6 & 6-6-6  & \default{\textbf{67.74}} &54.01 &78.70 &83.09 &78.49 &44.00 &80.90 &49.44 &73.32 &6.26\\
  & & 4-5-5 & 5-5-5 & \default{66.51} &51.54 &77.27 &81.99 &77.00 &44.40 &78.84 &48.46 &72.61 &5.87\\
\midrule
\noalign{\vspace{0.1em}}
 \textit{w/ QLoRA} & 32 &4-16-16 &16-16-16 & 68.31 &55.55 &80.39 &83.36 &78.65 &44.60 &81.28 &50.05 &72.61 &11.02\\
\noalign{\vspace{0.1em}}\hdashline[0.8pt/1pt]\noalign{\vspace{0.1em}}
\multirow{4}{*}{w/ GSQ-Tuning}  &\multirow{4}{*}{32} & 4-8-8 & 8-8-8  & \default{\textbf{68.45}} &55.72 &80.22 &83.43 &78.60 &45.00  &81.18 &50.20 &73.32 &7.23\\
& & 4-7-7 & 7-7-7  & \default{\textbf{68.29}} &54.95 &80.13 &83.36 &78.53 &44.80 &81.01 &50.20 &73.32 &6.84\\
& & 4-6-6 & 6-6-6  & \default{\textbf{68.08}} &55.29 &79.29 &83.55 &78.28 &45.80 &81.07 &49.39 &71.98 &6.46\\
  & & 4-5-5 & 5-5-5 & \default{66.48} &51.71 &77.69 &82.11 &76.91 &44.20 &79.43 &48.16 &71.67 &6.07\\
\midrule
 \textit{w/ QLoRA} & 64 & 4-16-16 &16-16-16 & 68.45 &55.63 &80.13 &83.67 &78.78 &44.80 &81.28 &50.41 &72.93 &11.64\\
\noalign{\vspace{0.1em}}\hdashline[0.8pt/1pt]\noalign{\vspace{0.1em}}
\multirow{4}{*}{w/ GSQ-Tuning}  &\multirow{4}{*}{64} & 4-8-8 & 8-8-8  & \default{\textbf{68.61}} &55.97 &80.22 &83.61 &78.68 &45.20 &81.50 &50.41 &73.32 &7.63\\
& & 4-7-7 & 7-7-7  & \default{\textbf{68.57}}  &55.97 &80.68 &83.73 &78.84 &45.20 &81.01 &50.26 &72.85 &7.24\\
& & 4-6-6 & 6-6-6  & \default{\textbf{68.22}}  &55.55 &79.29 &83.67 &78.47 &44.80 &80.90 &50.05 &73.09 &6.86\\
  & & 4-5-5 & 5-5-5 & \default{66.69} &54.10 &77.99 &81.65 &77.12 &43.80 &79.54 &47.90 &71.43 &6.47\\
\midrule
 \textit{w/ QLoRA} & 128 & 4-16-16 &16-16-16 & 68.77 &56.14 &80.56 &83.98 &79.03 &45.60 &81.34 &50.56 &72.93 &12.13\\
\noalign{\vspace{0.1em}}\hdashline[0.8pt/1pt]\noalign{\vspace{0.1em}}
\multirow{4}{*}{w/ GSQ-Tuning}  &\multirow{4}{*}{128} & 4-8-8 & 8-8-8  & \default{\textbf{68.72}} &56.57 &80.22 &83.82 &78.80 &45.40 &81.23 &50.41 &73.32 &8.43\\
& & 4-7-7 & 7-7-7  & \default{\textbf{68.71}} &56.48 &80.18 &83.88 &78.78 &45.80 &81.34 &50.36 &72.93 &8.04\\
& & 4-6-6 & 6-6-6  & \default{\textbf{68.67}} &56.91 &79.50 &83.79 &78.71 &46.60 &80.52 &50.36 &73.01 &7.66\\
  & &4-5-5 & 5-5-5 & \default{66.92} &52.47 &78.45 &82.63 &77.22 &44.60 &79.49 &48.52 &71.98&7.27\\
\midrule
 \textit{w/ QLoRA} & 256 & 4-16-16 &16-16-16 & 69.09 &56.74 &80.35 &84.56 &79.02 &45.20 &81.83 &50.92 &74.11&13.81\\
\noalign{\vspace{0.1em}}\hdashline[0.8pt/1pt]\noalign{\vspace{0.1em}}
\multirow{4}{*}{w/ GSQ-Tuning}  &\multirow{4}{*}{256} & 4-8-8 & 8-8-8  & \default{\textbf{69.04}} &56.57 &80.85 &84.07 &78.97 &45.40 &81.45 &51.28 &73.72 &10.03 \\
& & 4-7-7 & 7-7-7  & \default{\textbf{69.00}} &56.83 &80.89 &84.25 &78.96 &45.60 &81.50 &50.46 &73.56 &9.64\\
& & 4-6-6 & 6-6-6  & \default{\textbf{68.84}} &56.74 &79.80 &83.98 &78.84 &46.40 &81.12 &50.77 &73.09 &9.26\\
  & & 4-5-5 & 5-5-5 & \default{67.54} &53.33 &78.49 &83.21 &77.38 &44.60 &79.98 &48.93 &73.64 &8.87\\
\midrule
 \textit{w/ QLoRA} & 512 &4-16-16 &16-16-16 & 69.18 &57.17 &80.30 &84.65 &79.28 &46.40 &81.07 &50.36 &74.27 &16.81\\
\noalign{\vspace{0.1em}}\hdashline[0.8pt/1pt]\noalign{\vspace{0.1em}}
\multirow{4}{*}{w/ GSQ-Tuning}  &\multirow{4}{*}{512} & 4-8-8 & 8-8-8  & \default{\textbf{69.24}} &56.48 &80.47 &85.35 &79.13 &45.40 &81.56 &51.54 &74.03 &13.23\\
& & 4-7-7 & 7-7-7 & \default{\textbf{69.16}} &56.40 &80.68 &85.26 &79.10 &45.80 &81.28 &51.13 &73.64 &12.84\\
& & 4-6-6 & 6-6-6  & \default{\textbf{69.01}} &56.57 &80.01 &84.56 &78.84 &45.80 &81.23 &51.64 &73.48 &12.45\\
  & & 4-5-5 & 5-5-5 & \default{67.90} &54.38 &78.60 &83.97 &78.08 &45.30 &80.24 &50.00 &72.76 &12.07\\
\bottomrule
\end{tabular}}}
\end{table*}
\subsection{Detailed results on different rank setting:}
\label{sec:detailed_results}
Here, we also report the results of our GSQ-Tuning on different LlaMA model, including LlaMA2-7B (Tab.\ref{tab:llama2-7b}), LlaMA2-13B (Tab.\ref{tab:llama2-13b}), LlaMA2-70B(Tab.\ref{tab:llama2-70b}), LlaMA3-3B(Tab.\ref{tab:llama3-3b}), and LlaMA3-8B(Tab.\ref{tab:llama3-8b}). The results consistently demonstrated the effectiveness and efficiency of GSQ-Tuning.

% \begin{table*}[!t]
\renewcommand\arraystretch{1.0}
\centering
\caption{$0$-shot commonsense QA accuracy (\%) with respect to different quantization bits with 512 rank.}
\label{tab:common}
\setlength{\tabcolsep}{1.2mm}
{\resizebox{0.98\textwidth}{!}{
\begin{tabular}{lcccccccccccc|c}
\noalign{\vspace{0.3em}}
\toprule
\noalign{\vspace{0.1em}}
\textbf{Method} & \textbf{\#Bits} &\textbf{Avg.} & \textbf{ARC-c} & \textbf{ARC-e} & \textbf{BoolQ} & \textbf{HellaS.} & \textbf{OBQA} & \textbf{PIQA} & \textbf{SCIQ.} & \textbf{WinoG.} & \textbf{Mem.} \\
\midrule
\noalign{\vspace{0.1em}}
 LLaMA2-7B         &  \multirow{2}{*}{W4A16G16} & &56.3 & 78.2 & 67.1 & 67.3 & 38.2 & 72.9 & 28.4 & 58.3 \\
 \textit{+QLoRA} &  & & \textit{61.8} & \textit{78.1} & \textit{68.4} & \textit{75.8} & \textit{43.6} & \textit{73.7} & \textit{32.8} & \textit{62.0} \\
\noalign{\vspace{0.1em}}\hdashline[0.8pt/1pt]\noalign{\vspace{0.1em}}
\multirow{4}{*}{+GSQ-Tuning}  & W7A7G7  & & 54.5 & 76.5 & 66.9 & 66.1 & 36.9 & 70.9 & 27.4 & 57.0 \\
& W6A6G6  & & 54.5 & 76.5 & 66.9 & 66.1 & 36.9 & 70.9 & 27.4 & 57.0 \\
      & W5A5G5 & & 57.4 & 77.6 & 66.2 & 70.9 & 41.8 & 73.5 & 31.2 & 59.8 \\
\midrule
\noalign{\vspace{0.1em}}
 LLaMA2-13B         & \multirow{2}{*}{W4A16G16} & &56.3 & 78.2 & 67.1 & 67.3 & 38.2 & 72.9 & 28.4 & 58.3 \\
 \textit{+QLoRA} &  & & \textit{61.8} & \textit{78.1} & \textit{68.4} & \textit{75.8} & \textit{43.6} & \textit{73.7} & \textit{32.8} & \textit{62.0} \\
\noalign{\vspace{0.1em}}\hdashline[0.8pt/1pt]\noalign{\vspace{0.1em}}
 \multirow{4}{*}{W/ GSQ-Tuning}  & W7A7G7  & & 54.5 & 76.5 & 66.9 & 66.1 & 36.9 & 70.9 & 27.4 & 57.0 \\
 & W6A6G6 & & 57.4 & 77.6 & 66.2 & 70.9 & 41.8 & 73.5 & 31.2 & 59.8 \\
      & W5A5G5 & & 57.4 & 77.6 & 66.2 & 70.9 & 41.8 & 73.5 & 31.2 & 59.8 \\
\midrule
 LLaMA2-70B         & \multirow{2}{*}{W4A16G16} & &56.3 & 78.2 & 67.1 & 67.3 & 38.2 & 72.9 & 28.4 & 58.3 \\
 \textit{+QLoRA} &  & & \textit{61.8} & \textit{78.1} & \textit{68.4} & \textit{75.8} & \textit{43.6} & \textit{73.7} & \textit{32.8} & \textit{62.0} \\
\noalign{\vspace{0.1em}}\hdashline[0.8pt/1pt]\noalign{\vspace{0.1em}}
 \multirow{4}{*}{W/ GSQ-Tuning} & W7A7G7  & & 54.5 & 76.5 & 66.9 & 66.1 & 36.9 & 70.9 & 27.4 & 57.0 \\
 & W6A6G6 & & 57.4 & 77.6 & 66.2 & 70.9 & 41.8 & 73.5 & 31.2 & 59.8 \\
& W5A5G5 & & 57.4 & 77.6 & 66.2 & 70.9 & 41.8 & 73.5 & 31.2 & 59.8 \\
\midrule
 LLaMA3-3B         & \multirow{2}{*}{W4A16G16}& &56.3 & 78.2 & 67.1 & 67.3 & 38.2 & 72.9 & 28.4 & 58.3 \\
 \textit{+QLoRA} &  & & \textit{61.8} & \textit{78.1} & \textit{68.4} & \textit{75.8} & \textit{43.6} & \textit{73.7} & \textit{32.8} & \textit{62.0} \\
\noalign{\vspace{0.1em}}\hdashline[0.8pt/1pt]\noalign{\vspace{0.1em}}
 \multirow{4}{*}{W/ GSQ-Tuning} & W7A7G7  & & 54.5 & 76.5 & 66.9 & 66.1 & 36.9 & 70.9 & 27.4 & 57.0 \\
 & W6A6G6 & & 57.4 & 77.6 & 66.2 & 70.9 & 41.8 & 73.5 & 31.2 & 59.8 \\
& W5A5G5 & & 57.4 & 77.6 & 66.2 & 70.9 & 41.8 & 73.5 & 31.2 & 59.8 \\
\midrule
 LLaMA3-8B         & \multirow{2}{*}{W4A16G16} & &56.3 & 78.2 & 67.1 & 67.3 & 38.2 & 72.9 & 28.4 & 58.3 \\
 \textit{+QLoRA} &  & & \textit{61.8} & \textit{78.1} & \textit{68.4} & \textit{75.8} & \textit{43.6} & \textit{73.7} & \textit{32.8} & \textit{62.0} \\
\noalign{\vspace{0.1em}}\hdashline[0.8pt/1pt]\noalign{\vspace{0.1em}}
 \multirow{4}{*}{W/ GSQ-Tuning} & W7A7G7  & & 54.5 & 76.5 & 66.9 & 66.1 & 36.9 & 70.9 & 27.4 & 57.0 \\
 & W6A6G6 & & 57.4 & 77.6 & 66.2 & 70.9 & 41.8 & 73.5 & 31.2 & 59.8 \\
& W5A5G5 & & 57.4 & 77.6 & 66.2 & 70.9 & 41.8 & 73.5 & 31.2 & 59.8 \\
\bottomrule
\end{tabular}}}
\end{table*}

\subsection{Comparison with FP8 with 64 rank}
Here, we compare the designed GSE data format with FP8 in fully quantized fine-tuning framework with 32 rank setting. As shown in Tab.~\ref{tab:copare_fp8_r64}, the results still demonstrate that the designed GSE implemented in our GSQ-Tuning method achieves superior fine-tuning performance compared to FP8 while significantly reducing computation efficiency. Even under 5-bit settings, GSQ-Tuning maintains fine-tuning performance on par with FP8, validating its effectiveness.
\begin{table*}[!t]
\renewcommand\arraystretch{1.0}
\centering
\caption{$0$-shot accuracy comparison with FP8 in different quantization bits in 64 rank setting.}
\label{tab:copare_fp8_r64}
\setlength{\tabcolsep}{1.2mm}
{\resizebox{0.98\textwidth}{!}{
\begin{tabular}{lccccccccccccc|c}
\noalign{\vspace{0.3em}}
\toprule
\noalign{\vspace{0.1em}}
\textbf{Method} & LLMs branch &low-rank branch  &\textbf{Avg.} & \textbf{ARC-c} & \textbf{ARC-e} & \textbf{BoolQ} & \textbf{HellaS.} & \textbf{OBQA} & \textbf{PIQA} & \textbf{SCIQ.} & \textbf{WinoG.} & \textbf{Mem. (G)} \\
\midrule
\noalign{\vspace{0.1em}}
 LLaMA2-7B         &  16-16-16 & w/o & 64.13 &46.25 &74.62 &77.68 &76.01 &44.20 &79.11 &46.11 &69.06 &13.20\\
 \textit{w/ QLoRA} &  4-16-16 & 16-16-16 & 65.69 &47.14 &74.75 &79.50 &76.46 &45.50 &79.63 &50.26 &71.32 &9.73\\
\noalign{\vspace{0.1em}}\hdashline[0.8pt/1pt]\noalign{\vspace{0.1em}}
w/ FP8 & 4-8-8 & 8-8-8 & 64.46 &46.84 &73.61 &77.83 &76.03 &44.60 &79.65 &47.80 &69.38 &6.88\\
\noalign{\vspace{0.1em}}\hdashline[0.8pt/1pt]\noalign{\vspace{0.1em}}
\multirow{3}{*}{w/ GSQ-Tuning}  &4-8-8 & 8-8-8  & 65.60 &48.12 &74.24 &79.72 &76.00 &45.80 &79.60 &49.69 &71.67 &6.88\\
& 4-6-6 & 6-6-6  & 65.39 &47.70 &74.58 &79.24 &76.05 &44.60 &79.60 &50.41 &70.96 &6.17\\
  & 4-5-5 & 5-5-5 & 64.18 &45.14 &72.69 &75.20 &75.27 &46.40 &79.65 &48.62 &70.48 &5.81\\
\midrule
% \midrule
 LLaMA3-8B         & 16-16-16 & w/o & 67.18 &53.50 &77.74 &81.13 &79.20 &45.00 &80.63 &47.03 &73.24 &15.01\\
 \textit{w/ QLoRA} & 4-16-16 & 16-16-16 & 68.45 &55.63 &80.13 &83.67 &78.78 &44.80 &81.28 &50.41 &72.93 &11.71\\
\noalign{\vspace{0.1em}}\hdashline[0.8pt/1pt]\noalign{\vspace{0.1em}}
w/ FP8  & 4-8-8 & 8-8-8  & 66.46 &50.77 &76.39 &81.38 &78.19 &43.40 &79.92 &47.29 &74.35 &7.63\\
\noalign{\vspace{0.1em}}\hdashline[0.8pt/1pt]\noalign{\vspace{0.1em}}
 \multirow{3}{*}{w/ GSQ-Tuning} & 4-8-8 & 8-8-8  & 68.61 &55.97 &80.22 &83.61 &78.68 &45.20 &81.50 &50.41 &73.32 &7.63\\
 & 4-6-6 & 6-6-6 & 68.22 &55.55 &79.29 &83.67 &78.47 &44.80 &80.90 &50.05 &73.09 &6.86\\
& 4-5-5 & 5-5-5 & 66.69 &54.10 &77.99 &81.65 &77.12 &43.80 &79.54 &47.90 &71.43 &6.47\\
\bottomrule
\end{tabular}}}
\end{table*}