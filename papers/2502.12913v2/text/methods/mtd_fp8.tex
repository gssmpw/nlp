\paragraph{Low-bitwidth Floating Point (FP).}
Floating-point numbers are a commonly used data representation in deep learning. For instance, FP16 represents each number using 16 bits. Recently, lower-bit floating-point representations, such as FP8, have been introduced into the training processes of deep learning models~\cite{micikevicius2022fp8,baalen2023fp8}. FP8 operates in two modes: the E4M3 and E5M2 formats. In these formats, E represents the number of exponent bits and M denotes that of mantissa bits.

Similar to quantization methods, low-bitwidth FP formats can effectively reduce memory storage requirements and decrease the hardware area and energy consumption of computational units. However, we observe that low-bitwidth FP may not be the optimal solution for LoRA fine-tuning in large-scale models, primarily due to the following 3 reasons: (1) Neural network tensors exhibit spatial locality, meaning that adjacent elements within a tensor tend to have similar magnitudes, leading to redundancy in the exponent bits of FP representations. As illustrated in Fig.~\ref{fig:weights_std}, the standard deviation of the values in the weight tensor is considerably lower than the magnitude of the values, indicating a small local variation; (2) The limited number of mantissa bits in low-bitwidth FP formats constrains precision, potentially impairing model performance. For instance, the E5M2 format, which has only two mantissa bits, is incapable of representing certain integers below ten, such as 5, 7, and 9; (3) FP computation demonstrates less efficiency compared to INT computation, making it less suitable for resource-constrained environments. As shown in Table~\ref{tab:hardware_consumption}, FP formats incur considerably higher costs in  power and chip area compared to integer-based computation.


Due to the inherent characteristics of FP representations and the requirement for relatively high precision in training, it is crucial to explore other data format to reduce both hardware area and energy consumption in resource-constrained cases.

\begin{figure}[t]
    \centering
    \includegraphics[width=\linewidth]{figures/plot_max_magnitude_weights.pdf}
    \vspace{-10pt}
    \includegraphics[width=\linewidth]{figures/plot_std_weights_log.pdf}
    % \vspace{-10pt}
    \caption{\textbf{In each layer, the weights' magnitudes are similar.} The standard deviations of weights across layers are less than $2^{-2}$ by 3-$\sigma$ (about probability 99.7\%). The weights are from Vicuna-7B-v1.5.}
    \label{fig:weights_std}
\end{figure}