\documentclass[journal]{IEEEtran}
\usepackage{changepage, graphicx}
\usepackage{algorithm, multirow}
\usepackage{algorithmic, CJK}
\usepackage{amsmath, indentfirst}
\usepackage{amssymb}
\usepackage{float}
% \usepackage[colorlinks,linkcolor=blue,anchorcolor=blue,linktocpage=true,urlcolor=blue,citecolor=blue,citebordercolor={0  0  1}]{hyperref}
\usepackage{subfigure}
\usepackage{cite}
\usepackage{url}
\usepackage{color, soul}
\usepackage{color}
\usepackage{framed}
\usepackage{array}
\usepackage{mathrsfs}
\usepackage{booktabs}
\usepackage[colorlinks,urlcolor=black,linkcolor=black,citecolor=black,anchorcolor=black,filecolor=black,menucolor=black,runcolor=black]{hyperref} % 设置链接颜色 

\definecolor{shadecolor}{rgb}{1,0.94509804,0}
\hyphenation{op-tical net-works semi-conduc-tor}
\soulregister\cite7 % 针对\cite命令
\soulregister\citep7 % 针对\citep命令
\soulregister\citet7 % 针对\citet命令
\soulregister\ref7 % 针对\ref命令
\soulregister\pageref7 % 针对\pageref命令
\soulregister\math7 % 针对\pageref命令
\soulregister\begin7 % 针对\pageref命令

\makeatletter
\newenvironment{breakablealgorithm}
{% \begin{breakablealgorithm}
	\begin{center}
		\refstepcounter{algorithm}% New algorithm
		\hrule height.8pt depth0pt \kern2pt% \@fs@pre for \@fs@ruled
		\renewcommand{\caption}[2][\relax]{% Make a new \caption
			{\raggedright\textbf{\ALG@name~\thealgorithm} ##2\par}%
			\ifx\relax##1\relax % #1 is \relax
			\addcontentsline{loa}{algorithm}{\protect\numberline{\thealgorithm}##2}%
			\else % #1 is not \relax
			\addcontentsline{loa}{algorithm}{\protect\numberline{\thealgorithm}##1}%
			\fi
			\kern2pt\hrule\kern2pt
		}
	}{% \end{breakablealgorithm}
		\kern2pt\hrule\relax% \@fs@post for \@fs@ruled
	\end{center}
}

\begin{document}
\title{A Split-Window Transformer for Multi-Model Sequence Spammer Detection using Multi-Model Variational Autoencoder}
\author{
	Zhou~Yang, Yucai~Pang, Hongbo Yin, Yunpeng~Xiao 
	\thanks{This paper is partially supported by the National Natural Science Foundation of China (Grant No.62072066, 62006032), the Key Cooperation Project of Chongqing Municipal Education Commission(Grant No.HZ2021008) and Youth Innovation Group Support Program of ICE Discipline of CQUPT (Grant No.SCIE-QN-2022-05).\emph{(Corresponding author: Yucai Pang and Yunpeng Xiao.)}}
	
	\thanks{Z. Yang, Y. Pang, and Y. Xiao are with the School of Communications and Information Engineering, Chongqing University of Posts and Telecommunications, Chongqing, 400065, China (e-mail: yzhoul392@gmail.com;  pangyc@cqupt.edu.cn; xiaoyp@cqupt.edu.cn; ).}
	
	\thanks{H. Yin is with the School of Information and Communication Engineering, University of Electronic Science and Technology of China, Chengdu, 611731, China (e-mail: yinhub@yeah.net).}
}
\markboth{} %
{Shell \MakeLowercase{\textit{et al.}}: Bare Demo of IEEEtran.cls for IEEE Journals}
\maketitle
\begin{abstract}
	This paper introduces a new Transformer, called MS$^2$Dformer, that can be used as a generalized backbone for \textbf{m}ulti-modal \textbf{s}equence \textbf{s}pammer \textbf{d}etection. Spammer detection is a complex multi-modal task, thus the challenges of applying Transformer are two-fold. Firstly, complex multi-modal noisy information about users can interfere with feature mining. Secondly, the long sequence of users' historical behaviors also puts a huge GPU memory pressure on the attention computation. To solve these problems, we first design a user behavior Tokenization algorithm based on the multi-modal variational autoencoder (MVAE). Subsequently, a hierarchical split-window multi-head attention (SW/W-MHA) mechanism is proposed. The split-window strategy transforms the ultra-long sequences hierarchically into a combination of intra-window short-term and inter-window overall attention. Pre-trained on the public datasets, MS$^2$Dformer's performance far exceeds the previous state of the art. The experiments demonstrate MS$^2$Dformer's\footnote{\url{https://github.com/yzhouli/MSDformer.}} ability to act as a backbone.
	
	% 本文介绍了一种名为MSDformer的新型Transformer,它可以作为多模态序列垃圾邮件发送者检测的通用骨干。垃圾邮件发送者检测是一项复杂多模态任务,因此将Transformer应用在这项任务所面临的挑战有两项。首先,用户复杂多模态噪声信息会干扰特征挖掘。其次,超长用户历史行为序列也为注意力计算造成巨大显存压力。为了解决这些问题,我们首先设计了基于多模态变分自编码器(MVAE)的用户行为Token化算法。随后,提出分层分割窗口的多头注意力(SW-MHA)计算机制。分割窗口策略将超长序列分层转化为窗口内短期和窗口间总体注意力结合计算,显著降低了MHA计算过程的显存爆炸隐患,同时加速推理。垃圾邮件发送者通常采用长期隐藏或短期直接引导方式推动舆论发展。随后,在两个公开数据集(Weibo和Twitter)中进行预训练。MSDformer的性能大大超过了之前的技术水平,在Weibo中达到了+9% Acc.和+7% AUC,在Twitter中达到了+7% AUC。实验证明了MSDformer作为多模态垃圾邮件发送者检测骨干的能力。公开数据集和模型代码在https://github.com/yzhouli/Spammer_Weibo。 
\end{abstract}
\begin{IEEEkeywords}
	Spammer Detection, Multi-modal Representation, Multi-modal Variational Autoencoder, Split-Window Attention Mechanism, Social Network Analysis
\end{IEEEkeywords}
\section{Introduction}
%垃圾邮件发送者是引导社交舆论走向的重要推手。长期以来,制造垃圾邮件或虚假新闻的用户被定义为垃圾邮件发送者。信息传播通常以图结构为载体进行传播。因此,图神经网络(GNN)成为识别垃圾邮件发送者的通用骨干。随后,GNN进一步发展,通过结合卷积(GCN)和注意力(GAT、Graph Transformer)思想,GNN性能进一步加强。
\IEEEPARstart{S}pammers are important promoters of directing social opinion. Users who create spam or fake news for a long time are defined as spammers. Information dissemination is usually carried out by graph structures. Therefore, the graph neural network (GNN) becomes a generalized backbone for identifying spammers. Subsequently, GNN has been further developed. By combining the ideas of Convolution (GCN\cite{li2019spam}), Attention (GAT\cite{zhang2023detecting, jiang2024learning} and Graph Transformer\cite{chen2024gnn}), and Sampling (Graph-SAGE\cite{zhang2024predicting}), the performance of GNN is further improved.
\begin{figure}[htbp]
	\center{\includegraphics[width=1\linewidth] {./image/inspire.pdf}} 
	\caption{Some examples of complex cross-modal feature mining challenges. (a) and (b) are both standard cases, i.e., the core argument can be identified in the text modal, and the multi-modal provides auxiliary features. (c) the text modal does not directly provide the point of view, so further alignment and mining of the core argument in conjunction with the image is required. (d) the situation is most common in social behavior. User behavior from the text aspect is often accompanied by noise features, i.e., emojis and non-common characters (@, \#, and //, etc.). In special cases, it is also mixed with URL linking to elaborate the argument.}
	\label{fig-inspire}
\end{figure}
\begin{figure*}[h]
	\center{\includegraphics[width=1\linewidth]  {./image/MHA.pdf}} 
	\caption{Inspiration for hierarchical attention mechanisms. (a) is the classical multi-head attention mechanism (MHA). (b-h) are sparse multi-head attention (SMHA) for solving ultra-long sequence modeling. (b) and (c) are SMHAs based on split windows. The core idea of them all is to limit the receptive field of an individual element, thus reducing the computational effort of $QK^\text{T}$. Among them, (c) expands the windowed receptive field similarly to the dilated convolution. Because the CPU dominates the windowing process, the full SMHA computation is slow (see Table \ref{table-ab-memory}). To solve this problem, the researchers adjusted the sliding distance to be consistent with the window length, thus proposing the block split-window mechanism (see (d) and (h)). Meanwhile, considering the importance of CLS tokens, Longformer (e\cite{beltagy2020longformer}) proposes global attention based on (d). Subsequently, random sampling is also added (see (f) and (g)).}
	\label{fig-MHAs}
\end{figure*}
%复杂跨模态特征挖掘挑战举例。(a)和(b)均为标准情况,即在文本模态就可以识别核心论点,多模态提供辅助特征。(c)中文本模态并不能直接提供观点,因此需要结合图像进一步对齐和挖掘核心论点。(d)情况在社交行为中最常见。用户行为从文本方面常常伴随噪声特征,即表情包和非常见字符(@、#和\\等)。特殊情况,还会掺杂URL链接方式阐述论点。
%另一方面,学者认为垃圾邮件发送者通常采用突发性和短期连贯性的行为引导舆论。随后,他们则进行快速伪装,借助传播正能量等行为聚集粉丝,为下一次引导做准备。因此,序列建模策略近期也被应用在当前任务中。与GNN骨干不同,该策略重点关注用户历史行为长期或短期之间的隐藏相关性。如果将单个历史行为视为一个Token,那么整个历史行为序列建模问题就可以转化为自然语言处理(NLP)问题。在NLP领域中,Transformer专为序列建模和转译任务而设计,其显著特点是利用注意力对数据中的长程或短程依赖关系进行建模。这与序列建模需求高度一致。
\par On the other hand, scholars believe that spammers usually use sudden and short-term coherent behaviors to guide public opinion. Subsequently, they make quick disguises and gather followers with the help of behaviors such as spreading positive energy to prepare for the subsequent guidance. Meanwhile, they also use outdated information as a vehicle and add new claims to guide new public opinion. Moreover, the interval between outdated information and current activities is very long (see Fig. \ref{fig-ultra-long-term}). Therefore, sequence modeling strategies have also recently been applied to the current task. Unlike the GNN backbone, this strategy focuses on the hidden correlations between users' historical long-term and short-term behaviors. If a single historical behavior is considered a Token, the entire problem of modeling historical behavioral sequences can be transformed into the problem of natural language processing (NLP). In the field of NLP, Transformer is designed for sequence modeling and translation tasks, and its distinguishing feature is the use of attention to model long-term or short-term dependencies in the data. This is highly consistent with the sequence modeling needs of the task at hand.
%然而,NLP任务与垃圾邮件发送者识别任务存在差异。因此,将NLP领域中成熟的Transformer骨干应用在垃圾邮件发送者识别任务需要解决两个问题。1)多模态特征挖掘问题。众所周知,社交网络是典型的复杂网络。因此,用户行为的表现载体也是多种多样,即包括文本和多模态辅助特征。在NLP任务中,研究人员仅需要考虑如何挖掘语义和语境信息即可。但是,垃圾邮件发送者识别任务中还需要考虑多模态特征挖掘与模态对齐问题。2)超长历史行为序列表示问题。在NLP领域中,基于Transformer架构的模型支持的Token长度通常为512-2048之间(BERT(512 tokens)、GPT-2(1024 tokens)、GPT-3(2048 tokens)、Longformer(4096 tokens)、GPT-4(8192 tokens)等)。然而,用户历史行为序列长度远超这些长度。经过实验验证(看表1),序列长度为36768时效果最佳。此时,需要考虑的问题是如何解决多头注意力计算过程中%QK^T%矩阵造成的内存溢出问题。同时,由于每日社交用户过多,因此需要考虑使用小型化设计,因此不能无限制增加显卡内存需求。
\par However, there are differences between NLP and spammer detection tasks. Therefore, there are two issues that need to be addressed to apply the well-established Transformer backbone in the NLP domain for spammer detection tasks.

\begin{itemize}
	\setlength{\itemsep}{-1pt}
	\setlength{\parsep}{0pt}
	\setlength{\parskip}{0pt}
	\item[1)]
	The problem of multi-modal feature mining. Social networks are typically complex networks. Therefore, the representation carriers of user behaviors are also diverse, i.e., including text and multi-modal auxiliary features. In NLP tasks, researchers only need to consider how to mine semantic and contextual information. However, the problem of multi-modal feature mining and cross-modal alignment must also be considered in the spammer detection task (see Fig. \ref{fig-inspire}).
	\item[2)]
	The problem of representing ultra-long historical behavioral sequences. In the field of NLP, the classical MHA-based Transformer architecture supports token lengths typically between 512 and 8192 (BERT (512 tokens\cite{devlin2018bert}), GPT-2 (1024 tokens\cite{radford2019language}), GPT-3 (2048 tokens\cite{brown2020language}), GPT-4 (8192 tokens\cite{achiam2023gpt}), etc.). However, the length of historical user behavior sequences in the spammer detection task far exceeds this threshold. It has been experimentally verified (see Table 1) that the sequence length of 16, 384, or 36,768 is the most effective. At this point, the problem to be considered is solving the memory explosion caused by the $QK^T$ matrix during the multi-head attention computation. Meanwhile, miniaturized design must be considered because of the excessive number of daily social users. With the help of SMHA (see Fig. \ref{fig-MHAs}), the Transformer structure can handle ultra-long sequences. However, SMHA fails to meet the requirements at two levels of running efficiency and memory consumption (see Table \ref{table-ab-memory}).
\end{itemize}

%为了解决上述问题,我们构建了一种名为MS$^2$Dformer的新型Transformer,它可以作为多模态序列垃圾邮件发送者检测的通用骨干。同时,结合变分自编码器和层次系数多头注意力机制解决复杂多模态特征挖掘与超长序列建模。随后,建立Transformer骨干网络检测垃圾邮件发送者。本文贡献如下所示:
\par To solve the above problems, we construct a novel Transformer, called MS$^2$Dformer, that can be used as a generalized backbone for \textbf{m}ulti-modal \textbf{s}equence \textbf{s}pammer \textbf{d}etection. Meanwhile, the variational autoencoder and hierarchical sparse attention mechanism are combined to solve complex multi-modal feature mining and ultra-long sequence modeling. Subsequently, the Transformer backbone is built to detect spammers. The contributions of this article are shown as follows:
%1)提出一种基于多模态变分自编码器(MVAE)的用户行为Token化算法。首先,利用预训练模型(BERT和ViT)对多模态用户行为进行嵌入。其次,构建双通道的多模态特征编码和跨模态特征对齐。随后,建立共享跨模态特征的双通道解码组件。最后,跨模态编码特征同时参与特征重建和序列垃圾邮件发送者识别。
%2)提出分层分割窗口的多头注意力(SW-MHA)计算机制。首先,构建分割窗口策略将超长序列分层转化为窗口内短期注意力结合计算。此举既可以保证模型能处理超长Token序列,同时也进一步挖掘短期垃圾邮件发送者突发行为。并且,窗口采用重合滑动方式防止遗漏重要短期信息。随后,建立窗口间总体注意力从而深度聚合长期潜伏行为。SW-MHA显著降低了MHA计算过程的显存爆炸隐患,同时加速推理。
%3)在两个公开数据集,即Weibo和Twitter,中进行预训练。MS$^2$Dformer的性能大大超过了之前的技术水平,在Weibo中达到了+9% Acc.和+7% AUC,在Twitter中达到了+7% AUC。实验证明了MS$^2$Dformer作为多模态垃圾邮件发送者检测骨干的能力。

\begin{itemize}
	\setlength{\itemsep}{-1pt}
	\setlength{\parsep}{0pt}
	\setlength{\parskip}{0pt}
	\item[1)]
	A Tokenization algorithm is proposed for user behavior based on a multi-modal variational autoencoder (MVAE). Firstly, pre-trained models (BERT\cite{wang2024utilizing} and ViT\cite{liu2021swin, han2022survey}) are utilized to embed multi-modal user behaviors. Secondly, dual-channel multi-modal feature encoding and cross-modal feature alignment are constructed. Subsequently, a dual-channel decoding component that shares cross-modal features is built\cite{wang2024revisiting, fang2023unsupervised}. Finally, the cross-modal encoded features are involved in feature reconstruction and sequence spammer detection.
	\item[2)]
	A hierarchical split-window multi-head attention (SW/W-MHA, see Fig. \ref{fig-MHAs} (i)) computational mechanism is proposed. Firstly, the split-window strategy is constructed to transform the ultra-long sequences into short-term attention within the window by hierarchical transformation. This action ensures the model can handle ultra-long Token sequences while further mining short-term spammers' bursty behaviors. Moreover, the windows overlap and slide to prevent omitting important short-term information (SW-MHA). Subsequently, the overall attention between windows is built to aggregate the long-term latent behaviors (W-MHA) deeply. SW/W-MHA significantly reduces the explicit explosion potential of MHA computation and accelerates the inference.
	\item[3)]
	Pre-training is performed in two publicly available datasets. MS$^2$Dformer significantly outperforms the previous state of the art, achieving an accuracy improvement of +6.9/5.2\%. In the consumer-grade platform (RTX 4060 (16GB)), the model is built with more than 53M parameters and runs efficiently (see Table \ref{table-ab-memory}). The experiments demonstrate MS$^2$Dformer's ability to act as a multi-modal spammer detection backbone.
\end{itemize}

%本文其余章节的组织结构如下所示。第二节介绍了近期与垃圾邮件发送者检测、多模态变分自编码器和长序列Transformer架构建模任务相关的工作。第三节介绍构建MS$^2$Dformer模型所需的相关定义。第四节重点描述模型的细节。第五部分比较了MS$^2$Dformer模型与当前最先进算法在公开数据集预训练的训练对比。同时,构建消融研究验证模型组件的合理性。第六节总结全文并展望未来的研究工作。
\par The organization of the remaining sections of this article is shown as follows: Section II describes recent work related to the tasks of spammer detection, multi-modal variational autoencoder, and long sequence Transformer architecture. Section III describes the relevant definitions needed to construct the MS$^2$Dformer model. Section IV focuses on describing the details of the model. Section V compares the performance of the MS$^2$Dformer model with the current state-of-the-art algorithms pre-trained on a public dataset. Meanwhile, the ablation study is constructed to validate the rationality of the model components. Section VI summarizes the entire article and looks forward to future research work.

\section{Related Work}
%垃圾邮件发送者通过发送垃圾邮件(也有学者称为虚假新闻或谣言)引导舆论。因此,学术界定义垃圾邮件发送者检测包含两个子任务:短期突发用户检测和长期隐藏用户检测。前者,组合用户短期多个行为识别用户是否为垃圾邮件发送者。这项任务通常定义为垃圾邮件(虚假新闻或谣言)检测。此时,由于用户短期行为较少,因此通常结合传播空间以及GNN主干构建模型。后者,更适合识别长期潜伏的用户。同时,也能兼顾突发用户检测任务。并且,伴随着算力提升,多模态特征挖掘也被结合在这两项子任务中。
\par Spammers guide public opinion by sending spam (also called fake news or rumors by some scholars). Therefore, the academic definition of spammer detection contains two sub-tasks: short-term burst and long-term hidden user detection. In the former, several short-term behaviors of a user are combined to identify whether the user is a spammer or not. This task is usually called spam (fake news or rumor) detection. In this case, the model is usually constructed by combining the spread space and the GNN backbone because the user has fewer short-term behaviors. The latter is more suitable for identifying long-term potential users. Meanwhile, it can also consider the task of sudden user detection. Moreover, with the arithmetic power improvement, multi-modal feature mining is combined in these two sub-tasks.
%基于图主干的模型:社会传播过程通常以图为载体。因此,基于图的神经网络模型被广泛用于当前任务。例如,Bian 等人使用自上而下和自下而上的双向图卷积模型来解决虚假信息识别任务。随后,Wei 等人在 Bian 等人的基础上引入了随机化理论。结果,这一策略大大增强了基于图神经网络的虚假信息识别模型的普适性。同时,结合马尔可夫场的GNN主干也被应用在虚假信息制造者检测任务中。实时证明,基于GNN的模型在识别精度上已经达到了几乎不可被超越的表现。然而,随着用户历史行为增加,即识别长期隐藏的垃圾邮件发送者时,基于GNN的模型需要消耗大量GPU内存去推导可疑行为。由于每日社交行为数量数以亿计,所以构建模型是不得不考虑GPU内存问题。最好是可以在消费级GPU上能够运行。因此,在超长历史行为序列为背景的潜伏式垃圾邮件发送者检测任务中,GNN骨干网络不是最有效的。
\par \textbf{GNN Backbone-Based Models:} Social diffusion processes are usually carried out in graphs. Therefore, GNN-based models are popularly used for the task at hand. For instance, Bian et al.\cite{Bian2020Rumor} used top-down and bottom-up bi-directional GCN models to solve the task of fake news recognition. Subsequently, Wei et al.\cite{wei2024modeling} introduced randomization theory based on Bian et al. Consequently, this strategy dramatically enhances the generalizability of the fake news recognition model. Meanwhile, the GNN backbone incorporating Markov fields\cite{deng2023markov} was also applied to the task of fake maker detection. It has been proven that the GNN-based model has reached an almost unsurpassable performance in terms of recognition accuracy. However, as the historical user behavior increases, i.e., when identifying long-hidden spammers, the GNN-based model consumes a large amount of GPU memory to derive suspicious behaviors. With hundreds of millions of social behaviors every day, models must be built with GPU memory in mind and ideally run on consumer GPUs. Therefore, the GNN backbone is not the most effective in latent spammer detection against the background of ultra-long historical behavioral sequences.
%基于序列的模型:社交网络传播空间包含图结构和时序信息。因此,部分研究人员尝试采用序列建模策略构建检测模型。Yang 等人结合时间特征构建了基于情感熵的传播语音。随后,借助音频分类技术识别虚假信息。同样,Ma 等人提出了一种基于时间特征的传播响应信息建模策略,并借助 RNN(循环神经网络)识别了虚假信息。随后,Ma 等人在 TD-RvNNcite的基础上构建了一个 GAN(生成对抗网络)式模型。其中,Transformer 模型被用来对时间传播序列进行建模,并将其用作生成器。随后,引入 RNN 模型作为判别器来识别虚假信息。此时,作者主要解决短期行为衍生的传播空间建模任务。因此,他们使用了支持512tokens的NLP领域成熟Transformer架构。他们的模型在面临超长用户序列时将超过512tokens的行为删除。同时,时序传播图也被应用。例如,Sun 等人基于图卷积神经网络模型挖掘了传播子空间的结构信息。随后,基于时态特征构建了平展子空间结构特征的双向融合策略。他们同样没有解决消费级GPU平台训练和部署时序GNN主干网络。
\par \textbf{Sequence-Based Models:} Social network spread space contains graph structure and temporal sequence information. Therefore, some researchers try constructing a detection model using a sequence modeling strategy. By combining temporal features, Yang et al.\cite{Yang2024Topic} constructed a spreading audio based on emotional entropy. Subsequently, fake news is recognized with the help of audio classification techniques. Similarly, Ma et al.\cite{ma2016detecting} proposed a temporal feature-based modeling strategy for propagated response information and identified the fake news with the help of RNN (Recurrent Neural Network). Subsequently, Ma et al.\cite{ma2021improving} constructed a GAN (Generative Adversarial Network) style model based on TD-RvNN\cite{ma2016detecting}. In this case, the Transformer model is used to model time-propagated sequences, and it is used as a generator. Subsequently, the RNN model is introduced as a discriminator to identify fake news. At this point, the authors mainly address modeling the short-term behavior-derived spread space. Therefore, they used the well-established Transformer architecture of the NLP domain that supports 512 tokens. Their model removes behaviors exceeding 512 tokens when confronted with ultra-long user sequences. Meanwhile, temporal spreading graphs were also applied. For instance, Sun et al.\cite{Sun2022ddgcn} mined the structural information of the propagation subspace based on GCN models. Subsequently, a bidirectional fusion strategy for structural features of the spread subspace was constructed based on temporal features. They did not address the training and deployment of temporal GNN backbone networks on consumer GPU platforms.
%结合多模态的模型:随着计算机和通信技术快速发展,在线社交行为已经不局限于文本模型。因此,多模态数据挖掘与跨模态对齐成为当前研究的热点。在短期突发检测任务中,研究人员已经聚焦于多模态领域展开研究。例如,Wang等人引入对抗学习对齐跨模态特征。Zhang等人引入强化学习策略学习跨模态特征。然而,在长期隐藏用户检测子任务中,多模态特征并灭有被广泛使用。例如,Qu等人考虑使用多模态建模策略。但是,他们仅仅是将文本转化为三个通道的类似视觉信息,而并非真正的多模态数据。为此,我们从学列建模角度构建了多模态的垃圾邮件发送者建模通用骨干,称为MS$^2$Dformer。首先,使用多模态变分自编码构建双通道特征挖掘与对齐。随后,基于层次分割窗口注意力机制深度量化序列特征。结合掩码机制,MS$^2$Dformer在两个子任务中均表现极佳,证明了它作为垃圾邮件发送者检测任务通用骨干的潜力。
\par \textbf{Multi-Modal Models:} With the rapid development of computer and communication technologies, online social behaviors are no longer limited to text models. Therefore, multi-modal data mining and cross-modal alignment have become a hot topic in current research. In short-term burst detection, researchers have focused on the multi-modal domain. For instance, Wang et al.\cite{wang2023cross} introduced adversarial learning to align cross-modal features. Zhang et al.\cite{zhang2024reinforced} introduced a reinforcement learning strategy to learn cross-modal features. However, multi-modal features have not been widely used in the long-term hidden user detection sub-task. For instance, Qu et al.\cite{qu2024temporal} considered using a multi-modal modeling strategy. However, they only transformed the text into three channels of similar visual information, not accurate multi-modal data. To this end, we construct a generalized backbone for multi-modal spammer detection, called MS$^2$Dformer, from a sequence modeling perspective. Firstly, two-channel feature mining and alignment are constructed using multi-modal variational auto-encoding. Subsequently, sequence features are deeply quantified based on a hierarchical split-window attention mechanism.
\section{Problem Definition}
\subsection{Related Definitions}
% 垃圾邮件发送者检测是一项典型的多模态复杂任务。为此,本节从单个历史行为的文本和图像多模态特征提取出发,定义MS$^2$Dformer模型的输入序列数据。
\par Spammer detection is a typically multi-modal and complex task. Therefore, the input sequence data for the MS$^2$Dformer model is defined in this section starting from multi-modal feature extraction of text and images.
%%%%%%%%%%%%%%%%%%%%%%%%%%%%%%%%%%%%%%%%%%%%%%%%%%%%%%%%%%%%%%%%%%%%%%%%%%%%
\par \textbf{Definition 1.} Text modal feature \begin{math} T^v \end{math}
% 如图1所示,用户历史行为中文本模型的数据往往提供了非常重要的数据。因此,本节采用预训练的仅编码器结构的BERT模型对文本进行编码。公式如下所示:
%其中,$T$表示文本数据。$T^v\in \mathbb{R}^{768}$表示经过预训练BERT模型编码的“CLS”token的嵌入表示。特别的,NLP领域中文本数据在句首连接"CLS"token。因此,$CLS^{T}$表示这个token。
\par As shown in Fig. \ref{fig-inspire}, the data from the text model in the user's historical behavior often provides very important data. Therefore, in this section, a pre-trained BERT\footnote{\url{https://huggingface.co/google-bert/bert-base-uncased.}} model with only an encoder structure is used to embed the text. The equation is shown as follows:
\begin{equation}
	T^v = \text{BERT}(\text{CLS}^{\text{T}}+T) \in \mathbb{R}^{768}
	\label{eq-1}
\end{equation}
where $T$ represents the text data. In the NLP domain, the textual data is concatenated with the “CLS” token at the beginning of the sentence. Thus, $\text{CLS}^{\text{T}}$ represents this token. $T^v\in \mathbb{R}^{768}$ represents the embedded representation of the “CLS” token encoded by the pre-trained BERT model.
%%%%%%%%%%%%%%%%%%%%%%%%%%%%%%%%%%%%%%%%%%%%%%%%%%%%%%%%%%%%%%%%%%%%%%%%%%%%
\par \textbf{Definition 2.} Image modal feature \begin{math} I^v \end{math}
%仅仅通过文本模态引导舆论会引起目标群体不信任从而导致挑拨失败。所以,垃圾邮件发送者可能会提供剪辑后的图像等多模态信息作证他的观点。因此,在考虑文本特征的同时,图像特征同样不可缺少。为此,我们采用预训练的ViT模型提取图像特征。公式如下所示:
%其中,$I$表示输入的图像数据。$I^v\in \mathbb{R}^{768}$表示经过预训练ViT模型编码的“CLS”token的嵌入表示。由于,ViT在模型内部添加“CLS”token,因此不在公式中体现。
\par Guiding public opinion through textual modalities alone can cause the target group to trigger a crisis of confidence. Ultimately, it leads to a failure of provocation. Therefore, a spammer may provide multi-modal information, such as edited images, to testify his viewpoint. Thus, while considering text features, image features are equally indispensable. For this purpose, a pre-trained ViT\footnote{\url{https://huggingface.co/google/vit-base-patch16-224.}} model is used to extract image features. The equation is shown as follows:
\begin{equation}
	I^v = \text{ViT}(I) \in \mathbb{R}^{768}
	\label{eq-2}
\end{equation}
where $I$ represents the input image data. $I^v\in \mathbb{R}^{768}$ represents the embedded representation of the “CLS” token encoded by the pre-trained ViT model. Because ViT adds “CLS” tokens inside the model, they are not represented in Eq. (\ref{eq-2}).
%%%%%%%%%%%%%%%%%%%%%%%%%%%%%%%%%%%%%%%%%%%%%%%%%%%%%%%%%%%%%%%%%%%%%%%%%%%%
%%%%%%%%%%%%%%%%%%%%%%%%%%%%%%%%%%%%%%%%%%%%%%%%%%%%%%%%%%%%%%%%%%%%%%%%%%%%
\par \textbf{Definition 3.} User Historical Behavior Sequence \begin{math}S=\{(T^v_0, I^v_0),..., (T^v_l, I^v_l)\}\end{math}
%本文提出一种序列化的垃圾邮件发送者识别模型。因此,首先需要将用户$U_i$的历史行为序列$U_i=\{b_0, ..., b_l\}$转化为模型能够处理的标准表示。如图1所示,序列中任意行为包含文本和图像两种模态数据,即$b_i=(T_i, I_i)$,并且$i \in [0. l]$。特别的,存在历史行为$b_i$不包含图像数据时,填充一个空图像。随后,结合公式1和公式2将历史行为序列$\{b_0, ..., b_l\}$转化为标准形式。公式如下所示:
%其中,$S_i \in \mathbb{R}^{l\times2\times768}$表示用户$U_i$的标准输入序列。$l$表示模型支持的历史行为数量。当用户真实行为数量超过$l$时将会被截断。另一方面,不足的序列补充空行为。在模型中通过MASK机制消除这些空行为的影响。同时,为了加速提取多模态特征,采用$\text{BERT}(\{T_0, ..., T_l\})$和$\text{ViT}(\{I_0, ..., I_l\})$策略并行化提取特征。
\par This paper proposes a serialized model for spammer detection. Therefore, it is first necessary to transform a sequence of historical user behaviors $U=\{b_0, ..., b_l\}$ into a standard representation that the model can process. As shown in Fig. \ref{fig-inspire}, any behavior in the sequence contains both text and image modal data, i.e., $b_i=(T_i, I_i)$, and $i \in [0, l]$. In particular, a history behavior $b_i$ exists that fills an empty image when it does not contain image data. Subsequently, the sequence of history behaviors $\{b_0, ..., b_l\}$ into standard form. The equation is shown as follows:	
\begin{equation}
	\begin{split}
	S&=\text{BERT}(\{T_0, ..., T_l\})\ \text{and}\ \text{ViT}(\{I_0, ..., I_l\})\\
	   &=\{(T^v_0, I^v_0),..., (T^v_l, I^v_l)\} \in \mathbb{R}^{l\times2\times768}
	\label{eq-3}
	\end{split}
\end{equation}
where $S \in \mathbb{R}^{l\times2\times768}$ represents the standard sequence of inputs for user $U$. $l$ represents the length of the sequence supported by the model. The user's real behavior sequence will be truncated when its length exceeds $l$. On the other hand, insufficient sequences supplement the empty behaviors. The effect of these empty behaviors is eliminated in the model by the MASK mechanism. Meanwhile, to accelerate the extraction of multi-modal features, $\text{BERT}(\{T_0, ... , T_l\})$ and $\text{ViT}(\{I_0, ... , I_l\})$ strategies are employed to parallelize the extraction of features.
%%%%%%%%%%%%%%%%%%%%%%%%%%%%%%%%%%%%%%%%%%%%%%%%%%%%%%%%%%%%%%%%%%%%%%%%%%%%
\subsection{Problem Formulation}
%%%%%%%%%%%%%%%%%%%%%%%%%%%%%%%%%%%%%%%%%%%%%%%%%%%%%%%%%%%%%%%%%%%%%%%%%%%%
% 我们提出了一个新的Transformer,名为MS$^2$Dformer(看图2),以数学方式解决序列化的多模态垃圾邮件发送者检测问题。该模型将这个复杂问题分为四个阶段分别建模。因此,模型的总体表示如下:
\par To solve the problem of serialized multi-modal spammer detection mathematically, we propose a new Transformer, called MS$^2$Dformer. The model models this complex problem in four separate stages. Thus, the overall representation of the model is as follows:
%%%%%%%%%%%%%%%%%%%%%%%%%%%%%%%%%%%%%%%%%%%%%%%%%%%%%%%%%%%%%%%%%%%%%%%%%%%%
\begin{eqnarray}\left. {\begin{array}{*{20}{l}}
			U=\{b_0, ... , b_l\}\\
			b_i=(T_i, I_i)\\
	\end{array}} \right\} \Rightarrow \text{MS}^2\text{Dformer} \Rightarrow {P\{s,n \mid user\}}
\end{eqnarray}
 %%%%%%%%%%%%%%%%%%%%%%%%%%%%%%%%%%%%%%%%%%%%%%%%%%%%%%%%%%%%%%%%%%%%%%%%%%%%
\begin{figure*}[h]
	\center{\includegraphics[width=1\linewidth]  {./image/model1.pdf}} 
	\caption{The framework of the MS$^2$Dformer\_B model (MS$^2$Dformer Base Version, see Eq. (\ref{eq-base})).}
	\label{fig-2}
\end{figure*}
%%%%%%%%%%%%%%%%%%%%%%%%%%%%%%%%%%%%%%%%%%%%%%%%%%%%%%%%%%%%%%%%%%%%%%%%%%%%
\subsubsection{Model Input}\
% MS$^2$Dformer模型的输入数据如下所示:
\par The input data for the MS$^2$Dformer model is shown as follows:
\begin{itemize}
	\setlength{\itemsep}{-1pt}
	\setlength{\parsep}{0pt}
	\setlength{\parskip}{0pt}
	\item
	The sequence of historical user behaviors $U=\{b_0, ... , b_l\}$. $l$ represents the length of the sequence supported by the model.
	\item
	Individual user behavior $b_i=(T_i, I_i)$. The $i$-th behavior of user $U$ contains text $T_i$ and image $I_i$, and $i \in [0, l]$.
\end{itemize}

\subsubsection{Model Output}\ 
%基于上一节的模型输入数据,MS$^2$Dformer模型需要分阶段解决以下问题。
\par Based on the model input data from the previous section, the MS$^2$Dformer model (see Fig. \ref{fig-2}) needs to solve the following problems in stages.
\begin{itemize}
	\setlength{\itemsep}{-1pt}
	\setlength{\parsep}{0pt}
	\setlength{\parskip}{0pt}
	\item
	%多模态隐藏特征$H_k \in \mathbb{R}^{h\times2\times768}$。首先,模型输入序列$U_k=\{b_0, ... , b_l\}$经过公式(1)后的到嵌入表示$S_k=\{(T^v_0, I^v_0),..., (T^v_l, I^v_l)\} \in \mathbb{R}^{l\times2\times768}$。其次,在变分自编码器(VAE)的基础上构建双通道的多模态变分自编码器(MVAE)。随后,经过MVAE,$S_k \in \mathbb{R}^{l\times2\times768}$被转化为$S_k \in \mathbb{R}^{l\times2D}$。$D$表示MVAE嵌入部分的维度。最后,添加分类token“CLS$_{\text{S}}}$”。因此,$S_k \in \mathbb{R}^{l\times2\times768}$被转化为$S_k \in \mathbb{R}^{H\times2D}$,$H=l+1$。
	\textbf{Stage 1:} Multi-modal hidden features $S_1 \in \mathbb{R}^{H\times2D}$. Firstly, the model input sequence $U=\{b_0, ... , b_l\}$ after Eq. (\ref{eq-3}) to the embedding representation $S=\{(T^v_0, I^v_0),... , (T^v_l, I^v_l)\} \in \mathbb{R}^{l\times2\times768}$. Secondly, a two-channel multi-modal variational autoencoder (MVAE) is constructed from the variational autoencoder (VAE). After MVAE, $S \in \mathbb{R}^{l\times2\times768}$ is transformed into $\mathbb{R}^{l\times2D}$. $D$ represents the dimension of the MVAE embedding part. Finally, the classification token CLS$^{\text{S}}$ is added. Thus, $S \in \mathbb{R}^{l\times2\times768}$ is transformed into $S_1 \in \mathbb{R}^{H\times2D}$, and $H=l+1$.
	\item		
	% 解决超长序列建模问题。首先,建立层次分割窗口注意力机制(SW-MHA)。随后,基于第一层窗口注意力机制建立SW-MHA Transformer block。此时,$S_k \in \mathbb{R}^{H\times2D}$被SW-MHA转化为$S_k \in \mathbb{R}^{$\frac{H}{W}$\times2WD}$。随后,修改原始线性感知器(MLP)为SW-MLP。并且,$S_k \in \mathbb{R}^{$\frac{H}{W}$\times2WD}$被SW-MHA转化为$S_k \in \mathbb{R}^{$\frac{H}{W}$\times4D}$。其次,基于第二层窗间注意力机制建立多个W-MHA Transformer block。此时,W-MHA Transformer block用于深度挖掘窗间序列特征。因此,不对输入数据维度进行改变。
	\textbf{Stage 2:} Solve the problem of modeling ultra-long sequences. Firstly, the hierarchical split-window attention mechanism (SW/W-MHA) is constructed. Subsequently, the SW-MHA Transformer block is constructed based on the intra-window attention mechanism. In this case, $S_1 \in \mathbb{R}^{H\times2D}$ is transformed by SW-MHA into $\mathbb{R}^{\frac{H}{\lambda_{1}}\times2W_{1}D}$. $\lambda_{1}$ represents the stride size of the window. $W_{1}$ represents the length of the window. Afterwards, the original multi-layer linear perceptron (MLP) is modified to SW-MLP. In addition, $S_1 \in \mathbb{R}^{\frac{H}{\lambda_{1}}\times2W_{1}D}$ is transformed by SW-MLP to $\in \mathbb{R}^{\frac{H}{\lambda_{1}}\times4D}$. Secondly, multiple W-MHA Transformer blocks are constructed based on the inter-window attention mechanism. In this case, the W-MHA Transformer block is used to deeply mine the inter-window sequence features. Therefore, no changes are made to the input data dimensions, i.e., $S_2 \in \mathbb{R}^{\frac{H}{\lambda_{1}}\times4D}$.
	\item	
	% 深度序列特征挖掘问题。相较于Stage 2,Stage 3存在两点不同。首先,Stage 3中SW-MHA组件设置的窗口滑动步长W_{2}应该远小于W_{1}。其次,W-MHA组件采用更深层的Block结构挖掘序列特征。总的来看,经过Stage 3,$S_k \in \mathbb{R}^{$\frac{H}{W_{1}}$\times4D}$被转化为$S_k \in \mathbb{R}^{$\frac{H}{W_{1}W_{2}}$\times8D}$。
	\textbf{Stage 3:} Solve the problem of deep sequence feature mining. Compared with Stage 2, there are two differences in Stage 3. Firstly, the window sliding step $\lambda_{2}$ set by the SW-MHA component in Stage 3 should be much smaller than $\lambda_{1}$. Secondly, the W-MHA component uses a deeper Block structure to mine sequence features. Overall, $S_2 \in \mathbb{R}^{\frac{H}{\lambda_{1}}\times4D}$ is transformed into $\mathbb{R}^{\frac{H}{\lambda_{1}\lambda_{2}}\times8D}$, i.e., $S_3 \in \mathbb{R}^{\frac{H}{\lambda_{1}\lambda_{2}}\times8D}$.
	\item		
	% 解决垃圾邮件发送者检测问题。首先,选择$S_k \in \mathbb{R}^{\frac{H}{W_{1}W_{2}}\times8D}$的分类token CLS$_{\text{S}}} \in \mathbb{R}^{8D}$。最后,经过两层线性层将分类token映射为CLS$_{\text{S}}} \in \mathbb{R}^{2}$。因此,经过softmax函数识别垃圾邮件发送者。
	\textbf{Stage 4:} Solve the problem of spammer detection. Firstly, the classification token CLS$^{\text{S}} \in \mathbb{R}^{8D}$ in $S_3 \in \mathbb{R}^{\frac{H}{\lambda_{1}\lambda_{2}}\times8D}$ is selected. Lastly, the classification token is mapped to CLS$^{\text{S}} \in \mathbb{R}^{2}$ after two linear layers. Thus, the spammers are recognized by softmax function.
\end{itemize}

\section{MS$^2$Dformer Model}

\subsection{Overview}
% 我们构建了一种名为MS$^2$Dformer的新型Transformer,它可以作为多模态序列垃圾邮件发送者检测的通用骨干。模型分为四个阶段识别垃圾邮件发送者(看图2)。首先,阶段1基于双通道MVAE完成用户历史行为token化。其次,结合阶段2和阶段3对超长序列建模,并且深度挖掘序列特征。最好,阶段4采用线形层挖掘分类token CLS$^{\text{S}}特征,随后识别垃圾邮件发送者。
\par A novel Transformer, called MS$^2$Dformer, is constructed that can be used as a generalized backbone for multi-modal sequence spammer detection. The model is divided into four stages to identify spammers (see Fig. \ref{fig-2}). Firstly, stage 1 completes tokenizing the user's historical behavior based on two-channel MVAE. Secondly, ultra-long sequences are modeled with stages 2 and 3, and sequence features are deeply mined. Finally, stage 4 employs a linear layer to mine classification token CLS$^{\text{S}}$ features and subsequently identifies spammers.
\begin{figure*}[h]
	\center{\includegraphics[width=1\linewidth]  {./image/SW_MHA.pdf}} 
	\caption{The case of SW/W-MHA.}
	\label{fig-SW-MHA}
\end{figure*}
\subsection{MVAE-based Historical Behavior Tokenization}
% 序列化垃圾邮件发送者识别模型的首要待解决问题是如何将用户多模态历史行为token化。为此,我们构建了基于双通道MVAE的用户历史行为token化策略。随后,联合CLS$^{\text{S}} token完成整个用户历史行为的序列化。因此,本节包含三个步骤:输入嵌入、MVAE和行为token化。
\par The primary problem to be solved for the serialized spammer identification model is how to tokenize the user's multi-modal historical behavior. To this end, we construct a two-channel MVAE-based tokenization strategy for user history behaviors. Subsequently, the serialization of the entire user history behavior is completed by the joint CLS$^{\text{S}}$ token. Therefore, this section contains three steps: \textit{Input Embedding}, \textit{MVAE}, and \textit{Behavior Tokenization}.
% 由大型预训练模型生成的输入嵌入能够编码丰富的上下文信息。为此,我们使用BERT(看公式1)获得单个行为文本特征嵌入$T^v\in \mathbb{R}^{768}$,使用ViT(看公式2)获得图像嵌入$I^v\in \mathbb{R}^{768}$。随后,基于原始序列构建MVAE组件的输入嵌入序列$S_k \in \mathbb{R}^{l\times2\times768}$(看公式3)。
\par \textbf{Input Embedding:} Input embeddings generated by large pre-trained models are capable of encoding rich contextual information. To this end, we use BERT (see Eq. \ref{eq-1}) to obtain a single behavioral text feature embedding $T^v\in \mathbb{R}^{768}$ , and use ViT (see Eq. \ref{eq-2}) to obtain the image embedding $I^v\in \mathbb{R}^{768}$. Subsequently, the input embedding sequence $S \in \mathbb{R}^{l\times2\times768}$ of the MVAE component is constructed based on the original sequence $U$ (see Eq. \ref{eq-3}).
% Since the embedding already contains abundant semantic information, there is little need to further extract contextual information with complex and deep layers in VAE’s encoders. Thus, we decide to use one linear layer followed by batch normalization (BN) and dropout (𝑝 = 0.2) as the multimodal VAE’s encoder structure. BN is used to reduce internal covariance shift [15] by projecting the input to mean of zero and the variance of 1. We apply BN and dropout as they help overcome overfitting. Formally, the encoder is represented as:
% 由于嵌入已经包含了丰富的语义信息,因此在 VAE 的编码器中没有必要再通过复杂和深层的层来进一步提取上下文信息。因此,我们决定使用一个线性层,然后进行批量归一化(BN)和剔除(𝑝 = 0.2),作为多模态 VAE 的编码器结构。BN 用于通过将输入投影为均值为零和方差为 1 来减少内部协方差偏移。编码器的形式表示为
\par \textbf{MVAE:} Because the embedding already contains rich semantic information, there is no need to go through complex and deep layers to further extract contextual information in the MVAE's encoder. Therefore, we use two linear layers followed by Batch Normalization (BN) and Dropout (rate = 0.2) as the encoder structure for MVAE. BN is used to reduce the internal covariance bias by projecting the inputs to have mean zero and variance one. Subsequently, the formal representation of the encoder is shown below:
\begin{equation}
	z = \mu + \sigma \odot \epsilon
	\label{eq-4}
\end{equation}
\begin{equation}
	\mathcal{L} = \mathbb{E}_{q_\phi(z|x)}\left[\log p_\theta(x|z)\right] - \mathrm{D}_{\text{KL}}(q_\phi(z|x) \parallel p(z))
	\label{eq-5}
\end{equation}
where $\epsilon$ represents the noise sampled from the standard normal distribution $\mathcal{N}(0, I)$. $\odot$ represents the elemental product. $z$ denotes the latent variable. It is satisfying a Gaussian distribution. $\mu$ and $\sigma$ are the mean and variance of $z$, respectively. These parameters describe the conditional distribution $p_\theta(x|z)$ of the latent variable $z$. $\phi$ represents the parameters of the encoder. The objective of the decoder is to approximate the posterior distribution $q_\phi(z|x)$ to the posterior distribution $p_\theta(x|z)$.
% 其中$\epsilon$表示从标准正态分布$\mathcal{N}(0, I)$中采样的噪声。$\odot$表示元素乘积。$z$表示潜在变量,并且满足高斯分布。$\mu$和$\sigma$分别为$z$的均值和方差。这些参数描述了潜在变量$z$的条件分布$p_\theta(x|z)$。$\phi$表示编码器的参数。解码器的目标是将后验分布$q_\phi(z|x)$接近先验分布$p_\theta(x|z)$。
% 在MVAE中,输入数据$x$包含文本嵌入$T^v$和图像嵌入$I^v$,即$x=(T^v, I^v)$。随后,经过双通道的编码组件得到潜在变量$z^\text{T}$和$z^\text{I}$。然后,组合文本和图像两个通道的多模态潜在特征的到综合特征$z=\text{concat}(z^\text{T}, z^\text{I})$。
\par In MVAE, the input data $x$ contains text embedding $T^v$ and image embedding $I^v$, i.e., $x=(T^v, I^v)$. Subsequently, the potential variables $z^\text{T}$ and $z^\text{I}$ are obtained after the encoding component of the two channels. Then, the composite feature $z=\text{concat}(z^\text{T}, z^\text{I})$ is obtained by combining the two-channel multi-modal potential features.
% 随后,MVAE需要构建两个通道的解码器进行多模态特征重建。两个编码器共享一个潜在变量$z$。因此,MVAE的重建损失函数如下所示:
\par Afterwards, MVAE constructs two channel decoders for multi-modal feature reconstruction. The two encoders share a latent variable $z$. Therefore, the reconstruction loss function for MVAE is shown as follows:
\begin{equation}
	\begin{split}
	\mathcal{L}^\text{T} = \mathbb{E}_{q_\phi(z|T^v)}\left[\log p_\theta(T^v|z)\right] -\mathrm{D}_{\text{KL}}(q_\phi(z|T^v) \parallel p(z))
	\end{split}
	\label{eq-7}
\end{equation}
\begin{equation}
	\mathcal{L}^\text{I} = \mathbb{E}_{q_\phi(z|I^v)}\left[\log p_\theta(I^v|z)\right] - \mathrm{D}_{\text{KL}}(q_\phi(z|I^v) \parallel p(z))
	\label{eq-8}
\end{equation}
where $\mathcal{L}^\text{T}$ and $\mathcal{L}^\text{I}$ represent the loss for the text and image channels. $\mathbb{E}_{q_\phi(z|T^v)}\left[\log p_\theta(T^v|z)\right]$ and $\mathbb{E}_{q_\phi(z|I^v)}\left[\log p_\theta(I^v|z)\right]$ represent the reconstruction loss for the two channel. $\mathrm{D}_{\text{KL}}(q_\phi(z|T^v) \parallel p(z))$ and $\mathrm{D}_{\text{KL}}(q_\phi(z|I^v) \parallel p(z))$ represent the KL divergence Loss. $p(z)$ represents the standard normal distribution, i.e., $p(z)=\mathcal{N}(0, I)$.
% 其中$\mathcal{L}^\text{T}$和$\mathcal{L}^\text{I}$表示文本和图像通道的损失函数。$\mathbb{E}_{q_\phi(z|T^v)}\left[\log p_\theta(T^v|z)\right]$和$\mathbb{E}_{q_\phi(z|I^v)}\left[\log p_\theta(I^v|z)\right]$表示两个通道的重构损失。$\mathrm{D}_{\text{KL}}(q_\phi(z|T^v) \parallel p(z))$和$\mathrm{D}_{\text{KL}}(q_\phi(z|I^v) \parallel p(z))$表示两个通道的KL散度正则损失。$p(z)$表示标准正态分布,即$p(z)=\mathcal{N}(0, I)$。
\par \textbf{Behavior Tokenization:}
% 在MVAE中,潜在变量$z$包含跨多模态特征。同时,通过引入噪声$\epsilon$模拟图1出现的复杂情况。通过循环重建双通道特征,干扰特征会被排除。因此,将单个行为对应潜在变量$z$作为阶段2输入序列的token。
In MVAE, the latent variable $z$ is included across multi-modal features. Meanwhile, the complexities appearing in Fig. \ref{fig-inspire} are simulated by introducing the noise $\epsilon$. Subsequently, the interference features are eliminated by cyclically reconstructing the two-channel features. Therefore, the individual behavior corresponding to the latent variable $z$ is used as a token for the input sequence of stage 2.
% 随后,考虑到计算效率问题,采用并行化策略同步提取整个输入序列$S_k \in \mathbb{R}^{l\times2\times768}$的潜在变量$z \in \mathbb{R}^{l\times2D}$。其中,$D$表示MVAE编码器组件的最终表示维度。随后,考虑到传统Transformer架构中分类token的重要性。因此,我们构建了一个全零矩阵CLS$^{\text{S}} \in \mathbb{R}^{1\times2D}$充当分类token。最后,我们组合CLS$^{\text{S}}$和$z$作为阶段2的输入序列$S_k \in \mathbb{R}^{H\times2D}$。此时,为了方便计算,我们定义$H=l+1$。
\par Subsequently, a parallelization strategy is used to simultaneously extract the latent variables $z \in \mathbb{R}^{l\times2D}$ from the entire input sequence $S \in \mathbb{R}^{l\times2\times768}$. Where $D$ is a hyper-parameter indicating the final representation dimension of the MVAE encoder component. Later, the importance of classification tokens in the traditional Transformer architecture is taken into account. Therefore, we construct an all-zero matrix CLS$^{\text{S}} \in \mathbb{R}^{1\times2D}$ to act as the classification token. Lastly, we combine CLS$^{\text{S}}$ and $z$ to serve as the input sequence for stage 2, $S_1 \in \mathbb{R}^{H\times2D }$. In this case, to facilitate the computation, we define $H=l+1$.
\subsection{Ultra-long Behavior Sequence Mining}
% 序列建模任务中,多头注意力机制(MHA)展现了极其强大的能力。因此,我们使用基于MHA的Transformer架构作为序列特征挖掘的基本骨干。其中,经典MHA的计算过程如下所示:
% 其中,$X$表示输出序列。$\widehat{X}$表示经过MHA计算后的注意力。$D$表示MVAE编码器组件的最终表示维度。$H$表示序列长度。$Q$、$K$和$V$分别对应序列$S_k$的查询、键和值。$\text{Attention}(Q, K, V)$表示单个头的注意力计算。$K^\text{T}$表示$K$矩阵的转置。$d$表示键的维度。$head_i$表示第i个头的注意力。$n$表示MHA的头数量。$W^\text{Q}_{i}$、$W^\text{K}_{i}$、$W^\text{V}_{i}$和$W^\text{M}$表示可训练的参数矩阵。
\par \textbf{MHA:} The multi-head attention mechanism (MHA) has shown to be extremely powerful in sequence modeling tasks. Therefore, the MHA-based Transformer architecture is used as the basic backbone for sequence feature mining. Among them, the computational process of classical MHA is shown as follows:
\begin{equation}
	Q=XW^\text{Q},\ K=XW^\text{K},\ V=XW^\text{V}
	\label{eq-9}
\end{equation}
\begin{equation}
	\text{Attention}(Q, K, V)=\text{softmax}(\frac{QK^\text{T}}{\sqrt{d}})V
	\label{eq-10}
\end{equation}
\begin{equation}
	head_i=\text{Attention}(QW^\text{Q}_{i}, KW^\text{K}_{i}, VW^\text{V}_{i})
	\label{eq-11}
\end{equation}
\begin{equation}
	\begin{split}
		\widehat{X}&=\text{MHA}(Q, K, V)\\
		&=\text{Concat}(head_1,...,head_n)W^\text{M}
		\label{eq-12}
	\end{split}
\end{equation}
where $X$ represents the input sequence. $\widehat{X}$ represents the sequence after MHA computation. $D$ represents the final representation dimension of the MVAE encoder component. $H$ represents the sequence length. $Q$, $K$, and $V$ represent the query, key, and value of the sequence $X$, respectively. $\text{Attention}(Q, K, V)$ represents the attention computation for a individual head. $K^\text{T}$ represents the transpose of the $K$ matrix. $d$ represents the dimension of the key. $head_i$ represents the attention of the $i$-th head. $n$ represents the number of heads in the MHA. $W^\text{Q}_{i}$, $W^\text{K}_{i}$, $W^\text{V}_{i}$, and $W^\text{M}$ represent the trainable parameter matrices.
% 在垃圾邮件发送者检测任务中,用户历史行为序列是一个超长序列。因此,传统MHA计算中的$QK^\text{T}$(看公式8)会构建一个超大的矩阵从而造成显存崩溃。为此,我们提出了层次分割窗口注意力。核心思想是在$QK^\text{T}$中构建滑动窗口$\lambda$。随后,窗口$\lambda$以步长$W$滑动取样。经过SW-MHA计算,超长序列$H$会被转化为$\frac{H}{W}$。此时,超长序列就会被转化为标准序列建模问题。此时,分两个阶段对$QK^\text{T}$进行分块计算窗内和窗外注意力,从而近似计算整个序列的注意力。为此,我们提出了两层窗口注意力计算:SW-MHA和W-MHA。
\par In the spammer detection task, the sequence of historical user behaviors is an ultra-long sequence. Therefore, $QK^\text{T}$ in traditional MHA computation (see Eq. 8) will construct an oversized matrix thus causing GPU memory crash (see Table \ref{table-ab-memory}). For this reason, we propose hierarchical split window attention. The core idea is to construct sliding windows $W$ in $QK^\text{T}$. Subsequently, the window $W$ is sliding sampled in steps $\lambda$. Subsequently, the length $H$ of the ultra-long sequence is transformed into $\frac{H}{\lambda}$. Thus, the problem of modeling ultra-long sequences is transformed into the problem of modeling standard sequences. In this case, the intra-window and inter-window attention is computed in chunks for $QK^\text{T}$ at two levels, thus approximating the attention of the entire sequence. To this end, a two-level windowed attention computation is proposed: \textit{SW-MHA} and \textit{W-MHA}.
% 为了避免出现$QK^\text{T}$超大矩阵,因此需要在$Q$和$K$向量中进行滑动分窗。随后,并行计算窗口序列的窗内注意力。公式如下所示:
% 其中,$\text{SSW}$表示窗口滑动分割函数。假设,$Q$、$K$和$V$均为序列长度$H$和嵌入维度$\eta$的矩阵。那么,$Q^{\text{SW}}$、$K^{\text{SW}}$和$V^{\text{SW}}$被转化为序列长度$\frac{H}{W}$和嵌入维度$W\times\eta$的矩阵。
\par \textbf{SW-MHA:} To avoid $QK^\text{T}$ oversized matrices, sliding window splitting is performed in the $Q$ and $K$ vectors. Subsequently, the intra-window attention of the window sequence is computed in parallel. The equation is shown as follows:
\begin{equation}
	\begin{split}
	\widehat{Q}=\text{SW}(XW^\text{Q}), \widehat{K}=\text{SW}(XW^\text{K}), \widehat{V}=\text{SW}(XW^\text{V})
	\end{split}
	\label{eq-13}
\end{equation}
\begin{equation}
	\text{SW-Att}_i=\text{softmax}(\frac{\widehat{Q}_{i}\widehat{K}_{i}^\text{T}}{\sqrt{d}})\widehat{V}_{i}, i\in [1, k]
	\label{eq-13-1}
\end{equation}
\begin{equation}
	\text{SW-Attention}(Q, K, V)=\text{concat}(\text{SW-Att}_1, ..., \text{SW-Att}_k)
	\label{eq-13-2}
\end{equation}
where $\text{SW}$ represents the sliding window splitting function, and the implementation process is shown in Fig. \ref{fig-SW-MHA} (a). $k$ represents the length of the sequence after the split window. It is assumed that $Q$, $K$ and $V$ are matrices of sequence length $H$ and embedding dimension $\eta$. Then, $Q$, $K$ and $V$ are all transformed into $\mathbb{R}^{\frac{H}{\lambda}\times W\times\eta}$ by the SW algorithm. $W$ and $\lambda$ represent the window length and sliding step, respectively.
% 其中,$\text{SW}$ 表示窗口滑动分割函数,实现过程如图1(a所示)。假设,$Q$、$K$和$V$均为序列长度$H$和嵌入维度$\eta$的矩阵。那么,$Q$、$K$和$V$均被SW算法转化为$mathbb{R}^{\frac{H}{\lambda}\times\lambda\times\eta}$。
% 随后,基于公式(12)计算多头注意力。最后,为了满足Transformer架构计算,我们将$\widehat{X}$在最后两个维度展开。公式如下所示:
% 其中$X^{\text{SW}} \in \mathbb{R}^{\frac{H}{W}\times W\eta}}$表示经过窗内注意力计算后的特征矩阵。
\par Subsequently, the multi-head attention is computed in combination with Eq. (12). Finally, to satisfy the Transformer architecture calculation, $\widehat{X}$ is expanded in the last two dimensions. The equation is shown as follows:
\begin{equation}
	X^{\text{SW}}=\text{Flatten}(\widehat{X})
	\label{eq-14}
\end{equation}
where $X^{\text{SW}} \in \mathbb{R}^{\frac{H}{\lambda}\times W\eta}$ represents the feature matrix after intra-window attention (SW-MHA).
\par \textbf{W-MHA:} The computation process for inter-window attention uses traditional MHA (see Eqs. (\ref{eq-9}-\ref{eq-12})). Unlike MHA, the input sequence for inter-window attention is the features obtained after the calculation of intra-window attention. Therefore, inter-window attention does not change the shape of the input features.
\par \textbf{Transformer Block:} Subsequently, a comprehensive approximate representation of the overall attention is made based on the intra-window attention (SW-MHA) and inter-window attention (W-MHA). Given the effectiveness of the Transformer architecture, SW-Transformer Block and W-Transformer Block are constructed based on SW-MHA and W-MHA. The specific equations are shown as follows:
\begin{equation}
	\begin{split}
		X^{\text{SW}}&=\text{SW-MHA}(\text{LN}(X)))\\
		\xi&=\text{SW-MLP}(\text{LN}(X^{\text{SW}})))
	\end{split}
	\label{eq-15}
\end{equation}
\begin{equation}
	\begin{split}
		\widehat{\xi}&=\xi+\text{W-MHA}(\text{LN}(\xi)))\\
		X^{\text{O}}&=\widehat{\xi}+\text{MLP}(\text{LN}(\widehat{\xi})))
	\end{split}
	\label{eq-16}
\end{equation}
where Eq. (\ref{eq-15}) describes the architecture of SW-Transformer Block and Eq. (\ref{eq-16}) describes the architecture of W-Transformer Block. LN represents Layer Normalization. MLP represents standard multilayer linear perceptron. SW-MLP represents the multilayer linear perceptron specifically for SW-Transformer Block. Compared to MLP, SW-MLP acts not to increase the hidden state space but to decrease the feature dimension. For instance, the output feature $X^{\text{SW}} \in \mathbb{R}^{\frac{H}{\lambda}\times W\eta}$ of SW-MHA has an embedding dimension of $W\eta$. MLP does it by raising the dimension first to $2W\eta$ and subsequently lowering it to $W\eta$. The very large dimension space of $2W\eta$ may also cause GPU memory crash. Therefore, SW-MLP reduces the dimensionality first to $4\eta$ and subsequently to $2\eta$. In this case, SW-MLP prevents GPU memory explosion and also constructs the hidden state space. $\widehat{\xi}$ and $\xi$ represent intermediate variables. $X^{\text{O}}$ represents the Block output. Let's assume that the dimension of the input feature $X$ is $\mathbb{R}^{H\times\eta}$, then the dimensions of $\xi$, $\widehat{\xi}$, and $X^{\text{O}}$ are $\mathbb{R}^{\frac{H}{\lambda}\times2\eta}$.
% Stage 2的输入序列$X$为$S_k \in \mathbb{R}^{H\times2D}$。因此,Stage 2的主要任务是解决超长序列造成的GPU内存崩溃威胁。为此,Stage 2需要一个较大的窗口$\lambda$的滑动步长$W_1$用于窗内特征挖掘(SW-Transformer Block)。同时,为了挖掘窗间关系特征,设定3层W-Transformer Block。对比Stage 2,Stage 3的重点是挖掘序列特征。因此,Stage 2的SW-Transformer Block窗口$\lambda$的滑动步长$W_1$不需要过大。同时,需要设定更多的W-Transformer Block挖掘序列特征。
\par \textbf{Stage 2 and 3:} The input sequence $X$ for Stage 2 is $S_1 \in \mathbb{R}^{H\times2D}$. Therefore, the main task of Stage 2 is to address the threat of GPU memory crashes caused by ultra-long sequences. To this end, a larger sliding step $\lambda_1$ of the window $W_1$ for intra-window feature mining (SW-Transformer Block) is required in Stage 2. Meanwhile, to mine inter-window relationship features, 3 layers of W-Transformer Block are set. Comparing with Stage 2, Stage 3 focuses on mining sequence features. Therefore, the sliding step $\lambda_2$ of the SW-Transformer Block window $W_2$ in Stage 2 does not need to be too large. Meanwhile, more W-Transformer Block needs to be set to mine sequence features.

\subsection{Spammer Detection}
% 前两节分别描述了Stage 1-3的工作原理。此时,Stage 4的目标是识别垃圾邮件发送者,输入序列$S_k$维度为$\mathbb{R}^{\frac{H}{W_1W_2}\times8D}$。首先,从序列$S_k$中选择分类token$\text{CLS}^{\text{S}_k} \in 8D$。随后,将$\text{CLS}^{\text{S}_k}$输入两层线性层中进行特征学习和维度转换。最后,借助softmax函数识别垃圾邮件发送者。因此,模型的最终目标函数如下所示:
% 其中,$\widehat{Y_k}$表示模型输出。Liner表示线性层。Dropout表示Dropout层,其目的是为了充分训练模型参数。随后,模型分类损失采用交叉熵函数。公式如下所示:
% 其中,$N$表示一个批次模型输入样本数量。$Y_k$表示样本真实标签。$\widehat{Y_k}$表示模型预测。随后,结合MVAE双通道损失$\mathcal{L}^\text{I}$和$\mathcal{L}^\text{T}$得到模型总损失$\mathcal{L}^\text{Total}$。公式如下所示:
% 其中,$\psi_1$、$\psi_2$和$\psi_3$分表表示三个损失函数的衰减因子。
\par Stage 1-3 are described in the previous two sections, respectively. In this section, the objective of Stage 4 is to identify spammers with an input sequence $S_3$ of dimension $\mathbb{R}^{\frac{H}{\lambda_1\lambda_2}\times8D}$. Firstly, the classification token $\text{CLS}^{\text{S}} \in 8D$ is selected from the sequence $S_3$. Subsequently, $\text{CLS}^{\text{S}}$ is input into two linear layers for feature learning and dimension transformation. Finally, spammers are identified with the help of softmax function. Thus, the final objective function of the model is shown as follows:
\begin{equation}
	\widehat{Y}=\text{Liner}(\text{Dropout}(\text{Liner}(\text{CLS}^{\text{S}})))
	\label{eq-17}
\end{equation}
where $\widehat{Y}$ represents the model output. Liner represents the linear layer. Dropout represents the Dropout layer, which is intended to adequately train the model parameters. Subsequently, the model classification loss uses the cross-entropy function. The equation is shown as follows:
\begin{equation}
	\mathcal{L}=-\frac{1}{N}\sum^{N}_{k}Y_k\text{log}(\widehat{Y_k})
	\label{eq-18}
\end{equation}
where $N$ represents the number of input samples in a batch. $Y_k$ represents the true labels of the samples. $\widehat{Y_k}$ represents the model prediction. Subsequently, the model total loss $\mathcal{L}^\text{Total}$ is obtained by combining the MVAE two-channel loss $\mathcal{L}^\text{I}$ and $\mathcal{L}^\text{T}$. The equation is shown as follows:
\begin{equation}
	\mathcal{L}^\text{Total}=\psi_1\times\mathcal{L}+\psi_2\times\mathcal{L}^\text{I}+\psi_3\times\mathcal{L}^\text{T}
	\label{eq-19}
\end{equation}
where $\psi_1$, $\psi_2$, and $\psi_3$ represent the decay factors of the three losses, respectively.

\subsection{Learning Algorithm}
% 本文提出了一个新的Transformer,称为MS$^2$Dformer。模型中主要包含两种新的算法。首先,基于传统VAE建立了双通道的MVAE用于多模态行为量化(看算法1)。随后,提出基于分割窗口注意力机制的SW-Transformer Block(看算法2)。
\par A new Transformer, called MS$^2$Dformer, is proposed. Firstly, a two-channel MVAE for multi-modal behavior quantification is built based on the classical VAE (see Algorithm \ref{alg-1}). Subsequently, SW-Transformer Block based on the split-window attention mechanism is proposed (see Algorithm \ref{alg-2}).

\begin{algorithm}
	\renewcommand{\algorithmicrequire}{\textbf{Input:}}
	\renewcommand{\algorithmicensure}{\textbf{Output:}}
	\caption{{User Historical Behavior Tokenization}}
	\label{alg-1}
	\begin{algorithmic}[1]
		\REQUIRE \
		\par The sequence of historical user behaviors $U=\{b_0, ... , b_l\}$;
		\par Individual user behavior $b_i=(T_i, I_i)$;
		\ENSURE \ 
		\par Behavior tokens: $S_1$, Dual-channel loss: $\mathcal{L}^\text{T}$, and $\mathcal{L}^\text{I}$;
		
		\STATE $U \to \{(T^v_0, I^v_l),..., (T^v_l, I^v_l)\}=(T^v, I^v)$ by Eqs. (\ref{eq-1}-\ref{eq-3});
		\STATE $T^v \to z^\text{T}$ and $I^v \to z^\text{I}$ by dual-channel encoder;
		\STATE $z=\mu + \sigma \odot \epsilon=\text{concat}(z^\text{T}, z^\text{I})$;
		\STATE $z \to T^v_d$ and $z \to I^v_d$ by dual-channel decoder;
		\STATE $(T^v_d \to T^v) \to \mathcal{L}^\text{T}$ and $(I^v_d \to I^v) \to \mathcal{L}^\text{I}$;
		\STATE $S_1=\text{concat}(\text{CLS}^{\text{S}}, z)$;
		\STATE \textbf{return} $S_1$, $\mathcal{L}^\text{T}$, and $\mathcal{L}^\text{I}$;
	\end{algorithmic}
\end{algorithm}
\begin{algorithm}
	\renewcommand{\algorithmicrequire}{\textbf{Input:}}
	\renewcommand{\algorithmicensure}{\textbf{Output:}}
	\caption{{SW-MHA}}
	\label{alg-2}
	\begin{algorithmic}[1]
		\REQUIRE \
		\par Input Feature Sequence $S^\text{I} \in \mathbb{R}^{H\times\eta}$;
		\ENSURE \ 
		\par Output Feature Sequence $S^\text{O} \in \mathbb{R}^{\frac{H}{\lambda}\times W\eta}$;
		
		\STATE $Q=S^\text{I}W^\text{Q},\ K=S^\text{I}W^\text{K},\ V=S^\text{I}W^\text{V}$;
		\FOR{$i$ from $0$ to $H$ step $\lambda$)}
			\STATE $\widehat{Q}_i=Q[i\ \text{to}\ W], \widehat{K}_i=K[i\ \text{to}\ W], \widehat{V}_i=V[i\ \text{to}\ W]$;
		\ENDFOR \\
		// $\widehat{Q} \in \mathbb{R}^{\frac{H}{\lambda}\times W\times\eta}, \widehat{K} \in \mathbb{R}^{\frac{H}{\lambda}\times W\times\eta}, \widehat{V} \in \mathbb{R}^{\frac{H}{\lambda}\times W\times\eta}$  by window $W$ and sliding step $\lambda$;\\
		// To prevent cases where $H$ is not a multiple of $\lambda$, link $\lambda$ empty elements at the end of the sequence $S^\text{I}$, similar to the padding process used to CNNs, i.e. $S^\text{I} \in \mathbb{R}^{(H+\lambda)\times\eta}$;
		\STATE $head_i=\text{Attention}(\widehat{Q}W^\text{Q}_{i}, \widehat{K}W^\text{K}_{i}, \widehat{V}W^\text{V}_{i})$;
		\STATE $\widehat{X}=\text{MHA}(\widehat{Q}, \widehat{K}, \widehat{V}) \in \mathbb{R}^{\frac{H}{\lambda}\times W\times\eta}$\\
		\ \ \ \ \ $=\text{Concat}(head_1,...,head_n)W^\text{M}$;
		\STATE $X^{\text{SW}}=\text{Flatten}(\widehat{X})$;
		\STATE \textbf{return} $S^\text{O} \in \mathbb{R}^{\frac{H}{\lambda}\times W\eta}$
	\end{algorithmic}
\end{algorithm}
\subsection{Time Complexity Analysis}
% 如图2(a)所示,MVAE采用双通道的VAE结构。因此,MVAE的时间复杂度为$T_{\text{MVAE}}=T^{\txet{I}}_{\text{encoder}}+T^{\txet{I}}_{\text{decoder}}+T^{\txet{T}}_{\text{encoder}}+T^{\txet{T}}_{\text{decoder}}+T^{z}=O(4((H-1)\cdot 256(768+D)+D)\sim O(H-1)$。随后,SW-MHA的时间复杂度为$O(k(W^2\cdot \eta))$。其中,$k$表示窗口序列的长度。$W$表示窗口长度。$\eta$表示输入序列特征的维度。因此,Stage 2的时间复杂度为$T_{\text{Satge 1}}=T_{\text{SW-Block}}+b\cdot T_{\text{W-Block}}$。其中,$T_{\text{SW-Block}}=O(2D\cdot W_1^2+4D(2D+2DW_1))\sim O(D\cdot (W_1)^2+D^2W_1)$,$T_{\text{SW-Block}}=2DW_1\cdot (H/\lambda_1)^2+(4DW_1)^2\sim DW_1\cdot (H/\lambda_1)^2+(DW_1)^2$。因此,$T_{\text{Satge 1}}=O(D\cdot (H/\lambda_1)^2+D^2W_1)+O(b(DW_1\cdot (H/\lambda_1)^2+(DW_1)^2))\sim O(H/\lambda_1)^2$。类似的,Stage 3的时间复杂度为$O(H/(\lambda_1\cdot \lambda_2))^2$。因此,MS$^2$former的总体时间复杂度为$O(H-1)+O(H/\lambda_1)^2+O(H/(\lambda_1\cdot \lambda_2))^2\sim O(H/\lambda_1)^2$。与传统基于MHA的Transformer架构($O(H^2)$)相比,MS$^2$former在运行效率方面更胜一筹(see Table 4)。
\par As shown in Fig. \ref{fig-2} (a), MVAE adopts a two-channel VAE structure. Therefore, the time complexity of MVAE is $T_{\text{MVAE}}=T^{\text{I}}_{\text{encoder}}+T^{\text{I}}_{\text{decoder}}+T^{\text{T}}_{\text{encoder}}+T^{\text{T}}_{\text{decoder}}+T^{z}=O(4((H-1)\cdot 256(768+D))+D)\sim O(H-1)$. Subsequently, the time complexity of SW-MHA is $O(k(W^2\cdot \eta))$. Where $k$ denotes the length of the window sequence. $W$ denotes the window length. $\eta$ denotes the dimension of the input sequence features. Therefore, the time complexity of Stage 2 is $T_{\text{Satge 1}}=T_{\text{SW-Block}}+b\cdot T_{\text{W-Block}}$. where $T_{\text{SW-Block}}=O(2D\cdot W_1^2+4D(2D+2DW_1))\sim O(D\cdot (W_1)^2+D^2W_1)$, and $T_{\text{SW-Block}}=2DW_1\cdot (W_1)^2+( 4DW_1)^2\sim DW_1\cdot (H/\lambda_1)^2+(DW_1)^2$. Thus, $T_{\text{Satge 1}}=O(D\cdot (W_1)^2+D^2W_1)+O(b(DW_1\cdot (H/\lambda_1)^2+(DW_1)^2))\sim O((H/\lambda_1)^2)$. Similarly, the time complexity of Stage 3 is $O((H/(\lambda_1\cdot \lambda_2))^2)$. Therefore, the overall time complexity of MS$^2$former is $O(H-1)+O((H/\lambda_1)^2)+O((H/(\lambda_1\cdot \lambda_2))^2)\sim O((H/\lambda_1)^2)$. Compared with the traditional MHA-based Transformer architecture ($O(H^2)$), MS$^2$former is superior in terms of operational efficiency (see Table \ref{table-ab-memory}).

\begin{figure*}[h]
	\center{\includegraphics[width=1.03\linewidth]  {./image/Paramters_3.pdf}} 
	\caption{Statistics from two publicly available datasets.}
	\label{fig-Parameters-windows}
\end{figure*}
\begin{figure*}[h]
	\center{\includegraphics[width=1.03\linewidth]  {./image/Paramters_1.pdf}} 
	\caption{The influence of different numbers of historical behaviors on model training.}
	\label{fig-behavior}
\end{figure*}
\section{Experiment and Analysis}
%%%%%%%%%%%%%%%%%%%%%%%%%%%%%%%%%%%%%%%%%%%%%%%%%%%%%%%%%%%%%%%%%%%%%%%%%%%%
\subsection{Experimental Data}	
% 我们在两个公开的虚假信息检测数据集\cite{Yang2024model},即Weibo V1和Weibo V2,上进行模型训练。垃圾邮件发送者通常在经常发送虚假信息的用户群体中产生。但是,这些用户可能并不清楚事实仅仅只是感兴趣从而转发信息,因此不是任何发送过虚假信息的用户就是真正的垃圾邮件发送者。为此,我们聘请了多个领域的研究生专家对Yang等人公开的两个数据集中发布虚假信息的用户进一步判断,从而筛选出真正的垃圾邮件发送者。随后,我们构建并收集了Weibo 2023和Weibo 2024两个公开的垃圾邮件发送者检测基准数据集。其中,Weibo V1包含2022到2023年内被微博平台标记的1000条虚假信息和948个垃圾邮件发送者潜在用户,发送虚假信息超过2条的用户仅有47个。因此,Weibo 2023仅确定342个垃圾邮件发送者。类似的,Weibo V2包含2022到2024年内被微博平台标记的5661条虚假信息和5653个垃圾邮件发送者潜在用户。因此,Weibo 2023仅确定952个垃圾邮件发送者。两个数据集的统计如表1所示:
\begin{table}[t]
	\renewcommand{\arraystretch}{1.3}
	\centering
	\caption{The statistics of the two datasets}
	\begin{tabular}{c|c c}
		\toprule[1.5pt]
		Statistic & Weibo 2023  & Weibo 2024 \\ \hline \hline
		$\#$ of Users& $684$ & $1971$ \\ \hline
		$\#$ of Normal& $343$ & $1019$ \\ \hline
		$\#$ of Spammer & $341$ & $952$ \\ \hline
		Avg. Behavior Length& $26,192$ & $25,310$ \\ \hline
		Max Behavior Length& $292,491$ & $308,798$ \\ \hline
		Min Behavior Length& $15$ & $1$ \\ \hline
		\bottomrule[1.5pt]
	\end{tabular}
	\label{table-datasets}
\end{table}
\par We conducted model training on two publicly available fake news detection datasets \cite{Yang2024model}, i.e., Weibo V1\footnote{\url{https://github.com/yzhouli/DDCA-Rumor-Detection/tree/main/MisDerdect.}} and Weibo V2\footnote{\url{https://github.com/yzhouli/SocialNet/tree/master/Weibo.}}. Spammers are usually created by users who frequently send out fake news. However, these users may not be aware of the facts or be interested in forwarding the news, so not every user who has sent fake news is a real spammer. Therefore, we hired graduate experts in various fields to determine further the users who posted fake news in the two datasets disclosed by \cite{Yang2024model} to filter out the real spammers. Subsequently, we constructed and collected two publicly available spammer detection benchmark datasets, Weibo 2023 and Weibo 2024. Weibo V1 contains 1,000 fake news and 948 potential users of spammers flagged by the Weibo platform between 2022 and 2023, and only 47 users who sent more than two fake news. Therefore, Weibo 2023 identifies only 342 spammers. Similarly, Weibo V2 contains 5,661 fake news and 5,653 potential users of spammers that were flagged by the Weibo platform between 2022 and 2024. Therefore, Weibo 2023 identifies only 952 spammers. The statistics of the two datasets are shown in Table \ref{table-datasets} and Fig.\ref{fig-Parameters-windows}.
%%%%%%%%%%%%%%%%%%%%%%%%%%%%%%%%%%%%%%%%%%%%%%%%%%%%%%%%%%%%%%%%%%%%%%%%%%%%
\subsection{Parameters Setting}
% 我们提出了三个版本的算法,称为MS$^2$Dformer\_B,MS$^2$Dformer\_M,和MS$^2$Dformer\_L。他们分别对应MS$^2$Dformer架构的基础、中等和大型版本。具体设置如下:
\par \textbf{Model Variants:} Three variants of the MS$^2$Dformer model are constructed to cope with different usage environments:
\begin{equation}
	\begin{split}
		\text{MS$^2$Dformer}\_\text{B}:\{D&=16,W=\{64,64\},\\\lambda&=\{32, 4\},B=\{3, 3\}\}\\
	\end{split}
	\label{eq-base}
\end{equation}
\begin{equation}
	\begin{split}
		\text{MS$^2$Dformer}\_\text{M}:\{D&=16,W=\{128,64\},\\\lambda&=\{32, 4\},B=\{3, 11\}\}\\
	\end{split}
\end{equation}
\begin{equation}
	\begin{split}
		\text{MS$^2$Dformer}\_\text{L}:\{D&=64,W=\{128 ,64\},\\\lambda&=\{32, 4\},B=\{7, 17\}\}
	\end{split}
\end{equation}
where $D$ represents the embedding dimension of the encoder in MVAE. $W$ represents the SW-MHA window size in Stage 2-3. $\lambda$ represents the window sliding step. $B$ represents the number of SW-MHA or W-MHA Transformer Block.
% 我们在Tensorflow 2.9.0和Python 3.8的软件平台上编写模型源代码。并且,数据集以7:2:1的比例划分为训练、验证和测试集。同时,我们采用Adam作为模型的优化器,并且设置学习率为$1 \times 10^{-4}$。模型训练时长为60 epochs,同时引入早停机制。随后,我们根据信息传播平均时长设置窗口长度$\lambda=64/128$。之后,采用控制变量法验证损失函数多模态信息衰减因此的值。特别的,本工作的核心是检测垃圾邮件发送者,因此设置$\psi_1$为1。可以发现,$\psi_2=0.4\0.4$和$\psi_3=0.4\0.4$时效果最佳。
\par \textbf{Hyper-parameter Settings:} We have written the model source code on the software platform of Tensorflow 2.9.0 and Python 3.8. Moreover, the dataset is divided into training, validation, and testing sets in the ratio of 7:2:1. Meanwhile, we used Adam as the optimizer of the model and set the learning rate to $1 \times 10^{-4}$. The model training time is 60 epochs, while the early stopping mechanism is introduced. Subsequently, we set the window size $W=64/128$ based on the average length of information dissemination (see Fig. \ref{fig-Parameters-windows} (a)). Afterward, the value of the multi-modal information decay factor of the loss function is verified using the control variable method (see Fig. \ref{fig-Parameters-decay}). In particular, the core of this work is to detect spammers, so $\psi_1$ is set to 1. It can be found that the best results are obtained when $\psi_2=0.3/0.3$ and $\psi_3=0.4/0.4$.
\begin{figure}[t]
	\center{\includegraphics[width=1.03\linewidth]  {./image/Paramters_2.pdf}} 
	\caption{The influence of different values of $\psi_2$ and $\psi_3$ ($\psi_1=1$).}
	\label{fig-Parameters-decay}
\end{figure}
% 我们从三个维度选择了先进的基线算法。首先,在算法方面我们选择了传统的深度学习算法,包括GCN、GAT、Graph-SAGE、RNN、LSTM、GRU和MHA。其次,我们在GCN的基础上选择了Graph-U-Nets、R-GCN和ChebNet。最后,从任务层面出发,我们选择了专用的垃圾邮件发送者检测模型,即MDGCN和Adver-GCN。
\par \textbf{Baseline Settings:} We selected advanced baseline algorithms in three dimensions. Firstly, in terms of algorithms, we selected traditional deep learning algorithms, including GCN\cite{li2019spam}, GAT\cite{zhang2023detecting}, Graph-SAGE\cite{zhang2024predicting}, RNN\cite{zhang2023rumor}, LSTM\cite{babu2023efficient}, GRU\cite{GRU}, and MHA\cite{rao2023hybrid}. Secondly, we selected Graph-U-Nets\cite{gao2019graph}, R-GCN\cite{generale2022scaling}, and ChebNet\cite{he2022convolutional} on the basis of GCN. Finally, from the task dimension, we chose specialized spammer detection models, i.e., MDGCN\cite{deng2023markov}, Graph Transformer\cite{GraphTrans}, and Adver-GCN\cite{zhang2022detecting}.
\subsection{Validity Analysis of Historical Behavioral Length}
% 在社交平台中,用户的历史行为数量分布差异较大。如图b所示,历史行为数量为32768是该分布的拐点,超过70%的用户超过这个阈值。因此,我们验证了这个范围内的不同历史行为数量对于模型性能的影响(如图4所示)。其中,基于GNN的模型,即GCN、GAT和Graph-SAGE在超过2048这个行为数量阈值时性能急剧下降。此时,用户历史行为构建的传播图过大,传统GNN模型不能有效挖掘全局和局部结构信息。另一方面,基于序列建模的模型,即RNN、LSTM和GRU,超过128这个行为数量阈值时性能急剧下降。这是因为这些序列模型在面对超长序列时并不能有效衡量超长期和短期交互关系。最后,我们同样进行了传统多头注意力机制(MHA)面对不同历史行为序列的性能变化。可以发现,MHA可以有效衡量长期和短期交互关系。但是,受制于$QK^{\text{T}}$的存储压力,面对超长序列时MHA无法运行。特别的,图4中$\circ$和$\triangle$分别表示训练平台为PTX 3060 (16 GB)或A800 (80GB)。
\begin{table*}[htbp]
	\renewcommand{\arraystretch}{1.3}
	\caption{Comparison of the performance for different models on the Weibo 2023 dataset}
	\label{table_acc_v1}
	\centering	
	\begin{tabular}{cccccccc}
		\toprule[1.5pt]
		\multicolumn{1}{c|}{\multirow{2}*{Method}}&\multicolumn{1}{c|}{\multirow{2}*{Accuracy}}&\multicolumn{3}{c|}{Normal}&\multicolumn{3}{c}{Spammer}\\
		\cline{3-5} \cline{6-8}
		\multicolumn{1}{c|}{}&\multicolumn{1}{c|}{}&\multicolumn{1}{c}{\multirow{1}*{Precision}}&\multicolumn{1}{c}{Recall}&\multicolumn{1}{c}{F1}&\multicolumn{1}{c}{\multirow{1}*{Precision}}&\multicolumn{1}{c}{Recall}&\multicolumn{1}{c}{F1}\\
		\hline \hline
		GAT & $0.816_{\pm 0.007}$ & $0.83_{\pm 0.003}$ & $0.794_{\pm 0.015}$ & $0.812_{\pm 0.009}$ & $0.803_{\pm 0.011}$ & $0.838_{\pm 0.001}$ & $0.820_{\pm 0.006}$\\
		Graph-SAGE & $0.828_{\pm 0.003}$ & $0.821_{\pm 0.003}$ & $0.838_{\pm 0.014}$ & $0.830_{\pm 0.005}$ & $0.835_{\pm 0.011}$ & $0.817_{\pm 0.008}$ & $0.826_{\pm 0.002}$\\
		GCN & $0.832_{\pm 0.007}$ & $0.798_{\pm 0.001}$ & $0.890_{\pm 0.022}$ & $0.842_{\pm 0.010}$ & $0.874_{\pm 0.020}$ & $0.773_{\pm 0.008}$ & $0.820_{\pm 0.005}$\\
		Graph-U-Nets & $0.758_{\pm 0.008}$ & $0.788_{\pm 0.056}$ & $0.717_{\pm 0.103}$ & $0.743_{\pm 0.030}$ & $0.749_{\pm 0.044}$ & $0.800_{\pm 0.110}$ & $0.767_{\pm 0.036}$\\ 
		R-GCN & $0.820_{\pm 0.004}$ & $0.778_{\pm 0.019}$ & $0.897_{\pm 0.029}$ & $0.832_{\pm 0.002}$ & $0.881_{\pm 0.026}$ & $0.742_{\pm 0.037}$ & $0.804_{\pm 0.011}$\\
		MDGCN & $0.842_{\pm 0.007}$ & $0.840_{\pm 0.034}$ & $0.849_{\pm 0.037}$ & $0.843_{\pm 0.001}$ & $0.848_{\pm 0.020}$ & $0.834_{\pm 0.051}$ & $0.84_{\pm 0.016}$\\  
		ChebNet & $0.834_{\pm 0.001}$ & $0.826_{\pm 0.005}$ & $0.849_{\pm 0.007}$ & $0.837_{\pm 0.001}$ & $0.844_{\pm 0.005}$ & $0.820_{\pm 0.008}$ & $0.832_{\pm 0.001}$\\
		Adver-GCN & $0.844_{\pm 0.007}$ & $0.832_{\pm 0.016}$ & $0.866_{\pm 0.007}$ & $0.848_{\pm 0.004}$ & $0.858_{\pm 0.002}$ & $0.822_{\pm 0.022}$ & $0.840_{\pm 0.010}$\\ 
		Graph Transformer & $0.864_{\pm 0.007}$ & $0.899_{\pm 0.008}$ & $0.825_{\pm 0.021}$ & $0.862_{\pm 0.008}$ & $0.832_{\pm 0.015}$ & $0.904_{\pm 0.007}$ & $0.867_{\pm 0.007}$\\\hline \hline
		
		RNN &  $0.832_{\pm 0.022}$ & $0.816_{\pm 0.005}$ & $0.852_{\pm 0.052}$ & $0.834_{\pm 0.027}$ & $0.850_{\pm 0.043}$ & $0.810_{\pm 0.008}$ & $0.829_{\pm 0.017}$\\
		GRU & $0.863_{\pm 0.011}$ & $0.865_{\pm 0.003}$ & $0.858_{\pm 0.023}$ & $0.861_{\pm 0.013}$ & $0.862_{\pm 0.019}$ & $0.868_{\pm 0.001}$ & $0.864_{\pm 0.010}$\\
		LSTM & $0.866_{\pm 0.008}$ & $0.884_{\pm 0.020}$ & $0.843_{\pm 0.008}$ & $0.862_{\pm 0.006}$ & $0.852_{\pm 0.003}$ & $0.890_{\pm 0.022}$ & $0.870_{\pm 0.009}$\\ 
		MHA & $0.872_{\pm 0.007}$ & $0.871_{\pm 0.012}$ & $0.871_{\pm 0.001}$ & $0.871_{\pm 0.006}$ & $0.874_{\pm 0.003}$ & $0.874_{\pm 0.015}$ & $0.874_{\pm 0.009}$\\ \hline \hline
		
		MS$^2$Dformer\_B & $0.923_{\pm 0.004}$ & $0.913_{\pm 0.001}$ & $0.932_{\pm 0.007}$ & $0.923_{\pm 0.004}$ & $0.932_{\pm 0.007}$ & $\underline{0.912}_{\pm 0.001}$ & $0.922_{\pm 0.003}$\\
		MS$^2$Dformer\_M & $\underline{0.931}_{\pm 0.004}$ & $\underline{0.914}_{\pm 0.011}$ & $\underline{0.948}_{\pm 0.022}$ & $\underline{0.931}_{\pm 0.005}$ & $\underline{0.947}_{\pm 0.021}$ & $0.912_{\pm 0.015}$ & $\underline{0.929}_{\pm 0.003}$\\
		MS$^2$Dformer\_L & $\textbf{0.941}_{\pm 0.004}$ & $\textbf{0.926}_{\pm 0.018}$ & $\textbf{0.959}_{\pm 0.015}$ & $\textbf{0.942}_{\pm 0.002}$ & $\textbf{0.958}_{\pm 0.014}$ & $\textbf{0.923}_{\pm 0.022}$ & $\textbf{0.940}_{\pm 0.005}$\\ \hline
		\bottomrule[1.5pt]
	\end{tabular}
\end{table*}
%\begin{table*}[htbp]
%	\renewcommand{\arraystretch}{1.3}
%	\centering	
%	\caption{Performance comparison of different models on two public datasets}
%	\begin{tabular}{ccccccc}
%		\toprule[1.5pt]
%		\multicolumn{1}{c|}{\multirow{3}*{Method}}&\multicolumn{3}{c|}{Weibo 2023}&\multicolumn{3}{c}{Weibo 2024}\\
%		\cline{2-4} \cline{5-7}
%		\multicolumn{1}{c|}{          }&\multicolumn{1}{c|}{\multirow{2}*{Accuracy}}&\multicolumn{2}{c|}{F1-Score}&\multicolumn{1}{c|}{\multirow{2}*{Accuracy}}&\multicolumn{2}{c}{F1-Score}\\
%		\cline{3-4} \cline{6-7}
%		\multicolumn{1}{c|}{}&\multicolumn{1}{c|}{}&\multicolumn{1}{c}{\multirow{1}*{Normal}}&\multicolumn{1}{c|}{Spammer}&\multicolumn{1}{c|}{}&\multicolumn{1}{c}{\multirow{1}*{Normal}}&\multicolumn{1}{c}{Spammer}\\
%		\hline \hline
%		GAT & $0.816_{\pm 0.007}$ & $0.812_{\pm 0.009}$ & $0.820_{\pm 0.006}$ & $0.810_{\pm 0.003}$ & $0.820_{\pm 0.008}$ & $0.799_{\pm 0.006}$\\ 
%		Graph-SAGE & $0.828_{\pm 0.003}$ & $0.830_{\pm 0.005}$ & $0.826_{\pm 0.002}$& $0.842_{\pm 0.005}$ & $0.853_{\pm 0.009}$ & $0.828_{\pm 0.005}$\\ 
%		GCN & $0.832_{\pm 0.007}$ & $0.842_{\pm 0.010}$ & $0.820_{\pm 0.005}$& $0.806_{\pm 0.005}$ & $0.808_{\pm 0.010}$ & $0.802_{\pm 0.009}$\\ 
%		Graph-U-Nets & $0.758_{\pm 0.008}$ & $0.743_{\pm 0.030}$ & $0.767_{\pm 0.036}$& $0.744_{\pm 0.002}$ & $0.726_{\pm 0.011}$ & $0.761_{\pm 0.01}$\\ 
%		R-GCN & $0.820_{\pm 0.004}$ & $0.832_{\pm 0.002}$ & $0.804_{\pm 0.011}$& $0.842_{\pm 0.004}$ & $0.85_{\pm 0.006}$ & $0.832_{\pm 0.011}$\\ 
%		MDGCN & $0.842_{\pm 0.007}$ & $0.843_{\pm 0.001}$ & $0.84_{\pm 0.016}$& $0.841_{\pm 0.003}$ & $0.846_{\pm 0.001}$ & $0.835_{\pm 0.006}$\\ 
%		ChebNet & $0.834_{\pm 0.001}$ & $0.837_{\pm 0.001}$ & $0.832_{\pm 0.001}$& $0.812_{\pm 0.004}$ & $0.814_{\pm 0.010}$ & $0.808_{\pm 0.017}$\\ 
%		Adver-GCN & $0.844_{\pm 0.007}$ & $0.848_{\pm 0.004}$ & $0.840_{\pm 0.010}$& $0.851_{\pm 0.004}$ & $0.848_{\pm 0.002}$ & $0.854_{\pm 0.008}$\\ \hline \hline
%		
%		RNN &  $0.832_{\pm 0.022}$ & $0.834_{\pm 0.027}$ & $0.829_{\pm 0.017}$& $0.838_{\pm 0.004}$ & $0.837_{\pm 0.006}$ & $0.838_{\pm 0.013}$\\ 
%		GRU & $0.863_{\pm 0.011}$ & $0.861_{\pm 0.013}$ & $0.864_{\pm 0.010}$& $0.856_{\pm 0.008}$ & $0.853_{\pm 0.010}$ & $0.860_{\pm 0.005}$\\ 
%		LSTM & $0.866_{\pm 0.008}$ & $0.862_{\pm 0.006}$ & $0.870_{\pm 0.009}$& $0.875_{\pm 0.004}$ & $0.872_{\pm 0.008}$ & $0.878_{\pm 0.001}$\\ 
%		MHA & $0.872_{\pm 0.007}$ & $0.871_{\pm 0.006}$ & $0.874_{\pm 0.009}$& $0.883_{\pm 0.004}$ & $0.881_{\pm 0.003}$ & $0.884_{\pm 0.004}$\\ \hline \hline
%		
%		MS$^2$Dformer\_B & $0.923_{\pm 0.004}$ & $0.923_{\pm 0.004}$ & $0.922_{\pm 0.003}$& $0.917_{\pm 0.004}$ & $0.915_{\pm 0.006}$ & $0.919_{\pm 0.004}$\\
%		MS$^2$Dformer\_M & $0.931_{\pm 0.004}$ & $0.931_{\pm 0.005}$ & $0.929_{\pm 0.003}$& $0.926_{\pm 0.007}$ & $0.928_{\pm 0.007}$ & $0.924_{\pm 0.008}$\\
%		MS$^2$Dformer\_L & $0.941_{\pm 0.004}$ & $0.942_{\pm 0.002}$ & $0.940_{\pm 0.005}$& $0.935_{\pm 0.004}$ & $0.933_{\pm 0.004}$ & $0.935_{\pm 0.003}$\\ \hline
%		\bottomrule[1.5pt]
%	\end{tabular}
%	\label{table-acc}
%\end{table*}
\par In social platforms, the distribution of the number of historical behaviors of users varies widely. As shown in Fig. \ref{fig-Parameters-windows} (b), the number of historical behaviors of 16,384/32,768 is the inflection point of this distribution, with more than 60/70\% of users exceeding this threshold. Therefore, this range of historical behaviors is verified for the model performance (shown in Fig. \ref{fig-behavior}). In particular, the performance of the GNN-based models, i.e., GCN, GAT, and Graph-SAGE, decreases dramatically when the threshold of 2048, the number of behaviors, is exceeded. At this point, the spread graph constructed by the user's historical behaviors overlaps, and the traditional GNN models cannot effectively mine the global and local structural information. On the other hand, the performance of models based on sequence modeling, i.e., RNN, LSTM, and GRU, decreases dramatically when exceeding the threshold of 128 as the number of behaviors. This is because such sequence models do not effectively measure ultra-long-term and short-term interactions when confronted with ultra-long sequences. Finally, a similar exercise was conducted to examine the performance of the traditional Multi-head Attention Mechanism (MHA) when faced with different lengths of historical behavioral sequences. It can be found that MHA can effectively measure long- and short-term interactions. However, due to the memory pressure of $QK^{\text{T}}$, MHA cannot run when facing ultra-long sequences. Specifically, $\circ$ and $\triangle$ in Fig. \ref{fig-behavior} (a) and (c) represent that the training platform is RTX 4060 (16 GB) or A800 (80 GB).
\begin{table}[htbp]
	\renewcommand{\arraystretch}{1.3}
	\centering	
	\caption{Performance comparison of different multi-modal fusion strategies}
	\resizebox{0.48\textwidth}{!}{
		\begin{tabular}{ccccccc}
			\toprule[1.5pt]
			\multicolumn{1}{c|}{\multirow{2}*{Method}}&\multicolumn{3}{c|}{\multirow{1}*{Weibo 2023}}&\multicolumn{3}{c}{Weibo 2024}\\
			\cline{2-4} \cline{5-7}
			\multicolumn{1}{c|}{}&\multicolumn{1}{c}{Base}&\multicolumn{1}{c}{Middle}&\multicolumn{1}{c|}{Large}&\multicolumn{1}{c}{Base}&\multicolumn{1}{c}{Middle}&\multicolumn{1}{c}{Large}\\
			\hline \hline
			Our (MVAE \& (Bert+ViT)) & $\textbf{0.923}$ & $\textbf{0.931}$ & $\textbf{0.941}$ & $\underline{0.917}$ & $\textbf{0.926}$ & $\textbf{0.935}$ \\ \hline \hline
			-w Text \& w/o Image & $0.859$ & $0.863$ & $0.861$ & $0.863$ & $0.863$ & $0.881$ \\
			-w Image \& w/o Text & $0.904$ & $0.889$ & $0.897$ & $0.875$ & $0.884$ & $0.862$ \\
			-w Add Image \& Text & $0.889$ & $0.904$ & $0.924$ & $0.889$ & $0.897$ & $0.901$ \\ \hline \hline
			% Chinese CLIP: Contrastive Vision-Language Pretraining in Chinese
			-w CLIP \& w/o (Bert+ViT) & $0.909$ & $0.914$ & $0.933$ & $0.913$ & $0.914$ & $0.928$ \\
			% Scaling Up Visual and Vision-Language Representation Learning With Noisy Text Supervision
			-w ALIGN \& w/o (Bert+ViT) & $0.915$ & $0.923$ & $\underline{0.937}$ & $\textbf{0.917}$ & $0.921$ & $0.928$ \\
			% Blip-2: Bootstrapping language-image pre-training with frozen image encoders and large language models
			-w BLIP-2 \& w/o (Bert+ViT)  & $\underline{0.917}$ & $\underline{0.928}$ & $0.935$ & $0.916$ & $\underline{0.924}$ & $\underline{0.929}$ \\ 
			\hline
			\bottomrule[1.5pt]
	\end{tabular}}
	\label{table-ab-multi-model}
\end{table}
\begin{table*}[htbp]
	\renewcommand{\arraystretch}{1.3}
	\caption{Comparison of the performance for different models on the Weibo 2024 dataset}
	\label{table_acc_v2}
	\centering	
	\begin{tabular}{cccccccc}
		\toprule[1.5pt]
		\multicolumn{1}{c|}{\multirow{2}*{Method}}&\multicolumn{1}{c|}{\multirow{2}*{Accuracy}}&\multicolumn{3}{c|}{Normal}&\multicolumn{3}{c}{Spammer}\\
		\cline{3-5} \cline{6-8}
		\multicolumn{1}{c|}{}&\multicolumn{1}{c|}{}&\multicolumn{1}{c}{\multirow{1}*{Precision}}&\multicolumn{1}{c}{Recall}&\multicolumn{1}{c}{F1}&\multicolumn{1}{c}{\multirow{1}*{Precision}}&\multicolumn{1}{c}{Recall}&\multicolumn{1}{c}{F1}\\
		\hline \hline
		GAT & $0.810_{\pm 0.003}$ & $0.803_{\pm 0.018}$ & $0.840_{\pm 0.034}$ & $0.820_{\pm 0.008}$ & $0.821_{\pm 0.024}$ & $0.779_{\pm 0.034}$ & $0.799_{\pm 0.006}$\\
		Graph-SAGE & $0.842_{\pm 0.005}$ & $0.817_{\pm 0.014}$ & $0.894_{\pm 0.037}$ & $0.853_{\pm 0.009}$ & $0.876_{\pm 0.035}$ & $0.786_{\pm 0.029}$ & $0.828_{\pm 0.005}$\\ 
		GCN & $0.806_{\pm 0.005}$ & $0.823_{\pm 0.029}$ & $0.796_{\pm 0.042}$ & $0.808_{\pm 0.010}$ & $0.790_{\pm 0.025}$ & $0.816_{\pm 0.044}$ & $0.802_{\pm 0.009}$\\
		Graph-U-Nets & $0.744_{\pm 0.002}$ & $0.804_{\pm 0.023}$ & $0.664_{\pm 0.035}$ & $0.726_{\pm 0.011}$ & $0.702_{\pm 0.011}$ & $0.831_{\pm 0.037}$ & $0.761_{\pm 0.01}$\\
		R-GCN & $0.842_{\pm 0.004}$ & $0.832_{\pm 0.031}$ & $0.871_{\pm 0.047}$ & $0.85_{\pm 0.006}$ & $0.856_{\pm 0.038}$ & $0.811_{\pm 0.052}$ & $0.832_{\pm 0.011}$\\
		MDGCN & $0.841_{\pm 0.003}$ & $0.85_{\pm 0.012}$ & $0.842_{\pm 0.012}$ & $0.846_{\pm 0.001}$ & $0.831_{\pm 0.007}$ & $0.838_{\pm 0.018}$ & $0.835_{\pm 0.006}$\\ 
		ChebNet & $0.812_{\pm 0.004}$ & $0.838_{\pm 0.048}$ & $0.798_{\pm 0.062}$ & $0.814_{\pm 0.010}$ & $0.798_{\pm 0.034}$ & $0.828_{\pm 0.074}$ & $0.808_{\pm 0.017}$\\
		Adver-GCN & $0.851_{\pm 0.004}$ & $0.864_{\pm 0.021}$ & $0.832_{\pm 0.022}$ & $0.848_{\pm 0.002}$ & $0.840_{\pm 0.012}$ & $0.870_{\pm 0.029}$ & $0.854_{\pm 0.008}$\\ 
		Graph Transformer & $0.865_{\pm 0.008}$ & $0.898_{\pm 0.006}$ & $0.827_{\pm 0.023}$ & $0.861_{\pm 0.009}$ & $0.838_{\pm 0.017}$ & $0.902_{\pm 0.008}$ & $0.869_{\pm 0.006}$\\ \hline \hline
		RNN & $0.838_{\pm 0.004}$ & $0.839_{\pm 0.04}$ & $0.840_{\pm 0.052}$ & $0.837_{\pm 0.006}$ & $0.846_{\pm 0.033}$ & $0.836_{\pm 0.058}$ & $0.838_{\pm 0.013}$\\ 
		GRU & $0.856_{\pm 0.008}$ & $0.866_{\pm 0.003}$ & $0.840_{\pm 0.022}$ & $0.853_{\pm 0.010}$ & $0.848_{\pm 0.017}$ & $0.872_{\pm 0.008}$ & $0.860_{\pm 0.005}$\\
		LSTM & $0.875_{\pm 0.004}$ & $0.883_{\pm 0.016}$ & $0.863_{\pm 0.03}$ & $0.872_{\pm 0.008}$ & $0.869_{\pm 0.022}$ & $0.887_{\pm 0.022}$ & $0.878_{\pm 0.001}$\\ 
		MHA & $0.883_{\pm 0.004}$ & $0.885_{\pm 0.007}$ & $0.878_{\pm 0.003}$ & $0.881_{\pm 0.003}$ & $0.881_{\pm 0.001}$ & $0.887_{\pm 0.007}$ & $0.884_{\pm 0.004}$\\ \hline \hline
		
		MS$^2$Dformer\_B & $0.917_{\pm 0.004}$ & $\underline{0.928}_{\pm 0.013}$ & $0.903_{\pm 0.022}$ & $0.915_{\pm 0.006}$ & $0.907_{\pm 0.018}$ & $\underline{0.930}_{\pm 0.014}$ & $0.919_{\pm 0.004}$\\
		MS$^2$Dformer\_M & $\underline{0.926}_{\pm 0.007}$ & $0.899_{\pm 0.012}$ & $\textbf{0.959}_{\pm 0.015}$ & $\underline{0.928}_{\pm 0.007}$ & $\underline{0.917}_{\pm 0.015}$ & $0.893_{\pm 0.015}$ & $\underline{0.924}_{\pm 0.008}$\\
		MS$^2$Dformer\_L & $\textbf{0.935}_{\pm 0.004}$ & $\textbf{0.937}_{\pm 0.001}$ & $\underline{0.930}_{\pm 0.008}$ & $\textbf{0.933}_{\pm 0.004}$ & $\textbf{0.932}_{\pm 0.007}$ & $\textbf{0.938}_{\pm 0.004}$ & $\textbf{0.935}_{\pm 0.003}$\\ \hline
		\bottomrule[1.5pt]
	\end{tabular}
\end{table*}
\begin{figure}[t]
	\center{\includegraphics[width=1\linewidth]  {./image/Paramters.pdf}} 
	\caption{Comparison of model training based on optimal parameter settings.}
	\label{fig-models}
\end{figure}
\begin{table*}[t]
	\renewcommand{\arraystretch}{1.3}
	\centering	
	\caption{GPU  platform operation with different MHA mechanisms (behavior size =16384). SPU represents the number of \textbf{s}econds that the model consumes to \textbf{p}rocess a individual \textbf{u}ser.}
		\begin{tabular}{ccccccccc}
			\toprule[1.5pt]
			\multicolumn{1}{c|}{\multirow{2}*{Method}}&\multicolumn{1}{c|}{\multirow{2}*{GPU Platform }}&\multicolumn{1}{c|}{\multirow{2}*{Run}}&\multicolumn{2}{c|}{\multirow{1}*{Base}}&\multicolumn{2}{c|}{\multirow{1}*{Middle}}&\multicolumn{2}{c}{\multirow{1}*{Large}}\\
			\cline{4-9}
			\multicolumn{1}{c|}{}&\multicolumn{1}{c|}{}&\multicolumn{1}{c|}{}&\multicolumn{1}{c}{Params}&\multicolumn{1}{c|}{SPU}&\multicolumn{1}{c}{Params}&\multicolumn{1}{c|}{SPU}&\multicolumn{1}{c}{Params}&\multicolumn{1}{c}{SPU}\\
			\hline \hline
			Our (SW-MHA \& W-MHA) & RTX 4060 (16 GB) & $\surd$ & $2.21$ M & $0.29$ s & $3.66$ M & $0.32$ s & $53.8$ M & $0.35$ s \\ \hline \hline
			-w CNN \& w/o SW-MHA & RTX 4060 (16 GB) & $\surd$ & $1.97$ M & $0.20$ s & $2.53$ M & $0.23$ s & $20.9$ M & $0.27$ s \\ \hline \hline
			-w MHA \& w MVAE & A800 (80 GB) & $\times$ & $-$ & $-$ & $-$ & $-$ & $-$ & $-$  \\ \hline \hline
			-w SW-SMHA \& w MVAE & V100 (32 GB) & $\surd$ & $1.73$ M & $77.2$ s & $1.82$ M & $172.6$ s & $6.41$ M & $181.3$ s \\
			-w BRSW-SMHA \& w MVAE & V100 (32 GB) & $\surd$ & $1.73$ M & $0.87$ s & $1.82$ M & $1.61$ s & $6.41$ M & $3.20$ s \\
			-w BSW-SMHA \& w MVAE & V100 (32 GB) & $\surd$ & $1.73$ M & $1.05$ s & $1.82$ M & $1.22$ s & $6.41$ M & $3.78$ s \\
			-w BDW-SMHA \& w MVAE & V100 (32 GB) & $\surd$ & $1.73$ M & $1.55$ s & $1.82$ M & $1.56$ s & $6.41$ M & $3.20$ s \\
			-w BGSW-SMHA \& w MVAE & V100 (32 GB) & $\surd$ & $1.73$ M & $0.89$ s & $1.82$ M & $1.74$ s & $6.41$ M & $3.82$ s \\ \hline
			\bottomrule[1.5pt]
	\end{tabular}
	\label{table-ab-memory}
\end{table*}
% 基于图4验证的最佳历史行为长度训练MS$^2$Dformer和基线模型,训练过程如图6所示,训练结果如表2所示。对比GNN变体,MS$^2$Dformer\_L模型性能提升$+0.097$和$+0.084$,MHA模型性能提升$+0.028$和$+0.032$。可以发现,序列建模策略更适合当前任务。随后,对比序列模型变体,MS$^2$Dformer\_L模型性能提升$+0.069$和$+0.052$。MS$^2$Dformer架构能够有效挖掘序列特征,从而验证了它的有效性。
\subsection{Overall Performance Analysis}
\par The MS$^2$Dformer and baseline models are trained based on the optimal behavior length, the training process is shown in Fig. \ref{fig-models}, and the training results are shown in Table \ref{table_acc_v1}-\ref{table_acc_v2}. 
\par Comparing the GNN variants, the MS$^2$Dformer\_L model performance improves $+0.077$ and $+0.07$, and the MHA model performance improves $+0.008$ and $+0.018$. It can be found that the sequence modeling strategy is more suitable for the task at hand. Subsequently, comparing the sequence modeling variants, the MS$^2$Dformer\_L model performance is improved by $+0.069$ and $+0.052$. The MS$^2$Dformer architecture is able to efficiently mine the sequence features, thus validating its effectiveness.
\subsection{Ablation Study}
% 相较于单模态建模策略,即-w Text \& w/o Image和-w Image \& w/o Text,融合特征策略(MVAE \& (Bert+ViT))性能提升$+3.7$\%到$+8.2$\%。因此,可以证明多模态融合策略的有效性。随后,对比直接融合策略(-w Add Image \& Text),采用预训练模型的融合策略(-w CLIP/ALIGN/BLIP-2 w/o (Bert+ViT))性能也存在大幅度提升。例如,CLIP模型基于对抗学习策略构建模型。随后,在广泛的数据集学习到更加繁华的知识。因此,采用CLIP/ALIGN/BLIP-2三个预训练模型的效果更好。但是,他们无法更加专注与当前任务。因此,采用MVAE \& (Bert+ViT)的组合策略,并且进行全量训练达到了最佳性能。
\par \textbf{Multi-Modal Fusion:} As shown in Table \ref{table-ab-multi-model}, compared to the uni-modal modeling strategies, i.e., -w Text \& w/o Image and -w Image \& w/o Text, the fusion feature strategy (MVAE \& (Bert+ViT)) improves the performance by $+3.7$\% to $+8.2$\%. Thus, the validity of the multi-modal fusion strategy can be demonstrated. Subsequently, comparing the direct fusion strategy (-w Add Image \& Text), there also exists a substantial performance improvement in the fusion strategy (-w CLIP\cite{yang2022chinese}/ALIGN\cite{jia2021scaling}/BLIP-2\cite{li2023blip} \& w/o (Bert+ViT)) using pre-trained models. For instance, the CLIP model constructs models based on an adversarial learning strategy. Subsequently, more generalized knowledge is learned over a wide range of datasets. Therefore, the three pre-trained CLIP/ALIGN/BLIP-2 models are used with better results. However, they could not focus more on the current task. Therefore, a combination strategy of MVAE \& (Bert+ViT) and full training was used to achieve the best performance.
% 本工作的提出了一种全新的基于层次分割窗口的MHA机制。随后,修改后的MHA机制可以在较低显存的GPU平台上运行(看表1上)。当前,学术界提出了稀疏注意力机制解决超长序列的注意力计算问题。稀疏注意力在缓解GPU显存压力方面做出了卓越贡献(看表1下)。但是,稀疏注意力本质上无法有效挖掘超长序列的长期交互关系,因此它在当前任务表现不佳。特别的,为了环节超长序列建模问题,CNN被引入在Transformer 块之前,从而将输入序列降低到GPU平台支持的长度。本工作提出的SW-MHA机制与CNN有类似作用,但是CNN无法有效量化短期交互关系,因此性能较低。最后,如表1所示,我们提出的模型在显存最小的PTX 4060 (16 GB)平台完成了超过53百万参数的模型训练。并且,模型平均处理用户时间(FPS)较低。因此,我们提出的模型的有效性被再一次验证。
\begin{table}[t]
	\renewcommand{\arraystretch}{1.3}
	\centering	
	\caption{Performance comparison of different MHA mechanisms}
	\resizebox{0.48\textwidth}{!}{
		\begin{tabular}{ccccccc}
			\toprule[1.5pt]
			\multicolumn{1}{c|}{\multirow{2}*{Method}}&\multicolumn{3}{c|}{\multirow{1}*{Weibo 2023}}&\multicolumn{3}{c}{Weibo 2024}\\
			\cline{2-4} \cline{5-7}
			\multicolumn{1}{c|}{}&\multicolumn{1}{c}{Base}&\multicolumn{1}{c}{Middle}&\multicolumn{1}{c|}{Large}&\multicolumn{1}{c}{Base}&\multicolumn{1}{c}{Middle}&\multicolumn{1}{c}{Large}\\
			\hline \hline
			Our (SW-MHA \& W-MHA) & $\textbf{0.923}$ & $\textbf{0.931}$ & $\textbf{0.941}$ & $\textbf{0.917}$ & $\textbf{0.926}$ & $\textbf{0.935}$ \\ \hline \hline
			-w CNN \& w/o SW-MHA &  $0.911$ & $0.912$ & $0.919$ & $0.912$ & $0.910$ & $0.914$ \\ \hline \hline
			-w SW-SMHA \& w MVAE & $\underline{0.919}$ & $0.924$ & $0.927$ & $0.915$ & $\underline{0.921}$ & $\underline{0.928}$ \\
			-w BRSW-SMHA \& w MVAE & $0.907$ & $0.912$ & $0.913$ & $0.907$ & $0.913$ & $0.911$ \\
			-w BSW-SMHA \& w MVAE & $0.913$ & $0.918$ & $0.922$ & $0.913$ & $0.918$ & $0.918$ \\
			-w BDW-SMHA \& w MVAE & $0.912$ & $0.921$ & $0.923$ & $0.901$ & $0.919$ & $0.908$ \\
			-w BGSW-SMHA \& w MVAE & $0.918$ & $0.926$ & $\underline{0.929}$ & $\underline{0.916}$ & $0.917$ & $0.927$ \\ \hline
			\bottomrule[1.5pt]
	\end{tabular}}
	\label{table-ab-MHA}
\end{table}
\par \textbf{Sequence Modeling Variants:} This work proposes a new MHA mechanism based on hierarchical split windows. Subsequently, the modified MHA mechanism can run on GPU platforms with lower memory (see Table \ref{table-ab-memory} up). Currently, sparse attention mechanisms are proposed in academia to solve the problem of attention computation for ultra-long sequences. Sparse attention has made excellent contributions in relieving GPU memory pressure (see Table \ref{table-ab-memory} bottom). However, sparse attention inherently fails to effectively mine the long-term interactions of ultra-long sequences. Thus, it performs poorly in the current task (see Table \ref{table-ab-MHA}). In particular, to address the problem of modeling ultra-long sequences, CNN is introduced before the Transformer block, thus reducing the input sequence to a length supported by the GPU platform. The SW-MHA mechanism proposed in this work works similarly to the CNN. However, the CNN cannot quantify short-term interactions efficiently, thus performing less (see Table \ref{table-ab-MHA}). Finally, as shown in Table \ref{table-ab-memory}, our proposed model completes model training with more than 53 million parameters on the RTX 4060 (16 GB) platform with the smallest memory. Moreover, the model has a low SPU. Thus, the validity of our proposed model is validated once again.
\begin{table}[htbp]
	\renewcommand{\arraystretch}{1.3}
	\centering	
	\caption{Performance comparison of MS$^2$Dformer\_B models based on different position encoding}
	\resizebox{0.3\textwidth}{!}{
	\begin{tabular}{ccc}
		\toprule[1.5pt]
		Method & Weibo 2023 & Weibo 2024\\
		\hline \hline
		Our (APE \& APE) & $\textbf{0.923}$ & $\textbf{0.907}$\\ \hline \hline
		-w/o PE \& w/o PE & $0.896$ & $0.891$ \\
		-w/o PE \& w APE & $0.901$ & $0.897$ \\
		-w/o PE \& w TPE & $0.915$ & $0.899$ \\ \hline \hline
		-w APE \& w/o PE & $0.911$ & $0.901$ \\
		-w APE \& w TPE & $\underline{0.919}$ & $\underline{0.905}$ \\ \hline \hline
		-w TPE \& w/o PE & $0.896$ & $0.887$ \\
		-w TPE \& w APE & $0.909$ & $0.902$ \\
		-w TPE \& w TPE & $0.898$ & $0.903$ \\ \hline
		\bottomrule[1.5pt]
	\end{tabular}}
	\label{table-ab-PE}
\end{table}
% 基于序列建模的垃圾邮件发送者检测模型重点关注历史行为序列在时序特征上长期和短期交互特征。在MHA机制中,位置编码的目标是让模型能够知道序列的位置关系。因此,消除SW-MHA和W-MHA的位置编码(PE)时,模型能够急剧下降(看表1)。结合绝对位置编码(APE)时,模型能够有效感知时序特征。采用可训练的位置编码时(TPE)模型可能会获取更有效的位置信息。但是,TPE也可能破坏时间维度的长期和短期关系。因此,采用TPE机制的模型性能是不稳定的。综上,SW-MHA和W-MHA均采用APE时模型效果最佳。
\par \textbf{Position Encoding:} The spammer detection model based on sequence modeling focuses on the long-term and short-term interaction features of historical behavioral sequences in terms of temporal features. In the MHA mechanism, position encoding aims to enable the model to know the positional relationship of the sequences. Therefore, the model can decline dramatically when removing the position encoding (PE) for SW-MHA and W-MHA (see Table \ref{table-ab-PE}). When combined with absolute position encoding (APE), the model can perceive temporal features effectively. The model may acquire more effective position information when using trainable position encoding (TPE). However, TPE may also disrupt the time dimension's long- and short-term relationships. Therefore, the performance of models using the TPE mechanism is unstable. In summary, the model works best when APE is used for the SW-MHA and W-MHA.
% 在社交平台中,用户通常发布带有噪声的信息影响模型提取有意义的特征(看图1)。采用MVAE,我们提出的模型可以有效环节噪声影响(看表2)。同时,用户存在短期自身行为交互关系。为此,我们采用分割窗口的策略挖掘短期交互关系(看图2)。然而,由于在线社交方式已经经过多年发展,因此垃圾邮件发送者经常借助已经发布过的或过时的信息再次引导舆论(看图3)。这种超长期行为的交互关系对于识别潜伏的垃圾邮件发送者十分重要。传统MHA受到显存影响,因此无法处理超长期历史交互关系(如表3上)。为了解决这种问题,学者提出SMHA对超长序列建模。然而,它们均无法避免构建$QK^\text{T}$矩阵(看表3下)。最后,我们在MHA和SMHA的基础上提出了基于分割窗口的层次注意力机制,它在性能、算力消耗、SPU、参数量等指标上均存在巨大提升。
\begin{figure}[t]
	\center{\includegraphics[width=1\linewidth]  {./image/Case.pdf}} 
	\caption{The case of ultra-long history interactions with similar user behavior. (a) and (b) are selected from different historical behaviors of a individual user.}
	\label{fig-ultra-long-term}
\end{figure}
\subsection{Case Study}
\par In social platforms, users usually post information with noise affecting the model to extract meaningful features (see Fig. \ref{fig-inspire}). Using MVAE, our proposed model can effectively reduce the noise influence (see Table \ref{table-ab-multi-model}). Meanwhile, users have short-term self-behavioral interactions. Therefore, we adopt the strategy of split window to mine the short-term interaction relations (see Fig. \ref{fig-Parameters-windows}). However, because online social has been developed for many years, spammers often guide public opinion again with the help of already published or outdated information (see Fig. \ref{fig-ultra-long-term}). This interaction of ultra-long-term behavior is important for identifying latent spammers. Traditional MHA is affected by explicit memory, so it cannot handle ultra-long-term historical interactions (see Table \ref{table-ab-memory} up). To address this problem, scholars have proposed SMHA to model ultra-long sequences. However, none of them can avoid constructing the $QK^\text{T}$ matrix (see Fig. \ref{fig-MHAs}). Finally, we propose a hierarchical attention mechanism based on a split window according to MHA and SMHA, which exists a considerable performance improvement (see Table \ref{table_acc_v1}-\ref{table_acc_v2}), arithmetic consumption, SPU, and parameter size (see Table \ref{table-ab-memory}).
\section{Conclusions}
\par In this paper, we propose a model for multi-modal sequential spammer detection. To relieve the effect of multi-modal noise, a two-channel VAE mechanism is first constructed to complete the history behavior Tokenization. Subsequently, to model ultra-long historical behavior sequences, a hierarchical multi-head attention mechanism based on the split window is proposed for the first time. The MS$^2$Dformer architecture, with the largest parameters, lower processing time, and huge performance improvement, is trained in a low-computing power platform (RTX 4060 (16 GB)). In the future, the performance of the hierarchical multi-head attention mechanism will be tested for other ultra-long sequence representation tasks, such as recommendation systems.
\begin{thebibliography}{38}
\bibitem{li2019spam}
A. Li, Z. Qin, R. Liu, Y. Yang, and D. Li, “Spam review detection with graph convolutional networks,” in Proceedings of the 28th ACM international conference on information and knowledge management, 2019, pp. 2703–2711.

\bibitem{zhang2023detecting}
F. Zhang, J. Wu, P. Zhang, R. Ma, and H. Yu, “Detecting collusive spammers with heterogeneous graph attention network,” Information Processing \& Management, vol. 60, no. 3, p. 103282, 2023.

\bibitem{jiang2024learning}
B. Jiang, Z. Zhang, S. Ge, B. Wang, X. Wang, and J. Tang, “Learning graph attentions via replicator dynamics,” IEEE Transactions on Pattern Analysis and Machine Intelligence, vol. 46, no. 12, pp. 7720–7727, 2024.

\bibitem{chen2024gnn}
C. Chen, Y. Wu, Q. Dai, H.-Y. Zhou, M. Xu, S. Yang, X. Han, and Y. Yu, “A survey on graph neural networks and graph transformers in computer vision: A task-oriented perspective,” IEEE Transactions on Pattern Analysis and Machine Intelligence, vol. 46, no. 12, pp. 10297--10318, 2024.

\bibitem{zhang2024predicting}
X. Zhang and W. Gao, “Predicting viral rumors and vulnerable users with graph-based neural multi-task learning for infodemic surveillance,” Information Processing \& Management, vol. 61, no. 1, p. 103520, 2024.

\bibitem{devlin2018bert}
J.~Devlin, ``Bert: Pre-training of deep bidirectional transformers for language
  understanding,'' \emph{arXiv preprint arXiv:1810.04805}, 2018.

\bibitem{radford2019language}
A.~Radford, J.~Wu, R.~Child, D.~Luan, D.~Amodei, I.~Sutskever \emph{et~al.},
  ``Language models are unsupervised multitask learners,'' \emph{OpenAI blog},
  vol.~1, no.~8, p.~9, 2019.

\bibitem{brown2020language}
T.~B. Brown, ``Language models are few-shot learners,'' \emph{arXiv preprint
  arXiv:2005.14165}, 2020.

\bibitem{beltagy2020longformer}
I.~Beltagy, M.~E. Peters, and A.~Cohan, ``Longformer: The long-document
  transformer,'' \emph{arXiv preprint arXiv:2004.05150}, 2020.

\bibitem{achiam2023gpt}
J.~Achiam, S.~Adler, S.~Agarwal, L.~Ahmad, I.~Akkaya, F.~L. Aleman, D.~Almeida,
  J.~Altenschmidt, S.~Altman, S.~Anadkat \emph{et~al.}, ``Gpt-4 technical
  report,'' \emph{arXiv preprint arXiv:2303.08774}, 2023.

\bibitem{wang2024utilizing}
J.~Wang, J.~X. Huang, X.~Tu, J.~Wang, A.~J. Huang, M.~T.~R. Laskar, and
  A.~Bhuiyan, ``Utilizing bert for information retrieval: Survey, applications,
  resources, and challenges,'' \emph{ACM Computing Surveys}, vol.~56, no.~7,
  pp. 1--33, 2024.

\bibitem{liu2021swin}
Z.~Liu, Y.~Lin, Y.~Cao, H.~Hu, Y.~Wei, Z.~Zhang, S.~Lin, and B.~Guo, ``Swin
  transformer: Hierarchical vision transformer using shifted windows,'' in
  \emph{Proceedings of the IEEE/CVF international conference on computer
  vision}, 2021, pp. 10\,012--10\,022.

\bibitem{han2022survey}
K.~Han, Y.~Wang, H.~Chen, X.~Chen, J.~Guo, Z.~Liu, Y.~Tang, A.~Xiao, C.~Xu,
  Y.~Xu \emph{et~al.}, ``A survey on vision transformer,'' \emph{IEEE
  transactions on pattern analysis and machine intelligence}, vol.~45, no.~1,
  pp. 87--110, 2022.

\bibitem{wang2024revisiting}
Z.~Wang, C.~Pei, M.~Ma, X.~Wang, Z.~Li, D.~Pei, S.~Rajmohan, D.~Zhang, Q.~Lin,
  H.~Zhang \emph{et~al.}, ``Revisiting vae for unsupervised time series anomaly
  detection: A frequency perspective,'' in \emph{Proceedings of the ACM on Web
  Conference 2024}, 2024, pp. 3096--3105.

\bibitem{fang2023unsupervised}
L.~Fang, K.~Feng, K.~Zhao, A.~Hu, and T.~Li, ``Unsupervised rumor detection
  based on propagation tree vae,'' \emph{IEEE Transactions on Knowledge and
  Data Engineering}, vol.~35, no.~10, pp. 10\,309--10\,323, 2023.

\bibitem{Bian2020Rumor}
T.~Bian, X.~Xiao, T.~Xu, P.~Zhao, W.~Huang, Y.~Rong, and J.~Huang, ``Rumor
  detection on social media with bi-directional graph convolutional networks,''
  in \emph{Proceedings of the AAAI conference on artificial intelligence},
  2020, pp. 549--556.

\bibitem{wei2024modeling}
L.~Wei, D.~Hu, W.~Zhou, X.~Wang, and S.~Hu, ``Modeling the uncertainty of
  information propagation for rumor detection: A neuro-fuzzy approach,''
  \emph{IEEE Transactions on Neural Networks and Learning Systems}, vol.~35,
  no.~2, pp. 2522--2533, 2024.

\bibitem{deng2023markov}
L.~Deng, C.~Wu, D.~Lian, Y.~Wu, and E.~Chen, ``Markov-driven graph
  convolutional networks for social spammer detection,'' \emph{IEEE
  Transactions on Knowledge and Data Engineering}, vol.~35, no.~12, pp.
  12\,310--12\,322, 2023.

\bibitem{Yang2024Topic}
Z.~Yang, Y.~Pang, X.~Li, Q.~Li, S.~Wei, R.~Wang, and Y.~Xiao, ``Topic
  audiolization: A model for rumor detection inspired by lie detection
  technology,'' \emph{Information Processing \& Management}, vol.~61, no.~1, p.
  103563, 2024.

\bibitem{ma2016detecting}
J.~Ma, W.~Gao, P.~Mitra, S.~Kwon, B.~J. Jansen, K.-F. Wong, and M.~Cha,
  ``Detecting rumors from microblogs with recurrent neural networks,'' in
  \emph{Proceedings of the 25th International Joint Conference on Artificial
  Intelligence}, 2016, pp. 3818--3824.

\bibitem{ma2021improving}
J.~Ma, J.~Li, W.~Gao, Y.~Yang, and K.-F. Wong, ``Improving rumor detection by
  promoting information campaigns with transformer-based generative adversarial
  learning,'' \emph{IEEE Transactions on Knowledge and Data Engineering},
  vol.~35, no.~3, pp. 2657--2670, 2023.

\bibitem{Sun2022ddgcn}
M.~Sun, X.~Zhang, J.~Zheng, and G.~Ma, ``Ddgcn: Dual dynamic graph
  convolutional networks for rumor detection on social media,'' in
  \emph{Proceedings of the AAAI Conference on Artificial Intelligence}, 2022,
  pp. 4611--4619.

\bibitem{wang2023cross}
L.~Wang, C.~Zhang, H.~Xu, Y.~Xu, X.~Xu, and S.~Wang, ``Cross-modal contrastive
  learning for multimodal fake news detection,'' in \emph{Proceedings of the
  31st ACM international conference on multimedia}, 2023, pp. 5696--5704.

\bibitem{zhang2024reinforced}
L.~Zhang, X.~Zhang, Z.~Zhou, F.~Huang, and C.~Li, ``Reinforced adaptive
  knowledge learning for multimodal fake news detection,'' in \emph{Proceedings
  of the AAAI Conference on Artificial Intelligence}, vol.~38, no.~15, 2024,
  pp. 16\,777--16\,785.

\bibitem{qu2024temporal}
Z.~Qu, F.~Zhou, X.~Song, R.~Ding, L.~Yuan, and Q.~Wu, ``Temporal enhanced
  multimodal graph neural networks for fake news detection,'' \emph{IEEE
  Transactions on Computational Social Systems}, 2024.

\bibitem{Yang2024model}
Z.~Yang, Y.~Pang, Q.~Li, S.~Wei, R.~Wang, and Y.~Xiao, ``A model for early
  rumor detection base on topic-derived domain compensation and multi-user
  association,'' \emph{Expert Systems with Applications}, vol. 250, p. 123951,
  2024.

\bibitem{zhang2023rumor}
Q. Zhang, Y. Yang, C. Shi, A. Lao, L. Hu, S. Wang, and U. Naseem, “Rumor detection with hierarchical representation on bipartite ad hocevent trees,” IEEE Transactions on Neural Networks and Learning Systems, vol. 35, no. 10, pp. 14x112–-14124, 2024.

\bibitem{babu2023efficient}
R. Babu, J. Kannappan, B. V. Krishna, and K. Vijay, “An efficient spam detector model for accurate categorization of spam tweets using quantum chaotic optimization-based stacked recurrent network,” Nonlinear Dynamics, vol. 111, no. 19, pp. 18523–-18540, 2023.

\bibitem{GRU}
K. Yu, X. Zhu, Z. Guo, A. Tolba, J. J. P. C. Rodrigues, and V. C. Leung, “A cross-field deep learning-based fuzzy spamming detection approach via collaboration of behavior modeling and sentiment analysis,” IEEE Transactions on Fuzzy Systems, vol. 32, no. 12, pp. 7168–7182, 2024.

\bibitem{rao2023hybrid}
S. Rao, A. K. Verma, and T. Bhatia, “Hybrid ensemble framework with self-attention mechanism for social spam detection on imbalanced data,” Expert Systems with Applications, vol. 217, p. 119594, 2023.

\bibitem{gao2019graph}
H. Gao and S. Ji, “Graph u-nets,” in international conference on machine learning. PMLR, 2019, pp. 2083–-2092.

\bibitem{generale2022scaling}
A. Generale, T. Blume, and M. Cochez, “Scaling r-gcn training with graph summarization,” in Companion Proceedings of the Web Conference 2022, 2022, pp. 1073–-1082.

\bibitem{he2022convolutional}
M. He, Z. Wei, and J.-R. Wen, “Convolutional neural networks on graphs with chebyshev approximation, revisited,” Advances in neural information processing systems, vol. 35, pp. 7264–-7276, 2022.

\bibitem{GraphTrans}
Y. Cai, H. Wang, H. Cao, W. Wang, L. Zhang, and X. Chen, “Detecting spam movie review under coordinated attack with multi-view explicit and implicit relations semantics fusion,” IEEE Transactions on Information Forensics and Security, vol. 19, pp. 7588–-7603, 2024.

\bibitem{zhang2022detecting}
F. Zhang, S. Yuan, P. Zhang, J. Chao, and H. Yu, “Detecting review spammer groups based on generative adversarial networks,” Information Sciences, vol. 606, pp. 819–836, 2022.

\bibitem{yang2022chinese}
A. Yang, J. Pan, J. Lin, R. Men, Y. Zhang, J. Zhou, and C. Zhou, “Chinese clip: Contrastive vision-language pretraining in chinese,” arXiv preprint arXiv:2211.01335, 2022.

\bibitem{jia2021scaling}
C. Jia, Y. Yang, Y. Xia, Y.-T. Chen, Z. Parekh, H. Pham, Q. Le, Y.-H. Sung, Z. Li, and T. Duerig, “Scaling up visual and vision-language representation learning with noisy text supervision,” in International conference on machine learning. PMLR, 2021, pp. 4904–4916.

\bibitem{li2023blip}
J. Li, D. Li, S. Savarese, and S. Hoi, “Blip-2: Bootstrapping languageimage pre-training with frozen image encoders and large language models,” in International conference on machine learning. PMLR, 2023, pp. 19 730–19 742.
\end{thebibliography}

%\bibliographystyle{IEEEtran}
%\bibliography{reference}
\vspace{-15mm}
\begin{IEEEbiography}[{\includegraphics[width=1in,height=1.2in,clip,keepaspectratio]{./figures/author-ZhouY.pdf}}]{Zhou Yang}
	received the B.S degree in electronic Science and Technology from Zhengzhou University of Science and Technology, China, in 2021. He is currently working toward the M.S degree in information and communication engineering with Chongqing University of Posts and Telecommunications, China. His research interests include rumor detection, spammer detection, deep learning, and its application.
\end{IEEEbiography}
\vspace{-15mm}
\begin{IEEEbiography}[{\includegraphics[width=1in,height=1.2in,clip,keepaspectratio]{./figures/author-YucaiP.pdf}}]{Yucai Pang}
	received the Ph.D. degree from the Harbin Engineering University, Harbin, China. He has been working as an Associate Professor with the School of Communication and Information Engineering, Chongqing University of Posts and Telecommunications since 2017. His research interests include big data prediction, array signal processing, and social rumor detection.
\end{IEEEbiography}
\vspace{-15mm}
\begin{IEEEbiography}[{\includegraphics[width=1in,height=1.2in,clip,keepaspectratio]{./figures/author-HongboY.png}}]{Hongbo Yin}
	is currently working toward his doctoral degree in the School of Information and Communication Engineering, University of Electronic Science and Technology of China (UESTC), Chengdu, China. He received the M.Sc. degree in Chongqing University of Posts and Telecommunications, Chongqing, China, in 2024. His main research interests are social networks, spammer detection, edge computing networks.
\end{IEEEbiography}
\vspace{-15mm}
\begin{IEEEbiography}[{\includegraphics[width=1in,height=1.2in,clip,keepaspectratio]{./figures/author-XiaoYp.pdf}}]{Yunpeng Xiao}
	received the Ph.D. degree in computer science from Beijing University of Posts and Telecommunications, Beijing, China, in 2013. He is a professor and vice dean of the Institute of Electronic Information and Network Engineering, Chongqing University of Posts and Telecommunications, Chongqing, China. He was a visiting scholar of Georgia Institute of Technology from 2018 to 2019. His research interests include social networks, e-commerce and intelligent system.
\end{IEEEbiography}

\end{document}


