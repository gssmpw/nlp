\section{Related Work}
\label{sec:6}

\textbf{The Deep Learning Studies on fMRI Data.}\label{sec:6.1} Medical imaging analysis has seen considerable development over the last several decades. Thanks to the rapid progress, particularly convolutional neural networks (CNNs) \cite{DL2015}, towards medical imaging analysis \cite{WEN2020101694}. Impressive performance comparable to human experts on image classification, object detection, segmentation, registration, and other tasks \cite{LITJENS201760} has occurred. As one of the most popular modalities, most of the previous works are on resting-state fMRI (rs-fMRI) data. \cite{sarraf2016classification} used convolutional neural networks to classify Alzheimer's brain from the normal healthy brain. \cite{bai2017unsupervised} proposed an unsupervised matrix tri-factorization to discover an underlying network that consists of cohesive spatial regions (nodes) and relationships between those regions (edges) for brain imaging data. Such works on rs-fMRI focus on exploring the intrinsically functionally segregation or specialization of brain regions/networks \cite{logotjetis-fMRI} but are limited on identifying spatiotemporal brain patterns that are functionally involved in specific task performance. 

There exist some  work on using GNN for fMRI data \cite{10.1007/978-3-030-36683-4_65,Parisot2018DiseasePU} but it differs from our work in several important ways. Firstly, previous work is for diagnosis not prognosis, it is for resting state data  not task fMRI data and does not use the spectral embedding representation as we do. Perhaps most importantly it is for larger data sets with the later work \cite{Parisot2018DiseasePU} using the ADNI data set \cite{jack2008alzheimer} which contains thousands of instances not under one hundred like our work.

\textbf{The t-fMRI Studies.}\label{sec:6.2} Recently, the t-fMRI analysis is attracting more and more attention for its ability to connect human activities to brain functioning. In the work of \cite{schwartz2019inducing}, the subjects in the study are asked to read a chapter from a novel while the fMRI scans recording their brain activities are conducted. They fine-tuned a pre-trained BERT model to map the natural language to brain fMRIs. \cite{rieck2020uncovering} used time-varying persistence diagrams to represent the human brain activities when the subjects are watching the movie. \cite{10.1371/journal.pcbi.1006633} studies deep image reconstruction by decoding fMRI into the hierarchical features of a pre-trained deep neural network (DNN) for the same input image. The studies in schizophrenia diagnosis utilizing cognitive control tasks suffered from either small sample size or modest classification performance \cite{pmid29622496}. All these t-fMRI settings are different from the AX-CPT setting, for they don't have multiple types of repeated independent clinical trials to result in one combined evaluation. Instead, their tasks are sequence-to-sequence, guided by the inputs such as series of images and natural languages.

\textbf{The AX-CPT t-fMRI Studies.}\label{sec:6.3} The AX-CPT task is a clinical test on reactive and proactive control processes to identify human cognitive control deficits \cite{LESH2013590}. With modest classification accuracy, the first schizophrenia diagnosis study \cite{yoon2012automated} on the fMRI scans conducted while the cohort subjects completed the AX-CPT task suggests an application to discriminate disorganization level among the patients. \cite{doi:10.1176/appi.ajp.2019.18101126} began the studies on the prognosis of treatment of schizophrenia by analyzing the task-fMRI data. The task-fMRI scans of 82 subjects with psychotic disorders were collected and small regions of interest (ROI) were extracted from the scans for the study. The following work of \cite{https://doi.org/10.1002/hbm.25286} compared machine and naive deep learning-based algorithms for the prediction of clinical improvement in psychosis with the same tast-fMRI data. It achieved ROI voxelwise accuracy of $62.4\%$ using a logistic regression model and $72.6\%$ using a multilayer perceptron model which we used as the baseline to our work. These works highly rely on hand-crafted regions of interest segmentation and they are also analyzing the task-fMRI data on the average activation of some selected keyframes in the scans, which may contribute to a great amount of information loss. In our work, we use the same source of data \footnotemark as the two above works (\cite{doi:10.1176/appi.ajp.2019.18101126, https://doi.org/10.1002/hbm.25286, smucny2021comparing}) on prognosis but only the 51 scans in the 1st protocol (EP1) are included. Different from the above works, we don't need any handcrafted ROI segmentation and the model is working voxelwise on the full brain scans. 

\footnotetext{The data is freely available after requests but can not be publicly posted due to privacy concerns.}

\textbf{The Multi-view Learning and Multi-instance Learning.}\label{sec:6.4} Multi-view learning (\cite{yan2021deep}) and multi-instance learning (\cite{carbonneau2018multiple}) are prevalent in practice; for example, the text content of the web page and the links to the web page are two views of the web page; the gene sub-sequences can be seen as the multiple instances in a bag of the chromosome. Our approach fits the general multi-view multi-instance learning definition but still shows explicit differences from the other latest works in this scope. \cite{nguyen2014labeling, li2017multi, wang2020differentiating, cen2019representation} are non-deep matrix factorization methods or graph representation methods which are not applicable in very high dimensional feature space. In contrast to our multi-view multi-instance setting where a bag of instances represents a view of an example, \cite{yuan2018multi, yang2021deep, xing2019multi} study multi-instance learning on a bag of instances where each instance has multiple views. All these subtle differences invalidate their approaches to be used in our setting.