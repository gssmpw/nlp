\newcolumntype{P}[1]{>{\centering\arraybackslash}p{#1}}

\section{Appendix}
\label{sec:appendix}

\subsection{Dataset details}
\label{sec:appendix_dataset_details}

\subsubsection{Multi-task dataset}

\paragraph{Description}
The synthetic dataset used in this study was generated using Claude 3.5 Sonnet, a large language model developed by Anthropic. The dataset consists of sentences crafted to represent various combinations of emotions, topics, and pragmatic intents. The generation process was designed to create a diverse and balanced dataset suitable for studying representation changes in a multi-task setup. Example sentences covering all five category labels for each of the three subtasks can be found in \cref{tab:multitask_example_sentences}.

\begin{table*}[ht]
    \centering
    \begin{tabular}{P{0.45\linewidth} | P{0.13\linewidth} | P{0.15\linewidth} | P{0.15\linewidth}}
      \textbf{Sentence}  & \textbf{Sentiment} & \textbf{Topic} & \textbf{Intent} \\ \hline
      
    This concert has me over the moon - I'm having the time of my life! & Joy & Entertainment & Idiomatic \\ \hline 

    You'll watch as the winds of change blow away the last remnants of your faith in the system! & Sadness & Politics & Metaphorical \\ \hline 

   Tomorrow's blood test results might reveal something I'm not prepared to handle. & Fear & Health & Literal \\ \hline  

   My favorite team's strategy of constantly fumbling the ball was clearly the path to victory. & Anger & Sports & Sarcastic \\ \hline  

   I nearly fell out of my chair when my ancient printer suddenly sprang to life and started spewing out pages of binary code! & Surprise & Technology & Humorous \\ \hline
   
    \end{tabular}
    \caption{Example sentences from the synthetic multi-task dataset }
    \label{tab:multitask_example_sentences}
\end{table*}

\begin{figure*}
    \centering
    
\begin{tcolorbox}[  enhanced,
  interior style={
    top color=gray!5,
    bottom color=gray!5,
  },
  frame style={
    color=gray!35,
  },
  left=2pt,    % left padding
  right=2pt,   % right padding
  top=2pt,     % top padding
  bottom=2pt,  % bottom padding
]

You are a helpful assistant tasked with generating a dataset of sentences. Generate 2 sentences for each of the following categories of emotion:

1. Joy

2. Sadness

3. Anger

4. Fear

5. Surprise

\medskip

Please make sure all sentences are related to the topic ``\textbf{technology}'' and have \textbf{humorous} intent (\textbf{Each sentence is intended to be funny or amusing, often through clever use of language, unexpected connections, or playful exaggeration.}).

\medskip
There are a few requirements for the sentences:

Use \textbf{first-person} perspective.

Use \textbf{future} tense.


\medskip
Additionally, include at least one sentence that:

- \textbf{Sounds like a tweet}

\medskip

Very important instructions:

1. Convey the emotion through the situation, word choice, and tone. Do not directly state the emotion or use immediate synonyms.

2. Imply the topic through context and content, but do not explicitly mention the topic name.

3. Express the intent naturally without explicitly stating the type of intent being used.

\medskip

Format your response as follows:

Joy:

1. [Sentence 1]

2. [Sentence 2]

\smallskip
Sadness:

1. [Sentence 1]

2. [Sentence 2]

\smallskip
Anger:

1. [Sentence 1]

2. [Sentence 2]

\smallskip
Fear:

1. [Sentence 1]

2. [Sentence 2]

\smallskip

Surprise:

1. [Sentence 1]

2. [Sentence 2]

\medskip
Ensure each sentence is on a new line and numbered within its category.

Do not include any additional text or explanations outside of this format.

Very important: Remember to vary the syntax and structure of the sentences to make the dataset diverse and interesting! Do not use the same structure for all sentences.

\end{tcolorbox}
    \caption{Example prompt configuration used in generating the synthetic dataset (emotion-focused type). Text highlighted in \textbf{bold} represents parts of the prompt that were varied on each iteration to increase diversity of resulting sentences.}
    \label{fig:example-generation-prompt}
\end{figure*}




\paragraph{Generation procedure}

The dataset was generated through an iterative process, by cycling through three possible generation types:

\begin{itemize}
\item Emotion-focused: The model's goal was to generate 10 sentences (2 for each emotion), given a specific topic and intent.
\item Topic-focused: The model's goal was to generate 10 sentences (2 for each topic), given a specific emotion and intent.
\item Intent-focused: The model's goal was to generate 10 sentences (2 for each intent), given a specific emotion and topic.
\end{itemize}

Example of a full prompt an for emotion-focused iteration can be found in \cref{fig:example-generation-prompt}. Prompts for other two types were phrased analogously.

To ensure diversity of resulting sentences, each prompt included a specific constraint for the sentence to be of a certain perspective (first, second or third person) and tense (present, past, future or mixed), chosen at random.

Additionally, the model was instructed to make at least one generated sentence follow a special requirement, that was sampled randomly from a pool of 18 possible requirements (\cref{tab:special-requirements}) during prompt construction. 

\begin{table*}[ht]
    \centering
    \begin{tabular}{p{0.75\linewidth}}
        Sounds like a tweet \\
        Describes a hypothetical scenario \\
        Uses simple vocabulary as if spoken by a child \\
        Has a rhythmic or lyrical quality \\
        Sounds like a memorable quote \\
        Includes a question \\
        Includes a command or instruction \\
        Incorporates a well-known saying or proverb \\
        Structured like a headline \\
        Includes a number or statistic \\
        Imitates casual online comment style \\
        Uses formal language \\
        Starts with a gerund (-ing word) \\
        Includes a rhetorical question \\
        Uses the passive voice \\
        Includes a list or enumeration \\
        Employs repetition for emphasis \\
        Starts with a conditional (If...)
    \end{tabular}
    \caption{Possible special requirements during dataset generation}
    \label{tab:special-requirements}
\end{table*}



The generated sentences were parsed and stored with their corresponding labels.


\paragraph{Post processing}

After generation, the dataset underwent several post-processing steps:

\begin{itemize}
    \item Duplicate sentences were removed to ensure uniqueness
    \item Category labels were capitalized and ordered consistently.
    \item Letter codes and shuffled labels were assigned to each sentence for alternative labeling schemes
    \item Dataset was subsampled to 1000 sentences (500 for train and 500 for test splits) in a way to ensure uniform coverage of each of three subtasks (in each split: 100 sentences per category)
\end{itemize}


To introduce further variations, the following transformations were applied to both train and test sets:
\begin{itemize}
\item 10\% of sentences were converted to lowercase
\item 10\% of sentences were converted to uppercase
\end{itemize}
These transformations were applied randomly and independently to each set.




\subsubsection{Open datasets}

To ensure our findings were not solely a result of using a synthetic dataset generated by another language model, we replicated our single-task experiments using two open datasets, often used for text classification: \textbf{AG News} \cite{Zhang2015CharacterlevelCN}and \textbf{TREC (coarse)} \cite{li-roth-2002-learning, hovy-etal-2001-toward}. For the TREC dataset we removed the ``Abbreviation'' category, which had an insufficient number of samples for manifold analysis. Additionally, we created a balanced test partition with uniform representation across all categories. Resulting dataset sizes can be found in \cref{tab:open_datasets_samples}.

\begin{table*}[ht]
    \centering
    \begin{tabular}{c|c|c}
        \textbf{Dataset} & \textbf{Samples per category} & \textbf{Category labels} \\  \hline
        TREC coarse & 65 & Description, Entity, Human, Location, Numeric \\  \hline
        
        AG news & 63 & Business, World, Sports, Sci/Tech \\  \hline
    \end{tabular}
    \caption{Parameters of open dataset subsampling sizes used in experiments}
    \label{tab:open_datasets_samples}
\end{table*}