\section{Impact Statement}

This work theoretically analyzes adaptation as minimum entropy problem in terms of Lebesgue integral. 
Minimum entropy problem is one of the most widely used optimization problems across diverse fields 
like statistics, information theory and machine learning). 
This analysis has great potentials to improve data analyses, network communication and machine 
due to the general use of minimum entropy problem 
However, it also brings about the ethical considerations for the privacy about neural network 
since we discover the internal mechanism of minimum entropy problem and connects it to deep learning context.

In addition, our proposed method, URECA leverages the relationships between disentangled representations 
with the clusters estimated from the clustering process of URECA which simulates clustering disentangled 
representations based on simulation trick with transported logits.
This clustering algorithm introduces novel insights for efficient clustering 
since it simulates clustering with numeric calculations.
Although the efficiency makes rapid adaptation to distribution shift, it also accelerates the development
of attack from malicious users.

Although our work pushes the boundaries of machine learning , the distinctions also introduce new challenges
,which people should contemplate on for the proper utilization of advancements.  

