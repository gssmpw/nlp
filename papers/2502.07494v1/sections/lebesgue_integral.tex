\section{Lebesgue Integral}
Lebesgue integral is extension of riemannian integral(\cite{KwonY12}). 
Intuitively, lebesgue integral is integral across the partitions of image 
for the given function, while riemannian integral is across the partitions of domain. 
Lebesgue integral for the non-negative measurable function $f$ is defined as follows,
under the condition $0 \le s(x) \le f(x)$.

\begin{equation}
\label{APP_B_def:lebesgue_integral}
\int{f(x)  d\mu(x)} = \sup\{\int  s(x) d\mu(x)\}
\end{equation}
In this formula, $s$ is measurable simple function, which is linear combination of characteristic function $\chi_{E_k}$. 
$a_k$ is the element in range  and $E_k$ is pre-image of $s$.
Measurable function is function of which all pre-images are in the domain of that function.  
Definition of simple function is as follows.
\begin{equation}
\label{def:simple_function}
s(x) = \sum_k a_k \chi_{E_k}(x) 
\end{equation}
Characteristic function $\chi_{E_k}$ can be called as indicator function, which is defined as follows. 
\begin{align}
\label{def:charac_func}
\chi_{E_k}(x) = 
     \begin{cases}
       1 & (x \in E_k)\\
       0 & (x \notin E_k)\\
     \end{cases}
\end{align}