\begin{abstract} 
Adaptation is to make model learn the patterns shifted from the training distribution. 
In general, this adaptation is formulated as the minimum entropy problem. 
However, the minimum entropy problem has inherent limitation---shifted initialization cascade phenomenon. 
We extend the relationship between the minimum entropy problem and the minimum set cover problem via Lebesgue integral. 
This extension reveals that internal mechanism of the minimum entropy problem ignores the relationship between disentangled representations,
which leads to shifted initialization cascade.
From the analysis, we introduce a new clustering algorithm, Union-find based Recursive Clustering Algorithm~(URECA).
URECA is an efficient clustering algorithm for the leverage of the relationships between disentangled representations.
The update rule of URECA depends on Thresholdly-Updatable Stationary Assumption to dynamics as a released version of Stationary Assumption.
This assumption helps URECA to transport disentangled representations with no errors based on the relationships between disentangled representations.
URECA also utilize simulation trick to efficiently cluster disentangled representations. 
The wide range of evaluations show that URECA achieves consistent performance gains for the few-shot adaptation 
to diverse types of shifts along with advancement to State-of-The-Art performance in CoSQA 
in the scenario of query shift. 
\end{abstract}