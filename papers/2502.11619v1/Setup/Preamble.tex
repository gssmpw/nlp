% %\usepackage{microtype}      % better looking text
% %\usepackage[utf8]{inputenc}
% \usepackage{mathtools}
% \usepackage[warnings-off={mathtools-colon,mathtools-overbracket}]{unicode-math}
% \usepackage[T1]{fontenc}
% \usepackage{kpfonts}
% \usepackage{relsize}


%%%%%%%%%%%%%% FROM EXAMPLE %%%%%%%%%%%%%%%%%%%%%%

\usepackage{epsfig}
\usepackage{subcaption}
\usepackage{calc}
\usepackage{amssymb}
\usepackage{amstext}
\usepackage{amsmath}
\usepackage{amsthm}
\usepackage{multicol}
\usepackage{pslatex}
\usepackage{apalike}
\usepackage{algorithm2e}
\usepackage[bottom]{footmisc}

%%%%%%%%%%%%%%%%% OUR  PACKAGES %%%%%%%%%%%%%

\usepackage{algpseudocode}
\usepackage{multirow}
\usepackage{graphicx}
\usepackage{icomma}
\usepackage{multirow}
\usepackage{pgf}
\usepackage{pgffor}
\usepackage{pgfplots}
\usepgfplotslibrary{groupplots,dateplot}

\newcounter{piccount} % define a counter to keep track of the number of pictures

\usepackage{changepage}
\usepackage{threeparttable}
\usepackage[]{geometry}     % Package for changing page margins (before fancyhdr) 
% \usepackage{fancyhdr}       % Package to change header and footer
% \usepackage{parskip}        % Package to tweak paragraph skipping (instead of indents a small skip is added after every paragraph)
% \usepackage{titlesec}
\usepackage{tikz}           % Package for drawing
\usetikzlibrary{shapes.geometric, arrows.meta, shadows, positioning, patterns}
\colorlet{myred}{red!20!white}
\colorlet{myblue}{blue!20!white}
\colorlet{mygreen}{green!40!white}
\colorlet{mypurple}{blue!50!red!50!white}
\colorlet{mydarkred}{myred!40!black}
\colorlet{mydarkblue}{myblue!40!black}
\colorlet{mydarkgreen}{mygreen!40!black}
\tikzset{no shadows/.style={general shadow/.style=}}

\pgfplotsset{compat=newest}
\usepackage{etoolbox}
\usepackage{listofitems}
\usepackage{pgfplots}       % Package for creating graphs and charts
\usepackage{xcolor}         % Package for defining DTU colours to be used
% \usepackage{amsmath}        % For aligning equations among other
\DeclareMathOperator*{\argmax}{argmax}

% \usepackage{siunitx}        % SI units
% \usepackage{listings}       % Package for inserting code, (before cleveref)
\PassOptionsToPackage{hyphens}{url} % Ability to line break urls at hyphens
\usepackage{hyperref}       % Package for cross referencing (also loads url package)
\usepackage{orcidlink}
\usepackage{cleveref}       % improved cross referencing
\usepackage{textcomp}       % \textdegree = °C and other useful symbols
\usepackage[english]{babel} % localisation 
\usepackage{caption}        % better captions
% \usepackage{subfig}         
\usepackage{csquotes}       % For biblatex with babel
% \usepackage[backend=biber,style=numeric,sorting=none]{biblatex} % Package for bibliography (citing)
% \bibliography{bibliography.bib}
\usepackage{tabularx}       % for ability to adjust column spacing in tabular better
\usepackage{booktabs}       % for better tables
\usepackage{float}          % floating figures in correct places
\usepackage{calc}           % Adds ability for latex to calculate (3pt+2pt) 

\usepackage{leftindex}      % For math symbols

\usepackage{pdflscape}
\usepackage{pdfpages}

\hyphenation{con-sti-tu-tion-al}
\hyphenation{need-ed}
\hyphenation{dif-fusion}
\hyphenation{pop-ul-ar-i-ty}
\hyphenation{mil-li-ons}


\usepackage[toc,section=section, acronym]{glossaries} % Ordliste
% \setglossarystyle{listgroup} % listhypergroup - den her er også sej
\setacronymstyle{long-short}

\newacronym{AR}{AR}{Augmented Reality}
\newacronym{AI}{AI}{Artificial Intelligence}

\newacronym{BS}{BS}{Base Station}

\newacronym{CDF}{CDF}{cumulative distribution function}
\newacronym{CSI}{CSI}{Channel State Information}

\newacronym{D2D}{D2D}{Device-to-Device}
\newacronym{DC}{DC}{dual connectivity}
\newacronym{DSS}{DSS}{Dynamic Spectrum Sharing}

\newacronym{eNB}{eNB}{Evolved Node-B}
\newacronym{EPC}{EPC}{Evolved Packet Core}
\newacronym{eMTC}{eMTC}{enhanced MTC}

\newacronym{FDD}{FDD}{Frequency Domain Duplex}
\newacronym{FDM}{FDM}{Frequency Domain Multiplexing}

\newacronym{gNB}{gNB}{Next Generation Node-B}
\newacronym{Gbps}{Gbps}{gigabits per second}

\newacronym{HAP}{HAP}{High-Altitude Platform}

\newacronym{IEEE}{IEEE}{Institute of Electrical and Electronics Engineers}
\newacronym{IoE}{IoE}{Internet of Everything}
\newacronym{IoT}{IoT}{Internet of Things}
\newacronym{IIoT}{IIoT}{Industrial Internet of Things}
\newacronym{IoMT}{IoMT}{Internet of Medical Things}
\newacronym{IRS}{IRS}{Intelligent Reflecting Surfaces}
\newacronym{ITU}{ITU}{International Telecommunication Union}

\newacronym{KPI}{KPIs}{Key Performance Indicators}

\newacronym{LEO}{LEO}{Low Earth Orbit}
\newacronym{LTE}{LTE}{Long Term Evolution}
\newacronym{LTEAPRO}{LTE-A~Pro}{LTE Advanced Pro}

%\newacronym{MC}{MC}{molecular communications}
\newacronym{MCS}{MCS}{Modulation and Coding Scheme}
\newacronym{MM}{mmWave}{millimeter wave}
\newacronym{MIMO}{MIMO}{Multiple-Input Multiple-Output}
\newacronym{mMIMO}{mMIMO}{Massive Multiple-Input Multiple-Output}
\newacronym{mMTC}{mMTC}{massive machine-type communication}
\newacronym{MTC}{MTC}{Machine-Type Communication}
\newacronym{ML}{ML}{Machine Learning}
\newacronym{MU-MIMO}{MU-MIMO}{multi-user MIMO}
\newacronym{multi-TRP}{multi-TRP}{multiple transmission/reception point}

\newacronym{NR}{NR}{New Radio}
\newacronym{NTN}{NTN}{Non-Terrestrial Network}

\newacronym{PLS}{PLS}{Physical Layer Security}

\newacronym{QC}{QC}{Quantum Computing}
\newacronym{QoS}{QoS}{Quality of Service}

\newacronym{RAN}{RAN}{Radio Access Network}
\newacronym{RANs}{RANs}{Radio Access Networks}
\newacronym{RAT}{RAT}{Radio Access Technology}
\newacronym{RE}{RE}{Resource Element}
\newacronym{RHS}{RHS}{Reconfigurable Holographic Surface}
\newacronym{RIS}{RIS}{Reconfigurable Intelligent Surface}
\newacronym{RL}{RL}{Reinforcement Learning}
\newacronym{RU}{RU}{Radio Unit}
\newacronym{RT}{RT}{real-time}

\newacronym{TCI}{TCI}{Transmission Configuration Indicator}
\newacronym{TDD}{TDD}{Time Division Duplex}
\newacronym{TDM}{TDM}{Time Division Multiplexing}
\newacronym{TR}{TR}{Technical Report}
\newacronym{THz}{THz}{Terahertz}

\newacronym{UM-MIMO}{UM-MIMO}{ultra-massive MIMO}
\newacronym{URLLC}{URLLC}{ultra-reliable low-latency communication}
\newacronym{UAV}{UAVs}{Unmanned Aerial Vehicles}
\newacronym{UE}{UE}{User Equipment}
\newacronym{UL}{UL}{uplink}

\newacronym{V2V}{V2V}{Vehicle-to-Vehicle}
\newacronym{V2X}{V2X}{Vehicle-to-Everything}
\newacronym{VR}{VR}{Vitual Reality}

\newacronym{WLAN}{WLAN}{Wireless Local Area Network}
\newacronym{XR}{XR}{Extended Reality}
\newacronym{3GPP}{3GPP}{3rd Generation Partnership Project}
\newacronym{3D}{3D}{three-dimensional}

\newacronym{6GC}{6GC}{6G Core}
\newacronym{SBA}{SBA}{service-based architecture}
\newacronym{NF}{NF}{network function}
\newacronym{LLM}{LLM}{Large Language Model}
\newacronym{SBI}{SBI}{service-based interface}
\newacronym{RAG}{RAG}{Retrieval-Augmented Generation}
\newacronym{PCAP}{PCAP}{packet capture}
\newacronym{TS}{TS}{technical specification}
\newacronym{AF}{AF}{application function}
\newacronym{OAM}{OAM}{Operations, Administration, and Maintenance}
\newacronym{NWDAF}{NWDAF}{Network
Data Analytics Function}
\newacronym{AMF}{AMF}{Access and Mobility Management Function}
\newacronym{SMF}{SMF}{Session Management Function}
\newacronym{UPF}{UPF}{User Plane Function}
\newacronym{GUI}{GUI}{graphical user interface}
\newacronym{CLI}{CLI}{command-line interface}
\newacronym{NRF}{NRF}{NF Repository Function}
\newacronym{LNF}{LNF}{LLM-based Network Function}
\newacronym{RESTful}{RESTful}{REpresentational State Transfer}
\newacronym{NEF}{NEF}{Network Exposure Function}
\newacronym{LAF}{LAF}{LLM-based Application Function}
\newacronym{DN}{DN}{data network}
\newacronym{SLA}{SLA}{Service Level Agreement}
\newacronym{MNO}{MNO}{Mobile Network Operator}
\newacronym{NOC}{NOC}{Network Operations Center}
\newacronym{PFCP}{PFCP}{Packet Forwarding Control Protocol}
\newacronym{NGAP}{NGAP}{ Next Generation Application Protocol}
\newacronym{IVR}{IVR}{interactive voice response}
\newacronym{FAQ}{FAQ}{frequently asked question}
\newacronym{eMBB}{eMBB}{enhanced mobile broadband}
\newacronym{GBR}{GBR}{Guaranteed Bit Rate}
\newacronym{ARP}{ARP}{Allocation and Retention Priority}
\newacronym{PCF}{PCF}{Policy Control Function}
\newacronym{PDU}{PDU}{protocol data unit}
\newacronym{BERT}{BERT}{Bidirectional Encoder
Representations from Transformers}
\newacronym{SOTA}{SOTA}{state-of-the-art}
\newacronym{SON}{SON}{self-organizing network}
\newacronym{GPT}{GPT}{Generative Pre-trained Transformer}
\newacronym{LLaMA}{LLaMA}{Large Language Model Meta AI}
\newacronym{GPU}{GPU}{Graphics Processing Unit}
%\newacronym{operations, administration and maintenance}{OAM}{OAM}
\newacronym{CoT}{CoT}{Chain-of-Thought}
\newacronym{ToT}{ToT}{Tree of Thoughts}
\newacronym{Near-RT}{Near-RT}{near-real-time}
\newacronym{Non-RT}{Non-RT}{none-real-time}
\newacronym{IETF}{IETF}{Internet Engineering Task Force}
\newacronym{AISV}{AISV}{AI independent software vendor}
\newacronym{SbD}{SbD}{security-by-design}

\newacronym{IC}{IC}{Immersive Communication}
\newacronym{HRLLC}{HRLLC}{Hyper Reliable and Low Latency Communication}
\newacronym{MC}{MC}{Massive Communication}
