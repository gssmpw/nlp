\section{Related Work}
\subsection{Graph Collaborative Filtering}

GNNs, particularly spectral GNNs, 
exhibit superiority in collaborative filtering due to the bipartite graph structure inherent in the interaction data.
Since LightGCN____ simplifies NGCF____ by eliminating all trainable components except for the embeddings, 
more and more researchers have begun to recognize the significance of the spectrum____.
____ unified a series of classic works from a spectral perspective, 
highlighting the low-frequency enhancement function of LightGCN.
Furthermore, the highest-frequency signals are mined in BSPM____ and JGCF____ through graph diffusion and Jacobi polynomial bases, respectively.
However, these spectral GNNs are constrained to the bipartite interaction graph that adheres to the full-spectrum assumption. 
For the augmented graph examined in this paper, the underlying graph filters need to be modified to achieve optimal performance.

It is worth noting that spectral GNNs in the traditional graph community have evolved in a markedly different direction.
While early variants, 
such as APPNP____, 
heuristically design graph filters, 
subsequent efforts focus on learning the coefficients automatically. 
Notable examples include ChebyNet____, GPR-GNN____, JacobiConv____, FavardGNN____.
However, spectral GNNs with learnable coefficients encounter training difficulties and instability____.
Consequently, the extension of graph collaborative filtering to the augmented graph requires further consideration.


\subsection{Multimodal Recommendation}
Modern media, enriched with diverse modalities (\eg, textual descriptions, visual thumbnails, etc.), 
offer highly valuable information for real-world recommendation tasks.
Significant efforts have been made to integrate multimodal features into the traditional collaborative filtering framework. 
For instance, VBPR____ combines visual features extracted by a pretrained CNN for prediction. 
% VECF____ utilizes VGG____ and an attention mechanism to provide visual explanations for each recommendation. 
Moreover, MMSSL____ has emerged as one of the state-of-the-art methods through the application of self-supervised learning.

GNNs have also become indispensable components for integrating multimodal information.
MMGCN____ directly utilize raw modal features by aggregating neighbor representations for each modality.
However, these raw features often contain noise that is unrelated to recommendation____.
LATTICE____ constructs the item-item $\kappa$NN similarity graph and dynamically refines it over epochs, 
as this approach ensures that only the most relevant relationships are preserved.
The subsequent work, FREEDOM____, 
simplifies the $\kappa$NN graph and introduces a standard baseline comprising two distinct convolutional branches.
However, their generalizability is constrained, 
requiring specific modifications to adapt to tasks with diverse characteristics, statistical properties, and signal-to-noise ratios.
Recently, GUME____ attempted to incorporate multimodal information by utilizing an augmented graph, 
similar to the one we defined in Section~\ref{sec_augmented_graph}. 
Notably, there is a growing trend toward leveraging side information in a structured and integrated manner. 
However, GUME empirically adopts an equally weighted convolution framework, 
which fails to align with the spectrum of the augmented graph. 
In this work, our analysis of the oracle spectrum importance reveals this discrepancy, 
and we propose a spectrum shift correction approach to address this issue.



\subsection{Social Recommendation}

In addition to multimodal features, social relationships serve as valuable side information frequently utilized in real-world recommendation scenarios. 
The primary task addressed in this work remains the prediction of user-item behaviors based on known user-item interaction histories and user-user relationships. 
Although non-graph-based methods such as SoRec____, Social Regularization____, and TrustMF____ exist, 
GNNs dominate this area due to the inherently graph-structured nature of social networks.
GraphRec____ first applies graph convolution to social recommendation.
Several subsequent approaches____________ simplifies convolution framework and explore better convolutional ordering in terms of user-user and user-item graphs.
For instance, DiffNet++____ models a social diffusion process on the heterogeneous social network
and an interest diffusion process on the bipartite interaction graph, respectively.
% SMIN____ utilizes both social connections and multiple meta relationships.
Also, some methods intend to extract deeper information from graph topologies.
% MHCN____ classifies social relationships as several motifs and constructs hypergraph for convolution.
MHCN____ constructs hypergraph based on the motifs derived from the social and purchase relationships.
In view of the significant noise in social relationships, 
SHaRe____ refines the social graph by removing edges with low embedding similarity and adding edges with high similarity. 
As a representative self-supervised learning-based approach, 
DSL____ constructs a social view and an interaction view based on the respective graphs. 
These methods empirically extend the LightGCN convolution to social recommendation without accounting for the spectrum shift phenomenon.
In contrast to the progressively more complex designs, 
we propose a universal solution by equipping existing spectral GNNs with a correction mechanism to match the shifted spectrum. 
This approach is significantly simpler and more effective.