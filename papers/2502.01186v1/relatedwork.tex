\section{Related Works}
\label{sec:rw}
%Numerous tools in archaeology (litterature), but not adapted to coins.
%Numerous tools in numismatics (litterature, but not usable on ancient coins.
%%%% TODO Deep Learning frameworks: but they need large DS -> pretrained one in evaluation.

In the domain of coin die link detection, some recent works are very promising \cite{taylor2020computer, Base, Cohesion}.
However, the datasets used in these works are not publicly available, and the source codes for computing coin dissimilarities have not been released online. The first work \cite{taylor2020computer} uses \textit{Oriented FAST and rotated BRIEF} (ORB \cite{rublee2011orb}) to extract points of interest, also called \textit{keypoints}, from coin pictures, then brute force matching to match points between two coins, and finally averages the descriptor distances of the best matches to obtain a dissimilarity measure. The method used in \cite{Base} extracts keypoints using Gaussian processes \cite{Gaussian}, associates descriptors with VLFeat \cite{vedaldi2010vlfeat}, matches keypoints using a bounded distortion feature matching method \cite{lipman2014feature}, and finally computes a dissimilarity measure based on the Procrustes distance between these point sequences, and the number of matches. Finally, the procedure described in \cite{Cohesion} uses SIFT to obtain keypoints and descriptors, matches keypoints using the ratio test \cite{lowe2004distinctive} and bounded distortion feature matching, and finally combines the Procrustes distance, the number of matches, and the descriptors and average local gradients to construct a dissimilarity measure.

These methods computes dissimilarity measures between pictures, from local features, i.e. the keypoints and their associated descriptors. In this paper, we propose to focus on a global measure of the similarity between images, based on SSIM, in order to benefit from all the information contained in the images when comparing them.