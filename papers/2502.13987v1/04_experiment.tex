\section{Experiments}
\label{chap:experiments}

\begin{table*}[t]
  \centering
  \caption{
  Influence of the number of self-reference images. The \textbf{boldface} indicates the best scores while the \underline{underline} the second best scores.
  }
  \label{tab:refnum}
  \resizebox{\linewidth}{!}{
    \begin{tabular}{llccccccccccc}
      \toprule
      \multirow{2}{*}{Metric} & \multirow{2}{*}{Method} & \multicolumn{11}{c}{Target age} \\
      \cmidrule(lr){3-13}
      & & 1 & 5 & 8 & 12 & 17 & 25 & 35 & 45 & 60 & 80 & ALL \\
      \midrule
      \multirow{4}{*}{AGE~$\downarrow$} & Ours w/ 0 ref. & 8.92 & \underline{16.8} & 14.2 & \underline{9.80} & \textbf{11.0} & \textbf{11.4} & \underline{11.8} & 13.1 & \underline{7.28} & \textbf{6.52} & \textbf{11.1} \\
      & Ours w/ 1 ref. & \textbf{8.41} & \textbf{16.7} & \textbf{14.0} & 10.1 & 12.4 & 13.0 & 11.9 & \underline{12.2} & \textbf{6.95} & \underline{6.56} & \underline{11.2} \\
      & Ours w/ 3 ref. & 9.22 & 17.2 & \underline{14.1} & \textbf{9.54} & 12.3 & 12.4 & \textbf{11.7} & \textbf{11.9} & 7.42 & 7.07 & 11.3 \\
      & Ours w/ 5 ref. & \underline{8.89} & 17.8 & 14.4 & 10.3 & \underline{12.2} & \underline{12.3} & 12.2 & 12.5 & 7.48 & 7.31 & 11.5 \\
      \midrule
      \multirow{4}{*}{ID~$\downarrow$} & Ours w/ 0 ref. & \textbf{0.155} & \underline{0.124} & \underline{0.117} & \underline{0.106} & 0.0677 & 0.0656 & 0.0620 & 0.0680 & 0.0748 & 0.141 & 0.0981 \\
      & Ours w/ 1 ref. & 0.168 & 0.135 & 0.123 & 0.117 & 0.0698 & 0.0663 & 0.0623 & 0.0718 & 0.0757 & 0.134 & 0.102 \\
      & Ours w/ 3 ref. & \textbf{0.155} & \underline{0.124} & \textbf{0.115} & \textbf{0.105} & \textbf{0.0630} & \textbf{0.0545} & \textbf{0.0517} & \textbf{0.0613} & \textbf{0.0649} & \textbf{0.128} & \textbf{0.0923} \\
      & Ours w/ 5 ref. & \underline{0.156} & \textbf{0.122} & \textbf{0.115} & \underline{0.106} & \underline{0.0653} & \underline{0.0581} & \underline{0.0568} & \underline{0.0677} & \underline{0.0703} & \underline{0.129} & \underline{0.0946} \\
      \bottomrule
    \end{tabular}
  }
\end{table*}

\begin{table*}[t]
  \centering
  \caption{
    Quantitative comparison between our method and the existing methods. 
    }
  \label{tab:final}
  \resizebox{\linewidth}{!}{
  \begin{tabular}{llccccccccccc}
    \toprule
    \multirow{2}{*}{Metric} & \multirow{2}{*}{Method} & \multicolumn{11}{c}{Target age} \\
    \cmidrule(lr){3-13}
    & & 1 & 5 & 8 & 12 & 17 & 25 & 35 & 45 & 60 & 80 & ALL \\
    \midrule
    \multirow{4}{*}{AGE~$\downarrow$} & SAM~\cite{alaluf2021only} & 17.8 & \textbf{15.9} & \textbf{13.9} & \underline{11.8} & \textbf{8.74} & \textbf{7.79} & \textbf{10.8} & \underline{11.7} & 8.77 & \textbf{3.55} & \textbf{11.1} \\
    & CUSP~\cite{gomez2022custom} & 13.0 & 20.4 & 24.4 & 20.0 & 12.4 & \underline{10.6} & 12.2 & \textbf{10.2} & \textbf{6.37} & 7.11 & 13.7 \\
    & FADING~\cite{chen2023face} & \underline{11.7} & 21.7 & 20.3 & 15.4 & 16.3 & 15.5 & 15.9 & 14.7 & 8.86 & 7.72 & 14.8 \\
    & Ours & \textbf{9.22} & \underline{17.2} & \underline{14.1} & \textbf{9.54} & \underline{12.3} & 12.4 & \underline{11.7} & 11.9 & \underline{7.42} & \underline{7.07} & \underline{11.3} \\
    \midrule
    \multirow{4}{*}{ID~$\downarrow$} & SAM~\cite{alaluf2021only} & 0.270 & 0.268 & 0.266 & 0.264 & 0.261 & 0.261 & 0.264 & 0.266 & 0.271 & 0.271 & 0.266 \\
    & CUSP~\cite{gomez2022custom} & 0.208 & 0.204 & 0.208 & 0.201 & 0.195 & 0.145 & 0.140 & 0.146 & 0.167 & 0.283 & 0.190 \\
    & FADING~\cite{chen2023face} & \textbf{0.151} & \textbf{0.103} & \textbf{0.0987} & \textbf{0.0812} & \textbf{0.0603} & \underline{0.0606} & \underline{0.0638} & \underline{0.0714} & \underline{0.0885} & \textbf{0.121} & \textbf{0.0900} \\
    & Ours & \underline{0.155} & \underline{0.124} & \underline{0.115} & \underline{0.105} & \underline{0.0630} & \textbf{0.0545} & \textbf{0.0517} & \textbf{0.0613} & \textbf{0.0649} & \underline{0.128} & \underline{0.0923} \\
    \bottomrule
  \end{tabular}
  }
\end{table*}

\begin{figure*}[t]
  \centering
  \includegraphics[width=1.\linewidth]{fig/qual1_ia-eps-converted-to.pdf}
  \caption{
  Qualitative comparison between our method and the existing methods~\cite{alaluf2021only,gomez2022custom,chen2023face}. The upper right numbers on the input and self-reference images show the ages estimated by the age estimator. 
  }
  \label{fig:qualitative}
\end{figure*}

\begin{table*}[t]
  \centering
  \caption{
  Quantitative comparison of our method with and without our refined regularization set. 
  }
  \label{tab:dexreg}
  \resizebox{\textwidth}{!}{
  \begin{tabular}{llccccccccccc}
    \toprule
    \multirow{2}{*}{Metric} & \multirow{2}{*}{Method} & \multicolumn{11}{c}{Target age} \\
    \cmidrule(lr){3-13}
    & & 1 & 5 & 8 & 12 & 17 & 25 & 35 & 45 & 60 & 80 & ALL \\
    \midrule
    \multirow{2}{*}{AGE~$\downarrow$} & Ours w/o refined reg. & 9.51 & 17.3 & 14.4 & 9.91 & \textbf{12.0} & \textbf{12.1} & \textbf{11.7} & \textbf{11.7} & 7.52 & 7.77 & 11.4 \\
    & Ours & \textbf{9.22} & \textbf{17.2} & \textbf{14.1} & \textbf{9.54} & 12.3 & 12.4 & \textbf{11.7} & 11.9 & \textbf{7.42} & \textbf{7.07} & \textbf{11.3} \\
    \midrule
    \multirow{2}{*}{ID~$\downarrow$} & Ours w/o refined reg. & 0.163 & 0.131 & 0.123 & 0.113 & 0.0731 & 0.0649 & 0.0626 & 0.0714 & 0.0752 & 0.129 & 0.101 \\
    & Ours & \textbf{0.155} & \textbf{0.124} & \textbf{0.115} & \textbf{0.105} & \textbf{0.0630} & \textbf{0.0545} & \textbf{0.0517} & \textbf{0.0613} & \textbf{0.0649} & \textbf{0.128} & \textbf{0.0923} \\
    \bottomrule
  \end{tabular}
  }
\end{table*}

\paragraph{Experimental settings. }
Our method was implemented using Python, PyTorch, and Diffusers.
For both training and inference, we used an NVIDIA RTX A6000 GPU.
The input image size was $224\times224$ pixels.
Our method utilized Stable Diffusion v1.5~\cite{rombach2022high} as the pretrained diffusion model. The batch size was set to 2, and training was performed for 800 iterations.
We used AdamW~\cite{loshchilov2017decoupled} as the optimizer with a learning rate of $1.0\times10^{-6}$.
The LoRA rank was $r=16$. 
Training took approximately 35 minutes, while inference required around 60 seconds for Null-text Inversion and 15 seconds for Prompt-to-Prompt.


\paragraph{Datasets.}
For a regularization set, we used 594 out of 612 images from CelebA-Dialog~\cite{jiang2021talk}, which were properly aligned for the age estimator DEX~\cite{rothe2015dex}.
For self-reference images, we constructed a dataset based on AgeDB~\cite{moschoglou2017agedb}.
AgeDB is a dataset consisting of 16,488 images of 568 celebrities collected from the Internet, with an average of 29 images per individual.
It is labeled by DEX with integer age labels ranging from 0 to 101.
However, AgeDB images have a relatively low resolution of $112\times112$ pixels and contain low-quality images, including grayscale ones.
To address this, we colorized the grayscale images using an existing method~\cite{kang2023ddcolor} and applied super-resolution~\cite{lin2024diffbir} to all images.
We used this dataset for both training and inference in our experiment. 
Specifically, we selected 20 individuals from the dataset (10 males and 10 females) and trained the model using a few self-reference images per individual.
For inference, we randomly selected five images of the corresponding individual for each model. 

\paragraph{Evaluation metrics.}
We used AGE and ID as evaluation metrics~\cite{alaluf2021only}.
AGE represents the accuracy of age editing and is calculated as the mean absolute difference between the estimated age of the output image, obtained from Face++~\cite{facepp}, and the target age. 
ID measures identity preservation and is computed as the average cosine similarity between the feature vectors of the input and output images using ArcFace~\cite{deng2019arcface}.

\subsection{
Influence of the Number of Self-reference Images
}
\label{sec:eval_refnum}


First, we investigated the influence of the number of self-reference images.
As shown in Table~\ref{tab:refnum}, our method demonstrates improvements in ID using a few self-reference images. 
Notably, using three self-reference images achieves the best ID score in total (ALL) while suppressing deterioration in AGE, indicating a more balanced and stable adaptation. We set the number of self-reference images $M=3$ for all subsequent experiments. 


\subsection{Comparisons with Existing Methods}
\label{sec:comparison}

We compared our method with the GAN-based age editing methods, SAM~\cite{alaluf2021only} and CUSP~\cite{gomez2022custom}, as well as the diffusion-based method, FADING~\cite{chen2023face}.


\paragraph{Quantitative comparison.}
As shown in Table~\ref{tab:final}, our method demonstrates the second-best AGE score in total, following SAM.
Meanwhile, the GAN-based methods, SAM and CUSP, show significantly worse ID scores than our method.
While FADING performed the best in ID, it gets much worse in AGE than our method.
These results indicate that our method achieves accurate age editing without significantly degrading identity preservation. 


\paragraph{Qualitative comparison.}
Figure~\ref{fig:qualitative} presents the qualitative results.
SAM exhibits significant identity changes due to its low inversion performance of input images and fails to perform adequate age regression for younger targets. 
CUSP, while effective at editing toward younger ages, tends to struggle with identity preservation.
Additionally, as seen in the second row of the lower example, aging transformations sometimes introduce unintended attributes, such as the addition of glasses due to attribute entanglement. 
FADING suffers from noticeable artifacts at younger ages and exhibits abrupt transitions around the 12-25 age range.
In contrast, our method successfully produces convincing age-editing results while preserving the distinct characteristics of the given self-reference images. 
More comparisons are shown in Appendix. 

\subsection{Ablation Studies}
This section validates the effectiveness of our improvements described in Sections~\ref{sec:int_age},~\ref{sec:prompt_design}, and~\ref{sec:lora}.
\label{sec:ablation}

\subsubsection{Regularization set refinement}
Table~\ref{tab:dexreg} presents a quantitative comparison of our method with and without regularization set refinement.
The results indicate improvements in both AGE and ID.
Notably, ID improves across all target ages, demonstrating the effectiveness of the regularization set.

\begin{table*}[t]
  \centering
  \caption{
  Quantitative comparison of our method with and without LoRA~\cite{hu2021lora}. 
  }
  \label{tab:lora_existence}
  \resizebox{\linewidth}{!}{
  \begin{tabular}{llccccccccccc}
    \toprule
    \multirow{2}{*}{Metric} & \multirow{2}{*}{Method} & \multicolumn{11}{c}{Target age} \\
    \cmidrule(lr){3-13}
    & & 1 & 5 & 8 & 12 & 17 & 25 & 35 & 45 & 60 & 80 & ALL \\
    \midrule
    \multirow{2}{*}{AGE~$\downarrow$} & Ours w/o LoRA & 9.68 & 17.5 & 16.4 & 10.6 & \textbf{9.45} & \textbf{10.4} & \textbf{9.12} & \textbf{10.7} & 9.36 & \textbf{5.74} & \textbf{10.9} \\
    & Ours & \textbf{9.22} & \textbf{17.2} & \textbf{14.1} & \textbf{9.54} & 12.3 & 12.4 & 11.7 & 11.9 & \textbf{7.42} & 7.07 & 11.3 \\
    \midrule
    \multirow{2}{*}{ID~$\downarrow$} & Ours w/o LoRA & 0.297 & 0.327 & 0.325 & 0.310 & 0.253 & 0.243 & 0.252 & 0.262 & 0.266 & 0.300 & 0.283 \\
    & Ours & \textbf{0.155} & \textbf{0.124} & \textbf{0.115} & \textbf{0.105} & \textbf{0.0630} & \textbf{0.0545} & \textbf{0.0517} & \textbf{0.0613} & \textbf{0.0649} & \textbf{0.128} & \textbf{0.0923} \\
    \bottomrule
  \end{tabular}
  }
\end{table*}

\begin{figure*}[t]
  \centering
  \includegraphics[width=\linewidth]{fig/wolora_2_ia.pdf}
  \caption{
    Qualitative comparison of our method with and without LoRA~\cite{hu2021lora}. 
  }
  \label{fig:lora_existence}
\end{figure*}

\subsubsection{
Overfitting avoidance with LoRA
}
We evaluated the impact of introducing LoRA into our method.
As shown in Table~\ref{tab:lora_existence}, incorporating LoRA improves ID across all target ages.
While our method without LoRA shows better AGE scores in some age ranges, it significantly sacrifices ID.
As shown in the qualitative results in Figure~\ref{fig:lora_existence}, without LoRA, the facial appearance is changed into a painterly style, leading to noticeable quality degradation of the output images.

\subsubsection{Prompt design}

\paragraph{
Modification of age representation.}
\label{sec:eval_yearold}

\begin{table*}[t]
  \centering
  \caption{
    Quantitative comparison of our method with and without hyphenation of age representation and the corresponding cross-attention replacement. 
    }
  \label{tab:yearold}
  \resizebox{\linewidth}{!}{
    \begin{tabular}{llccccccccccc}
      \toprule
      \multirow{2}{*}{Metric} & \multirow{2}{*}{Method} & \multicolumn{11}{c}{Target age} \\
      \cmidrule(lr){3-13}
      & & 1 & 5 & 8 & 12 & 17 & 25 & 35 & 45 & 60 & 80 & ALL \\
      \midrule
      \multirow{2}{*}{AGE~$\downarrow$} & Ours w/o hyphens & \textbf{7.65} & \textbf{16.6} & \textbf{13.9} & 9.68 & \textbf{8.57} & 12.9 & 15.1 & 12.7 & 7.71 & 7.70 & \textbf{11.2} \\
      & Ours & 9.22 & 17.2 & 14.1 & \textbf{9.54} & 12.3 & \textbf{12.4} & \textbf{11.7} & \textbf{11.9} & \textbf{7.42} & \textbf{7.07} & 11.3 \\
      \midrule
      \multirow{2}{*}{ID~$\downarrow$} & Ours w/o hyphens & 0.176 & 0.139 & 0.134 & 0.120 & 0.0797 & 0.0611 & 0.0604 & 0.0722 & 0.0791 & 0.135 & 0.106 \\
      & Ours & \textbf{0.155} & \textbf{0.124} & \textbf{0.115} & \textbf{0.105} & \textbf{0.0630} & \textbf{0.0545} & \textbf{0.0517} & \textbf{0.0613} & \textbf{0.0649} & \textbf{0.128} & \textbf{0.0923} \\
      \bottomrule
    \end{tabular}
  }
\end{table*}

We evaluated the impact of modifying the age representation prompt from ``\texttt{$\alpha$ year old}" to ``\texttt{$\alpha$-year-old}" and the corresponding cross-attention replacement. 
Table~\ref{tab:yearold} presents the quantitative comparison.
The results show that this modification allows our method to improve ID while maintaining AGE. Notably, ID preservation improved across all target ages.

\paragraph{
Use of self-reference image age.}
\begin{table*}[t]
  \centering
  \caption{
    Quantitative comparison of our method with and without using self-reference image age. 
    }
  \label{tab:refage}
  \resizebox{\linewidth}{!}{
    \begin{tabular}{llccccccccccc}
      \toprule
      \multirow{2}{*}{Metric} & \multirow{2}{*}{Method} & \multicolumn{11}{c}{Target age} \\
      \cmidrule(lr){3-13}
      & & 1 & 5 & 8 & 12 & 17 & 25 & 35 & 45 & 60 & 80 & ALL \\
      \midrule
      \multirow{2}{*}{AGE~$\downarrow$} & Ours w/o $\alpha_\mathit{ref}$ & \textbf{8.69} & \textbf{16.6} & \textbf{13.9} & \textbf{9.44} & \textbf{10.6} & \textbf{11.9} & 11.8 & \textbf{11.6} & 7.57 & \textbf{6.94} & \textbf{10.9} \\
      & Ours & 9.22 & 17.2 & 14.1 & 9.54 & 12.3 & 12.4 & \textbf{11.7} & 11.9 & \textbf{7.42} & 7.07 & 11.3 \\
      \midrule
      \multirow{2}{*}{ID~$\downarrow$} & Ours w/o $\alpha_\mathit{ref}$ & 0.163 & 0.129 & 0.119 & 0.112 & 0.0675 & 0.0639 & 0.0605 & 0.0699 & 0.0769 & 0.139 & 0.100 \\
      & Ours & \textbf{0.155} & \textbf{0.124} & \textbf{0.115} & \textbf{0.105} & \textbf{0.0630} & \textbf{0.0545} & \textbf{0.0517} & \textbf{0.0613} & \textbf{0.0649} & \textbf{0.128} & \textbf{0.0923} \\
      \bottomrule
    \end{tabular}
  }
\end{table*}

Table~\ref{tab:refage} presents the performance changes resulting from incorporating the age of self-reference images as input.
The results indicate that, although overall AGE slightly deteriorates, ID improves across all target ages.
This suggests that by incorporating the age of self-reference images, our method effectively disentangles age information from identity, enabling more efficient identity learning.

\paragraph{
Token replacement for extreme age. 
}
\label{sec:eval_baby}

\begin{table}[t]
  \centering
  \caption{
    Quantitative comparison with and without token replacement for extreme age. 
    }
  \label{tab:low_high}
  \resizebox{1\linewidth}{!}{
  \begin{tabular}{llccc}
    \toprule
    \multirow{2}{*}{Metric} & \multirow{2}{*}{Method} & \multicolumn{3}{c}{Target age} \\
    \cmidrule(lr){3-5}
    & & 1 & 80 & ALL \\
    \midrule
    \multirow{2}{*}{AGE~$\downarrow$} & Ours w/o ``\texttt{baby}"/``\texttt{elderly}" & 13.9 & 10.6 & 12.5 \\
    & Ours & \textbf{9.22} & \textbf{7.07} & \textbf{11.3} \\
    \midrule
    \multirow{2}{*}{ID~$\downarrow$} & Ours w/o ``\texttt{baby}"/``\texttt{elderly}" & \textbf{0.142} & \textbf{0.0944} & \textbf{0.0886} \\
    & Ours & 0.155 & 0.128 & 0.0923 \\
    \bottomrule
  \end{tabular}
  }
\end{table}

Table~\ref{tab:low_high} demonstrates significant improvements in AGE for the target ages (1 and 80) affected by our token replacement, with only a small drop in ID.
Figure~\ref{fig:low_high} shows the qualitative results.
The qualitative analysis also confirms that the method achieves more pronounced age modifications for both lower and higher ages.
For younger ages, facial features become rounder. 
For older ages, a part of the hair changes to white, and wrinkles increase, demonstrating realistic age transformations.

\begin{figure}
  \centering
  \includegraphics[width=\linewidth]{fig/wobaby_ia_f-eps-converted-to.pdf}
  \caption{
  Qualitative comparison with and without token replacement for extreme age. 
  }
  \label{fig:low_high}
\end{figure}
