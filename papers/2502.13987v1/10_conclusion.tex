\section{Conclusion}
\label{chap:conclusion}
In this paper, we proposed the first diffusion model-based method for personalized age transformation, which enhances the performance of both age editing and identity preservation. 
Our method fine-tunes a pretrained LDM using self-reference images and their corresponding ages, adapting the model to a specific individual.
Simultaneously, it specializes in age transformations by learning regularization images labeled with fine-grained ages.
To prevent overfitting and ensure stable age transformations, we employed LoRA during training and inference.
Furthermore, we developed effective prompt designs such as modification of age representation, use of self-reference image age, and token replacement for extreme age. 
Quantitative and qualitative evaluations demonstrated that our method achieves age editing performance comparable to state-of-the-art approaches while effectively preserving identity.

\paragraph{Limitations and future work.}
\begin{figure*}[t]
  \centering
  \includegraphics[width=\linewidth]{fig/failed_ia.pdf}
  \caption{
  Our failure cases (red boxes). In the top row, the shape and color of the teeth, as well as the shadow under the nose, appear unnatural. In the bottom row, the gender changes during age editing when transforming to an older age.
  }
  \label{fig:failed}
\end{figure*}

Figure~\ref{fig:failed} shows failure cases of our method.
One limitation is that artifacts may still occur when performing extreme age transformations, particularly when editing toward younger ages, which require significant shape deformation.
These artifacts are most commonly observed in regions where structural changes are expected, such as the jawline, mouth, and nose.
Additionally, in rare cases, the gender of the subject changes unintentionally during editing. 
This is probably because our method replaces the term ``\texttt{person}'' in the prompt based solely on age without considering gender; we used gender-neutral terms like ``\texttt{baby}" or ``\texttt{elderly}" for extreme age groups, unlike gender-specific terms like ``\texttt{man}" or ``\texttt{woman}". 
Future improvements in prompt design could potentially address this issue and further enhance the reliability of age transformations. 