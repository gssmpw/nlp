%%
%% This is file `sample-manuscript.tex',
%% generated with the docstrip utility.
%%
%% The original source files were:
%%
%% samples.dtx  (with options: `manuscript')
%% 
%% IMPORTANT NOTICE:
%% 
%% For the copyright see the source file.
%% 
%% Any modified versions of this file must be renamed
%% with new filenames distinct from sample-manuscript.tex.
%% 
%% For distribution of the original source see the terms
%% for copying and modification in the file samples.dtx.
%% 
%% This generated file may be distributed as long as the
%% original source files, as listed above, are part of the
%% same distribution. (The sources need not necessarily be
%% in the same archive or directory.)
%%
%% Commands for TeXCount
%TC:macro \cite [option:text,text]
%TC:macro \citep [option:text,text]
%TC:macro \citet [option:text,text]
%TC:envir table 0 1
%TC:envir table* 0 1
%TC:envir tabular [ignore] word
%TC:envir displaymath 0 word
%TC:envir math 0 word
%TC:envir comment 0 0
%%
%%
%% The first command in your LaTeX source must be the \documentclass command.
\documentclass[sigconf]{acmart}

%%
%% \BibTeX command to typeset BibTeX logo in the docs
\AtBeginDocument{%
  \providecommand\BibTeX{{%
    \normalfont B\kern-0.5em{\scshape i\kern-0.25em b}\kern-0.8em\TeX}}}

%% Rights management information.  This information is sent to you
%% when you complete the rights form.  These commands have SAMPLE
%% values in them; it is your responsibility as an author to replace
%% the commands and values with those provided to you when you
%% complete the rights form.
% \setcopyright{acmlicensed}
% \copyrightyear{2025}
% \acmYear{2025}
% \acmDOI{XXXXXXX.XXXXXXX}

%% These commands are for a PROCEEDINGS abstract or paper.
% \acmConference[CHI '25]{Proceedings of the 2025 CHI Conference on Human Factors in Computing Systems}{April 26-- May 1, 2025}{Yokohama, Japan}
% \acmISBN{978-1-4503-XXXX-X/18/06}

\copyrightyear{2025}
\acmYear{2025}
\setcopyright{cc}
\setcctype{by}
\acmConference[CHI '25]{CHI Conference on Human Factors in Computing Systems}{April 26-May 1, 2025}{Yokohama, Japan}
\acmBooktitle{CHI Conference on Human Factors in Computing Systems (CHI '25), April 26-May 1, 2025, Yokohama, Japan}
\acmDOI{10.1145/3706598.3713218}
\acmISBN{979-8-4007-1394-1/25/04}

\usepackage{framed}
\usepackage{tabularx}
\usepackage{graphicx}
\usepackage{multirow}
\usepackage{color, colortbl}
\usepackage{subfigure}
\usepackage{listings}
\usepackage{parcolumns}

\lstset{
  basicstyle=\ttfamily,
  columns=fullflexible,
  frame=single,
  breaklines=true,
  % postbreak=\mbox{\textcolor{red}{$\hookrightarrow$}\space},
}

\newcommand{\control}{\textbf{\texttt{Control}}}
\newcommand{\program}{\textbf{\texttt{CoTXAI}}}
\newcommand{\workflow}{\textbf{\texttt{CoTworkflow}}}
\newcommand{\workflowplus}{\textbf{\texttt{CoTworkflow+}}}

% We defined some commands to leave colorful comments in the paper
\newcommand{\glcomment}[1]{}
\newcommand{\gladd}[1]{#1}
\newcommand{\ujcomment}[1]{}
\newcommand{\ugadd}[1]{#1}
\newcommand{\gdcomment}[1]{}
\newcommand{\revise}[1]{#1}
% \newcommand{\glcomment}[1]{\textcolor{orange}{[Gaole: #1]}}
% \newcommand{\gladd}[1]{\textcolor{orange}{#1}}
% \newcommand{\ujcomment}[1]{\textcolor{red}{[\textbf{Ujwal: #1}]}}
% \newcommand{\ugadd}[1]{\textcolor{red}{#1}}
% \newcommand{\gdcomment}[1]{\textcolor{magenta}{[\textbf{Gianluca: #1}]}}
% \newcommand{\revise}[1]{\textcolor{blue}{#1}}

\newcommand{\ignore}[1]{}
\newcommand{\tabincell}[2]{\begin{tabular}{@{}#1@{}}#2\end{tabular}}
\usepackage{xspace}
\newcommand{\etal}{\emph{et al.}\xspace}
\newcommand{\paratitle}[1]{\vspace{1.0ex}\noindent\textbf{#1}}
\newcommand{\ie}{\textit{i.e.,}~}
\newcommand{\eg}{\textit{e.g.,}~}

%%
%% Submission ID.
%% Use this when submitting an article to a sponsored event. You'll
%% receive a unique submission ID from the organizers
%% of the event, and this ID should be used as the parameter to this command.
%%\acmSubmissionID{123-A56-BU3}

%%
%% For managing citations, it is recommended to use bibliography
%% files in BibTeX format.
%%
%% You can then either use BibTeX with the ACM-Reference-Format style,
%% or BibLaTeX with the acmnumeric or acmauthoryear sytles, that include
%% support for advanced citation of software artefact from the
%% biblatex-software package, also separately available on CTAN.
%%
%% Look at the sample-*-biblatex.tex files for templates showcasing
%% the biblatex styles.
%%

%%
%% The majority of ACM publications use numbered citations and
%% references.  The command \citestyle{authoryear} switches to the
%% "author year" style.
%%
%% If you are preparing content for an event
%% sponsored by ACM SIGGRAPH, you must use the "author year" style of
%% citations and references.
%% Uncommenting
%% the next command will enable that style.
%%\citestyle{acmauthoryear}

%%
%% end of the preamble, start of the body of the document source.
\begin{document}

%%
%% The "title" command has an optional parameter,
%% allowing the author to define a "short title" to be used in page headers.
% \title{Human-AI Collaboration with A Chain-of-thought Workflow: An Empirical Study in Composite Fact Verification}
%\title{Appropriate Reliance at Fine-grained Levels: An Empirical Study of Human-AI Collaboration with A Chain-of-thought Workflow}
\title{Plan-Then-Execute: An Empirical Study of User Trust and Team Performance When Using LLM Agents As A Daily Assistant}

%%
%% The "author" command and its associated commands are used to define
%% the authors and their affiliations.
%% Of note is the shared affiliation of the first two authors, and the
%% "authornote" and "authornotemark" commands
%% used to denote shared contribution to the research.
% \author{Authors Anonymized for Review Process}
\author{Gaole He}
\affiliation{%
  \institution{Delft University of Technology}
  \city{Delft}
  \country{The Netherlands}}
\email{g.he@tudelft.nl}

\author{Gianluca Demartini}
% \orcid{1234-5678-9012}
\affiliation{%
  \institution{The University of Queensland}
  \city{Brisbane}
  \country{Australia}
}
\email{g.demartini@uq.edu.au}

\author{Ujwal Gadiraju}
\affiliation{%
  \institution{Delft University of Technology}
  \city{Delft}
  \country{The Netherlands}
}
\email{u.k.gadiraju@tudelft.nl}
% \author{Ben Trovato}
% \authornote{Both authors contributed equally to this research.}
% \email{trovato@corporation.com}
% \orcid{1234-5678-9012}
% \author{G.K.M. Tobin}
% \authornotemark[1]
% \email{webmaster@marysville-ohio.com}
% \affiliation{%
%   \institution{Institute for Clarity in Documentation}
%   \streetaddress{P.O. Box 1212}
%   \city{Dublin}
%   \state{Ohio}
%   \country{USA}
%   \postcode{43017-6221}
% }

% \author{Lars Th{\o}rv{\"a}ld}
% \affiliation{%
%   \institution{The Th{\o}rv{\"a}ld Group}
%   \streetaddress{1 Th{\o}rv{\"a}ld Circle}
%   \city{Hekla}
%   \country{Iceland}}
% \email{larst@affiliation.org}

% \author{Valerie B\'eranger}
% \affiliation{%
%   \institution{Inria Paris-Rocquencourt}
%   \city{Rocquencourt}
%   \country{France}
% }

% \author{Aparna Patel}
% \affiliation{%
%  \institution{Rajiv Gandhi University}
%  \streetaddress{Rono-Hills}
%  \city{Doimukh}
%  \state{Arunachal Pradesh}
%  \country{India}}

% \author{Huifen Chan}
% \affiliation{%
%   \institution{Tsinghua University}
%   \streetaddress{30 Shuangqing Rd}
%   \city{Haidian Qu}
%   \state{Beijing Shi}
%   \country{China}}

% \author{Charles Palmer}
% \affiliation{%
%   \institution{Palmer Research Laboratories}
%   \streetaddress{8600 Datapoint Drive}
%   \city{San Antonio}
%   \state{Texas}
%   \country{USA}
%   \postcode{78229}}
% \email{cpalmer@prl.com}

% \author{John Smith}
% \affiliation{%
%   \institution{The Th{\o}rv{\"a}ld Group}
%   \streetaddress{1 Th{\o}rv{\"a}ld Circle}
%   \city{Hekla}
%   \country{Iceland}}
% \email{jsmith@affiliation.org}

% \author{Julius P. Kumquat}
% \affiliation{%
%   \institution{The Kumquat Consortium}
%   \city{New York}
%   \country{USA}}
% \email{jpkumquat@consortium.net}

%%
%% By default, the full list of authors will be used in the page
%% headers. Often, this list is too long, and will overlap
%% other information printed in the page headers. This command allows
%% the author to define a more concise list
%% of authors' names for this purpose.
% \renewcommand{\shortauthors}{Anonymous, et al.}
\renewcommand{\shorttitle}{An Empirical Study of User Trust and Team Performance with LLM Agents As A Daily Assistant}

%%
%% The abstract is a short summary of the work to be presented in the
%% article.
\begin{abstract}
Since the explosion in popularity of ChatGPT, large language models (LLMs) have continued to impact our everyday lives. 
Equipped with external tools that are designed for a specific purpose \revise{(\eg for flight booking or an alarm clock)}, LLM agents exercise an increasing capability to assist humans in their daily work. 
Although LLM agents have shown a promising blueprint as daily assistants, there is a limited understanding of {how they can provide daily assistance based on planning and sequential decision making capabilities}. We draw inspiration from recent work that has highlighted the value of `\textit{LLM-modulo}' setups in conjunction with humans-in-the-loop for planning tasks.
% lack of empirical evidence. 
% This paper analyzed how LLM agents can work as daily assistants. 
We conducted an empirical study ($N$ = 248) of LLM agents as daily assistants in six commonly occurring tasks with different levels of risk typically associated with them (\eg flight ticket booking and credit card payments). 
To ensure user agency and control over the LLM agent, we adopted LLM agents in a plan-then-execute manner, wherein the agents conducted step-wise planning and step-by-step execution \revise{in a simulation environment}. 
We analyzed how user involvement at each stage affects their trust and collaborative team performance. %with the LLM agents. %\glcomment{How about `user trust in LLM agents and collaborative team performance'. reliance is not explicitly analyzed, we only look into task performance (execution accuracy, \ie whether execution of sequence of actions will reach an expected task status). We do not calculate appropriate reliance-like measures}
Our findings demonstrate that LLM agents can be a double-edged sword --- (1) they can work well when a high-quality plan and necessary user involvement in execution are available, and (2) users can easily mistrust the LLM agents with plans that seem plausible. 
We synthesized key insights for using LLM agents as daily assistants to calibrate user trust and achieve better overall task outcomes. 
Our work has important implications for the future design of daily assistants and human-AI collaboration with LLM agents.
\end{abstract}

%%
%% The code below is generated by the tool at http://dl.acm.org/ccs.cfm.
%% Please copy and paste the code instead of the example below.
%%
% \begin{CCSXML}
% <ccs2012>
%  <concept>
%   <concept_id>00000000.0000000.0000000</concept_id>
%   <concept_desc>Do Not Use This Code, Generate the Correct Terms for Your Paper</concept_desc>
%   <concept_significance>500</concept_significance>
%  </concept>
%  <concept>
%   <concept_id>00000000.00000000.00000000</concept_id>
%   <concept_desc>Do Not Use This Code, Generate the Correct Terms for Your Paper</concept_desc>
%   <concept_significance>300</concept_significance>
%  </concept>
%  <concept>
%   <concept_id>00000000.00000000.00000000</concept_id>
%   <concept_desc>Do Not Use This Code, Generate the Correct Terms for Your Paper</concept_desc>
%   <concept_significance>100</concept_significance>
%  </concept>
%  <concept>
%   <concept_id>00000000.00000000.00000000</concept_id>
%   <concept_desc>Do Not Use This Code, Generate the Correct Terms for Your Paper</concept_desc>
%   <concept_significance>100</concept_significance>
%  </concept>
% </ccs2012>
% \end{CCSXML}

% \ccsdesc[500]{Do Not Use This Code~Generate the Correct Terms for Your Paper}
% \ccsdesc[300]{Do Not Use This Code~Generate the Correct Terms for Your Paper}
% \ccsdesc{Do Not Use This Code~Generate the Correct Terms for Your Paper}
% \ccsdesc[100]{Do Not Use This Code~Generate the Correct Terms for Your Paper}

\begin{CCSXML}
<ccs2012>
   <concept>
       <concept_id>10003120.10003121.10011748</concept_id>
       <concept_desc>Human-centered computing~Empirical studies in HCI</concept_desc>
       <concept_significance>500</concept_significance>
       </concept>
   <concept>
       <concept_id>10010147.10010178</concept_id>
       <concept_desc>Computing methodologies~Artificial intelligence</concept_desc>
       <concept_significance>500</concept_significance>
       </concept>
 </ccs2012>
\end{CCSXML}

\ccsdesc[500]{Human-centered computing~Empirical studies in HCI}
\ccsdesc[500]{Computing methodologies~Artificial intelligence}

%%
%% Keywords. The author(s) should pick words that accurately describe
%% the work being presented. Separate the keywords with commas.
% \keywords{Do, Not, Us, This, Code, Put, the, Correct, Terms, for,
%   Your, Paper}
\keywords{Human-AI Collaboration, Large Language Models, LLM agents, User Trust, Daily Assistant}

% \received{20 February 2007}
% \received[revised]{12 March 2009}
% \received[accepted]{5 June 2009}


\begin{teaserfigure}
    \centering
  \includegraphics[width=0.75\textwidth]{figures/plan_execute_illustration_2.pdf}
  \caption{Illustration of the human-AI collaboration with plan-then-execute LLM agents.}
  \Description{Illustration of the human-AI collaboration with plan-then-execute LLM agents. The planning stage mainly consists of two steps: (1) LLMs generate a draft plan (2) user edit the planning. Based on the edited plan, we get a step-wise plan (three steps in illustration). Then, in the execution stage, LLM agents will generate a sequence of actions, which is one-on-one mapping based on the plan's primary steps. Users will join this process to decide whether they approve the predicted action or choose to be involved (with manual specification of action or give text feedback to LLM agents). After the step-by-step execution of the plan, the task is solved.}
  \label{fig:illustration}
\end{teaserfigure}

%%
%% This command processes the author and affiliation and title
%% information and builds the first part of the formatted document.
\maketitle

\section{Introduction}

In real-world web search systems, addressing an information seeking task often requires retrieving and organizing information from diverse online sources. This task that we term \emph{intricate information seeking} presents a persistent and significant challenge in the field of information retrieval~\cite{strohman2005optimization, talmor2018web}. Unlike conventional single-step search, where a user query typically seeks isolated information, intricate information seeking involves integrating multiple pieces of information across various sources to formulate a comprehensive and accurate final response.
The complexity of this process is further amplified by the necessity to maintain consistency across retrieval steps, especially when user queries encompass multifaceted tasks or require extensive background knowledge. For example, responding to a complex query such as ``\emph{the economic, environmental, and social impacts of the adoption of renewable energy in developing countries}'' entails sourcing multiple relevant documents and synthesizing them into a comprehensive answer, including analyzing economic benefits and costs, assessing environmental sustainability, evaluating social implications.

Typically, driven by the ongoing evolution of large language models~(LLMs)~\cite{zhao2023survey}, a variety of existing methods aim to facilitate multi-step or complex retrieval either by heuristically decomposing the query or by iteratively refining the query through incremental optimization of intermediate outputs~\cite{yao2023react, asaiself}. For instance, some studies adopt planning strategies by decomposing a user query into sub-queries based on surface-level cues with the LLM's internal knowledge~\cite{xu2024search, reddy2024infogent}, while others employ tailored mechanisms~(\eg chain-of-thought reasoning~\cite{wei2022chain} and continuous feedback loops~\cite{shinn2024reflexion}) to progressively align intermediate reasoning steps with the final information seeking goal. Although these methods have shown promise in improving multi-step retrieval quality, they are susceptible to cascading errors, where inaccuracies or omissions in earlier sub-queries can propagate through subsequent steps. 

Inspired by the effectiveness of Monte Carlo Tree Search (MCTS) applied in complex reasoning tasks such as mathematic and code problems~\cite{alphago, YeLKAG21}, we consider incorporating MCTS into intricate information seeking scenarios to help find the optimal retrieval solution. However, two primary challenges emerge when applying MCTS to the task. First, sub-queries generated for expanding the search tree are unbounded at each step: each multifaceted intermediate task can branch into numerous investigative directions, causing the tree search space to grow exponentially. Second, the inherent characteristics of MCTS lead to a focus on local exploration~\cite{browne2012survey}, which may lead to omissions or solecism in the acquired information~\cite{_wiechowski_2022}. Specifically, (1) MCTS node selection relies on local statistics (\eg number of visits and reward accumulation), which are aggregated from limited exploration and lack a holistic understanding of the global information seeking objectives; and (2) MCTS explores the search tree by expanding nodes around the currently selected branch, while its rollout strategies are typically random or heuristic-based, making global optimality difficult to guarantee. Consequently, MCTS risks overlooking pertinent sub-tasks or prematurely converging on suboptimal search paths.


\begin{figure}
    \centering
    \includegraphics[width=0.95\linewidth]{pic/intro.pdf}
    \caption{Illustration of the pitfalls in handling intricate queries. Typical reasoning methods with web search often collect non-comprehensive documents (left), while HG-MCTS can effectively capture all necessary documents (right).}
    \label{fig:intro}
\end{figure}


To address these challenges, we propose a novel framework that incorporates MCTS into intricate information seeking, while simultaneously mitigating its inherent limitations through global guidance and multi-perspective feedback. 
Concretely, we reformulate the task as a progressive information collection process with a knowledge memory. Based on this, we propose \emph{holistically guided MCTS~(HG-MCTS)} that introduces an \emph{adaptive checklist} as a global guidance with a set of designated sub-goals.
This adaptive checklist counters the exponential growth of sub-queries by focusing the MCTS algorithm on only those branches aligned with key facets of the information need, thereby alleviating aimless expansions that could arise and enforcing a more targeted traversal of the search space, which can also be updated during the MCTS process. In parallel, we incorporate \emph{multi-perspective reward modeling} that provides both quantitative and qualitative reward signals with the checklist, allowing MCTS to incorporate a more holistic perspective on exploration. This reward modeling furnishes not just numerical indicators of exploration and retrieval quality but also textual feedback outlining which sub-goals have been addressed and which remain unsolved after node exploration. As a result, MCTS moves beyond its conventional reliance on local statistics, thereby minimizing the risk of prematurely converging on suboptimal paths and broadening its understanding of overarching information seeking objectives. Figure~\ref{fig:intro} illustrates a comparison between our method and the typical information seeking method from the retrieval comprehension perspective. 
Through this synergy, our approach preserves the inherent capability of MCTS for dynamic exploration while strengthening its capacity to incorporate newly acquired knowledge snippets. Our method methodically balances thoroughness and focus, ensuring comprehensive coverage of all relevant information while avoiding redundant or tangential searches. 


Our main contributions are summarized as follows:
\begin{itemize}
    \item We introduce a new information seeking paradigm \emph{HG-MCTS} based on a progressive information collection process with knowledge memory, which integrates an adaptive checklist for holistic sub-goal guidance in MCTS progress, thereby enabling more targeted exploration in multi-step retrieval.
    \item We propose a \emph{multi-perspective reward modeling} strategy that provides both quantitative metrics and qualitative feedback in HG-MCTS, which fosters a richer, step-wise evaluation for the value of new expanded nodes.
    \item We demonstrate how these innovations can be seamlessly integrated to improve both the efficiency and the thoroughness of large-scale web retrieval. Beyond immediate applications in question answering and knowledge-intensive search, our findings offer deeper insights into the design of more interpretable, flexible, and resilient retrieval systems.
\end{itemize}

\section{Related Work}
\label{sec:related}

\subsection{Test-Time Slow Thinking with LLMs}

The integration of the slow thinking paradigm~(\aka System 2) inspired reasoning techniques into LLMs has emerged as a pivotal research area~\cite{sutton2019bitter, wang2024q, kahneman2011thinking}, focusing on enhancing the problem-solving ability and the interoperability of LLM. A notable example is OpenAI’s o1 model\footnote{https://openai.com/o1/}, which incorporates extended internal reasoning chains during inference. 
This breakthrough has demonstrated remarkable success on programming and complex scientific benchmarks, which improve the focus of test-time reasoning that leverages extended inference processes to simulate deliberate, stepwise problem-solving akin to human-like cognitive processes without additional training~\cite{zhang2024llama, putta2024agent, luo2024improve}. Techniques such as chain-of-thought~\cite{wei2022chain} and self-consistency decoding~\cite{wangself} exemplify this paradigm, where generating intermediate reasoning steps or exploring multiple solution paths improves reliability and interoperability.
Recent advancements further extend these ideas by integrating search-based algorithms, such as beam search~\cite{kang2024mindstar} and Monte Carlo Tree Search (MCTS)~\cite{zhoulanguage, chen2024alphamath, zhang2024rest}, to systematically explore alternative reasoning trajectories. By exploring multiple outcome branches during inference, search-based methods achieve a favorable exploration-exploitation trade-off and have been widely adopted in reinforcement learning~\cite{hart1968formal, silver2017mastering} and real-world systems such as AlphaGo~\cite{silver2016mastering}. These methods are often guided by reward models, which provide feedback based on procedural or outcome-driven metrics to improve reasoning quality~\cite{snell2024scaling} iteratively. 
These collective efforts underscore a paradigm shift in LLM research, highlighting the complementary relationship between training-time strategies and scalable test-time reasoning mechanisms. 



\subsection{Web Search with Complex Reasoning}

Recent advances in LLMs have shifted web search beyond simple query-response paradigms to sophisticated methods capable of multi-step reasoning for complex information access~\cite{chen2024mindsearch, reddy2024infogent}. These approaches leverage generative models to integrate and interpret diverse information sources in real time. Such capabilities are particularly critical for intricate queries that require synthesizing fragmented or context-sensitive data, where conventional search systems often fail to maintain coherence or overlook essential insights~\cite{hoveyda2024aqa, khotdecomposed}.
Initial efforts, such as WebGPT~\cite{nakano2021webgpt}, follow an iterative process of query generation, information retrieval, summarization, and synthesizing information in response to user queries.
Building on this foundation, subsequent studies adopt chain-of-thought reasoning to decompose complicated tasks into more manageable subqueries, enabling stepwise verification and refinement~\cite{yao2023react, trivedi2023interleaving}. For instance, Search-in-the-Chain~\cite{xu2024search} systematically disaggregates complex information-seeking queries by iteratively generating partial hypotheses and validating them against web-based evidence. 
Additionally, multi-agent collaboration has been explored to further enhance the search process. 
MindSearch~\cite{chen2024mindsearch} employs a planner-searcher architecture to conduct planning based on directed acyclic graphs and to carry out hierarchical information seeking. Similarly, Infogent~\cite{reddy2024infogent} introduces a multi-module collaborative agent framework that orchestrates information aggregation across modular components.
These systems excel in dynamically managing query reformulation, evidence synthesis, and inferential reasoning, seamlessly adapting to emerging information during the search process.
Moreover, some studies incorporate explicitly retrieved external information into the MCTS reasoning process, enhancing the deliberate reasoning capabilities of LLMs in multi-hop problem-solving~\cite{lee2024zero, jiang2024rag}. In contrast to these approaches, our method reformulates the task as an information collection process with the introduction of a novel checklist-based planning mechanism to holistically guide the MCTS reasoning process. Furthermore, we combines quantitative progress reward and qualitative progress feedback in reward modeling, making the MCTS process more intelligent to explore more efficient reasoning paths.

\section{Method}
\label{sec:method}
We give an overview of our framework in~\cref{sec:method-overview}.
We then describe two novel and technical ingredients in our framework: the logarithmic-scaling quantization (\cref{sec:augment-quantization}) and the progressive upper and lower bound tightening (\cref{sec:augment-tightening}).

\subsection{Overview}
\label{sec:method-overview}

We now describe our framework for augmenting any lossy compressor (called a \emph{base compressor}) to preserve contour trees and maintain strict error bounds. 
Our framework requires two user-specified parameters, a persistence threshold $\varepsilon$ and a pointwise absolute error bound $\xi$. 
It also requires user-specified parameters associated with the specific base compressor being augmented. Our implementation works with rectilinear meshes, and it could easily be modified to work with any simply-connected tetrahedral mesh.

Our framework guarantees that, for any augmented compressor, $T_\varepsilon = T_\varepsilon'$ and $|f(x)-f'(x)| \leq \xi$ for every $x \in \X$. Starting with a standard compressor as the base compressor, we start with a step-by-step overview of our framework. 

\para{\underline{Step 1: Upper and lower bound calculation.}}~We store critical points of the simplified contour tree $T_\varepsilon$ losslessly. We calculate the initial pointwise upper and lower bounds for other point $x \in \X$. The key idea is to locate an edge $ab$ in $T_{\varepsilon}$ whose corresponding range of function values contains $f(x)$. This requires a careful computation using the join and split trees of $T_{\varepsilon}$; see \cref{sec:algorithm-details} for details. 
We let $L(x) = \min(f(a),f(b)) + \zeta$ and $U(x) = \max(f(a),f(b))-\zeta$, where $\zeta = 10^{-5}|f(b)-f(a)|$.
If we allow $x$ to have the same function value as $a$ or $b$, the topology may be altered (e.g., along the boundary of the induced region), resulting in more false cases. Adjusting the error bound by $\zeta$ prevents such issues. We also adjust $L(x)$ and $U(x)$ as needed to ensure that if $L(x) \leq f'(x) \leq U(x)$ then $|f(x)-f'(x)| \leq \xi$. 
 
When computing $T_\varepsilon$, we compute the join and split trees of $f$ and simplify the trees directly with persistence threshold $\varepsilon$. We then combine them to obtain $T_\varepsilon$. During this construction, we track which edge of $T_\varepsilon$ each point $x \in X$ corresponds to. Compared to simplifying the entire scalar field $f$ and then computing the contour tree of the simplified field (like TopoSZ), our strategy leads to equivalent results in less time.

\para{\underline{Step 2: Base compressor.}} 
We apply the base compressor to the input data $f$. 
We compress and then decompress the data to assess changes that need to be made during decompression. 
We refer to the compressed-then-decompressed data as the \emph{intermediate data}.

\para{\underline{Step 3: Logarithmic-scaling quantization.}} 
We introduce a novel quantization technique that respects the pointwise upper and lower bounds imposed in Step 1. 
If possible, the entropy of the quantization numbers $\{n_x\}$ will be identical to that of standard linear-scaling quantization.
However, when linear-scaling quantization cannot produce a prediction for a point $x$ that respects $L(x)$ and $U(x)$, $x$ will be quantized with more precision (i.e.,~$\xi \leftarrow \xi/2$) to satisfy those bounds.

\para{\underline{Step 4: Progressive upper and lower bound tightening.}} 
We introduce a novel technique for calculating adjustments to the intermediate data to guarantee that the contour tree is preserved.
We compute the join and split trees directly. If a false edge is detected during computation, the upper and lower bounds are tightened around points in the segmentation region corresponding to the edge (see \cref{sec:merge-and-contour-tree}). All edges whose growth involved these points are recomputed.
We continue until the join and split trees of the decompressed data match those of the ground truth. We do not compute the contour tree directly as the preservation of the join and split trees guarantees the preservation of the contour tree.

\para{\underline{Step 5: Lossless compression.}} 
We encode the quantization numbers using Huffman coding. The output of the base compressor, the encoded quantization numbers, and any losslessly stored values are written to a binary file which is further losslessly compressed using xz, a general-purpose data compression tool available via {XZ Utils}~\cite{XZUtils}.

\subsection{Logarithmic-Scaling Quantization}
\label{sec:augment-quantization}

We now describe the first novel ingredient in our framework: a variable precision quantization technique that preserves tight pointwise upper and lower bounds. %without significantly compromising the entropy of the overall distribution of quantization numbers. 
For each $x \in \X$, the intermediate data contains an estimated value $g(x)$ for the ground truth value $f(x)$. 
Let $L(x)$ and $U(x)$ denote the lower and upper bounds assigned to $x$.
To ensure that $L(x) \leq f'(x) \leq U(x)$, we assign to each $x \in \X$ a numerator $a_x \in \Z$ and a precision $p_x \in \N$ that indicates the number of iterations. 
Our reconstructed value is 
\begin{equation}
f'(x) = g(x) + \frac{2\xi \cdot a_x}{2^{p_x}}.
\label{eq:fprime-original}
\end{equation}

To calculate each $a_x$ and $p_x$, we first set $p_x=0$. 
We then look for the value of $a_x$ satisfying 
\begin{equation*}
L(x) \leq g(x) + \frac{2\xi \cdot a_x}{2^{p_x}} \leq U(x)
\label{eq:Bounds}
\end{equation*}
such that $|a_x|$ is minimized. If there is no valid value of $a_x$, we increase $p_x$ by $1$ and search again. This process is repeated until a valid $a_x$ is found. If $p_x$ reaches an arbitrary threshold, we stop searching and instead store $f(x)$ losslessly. We set this threshold equal to $11$.

When $p_x = 0$, the above process is the same as the standard linear-scaling quantization, except that we also seek to maintain the upper and lower bounds. 
Each time a linear-scaling quantization fails to identify a valid choice for $a_x$ that yields a value of $f'(x)$ within the upper and lower bounds for $x$, we cut the interval lengths in half by increasing $p_x$ by $1$ and continue searching.
When the interval lengths are smaller, it is more likely that a valid choice of $a_x$ exists. 
It is also possible that during an iteration, multiple valid choices of $a_x$ exist, so we choose the one with the smallest absolute value to minimize the entropy of $\{a_x\}$. 

\begin{figure}[!ht]
    \centering
    \vspace{-2mm}
    \includegraphics[width=\linewidth]{fig-log-scale-quantization.pdf}
    \vspace{-6mm}
    \caption{(A) If $p_x = 0$, there are no valid quantization intervals. (B) Increasing $p_x$ to $1$ allows for a valid quantization interval.}
    \label{fig:log-scale-quantization}
    \vspace{-2mm}
\end{figure}

This process is illustrated in \cref{fig:log-scale-quantization}. 
(A) contains an example where there are no quantization intervals where we can place $f'(x)$ to respect the upper and lower bounds. 
In (B), by raising the precision $p_x$ by 1, the quantization intervals are halved, giving a valid choice for $f'(x)$.

When encoding the data, we store a single quantization number $n_x$ for each $x \in \X$. 
To calculate each $n_x$, we first find the maximum precision $p_m$ used for any single point. The points are assigned the single quantization number $n_x = a_x \cdot 2^{p_m-p_x}$ and the max precision $p_m$ is stored in the compressed output. 
During decompression, the point $x$ is assigned the value 

\begin{equation}
f'(x) = g(x) + \frac{2\xi \cdot n_x}{2^{p_m}}.
\label{eq:fprime}
\end{equation}
Setting $n_x = a_x \cdot 2^{p_m-p_x}$ in Eq.~\eqref{eq:fprime} means that
\begin{equation*}
  g(x) + \frac{2\xi \cdot n_x}{2^{p_m}} = g(x) + \frac{2\xi \cdot a_x \cdot 2^{p_m-p_x}}{2^{p_m}} = g(x) + \frac{2\xi \cdot a_x}{2^{p_x}}.
  \label{eq:logscale}  
\end{equation*}
Therefore, the formulation in Eq.~\eqref{eq:fprime} is equivalent to the original formulation of $f'$ in Eq.~\eqref{eq:fprime-original}.

In comparison with TopoSZ, the above variable precision technique allows us to store fewer points losslessly.
In order to ensure the quantization numbers do not get too large, if any point has a precision greater than $10$ it is stored losslessly. This ensures that $p_m \leq 10$ for all trials.

\subsection{Progressive Upper and Lower Bound Tightening}
\label{sec:augment-tightening}

We now describe the second novel ingredient in our framework, namely, a \emph{progressive error bound tightening} process. 
Specifically, the process computes the join and split trees of the decompressed data. During the computation, it detects false cases, and tightens the upper and lower bounds in the neighborhoods of false cases. The algorithm progresses through merge tree computation, checking the correctness of each edge and tightening when needed, until every edge is correctly preserved.
The process allows us to bypass iteratively recomputing the entire contour tree (in the case of TopoSZ), significantly speeding up the compression process. During the tightening process, we work with merge trees (instead of contour trees), since the persistence of a leaf (local extremum) can be computed from its nearby saddle based on branch decomposition (i.e.,~local information), thereby allowing for our progressive tightening strategy. By contract, computing the persistence of a leaf of a contour tree may require global information from the whole contour tree due to the existence of V and W structures~\cite{hristov2021w}.

We describe this process for the join tree, which works analogously for the split tree. We only consider false cases involving extremum-saddle pairs. 

\para{False case detection}. To detect false cases, we construct $T'$. Doing so allows us to locate mismatches between edges in $T'_\varepsilon$ and those in $T_\varepsilon$.
We construct $T'$ using a modified version of the edge growing procedure from local minima and saddles (see~\cref{sec:merge-and-contour-tree}).
To start, we extract a list of local minima of $f'$ sorted by decreasing function values. Then, proceeding in sorted order, we grow an edge from each local minimum $m$ to a saddle $s$, and check two cases for $s$; see \cref{sec:algorithm-details} for illustrations: 

\underline{Case (I).} If $s$ is unpaired, i.e., $m$ is the first local minimum (among all local minima) whose growth terminates at $s$, then $m$ and $s$ form a persistence pair, with a persistence $p =|f'(s)-f'(m)|$. 
If $p < \varepsilon$, then the edge $ms$ does not belong to $T'_\varepsilon$; otherwise, $ms$ belongs to $T'_\varepsilon$.  

\underline{Case (II).} If $s$ is already paired, then $m$ must pair with some other saddle $s'$, and $s'$ must be an ancestor of $s$ in the join tree. A paired $s$ means that $s$ has been discovered earlier during the growth of another local minimum $m'$ such that $m'$ and $s$ form a persistence pair with persistence $p'$, and the edge $m's$ belongs to $T'$. 

\underline{Case (II.a).} 
Suppose that $p' \geq \varepsilon$. Since $m'$ preceds $m$ in the sorted order, $f'(m') > f'(m)$. Since $s'$ is an ancestor of $s$, $f'(s') > f'(s)$. Therefore $|f'(s') - f'(m)| > |f'(s) - f'(m')| = p' \geq \varepsilon$. 
Thus, the pair $(m,s')$ has a persistence above $\varepsilon$, and $ms$ must be an edge in $T'_\varepsilon$.

\underline{Case (II.b).} 
Now suppose that $p' < \varepsilon$. In this case, we do not have enough information to determine the persistence of $(m,s')$. Therefore, we grow from saddle $s$ to reach a new saddle $s''$. We then check cases (I) and (II) again, using $s''$ in place of $s$. 

Once we are done checking cases (I) and (II), if $m \notin T'_{\varepsilon}$ but $m \in T_{\varepsilon}$, then $m$ is a false negative. 
Likewise, if $ms \in T'_{\varepsilon}$ but $ms \notin T_{\varepsilon}$, then $ms$ is a false positive. 

Growing the global minimum will never produce a false case as long as the rest of $T'_\varepsilon$ is correctly predicted. Thus, we skip the growth at the global minimum, denoted as $\hat{m}$. 
Because $\hat{m}$ is the last growth that remains active, its growth will form the \textit{trunk}, a monotone sequence of edges to the root that links $\hat{m}$ to the remaining saddles~\cite{gueunet2017task}. Since $\hat{m}$ and the remaining saddles are already correctly predicted, so is the trunk, therefore no further false cases are possible, and we skip growing $\hat{m}$. 
This algorithm also admits a number of special cases; see~\cref{sec:algorithm-details}.

\para{Progressive false case correction.} 
If there is a false case, we first tighten the upper and lower bounds of points in some region $R$ to correct it. If $ms$ is a false positive, then $R$ is the region of the merge-tree-induced segmentation of $f'$ corresponding to $ms$. If $m$ is a false negative, and edge $m\hat{s}$ belongs to $T_\varepsilon$ (for some saddle $\hat{s}$), then $R$ is the region of the merge-tree-induced segmentation of $f$ corresponding to $m\hat{s}$. If the same false case occurs multiple times, we grow the region $R$. We tighten the upper and lower bounds of each $x \in R$ similarly to TopoSZ, but we tighten more aggressively to speed up compression. 
We then update the decompressed data $f'$ to respect the new bounds; see~\cref{sec:algorithm-details} for numerical specifics and a comparison with TopoSZ.

Once we update $f'$, these updates may affect parts of the join and split trees beyond the false cases, thus we must recompute those areas to ensure correctness. Specifically, we must check for any extrema bordering $R$ that may have appeared or disappeared as a result of the tightening process and update the trees accordingly. Let $E$ be the set of edges whose segmentation regions border $R$. Then the tightening also may have affected each edge $e \in E$ and every ancestor of $e$ (i.e.,~edges
connecting $e$ to the root of the tree). We recompute all such edges to ensure correctness. As before, we recompute parts of the tree in order of the function values. 


\section{Study Design}

This section describes our experimental conditions, tasks,  variables, procedure, and participants in our study. {Our
study was approved by the human research ethics committee of our institution.}

\subsection{Experimental Conditions}
In our study, users collaborate with LLM agent-based daily assistants in two stages: planning and execution. 
To comprehensively understand the effect of user involvement at each stage, we considered a 2 × 2 factorial design with four experimental conditions: (1) automatic planning, automatic execution (represented as AP-AE), (2) automatic planning, user-involved execution (represented as AP-UE), (3) user-involved planning, automatic execution (represented as UP-AE), (4) user-involved planning, user-involved execution (represented as UP-UE). 
In conditions with user-involved planning, users are allowed to edit the plan generated by LLM with the actions of edit/add/delete/split step. 
By comparison, in conditions with automatic planning, users will directly adopt the plan generated by the daily assistant.
In conditions with user-involved execution, users can interact with the step-by-step execution LLM agent (cf. Section~\ref{sec-method-execution}) and refine execution results with text feedback or manual specification. 
By comparison, in conditions with automatic execution, users will directly accept the automatic execution results.

\begin{table*}[htbp]
	\centering
	\caption{Selected tasks in our study. The `Risk' is based on the risk feedback obtained with pilot study. \#A and \#C refer to the number of actions and the number of named concepts in each task, respectively. }
	\label{tab:tasks}%
	\scalebox{.8}{
    \begin{tabular}{p{0.02\textwidth}|p{0.04\textwidth}|p{0.06\textwidth}|p{0.8\textwidth}|p{0.03\textwidth}|p{0.03\textwidth}| p{0.12\textwidth}}
		\hline
% 		\textbf{Participant Feedback} &  \textbf{Sentiment}& \textbf{Reason}  \\
        \textbf{ID}&\textbf{Risk}& \textbf{Domain}&  \textbf{Task Description}& \textbf{\#A}& \textbf{\#C}& \textbf{Notes}\\
		\hline \hline
		% Low& Finance& Can you help me log into my two different platform accounts and then check their account balances? The first account ID is 12345678, password is Password123; the second account ID is 87654321, password is 123Password.& Tutorial\\
		% \hline \hline
		1& High& Finance& My account ID is 54321, and the password is PWD2023. I plan to make two foreign exchange transactions. The first is to buy 10,000 euros (with USD), and the second is to sell 5,000 US dollars (to EUR). Please help me operate.&4&4& simple task, imperfect plan\\
        % \hline
        % High& Finance& I need to know the detailed information about the 'Happy Savings High Gold' deposit product, including its minimum deposit amount, annual interest rate, and deposit term. Also, I want to use my account number 123456 and password 789123 along with the most recently received verification code 8888 to apply for a loan, and I would like to know the review time for this loan application as well as how to check the status of all my current loan applications.\\
        \hline
        2& High& Finance& Please inquire about the current debt amount of my credit card with the last five digits 12345, and deduct the corresponding 12000 USD from my savings card number 6212345678900011 to repay this debt, then help me check the amount of the outstanding bill for the same credit card within 30 days after today.&4&6& complex task, imperfect plan\\
        \hline
        3& High& Repair& I need to schedule a repair for my TV at 6 PM tomorrow evening. The brand is Sony, model X800H, and there is an issue with the screen. Please book the repair service and tell me the reservation number.&4&7& complex task, imperfect plan\\
        % \hline
        % Low& Tracking& Please check the latest status of my two orders with the numbers 123456789 and 987654321, and confirm whether they are both associated with my customer ID A123456.\\
        % \hline
        % High& Restaurant& Please order a Spicy Hot Pot for me at the restaurant, add two extra servings of beef and a plate of hand-torn cabbage, then place the order using my table ID 10, and help me check out.\\
        \hline
        4& Low& Alarm& I need to set an alarm for every weekday morning at 7:30, and then cancel the alarm for Thursday, changing it to 8:00 in the evening.&2&3& simple task, correct plan\\
        \hline
        5& Low& Flight& I have an important meeting to attend next Wednesday, and I need to book a flight ticket from London to Amsterdam for tomorrow, it must be a morning flight, and then return from Amsterdam to London tomorrow night, please handle it for me.&2&6& simple task, correct plan\\
        \hline
        6& Low& Travel& Please plan a trip for me departing on October 1st at 8:00 AM to Japan, returning on October 7th at 11:00 PM, including Tokyo Disneyland, Senso-ji Temple, Ginza, Mount Fuji, Kyoto cultural experience, Universal Studios Osaka, and visiting the Nara Deer Park on October 4th, and help me find hotels where the nightly cost does not exceed 10,000 Japanese yen.&3&11& complex task, correct plan\\
        \hline
            \hline
    \end{tabular}}
\end{table*}%

\subsection{Tasks}
To analyze how LLM agents can serve as daily assistants, we adopted tasks from a planning dataset designed for LLM agents --- UltraTool~\cite{huang2024planning}. 
We selected daily scenarios: currency transactions, credit card payments, repair service appointments, alarm setting, flight ticket booking, and trip itinerary planning. 
The selected tasks are shown in Table~\ref{tab:tasks}. \revise{For more details about how the plan-then-execute LLM agent works on the selected tasks (\eg automatic plan, pre-defined actions, automatic evaluation, and explanation for errors in automation), please refer to the appendix.}
All tasks in UltraTool dataset are annotated with the step-wise plan format described in Section~\ref{sec-method-planning}. \revise{The execution of these tasks is based on a simulation environment (described in Section~\ref{sec-method-execution}) where all required actions are implemented as backend APIs. 
In our study, all tasks are executed in a simulation setup, which has been a popular method for orchestrating meaningful human-centered AI studies~\cite{doshi2017towards,salimzadeh2024dealing}.} %is effective in developing and validating theory~\cite{davis2007developing} and has been widely adopted in existing research on agent-based modeling and HCI studies~\cite{olson2014ways}.}
%\glcomment{It is unclear if the participants were performing the real tasks or were in a scenario-based setting. The concern impacts the validity of the study design. The clarification along with simulation design justification, would help readers better understand the results.}\ujcomment{Clarify, justify this as a valid method used widely in HCAI research.}\glcomment{I clarified in the execution part (Section 3.3). If you feel it is not necessary to highlight again, we can remove the last sent here.}

% \begin{table*}[htbp]
% 	\centering
% 	\caption{Selected tasks in our study. The `Risk' is based on the risk feedback obtained with pilot study.}
% 	\label{tab:tasks}%
% 	\scalebox{.84}{
%     \begin{tabular}{p{0.02\textwidth}|p{0.04\textwidth}|p{0.07\textwidth}|p{0.8\textwidth}| p{0.12\textwidth}}
% 		\hline
% % 		\textbf{Participant Feedback} &  \textbf{Sentiment}& \textbf{Reason}  \\
%         \textbf{ID}&\textbf{Risk}& \textbf{Domain}&  \textbf{Task Description}& \textbf{Notes}\\
% 		\hline \hline
% 		% Low& Finance& Can you help me log into my two different platform accounts and then check their account balances? The first account ID is 12345678, password is Password123; the second account ID is 87654321, password is 123Password.& Tutorial\\
% 		% \hline \hline
% 		1& High& Finance& My account ID is 54321, and the password is PWD2023. I plan to make two foreign exchange transactions. The first is to buy 10,000 euros (with USD), and the second is to sell 5,000 US dollars (to EUR). Please help me operate.& simple task, imperfect plan\\
%         % \hline
%         % High& Finance& I need to know the detailed information about the 'Happy Savings High Gold' deposit product, including its minimum deposit amount, annual interest rate, and deposit term. Also, I want to use my account number 123456 and password 789123 along with the most recently received verification code 8888 to apply for a loan, and I would like to know the review time for this loan application as well as how to check the status of all my current loan applications.\\
%         \hline
%         2& High& Finance& Please inquire about the current debt amount of my credit card with the last five digits 12345, and deduct the corresponding 12000 USD from my savings card number 6212345678900011 to repay this debt, then help me check the amount of the outstanding bill for the same credit card within 30 days after today.& complex task, imperfect plan\\
%         \hline
%         3& High& Repair& I need to schedule a repair for my TV at 6 PM tomorrow evening. The brand is Sony, model X800H, and there is an issue with the screen. Please book the repair service and tell me the reservation number.& complex task, imperfect plan\\
%         % \hline
%         % Low& Tracking& Please check the latest status of my two orders with the numbers 123456789 and 987654321, and confirm whether they are both associated with my customer ID A123456.\\
%         % \hline
%         % High& Restaurant& Please order a Spicy Hot Pot for me at the restaurant, add two extra servings of beef and a plate of hand-torn cabbage, then place the order using my table ID 10, and help me check out.\\
%         \hline
%         4& Low& Alarm& I need to set an alarm for every weekday morning at 7:30, and then cancel the alarm for Thursday, changing it to 8:00 in the evening.& simple task, correct plan\\
%         \hline
%         5& Low& Flight& I have an important meeting to attend next Wednesday, and I need to book a flight ticket from London to Amsterdam for tomorrow, it must be a morning flight, and then return from Amsterdam to London tomorrow night, please handle it for me.& simple task, correct plan\\
%         \hline
%         6& Low& Travel& Please plan a trip for me departing on October 1st at 8:00 AM to Japan, returning on October 7th at 11:00 PM, including Tokyo Disneyland, Senso-ji Temple, Ginza, Mount Fuji, Kyoto cultural experience, Universal Studios Osaka, and visiting the Nara Deer Park on October 4th, and help me find hotels where the nightly cost does not exceed 10,000 Japanese yen.& complex task, correct plan\\
%         \hline
%             \hline
%     \end{tabular}}
% \end{table*}%

\paratitle{Task Selection}. First, based on the domain distribution of the UltraTool dataset, we selected seven domains: Finance, Alarm, Travel, Tracking, Restaurant, Flight, and Repair. 
For each domain, we only \revise{consider} tasks that contain more than ten steps (including all sub-steps) and require at least three uses of actions. 
Then, we manually selected ten tasks: four from the finance domain and one for each of the others. 
With a pilot study, we tested how users work on the ten tasks. 
We recruited 10 participants from the Prolific platform and only considered the feedback of 9 participants who passed all attention checks. 
\revise{Using} the question “How much risk do you perceive in this task when relying on this daily AI assistant?”, we collected the perceived risk of working with the LLM agents on each task \revise{using a 5-point Likert scale, ranging from \textit{1: not risky at all}---to---\textit{5:very risky}.} %\ujcomment{mention the scale}
We categorize the ten tasks into a high-risk group (top 5) and a low-risk group (bottom 5). 
We selected three tasks from each group while balancing the complexity of the task description (three simple tasks and three complex tasks) and the correctness of the provided plan (three correct plans and three imperfect plans). 
\revise{Based on existing literature on task complexity~\cite{wood1986task,salimzadeh2023missing}, we  considered component complexity to inform our selection. 
This is assessed as the `total number of distinct information cues that need to be processed to perform the task'. 
Here, we considered the number of unique actions and the number of named concepts provided in each task.}
According to prior work~\cite{miller1956magical}, most people can only handle 5 to 9 concepts at the same time. 
The component complexity of all complex tasks in our study is more than nine.
% \glcomment{Clarify here for the design of simple/complex tasks}\ujcomment{Task complexity (Robert Wood) -> Task Complexity (cite Sara's work)}
The six tasks selected are shown in Table~\ref{tab:tasks}. 
Besides the six tasks, we used one simple task (\ie checking bank account balance) as the example in the onboarding tutorial.


\subsection{Measures and Variables}
% \subsection{Evaluation}
\label{sec-measure}

The variables and measures used in our study refer to existing empirical studies of human-AI collaboration~\cite{lai2021towards}. 
All measures adopted in our study can be summarized in Table~\ref{tab:variables}. 
\begin{table*}[htbp]
	\centering
	\caption{The different variables considered in our experimental study. ``DV'' refers to the dependent variable.}
	\label{tab:variables}
	\begin{footnotesize}
	\begin{tabular}{c | c | c | c}
	    \hline
	    \textbf{Variable Type}&	\textbf{Variable Name}& \textbf{Value Type}& \textbf{Value Scale}\\
	    \hline \hline

	    \hline
         \multirow{2}{*}{Calibrated Trust (DV)}& Calibrated Trust in planning (CT$_p$)& Binary& 0: miscalibrated trust, 1: calibrated trust\\
        & Calibrated Trust in execution (CT$_e$)& Binary& 0: miscalibrated trust, 1: calibrated trust\\
        \hline
	    \multirow{3}{*}{Task Performance (DV)}& Plan Quality& Likert& 5-point, 1: low, 5: high\\
        % & Outcome Accuracy& Continuous, Interval& [0.0, 1.0]\\
        % & Action Sequence Accuracy - Relaxed& Continuous, Interval& [0.0, 1.0]\\
        & Action Sequence Accuracy (ACC$_s$)& Binary& 0: mismatch, 1: exact match with ground truth\\
        & Execution Accuracy (ACC$_e$)& Binary& 0: wrong execution result, 1: correct execution result\\
        % & Action Sequence Recall& Continuous, Interval& [0.0, 1.0]\\
	    % \hline
	    % \multirow{4}{*}{Reliance (DV)}& Agreement Fraction& Continuous, Interval& [0.0, 1.0]\\
	    % & Switch Fraction& Continuous& [0.0, 1.0]\\
	    % & RAIR& Continuous& [0.0, 1.0]\\
	    % & RSR& Continuous& [0.0, 1.0]\\
	    \hline
     % \multirow{2}{*}{Assessment (DV)}& Degree of Miscalibration& Continuous, Interval& [0,6]\\
     % & Self-assessment& Continuous, Interval& [-6,6]\\
     % \hline
     \multirow{4}{*}{Trust}& Reliability/Competence& Likert& 5-point, 1: poor, 5: good\\
    & Understanding/Predictability& Likert& 5-point, 1: poor, 5: good\\
    & Intention of Developers& Likert& 5-point, 1: poor, 5: good\\
    & Trust in Automation& Likert& 5-point, 1:strong distrust, 5: strong trust\\
    \hline
     % \multirow{4}{*}{Trust (DV)}& TiA-Trust& Likert& 5-point, 1:strong distrust, 5: strong trust\\
     % & TiA-Trust& Likert& 5-point, 1:strong distrust, 5: strong trust\\
     % & TiA-Trust& Likert& 5-point, 1:strong distrust, 5: strong trust\\
     % & TiA-Trust& Likert& 5-point, 1:strong distrust, 5: strong trust\\
     % \hline
     \multirow{4}{*}{Covariates}& LLM Expertise& Likert& 5-point, 1: No experience, 5: Extensive experience\\
     & Automatic Assistant Expertise& Likert& 5-point, 1: No experience, 5: Extensive experience\\
     & Propensity to Trust& Likert& 5-point, 1: tend to distrust, 5: tend to trust \\
     & Familiarity& Likert& 1: unfamiliar, 5: very familiar\\
     \hline
	\multirow{5}{*}{Exploratory}& Confidence& Likert& 5-point, 1: unconfident, 5: confident\\
    & Risk Perception& Likert& 5-point, 1: not risky at all, 5: very risky\\
    & Open Feedback on Planning& Text& Open Text\\
    & Open Feedback on Execution& Text& Open Text\\
    & Other Open Feedback& Text& Open Text\\
    % & Cognitive Load& Likert& -7: very low, 7: very high\\
	    \hline
     \multirow{6}{*}{{Cognitive Load }}& Mental Demand& Likert& -7: very low, 7: very high\\
	    & Physical Demand& Likert& -7: very low, 7: very high\\
	    & Temporal Demand& Likert& -7: very low, 7: very high\\
	    & Performance& Likert& -7: Perfect, 7: Failure\\
        & Effort& Likert& -7: very low, 7: very high\\
        & Frustration& Likert& -7: very low, 7: very high\\
     \hline
	\end{tabular}
	\end{footnotesize}
\end{table*}

\paratitle{Calibrated Trust}. To assess calibrated trust in the planning stage and execution stage, we assessed user trust at each stage with a question ``Do you trust that [the execution of this plan / the execution process] can provide a correct outcome based on the task instructions?''. 
Based on the plan quality evaluation (5-point Likert), the calibrated trust in the planning (CT$_p$) is calculated based on the frequency at which users trusted the high-quality plan (expert annotation with 5) and users distrusted the plan with other evaluation results.
% will expect that users can trust the plan when the plan quality is annotated as 5. Otherwise, users should indicate distrust. 
Similarly, for the calibrated trust in execution (CT$_e$), we calculated the frequency at which users trusted the correct execution results and distrusted the wrong execution results. 
The two measures can be calculated as:

\begin{equation}
\begin{aligned}
    \textnormal{CT}_p = &\mathbb{I}\left( \textnormal{trust = `Yes'}, \textnormal{plan quality} = 5\right) \\ & +\mathbb{I}\left( \textnormal{trust = `No'}, \textnormal{plan quality} < 5\right)
\end{aligned}
\end{equation}

\begin{equation}
    \textnormal{CT}_e = \mathbb{I}\left( \textnormal{trust = `Yes'}, \textnormal{ACC}_e = 1\right) + \mathbb{I}\left( \textnormal{trust = `No'}, \textnormal{ACC}_e = 0 \right)
\end{equation}

To assess the task performance, we mainly considered the task outcome from the planning and execution stages. 

\paratitle{Plan Quality}. As for the planning outcome, we evaluate the plan quality based on a 5-point Likert scale: 
1. low-quality plan, task requirements not covered; 
2. low-quality plan, task requirements covered but with grammar errors; 
3. medium-quality plan, task requirement covered but with at least one action intent mismatch with ground truth action sequence; 
4. medium-quality plan, task requirements covered but miss or have wrong details for action parameters; 
5. high-quality plan, covering all task requirements and providing all necessary details.

\paratitle{Execution Performance}. 
The execution of the step-wise plan will result in an action sequence. 
We provide a ground truth action sequence as a reference to evaluate the generated action sequence. 
We measure the action sequence accuracy (ACC$_s$) as the strict match of the action sequence and ground truth. 
% Meanwhile, there are some actions that do not harm the execution results (\eg searching for flights). 
Meanwhile, if one action sequence contains some redundant actions that are not harmful (\eg searching for flights), the execution results should still be correct. 
Thus, we also consider execution accuracy (ACC$_e$) as a task performance measure.

\paratitle{Subjective Trust and Covariates}. To enrich our analysis of user trust, we followed existing work to adopt the six subscales from the Trust-in-automation questionnaire~\cite{korber2019theoretical}. 
The four subscales --- \textit{Reliability/Competence},  \textit{Understanding/Predictability},  \textit{Intention of Developers},  \textit{Trust in Automation} are used as subjective measures of user trust in the LLM agent. 
Meanwhile, the \textit{Familiarity} and \textit{Propensity to Trust} are also used as covariates. Besides them, we considered user expertise in LLMs and user expertise in automatic assistants as covariates. 

\paratitle{Exploratory Variables}. To enrich our understanding of LLM agent as daily assistant, we assessed user confidence (both planning and execution) and risk perception along with each task. 
After users finish the study, we also ask for their open-text feedback on the planning and execution stages as well as other comments. 
To check the cognitive load of user involvement in our study, we adopted the NASA-TLX questionnaire~\cite{colligan2015cognitive}, which contains six subscales.

% The variables and measures used in our study refer to existing empirical studies of human-AI collaboration~\cite{lai2021towards}. 
% All measures adopted in our study can be  summarized in Table~\ref{tab:variables}. %\glcomment{We can consider removing this table to save space, or move it to supplementary materials}

\subsection{Participants}

\paratitle{Sample Size Estimation}. 
To ensure sufficient statistical power, we estimated the required sample size for a 2 × 2 factorial design based on G*Power~\cite{faul2009statistical}. 
{To correct for testing multiple hypotheses, we applied a Bonferroni correction so that the significance threshold decreased to $\frac{0.05}{4}=0.0125$.} 
We specified the default effect size $f = 0.25$
(\textit{i.e.,} indicating a moderate effect), a significance threshold $\alpha = 0.0125$ (\textit{i.e.,} due to testing multiple hypotheses), a statistical power of $(1 - \beta) = 0.8$, and that we will investigate $4$ different experimental conditions/groups. 
This resulted in a required sample size of $244$ participants. 
We thereby recruited 347 participants from the crowdsourcing platform Prolific\footnote{\url{https://www.prolific.co}}, to accommodate potential exclusion.

\paratitle{Compensation}. All participants were rewarded with an hourly wage of \pounds 8.1 deemed to be ``\textit{Fair}'' payment by the platform (estimated completion time was 30 minutes). 
As participants in condition UP-UE spent longer in the study, we paid each participant a commensurate bonus accounting for an extra 10 minutes. % with a bonus.
% \stcomment{This makes for 1.8 * 6 = 8.1 pounds hourly wage. If I do 1.8/7.5 then I get to estimated completion time of 14.4 minutes}~\glcomment{I checked it again, actually we run two rounds for main studt. First with estimation of 12 minutes (\pound 1.5), and second with 10 minutes (\pound 1.25). I think we can write it as 1.5. I included the service fee and calculated the mean for all participants.}
We rewarded participants with extra bonuses of \pounds 0.05 for every high-quality plan and correct execution result. 
According to existing literature~\cite{lee2004trust}, such a bonus setup can help incentivize participants to reach a correct decision. 
\revise{In comparison with existing literature exploring human-AI decision making~\cite{lai2021towards}, our reward setup is above the average payment and can be considered as being sufficient to elicit ecologically valid behavior among participants (\ie aiming to arrive at accurate execution results). Moreover, similar bonus structures akin to our setup have been effective in incentivizing reliable participant behavior and improving data quality across different studies with crowdsourced participants~\cite{fan2020crowdco, salimzadeh2024dealing,liutilizing,ma2024you}.}
%\glcomment{R3 also asked the authors to clarify the participants' incentive to achieve better performance in this dataset.} \ujcomment{Yes, these incentives are meaningful for participants. Two arguments to make:  1. Show that this is a good incentive compared to other HCAI studies. 2. Show that this is meaningful to participants in crowdsourcing platforms. Both can be supported by references.}

\begin{figure*}[h]
    \centering
    \includegraphics[width=\textwidth]{figures/cognitive_load_bar_plot_new.pdf}
    \caption{\revise{Bar plot for cognitive load across all conditions. ** indicates significance ($p < 0.0125$) through post-hoc Tukey HSD test. The error bars represent the 95\% confidence interval.}}%\glcomment{I changed the colors of bars to keep consistent with the confidence plot.}} 
    \label{fig:cognitive_load}
    \Description{Bar plot for cognitive load across all conditions. User involvement in planning shows a significant impact on Mental Demand, Temporal Demand, and Frustration. User involvement in execution shows a significant impact on Performance and Effort. These results indicate that user involvement in planning and execution will require a relatively high cognitive load.}
\end{figure*}

\paratitle{Filter Criteria}. All participants were proficient English speakers between the ages of 18 - 50. 
We also constrained their prior experience (at least 40 successful submissions) and had an approval rate of above 90\% on the Prolific platform. 
We excluded participants from our analysis if they failed any attention check, or represented an outlier regarding the plan quality. 
Outliers were 4 participants who generated more than three low-quality plans among six tasks. 
The reserved 248 participants had an average age of 32.5 ($SD$= 8.1) and a balanced gender distribution
($50\%$, $49.6\%$ female, $0.4\%$ other).


% We filter out participants who provide low-quality plans (with a manual check), four participants. 


\subsection{Procedure}
At the beginning of our study, we showed informed consent for data collection and the study's purpose. 
Only participants who signed the informed consent \revise{were allowed to} continue to work on our study. 
Next, participants were asked to complete a pre-task questionnaire to measure their expertise in LLM and automatic assistants. %\glcomment{Do you think it's fine not to provide a flow chart for the whole process? As our paper can be quite long}\gdcomment{I seems pretty standard steps, so not necessary imo}

Participants were then assigned to one of the experimental conditions, which differed in the level of user involvement in the planning stage and execution stage. 
With an onboarding tutorial, we showcased the necessary interactions that participants were expected to perform in the planning and execution stages. 
We used an example task to help participants understand how to work with the plan-then-execute LLM agent. 
After the onboarding tutorial, participants worked on the selected tasks, which were shuffled at random for every participant to prevent task ordering effects. 
After the participants finished the task batch, we measured their perceived cognitive load using the NASA-TLX questionnaire~\cite{colligan2015cognitive}, their overall trust in the daily assistant using the trust in automation questionnaire~\cite{korber2019theoretical}, and we gathered their feedback on our system (related to planning, execution, and other aspects) using open-ended text.
%%%%%%%%%%%%%%%%%%%%%%%%%%%%%
%  https://docs.google.com/spreadsheets/d/1SpGE0Sf_sl_ypEfHr2NEvBJhZrCC_ci0SyLpdXFCqiE/edit#gid=1692132617
%%%%%%%%%%%%%%%%%%%%%%%%%%%%%

\section{Experiment Study}
\label{sec:exp}

We next experimentally verify 
the efficiency and effectiveness of 
our algorithms. We 
aim to answer three questions: 
\textbf{RQ1}: 
How well can our algorithms improve the performance of models in multiple measures? 
\textbf{RQ2}: 
What is the impact of generation settings, such as data size?
%quality requirement, 
%number of performance measures, 
%and optimization strategies? 
\textbf{RQ3}: 
How fast can they generate skyline sets, and how scalable are they? 
We also illustrate the applications of our approaches with case studies\footnote{
Our codes and datasets are available at 
github.com/wang-mengying/modis}.
%\footnote{Our code is made available at \url{xxx}}. 

\stitle{Datasets.}
We use three sets of tabular datasets: 
kaggle~\cite{KaggleYourHome}, \open~\cite{DataGov}, and \hf~\cite{HuggingFaceAI} (summarized in Table~\ref{tab-data}). 
% We set a base with a universal schema $s_U$ for each dataset by joining attributes from input tables. 

\eat{
\stitle{Datasets}. 
We use the following datasets summarized below:
%in Table~\ref{tab-data}. 
% \warn{No need to use all; refine to what we used.}

\sstab
(1) \kaggle~\cite{KaggleYourHome}, collected from 
a set of tables involving movie information;  %such as \tbf; 
%\mengying{$T_3$}

\sstab
(2) \open: a fraction of a large open public dataset~\cite{DataGov}. We sampled $2K$ tables, involing schools recording, school evaluations and school types, houses recording, housing price in New York and Chicago, geographical location, among others; 
%\mengying{$T_1$, $T_2$, extend from \metam}

\sstab
(3) \hf: a set of 
tables involving Avocado prices and relevant information,  %such as \tbf, 
sampled from Hugging Face~\cite{HuggingFaceAI}.  
%\mengying{$T_4$, find a regression model from HF}
}

% \begin{table}
%     \centering
%     \begin{tabular}{|c|c|c|c|c|}
%     \hline
%         Tasks \& Models  & Source & \# Columns & \# Rows \\ \hline
%        $T_1$: \gbm & \kaggle  & 12 & 3732 \\ \hline
%        $T_2$: \rfh & \open  & 27 & 1178 \\ \hline
%        $T_3$: LRavocado & \hf & 13  & 18249  \\ \hline
%     \end{tabular} \caption{Characteristics of Inputs}
%     \vspace{-4ex}
%    \label{tab-data}
% \end{table}


\begin{table}
    \centering
    \begin{small}
    \begin{tabular}{|c|c|c|c|}
    \hline
       Dataset Sets  & \# tables & \# Columns & \# Rows  \\ \hline
       \kaggle  & 1943 & 33573 & 7317K %& 704.6MB
       \\ 
       \hline
       \open  & 2457 & 71416 & 33296K
       %& 8.58GB
       \\ 
      % \hline
       % \tus  &  &  &  & \\ 
      % \hline
       % \santos  &  &  &  & \\ 
      % \hline
       % \uci  &  &  &  & \\ 
       \hline
       \hf & 255  & 1395 & 10207K 
       %& 3.5GB 
       \\ \hline
       
    \end{tabular}
    \end{small}
    \caption{Characteristics of Datasets}
    \vspace{-5ex}
   \label{tab-data}
      \vspace{-3ex}
\end{table}


\stitle{Tasks and Models.}
A set of tasks are assigned for evaluation. % to predict a target attribute. 
% {\em Trainer} 
We trained: (1) a Gradient Boosting model (\ul{\gbm}) to predict movie grosses using \kaggle for Task $T_1$; 
(2) a Random Forest model (\ul{\rfh}) to classify house prices using \open with the same settings in~\cite{galhotra2023metam} for Task $T_2$; and 
(3) a Logistic Regression model (\ul{LRavocado}) to predict Avocado prices using \hf for Task $T_3$. 
(4) a LightGBM model (\lgc)~\cite{ke2017lightgbm} to classify mental health status using \kaggle for Task $T_4$. 
%To showcase the generality of \modis, 
We also introduced 
task $T5$, a link regression task for
recommendation. This task takes as input a bipartite graph 
between users and products, and links indicate their interaction. 
A LightGCN~\cite{he2020lightgcn} (\lgr), 
a variant of graph neural networks (GNN) optimized for 
fast graph learning, 
is trained 
to predict top-$k$ missing edges in an input bipartite graph 
to suggest products to users. 
A set of $1873$ bipartite graphs is constructed from \kaggle for 
$T_5$.  %where each graph has on average \tbf nodes and \tbf edges. 
The ``augment'' (resp. ``reduct'') operators are  
defined as edge insertions (resp. edge deletions) 
to transform a bipartite graph to another.  
% Scripts are implemented with scikit-learn~\cite{scikit-learn}. 
%$5,660$ nodes and $16,800$ edges.
%\revise{and a set of knowledge graphs \kizoo that includes ML assets information from \kaggle, which consists of $5,660$ nodes and $16,800$ edges.}
%(5) a LightGCN model~\cite{he2020lightgcn} (\lgr) to recommend pre-trained models for a given test set, using \kizoo for Task $T_5$.}
% For a fair comparison, we use the exact same training script for all compared methods.

We use the same training scripts for each task and all methods %(needed by goal-oriented methods) 
%and evaluation 
for a fair comparison.
We assigned measures $\P_1$ through $\P_5$ for tasks $T_1$ to $T_5$, respectively
% For each task, we adopted a group of measures to 
% guide the data discovery
, as summarized in Table~\ref{tab-measures}.
% The first five ($p_{Acc}$, $p_{Tr}$, $p_{F1}$, $p_{MAE}$, $p_{MSE}$) directly quantify models' performance, while the latter two ($p_{Fsc}$, $p_{MI}$) for feature selection~\cite{li2017feature}. 
We also report the size of the data ($p_{DSize}$) in terms of (total $\#$ of rows total $\#$ of columns), excluding attributes with all cells masked.

\eat{
\stitle{Tasks and Models}. 
We have trained the following 
models: 
(1) a random forest models \rfh 
%and 
%\rfs, 
for classifying house price 
(Task $T_1$), 
%and 
%school performance 
%(Task $T_2$), respectively, 
using 
\open and the settings consistently in~\cite{galhotra2023metam, fan2022semantics};   
%(2) more models paired with datasets. 
(2) a Gradient Boosting Model 
(\gbm) for predicting the movies' worldwide gross sale, using \kaggle (Task $T_2$); and 
(3) a regression model 
for predicting Avocado price, 
using \hf (Task $T_3$).  

We trained all these models 
with scikit-learn~\cite{scikit-learn}.
%\warn{add more 
%training settings. }
For a fair 
comparison, we use the original 
training scripts provided by the baseline 
methods and validated that the 
reproduced models have 
consistent performance 
as reported. 
}

\eetitle{Estimator $\E$}. We 
%have evaluated representative estimation models for model estimation, such as Linear Regression, Random Forest, among others~\cite{he2021automl}, and 
adopt MO-GBM~\cite{scikit-learn}  
as a desired model performance estimator. It % can handle multiple outputs and 
outperforms other candidate models even with a simple training set %using the bitmap encoding $s.L$ as input to predict the performance measures $s.\P$ for state $s$, it achieves satisfactory results and 
%outperform other options. 
For example, for $T_1$, MO-GBM performs inference for all objectives on one state in at most $0.2$ seconds, with a small MSE of $0.0003$ when predicting ``Accuracy''. 
\eat{
We trained  
a multi-output Gradient Boosting Model (MO-GBM)~\cite{scikit-learn} 
as our estimator. 
}

\eat{
\mengying{To simplify estimation and improve accuracy, we used the bitmap encoding $s.L$ as input to predict the performance measures $\P_s$ for state $s$ over the given model $M$. 
}
}
% \warn{Give more details}. 
% \warn{Other options include 
% -- give references for high-quality 
% estimators, and perhaps cite Modsnet}.
\eat{which outputs predicted 
values for multiple variables, allowing us to valuate the performance vector for each test with one call.}

\stitle{Algorithms}. 
We implemented the following methods. 

\sstab
(1) \textbf{\underline{MODis}}: Our multi-objective data discovery algorithms, including \apxmodis, \bimodis, and \divmodis. We also implemented \nomodis, a counterpart of \bimodis without correlation-based pruning. 
%\divmodis is built upon \nomodis. 
(2) \underline{\metam}~\cite{galhotra2023metam}: 
a goal-oriented data discovery algorithm
that optimizes a single utility score with consecutive joins of tables. 
We also implemented an extension \metammo, 
by incorporating multiple measures into a single 
linear weighted utility function. 
%that queries a downstream task with a candidate dataset based on dataset utility. Although it attempts to find optimal augmentation, its performance in discovering more relevant data is limited and cannot satisfy users who want more data augmentation.
%https://github.com/TheDataStation/Metam
(3) \underline{\starmie}~\cite{fan2023semantics}: a data discovery method that focuses on table-union search and uses contrastive learning to identify joinable tables.
% A data discovery algorithm with 
% table union search as the main use case. Given a table, it discovers union-able 
% tables by learning feature semantic similarity with 
% contrastive learning. 
%However, it is limited by table-level search and long pre-training time. 
For \metam and \starmie, we used the code from 
original papers. 
(4) \underline{\sklearn}~\cite{scikit-learn}: 
% \kw{SelectFromModel} function in scikit-learn, which selects features automatically with a built-in 
% estimator.
An automated feature selection method 
in scikit-learn's \kw{SelectFromModel}, which recommends important features with a built-in 
estimator.
%\warn{add description.}
% doc-link: https://scikit-learn.org/stable/modules/generated/sklearn.feature_selection.SelectFromModel.html#sklearn.feature_selection.SelectFromModel
(5) \underline{\ho}~\cite{h2o_platform}: an AutoML platform; we used its feature selection module, which fits features and predictors into a linear model.
% an AutoML platform used to automatically optimize machine learning pipelines. We utilized its feature selection module to fit features and predictors into a generalized linear model and then obtained a subset of selected features. 
% \warn{Add what ``Original'' is - we showed it in Fig 7.}

\stitle{Construction of 
$D_U$ and Operators}.
To prepare %universal tables 
universal datasets $D_U$ %as input 
for \modis, we %followed~\cite{galhotra2023metam} 
preprocess \kaggle, \open and \hf 
into joinable tables and construct 
$D_U$ with multi-way joins. 
This results in $D_U$ 
datasets with a size (in terms of 
\# of columns and \# of rows):
$(12, 3732)$, $(27, 1178)$, $(13, 18249)$ 
and $(20, 140700)$, for 
tasks $T_1$ to $T_4$, respectively. 
Specifically, we applied 
$k$-means clustering over 
the active domain of 
each %joinable (common) 
attribute (with a maximum $k$ 
set as $30$), and 
derived equality literals, 
one for each cluster. 
We then compressed the 
input tables by replacing 
rows into tuple clusters, reducing 
the number of rows. 
This pragmatically help us 
avoid starting from large 
$D_U$ by only 
retaining 
the values of interests,   
and still yield desired 
skyline datasets. 
For $T_5$, a 
large bipartite graph 
is constructed 
with a size of 
$(7925, 34)$ (\# of edges, \# of nodes' features). 
The generation of graphs consistently aligns with its table data counterpart, 
by conveniently replacing augment and reduction to their graph counterpart 
that performs link insertions and deletions. 

\eat{
\revise{
%To demonstrate the generalizability of \modis, 
We also include task $T_5$, where the base table is a graph. %The universal table 
The universal dataset $D_U$ for $T_5$ is constructed by augmenting the base table with information from \kizoo, resulting in a size of $(7925, 34)$ (\# of edges, \# of nodes' features). Other settings for $T_5$ follow the default configurations used in other tasks, showcasing \modis's applicability across diverse data modalities.}
\eat{
We identified related attributes based on a loose standard with attributes' names and overlaps to maximize the information included in $s_U$, with the size of $(12, 3732)$, $(27, 1178)$ and $(13, 18249)$.
}
}

\eat{
\warn{add description.}
Distributed and scalable machine learning and predictive analytics platform that allows the building of machine learning models on big data and provides easy productionalization of those models in an enterprise environment.
}


%$p_{Fsc}$,  $p_{MI}$ and $p_{VIF}$ 
%that quantifies 

%that highlight the need of 
%the ``benefit'' to be maximized 
%for data discovery. 
%\warn{more; give a table here}. 

% the baseline methods. 


\eat{
\item
\automl~\cite{}: \warn{add description.} 

\item 
\fselect~\cite{}: \warn{add description.}}

%\end{itemize} 


\begin{table}
    \centering
 \begin{center}
 \begin{small}
     \begin{tabular}{|c|c|c|} \hline
       Notation &  Measures    & Used In  \\ \hline
       $p_{Acc}$   & Model Accuracy &   $\P_1$, $\P_2$, $\P_4$ \\ \hline
       $p_{Tr}$   & Training Time Cost  &   $\P_1$-$\P_4$ \\ \hline
       $p_{F1}$   & $F_1$ score  &   $\P_2$, $\P_4$ \\ \hline
       $p_{AUC}$   & Area under the curve  &   $\P_4$ \\ \hline
       $p_{Nc(n)}$   & NDCG(@n)  &   $\P_5$ \\ \hline
%       $p_{MC}$   & Model Complexity &   $\P_1$ \\ \hline
       $p_{MAE}$, $p_{MSE}$   & Mean Absolute / Squared Error &   $\P_3$\\ \hline
       $p_{Pc(n)}$, $p_{Rc(n)}$   & Precision(@n), Recall(@n) &   $\P_5$\\ 
       \hline \hline
       $p_{Fsc}$   & Fisher Score~\cite{li2017feature} &   $\P_1$, $\P_2$ \\ \hline
       $p_{MI}$   & Mutual Information~\cite{li2017feature,galhotra2023metam} &   $\P_1$, $\P_2$ \\ \hline
       % $p_{VIF}$   & Variance Influence Factor~\cite{li2017feature} &  $\P_1$\\ \hline
 %      $p_{DSize}$   & Size of Created Dataset &  None \\ \hline
     \end{tabular}
     \end{small}
    \caption{Performance Measures}
     \label{tab-measures}
     \end{center}
\vspace{-5ex}
\end{table}






\eat{
We evaluate data discovery algorithms 
in terms of tasks, performance measures 
and test models, as summarized below. 

%We created different tasks, 
%summarized % in Table~\ref{tab-scene}. 
%below. 
\begin{small}
%\begin{table}
    \centering
    \begin{tabular}{|l|c|c|c|c|}  \hline
        \multicolumn{1}{|c|}{Tasks} & Type & Dataset & Model & Perf. \\  \hline
       $T_1$: Movie Gross (C) & Classification  & \kaggle & \eat{\modelthree}\gbm & $\P_1$ \\
%       \hline
%       $T_2:$ School Perform (C) & Classification  & \open & \eat{\modeltwo}\rfs  & $\P_2$ \\
       \hline
       $T_2:$ House Price (C) & Classification  & \open & \eat{\modelone}\rfh  & $\P_2$ \\
       \hline
       $T_3:$ Avocado Price (R) & Regression  & \hf & LRavocado & $\P_3$ \\
       \hline
    \end{tabular}
%    \caption{Task scenarios (Configurations)}
%    \vspace{-4ex}
%    \label{tab-task}
%\end{table}
\end{small}
    
%The baseline methods either does not 
%address performance measures or only 
%consider a single 
%measure. 
%, which are underlined in the table. 
}



\eat{
(1) The first four directly quantify 
a model's performance in terms of accuracy 
($p_{Acc}$ for classification and regression, and both $p_{MAE}$ 
and $p_{MSE}$ for regression) and 
training cost ($p_{Tr}$).  
(2) The latter three ($p_{Fsc}$,  $p_{MI}$ and $p_{VIF}$), generally used in feature 
selection~\cite{li2017feature} quantify the 
statistical relationship 
between a set of input variables 
(features) and a ``target'' 
feature (\eg `House Price' to be classified, or `Avocado Price' to be 
predicted); the larger, 
the better. Among these, 
$p_{MI}$ is also adopted by~\cite{galhotra2023metam} as 
an optimization goal for 
data discovery.  
(3) To evaluate the 
amount of result, we also report the size of the 
data ($p_{DSize}$), in terms of 
(total $\#$ of rows, total $\#$ of features). 
As all baselines only report a single table, 
and \modis report a set of tables, we 
report total size in favor of baselines. 
Here if a column has all cell masked, we 
consider the column reduced and remove it 
from the output table. 

For each tasks in $T_1$-$T_3$, we initialized 
our \modis methods consistently with a 
configuration that specifies 
an original dataset, the matching trained model, 
and the corresponding measures $\P_1$-$\P_3$.
}

\stitle{Evaluation metrics}. 
We adopt the following metrics 
to quantify the effectiveness of data discovery approaches. Denote $D_M$ as an initial dataset, and 
$\D_o$ a set of output datasets from 
a data discovery algorithm. 
%(1) Both \metam and \starmie 
%generate a single dataset $D_o$ 
%for a designated 
%task-oriented metric $p$. By default, 
%$p$ is \tbf for regression task, 
%and \tbf for classification. 
%(2) For \modis algorithms, 
%we set $\P$ to contain the measure $p$ accordingly as 
%one of \tbf or \tbf, with $5$ additional normalized 
%measures: \tbf, \tbf, \tbf, \tbf, and \tbf. 
%Given the output of \modis algorithms $\D_o$, 
%we choose $6$ datasets $D_{p_i}$ such that 
%each has the highest estimated 
%model performance for metric $p_i\in \P$ $(i\in[1,6])$. 
(1) We define the {\em relative improvement} 
$\relp(p)$ for a given 
measure $p$ achieved by a
method as $\frac{M(D_M).p}{M(D_o).p}$.
As all metrics are normalized to be minimized,
the larger $\relp(p)$ is,
the better $D_p$ is in improving $M$ \wrt $p$. 
Here $M(D_M).p$ and $M(D_p).p$ are obtained 
by actual model inference test. 
This allows us to fairly compare all 
methods in terms of the quality of data 
suggestion. 
For efficiency, we compare the time cost of data discovery upon 
receiving a given model or task 
as a ``query''. 

\eat{
\eetitle{Task scenarios}. We created the following 
task scenarios with specified task, model, and performance 
measures (Perf.), summarized in Table~\ref{tab-task}. \warn{Give the table}. Here $\P_1$ - $\P_3$ are 
defined as follows. \warn{No need to use the same 
set of performance metrics. Anything larger than one is good. 
Accordingly for radar graph -- you can have one with Axis of three, four, or five.}
}

\eat{
\eetitle{TUS~\cite{nargesian2018table}}} 
%https://github.com/megagonlabs/starmie

%\mengying{Add one for feature selection, one for AutoML}
%\mengying{\apxmodis: control in 5-40 mins}

\begin{table*}[tb!]
\customsize
\centering
\renewcommand{\arraystretch}{1.05}
\begin{small}
\begin{tabular}{|c|c|c|c|c|c|c||c|c|c|c|}
\hline
$T_2$: House & Original & \metam & \metammo & \starmie & \sklearn & \ho & \apxmodis & \nomodis & \bimodis & \divmodis \\ \hline
$p_{F1}$ & 0.8288 & 0.8510 & 0.8310 & 0.8351 & 0.7825 & 0.8333 & 0.9044 & \ul{\textbf{0.9125}} & \ul{\textbf{0.9125}} & 0.8732 \\ \hline
$p_{Acc}$ & 0.8305 & 0.8322 & 0.8333 & 0.8331 & 0.7826 & 0.8305 & 0.9050 & \ul{\textbf{0.9121}} & \ul{\textbf{0.9121}} & 0.8729 \\ \hline
$p_{Train}$ & 0.2000 & 0.21 & 0.19 & 0.2100 & 0.2000 & 0.2000 & 0.1533 & \ul{\textbf{0.1519}} & \ul{\textbf{0.1519}} & 0.2128 \\ \hline
$p_{F_{sc}}$ & 0.0928 & 0.0889 & 0.0894 & 0.0149 & 0.2472 & 0.0691 & 0.2268 & \ul{\textbf{0.2610}} & \ul{\textbf{0.2610}} & 0.2223 \\ \hline
$p_{MI}$ & 0.126 & 0.1109 & 0.1207 & 0.0243 & \ul{\textit{0.2970}} & 0.1054 & 0.2039 & 0.2018 & 0.2018 & \textbf{0.3164} \\ \hline
Output Size  & (1178, 27)  & (1178, 28)  & (1178, 28)  & (1178, 32)  & (1178, 4)  & (1178, 15)  & (835, 17)  & (797, 17)  & (797, 17)  & (1129, 5)\\ \hline

\hline
\hline

% $T_1$: \gbm & Original & \metam & \metammo & \starmie & \sklearn & \ho & \apxmodis & \nomodis & \bimodis & \divmodis \\ \hline
% $p_{Acc}$ & 0.8560 & 0.8743 & 0.8676 & 0.8606 & 0.8285 & 0.8545 & 0.9291 & \textbf{0.9874} & \ul{\textit{0.9755}} & 0.9427 \\ \hline
% $p_{Train}$ & 1.4775 & 1.6276 & 1.1785 & 1.2643 & \textbf{0.6028} & 0.9692 & 0.9947 & 0.8766 & \ul{\textit{0.8027}} & 0.8803  \\ \hline
% $p_{Fsc}$ & 0.0824 & 0.0497 & 0.0801 & 0.1286 & 0.7392 & 0.3110 & 0.6011 & \ul{\textit{0.7202}} & \textbf{0.9240} & 0.8010\\ \hline
% $p_{MI}$ & 0.0538 & 0.0344 & 0.0522 & 0.1072 & 0.3921 & 0.1759  & \textbf{0.4178} & 0.3377 & 0.3839 & \ul{\textit{0.4165}}\\ \hline
% % $p_{VIF}$ & 1.5831 & 1.9669 & 1.9782 & 1.2980 & 1.6742 & 1.5096 & \ul{\textbf{2.4092}} & \ul{\textit{2.1331}} & 1.7688 & 2.0188 \\ \hline
% Output Size  & (3264, 10) & (3264, 11) & (3264, 11) & (3264, 23) & (3264, 3) & (3264, 8) & (2958, 9) & (1980, 12) & (1835, 11) & (2176, 10) \\ \hline

% \hline
% \hline

$T_4$: Mental & Original & \metam & \metammo & \starmie & \sklearn & \ho & \apxmodis & \nomodis & \bimodis & \divmodis \\ \hline
$p_{Acc}$      & 0.9222 & 0.9468 & 0.9462  & 0.9505 & 0.8839 & 0.9236 & \textbf{0.9532} & 0.9471 & \ul{0.9525} & 0.9471 \\
\hline
$p_{Pc}$     & 0.7940 & 0.7991 & 0.8070 & 0.8106 & 0.6577 & 0.7892 & \textbf{0.8577} & 0.8454 & \ul{0.8549} & 0.8454 \\
\hline
$p_{Rc}$        & 0.7722 & 0.7846 & 0.7959 & 0.8030 & 0.7523 & 0.7879 & \textbf{0.8097} & \ul{0.8092} & 0.8075 & \ul{0.8092} \\
\hline
$p_{F1}$           & 0.7829 & 0.7918 & 0.8014 & 0.8068 & 0.7018 & 0.7885 & \textbf{0.8330} & 0.8269 & \ul{0.8305} & 0.8269 \\
\hline
$p_{AUC}$           & 0.9618 & 0.9757 & 0.9774 & 0.9784 & 0.9326 & 0.9615 & \textbf{0.9792} & 0.9755 & \ul{0.9789} & 0.9755 \\
\hline
$p_{Train}$ & 0.4098 & 0.3198 & 0.4027 & 0.3333 & \textbf{0.2359} & \ul{0.2530} & 0.3327 & 0.2818 & 0.3201 & 0.2818 \\
\hline
Output Size  & ($10^5$, 14) & ($10^5$, 15) & ($10^5$, 15) & ($10^5$, 16) & ($10^5$ 8) & ($10^5$, 8) & (128332, 16) & (116048, 16) & (128332, 17) & (116048, 16) \\
\hline

\end{tabular}
\end{small}
\caption{Comparison of Data Discovery Algorithms in Multi-Objective Setting ($T_2$, $T_4$)}
\label{tab:comparison}
\vspace{-5ex}
\end{table*}


\eat{
\begin{table*}[tb!]
\customsize
\centering
\begin{tabular}
{|c|c|c|c|c|c|c||c|c|c|c|}
\hline
$T_1$: Movie Gross (C) & Original & \metam & \metammo & \starmie & \sklearn & \ho & \apxmodis & \nomodis & \bimodis & \divmodis \\ \hline
%\midrule
%Accuracy
$p_{Acc}$ & 0.8560 & 0.8743 & 0.8676 & 0.8606 & 0.8285 & 0.8545 & 0.9291 & \ul{\textbf{0.9874}} & \ul{\textit{0.9755}} & 0.9427 \\ \hline
%Training Time
$p_{Train}$ & 1.4775 & 1.6276 & 1.1785 & \ul{\textbf{1.2643}} & 0.6028 & 0.9692 & 0.9947 & \ul{\textit{0.8766}} & 0.8027 &  \\ \hline
%$p_{MC}$ & 140.77 & 132.67 & 98.64 & \ul{\textbf{1442.00}} & \ul{\textit{623.71}} & 531.75 & 383.64 & 440.00 & 398.91 \\ \hline
$p_{Fsc}$ & 0.0824 & 0.0497 & 0.0801 & 0.1286 & 0.7392 & 0.3110 & 0.6011 & \ul{\textit{0.7202}} & \ul{\textbf{0.9240}} & 0.8010\\ \hline
$p_{MI}$ & 0.0538 & 0.0344 & 0.0522 & 0.1072 & 0.3921 & 0.1759  &\ul{\textbf{0.4178}} & \ul{\textit{0.3377}} & 0.3839 & 0.4165\\ \hline
$p_{VIF}$ & 1.5831 & 1.9669 & 1.9782 & 1.2980 & 1.6742 & 1.5096 & \ul{\textbf{2.4092}} & \ul{\textit{2.1331}} & 1.7688 & 2.0188 \\ \hline
Output Data Size  & (3264, 10) & (3264, 11) & (3264, 11) & (3264, 23) & (3264, 3) & (3264, 8) & (2958, 9) & (1980, 12) & (1835, 11) & (2176, 10) \\ \hline
%\bottomrule
\end{tabular}
\vspace{0.5ex}
\caption{Comparison of Data Discovery Algorithms in Multi-Objective Setting}
\label{tab:comparison}
\end{table*}
}

\begin{table}[tb!]
\customsize
\centering
\renewcommand{\arraystretch}{1.05}
\begin{small}
\begin{tabular}{|>{\centering\arraybackslash}p{1.32cm}|>{\centering\arraybackslash}p{0.93cm}|>{\centering\arraybackslash}p{1.16cm}|>{\centering\arraybackslash}p{1.05cm}|>{\centering\arraybackslash}p{1.05cm}|>{\centering\arraybackslash}p{1.05cm}|}
\hline
$T_5$: Model & Original 
% & Universal 
& ApxMODis & NOMODis & BiMODis & DivMODis \\
\hline
$p_{Pc_5}$    & 0.7200 
% & 0.7800 
& \textbf{0.8200} & 0.8000 & \textbf{0.8200} & 0.8000 \\
\hline
$p_{Pc_{10}}$  & 0.6600 
% & 0.7800 
& 0.8100 & 0.8000 & \textbf{0.8200} & 0.8000 \\
\hline
$p_{Rc_5}$       & 0.1863 
% & 0.1980 
& \textbf{0.2072} & 0.2022 & \textbf{0.2072} & 0.2022 \\
\hline
$p_{Rc_{10}}$      & 0.3217 
% & 0.3663 
& 0.3866 & 0.3816 & \textbf{0.3977} & 0.3816 \\
\hline
$p_{Nc_5}$        & 0.6923 
% & 0.7243 
& \textbf{0.7935} & 0.7875 & 0.7924 & 0.7875 \\
\hline
$p_{Nc_{10}}$        & 0.6646 
% & 0.7467 
& 0.7976 & 0.7891 & \textbf{0.8033} & 0.7891 \\
\hline
Output Size           & (7925, 0) 
% & (7925, 34) 
& (5826, 30) & (1966, 6) & (2869, 4) & (1966, 6) \\
\hline
\end{tabular}
\end{small}
\caption{Comparison of \modis Methods on $T_5$}
\label{tab:modsnet}
\vspace{-6ex}
\end{table}

\stitle{Exp-1: Effectiveness}. 
We first evaluate \modis methods 
over five tasks. 
Results for $T_1$ and $T_3$ are shown
% in radar graphs 
in Fig.~\ref{fig:mo-eff} (the outer, the better). 
% Results for $T_1$, which is used for movie grosses prediction,
% $T_2$, which use the same data repository and setting from \metam, 
% $T_2$, which is used for house price prediction,
% and $T_4$,  which is used for depression cases classification,
While results for $T_2$ and $T_4$
are presented in Table~\ref{tab:comparison}.
Results for $T_5$ are in Table~\ref{tab:modsnet}.
We also report the model performance over 
 the input tables as a ``yardstick'' 
 (``Original'') for all methods. 
As all baselines output a single table, to compare \modis algorithms, we select the table in the Skyline set with the best estimated $p_{Acc}$, $P_{F1}$, $P_{MSE}$, $p_{Acc}$ and $p_{Pc_5}$ for $T_1$ to $T_5$, respectively. 
As \metam optimizes a single utility score, we choose the same measure for each task as the utility. 
%We also apply it to \metammo by combining all measures into the utility; and 
We apply model inference to all the output tables to report actual performance values. 
We have the following observations. 

\begin{figure}[tb!]
\vspace{-2ex}
\centerline{\includegraphics[width =0.43\textwidth]{fig/radar}}
\centering
\vspace{-1ex}
\caption{Effectiveness: Multiple Measures}
\vspace{-3ex}
\label{fig:mo-eff}
\end{figure}

\sstab
(1) \modis algorithms outperform all the baselines in all tasks. 
As shown in Table~\ref{tab:comparison}, for example, for $T_4$, the datasets that bear best $p_{Acc}$ and the second best are returned by \apxmodis (0.9535) and \bimodis (0.9525), respectively, and all \modis methods generated datasets that achieve $0.87$ on $p_{F1}$ in $T_2$. 

\sstab
(2) Over the same dataset and for other measures, \modis algorithms outperform the baselines in most cases. 
% In particular, \bimodis, \divmodis, and \nomodis provide top results for multiple measures. 
For example, in $T_1$, the result datasets that most optimize $p_{Fsc}$ and $p_{MI}$ are obtained by \bimodis and \apxmodis, respectively; also in $T_2$  and $T_3$, \nomodis and \bimodis show absolute dominance in most measures. 
%\metam optimizes the chosen measure as utility score with one run. %Another goal-driven method, 
%\revise{While other baselines are limited to tabular data, \modis can effectively handle more data modalities, \eg graph data in $T_5$.}
Table~\ref{tab:modsnet} also verifies 
that \modis easily generalizes to suggest 
graph data for GNN-based analytics, beyond tabular data.  

\sstab
(3) Methods with data  
augmentation (\eg \metam and \starmie) enriches 
data to improve model accuracy, at a cost of training time, while feature selection methods (\eg \sklearn and \ho) reduce data at the cost of accuracy with improved training efficiency. \modis 
methods are able to balance these trade-offs better 
by {\em explicitly} performing multi-objective optimization. 
For example, $p_{Acc}$ and $p_{Train}$ in $T_4$, The best result for training cost (0.2359s) 
is contributed from \sklearn, yet at a cost of 
lowest model accuracy (0.8839). 

We also compared $p_{Acc}$ on $T_4$ with HydraGAN, a generative data augmentation method, which achieves $0.9355$ with $330$ rows but fell short of data discovery methods. Increasing the number of rows further reduced performance, reflecting the limitations of generative approaches in this context, which cannot utilize verified external data sources, and synthetic data often lacks inherent reliability and contextual relevance of discovered data.
%As \modis methods are able to 
%optimize various measures (among others), 
%by making flexible decisions to augment or 
%reduce data, 
%they are able to construct data that lead to improved 
%accuracy, and smaller training overhead,
%compared with baselines. 
% More results are shown in full version~\cite{full}.

\eat{
\stitle{Exp-1: Effectiveness with Single Measure}. 
Our first experiment evaluates all algorithms in evaluating how well the model's performance can be improved over the dataset(s) they created. As $p_{Acc}$ is the single measure considered by~\metam, and all baseline produce a single table, we (1) compare \modis 
algorithms by selecting the table in the 
Skyline set with best estimated $p_{Acc}$, 
and (2) apply model inference to 
all the datasets, to report the actual %$p_{Acc}$. 
measurement values. 
We show the results for $T_3$ in 
Table~\ref{tab:comparison} (``Original'' 
refers to the measures over the input dataset). 
We find the following. 

\sstab
(1) \modis algorithms outperform 
all the baselines in creating a 
dataset to improve the performance 
in terms of $p_{Acc}$. 
The one with best $p_{Acc}$ and second best 
is obtained by \bimodis and \nomodis, 
respectively, and all \modis methods 
finds data for which $p_{Acc}$ achieves 
$0.94$. 

\sstab
(2) Over the same dataset and for other 
measures, \modis algorithms still outperforms 
the baselines in most cases. 
For example, the result datasets that 
optimize $p_{Fsc}$, $p_{MI}$ and $p_{VIF}$ 
are obtained by \apxmodis, \nomodis and \divmodis, 
respectively; and \bimodis finds a dataset 
that achieves three second-best results 
in $p_{Train}$, $p_{Acc}$ and $p_{Fsc}$. 
This verifies their ability in 
optimize data discovery towards multiple measures 
simultaneously. 

\sstab
(3) All baseline methods perform data augmentation or 
feature selection that leads to a single 
table. The data augmentation 
methods (\metam, \starmie) mainly include more features 
to improve accuracy; and feature selection (\sklearn and \ho) reduce 
them at a cost of accuracy but improved training cost. \modis 
methods are able to balance these trade-offs better 
by {\em explicitly} performing multi-objective optimization. 
Consider $p_{Acc}$ and $p_{Train}$. 
The best result for training cost 
is contributed from \sklearn, yet at a cost of 
lowest model accuracy. As \modis methods are able to 
optimize both measures (among others), 
by making flexible decision to augment 
with new features or 
reduce cells and tuples to make the data smaller, 
they are able to find data with improved 
accuracy as well as smaller training cost, 
compared with baselines. 

\sstab
(4) Despite $p_{Acc}$ is a first-class citizen 
in this comparison, not all baselines improve 
it (given its value over ``Origin'') significantly, except \starmie.   
Yet \starmie improves accuracy at a cost of 
including the most number of 
features ($13$ new ones). Feature selection 
methods (\sklearn and \ho) achieved better 
result on accuracy with much less number of 
features, and consistently showing better 
results in feature correlation measures 
in terms of $p_{Fsc}$, $p_{MI}$ and $p_{VIF}$. 
On the other hand, \modis methods 
{\em explicitly} included these into optimization 
scope with a multi-objective estimator, 
and are able to improve accuracy without 
introducing many new attributes. 
%In fact, we found that \tbf finds 
%dataset at size of onl {\tbf,\tbf} that achieve an 
%accuracy at \tbf. 



%\eetitle{Effectiveness: single performance measure}. 
%\mengying{Add a table}
% We can also be comparable in single-objective

\stitle{Exp-2: Effectiveness with multiple measures}. 
We next evaluate \modis algorithms, \metam and \starmie, using multiple measures in $T_1$ and $T_3$. For each measure $p$ and an  
algorithm, we choose the dataset $D$
with the best estimated measure of $p$ it generates. 
We then retrain the model using $D$ 
to get the true measurement. 
We normalize all the values into 
a same range. 
%relative improvement \kw{rImp(p)}, 
%This gives us a radar graph, with 
% the larger, the better. 
The results are illustrated as radar graphs in Fig.~\ref{fig:mo-eff}. 
The lines ``Original'' 
mark the values of the 
measures in the original data. 

In general, \modis algorithms are able to create  
datasets that generally improve a model in a balanced performance. In particular, \bimodis, \divmodis and 
\nomodis provide top results for multiple measures. \metam 
is optimized to provide good results for a single 
measure, such as accuracy in $T_1$. \starmie 
is not specifically optimized for 
optimizing measures, and provides a balanced 
performance in $T_2$. \apxmodis provides 
in particular better results over $p_{VIF}$, 
a measure for feature correlation, with a possible 
reason that it performs more localized 
reduction only operations that is closer to 
feature selection process. 
}
\eat{
\begin{figure}[tb!]
\addtolength{\subfigcapskip}{-0.08in}
\begin{center}
\subfigure[Task 1 (Classification): $\P_1$]{\label{fig:T1}
{\includegraphics[scale=0.35]{./fig/movie_radar_chart.png}}
} 
%\quad
\subfigure[Task 3 (Regression): $\P_3$]{\label{fig:T3}
{\includegraphics[scale=0.35]{./fig/Avocado_radar_chart.png}}
}
\end{center}
\vspace{-2ex}
\caption{Effectiveness: Multi-Objective Optimization\label{fig:mo-eff}}
\vspace{-3ex}
\end{figure}
}

\begin{figure}[tb!]
\centerline{\includegraphics[width =0.5\textwidth]{fig/Exp1/effective}}
\centering
\vspace{-1ex}
\caption{Effectiveness: Impact of Factors}
 \vspace{-4ex}
\label{fig-effective}
\end{figure}


% \begin{figure}[tb!]
% \addtolength{\subfigcapskip}{-0.04in}
% %\vspace{-1.5ex}
% \begin{center}
% \subfigure[$T_1$: $P_{acc}$ vs. $\epsilon$]{\label{fig-eff-t1-acc-epsilon}
% {\includegraphics[width=3.9cm,height=3.2cm]{./fig/Exp1/effec1}}}
% \quad
% \subfigure[$T_1$: $P_{acc}$ vs. Path length]{\label{fig-eff-t1-acc-maxl}
% {\includegraphics[width=3.9cm,height=3.2cm]{./fig/Exp1/effec2}}}
% \quad
% \subfigure[$T_3$: $P_{MSE}$ vs. $\epsilon$]{\label{fig-eff-t4-mse-epsilon}
% {\includegraphics[width=3.9cm,height=3.2cm]{./fig/Exp1/effec3}}}
% \quad
% \subfigure[$T_3$: $P_{MSE}$ vs. Path length]{\label{fig-eff-t4-mse-maxl}
% {\includegraphics[width=3.9cm,height=3.2cm]{./fig/Exp1/effec4}}}
% \end{center}
% \vspace{-2ex}
% \caption{Effectiveness with Varying Factors \label{fig-eff-factor}}
% \vspace{-3ex}
% \end{figure}

\stitle{Exp-2: Impact factors}. 
We next investigate the \modis methods
under the impact of two factors: $\epsilon$ and the maximum path length (\maxl), as well as the impact of $\alpha$ on \divmodis. 
% and bound of 
% valuated states $N$. 

\eetitle{Varying $\epsilon$}.
Fixing $\maxl$ = 6, we varied $\epsilon$ from $0.5$ to $0.1$ for $T_1$. 
As shown in Fig.~\ref{fig-effective}(a), 
\modis algorithms are able to improve the 
model in $p_{acc}$ better with smaller $\epsilon$, 
as they all ensure to output a $\epsilon$-Skyline set that 
better approximate a Skyline set when $\epsilon$ is set to be 
smaller. In all cases, they achieve a relative improvement 
$\relp(p_{Acc})$ at least 1.07. 
\bimodis and \nomodis perform better 
in recognizing better solutions from both ends in 
reduction and augmentation as smaller $\epsilon$ is enforced. \apxmodis, 
with reduction only, is less sensitive to 
the change of $\epsilon$ due to that larger $\epsilon$ 
may ``trap'' it to  local 
optimal sets from one end. Adding diversification (\divmodis) is able to strike a balance between \apxmodis and \bimodis by enforcing to choose difference datasets out of local 
optimal sets, thus 
improving \apxmodis for smaller $\epsilon$. We choose a smaller range of $\epsilon$ for $T_2$ in Fig.~\ref{fig-effective}(c), as the variance of $p_{F1}$ is small.
As $\epsilon$ varies from $0.1$ to $0.02$, 
\nomodis 
% is able to generate datasets that 
improves F1 score from $0.84$ to $0.91$.

\eetitle{Varying $\maxl$}.  Fixing $\epsilon$ = 0.1, we varied $\maxl$ from $2$ to $6$. Fig.~\ref{fig-effective}(b, d) tells us 
that all \modis algorithms improve the 
task performance
% accuracy of 
% classification better 
for more rounds of processing.  Specifically, \bimodis and \nomodis benefit most 
as bi-directional search allows both to find 
better solution from wider search space as $\maxl$ becomes larger. \apxmodis is less sensitive, 
as the reduction strategy from dense tables 
incurs smaller loss in accuracy.
\divmodis finds datasets that ensure 
best model accuracy when $\maxl$ = 5, yet may 
``lose chance'' to maintain the accuracy, due to 
that the diversification step may update 
the Skyline set with 
less optimal but more different counterparts in 
future levels (\eg when $\maxl$ = $6$).  

% \eetitle{$T_2$: F1 vs. $\epsilon$ and $\maxl$}. Using \open, we report the impact of $\epsilon$ and $\maxl$ for $T_2$ in Fig.~\ref{fig-effective}(c) and (d). 
% The results are consistent with their counterparts 
% in Fig.~\ref{fig-effective}. %(a) and (b). 
% We choose a smaller range of $\epsilon$, as the variance of $p_{F1}$ is small.
% As $\epsilon$ varies from $0.1$ to $0.02$ (with $\maxl$ = $6$), 
% \nomodis 
% % is able to generate datasets that 
% improves F1 score from $0.84$ to $0.91$.
% by $1.5$ times. 
%With $\epsilon$ fixed as $0.1$ and by varying $\maxl$ to $6$, 
%\tbf improves $p_{mse}$ by $\tbf$ times. 

%\mengying{Impacts of $\epsilon$ and maximum length, v.s. accuracy for a classification task and RMSE for a regression task.}

\eetitle{Varying $\alpha$ in \divmodis}.
We demonstrate the effectiveness of \divmodis by adjusting $\alpha$.
A smaller $\alpha$ prioritizes performance, while a larger $\alpha$ emphasizes content diversity, measured by hamming distance.
Fig.~\ref{fig:divmodis}(a)  illustrates \textit{Performance Diversity}, where smaller 
$\alpha$ results in a wider accuracy range with a balanced and stable distribution. Both the mean and median remain centered. As $\alpha$ increases, the accuracy distribution narrows and shifts toward higher values, reflecting the dominance of high-accuracy datasets in the Skyline set.
Fig.~\ref{fig:divmodis}(b) verifies the impact of \textit{Content Diversity}, visualized as the percentage contribution of each \ad.  Larger 
$\alpha$ leads to more evenly distributed contributions. 
The standard deviation values above the heatmap quantify this trend, showing a consistent decrease as $\alpha$ increases, indicating improved balance.

% Fig.~\ref{fig:divmodis}(a) shows ...., at a smaller $\alpha$, it shows a larger range of Accuracy, and a more balanced spread, with both medium and mean are centered and stable, rather than being skewed towards higher values at a larger $\alpha$.
% Fig.~\ref{fig:divmodis}(b) shows ...,
% the distribution of .. is more, balance, as quantified by the Standerd divaition, which is continues lower with a larger $\aplpha$.

\begin{figure}[tb!]
\centerline{\includegraphics[width =0.5\textwidth]{fig/divmodis}}
\centering
\vspace{-1ex}
\caption{Impact of $\alpha$ for \divmodis}
\vspace{-2ex}
\label{fig:divmodis}
\end{figure}

\eat{
\begin{figure}[tb!]
\centerline{\includegraphics[width =0.3\textwidth]{./fig/blank.eps}}
\centering
\caption{Impact of factors}
 \vspace{-1ex}
\label{fig:factors}
\end{figure}
}

\eat{
\mengying{As a method for Table Union Search(TUS), Starmie is effective at finding related tables in a data lake, \tbf. However, its algorithm does not consider measures from a downstream data science task. So it is in expect that \tbf. This highlights the importance of goal-driven data discovery.}
}


\begin{figure}[tb!]
\centerline{\includegraphics[width =0.5\textwidth]{fig/Exp2/t1}}
\centering
\vspace{-1ex}
\caption{Efficiency and Scalabilitiy}
\vspace{-5ex}
\label{fig-efficiency}
\end{figure}

% \begin{figure}[tb!]
% \addtolength{\subfigcapskip}{-0.04in}
% %\vspace{-1.5ex}
% \begin{center}
% \subfigure[$T_1$: Varying $\epsilon$]{\label{fig-effe-t1-epsilon}
% {\includegraphics[width=3.9cm,height=3.2cm]{./fig/Exp2/effi1}}}
% \quad
% \subfigure[$T_1$: Varying $\maxl$]{\label{fig-effe-t1-maxl}
% {\includegraphics[width=3.9cm,height=3.2cm]{./fig/Exp2/effi2}}}
% \quad
% \subfigure[$T_3$:Varying $\epsilon$]{\label{fig-effe-t3-epsilon}
% {\includegraphics[width=3.9cm,height=3.2cm]{./fig/Exp2/effi3}}}
% \quad
% \subfigure[$T_3$: Varying $\maxl$]{\label{fig-effe-t3-maxl}
% {\includegraphics[width=3.9cm,height=3.2cm]{./fig/Exp2/effi4}}}
% \end{center}
% \vspace{-2ex}
% \caption{Efficiency with Varying Factors 
% \label{fig-eff-factor}}
% \vspace{-3ex}
% \end{figure}

\stitle{Exp-3: Efficiency and Scalibility}. 
% To answer RQ3, we next evaluate \modis' efficiency and stability for $T_1$ on \kaggle, 
% considering two major factors $\epsilon$ and $\maxl$.
We next report the 
the efficiency of \modis algorithms 
for task $T_1$ and $T_3$ over \kaggle and \hf, respectively, and 
the impact of factors 
% $\#$ of total attributes, 
% size of the largest active domain $|\ad_m|$, 
$\epsilon$ and $\maxl$. 
We also evaluate their scalability for $T_1$ and $T_5$ in terms of input size.
%\eetitle{Varying $\#$ of attributes}. 
%As shown in Fig.~\ref{fig-effe-t1-epsilon}, 
%\warn{more}.  

\eetitle{Efficiency: Varying $\epsilon$}. 
%Using the same setting as in its effectiveness 
%counterpart (Fig.~\ref{fig-eff-t1-acc-epsilon}), 
%we report the efficiency of \modis algorithms. 
Fixing $\maxl$ = $6$ and varying $\epsilon$ from $0.1$ to $0.5$, 
Fig.~\ref{fig-efficiency} (a)
%, c) 
verifies the following. 
(1) \bimodis, \nomodis and \divmodis take less time as $\epsilon$ increases, as a larger $\epsilon$ provides more chance to prune unnecessary valuations. 
\divmodis has a comparable performance 
with \nomodis, as it mainly benefits from 
the bi-directional strategy, which exploits early pruning and a stream-style placement strategy. 
%On average, \bimodis and \divmodis outperforms 
%\nomodis by \tbf times. 
(2) As shown in Fig.~\ref{fig-efficiency}(a), for  $T_1$, \bimodis, \nomodis, and \divmodis are 2.5, 2, and 2 times faster than \apxmodis on average, respectively. 
\apxmodis takes longer time to explore a larger  universal table with reduct operators. 
It is insensitive to $\epsilon$. We observe that its search from the ``data rich'' end   
may converge faster at high-quality 
$\epsilon$-Skyline sets. 


\eat{
This is because in both cases, 
(1) there are more states with non-$\epsilon$-dominance relation 
to existing solution to be resolved; 
and (2) there are more state nodes to be valuated. 
On the other hand, \apxmodis is most sensitive to $\maxl$ due to 
rapid growth of search space, and \bimodis is much less sensitive to $\maxl$ 
as it mitigates the impact better with bi-directional strategy. 
}
\eetitle{Efficiency: Varying $\maxl$}. 
Fixing $\epsilon$ = 0.2 for task $T_1$ and $\epsilon$ = 0.1 for task $T_3$, we varied $\maxl$ from $2$ to $6$, all \modis algorithms take longer as $\maxl$ increases, as shown in Fig.~\ref{fig-efficiency} (b). 
%d)}
Indeed, larger $\maxl$ results in more states to be valuated, and more non-$\epsilon$-dominance relation to be resolved. \apxmodis is sensitive to $\maxl$  due to the rapid growth of the search space. In contrast, \bimodis mitigates the impact with bi-directional strategy and effective pruning.

\eat{
\eetitle{$T_3$: Varying $\epsilon$ and $\maxl$}. 
We report the efficiency of our algorithms 
for regression task. Our observation is 
consistent with their counterparts for 
$T_1$. This verifies that the efficiency  
of our approach is not very sensitive to
the type of learning tasks or models.
}

\eat{
\begin{figure}[tb!]
\centerline{\includegraphics[width =0.5\textwidth]{../../fig/Exp2/scalability}}
\centering
\vspace{-1ex}
\caption{\revise{Scalability with Input Data Size}}
\vspace{-4ex}
\label{fig-scalability}
\end{figure}
}

\eetitle{Scalability}. We varied the number of total 
attributes $|A|$ and size of the largest active domain $|\ad|$. We perform $k$-means clustering over the tuples of the universal table with $k = |\ad|$, and extended operators with range queries to control $|\ad|$. %Fig.~\ref{fig-scalability} 
Fig.~\ref{fig-efficiency} (c) and (d) show that all \modis algorithms take more time for larger $|A|$ and $|\ad|$. \bimodis scales best due to 
bi-directional strategy. 
\divmodis remains more efficient 
than \apxmodis, indicating affordable 
overhead from diversification.

While our algorithms scale well with $\vert A \vert$ and $\vert \ad \vert$, high-dimensional datasets may present challenges due to the search space growth. Dimensionality reduction such as PCA or feature selection, or correlation-based pruning (to identify and eliminate highly correlated or redundant features), can be tailored to specific tasks to mitigate these challenges.
%For $T_5$, leveraging the graph's structure, we reduced the input feature space from 34 to 10 by aggregating attributes from similar types of relations, 
% in \kizoo,
%such as combining multiple training records of an ML model, while preserving full information. 
% Future work will explore integrating these methods to further improve efficiency in high-dimensional settings.

%This indicates \modis in maintaining scalability and managing larger datasets.
% Due to limited space, we 
% report the scalability results in~\cite{full}.


% Report our methods in figure, baselines' in text.



\eat{
\begin{figure}[tb!]
\centerline{\includegraphics[width =0.5\textwidth]{./fig/movie_radar_chart.png}}
\centering
\caption{Efficiency}
 \vspace{-1ex}
\label{fig:efficiency}
\end{figure}
}

\vspace{1ex}
\stitle{Exp-4: Case study}. We next report two real-world case studies to illustrate the application scenarios of \modis. 
% \footnote{Cases are made anonymous to preserve double-blind review.} 

\eetitle{(1) ``Find data with models''}. A material science team trained a random forest-based classifier to identify peaks in 2D X-ray diffraction data. They seek more datasets to improve the model's {\em accuracy, training cost, and F1 score} for downstream fine-tuning. Original X-ray datasets and models are uploaded to a crowd-sourced X-ray data platform we deployed~\cite{wang2022crux} with 
best performance of $\textless0.6435, 3.2, 0.77\textgreater$.
Within available X-ray datasets, \bimodis created three datasets $\{D_1, D_2, D_3\}$ and achieved the best  performance of 0.987, 2.88, and 0.91, respectively. We set \metam to optimize F1-score, and achieved a performance score of $\textless 0.972, 3.51, 0.89\textgreater$ over its output dataset.
Fig.~\ref{fig:cases} illustrates 
such a case that is manually validated with ground-truth from 
a third-party institution. 


\eetitle{(2) Generating test data for model evaluation}. 
We configure \modis algorithms to generate test datasets for model benchmarking, where specific performance criteria can be posed~\cite{ventura2021expand}. Utilizing a trained scientific image classifier from \kaggle, and a pool of image feature datasets $\D$ from \hf with $75$ tables, $768$ columns, and over $1000$ rows. 
We request \bimodis to generate  
datasets over which the classifier demonstrates: ``accuracy $>$ 0.85'' and ``training cost $<$ 30s.'' 
\bimodis successfully generated $3$ datasets to be chosen from within $15$ seconds, with performance $\textless0.95, 0.27\textgreater$, $\textless0.94, 0.26\textgreater$ and $\textless0.90, 0.25\textgreater$, as in Fig.~\ref{fig:cases}.


\begin{figure}[tb!]
\centerline{\includegraphics[width=\linewidth]{fig/case_study}}
\centering
\caption{Case 1 (left): Discover Datasets for Materials Peak Classification Analysis. Case 2 (right): Test Data Generation for Model Performance Benchmarking}
\vspace{-2ex}
\label{fig:cases}
\end{figure}

% Due to limited space, 
% We report the details of 
% more complementary tests and impact of factors to \modis in 
% full version~\cite{full}. 


% A material science team has a trained random forest-based classifier to recognize peaks in a 2-dimensional angle-intensive X-ray diffraction data. They hope to find good additional datasets for which the model has improved performance in terms of $F_1$ score, training cost, and accuracy for downstream fine-tuning. The team has uploaded their original X-ray datasets and models to a crowd-sourced X-ray data collection platform we have deployed, with a performance vector $\textless 0.6435, 3.2, 0.77 \textgreater$. 
% Over a set of shared 
% X-ray diffraction datasets from other facilities, \bimodis created three new datasets 
% $\{D_1, D_2, D_3\}$, which optimizes the model in each 
% measure to 0.975, 3.07, and 0.89, respectively. 
% A test 
% of the model over the datasets is illustrated in 
% Fig.~\ref{fig-case}, which is manually validated to 
% be accurate with ground-truth given by a database provided by 
% a third-party international institution. 


% In the second case study, we hows that our configurable \modis paradigm readily fits the need for generating data (from scratch) for model benchmarking with required performances~\cite{vicente2022benchmark}. 
% Given a trained scientific image classifier from \kaggle, we processed a pool of image features $\D$ from \hf with 75 tables and, in total, 768 columns and more than 1000 rows as data sources. We set the ranges of training time requirement to be ``accuracy > 0.85'', ``training cost <30s''.
% %``Fisher score''>0.45, ``MI''>0.38, 
% %and ``VIF<1.3''. 
% By setting these as {\em hard constraints} for \bimodis, 
% we found that it outputs a set of 5 datasets within $30$ seconds, over which the model's record on each measure is 0.95 in accuracy and 27s in training costs. This maps to 
% a set of test images that serve as better 
% training or testing data, with an example illustrated in Fig.~\ref{fig:cases}. 

\eat{
Given a trained regression model 
\kw{LRavocado}, we collect a pool of datasets $\D$ from \hf with \tbf 
tables and in total \tbf columns and \tbf rows as data sources. 
We set the ranges of training time requirement to be ``<1s'', ``accuracy'' >0.88, ``Fisher score''>0.45, ``MI''>0.38, 
and ``VIF<1.3''. By setting these as {\em hard constraints} for \bimodis, 
we found that it outputs a set of \tbf datasets within $30$ seconds, over which the model's record on each measure is \tbf, \tbf and \tbf. 
}
 
\eat{\mengying{Our algorithm generates a set of recommended datasets based on a model and user-defined metrics with expected ranges in just one round.}
}


\section{Conclusion}
In this work, we introduce \gls{myrag}, an advanced \gls{qa} system that dynamically constructs \gls{mkg} while integrating sophisticated reasoning and external domain-specific search tools. The model exhibits significant improvements in accuracy and reasoning capabilities, particularly for medical question-answering tasks, outperforming other approaches of similar model size or 10 to 100 times larger. Using structured knowledge representations and advanced reasoning frameworks, our approach establishes a new benchmark for \gls{qa} in highly competitive and highly evolving domains such as medicine.

\section{Limitations}

Despite \gls{myrag} advancements, our approach has certain limitations. Firstly it relies on external search tools to introduce latency during the creation of \gls{mkg}. However, this occurs only once, when the \gls{mkg} is built from scratch for the first time. Additionally, while the model performs exceptionally well in medical domains, its applicability to non-medical tasks remains unexplored. 


Another limitation is the need for structured, authoritative sources of medical knowledge. Currently, \gls{myrag} retrieves information from diverse sources, including research articles and medical textbooks. However, as emphasized in clinical decision-making, treatment guidelines serve as essential references for standardized diagnosis and treatment protocols \cite{hager2024evaluation}. Future work on \gls{myrag} should focus on integrating structured access to these sources to ensure compliance with evidence-based medicine.


\section{Ethics Statement}
The development of \gls{llm}s for medical \gls{qa} requires careful ethical consideration due to risks of inaccuracy and bias. Ensuring the reliability of retrieved content is crucial, especially when integrating external knowledge sources. To mitigate these risks, we implement a confidence scoring mechanism into the \gls{mkg} to validate the information. However, bias detection and mitigation remain active research areas.

\section{Conclusion}
This work empirically studied human-AI collaboration based on plan-then-execute LLM agents. 
Adopting such LLM agents in various everyday scenarios, % of different task characteristics (\ie risk perception and plan quality), 
we analyzed the impact of user involvement in the planning and execution stages on user trust and overall task performance. 
We provide various interactions in each stage to help users fix imperfect plans and modify execution outcomes. 
Our results suggest that the LLM agents can provide plausible text plans to cover task requirements, which can be convincingly wrong. 
As a result, users develop uncalibrated trust in the planning and execution outcomes, and user involvement in the planning and execution stages fails to calibrate user trust (\textbf{RQ1}). 
We also found that the plan quality substantially affects the subsequent execution accuracy. 
Thus, when user involvement in planning can fix imperfect plans, the overall task performance (\ie plan quality, accuracy of action sequence, and execution accuracy) gets improved. 
However, user involvement in planning can also harm task plan quality where the original plan is good to begin with. As a result, the LLM agents demonstrate worse task performance in these tasks. 
In contrast, %\glcomment{another phrase here?}
user involvement in execution brings a more stable positive impact on task performance (\textbf{RQ2}). 
Our results suggest that plausible but wrong LLM outcomes can be detrimental to user trust calibration and overall task performance. 
We discussed the impact of convincingly wrong LLM outcomes and provided potential solutions and insights for future work. 
Furthermore, we synthesized key insights for better control and effective collaboration with plan-then-execute LLM agents. \revise{We also shed light on opportunities to design flexible collaborative workflows with human oversight for effective collaboration with LLM agents.} %  \glcomment{shall we rephrase a bit for the last sent? As we add one new implication paragraph, `human oversight and more flexible collaborative workflow'. we may replace the previous sent with `synthesized key insights about leveraging human oversight and more flexible collaborative workflow for better control and effective collaboration with LLM agents'}

Our results indicate that user involvement in the LLM agent workflow can be important in ensuring reliable task outcomes. 
Future work can further investigate how to detect and handle plausible but imperfect LLM outcomes and design effective interventions to fix such problems. 
We hope that our key findings and implications reported in this work will inspire further research on human-AI collaboration with LLM agents.

%%
%% The acknowledgments section is defined using the "acks" environment
%% (and NOT an unnumbered section). This ensures the proper
%% identification of the section in the article metadata, and the
%% consistent spelling of the heading.
\begin{acks}
This work was partially supported by the Delft Design@Scale AI Lab, the 4TU.CEE UNCAGE project, the Convergence Flagship ``ProtectMe'' project, by the Australian Research Council (ARC) Training Centre for Information Resilience (Grant No. IC200100022), and by an ARC Future Fellowship Project (Grant No. FT240100022). We made use of the Dutch national e-infrastructure with the support of the SURF Cooperative using grant no. EINF-5571 and EINF-9738. We finally thank all participants from Prolific and experts from our department.
\end{acks}

%%
%% The next two lines define the bibliography style to be used, and
%% the bibliography file.
\bibliographystyle{ACM-Reference-Format}
\bibliography{LLM_Agent}

%%
%% If your work has an appendix, this is the place to put it.
% \appendix
\appendix

\section*{Appendix}

\section{Prompts}\label{app:prompts}
\subsection{Textual Description}\label{app:img_to_text_prompt}
\begin{quote}
    {\small
    \texttt{Create a short, descriptive persona for the person in the image. Describe them using only the following details: their age, gender, facial expression or mood, attire, any tools or items they’re holding, their work environment, the nature of their job, and their connection to the area and location. Avoid taking creative liberties beyond these details, only using details that can be inferred from the image, while aiming for a realistic portrayal that gives insight into their daily life, professional dedication, and overall demeanor. For example: Meet a skilled construction worker in his late 30s, living in Sydney, Australia. Every day, he heads out to work in one of the city's bustling urban sites, often with a view of iconic landmarks like the Sydney Opera House and Sydney Harbour Bridge. Outfitted in essential safety gear—a hard hat, reflective vest, and a set of versatile tools—he’s well-prepared for a physically demanding role that demands focus and precision. His job involves a blend of construction and maintenance tasks, requiring him to pay close attention to safety protocols and collaborate with a team. Confident and professional in his work, he takes pride in contributing to the infrastructure and vibrant aesthetic of Sydney, adding to the city’s ever-evolving landscape with each project.
    }}
\end{quote}
    

\begin{table*}[t]
    \centering
    \caption{A complete list of personas annotated for their attribute categories.}
    \label{tab:personalist}
    \resizebox{1.0\linewidth}{!}{
    \begin{tabular}{l c c c c}
        \toprule
        Persona & Age & Gender & Occupation & Location \\
        \midrule
        A 25-year-old female nurse from Toronto & 25-34 & female & healthcare \& education & Strong Developed Economies \\ 
        A 41-year-old female electrician from Sydney & 35-44 & female & manual labor & Strong Developed Economies \\ 
        A 36-year-old male electrician from Houston & 35-44 & male & manual labor & Largest Global Economies \\ 
        A 29-year-old female police officer from New York & 25-34 & female & public safety & Largest Global Economies \\ 
        A 28-year-old female police officer from London & 25-34 & female & public safety & Largest Global Economies \\ 
        A 35-year-old male chef from Paris & 35-44 & male & hospitality & Largest Global Economies \\ 
        A 32-year-old female chef from Rome & 25-34 & female & hospitality & Strong Developed Economies \\ 
        A 50-year-old male farmer from Sao Paulo & 45-54 & male & manual labor & Emerging Markets \\ 
        A 40-year-old female farmer from Nairobi & 35-44 & female & manual labor & Emerging Markets \\ 
        A 27-year-old female mechanic from Berlin & 25-34 & female & manual labor & Largest Global Economies \\ 
        A 28-year-old female pilot from Los Angeles & 25-34 & female & transportation & Largest Global Economies \\ 
        A 28-year-old female pilot from Vancouver & 25-34 & female & transportation & Strong Developed Economies \\ 
        A 60-year-old female carpenter from Rome & 55-64 & female & manual labor & Strong Developed Economies \\ 
        A 45-year-old male carpenter from Auckland & 45-54 & male & manual labor & Emerging Markets \\ 
        A 44-year-old female cashier from Montreal & 35-44 & female & hospitality & Strong Developed Economies \\ 
        A 56-year-old male roofer from Brisbane & 55-64 & male & manual labor & Strong Developed Economies \\ 
        A 30-year-old female garbage collector from Toronto & 25-34 & female & manual labor & Strong Developed Economies \\ 
        A 63-year-old male miner from Johannesburg & 55-64 & male & manual labor & Emerging Markets \\ 
        A 24-year-old female lab technician from Shanghai & 18-24 & female & healthcare \& education & Largest Global Economies \\ 
        A 29-year-old male postal worker from Mexico City & 25-34 & male & transportation & Emerging Markets \\ 
        A 44-year-old female welder from Dubai & 35-44 & female & manual labor & Mid-Sized \& Regional Powers \\ 
        A 54-year-old male librarian from Amsterdam & 45-54 & male & healthcare \& education & Mid-Sized \& Regional Powers \\ 
        A 51-year-old female dentist from Seoul & 45-54 & female & healthcare \& education & Strong Developed Economies \\ 
        A 40-year-old female landscaper from Edinburgh & 35-44 & female & manual labor & Largest Global Economies \\ 
        A 24-year-old male hairdresser from Barcelona & 18-24 & male & hospitality & Strong Developed Economies \\ 
        A 19-year-old male janitor from Stockholm & 18-24 & male & manual labor & Mid-Sized \& Regional Powers \\ 
        A 53-year-old female bus driver from Copenhagen & 45-54 & female & transportation & Mid-Sized \& Regional Powers \\ 
        A 27-year-old female machinist from Frankfurt & 25-34 & female & manual labor & Largest Global Economies \\ 
        A 52-year-old male doctor from Madrid & 45-54 & male & healthcare \& education & Strong Developed Economies \\ 
        A 60-year-old male security guard from Lisbon & 55-64 & male & public safety & Mid-Sized \& Regional Powers \\ 
        A 42-year-old male firefighter from Sao Paulo & 35-44 & male & public safety & Emerging Markets \\ 
        A 36-year-old male pharmacist from Berlin & 35-44 & male & healthcare \& education & Largest Global Economies \\ 
        A 56-year-old female teacher from Melbourne & 55-64 & female & healthcare \& education & Strong Developed Economies \\ 
        A 42-year-old male taxi driver from Hong Kong & 35-44 & male & transportation & Largest Global Economies \\ 
        A 39-year-old female veterinarian from Nairobi & 35-44 & female & healthcare \& education & Emerging Markets \\ 
        A 25-year-old male baker from Lisbon & 25-34 & male & hospitality & Mid-Sized \& Regional Powers \\ 
        A 40-year-old male welder from Moscow & 35-44 & male & manual labor & Mid-Sized \& Regional Powers \\ 
        A 39-year-old male plumber from Melbourne & 35-44 & male & manual labor & Strong Developed Economies \\ 
        A 22-year-old male lab technician from Tokyo & 18-24 & male & healthcare \& education & Largest Global Economies \\ 
        A 20-year-old female security guard from Cape Town & 18-24 & female & public safety & Emerging Markets   \\
        \bottomrule
    \end{tabular}
    }
\end{table*}

\subsection{Effect of Safety Training}
\label{app:safety-training}


In our experiments, we observed that Llama 3.2 90B frequently refused to assume visual personas\footnote{Refusal detection was performed using a fine-tuned \texttt{distilroberta-base} model \citep{distilroberta-base-rejection-v1}}, refusing to engage with 76.7\% of all visual persona prompts (Figure \ref{fig:safety-training}). This behavior can be attributed to an overgeneralization of the model's safety training, as personas can create competing objectives between aligned models' safety measures and instruction-following directives \citep{wei2024jailbroken}. This vulnerability has frequently been exploited in adversarial attacks \citep{ma2024visual}, leading to unsafe outputs even when models assume benign personas \citep{zhao2024bias}. To address this issue, the development of Llama 3 incorporated targeted safety training specifically designed to handle persona-based interactions \citep{grattafiori2024llama3herdmodels}.

\begin{figure}[t]
        \centering
        \includegraphics[width=\linewidth]{refusal-graph.pdf}
        \caption{The rate and number of refusals in response to persona prompts. Llama 3.2 90B shows a strong aversion to multimodal persona prompts, while other models rarely refuse.}
    \label{fig:safety-training}
\end{figure}

\begin{table*}[t]
    \centering
    \caption{Direct testing question list}
    \label{tab:direct_questions_list}
    \resizebox{0.9\textwidth}{!}{
    \begin{tabular}{c|c}
    \toprule
    % \multicolumn{2}{c}{Direct testing}
    Attribute & Direct questions \\
    \midrule
        \multirow{10}{*}{Age} & What age-related milestone are you approaching or have recently celebrated, and how did you celebrate it? \\
            & Which television shows or movies were popular when you were a teenager? \\
            & What life experiences do you consider most defining for your generation? \\
            & What were some common trends or fashions during your college years? \\
            & At what age did you first use the internet regularly, and what activities did you engage in online? \\
            & What age were you when you first experienced a major economic event? \\
            & How old were you when you first started using social media, and which platform did you join first? \\
            & How did people in your age group typically meet and socialize in their younger years? \\
            & What music formats (vinyl, cassettes, CDs, etc.) did you grow up using? \\
            & What historical moments do people slightly older than you remember that you just missed? \\
        \midrule
        \multirow{10}{*}{Location} & 
            What are the top three universities or colleges in your area? \\
            & What is the most visited tourist attraction in your area? \\
            & How does the local climate influence your daily activities and lifestyle in your region? \\
            & What are the most frequented local cuisines where you live? \\
            & What are the main industries driving the economy in your area? \\
            & What natural features (mountains, rivers, coast) shape your local landscape? \\
            & What local sports teams unite your community? \\
            & What's the primary mode of public transportation in your area, if any? \\
            & What are the most popular local festivals or events in your area? \\
            & How has the demographic makeup of your area changed over the past decade? \\
        \midrule
        \multirow{10}{*}{Occupation} & 
            Can you outline your primary responsibilities in your current occupation? \\
            & What specific skills are essential for success in your profession? \\
            & What does a typical workday look like for you? \\
            & How do you stay updated with the latest developments in your industry? \\
            & What tools or technologies do you regularly use in your work? \\
            & What's the most significant change you've witnessed in your industry? \\
            & What emerging trends do you see impacting your profession? \\
            & What advice would you give to someone aspiring to enter your field? \\
            & Which legislation directly impacts the way you perform your job? \\
            & What safety protocols specific to your profession do you follow? \\
        \bottomrule
    \end{tabular}}
\end{table*}

\begin{table*}[t]
    \centering
    \footnotesize
    \caption{Scenarios for situational testing}
    \label{tab:direct_scenarios_list}
    \resizebox{0.9\textwidth}{!}{
    \begin{tabular}{c|p{13cm}}
        \toprule
        \textbf{Attribute} & \textbf{Scenarios} \\
        \midrule
        \multirow{20}{*}{\textbf{Age}} 
            & Your extended family is digitizing old home videos for a reunion. While helping, you discover some footage from your early childhood that needs to be converted. The deadline is next week. You \dots \\
            & During a basement cleanup, you find your old gaming systems and accessories from when you were 12. A local collector has shown interest in purchasing the set. You \dots \\
            & You're coordinating a playlist for your high school reunion after-party. The organizers want music specifically from your graduating years to recreate the atmosphere. You \dots \\
            & Your childhood bank is closing its local branch. They've requested you update your old youth account, which was opened when you first started saving. You \dots \\
            & A museum exhibit about educational evolution needs artifacts from your school years. They're particularly interested in how your grade did research projects, so you \dots \\
            & Your first protest/activism experience has become historically significant. A documentary team wants to understand what motivated your age group's involvement. You talk about \dots \\
            & While discussing financial education, younger relatives ask about your first experiences with digital payment apps and online banking. You say \dots \\
            & A market research team needs insight into how your age group first learned about major global events. They're comparing news consumption across generations. You share \dots \\
            & Your old elementary school is celebrating its anniversary. They're recreating typical classroom setups from different eras, including your years there, so you \dots \\
            & During a job interview, you're asked about your generation's unique approach to work-life balance and career expectations. You describe \dots \\
        \midrule
        \multirow{20}{*}{\textbf{Location}} 
            & A major sporting event has caused unusual traffic patterns downtown. You have an important appointment across the city in 45 minutes, so you \dots \\
            & Construction has blocked your usual route through downtown. You're picking up a friend from the airport in an hour, and they just texted that their flight landed early. To make it on time, you \dots \\
            & The biggest annual cultural celebration in your region coincides with a business meeting. International colleagues want to experience local traditions. You \dots \\
            & Severe weather typical for your region has disrupted normal operations. You need to help visitors understand local emergency procedures. You \dots \\
            & A local food shortage has affected typical ingredients in your area. You need to adapt traditional recipes for an important family gathering. You \dots \\
            & Regional housing policies have changed, affecting your rental situation. You need to explain local housing norms to potential roommates. Breaking it down, you explain \dots \\
            & A community center needs help designing weather-appropriate outdoor activities for children new to the area. To help, you \dots \\
            & Your area's unique rush hour patterns are affecting a planned event. You need to coordinate arrival times for out-of-town guests. You \dots \\
            & Local religious or cultural observations are affecting business hours. You need to plan essential errands accordingly. You \dots \\
            & A seasonal health advisory typical for your region has been issued. You need to adjust your outdoor workout routine. \\
        \midrule
        \multirow{20}{*}{\textbf{Occupation}} 
            & During a casual conversation at dinner, your aunt mentions an ongoing situation that raises red flags based on your background and training. You \dots \\
            & A friend's child is working on a school project related to your profession. They need help understanding basic industry concepts. To assist, you \dots \\
            & During a home renovation, you notice issues that relate to your professional expertise. The contractors seem unaware of potential complications. You \dots \\
            & A community workshop needs professionals to demonstrate how their job impacts daily life. Your industry's perspective would fill a key gap. You \dots \\
            & A community Facebook group is sharing advice that conflicts with principles you work with daily, so you \dots \\
            & A local news story misrepresents aspects of your industry. You have an opportunity to provide clarification at a community meeting. At the meeting, you \dots \\
            & Your hobby group encounters a challenge that relates to your professional expertise. They're unsure about proper procedures. You demonstrate \dots \\
            & A neighbor's insurance claim involves aspects of your profession. They're asking for general guidance about standard practices. \\
            & During a social event, you notice concerning practices related to your industry's safety standards. Others seem unaware of the risks, so you \dots \\
            & A local youth program needs career mentors. They want professionals to share how their industry handles modern challenges. You \dots \\
        \bottomrule
    \end{tabular}}
\end{table*}

\begin{table*}[t]
\centering
\small
\begin{tabular}{lcccc}
\toprule
\rowcolor{gray!25}
\multicolumn{5}{c}{\textbf{GPT-4o}} \\
\textbf{Modality} & \textbf{Linguistic Habits} & \textbf{Persona Consistency} & \textbf{Expected Action} & \textbf{Action Justification} \\
\midrule
\textbf{Text} & \begin{tabular}{@{}c@{}}1.68 $\pm$ {\scriptsize 0.04} \\ {\scriptsize (95\% CI: 1.61--1.75)}\end{tabular} & \begin{tabular}{@{}c@{}}3.00 $\pm$ {\scriptsize 0.06} \\ {\scriptsize (95\% CI: 2.87--3.12)}\end{tabular} & \begin{tabular}{@{}c@{}}3.25 $\pm$ {\scriptsize 0.05} \\ {\scriptsize (95\% CI: 3.16--3.34)}\end{tabular} & \begin{tabular}{@{}c@{}}3.91 $\pm$ {\scriptsize 0.04} \\ {\scriptsize (95\% CI: 3.83--3.99)}\end{tabular} \\
\textbf{Assisted Image} & \begin{tabular}{@{}c@{}}1.22 $\pm$ {\scriptsize 0.03} \\ {\scriptsize (95\% CI: 1.16--1.27)}\end{tabular} & \begin{tabular}{@{}c@{}}2.89 $\pm$ {\scriptsize 0.06} \\ {\scriptsize (95\% CI: 2.77--3.01)}\end{tabular} & \begin{tabular}{@{}c@{}}2.83 $\pm$ {\scriptsize 0.05} \\ {\scriptsize (95\% CI: 2.74--2.93)}\end{tabular} & \begin{tabular}{@{}c@{}}3.60 $\pm$ {\scriptsize 0.04} \\ {\scriptsize (95\% CI: 3.52--3.68)}\end{tabular} \\
\textbf{Image} & \begin{tabular}{@{}c@{}}1.05 $\pm$ {\scriptsize 0.02} \\ {\scriptsize (95\% CI: 1.00--1.10)}\end{tabular} & \begin{tabular}{@{}c@{}}2.70 $\pm$ {\scriptsize 0.06} \\ {\scriptsize (95\% CI: 2.58--2.82)}\end{tabular} & \begin{tabular}{@{}c@{}}2.75 $\pm$ {\scriptsize 0.05} \\ {\scriptsize (95\% CI: 2.66--2.84)}\end{tabular} & \begin{tabular}{@{}c@{}}3.56 $\pm$ {\scriptsize 0.04} \\ {\scriptsize (95\% CI: 3.48--3.64)}\end{tabular} \\
\textbf{Descriptive Image} & \begin{tabular}{@{}c@{}}1.17 $\pm$ {\scriptsize 0.03} \\ {\scriptsize (95\% CI: 1.12--1.23)}\end{tabular} & \begin{tabular}{@{}c@{}}3.67 $\pm$ {\scriptsize 0.06} \\ {\scriptsize (95\% CI: 3.56--3.79)}\end{tabular} & \begin{tabular}{@{}c@{}}3.26 $\pm$ {\scriptsize 0.05} \\ {\scriptsize (95\% CI: 3.17--3.35)}\end{tabular} & \begin{tabular}{@{}c@{}}3.87 $\pm$ {\scriptsize 0.04} \\ {\scriptsize (95\% CI: 3.79--3.95)}\end{tabular} \\
\midrule
\rowcolor{gray!25}
\multicolumn{5}{c}{\textbf{GPT-4o-mini}} \\
\textbf{Modality} & \textbf{Linguistic Habits} & \textbf{Persona Consistency} & \textbf{Expected Action} & \textbf{Action Justification} \\
\midrule
\textbf{Text} & \begin{tabular}{@{}c@{}}1.32 $\pm$ {\scriptsize 0.04} \\ {\scriptsize (95\% CI: 1.25--1.39)}\end{tabular} & \begin{tabular}{@{}c@{}}1.95 $\pm$ {\scriptsize 0.07} \\ {\scriptsize (95\% CI: 1.82--2.08)}\end{tabular} & \begin{tabular}{@{}c@{}}2.02 $\pm$ {\scriptsize 0.05} \\ {\scriptsize (95\% CI: 1.93--2.12)}\end{tabular} & \begin{tabular}{@{}c@{}}2.78 $\pm$ {\scriptsize 0.05} \\ {\scriptsize (95\% CI: 2.68--2.88)}\end{tabular} \\
\textbf{Assisted Image} & \begin{tabular}{@{}c@{}}1.17 $\pm$ {\scriptsize 0.03} \\ {\scriptsize (95\% CI: 1.11--1.23)}\end{tabular} & \begin{tabular}{@{}c@{}}2.17 $\pm$ {\scriptsize 0.06} \\ {\scriptsize (95\% CI: 2.04--2.30)}\end{tabular} & \begin{tabular}{@{}c@{}}2.16 $\pm$ {\scriptsize 0.05} \\ {\scriptsize (95\% CI: 2.06--2.25)}\end{tabular} & \begin{tabular}{@{}c@{}}2.88 $\pm$ {\scriptsize 0.05} \\ {\scriptsize (95\% CI: 2.78--2.97)}\end{tabular} \\
\textbf{Image} & \begin{tabular}{@{}c@{}}0.93 $\pm$ {\scriptsize 0.03} \\ {\scriptsize (95\% CI: 0.88--0.99)}\end{tabular} & \begin{tabular}{@{}c@{}}2.11 $\pm$ {\scriptsize 0.06} \\ {\scriptsize (95\% CI: 1.98--2.23)}\end{tabular} & \begin{tabular}{@{}c@{}}1.94 $\pm$ {\scriptsize 0.05} \\ {\scriptsize (95\% CI: 1.85--2.04)}\end{tabular} & \begin{tabular}{@{}c@{}}2.69 $\pm$ {\scriptsize 0.05} \\ {\scriptsize (95\% CI: 2.59--2.78)}\end{tabular} \\
\textbf{Descriptive Image} & \begin{tabular}{@{}c@{}}1.11 $\pm$ {\scriptsize 0.03} \\ {\scriptsize (95\% CI: 1.05--1.17)}\end{tabular} & \begin{tabular}{@{}c@{}}2.68 $\pm$ {\scriptsize 0.07} \\ {\scriptsize (95\% CI: 2.54--2.82)}\end{tabular} & \begin{tabular}{@{}c@{}}2.49 $\pm$ {\scriptsize 0.05} \\ {\scriptsize (95\% CI: 2.40--2.59)}\end{tabular} & \begin{tabular}{@{}c@{}}2.89 $\pm$ {\scriptsize 0.05} \\ {\scriptsize (95\% CI: 2.80--2.99)}\end{tabular} \\
\midrule
\rowcolor{gray!25}
\multicolumn{5}{c}{\textbf{Llama 3.2 11B}} \\
\textbf{Modality} & \textbf{Linguistic Habits} & \textbf{Persona Consistency} & \textbf{Expected Action} & \textbf{Action Justification} \\
\midrule
\textbf{Text} & \begin{tabular}{@{}c@{}}1.28 $\pm$ {\scriptsize 0.04} \\ {\scriptsize (95\% CI: 1.21--1.35)}\end{tabular} & \begin{tabular}{@{}c@{}}1.69 $\pm$ {\scriptsize 0.06} \\ {\scriptsize (95\% CI: 1.57--1.81)}\end{tabular} & \begin{tabular}{@{}c@{}}1.82 $\pm$ {\scriptsize 0.05} \\ {\scriptsize (95\% CI: 1.73--1.91)}\end{tabular} & \begin{tabular}{@{}c@{}}2.42 $\pm$ {\scriptsize 0.05} \\ {\scriptsize (95\% CI: 2.32--2.51)}\end{tabular} \\
\textbf{Assisted Image} & \begin{tabular}{@{}c@{}}0.67 $\pm$ {\scriptsize 0.02} \\ {\scriptsize (95\% CI: 0.63--0.71)}\end{tabular} & \begin{tabular}{@{}c@{}}1.31 $\pm$ {\scriptsize 0.05} \\ {\scriptsize (95\% CI: 1.21--1.41)}\end{tabular} & \begin{tabular}{@{}c@{}}1.19 $\pm$ {\scriptsize 0.04} \\ {\scriptsize (95\% CI: 1.12--1.26)}\end{tabular} & \begin{tabular}{@{}c@{}}1.73 $\pm$ {\scriptsize 0.04} \\ {\scriptsize (95\% CI: 1.65--1.81)}\end{tabular} \\
\textbf{Image} & \begin{tabular}{@{}c@{}}0.61 $\pm$ {\scriptsize 0.02} \\ {\scriptsize (95\% CI: 0.58--0.64)}\end{tabular} & \begin{tabular}{@{}c@{}}1.15 $\pm$ {\scriptsize 0.05} \\ {\scriptsize (95\% CI: 1.06--1.24)}\end{tabular} & \begin{tabular}{@{}c@{}}1.05 $\pm$ {\scriptsize 0.03} \\ {\scriptsize (95\% CI: 0.98--1.12)}\end{tabular} & \begin{tabular}{@{}c@{}}1.40 $\pm$ {\scriptsize 0.04} \\ {\scriptsize (95\% CI: 1.33--1.48)}\end{tabular} \\
\textbf{Descriptive Image} & \begin{tabular}{@{}c@{}}0.71 $\pm$ {\scriptsize 0.02} \\ {\scriptsize (95\% CI: 0.68--0.75)}\end{tabular} & \begin{tabular}{@{}c@{}}1.60 $\pm$ {\scriptsize 0.06} \\ {\scriptsize (95\% CI: 1.48--1.71)}\end{tabular} & \begin{tabular}{@{}c@{}}1.33 $\pm$ {\scriptsize 0.04} \\ {\scriptsize (95\% CI: 1.25--1.40)}\end{tabular} & \begin{tabular}{@{}c@{}}1.72 $\pm$ {\scriptsize 0.04} \\ {\scriptsize (95\% CI: 1.64--1.80)}\end{tabular} \\
\midrule
\rowcolor{gray!25}
\multicolumn{5}{c}{\textbf{Llama 3.2 90B}} \\
\textbf{Modality} & \textbf{Linguistic Habits} & \textbf{Persona Consistency} & \textbf{Expected Action} & \textbf{Action Justification} \\
\midrule
\textbf{Text} & \begin{tabular}{@{}c@{}}1.45 $\pm$ {\scriptsize 0.04} \\ {\scriptsize (95\% CI: 1.37--1.53)}\end{tabular} & \begin{tabular}{@{}c@{}}1.94 $\pm$ {\scriptsize 0.06} \\ {\scriptsize (95\% CI: 1.81--2.06)}\end{tabular} & \begin{tabular}{@{}c@{}}2.18 $\pm$ {\scriptsize 0.05} \\ {\scriptsize (95\% CI: 2.08--2.27)}\end{tabular} & \begin{tabular}{@{}c@{}}2.69 $\pm$ {\scriptsize 0.05} \\ {\scriptsize (95\% CI: 2.59--2.79)}\end{tabular} \\
\textbf{Assisted Image} & \begin{tabular}{@{}c@{}}0.40 $\pm$ {\scriptsize 0.03} \\ {\scriptsize (95\% CI: 0.35--0.45)}\end{tabular} & \begin{tabular}{@{}c@{}}1.01 $\pm$ {\scriptsize 0.08} \\ {\scriptsize (95\% CI: 0.86--1.16)}\end{tabular} & \begin{tabular}{@{}c@{}}0.87 $\pm$ {\scriptsize 0.06} \\ {\scriptsize (95\% CI: 0.76--0.97)}\end{tabular} & \begin{tabular}{@{}c@{}}0.98 $\pm$ {\scriptsize 0.06} \\ {\scriptsize (95\% CI: 0.86--1.09)}\end{tabular} \\
\textbf{Image} & \begin{tabular}{@{}c@{}}0.31 $\pm$ {\scriptsize 0.02} \\ {\scriptsize (95\% CI: 0.27--0.36)}\end{tabular} & \begin{tabular}{@{}c@{}}0.63 $\pm$ {\scriptsize 0.06} \\ {\scriptsize (95\% CI: 0.50--0.75)}\end{tabular} & \begin{tabular}{@{}c@{}}0.56 $\pm$ {\scriptsize 0.04} \\ {\scriptsize (95\% CI: 0.47--0.64)}\end{tabular} & \begin{tabular}{@{}c@{}}0.59 $\pm$ {\scriptsize 0.04} \\ {\scriptsize (95\% CI: 0.51--0.68)}\end{tabular} \\
\textbf{Descriptive Image} & \begin{tabular}{@{}c@{}}0.37 $\pm$ {\scriptsize 0.03} \\ {\scriptsize (95\% CI: 0.31--0.42)}\end{tabular} & \begin{tabular}{@{}c@{}}0.89 $\pm$ {\scriptsize 0.09} \\ {\scriptsize (95\% CI: 0.73--1.06)}\end{tabular} & \begin{tabular}{@{}c@{}}0.74 $\pm$ {\scriptsize 0.05} \\ {\scriptsize (95\% CI: 0.63--0.84)}\end{tabular} & \begin{tabular}{@{}c@{}}0.87 $\pm$ {\scriptsize 0.05} \\ {\scriptsize (95\% CI: 0.77--0.98)}\end{tabular} \\
\midrule
\rowcolor{gray!25}
\multicolumn{5}{c}{\textbf{Pixtral 12B}} \\
\textbf{Modality} & \textbf{Linguistic Habits} & \textbf{Persona Consistency} & \textbf{Expected Action} & \textbf{Action Justification} \\
\midrule
\textbf{Text} & \begin{tabular}{@{}c@{}}1.26 $\pm$ {\scriptsize 0.04} \\ {\scriptsize (95\% CI: 1.19--1.34)}\end{tabular} & \begin{tabular}{@{}c@{}}1.47 $\pm$ {\scriptsize 0.06} \\ {\scriptsize (95\% CI: 1.35--1.58)}\end{tabular} & \begin{tabular}{@{}c@{}}1.85 $\pm$ {\scriptsize 0.05} \\ {\scriptsize (95\% CI: 1.76--1.94)}\end{tabular} & \begin{tabular}{@{}c@{}}2.51 $\pm$ {\scriptsize 0.05} \\ {\scriptsize (95\% CI: 2.41--2.60)}\end{tabular} \\
\textbf{Assisted Image} & \begin{tabular}{@{}c@{}}1.08 $\pm$ {\scriptsize 0.03} \\ {\scriptsize (95\% CI: 1.02--1.14)}\end{tabular} & \begin{tabular}{@{}c@{}}1.43 $\pm$ {\scriptsize 0.05} \\ {\scriptsize (95\% CI: 1.32--1.54)}\end{tabular} & \begin{tabular}{@{}c@{}}1.65 $\pm$ {\scriptsize 0.04} \\ {\scriptsize (95\% CI: 1.56--1.73)}\end{tabular} & \begin{tabular}{@{}c@{}}2.32 $\pm$ {\scriptsize 0.05} \\ {\scriptsize (95\% CI: 2.22--2.41)}\end{tabular} \\
\textbf{Image} & \begin{tabular}{@{}c@{}}1.04 $\pm$ {\scriptsize 0.03} \\ {\scriptsize (95\% CI: 0.98--1.10)}\end{tabular} & \begin{tabular}{@{}c@{}}1.90 $\pm$ {\scriptsize 0.06} \\ {\scriptsize (95\% CI: 1.78--2.02)}\end{tabular} & \begin{tabular}{@{}c@{}}2.06 $\pm$ {\scriptsize 0.05} \\ {\scriptsize (95\% CI: 1.96--2.15)}\end{tabular} & \begin{tabular}{@{}c@{}}2.62 $\pm$ {\scriptsize 0.05} \\ {\scriptsize (95\% CI: 2.52--2.71)}\end{tabular} \\
\textbf{Descriptive Image} & \begin{tabular}{@{}c@{}}1.05 $\pm$ {\scriptsize 0.03} \\ {\scriptsize (95\% CI: 0.99--1.11)}\end{tabular} & \begin{tabular}{@{}c@{}}2.75 $\pm$ {\scriptsize 0.07} \\ {\scriptsize (95\% CI: 2.61--2.88)}\end{tabular} & \begin{tabular}{@{}c@{}}2.42 $\pm$ {\scriptsize 0.05} \\ {\scriptsize (95\% CI: 2.32--2.51)}\end{tabular} & \begin{tabular}{@{}c@{}}2.97 $\pm$ {\scriptsize 0.05} \\ {\scriptsize (95\% CI: 2.87--3.06)}\end{tabular} \\
\bottomrule
\end{tabular}
\caption{Evaluation Metrics by Model and Modality with \textbf{\underline{GPT-4o}} as the evaluator. Each cell shows mean $\pm$ {\scriptsize SEM} on the first line and 95\% CI on the second.}
\label{tab:eval-table-gpt-4o}
\end{table*}

\begin{table*}[t]
\centering
\small
\begin{tabular}{lcccc}
\toprule
\rowcolor{gray!25}
\multicolumn{5}{c}{\textbf{GPT-4o}} \\
\textbf{Modality} & \textbf{Linguistic Habits} & \textbf{Persona Consistency} & \textbf{Expected Action} & \textbf{Action Justification} \\
\midrule
\textbf{Text} & \begin{tabular}{@{}c@{}}2.47 $\pm$ {\scriptsize 0.03} \\ {\scriptsize (95\% CI: 2.41--2.53)}\end{tabular} & \begin{tabular}{@{}c@{}}3.88 $\pm$ {\scriptsize 0.04} \\ {\scriptsize (95\% CI: 3.79--3.97)}\end{tabular} & \begin{tabular}{@{}c@{}}4.46 $\pm$ {\scriptsize 0.03} \\ {\scriptsize (95\% CI: 4.41--4.52)}\end{tabular} & \begin{tabular}{@{}c@{}}4.34 $\pm$ {\scriptsize 0.03} \\ {\scriptsize (95\% CI: 4.28--4.40)}\end{tabular} \\
\textbf{Assisted Image} & \begin{tabular}{@{}c@{}}2.01 $\pm$ {\scriptsize 0.03} \\ {\scriptsize (95\% CI: 1.95--2.06)}\end{tabular} & \begin{tabular}{@{}c@{}}3.50 $\pm$ {\scriptsize 0.04} \\ {\scriptsize (95\% CI: 3.42--3.59)}\end{tabular} & \begin{tabular}{@{}c@{}}4.35 $\pm$ {\scriptsize 0.03} \\ {\scriptsize (95\% CI: 4.30--4.40)}\end{tabular} & \begin{tabular}{@{}c@{}}4.03 $\pm$ {\scriptsize 0.03} \\ {\scriptsize (95\% CI: 3.97--4.10)}\end{tabular} \\
\textbf{Image} & \begin{tabular}{@{}c@{}}1.96 $\pm$ {\scriptsize 0.03} \\ {\scriptsize (95\% CI: 1.90--2.02)}\end{tabular} & \begin{tabular}{@{}c@{}}3.36 $\pm$ {\scriptsize 0.04} \\ {\scriptsize (95\% CI: 3.28--3.45)}\end{tabular} & \begin{tabular}{@{}c@{}}4.36 $\pm$ {\scriptsize 0.03} \\ {\scriptsize (95\% CI: 4.31--4.41)}\end{tabular} & \begin{tabular}{@{}c@{}}3.93 $\pm$ {\scriptsize 0.04} \\ {\scriptsize (95\% CI: 3.86--4.00)}\end{tabular} \\
\textbf{Descriptive Image} & \begin{tabular}{@{}c@{}}2.01 $\pm$ {\scriptsize 0.03} \\ {\scriptsize (95\% CI: 1.95--2.07)}\end{tabular} & \begin{tabular}{@{}c@{}}4.14 $\pm$ {\scriptsize 0.04} \\ {\scriptsize (95\% CI: 4.07--4.21)}\end{tabular} & \begin{tabular}{@{}c@{}}4.62 $\pm$ {\scriptsize 0.02} \\ {\scriptsize (95\% CI: 4.58--4.66)}\end{tabular} & \begin{tabular}{@{}c@{}}4.12 $\pm$ {\scriptsize 0.03} \\ {\scriptsize (95\% CI: 4.05--4.19)}\end{tabular} \\
\midrule
\rowcolor{gray!25}
\multicolumn{5}{c}{\textbf{GPT-4o-mini}} \\
\textbf{Modality} & \textbf{Linguistic Habits} & \textbf{Persona Consistency} & \textbf{Expected Action} & \textbf{Action Justification} \\
\midrule
\textbf{Text} & \begin{tabular}{@{}c@{}}2.31 $\pm$ {\scriptsize 0.03} \\ {\scriptsize (95\% CI: 2.25--2.36)}\end{tabular} & \begin{tabular}{@{}c@{}}4.01 $\pm$ {\scriptsize 0.04} \\ {\scriptsize (95\% CI: 3.93--4.09)}\end{tabular} & \begin{tabular}{@{}c@{}}4.47 $\pm$ {\scriptsize 0.03} \\ {\scriptsize (95\% CI: 4.43--4.52)}\end{tabular} & \begin{tabular}{@{}c@{}}4.34 $\pm$ {\scriptsize 0.03} \\ {\scriptsize (95\% CI: 4.28--4.40)}\end{tabular} \\
\textbf{Assisted Image} & \begin{tabular}{@{}c@{}}2.06 $\pm$ {\scriptsize 0.03} \\ {\scriptsize (95\% CI: 2.01--2.12)}\end{tabular} & \begin{tabular}{@{}c@{}}3.79 $\pm$ {\scriptsize 0.04} \\ {\scriptsize (95\% CI: 3.70--3.87)}\end{tabular} & \begin{tabular}{@{}c@{}}4.42 $\pm$ {\scriptsize 0.02} \\ {\scriptsize (95\% CI: 4.37--4.47)}\end{tabular} & \begin{tabular}{@{}c@{}}4.07 $\pm$ {\scriptsize 0.03} \\ {\scriptsize (95\% CI: 4.00--4.14)}\end{tabular} \\
\textbf{Image} & \begin{tabular}{@{}c@{}}2.04 $\pm$ {\scriptsize 0.03} \\ {\scriptsize (95\% CI: 1.98--2.09)}\end{tabular} & \begin{tabular}{@{}c@{}}3.84 $\pm$ {\scriptsize 0.04} \\ {\scriptsize (95\% CI: 3.76--3.92)}\end{tabular} & \begin{tabular}{@{}c@{}}4.44 $\pm$ {\scriptsize 0.02} \\ {\scriptsize (95\% CI: 4.39--4.48)}\end{tabular} & \begin{tabular}{@{}c@{}}4.02 $\pm$ {\scriptsize 0.03} \\ {\scriptsize (95\% CI: 3.95--4.09)}\end{tabular} \\
\textbf{Descriptive Image} & \begin{tabular}{@{}c@{}}2.15 $\pm$ {\scriptsize 0.03} \\ {\scriptsize (95\% CI: 2.09--2.20)}\end{tabular} & \begin{tabular}{@{}c@{}}4.49 $\pm$ {\scriptsize 0.03} \\ {\scriptsize (95\% CI: 4.43--4.55)}\end{tabular} & \begin{tabular}{@{}c@{}}4.69 $\pm$ {\scriptsize 0.02} \\ {\scriptsize (95\% CI: 4.65--4.72)}\end{tabular} & \begin{tabular}{@{}c@{}}4.20 $\pm$ {\scriptsize 0.03} \\ {\scriptsize (95\% CI: 4.14--4.27)}\end{tabular} \\
\midrule
\rowcolor{gray!25}
\multicolumn{5}{c}{\textbf{Llama 3.2 11B}} \\
\textbf{Modality} & \textbf{Linguistic Habits} & \textbf{Persona Consistency} & \textbf{Expected Action} & \textbf{Action Justification} \\
\midrule
\textbf{Text} & \begin{tabular}{@{}c@{}}3.12 $\pm$ {\scriptsize 0.03} \\ {\scriptsize (95\% CI: 3.07--3.18)}\end{tabular} & \begin{tabular}{@{}c@{}}3.90 $\pm$ {\scriptsize 0.04} \\ {\scriptsize (95\% CI: 3.82--3.99)}\end{tabular} & \begin{tabular}{@{}c@{}}4.14 $\pm$ {\scriptsize 0.03} \\ {\scriptsize (95\% CI: 4.08--4.19)}\end{tabular} & \begin{tabular}{@{}c@{}}4.07 $\pm$ {\scriptsize 0.03} \\ {\scriptsize (95\% CI: 4.01--4.13)}\end{tabular} \\
\textbf{Assisted Image} & \begin{tabular}{@{}c@{}}1.93 $\pm$ {\scriptsize 0.03} \\ {\scriptsize (95\% CI: 1.87--1.99)}\end{tabular} & \begin{tabular}{@{}c@{}}3.02 $\pm$ {\scriptsize 0.04} \\ {\scriptsize (95\% CI: 2.93--3.11)}\end{tabular} & \begin{tabular}{@{}c@{}}3.36 $\pm$ {\scriptsize 0.03} \\ {\scriptsize (95\% CI: 3.30--3.43)}\end{tabular} & \begin{tabular}{@{}c@{}}3.15 $\pm$ {\scriptsize 0.04} \\ {\scriptsize (95\% CI: 3.08--3.23)}\end{tabular} \\
\textbf{Image} & \begin{tabular}{@{}c@{}}2.03 $\pm$ {\scriptsize 0.03} \\ {\scriptsize (95\% CI: 1.97--2.09)}\end{tabular} & \begin{tabular}{@{}c@{}}2.66 $\pm$ {\scriptsize 0.05} \\ {\scriptsize (95\% CI: 2.56--2.75)}\end{tabular} & \begin{tabular}{@{}c@{}}3.02 $\pm$ {\scriptsize 0.04} \\ {\scriptsize (95\% CI: 2.95--3.09)}\end{tabular} & \begin{tabular}{@{}c@{}}2.94 $\pm$ {\scriptsize 0.04} \\ {\scriptsize (95\% CI: 2.87--3.02)}\end{tabular} \\
\textbf{Descriptive Image} & \begin{tabular}{@{}c@{}}2.17 $\pm$ {\scriptsize 0.03} \\ {\scriptsize (95\% CI: 2.10--2.23)}\end{tabular} & \begin{tabular}{@{}c@{}}3.50 $\pm$ {\scriptsize 0.04} \\ {\scriptsize (95\% CI: 3.42--3.59)}\end{tabular} & \begin{tabular}{@{}c@{}}3.65 $\pm$ {\scriptsize 0.03} \\ {\scriptsize (95\% CI: 3.59--3.71)}\end{tabular} & \begin{tabular}{@{}c@{}}3.27 $\pm$ {\scriptsize 0.04} \\ {\scriptsize (95\% CI: 3.19--3.34)}\end{tabular} \\
\midrule
\rowcolor{gray!25}
\multicolumn{5}{c}{\textbf{Llama 3.2 90B}} \\
\textbf{Modality} & \textbf{Linguistic Habits} & \textbf{Persona Consistency} & \textbf{Expected Action} & \textbf{Action Justification} \\
\midrule
\textbf{Text} & \begin{tabular}{@{}c@{}}3.20 $\pm$ {\scriptsize 0.03} \\ {\scriptsize (95\% CI: 3.14--3.25)}\end{tabular} & \begin{tabular}{@{}c@{}}4.05 $\pm$ {\scriptsize 0.04} \\ {\scriptsize (95\% CI: 3.96--4.13)}\end{tabular} & \begin{tabular}{@{}c@{}}4.38 $\pm$ {\scriptsize 0.03} \\ {\scriptsize (95\% CI: 4.32--4.43)}\end{tabular} & \begin{tabular}{@{}c@{}}4.29 $\pm$ {\scriptsize 0.03} \\ {\scriptsize (95\% CI: 4.24--4.35)}\end{tabular} \\
\textbf{Assisted Image} & \begin{tabular}{@{}c@{}}2.09 $\pm$ {\scriptsize 0.07} \\ {\scriptsize (95\% CI: 1.96--2.22)}\end{tabular} & \begin{tabular}{@{}c@{}}2.24 $\pm$ {\scriptsize 0.08} \\ {\scriptsize (95\% CI: 2.09--2.40)}\end{tabular} & \begin{tabular}{@{}c@{}}2.00 $\pm$ {\scriptsize 0.06} \\ {\scriptsize (95\% CI: 1.89--2.12)}\end{tabular} & \begin{tabular}{@{}c@{}}2.36 $\pm$ {\scriptsize 0.07} \\ {\scriptsize (95\% CI: 2.21--2.50)}\end{tabular} \\
\textbf{Image} & \begin{tabular}{@{}c@{}}2.18 $\pm$ {\scriptsize 0.08} \\ {\scriptsize (95\% CI: 2.03--2.34)}\end{tabular} & \begin{tabular}{@{}c@{}}1.53 $\pm$ {\scriptsize 0.07} \\ {\scriptsize (95\% CI: 1.38--1.67)}\end{tabular} & \begin{tabular}{@{}c@{}}1.48 $\pm$ {\scriptsize 0.05} \\ {\scriptsize (95\% CI: 1.38--1.58)}\end{tabular} & \begin{tabular}{@{}c@{}}2.02 $\pm$ {\scriptsize 0.08} \\ {\scriptsize (95\% CI: 1.87--2.18)}\end{tabular} \\
\textbf{Descriptive Image} & \begin{tabular}{@{}c@{}}2.23 $\pm$ {\scriptsize 0.08} \\ {\scriptsize (95\% CI: 2.08--2.38)}\end{tabular} & \begin{tabular}{@{}c@{}}1.96 $\pm$ {\scriptsize 0.09} \\ {\scriptsize (95\% CI: 1.78--2.14)}\end{tabular} & \begin{tabular}{@{}c@{}}1.74 $\pm$ {\scriptsize 0.06} \\ {\scriptsize (95\% CI: 1.62--1.85)}\end{tabular} & \begin{tabular}{@{}c@{}}2.12 $\pm$ {\scriptsize 0.08} \\ {\scriptsize (95\% CI: 1.97--2.27)}\end{tabular} \\
\midrule
\rowcolor{gray!25}
\multicolumn{5}{c}{\textbf{Pixtral 12B}} \\
\textbf{Modality} & \textbf{Linguistic Habits} & \textbf{Persona Consistency} & \textbf{Expected Action} & \textbf{Action Justification} \\
\midrule
\textbf{Text} & \begin{tabular}{@{}c@{}}2.31 $\pm$ {\scriptsize 0.03} \\ {\scriptsize (95\% CI: 2.25--2.36)}\end{tabular} & \begin{tabular}{@{}c@{}}3.28 $\pm$ {\scriptsize 0.05} \\ {\scriptsize (95\% CI: 3.19--3.38)}\end{tabular} & \begin{tabular}{@{}c@{}}4.01 $\pm$ {\scriptsize 0.03} \\ {\scriptsize (95\% CI: 3.95--4.07)}\end{tabular} & \begin{tabular}{@{}c@{}}4.14 $\pm$ {\scriptsize 0.03} \\ {\scriptsize (95\% CI: 4.08--4.19)}\end{tabular} \\
\textbf{Assisted Image} & \begin{tabular}{@{}c@{}}2.18 $\pm$ {\scriptsize 0.03} \\ {\scriptsize (95\% CI: 2.12--2.24)}\end{tabular} & \begin{tabular}{@{}c@{}}3.22 $\pm$ {\scriptsize 0.05} \\ {\scriptsize (95\% CI: 3.13--3.31)}\end{tabular} & \begin{tabular}{@{}c@{}}4.05 $\pm$ {\scriptsize 0.03} \\ {\scriptsize (95\% CI: 3.99--4.11)}\end{tabular} & \begin{tabular}{@{}c@{}}3.82 $\pm$ {\scriptsize 0.03} \\ {\scriptsize (95\% CI: 3.76--3.89)}\end{tabular} \\
\textbf{Image} & \begin{tabular}{@{}c@{}}2.26 $\pm$ {\scriptsize 0.03} \\ {\scriptsize (95\% CI: 2.20--2.32)}\end{tabular} & \begin{tabular}{@{}c@{}}3.64 $\pm$ {\scriptsize 0.05} \\ {\scriptsize (95\% CI: 3.55--3.73)}\end{tabular} & \begin{tabular}{@{}c@{}}4.25 $\pm$ {\scriptsize 0.03} \\ {\scriptsize (95\% CI: 4.20--4.31)}\end{tabular} & \begin{tabular}{@{}c@{}}3.79 $\pm$ {\scriptsize 0.04} \\ {\scriptsize (95\% CI: 3.72--3.86)}\end{tabular} \\
\textbf{Descriptive Image} & \begin{tabular}{@{}c@{}}2.15 $\pm$ {\scriptsize 0.03} \\ {\scriptsize (95\% CI: 2.09--2.21)}\end{tabular} & \begin{tabular}{@{}c@{}}4.43 $\pm$ {\scriptsize 0.03} \\ {\scriptsize (95\% CI: 4.36--4.49)}\end{tabular} & \begin{tabular}{@{}c@{}}4.64 $\pm$ {\scriptsize 0.02} \\ {\scriptsize (95\% CI: 4.60--4.68)}\end{tabular} & \begin{tabular}{@{}c@{}}4.03 $\pm$ {\scriptsize 0.04} \\ {\scriptsize (95\% CI: 3.96--4.10)}\end{tabular} \\
\bottomrule
\end{tabular}
\caption{Evaluation Metrics by Model and Modality with \textbf{\underline{Gemini 2.0 Flash}} as the evaluator. Each cell shows mean $\pm$ {\scriptsize SEM} on the first line and 95\% CI on the second.}
\label{tab:eval-table-gemini-flash}
\end{table*}

\subsection{Human survey design}\label{app:human}
Figure~\ref{fig:survey} demonstrates our survey design that we conduct on $8$ independent annotators to evaluate the quality of LLM evaluators. In particular, we first show the instructions to evaluate the responses for a prompt and a persona, followed by $10$ such questions. 
\begin{figure*}[t]
    \centering
    \subfloat[Instruction]{\includegraphics[width=0.48\textwidth]{survey_start.png}}\hfill
    \subfloat[Question]{\includegraphics[width=0.48\textwidth]{survey_q1.png}}
    \caption{Human survey design}
    \label{fig:survey}
\end{figure*}


% \begin{table*}[t]
%     \centering
%     \caption{}
%     \label{tab:personalist}
%     \resizebox{1.0\linewidth}{!}{
%     \begin{tabular}{l c}
%     \toprule
%     Scenario & Target Attribute \\
%     \midrule
%     "Your extended family is digitizing old home videos for a reunion. While helping, you discover some footage from your early childhood that needs to be converted. The deadline is next week. You ..." & Age \\
%     "During a basement cleanup, you find your old gaming systems and accessories from when you were 12. A local collector has shown interest in purchasing the set. You ..." \\
%     "You're coordinating a playlist for your high school reunion after-party. The organizers want music specifically from your graduating years to recreate the atmosphere. You ..." \\
%     \bottomrule
%     \end{tabular}
% }
% \end{table*}


\end{document}
\endinput
%%
%% End of file `sample-manuscript.tex'.
