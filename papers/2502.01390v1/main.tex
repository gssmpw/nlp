%%
%% This is file `sample-manuscript.tex',
%% generated with the docstrip utility.
%%
%% The original source files were:
%%
%% samples.dtx  (with options: `manuscript')
%% 
%% IMPORTANT NOTICE:
%% 
%% For the copyright see the source file.
%% 
%% Any modified versions of this file must be renamed
%% with new filenames distinct from sample-manuscript.tex.
%% 
%% For distribution of the original source see the terms
%% for copying and modification in the file samples.dtx.
%% 
%% This generated file may be distributed as long as the
%% original source files, as listed above, are part of the
%% same distribution. (The sources need not necessarily be
%% in the same archive or directory.)
%%
%% Commands for TeXCount
%TC:macro \cite [option:text,text]
%TC:macro \citep [option:text,text]
%TC:macro \citet [option:text,text]
%TC:envir table 0 1
%TC:envir table* 0 1
%TC:envir tabular [ignore] word
%TC:envir displaymath 0 word
%TC:envir math 0 word
%TC:envir comment 0 0
%%
%%
%% The first command in your LaTeX source must be the \documentclass command.
\documentclass[sigconf]{acmart}

%%
%% \BibTeX command to typeset BibTeX logo in the docs
\AtBeginDocument{%
  \providecommand\BibTeX{{%
    \normalfont B\kern-0.5em{\scshape i\kern-0.25em b}\kern-0.8em\TeX}}}

%% Rights management information.  This information is sent to you
%% when you complete the rights form.  These commands have SAMPLE
%% values in them; it is your responsibility as an author to replace
%% the commands and values with those provided to you when you
%% complete the rights form.
% \setcopyright{acmlicensed}
% \copyrightyear{2025}
% \acmYear{2025}
% \acmDOI{XXXXXXX.XXXXXXX}

%% These commands are for a PROCEEDINGS abstract or paper.
% \acmConference[CHI '25]{Proceedings of the 2025 CHI Conference on Human Factors in Computing Systems}{April 26-- May 1, 2025}{Yokohama, Japan}
% \acmISBN{978-1-4503-XXXX-X/18/06}

\copyrightyear{2025}
\acmYear{2025}
\setcopyright{cc}
\setcctype{by}
\acmConference[CHI '25]{CHI Conference on Human Factors in Computing Systems}{April 26-May 1, 2025}{Yokohama, Japan}
\acmBooktitle{CHI Conference on Human Factors in Computing Systems (CHI '25), April 26-May 1, 2025, Yokohama, Japan}
\acmDOI{10.1145/3706598.3713218}
\acmISBN{979-8-4007-1394-1/25/04}

\usepackage{framed}
\usepackage{tabularx}
\usepackage{graphicx}
\usepackage{multirow}
\usepackage{color, colortbl}
\usepackage{subfigure}
\usepackage{listings}
\usepackage{parcolumns}

\lstset{
  basicstyle=\ttfamily,
  columns=fullflexible,
  frame=single,
  breaklines=true,
  % postbreak=\mbox{\textcolor{red}{$\hookrightarrow$}\space},
}

\newcommand{\control}{\textbf{\texttt{Control}}}
\newcommand{\program}{\textbf{\texttt{CoTXAI}}}
\newcommand{\workflow}{\textbf{\texttt{CoTworkflow}}}
\newcommand{\workflowplus}{\textbf{\texttt{CoTworkflow+}}}

% We defined some commands to leave colorful comments in the paper
\newcommand{\glcomment}[1]{}
\newcommand{\gladd}[1]{#1}
\newcommand{\ujcomment}[1]{}
\newcommand{\ugadd}[1]{#1}
\newcommand{\gdcomment}[1]{}
\newcommand{\revise}[1]{#1}
% \newcommand{\glcomment}[1]{\textcolor{orange}{[Gaole: #1]}}
% \newcommand{\gladd}[1]{\textcolor{orange}{#1}}
% \newcommand{\ujcomment}[1]{\textcolor{red}{[\textbf{Ujwal: #1}]}}
% \newcommand{\ugadd}[1]{\textcolor{red}{#1}}
% \newcommand{\gdcomment}[1]{\textcolor{magenta}{[\textbf{Gianluca: #1}]}}
% \newcommand{\revise}[1]{\textcolor{blue}{#1}}

\newcommand{\ignore}[1]{}
\newcommand{\tabincell}[2]{\begin{tabular}{@{}#1@{}}#2\end{tabular}}
\usepackage{xspace}
\newcommand{\etal}{\emph{et al.}\xspace}
\newcommand{\paratitle}[1]{\vspace{1.0ex}\noindent\textbf{#1}}
\newcommand{\ie}{\textit{i.e.,}~}
\newcommand{\eg}{\textit{e.g.,}~}

%%
%% Submission ID.
%% Use this when submitting an article to a sponsored event. You'll
%% receive a unique submission ID from the organizers
%% of the event, and this ID should be used as the parameter to this command.
%%\acmSubmissionID{123-A56-BU3}

%%
%% For managing citations, it is recommended to use bibliography
%% files in BibTeX format.
%%
%% You can then either use BibTeX with the ACM-Reference-Format style,
%% or BibLaTeX with the acmnumeric or acmauthoryear sytles, that include
%% support for advanced citation of software artefact from the
%% biblatex-software package, also separately available on CTAN.
%%
%% Look at the sample-*-biblatex.tex files for templates showcasing
%% the biblatex styles.
%%

%%
%% The majority of ACM publications use numbered citations and
%% references.  The command \citestyle{authoryear} switches to the
%% "author year" style.
%%
%% If you are preparing content for an event
%% sponsored by ACM SIGGRAPH, you must use the "author year" style of
%% citations and references.
%% Uncommenting
%% the next command will enable that style.
%%\citestyle{acmauthoryear}

%%
%% end of the preamble, start of the body of the document source.
\begin{document}

%%
%% The "title" command has an optional parameter,
%% allowing the author to define a "short title" to be used in page headers.
% \title{Human-AI Collaboration with A Chain-of-thought Workflow: An Empirical Study in Composite Fact Verification}
%\title{Appropriate Reliance at Fine-grained Levels: An Empirical Study of Human-AI Collaboration with A Chain-of-thought Workflow}
\title{Plan-Then-Execute: An Empirical Study of User Trust and Team Performance When Using LLM Agents As A Daily Assistant}

%%
%% The "author" command and its associated commands are used to define
%% the authors and their affiliations.
%% Of note is the shared affiliation of the first two authors, and the
%% "authornote" and "authornotemark" commands
%% used to denote shared contribution to the research.
% \author{Authors Anonymized for Review Process}
\author{Gaole He}
\affiliation{%
  \institution{Delft University of Technology}
  \city{Delft}
  \country{The Netherlands}}
\email{g.he@tudelft.nl}

\author{Gianluca Demartini}
% \orcid{1234-5678-9012}
\affiliation{%
  \institution{The University of Queensland}
  \city{Brisbane}
  \country{Australia}
}
\email{g.demartini@uq.edu.au}

\author{Ujwal Gadiraju}
\affiliation{%
  \institution{Delft University of Technology}
  \city{Delft}
  \country{The Netherlands}
}
\email{u.k.gadiraju@tudelft.nl}
% \author{Ben Trovato}
% \authornote{Both authors contributed equally to this research.}
% \email{trovato@corporation.com}
% \orcid{1234-5678-9012}
% \author{G.K.M. Tobin}
% \authornotemark[1]
% \email{webmaster@marysville-ohio.com}
% \affiliation{%
%   \institution{Institute for Clarity in Documentation}
%   \streetaddress{P.O. Box 1212}
%   \city{Dublin}
%   \state{Ohio}
%   \country{USA}
%   \postcode{43017-6221}
% }

% \author{Lars Th{\o}rv{\"a}ld}
% \affiliation{%
%   \institution{The Th{\o}rv{\"a}ld Group}
%   \streetaddress{1 Th{\o}rv{\"a}ld Circle}
%   \city{Hekla}
%   \country{Iceland}}
% \email{larst@affiliation.org}

% \author{Valerie B\'eranger}
% \affiliation{%
%   \institution{Inria Paris-Rocquencourt}
%   \city{Rocquencourt}
%   \country{France}
% }

% \author{Aparna Patel}
% \affiliation{%
%  \institution{Rajiv Gandhi University}
%  \streetaddress{Rono-Hills}
%  \city{Doimukh}
%  \state{Arunachal Pradesh}
%  \country{India}}

% \author{Huifen Chan}
% \affiliation{%
%   \institution{Tsinghua University}
%   \streetaddress{30 Shuangqing Rd}
%   \city{Haidian Qu}
%   \state{Beijing Shi}
%   \country{China}}

% \author{Charles Palmer}
% \affiliation{%
%   \institution{Palmer Research Laboratories}
%   \streetaddress{8600 Datapoint Drive}
%   \city{San Antonio}
%   \state{Texas}
%   \country{USA}
%   \postcode{78229}}
% \email{cpalmer@prl.com}

% \author{John Smith}
% \affiliation{%
%   \institution{The Th{\o}rv{\"a}ld Group}
%   \streetaddress{1 Th{\o}rv{\"a}ld Circle}
%   \city{Hekla}
%   \country{Iceland}}
% \email{jsmith@affiliation.org}

% \author{Julius P. Kumquat}
% \affiliation{%
%   \institution{The Kumquat Consortium}
%   \city{New York}
%   \country{USA}}
% \email{jpkumquat@consortium.net}

%%
%% By default, the full list of authors will be used in the page
%% headers. Often, this list is too long, and will overlap
%% other information printed in the page headers. This command allows
%% the author to define a more concise list
%% of authors' names for this purpose.
% \renewcommand{\shortauthors}{Anonymous, et al.}
\renewcommand{\shorttitle}{An Empirical Study of User Trust and Team Performance with LLM Agents As A Daily Assistant}

%%
%% The abstract is a short summary of the work to be presented in the
%% article.
\begin{abstract}
Since the explosion in popularity of ChatGPT, large language models (LLMs) have continued to impact our everyday lives. 
Equipped with external tools that are designed for a specific purpose \revise{(\eg for flight booking or an alarm clock)}, LLM agents exercise an increasing capability to assist humans in their daily work. 
Although LLM agents have shown a promising blueprint as daily assistants, there is a limited understanding of {how they can provide daily assistance based on planning and sequential decision making capabilities}. We draw inspiration from recent work that has highlighted the value of `\textit{LLM-modulo}' setups in conjunction with humans-in-the-loop for planning tasks.
% lack of empirical evidence. 
% This paper analyzed how LLM agents can work as daily assistants. 
We conducted an empirical study ($N$ = 248) of LLM agents as daily assistants in six commonly occurring tasks with different levels of risk typically associated with them (\eg flight ticket booking and credit card payments). 
To ensure user agency and control over the LLM agent, we adopted LLM agents in a plan-then-execute manner, wherein the agents conducted step-wise planning and step-by-step execution \revise{in a simulation environment}. 
We analyzed how user involvement at each stage affects their trust and collaborative team performance. %with the LLM agents. %\glcomment{How about `user trust in LLM agents and collaborative team performance'. reliance is not explicitly analyzed, we only look into task performance (execution accuracy, \ie whether execution of sequence of actions will reach an expected task status). We do not calculate appropriate reliance-like measures}
Our findings demonstrate that LLM agents can be a double-edged sword --- (1) they can work well when a high-quality plan and necessary user involvement in execution are available, and (2) users can easily mistrust the LLM agents with plans that seem plausible. 
We synthesized key insights for using LLM agents as daily assistants to calibrate user trust and achieve better overall task outcomes. 
Our work has important implications for the future design of daily assistants and human-AI collaboration with LLM agents.
\end{abstract}

%%
%% The code below is generated by the tool at http://dl.acm.org/ccs.cfm.
%% Please copy and paste the code instead of the example below.
%%
% \begin{CCSXML}
% <ccs2012>
%  <concept>
%   <concept_id>00000000.0000000.0000000</concept_id>
%   <concept_desc>Do Not Use This Code, Generate the Correct Terms for Your Paper</concept_desc>
%   <concept_significance>500</concept_significance>
%  </concept>
%  <concept>
%   <concept_id>00000000.00000000.00000000</concept_id>
%   <concept_desc>Do Not Use This Code, Generate the Correct Terms for Your Paper</concept_desc>
%   <concept_significance>300</concept_significance>
%  </concept>
%  <concept>
%   <concept_id>00000000.00000000.00000000</concept_id>
%   <concept_desc>Do Not Use This Code, Generate the Correct Terms for Your Paper</concept_desc>
%   <concept_significance>100</concept_significance>
%  </concept>
%  <concept>
%   <concept_id>00000000.00000000.00000000</concept_id>
%   <concept_desc>Do Not Use This Code, Generate the Correct Terms for Your Paper</concept_desc>
%   <concept_significance>100</concept_significance>
%  </concept>
% </ccs2012>
% \end{CCSXML}

% \ccsdesc[500]{Do Not Use This Code~Generate the Correct Terms for Your Paper}
% \ccsdesc[300]{Do Not Use This Code~Generate the Correct Terms for Your Paper}
% \ccsdesc{Do Not Use This Code~Generate the Correct Terms for Your Paper}
% \ccsdesc[100]{Do Not Use This Code~Generate the Correct Terms for Your Paper}

\begin{CCSXML}
<ccs2012>
   <concept>
       <concept_id>10003120.10003121.10011748</concept_id>
       <concept_desc>Human-centered computing~Empirical studies in HCI</concept_desc>
       <concept_significance>500</concept_significance>
       </concept>
   <concept>
       <concept_id>10010147.10010178</concept_id>
       <concept_desc>Computing methodologies~Artificial intelligence</concept_desc>
       <concept_significance>500</concept_significance>
       </concept>
 </ccs2012>
\end{CCSXML}

\ccsdesc[500]{Human-centered computing~Empirical studies in HCI}
\ccsdesc[500]{Computing methodologies~Artificial intelligence}

%%
%% Keywords. The author(s) should pick words that accurately describe
%% the work being presented. Separate the keywords with commas.
% \keywords{Do, Not, Us, This, Code, Put, the, Correct, Terms, for,
%   Your, Paper}
\keywords{Human-AI Collaboration, Large Language Models, LLM agents, User Trust, Daily Assistant}

% \received{20 February 2007}
% \received[revised]{12 March 2009}
% \received[accepted]{5 June 2009}


\begin{teaserfigure}
    \centering
  \includegraphics[width=0.75\textwidth]{figures/plan_execute_illustration_2.pdf}
  \caption{Illustration of the human-AI collaboration with plan-then-execute LLM agents.}
  \Description{Illustration of the human-AI collaboration with plan-then-execute LLM agents. The planning stage mainly consists of two steps: (1) LLMs generate a draft plan (2) user edit the planning. Based on the edited plan, we get a step-wise plan (three steps in illustration). Then, in the execution stage, LLM agents will generate a sequence of actions, which is one-on-one mapping based on the plan's primary steps. Users will join this process to decide whether they approve the predicted action or choose to be involved (with manual specification of action or give text feedback to LLM agents). After the step-by-step execution of the plan, the task is solved.}
  \label{fig:illustration}
\end{teaserfigure}

%%
%% This command processes the author and affiliation and title
%% information and builds the first part of the formatted document.
\maketitle

%=========================================================================
% SparseZipper: Introduction
%=========================================================================

\section{Introduction}
\label{sec-spz-intro}

% *** Hardware trend: matrix extensions on CPUs ***
%   + Matrix extensions for CPUs are coming
%   + Why? --> accelerate dense matrix-matrix multiplications
%   + How? --> in the form of large 2D MAC array and work closely with existing vector architectures
%   + Benefits? --> being close to CPUs

General matrix multiply (GEMM) is a key building block in many different
domains including machine learning, graph analytics, and scientific
computing. Therefore, numerous domain-specific architectures have been
proposed to accelerate dense-dense GEMM (i.e., most values in both input
matrices are non-zeros) with various trade-offs in programmability,
performance, and energy
efficiency~\cite{jouppi-datacenter-isca2017,teich-google-tpu-v2-blog2018,chen-eyeriss-v2-jetcas2019,jouppi-google-tpu-v2-v3-cacm2020,choquette-tensor-core-nvidia-ieeemicro2021}.
In addition to coarse-grain accelerators, CPU vendors have recently
introduced matrix extensions (e.g., Intel's Advanced Matrix Extension
(AMX)~\cite{intel-amx-web,nassif-intel-sapphire-isscc2022,jeong-rasa-dac2021},
Arm's Scalable Matrix Extension (SME)~\cite{arm-sme-web}, RISC-V's matrix
extension proposal~\cite{riscv-mtx-ext-proposal-web}, and IBM's
Matrix-Multiply Assist (MMA)~\cite{ibm-mmx-assist-web}) to their ISAs for
dense-dense GEMM acceleration. Such matrix extensions attempt to strike a
balance between programmability and efficiency, and they are often
implemented using systolic-array-based
micro-architectures~\cite{intel-amx-web,nassif-intel-sapphire-isscc2022}.

% *** Software trend: Sparse computation ***
%   + Data are sparse and in many cases extremely sparse -> motivate the challenge
%   + Unstructured sparsity -> motivate the importance of a general solution
%   + Why dense hardware is no longer enough to do sparse?

However, matrices in workloads are not always dense. In fact, many recent
neural network
models~\cite{reddi-mlperf-isca2020,naumov-dnn-model-arxiv2019,han-deep-compress-arxiv2015,jouppi-datacenter-isca2017,wu-ml-facebook-hpca2019},
real-world graph
analytics~\cite{davis-graphblas-tmos2019,hoefler2011generic,shun-multicore-tc-2015},
and scientific
simulations~\cite{canning-sparse-sim-1996,galli-quantum-sim-1996} operate
on sparse matrices where the majority of values are zeros. In addition,
matrix densities (i.e., the percentage of non-zero values in a matrix)
vary dramatically across domains (e.g., from $10^{-6}\%$ density in
matrices representing social graphs to 50\% density in matrices used in
neural network models~\cite{hegde-extensor-micro2019}). Such low matrix
densities prevent computing GEMM for sparse matrices efficiently on CPUs
using the recently introduced matrix extensions since most
multiplications will involve at least one input value which is zero.
Moreover, sparse matrices are typically stored in compact formats with
metadata indicating positions of non-zero values for space efficiency, so
they are not directly compatible with existing built-in matrix engines
specialized for processing matrices stored in a dense format.

% *** Existing solutions (only closely related work) ***
%   + SparseCore - specialized ISA extension just for sparse computation
%   + VEGETA
%   + SparseTPU? Should we include this? It's not really an ISA extension

In addition to numerous domain-specific sparse-sparse GEMM (SpGEMM)
accelerators~\cite{qin-sigma-hpca2020, hegde-extensor-micro2019,
  zhang-sparch-hpca2020, srivastava-matraptor-micro2020,
  zhang-gamma-asplos2021}, previous work has proposed several ISA
extensions to accelerate sparse computations.
SparseCore~\cite{rao-sparsecore-asplos2022} is a stream-based ISA
extension designed specifically for sparse computations at the cost of
extra hardware for stream registers and stream processing units without
efficiently supporting dense-dense GEMM. VEGETA extends a matrix
extension to accelerate sparse-dense matrix-matrix multiplication (SpMM)
in addition to dense computations~\cite{jeong-vegeta-hpca2023}. However,
VEGETA is limited to SpMM and DNN-specific sparsity structures, so it is
not efficient when multiplying two highly sparse (i.e., less than 1\%
density) matrices with unstructured sparsity structures, which is critical in
various workload domains including graph analytics (e.g., multi-source
breadth-first search, peer pressure clustering, cycle detection, triangle
counting,
etc.)~\cite{dalberto-all-pair-spgemm-2007,shah-graph-spgemm-2007,rabin-maximum-matching-alg-1989,azad-triangle-count-2015},
hybrid linear solvers (e.g., Schur complement method and algebraic multi-grid
methods)~\cite{yamazaki-spgemm-schur-2010}, context-free grammar
parsing~\cite{penn-context-free-grammar-2006}, molecular dynamics
simulatio~\cite{itoh-order-n-spgemm-1995}, and interior point
methods~\cite{karypis-interior-point-alg-1994}.

% *** Our solution ***
% Augment matrix extensions to accelerate SpGEMM

In this work, we propose SparseZipper that minimally extends existing
matrix ISAs and systolic-array-based micro-architecture specialized for
dense-dense GEMM to accelerate SpGEMM operating on highly sparse matrices
with unstructured sparsity structures. SparseZipper targets a conventional
row-wise dataflow SpGEMM algorithm (i.e., Gustavson algorithm) with sparse
matrices represented in commonly used compressed sparse row/column
(CSR/CSC) formats. The abstraction and micro-architecture of SparseZipper
are specialized for accelerating the algorithm's main performance
bottleneck which involves merging multiple sparse vectors represented as
streams of indices (i.e., keys) and data (i.e., values).
By leveraging existing matrix registers for storing key-value streams and a
systolic array for merging multiple streams, SparseZipper incurs minimal area
overhead.
Our performance evaluation shows SparseZipper achieves 5.98$\times$ and
2.61$\times$ speedup over a scalar hash-based implementation of SpGEMM and a
state-of-the-art vectorized SpGEMM version, respectively.
Our component-level area evaluation shows SparseZipper increases the area of a
baseline 16$\times$16 systolic array by only 12.7\%.
This overhead would be much lower when considering an entire processor and its
caches.

%SparseZipper targets the key bottleneck, which is merging partial sparse
%vectors, in a conventional SpGEMM algorithm for data-parallel
%architectures~\cite{li-merge-spmspv-vectorarch-2018,li-spgemm-vector-arch-mchpc2019,fevre-spgemm-rvv-arxiv2023,winter-adaptive-spgemm-gpu-ppopp2019,liu-efficient-spgemm-gpu-ipdps2014,dalton-optimizing-spgemm-gpu-ipdps2015}.
%Each partial sparse vector is considered as a stream of keys (i.e.,
%representing row/column indices of non-zeros in a matrix) and corresponding
%non-zero values.
%At the core of SparseZipper is its ability to efficiently merge such streams in
%parallel by leveraging in-place matrix registers to store parts of concurrent
%streams and built-in systolic array to merge those streams together.
%In order to facilitate that merge operation, we propose a minimal set of
%additional architectural states to keep track of active streams and matrix
%instructions to move streams between matrix registers and memory.
%Our performance evaluations show SparseZipper achieves 5.98$\times$ and
%2.61$\times$ speedup over a scalar hash-based implementation of SpGEMM and a
%vectorized SpGEMM version respectively.
%Our post-synthesis area evaluation shows SparseZipper incurs less than XX\%
%area overhead compared to the baseline matrix engine designed for dense GEMM.

% *** Our contributions ***
\BF{Contributions --} Our key contributions include: (1)~a SparseZipper
ISA extension that enhances an existing matrix ISA to efficiently support
merging multiple key-value streams, the main performance bottleneck in
the conventional row-wise dataflow SpGEMM algorithm; (2)~a minimal set of
micro-architectural changes to a systolic array to support the new
SparseZipper instructions; and (3)~a detailed cycle-level evaluation
demonstrating the performance benefits of SparseZipper and a first-order
area evaluation demonstrating the minimal additional hardware needed for
SparseZipper.

%% Motivate sparse computation
%% + Important in various domains
%% + Example in graph analytics
%
%\paragraph{Motivating sparse matrix computations} Computation on sparse data,
%where a majority of values are zeros, lies at the heart of many important
%application domains such as scientific
%computing~\cite{gilbert-hpc-graphs-2007}, graph
%analytics~\cite{davis-graphblas-tmos2019,hoefler2011generic,shun-multicore-tc-2015},
%machine
%learning~\cite{reddi-mlperf-isca2020,naumov-dnn-model-arxiv2019,han-deep-compress-arxiv2015,jouppi-datacenter-isca2017,wu-ml-facebook-hpca2019},
%and simulation~\cite{canning-sparse-sim-1996,galli-quantum-sim-1996}.
%For instance, in the domain of graph analytics, a graph can be represented as a
%sparse adjacency matrix in which non-zero values in the matrix represent
%connections among vertices.
%Primitive graph operations (e.g., traversal and triangle counting) can be
%expressed in algebraic kernels such as sparse-matrix sparse-vector (SpMSpV) and
%sparse-matrix sparse-matrix (SpMSpM) multiplication~\cite{}.
%Such adjacency matrices are often highly sparse (e.g., only 0.2\% of elements
%in WikiVote network's adjacency matrix are non-zeros).
%
%% Why is it challenging?
%% + Sparse data -> most are zeros -> low arithmetic intensity & memory-bound
%% + Data dependent: matrix sparsity, structure, etc.
%% + Domain dependent
%% + Various data flows
%
%\paragraph{Challenges in sparse matrix computations} The high sparsity of data
%represented in matrix and vector format pose several computational challenges.
%First, using algorithms designed for processing dense data is algorithmically
%inefficient due to wasted multiplications with zero values.
%Therefore, sparse matrices and vectors are often represented in compact formats
%such as compressed sparse row/column (CSR/CSC) so that only non-zero values are
%stored and effectively processed.
%Second, computation on sparse data requires only few arithmetic operations per
%loaded data, which makes it inherently memory-bound.
%Third, the efficiency of sparse computation is largely dependent on input data.
%Previous work has proposed various custom compact matrix formats~\cite{} to
%exploit certain structures of non-zero values and a wide range of data-flows
%(e.g., inner product, outer product, and Gustavsons for SpMSpM) for particular
%levels of matrix sparsity.
%
%% Approach #1: General-purpose compute: GPGPU, vector, and multi-core
%
%\paragraph{General-purpose compute} multi-core CPU~\cite{},
%GPGPU~\cite{merrill-merge-spmv-gpu-sc2016}, vector
%architectures~\cite{nagasaka-intel-knl-icpp2018,li-merge-spmspv-vectorarch-2018}
%
%% Approach #2: Domain-specific accelerator
%
%\paragraph{Domain-specific accelerators} MatRaptor~\cite{}, GraphLily~\cite{},
%OuterSPACE~\cite{}, SpArch~\cite{}, Spada~\cite{}, ExTensor~\cite{},
%GAMMA~\cite{}, SIGMA~\cite{}
%
%% Approach #3: ISA extensions: sparse core
%%   - Middle ground
%%   - Try to argue that this is a right approach but existing work is not good
%%   enough (e.g., SparseCore - substantial architectural changes, VEGETA -
%%   specialized for a certain domain, SAVE - extending vector engine)
%
%\paragraph{ISA extensions} SparseCore~\cite{rao-sparsecore-asplos2022},
%VEGETA~\cite{jeong-vegeta-hpca2023}, SAVE~\cite{gong-save-micro2020}
%
%% Emerging matrix architectures
%%   - Why? and how?
%
%\paragraph{Emerging matrix architectures} Intel
%AMX~\cite{intel-amx-web,nassif-intel-sapphire-isscc2022}, Arm
%SME~\cite{arm-sme-web}, IBM matrix-multiply assist~\cite{ibm-mmx-assist-web},
%and RISC-V matrix extension proposal~\cite{riscv-mtx-ext-proposal-web}
%
%% Our solution
%%
%\paragraph{Our solution}
%
%% Our contributions
%\paragraph{Contributions} Our key contributions include:
%(1) an ISA extension that includes new matrix instructions to efficiently
%support merging streams of keys and values, a key operation in
%merge-based sparse matrix computations,
%(2) an implementation of SpMSpV and SpMSpM using the new matrix instructions
%and existing vector instructions,
%(3) a minimal set of micro-architectural changes to a systolic array-based
%matrix engine to support the new stream merging operation, and
%(4) a detailed cycle-level performance evaluation.

\section{Background and Related Work}
Our work proposes to analyze how user involvement in the planning and execution stages of LLM agents shapes user trust in the LLM agents and the overall task performance of LLM agents. 
Thus, we position our work in \revise{three realms} of related literature: human-AI collaboration (\S~\ref{sec-rel-collaboration-LLM}), \revise{trust and reliance on AI systems (\S~\ref{sec-rel-trust-reliance}),} task support with LLMs and LLM agents (\S~\ref{sec-rel-LLM-agent}). %\glcomment{Be careful with the human-AI collaboration. I find it a bit distracting from our focus}

%\glcomment{the paper did not provide a strong review of literature surrounding AI trust. While I am not an expert in understanding trust in AI, I do know there is rich literature in this area. Because this review was not included, it is difficult to evaluate the originality of this work.}\ujcomment{Improve the section, but also point towards recent reviews for readers to get a more comprehensive view (e.g., Siddharth Mehrotra's recent trust review paper)}

\subsection{Human-AI Collaboration}
\label{sec-rel-collaboration-LLM}
% LLM agent in CHI~\cite{zhang2024s}
% \ujcomment{- General human-AI collaboration; delegation; AI-assisted decision making; trust/reliance; complementary performance; 
% - Metrics introduced in recent years;
% - What has actually worked or shown promise in facilitating optimal human-AI collaboration?
% - How do we position our work in this context?}

% \paratitle{Delegation, algorithm appreciation, algorithm aversion, control}. 
In recent decades, deep learning-based AI systems have shown promising performance across various domains~\cite{yang2022survey,fernando2021deep} and applications~\cite{pouyanfar2018survey,dong2021survey}. 
However, such AI systems are not good at dealing with out-of-distribution data~\cite{jia2017adversarial,mccoy-etal-2019-right}, and their intrinsic probabilistic nature brings much uncertainty in %practical service
practice~\cite{ghahramani2015probabilistic}. 
Such observations raise wide concerns about the accountability and reliability of AI systems~\cite{kaur2022trustworthy}. 
Under such circumstances, human-AI collaboration has been recognized as a well-suited approach %one promising approach 
to taking advantage of their promising predictive power and ensuring trustworthy outcomes~\cite{lai2021towards,jiang2021supporting}. 
While humans can provide more reliable and accountable task outcomes, too much user involvement to check and control AI outcomes is undesirable~\cite{lai2022human}. 
It goes against the premise that AI systems are introduced to reduce human workload. 
In that context, researchers have theorized and empirically analyzed when and where users could and should delegate to AI systems~\cite{lai2022human,lubars2019ask}. 

\paratitle{Task Delegation}. While humans prefer to play the leading role in human-AI collaboration~\cite{lubars2019ask}, delegating to AI systems can bring benefits like cost-saving and higher efficiency. 
Apart from manual delegation decisions, it is common to apply automatic rules for human delegation (\eg heuristics obtained from domain expertise or manually crafted rules~\cite{lai2022human}).
% Humans can delegate to AI decisions based on . 
Many user factors like trust~\cite{lubars2019ask}, human expertise domain~\cite{erlei2024understanding}, and AI knowledge~\cite{pinski2023ai}) have a substantial impact on human delegation behaviors. 
% With an empirical study, Erlei \etal~\cite{erlei2024understanding} found that the human expertise domain impacts human delegation behaviors, and the choice independence will be violated when users consider AI performance in an unrelated task.
% Erlei \etal conducted an empirical study to analyze the impact of choice independence and error type in the appropriate delegation behaviors~\cite{erlei2024understanding}. 
%Besides human delegation to AI systems, 
Another relevant stream of recent research has explored AI delegation to humans~\cite{madras2018predict,fugener2022cognitive,pinski2023ai}. 
Researchers have investigated the conditions under which AI systems should defer to a human decision maker, which may bring benefits of improved fairness~\cite{madras2018predict}, accuracy~\cite{narasimhan2022post}, and complementary teaming~\cite{ijcai2022p344}. 
Compared to human delegation, AI delegation has been observed to achieve more consistent benefits in team performance~\cite{fugener2022cognitive,hemmer2023human}. {In collaboration with LLM agents, users need to determine when they should be involved in high-level planning and real-time execution. Such involvement decisions are similar to the delegation choices made by users. While task delegation is not the focus of our study, future work can explore this further.}% within human-LLM agent collaboration.}


\paratitle{AI-assisted Decision Making} has attracted a lot of research focus in human-AI collaboration literature. 
Most existing work has conducted empirical studies~\cite{lai2021towards} and structured interviews~\cite{jiang2021supporting} to understand how factors surrounding the user, task, and AI systems affect human-AI collaboration. 
User factors like AI literacy~\cite{Chiang-IUI-2022}, cognitive bias~\cite{rastogi2022deciding}, and risk perception~\cite{fogliato2021impact,green2021algorithmic} have been observed to substantially impact user trust and reliance on the AI system. 
Similarly, task characteristics like task complexity and uncertainty~\cite{salimzadeh2023missing,salimzadeh2024dealing} and factors of the AI system (\eg performance feedback~\cite{bansal2019beyond,Lu-CHI-2021}, AI transparency~\cite{vossing2022designing} and confidence of AI advice~\cite{tomsett2020rapid,Zhang-FAT-2020}) also affect user trust and reliance on the AI system. 
For a more comprehensive survey of existing work on AI-assisted decision making, readers can refer to~\cite{lai2021towards}.

% Typically, user trust is operationalized as a subjective attitude toward AI systems/AI advice within the literature on human-AI collaboration. In comparison, user reliance on AI systems is based on user behaviors (\eg adoption of AI advice and modification of AI outcomes). 
% Such formulation can even be dated back to trust and reliance on automation systems~\cite{lee2004trust}.

% \paratitle{Calibrated Trust and Appropriate Reliance}. User trust in the context of human-AI collaboration is typically operationalized as a subjective attitude toward AI systems/AI advice~\cite{lee2004trust}. In comparison, user reliance on AI systems is based on user behaviors (\eg adoption of AI advice and modification of AI outcomes). 
% As pointed out by existing work on trust in algorithmic/automated systems, user trust can substantially affect user reliance~\cite{lee2004trust}. 
% While trust calibration is an important goal in human-AI collaboration, it may be not enough to ensure complementary team performance. 
% Through empirical user studies with different confidence levels of AI predictions, Zhang \etal~\cite{Zhang-FAT-2020} found that ``trust calibration alone is not sufficient to improve AI-assisted decision making''. 
% To achieve optimal human-AI collaboration, humans and AI systems need to play complementary roles~\cite{hemmer2021human,hemmer2024complementarity}, and humans need to know when they should adopt AI assistance. 
% In other words, humans should rely on AI advice when AI systems are correct and outperform them, and override AI advice when AI systems are incorrect or less capable than humans. 
% Such user reliance patterns are denoted as \textit{appropriate reliance}~\cite{schemmer2022should,schemmer2023appropriate}, which is the key to
% achieving complementary team performance. 

% Compared with human assistance, users can easily lose confidence in AI systems after seeing them make the same mistakes~\cite{dietvorst2015algorithm}. 
% Such algorithm aversion can be overcome by enabling users to modify the AI predictions~\cite{dietvorst2018overcoming}. 
% As a result of these 
% under-reliance (disuse AI assistance when AI systems outperform humans) and over-reliance (misuse AI assistance when AI systems are wrong or perform worse than humans).
% , users show contradicting attitudes towards AI assistance: algorithm appreciation~\cite{logg2019algorithm,hou2021expert} and algorithm aversion~\cite{dietvorst2015algorithm,dietvorst2018overcoming}. 

% \paratitle{User Trust}. 
% Most existing work has conducted empirical studies~\cite{lai2021towards} and structured interviews~\cite{jiang2021supporting} to understand user trust in AI systems. 



% As a result of uncalibrated trust, users also show sub-optimal reliance on the AI systems: under-reliance (disuse AI assistance when AI systems outperform humans) and over-reliance (misuse AI assistance when AI systems are wrong or perform worse than humans).\glcomment{Is this claim true: unexpected reliance due to uncalibrated trust?}

% To achieve optimal human-AI collaboration, humans and AI systems are supposed to play complementary roles~\cite{hemmer2021human,hemmer2024complementarity}, and humans know when they should adopt AI assistance. 
% In other words, humans should rely on AI advice when AI systems are correct and outperform them, and humans should override AI advice when AI systems are incorrect or less capable than humans. 
% Such user reliance patterns are denoted as appropriate reliance~\cite{schemmer2022should,schemmer2023appropriate}, which is the key to
% achieving complementary team performance. 
% Many factors like cognitive bias~\cite{he2023knowing}, 
% \paratitle{Interventions to Facilitate Calibrated Trust and Appropriate Reliance} 
% The main issues that lead to sub-optimal human-AI collaboration are: under-reliance (\ie disuse AI assistance when AI systems outperform humans) and over-reliance (\ie misuse AI assistance when AI systems are wrong or perform worse than humans)~\cite{schemmer2022should}. 
% %These reliance behaviors are also highly relevant to uncalibrated user trust. 
% Users with an uncalibrated trust in the AI system can be easily misled to disuse or misuse AI systems~\cite{jacovi2021formalizing}. 
%For example, compared with human assistance, users can easily develop a negative impression of AI systems and lose confidence in AI systems. Such phenomenon is called algorithm aversion~\cite{dietvorst2015algorithm}. 
%By contrast, some users were influenced more by algorithmic decisions instead of human decisions, and they first coined the notion of ``Algorithm Appreciation''~\cite{logg2019algorithm}. 
% Researchers have proposed various interventions to promote appropriate reliance~\cite{he2023knowing,Lu-CHI-2021,lu2024does,chiang2021you,Chiang-IUI-2022} and calibrate user trust in AI systems~\cite{buccinca2021trust,Zhang-FAT-2020}.  
% % We bring some representative interventions here.
% %\glcomment{Here is not good enough. I don't plan to bring too many examples. To check how to improve}
% For example, explainable AI methods have been shown to help reduce over-reliance~\cite{vasconcelos2023explanations} and under-reliance~\cite{wang2021explanations} in different scenarios albeit with little consistency across contexts. 
% Another example is tutorial interventions, which have shown effectiveness in user onboarding~\cite{lai2020chicago}, mitigating cognitive biases~\cite{he2023knowing} and developing AI literacy~\cite{Chiang-IUI-2022}. 
% For a more comprehensive overview of interventions to facilitate trust calibration and appropriate reliance, readers can refer to ~\cite{lai2021towards,eckhardt2024survey}.

% In this work, we analyze how user involvement in the planning and execution stages of LLM agents will shape user trust and affect overall task performance. 
% It is highly relevant to existing studies of human-AI collaboration about user trust and appropriate reliance. 
While machine learning and deep learning methods have been extensively analyzed in existing human-AI collaboration literature, to our knowledge, human-AI collaboration with LLM agents is still under-explored. 
Unlike previous studies where AI systems only follow a fixed mode to generate advice, LLM agents can be equipped with more logical clarity and can provide a step-wise plan and can follow a step-by-step execution. 
With such a plan-then-execute setup, LLM agents can bring high flexibility as well as uncertainty in high-level planning and real-time execution. Little is known about
%Meanwhile, it is unclear 
how well LLM agents can work as daily assistants while handling tasks entailing varying stakes and potential risks. %where wrong actions may cause a loss. 
In our study, we analyzed the impact of user involvement in such AI systems by adjusting their intermediate outcomes (plan and step-by-step execution) to calibrate their trust and improve task outcomes. 
Our findings and implications can help advance the understanding of the effectiveness of LLM agents in human-AI collaboration.
% with humans.

%\glcomment{I find the positioning of our work in the space of human-AI collaboration is a bit challenging. Our major claim is: human-AI collaboration with LLM agents is under-explored}

\subsection{\revise{Trust and Reliance on AI systems}}
\label{sec-rel-trust-reliance}
\revise{Trust and reliance have been important research topics since human adoption of automation systems~\cite{lee2004trust,dzindolet2003role}. Due to the widespread integration of AI systems in nearly all walks of society, %In recent years, 
there has been a growing interest in understanding user %researchers have developed a strong interest in 
trust~\cite{vereschak2021evaluate,mehrotra2024systematic} and reliance~\cite{eckhardt2024survey} on AI systems.}
User trust in the context of human-AI collaboration is typically operationalized as a subjective attitude toward AI systems/AI advice~\cite{lee2004trust}. 
In comparison, user reliance on AI systems is based on user behaviors (\eg adoption of AI advice and modification of AI outcomes). 
\revise{The two constructs have been shown to be highly related~\cite{lee1992trust,lee2004trust}: for example, user trust can substantially affect user reliance~\cite{lee2004trust}. 
However, they are intrinsically different and cannot be viewed as a direct reflection of each other~\cite{kahr2024understanding}. 
Most existing work has, therefore, studied the two constructs separately in terms of subjective trust and objective reliance.}
% As pointed out by existing work on trust in algorithmic/automated systems, user trust can substantially affect user reliance~\cite{lee2004trust}. 

\revise{%In an early analysis of human-AI trust, most literature 
Earlier work exploring human-AI trust primarily focused on the impact of different contextual factors surrounding user (\eg risk perception~\cite{green2021algorithmic}), task (\eg task complexity~\cite{salimzadeh2023missing}), and system (\eg stated accuracy~\cite{yin2019understanding,Zhang-FAT-2020}). 
% Among this literature, performance indicators of the AI system (\eg stated accuracy~\cite{yin2019understanding,Zhang-FAT-2020} and confidence~\cite{rechkemmer2022confidence}) have been extensively studied. 
% For trust calibration, researchers have proposed different interventions like explanations~\cite{Zhang-FAT-2020}, educational tutorials~\cite{Chiang-IUI-2022,chiang2021you,lai2020chicago}, competence comparison~\cite{ma2023should}. 
%According to empirical studies about AI-assisted decision making~\cite{yin2019understanding}, 
Empirical studies have shown that most users tend to trust AI systems that are perceived to be highly accurate~\cite{yin2019understanding}. 
Such trust is vulnerable, as the AI system may provide an illusion of competence with persuasive technology (\eg explanations~\cite{chromik2021think,He-IUI-2025}) or overclaimed performance~\cite{yin2019understanding}. 
Even if the AI systems are accurate on specific datasets, they still suffer from out-of-distribution data~\cite{liu2021understanding,chiang2021you}. 
The misplaced trust in the AI systems can lead to misuse of the systems.
% But such trust can be fragile. 
Several empirical studies~\cite{tolmeijer2021second} have shown that once users realize the AI system errs or performs worse than expected, their trust in the AI system can be violated, %In the extreme case, it can 
even resulting in the disuse of the AI system. 
Both the misuse and disuse of the AI system hinder optimal human-AI collaboration. 
}

\revise{%To calibrate user trust in the AI system, 
To address such concerns, researchers have explored how to help users calibrate their trust in the AI system. %researchers proposed 
Different techniques to help users realize the trustworthiness of the AI system have been proposed~\cite{kaur2022trustworthy,rechkemmer2022confidence,ma2023should}. 
For example, increasing the transparency of AI systems by providing confidence scores~\cite{Zhang-FAT-2020}, explanations~\cite{wang2021explanations}, trustworthiness cues~\cite{liao2022designing}, and uncertainty communication~\cite{Sunnie-FAccT-2024}. 
However, the actual trustworthiness of the AI system does not always align with user perception. 
As found by \citet{banovic2023being}, untrustworthy AI systems can deceive end users to gain their trust. 
Another example is that users can develop an illusion of explanatory depth brought by explainable AI techniques~\cite{chromik2021think}, which leads to uncalibrated trust in the AI system. 
Even if users have indicated trust in the AI system, they may turn to rely more on themselves in final decision-making. 
The reasons are complex, and many factors, such as accountability concerns~\cite{lima2021human,tolmeijer2022capable} and cognitive bias~\cite{he2023knowing}, may affect user reliance behaviors. %Much research effort is required to further our understanding of trust calibration in AI systems.
% On the other hand, research has dived deep into calibrating user trust in AI systems.
}

While trust calibration is an important goal in human-AI collaboration, it may be not enough to ensure complementary team performance. 
Through empirical user studies with different confidence levels of AI predictions, Zhang \etal~\cite{Zhang-FAT-2020} found that ``trust calibration alone is not sufficient to improve AI-assisted decision making''. 
To achieve optimal human-AI collaboration, humans and AI systems need to play complementary roles~\cite{hemmer2021human,hemmer2024complementarity}, and humans need to know when they should adopt AI assistance. 
In other words, humans should rely on AI advice when AI systems are correct and outperform them, and override AI advice when AI systems are incorrect or less capable than humans. 
Such user reliance patterns are denoted as \textit{appropriate reliance}~\cite{schemmer2022should,schemmer2023appropriate}, which is the key to
achieving complementary team performance. 

The main issues that lead to sub-optimal human-AI collaboration are: under-reliance (\ie disuse AI assistance when AI systems outperform humans) and over-reliance (\ie misuse AI assistance when AI systems are wrong or perform worse than humans)~\cite{schemmer2022should}. 
Users with an uncalibrated trust in the AI system can be easily misled to disuse or misuse AI systems~\cite{jacovi2021formalizing}. 
Researchers have proposed various interventions to promote appropriate reliance~\cite{he2023knowing,Lu-CHI-2021,lu2024does,chiang2021you,Chiang-IUI-2022} and calibrate user trust in AI systems~\cite{buccinca2021trust,Zhang-FAT-2020}.  
For example, explainable AI methods have been shown to help reduce over-reliance~\cite{vasconcelos2023explanations} and under-reliance~\cite{wang2021explanations} in different scenarios albeit with little consistency across contexts. 
Another example is tutorial interventions, which have shown effectiveness in user onboarding~\cite{lai2020chicago}, mitigating cognitive biases~\cite{he2023knowing} and developing AI literacy~\cite{Chiang-IUI-2022}. 
For a more comprehensive overview of interventions to facilitate trust calibration and appropriate reliance, readers can refer to ~\cite{lai2021towards,eckhardt2024survey,mehrotra2024systematic,kahr2024understanding}.

\revise{LLM agents~\cite{wang2024survey} have gained much popularity in recent years, distinguishing them from most prior AI systems. 
They can communicate through conversation, plan logically, and can be built to leverage powerful external tools to achieve complex functions.
% The interaction between human and AI systems is relatively limited. 
% Users mostly develop trust and reliance on the AI systems via provided information cues (\eg, explanations and references). 
While trust and reliance have been extensively analyzed in existing human-AI collaboration literature, it is still unclear how users trust and rely on AI systems powered by LLM agents. 
% This work addressed this gap with empirical studies to analyze user trust and reliance on the plan-then-execute LLM agents. 
In our work, calibrated trust is adopted as an important goal for human-AI collaboration in the planning and execution stage. 
Meanwhile, users are expected to fix potential errors in the planning and execution stages, reflecting their reliance on the AI system. 
Our work can substantially advance the understanding of trust and reliance on plan-then-execute LLM agents.
}

% \paratitle{position our work}.

\subsection{Task Support with LLMs and LLM Agents}
\label{sec-rel-LLM-agent}
% \glcomment{Do we need to give more context about LLM Agent? It seems more technical}
LLMs and LLM agents bring new opportunities and challenges to human-AI collaboration~\cite{bommasani2021opportunities}. 
%On the one hand, based on the text generation capability of LLMs, it would be possible for humans to directly give text responses and communicate with any AI systems that take LLMs as their backbone. 
%To this end, 
%LLMs and LLM agents bring new opportunities for more flexible and natural interactions with humans. 
% Meanwhile, the natural language understanding and learning capabilities also enable LLMs to evolve with user feedback. 
It is evident that their generation capabilities can help reduce the cognitive effort from humans. %On the other hand, the LLMs also bring new challenges like dealing with 
But LLMs are also riddled with challenges such as hallucination~\cite{ji2023survey} (\ie generated text seems plausible but is factually incorrect). 
% Fatal errors can be made if users get misled by such persuasive technology, resulting in unaffordable costs (\eg medical diagnosis and financial decisions).
Failure to handle such issues may bring fatal errors with unaffordable costs depending on the context (\eg medical diagnosis). 
%As our work is within the scope of human-AI collaboration, users can refer to corresponding literature reviews to obtain more technical background about LLMs~\cite{zhao2023survey} and LLM agents~\cite{xi2023rise,wang2024survey}.

\begin{figure*}[h]
    \centering
    \includegraphics[width=0.75\textwidth]{figures/Screenshot-planning.png}
    \caption{Screenshot of user-involved planning interface.} 
    \label{fig:planning}
    \Description{Screenshot of user-involved planning interface. At the top, we show the task description along with three buttons: show potential actions, plan edit instruction, and add one step. At the bottom, we show a step-wise plan for setting alarms. Users can click these buttons to achive the function we described in the user-involved planning to edit the plan.}
\end{figure*}

% In recent years, LLMs have gained an explosion of popularity among academia and the industry community. 
Due to the capability of generating coherent, knowledgeable, and high-quality responses to diverse human input~\cite{wei2022emergent}, a wide community of human-computer interaction researchers has paid attention to large language models~\cite{liao2023ai}.
% With large volumes of data, large language models can obtain capabilities to help humans in writing. 
% Ideally, any complex task that can be modularized into a chain of different functions can also be achieved with chaining LLMs.
% LLMs can be used to generate dynamic user interface ~\cite{wang2023enabling}, support scientific writing~\cite{shen2023convxai}, and obtain high-quality data annotation~\cite{gilardi2023chatgpt,wang2024human,he2024if}. 
Researchers have actively explored how LLMs can assist users in various tasks like data annotation~\cite{wang2024human,he2024if}, programming~\cite{omidvar2024evaluating}, scientific writing~\cite{shen2023convxai}, and fact verification~\cite{si2024large}. 
All the above functions can be achieved with elaborate prompt engineering using a single LLM. 
By chaining multiple LLMs with different functions, humans can customize task-specific workflows to solve complex tasks~\cite{wu2022ai}. 
Apart from obtaining answers with a one-shot text generation, LLMs also provide convenient conversational interactions. 
Through empirical studies, such conversational interactions have been shown to be effective in human-AI collaboration with multiple applications, such as decision making~\cite{slack2023explaining,lin2024decision,ma2024towards}, scientific writing~\cite{shen2023convxai}, and mental health support~\cite{sharma2023human}. 
With the growing popularity of LLMs, more and more humans have begun to adopt LLMs (\eg ChatGPT) to boost their work efficiency and productivity %in their everyday work
~\cite{zhao2023survey}.

% Meanwhile, researchers also analyzed different system factors associated with LLMs. 
% For example, Kim \etal~\cite{Sunnie-FAccT-2024} found that the LLM's uncertainty Expression can decrease user trust in wrong AI advice, which helps reduce over-reliance.


LLM agents have been shown to have good planning, memory, and toolkit usage capabilities~\cite{xi2023rise, wang2024survey}. 
% External toolkits greatly increase the impact of human-AI collaboration on the real world. 
When suitable toolkits are provided, LLM agents can readily generate a task-specific plan and solve the tasks using toolkits. 
Attracted by the promise of LLM agents, there have been some early explorations~\cite{geissler2024concept,zheng2023synergizing,zhang2024s} of adopting them in human-AI collaboration contexts. 
%To our knowledge, only a few works~\cite{geissler2024concept,zheng2023synergizing,zhang2024s} have explored human-AI collaboration with LLM agents. 
% These works mainly analyzed how LLM agents can serve specific use cases (\eg design creation~\cite{geissler2024concept}) or conducted structured interviews to obtain expert insights~\cite{zhang2024s,zheng2023synergizing}. 
These works were mostly analyzed in specific use cases (\eg design creation~\cite{geissler2024concept}). 
% There is a lack of empirical studies on user trust and team performance in collaboration with LLM agents.
It is unclear how user trust and team performance are affected by user interactions with LLM agents in a sequential decision making setup (\ie solving a task by executing a sequence of actions) where users can be in control of the execution. %\glcomment{It is the only mention of sequential decision making. In my current framing, I try to give a very specific definition of what type of tasks we are focusing on. I mainly motivate LLM agents to provide daily assistance and facilitate daily life. Do you think we want to highlight sequential decision making?}
To fill this research gap and advance our understanding of user control over LLM agents, we carried out a quantitative empirical study. %to obtain empirical evidence.



% \ujcomment{- How do we position our work in this context?}

% \subsection{Human-agent collaboration}
% \glcomment{More traditional work of human-autonomous agent collaboration? Is it too far from our topic?}

% \subsection{User Trust and Reliance in Human-AI Collaboration}
% \label{sec-rel-trust}

\section{NumericBench Generation}
In this section, we present our created  NumericBench, which is specifically designed to evaluate fundamental numerical capabilities of LLMs. 
NumericBench consists of diverse datasets and tasks, 
enabling a systematic and comprehensive evaluation.
We discuss the datasets included in NumericBench, the key abilities it evaluates, and the methodology for benchmark generation.

\begin{table*}[t]
	\caption{NumericBench statistics. R: contextual retrieval, C: comparison, S: summary, L: logical reasoning. The token count is calculated based on tiktoken, which is the tokenizer used by Llama3~\cite{grattafiori2024llama3herdmodels}. The sentences used for token calculation include both the context and the question.}
	\centering
	\renewcommand{\arraystretch}{1.15} % 设置行间距为默认的 1.15 倍
	\setlength{\tabcolsep}{1.5pt} % 将列间距设置为 1pt
\resizebox{\textwidth}{!}{
	\begin{tabular}{c|c|c|c|c}
		\toprule
		\textbf{Data} & \textbf{Format} & \textbf{Questions} & \textbf{\# Instance} & \textbf{Avg Token} \\ \midrule
		
		\multirow{3}{*}{} 
		& \multirow{3}{*}{} 
		& \begin{tabular}[c]{@{}c@{}}R: What is the index of the first occurrence\\ of the number -3095 in the list?\end{tabular} 
		& 1000 & 3704.23 \\ \cline{3-5}
		
		\textbf{\begin{tabular}[c]{@{}c@{}}Number\\ List\end{tabular}}
		& $[69, -1, 6.1, \ldots, 5.7]$
		& \begin{tabular}[c]{@{}c@{}}C: Which index holds the smallest number\\
			 in the list between the indices 20 and 80?\end{tabular} 
		& 1000 & 3685.57  \\ \cline{3-5}
		
		& & \begin{tabular}[c]{@{}c@{}}S: What is the average of the index of\\
			 top 30 largest numbers in the list?\end{tabular} 
		& 1000 & 3654.78 \\ \midrule
		
		\multirow{3}{*}{} 
		& \multirow{3}{*}{
		\begin{tabular}[c]{@{}c@{}}
			\{date: 2024-06-19,\\
			close\_price: 9.79, \\
			open\_price: 9.4, \\
			\ldots \\
			PE\_ratio: 4.5416\}
		\end{tabular}
		} 
		& \begin{tabular}[c]{@{}c@{}}
			R: On which date did the close price\\
			 of stock firstly reach 61.76 yuan?
		\end{tabular}
		& 1000 & 27585.35 \\ \cline{3-5}
		
		\textbf{Stock}
		& 
		& \begin{tabular}[c]{@{}c@{}}
			C: Among the top-45 trading value days, which\\
			 date did the stock have the lowest close price?
		\end{tabular}
		 & 1000 & 27595.40 \\ \cline{3-5}
		
		& & \begin{tabular}[c]{@{}c@{}} 
			S: How many days had the close price higher than\\
			 the open price from 2024-07-31 to 2024-12-13?
		\end{tabular}	
		& 1000 & 27561.29 \\ \midrule
		
		\multirow{3}{*}{} 
		& \multirow{3}{*}{
		\begin{tabular}[c]{@{}c@{}}
			\{date: 2024-07-21,\\
			pressure\_msl: 999.96,\\
			dew\_point\_2m: 26.25,\\
			\ldots \\
			cloud\_cover: 61.5\}
		\end{tabular}
		} 
		& \begin{tabular}[c]{@{}c@{}} 
			R: On which date did the dew point temperature\\
			 at two meters firstly drop below 9.15°C?
		\end{tabular}
		& 1000 & 27359.26 \\ \cline{3-5}
		
		\textbf{Weather}
		& & \begin{tabular}[c]{@{}c@{}} 
			C: On which date did the MSL pressure reach its\\
			highest value when the cloud cover was below 9\%?
		\end{tabular}
		& 1000 & 27368.19 \\ \cline{3-5}
		
		& & \begin{tabular}[c]{@{}c@{}} 
			S: What was the average temperature at two meters\\
			when the relative humidity exceeded 78.56\%?
		\end{tabular}
		& 1000 & 27331.21 \\ \midrule
		
		\textbf{Sequence} 
		& $[0.34, 3, 6, \ldots, 111]$ 
		& L: What is the next number in the sequence? & 500 & 677.57 \\ \midrule
		
		\textbf{\begin{tabular}[c]{@{}c@{}}Arithmetic \\Operation\end{tabular}} 
		& \begin{tabular}[c]{@{}c@{}} 
		$a: 6.755,
		b: -1.225$
		\end{tabular}
		& \begin{tabular}[c]{@{}c@{}} 
		 $Q_{oper}$: What is the result of $a + b$?\\
		 $Q_{context}$: What is the result of $a $ plus $b$?
		 
		\end{tabular}
		& 12000 & 112.00 \\ \midrule
		
		\textbf{\begin{tabular}[c]{@{}c@{}}Mixed-number-string\\ Sequence\end{tabular}} 
		& \begin{tabular}[c]{@{}c@{}} 
		$effV2\ldots x98o7Lo$
		\end{tabular}
		& \begin{tabular}[c]{@{}c@{}} 
		How many numbers are there in the string? Note\\
		that a sequence like 'a243b' counts as a single number.
		\end{tabular}
		& 2000 & 196.53 \\ \bottomrule

	\end{tabular}
}
	\label{tab:data_stat}
	
\end{table*}

 

\subsection{Numeric Dataset Collection}
NumericBench offers a diverse collection of numerical datasets and questions designed to reflect real-world scenarios. 
This variety ensures that LLMs are thoroughly tested on their fundamental  abilities on numerical data.

\noindent\textbf{Number List Dataset.}
The synthetic number list dataset consists of simple collections of numerical values (integer and floats) 
presented as ordered or unordered lists.
Numbers in lists are one of the simplest and most fundamental data representations encountered in real-world scenarios.
Despite their simplicity, retrieving, indexing,  comparison, and summary on numbers can verify the fundamental numerical ability of LLMs. 
This dataset serves as a fundamental dataset of how well LLMs understand numerical values as discrete entities.



\noindent\textbf{Stock Dataset.}
The time-series  stock dataset is crawled from Eastmoney website~\cite{eastmoney}, 
which has eighteen attributes, such as stock close prices, open price,  trading volumes, and price-earnings ratio, over time.
Stock  data reflects dynamic, real-world numerical reasoning challenges that involve trend analysis, comparison, and decision-making under uncertainty,  representing real-world financial workflows.
 




\noindent\textbf{Weather Dataset.}
The weather dataset is crawled from Open-Meteo python API~\citep{openmeteo}, which includes data related to weather metrics, such as temperature, precipitation, humidity, and wind speed. 
The data is presented across various longitude and latitude.
 
 




\noindent\textbf{Numeric Sequence  Dataset.}
The synthetic numeric sequence dataset comprises sequences of numbers generated by arithmetic or geometric progression, complex patterns, or noisy inputs. 
Tasks require identifying patterns, predicting the next number, or reasoning about relationships between numbers.
Numerical sequences test the logical reasoning capabilities of LLMs, requiring pattern recognition and multi-step reasoning. This dataset introduces structured challenges that are common in mathematics and algorithmic reasoning.


 
\noindent\textbf{Arithmetic Operation Dataset.}
The dataset comprises 12,000 pairs of simple numbers, each undergoing addition, subtraction, multiplication, and division operations. Each pair of numbers, $a$ and $b$, consists of $k$-digit integers with three decimal places, where $k \in \{1, 2, \cdots, 6\}$. 
For each value of $k$, there are 2,000 pairs, evenly distributed across the four basic operations (i.e, $+, -,  *, /$), with 500 pairs per operation. 
This dataset is to evaluate the fundamental mathematical operation capabilities of LLMs, simulating the majority of mathematical calculation requirements in real-world scenarios.

\noindent\textbf{Mixed-number-string Sequence Dataset.}
The dataset consists of alphanumeric strings of varying lengths $\{50, 100, 150, 200\}$, each containing a randomized mix of letters and digits. For each string length, 500 samples are generated, resulting in a total of 2,000 samples. Each sample includes a query asking for the count of contiguous numeric sequences within the string, where sequences like "a243b" count as a single number. This dataset is designed to assess the ability of LLMs to identify and count numeric sequences.
 







\subsection{Fundamental Numerical Ability}
NumericBench is designed to comprehensively evaluate six fundamental numerical reasoning abilities of LLMs, which is 
%These three fundamental abilities are 
essential for solving real-world numeric-related tasks.
%such as numeric data summary and financial price analysis.


\noindent\textbf{Contextual Retrieval Ability.}
Contextual retrieval ability evaluates how well LLMs can locate, extract, and identify specific numerical values or their positions within structured or unstructured data. 
This includes tasks like finding a specific number in a list, retrieving values , and indexing numbers based on their order.
For example, as shown in Table~\ref{tab:data_stat}, it evaluates LLMs on tasks such as retrieving stock prices and identifying key values within numerical lists or domain-specific data (e.g., stock market and weather-related information).
This ability is fundamental to numerical reasoning because it forms the foundation for higher-order tasks, such as comparison, aggregation, and logical reasoning. 
 
 



\noindent\textbf{Comparison Ability.}
Comparison ability evaluates how well LLMs can compare numerical values to determine relationships such as greater than, less than, or equal to, and identify trends or differences in datasets. 
Comparison is vital for logical reasoning and decision-making, as many real-world tasks depend on accurate numerical evaluation. 
For instance,  as shown in Table~\ref{tab:data_stat},   comparing prices is essential in stock  for assessing performance, while weather forecasting requires analysis of temperature or precipitation trends over time. 
 



\noindent\textbf{Summary Ability.}
Summary ability assesses the LLM’s capacity to aggregate numerical data into concise insights, such as calculating totals, averages, or other statistical metrics. 
Summarization is critical for condensing large datasets into actionable information, enabling decision-making based on aggregated insights rather than raw data. 
This ability is indispensable in domains like electricity usage analysis, where summarizing hourly or daily consumption helps forecast bills, in business reporting for aggregating sales and revenue data to evaluate performance, 
and in healthcare analytics to monitor trends in patient metrics over time.



\noindent\textbf{Logic Reasoning Ability.}
Logical Reasoning Ability measures the LLM’s ability to perform multi-step operations involving numerical data, 
such as recognizing patterns, inferring rules, and applying arithmetic or geometric reasoning to solve complex problems. Logical reasoning extends beyond simple numerical tasks and reflects the LLM’s capacity for deeper, structured thinking. 
This ability is crucial for algorithm design, where solving problems involving numeric sequences or patterns is essential, in scientific research for identifying relationships and correlations in data.

\noindent\textbf{Arithmetic Operation Ability.}
It reflects the LLM's capacity to perform mathematical calculations accurately. Such ability is essential for tasks involving numerical computations, such as  automated machine learning through LLMs.





\noindent\textbf{Number Recognition  Ability.}
This measures the LLM's proficiency in identifying and interpreting numerical information within a given context. It represents a fundamental requirement for handling numeric-based tasks effectively.




\subsection{NumericBench Generation}
We use the number list, stock, and weather datasets to evaluate the contextual retrieval, comparison, and summary abilities of LLMs. 
Specifically, for each ability and each dataset, we prepare a set of questions designed to assess the corresponding target ability.
As shown in Table~\ref{appx:number_question}, Table~\ref{appx:stock_question}, and Table~\ref{appx:weather_question} in Appendix, there are nine question sets in total, covering three abilities across three datasets. 
When evaluating a specific ability (e.g., contextual retrieval) on a specific dataset (e.g., stock data), we randomly select one question from the corresponding question set for each data instance (e.g., a stock instance) 
and manually label the answer. This approach enables us to generate question-answer pairs for each ability on the number list, stock, and weather datasets.

For arithmetic operations and number counting in the strings dataset, the question format is straightforward, as illustrated in Table~\ref{tab:data_stat}. These questions are designed to evaluate the basic arithmetic operation and number recognition abilities of LLMs.



\section{Study Design}

This section describes our experimental conditions, tasks,  variables, procedure, and participants in our study. {Our
study was approved by the human research ethics committee of our institution.}

\subsection{Experimental Conditions}
In our study, users collaborate with LLM agent-based daily assistants in two stages: planning and execution. 
To comprehensively understand the effect of user involvement at each stage, we considered a 2 × 2 factorial design with four experimental conditions: (1) automatic planning, automatic execution (represented as AP-AE), (2) automatic planning, user-involved execution (represented as AP-UE), (3) user-involved planning, automatic execution (represented as UP-AE), (4) user-involved planning, user-involved execution (represented as UP-UE). 
In conditions with user-involved planning, users are allowed to edit the plan generated by LLM with the actions of edit/add/delete/split step. 
By comparison, in conditions with automatic planning, users will directly adopt the plan generated by the daily assistant.
In conditions with user-involved execution, users can interact with the step-by-step execution LLM agent (cf. Section~\ref{sec-method-execution}) and refine execution results with text feedback or manual specification. 
By comparison, in conditions with automatic execution, users will directly accept the automatic execution results.

\begin{table*}[htbp]
	\centering
	\caption{Selected tasks in our study. The `Risk' is based on the risk feedback obtained with pilot study. \#A and \#C refer to the number of actions and the number of named concepts in each task, respectively. }
	\label{tab:tasks}%
	\scalebox{.8}{
    \begin{tabular}{p{0.02\textwidth}|p{0.04\textwidth}|p{0.06\textwidth}|p{0.8\textwidth}|p{0.03\textwidth}|p{0.03\textwidth}| p{0.12\textwidth}}
		\hline
% 		\textbf{Participant Feedback} &  \textbf{Sentiment}& \textbf{Reason}  \\
        \textbf{ID}&\textbf{Risk}& \textbf{Domain}&  \textbf{Task Description}& \textbf{\#A}& \textbf{\#C}& \textbf{Notes}\\
		\hline \hline
		% Low& Finance& Can you help me log into my two different platform accounts and then check their account balances? The first account ID is 12345678, password is Password123; the second account ID is 87654321, password is 123Password.& Tutorial\\
		% \hline \hline
		1& High& Finance& My account ID is 54321, and the password is PWD2023. I plan to make two foreign exchange transactions. The first is to buy 10,000 euros (with USD), and the second is to sell 5,000 US dollars (to EUR). Please help me operate.&4&4& simple task, imperfect plan\\
        % \hline
        % High& Finance& I need to know the detailed information about the 'Happy Savings High Gold' deposit product, including its minimum deposit amount, annual interest rate, and deposit term. Also, I want to use my account number 123456 and password 789123 along with the most recently received verification code 8888 to apply for a loan, and I would like to know the review time for this loan application as well as how to check the status of all my current loan applications.\\
        \hline
        2& High& Finance& Please inquire about the current debt amount of my credit card with the last five digits 12345, and deduct the corresponding 12000 USD from my savings card number 6212345678900011 to repay this debt, then help me check the amount of the outstanding bill for the same credit card within 30 days after today.&4&6& complex task, imperfect plan\\
        \hline
        3& High& Repair& I need to schedule a repair for my TV at 6 PM tomorrow evening. The brand is Sony, model X800H, and there is an issue with the screen. Please book the repair service and tell me the reservation number.&4&7& complex task, imperfect plan\\
        % \hline
        % Low& Tracking& Please check the latest status of my two orders with the numbers 123456789 and 987654321, and confirm whether they are both associated with my customer ID A123456.\\
        % \hline
        % High& Restaurant& Please order a Spicy Hot Pot for me at the restaurant, add two extra servings of beef and a plate of hand-torn cabbage, then place the order using my table ID 10, and help me check out.\\
        \hline
        4& Low& Alarm& I need to set an alarm for every weekday morning at 7:30, and then cancel the alarm for Thursday, changing it to 8:00 in the evening.&2&3& simple task, correct plan\\
        \hline
        5& Low& Flight& I have an important meeting to attend next Wednesday, and I need to book a flight ticket from London to Amsterdam for tomorrow, it must be a morning flight, and then return from Amsterdam to London tomorrow night, please handle it for me.&2&6& simple task, correct plan\\
        \hline
        6& Low& Travel& Please plan a trip for me departing on October 1st at 8:00 AM to Japan, returning on October 7th at 11:00 PM, including Tokyo Disneyland, Senso-ji Temple, Ginza, Mount Fuji, Kyoto cultural experience, Universal Studios Osaka, and visiting the Nara Deer Park on October 4th, and help me find hotels where the nightly cost does not exceed 10,000 Japanese yen.&3&11& complex task, correct plan\\
        \hline
            \hline
    \end{tabular}}
\end{table*}%

\subsection{Tasks}
To analyze how LLM agents can serve as daily assistants, we adopted tasks from a planning dataset designed for LLM agents --- UltraTool~\cite{huang2024planning}. 
We selected daily scenarios: currency transactions, credit card payments, repair service appointments, alarm setting, flight ticket booking, and trip itinerary planning. 
The selected tasks are shown in Table~\ref{tab:tasks}. \revise{For more details about how the plan-then-execute LLM agent works on the selected tasks (\eg automatic plan, pre-defined actions, automatic evaluation, and explanation for errors in automation), please refer to the appendix.}
All tasks in UltraTool dataset are annotated with the step-wise plan format described in Section~\ref{sec-method-planning}. \revise{The execution of these tasks is based on a simulation environment (described in Section~\ref{sec-method-execution}) where all required actions are implemented as backend APIs. 
In our study, all tasks are executed in a simulation setup, which has been a popular method for orchestrating meaningful human-centered AI studies~\cite{doshi2017towards,salimzadeh2024dealing}.} %is effective in developing and validating theory~\cite{davis2007developing} and has been widely adopted in existing research on agent-based modeling and HCI studies~\cite{olson2014ways}.}
%\glcomment{It is unclear if the participants were performing the real tasks or were in a scenario-based setting. The concern impacts the validity of the study design. The clarification along with simulation design justification, would help readers better understand the results.}\ujcomment{Clarify, justify this as a valid method used widely in HCAI research.}\glcomment{I clarified in the execution part (Section 3.3). If you feel it is not necessary to highlight again, we can remove the last sent here.}

% \begin{table*}[htbp]
% 	\centering
% 	\caption{Selected tasks in our study. The `Risk' is based on the risk feedback obtained with pilot study.}
% 	\label{tab:tasks}%
% 	\scalebox{.84}{
%     \begin{tabular}{p{0.02\textwidth}|p{0.04\textwidth}|p{0.07\textwidth}|p{0.8\textwidth}| p{0.12\textwidth}}
% 		\hline
% % 		\textbf{Participant Feedback} &  \textbf{Sentiment}& \textbf{Reason}  \\
%         \textbf{ID}&\textbf{Risk}& \textbf{Domain}&  \textbf{Task Description}& \textbf{Notes}\\
% 		\hline \hline
% 		% Low& Finance& Can you help me log into my two different platform accounts and then check their account balances? The first account ID is 12345678, password is Password123; the second account ID is 87654321, password is 123Password.& Tutorial\\
% 		% \hline \hline
% 		1& High& Finance& My account ID is 54321, and the password is PWD2023. I plan to make two foreign exchange transactions. The first is to buy 10,000 euros (with USD), and the second is to sell 5,000 US dollars (to EUR). Please help me operate.& simple task, imperfect plan\\
%         % \hline
%         % High& Finance& I need to know the detailed information about the 'Happy Savings High Gold' deposit product, including its minimum deposit amount, annual interest rate, and deposit term. Also, I want to use my account number 123456 and password 789123 along with the most recently received verification code 8888 to apply for a loan, and I would like to know the review time for this loan application as well as how to check the status of all my current loan applications.\\
%         \hline
%         2& High& Finance& Please inquire about the current debt amount of my credit card with the last five digits 12345, and deduct the corresponding 12000 USD from my savings card number 6212345678900011 to repay this debt, then help me check the amount of the outstanding bill for the same credit card within 30 days after today.& complex task, imperfect plan\\
%         \hline
%         3& High& Repair& I need to schedule a repair for my TV at 6 PM tomorrow evening. The brand is Sony, model X800H, and there is an issue with the screen. Please book the repair service and tell me the reservation number.& complex task, imperfect plan\\
%         % \hline
%         % Low& Tracking& Please check the latest status of my two orders with the numbers 123456789 and 987654321, and confirm whether they are both associated with my customer ID A123456.\\
%         % \hline
%         % High& Restaurant& Please order a Spicy Hot Pot for me at the restaurant, add two extra servings of beef and a plate of hand-torn cabbage, then place the order using my table ID 10, and help me check out.\\
%         \hline
%         4& Low& Alarm& I need to set an alarm for every weekday morning at 7:30, and then cancel the alarm for Thursday, changing it to 8:00 in the evening.& simple task, correct plan\\
%         \hline
%         5& Low& Flight& I have an important meeting to attend next Wednesday, and I need to book a flight ticket from London to Amsterdam for tomorrow, it must be a morning flight, and then return from Amsterdam to London tomorrow night, please handle it for me.& simple task, correct plan\\
%         \hline
%         6& Low& Travel& Please plan a trip for me departing on October 1st at 8:00 AM to Japan, returning on October 7th at 11:00 PM, including Tokyo Disneyland, Senso-ji Temple, Ginza, Mount Fuji, Kyoto cultural experience, Universal Studios Osaka, and visiting the Nara Deer Park on October 4th, and help me find hotels where the nightly cost does not exceed 10,000 Japanese yen.& complex task, correct plan\\
%         \hline
%             \hline
%     \end{tabular}}
% \end{table*}%

\paratitle{Task Selection}. First, based on the domain distribution of the UltraTool dataset, we selected seven domains: Finance, Alarm, Travel, Tracking, Restaurant, Flight, and Repair. 
For each domain, we only \revise{consider} tasks that contain more than ten steps (including all sub-steps) and require at least three uses of actions. 
Then, we manually selected ten tasks: four from the finance domain and one for each of the others. 
With a pilot study, we tested how users work on the ten tasks. 
We recruited 10 participants from the Prolific platform and only considered the feedback of 9 participants who passed all attention checks. 
\revise{Using} the question “How much risk do you perceive in this task when relying on this daily AI assistant?”, we collected the perceived risk of working with the LLM agents on each task \revise{using a 5-point Likert scale, ranging from \textit{1: not risky at all}---to---\textit{5:very risky}.} %\ujcomment{mention the scale}
We categorize the ten tasks into a high-risk group (top 5) and a low-risk group (bottom 5). 
We selected three tasks from each group while balancing the complexity of the task description (three simple tasks and three complex tasks) and the correctness of the provided plan (three correct plans and three imperfect plans). 
\revise{Based on existing literature on task complexity~\cite{wood1986task,salimzadeh2023missing}, we  considered component complexity to inform our selection. 
This is assessed as the `total number of distinct information cues that need to be processed to perform the task'. 
Here, we considered the number of unique actions and the number of named concepts provided in each task.}
According to prior work~\cite{miller1956magical}, most people can only handle 5 to 9 concepts at the same time. 
The component complexity of all complex tasks in our study is more than nine.
% \glcomment{Clarify here for the design of simple/complex tasks}\ujcomment{Task complexity (Robert Wood) -> Task Complexity (cite Sara's work)}
The six tasks selected are shown in Table~\ref{tab:tasks}. 
Besides the six tasks, we used one simple task (\ie checking bank account balance) as the example in the onboarding tutorial.


\subsection{Measures and Variables}
% \subsection{Evaluation}
\label{sec-measure}

The variables and measures used in our study refer to existing empirical studies of human-AI collaboration~\cite{lai2021towards}. 
All measures adopted in our study can be summarized in Table~\ref{tab:variables}. 
\begin{table*}[htbp]
	\centering
	\caption{The different variables considered in our experimental study. ``DV'' refers to the dependent variable.}
	\label{tab:variables}
	\begin{footnotesize}
	\begin{tabular}{c | c | c | c}
	    \hline
	    \textbf{Variable Type}&	\textbf{Variable Name}& \textbf{Value Type}& \textbf{Value Scale}\\
	    \hline \hline

	    \hline
         \multirow{2}{*}{Calibrated Trust (DV)}& Calibrated Trust in planning (CT$_p$)& Binary& 0: miscalibrated trust, 1: calibrated trust\\
        & Calibrated Trust in execution (CT$_e$)& Binary& 0: miscalibrated trust, 1: calibrated trust\\
        \hline
	    \multirow{3}{*}{Task Performance (DV)}& Plan Quality& Likert& 5-point, 1: low, 5: high\\
        % & Outcome Accuracy& Continuous, Interval& [0.0, 1.0]\\
        % & Action Sequence Accuracy - Relaxed& Continuous, Interval& [0.0, 1.0]\\
        & Action Sequence Accuracy (ACC$_s$)& Binary& 0: mismatch, 1: exact match with ground truth\\
        & Execution Accuracy (ACC$_e$)& Binary& 0: wrong execution result, 1: correct execution result\\
        % & Action Sequence Recall& Continuous, Interval& [0.0, 1.0]\\
	    % \hline
	    % \multirow{4}{*}{Reliance (DV)}& Agreement Fraction& Continuous, Interval& [0.0, 1.0]\\
	    % & Switch Fraction& Continuous& [0.0, 1.0]\\
	    % & RAIR& Continuous& [0.0, 1.0]\\
	    % & RSR& Continuous& [0.0, 1.0]\\
	    \hline
     % \multirow{2}{*}{Assessment (DV)}& Degree of Miscalibration& Continuous, Interval& [0,6]\\
     % & Self-assessment& Continuous, Interval& [-6,6]\\
     % \hline
     \multirow{4}{*}{Trust}& Reliability/Competence& Likert& 5-point, 1: poor, 5: good\\
    & Understanding/Predictability& Likert& 5-point, 1: poor, 5: good\\
    & Intention of Developers& Likert& 5-point, 1: poor, 5: good\\
    & Trust in Automation& Likert& 5-point, 1:strong distrust, 5: strong trust\\
    \hline
     % \multirow{4}{*}{Trust (DV)}& TiA-Trust& Likert& 5-point, 1:strong distrust, 5: strong trust\\
     % & TiA-Trust& Likert& 5-point, 1:strong distrust, 5: strong trust\\
     % & TiA-Trust& Likert& 5-point, 1:strong distrust, 5: strong trust\\
     % & TiA-Trust& Likert& 5-point, 1:strong distrust, 5: strong trust\\
     % \hline
     \multirow{4}{*}{Covariates}& LLM Expertise& Likert& 5-point, 1: No experience, 5: Extensive experience\\
     & Automatic Assistant Expertise& Likert& 5-point, 1: No experience, 5: Extensive experience\\
     & Propensity to Trust& Likert& 5-point, 1: tend to distrust, 5: tend to trust \\
     & Familiarity& Likert& 1: unfamiliar, 5: very familiar\\
     \hline
	\multirow{5}{*}{Exploratory}& Confidence& Likert& 5-point, 1: unconfident, 5: confident\\
    & Risk Perception& Likert& 5-point, 1: not risky at all, 5: very risky\\
    & Open Feedback on Planning& Text& Open Text\\
    & Open Feedback on Execution& Text& Open Text\\
    & Other Open Feedback& Text& Open Text\\
    % & Cognitive Load& Likert& -7: very low, 7: very high\\
	    \hline
     \multirow{6}{*}{{Cognitive Load }}& Mental Demand& Likert& -7: very low, 7: very high\\
	    & Physical Demand& Likert& -7: very low, 7: very high\\
	    & Temporal Demand& Likert& -7: very low, 7: very high\\
	    & Performance& Likert& -7: Perfect, 7: Failure\\
        & Effort& Likert& -7: very low, 7: very high\\
        & Frustration& Likert& -7: very low, 7: very high\\
     \hline
	\end{tabular}
	\end{footnotesize}
\end{table*}

\paratitle{Calibrated Trust}. To assess calibrated trust in the planning stage and execution stage, we assessed user trust at each stage with a question ``Do you trust that [the execution of this plan / the execution process] can provide a correct outcome based on the task instructions?''. 
Based on the plan quality evaluation (5-point Likert), the calibrated trust in the planning (CT$_p$) is calculated based on the frequency at which users trusted the high-quality plan (expert annotation with 5) and users distrusted the plan with other evaluation results.
% will expect that users can trust the plan when the plan quality is annotated as 5. Otherwise, users should indicate distrust. 
Similarly, for the calibrated trust in execution (CT$_e$), we calculated the frequency at which users trusted the correct execution results and distrusted the wrong execution results. 
The two measures can be calculated as:

\begin{equation}
\begin{aligned}
    \textnormal{CT}_p = &\mathbb{I}\left( \textnormal{trust = `Yes'}, \textnormal{plan quality} = 5\right) \\ & +\mathbb{I}\left( \textnormal{trust = `No'}, \textnormal{plan quality} < 5\right)
\end{aligned}
\end{equation}

\begin{equation}
    \textnormal{CT}_e = \mathbb{I}\left( \textnormal{trust = `Yes'}, \textnormal{ACC}_e = 1\right) + \mathbb{I}\left( \textnormal{trust = `No'}, \textnormal{ACC}_e = 0 \right)
\end{equation}

To assess the task performance, we mainly considered the task outcome from the planning and execution stages. 

\paratitle{Plan Quality}. As for the planning outcome, we evaluate the plan quality based on a 5-point Likert scale: 
1. low-quality plan, task requirements not covered; 
2. low-quality plan, task requirements covered but with grammar errors; 
3. medium-quality plan, task requirement covered but with at least one action intent mismatch with ground truth action sequence; 
4. medium-quality plan, task requirements covered but miss or have wrong details for action parameters; 
5. high-quality plan, covering all task requirements and providing all necessary details.

\paratitle{Execution Performance}. 
The execution of the step-wise plan will result in an action sequence. 
We provide a ground truth action sequence as a reference to evaluate the generated action sequence. 
We measure the action sequence accuracy (ACC$_s$) as the strict match of the action sequence and ground truth. 
% Meanwhile, there are some actions that do not harm the execution results (\eg searching for flights). 
Meanwhile, if one action sequence contains some redundant actions that are not harmful (\eg searching for flights), the execution results should still be correct. 
Thus, we also consider execution accuracy (ACC$_e$) as a task performance measure.

\paratitle{Subjective Trust and Covariates}. To enrich our analysis of user trust, we followed existing work to adopt the six subscales from the Trust-in-automation questionnaire~\cite{korber2019theoretical}. 
The four subscales --- \textit{Reliability/Competence},  \textit{Understanding/Predictability},  \textit{Intention of Developers},  \textit{Trust in Automation} are used as subjective measures of user trust in the LLM agent. 
Meanwhile, the \textit{Familiarity} and \textit{Propensity to Trust} are also used as covariates. Besides them, we considered user expertise in LLMs and user expertise in automatic assistants as covariates. 

\paratitle{Exploratory Variables}. To enrich our understanding of LLM agent as daily assistant, we assessed user confidence (both planning and execution) and risk perception along with each task. 
After users finish the study, we also ask for their open-text feedback on the planning and execution stages as well as other comments. 
To check the cognitive load of user involvement in our study, we adopted the NASA-TLX questionnaire~\cite{colligan2015cognitive}, which contains six subscales.

% The variables and measures used in our study refer to existing empirical studies of human-AI collaboration~\cite{lai2021towards}. 
% All measures adopted in our study can be  summarized in Table~\ref{tab:variables}. %\glcomment{We can consider removing this table to save space, or move it to supplementary materials}

\subsection{Participants}

\paratitle{Sample Size Estimation}. 
To ensure sufficient statistical power, we estimated the required sample size for a 2 × 2 factorial design based on G*Power~\cite{faul2009statistical}. 
{To correct for testing multiple hypotheses, we applied a Bonferroni correction so that the significance threshold decreased to $\frac{0.05}{4}=0.0125$.} 
We specified the default effect size $f = 0.25$
(\textit{i.e.,} indicating a moderate effect), a significance threshold $\alpha = 0.0125$ (\textit{i.e.,} due to testing multiple hypotheses), a statistical power of $(1 - \beta) = 0.8$, and that we will investigate $4$ different experimental conditions/groups. 
This resulted in a required sample size of $244$ participants. 
We thereby recruited 347 participants from the crowdsourcing platform Prolific\footnote{\url{https://www.prolific.co}}, to accommodate potential exclusion.

\paratitle{Compensation}. All participants were rewarded with an hourly wage of \pounds 8.1 deemed to be ``\textit{Fair}'' payment by the platform (estimated completion time was 30 minutes). 
As participants in condition UP-UE spent longer in the study, we paid each participant a commensurate bonus accounting for an extra 10 minutes. % with a bonus.
% \stcomment{This makes for 1.8 * 6 = 8.1 pounds hourly wage. If I do 1.8/7.5 then I get to estimated completion time of 14.4 minutes}~\glcomment{I checked it again, actually we run two rounds for main studt. First with estimation of 12 minutes (\pound 1.5), and second with 10 minutes (\pound 1.25). I think we can write it as 1.5. I included the service fee and calculated the mean for all participants.}
We rewarded participants with extra bonuses of \pounds 0.05 for every high-quality plan and correct execution result. 
According to existing literature~\cite{lee2004trust}, such a bonus setup can help incentivize participants to reach a correct decision. 
\revise{In comparison with existing literature exploring human-AI decision making~\cite{lai2021towards}, our reward setup is above the average payment and can be considered as being sufficient to elicit ecologically valid behavior among participants (\ie aiming to arrive at accurate execution results). Moreover, similar bonus structures akin to our setup have been effective in incentivizing reliable participant behavior and improving data quality across different studies with crowdsourced participants~\cite{fan2020crowdco, salimzadeh2024dealing,liutilizing,ma2024you}.}
%\glcomment{R3 also asked the authors to clarify the participants' incentive to achieve better performance in this dataset.} \ujcomment{Yes, these incentives are meaningful for participants. Two arguments to make:  1. Show that this is a good incentive compared to other HCAI studies. 2. Show that this is meaningful to participants in crowdsourcing platforms. Both can be supported by references.}

\begin{figure*}[h]
    \centering
    \includegraphics[width=\textwidth]{figures/cognitive_load_bar_plot_new.pdf}
    \caption{\revise{Bar plot for cognitive load across all conditions. ** indicates significance ($p < 0.0125$) through post-hoc Tukey HSD test. The error bars represent the 95\% confidence interval.}}%\glcomment{I changed the colors of bars to keep consistent with the confidence plot.}} 
    \label{fig:cognitive_load}
    \Description{Bar plot for cognitive load across all conditions. User involvement in planning shows a significant impact on Mental Demand, Temporal Demand, and Frustration. User involvement in execution shows a significant impact on Performance and Effort. These results indicate that user involvement in planning and execution will require a relatively high cognitive load.}
\end{figure*}

\paratitle{Filter Criteria}. All participants were proficient English speakers between the ages of 18 - 50. 
We also constrained their prior experience (at least 40 successful submissions) and had an approval rate of above 90\% on the Prolific platform. 
We excluded participants from our analysis if they failed any attention check, or represented an outlier regarding the plan quality. 
Outliers were 4 participants who generated more than three low-quality plans among six tasks. 
The reserved 248 participants had an average age of 32.5 ($SD$= 8.1) and a balanced gender distribution
($50\%$, $49.6\%$ female, $0.4\%$ other).


% We filter out participants who provide low-quality plans (with a manual check), four participants. 


\subsection{Procedure}
At the beginning of our study, we showed informed consent for data collection and the study's purpose. 
Only participants who signed the informed consent \revise{were allowed to} continue to work on our study. 
Next, participants were asked to complete a pre-task questionnaire to measure their expertise in LLM and automatic assistants. %\glcomment{Do you think it's fine not to provide a flow chart for the whole process? As our paper can be quite long}\gdcomment{I seems pretty standard steps, so not necessary imo}

Participants were then assigned to one of the experimental conditions, which differed in the level of user involvement in the planning stage and execution stage. 
With an onboarding tutorial, we showcased the necessary interactions that participants were expected to perform in the planning and execution stages. 
We used an example task to help participants understand how to work with the plan-then-execute LLM agent. 
After the onboarding tutorial, participants worked on the selected tasks, which were shuffled at random for every participant to prevent task ordering effects. 
After the participants finished the task batch, we measured their perceived cognitive load using the NASA-TLX questionnaire~\cite{colligan2015cognitive}, their overall trust in the daily assistant using the trust in automation questionnaire~\cite{korber2019theoretical}, and we gathered their feedback on our system (related to planning, execution, and other aspects) using open-ended text.
\section{Results} 
In this section, we will present the main experimental results and exploratory analysis for our study. %\glcomment{R2 suggests a clearer results section, especially when reporting null effects. The current results section may confuse reviewers by interpreting null results.}\ujcomment{1. Look at the text once again to make sure our claims are consistent with the hypotheses tests. Consider highlighting the fact that there are null results in some cases. 2. Consider adding a table that summarizes the hypotheses findings.}\glcomment{I would recommend inviting the authors to revise the paper to rewrite the results section with interpretations that are faithful to the otherwise frequentist approach of the paper.}

\subsection{Descriptive Statistics}

In total, our analysis is based on 248 participants, who are balanced across conditions: AP-AE (63), AP-UE (64), UP-AE (61), and UP-UE (60). 
All edited plans in user-involved planning conditions are evaluated by the authors following the plan quality criteria \revise{described} in Section~\ref{sec-measure}. 
% And we keep a balanced gender distribution: 124 males (), 123 females ($49.6\%$), and 1 other. The participants age range from 

\paratitle{Distribution of Covariates}. In our study, most participants claimed to have some experience with using large language models ($M=3.6, SD=1.0$) and automatic assistants ($M=3.4, SD=1.1$). In the trust in automation questionnaire, participants indicated a medium level of \textit{Familiarity} ($M=2.9, SD=1.2$) and \textit{Propensity to Trust} ($M=3.0, SD=0.7$).

\paratitle{Performance Overview}. Overall, users show calibrated trust in the planning ($M= 0.50, SD=0.13$) and calibrated trust in the execution ($M=0.64, SD=0.19$). 
For the execution outcome, we find that although it is tricky to obtain a ground truth action sequence ($M=0.48, SD=0.17$), the action sequence has a relatively high recall of ground truth actions ($M=0.77, SD=0.11$). %\glcomment{recall only used once here, shall we remove it?}\gdcomment{I would keep it}
The successful rate for correct execution ($M=0.52, SD=0.18$) is higher than the strict evaluation of the action sequence. 
We also collected user subjective trust with four subscales of the trust in automation questionnaire: \textit{Reliability/Competence} ($M=3.49, SD=0.77$),  \textit{Understanding/Predictability} ($M=3.30, SD=0.56$),  \textit{Intention of Developers} ($M=3.61, SD=0.81$),  \textit{Trust in Automation} ($M=3.52, SD=1.01$). 
With a two-way ANOVA analysis considering user involvement in planning and execution, we do not find any significant impact of user involvement on subjective user trust in AI systems across conditions.

% \begin{figure*}[h]
%     \centering
%     \includegraphics[width=\textwidth]{figures/cognitive_load_bar_plot_new.pdf}
%     \caption{\revise{Bar plot for cognitive load across all conditions. ** indicates significance ($p < 0.0125$) through post-hoc Tukey HSD test. The error bars represent the 95\% confidence interval.}}%\glcomment{I changed the colors of bars to keep consistent with the confidence plot.}} 
%     \label{fig:cognitive_load}
%     \Description{Bar plot for cognitive load across all conditions. User involvement in planning shows a significant impact on Mental Demand, Temporal Demand, and Frustration. User involvement in execution shows a significant impact on Performance and Effort. These results indicate that user involvement in planning and execution will require a relatively high cognitive load.}
% \end{figure*}

\paratitle{Cognitive Load}. The cognitive load of participants across the four experimental conditions is shown in Figure~\ref{fig:cognitive_load}. 
% Based on the bar plot, we can infer that involvement in both planning and execution will pose a higher cognitive load on participants. 
% Overall, user involvement in planning seems to have more impact on cognitive load (reflected by dimensions of mental demand, temporal demand, and frustration). 
Based on two-way ANOVA, we analyzed the impact of user involvement in planning and execution affect user cognitive load. 
User involvement in planning shows a significant impact on \textit{Mental Demand}, \textit{Temporal Demand}, and \textit{Frustration}. User involvement in execution shows a significant impact on \textit{Performance} and \textit{Effort}. 
With post-hoc Tukey HSD test, we confirmed such impact --- involvement in both planning and execution  posed a higher cognitive load on participants.

% \subsubsection{Qualitative Analysis of User Involvement} We also look into, how user interact with the AI system in the planning and execution stages.

\begin{table*}[h]
	\centering
	\caption{Task-specific evaluation results for user-involvement in planning on calibrated trust in planning (CT$_p$) and plan quality. We also report the mean value for each measure on each condition.}
	\label{tab:h1-res}
    \begin{small}
	\begin{tabular}{c | c c c  c | c | c  c c c | c}
	    \hline
        &     \multicolumn{5}{c|}{CT$_p$}&  \multicolumn{5}{c}{Plan Quality}\\
        \cline{2-11}
        Tasks& AP-AE& AP-UE& UP-AE& UP-UE& Post-hoc results& AP-AE& AP-UE& UP-AE& UP-UE& Post-hoc results\\
        \hline 
        \hline
        Avg & 0.51& 0.50& 0.50& 0.50& - & 3.8& 3.8& 3.6& 3.7& AP > UP\\
        \hline
        task-1& 0.11 & 0.20 & 0.13 & 0.27& -& 2.0 & 2.0 & 2.3 & 2.4& AP < UP\\
        task-2 &0.21 & 0.11 & 0.20 & 0.17& -& 3.0 & 3.0 & 2.9 & 2.9& -\\
        task-3 & 0.10 & 0.03 & 0.10 & 0.07& -& 3.0 & 3.0 & 2.7 & 2.9& AP > UP\\
        task-4 & 0.94 & 0.97 & 0.80 & 0.90& AP > UP& 5.0 & 5.0 & 4.3 & 4.8& AP > UP\\
        task-5 & 0.87 & 0.84 & 0.90 & 0.82& -& 5.0 & 5.0 & 4.6 & 4.8& AP > UP\\
        task-6 & 0.81 & 0.81 & 0.85 & 0.75& -& 5.0 & 5.0 & 4.7 & 4.6& AP > UP\\
    \hline
	\end{tabular}
 \end{small}
\end{table*}

\begin{table*}[h]
	\centering
	\caption{Task-specific evaluation results for user-involvement in planning on task performance. ACC$_s$ denotes the strict accuracy of an action sequence, and ACC$_e$ denotes the correctness of execution results. Bold fonts are used to highlight the best performance across conditions.}
	\label{tab:h2-res}
    \begin{small}
	\begin{tabular}{c | c c c  c | c | c  c c c | c}
	    \hline
        \multirow{2}{*}{Tasks}&     \multicolumn{5}{c|}{ACC$_s$}&  \multicolumn{5}{c}{ACC$_e$}\\
        \cline{2-11}
        & AP-AE& AP-UE& UP-AE& UP-UE& Post-hoc results& AP-AE& AP-UE& UP-AE& UP-UE& Post-hoc results\\
        \hline 
        \hline
        Avg& \textbf{0.53} & 0.46 & 0.46 & 0.48 & - & 0.54 & 0.53 & 0.47 & \textbf{0.56} & -\\
        \hline
        task-1& 0.00 & 0.00 & 0.10 & \textbf{0.12} & AP < UP & 0.00 & 0.00 & 0.10 & \textbf{0.13} & AP < UP\\
        task-2 &\textbf{0.78} & 0.64 & 0.61 & 0.57 & - & \textbf{0.78} & 0.72 & 0.66 & 0.75 & - \\
        task-3 & \textbf{0.44} & 0.12 & 0.36 & 0.28 & - & 0.44 & 0.42 & 0.36 & \textbf{0.52} & - \\
        task-4 & \textbf{0.95} & 0.89 & 0.75 & 0.82 & AP > UP & \textbf{0.95} & 0.89 & 0.75 & 0.82 & AP > UP \\
        task-5 & \textbf{0.98} & 0.91 & 0.90 & 0.90 & - & \textbf{0.98} & 0.91 & 0.92 & 0.90 & - \\
        task-6 & 0.05 & \textbf{0.22} & 0.02 & 0.18 & - & 0.06 & \textbf{0.23} & 0.03 & 0.22 & - \\
    \hline
	\end{tabular}
 \end{small}
\end{table*}

\paratitle{User Involvement}. Among 121 participants in conditions with user-involved planning, 104 participants edited at least one task plan. Meanwhile, 90 participants used the provided buttons (\ie add/delete/split step) in our study. In total, \textit{delete step} is used 394 times, \textit{add step} is used 183 times, \textit{split step} is used 126 times. Among 124 participants in conditions with user-involved execution, 114 participants interacted with the conversation interface to change action prediction (\ie have at least one task where they choose to give feedback or override predicted action). 
Meanwhile, 105 participants specified at least one action in the task batch. 
In total, \textit{Specify Action} is used 445 times, feedback to the LLM agent is used 91 times before action execution, and feedback to the LLM agent is used 163 times after execution.


\subsection{Hypothesis Verification}
As the tasks selected in our study are of different initial plan quality and risk levels, we conducted a task-specific analysis in each hypothesis verification.

\subsubsection{The Impact of User Involvement in Planning on Calibrated Trust} 
% To verify the impact of user involvement in the planning stage on calibrated trust in the planning outcome (\textbf{H1}), we need to also consider the potential impact of user involvement in the execution stage. 
To verify \textbf{H1}, we adopted the one-way ANOVA test and post-hoc Tukey HSD test on the calibrated user trust in planning (\ie CT$_p$). 
The results are shown in Table~\ref{tab:h1-res}. 
Only in task-4, we found user involvement in planning will have a negative impact on calibrated trust in planning. 
To avoid a potential impact of user involvement in the execution stage, we conducted a two-way ANOVA test to confirm the findings. 
% Overall, the test results are highly consistent: there is only a significant difference on task-4.
We only find a significant difference in task-4.
Post-hoc Tukey HSD results show that participants in conditions with automatic planning (AP) showed significantly higher calibrated trust in planning outcomes than those in conditions with user-involved planning (UP). 
Thus, our experimental results do not support \textbf{H1}.
% \glcomment{show task-specific analysis / mean values. Then mention the aggregation of results for high-risk tasks and low-risk tasks}

% \begin{table*}[h]
% 	\centering
% 	\caption{Task-specific evaluation results for user-involvement in planning on calibrated trust in planning (CT$_p$) and plan quality. We also report the mean value for each measure on each condition.}
% 	\label{tab:h1-res}
%     \begin{small}
% 	\begin{tabular}{c | c c c  c | c | c  c c c | c}
% 	    \hline
%         &     \multicolumn{5}{c|}{CT$_p$}&  \multicolumn{5}{c}{Plan Quality}\\
%         \cline{2-11}
%         Tasks& AP-AE& AP-UE& UP-AE& UP-UE& Post-hoc results& AP-AE& AP-UE& UP-AE& UP-UE& Post-hoc results\\
%         \hline 
%         \hline
%         Avg & 0.51& 0.50& 0.50& 0.50& - & 3.8& 3.8& 3.6& 3.7& AP > UP\\
%         \hline
%         task-1& 0.11 & 0.20 & 0.13 & 0.27& -& 2.0 & 2.0 & 2.3 & 2.4& AP < UP\\
%         task-2 &0.21 & 0.11 & 0.20 & 0.17& -& 3.0 & 3.0 & 2.9 & 2.9& -\\
%         task-3 & 0.10 & 0.03 & 0.10 & 0.07& -& 3.0 & 3.0 & 2.7 & 2.9& AP > UP\\
%         task-4 & 0.94 & 0.97 & 0.80 & 0.90& AP > UP& 5.0 & 5.0 & 4.3 & 4.8& AP > UP\\
%         task-5 & 0.87 & 0.84 & 0.90 & 0.82& -& 5.0 & 5.0 & 4.6 & 4.8& AP > UP\\
%         task-6 & 0.81 & 0.81 & 0.85 & 0.75& -& 5.0 & 5.0 & 4.7 & 4.6& AP > UP\\
%     \hline
% 	\end{tabular}
%  \end{small}
% \end{table*}

We noticed that the calibrated trust in planning is quite low in the high-risk tasks where all initial plans are imperfect. 
This indicates that many users across all conditions consider the generated plan trustworthy. 
On tasks with low risk, where the initial plan is of high quality, users achieved much higher calibrated trust in the planning outcome. 
We also find that conditions with user-involved execution (UE) show slightly higher CT$_p$ in task-1 and task-4 than conditions with automatic execution (AE). With the same statistical test as \textbf{H1} analysis, such differences are not significant.

% \begin{table}[h]
% 	\centering
% 	\caption{Task-specific evaluation results for user-involvement in planning on calibrated trust in planning (CT$_p$) and plan quality. We also report the mean value for each measure on each condition. \glcomment{We can consider use `Calibrated Trust-planning' directly}}
% 	\label{tab:h1-res}
%     \begin{small}
% 	\begin{tabular}{c | c c c  c | c | c  c c c | c}
% 	    \hline
%         &     \multicolumn{5}{c|}{CT$_p$}&  \multicolumn{5}{c}{Plan Quality}\\
%         \cline{2-11}
%         Tasks& AP-AE& AP-UE& UP-AE& UP-UE& Post-hoc results& AP-AE& AP-UE& UP-AE& UP-UE& Post-hoc results\\
%         \hline 
%         \hline
%         Avg & 0.51& 0.50& 0.50& 0.50& - & 3.8& 3.8& 3.6& 3.7& AP > UP\\
%         \hline
%         task-1& 0.11 & 0.20 & 0.13 & 0.27& -& 2.0 & 2.0 & 2.3 & 2.4& AP < UP\\
%         task-2 &0.21 & 0.11 & 0.20 & 0.17& -& 3.0 & 3.0 & 2.9 & 2.9& -\\
%         task-3 & 0.10 & 0.03 & 0.10 & 0.07& -& 3.0 & 3.0 & 2.7 & 2.9& AP > UP\\
%         task-4 & 0.94 & 0.97 & 0.80 & 0.90& AP > UP& 5.0 & 5.0 & 4.3 & 4.8& AP > UP\\
%         task-5 & 0.87 & 0.84 & 0.90 & 0.82& -& 5.0 & 5.0 & 4.6 & 4.8& AP > UP\\
%         task-6 & 0.81 & 0.81 & 0.85 & 0.75& -& 5.0 & 5.0 & 4.7 & 4.6& AP > UP\\
%     \hline
% 	\end{tabular}
%  \end{small}
% \end{table}



% \begin{table}[h]
% 	\centering
% 	\caption{Task-specific Evaluation Results for calibrated trust. CT$_p$ denotes the calibrated trust in the planning outcome, and CT$_e$ denotes the calibrated trust in the execution outcome.\glcomment{Shall we split calibrated trust into two measures?}}
% 	\label{tab:calibrated_trust}
%     \begin{small}
% 	\begin{tabular}{c | c c | c c| c c| c c| c c| c c | c c}
% 	    \hline
%         Tasks&\multicolumn{2}{c|}{Avg}&\multicolumn{2}{c|}{Task-1}&\multicolumn{2}{c|}{Task-2}&\multicolumn{2}{c|}{Task-3}&\multicolumn{2}{c|}{Task-4}&\multicolumn{2}{c|}{Task-5}&\multicolumn{2}{c}{Task-6}\\
%         \hline
%         % Characteristics& Risk& Plan& Risk& Plan& Risk& Plan& Risk& Plan& Risk& Plan& Risk& Plan& Risk& Plan\\\
%         % Notes& -& -& High& 2& High& 3& High& 3& Low& 5& Low& 5& Low& 5\\
%         % \hline
%         Conditions& CT$_p$& CT$_e$& CT$_p$& CT$_e$& CT$_p$& CT$_e$& CT$_p$& CT$_e$& CT$_p$& CT$_e$& CT$_p$& CT$_e$& CT$_p$& CT$_e$\\
%         \hline 
%         \hline
%         AP-AE& 0.51& 0.66& 0.11 & 0.48 & 0.21 & 0.78 & 0.10 & 0.51 & 0.94 & 0.94 & 0.87 & 0.89 & 0.81 & 0.37\\
%         AP-UE & 0.50& 0.65& 0.20 & 0.44 & 0.11 & 0.83 & 0.03 & 0.41 & 0.97 & 0.92 & 0.84 & 0.92 & 0.81 & 0.38\\
%         UP-AE & 0.50& 0.62&  0.13 & 0.49 & 0.20 & 0.67 & 0.10 & 0.56 & 0.80 & 0.77 & 0.90 & 0.90 & 0.85 & 0.30\\
%         UP-UE & 0.49& 0.63& 0.27 & 0.48 & 0.17 & 0.75 & 0.07 & 0.45 & 0.90 & 0.82 & 0.82 & 0.90 & 0.75 & 0.40\\
%     \hline
%     % Plan& Imperfect& 
%     % \hline 
% 	\end{tabular}
%  \end{small}
% \end{table}

\subsubsection{The Impact of User Involvement in Planning on Task Performance} \label{sec-exp-h2}
To verify \textbf{H2}, we considered plan quality, the accuracy of action sequences (ACC$_s$), and the execution accuracy of the plan (ACC$_e$) for analysis.
For plan quality (cf. Table~\ref{tab:h1-res}), we conducted one-way ANOVA on plan quality considering the user involvement in the planning stage. 
We found that overall user involvement in the planning stage caused a decrease plan quality, especially on tasks with a perfect plan (\ie task 4, 5, 6, where plan quality = 5) and task-3. 
However, in task-1, where the original plan contains a grammar error, we find that user involvement in planning can improve the plan quality. 
As the action sequence accuracy (ACC$_s$) and execution accuracy (ACC$_e$) are not normally distributed, we conducted the Kruskal-Wallis H-test by considering the user involvement in the planning as the independent variable. \revise{The results are shown in Table~\ref{tab:h2-res}.}
With further post-hoc Mann-Whitney tests, we found that while participants achieved a relatively higher accuracy of action sequences in condition AP-AE, the condition UP-UE achieved the best execution accuracy. 
In most tasks, condition UP-UE achieved better or compatible performance as other conditions. The only exception is task-4, where user involvement in the planning caused a significantly worse performance (both ACC$_s$ and ACC$_e$). 
% While user involvement can not consistently bring benefits to the task performance, we did find it can help in specific scenarios (\eg task-1, where the plan is with a grammar error). 
\revise{As user involvement does not consistently lead to improved performance, these results are not enough to support \textbf{H2}.}
% Thus, \textbf{H2} is partially supported.\glcomment{I think we can still claim partial support for H2 instead of no support. right? @Ujwal} \ujcomment{To align with the reviewer comments (esp. regarding null results), maybe it is better to say no support for H2 but leave the other points and discussion as they are.}

% \revise{We found that in task-1 and task-6 most participants in the AP-AE condition achieved a very low success rate. This is mainly due to the imperfect plans and imperfect execution generated by LLMs. 
% In task-1, the plan generated by LLMs includes one step which contains two actions to execute. 
% Due to the inability to edit the plan, the LLM agent execution missed one transaction in conditions with automatic planning. 
% In task-6, the plan generated by LLMs is correct. 
% However, in the automatic execution of step 2 of the plan (\ie selecting an itinerary suggested), the LLM agent has a high probability of choosing an itinerary that does not match the task description. 
% If the participants do not carefully check the task description, and correct this agent behavior, the execution results would be wrong.
% This also helps explain why user involvement substantially improves the task outcome accuracy in task-6. More details about tasks can be found in supplementary materials.}
%\glcomment{I was struck by the difference in results for the six tasks shown in Table 4. For task-1, which was simple, but had an imperfect plan, basically all trials were failures. For task-6, which was complex but had a correct plan, failure was also extremely high. I was left wondering how these tasks differed from each other. I don't quite trust the simple argument that "user involvement can help" when the failure rate is around 90\%. I was hoping for something that can explain what is happening. Is the problem in the interests, capabilities, capacities, or incentives of the participants? 
%}\glcomment{After addressing this comment, I feel it would be better if we could provide the generated plans and further explain the flaws of automation of LLM agent in each task in supplementary materials.}
% We found that, for the ACC$_s$ over the task batch, AP-AE > UP-AE, AP-UE. \glcomment{Shall we only conduct one-way ANOVA for plan quality?} \glcomment{only compare AP with UP}

% \begin{table}[h]
% 	\centering
% 	\caption{Task-specific Evaluation Results for calibrated trust. ACC$_s$ denotes the strict accuracy of an action sequence, and ACC$_e$ denotes the correctness of execution results.}
% 	\label{tab:h2-res}
%     \begin{small}
% 	\begin{tabular}{c | c | c | c}
% 	    \hline
%         Tasks& Plan Quality& ACC$_s$& ACC$_e$\\
%         \hline 
%         \hline
%         Avg& -& - & -\\
%         task-1 & AP < UP & AP < UP& AP < UP\\
%         task-2 & -& -& -\\
%         task-3 & AP > UP& AE > UE& -\\
%         task-4 & AP > UP& AP > UP& AP > UP\\
%         task-5 & AP > UP& -& -\\
%         task-6 & AP > UP& AE < UE & AE < UE\\
%     \hline
% 	\end{tabular}
%  \end{small}
% \end{table}

% \begin{table*}[h]
% 	\centering
% 	\caption{Task-specific evaluation results for user-involvement in planning on task performance. ACC$_s$ denotes the strict accuracy of an action sequence, and ACC$_e$ denotes the correctness of execution results. Bold fonts are used to highlight the best performance across conditions.}
% 	\label{tab:h2-res}
%     \begin{small}
% 	\begin{tabular}{c | c c c  c | c | c  c c c | c}
% 	    \hline
%         \multirow{2}{*}{Tasks}&     \multicolumn{5}{c|}{ACC$_s$}&  \multicolumn{5}{c}{ACC$_e$}\\
%         \cline{2-11}
%         & AP-AE& AP-UE& UP-AE& UP-UE& Post-hoc results& AP-AE& AP-UE& UP-AE& UP-UE& Post-hoc results\\
%         \hline 
%         \hline
%         Avg& \textbf{0.53} & 0.46 & 0.46 & 0.48 & - & 0.54 & 0.53 & 0.47 & \textbf{0.56} & -\\
%         \hline
%         task-1& 0.00 & 0.00 & 0.10 & \textbf{0.12} & AP < UP & 0.00 & 0.00 & 0.10 & \textbf{0.13} & AP < UP\\
%         task-2 &\textbf{0.78} & 0.64 & 0.61 & 0.57 & - & \textbf{0.78} & 0.72 & 0.66 & 0.75 & - \\
%         task-3 & \textbf{0.44} & 0.12 & 0.36 & 0.28 & - & 0.44 & 0.42 & 0.36 & \textbf{0.52} & - \\
%         task-4 & \textbf{0.95} & 0.89 & 0.75 & 0.82 & AP > UP & \textbf{0.95} & 0.89 & 0.75 & 0.82 & AP > UP \\
%         task-5 & \textbf{0.98} & 0.91 & 0.90 & 0.90 & - & \textbf{0.98} & 0.91 & 0.92 & 0.90 & - \\
%         task-6 & 0.05 & \textbf{0.22} & 0.02 & 0.18 & - & 0.06 & \textbf{0.23} & 0.03 & 0.22 & - \\
%     \hline
% 	\end{tabular}
%  \end{small}
% \end{table*}

\revise{We found that in task-1 and task-6 most participants in the AP-AE condition achieved a very low success rate. This is mainly due to the imperfect plans and imperfect execution generated by LLMs. 
In task-1, the plan generated by LLMs includes one step which contains two actions to execute. 
Due to the inability to edit the plan, the LLM agent execution missed one transaction in conditions with automatic planning. 
In task-6, the plan generated by LLMs is correct. 
However, in the automatic execution of step 2 of the plan (\ie selecting an itinerary suggested), the LLM agent has a high probability of choosing an itinerary that does not match the task description. 
If the participants do not carefully check the task description, and correct this agent behavior, the execution results would be wrong.
This also helps explain why user involvement substantially improves the task outcome accuracy in task-6. More details about tasks can be found in the appendix.}

\begin{table*}[h]
	\centering
	\caption{Task-specific evaluation results for user-involvement in execution on task performance. Bold fonts are used to highlight the best performance across conditions.}
	\label{tab:h4-res}
    \begin{small}
	\begin{tabular}{c | c c c  c | c | c  c c c | c}
	    \hline
        \multirow{2}{*}{Tasks}&     \multicolumn{5}{c|}{ACC$_s$}&  \multicolumn{5}{c}{ACC$_e$}\\
        \cline{2-11}
        & AP-AE& AP-UE& UP-AE& UP-UE& Post-hoc results& AP-AE& AP-UE& UP-AE& UP-UE& Post-hoc results\\
        \hline 
        \hline
        Avg& \textbf{0.53} & 0.46 & 0.50 & 0.51 & - & 0.54 & 0.53 & 0.50 & \textbf{0.58} & -\\
        \hline
        task-1& 0.00 & 0.00 & 0.10 & \textbf{0.12} & - & 0.00 & 0.00 & 0.10 & \textbf{0.14} & -\\
        task-2 &\textbf{0.78} & 0.64 & 0.67 & 0.62 & - & \textbf{0.78} & 0.72 & 0.69 & \textbf{0.78} & - \\
        task-3 & \textbf{0.44} & 0.12 & 0.42 & 0.29 & AE > UE & 0.44 & 0.42 & 0.42 & \textbf{0.53} & - \\
        task-4 & \textbf{0.95} & 0.89 & 0.94 & 0.88 & - & \textbf{0.95} & 0.89 & 0.94 & 0.88 & - \\
        task-5 & 0.98 & 0.91 & \textbf{1.00} & 0.98 & - & 0.98 & 0.91 & \textbf{1.00} & 0.98 & -\\
        task-6 & 0.05 & \textbf{0.22} & 0.02 & 0.19 & AE < UE & 0.06 & \textbf{0.23} & 0.04 & \textbf{0.23} & AE < UE\\
    \hline
	\end{tabular}
 \end{small}
\end{table*}

\subsubsection{The Impact of User Involvement in Execution on Calibrated Trust in Execution Outcome}
As we observe in Table~\ref{tab:h1-res}, user involvement in planning can have some negative impact on the plan quality, which further impacts the execution stage. 
To control such impact, we filtered out the tasks where plan quality decreased after user-involved planning in the analysis of user involvement in the execution stage. 
To verify \textbf{H3}, we conducted one-way ANOVA on calibrated trust in execution outcome (CT$_e$). The results are shown in Table~\ref{tab:h3-res}. 
We found that user involvement in execution causes no significant difference across conditions. 
Thus, \textbf{H3} is not supported by our experimental results.
% However, such results are heavily impacted by the plan quality, which is impacted by the user involvement in the planning stages. 
% To control such impact, we filtered out tasks where users generate a low-quality plan (plan quality = 1, 2). 
% \glcomment{fair comparison, AP-AE vs AP-UE; unfair comparison, UP-AE vs UP-UE?} \glcomment{how to control plan quality in UP conditions? >= initial plan?}


\begin{table}[h]
	\centering
	\caption{Task-specific evaluation results for user-involvement in execution on calibrated trust in execution (CT$_e$). We also report the mean value for each measure on each condition.}%\glcomment{Shall we adopt calibrated trust based on two dimensions? correct, wrong?}}
	\label{tab:h3-res}
    \begin{small}
	\begin{tabular}{c | c c c  c | c }
	    \hline
        \multirow{2}{*}{Tasks}&     \multicolumn{5}{c}{CT$_e$}\\
        \cline{2-6}
        & AP-AE& AP-UE& UP-AE& UP-UE& Post-hoc results\\
        \hline 
        \hline
        Avg & \textbf{0.66} & 0.65 & 0.64 & 0.65 & - \\
        \hline
        task-1& 0.48 & 0.44 & \textbf{0.51} & 0.49 & -\\
        task-2 &0.78 & \textbf{0.83} & 0.71 & 0.80 & -\\
        task-3 & 0.51 & 0.41 & \textbf{0.60} & 0.47 & -\\
        task-4 & \textbf{0.94} & 0.92 & 0.88 & 0.86 & -\\
        task-5 & 0.89 & 0.92 & \textbf{0.96} & 0.94 & -\\
        task-6 & 0.37 & 0.38 & 0.28 & \textbf{0.42} & -\\
    \hline
	\end{tabular}
 \end{small}
\end{table}


%\glcomment{All of section 5.2.4 is riddled with interpretations that confuse the reader, such as “One point to note is that on the three tasks with high risks (tasks 1, 2, 3), condition UP-UE achieved the best execution accuracy (non-significant difference between AE and UE)”. My interpretation of 5.2.4 at first glance was that there is no support for H4, counter to what the paper reports through selective interpretation. Reporting null effects is not a problem: it is also an interesting finding if you observe no differences between when there is and isn’t user involvement. The somewhat convoluted interpretation makes it harder to understand the actual results and undermines credence in the findings.}

\subsubsection{The Impact of User Involvement on Overall Task Performance} Similar to the verification of \textbf{H3}, we excluded the tasks where plan quality decreased after user-involved planning in this analysis. 
As the plan is generated before user involvement in the execution, we only considered ACC$_s$ and ACC$_e$ in the analysis of user involvement in the execution stage. 
To verify \textbf{H4}, we conducted Kruskal-Wallis H-test by considering the user involvement in the execution as the independent variable. The results are shown in Table~\ref{tab:h4-res}. 
With post-hoc Mann-Whitney tests, we found that user involvement in the execution stage showed significantly higher ACC$_s$ and ACC$_e$ in task-6 (where the LLM assistant mainly failed to choose the most suitable itinerary plan). 
% \revise{While ACC$_s$ assesses how often the execution action sequence strictly the same as ground truth (annotated by the authors).}
% While participants achieved a relatively higher accuracy of action sequences in condition AP-AE, condition UP-UE achieved the best execution accuracy. 
\revise{We found that participants in the AP-AE condition achieved the best accuracy of action sequences (\ie ACC$_s$), and participants in condition UP-UE achieved the best execution accuracy (\ie ACC$_e$). 
In other words, the executed action sequence in condition AP-AE is more aligned with the ground truth action sequence annotated by the authors. 
However, with user involvement in the execution stage, participants in condition UP-UE have a better opportunity to obtain correct task outcomes by correcting potentially flawed actions. 
Such a difference is due to our measure of ACC$_e$, which tolerates the non-risky actions (\eg search flight) and failure of action predictions. 
In contrast, our measure of ACC$_s$ considers this as a wrong action sequence. 
% being tolerant in the evaluation of task outcome with ACC$_e$. %\ujcomment{This prev. sentence is not clear.}
} %\glcomment{To check in our online meeting}
\revise{Thus,} in task-3, \revise{even if} we find automatic execution achieved \revise{significantly} better ACC$_s$ than user-involved execution, participants in condition AP-UE and UP-UE %\revise{still} 
obtained comparable or higher execution accuracy \revise{(\ie ACC$_e$) than conditions with automatic execution}. 
% One point to note is that on the three tasks with high risks (tasks 1, 2, 3), condition UP-UE achieved the best execution accuracy (non-significant difference between AE and UE). 
% Meanwhile, we also noticed that after filtering out low-quality plans, participants in condition UP-AE achieved $100\%$ accuracy in both action sequence generation and execution. 
% Thus, on the whole, we find support for \textbf{H4}.
\revise{While user involvement shows some positive impact on the execution accuracy, such impact is not significant and consistent across all tasks. 
Only in task-6, where users can correct the errors made by the LLM agent (\ie the wrong itinerary selection mentioned in Section~\ref{sec-exp-h2}), user involvement in the execution shows a significant contribution to the task performance. 
% Thus, these results only provide partial support for \textbf{H4}.
Thus, these results are not enough to strictly support \textbf{H4}.
}
%\glcomment{I still take this as a supporting signal to H4 instead of a null effect. Meanwhile, I removed many sentences of claiming the better performance achieved by UE (such as UP-AE achieved 100 percent accuracy and task 1, 2, 3 are of high risks). I guess these sents may bring the confusion.How do you think about it?} \ujcomment{Same as previous comment for H2. Let's discuss how to best win over this reviewer.}
% \glcomment{Kruskal. As the execution results are affected by the plan quality. Shall we control the plan quality, then compare them? For example, compare AP-AE with AP-UE; compare UP-AE with UP-UE}

% \paratitle{Task-specific Analysis}. 

% \begin{table}[tbp]
% 	\centering
% 	\caption{Task-specific evaluation results for overall task performance. ACC$_s$ denotes the strict accuracy of an action sequence, and ACC$_e$ denotes the correctness of execution results.}
% 	\label{tab:task_execution_res}
%     \begin{small}
%     \scalebox{0.9}{
% 	\begin{tabular}{c| c c | c c| c c| c c| c c| c c | c c}
% 	    \hline
%         Tasks&\multicolumn{2}{c|}{Avg}&\multicolumn{2}{c|}{Task-1}&\multicolumn{2}{c|}{Task-2}&\multicolumn{2}{c|}{Task-3}&\multicolumn{2}{c|}{Task-4}&\multicolumn{2}{c|}{Task-5}&\multicolumn{2}{c}{Task-6}\\
%         \hline
%         Conditions& ACC$_s$& ACC$_e$& ACC$_s$& ACC$_e$& ACC$_s$& ACC$_e$& ACC$_s$& ACC$_e$& ACC$_s$& ACC$_e$& ACC$_s$& ACC$_e$& ACC$_s$& ACC$_e$\\
%         \hline 
%         \hline
%         AP-AE & 0.53& 0.54& 0.00 & 0.00 & 0.78 & 0.78 & 0.44 & 0.44 & 0.95 & 0.95 & 0.98 & 0.98 & 0.05 & 0.06\\
%         AP-UE & 0.46& 0.53& 0.00 & 0.00 & 0.64 & 0.72 & 0.12 & 0.42 & 0.89 & 0.89 & 0.91 & 0.91 & 0.22 & 0.23\\
%         UP-AE & 0.46& 0.47& 0.10 & 0.10 & 0.61 & 0.66 & 0.36 & 0.36 & 0.75 & 0.75 & 0.90 & 0.92 & 0.02 & 0.03\\
%         UP-UE & 0.48& 0.56& 0.12 & 0.13 & 0.57 & 0.75 & 0.28 & 0.52 & 0.82 & 0.82 & 0.90 & 0.90 & 0.18 & 0.22\\
%     \hline
%     % Plan& Imperfect& 
%     % \hline 
% 	\end{tabular}
%  }
%  \end{small}
% \end{table}

% \begin{table*}[h]
% 	\centering
% 	\caption{Task-specific evaluation results for user-involvement in execution on task performance. Bold fonts are used to highlight the best performance across conditions.}
% 	\label{tab:h4-res}
%     \begin{small}
% 	\begin{tabular}{c | c c c  c | c | c  c c c | c}
% 	    \hline
%         \multirow{2}{*}{Tasks}&     \multicolumn{5}{c|}{ACC$_s$}&  \multicolumn{5}{c}{ACC$_e$}\\
%         \cline{2-11}
%         & AP-AE& AP-UE& UP-AE& UP-UE& Post-hoc results& AP-AE& AP-UE& UP-AE& UP-UE& Post-hoc results\\
%         \hline 
%         \hline
%         Avg& \textbf{0.53} & 0.46 & 0.50 & 0.51 & - & 0.54 & 0.53 & 0.50 & \textbf{0.58} & -\\
%         \hline
%         task-1& 0.00 & 0.00 & 0.10 & \textbf{0.12} & - & 0.00 & 0.00 & 0.10 & \textbf{0.14} & -\\
%         task-2 &\textbf{0.78} & 0.64 & 0.67 & 0.62 & - & \textbf{0.78} & 0.72 & 0.69 & \textbf{0.78} & - \\
%         task-3 & \textbf{0.44} & 0.12 & 0.42 & 0.29 & AE > UE & 0.44 & 0.42 & 0.42 & \textbf{0.53} & - \\
%         task-4 & \textbf{0.95} & 0.89 & 0.94 & 0.88 & - & \textbf{0.95} & 0.89 & 0.94 & 0.88 & - \\
%         task-5 & 0.98 & 0.91 & \textbf{1.00} & 0.98 & - & 0.98 & 0.91 & \textbf{1.00} & 0.98 & -\\
%         task-6 & 0.05 & \textbf{0.22} & 0.02 & 0.19 & AE < UE & 0.06 & \textbf{0.23} & 0.04 & \textbf{0.23} & AE < UE\\
%     \hline
% 	\end{tabular}
%  \end{small}
% \end{table*}


\subsection{Exploratory Analysis}

\subsubsection{The Impact of Covariates}
For further insights into all user factors on user trust and team performance, we calculated Spearman rank-order correlation coefficients for user trust, calibrated trust, risk perception, and task performance. 
As can be seen in Table~\ref{tab:correlation}, we found these covariates mainly show correlations with subjective user trust, calibrated trust in execution, and risk perception. 
% However, these user factors do not significantly correlate with task performance. 
First, all covariates (\ie user factors) positively correlated with user trust (four subscales in the trust in automation questionnaire~\cite{korber2019theoretical}) and negatively correlated with perceived risk (average over six tasks). 
It indicates that users with more expertise or familiarity with such systems tend to trust the daily assistant and show less perceived risk when using it. 
Meanwhile, users with a general propensity to trust also tend to trust the AI system. 
Besides user trust, \textit{Assistant Expertise} and \textit{Propensity to Trust} show a significant negative correlation with calibrated trust in the execution outcome. 
Apart from the above correlation, these user factors do not significantly correlate with task performance measures or calibrated trust in the planning outcome.

\begin{table*}[ht]
% the environment \color{blue} change all cell color
	\centering
	\caption{Spearman rank-order correlation coefficient for covariates level on dependent variables. All measures are calculated based on average over task batch. ``${\dagger}$'' and ``${\dagger\dagger}$'' indicate the effect of the variable is significant at the level of 0.05 and 0.0125, respectively.}
	\label{tab:correlation}%
	\begin{small}
        \scalebox{0.95}{
	\begin{tabular}{l | l| c c | c c | c c | c c}
	    \hline
	    \multicolumn{2}{c|}{\textbf{Covariates}}& \multicolumn{2}{c|}{llm expertise} &  \multicolumn{2}{c|}{assistant expertise}& \multicolumn{2}{c|}{Familiarity}& \multicolumn{2}{c}{Propensity to Trust}\\
     \hline
     \textbf{Category} &\textbf{Variables}&  $r$& $p$ &  $r$& $p$ &  $r$& $p$ &  $r$& $p$ \\
	    \hline \hline
    % Mental demand & -0.046 & .472& -0.003 & .968& -0.163 &\textbf{.010}$^{\dagger\dagger}$& -0.188 &\textbf{.003}$^{\dagger\dagger}$\\
    % Physical demand & 0.314 &\textbf{.000}$^{\dagger\dagger}$& 0.267 &\textbf{.000}$^{\dagger\dagger}$& 0.166 &\textbf{.009}$^{\dagger\dagger}$& 0.219 &\textbf{.001}$^{\dagger\dagger}$\\
    % Temporal demand & 0.031 & .623& 0.034 & .593& -0.077 & .225& -0.039 & .542\\
    % Performance & -0.133 & .036$^{\dagger}$& -0.154 & .016$^{\dagger}$& -0.154 & .015$^{\dagger}$& -0.080 & .209\\
    % Effort & 0.037 & .562& 0.109 & .087& -0.046 & .470& -0.013 & .839\\
    % Frustration & -0.249 &\textbf{.000}$^{\dagger\dagger}$& -0.131 & .039$^{\dagger}$& -0.296 &\textbf{.000}$^{\dagger\dagger}$& -0.348 &\textbf{.000}$^{\dagger\dagger}$\\
    % \hline
    \multirow{4}{*}{\textbf{User Trust}}& Reliability/Competence & 0.334 &\textbf{.000}$^{\dagger\dagger}$& 0.245 &\textbf{.000}$^{\dagger\dagger}$& 0.321 &\textbf{.000}$^{\dagger\dagger}$& 0.679 &\textbf{.000}$^{\dagger\dagger}$\\
    &Understanding/Predictability & 0.307 &\textbf{.000}$^{\dagger\dagger}$& 0.164 &\textbf{.010}$^{\dagger\dagger}$& 0.208 &\textbf{.001}$^{\dagger\dagger}$& 0.380 &\textbf{.000}$^{\dagger\dagger}$\\
    &Intention of Developers & 0.406 &\textbf{.000}$^{\dagger\dagger}$& 0.324 &\textbf{.000}$^{\dagger\dagger}$& 0.362 &\textbf{.000}$^{\dagger\dagger}$& 0.517 &\textbf{.000}$^{\dagger\dagger}$\\
    &Trust in Automation & 0.380 &\textbf{.000}$^{\dagger\dagger}$& 0.278 &\textbf{.000}$^{\dagger\dagger}$& 0.356 &\textbf{.000}$^{\dagger\dagger}$& 0.698 &\textbf{.000}$^{\dagger\dagger}$\\
    % &Trust-p & 0.099 & .119& 0.163 &\textbf{.010}$^{\dagger\dagger}$& 0.233 &\textbf{.000}$^{\dagger\dagger}$& 0.137 & .030$^{\dagger}$\\
    % &Trust-e & 0.096 & .130& 0.168 &\textbf{.008}$^{\dagger\dagger}$& 0.091 & .154& 0.292 &\textbf{.000}$^{\dagger\dagger}$\\
	    \hline
    \multirow{2}{*}{\textbf{Calibrated Trust}}&CT$_p$ & 0.053 & .404& 0.053 & .402& 0.056 & .378& 0.037 & .566\\
    &CT$_e$ & -0.120 & .059& -0.195 &\textbf{.002}$^{\dagger\dagger}$& -0.032 & .621& -0.174 &\textbf{.006}$^{\dagger\dagger}$\\
    \hline
    \textbf{Risk Perception}& Perceived Risk & -0.187 &\textbf{.003}$^{\dagger\dagger}$& -0.180 &\textbf{.004}$^{\dagger\dagger}$& -0.237 &\textbf{.000}$^{\dagger\dagger}$& -0.363 &\textbf{.000}$^{\dagger\dagger}$\\
    \hline
    
    % &Recall & -0.075 & .238& -0.077 & .230& -0.003 & .965& -0.013 & .843\\
    \multirow{3}{*}{\textbf{Task Performance}}&ACC$_s$ & 0.037 & .560& -0.014 & .823& 0.110 & .085& 0.018 & .772\\
    &ACC$_e$ & -0.000 & .995& -0.037 & .567& 0.085 & .184& 0.007 & .911\\
    &Plan Quality& -0.035 & .587& -0.037 & .560& 0.080 & .211& -0.032 & .611\\
    \hline
	\end{tabular}
        }
	\end{small}
\end{table*}

\begin{table}[ht]
% the environment \color{blue} change all cell color
	\centering
	\caption{Task-specific spearman rank-order correlation coefficient for plan quality and risk perception. ``${\dagger}$'' and ``${\dagger\dagger}$'' indicate the effect of the variable is significant at the level of 0.05 and 0.0125, respectively.}
	\label{tab:correlation-2}%
	\begin{small}
        \scalebox{0.95}{
	\begin{tabular}{l | l| c c | c c }
	    \hline
	    \multirow{2}{*}{\textbf{Category}}&\multirow{2}{*}{\textbf{Variables}}& \multicolumn{2}{c|}{Plan Quality} &  \multicolumn{2}{c}{Risk Perception}\\
     \cline{3-6}
     & &  $r$& $p$ &  $r$& $p$\\
	    \hline \hline
    \multirow{2}{*}{\textbf{User Trust}}&Trust-p & 0.056 & .032$^{\dagger}$& -0.293 &\textbf{.000}$^{\dagger\dagger}$\\
    &Trust-e & 0.258 &\textbf{.000}$^{\dagger\dagger}$& -0.160 &\textbf{.000}$^{\dagger\dagger}$\\
    \hline
    \multirow{2}{*}{\textbf{Calibrated Trust}}& CT$_p$ & 0.723 &\textbf{.000}$^{\dagger\dagger}$& -0.102 &\textbf{.000}$^{\dagger\dagger}$\\
    & CT$_e$& 0.221 &\textbf{.000}$^{\dagger\dagger}$& 0.000 & .995\\
    \hline
    \multirow{3}{*}{\textbf{Task Performance}}&Plan Quality & - &-& -0.141 &\textbf{.000}$^{\dagger\dagger}$\\
     % \cline{3-6}
    &ACC$_e$ & 0.400 &\textbf{.000}$^{\dagger\dagger}$& -0.110 &\textbf{.000}$^{\dagger\dagger}$\\
    &ACC$_s$ & 0.446 &\textbf{.000}$^{\dagger\dagger}$& -0.096 &\textbf{.000}$^{\dagger\dagger}$\\
    \hline
    \multirow{2}{*}{\textbf{Confidence}}&Confidence-p & 0.137 &\textbf{.000}$^{\dagger\dagger}$& -0.532 &\textbf{.000}$^{\dagger\dagger}$\\
    &Confidence-e & 0.225 &\textbf{.000}$^{\dagger\dagger}$& -0.271 &\textbf{.000}$^{\dagger\dagger}$\\
    \hline
	\end{tabular}
        }
	\end{small}
\end{table}

\subsubsection{Impact of Plan Quality and Risk Percetion.} Besides the measures calculated over task batch, a task-level 
%~\glcomment{Is `task-specific' the correct phrase? I mean calculating the correlation at the level of each task, but not really check how these variables correlate at each task}
analysis of plan quality and risk perception can deepen our understanding of their impacts. 
Besides measures adopted in Table~\ref{tab:correlation}, we include task-level confidence in this analysis and exclude the subscales from the trust in automation questionnaire.
Thus, we calculated Spearman rank-order correlation coefficients for task-level measures across all groups of participants (shown in Table~\ref{tab:correlation-2}). 
As we can see, both plan quality and risk perception significantly correlate with user trust, calibrated trust, task performance, and user confidence. 
The \textit{plan quality} shows a significant positive correlation with most measures, which indicates users perform better and calibrate their trust in the LLM agents in tasks with a high-quality plan. 
By contrast, the \textit{risk perceptions} shows a negative correlation with most measures and also a negative correlation with the plan quality. %\gladd{Impacts of these factors will be further discussed in Section~\ref{sec:discussion-findings}.}

% \begin{table*}[ht]
% % the environment \color{blue} change all cell color
% 	\centering
% 	\caption{Task-specific spearman rank-order correlation coefficient for plan quality and risk perception. ``${\dagger}$'' and ``${\dagger\dagger}$'' indicate the effect of the variable is significant at the level of 0.05 and 0.0125, respectively.}
% 	\label{tab:correlation-2}%
% 	% \begin{small}
%         % \scalebox{0.95}{
% 	\begin{tabular}{l | l| c c | c c }
% 	    \hline
% 	    \multirow{2}{*}{\textbf{Category}}&\multirow{2}{*}{\textbf{Variables}}& \multicolumn{2}{c|}{Plan Quality} &  \multicolumn{2}{c}{Risk Perception}\\
%      \cline{3-6}
%      & &  $r$& $p$ &  $r$& $p$\\
% 	    \hline \hline
%     \multirow{2}{*}{\textbf{User Trust}}&Trust-p & 0.056 & .032$^{\dagger}$& -0.293 &\textbf{.000}$^{\dagger\dagger}$\\
%     &Trust-e & 0.258 &\textbf{.000}$^{\dagger\dagger}$& -0.160 &\textbf{.000}$^{\dagger\dagger}$\\
%     \hline
%     \multirow{2}{*}{\textbf{Calibrated Trust}}& CT$_p$ & 0.723 &\textbf{.000}$^{\dagger\dagger}$& -0.102 &\textbf{.000}$^{\dagger\dagger}$\\
%     & CT$_e$& 0.221 &\textbf{.000}$^{\dagger\dagger}$& 0.000 & .995\\
%     \hline
%     \multirow{3}{*}{\textbf{Task Performance}}&Plan Quality & - &-& -0.141 &\textbf{.000}$^{\dagger\dagger}$\\
%      % \cline{3-6}
%     &ACC$_e$ & 0.400 &\textbf{.000}$^{\dagger\dagger}$& -0.110 &\textbf{.000}$^{\dagger\dagger}$\\
%     &ACC$_s$ & 0.446 &\textbf{.000}$^{\dagger\dagger}$& -0.096 &\textbf{.000}$^{\dagger\dagger}$\\
%     \hline
%     \multirow{2}{*}{\textbf{Confidence}}&Confidence-p & 0.137 &\textbf{.000}$^{\dagger\dagger}$& -0.532 &\textbf{.000}$^{\dagger\dagger}$\\
%     &Confidence-e & 0.225 &\textbf{.000}$^{\dagger\dagger}$& -0.271 &\textbf{.000}$^{\dagger\dagger}$\\
%     \hline
% 	\end{tabular}
%         % }
% 	% \end{small}
% \end{table*}

% \subsubsection{Risk perception} \glcomment{Task-specific analysis? What is the connection between user trust and risk perception}

\subsubsection{Failure Analysis}
\label{sec-failure-analysis}
% For tasks where the plan is imperfect (plan quality $<5$), we identify the imperfect plan to be the main cause of execution failure. 
As we find that plan quality substantially affects task execution accuracy, we look into task performance across different plan qualities. 
For the tasks with low-quality plans (plans fail to cover task information or plan with grammar errors, \ie plan quality=1, 2), the execution accuracy is $1.8\%$. 
While for tasks with a plan that may mislead action prediction (plan quality = 3, 4), our LLM agent-based daily assistant achieved $59\%$ execution accuracy. 
The average execution accuracy for tasks with a high-quality plan (plan quality =5) is $66.7\%$.

We further check 717 tasks where a high-quality plan (plan quality = 5) is provided. Among them, 235 tasks provide wrong execution results. The main causes are: 
(1) Wrong action parameter prediction ($48.9\%$). 
While action names match, one or more parameters mismatch the expected value at some step of the action sequence. 
(2) Invalid actions ($48.5\%$). Given a perfect plan, the LLM agent failed to predict one valid action (failed to predict one action name or failed to predict some action parameter value) to execute in some steps. 
(3) Wrong action name prediction ($2.6\%$). The generated action sequence has at least one action name prediction that mismatches the ground truth. %\glcomment{To discuss, these problems can be covered by the user. Shall we further discuss the errors made in each condition?}

\begin{figure}[h]
    \centering
    \includegraphics[width=0.48\textwidth]{figures/confidence_planning_execution_barplot.pdf}
    \caption{\revise{Bar plot for confidence dynamics, the x-axis denotes the task ordering index (shuffled for every participant). The error bars represent the 95\% confidence interval.}} 
    \label{fig:confidence}
    \Description{Bar plot for confidence dynamics, the x-axis denotes the task ordering index (shuffled for every participant). We observe that participants show higher confidence in the task execution outcomes, compared to planning outcomes. The confidence gap between the two stages get narrowed with user involvement in the execution stages.}
\end{figure}

\begin{table*}[htbp]
	\centering
	\caption{Excerpts from participants' responses to open questions regarding opinion.}
	\label{tab:example-opinion}%
	\scalebox{.9}{
    \begin{tabular}{p{0.72\textwidth}| p{0.08\textwidth} |p{0.24\textwidth}}
		\hline
		\textbf{Opinion towards Planning} &  \textbf{Sentiment}& \textbf{Reason}  \\
		\hline \hline
		I really like how organized it is. The step by step and numerical planning allows it to make sense in a clear and structured way, meaning there is less room for errors or misinformation& Positive&  helpful with reducing error\\
        \hline
        It was remarkable how quickly. It was able to achieve the goals which was set out in the tasks. I quite liked it I would definitely want something like this in my life as It would my my life much easier& Positive& Effective and make life easy\\
		% \hline
		% I'm not sure if that was supposed to be very mentally challenging but it was! I felt like I needed to understand coding or be some kind of computer nerd& Negative& high cognitive load and difficult for laypeople\\
		% \hline
		% While certainly useful, in its current form it is likely too complex for the layperson, who could easily fall into grievous error using such a system.& Negative& difficult for laypeople\\
		\hline
		As I said previously, it's far, far too detailed in an unnecessary way.  I'm not sure people need the entire plan of what the AI will do, as long as the job gets done.& Negative& too detailed\\
  \hline
        I found it really helpful, but made me slightly nervous thinking all my plans being successful are in the hands of ai tech& Mixed& helpful assistant, agency concerns \\
  \hline
  \textbf{Opinion towards Execution} &  \textbf{Sentiment}& \textbf{Reason}  \\
		\hline \hline
        The execution stage was amazing. I feel like this could be the future and we wont need to call or talk to people to get this kind of thing done ever again.& Positive& promising future\\
        \hline
        The execution stage went smoothly, except for a few rare instances of an error response before also saying the AI's automatic reply (which was correct).& Mixed& Smooth user experience, error response\\
		\hline
		I found it clunky and nit that user friendly& Negative& clunky, not user-friendly\\
        \hline
        This bit is user friendly, but very robotic, which makes it difficult to trust& Mixed& user-friendly, distrust due to robotic nature\\
		\hline
    \end{tabular}}
\end{table*}%

\subsubsection{Confidence Dynamics}
To visualize the user confidence in the planning and execution stage, we draw \revise{point plots} (see Figure~\ref{fig:confidence}) for user confidence in the task order. 
Overall, condition AP-AE shows the highest confidence in both the planning and execution stages. 
To verify the impact of user involvement in confidence, we adopted two-way ANOVA and post-hoc Tukey HSD test. 
We find that: (1) with user involvement in the planning, participants showed significantly lower confidence in planning (AP-AE > UP-AE, UP-UE); (2) with user involvement in the execution, participants showed a significantly lower confidence in execution (AP-AE > AP-UE, UP-UE). 
Meanwhile, users typically showed a higher confidence in the execution stage. 
Compared with conditions with automation execution (\ie condition AP-AE and UP-AE), the confidence gap narrows down in the conditions with user-involved execution (\ie condition AP-UE and UP-UE). %\glcomment{Try to combine user trust with confidence, but it seems not very explicit. I decide to first keep it as is}

% \begin{figure}[h]
%     \centering
%     \includegraphics[width=\textwidth]{figures/confidence_dynamics_barplot.pdf}
%     \caption{Line plot for confidence dynamics, the x-axis denotes the task ID.\glcomment{How about the new barplot. Besides, I want to check with you, which one works better? confidence based on each of the tasks (based on task ID) or confidence based on task ordering?}} 
%     \label{fig:confidence}
%     \Description{Line plot for confidence dynamics, the x-axis denotes the task ordering index (shuffled for every participant). We observe that participants show higher confidence in the task execution outcomes, compared to planning outcomes. The confidence gap between the two stages get narrowed with user involvement in the execution stages.}
% \end{figure}

% \begin{figure*}[h]
%     \centering
%     \includegraphics[width=\textwidth]{figures/confidence_planning_execution_pointplot.pdf}
%     \caption{\revise{Point plot for confidence dynamics, the x-axis denotes the task ordering index (shuffled for every participant). The error bars represent the 95\% confidence interval.}} 
%     \label{fig:confidence}
%     \Description{Point plot for confidence dynamics, the x-axis denotes the task ordering index (shuffled for every participant). We observe that participants show higher confidence in the task execution outcomes, compared to planning outcomes. The confidence gap between the two stages get narrowed with user involvement in the execution stages.}
% \end{figure*}

\subsection{Analysis of Open Feedback} At the end of our study, we collected open feedback regarding the planning stage, execution stage, and any other feedback using the following question: %is framed as :
`\textit{Please share any comments, remarks or suggestions regarding the planning/execution stage of LLM Assistant}' and `\textit{Do you have any other comments, remarks or suggestions regarding the study?}'. 
Overall, we analyzed all the feedback based on user opinions (positive, negative, mixed, neutral) and their suggestions. 
In our analysis, we ignored all phrases without any useful information like `None', `N/A', and `No comment'. 

\paratitle{\revise{Feedback and Suggestions}}. While most comments tended to show positive opinions (more than $80\%$) towards LLM agents as daily assistants, there are also negative opinions regarding the difficulty, expertise, trust, etc. We provide example excerpts from participants in Table~\ref{tab:example-opinion}. Besides opinions towards the system, some participants also appreciated our user-centric setup:"\textit{The study does a good job of emphasizing user experience by asking about perceptions of risk, trust, and confidence. This approach ensures that the evaluation is user-centric, which is important for assessing the real-world applicability of the LLM Assistant}."

% \begin{table*}[htbp]
% 	\centering
% 	\caption{Excerpts from participants' responses to open questions regarding opinion.}
% 	\label{tab:example-opinion}%
% 	\scalebox{.9}{
%     \begin{tabular}{p{0.7\textwidth}| p{0.1\textwidth} |p{0.2\textwidth}}
% 		\hline
% 		\textbf{Opinion towards Planning} &  \textbf{Sentiment}& \textbf{Reason}  \\
% 		\hline \hline
% 		I really like how organized it is. The step by step and numerical planning allows it to make sense in a clear and structured way, meaning there is less room for errors or misinformation& Positive&  helpful with reducing error\\
%         \hline
%         It was remarkable how quickly. It was able to achieve the goals which was set out in the tasks. I quite liked it I would definitely want something like this in my life as It would my my life much easier& Positive& Effective and make life easy\\
% 		% \hline
% 		% I'm not sure if that was supposed to be very mentally challenging but it was! I felt like I needed to understand coding or be some kind of computer nerd& Negative& high cognitive load and difficult for laypeople\\
% 		% \hline
% 		% While certainly useful, in its current form it is likely too complex for the layperson, who could easily fall into grievous error using such a system.& Negative& difficult for laypeople\\
% 		\hline
% 		As I said previously, it's far, far too detailed in an unnecessary way.  I'm not sure people need the entire plan of what the AI will do, as long as the job gets done.& Negative& too detailed\\
%   \hline
%         I found it really helpful, but made me slightly nervous thinking all my plans being successful are in the hands of ai tech& Mixed& helpful assistant, agency concerns \\
%   \hline
%   \textbf{Opinion towards Execution} &  \textbf{Sentiment}& \textbf{Reason}  \\
% 		\hline \hline
%         The execution stage was amazing. I feel like this could be the future and we wont need to call or talk to people to get this kind of thing done ever again.& Positive& promising future\\
%         \hline
%         The execution stage went smoothly, except for a few rare instances of an error response before also saying the AI's automatic reply (which was correct).& Mixed& Smooth user experience, error response\\
% 		\hline
% 		I found it clunky and nit that user friendly& Negative& clunky, not user-friendly\\
%         \hline
%         This bit is user friendly, but very robotic, which makes it difficult to trust& Mixed& user-friendly, distrust due to robotic nature\\
% 		\hline
%     \end{tabular}}
% \end{table*}%

Some participants also provided suggestions on how to further improve the design of LLM agent-based daily assistants. 
% Example excerpts from participants in Table~\ref{tab:example-suggestion}. 
Regarding the plan edit, participants hope we can provide more convenient edit operations like `drop/drag' to adjust plan text ordering and `undo' operation to tolerate unexpected mistakes. 
Some participants also found the plans too detailed, which could increase the cognitive load (cf. Table~\ref{tab:example-opinion} except 3). 
%\revise{The granularity of the plan can be a factor that may affect the effectiveness of human-AI collaboration. For example, a detailed plan can be useful when precise parameters are required, but a simple plan may prove to be more effective when to m.} 
As for the execution, many participants found it to be smooth. %think it works smoothly. 
At the same time, they think additional verification in each step may further enhance the reliability of daily assistants: ``\textit{For the execution stage, I commend it for creating an input formatting box to execute the user's request validating each requirement}.''
There are also comments about the whole plan-then-execute workflow: ``\textit{The planning was really challenging, and I mostly left the default plans (they looked fine). 
This worked in the main, but a couple clearly needed revisiting. I would approach this iteratively: plan, test, observe, back to planning, then another test, before reaching the desired outcome}.'' \revise{Our findings suggest open research opportunities to explore more effective ways to provide an overview of plans that trade-off user cognitive load resulting from granular descriptions, with the need to provide details to help users identify flaws. For example, we can consider developing methods to interactively allow users to flesh out further details in a plan. }
%Iteration upon the current plan-then-execute workflow to ensure the correct final outcome can be a future direction to explore.


% \begin{table}[htbp]
% 	\centering
% 	\caption{Excerpts from participants' responses to open questions as suggestions.}
% 	\label{tab:example-suggestion}%
% 	\scalebox{.85}{
%     \begin{tabular}{p{0.65\textwidth}| p{0.1\textwidth} |p{0.3\textwidth}}
% 		\hline
%   \textbf{Suggestion towards Planning} &  \textbf{Sentiment}& \textbf{Reason}  \\
% 		\hline \hline
%     The planning was really challenging, and I mostly left the default plans (they looked fine). This worked in the main, but a couple clearly needed revisiting. I would approach this iteratively : plan, test, observe, back to planning, then another test, before reaching the desired outcome.& &\\
%     \hline
%     I think being able to expand or collapse the steps in the planning stage would help make it easier. Also, a drag and drop facility with the required fields would also help.& &\\
%     \hline
%     Well thought out study and is explained well. I still would like see the instruction have a video-format for people less "technologically able". But I am very interested to see what comes next.& &\\
%   \hline
%   \textbf{Suggestion towards Execution} &  \textbf{Sentiment}& \textbf{Reason}  \\
% 		\hline \hline
%   The study provides a structured approach to task execution with an emphasis on clarity and verification. Incorporating real-time adjustments based on feedback could further improve its effectiveness.& & \\ \hline
%   Additional verification in each step would have been helpful.& & \\
%   \hline
%     \end{tabular}}
% \end{table}%

% \begin{table}[htbp]
% 	\centering
% 	\caption{Resulting main themes from the thematic analysis of participants' responses to the open questions pertaining to analogy-based explanations across stages.} %\glcomment{@Ujwal,@Gianluca. If you find some comments for the topic `Tutorial and Instruction' is not desirable, we can remove them}}
% 	\label{tab:example-topic}%
% 	\scalebox{.9}{
%     \begin{tabular}{p{0.1\textwidth}|p{0.3\textwidth}|p{0.3\textwidth}|p{0.3\textwidth}}
% 		\hline
% % 		\textbf{Participant Feedback} &  \textbf{Sentiment}& \textbf{Reason}  \\
% 		\multirow{2}{*}{\textbf{Topic}}&\multicolumn{3}{c}{\textbf{Participant Feedback}}\\
% 		\cline{2-4}
% 		&Planning& Execution& Other\\
% 		\hline \hline
% 		Expertise and Familiarity& %(1) the whole study felt like it was geared toward a computer science course rather than the average person 
%   (1) I felt like I needed to understand coding or be some kind of computer nerd (2) While certainly useful, in its current form it is likely too complex for the layperson, who could easily fall into grievous error using such a system.& (1) There is little knowledge on how The LLM Assistant executes actions and why it may not complete the whole agenda, making it difficult to identify the error when it makes mistakes. (2) I've used booking systems like this in the past. They seem very reliable. They often work well with a human back up at the end of the line.& It was a little hard for someone with little experience of the LLM Assistant. \\
% 		\hline
% 		Work Load& %(1) I'm not sure if that was supposed to be very mentally challenging but it was! 
%   (1) The planning stage required a fair bit of effort to complete. %(2) The planning was very time-consuming. 
%   (2) it took some time and effort to double check if the AI's planning was accurate & (1) This also seems very hands on, defeating the purpose. (2) An AI assistant should be streamlined and easy to use.  All that reading of the tasks and checking, a person might as well do the task themselves if they have to approve every step.& (1) The task was mentally exhausting. (2) Not enough reward payment for a lot of cognitive load and a long time of work (3) It was mentally challenging and time-consuming (4) The study was too long for the assigned time hence creating anxiety about the completion of the study.\\
% 		\hline
% 		% Time concern& & & (1) The study was too long for the assigned time hence creating anxiety about the completion of the study. (2) I felt like this study was interesting but took much longer than stated.\\
%   %       \hline
%         Tutorial and Instruction& (1) after doing the planning several times, I found that easier to understand. (2) The step by step instruction in the planning stage helps to understand the big picture and the elements or variables that require to be captured to achieve the outcome. (3) Instructions were wordy and would benefit from being reduced (4) The planning stage is clear but could use more examples to guide when to split, add, or delete steps.& (1) The execution stage was more intuitive. The tutorial made it seem more complicated than it actually was. (2) Instructions during the tutorial are a bit confusing. I was only able to understand fully when interacting with the execution system and through the tasks (3) It wasn't totally clear when something had failed if the system had corrected itself. and I got into an instruction loop a few times & (1) The instructions need to be "dumbed down". (2) More examples to start with and maybe start with the simpler tasks first. (3) I would rather a hand holding guide at the start to take me through the steps piece by piece. I found that I learned as I engaged in the task rather than reading the instructions at the start.\\
%         \hline
%         Trust and Uncertainty& (1) the planning stage of the LLM assistant was helpful and trustworthy. (2) A bit unreliable as it can make mistakes.& (1) This bit is user-friendly, but very robotic, which makes it difficult to trust (2) Seemed to work well but unsure if I'd trust it. (3) I would still rather do things myself& I would not trust LLM to do the tasks in the examples, I would prefer to do them myself.\\
% 		\hline
%     \end{tabular}}
% \end{table}%


% \paratitle{Themes}. 
% To dive deeper into the user feedback, we conducted thematic analysis using the ATLAS.TI tool. The thematic analysis results are presented in Table~\ref{tab:example-topic}. Most participants had positive experiences. However, we present a balanced set of excerpts in the 
% %While most participants show positive opinions, we tend to show more criticism in 
% Table~\ref{tab:example-topic}. The complete anonymized data  will be shared publicly.
% %Such reviews can give 
% This presents a critical reflection of our plan-then-execute LLM agent-based daily assistant and can help point out future directions worth exploring. 
% In summary, we found that the participants mainly gave feedback surrounding four aspects: (1) expertise and familiarity, (2) workload, (3) tutorial and instruction, and (4) trust and uncertainty. 
% We summarize the feedback into four themes. 
% First, some participants argue this daily assistant may be difficult for laypeople, who lack knowledge about computer science or LLM, to use. 
% Second, some participants feel the whole study is too long and mentally challenging. Third, some participants hope there can be a simplified tutorial with clear instructions, and more examples and practice will help user onboarding. 
% Last, while some participants think the daily assistant is trustworthy and accurate, others doubt its reliability and prefer to do things themselves. \glcomment{Check whether we should remove some negative feedback, especially about tutorial and instruction}
\section{Conclusion}
In this work, we introduce \gls{myrag}, an advanced \gls{qa} system that dynamically constructs \gls{mkg} while integrating sophisticated reasoning and external domain-specific search tools. The model exhibits significant improvements in accuracy and reasoning capabilities, particularly for medical question-answering tasks, outperforming other approaches of similar model size or 10 to 100 times larger. Using structured knowledge representations and advanced reasoning frameworks, our approach establishes a new benchmark for \gls{qa} in highly competitive and highly evolving domains such as medicine.

\section{Limitations}

Despite \gls{myrag} advancements, our approach has certain limitations. Firstly it relies on external search tools to introduce latency during the creation of \gls{mkg}. However, this occurs only once, when the \gls{mkg} is built from scratch for the first time. Additionally, while the model performs exceptionally well in medical domains, its applicability to non-medical tasks remains unexplored. 


Another limitation is the need for structured, authoritative sources of medical knowledge. Currently, \gls{myrag} retrieves information from diverse sources, including research articles and medical textbooks. However, as emphasized in clinical decision-making, treatment guidelines serve as essential references for standardized diagnosis and treatment protocols \cite{hager2024evaluation}. Future work on \gls{myrag} should focus on integrating structured access to these sources to ensure compliance with evidence-based medicine.


\section{Ethics Statement}
The development of \gls{llm}s for medical \gls{qa} requires careful ethical consideration due to risks of inaccuracy and bias. Ensuring the reliability of retrieved content is crucial, especially when integrating external knowledge sources. To mitigate these risks, we implement a confidence scoring mechanism into the \gls{mkg} to validate the information. However, bias detection and mitigation remain active research areas.

\section{Conclusion}
This work empirically studied human-AI collaboration based on plan-then-execute LLM agents. 
Adopting such LLM agents in various everyday scenarios, % of different task characteristics (\ie risk perception and plan quality), 
we analyzed the impact of user involvement in the planning and execution stages on user trust and overall task performance. 
We provide various interactions in each stage to help users fix imperfect plans and modify execution outcomes. 
Our results suggest that the LLM agents can provide plausible text plans to cover task requirements, which can be convincingly wrong. 
As a result, users develop uncalibrated trust in the planning and execution outcomes, and user involvement in the planning and execution stages fails to calibrate user trust (\textbf{RQ1}). 
We also found that the plan quality substantially affects the subsequent execution accuracy. 
Thus, when user involvement in planning can fix imperfect plans, the overall task performance (\ie plan quality, accuracy of action sequence, and execution accuracy) gets improved. 
However, user involvement in planning can also harm task plan quality where the original plan is good to begin with. As a result, the LLM agents demonstrate worse task performance in these tasks. 
In contrast, %\glcomment{another phrase here?}
user involvement in execution brings a more stable positive impact on task performance (\textbf{RQ2}). 
Our results suggest that plausible but wrong LLM outcomes can be detrimental to user trust calibration and overall task performance. 
We discussed the impact of convincingly wrong LLM outcomes and provided potential solutions and insights for future work. 
Furthermore, we synthesized key insights for better control and effective collaboration with plan-then-execute LLM agents. \revise{We also shed light on opportunities to design flexible collaborative workflows with human oversight for effective collaboration with LLM agents.} %  \glcomment{shall we rephrase a bit for the last sent? As we add one new implication paragraph, `human oversight and more flexible collaborative workflow'. we may replace the previous sent with `synthesized key insights about leveraging human oversight and more flexible collaborative workflow for better control and effective collaboration with LLM agents'}

Our results indicate that user involvement in the LLM agent workflow can be important in ensuring reliable task outcomes. 
Future work can further investigate how to detect and handle plausible but imperfect LLM outcomes and design effective interventions to fix such problems. 
We hope that our key findings and implications reported in this work will inspire further research on human-AI collaboration with LLM agents.

%%
%% The acknowledgments section is defined using the "acks" environment
%% (and NOT an unnumbered section). This ensures the proper
%% identification of the section in the article metadata, and the
%% consistent spelling of the heading.
\begin{acks}
This work was partially supported by the Delft Design@Scale AI Lab, the 4TU.CEE UNCAGE project, the Convergence Flagship ``ProtectMe'' project, by the Australian Research Council (ARC) Training Centre for Information Resilience (Grant No. IC200100022), and by an ARC Future Fellowship Project (Grant No. FT240100022). We made use of the Dutch national e-infrastructure with the support of the SURF Cooperative using grant no. EINF-5571 and EINF-9738. We finally thank all participants from Prolific and experts from our department.
\end{acks}

%%
%% The next two lines define the bibliography style to be used, and
%% the bibliography file.
\bibliographystyle{ACM-Reference-Format}
\bibliography{LLM_Agent}

%%
%% If your work has an appendix, this is the place to put it.
% \appendix
\clearpage
 
\definecolor{exampleblue}{RGB}{0, 114, 188} % Blue for header
\definecolor{exampleborder}{RGB}{0, 114, 188} % Blue for border
\definecolor{redtext}{RGB}{204, 0, 0}         % Red text for emphasis

\section{Appendix}
In this appendix, we provide additional details about the design of \textbf{NumericBench}, along with supplementary experimental results and case studies. The organization of the supplementary materials in this appendix is as follows:

\begin{enumerate}[leftmargin=*]
	
	\item \textbf{Question formats for contextual retrieval, comparison, and summary abilities.}  
As shown in Table~\ref{appx:number_question}, Table~\ref{appx:stock_question}, and Table~\ref{appx:weather_question}, 
we designed diverse question types tailored to each dataset to evaluate the three fundamental numerical abilities of LLMs: contextual retrieval, comparison, and summary. contextual retrieval  assesses the model’s capacity to accurately extract relevant numerical information from complex contexts; comparison tests the ability to analyze and compare numerical values;  Summary evaluates the synthesis of numerical information into concise and meaningful insights for tasks like reporting or trend analysis.


By designing tailored questions for each dataset, we ensure a comprehensive evaluation of LLMs’ numerical reasoning abilities across varying scenarios and complexities.
	\item \textbf{Basic numerical questions answered incorrectly by GPT-4o.}  
	As illustrated in Figure~\ref{fig:number_compare}, Figure~\ref{fig:multiplication}, and Figure~\ref{fig:number_couting}, GPT-4o failed to answer three basic numerical questions correctly. This result is surprising, considering GPT-4o's impressive performance in real-world applications. However, these findings highlight the weak fundamental numerical abilities of LLMs.
	
	\item \textbf{Token counts for short and long contexts.}  
	As shown in Table~\ref{tab:data_stat_short_long}, the token counts of long and short contexts differ significantly. This distinction enables a more thorough evaluation of LLM performance across scenarios involving varying context lengths. Short contexts are designed to test the model's ability to process and understand concise information, focusing on immediate comprehension and reasoning. In contrast, long contexts present a more complex challenge, requiring the model to handle extended sequences of information, maintain coherence over a larger context window, 
	and retrieve relevant details from earlier parts of the input. Such two type length can more comprehensively evaluate LLMs. 
	
	\item \textbf{Additional experimental results on noisy and varying-length contexts.}  
	As shown in Figure~\ref{fig:length_stock} and Figure~\ref{fig:length_weather}, existing LLMs perform poorly on the stock and weather datasets, although they achieve better performance compared to their results on short contexts. 
	Similarly, as shown in Figure~\ref{fig:noisy_weather}, LLMs perform poorly on noisy weather data.
 
 \item \textbf{Real failure cases of math-oriented LLMs.} In this paper, we do not compare existing math-oriented LLMs, such as Metamath-Llemma-7B~\citep{yu2023metamath}, Deepseek-Math-7B-Instruct~\citep{deepseek-math}, InternLM2-Math-7B~\citep{ying2024internlmmathopenmathlarge}, and MAmmoTH-7B~\citep{yue2023mammoth}. 
 This is primarily because these math-oriented LLMs are designed for specialized geometric and structured mathematical problems. They are unable to understand the tasks in NumericBench, fail to follow a correct reasoning process, and directly produce meaningless outputs. These failure cases are illustrated in Figure~\ref{fig:fail_internlm}, Figure~\ref{fig:fail_ds_math}, Figure~\ref{fig:fail_llemma}, and Figure~\ref{fig:fail_mammoth}.
 
\end{enumerate}

\noindent \textbf{The Use of AI Tools.} When writing  this paper, we use Grammarly\footnote{https://www.grammarly.com/} for automated spell checking and use GPT-4o\footnote{https://platform.openai.com/docs/models/gpt-4o} to refine several sentences.


\clearpage
 
 

\begin{table*}[!h]
	\centering
	
	\caption{Question format on number list dataset. R: contextual retrieval, C: comparison, S: summary. In the contextual retrieval task, a number $x$ is randomly selected from the given number list. For the comparison task, the $k$-th largest number is randomly generated within the range of one to the length of the number list. The indices $x$ corresponds to twenty percent of the length of the number list, while $y$ corresponds to eighty percent of the length. The number $z$ is randomly chosen within the range $(\min(\text{list}), \max(\text{list}))$. For the summary task, the top $k$ is set to thirty percent of the length of the number list.}
 
	\renewcommand{\arraystretch}{1.15}  
	\setlength{\tabcolsep}{1.5pt}  
	\begin{tabular}{c|c}
		\toprule
		\textbf{Ability}    & \textbf{Question Format} \\ \midrule
		\textit{\textbf{R}} &  \begin{tabular}[c]{@{}l@{}}
			$Q_0$: What is the index of the first occurrence of the number $x$ in the list?\\
			$Q_1$: What is the index of the last occurrence of the number $x$ in the list?\\
			$Q_2$: What is the number after the first occurrence of the number $x$ in the list?\\
			$Q_3$: What is the number before the last occurrence of the number $x$ in the list?\\
			$Q_4$: What is the index of the first even number in the list?\\
			$Q_5$: What is the index of the first odd number in the list?\\
			$Q_6$: What is the index of the last even number in the list?\\
			$Q_7$: What is the index of the last odd number in the list?
		\end{tabular} \\ \midrule
		\textit{\textbf{C}} &  \begin{tabular}[c]{@{}l@{}}
			$Q_8$: What is the index of the first occurrence of the $k$-th largest number in the given list?\\
			$Q_9$: Which index holds the greatest number in the list between the indices $x$ and $y$?\\
			$Q_{10}$: Which index holds the smallest number in the list between the indices $x$ and $y$?\\
			$Q_{11}$: Which number is closest to $z$ in the list between the indices $x$ and $y$?\\
			$Q_{12}$: Which number is furthest from $z$ in the list between the indices $x$ and $y$?\\
			$Q_{13}$: Which number is the largest among those less than $z$ in the list?\\
			$Q_{14}$: Which number is the smallest among those greater than $z$ in the list?
		\end{tabular} \\ \midrule
		\textit{\textbf{S}} &  \begin{tabular}[c]{@{}l@{}}
			$Q_{15}$: What is the maximum sum of any two consecutive items in the list?\\
			$Q_{16}$: What is the maximum sum of any three consecutive items in the list?\\
			$Q_{17}$: What is the maximum absolute difference between two consecutive items in the list?\\
			$Q_{18}$: What is the sum of the indices of the top $k$ largest numbers in the list?\\
			$Q_{19}$: What is the sum of the indices of the top $k$ smallest numbers in the list?\\
			$Q_{20}$: What is the average of the indices of the top $k$ largest numbers in the list?\\
			$Q_{21}$: What is the average of the indices of the top $k$ smallest numbers in the list?\\
			$Q_{22}$: How many times do numbers consecutively increase for more than five times?\\
			$Q_{23}$: How many times do numbers consecutively decrease for more than five times?\\
			$\cdots \cdots$ \\
		\end{tabular} \\ \bottomrule
	\end{tabular}	
	\label{appx:number_question}
\end{table*}
\clearpage


 

\begin{table*}[]
	
	\caption{Question format on stock dataset. R: contextual retrieval, C: comparison, S: summary. $x$ and $y$ lie within the minimum and maximum ranges of their respective attributes. The top $k$ corresponds to thirty percent of the number list. $date_1$ represents the day at the twentieth percentile of the stock history, while $date_2$ corresponds to the day at the eightieth percentile.}
	\centering
	\renewcommand{\arraystretch}{1.15} % 设置行间距为默认的 1.15 倍
	\setlength{\tabcolsep}{1.5pt} % 将列间距设置为 1pt
	\begin{tabular}{c|c}
		\toprule
		\textbf{Ability}    & \textbf{Question Format} \\ \midrule
		\textit{\textbf{R}} &  \begin{tabular}[c]{@{}l@{}}
			$Q_0$: On which date did the close price of the stock first reach $x$ yuan?\\
			$Q_1$: On which date did the highest price of the stock first reach $x$ yuan?\\
			$Q_2$: On which date did the volume of the stock first reach $x$ lots?\\
			$Q_3$: On which date did the value of the stock first reach $x$ thousand yuan?\\
			$Q_4$: On which date did the price change rate of the stock first reach $x$\%?\\
			$Q_5$: On which date did the price change of the stock first reach $x$ yuan?\\
		\end{tabular} \\ \midrule
		\textit{\textbf{C}} &  \begin{tabular}[c]{@{}l@{}}
			\begin{tabular}[c]{@{}l@{}}
				$Q_6$: On which date did the stock have the highest turnover rate when the close \\price was greater than $x$ yuan?
			\end{tabular}\\
			
			\begin{tabular}[c]{@{}l@{}}
				$Q_7$: On which date did the stock have the highest quantity relative ratio when \\the open price was less than $x$ yuan?
			\end{tabular}\\
			
			\begin{tabular}[c]{@{}l@{}}
				$Q_8$: On which date did the stock have the highest difference between the highest \\and lowest prices when the trading volume exceeded $x$ lots?
			\end{tabular}\\
			
			\begin{tabular}[c]{@{}l@{}}
				$Q_9$: On which date did the stock record the highest daily average price, calculated \\as 'value' divided by 'volume,' when the PE ratio was less than $x$?
			\end{tabular}\\
			
			\begin{tabular}[c]{@{}l@{}}
				$Q_{10}$: Among the top-$k$ trading value days, on which date did the stock have the \\lowest close price?
			\end{tabular}\\
			
			\begin{tabular}[c]{@{}l@{}}
				$Q_{11}$: When the quantity relative ratio exceeded $x$, on which date did the stock \\have the highest sum of the open price and close price?
			\end{tabular}\\
			
			\begin{tabular}[c]{@{}l@{}}
				$Q_{12}$: When the absolute price change rate exceeded $x$\%, on which date did the \\stock have the highest difference between the highest and lowest prices?
			\end{tabular}
		\end{tabular} \\ \midrule
		\textit{\textbf{S}} &  \begin{tabular}[c]{@{}l@{}}
			\begin{tabular}[c]{@{}l@{}}
				$Q_{13}$: How many days had a volume greater than $x$ from $date_1$ to $date_2$?
			\end{tabular}\\
			
			\begin{tabular}[c]{@{}l@{}}
				$Q_{14}$: How many days had the close price higher than the open price from \\$date_1$ to $date_2$?
			\end{tabular}\\
			
			\begin{tabular}[c]{@{}l@{}}
				$Q_{15}$: How many days had a close price higher than the open price, with the \\quantity relative ratio exceeding $x$\%?
			\end{tabular}\\
			
			\begin{tabular}[c]{@{}l@{}}
				$Q_{16}$: How many days had the close price reach $x$ yuan with the absolute price \\change rate exceeding $x$\%?
			\end{tabular}\\
			
			\begin{tabular}[c]{@{}l@{}}
				$Q_{17}$: What was the average trading volume when both the turnover rate \\exceeded $x$\% and the price change rate was greater than $y$\%?
			\end{tabular}\\
			
			\begin{tabular}[c]{@{}l@{}}
				$Q_{18}$: Excluding non-trading days, how many times did the open price of \\the stock rise for three or more consecutive days?
			\end{tabular}\\
			
			\begin{tabular}[c]{@{}l@{}}
				$Q_{19}$: Excluding non-trading days, how many times did the close price of \\the stock rise for three or more consecutive days?
			\end{tabular}\\
			
			\begin{tabular}[c]{@{}l@{}}
				$Q_{20}$: Excluding non-trading days, how many times did the open price and \\close price of the stock both rise for three or more consecutive days?
			\end{tabular}\\
		
			\begin{tabular}[c]{@{}l@{}}
			$\cdots \cdots$
		\end{tabular}
		
		\end{tabular} \\ \bottomrule
	\end{tabular}
\label{appx:stock_question}
\end{table*}
\clearpage
 

\begin{table*}[]
	\centering
	\caption{Question format on weather dataset.  R: contextual retrieval, C: comparison, S: summary. The value of $x$ falls within the minimum and maximum ranges of its respective attribute. $date_1$ represents the day at the twentieth percentile of the stock history, while $date_2$ represents the day at the eightieth percentile.}
	\begin{tabular}{c|c}
		\toprule
		\textbf{Ability}    & \textbf{Question Format} \\ \midrule
		\textit{\textbf{R}} &  \begin{tabular}[c]{@{}l@{}}
			$Q_0$: On which date did the temperature at two meters first reach $x$°C?\\
			$Q_1$: On which date did the relative humidity at two meters first exceed $x$\%?\\
			$Q_2$: On which date did the dew point temperature at two meters first drop below $x$°C?\\
			$Q_3$: On which date did the precipitation first exceed $x$ mm?\\
			$Q_4$: On which date did the sea-level air pressure first exceed $x$ hPa?\\
			$Q_5$: On which date did the cloud cover first reach $x$\%?\\
			$Q_6$: On which date did the wind speed at 10 meters first exceed $x$ m/s?
		\end{tabular} \\ \midrule
		\textit{\textbf{C}} &  \begin{tabular}[c]{@{}l@{}}
			\begin{tabular}[c]{@{}l@{}}
				$Q_7$: On which date did the temperature at two meters reach its highest value \\ 
				when the relative humidity was below $x$\%? 
			\end{tabular} \\
			
			\begin{tabular}[c]{@{}l@{}}
				$Q_8$: On which date did the relative humidity at two meters reach its lowest value \\ 
				when the temperature at two meters was above $x^\circ$C?
			\end{tabular} \\
			
			\begin{tabular}[c]{@{}l@{}}
				$Q_9$: On which date did the difference between the temperature and dew point \\ 
				at two meters reach its maximum when the cloud cover was below $x$\%? 
			\end{tabular} \\
			
			\begin{tabular}[c]{@{}l@{}}
				$Q_{10}$: On which date did the precipitation reach its highest value \\ 
				when the temperature at two meters was below $x^\circ$C? 
			\end{tabular} \\
			
			\begin{tabular}[c]{@{}l@{}}
				$Q_{11}$: On which date did the cloud cover reach its lowest value \\ 
				when the wind speed at 10 meters exceeded $x$ m/s? 
			\end{tabular} \\
			
			\begin{tabular}[c]{@{}l@{}}
				$Q_{12}$: On which date did the wind speed at 10 meters reach its highest value \\ 
				when the precipitation exceeded $x$ mm? 
			\end{tabular} \\
			
			\begin{tabular}[c]{@{}l@{}}
				$Q_{13}$: On which date did the sea-level air pressure reach its highest value \\ 
				when the cloud cover was below $x$\%? 
			\end{tabular}
		\end{tabular} \\ \midrule
		\textit{\textbf{S}} &  \begin{tabular}[c]{@{}l@{}}
			\begin{tabular}[c]{@{}l@{}}
				$Q_{14}$: How many days had a temperature at two meters greater than $x^\circ$C \\from $date_1$ to $date_2$? 
			\end{tabular} \\
			
			\begin{tabular}[c]{@{}l@{}}
				$Q_{15}$: How many days had a relative humidity at two meters exceeding $x$\% \\from $date_1$ to $date_2$? 
			\end{tabular} \\
			
			\begin{tabular}[c]{@{}l@{}}
				$Q_{16}$: How many days had a precipitation greater than $x$ mm from $date_1$ \\to $date_2$? 
			\end{tabular} \\
			
			\begin{tabular}[c]{@{}l@{}}
				$Q_{17}$: What was the average temperature at two meters when the relative \\humidity exceeded $x$\%? 
			\end{tabular} \\
			
			\begin{tabular}[c]{@{}l@{}}
				$Q_{18}$: What was the average wind speed at 10 meters when the precipitation \\exceeded $x$ mm? 
			\end{tabular} \\
			
			\begin{tabular}[c]{@{}l@{}}
				$Q_{19}$: How many times did the temperature at two meters rise for three or more \\consecutive days? 
			\end{tabular} \\
			
			\begin{tabular}[c]{@{}l@{}}
				$Q_{20}$: How many times did the relative humidity at two meters drop for \\three or more consecutive days? 
			\end{tabular} \\
		
					
		\begin{tabular}[c]{@{}l@{}}
			$\cdots \cdots$
		\end{tabular} \\
	
		\end{tabular} \\ \bottomrule
	\end{tabular}
\label{appx:weather_question}
\end{table*}

\clearpage
\begin{figure*}[t]
	\centering	
	\vspace{-1em}
	\frame{
		\includegraphics[width = 0.9\textwidth]{image/intro_example/number_compare.png}
	}
	%	\captionsetup{labelformat=empty}
	%	\addtocounter{figure}{-1}
	\caption{Number comparisons on GPT-4o. The correct answer is -9.11. }
	\label{fig:number_compare}
\end{figure*}

\begin{figure*}[t]
	\centering	
	\vspace{-1em}
	\frame{
		\includegraphics[width = 0.9\textwidth]{image/intro_example/multiplication.png}
	}
	%	\captionsetup{labelformat=empty}
	%	\addtocounter{figure}{-1}
	\caption{Number multiplication on GPT-4o. The correct answer is 102244.12. }
	\label{fig:multiplication}
\end{figure*}
\begin{figure*}[t]
	\centering	
	\vspace{-1em}
	\frame{
		\includegraphics[width = 0.9\textwidth]{image/intro_example/number_counting.jpg}
	}
	%	\captionsetup{labelformat=empty}
	%	\addtocounter{figure}{-1}
	\caption{Number counting on GPT-4o, which is required directly give answer. The correct answer is 4. }
	\label{fig:number_couting}
\end{figure*}

\clearpage


%\subsection{Additional Experiment Results}
%\subsubsection{Additional results on context length evaluation for stock and weather data}\label{appx:sssec:length}
		\begin{figure*}[t]
		\centering 	
		\subfloat[Contextual Retrieval]	
		{\centering\includegraphics[width=0.33\linewidth]{image/main_fig/retrieval-stock.pdf}}
		\hfill
		\subfloat[Comparison]
		{\centering\includegraphics[width=0.33\linewidth]{image/main_fig/compare-stock.pdf}}
		\subfloat[Summary]	
		{\centering\includegraphics[width=0.33\linewidth]{image/main_fig/summary-stock.pdf}}
		\hfill
		%	\subfloat[MUTAG]
		%	{\centering\includegraphics[width=0.25\linewidth, height=3.05cm]{image/g1-4.pdf}}	
		%	\hfill
		%	
		\caption{Evaluation on short and long context on stock dataset. Due to the input sequence length limit of Qwen2.5-72B-Inst on the API platform, the long dataset of all three abilities cannot be evaluated using this model.}
		\label{fig:length_stock}
	\end{figure*}
	
	
	
	\begin{figure*}[t]
		\centering 	
		\subfloat[Contextual Retrieval]	
		{\centering\includegraphics[width=0.33\linewidth]{image/main_fig/retrieval-weather.pdf}}
		\hfill
		\subfloat[Comparison]
		{\centering\includegraphics[width=0.33\linewidth]{image/main_fig/compare-weather.pdf}}
		\subfloat[Summary]	
		{\centering\includegraphics[width=0.33\linewidth]{image/main_fig/summary-weather.pdf}}
		\hfill
		%	\subfloat[MUTAG]
		%	{\centering\includegraphics[width=0.25\linewidth, height=3.05cm]{image/g1-4.pdf}}	
		%	\hfill
		%	
		\caption{Evaluation on short and long context on weather dataset. Due to the input sequence length limit of Qwen2.5-72B-Inst on the API platform, the long dataset of all three abilities cannot be evaluated using this model.}
		\label{fig:length_weather}
		
	\end{figure*}

		\begin{figure*}[t]
		
		\centering 	
		\subfloat[Contextual  Retrieval]	
		{\centering\includegraphics[width=0.33\linewidth]{image/noisy_dataset_fig/retrieval-noisy-weather.pdf}}
		\hfill
		\subfloat[Comparison]
		{\centering\includegraphics[width=0.33\linewidth]{image/noisy_dataset_fig/compare-noisy-weather.pdf}}
		\subfloat[Summary]	
		{\centering\includegraphics[width=0.33\linewidth]{image/noisy_dataset_fig/summary-noisy-weather.pdf}}
		\hfill
		%	\subfloat[MUTAG]
		%	{\centering\includegraphics[width=0.25\linewidth, height=3.05cm]{image/g1-4.pdf}}	
		%	\hfill
		%	
		\caption{Evaluation on  noisy weather dataset. Due to the input sequence length limit of Qwen2.5-72B-Inst on the API platform, the data containing 4 and 6 irrelevant attributes cannot be evaluated using this model.}
		\label{fig:noisy_weather}
		
	\end{figure*}
\clearpage
	
	
	
	
%	\begin{figure*}[t]
%		
%		\centering 	
%		\subfloat[Context Retrieval]	
%		{\centering\includegraphics[width=0.33\linewidth]{image/multi_dataset_fig/retrieval-multi-stock.pdf}}
%		\hfill
%		\subfloat[Comparison]
%		{\centering\includegraphics[width=0.33\linewidth]{image/multi_dataset_fig/compare-multi-stock.pdf}}
%		\subfloat[Summary]	
%		{\centering\includegraphics[width=0.33\linewidth]{image/multi_dataset_fig/summary-multi-stock.pdf}}
%		\hfill
%		%	\subfloat[MUTAG]
%		%	{\centering\includegraphics[width=0.25\linewidth, height=3.05cm]{image/g1-4.pdf}}	
%		%	\hfill
%		%	
%		\caption{Evaluation on multi-turn QA on stock dataset. Due to the input sequence length limit of Qwen2.5-72B-Inst on the API platform, the model cannot generate outputs in the third turn of the conversation. }
%		\label{fig:multurn_stock}
%		
%	\end{figure*}
	
	\begin{table*}[]
		\caption{The average token number on short and long instances for each data.}
		\centering
		\begin{tabular}{c|c|cc|cc}
			\toprule
			\multirow{2}{*}{\textbf{Dataset}}                                               & \multirow{2}{*}{\textbf{Ability}} & \multicolumn{2}{c|}{\textbf{Short}}                            & \multicolumn{2}{c}{\textbf{Long}}                             \\ \cmidrule{3-6} 
			
			&                                   & \multicolumn{1}{c|}{\textbf{\# Instance}} & \textbf{Avg Token} & \multicolumn{1}{c|}{\textbf{\# Instance}} & \textbf{Avg Token} \\ \midrule
			
			\multirow{3}{*}{\textbf{\begin{tabular}[c]{@{}c@{}}Number\\ List\end{tabular}}} & \textit{Contextual Retrieval}

                  & \multicolumn{1}{c|}{500}                     &        809.12     & \multicolumn{1}{c|}{500}                     &         6599.34      \\   
			
			& \textit{Comparison}                        & \multicolumn{1}{c|}{500}                     &     804.86     & \multicolumn{1}{c|}{500}                     &        6566.27      \\ 
			
			& \textit{Summary}



                  & \multicolumn{1}{c|}{500}                     &       822.49      & \multicolumn{1}{c|}{500}                     &       6487.07       \\ \midrule
			
			\multirow{3}{*}{\textbf{Stock}}                                                 & \textit{Contextual Retrieval}

                  & \multicolumn{1}{c|}{500}                     &        18529.07      & \multicolumn{1}{c|}{500}                     &      36641.63     \\  
			& \textit{Comparison}                        & \multicolumn{1}{c|}{500}                     &    18539.58     & \multicolumn{1}{c|}{500}                     &      36651.22      \\ 
			& \textit{Summary}

                  & \multicolumn{1}{c|}{500}                     &      18504.51      & \multicolumn{1}{c|}{500}                     &       36618.07      \\ \midrule
			
			\multirow{3}{*}{\textbf{Weather}}                                               & \textit{Contextual Retrieval}

                  & \multicolumn{1}{c|}{500}                     &        18362.38        & \multicolumn{1}{c|}{500}                     &        36356.13    \\  
			& \textit{Comparison}                        & \multicolumn{1}{c|}{500}                     &        18371.11    & \multicolumn{1}{c|}{500}                     &       36365.27     \\ 
			& \textit{Summary}



                  & \multicolumn{1}{c|}{500}                     &        18334.15     & \multicolumn{1}{c|}{500}                     &        36328.27    \\ \bottomrule
		\end{tabular}
		\label{tab:data_stat_short_long}
	\end{table*}
	\clearpage
%	\begin{figure*}[t]
%		
%		\centering 	
%		\subfloat[Context Retrieval]	
%		{\centering\includegraphics[width=0.33\linewidth]{image/multi_dataset_fig/retrieval-multi-weather.pdf}}
%		\hfill
%		\subfloat[Comparison]
%		{\centering\includegraphics[width=0.33\linewidth]{image/multi_dataset_fig/compare-multi-weather.pdf}}
%		\subfloat[Summary]	
%		{\centering\includegraphics[width=0.33\linewidth]{image/multi_dataset_fig/summary-multi-weather.pdf}}
%		\hfill
%		%	\subfloat[MUTAG]
%		%	{\centering\includegraphics[width=0.25\linewidth, height=3.05cm]{image/g1-4.pdf}}	
%		%	\hfill
%		%	
%		\caption{Evaluation on multi-turn QA on weather dataset.}
%		\label{fig:multurn_weather}
%		
%	\end{figure*}
	

	
	
%\noindent\textbf{Case Study}
	\newtcolorbox[auto counter, number within=section]{examplebox}[2][]{%
		colback=white,        % Background color
		colframe=exampleborder, % Border color
		coltitle=white,       % Header text color
		fonttitle=\bfseries,  % Header font
		colbacktitle=exampleblue, % Header background color
		title=Example~\thetcbcounter: #2, % Title format
	}
	\begin{figure*}
		\begin{examplebox}{InternLM2-Math-7B}
			\textbf{Failure:} \\Model repeatedly output the same number until the output sequence ends.
			
			\vspace{0.2cm}
			\textbf{User:} \\ Question: What is the maximum sum of any three consecutive items in the list? Options: A: 21501.9737231253, B: 21502.8144848529, C: 21501.9642201794, D: 21502.7354388437, E: 21502.0327435643, F: 21501.7526575621, G: 21501.7057960155, H: 21502 Data: [-5279, ....(100 numbers)]
			
			\vspace{0.3cm}
			\textbf{InternLM2-Math-7B Model:} \\
			Step 1: To find the maximum sum of any three consecutive items in the list, we need to identify the three largest numbers in the list and add them together. <PRM +> \\Step 2: We can start by sorting the list in ascending order to make it easier to identify the largest numbers. <PRM +> \\Step 3: After sorting, the list becomes: [-8509, -8509, -8509, -8509, -8509, -8509, -8509, -8509, -8509, -8509, -8509, -8509, -8509, -8509, -8509, -8509, -8509, -8509, -8509, -8509, -8509, -8509, -8509, -8509, -8509, -8509, -8509, -8509...(output too long)
		\end{examplebox}
		\caption{Failure type and conversation record of InternLM2-Math-7B. The model constantly generate the same "-8509" text until the output sequence was too long and truncated, leaving the problem unsolved.}
		\label{fig:fail_internlm}
	\end{figure*}
	
	\begin{figure*}
		\begin{examplebox}{DeepSeek-Math-7B}
				\textbf{Failure:} \\ Model outputs sequence without meaning, probably due to overly long input sequences. 
				
				\vspace{0.2cm}
				\textbf{User:} \\Question: What is the sum of the index of top 300 smallest numbers in the list? Options: A: 146734, B: 146738, C: 146733, D: 146724, E: 146728, F: 146727, G: 146721, H: 146729 Data: [3640, 1560, ...(100 numbers)]
				
				\vspace{0.3cm}
				\textbf{Deepseek-Math-7B Model:} \\
				31,3,9,3,1,1,3,3,3,3,3,3,3,3, 3,3,1,3,3,3,1,3,3,3, 3, 3,3,3, 1, 1,3,3, 3,1,3, 3,3,3, 1, 3,3,3, 3,3,1,3,1,3,91, 3,3,3, 3,31,3,3,3, 3, 339, 3, 3, 311, 1, 3, 1,3,3, 3,3,6,3,3, 3,31,3,3,1, 1, 1, 1, 3,1,1, 3,3, 1,1, 3,1,3,1, 1, 1, 1, 1, 19,1, 1, 1,191,1, 1,1, 19, 31,1,1, 1, 3,1, 1,3, 1133,3, 1, 1,1, 11, 11,3, 31, 3, 3,1,1,3,1, 1,33, 11,3,1911,3, 1,13,3131311119191,1, 3,1,11,1,19,11,1,1,1....(keeps outputting numbers)
		\end{examplebox}
		\caption{Failure type and conversation record of DeepSeek-Math-7B. The model only generates a series of numbers throughout its output, and we're unable to acquire any useful information from the output.}
		\label{fig:fail_ds_math}
	\end{figure*}	

	\begin{figure*}
		\begin{examplebox}{Metamath-Llemma-7B}
			\textbf{Failure:}\\ Model outputs ineffective chain of thoughts, and giving the wrong answer in the wrong format(ground truth is B))
			
			\vspace{0.2cm}
			\textbf{User:}\\ Question: What is the sum of the index of top 30 smallest numbers in the list? Options: A: 1566, B: 1572, C: 1575, D: 1573, E: 1577, F: 1564, G: 1569, H: 1567 Data: [-2568.1542995196, 1035.3645340299...(100 numbers)]
			
			\vspace{0.3cm}
			\textbf{Metamath-Llemma-7B Model:} \\ The data you receive is a list of numbers. To find the sum of the index of the top 30 smallest numbers, we need to find the 30 smallest numbers in the list and then calculate their sum. To find the 30 smallest numbers, we can use the min function in Python. The min function takes a list as input and returns the smallest number in the list. We can use the min function to find the 30 smallest numbers in the list. Now, we need to calculate the sum of these 30 smallest numbers. To do this, we can use the sum function in Python. The sum function takes a list as input and returns the sum of all the numbers in the list. We can use the sum function to calculate the sum of the 30 smallest numbers. Finally, we need to find the index of the sum. To do this, we can use the index function in Python. The index function takes a number as input and returns the index of that number in the list. We can use the index function to find the index of the sum. Therefore, the answer is (correct option). \#\#\#\# 1566 The answer is: 1566
		\end{examplebox}
		\caption{Failure type and conversation record of Metamath-Llemma-7B. The model attempts to use programming language to solve the problem, but it does not provide executable lines of code, only providing vague description of certain functions instead. After outputting the vague solution, the model outputs an answer, which is different from the ground truth and not in the expected format(We expect the model to provide the answer with a capital letter corresponding to the correct option).}
		\label{fig:fail_llemma}
	\end{figure*}

	\begin{figure*}
	\begin{examplebox}{MAmmoTH-13B}
		\textbf{Failure:}\\ Model outputs ineffective chain of thoughts, and giving the wrong answer in the wrong format(ground truth is B))
		
		\vspace{0.2cm}
		\textbf{User:}\\ Question: What is the sum of the index of top 30 smallest numbers in the list? Options: A: 1566, B: 1572, C: 1575, D: 1573, E: 1577, F: 1564, G: 1569, H: 1567 Data: [-2568.1542995196, 1035.3645340299...(100 numbers)]
		
		\vspace{0.3cm}
		\textbf{MAmmoTH-13B Model:}\\  Notice that there are two numbers in your list that have the smallest index: 4346.5174909575 and 8169.6394989672. They have the smallest index of 4346 and 8169 respectively. Since in the options the first number is the index of the smallest number, then the correct option is D.\\             Option: D
	\end{examplebox}
	\caption{Failure type and conversation record of Mammoth-13B. The model fail to understand the problem's instruction, and does not provide reasonable output. Moreover, it provides the wrong answer in incorrect format, just like Metamath-Llemma-7B(We expect the model to provide the answer with a capital letter corresponding to the correct option).}
	\label{fig:fail_mammoth}
	\end{figure*}

\end{document}
\endinput
%%
%% End of file `sample-manuscript.tex'.
