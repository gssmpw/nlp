\clearpage
\appendix

\section{Details about Tasks}

% \begin{table}[htbp]
% 	\centering
% 	\caption{Selected tasks in our study. The `Risk' is based on the risk feedback obtained with pilot study.}
% 	\label{tab:tasks}%
% 	\scalebox{.84}{
%     \begin{tabular}{p{0.02\textwidth}|p{0.4\textwidth}|p{0.4\textwidth}|p{0.2\textwidth}}
% 		\hline
% % 		\textbf{Participant Feedback} &  \textbf{Sentiment}& \textbf{Reason}  \\
%         \textbf{ID}& \textbf{Task Description}& \textbf{Primary Plan}& \textbf{Execution}\\
% 		\hline \hline
% 		1& My account ID is 54321, and the password is PWD2023. I plan to make two foreign exchange transactions. The first is to buy 10,000 euros (with USD), and the second is to sell 5,000 US dollars (to EUR). Please help me operate.& 1. Log into the first account; 1.1 Get login information for the account (Account ID: 12345678, Account Password: Password123)& simple task, imperfect plan\\
%         \hline
%         2& Please inquire about the current debt amount of my credit card with the last five digits 12345, and deduct the corresponding 12000 USD from my savings card number 6212345678900011 to repay this debt, then help me check the amount of the outstanding bill for the same credit card within 30 days after today.& complex task, imperfect plan\\
%         \hline
%         3& I need to schedule a repair for my TV at 6 PM tomorrow evening. The brand is Sony, model X800H, and there is an issue with the screen. Please book the repair service and tell me the reservation number.& complex task, imperfect plan\\
%         \hline
%         4& I need to set an alarm for every weekday morning at 7:30, and then cancel the alarm for Thursday, changing it to 8:00 in the evening.& simple task, correct plan\\
%         \hline
%         5& I have an important meeting to attend next Wednesday, and I need to book a flight ticket from London to Amsterdam for tomorrow, it must be a morning flight, and then return from Amsterdam to London tomorrow night, please handle it for me.& simple task, correct plan\\
%         \hline
%         6& Please plan a trip for me departing on October 1st at 8:00 AM to Japan, returning on October 7th at 11:00 PM, including Tokyo Disneyland, Senso-ji Temple, Ginza, Mount Fuji, Kyoto cultural experience, Universal Studios Osaka, and visiting the Nara Deer Park on October 4th, and help me find hotels where the nightly cost does not exceed 10,000 Japanese yen.& complex task, correct plan\\
%         \hline
%             \hline
%     \end{tabular}}
% \end{table}%

Here, we provide more details about the selected task used in our study. For each task, we provide the automatic planning (\ie stepwise text plan) and execution (\ie action sequence) generated with our plan-then-execute LLM agent.

% \paratitle{Example task in tutorial}: Can you help me log into my two different platform accounts and then check their account balances? The first account ID is 12345678, password is Password123; the second account ID is 87654321, password is 123Password.

% \paratitle{Automatic plan}:
% \begin{itemize}
%     \item 1. Log into the first account
%     \begin{itemize}
%         \item 1.1 Get login information for the account (Account ID: 12345678, Account Password: Password123)
%         \item 1.2 Log into the account
%         \item 1.3 Confirm whether the first account is successfully logged in
%     \end{itemize}
%     \item 2. Check the first account balance
%     \begin{itemize}
%         \item 2.1 Get account balance inquiry information (Account ID: 12345678)
%         \item 2.2 Inquire about account balance
%         \item 2.3 Get the first account balance information (Account Balance: First account balance information)
%     \end{itemize}
%     \item 3. Log into the second account
%     \begin{itemize}
%         \item 3.1 Get login information for the account (Account ID: 87654321, Account Password: 123Password)
%         \item 3.2 Log into the account
%         \item 3.3 Confirm whether the second account is successfully
%     \end{itemize}
%     \item 4. Check the second account balance
%     \begin{itemize}
%         \item 4.1 Get account balance inquiry information (Account ID: 87654321)
%         \item 4.2 Inquire about account balance
%         \item 4.3 Get the second account balance information (Account Balance: Second account balance information)
%     \end{itemize}
% \end{itemize}

% \paratitle{Automatic execution}:
% \begin{itemize}
%     \item bank\_account\_login( account=12345678, passwd=Password123)
%     \item check\_balance(account=12345678, passwd=Password123)
%     \item bank\_account\_login(account=87654321, passwd=123Password)
%     \item check\_balance(account=87654321, passwd=123Password)
% \end{itemize}

% \paratitle{Summary}: both planning and execution are correct.

\paratitle{Task-1}: My account ID is 54321, and the password is PWD2023. I plan to make two foreign exchange transactions. The first is to buy 10,000 euros (with USD), and the second is to sell 5,000 US dollars (to EUR). Please help me operate.

\paratitle{Automatic Plan}:
\begin{itemize}
    \item 1. Log in to user account
    \begin{itemize}
        \item 1.1 Obtain account login information (Account ID: 54321, Account Password: PWD2023)
        \item 1.2 Log in to the account
        \item 1.3 Confirm successful login
    \end{itemize}
    \item 2. Conduct foreign exchange transactions
    \begin{itemize}
        \item 2.1 Buy euros
        \begin{itemize}
            \item 2.1.1 Obtain information for buying euros (Currency Type: EUR, Purchase Amount: 10000)
            \item 2.1.2 Buy the specified amount of euros
            \item 2.1.3 Confirm successful euro purchase
        \end{itemize}
        \item 2.2 Sell US dollars
        \begin{itemize}
            \item 2.2.1 Obtain information for selling US dollars (Currency Type: USD, Sell Amount: 5000)
            \item 2.2.2 Check the US dollar holdings
            \item 2.2.3 Obtain US dollar holdings information (Foreign Exchange Holdings Information: Obtained US dollar holdings information)
            \item 2.2.4 Sell the specified amount of US dollars
            \item 2.2.5 Confirm successful US dollar sale
        \end{itemize}
    \end{itemize}
\end{itemize}

\paratitle{Automatic execution}:
\begin{itemize}
    \item bank\_account\_login(account=54321, passwd=PWD2023)
    \item buy\_currency(account=54321, passwd=PWD2023, sourceCurrency=USD, targetCurrency=EUR, amount=10000)
\end{itemize}

\paratitle{Summary}: The plan is imperfect. Users are supposed to split the primary step 2 into two primary steps. Then, the LLM agent can conduct two transactions instead of one.

\paratitle{Task-2}: Please inquire about the current debt amount of my credit card with the last five digits 12345, and deduct the corresponding 12000 USD from my savings card number 6212345678900011 to repay this debt, then help me check the amount of the outstanding bill for the same credit card within 30 days after today.

\paratitle{Automatic Plan}:
\begin{itemize}
    \item 1. Inquire about the credit card debt amount
    \begin{itemize}
        \item 1.1 Obtain credit card information (Account ID: Account ID, Account Password: Account Password, Credit Card: Last five digits 12345, Verification Code: Verification Code)
        \item 1.2 Log in to the banking service platform
        \item 1.3 Inquire about the current credit card debt amount
        \item 1.4 Confirm the debt amount inquiry is successful and obtain the amount (Debt Amount: Inquired Debt Amount)
    \end{itemize}
    \item 2. Deduct funds to repay the credit card debt
    \begin{itemize}
        \item 2.1 Obtain savings card information (Savings Card Number: 6212345678900011)
        \item 2.2 Transfer from savings card to credit card
        \item 2.3 Confirm the transfer is successful
    \end{itemize}
    \item 3. Check the credit card's future outstanding bills
    \begin{itemize}
        \item 3.1 Obtain date information (Date Range: Within 30 days after today)
        \item 3.2 Inquire about the credit card's outstanding bill amount within the specified date range
        \item 3.3 Confirm the future outstanding bill amount inquiry is successful and obtain the amount (Outstanding Bill Amount: Inquired Outstanding Bill Amount)
    \end{itemize}
\end{itemize}

\paratitle{Automatic execution}:
\begin{itemize}
    \item check\_balance(card=12345)
    \item pay\_credit\_card(source\_card=6212345678900011, target\_card =12345, amount=12000)
    \item check\_credit\_card\_bills(card=12345, period=30)
\end{itemize}

\paratitle{Summary}: The authors need to revise the plan and add one primary step to search for the credit card, which ends with `12345'. Due to missing this step, the LLM agent can not correctly execute all tasks because the correct credit card ID (\ie 4980981092312345) is missing.

\paratitle{Task-3}: I need to schedule a repair for my TV at 6 PM tomorrow evening. The brand is Sony, model X800H, and there is an issue with the screen. Please book the repair service and tell me the
reservation number.

\paratitle{Automatic Plan}:
\begin{itemize}
    \item 1. Obtain user contact information
    \begin{itemize}
        \item 1.1 Obtain user's name (Name: User's name)
        \item 1.2 Obtain user's contact number (Contact number: User's phone number)
        \item 1.3 Obtain the installation address of the user's TV (Installation address: User's address)
    \end{itemize}
    \item 2. Find repair service
    \begin{itemize}
        \item 2.1 Determine the repair service provider
        \begin{itemize}
            \item 2.1.1 Search for authorized Sony TV repair service providers
            \item 2.1.2 Select the nearest service provider to the user (Service provider list: Authorized Sony TV repair service provider list, User address: User's address)
        \end{itemize}
        \item 2.2 Obtain the contact information of the repair service provider (Service provider contact information: Nearest service provider's contact information to the user)
    \end{itemize}
    \item 3. Book repair service
    \begin{itemize}
        \item 3.1 Contact the repair service provider
        \begin{itemize}
            \item 3.1.1 Provide the brand, model, and issue description of the TV (Brand: Sony, Model: X800H, Issue description: Screen issue)
            \item 3.1.2 Provide the user's contact information (Name: User's name, Contact number: User's phone number, Installation address: User's address)
            \item 3.1.3 Obtain the appointment time (Appointment time: Tomorrow's date 18:00:00)
            \item 3.1.4 Book the repair service
        \end{itemize}
        \item 3.2 Obtain the reservation number (Reservation status: Repair request status, Reservation number: Repair request number)
    \end{itemize}
    \item 4. Notify the user
    \begin{itemize}
        \item 4.1 Send the reservation number and the contact information of the repair service provider to the user (Reservation number: Repair request number, Service provider contact information: Repair service provider's contact information)
    \end{itemize}
\end{itemize}

\paratitle{Automatic execution}:
\begin{itemize}
    \item obtain\_user\_info()
    \item search\_service\_provider(brand=Sony)
    \item appliance\_repair\_request(applianceModel=X800H, applianceType=TV, appointment\_time=Tomorrow 6 PM,issueDescription =Screen issue, servicepointID=[Sony01, Sony03])
    \item appliance\_repair\_request(applianceModel=X800H, applianceType=TV, appointment\_time=Tomorrow 6 PM,issueDescription =Screen issue, servicepointID=[Sony01, Sony03])
\end{itemize}

\paratitle{Summary}: While the first three primary steps are clear, the last primary step is not aligned with any provided actions. Users are supposed to delete the last primary step or override the potentially unnecessary actions after the first three steps.

\paratitle{Task-4}: I need to set an alarm for every weekday morning at 7:30, and then cancel the alarm for Thursday, changing it to 8:00 in the evening.

\paratitle{Automatic Plan}:
\begin{itemize}
    \item 1. Set the alarm for every day
    \begin{itemize}
        \item 1.1 Get the alarm setting information (Time: 07:30 AM, Frequency: Monday to Friday)
        \item 1.2 Set the alarm
        \item 1.3 Confirm whether the alarm is set successfully
    \end{itemize}
    \item 2. Cancel the Thursday alarm
    \begin{itemize}
        \item 2.1 Get the information of the alarm to be cancelled (Time: 07:30 AM, Frequency: Thursday)
        \item 2.2 Cancel the alarm
        \item 2.3 Confirm whether the alarm is cancelled successfully
    \end{itemize}
    \item 3. Set the new alarm time
    \begin{itemize}
        \item 3.1 Get the information of the alarm to be adjusted (New Time: 08:00 PM, Frequency: Thursday)
        \item 3.2 Set the new alarm
        \item 3.3 Confirm whether the new alarm is set successfully
    \end{itemize}
\end{itemize}

\paratitle{Automatic execution}:
\begin{itemize}
    \item create\_alarm(hour=7, minute=30, repeat=Weekly, frequency =Weekday)
    \item cancel\_alarm(hour=7, minute=30, repeat=Weekly, frequency =Thursday)
    \item create\_alarm(hour=20, minute=0, repeat=Weekly, frequency =Thursday)
\end{itemize}

\paratitle{Summary}: Both the automatic planning and execution are correct.

\paratitle{Task-5}: I have an important meeting to attend next Wednesday, and I need to book a flight ticket from London to Amsterdam for tomorrow, it must be a morning flight, and then return from
Amsterdam to London tomorrow night, please handle it for me.

\paratitle{Automatic Plan}:
\begin{itemize}
    \item 1. Search outbound flight ticket
    \begin{itemize}
        \item 1.1 Obtain outbound flight search information (Departure City: London, Arrival City: Amsterdam, Departure Date: Tomorrow's Date, Departure Time Slot: Morning)
        \item 1.2 Search for outbound flights
        \item 1.3 Select a suitable outbound flight (Flight Number: Selected Flight Number)
    \end{itemize}
    \item 2. Book outbound flight ticket
    \begin{itemize}
        \item 2.1 Obtain outbound passenger information (Name: User Name, ID Number: User ID Number)
        \item 2.2 Book outbound flight ticket
        \item 2.3 Confirm whether the outbound flight ticket is successfully booked
    \end{itemize}
    \item 3. Search return flight ticket
    \begin{itemize}
        \item 3.1 Obtain return flight search information (Departure City: Amsterdam, Arrival City: London, Departure Date: Tomorrow's Date, Departure Time Slot: Evening)
        \item 3.2 Search for return flights
        \item 3.3 Select a suitable return flight (Flight Number: Selected Flight Number)
    \end{itemize}
    \item 4. Book return flight ticket
    \begin{itemize}
        \item 4.1 Obtain return passenger information (Name: User Name, ID Number: User ID Number)
        \item 4.2 Book return flight ticket
        \item 4.3 Confirm whether the return flight ticket is successfully booked
    \end{itemize}
\end{itemize}

\paratitle{Automatic execution}:
\begin{itemize}
    \item search\_flight(departure\_city=London, arrival\_city=Amsterdam, date=Tomorrow, timePeriod='Morning')
    \item book\_flight(flight\_id=flight\_id, flight\_class=economics)
    \item search\_flight(departure\_city=Amsterdam, arrival\_city=London, date=Tomorrow, timePeriod='Evening')
    \item book\_flight(flight\_id=flight\_id, flight\_class=economics)
\end{itemize}

\paratitle{Summary}: Both the automatic planning and execution are correct.

\paratitle{Task-6}: Please plan a trip for me departing on October 1st at 8:00 AM to Japan, returning on October 7th at 11:00 PM, including Tokyo Disneyland, Senso-ji Temple, Ginza, Mount Fuji, Kyoto cultural experience, Universal Studios Osaka, and visiting the Nara Deer Park on October 4th, and help me find hotels where the nightly cost does not exceed 10,000 Japanese yen.

\paratitle{Automatic Plan}:
\begin{itemize}
    \item 1. Itinerary planning
    \begin{itemize}
        \item 1.1 Obtain itinerary information (destination: Japan, starting from October 1st 8:00 AM, return around October 7th 11:00 PM, including Tokyo Disneyland, Senso-ji Temple, Ginza, Mount Fuji, Kyoto cultural experience, Universal Studios Osaka, and visiting the Nara Deer Park on October 4th)
        \item 1.2 Plan the itinerary according to the info
        \item 1.3 show all planned itineraries
    \end{itemize}
    \item 2. Itinerary Selection
    \begin{itemize}
        \item 2.1 Show users the planned itinerary
        \item 2.2 ask users to select one route from them
        \item 2.3 confirm the selected itinerary and show the schedule
    \end{itemize}
    \item 3. Book hotels
    \begin{itemize}
        \item 3.1 Obtain travel budget (Accommodation budget: Not exceeding 10,000 yen per night)
        \item 3.2 Search for hotels within budget
        \item 3.3 arrange hotels for each night within budget (nightly cost does not exceed 10,000 Japanese yen)
        \item 3.4 Complete hotel booking and show the arranged hotel plan
    \end{itemize}
\end{itemize}

\paratitle{Automatic execution}:
\begin{itemize}
    \item travel\_itinerary\_planner(destination=Japan, departure\_time= October 1st 8:00 AM, return\_time=October 7th 11:00 PM, interests=Tokyo Disneyland, Senso-ji Temple, Ginza, Mount Fuji, Kyoto cultural experience, Universal Studios Osaka, and visiting the Nara Deer Park on October 4th)
    \item select\_itinerary(selection=route-1)
    \item book\_hotel(selection=route-1)
\end{itemize}

\paratitle{Summary}: The plan is correct. However, the LLM agent makes mistakes in selecting the route. The expected route should be route-2 instead of route-1. The wrong parameters will further affect the hotel booking in step 3.
