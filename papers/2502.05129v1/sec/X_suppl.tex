\clearpage
\setcounter{page}{1}
\maketitlesupplementary


% taken from related work; TODO remove overlapping content
\section{Salmon monitoring additional information}


Several methods exist for monitoring salmonid escapement, each with their own trade-offs. In narrow, shallow streams, constructing weirs (fences with a gate controlled by a technician) allows technicians to count salmon one by one as they pass through. Counting towers---structures built on a stream bank which give technicians an unobstructed, overhead view of their side of the stream---serve a similar purpose. To sample a greater number of streams in a certain area, fisheries also conduct manual aerial surveys from aircraft. In large or turbid rivers, however, these methods fail to produce reliable results. In addition, these techniques are often only applied to a subset of the duration of fish passage---\eg the first ten minutes of the hour, every hour---with results extrapolated to a 24 hour period~\cite{AlaskaNonSonar, Key2017}. 

% Sonar-based monitoring has grown in popularity due to its potential to address the limitations of \jk{...TODO...} 
For several decades, split-beam sonar has been used in rivers in Alaska and the Pacific Northwest to monitor salmon populations. It functions even at night or in turbid waters and provides bank-to-bank coverage~\cite{Key2017}.

Recently, imaging sonar with multiple beams (Fig \ref{fig:sonar}) has become more popular since it provides a higher-resolution view of the river. It is now the most popular method in rivers such as the Kenai in Alaska or the Eel in California. 
Upon manual review and under careful placement of sonar cameras to e.g. minimize blind spots, imaging sonar can produce counts with high precision ($<$3\%) and with similar accuracy to weir-based counting methods~\cite{holmes2006accuracy}. Fish lengths can also be determined with high precision~\cite{COOK201959}. Both of these findings are complicated by the presence of excessive debris or the passage of high-density schools, which can confuse or obstruct individual fish.

\begin{figure}[t]
  \centering
  \includegraphics[width=0.4\linewidth]{sonar1.png}
  \includegraphics[width=0.5\linewidth]{kr_kl.png}
  \caption{Left: depiction of the horizontal plane of a multi-beam sonar configuration; right: camera placement on the left and right banks of the Kenai river in Alaska~\cite{Key2017}.
  }
  \label{fig:sonar}
\end{figure}


\subsection{Sonar data review process}

Sonar-based monitoring comes with its own challenges: in some cases the cameras can produce over 30GB of data a day, which need to be manually reviewed. Depending on the specific needs of the river (\eg conservation vs recreational fishing management), sonar footage can be reviewed in bulk after the season or almost in real time. This process of manual review---finding portions with fish, and measuring fish lengths so as to filter by species---is time-intensive for the technicians~\cite{AlaskaSonar}.

One step that helps streamline the process is the usage of an \textit{echogram}, a 2D representation of the entire length of the clip: each column of an echogram represents one frame of a clip, with pixel intensity corresponding to the maximum intensity across all sonar beams at that range. Technicians use this representation, generated by the proprietary DIDSON or ARIS software which processes camera footage, to identify parts of the clip that are worth watching in full to perform counts and length measurements (Fig \ref{fig:aris}). Still, especially in rivers with high fish passage, obtaining full counts even in a subset of the data stream is a time-intensive and error-prone process.

\begin{figure}[b]
  \centering
  \includegraphics[width=0.7\linewidth]{aris_echogram.png}
  \caption{ARIS display software used by sonar technicians showing the echogram view and corresponding frame in sonar video~\cite{Key2017}.
  }
  \label{fig:aris}
\end{figure}


\section{Data preprocessing}

All images in the training, validation, and test sets are subject to a sequence of transformations to standardize the input format. The available transformations are described below in order of application.

\begin{enumerate}
    \item \textbf{Shift and rescale all channels.} Shifts all pixel values (initially lying between 0 and 1) down by 0.5 and divides the result by 0.25.
    \item \textbf{Resize to 200 by 800 pixels.}
\end{enumerate}