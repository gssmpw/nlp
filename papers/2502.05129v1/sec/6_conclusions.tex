\section{Conclusions}

We introduce a new method for salmon population monitoring based on an \textit{echogram}, a 2D representation of an entire sonar video clip, that is more computationally efficient than existing methods for determining fish counts which are applied to individual frames of a sonar video. Our initial results are promising: a lightweight ResNet-18 model achieves significant reductions in nMAE which bring us to count errors comparable to proofs of concept of more expensive models, through appropriate dataset selection, echogram generation, and data augmentation.

Future evaluations and iterations on this model should address the class imbalance between upstream- and downstream-moving fish, develop a larger and more diverse validation set, and fine-tune the echogram generation and data augmentation procedures. % As a format which incorporates information about larger time scales which are missing from the detector and tracker pipeline developed in \cite{kay2022caltechfishcountingdataset}, the echogram could also serve as useful context for a full pipeline. 