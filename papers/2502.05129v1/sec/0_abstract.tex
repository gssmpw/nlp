\begin{abstract}
Accurate estimates of salmon \textit{escapement}---the number of fish migrating upstream to spawn---are key data for conservation and fishery management. Existing methods for salmon counting using high-resolution imaging sonar hardware are non-invasive and compatible with computer vision processing. Prior work in this area has utilized object detection and tracking based methods for automated salmon counting. However, these techniques remain inaccessible to many sonar deployment sites due to limited compute and connectivity in the field. We propose an alternative lightweight computer vision method for fish counting based on analyzing \textit{echograms}---temporal representations that compress several hundred frames of imaging sonar video into a single image. We predict upstream and downstream counts within 200-frame time windows directly from echograms using a ResNet-18 model, and propose a set of domain-specific image augmentations and a weakly-supervised training protocol to further improve results. We achieve a count error of 23\% on representative data from the Kenai River in Alaska, demonstrating the feasibility of our approach.

% The ABSTRACT is to be in fully justified italicized text, at the top of the left-hand column, below the author and affiliation information.
% Use the word ``Abstract'' as the title, in 12-point Times, boldface type, centered relative to the column, initially capitalized.
% The abstract is to be in 10-point, single-spaced type.
% Leave two blank lines after the Abstract, then begin the main text.
% Look at previous \confName abstracts to get a feel for style and length.
\end{abstract}