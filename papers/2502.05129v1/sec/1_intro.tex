\section{Introduction}
\label{sec:intro}

% Salmon are ecologically important to these regions as they provide a food source during spawning months and fertilize the riverbed after their migration in from the ocean. Recently, salmonid populations in some rivers have declined, and since salmon always return to their home river to spawn, even if the population at large is healthy, ecosystems around individual rivers may suffer from a decline in salmonid levels. 

% - brief overview of current pipeline

% - motivation of using echograms instead as a first filter

% - first attempt at encoding temporal info - using motion artifacts to determine which frames most likely contain fish - while also increasing efficiency

% Can we use the echogram representation of a sonar clip to infer counts directly, and could incorporating these count predictions in an ensemble model lead to more robust final count statistics?

Accurate salmon population monitoring enables %allows fisheries and state departments to make 
data-driven fishery management and conservation. In particular, fishery managers and conservationists are interested in salmon \textit{escapement}: the abundance of migrating salmon returning from the sea that successfully spawn. Several methods exist for monitoring migrating salmon (see \cref{sec:related}). 
Sonar-based monitoring has recently grown in popularity due to its non-invasive nature and ability to collect data at high temporal resolution under a variety of conditions.
However, sonar cameras produce large amounts of data---in some cases over 30GB of data a day~\cite{AlaskaSonar}---and 
% Depending on the specific needs of the river (\eg conservation vs. fishery management), sonar footage can be reviewed in bulk after the season or almost in real time. 
reviewing this data
% ---finding portions with fish, and measuring fish lengths so as to filter by species---
is time-intensive for technicians, with no existing alternative that generalizes across sites.

Computer vision has the potential to more efficiently and accurately analyze sonar video for escapement monitoring. Prior work has introduced automated approaches based on object detection and multi-object tracking~\cite{kay2022caltechfishcountingdataset}. These approaches achieve counting errors of under 10\%; however, they rely upon processing each video frame independently with deep networks (\eg YOLOv5m with 21.2M params~\cite{kay2022caltechfishcountingdataset}), making them currently unsuitable for deployment in locations with limited compute and connectivity.

% on independently and identically distributed data. However, 
% due to their reliance upon very deep neural networks, these techniques %remain constrained to high-performance computing environments and 
% are not yet suitable for reliable deployment in locations with limited compute and connectivity.
% In these approaches, an object detection model is run on individual frames from a given clip and then a multi-object tracker is used to link together predicted bounding boxes into tracks. Once these tracks are determined, different heuristics may be used to determine fish counts. \jk{These approaches have been shown to be accurate but computationally expensive, ......}

In this paper, we explore an alternative approach to automated salmon counting in sonar video that harnesses a temporal representation called an \textit{echogram}. Echograms compress a multi-beam sonar video into a 2D image (see \cref{fig:echogram}). The $x$-axis of an echogram represents time. At each $x$-value, a column vector represents a compressed view of an entire frame of video. In this column vector, % is a 2D representation of the entire length of the clip: each column of an echogram represents one frame of a clip, with 
the pixel intensity at each $y$-value corresponds to the maximum intensity across all sonar beams at the corresponding range. If fish are present, this will result in a noticeable visual signature. Sonar technicians use these echogram visualizations during data review, %cross-reference this representation with the full video clip, 
both to identify temporal regions of interest %to perform counts and length measurements, 
and to cross-check challenging counts. %Still, especially in rivers with high fish passage, obtaining full counts even in a subset of the data stream is a time-intensive and error-prone process.

We propose and evaluate the feasibility of a method for analyzing echograms with computer vision to directly predict fish counts, providing a low-compute alternative to object detection and tracking pipelines. Our method takes as input a 200px wide echogram image and predicts the number of fish moving upstream or downstream during the corresponding timeframe, thus requiring only a single forward pass every 200 frames through a lightweight backbone (\eg a ResNet-18 with 11.7M params~\cite{he2016deep}) to compute counts. We further propose a set of domain-specific image augmentations as well as a weakly-supervised training protocol that incorporates annotations generated ahead of time by an object detector and tracker. 

Our initial model achieves counting error rates of 23\% on a validation set that is in-distribution with respect to the training set and 30.7\% on an out-of-distribution test set, nearly matching initial proofs of concept for much more computationally expense tracking-by-detection approaches~\cite{kulits2020automated}. We perform quantitative and qualitative analyses to identify challenges in echogram-based approaches as well as promising areas for future work.