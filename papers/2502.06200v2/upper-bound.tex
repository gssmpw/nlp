\section{The upper bound}\label{sec:ub}

In this section we prove \Cref{thm:main-ub}. We construct a distribution $\pi$  that satisfies $\DTV(\mu,\pi)\le \frac{\eps}{2}$, has smoothness close to that of $\mu$, is of bounded moment, and whose \Poincare constant is at least $\approx \tp{\frac{LM}{d\eps}}^{-O(d)}$. Then we call known Langevin-based algorithm to sample from $\pi$, whose sample complexity is directly related to the \Poincare constant. Our strategy for the construction of $\pi$ is as follows.

\begin{itemize}
    \item First, observe the following comparison result. Let $p_1(x)$ and $p_2(x)$ be the densities of two distributions supported on $\bb R^n$. If $1/C\le \frac{p_1(x)}{p_2(x)}\le C$ for every $x\in \bb R^n$, then the ratio of their \Poincare constants is at least $C^{-\+O(1)}$. Therefore, we only need to construct a distribution $\pi$ with appropriate smoothness, whose density is pointwise close to a suitable Gaussian. The range of ratios in the densities that we can tolerate is of the order $\tp{\frac{LM}{d\eps}}^{\+O(d)}$.
    \item Clearly the density $p_\mu(x)\propto e^{-f_\mu(x)}$ of $\mu$ does not satisfy our requirement due to the possible existence of certain regions with extremely small probability. The value of $f_\mu$ may be very large in these regions. On the other hand, the measure of these regions under $\mu$ is small, so we can \emph{truncate} $f_\mu$ appropriately to ensure its value is well upper bounded without affecting the measure $\mu$ much. We then use the truncated function to define the distribution $\pi$.
    \item In order to truncate $f_\mu$ appropriately, we need to estimate its minimum value $f^*\defeq \min_{x\in\bb R^d} f_\mu(x)$ and the partition function $Z_\mu \defeq \int_{\bb R^n} \exp\tp{-f_\mu(x)} \d x$ within a certain accuracy. To this end, we divide a compact set containing most mass of $\mu$ into cubes and approximate $\mu$ in each cube respectively using queries to $f_\mu$.
\end{itemize}
%\htodo{We only use queries to $f_\mu$, no $\grad f_{\mu}$ in the estimation?}

We will give the construction of $\pi$ in \Cref{sec:construction-of-pi} and prove its properties in \Cref{sec:properties-of-pi}. Then we show how to estimate the key parameters in our construction in \Cref{sec:estimate-of-pi}. Finally, we combine everything and prove \Cref{thm:main-ub} in \Cref{sec:proof-of-ub}.

\subsection{The construction of $\pi$}\label{sec:construction-of-pi}

The purpose of this section is to construct a distribution $\pi$ whose density function is close to that of $\gamma\sim\+N\tp{0,\frac{M}{\eps d}\!{Id}_d}$ \emph{pointwise} and $\DTV(\mu,\pi)\le \frac{\eps}{2}$. We assume the density of $\pi$ is \emph{proportional to} $\exp\tp{-f_\pi(x)}$. Let $f_\gamma\colon \bb R^d\to\bb R$ be the function $x\mapsto \frac{\eps d\norm{x}^2}{2M}+\frac{d}{2}\log\frac{2\pi M}{d\eps}$. Then the density of $\gamma$ is \emph{equal to} $\exp\tp{-f_\gamma(x)}$.

We use $Z_\pi \defeq \int_{\bb R^d} \exp\tp{-f_\pi(x)}\d x$ and $Z_\mu\defeq \int_{\bb R^d} \exp\tp{-f_\mu(x)} \d x$ to denote the two normalizing factors. Then the density of $\pi$ and $\mu$ are
\[
    p_\pi(x) = \exp\tp{-f_\pi(x)} / Z_\pi \mbox{ and } p_{\mu}(x) = \exp\tp{-f_{\mu}(x)} / Z_\mu
\]
respectively.

Note that the second moment of a random variable $X$ with law $\mu$ is at most $M$. By Markov's inequality, for $R=\sqrt{\frac{32M}{\eps}}$,
\begin{equation}\label{eqn:markov-mu}
    \Pr{\norm{X}^2>R^2} \le \frac{\E{\norm{X}^2}}{R^2}\le \frac{M}{R^2} = \frac{\eps}{32},
\end{equation}
meaning outside a ball of radius $\Theta(R)$, the mass of $\mu$ is $O(\eps)$. We let $f_\pi(x) = f_\gamma(x) - \log\eps$ for $x\in \bb R^d \setminus \+B_{2R}$. We will then construct a function $f_\pi^{\le 2R}$ with support $\+B_{2R}$ and define
\[
    f_\pi(x) = (1-\mathfrak{g}_{[R,2R]}(x)) \cdot f_\pi^{\le 2R}(x) + \mathfrak{g}_{[R,2R]}(x) \cdot \tp{f_\gamma(x)-\log\eps},
\]
where $\mathfrak{g}_{[R,2R]}\defeq q_{\!{mol}}\tp{\frac{\norm{x}^2-R^2}{(2R)^2-R^2}}$ is the smooth function interpolating $f_\pi^{\le 2R}$ and $f_\gamma - \log\eps$ in the region $\norm{x}\in [R,2R]$. 

%Let $Z_\pi\defeq \int_{\bb R^n}$ The construction of $f_\pi^{\le 2R}$ should meet the 
As discussed before, for $x\in \+B_R$, ideally $p_{\pi}(x) = \exp\tp{-f_{\pi}^{\le 2R}(x)} / Z_\pi$ should be close to $p_{\mu}(x) = \exp\tp{-f_\mu(x)} / Z_\mu$ with those points of extremely small probability smoothly truncated. As a result, we first assume that we can find an approximation of $Z_\mu$, denoted as $\wh Z_\mu$. In general calculating a good approximation for $Z_\mu$ is computationally equivalent to sampling from $p_\mu$. However, as our target is a bound for the \Poincare constant of order exponential in $d$, our requirement for the accuracy of the approximation is very loose. We also assume an approximation $\wh f^*$ of the minimum $f^*\defeq \inf_{x\in \+B_{2R}} f_\mu(x)$. In fact, we will prove the following proposition in \Cref{sec:estimate-of-pi}.

\begin{proposition} \label{prop:Z-and-fmin}
    Within $\+O\tp{\frac{LM}{\eps d}}^d$ queries to $f_\mu(x)$, one can find
    \begin{itemize}
        \item a number $\wh Z_\mu$ satisfying $\frac12 e^{-d}\le \frac{\wh Z_\mu}{Z_\mu}\le 1$, and
        \item a number $\wh f^*$ satisfying $f^*\le \wh f^*\le f^*+d$.
    \end{itemize}    
\end{proposition}

Then we turn to truncate the small value of $\exp(-f_\mu(x))$ or equivalently the large value of $f_\mu(x)$ in $\+B_{2R}$. Define two constants
\[
    h_1 \defeq \wh f^* +\log\!{vol}(\+B_{2R}) + \frac{d}{2}\log L+\log\frac{4}{\eps},\; h_2 \defeq h_1+\frac{d}{2}\log\frac{LM}{d\eps}.
\]
We remark that $h_1$ is our threshold for the truncation. The term $\frac{d}{2}\log L$ term is used to guarantee that $\log\!{vol}(\+B_{2R}) + \frac{d}{2}\log L$ is nonnegative and therefore $h_1\ge f^*$. In order to keep the truncated function smooth, we define a \emph{soft threshold} $h_2$ above $h_1$. 

Define the interpolation function $\mathfrak{g}_{[h_1,h2]}(x) \defeq q_{\!{mol}}\tp{\frac{h_2-f_{\mu}(x)}{h_2-h_1}}$. We define
\[
    \ol{f_{\pi}^{\le 2R}}(x)\defeq \mathfrak{g}_{[h_1,h_2]}(x)\cdot f_\mu(x) + (1-\mathfrak{g}_{[h_1,h_2]}(x))\cdot h_2
\]
In other words, for those $x$ with $f_\mu(x)\le h_1$, $\ol{f_{\pi}^{\le 2R}}(x) = f_\mu(x)$; for those $x$ with $f_\mu(x)\ge h_2$, $\ol{f_{\pi}^{\le 2R}}(x) = h_2$; for those $x$ with $f_\mu(x)\in [h_1,h_2]$, the value of $\ol{f_{\pi}^{\le 2R}(x)}$ is smoothly interpolated between $h_1$ and $h_2$. The function $f_\pi^{\le 2R}$ is illustrated in \Cref{fig:ub}. %\ctodo{A figure here.}

Finally, we \emph{approximately normalize} $\exp\tp{-\ol{f_{\pi}^{\le 2R}}(x)}$ into a ``probability'' by dividing our estimate $\wh Z_\mu$, namely that for every $x\in \bb R^d$, let
\[
    f^{\le 2R}_\pi(x) \defeq \ol{f^{\le 2R}_\pi}(x) + \log \wh Z_{\mu}.
\]


\subsection{Properties of $\pi$}\label{sec:properties-of-pi}

In this section we prove some useful properties of the distribution $\pi$ just constructed. We begin with three useful technical lemmas in \Cref{sec:ub-tech}. Then we prove key properties of $\pi$, including showing its closeness to $\mu$ in terms of total variation distance in \Cref{sec:ub-closeness}, bounding its \Poincare constant in \Cref{sec:ub-poincare} and analyzing its smoothness in \Cref{sec:ub-smooth}.

\subsubsection{Technical lemmas}\label{sec:ub-tech}

Recall that $p_\mu(x)\propto \exp\tp{-f_\mu(x)}$ is the density of the $L$-log-smooth distribution $\mu$ and $Z_\mu = \int_{\bb R^d} p_\mu(x)\d x$ is the normalizing factor. The first lemma says that $p_\mu(x)$ has an upper bound since $\mu$ is $L$-log-smooth. 

\begin{lemma}\label{lem:mu-bound}
    $\forall x\in \bb R^d,\;p_\mu(x) \le \tp{\frac{2\pi}{L}}^{-\frac{d}{2}}$.
\end{lemma}
\begin{proof}
    Let $x^* = \argmax_{x\in\bb R^d} \exp\tp{-f_\mu(x)}$. Since $f_\mu$ is $L$-smooth, for each $y\in \bb R^d$,
    \begin{align*}
        f_\mu(y)
        &\le f_\mu(x^*) + \grad f_\mu(x^*)\top (y-x^*) + \frac{L}{2}\norm{y-x^*}^2\\
        \mr{$\grad f_\mu(x^*)=0$}
        &=f_\mu(x^*)+ \frac{L}{2}\norm{y-x^*}^2.
    \end{align*}
    On the other hand, 
    \begin{align*}
        1
        &=Z_\mu^{-1} \int_{\bb R^d} \exp\tp{-f_\mu(y)} \d y\\
        &\ge Z_{\mu}^{-1}\int_{\bb R^d} \exp\tp{-f_\mu(x^*)-\frac{L}{2}\norm{y-x^*}^2} \d y\\
        &= p_\mu(x^*)\cdot\int_{\bb R^d} \exp\tp{-\frac{L}{2}\norm{y-x^*}^2}\d y.
    \end{align*}
    Since $\int_{\bb R^d} \exp\tp{-\frac{L}{2}\norm{y-x^*}^2}\d y = \tp{\frac{2\pi}{L}}^{\frac{d}{2}}$, we conclude the proof. 
\end{proof}

The second lemma shows that the function $\ol{f^{\le 2R}_{\pi}}$, our truncation for $f_\mu$, does not change the mass in the $\+O(R)$-ball much. 

\begin{lemma}\label{lem:Z_R-close-to-one}
    Assume $d\geq 3$. The following holds.
    \[
        Z_{\mu}^{-1}\int_{\+B_{2R}} \exp\tp{-\ol{f^{\le 2R}_{\pi}}(x)}\d x \le 1+\frac{\eps}{32}, \mbox{ and } Z_{\mu}^{-1}\int_{\+B_R} \exp\tp{-\ol{f^{\le 2R}_{\pi}}(x)}\d x \ge 1-\frac{\eps}{16}.
    \]
\end{lemma}
\begin{proof}
    Let $\+L =\set{x\in \+B_{2R}\cmid f_\mu(x)\ge h_1}$ be the set of points in $\+B_{2R}$ where the truncation occurs. Clearly
    \begin{align*}
        Z_{\mu}^{-1}\int_{\+B_{2R}} \exp\tp{-\ol{f^{\le 2R}_\pi}(x)}\d x
        &\leq Z_{\mu}^{-1}\int_{\+B_{2R}\setminus \+L} \exp\tp{-f_\mu(x)}\d x + Z_{\mu}^{-1}\int_{\+L} \exp\tp{-h_1}\d x\\
        &\le Z_{\mu}^{-1}\int_{\bb R^d} \exp\tp{-f_\mu(x)}\d x + Z_\mu^{-1}\cdot \!{vol}(\+B_{2R})\cdot \exp(-h_1)\\
        \mr{by \Cref{lem:mu-bound}, $Z_\mu^{-1}\exp\tp{-f^*} \le \tp{\frac{2\pi}{L}}^{-\frac{d}{2}}$}
        &\le 1+\frac{\eps}{4}\cdot (2\pi)^{-\frac{d}{2}}\\
        \mr{$d\geq 3$} 
        &\le 1+\frac{\eps}{32}.
    \end{align*}
    For the lower bound, we can calculate that
    \begin{align*}
        Z_\mu^{-1}\int_{\+B_R} \exp\tp{-\ol{f^{\le 2R}_\pi}(x)}\d x
        &\ge Z_\mu^{-1}\int_{\+B_R\setminus \+L} \exp\tp{-\ol{f^{\le 2R}_\pi}(x)}\d x\\
        \mr{$\ol{f^{\le 2R}_\pi}(x) = f_\mu(x)$ for $x\in\+B_{2R}\setminus\+L$} 
        &=Z_\mu^{-1}\int_{\+B_R\setminus\+L} \exp\tp{-f_\mu(x)} \d x\\
        &=Z_\mu^{-1}\int_{\+B_R} \exp\tp{-f_\mu(x)} \d x - Z_\mu^{-1}\int_{\+L} \exp\tp{-f_\mu(x)} \d x.
    \end{align*}
    By Markov's inequality,
    \begin{equation}\label{eqn:1st}
        Z_\mu^{-1}\int_{\+B_R} \exp\tp{-f_\mu(x)} \d x = \Pr[X\sim\mu]{X\in \+B_R}\ge 1-\frac{M}{R^2} = 1-\frac{\eps}{32}.
    \end{equation}
    By our definition of $h_1$, 
    \begin{equation}\label{eqn:2nd}
        Z_\mu^{-1}\int_{\+L} \exp\tp{-h_1}\le \!{vol}(\+B_{2R})\cdot Z_\mu^{-1}\exp\tp{-h_1}\le \frac{\eps}{4}\cdot (2\pi)^{-\frac{d}{2}}\le \frac{\eps}{32}.
    \end{equation}
    Combining~\eqref{eqn:1st} and~\eqref{eqn:2nd} finishes the proof.
\end{proof}

Recall that our definition for $f^{\le 2R}_{\pi}$ is an \emph{approximately normalized} $\ol{f^{\le 2R}_{\pi}}$ using our estimate $\wh Z_\mu$ for $Z_\mu$. The following lemma states that provided the estimate is accurate enough, $Z_\pi$ is close to $1$.

\begin{lemma}\label{lem:Zpi-close-to-one}
    Assume $d\geq 3$. It holds that
    \begin{itemize}
        \item $1-\frac{\eps}{16} \le Z_\pi \cdot \frac{\wh Z_\mu}{Z_\mu} \le 1+\frac{\eps}{16}$.
        \item $\frac12\le Z_\pi\le 4e^d$.
    \end{itemize}
\end{lemma}
\begin{proof}
    On the one hand, from \Cref{lem:Z_R-close-to-one},
    \[
        Z_{\pi} \geq \int_{\+B_R} \exp\tp{-f_\pi(x)} \d x = \frac{Z_\mu}{\wh Z_{\mu}} \cdot Z_\mu^{-1}\int_{\+B_R} \exp\tp{-\ol{f^{\le 2R}_\pi}(x)} \d x \geq \frac{Z_\mu}{\wh Z_{\mu}} \cdot \tp{1 - \frac{\eps}{16}}.
    \]
    On the other hand, 
    \begin{align*}
        Z_{\pi} &\leq \int_{\bb R^d\setminus \+B_R} e^{-f_\gamma(x)+\log\eps} \d x + \int_{\+B_{2R}} \exp\tp{-f^{\le 2R}_\pi(x)} \d x\\
        &\le \eps\Pr[X\sim \+N(0,\frac{M}{d\eps}\cdot \!{Id}_d)]{ \|X\|^2 \geq R^2} + \frac{Z_\mu}{\wh Z_{\mu}} \cdot Z_\mu^{-1}\int_{\+B_{2R}} e^{-\ol{f^{\le 2R}_\pi}(x)} \d x \\
        \mr{\Cref{lem:Z_R-close-to-one}}
        &\leq \frac{\eps}{32} + \frac{Z_\mu}{\wh Z_{\mu}} \cdot \tp{1+\frac{\eps}{32}}.
    \end{align*}
    The lemma then follows from \Cref{prop:Z-and-fmin}.
\end{proof} 

\subsubsection{Distance between $\pi$ and $\mu$}\label{sec:ub-closeness}

We now prove that the total variation distance between $\pi$ and $\mu$ is at most $\frac{\eps}{2}$. 

\begin{lemma} \label{lem:pi-mu-close}
    Assume $d\geq 3$. We have $\DTV(\pi,\mu)\le \frac{\eps}{2}$.
\end{lemma}
\begin{proof}
    We still let $\+L = \set{x\in \+B_{2R}\cmid f_{\mu}(x)\ge h_1}$ denote those points that have been truncated. Clearly
    \[
        \DTV(\pi,\mu) = \frac{1}{2}\Big(\underbrace{\int_{\bb R^d\setminus \+B_R}  \abs{p_\pi(x)-p_\mu(x)} \d x}_{\mbox{(a)}} + \underbrace{\int_{ \+B_R\setminus \+L }  \abs{p_\pi(x)-p_\mu(x)} \d x}_{\mbox{(b)}} + \underbrace{\int_{\+L\cap \+B_R}  \abs{p_\pi(x)-p_\mu(x)} \d x}_{\mbox{(c)}}\Big).
    \]
    We then bound terms (a), (b) and (c) respectively. For (a), we have
    \begin{align*}
        \mbox{(a)}
        &\le \int_{\bb R^d\setminus \+B_R} p_\mu(x) \d x + \int_{\bb R^d\setminus \+B_R} p_\pi(x)\d x\\
        &= \Pr[X\sim\mu]{\norm{X}^2>R^2} +\tp{1-Z_\pi^{-1} \int_{\+B_R} \exp\tp{-f^{\le 2R}_\pi(x)} \d x}\\
        \mr{\eqref{eqn:markov-mu} and definition of $f^{\le 2R}_\pi$}
        &\le \frac{\eps}{32} + 1-Z_{\pi}^{-1} \cdot \frac{Z_\mu}{\wh Z_\mu}\cdot Z_\mu^{-1}\cdot \int_{\+B_R} \exp\tp{-\ol{f^{\le 2R}_\pi}(x)} \d x\\
        \mr{\Cref{lem:Z_R-close-to-one}}
        &\le\frac{\eps}{32} + 1-\frac{Z_\mu}{\wh Z_\mu}\cdot Z_\pi^{-1}\tp{1-\frac{\eps}{16}}\\
        \mr{\Cref{lem:Zpi-close-to-one}}
        &\le \frac{\eps}{32}+\frac{\eps}{8} = \frac{5\eps}{32}.
    \end{align*}
    By our construction, for $x\in \+B_R\setminus\+L$, we have that $f_\pi(x) = f_{\mu}(x)+\log\wh Z_\mu$. Therefore, for the term (b), we have
    \begin{align*}
        \mbox{(b)}
        &=\int_{\+B_R\setminus\+L}\abs{Z_\mu^{-1}\exp\tp{-f_{\mu}(x)}-Z_\pi^{-1}\cdot \wh Z_\mu^{-1}\exp\tp{-f_{\mu}(x)}} \d x\\
        &=\abs{1-\frac{Z_\mu}{\wh Z_\mu}\cdot Z_\pi^{-1}} \cdot Z_\mu^{-1}\int_{\+B_R\setminus\+L} \exp\tp{-f_{\mu}(x)} \d x\\
        &\le \abs{1-\frac{Z_\mu}{\wh Z_\mu}\cdot Z_\pi^{-1}}\\
        \mr{\Cref{lem:Zpi-close-to-one}}
        &\le \frac{\eps}{8}.
    \end{align*}
    Finally, for the term (c), we have
    \begin{align*}
        \mbox{(c)}
        &\le \int_{\+L\cap \+B_R} p_\mu(x)\d x + \int_{\+L\cap \+B_R} p_\pi(x) \d x\\
        &\le Z_\mu^{-1}\int_{\+B_R} \exp\tp{-h_1}\d x + Z_\pi^{-1}\int_{\+L\cap \+B_R} \exp\tp{-h_1-\log \wh Z_\mu}\d x\\
        &\le \tp{1+\frac{Z_\mu}{\wh Z_\mu\cdot Z_\pi}}\cdot Z_\mu^{-1}\int_{\+B_R} \exp\tp{-h_1}\d x\\
        \mr{\Cref{lem:Zpi-close-to-one}}
        &\le 3\cdot\!{vol}(\+B_R)\cdot Z_\mu^{-1} \exp\tp{-h_1}\\
        \mr{Definition of $h_1$ and \Cref{lem:mu-bound}}
        &\le\frac{3\eps}{4}\cdot\tp{2\pi}^{-\frac{d}{2}} \le \frac{3\eps}{32}.
    \end{align*}
    In total, we have $\DTV(\pi,\mu) \le \frac{5\eps}{32}+\frac{\eps}{8}+\frac{3\eps}{32}<\frac{\eps}{2}$.
\end{proof}

\subsubsection{The \Poincare constant of $\pi$}\label{sec:ub-poincare}

In this section we bound the \Poincare constant of $\pi$. Recall that the density $p_\gamma$ of $\gamma\sim \+N\tp{0,\frac{M}{\eps d}\!{Id}_d}$ is $p_\gamma(x) = \exp\tp{-f_\gamma(x)}$ where $f_\gamma(x) = \frac{\eps d\norm{x}^2}{2M}+\frac{d}{2}\log\frac{2\pi M}{d\eps}$.  We will show that $p_\pi$ is close to $p_\gamma$ pointwise. 

\begin{lemma}\label{lem:fclose}
    Assume $d\geq 3$. For every $x\in \bb R^d$, $\abs{f_\gamma(x)-f_\pi(x)} \le \+O\tp{d\log\frac{LM}{d\eps}}$.
\end{lemma}
\begin{proof}
    Outside $\+B_{2R}$, we have
    \[
        \abs{f_\pi(x) - f_\gamma(x)} = \log\frac{1}{\eps} = \+O\tp{d\log\frac{LM}{d\eps}}.
    \]
    For $x\in \+B_{2R}$, or equivalently $\norm{x}^2\le \frac{128M}{\eps}$, 
    % $\abs{f_\gamma(x)} = \+O\tp{d\log\frac{LM}{d\eps}}$. 
    \[
        \frac{d}{2}\log\frac{2\pi M}{d\eps} \leq f_\gamma(x) \leq 64d + \frac{d}{2}\log\frac{2\pi M}{d\eps}.
    \]
    
    % It remains to show that $\abs{f_\pi(x)} = \+O\tp{d\log\frac{LM}{d\eps}}$ as well.
    It remains to bound $f_\pi(x)$ inside $\+B_{2R}$. Note that $f_\pi(x) = f^{\le 2R}_\pi(x)$ for $x\in \+B_R$ and $f_\pi(x)$ is an interpolation of $f^{\le 2R}_\pi(x)$ and $f_\gamma(x)-\log\eps$ for $x\in \+B_{2R}\setminus \+B_R$, we only need to bound $f^{\le 2R}_{\pi}$.
    % we only need to verify that $f^{\le 2R}_{\pi} = \+O\tp{d\log\frac{LM}{d\eps}}$ for $x\in \+B_{2R}$. 

    Recall that ${f^{\le 2R}_{\pi}(x)} = {\ol{f^{\le 2R}_{\pi}}(x)+\log \wh Z_\mu}= {\ol{f^{\le 2R}_{\pi}}(x)+\log Z_\mu}+{\log\frac{\wh Z_\mu}{Z_\mu}}$. By \Cref{prop:Z-and-fmin}, $-d -1 \leq \log\frac{\wh Z_\mu}{Z_\mu} \leq 0$.
    % we only need to bound $\abs{\ol{f^{\le 2R}_{\pi}}(x)+\log Z_\mu}$.
    By our construction, for all $x\in \+B_{2R}$, 
    \[
        f^*+\log Z_{\mu} \leq \ol{f^{\le 2R}_{\pi}}(x)+\log Z_{\mu} \leq h_2 + \log Z_{\mu}.
    \]
    From \Cref{lem:mu-bound}, $f^*+\log Z_{\mu} \geq \frac{d}{2}\log\frac{2\pi}{L}$. On the other hand, $h_2 + \log Z_{\mu} \leq f^*+\log Z_{\mu} + d + \log \!{vol}(\+B_{2R}) + \frac{d}{2}\log\frac{L^2M}{d\eps} + \log \frac{4}{\eps}$. Since
    \[
        \!{vol}(\+B_R)\cdot e^{-f^* - \log Z_{\mu}} = \int_{\+B_R} e^{-f^* - \log Z_{\mu}} \dd x \geq \Pr[X\sim \mu]{X\in \+B_R} \geq 1-\frac{\eps}{32}, 
    \]
    we have $f^*+\log Z_{\mu} \leq \log \!{vol}(\+B_R) + 1$. Therefore, 
    \begin{align*}
        \ol{f^{\le 2R}_{\pi}}(x)+\log Z_{\mu} \leq h_2 + \log Z_{\mu} &\leq \log \!{vol}(\+B_R) + \log \!{vol}(\+B_{2R}) + d+1 + \frac{d}{2}\log\frac{L^2M}{d\eps} + \log \frac{4}{\eps} \\
        \mr{\Cref{prop:Gamma}}
        &\leq \log \frac{4}{\eps} + d+1 + \frac{d}{2}\log \frac{ L^2 M}{d\eps} + \frac{d}{2}\log \frac{64e\pi M}{d\eps} + \frac{d}{2}\log \frac{4\cdot 64e\pi M}{d\eps} \\
        & \leq \log \frac{4}{\eps} + d+ 1 + d\log \frac{8 L M}{d\eps} + \frac{d}{2}\log \frac{4\cdot 64 e^2\pi^2 M}{d\eps}.
        % & = \+O\tp{d\log\frac{LM}{d\eps}}.
    \end{align*}

    Combining the above calculations, for $x\in \+B_{2R}$,
    \begin{align*}
        f_\pi(x) - f_\gamma(x) &\leq f^{\le 2R}_{\pi}(x) - f_\gamma(x) + \log \frac{1}{\eps}\\
        &\leq \ol{f^{\le 2R}_{\pi}}(x)+\log Z_{\mu} - f_\gamma(x) + \log\frac{\wh Z_\mu}{Z_\mu} + \log \frac{1}{\eps}\\
        &\leq \log \frac{4}{\eps} + d+ 1 + d\log \frac{8 L M}{d\eps} + \frac{d}{2}\log \frac{4\cdot 64 e^2\pi^2 M}{d\eps} - \frac{d}{2}\log\frac{2\pi M}{d\eps} + \log \frac{1}{\eps}\\
        &= \+O\tp{d\log \frac{LM}{d\eps}}
    \end{align*}
    and 
    \begin{align*}
        f_\pi(x) - f_\gamma(x) &\geq f^{\le 2R}_{\pi}(x) - f_\gamma(x) \\
        &= \ol{f^{\le 2R}_{\pi}}(x)+\log Z_{\mu} - f_\gamma(x) + \log\frac{\wh Z_\mu}{Z_\mu} \\
        &\geq \frac{d}{2}\log\frac{2\pi}{L} - 64d - \frac{d}{2}\log\frac{2\pi M}{d\eps} - d - 1\\
        &= - \+O\tp{d\log \frac{LM}{d\eps}}.
    \end{align*}
\end{proof}

Since $\abs{\log p_\pi(x) - \log p_\gamma(x)} = \abs{f_\gamma(x)-f_\pi(x)-\log Z_\pi} \le \abs{f_\gamma(x)-f_\pi(x)}+\abs{\log Z_\pi}$, by \Cref{lem:Zpi-close-to-one}, we have the following corollary.

\begin{corollary}\label{cor:pclose}
    For every $x\in\bb R^d$, $\tp{\frac{LM}{d\eps}}^{-\+O\tp{d}}\le \frac{p_\pi(x)}{p_\gamma(x)} \le \tp{\frac{LM}{d\eps}}^{\+O\tp{d}}$.
\end{corollary}

Then we come to the bound for the \Poincare constant of $\pi$. 

\begin{lemma}\label{lem:pi-PI}
    $C_{\!{PI}}(\pi) = \frac{2d\eps}{M}\cdot \tp{\frac{LM}{d\eps}}^{-\+O(d)}$.
\end{lemma}

\begin{proof}
    By definition, $C_{\!{PI}}(\pi) = \inf_{h\colon\bb R^d \to \bb R} \frac{\E[\pi]{\|\grad h\|^2}}{\Var[\pi]{h^2}}$. For each $h\colon\bb R^d \to \bb R$,
    \begin{align*}
        \frac{\E[\pi]{\|\grad h\|^2}}{\Var[\pi]{h}} 
        & = \frac{2\int_{\bb R^d} \|\grad h(x)\|^2 p_\pi(x) \dd x}{\int_{\bb R^d\times \bb R^d} \tp{h(x)-h(y)}^2 p_\pi(x)p_\pi(y) \dd x \dd y} \\
        \mr{\Cref{cor:pclose}}
        &\geq \tp{\frac{LM}{d\eps}}^{\+O(d)}\cdot \frac{2\int_{\bb R^d} \|\grad h(x)\|^2 p_\gamma(x) \dd x}{\int_{\bb R^d\times \bb R^d} \tp{h(x)-h(y)}^2 p_\gamma(x)p_\gamma(y) \dd x \dd y}\\
        &= \tp{\frac{LM}{d\eps}}^{\+O(d)}\cdot \frac{\E[\gamma]{\|\grad h\|^2}}{\Var[\gamma]{h}}.
    \end{align*}
    Since $C_{\!{PI}}(\gamma) = \inf_{h\colon\bb R^d \to \bb R} \frac{\E[\gamma]{\|\grad h\|^2}}{\Var[\gamma]{h}} = \frac{2d\eps}{M}$ \cite{HE76}, we know that $C_{\!{PI}}(\pi) \geq \frac{2d\eps}{M}\cdot \tp{\frac{LM}{d\eps}}^{\+O(d)}$.

\end{proof}

\subsubsection{The smoothness and first moment of $\pi$} \label{sec:ub-smooth}

In this section, we prove the smoothness property and bound the first moment of $\pi$. These properties are important in the algorithm to sample from $\pi$ in \Cref{sec:proof-of-ub}. Remember that we assumed $\grad f_\mu(0) = 0$. 

\begin{lemma}\label{lem:smooth1}
    % We have $\grad \ol{f^{\le 2R}_\pi}(0)=0$ and for each $x\in \+B_{2R}$, $\| \grad^2 \ol{f^{\le 2R}_\pi}(x) \| =  \+O\tp{\frac{L^3R^4}{\tp{h_2-h_1}^2}}$ and $\| \grad \ol{f^{\le 2R}_\pi}(x) \| = \+O\tp{\frac{L^2R^3}{h_2-h_1}}$.  
    We have $\grad \ol{f^{\le 2R}_\pi}(0)=0$ and for any $x,y \in \+B_{2R}$, $\| \grad \ol{f^{\le 2R}_\pi}(x) \| = \+O\tp{\frac{L^2R^3}{h_2-h_1}}$ and $\| \grad \ol{f^{\le 2R}_\pi}(x) - \grad \ol{f^{\le 2R}_\pi}(y) \| =  \+O\tp{\frac{L^3R^4}{\tp{h_2-h_1}^2}}\cdot \|x-y\|$.  
\end{lemma}
\begin{proof}
        By the definition of $\ol{f^{\le 2R}_\pi}$, for each $x,y\in \+B_{2R}$, direct calculation gives
        \begin{align*}
            \grad \ol{f^{\le 2R}_\pi}(x) &= \grad \mathfrak{g}_{[h_1,h_2]}(x) 
            \cdot \tp{f_\mu(x) - h_2} + \grad f_\mu(x) \cdot \mathfrak{g}_{[h_1,h_2]}(x)
            % ,\\
            % \grad^2 \ol{f^{\le 2R}_\pi}(x) &= \grad^2 \mathfrak{g}_{[h_1,h_2]}(x)\cdot \tp{f_\mu(x) - h_2} + \mathfrak{g}_{[h_1,h_2]}(x)\cdot \grad^2 f_\mu(x)\\
            % &\quad\quad +\grad \mathfrak{g}_{[h_1,h_2]}(x) \grad f_\mu(x)^{\top} + \grad f_\mu(x)\cdot \grad \mathfrak{g}_{[h_1,h_2]}(x)^{\top}
        \end{align*}
        and 
        \begin{align}
            \grad \ol{f^{\le 2R}_\pi}(x) - \grad \ol{f^{\le 2R}_\pi}(y) &= \tp{\grad \mathfrak{g}_{[h_1,h_2]}(x) - \grad \mathfrak{g}_{[h_1,h_2]}(y) }\cdot \tp{f_\mu(x) - h_2} + \grad \mathfrak{g}_{[h_1,h_2]}(y) \cdot (f_\mu(x)-f_{\mu}(y)) \notag \\
            &\quad\quad +\mathfrak{g}_{[h_1,h_2]}(x) \tp{\grad f_\mu(x) - \grad f_\mu(y)} + \grad f_\mu(y)\cdot \tp{\mathfrak{g}_{[h_1,h_2]}(x) - \mathfrak{g}_{[h_1,h_2]}(y)}. \notag
        \end{align}
        By the definition of $\mathfrak{g}_{[h_1,h_2]}$, we have
        \[
            \grad \mathfrak{g}_{[h_1,h_2]}(x) = \frac{-\grad f_\mu(x)}{h_2-h_1} \cdot q_{\!{mol}}'\tp{\frac{ h_2 - f_\mu(x) }{ h_2 - h_1 }}.
        \]
        % and
        % \[
        %     \grad^2 \mathfrak{g}_{[h_1,h_2]}(x) = \frac{\grad f_\mu(x)\cdot \grad f_\mu(x)^{\top}}{(h_2-h_1)^2} \cdot q_{\!{mol}}''\tp{\frac{ h_2 - f_\mu(x) }{ h_2 - h_1 }} - \frac{\grad^2 f_{\mu}(x)}{h_2-h_1} \cdot q_{\!{mol}}'\tp{\frac{ h_2 - f_\mu(x) }{ h_2 - h_1 }}.
        % \]
        It is easy to see $\grad \ol{f^{\le 2R}_\pi}(0)=0$. Since $f$ is $L$-smooth and $\grad f_\mu(0)=0$, for $x\in \+B_{2R}$, $\|\grad f_\mu(x) \| \leq L\|x\| \leq 2LR$ and $\|\grad f_\mu(x) - \grad f_\mu(y) \|\leq L\|x-y\|$. Recall that $q'_{\!{mol}}$ is always $O(1)$. We have $\|\grad \mathfrak{g}_{[h_1,h_2]}(x)\| =\+O\tp{\frac{LR}{h_2-h_1}}$ and 
        \begin{align*}
            \|\grad \mathfrak{g}_{[h_1,h_2]}(x) - \grad \mathfrak{g}_{[h_1,h_2]}(y) \| &\leq \norm{\frac{\grad f_\mu(x)-\grad f_\mu(y)}{h_2-h_1}}\cdot q_{\!{mol}}'\tp{\frac{ h_2 - f_\mu(x) }{ h_2 - h_1 }} \\
            &\quad + \frac{\|\grad f_{\mu}(y)\|}{h_2-h_1} \cdot \abs{q_{\!{mol}}'\tp{\frac{ h_2 - f_\mu(x) }{ h_2 - h_1 }} - q_{\!{mol}}'\tp{\frac{ h_2 - f_\mu(y) }{ h_2 - h_1 }}}\\
            &\leq \+O\tp{\frac{L}{h_2-h_1}}\cdot \|x-y\| + \+O\tp{\frac{LR}{h_2-h_1}\cdot \frac{LR}{h_2-h_1}}\cdot \|x-y\|\\
            &= \+O\tp{\frac{L^2R^2}{(h_2-h_1)^2}}\cdot \|x-y\|.
        \end{align*}
        Consequently, $\abs{f_\mu(x)-f_\mu(y)} \leq \+O(LR)\|x-y\|$ and $\abs{ \mathfrak{g}_{[h_1,h_2]}(x) - \mathfrak{g}_{[h_1,h_2]}(y) } \leq \+O\tp{\frac{LR}{h_2-h_1}}\cdot \|x-y\|$.
        
         Let $x^* = \arg\min_{x\in \+B_{2R}} f_\mu(x)$. We have for any $x\in \+B_{2R}$
         \begin{equation}\label{eq:x*}
             f^*\leq f_\mu(x) \leq f_\mu(x^*) + \grad f_\mu(x^*)\cdot (x-x^*) + \frac{L}{2}\cdot \|x-x^*\|^2 \leq f^* + 16LR^2. 
         \end{equation}
        Recall the definition of $h_2 =  \wh f^* + \log \!{vol}(\+B_{2R}) + \frac{d}{2}\log L + \log \frac{4}{\eps} + \frac{d}{2}\log\frac{LM}{d\eps}$. According to \Cref{prop:Z-and-fmin} and~\eqref{eq:x*}, for $x\in \+B_{2R}$, 
        \[
             f_\mu(x)-h_2\geq - \frac{d}{2}\log \frac{LM}{d\eps} - \log\frac{4}{\eps} + \log\Gamma\tp{\frac{d}{2}+1} - \frac{d}{2}\log \frac{4\cdot 32\pi LM}{\eps} - d,
        \]
        and
        \[
             f_\mu(x)-h_2\leq - \frac{d}{2}\log \frac{LM}{d\eps} - \log\frac{4}{\eps} + \log\Gamma\tp{\frac{d}{2}+1} - \frac{d}{2}\log \frac{4\cdot 32\pi LM}{\eps} + 16LR^2.
        \]    
        Therefore, $\abs{f_\mu(x)-h_2}= \+O(LR^2)$.

        Combining all above, we have
        \[
             \| \grad \ol{f^{\le 2R}_\pi}(x) \| \leq 2LR \cdot \+O\tp{\frac{\abs{f_\mu(x)-h_2}}{h_2-h_1}}  + 2LR = \+O\tp{\frac{L^2R^3}{h_2-h_1}},
        \]
        and 
        \begin{align*}
            \|\grad \ol{f^{\le 2R}_\pi}(x) - \grad \ol{f^{\le 2R}_\pi}(y)\| &\leq \+O\tp{\frac{L^3R^4}{\tp{h_2-h_1}^2} + \frac{L^2R^2}{h_2-h_1} + L } = \+O\tp{\frac{L^3R^4}{\tp{h_2-h_1}^2}}\cdot \|x-y\|.
        \end{align*}
\end{proof}
% \begin{proof}
%         By the definition of $\ol{f^{\le 2R}_\pi}$, for each $x,y\in \+B_{2R}$, direct calculation gives
%         \begin{align*}
%             \grad \ol{f^{\le 2R}_\pi}(x) &= \grad \mathfrak{g}_{[h_1,h_2]}(x) 
%             \cdot \tp{f_\mu(x) - h_2} + \grad f_\mu(x) \cdot \mathfrak{g}_{[h_1,h_2]}(x)
%             % ,\\
%             % \grad^2 \ol{f^{\le 2R}_\pi}(x) &= \grad^2 \mathfrak{g}_{[h_1,h_2]}(x)\cdot \tp{f_\mu(x) - h_2} + \mathfrak{g}_{[h_1,h_2]}(x)\cdot \grad^2 f_\mu(x)\\
%             % &\quad\quad +\grad \mathfrak{g}_{[h_1,h_2]}(x) \grad f_\mu(x)^{\top} + \grad f_\mu(x)\cdot \grad \mathfrak{g}_{[h_1,h_2]}(x)^{\top}
%         \end{align*}
%         and 
%         \begin{align*}
%             \grad \ol{f^{\le 2R}_\pi}(x) - \grad \ol{f^{\le 2R}_\pi}(y) &= \tp{\grad \mathfrak{g}_{[h_1,h_2]}(x) - \grad \mathfrak{g}_{[h_1,h_2]}(y) }\cdot \tp{f_\mu(x) - h_2} + \mathfrak{g}_{[h_1,h_2]}(x)\cdot \grad^2 f_\mu(x)\\
%             &\quad\quad +\grad \mathfrak{g}_{[h_1,h_2]}(x) \grad f_\mu(x)^{\top} + \grad f_\mu(x)\cdot \grad \mathfrak{g}_{[h_1,h_2]}(x)^{\top}
%         \end{align*}
%         By the definition of $\mathfrak{g}_{[h_1,h_2]}$, we have
%         \[
%             \grad \mathfrak{g}_{[h_1,h_2]}(x) = \frac{-\grad f_\mu(x)}{h_2-h_1} \cdot q_{\!{mol}}'\tp{\frac{ h_2 - f_\mu(x) }{ h_2 - h_1 }},
%         \]
%         and
%         \[
%             \grad^2 \mathfrak{g}_{[h_1,h_2]}(x) = \frac{\grad f_\mu(x)\cdot \grad f_\mu(x)^{\top}}{(h_2-h_1)^2} \cdot q_{\!{mol}}''\tp{\frac{ h_2 - f_\mu(x) }{ h_2 - h_1 }} - \frac{\grad^2 f_{\mu}(x)}{h_2-h_1} \cdot q_{\!{mol}}'\tp{\frac{ h_2 - f_\mu(x) }{ h_2 - h_1 }}.
%         \]
%         It is easy to see $\grad \ol{f^{\le 2R}_\pi}(0)=0$. Since $f$ is $L$-smooth and $\grad f_\mu(0)=0$, for $x\in \+B_{2R}$, $\|\grad f_\mu(x) \| \leq L\|x\| \leq 2LR$ and $\|\grad^2 f_\mu(x) \|\leq L$. 
        
%          Let $x^* = \arg\min_{x\in \+B_{2R}} f_\mu(x)$. We have for any $x\in \+B_{2R}$
%          \begin{equation}\label{eq:x*}
%              f^*\leq f_\mu(x) \leq f_\mu(x^*) + \grad f_\mu(x^*)\cdot (x-x^*) + \frac{L}{2}\cdot \|x-x^*\|^2 \leq f^* + 16LR^2. 
%          \end{equation}
%         Recall the definition of $h_2 =  \wh f^* + \log \!{vol}(\+B_{2R}) + \frac{d}{2}\log L + \log \frac{4}{\eps} + \frac{d}{2}\log\frac{LM}{d\eps}$. According to \Cref{prop:Z-and-fmin} and~\eqref{eq:x*}, for $x\in \+B_{2R}$, 
%         \[
%              f_\mu(x)-h_2\geq - \frac{d}{2}\log \frac{LM}{d\eps} - \log\frac{4}{\eps} + \log\Gamma\tp{\frac{d}{2}+1} - \frac{d}{2}\log \frac{4\cdot 32\pi LM}{\eps} - d,
%         \]
%         and
%         \[
%              f_\mu(x)-h_2\leq - \frac{d}{2}\log \frac{LM}{d\eps} - \log\frac{4}{\eps} + \log\Gamma\tp{\frac{d}{2}+1} - \frac{d}{2}\log \frac{4\cdot 32\pi LM}{\eps} + 16LR^2.
%         \]    
%         Combining all above, we have
%         \[
%              \| \grad \ol{f^{\le 2R}_\pi}(x) \| \leq 2LR \cdot \+O\tp{\frac{\abs{f_\mu(x)-h_2}}{h_2-h_1}}  + 2LR = \+O\tp{\frac{L^2R^3}{h_2-h_1}},
%         \]
%         and 
%         \begin{align*}
%              \| \grad^2 \ol{f^{\le 2R}_\pi}(x) \| &\leq \+O\tp{\frac{(2LR)^2\cdot \abs{f_\mu(x) - h_2}}{(h_2-h_1)^2}} + \+O\tp{\frac{(LR)^2}{h_2-h_1}} = \+O\tp{\frac{L^3R^4}{\tp{h_2-h_1}^2}}.
%         \end{align*}
% \end{proof}
    

\begin{lemma}\label{lem:f2smooth}
    % The function $f_\pi$ is $\+O\tp{\frac{L^3R^4}{\tp{h_2-h_1}^2} + \frac{d\eps}{M}}$-smooth.
    The function $f_\pi$ is $\+O\tp{\frac{L^3R^4}{\tp{h_2-h_1}^2}}$-smooth.
\end{lemma}
\begin{proof}
    We divide $\bb R^d$ into three parts, $\+B_R$, $\+B_{2R}\setminus \+B_{R}$ and $\bb R^d \setminus \+B_{2R}$. Since our construct guarantees that $f_\pi$ and $\grad f_{\pi}$ are continuous functions, to prove the smoothness of $f_{\pi}$, we only need to bound $\|\grad f_{\pi}(x) - \grad f_{\pi}(y)\|$ for those $x,y$ from the same part. For $x,y$ from different parts, for example, if $x\in \+B_R$ and $y\in \+B_{2R}\setminus \+B_{R}$, we can find a $z$ at the intersection of this two parts such that $\|x-y\|=\|x-z\|+\|z-y\|$ and bounding $\|\grad f_{\pi}(x) - \grad f_{\pi}(y)\|$ can be transformed to bounding $\|\grad f_{\pi}(x) - \grad f_{\pi}(z)\|$ and $\|\grad f_{\pi}(z) - \grad f_{\pi}(y)\|$ respectively.

    By construction, for $x,y\in \+B_R$, $\norm{\grad f_\pi(x)-\grad f_\pi(y)}$ is bounded by \Cref{lem:smooth1}. For $x,y\in \bb R^d \setminus \+B_{2R}$, we know $\|\grad f_\pi(x)-\grad f_\pi(y)\|\leq \frac{\eps d}{M}\cdot \|x-y\|$ and $\frac{\eps d}{M} = \+O\tp{\frac{L^3R^4}{\tp{h_2-h_1}^2}}$. It remains to deal with those $x,y\in \+B_{2R}\setminus \+B_{R}$.

    For $x,y\in \+B_{2R}\setminus \+B_{R}$,
    \[
        \grad f_\pi(x) = - \grad f^{\le 2R}_\pi(x) \cdot \mathfrak{g}_{[R,2R]}(x) - \grad \mathfrak{g}_{[R,2R]}(x) \cdot f^{\le 2R}_\pi(x) + \grad \mathfrak{g}_{[R,2R]}(x) \cdot (f_{\gamma}(x)-\log \eps) + \grad f_{\gamma}(x) \cdot  \mathfrak{g}_{[R,2R]}(x)
    \] 
    and
    \begin{align}
        \|\grad f_\pi(x) - \grad f_\pi(y)\| &\leq  \norm{\grad f^{\le 2R}_\pi(x) - \grad f^{\le 2R}_\pi(y)}\cdot \abs{\mathfrak{g}_{[R,2R]}(x)} + \norm{\grad f^{\le 2R}_\pi(y)}\cdot \abs{\mathfrak{g}_{[R,2R]}(x) - \mathfrak{g}_{[R,2R]}(y)} \notag \\
        &\quad + \abs{f^{\le 2R}_\pi(x) - f^{\le 2R}_\pi(y)}\cdot \|\grad \mathfrak{g}_{[R,2R]}(x)\|  + \norm{\grad \mathfrak{g}_{[R,2R]}(x)}\cdot \abs{f_{\gamma}(x) - f_{\gamma}(y)} \notag  \\
        &\quad + \norm{\grad f_{\gamma}(x) - \grad f_{\gamma}(y)}\cdot \abs{\mathfrak{g}_{[R,2R]}(x)} + \norm{\grad f_{\gamma}(y)}\cdot \abs{\mathfrak{g}_{[R,2R]}(x) - \mathfrak{g}_{[R,2R]}(y)}.\notag  \\
        &\quad + \norm{\grad \mathfrak{g}_{[R,2R]}(x) - \grad \mathfrak{g}_{[R,2R]}(y)} \cdot \abs{f^{\le 2R}_\pi(y) -f_{\gamma}(y)+\log \eps} . \label{eq:grad2}
    \end{align}
    For the first term in \Cref{eq:grad2}, we know from \Cref{lem:smooth1} and the fact $\mathfrak{g}_{[R,2R]}(x)=\+O(1)$ that $\norm{\grad f^{\le 2R}_\pi(x) - \grad f^{\le 2R}_\pi(y)}\cdot \abs{\mathfrak{g}_{[R,2R]}(x)} = \+O\tp{\frac{L^3R^4}{\tp{h_2-h_1}^2}}\cdot \|x-y\|$. For the fifth term, similarly, we have $\norm{\grad f_{\gamma}(x) - \grad f_{\gamma}(y)}\cdot \abs{\mathfrak{g}_{[R,2R]}(x)} = \+O\tp{\frac{\eps d}{M}}\cdot \|x-y\|$.

    By the definition of $\mathfrak{g}_{[R,2R]}$, we have $\grad \mathfrak{g}_{[R,2R]}(x) = \frac{2x}{(2R)^2 - R^2} \cdot q'_{\!{mol}}\tp{\frac{\|x\|^2 - R^2}{(2R)^2 - R^2}} = \+O\tp{\frac{1}{R}}$ for $x\in \+B_{2R}$. Therefore, we can bound the second term in \Cref{eq:grad2} by $\norm{\grad f^{\le 2R}_\pi(y)}\cdot \abs{\mathfrak{g}_{[R,2R]}(x) - \mathfrak{g}_{[R,2R]}(y)} = \+O\tp{\frac{L^2R^2}{h_2-h_1}}\cdot \|x-y\|$ and also bound the third term by $\abs{f^{\le 2R}_\pi(x) - f^{\le 2R}_\pi(y)}\cdot \|\grad \mathfrak{g}_{[R,2R]}(x)\| = \+O\tp{\frac{L^2R^2}{h_2-h_1}}\cdot \|x-y\|$. 

    Since $\grad f_{\gamma}(x) = \frac{\eps d x}{M}$, for any $x\in \+B_{2R}$, $\norm{\grad f_{\gamma}(x)} = \+O\tp{\frac{\eps d R }{M}}= \+O\tp{\frac{d }{R}}$. Then both the fourth and the sixth term in \Cref{eq:grad2} can be bounded by $\+O\tp{\frac{d }{R^2}}\cdot \|x-y\|$.

    Still by the definition of $\mathfrak{g}_{[R,2R]}$, for $x\in \+B_{2R}$,
    \begin{align*}
        \|\grad \mathfrak{g}_{[R,2R]}(x) - \grad \mathfrak{g}_{[R,2R]}(y)\| &= \frac{2\|x-y\|}{(2R)^2 - R^2} \cdot \abs{q'_{\!{mol}}\tp{\frac{\|x\|^2 - R^2}{(2R)^2 - R^2}}} \\
        &\quad + \frac{2\|y\|}{(2R)^2 - R^2} \cdot \abs{q'_{\!{mol}}\tp{\frac{\|x\|^2 - R^2}{(2R)^2 - R^2}} - q'_{\!{mol}}\tp{\frac{\|y\|^2 - R^2}{(2R)^2 - R^2}}} \\
        &\leq \+O\tp{\frac{1}{R^2}}\cdot \|x-y\|.
    \end{align*}
    Therefore, from \Cref{lem:fclose} the last term in \Cref{eq:grad2} can be bounded by  $\+O\tp{\frac{d\log \frac{LM}{d\eps}}{R^2}}\cdot \|x-y\|$.

    % Combining the above equation and \Cref{lem:fclose}, we have $\norm{\grad^2 \mathfrak{g}_{[R,2R]}(x) \cdot \tp{f_\gamma(x) - f^{\le 2R}_\pi(x) - \log\eps}} \leq \+O\tp{\frac{d\eps}{M}\cdot \log\frac{LM}{d\eps}}$. From \Cref{lem:smooth1}, we have $\norm{\grad \mathfrak{g}_{[R,2R]}(x) \grad f^{\le 2R}_\pi(x)^\top},\norm{\grad f^{\le 2R}_\pi(x)\grad \mathfrak{g}_{[R,2R]}(x)^\top} \leq \+O\tp{\frac{L^2R^2}{h_2-h_1}}$. For $\norm{\grad \mathfrak{g}_{[R,2R]}(x) \grad f_\gamma(x)^\top}$ and $\norm{ \grad f_\gamma(x)\grad \mathfrak{g}_{[R,2R]}(x)^\top}$, they can be bounded by $\+O\tp{\frac{d\eps}{M}}$.
    In total, $\grad f_\pi(x)$ is $ \+O\tp{\frac{L^3R^4}{\tp{h_2-h_1}^2}}$-Lipschitz.
\end{proof}
% \begin{proof}
%     By construction, for $x,y\in \+B_R$, $\norm{f_\pi(x)-f_\pi(y)}$ is bounded by \Cref{lem:smooth1}. For $x\in \bb R^d \setminus \+B_{2R}$, we know $f_\pi$ is $\frac{\eps d}{M}$-smooth and $\frac{\eps d}{M} = \+O\tp{\frac{L^3R^4}{\tp{h_2-h_1}^2}}$. It remains to deal with those $x\in \+B_{2R}\setminus \+B_{R}$.

%     For $x\in \+B_{2R}\setminus \+B_{R}$, 
%     \begin{align*}
%         \grad^2 f_\pi(x) &= \grad^2 \mathfrak{g}_{[R,2R]}(x) \cdot \tp{f_\gamma(x) - f^{\le 2R}_\pi(x)-\log\eps} + \mathfrak{g}_{[R,2R]}(x) \cdot \grad^2 f_\gamma(x) - \mathfrak{g}_{[R,2R]}(x)\cdot \grad^2 f^{\le 2R}_{\pi}(x) \\
%         &\quad - \grad \mathfrak{g}_{[R,2R]}(x) \grad f^{\le 2R}_\pi(x)^\top - \grad f^{\le 2R}_\pi(x) \grad \mathfrak{g}_{[R,2R]}(x)^\top\\
%         &\quad + \grad \mathfrak{g}_{[R,2R]}(x) \grad f_\gamma(x)^{\top} + \grad f_\gamma(x) \grad\mathfrak{g}_{[R,2R]}(x)^\top\\
%         &\quad +\grad^2 f^{\le 2R}_\pi.
%     \end{align*}
%     It is easy to know that $\norm{\mathfrak{g}_{[R,2R]}(x) \cdot \grad^2 f_\gamma(x)} \leq \frac{\eps d}{M} = \+O\tp{\frac{L^3R^4}{\tp{h_2-h_1}^2}}$. From \Cref{lem:smooth1}, $\norm{\mathfrak{g}_{[R,2R]}(x)\cdot \grad^2 f^{\le 2R}_{\pi}(x)} \leq \norm{\grad^2 f^{\le 2R}_{\pi}(x)} = \+O\tp{\frac{L^3R^4}{\tp{h_2-h_1}^2}}$.

%     By the definition of $\mathfrak{g}_{[R,2R]}$, we have $\grad \mathfrak{g}_{[R,2R]}(x) = \frac{2x}{(2R)^2 - R^2} \cdot q'_{\!{mol}}\tp{\frac{\|x\|^2 - R^2}{(2R)^2 - R^2}}$ and 
%     \[
%         \grad^2 \mathfrak{g}_{[R,2R]}(x) = \frac{2\!{Id}_d}{(2R)^2 - R^2} \cdot q'_{\!{mol}}\tp{\frac{\|x\|^2 - R^2}{(2R)^2 - R^2}} + \frac{4xx^\top}{\tp{(2R)^2 - R^2}^2} \cdot q''_{\!{mol}}\tp{\frac{\|x\|^2 - R^2}{(2R)^2 - R^2}}.
%     \]
%     Combining the above equation and \Cref{lem:fclose}, we have $\norm{\grad^2 \mathfrak{g}_{[R,2R]}(x) \cdot \tp{f_\gamma(x) - f^{\le 2R}_\pi(x) - \log\eps}} \leq \+O\tp{\frac{d\eps}{M}\cdot \log\frac{LM}{d\eps}}$. From \Cref{lem:smooth1}, we have $\norm{\grad \mathfrak{g}_{[R,2R]}(x) \grad f^{\le 2R}_\pi(x)^\top},\norm{\grad f^{\le 2R}_\pi(x)\grad \mathfrak{g}_{[R,2R]}(x)^\top} \leq \+O\tp{\frac{L^2R^2}{h_2-h_1}}$. For $\norm{\grad \mathfrak{g}_{[R,2R]}(x) \grad f_\gamma(x)^\top}$ and $\norm{ \grad f_\gamma(x)\grad \mathfrak{g}_{[R,2R]}(x)^\top}$, they can be bounded by $\+O\tp{\frac{d\eps}{M}}$.
%     In total, we have $\norm{\grad^2 f_\pi(x)} \leq \+O\tp{\frac{L^3R^4}{\tp{h_2-h_1}^2}}$.
% \end{proof}

The following result is a corollary of the above lemmas.
\begin{corollary}\label{coro:gap}
    The function $f_{\pi}$ satisfies $f_{\pi}(0) - \min_{x\in \bb R^d} f_{\pi}(x) = \+O\tp{\frac{L^3R^6}{\tp{h_2-h_1}^2}}$.
\end{corollary}
\begin{proof}
    By the definition of $f_\pi$, it is increasing as $\|x\|$ increases outside $\+B_{2R}$. Therefore, $f_{\pi}(0) - \min_{x\in \bb R^d} f_{\pi}(x) = f_{\pi}(0) - \min_{x\in \+B_{2R}} f_{\pi}(x)$. Since $\grad f_{\pi}(0) = \grad \ol{f^{\le 2R}_\pi}(0)=0$ and $f_{\pi}$ is $\+O\tp{\frac{L^3R^4}{\tp{h_2-h_1}^2}}$-smooth from \Cref{lem:f2smooth}, for arbitrary $x\in \+B_{2R}$, $f_{\pi}(0) - f_{\pi}(x) \leq \+O\tp{\frac{L^3R^6}{\tp{h_2-h_1}^2}}$ and consequently, $f_{\pi}(0) - \min_{x\in \bb R^d} f_{\pi}(x) = \+O\tp{\frac{L^3R^6}{\tp{h_2-h_1}^2}}$.
\end{proof}

Then we bound the first moment of $\pi$.
\begin{lemma}\label{lem:f2moment}
    The first moment of $\pi$ is bounded by $\+O(\sqrt{M})$.
\end{lemma}
\begin{proof}
    By the definition of $\pi$ and $f_\gamma$, we have
    \begin{align*}
        \E[X \sim \pi]{\|X\|}^2 &\leq \E[X \sim \pi]{\|X\|^2} \\
        & \le \frac{1}{Z_{\pi}} \tp{\int_{\bb R^d} \|x\|^2\cdot \exp\tp{-f_\gamma(x)+\log \eps} \dd x + \frac{Z_{\mu}}{\wh Z_{\mu}}\cdot \frac{1}{Z_{\mu}}\cdot \int_{\+B_{2R}} \|x\|^2\cdot \exp\tp{-\ol{f^{\le 2R}_\pi}(x)} \dd x } \\
        \mr{\Cref{lem:Zpi-close-to-one}}
        & \leq \frac{M}{Z_{\pi}} + \frac{2}{Z_{\mu}} \tp{\int_{\bb R^d} \|x\|^2\cdot \exp\tp{-f_\mu(x)} \dd x + \int_{\+B_{2R}} \|x\|^2\cdot \exp\tp{-h_1} \dd x } \\
        &\leq \frac{M}{ Z_{\pi}} + 2M + \frac{2}{Z_\mu}\cdot \int_{\+B_{2R}} (2R)^2\cdot \exp\tp{-h_1} \dd x  \\
        \mr{Definition of $h_1$} &\leq \frac{M}{Z_{\pi}} + 2M + \frac{8R^2}{Z_\mu}\cdot \exp\tp{-\wh f^* - \frac{d}{2}\log L - \log \frac{4}{\eps}} \\
        \mr{\Cref{lem:mu-bound}}
        &\leq \frac{M}{Z_{\pi}} + 2M + 2\eps R^2 \cdot (2\pi)^{-\frac{d}{2}}\\
        \mr{\Cref{lem:Zpi-close-to-one}}
        &= \+O\tp{M}.
    \end{align*}
\end{proof}

% \begin{lemma}\label{lem:f2moment}
%     The second moment of $\pi$ is bounded by $\+O(M)$.
% \end{lemma}
% \begin{proof}
%     By the definition of $\pi$ and $f_\gamma$, we have
%     \begin{align*}
%         \E[X \sim \pi]{\|X\|^2} 
%         & \le \frac{1}{Z_{\pi}} \tp{\int_{\bb R^d} \|x\|^2\cdot \exp\tp{-f_\gamma(x)+\log \eps} \dd x + \frac{Z_{\mu}}{\wh Z_{\mu}}\cdot \frac{1}{Z_{\mu}}\cdot \int_{\+B_{2R}} \|x\|^2\cdot \exp\tp{-\ol{f^{\le 2R}_\pi}(x)} \dd x } \\
%         \mr{\Cref{prop:Z-and-fmin}}
%         & \leq \frac{M}{Z_{\pi}} + \frac{2}{Z_{\mu}} \tp{\int_{\bb R^d} \|x\|^2\cdot \exp\tp{-f_\mu(x)} \dd x + \int_{\+B_{2R}} \|x\|^2\cdot \exp\tp{-h_1} \dd x } \\
%         &\leq \frac{M}{ Z_{\pi}} + 2M + \frac{2}{Z_\mu}\cdot \int_{\+B_{2R}} (2R)^2\cdot \exp\tp{-h_1} \dd x  \\
%         \mr{Definition of $h_1$} &\leq \frac{M}{Z_{\pi}} + 2M + \frac{8R^2}{Z_\mu}\cdot \exp\tp{-\wh f^* - \frac{d}{2}\log L - \log \frac{4}{\eps}} \\
%         \mr{\Cref{lem:mu-bound}}
%         &\leq \frac{M}{Z_{\pi}} + 2M + 2\eps R^2 \cdot (2\pi)^{-\frac{d}{2}}\\
%         \mr{\Cref{lem:Zpi-close-to-one}}
%         &= \+O\tp{M}.
%     \end{align*}
% \end{proof}


\subsection{Estimate $f^*$ and $Z_\mu$}\label{sec:estimate-of-pi}

In this section, we prove \Cref{prop:Z-and-fmin}, namely to show that how to get estimators $\wh f^*$ and $\wh Z_{\mu}$ satisfying
\[
    f^* \leq \wh f^* \leq f^* + d \quad \mbox{and}\quad 
    \frac12 e^{-d}\le \frac{\wh Z_{\mu}}{Z_{\mu}} \le 1.
\]
The idea is to discretize $\bb R^d$ into cubes of side length $\ell$ and use information in each cube to construct the estimation. Let $\ell = \frac{1}{64}\sqrt{\frac{d\eps}{L^2 M}}$ and $R_0 = 2R + \sqrt{d \cdot \ell^2} $. Let $\+Z_{R_0} = \+B_{R_0}\cap \ell \bb Z^d$ be the collection of vertices of the cubes in $\+B_{R_0}$.
%\set{x\in \+B_{R_0}:\ x / \ell\in \bb Z^d}$.

\begin{lemma}\label{lem:cubes}
    There are at most $\tp{\frac{2^{10}\cdot 5 LM}{d\eps}}^{d}$ cubes with side length $\ell$ in $\+B_{R_0}$ whose vertices are all in $\+Z_{R_0}$. 
\end{lemma}
\begin{proof}
    From \Cref{cor:dballvolbound}, $\!{vol}(\+B_{R_0}) = \frac{\tp{\pi R_0^2}^{\frac{d}{2}}}{\Gamma\tp{ \frac{d}{2}+1} } \leq \tp{\frac{2\pi e R_0^2}{d}}^{\frac{d}{2}}$. The volume of a cube with side length $\ell$ is $\tp{\frac{d\eps}{2^{12}\cdot L^2 M}}^{\frac{d}{2}}$. For the cubes whose vertices are all in $\+Z_{R_0}$, the overlapping area is $0$. Therefore, the total number of such cubes is no larger than 
    \begin{align*}
        \tp{2\pi e R_0^2 \cdot \frac{2^{12}\cdot L^2 M}{\eps d^2}}^{\frac{d}{2}} &= \tp{2\pi e \cdot \frac{2^{12}\cdot L^2 M}{\eps d^2}\cdot \tp{4R^2 + d\ell^2 + 4R\cdot \sqrt{d\ell^2}}}^{\frac{d}{2}}\\
        &= \tp{2\pi e \cdot \frac{2^{12}\cdot L^2 M}{\eps d^2}\cdot \tp{\frac{4\cdot 32 M}{\eps} + \frac{d^2\eps}{64^2\cdot L^2 M} + \frac{\sqrt{2}d}{4L}}}^{\frac{d}{2}}\\
        &= \tp{2\pi e \cdot \tp{\frac{2^{19}L^2 M^2}{\eps^2 d^2} + \frac{1}{4} + \frac{2^8\sqrt{2} LM}{\eps d}}}^{\frac{d}{2}}\\
        &\leq \tp{\frac{2^{10}\cdot 5 LM}{d\eps}}^{d}.
    \end{align*}
    % $\tp{2\pi e R_0^2 \cdot \frac{8\cdot 128 L^2 M}{\eps d^2}}^{\frac{d}{2}} = \tp{ 2\pi e \tp{\frac{8\cdot 128^2 M^2 L^2}{\eps^2 d^2} + 1 }}^{\frac{d}{2}} \leq \tp{\frac{2^{10}\cdot 5 LM}{d\eps}}^{d}$.
\end{proof}

We consider those cubes with side length $\ell$ and all vertices in $\+Z_{R_0}$. Let $n$ be the total number of such cubes. From \Cref{lem:cubes}, $n\leq \tp{\frac{2^{10}\cdot 5 LM}{d\eps}}^{d}$. Denote these cubes as $C_1,C_2,\dots,C_n$ and let $v_1,v_2,\dots, v_n$ be the center of these cubes. We first show that these cubes well cover the ball $\+B_{2R}$.
\begin{lemma}\label{lem:cubecover}
    For each $x\in \+B_{2R}$, there exists $i\in[n]$ such that $x\in C_i$.
    % $\|x - v_i\|\leq 2\sqrt{d}\ell$.
\end{lemma}
\begin{proof}
    % Assume in contradiction that we cannot find such a $v_i$ for some point $y\in \+B_{R'}$. 
    For each point $x\in \+B_{2R}$, we define $\ol x\in \bb R^d$ as
    $
        \forall j\in[d],\  \ol x(j) = \begin{cases}
            \lfloor \frac{x(j)}{\ell} \rfloor \cdot \ell, &\mbox{ if } x(j)\geq 0\\
            \lceil \frac{x(j)}{\ell} \rceil\cdot \ell, &\mbox{ if } x(j)< 0
        \end{cases}.
    $
    Consider the following cube
    \[
        C_x = \set{y\in \bb R^d:\ \forall j \in [d], y(j)\in \begin{cases}
            [\ol x(j), \ol x(j) + \ell], &\mbox{ if }\ol x(j)\ge 0\\
            [\ol x(j) - \ell, \ol x(j)], &\mbox{ if }\ol x(j)< 0
        \end{cases}}.
    \]
    It is obvious that $x\in C_x$ and for each vertex $y\in C_x$,
    \begin{align*}
        \|y\|^2 &\leq \sum_{j=1}^d \tp{\ol x(j)^2 + \ell^2 + 2\ell \cdot \abs{\ol x(j)}} \\
        &= \|\ol x\|^2 + d\ell^2 + 2\sqrt{d\ell^2}\cdot \frac{\sum_{j=1}^n \abs{\ol x(j)}}{\sqrt{d}}\\
        \mr{Cauchy-Schwartz inequality}&\leq \|\ol x\|^2 + d\ell^2 + 2\sqrt{d\ell^2}\cdot \|\ol x\|
        \\ 
        &\leq 4R^2 + d\ell^2 + 2\sqrt{d\ell^2}\cdot 2R
        \\ 
        &\leq R_0^2.
    \end{align*}
    Therefore $y\in \+Z_{R_0}$ and there exists $i\in [n]$ such that $C_i=C_x$.

    % We have $\ol x\in \+B_{2R} \cap \+Z_{R_0}$ and $\|\ol x - x\|^2 \leq d\ell^2$. Consider the following cube
    % \[
    %     C_x = \set{y\in \bb R^d:\ \forall j \in [d], y(j)\in \begin{cases}
    %         [\ol x(j), \ol x(j) + \ell], &\mbox{ if }\ol x(j)< 0\\
    %         [\ol x(j) - \ell, \ol x(j)], &\mbox{ if }\ol x(j)\geq 0
    %     \end{cases}}.
    % \]
    % It is obvious that there exists $i\in [n]$ such that $C_i=C_x$ and $\|x - v_i\|\leq \|x - \ol x\| + \|\ol x - v_i\| \leq 2\sqrt{d}\ell$.
\end{proof}

In the following, we will write $v_x$ for $v_i$ where $i$ is the unique $i$ such that $x\in C_i$.
%For each $x\in \+B_{2R}$, we denote $v_x$ as the vector $v_i$ where $i$ is the smallest index in $[n]$ such that $x\in C_i$. 
% For $x\in \+B_{2R} \setminus \tp{\bigcup_{j\in[n]} C_j}$, let $v_i$ be the closest point to $x$ in $\set{v_j}_{j\in[n]}$. 
% We denote the unique $v_i$ associated with each $x$ as $v_x$. 
%Let $\+J = \set{i\in[n]: \exists x\in \+B_{2R}, v_x = v_i}$. 
Let $\+J\defeq \set{i\in [n]\cmid C_i\cap \+B_{2R}\ne\emptyset}$. To estimate $f^*$ and $Z_{\mu}$, we query the value of each $f_\mu(v_i)$ and assign
\begin{equation}\label{eqn:hat}
    \wh f^* = \min_{i\in \+J} f_\mu(v_i) + \frac{d}{2},\quad \wh Z_{\mu} = \sum_{i=1}^n \!{vol}(C_i) \cdot \exp\tp{-f_\mu(v_i) - \frac{d}{2}}.
\end{equation}

%\mn{Here we add a multiplier $\frac{1}{\eps}$ to $\hat Z_{\mu}$ artificially to make $\hat Z_{\mu}$ large enough.}
Then the query complexity to determine $\wh f^*$ and $\wh Z_{\mu}$ is at most $\tp{\frac{2^{10}\cdot 5 LM}{d\eps}}^{d}$. In the following, we show that our construction satisfies the accurancy requirement, namely that



% The following lemma shows that such constructions indeed satisfy \Cref{assump:Z-and-fmin}.
 \begin{lemma}\label{lem:estimate}
     The construction of $\wh f^*$ and $\wh Z_{\mu}$ in~\eqref{eqn:hat} satisfies
     \[
    f^* \leq \wh f^* \leq f^* + d \quad \mbox{and}\quad 
    \frac12 e^{-d}\le \frac{\wh Z_{\mu}}{Z_{\mu}} \le 1.
    \]
 \end{lemma}
 \begin{proof}
    Since the function $f_\mu$ is $L$-smooth and $\grad f_\mu(0)=0$, for each $x\in \+B_{R_0}$, $\|\grad f_\mu(x)\|\leq L\|x\| \leq LR_0$. From \Cref{lem:cubecover}, we always have $\|x - v_x\| \leq \sqrt{d}\ell$. Then by the definition of $L$-smooth,
    \begin{align*}
        f_\mu(v_x) &\leq f_\mu(x) + \grad f_\mu(x)^{\top} (x-v_x) + \frac{L}{2} \|x - v_x\|^2\\
        & \leq f_\mu(x) + \|\grad f_\mu(x)\| \cdot \|x-v_x\| + \frac{L}{2} \|x - v_x\|^2 \\
        &\leq f_\mu(x) + LR_0\cdot \sqrt{d}\ell + \frac{L}{2}\cdot d\ell^2 \leq f_\mu(x) + \frac{d}{2}
    \end{align*}
    % \htodo{The universal constants here might be inaccurate.}
    and similarly
    \begin{align*}
        f_\mu(x) & \leq f_\mu(v_x) + \grad f_\mu(v_x)^{\top} (v_x - x) + \frac{L}{2} \|x - v_x\|^2 \leq f_\mu(v_x) + \frac{d}{2}.
    \end{align*}
    Therefore, $f^* = \min_{x\in \+B_{2R}} f_\mu(x) \leq \min_{x\in \+B_{2R}} f_\mu(v_x) + \frac{d}{2} = \wh f^*$ and $f^* \geq \min_{x\in \+B_{2R}} f_\mu(v_x) - \frac{d}{2} = \wh f^* - d$.  From the same calculation, we know for each $x\in C_i$, $f_\mu(v_i) - \frac{d}{2} \leq f_\mu(x)\leq f_\mu(v_i) + \frac{d}{2}$. For $\wh Z_{\mu}$, we have
    \[
        Z_{\mu} \geq \sum_{i=1}^n \int_{C_i} \exp\tp{-f_\mu(x)} \d x \geq \sum_{i=1}^n \!{vol}(C_i)\cdot \exp\tp{-f_\mu(v_i) - \frac{d}{2}} = \wh Z_{\mu}.
    \]
    On the other hand, since $\+B_R\subseteq \bigcup_{i\in[n]} C_i$, we have $\int_{\bb R^d \setminus \tp{\bigcup_{i\in[n]} C_i}} \exp\tp{-f_\mu(x)} \d x < \int_{\bigcup_{i\in[n]} C_i} \exp\tp{-f_\mu(x)} \d x$. Therefore,
    \[
        Z_{\mu}\leq 2\sum_{i=1}^n \int_{C_i} \exp\tp{-f_\mu(x)} \d x \leq 2\sum_{i=1}^n \!{vol}(C_i)\cdot \exp\tp{-f_\mu(v_i) + \frac{d}{2}} = 2e^d\cdot \wh Z_{\mu}.
    \]
\end{proof}

\subsection{Proof of \Cref{thm:main-ub}} \label{sec:proof-of-ub}

From previous sections we know that the distribution $\pi$ is $\+O\tp{\frac{L^3R^4}{\tp{h_2-h_1}^2}}$-log-smooth (\Cref{lem:f2smooth}), has its first moment bounded by $\+O(\sqrt{M})$ (\Cref{lem:f2moment}), satisfies $\DTV\tp{\pi,\mu}= \frac{\eps}{2}$ (\Cref{lem:pi-mu-close}) and satisfies $C_{\!{PI}}\ge \frac{2d\eps}{M}\cdot \tp{\frac{LM}{d\eps}}^{-\+O(d)}$ (\Cref{lem:pi-PI}). Moreover, we can query $f_\pi(x)$ and $\grad f_\pi(x)$ efficiently, provided query access to $f_\mu$ and $\grad f_\mu$. 

Therefore, we can use the algorithm in~\cite{BCE+22} to sample from $\pi$ (see also~\cite[Chapter 11]{Che24}). Let $N$ be the total steps and $h$ be the step size. To sample from a target distribution $\nu\propto e^{-f}$, their algorithm acts as follows:
\begin{itemize}
    \item[1.] Pick a time $t_0\in[0,Nh]$ uniformly at random.
    \item[2.] Let $k_0$ be the largest integer such that $k_0 h<t$. For each $t<t_0$ and $k\leq k_0$, the process evolves as 
    \begin{equation}
        X_t = X_{kh} - (t-kh) \grad f(X_{kh}) + \sqrt{2}(B_t-B_{kh}), \label{eq:LMC2}
    \end{equation} 
    where $\set{B_t}_{t\geq 0}$ is the standard Brownian motion.
    \item[3.] Output $X_{t_0}$.
\end{itemize}

\begin{theorem}[A direct consequence of Corollary 8 in \cite{BCE+22}]\label{thm:LMCforPI}
    Let $\set{\mu_t}_{t\geq 0}$ denote the law of the interpolation \Cref{eq:LMC2} of LMC. Assume the potential function $f$ is $\+L$-smooth and the target distribution $\nu\propto e^{-f}$ satisfies the \Poincare inequality with constant $\alpha>0$. If $\!{KL}(\mu_0 \| \nu)\leq K_0$, choosing step size $h=\frac{\sqrt{K_0}}{2\+L\sqrt{dN}}$, then for $N\geq \max\set{\frac{32^2 \alpha^{-2} \+L^2 dK_0}{\delta^4}, \frac{9K_0}{d}}$ and $\ol \mu_{Nh}\defeq \frac{\int_0^{Nh} \mu_t \d t}{Nh}$, $\DTV(\ol \mu_{Nh},\nu)\leq \delta$.
\end{theorem}

To get a convergence guarantee for our target distribution $\pi$, it remains to find an initial distribution $\mu_0$ such that $\!{KL}(\mu_0 \| \pi)$ is bounded. \Cref{lem:f2smooth} shows that $f_{\pi}$ is $\+L$-smooth for $\+L=\+O\tp{\frac{L^3R^4}{\tp{h_2-h_1}^2}}$. By choosing $\mu_0$ as $\+N\tp{0, \frac{\!{Id}_d}{2\+L}}$, we can bound $\!{KL}(\mu_0 \| \pi)$ using the following lemma.
\begin{lemma}[A direct corollary of Lemma 32 in \cite{CEL+24}]\label{lem:initial}
    Suppose $\grad f(0)= 0$ and $f$ is $\+L$-smooth. Let $m = \E[X\sim \nu]{\|X\|}$ be the first moment of $\nu \propto e^{-f}$. Then for $\mu_0=\+N\tp{0, \frac{\!{Id}_d}{2\+L}}$,
    \[
        \log \tp{\sup \frac{\d \mu_0}{\d \nu}} \leq 2+\+L + f(0)-\min_{x\in \bb R^d} f(x) + \frac{d}{2}\log \tp{4m^2\+L}.
    \]
\end{lemma}

Combining \Cref{coro:gap}, \Cref{lem:f2moment} and \Cref{lem:initial}, $\!{KL}(\mu_0\|\pi)$ can be bounded by $\!{poly}(d,M,L,\eps^{-1})$. Therefore, we can choose $\delta = \frac{\eps}{2}$ in \Cref{thm:LMCforPI} to sample from a distribution whose total variation distance is at most $\frac{\eps}{2}$ to $\pi$ with $\!{poly}(L,M, d,\eps^{-1})\cdot \tp{\frac{LM}{\eps d}}^{\+O(d)}$ queries to $f_\mu$ and $\grad f_{\mu}$.

% Therefore, we can use the algorithm in~\cite{VW19} to sample from $\pi$.

% \begin{theorem}[A direct consequence of Theorem 1 in~\cite{VW19}]\label{thm:ULA}
%     Let $\nu\propto \exp\tp{-f}$ satisfy log-Sobolev inequality with constant $\alpha>0$ and is $L$-smooth.  Assume $\grad f(0)=0$ and $f(0)=0$. The unadjusted Langevin algorithm outputs a sample from a distribution $\tilde \nu$ with $D_{\!{KL}}(\tilde \nu,\nu)\le \delta$ after $\tilde\Theta\tp{\frac{L^2d}{\alpha^2 n}}$ iterations. 
% \end{theorem}\ctodo{So maybe we shoudl assume $f(0)=0$ as well?}

% Note that we can assume without loss of generality that $f_\pi(0)=0$. We can pick $\delta = \frac{\eps^2}{2}$ and apply \Cref{thm:ULA} to sample from a distribution whose KL divergence is at most $\delta$ to $\pi$ with $\!{poly}(LM, d,\eps^{-1})\cdot \tp{\frac{LM}{\eps d}}^{\+O(d)}$ queries to $\grad f_\mu$ and $\grad f_{\mu}$. Then its total variation distance to $\pi$ is at most $\frac{\eps}{2}$ by Pinsker's inequality.

% Let $V:\bb R^d\to R$ be the potential function and distribution $\pi\propto e^{-V}$.  


% The following theorem is a direct corollary of Theorem 7 in \cite{CEL+24}.

% To Define $\hat V(x) = V(x) + \frac{\gamma}{2}\tp{\|x\|-R}^2_{+}$, where $R$ is chosen to satisfy $R= 2\*m = 2\int_{\bb R^d}\|x\| \d \pi(x)$ and $\gamma = \frac{1}{768 kh}$ ($h$ is the step size and $k$ is total steps).
% \begin{theorem}[A consequence of Theorem 7, Lemma 32 and 33 in \cite{CEL+24}]\label{thm:LMCforPI}
% Assume potential function $f$ is $L$-smooth, $\grad f(0)=0$, and the distribution $\nu\propto e^{-f(x)}$ satisfy a \Poincare inequality with constant $C_{\!{PI}}$. Let $\*m\defeq \int_{\bb R}\|x\| \dd \nu(x)$. Assume $\eps^{-1}, L, \*m,\ C_{PI}^{-1}\geq 1$. Then the LMC with a proper step size satisfy $R_2(p_{k}\|\pi)\leq \delta$ after
% \[
%     k = C_{\!{PI}}^2 \cdot \!{poly}\tp{d,m,L,\delta^{-1}, f(0)-\min_{x\in \bb R^d} f(x)}
% \]
% % \[
% %     k=\Theta\tp{\frac{dL^2 C_{\!{PI}}^2}{\eps}\cdot  R_3(p_0\|\pi)^2 \cdot \max\set{1, \frac{m}{d}, \frac{\sqrt{R_2(p_0\|\hat \pi)}}{d}}}
% % \]
% steps.
% \end{theorem}

% Assume the start distribution of LMC is $p_0$. 

% For $q=2$, we know that $R_2(\nu\| \pi) = \ln\tp{1+\chi^2(\nu\|\pi)}$ for any probability measure $\pi$ and $\nu$.
% We can find $p_0$ such that $R_3(p_0\|\pi)=\tilde O(d)$ and $R_2(p_0\|\hat \pi)=\tilde O(d)$. See discussions in Appendix A in \cite{CEL+24}.

% We can use $R_2(\nu\| \pi)$ to upper bound $D_{TV}(\nu,\pi)$. This is because 
% \begin{align}
%     4D_{TV}(\nu,\pi)^2 &= \tp{\int_{\bb R^d} \abs{\pi(x) - \nu(x)} \dd x}^2 \notag \\
%     \mr{Cauchy-Schwartz inequality}
%     &\leq \int_{\bb R^d} \tp{\frac{{\pi(x)-\nu(x)}}{\sqrt{\pi(x)}}}^2 \dd x \cdot \int_{\bb R^d} \pi(x) \dd x \notag\\
%     &= e^{R_2(\nu\|\pi)} - 1.\label{eq:TV}
% \end{align}
% Therefore, we can choose $\delta = 2\eps$ in \Cref{thm:LMCforPI} to sample from a distribution whose $R_2$ divergence is at most $\delta$ to $\pi$ with $\!{poly}(L,M, d,\eps^{-1})\cdot \tp{\frac{LM}{\eps d}}^{\+O(d)}$ queries to $\grad f_\mu$ and $\grad f_{\mu}$. Then its total variation distance to $\pi$ is at most $\frac{\eps}{2}$ since $2\eps \leq \log (1+\eps)$ for $\eps\in(0,1/32)$.


%\ctodo{Remember to show that each step of the construction can be done efficiently.}