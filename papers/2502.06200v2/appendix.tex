

\section{Auxiliary lemmas}

\subsection{The volume of $d$-balls}

% \ctodo{Can we find a reference for these lemmas?}

\begin{proposition}\label{prop:Gamma}
    For positive integer $d\geq 8$, $\tp{\frac{d}{2e}}^{\frac{d}{2}} \leq \Gamma\tp{\frac{d}{2}+1} \leq \tp{\frac{d}{e}}^{\frac{d}{2}}$.
\end{proposition}
\begin{proof}
    When $d$ is even, $\Gamma\tp{\frac{d}{2}+1} = \tp{\frac{d}{2}}!$. From Stirling's approximation,
    \[
        \tp{\frac{d}{2e}}^{\frac{d}{2}} \leq \sqrt{\pi d} \cdot \tp{\frac{d}{2e}}^{\frac{d}{2}} \leq \tp{\frac{d}{2}}! \leq 2\sqrt{\pi d} \cdot \tp{\frac{d}{2e}}^{\frac{d}{2}} \leq \tp{\frac{d}{e}}^{\frac{d}{2}}.
    \]
    % \mn{The Stirling's approximation states that for any positive integer $m$,
    % \[
    %     m!\leq \sqrt{2\pi m}\cdot \tp{\frac{m}{e}}^m \cdot e^{\frac{1}{12m}},
    % \] and
    % \[
    %     m!\geq \sqrt{2\pi m}\cdot \tp{\frac{m}{e}}^m \cdot e^{\frac{1}{12m+1}}.
    % \]}
    When $d$ is odd, $\Gamma\tp{\frac{d}{2}+1} = \frac{\sqrt{\pi}}{2^d}\cdot \frac{d!}{\tp{\frac{d-1}{2}}!}$. Similarly we have
    \[
        \frac{\sqrt{\pi}}{2^d}\cdot \frac{d!}{\tp{\frac{d-1}{2}}!} \leq \frac{\sqrt{\pi}}{2^d}\cdot \frac{\sqrt{2\pi d} \tp{\frac{d}{e}}^d }{ \sqrt{\pi(d-1)}\tp{\frac{d-1}{2e}}^{\frac{d-1}{2}}} \cdot e^{\frac{1}{12d} - \frac{1}{6(d-1)+1}} \leq \sqrt{ \frac{2\pi d}{d-1}}\cdot \tp{\frac{d-1}{d}}^{\frac{d-1}{2}} \tp{\frac{d}{2e}}^{\frac{d+1}{2}} \leq \tp{\frac{d}{e}}^{\frac{d}{2}},
    \]
    and 
    \[
        \frac{\sqrt{\pi}}{2^d}\cdot \frac{d!}{\tp{\frac{d-1}{2}}!} \geq \frac{\sqrt{\pi}}{2^d}\cdot \frac{\sqrt{2\pi d} \tp{\frac{d}{e}}^d }{ \sqrt{\pi(d-1)}\tp{\frac{d-1}{2e}}^{\frac{d-1}{2}}} \cdot e^{\frac{1}{12d+1} - \frac{1}{6(d-1)}} \geq \sqrt{2\pi}\cdot \tp{\frac{d}{2e}}^{\frac{d+1}{2}} \cdot \frac{1}{2} \geq \tp{\frac{d}{2e}}^{\frac{d}{2}}.
    \]
\end{proof}

\begin{proposition}
    The volume of $\+B_R$ is $\frac{(\pi R^2)^\frac{d}{2}}{\Gamma\tp{\frac{d}{2}+1}}$.
\end{proposition}

\begin{corollary}\label{cor:dballvolbound}
    \[
        \tp{\frac{e\pi R^2}{d}}^{\frac{d}{2}}\le\!{vol}\tp{\+B_R} \le \tp{\frac{2e\pi R^2}{d}}^{\frac{d}{2}}.
    \]
\end{corollary}

% \ctodo{Clean here}

\subsection{The trigonometric functions in \Cref{lem:disjointcap}}

For two vectors $u,v\in \bb R^d$, let $\theta\tp{u,v}$ denote the angle between $u$ and $v$. The cosine of this angle is defined as $\cos\tp{\theta\tp{u,v}} = \frac{\inner{u}{v}}{\|u\|\|v\|}$. In this section, we prove the results about trigonometric functions used in \Cref{lem:disjointcap}.

\begin{lemma}\label{lem:cos}
    For any unit vectors $v,x,w\in \bb R^d$, we have 
    \[
        \cos\tp{\theta(v,w)}= \cos\tp{\theta(v,x)}\cdot \cos\tp{\theta(x,w)} +\sin\tp{\theta(v,x)}\cdot \sin\tp{\theta(x,w)}.
    \]
\end{lemma}
\begin{proof}
    We divide $v$ into two parts, $v_{\|}$ and $v_{\perp}$, which are parallel and perpendicular to $x$ respectively. Similarly, we divide $w$ into $w_{\|}$ and $w_{\perp}$. Therefore,
    \begin{align*}
        \cos\tp{\theta(v,w)} &= \inner{v}{w}= \inner{v_\| + v_\perp}{w_\| + w_\perp}\\
        &= \inner{v_\|}{w_\|} + \inner{v_\perp}{w_\perp}\\
        &=\cos\tp{\theta(v,x)}\cdot \cos\tp{\theta(x,w)} + \sin\tp{\theta(v,x)}\cdot \sin\tp{\theta(x,w)}
    \end{align*}
\end{proof}

\begin{lemma}\label{lem:cosinBall}
    Let $u,v\in \bb R^d$ be a vector with norm $\|u\|=\|v\|=\frac{3R}{4}$. Let $\ell=\frac{\sqrt{\tp{\frac{3R}{4}}^2-2r_2^2}}{\frac{3R}{4}}$. If the caps $C_u=\set{x\in \bb R: \|x\|=\frac{3R}{4}, \cos\tp{\theta(x,u)}\geq \ell}$ and $C_v=\set{x\in \bb R: \|x\|=\frac{3R}{4}, \cos\tp{\theta(x,v)}\geq \ell}$ are disjoint. Then the balls $\+B(v,r_2)$ and $\+B(u,r_2)$ are also disjoint.
\end{lemma}
\begin{proof}
    Assume in contrast that there exists $x\in \+B(u,r_2)\cap \+B(v,r_2)$. Let $y=\frac{3R}{4}\cdot \frac{x}{\|x\|}$. We show that $y\in C_u\cap C_v$.

    Note that since $x\in \+B(u,r_2)$, 
    \[
        r_2^2\geq \|x-u\|^2=\|x\|^2 + \|u\|^2 - 2\inner{x}{u} = \|x\|^2 + \|u\|^2 - 2\|x\|\|u\|\cdot \cos\tp{\theta(x,u)}.
    \]
    So we have
    \begin{align*}
        \cos\tp{\theta(x,u)}&\geq \frac{\|x\|^2+\|u\|^2}{2\|x\|\|u\|} - \frac{r_2^2}{2\|x\|\|u\|} \\
        &\geq 1 - \frac{r_2^2}{2\|x\|\|u\|} \\
        \mr{$\|x\|\geq \|u\|-\|x-u\|\geq \frac{3R}{4}-r_2$} &\geq 1 - \frac{r_2^2}{2\tp{\frac{3R}{4}-r_2}\cdot \frac{3R}{4}} \\
        \mr{$R\geq 4r_2$}&\geq 1-\frac{r_2^2}{\tp{\frac{3R}{4}}^2}\geq \ell.
    \end{align*}
    Similarly, $\cos\tp{\theta(x,v)}\geq \ell$. This indicates $y\in C_u\cap C_v$, which leads to a conflict.
\end{proof}

\subsection{The properties of the smoothness parameter and the second moment}
The following lemma shows that scaling the domain does not change $L\cdot M$.
\begin{lemma}\label{lem:LM}
    For an $L$-smooth function $f\colon \bb R^d\to \bb R$, assume the second moment of the distribution $\mu$ with density $p_\mu(x)\propto e^{-f(x)}$ is $M$. Consider a scaled function $f'(x)=f\tp{\frac{x}{\sqrt{L}}}$. Then $f'$ is $1$-smooth and the second moment of $\nu$ with density $\propto e^{-f'(x)}$ is $L\cdot M$.
\end{lemma}
\begin{proof}
    From the chain rule, $\grad^2 f'(x) = \frac{1}{L} \grad^2 f(y)\Big|_{y=\frac{x}{\sqrt{L}}}$. This indicates that the function $f'$ is $1$-smooth. 

    For the second moment, we have
    \begin{align*}
        \E[X\sim \nu]{\|X\|^2} &= \frac{\int_{\bb R^d} \|x\|^2 e^{-f'(x)}\d x}{\int_{\bb R^d} e^{-f'(x)}\d x}\\
        \mr{substituting $x$ by $\sqrt{L}y$}&= \frac{\int_{\bb R^d} \|\sqrt{L}y\|^2 e^{-f'(\sqrt{L}y)}\cdot \tp{\sqrt{L}}^d \d y}{\int_{\bb R^d} e^{-f'(\sqrt{L}y)}\cdot \tp{\sqrt{L}}^d\d y}\\
        &=L\cdot \frac{\int_{\bb R^d} \|y\|^2 e^{-f(y)}\d y}{\int_{\bb R^d} e^{-f(y)}\d y}\\
        &=L\cdot \E[Y\sim \mu]{\|Y\|^2} \\
        &=L\cdot M.
    \end{align*}
\end{proof}

The following lemma gives a lower bound for $L\cdot M$ for any distribution $\mu$ with density $\propto e^{-f}$ such that $\grad f(0)=0$.
% The following lemma shows that for any distribution $\nu\propto e^{-f}$ satisfying \Cref{assump:moment} and \ref{assump:smooth} with $\grad f(0)=0$, we have $LM\geq d$.
\begin{lemma}\label{lem:lb-LM}
    Let $f:\bb R^d\to \bb R$ be an $L$-smooth function such that $e^{-f}$ is integrable and $\grad f(0)=0$. Let $\mu$ be the distribution with density $p_{\mu}\propto e^{-f}$. Then $\E[\mu]{\|X\|^2}\geq \frac{d}{L}$.
\end{lemma}
\begin{proof}
    By the definition of $L$-smoothness, for each $x\in \bb R^d$, 
    \[
        L\|x-0\| \geq \|\grad f(x) - \grad f(0)\| = \|\grad \log \mu(x)\| = \frac{\|\grad \mu(x)\|}{\mu(x)}.
    \]
    Therefore,
    \begin{align*}
        \E[\mu]{\|X\|^2} &= \int_{\bb R^d} \|x\|^2 p_\mu(x) \dd x\\
        &\geq \int_{\bb R^d} \|x\|\cdot \frac{1}{L}\cdot \frac{\|\grad p_\mu(x)\|}{p_\mu(x)}  \cdot p_\mu(x) \dd x \\
        &= \frac{1}{L} \int_{\bb R^d} \|x\|\cdot \|\grad p_\mu(x)\|\dd x  \\
        \mr{Cauchy-Schwartz inequality}&\geq -\frac{1}{L} \int_{\bb R^d} \sum_{i=1}^d x_i\cdot \grad p_\mu(x)_i\dd x \\
        &= -\frac{1}{L}\sum_{i=1}^d \int_{\bb R} \int_{\bb R}\cdots \int_{\bb R} x_i\cdot \grad p_\mu(x)_i\dd x_i \dd x_1\cdots \dd x_d \\
        \mr{integration by parts}&= \frac{1}{L}\sum_{i=1}^d \int_{\bb R} \int_{\bb R}\cdots \int_{\bb R}  p_\mu(x)\dd x_i \dd x_1\cdots \dd x_d \\
        &= \frac{d}{L}.
    \end{align*}
\end{proof}
