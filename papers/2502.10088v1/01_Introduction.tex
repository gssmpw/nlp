\label{sec:Introduction}

Trust and acceptance are critical for the widespread adoption of autonomous systems, whether in healthcare~\cite{alqudah2021technology,liu2022roles} or in other fields~\cite{choi2015investigating,parsonage2023trust}. \revised{In contexts such as self-driving cars, where automation is designed to minimize human error and enhance safety, public hesitation often arises from a lack of trust~\cite{liu2019evaluating}. }This reluctance often stems from a lack of trust—people are uncomfortable with the idea of handing over something as personal and potentially dangerous as driving to a machine, even if the machine is proven to be highly effective. In healthcare, this issue is even more sensitive, as patients’ physical well-being and emotional comfort are directly impacted by their interactions with robotic systems. Studies show that patients’ hesitation towards robotic systems stems from concerns about trust, privacy, and ethical issues, which directly affect their behavioral intentions toward accepting these technologies in healthcare settings~\cite{kamani2023patients}. Particularly in procedures where human touch and empathy traditionally play a large role, without acceptance, robotic systems may be met with hesitation or rejection by patients~\cite{xu2018would}. \revised{In particularly robotic ultrasound diagnosis, where the patient is conscious throughout the procedure, these concerns become even more pronounced. Robotic ultrasound systems offer advantages in terms of precision and repeatability~\cite{jiang2023robotic,bi2024machine}, but their acceptance has lagged due to the disconnect patients feel when interacting with machines instead of human operators~\cite{eilers2023importance}. This discomfort is magnified by the fact that patients undergoing ultrasound procedures are accustomed to direct human interaction—something robots inherently lack. Therefore, the challenge in designing robotic medical systems is not merely one of technical accuracy but also of creating an experience that is more human-centered, one that addresses patient comfort, trust, and acceptance.}


%This lack of acceptance is often rooted in the impersonal nature of automation—robots, however efficient, can seem cold and detached. This perceived absence of human involvement can lead to increased anxiety and discomfort~\cite{iroju2017state}, especially when the patient is conscious during the procedure.



One \revised{promising solution to} bridge between robotic efficiency and human empathy is the integration of virtual agents that can interact with patients in real-time, offering the reassurance that many patients need in clinical settings~\cite{lisetti2015now}. \add{Recent studies highlight that virtual agents not only improve user engagement but also play a crucial role in fostering trust, particularly when tasks are delegated to these agents~\cite{sun2024digital}. \revised{These agents create a} trust-based interaction framework, which combines rational, emotional, and technological dimensions, underscores the importance of designing agents that balance efficiency with empathy, thereby bridging the gap between automation and human-centric care.} Such agents, when combined with immersive technologies like augmented reality (AR) and virtual reality (VR), can create a hybrid experience where the precision of robotics is augmented by the comforting presence of an AI-driven virtual assistant. This idea forms the core of our approach\add{, aiming to make robotic medical systems more human-centered and patient-friendly}.

In this paper, we present a unique solution aimed at enhancing patient acceptance of robotic ultrasound procedures by combining an AI-based virtual human assistant with immersive visualizations. The virtual assistant interacts with patients throughout the procedure, offering reassurance and explaining the process. It also appears to control the robotic ultrasound probe, giving the impression of human-like involvement. By simulating human presence, we aim to bridge the gap between robotic automation and patient comfort.

To achieve this, we build upon proposed robotic ultrasound system and introduce three immersive visualizations, each designed to enhance the patient experience during the robotic ultrasound. The first is an AR visualization, where the patient sees the real world with the virtual assistant superimposed in the environment, as if present in the room (see Fig.~\ref{fig:teaser}a). The second is \replaced{a VR visualization with a passthrough window}{an Augmented Virtuality (AV) visualization}, which shows the body part being scanned in the real world while the rest of the environment is represented by a 3D Gaussian Splatting model of the room, excluding the robot (see Fig.~\ref{fig:teaser}b). Finally, the fully immersive VR visualization creates a complete virtual environment where the patient views a virtual version of their own body being scanned and synchronized with the real-world procedure (see Fig.~\ref{fig:teaser}c).
These immersive visualizations are designed to address the core issue of patient acceptance by providing a more humanized interaction during robotic ultrasound procedures and enhancing the overall experience.
The key contributions of this paper are as follows:

\begin{itemize}
  \item A novel pipeline that integrates a virtual human assistant and immersive visualizations into robotic ultrasound procedures, providing a structured approach to enhancing patient interaction and comfort.
  \item Three novel visualization modalities designed to enhance patient acceptance of robotic ultrasound systems.
  \item A user study in which we evaluate the impact of these visualizations on patient acceptance and mental workload, demonstrating the significant benefits of our approach.
  \item Insights for future design of mixed reality visualizations and virtual agents to enhance patient comfort and acceptance in autonomous medical procedures.
\end{itemize}


\add{Building on these contributions, the study evaluates three hypotheses that explore how the proposed conversational virtual agents and immersive visualizations address stress, trust, comfort, and usability challenges in robotic ultrasound systems, aligning with the goal of enhancing patient-centered care.} In addition, we make the code publicly available~\footnote{\url{https://github.com/stytim/Robotic-US-with-Virtual-Agent}} to encourage further research and development in enhancing patient acceptance of robotic medical systems.