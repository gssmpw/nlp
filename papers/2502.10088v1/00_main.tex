\documentclass[journal]{vgtc}                     % final (journal style)
%\documentclass[journal,hideappendix]{vgtc}        % final (journal style) without appendices
%\documentclass[review,journal]{vgtc}              % review (journal style)
%\documentclass[review,journal,hideappendix]{vgtc} % review (journal style)
%\documentclass[widereview]{vgtc}                  % wide-spaced review
%\documentclass[preprint,journal]{vgtc}            % preprint (journal style)






%% Uncomment one of the lines above depending on where your paper is
%% in the conference process. ``review'' and ``widereview'' are for review
%% submission, ``preprint'' is for pre-publication in an open access repository,
%% and the final version doesn't use a specific qualifier.

%% If you are submitting a paper to a conference for review with a double
%% blind reviewing process, please use one of the ``review'' options and replace the value ``0'' below with your
%% OnlineID. Otherwise, you may safely leave it at ``0''.
\onlineid{1049}

%% In preprint mode you may define your own headline. If not, the default IEEE copyright message will appear in preprint mode.
%\preprinttext{To appear in IEEE Transactions on Visualization and Computer Graphics.}

%% In preprint mode, this adds a link to the version of the paper on IEEEXplore
%% Uncomment this line when you produce a preprint version of the article 
%% after the article receives a DOI for the paper from IEEE
%\ieeedoi{xx.xxxx/TVCG.201x.xxxxxxx}

%% declare the category of your paper, only shown in review mode
\vgtccategory{Research}

%% please declare the paper type of your paper to help reviewers, only shown in review mode
%% choices:
%% * algorithm/technique
%% * application/design study
%% * evaluation
%% * system
%% * theory/model
\vgtcpapertype{please specify}

%% Paper title.
\title{Enhancing Patient Acceptance of Robotic Ultrasound \\through Conversational Virtual Agent and Immersive Visualizations}

%% Author ORCID IDs should be specified using \authororcid like below inside
%% of the \author command. ORCID IDs can be registered at https://orcid.org/.
%% Include only the 16-digit dashed ID.
\author{%
  \authororcid{Tianyu Song}{0000-0002-8428-9651},
  Felix Pabst, 
  \authororcid{Ulrich Eck}{0000-0002-5322-4724}, and 
  \authororcid{Nassir Navab}{0000-0002-6032-5611}
}

\authorfooter{
  %% insert punctuation at end of each item
  \item
  	Tianyu Song, Felix Pabst, Ulrich Eck and Nassir Navab are with Technical University of Munich.
  	
  	E-mail: \{tianyu.song, felix.pabst, ulrich.eck, nassir.navab\}@tum.de.

}

%% Abstract section.
\abstract{%
    %% We recommend that you link to your supplemental material here in the abstract, as well
  %% as in the Supplemental Materials section at the end.
  Robotic ultrasound systems have the potential to improve medical diagnostics, but patient acceptance remains a key challenge. 
To address this, we propose a novel system that combines an AI-based virtual agent, powered by a large language model (LLM), with three mixed reality visualizations aimed at enhancing patient comfort and trust. The LLM enables the virtual assistant to engage in natural, conversational dialogue with patients, answering questions in any format and offering real-time reassurance, creating a more intelligent and reliable interaction. The virtual assistant is animated as controlling the ultrasound probe, giving the impression that the robot is guided by the assistant.
The first visualization employs augmented reality (AR), allowing patients to see the real world and the robot with the virtual avatar superimposed. 
The second visualization is \replaced{a virtual reality (VR) environment with a passthrough window}{an augmented virtuality (AV) environment}, where the real-world body part being scanned is visible, while a 3D Gaussian Splatting reconstruction of the room, excluding the robot, forms the virtual environment. 
The third is a fully immersive virtual reality (VR) experience, featuring the same 3D reconstruction but entirely virtual, where the patient sees a virtual representation of their body being scanned in a robot-free environment. In this case, the virtual ultrasound probe, mirrors the movement of the probe controlled by the robot, creating a synchronized experience as it touches and moves over the patient’s virtual body. 
We conducted a comprehensive agent-guided robotic ultrasound study with all participants, comparing these visualizations against a standard robotic ultrasound procedure. Results showed significant improvements in patient trust, acceptance, and comfort. Based on these findings, we offer insights into designing future mixed reality visualizations and virtual agents to further enhance patient comfort and acceptance in autonomous medical procedures.
}

%% Keywords that describe your work. Will show as 'Index Terms' in journal
%% please capitalize first letter and insert punctuation after last keyword
\keywords{Mixed Reality, Virtual Agent, Robotic Ultrasound, Trust and Acceptance}

%% A teaser figure can be included as follows
\teaser{
  \centering
  \includegraphics[width=\linewidth]{figures/teaser.png}
  \caption{\textbf{Proposed Visualizations with a Conversational Virtual Agent Guiding the Robotic Ultrasound Procedure.} The figure shows three immersive visualizations as seen through the head-mounted display (from left to right): augmented reality, \replaced{virtual reality with a passthrough window}{augmented virtuality}, and fully immersive virtual reality. In both \add{augmented virtuality and }virtual reality visualizations, a realistic 3D Gaussian Splatting model of the actual room, excluding the robot, was used as the virtual environment. In each mode, the virtual human assistant communicates with the patient, offering explanations and reassurance to reduce stress and improve the patient’s experience during the autonomous ultrasound procedure.}
  \label{fig:teaser}
}
%% Uncomment below to disable the manuscript note
%\renewcommand{\manuscriptnotetxt}{}

%% Copyright space is enabled by default as required by guidelines.
%% It is disabled by the 'review' option or via the following command:
%\nocopyrightspace


%%%%%%%%%%%%%%%%%%%%%%%%%%%%%%%%%%%%%%%%%%%%%%%%%%%%%%%%%%%%%%%%
%%%%%%%%%%%%%%%%%%%%%% LOAD PACKAGES %%%%%%%%%%%%%%%%%%%%%%%%%%%
%%%%%%%%%%%%%%%%%%%%%%%%%%%%%%%%%%%%%%%%%%%%%%%%%%%%%%%%%%%%%%%%

%% Tell graphicx where to find files for figures when calling \includegraphics.
%% Note that due to the \DeclareGraphicsExtensions{} call it is no longer necessary
%% to provide the the path and extension of a graphics file:
%% \includegraphics{diamondrule} is completely sufficient.
\graphicspath{{figs/}{figures/}{pictures/}{images/}{./}} % where to search for the images

%% Only used in the template examples. You can remove these lines.
\usepackage{tabu}                      % only used for the table example
\usepackage{booktabs}                  % only used for the table example
\usepackage{lipsum}                    % used to generate placeholder text
\usepackage{mwe}                       % used to generate placeholder figures

%% We encourage the use of mathptmx for consistent usage of times font
%% throughout the proceedings. However, if you encounter conflicts
%% with other math-related packages, you may want to disable it.
\usepackage{mathptmx}                  % use matching math font
\usepackage{soul}
%\newcommand{\todo}[1]{\textcolor{red}{TODO: #1}}
\newcommand{\revised}[1]{\textcolor{black}{#1}}
\newcommand{\add}[1]{\textcolor{black}{#1}}
\newcommand{\deleted}[1]{\textcolor{red}{\st{#1}}}
%
%% Define the \replaced command
\newcommand{\replaced}[2]{%
%  \textcolor{red}{\st{#1}}%
  \textcolor{black}{#2}%
}


\begin{document}

%%%%%%%%%%%%%%%%%%%%%%%%%%%%%%%%%%%%%%%%%%%%%%%%%%%%%%%%%%%%%%%%
%%%%%%%%%%%%%%%%%%%%%% START OF THE PAPER %%%%%%%%%%%%%%%%%%%%%%
%%%%%%%%%%%%%%%%%%%%%%%%%%%%%%%%%%%%%%%%%%%%%%%%%%%%%%%%%%%%%%%%

%% The ``\maketitle'' command must be the first command after the
%% ``\begin{document}'' command. It prepares and prints the title block.
%% the only exception to this rule is the \firstsection command
\firstsection{Introduction}

\maketitle


\section{Introduction}

\begin{figure*}
    \centering
    \includegraphics[width=\textwidth]{figures/Introduction.pdf}
    \caption{Showing the novel problem statement applied to traffic prediction use case. Multiple unstructured observations from the past are used to reconstruct a hidden traffic state from which a full traffic state is forecast with a set of query locations. }
    \label{fig:intro}
\end{figure*}

% Was sagen denn die anderen warum Traffic Prediction gut ist? 
Forecasting the traffic in the near future is an important task for city management.
Data from the near past is used to predict future traffic states with spatio-temporal Graph Neural Networks \cite{bui22}.
Accurate prediction provides the opportunity to optimize traffic flow, reduce traffic jams and increase air quality \cite{Po19}.

% Wieso ist Sparsity in allen Dimensionen wichtig.
While traffic prediction relies on the availability of data from traffic sensors, there exists a plethora of reasons why sensors may stop working temporarily, such as simple errors, energy saving, or overloaded communication systems.
Considering small- or medium-sized cities, the coverage of sensors may be low because the sensors are too expensive or not available.
Also, the sensors are typically static and do not adapt to changes in the traffic flow (e.g. caused by a construction site), which motivates moving sensors that for example could be mounted on cars. 
However, both missing and moving sensors introduce sparsity, since measurements may not be available for all locations at all times.
This sparsity must be explicitly addressed in traffic prediction for a realistic application scenario, which is illustrated in figure \ref{fig:intro}.
From one hour of data on Sunday morning, only few observations of the traffic state are available at each timestep.
The number of observations may differ throughout the observed time and the observation itself can be distributed arbitrarily in the city. 
We assume a relatively low number of sensors to account for resource saving and sensor failure in our proposed framework SUSTeR.
The task is to predict the dense traffic state one timestep after the observations at all possible sensor locations.
We study this problem on the traffic dataset Metr-LA and PEMS-BAY to test our assumption that only a fraction of the sensor values would be enough for good predictions.
By modifying an existing traffic dataset, we are able to compare our results from very sparse observations to the bottom line with all information available.
A successful study will provide insights in how sensors in new cities can be reduced before installing them and further mobile sensors would save more resources and are able to adapt to new traffic situations.
We argue that in order to be adaptable to other cities and changes in traffic flows, prior information like the road network should be neglected and just the sparse observations considered.
This comes with the added benefit of making our solution applicable in regions where no openly available road network is maintained or pathways change frequently (e.g. flood areas, animal observations). 


The aforementioned problem is novel and more challenging than the commonly considered traffic prediction problem, since there exist very few observations in each input sample.
Current works for the traffic prediction problem do not consider any missing values. \cite{Li2021, Shao22}
A common method among state of the art approaches is the usage of Graph Neural Networks on graphs that model the sensor network \cite{bui22}.
The values of a sensor are applied to the same graph node for each timestep which prohibits any non-stationary sensors . 
With fixed sensor locations, the resulting sensor network is highly correlated with the road network.
Streets connecting two intersections with sensors should be also an interesting point for correlations in the sensor network.
However, variable observations and high temporal sparsity rates can not be modeled adequately in a static network.
We show in our experiments that the road network has only a small influence on the traffic predictions.

Besides the traffic prediction for future timesteps, some works explore the field of traffic speed imputation \cite{Cini22, Cuza22} where missing sensor values are predicted.
But the amount of missing values is assumed to be at most 80\%, which on average are still over 40 given sensors in each timestep in the Metr-LA dataset with a total of 207 sensors.
We consider up to 99.9\% missing values which are on average 2.4 observations in each timestep that are used as input.
Such high sparsity rates drastically decrease the chance that multiple values are present in one input sample from the same sensor location, which makes it challenging to recognize and learn temporal correlations for each location on its own.

High sparsity rates (>95\%) result in few sensor values, but if a reconstruction of the traffic state would be possible, we question if spatio-temporal graphs require nodes for each sensor.
In SUSTeR we utilize only a small amount of graph nodes for the encoding of information and do not relate such nodes to the sensor network.
We call this the hidden graph (see figure \ref{fig:intro}), which is still able to reconstruct the complete traffic state.
Due to the reduced number of nodes SUSTeR achieves faster runtimes, as shown in the experiments.
This hidden graph is not embedded directly in the spatial domain, which is why the assignment of observations, as well as the querying of the future traffic, is done with an encoder and a decoder, implemented as neural networks.
The decoding from the hidden graph to future values depends on a set of query locations.
Figure \ref{fig:intro} shows the query locations as given from outside and in combination with the reconstructed traffic state the future values are predicted.

To construct the hidden graph we encode observations from each timestep into from multiple graphs, one for each timestep. 
The graphs are created in a residual style and information is added to the node embeddings from the previous timesteps.
We choose this method to incorporate all timesteps equally into the hidden state because the redundant information along the past is non-existing for high sparsity rates.
From the sequence of graphs where our framework inserted the observations step by step we apply STGCN \cite{Yu18}, an algorithm for traffic prediction to find and learn the spatio-temporal correlations on our small number of graph nodes.
The first future timestep of the STGCN is our hidden graph in which the traffic state is reconstructed. 

% Recent work has an implicit embedding of the graph nodes into the spatial domain as the assignment from the sensor to graph node is fixed one by one.
% Because the graph has the same structure as the road network spatio-temporal correlations can be learned between those sensors.
% We reduce the number of nodes and use a non-linear assignment learned data-driven from the observations.

We find in the experiments that SUSTeR outperforms the plain STGCN and modern traffic prediction frameworks like D2STGNN for high sparsity rates $(\geq 99\%)$.
This is equivalent to only $0.2$ to $2.4$ observation for each timestep on average.
SUSTeR uses fewer parameters than the baselines and can train faster and with less training data.
Our main contributions can be summarized as follows:
\begin{itemize}
    \item We introduce a sparse and unstructured variant of the traffic prediction problem with sparsity in all dimensions. The sensors report only a fraction of their values and are arbitrarily distributed in the spatial domain.
    \item We propose SUSTeR, a framework around the STGCN architecture, which maps sparse observations onto a dense hidden graph to reconstruct the complete traffic state.
    Our code is available at github.\footnote{https://github.com/ywoelker/SUSTeR}
    \item We conducts experiments that show that SUSTeR outperforms the baselines in very sparse situations ($\geq 95\%$) and has a competitive performance in low sparsity rates.
    % \item SUSTeR trains a third faster than the next competitor.
\end{itemize}

\section{Related Work}

Recent advancements in LLMs have significantly expanded their application across diverse academic fields such as social science \citep{aher_using_2023, gao_s3_2023}, behavioral economics \citep{horton_large_2023}, and human-computer interaction \citep{hamalainen_evaluating_2023}, primarily through the creation and deployment of diverse agent personas. These LLM-created personas aim to simulate complex human and social behaviors \citep{park_social_2022, park_generative_2023}, enabling the development of increasingly personalized applications such as recommendation systems \citep{wang_user_2023}.

Existing frameworks for persona creation primarily emphasize isolated human traits, focusing mainly on socio-demographic characteristics \citep{chen_empathy_2024, zhang_speechagents_2024, chuang-etal-2024-simulating} for modeling specific human subpopulations \citep{argyle_out_2023}. Although researchers have expanded these frameworks to incorporate other dimensions such as personality traits \citep{jiang-etal-2024-personallm, liu2024skepticism, xie_human_2024, yuan_evaluating_2024} and value systems \citep{zhou_sotopia_2024, xie2024largelanguagemodelagents, kang_values_2023}, their approaches often yield biased or incomplete representations, as evidenced by homogeneous depictions of socially underrepresented groups \citep{petrov_limited_2024, gupta_bias_2023, deusex_2024, cheng_compost_2023, lee_large_2024}.

Research in social psychology demonstrates that an individual's self-concept emerges from the dynamic interplay of multiple identity dimensions, including personal traits, social interactions, and lived experiences \citep{mead1934mind}. This fundamental understanding highlights the critical importance of incorporating such multidimensional aspects in developing LLM agents that authentically reflect real-world individuals and their behavioral and thought patterns \citep{xiao_how_2023}.

To address these limitations, we introduce the SPeCtrum framework (see Figure \ref{fig:1}), which enables structured and authentic persona representations through (a) identifying essential elements of multidimensional self-concept based on social science theories and research methodologies and (b) developing systematic pipelines for integrating diverse identity sources. Moving beyond the dominant focus on isolated traits, our framework emphasizes the dynamic interactions between identity components to create LLM-based agents that capture the rich complexity of real-world individuals.


\section{Methods}\label{sec:Methods}

\begin{figure*}
    \centering
    \includegraphics[width=\textwidth]{figures/Pipeline2.png}
    \caption{\textbf{Overview of the Proposed Pipeline.} The pipeline adds a visualization module, including augmented reality and virtual reality modalities, enhanced by a virtual human assistant. The system architecture facilitates real-time interaction and synchronization between the robotic ultrasound system, the virtual agent, and the patient, providing a more engaging and patient-centered experience.}
    \label{fig:overview}
\end{figure*}


In this work, we propose a pipeline, as illustrated in Fig.~\ref{fig:overview}, that introduces a visualization module into the existing robotic ultrasound system (RUS), enhancing the patient experience through conversational virtual agent and immersive visualizations. 
In the following subsections, we detail the components and theoretical frameworks behind the conversational virtual agent, the different visualization modalities (AR, \revised{AV}, and fully immersive VR), as well as the registration process that ensures the seamless integration of virtual and real-world elements.

\subsection{Force-Compliant Robotic Ultrasound System}

Our robotic ultrasound system employs a force-compliant control approach based on impedance control~\cite{jiang2020automatic,jiang2021autonomous,jiang2021deformation}, which ensures safe and accurate probe positioning during the ultrasound procedure by regulating both contact force and probe orientation. This control system operates as a spring-like mechanism with predefined stiffness values, ensuring that if an obstacle, such as the patient’s body, is encountered, the robot will either bypass with a reduced force or stop at the location to avoid applying excessive force.

Impedance control modulates the interaction forces between the robotic arm and the patient’s tissue by adjusting the robot’s stiffness, damping, and inertia parameters. The control law is defined as:
\begin{equation}
\tau = J^T \left[ F_d + K_m e + D \dot{e} + M \ddot{e} \right]
\end{equation}
where $\tau$ represents the computed joint torques, $J^T$ is the transposed Jacobian matrix, and $e$ is the pose error between the current and target positions. The stiffness, damping, and inertia matrices are denoted by $K_m$, $D$, and $M$, respectively. The desired force $F_d$ controls the applied contact force. To ensure stable force along the probe’s centerline, the robot uses a 1-DOF compliant controller for force control and a 5-DOF position controller for precise positioning.

This force-compliant control \revised{follows the same formulation used in Hennersperger et al.\cite{hennersperger2017towards} and Jiang et al.\cite{jiang2021autonomous}, which} allows the system to adapt to soft tissue deformations and variations in probe position, preventing excessive force that could cause discomfort or harm. \add{To enable communication between the RUS and the visualization module, messages are serialized and passed between the two. This ensures that the visualization module is always aware of the robot’s state and the current phase of the procedure. Additionally, commands produced through user interactions with the conversational virtual agent in the visualization module can be sent back to the RUS via serialized messages. This bidirectional communication enables real-time synchronization and seamless integration of robotic control and immersive visualizations.}


\subsection{Conversational Virtual Agent}

The conversational virtual agent serves as a central component of the proposed visualizations, aiming to enhance patient interaction by natural communication and reduce the sense of isolation commonly associated with robotic procedures~\cite{almeras2019operating}, therefore creating a more humanized and engaging experience.

At the core of the agent’s functionality is a speech-to-text (STT) system that transcribes the patient’s spoken words into textual input. The transcribed text is then processed by a large language model (LLM), which generates contextually appropriate responses based on pre-configured prompt. The use of an LLM ensures that the agent provides responses that are both relevant to the patient’s situation and emotionally supportive, fostering a sense of trust and ease during the procedure.
Once the response is generated, it is converted back into audible speech through a text-to-speech (TTS) engine, which provides natural output. The TTS system matches the assistant’s appearance and personality, ensuring consistent and realistic verbal interactions. \add{Similar technical pipelines of the conversational intelligent agent have been used in domains such as virtual museums~\cite{garcia2024speaking} and educational VR applications~\cite{yang2024effects}.}

In addition to the agent’s verbal responses, various animations are employed to enhance its realism and human-like presence~\cite{yu2021avatars}. These include eye blinking, mouth movements synchronized with speech generated by the TTS system, and a subtle idle breathing animation to give the avatar a more natural appearance. Beyond these baseline animations, inverse kinematics (IK) is used to further enhance realism by controlling the assistant’s head and hand movement. The IK algorithm calculates natural, human-like gestures in response to the patient’s position and the state of the robot. \revised{Studies have shown that non-verbal cues such as gestures, head movements, and subtle body language significantly enhance user immersion and realism in virtual environments~\cite{etienne2023perception, beck2012emotional}. Adapting these findings to our robotic ultrasound use case, we implemented a set of non-verbal behaviors to increase the agent’s presence and engagement.} For example, the assistant’s head turns toward the patient when they speak, and its hand animates to hold the ultrasound probe when the robot moves it within the agent’s arm reach, as though the virtual assistant were guiding the procedure. This visual and behavioral consistency strengthens the impression of the virtual agent being physically present and actively engaged with the patient.

Together, these elements form the foundation of the conversational virtual agent framework. The integration of real-time speech, intelligent language processing, and naturalistic physical interaction ensures that the agent serves as a comforting presence, enhancing the overall patient experience.

\subsection{Augmented Reality Visualization}

The AR visualization, illustrated in Fig.~\ref{fig:teaser}a, is designed to blend virtual elements with the patient’s real-world environment, allowing for seamless integration of the guidance from virtual human agent within a familiar, physical space. This approach maintains situational awareness by allowing the patient to see both their surroundings and the robotic arm at work while interacting with the assistant. 
In this visualization, the avatar appears seated next to the patient, engaging with them through real-time conversation. Through the HMD, the patient can see both the virtual assistant and the ultrasound probe, which the avatar is holding and guiding over the patient’s body. As the robotic arm moves the ultrasound probe in the physical world, the avatar’s virtual hand mimics these motions synchronously. This visual configuration provides the patient with the comforting illusion that the human-like avatar is controlling the robotic procedure, reducing the sense of detachment associated with the automated process.


\subsection{\revised{Augmented Virtuality} Visualization}

The \replaced{VR Passthrough}{AV} visualization, as demonstrated in Fig.~\ref{fig:teaser}b, combines an immersive virtual environment with selective real-world visibility through a passthrough window. This allows the patient to remain visually engaged with the area of their body being scanned while still interacting with the virtual human assistant in the virtual world. The virtual environment mirrors the physical room, but the robotic ultrasound system itself is not visible; only the ultrasound probe becomes visible when it enters the passthrough window, allowing the patient to see it during the scan.
Within the virtual environment, the patient can also see a virtual representation of the ultrasound probe, which mirrors the position of the real probe. As the robotic arm drives the real ultrasound probe toward the patient, the virtual probe moves correspondingly, providing a visual cue that the probe is approaching. When the probe transitions into the passthrough window, the virtual probe seamlessly aligns with and transitions into the real probe, ensuring continuity and reassuring the patient that the procedure is progressing as expected.

In this visualization, the virtual human assistant interacts with the patient in the same way as in the AR setting. The assistant is positioned to give the impression of guiding the robotic ultrasound. The patient’s view of the scan area remains in the passthrough window.
This visualization preserves the patient’s sense of presence and control by allowing them to observe their body during the scan. At the same time, the virtual environment and virtual human assistant help to reduce anxiety with guidance and interaction throughout the procedure.

\subsection{Fully Immersive Virtual Reality Visualization}

The fully immersive VR visualization, as shown in Fig.~\ref{fig:teaser}c, places the patient entirely in a virtual environment, removing any direct visual connection to the real world. In this setting, the virtual environment mirrors the physical room, but the robotic ultrasound system is absent as well, reinforcing the illusion that the virtual assistant is in full control of the procedure. The patient perceives the virtual agent as the sole operator, guiding the ultrasound probe, which enhances the feeling of human involvement and control.

In this visualization, the patient sees a virtual replica of their own body. As the robot moves the ultrasound probe on the patient’s actual body, the virtual probe in the VR environment moves in sync on the virtual body. This synchronization ensures that the patient feels a consistent tactile and visual connection between what they see in the virtual space and what they feel in the real world.
This immersive environment is designed to reduce anxiety by eliminating the often sterile, impersonal atmosphere of traditional robotic procedures, replacing it with a comforting virtual experience where the patient feels human involvement and control throughout. By maintaining real-time synchronization with their physical body and providing a strong sense of presence through the virtual human assistant, the visualization offers a fully immersive alternative that fosters comfort and trust in the procedure.

\subsection{Registration of Virtual and Real Components}

\begin{figure}
    \centering
    \includegraphics[width=0.85\columnwidth]{figures/registration.png}
    \caption{\add{\textbf{Registration of Virtual and Real Robot Using Predefined Points.} Predefined points on the virtual robot are shown in green. We marked the corresponding points on the real robot with the HMD, shown in orange. The dashed lines represent the transformation matrix $W_T$ to be solved, which aligns the two point sets.}}
    \label{fig:registration}
\end{figure}


Accurate registration between the virtual and real-world components is crucial for maintaining spatial coherence across all visualizations. To achieve this, we utilize a set of predefined, ordered points on both the virtual and physical elements to perform 3D-to-3D registration, similar to the approaches used in previous works~\cite{song2022happy,yu2022duplicated}. For instance, several key points $P_{virtual}(i)$ are selected on the virtual robot, which correspond to matching points marked on the real robotic system $P_{real}(i)$. Similarly, for the virtual environment, corner points of the physical room are used to align the virtual representation of the room with the real one. The transformation matrix  $W_T$  between these point sets is computed to minimize the difference between their positions. This can be expressed mathematically as:

\begin{equation}
	W_T = \arg\min_{\widehat{W}_T} d\left(\widehat{W}_T P_{virtual}(i), P_{real}(i)\right)
\end{equation}
where $W_T$ is the estimated transformation matrix between the real and virtual spaces, and $d$ represents the distance between corresponding points in the two spaces. \add{The registration process is visually illustrated in Fig.~\ref{fig:registration}.} We further decompose this transformation into a rotation component $W_R$ and a translation component $W_t$, solving each sequentially using the Kabsch algorithm~\cite{kabsch1993automatic}.

After the initial alignment is achieved, we can leverage the spatial anchor feature offered by modern HMDs to maintain this alignment between sessions. This means the calibration procedure only needs to be performed once, unless fine-tuning or refinements are required in future sessions. By using spatial anchors, the system ensures persistent alignment, enhancing the immersive experience for the patient without the need for repeated calibrations.

\section{Experiments}
\label{sec:experiment}
\begin{table}[t]
\sisetup{table-format = 2.2, round-mode = places, round-precision=2, round-pad = false}
\caption{Comparison of test accuracy between the DUAL score method and existing techniques using ResNet-18 on CIFAR-10 and CIFAR-100 datasets. Training the model on the full dataset achieves an average test accuracy of 95.30\% on CIFAR-10 and 78.91\% on CIFAR-100. The best result in each pruning ratio is highlighted in bold.}
\setlength{\tabcolsep}{3.1pt}
\centering
\resizebox{\linewidth}{!}{
\begin{tabular}{l c@{}cc@{}cc@{}cc@{}cc@{}c | c@{}cc@{}cc@{}cc@{}cc@{}c}
    \toprule
    \textbf{Dataset ($\rightarrow$)} & \multicolumn{10}{c}{\textbf{CIFAR10}} & \multicolumn{10}{c}{\textbf{CIFAR100}}\\
    \cmidrule(lr){2-21}
    
    \textbf{Pruning Rate ($\rightarrow$)} & \multicolumn{2}{c}{\textbf{30\%}} & \multicolumn{2}{c}{\textbf{50\%}} & \multicolumn{2}{c}{\textbf{70\%}} & \multicolumn{2}{c}{\textbf{80\%}} & \multicolumn{2}{c}{\textbf{90\%}} & \multicolumn{2}{c}{\textbf{30\%}} & \multicolumn{2}{c}{\textbf{50\%}} & \multicolumn{2}{c}{\textbf{70\%}} & \multicolumn{2}{c}{\textbf{80\%}} & \multicolumn{2}{c}{\textbf{90\%}} \\
    \midrule
    
    \textbf{Random} & 94.39 & {\scriptsize \num{ +-0.2275}} & 93.20 & {\scriptsize \num{ +-0.1188}} & 90.47 & {\scriptsize \num{ +-0.1678}} & 88.28 & {\scriptsize \num{ +-0.1731}} & 83.74 & {\scriptsize \num{ +-0.2051}} & 75.15 & {\scriptsize \num{ +-0.2825}} & 71.68 & {\scriptsize \num{ +-0.3065}} & 64.86 & {\scriptsize \num{ +-0.3939}} & 59.23 & {\scriptsize \num{ +-0.6189}} & 45.09 & {\scriptsize \num{ +-1.2610}} \\
    
    \textbf{Entropy} & 93.48  & {\scriptsize \num{ +-0.0566 }} & 92.47 & {\scriptsize \num{ +-0.1707}} & 89.54 & {\scriptsize \num{ +-0.1753}} & 88.53 & {\scriptsize \num{ +-0.1864}} & 82.57 & {\scriptsize \num{ +-0.3645}} & 75.20 & {\scriptsize \num{ +-0.2486 }} & 70.90 & {\scriptsize \num{ +-0.3464}} & 61.70 & {\scriptsize \num{ +-0.4669}} & 56.24 & {\scriptsize \num{ +-0.5082}} & 42.25 & {\scriptsize \num{ +-0.3915}} \\
    
    \textbf{Forgetting} & 95.48  & {\scriptsize \num{ +-0.1412 }} & 94.94 & {\scriptsize \num{ +-0.2116}} & 89.55 & {\scriptsize \num{ +-0.6456}} & 75.47 & {\scriptsize \num{ +-1.2726}} & 46.64 & {\scriptsize \num{ +-1.9039}} & 77.52 & {\scriptsize \num{ +-0.2585 }} & 70.93 & {\scriptsize \num{ +-0.3700}} & 49.66 & {\scriptsize \num{ +-0.1962}} & 39.09 & {\scriptsize \num{ +-0.4065}} & 26.87 & {\scriptsize \num{ +-0.7318}} \\
    
    \textbf{EL2N} & 95.44  & {\scriptsize \num{ +-0.0628 }} & 95.19 & {\scriptsize \num{ +-0.1134}} & 91.62 & {\scriptsize \num{ +-0.1397}} & 74.70 & {\scriptsize \num{ +-0.4523}} & 38.74 & {\scriptsize \num{ +-0.7506}} & 77.13 & {\scriptsize \num{ +-0.2348}} & 68.98 & {\scriptsize \num{ +-0.3539}} & 34.59 & {\scriptsize \num{ +-0.4824}} & 19.52 & {\scriptsize \num{ +-0.7925}} & 8.89 & {\scriptsize \num{ +-0.2774}} \\
    
    \textbf{AUM} & 90.62  & {\scriptsize \num{ +-0.0921 }} & 87.26 & {\scriptsize \num{ +-0.1128}} & 81.28 & {\scriptsize \num{ +-0.2564}} & 76.58 & {\scriptsize \num{ +-0.3458}} & 67.88 & {\scriptsize \num{ +-0.5275}} & 74.34 & {\scriptsize \num{ +-0.1424 }} & 69.57 & {\scriptsize \num{ +-0.2100}} & 61.12 & {\scriptsize \num{ +-0.2004}} & 55.80 & {\scriptsize \num{ +-0.3256}} & 45.00 & {\scriptsize \num{ +-0.3694}} \\
    
    \textbf{Moderate} & 94.26  & {\scriptsize \num{ +-0.0904 }} & 92.79 & {\scriptsize \num{ +-0.0856}} & 90.45 & {\scriptsize \num{ +-0.2110}} & 88.90 & {\scriptsize \num{ +-0.1684}} & 85.52 & {\scriptsize \num{ +-0.2906}} & 75.20  & {\scriptsize \num{ +-0.2486 }} & 70.90 & {\scriptsize \num{ +-0.3464}} & 61.70 & {\scriptsize \num{ +-0.4699}} & 56.24 & {\scriptsize \num{ +-0.5082}} & 42.25 & {\scriptsize \num{ +-0.3915}} \\
    
    \textbf{Dyn-Unc} & 95.49  & {\scriptsize \num{ +-0.2061 }} & \textbf{95.35} & {\scriptsize \num{ +-0.1205}} & 91.78 & {\scriptsize \num{ +-0.6516}} & 83.32 & {\scriptsize \num{ +-0.9391}} & 59.67 & {\scriptsize \num{ +-1.7929}} & 77.67  & {\scriptsize \num{ +-0.1381}} & 74.23 & {\scriptsize \num{ +-0.2214}} & 64.30 & {\scriptsize \num{ +-0.1333}} & 55.01 & {\scriptsize \num{ +-0.5465}} & 34.57 & {\scriptsize \num{ +-0.6920}} \\
    
    \textbf{TDDS} & 94.42  & {\scriptsize \num{ +-0.1252 }} & 93.11 & {\scriptsize \num{ +-0.1377}} & 91.02 & {\scriptsize \num{ +-0.1908}} & 88.25 & {\scriptsize \num{ +-0.2385}} & 82.49 & {\scriptsize \num{ +-0.2799}} & 75.02  & {\scriptsize \num{ +-0.3682}} & 71.80 & {\scriptsize \num{ +-0.3323}} & 64.61 & {\scriptsize \num{ +-0.2431}} & 59.88 & {\scriptsize \num{ +-0.2110}} & 47.93 & {\scriptsize \num{ +-0.2147}} \\
    
    \textbf{CCS} & 95.31  & {\scriptsize \num{ +-0.2238}} & 95.06 & {\scriptsize \num{ +-0.1547}} & 92.68 & {\scriptsize \num{ +-0.1704}} & 91.25 & {\scriptsize \num{ +-0.2073}} & 85.92 & {\scriptsize \num{ +-0.3901}} & 77.15 & {\scriptsize \num{ +-0.2816}} & 73.83 & {\scriptsize \num{ +-0.2073}} & 68.65 & {\scriptsize \num{ +-0.3130}} & 64.06 & {\scriptsize \num{ +-0.2084}} & 54.23 & {\scriptsize \num{ +-0.4813}} \\
    
    \textbf{D2} & 94.13  & {\scriptsize \num{ +-0.2033}} & 93.26 & {\scriptsize \num{ +-0.1623}} & 92.34 & {\scriptsize \num{ +-0.1786}} & 90.38 & {\scriptsize \num{ +-0.3376}} & 86.11 & {\scriptsize \num{ +-0.2072}} & 76.47  & {\scriptsize \num{ +-0.2934}} & 73.88 & {\scriptsize \num{ +-0.2780}} & 62.99 & {\scriptsize \num{ +-0.2775}} & 61.48 & {\scriptsize \num{ +-0.3361}} & 50.14 & {\scriptsize \num{ +-0.8951}} \\
    
    \midrule
    
    \textbf{DUAL} & 95.25  & {\scriptsize \num{ +-0.17 }} & 94.95 & {\scriptsize \num{ +-0.22 }} & 91.75 & {\scriptsize \num{ +-0.98 }} & 82.02 & {\scriptsize \num{ +-1.85 }} & 54.95 & {\scriptsize \num{ +-0.42 }} & 77.43 & {\scriptsize \num{ +-0.18 }} & 74.62 & {\scriptsize \num{ +-0.47 }} & 66.41 & {\scriptsize \num{ +-0.52 }} & 56.57 & {\scriptsize \num{ +-0.57 }} & 34.38 & {\scriptsize \num{ +-1.39 }} \\
    
    \textbf{DUAL+$\beta$ sampling} & \textbf{95.51}  & {\scriptsize \num{ +-0.0634}} & 95.23 & {\scriptsize \num{ +-0.0796}} & \textbf{93.04} & {\scriptsize \num{ +-0.4282}} & \textbf{91.42} & {\scriptsize \num{ +-0.352}} & \textbf{87.09} & {\scriptsize \num{ +-0.3599}} & \textbf{77.86}  & {\scriptsize \num{ +-0.1186}} & \textbf{74.66} & {\scriptsize \num{ +-0.1173}} & \textbf{69.25} & {\scriptsize \num{ +-0.2156}} & \textbf{64.76} & {\scriptsize \num{ +-0.2272}} & \textbf{54.54} & {\scriptsize \num{ +-0.0884}} \\
    
    \bottomrule
\end{tabular}
}
\label{tbl:main_cifar}
\end{table}

\subsection{Experimental Settings}
We assessed the performance of our proposed method in three key scenarios: image classification, image classification with noisy labels and corrupted images. In addition, we validate cross-architecture generalization on three-layer CNN, VGG-16~\citep{simonyan2015deepconvolutionalnetworkslargescale}, ResNet-18 and ResNet-50~\citep{he2015deepresiduallearningimage}.

\textbf{Hyperparameters.}
For training CIFAR-10 and CIFAR-100, we train ResNet-18 for 200 epochs with a batch size of 128. SGD optimizer with momentum of 0.9 and weight decay of 0.0005 is used. The learning rate is initialized as 0.1 and decays with the cosine annealing scheduler. As \citet{zhang2024spanning} show that smaller batch size boosts performance at high pruning rates, we also halved the batch size for 80\% pruning, and for 90\% we reduced it to one-fourth. For ImageNet-1k, ResNet-34 is trained for 90 epochs with a batch size of 256 across all pruning ratios. An SGD optimizer with a momentum of 0.9, a weight decay of 0.0001, and an initial learning rate of 0.1 is used, combined with a cosine annealing scheduler.

\textbf{Baselines.} The baselines considered in this study are listed as follows\footnote{Infomax~\citep{tan2025data} was excluded as it employs different base hyperparameters in the original paper compared to other baselines and does not provide publicly available code. See Appendix~\ref{Appendix_Technical_Details_of_Baselines} for more discussion.}: (1) Random; (2) Entropy~\citep{coleman2020selectionproxyefficientdata}; (3) Forgetting~\citep{toneva2018empirical}; (4) EL2N~\citep{gordon2021data}; (5) AUM~\citep{pleiss2020identifyingmislabeleddatausing}; (6) Moderate~\citep{xia2022moderate}; (7) Dyn-Unc~\citep{he2024large}; (8) TDDS~\citep{zhang2024spanning}; (9) CCS\mbox{~\citep{zheng2022coverage}}; and (10) $\mathbb{D}^2$~\citep{maharana2023d2}. To ensure a fair comparison, all methods were trained with the same base hyperparameters for training, and the best hyperparameters reported in their respective original works for scoring. Technical details are provided in the \cref{Appendix_Technical_Details_of_Baselines}. 


\begin{wrapfigure}[16]{R}{0.5\textwidth}
\vspace{-15pt}
    \begin{center}
        \includegraphics[width=0.9\linewidth]{Figures/total_computation_time.pdf}
    \end{center}
    \vspace{-8pt}
    \caption{Comparison in total time spent on CIFAR datasets.}
    \label{fig:total_time_consumed}
    % \vspace{-10pt}
\end{wrapfigure}
\vspace{3pt}
\subsection{Image Classification Benchmarks}
\cref{tbl:main_cifar} presents the test accuracy for image classification results on CIFAR-10 and CIFAR-100. Our pruning method consistently outperforms other baselines, particularly when combined with Beta sampling. While the DUAL score exhibits competitive performance in lower pruning ratios, its accuracy degrades with more aggressive pruning. Our Beta sampling effectively mitigates this performance drop.

Notably, the DUAL score only requires training a single model for \emph{only 30 epochs}, significantly reducing the computational cost. In contrast, the second-best methods, Dyn-Unc and CCS, rely on scores computed over a full 200-epoch training cycle, making them considerably less efficient. Even considering subset selection, score computation, and subset training, the total time remains less than a single full training run, as shown in Figure~\ref{fig:total_time_consumed}. Specifically, on CIFAR-10, our method achieves lossless pruning up to a 50\% pruning ratio while saving 35.5\% of total training time. 

\begin{wrapfigure}[20]{L}{0.5\textwidth}
\vspace{2mm}
    \centering
    \captionof{table}{\label{tab:imagenet_results} Comparison of test accuracy on ImageNet-1k. The model trained with the full dataset achieves 73.1\% test accuracy. The best result in each pruning ratio is highlighted in bold.}
    \small
    \setlength{\tabcolsep}{3pt}
    \begin{tabular}{lccccc}
        \toprule
        \textbf{Pruning Rate} & \textbf{30\%} & \textbf{50\%} & \textbf{70\%} & \textbf{80\%} & \textbf{90\%} \\
        \midrule
        \textbf{Random} & 72.2 & 70.3 & 66.7 & 62.5 & 52.3 \\
        \textbf{Entropy} & 72.3 & 70.8 & 64.0 & 55.8 & 39.0 \\
        \textbf{Forgetting} & 72.6 & 70.9 & 66.5 & 62.9 & 52.3 \\
        \textbf{EL2N} & 72.2 & 67.2 & 48.8 & 31.2 & 12.9 \\
        \textbf{AUM} & 72.5 & 66.6 & 40.4 & 21.1 & 9.9 \\
        \textbf{Moderate} & 72.0 & 70.3 & 65.9 & 61.3 & 52.1 \\
        \textbf{Dyn-Unc} & 70.9 & 68.3 & 63.5 & 59.1 & 49.0 \\
        \textbf{TDDS} & 70.5 & 66.8 & 59.4 & 54.4 & 46.0 \\
        \textbf{CCS} & 72.3 & 70.5 & 67.8 & 64.5 & 57.3 \\
        \textbf{D2} & 72.9 & 71.8 & 68.1 & 65.9 & 55.6 \\
        \midrule
        \textbf{DUAL} & 72.8 & 71.5 & 68.6 & 64.7 & 53.1 \\
        \textbf{DUAL+$\beta$ sampling} & \textbf{73.3} & \textbf{72.3} & \textbf{69.4} & \textbf{66.5} & \textbf{60.0} \\
        \bottomrule
    \end{tabular}
\end{wrapfigure}
\vspace{2pt}
We also evaluate our pruning method on the large-scale dataset, ImageNet-1k. The DUAL score is computed during training, specifically at epoch 60, which is 33\% earlier than the original train epoch used to compute scores for other baseline methods. As shown in \cref{tab:imagenet_results}, Dyn-Unc performs worse than random pruning across all pruning ratios, and we attribute this undesirable performance to its limited total training epochs (only 90), which is insufficient for Dyn-Unc to fully capture the training dynamics of each sample. In contrast, our DUAL score, combined with Beta sampling, outperforms all competitors while requiring the least computational cost. The DUAL score's ability to consider both training dynamics and the difficulty of examples enables it to effectively identify uncertain samples early in the training process, even when training dynamics are limited. Remarkably, for 90\% pruned Imagenet-1K, it maintains a test accuracy of 60.0\%, surpassing the previous state-of-the-art (SOTA) by a large margin.
\vspace{15pt}
\subsection{Experiments under More Realistic Scenarios}
\subsubsection{Label Noise and Image Corruption}
Data affected by label noise or image corruption are difficult and unnecessary samples that hinder model learning and degrade generalization performance. Therefore, filtering out these samples through data pruning is crucial. Most data pruning methods, however, either focus solely on selecting difficult samples based on example difficulty~\citep{gordon2021data, pleiss2020identifyingmislabeleddatausing, coleman2020selectionproxyefficientdata} or prioritize dataset diversity~\citep{zheng2022coverage, xia2022moderate}, making them unsuitable for effectively pruning such noisy and corrupted samples.

In contrast, methods that select uncertain samples while considering training dynamics, such as Forgetting \citep{toneva2018empirical} and Dyn-Unc~\citep{he2024large}, demonstrate robustness by pruning both the hardest and easiest samples, ultimately improving generalization performance, as illustrated in Figure~\ref{fig:label_noise_20_ratio}. However, since noisy samples tend to be memorized after useful samples are learned~\citep{arpit2017closer, jiang2020characterizing}, there is a possibility that those noisy samples may still be treated as uncertain in the later stages of training and thus be included in the selected subset.

The DUAL score aims to identify high-uncertainty samples early in training by considering both training dynamics and example difficulty. Noisy data, typically under-learned compared to other challenging samples during this phase, exhibit lower uncertainty (Figure~\ref{fig:label_noise_visualization}, Appendix~\ref{Appendix_labelnoise_experiments}). Consequently, our method effectively prunes these noisy samples.

To verify this, we evaluate our method by introducing a specific proportion of symmetric label noise~\citep{patrini2017making, xia2020robust, li2022selective} and applying five different types of image corruptions~\citep{wang2018iterative, hendrycks2019benchmarking, xia2021instance}. We use CIFAR-100 with ResNet-18 and Tiny-ImageNet with ResNet-34 for these experiments. On CIFAR-100, we test label noise and image corruption ratios of 20\%, 30\%, and 40\%. For Tiny-ImageNet, we use a 20\% ratio of label noise and image corruption. We prune the label noise-added dataset using a model trained for 50 epochs and the image-corrupted dataset with a model trained for 30 epochs using DUAL pruning—both significantly lower than the 200 epochs used by other methods. For detailed experimental settings, please refer to Appendix~\ref{Appendix_Technical_Details_of_Ours}.
\begin{figure*}[t]
    \centering
    \begin{subfigure}{0.325\textwidth}
        \centering
        \includegraphics[width=\textwidth]{Figures/label_noise20_ratio.pdf}
        \caption{\label{fig:label_noise_20_ratio}Pruned mislabeled data ratio}
    \end{subfigure}
    \hfill
    \begin{subfigure}{0.325\textwidth}
        \centering
        \includegraphics[width=\textwidth]{Figures/label_noise20_testacc.pdf}
        \caption{\label{fig:label_noise_20_testacc}Test accuracy under label noise}
    \end{subfigure}
    \hfill
    \begin{subfigure}{0.325\textwidth}
        \centering
        \includegraphics[width=\textwidth]{Figures/imagecorruption_20_testacc.pdf}
    \caption{\label{fig:imagecorruption_20_testacc}Test accuracy under image noise}
    \end{subfigure}
    \caption{\label{fig:labelnoise_main}The left figure shows the ratio of pruned mislabeled data under 20\% label noise on CIFAR-100 trained with ResNet-18. When label noise is 20\%, the optimal value (black dashed line) corresponds to pruning 100\% of mislabeled data at a 20\% pruning ratio. The middle and right figures depict test accuracy under 20\% label noise and 20\% image corruption, respectively. Our method effectively prunes mislabeled data near the optimal value while maintaining strong generalization performance. Results are averaged over five random seeds.}
    \vspace{-10pt}
\end{figure*}
As shown in Figure~\ref{fig:labelnoise_main}, the left plot demonstrates that DUAL pruning effectively removes mislabeled data at a ratio close to the optimal. Notably, when the pruning ratio is 10\%, nearly \emph{all pruned samples are mislabeled data}.
Consequently, as observed in Figure~\ref{fig:label_noise_20_testacc}, DUAL pruning leads to improved test accuracy compared to training on the full dataset, even up to a pruning ratio of 70\%. At lower pruning ratios, performance improves as mislabeled data are effectively removed, highlighting the advantage of our approach in handling label noise.
Similarly, for image corruption, our method prunes more corrupted data across all corruption rates compared to other methods, as shown in Figure~\ref{fig:imagecorruption_203040_ratio},~\ref{fig:imagecorruption_all} in Appendix~\ref{Appendix_imagecorruption_experiments}. As a result, this leads to higher test accuracy, as demonstrated in Figure~\ref{fig:imagecorruption_20_testacc}. 

Detailed results, including exact numerical values for different corruption rates and Tiny-ImageNet experiments, can be found in Appendix~\ref{Appendix_labelnoise_experiments} and \ref{Appendix_imagecorruption_experiments}.  Across all experiments, DUAL pruning consistently shows \emph{strong noise robustness} and outperforms other methods by a substantial margin.

\subsubsection{Cross-Architecture Generalization}
We also evaluate the ability to transfer scores across various model architectures. To be specific, if we can get high-quality example scores for pruning by using a simpler architecture than one for the training, our DUAL pruning would become even more efficient in time and computational cost. Therefore, we focus on the cross-architecture generalization from relatively small networks to larger ones with three-layer CNN, VGG-16, ResNet-18, and ResNet-50. Competitors are selected from each categorized group of the pruning approach: EL2N from difficulty-based, Dyn-Unc from uncertainty-based, and CCS from the geometry-based group. 

For instance, we get training dynamics from the ResNet-18 and then calculate the example scores. Then, we prune samples using scores calculated from ResNet-18, and train selected subsets on ResNet-50. The result with ResNet-18 and ResNet-50 is described in Table~\ref{tab:04-cross-arch-r18-r50}. Surprisingly, the coreset shows competitive performance to the baseline, where the baseline refers to the test accuracy after training a coreset constructed based on the score calculated from ResNet-50.
For all pruning cases, we observe that our methods reveal the highest performances. Specifically, when we prune 70\% and 90\% of the original dataset, we find that all other methods fail, showing worse test accuracies than random pruning. 


\begin{table}[t]
\caption{Cross-architecture generalization performance on CIFAR-100 from ResNet-18 to ResNet-50. We report an average of five runs. `R50 $\rightarrow$ R50' stands for score computation on ResNet-50, as a baseline.}
\label{tab:04-cross-arch-r18-r50}
\setlength{\tabcolsep}{3.1pt}
\centering
\begin{tabular}{lcccc}
    \toprule
    \multicolumn{1}{c}{} & \multicolumn{4}{c}{ResNet-18 $\rightarrow$ ResNet-50} \\
    \midrule
    Pruning Rate ($\rightarrow$) & 30\% & 50\% & 70\% & 90\% \\
    \hline
    \midrule
    Random & 74.47 \scriptsize{$\pm 0.67$} & 70.09 \scriptsize{$\pm 0.42$} & 60.06 \scriptsize{$\pm 0.99$} & 41.91 \scriptsize{$\pm 4.32 $} \\
    EL2N  & 76.42 \scriptsize{$\pm 1.00$} & 69.14 \scriptsize{$\pm 1.00$} & 45.16 \scriptsize{$\pm 3.21$} & 19.63 \scriptsize{$\pm 1.15 $} \\
    Dyn-Unc  & 77.31 \scriptsize{$\pm 0.34$} & 72.12 \scriptsize{$\pm 0.68$} & 59.38 \scriptsize{$\pm 2.35$} & 31.74 \scriptsize{$\pm 2.31 $} \\
    CCS  & 74.78 \scriptsize{$\pm 0.66$} & 69.98 \scriptsize{$\pm 1.18$} & 59.75 \scriptsize{$\pm 1.41$} & 41.54 \scriptsize{$\pm 3.94 $} \\
    \midrule
    DUAL   & \textbf{78.03} \scriptsize{$\pm 0.83$} & 72.82 \scriptsize{$\pm 1.46$} & 63.08 \scriptsize{$\pm 2.45$} & 33.65 \scriptsize{$\pm 2.92 $} \\
    DUAL +$\beta$  & 77.82 \scriptsize{$\pm 0.65$} & \textbf{73.98} \scriptsize{$\pm 0.62$} & \textbf{66.36} \scriptsize{$\pm 1.66$} & \textbf{49.90} \scriptsize{$\pm 2.56 $} \\
    \hline
    \midrule
    DUAL (R50$\rightarrow$R50)  & 77.82 \scriptsize{$\pm 0.64$} & 73.66 \scriptsize{$\pm 0.85$} & 52.12 \scriptsize{$\pm 2.73 $} & 26.13 \scriptsize{$\pm 1.96 $} \\
    DUAL (R50$\rightarrow$R50)+$\beta$  & 77.57 \scriptsize{$\pm 0.23$} & 73.44 \scriptsize{$\pm 0.87$} & 65.17 \scriptsize{$\pm 0.96$} & 47.63 \scriptsize{$\pm 2.47 $} \\
    \bottomrule
    \vspace{-10pt}
\end{tabular}
\end{table}

We also test the cross-architecture generalization performance with three-layer CNN, VGG-16, and ResNet-18 in Appendix~\ref{Appendix_cross_architecture}. 
Even for a simple model like three-layer CNN, we see our methods show consistent performance, as can be seen in Table~\ref{tab:cnn-to-resnet18} in Appendix~\ref{Appendix_cross_architecture}. This observation gives rise to an opportunity to develop some small proxy networks to get example difficulty with less computational cost. 
Transfer across models with similar capacities, $e.g.$ from VGG-16 to ResNet-18 and vice versa, also supports the verification of cross-architecture compatibility.   

\subsection{Ablation Studies}
\paragraph{Hyperparameter Analysis.}
\begin{wrapfigure}[10]{r}{0.5\textwidth}
\vspace{-5mm}
    \begin{center}
        \includegraphics[width=0.92\linewidth]{Figures/TestAcc_TC_varying.pdf}
    \end{center}
    \vspace{-5mm}
    \caption{\textbf{Left}: Varying T with $J=10$ and $c_\gD=4$. \textbf{Right}: Varying $c_\gD$ with $T=30$ and $J=10$.}
    \label{fig:TC_varying}
\end{wrapfigure}
Here, we investigate the robustness of our hyperparameters, $T$, $J$, and $c_\gD$. We fix $J$ across all experiments, as it has minimal impact on selection, indicating its robustness (Fig~\ref{fig:J_varying}, Appendix~\ref{Appendix_Experiments}). In Figure~\ref{fig:TC_varying}, we assess the robustness of $T$ by varying it from 20 to 200 on CIFAR-100. We find that while $T$ remains highly robust in earlier epochs, increasing $T$ degrades generalization performance. This is expected, as larger $T$ overemphasizes difficult samples due to our difficulty-aware selection.  Thus, pruning in earlier epochs (from 30 to 50) proves to be more effective and robust. For the $c_\gD$, we vary it from 3 to 7 and find robustness, especially in the aggressive pruning regime. All results are averaged across three runs.

\paragraph{Beta Sampling Analysis.}
Next, we study the impact of our proposed pruning-ratio-adaptive Beta sampling on existing score metrics. We apply our Beta sampling strategy to prior score-based methods, including Forgetting, EL2N, and Dyn-Unc, using the CIFAR10 and CIFAR100 datasets. By comparing our sampling approach with vanilla threshold pruning, which selects only the highest-scoring samples, we observe that prior score-based methods become remarkably comparable to random pruning after Beta sampling is adjusted (see \cref{tab:abl_beta_cifar10_100_90_main}).

\begin{table*}[t]
\caption{Comparison on CIFAR-10 and CIFAR-100 for $90\%$ pruning rate. 
We report average accuracy with five runs. The best performance is in bold in each column.}
\label{tab:abl_beta_cifar10_100_90_main}
\setlength{\tabcolsep}{4.5pt}
\centering
\begin{tabular}{lcc|cc}
    \toprule
    \multicolumn{1}{c}{} & \multicolumn{2}{c}{CIFAR-10} & \multicolumn{2}{c}{CIFAR-100} \\
    \midrule
    Method & Thresholding & $\beta$-Sampling & Thresholding & $\beta$-Sampling \\
    \midrule
    Random &  \textbf{83.74}\scriptsize{$\pm$0.21} & 83.31 (-0.43) \scriptsize{$\pm$0.14} & \textbf{45.09} \scriptsize{$\pm$1.26} & 51.76 (+6.67) \scriptsize{$\pm$0.25} \\
    EL2N &  38.74 \scriptsize{$\pm$0.75} & 87.00 (+48.26) \scriptsize{$\pm$0.45} & 8.89 \scriptsize{$\pm$0.28} & 53.97 (+45.08)  \scriptsize{$\pm$0.63}  \\
    Forgetting &  46.64 \scriptsize{$\pm$1.90} & 85.67 (+39.03) \scriptsize{$\pm$0.13} & 26.87 \scriptsize{$\pm$0.73} & 52.40 (+25.53) \scriptsize{$\pm$0.43} \\
    Dyn-Unc &  59.67 \scriptsize{$\pm$1.79} & 85.33 (+32.14) \scriptsize{$\pm$0.20} & 34.57 \scriptsize{$\pm$0.69} & 51.85 (+17.28) \scriptsize{$\pm$0.35}   \\
    \midrule
    Ours & 54.95 \scriptsize{$\pm$0.42} & \textbf{87.09} (+31.51) \scriptsize{$\pm$0.36} & 34.28 \scriptsize{$\pm$1.39} & \textbf{54.54} (+20.26) \scriptsize{$\pm$0.09}  \\
    \bottomrule
\end{tabular}
\end{table*}


Even adapted for random pruning, our Beta sampling proves to perform well. Notably, EL2N, which performs poorly on its own, becomes significantly more effective when combined with our sampling method. Similar improvements are also seen with Forgetting and Dyn-Unc scores. This is because our proposed Beta sampling enhances the diversity of selected samples in turn, especially when used with example difficulty-based methods. More results conducted for 80\% pruning cases are included in the \cref{Appendix_beta_samapling}.
    

\paragraph{Additional Analysis.}
In addition to the main results presented in this paper, we also conducted various experiments to further explore the effectiveness of our method. These additional results include an analysis of coreset performance under a time budget ($e.g.$ other score metrics are also computed by using training dynamics up to epoch 30) and Spearman rank correlation was calculated between individual scores and the averaged score across five runs to assess the consistency of scores for each sample. Furthermore, additional results in extreme cases, 30\% and 40\% of label noise and image corruption can be found in \cref{Appendix_Experiments}.
\subsection{Effect Game Play on Self Reported Discernment of Misinformation}
\subsubsection {Descriptive Statistics}
Preliminary data analysis reveals significant violations of the normality assumption in pre-tested VOI and NMLS scales. Considering the rather small dataset, we decided to proceed with a non-parametric repeated measures approach (Related-Samples Wilcoxon Signed Rank Test to the results of the scales' pre- and post-evaluations).

\textbf{Effect of the educational game on the verification practicies (VOI-7)}
%To determine if the game practice significantly changed the intention to apply the direct and indirect verification practices, we ran Related-Samples Wilcoxon Signed Rank Test. 
The test revealed significant differences between pre and post-gaming VOI scores (N = 42, Z = 4.361, p < .001). The results suggested that the game positively affected the repertoire of used practices and/or the perceived will to use these practices. 

\textbf{Effect of the educational game on Media Literacy}
To measure the effects of the game on Media Literacy, we first ran the Related-Samples Wilcoxon Signed Rank Test on the full scale. Then, to determine which components of Media Literacy were most affected by the game, we conducted separate subscale tests to analyze changes in each of the four subdomains of Media Literacy. The results demonstrated significant differences in Media Literacy scale results (N = 47, Z = 3.083, p = .002). The analysis revealed the following differences: the game significantly improved both functional consuming   Z = 2.064, p = .039 and critical consuming Z = 3.344, p <.001 ), but not the functional prosuming Z = .435. p = .664 and critical prosuming Z = 1.868, p = .062. Therefore, the results suggest the game improves Media Literacy in the domains connected to understanding the content of the media and being able to critically evaluate the content of the media; however, it has not significantly improved the ability to produce media content which can be influential to others and convey author's ideas \cite{koc2016development}.

\textbf{Effect of the educational game on self-efficacy towards misinformation}
We did not find significant differences in self-efficacy between pre and post-game measurements (Z = .743, p = .458). 

\textbf{Effect of the educational game on the ability to recognise misinformation
}
We took the naive approach to calculate the MIST score, taking it as the sum of the right answers on all 20 questions \cite{maertens2024misinformation}. The results showed that participating in the game significantly improved the participant's ability to discriminate between fake and real news (Z = 2.702, p = .007) 

%A total of 48 participants registered for our game study and completed the informed consent process and pre-survey. These participants were matched into 24 pairs and scheduled for the gameplay experiment. 
%After completing the gameplay session, participants were given a short break before completing a post-survey and participating in a follow-up interview. All 24 pairs successfully completed the gameplay experiment. 


%However, data from one participant was removed due to [reason], leaving 47 participants whose interview responses and gameplay logs (including the outputs of both players and the LLM responses across four rounds) were included in the qualitative analysis for this paper.
\subsection{Qualitative Results}
In this section, we present participants' perceptions and understanding of misinformation, including how they learned to generate it and identify it through gameplay, as well as the strategies they developed for debunking misinformation. %to answer RQ1. 

%We employed a combined inductive-deductive approach to analyze the qualitative interview transcripts and gameplay logs\cite{kuckartz2019analyzing}. This approach ensured a comprehensive understanding of the gameplay experience. Our primary objectives were to understand how participants perceived and understood misinformation through the game, how they learned to distinguish and apply debunking strategies during gameplay, and how interactions with other players influenced their behavior and learning. The analysis process began with inductive coding. Two researchers independently coded a subset of the data, identified themes, and then discussed and reconciled any coding discrepancies, iterating on the coding system as needed. Once the coding system was established, the two researchers independently coded the full dataset. A third researcher then reviewed the coded data, and any differences in interpretation were discussed until a consensus was reached.


\subsubsection{Raise Awareness of Misinformation though gameplay}
Participants reported an increased awareness of misinformation through both the confrontational mechanics of the game and the news narratives presented within it. The game enhanced players' awareness of misinformation in two key ways.

\textbf{In-game News as a Reflection of Real-world Misinformation} 23 out of 47 participants noted that the in-game news mirrored real-world situations, thereby heightening their awareness of the characteristics of misinformation. A common observation was that news is rarely entirely true or false; instead, it often presents a mixture of both. This complexity makes genuine misinformation more challenging to detect. As one participant stated:
\begin{quote}
\textbf{N23}:
Nowadays, news often presents both positive and negative sides of a story, so I believe this game reflects real-life situations quite accurately.
\end{quote}
However, some participants acknowledged that the misinformation in the game appeared more overtly false compared to the more subtle nature of misinformation encountered in real life.

\textbf{The Competition Game Mechanics positively influence learning:} The PvP mechanics enhanced learning by requiring players to identify flaws in each other’s messages and respond effectively to achieve success. This repeated process helped deepen their understanding and sharpen their skills in distinguishing misinformation. As one participant noted:
\begin{quote}
\textbf{N20}:
In the process, I was able to see first-hand some of the flaws in the information (posted by others) and some of the claims made in an attempt to deceive people. And then it's also more accurate for me to judge the misinformation afterwards.
\end{quote}
Participants also learned from observing their opponents. For example, N22, who played the role of a debunker, noticed how the misinformation creator crafted and disseminated false information to persuade others:
\begin{quote}
    \textbf{N22}:
    When I was playing this round, I didn't score as high as my opponent, so I knew what they were saying and how they were letting the false information spread. Next time I come across such information, I will know that it is false.
\end{quote}
%In analyzing the gamelog, we identified several key strategies employed by Player 1 (influencer) to generate misinformation based on unverified evidence and rumors. For example, in Round 4, Player 1 spread rumors about a doctor’s death, which directly triggered public panic (N1). Many participants demonstrated a strong ability for emotional manipulation, frequently inciting fear and sympathy. Across all rounds (N6, N8), Player 1 used emotional appeals and personal stories to enhance the perceived credibility of the misinformation. Exaggerated claims, such as asserting the R drug’s "100\% effectiveness," were also employed to mislead (N36). Additionally, fostering cultural pride and heritage proved to be a powerful emotional tactic to build trust in the misinformation (N4). Player 1 further increased the complexity of the misinformation by incorporating celebrity endorsements(N9). A key example of this strategy was Player 1’s effective use of social media to amplify emotional narratives, which led personas like Emily and Maria to place strong trust in their misinformation (N21).---I took this part to emotion. 

%We observed that Player 2 employed a rational, evidence-based approach to counter misinformation, relying heavily on scientific facts and reasoning. For example, in Round 2, Player 2 (N2) effectively challenged Player 1’s misleading claims by referencing medical expert opinions and providing scientific explanations for patient deaths, underscoring a strong evidence-driven approach. Similarly, Player 2(N3) questioned the lack of clinical data supporting the R medicine and pointed out the potential commercial motivations. As the emotional complexity of the misinformation increased, Player 2's information became more structured and precise, with deeper critiques grounded in both scientific and ethical considerations (N3). In addition, Player 2 (N7) incorporated underlying factual evidence, such as economic motivations, to further strengthen their counterarguments. In some instances, Player 2 (N10) shifted focus from purely scientific critiques to address regulatory frameworks, leveraging an official perspective to challenge the spread of misinformation.

%During the game, some players encountered challenging narrative contexts. In Round 3 where Player 1 was at disadvantage, they responded by adopting a positive storytelling strategy, portraying characters in an optimistic and proactive manner to build trust in the misinformation. Instead of using fake scientific evidence, Player 1 used emotional manipulation, emphasizing values such as "cultural tradition" and "ancestral wisdom" to downplay the role of science and overcome the unfavorable situation.

%Player 2 encountered similar difficulties when anecdotal evidence was used to spread misinformation, particularly in Rounds 2 and 4. In these cases, Player 2 had to maintain logical reasoning despite the strong emotional appeals. Some players countered misinformation by clarifying facts and directly addressing conspiracy theories, while others employed a more cautious approach, appealing public to wait for more evidence to prove. This strategy allowed Player 2 to weaken Player 1’s influence.


\subsubsection{Identifying Misinformation through Source Evaluation:}
After the gameplay sessions, participants reported increased awareness of the varying credibility of different information sources. The game helped them to realize that producers of misinformation often seeks to enhance credibility by deliberately referencing authoritative organizations. One participant reflected on this realization:
\begin{quote}
    \textbf{N40}:
    After playing the game, I found that it was indeed the same as in the experiment. Some news did mention authoritative organizations as references, but I could tell that this was intentional…. (The game) may make my suspicions more valid.
\end{quote}
In addition, participants acknowledged that information from seemingly authoritative sources is not always reliable. It requires information to be cross-checked from multiple sources to verify its authenticity. As one participant noted: 
\begin{quote}
\textbf{N13}:
    I used to trust information from authoritative sources and reputable publications. But the game showed me that even these can be false, as my opponents used fake evidence from supposed authorities.
\end{quote}

\subsubsection{Identifying Misinformation through Emotional Manipulation Tactics}
Participants learned various tactics for both creating and debunking misinformation through the game’s instructions and their in-game experiences. A particularly commonly identified tactic was emotional manipulation, which was noted by 35 out of 47 participants (18 misinformation creators and 17 debunkers). By analyzing the game logs, we identified common emotional manipulation strategies used in the game. Most players crafted messages designed to evoke anxiety and fear, while some also attempted to generate feelings of hope. For example, in Round 4, Player 1 spread rumors about a doctor's death, which directly incited public panic (N1). Across all rounds, Player 1 frequently used emotional appeals and personal stories to enhance the perceived credibility of the misinformation (N6, N8). Additionally, invoking cultural pride and heritage was a powerful tactic used to build trust in misinformation (N4), while celebrity endorsements further increased the complexity and believability of the misinformation (N9). As illustrated in \autoref{fig:emotion}, players significantly increased public trust in their information by using emotionally charged language (N31). These strategies align closely with the characteristics of misinformation, where emotional appeals are commonly used to influence public opinion \cite{chuai2022really}. 
%By analysing gamelog, we identified the common emotional manipulation tactics player used in game. Most players aimed to craft messages that evoked anxiety and fear, while some also attempted to generate feelings of hope. For example, in Round 4, Player 1 spread rumors about a doctor’s death, which directly triggered public panic (N1). Across all rounds, Player 1 used emotional appeals and personal stories to enhance the perceived credibility of the misinformation (N6, N8). Additionally, fostering cultural pride and heritage proved to be a powerful emotional tactic to build trust in the misinformation (N4). Player 1 further increased the complexity of the misinformation by incorporating celebrity endorsements(N9). A example as shown in Figure 7, players were able to significantly increase public trust in their information by using emotionally charged words(N31) . These strategies closely align with the characteristics of misinformation, where emotional appeals are often used to sway public opinion\cite{chuai2022really}.  Through their use of emotional manipulation within the game, participants became more attuned to recognizing these tactics in real life. 

In follow-up interviews, many reported an increased awareness of the emotional undertones embedded in messages, which made them more suspicious of such content. They learned to identify emotionally charged language, such as messages that were \textit{“overly positive,”} \textit{“overly exaggerated,”} or \textit{“overly one-sided about an overly positive point of view.”}

Interestingly, when playing the role of the debunker, participant can reflected on the emotionally inflammatory language used by the misinformation creator and helped them develop a more clear strategy for addressing misinformation. This approach involved separating the factual content of a message from its emotional manipulations and focusing more on the factual aspects, as one participants explained 
\begin{quote}
    \textbf{N2}:
    When reading a story, it is important to put more focus on what is going on at the factual level rather than what is going on above the author's own views and emotions. 
\end{quote}
\begin{figure}
    \centering
    \includegraphics[width=1\linewidth]{figs/results_emotion_RR.jpg}
    \caption{An example of how emotional manipulation tactics from N31 effectively works on public opinion.}
    \label{fig:emotion}
\end{figure}

%Gameplay about emotion
%example of hope:
%“他们真的在尽最大努力。每一次实验、每一组数据都是为了确保我们每个人都能得到更安全、更有效的治疗。他们不仅在做研究,也在为我们的亲人和朋友而战。
%这些科学家也是有家室的人。他们比任何人都清楚,每一天都弥足珍贵。他们的辛勤工作不仅是为了科学的进步,也是为了那些应该享受生活的年轻人,为了我们宝贵的孩子,为了我们的父母。他们理应安享晚年。作为医务工作者,我们要带头给大家信心。我们最清楚,科学需要时间,但每一次进步都是朝着光明的方向。我们都看到了艾米医生团队的专业和坚持。很快,他们就会带来更多的好消息,届时,我们就能更有效地对抗这种病毒了。让我们一起欢呼,支持他们的工作,因为美好的日子就在前方。无论您是年轻人、中年人还是老年人,这项工作都是为了我们所有人。我们的团结是我们最大的力量。让我们共同迈向更健康、更安全的未来!(0819_E组-N19 N20, Pos. 225-228)”
%example of combining emotion and hard science
%Round3 p2:
%[2024-08-20 11:02:49] Player 2 input: R drugs can cure inflammation? Shameful to use data to falsify! Exploiting national cultural pride is even more shameful!
%According to incomplete statistics, nearly more than two-thirds of the European population has died from inflammation in the last 500 years of history, and the R-drug has not produced a definitive positive feedback on historical inflammation cures since its introduction in research. At the same time, taking advantage of the public's recognition of the anecdotal medical culture and its legacy, Dr Amy's medical organisation and its financial consortium have ignored the fact that Eternal Wellness has not conducted any new or further research or testing of the R-drug, and that the R-drug provides only temporary symptomatic relief, is not a complete cure for Zinc, and that the adverse effects of the R-drug are not yet known. What are the intentions of Eternal Wellness? (8.20 L組 N25N26, Pos. 175-177)

%example of fear (N33N34)
%N33 attempted to instill fear in the public regarding Max, using tactics designed to create doubt and anxiety about the medicine. However, N34  countered by directly refuting the claims and offering clear, factual explanations to debunk the misinformation, shifting the focus back to evidence-based reasoning.
%example of exaggeration
%N35 effectively countered N36's exaggerated claims about the R treatment virus and the new medicine MAX by highlighting the lack of evidence, media coverage, and credible reports, while casting doubt on anecdotal accounts like the widow's story to sw
%example of culture (N35N36)
\subsubsection{Critical Thinking about Misinformation Motives}
After the gameplay sessions, many participants (31 out of 47) demonstrated an enhanced understanding of the intricacy of information and the varied perspectives it can convey. This experience increased their awareness of the importance of considering the motivations behind messages. Participants also mentioned that they are now more inclined to think critically about the goals behind the information they encounter, especially in real-life situations where such considerations are common.
One participant explained how the game illustrated the pre-determined nature of many messages:
\begin{quote}
    \textbf{N9}:
    One of the most direct ways is that [the game] lets me know that what I'm reading is very likely to be pre-determined. It's like the rules of the game itself, which is that I'm playing as someone in camp A, and I'm across from someone in camp B, and we are both posting messages for the benefit of our camps. Those messages may take on various styles or appearances, but they are all ultimately very purposeful. This, I think, is a strong point to learn.
\end{quote}
Participants also found that this new perspective would be useful in their future interactions with information. They felt that applying this critical mindset could help them better understand the underlying goals and potential financial motivations behind the messages they encounter:
\begin{quote}
    \textbf{N42}:
    It feels like one of the more educational aspects of the game is that [through this game] it's like I can think about what their ultimate goal is from a reverse mindset, and then look at a lot of information in life with that mindset.
\end{quote}
%N32P2 R2面对不利背景故事情况的时候,未使用提示情况下质疑:“在死亡人群中有“几个人”使用新药无法说明药物是否无效,这与个人体质有很大关系,需要足够多的试验对比,而不是凭借患者家属个人描述。 (N31 N32 0821 Group P, 位置243)”在后续interview时候也提到自己“从我个人角度来说,我认为这种假新闻或者说片面性的新闻,它是故意在舆论上去引导一些信息 (N31 N32 0821 Group P Interview, 位置126)”

\subsubsection{The Impact of Role-Playing as a Misinformation Creator on Learning}
Through the experience of playing the role of the misinformation creator (Player 1), some participants became aware of just how low the barriers are for creating misinformation. This made them more cautious about the influence of certain public figures, particularly online "influencers." As one participant noted: 
\begin{quote}
    \textbf{N45}:
    I'm Player 1, and I realized that the cost of creating rumors is so low. If I were an online celebrity or someone with the ability to influence public opinion, and my job wasn't that of a journalist, I might not need to be very responsible for spreading these kinds of rumors.
\end{quote}
Another participant reflected on how playing as a misinformation creator broadened their perception of misinformation, particularly regarding how easily false information can be fabricated. This experience expanded their understanding of the boundaries of misinformation, making them more aware of how easily those boundaries can be crossed: 
\begin{quote}
    \textbf{N46}:
    I used to think I could recognize information with a stance, and information without a stance. But after playing the game this time, I've realized that it’s something that can be fabricated. The boundaries of awareness of false information have been expanded, and the bottom line has been lowered. That's probably how it feels.
\end{quote}
These findings indicate that when participants took on the role of distributing misinformation, it helped them to better grasp how misinformation is produced and emphasized how easily it can spread. This is further proved in the game. As shown in \autoref{fig:influencer}, Player 1 employed certain strategies when faced with an unfavorable context, such as avoiding or distorting facts and creating a positive image. In the interview, Player N31 also reported that as the game progressed, they felt increasingly confident in their ability to generate misinformation.
\begin{figure}
    \centering
    \includegraphics[width=1\linewidth]{figs/results_influencer_RR.jpg}
    \caption{Strategies of influencers used to deal with unfavorable situations in the game}
    \label{fig:influencer}
\end{figure}
\subsubsection{Tailoring Debunking Strategies to Audience Characteristics}
Many participants found the responses of the LLM-simulated personas to be particularly engaging. They analyzed these responses to understand the reasons behind changes in opinion, how the output of other players influenced these shifts, and what the personas now trusted. Participants noted that the LLM-simulated personas provided clear trust level scores and reactions, which were helpful in organizing their responses. As one participant observed:
\begin{quote}
    \textbf{N23}:
    What I found most interesting was the change in their opinions. They would follow the different points we made and then express their own opinions from various points of view. At first, I didn’t think what they said had any effect on me, but later on, I adjusted my strategy according to their thoughts and used them to control the score (trust level score).
\end{quote}
This insight into the personas' dynamic responses helped players refine their strategies. For example, players noticed that different personas reacted differently to emotional and logical appeals. While three personas were easily swayed by emotional arguments, the other two preferred rational, science-based evidence. Recognizing these tendencies, Player 2 successfully countered emotional tactics through logical analysis and evidence (N35). However, even when Player 2 consistently employed logical reasoning and evidence throughout the rounds, it didn’t always succeed in shifting all five personas' trust levels in their favor.
Participants further realized that using evidence to dispute misinformation wasn’t always effective. For instance, one persona was a traditional-minded person who resisted new scientific findings. As shown in \autoref{fig:debunk}, after several rounds of gameplay, players adapted their strategies to persuade the persona by considering her perspective. In the follow-up interview, a participant reflected:
\begin{quote}
    \textbf{N3}:
    There's a housewife who has always been a supporter of traditional medicine. I felt very confident of being able to persuade her because I observed players' struggles with her, and I saw the issues they sought to have resolved. So, in my final round, I focused specifically on her. I took the view that the best way to address this challenge was to rely on scientific evidence. It think that to have done otherwise would in itself have been a form of misinformation.
\end{quote}


%by tracking the trust level scores, players noticed that different personas reacted variably to emotional and logical appeals. Specifically, three out of five personas demonstrated strong trust in misinformation due to Player 1's emotional manipulation, while the other two like John and Alex did not. In this way, Player 2 successfully countered these emotional tactics through logical analysis and evidence (N35). 
%Furthermore, Player 2’s rational approach (N22) across every round in the game also proved insufficient to fully shift the perspectives of those with entrenched beliefs, as indicated by the trust level scores. Incorporating stronger emotional appeals may improve Player 2's influence on personas who are more receptive to emotional narratives.

%N10 "My main strategy is to persuade those five audiences."
%N5 "I tried to focus more on the reactions of different audiences than on my opponent's, especially when I need to clarify specific points for them. "
%Similarly, N12 (P1) reported that interacting with the personas helped them gain a more precise understanding of certain groups, both in the game and in real life. This understanding enabled them to adjust their strategies more effectively to persuade these groups during the game and in future real-life scenarios.%N21 "The young lady always trust my words. This is very similar to reality as some teenagers are easily influenced by online misinformation."
%Additionally, N10 reflected on the reactions from personas, "For example, the 78-year-old man. Outside his field of expertise, he might react differently. While his strong resistance to misinformation comes from his professional background, in another industry, he could be more easily misled."
\begin{figure}
    \centering
    \includegraphics[width=1\linewidth]{figs/results_debunk_RR.jpg}
    \caption{An example of tailored debunking strategy.}
    \label{fig:debunk}
\end{figure}


\subsubsection{Improving Confidence in Debunking}
Many participants reported an increase in their confidence to debunk misinformation in real-life scenarios. %This boost in confidence stemmed from their success in debunking misinformation within the game. 
The positive outcomes they experienced in the game appeared to strengthen their ability to challenge misinformation outside the game environment. For example, in Round 3, N7 purchased a hint to access supporting data for their argument. However, N8 used the same data to identify flaws and construct a counterargument. By leveraging data-driven reasoning, N8 effectively challenged N7’s claims and strengthened their own position, as shown in \autoref{fig:confidence}. In the post-game interview, N8 reported their increased confidence in using evidence-based rebuttals: 
\begin{quote}
    \textbf{N8:}
    I'd say my confidence went up to around 70\% or 80\%, because I realized that presenting evidence and discuss this is really effective.
\end{quote}
%For example, N11 (P2) noted that the game provided positive feedback after successfully debunking misinformation, which led to higher scores. 
%\begin{quote}

    %\textbf{N11}:
    %I believe my confidence grew as the game moves forward. Each time I counter my opponent's misinformation, I noticed that the game gave me positive feedback and the personas recognize my efforts. This made me more confident in the newly learned debunking methods in the game, as well as my previous approach to distinguish misinformation.
%\end{quote}
%This validation of their debunking strategies through in-game feedback contributed to their increased confidence. Additionally, some players noted that their in-game effective debunking strategies also led to an increase in their confidence. For instance, N8 would compare the data and previous information to debunk the misinformation, as shown in \autoref{fig:confidence}.

However, some participants expressed concern that debunking misinformation in real life would be more challenging. Unlike in the game, where the personas’ cognitive levels and trust levels were explicitly shown, it is much harder to gauge these factors in real-world interactions. This uncertainty could make it more difficult to challenge misinformation effectively in everyday situations.
\begin{figure}
    \centering
    \includegraphics[width=1\linewidth]{figs/results_confidence_RR.jpg}
    \caption{An example of how a debunker identifies logical flaws in the information by comparing different pieces of information.}
    \label{fig:confidence}
\end{figure}

%In round 3, N7 purchased a hint, gaining access to data to support their argument. However, N8 used this data to find flaws and construct a counterargument: "Previously, Eternal Health claimed that 20\% of R drug sales would be used for Zinc virus research and prevention, and that R drug sales were so successful that they sold out. So why does Jack still need to invest 65\% of his funds into building advanced technology and laboratories for the traditional medicine institution producing R drug? It seems that such a large investment shouldn’t be necessary. The reasons behind this are worth considering—perhaps there is no real connection or proper investment." (N8, Round 3) By leveraging data-driven arguments, N8 was able to effectively challenge N7’s claims and strengthen their own position. In the post-game interview, N8 reported their increasing confidence in using evidence-based rebuttals: "I’d say my confidence went up to around 70 or 80, because I realized that presenting evidence and discuss them is really effective." 


\subsubsection{Impact of Gameplay on Future Debunking Actions}
A few participants (4/47) shared that the game increased their likelihood of taking action against misinformation in the future. This change in attitude was driven either by participants' previous negative personal experiences with misinformation or by their realization during gameplay of the serious consequences misinformation can have.  The gameplay experience enhanced their willingness to invest time and effort into distinguishing and debunking false information. As one participant said:
\begin{quote}
\textbf{N41}
    In real life, there is a lot of false information, especially in advertising, media, and even those semi-official accounts, which can lead to changes in public opinion under the influence of these accounts, and in that case, it will definitely have some impact on some ordinary people. The game has strengthened my hatred for this kind of false information, so that I can be more awake and rational in my judgement.
\end{quote}

However, the majority of participants indicated that they might not actively debunk misinformation on social media after playing the game. The primary reasons were a dislike of online debates and the belief that it’s not their responsibility to engage in debunking efforts. These findings align with previous research, which suggests that most users are reluctant to take action to debunk misinformation publicly\cite{tang2024knows}.

\subsubsection{Challenges and Negative Effects of Gameplay}
While some participants gained confidence in their ability to debunk misinformation, others experienced a decrease in confidence (5/47). These participants observed that the game was close to real life, particularly that some individuals held strong pre-existing beliefs that were difficult to challenge. The game reinforced this reality, leading to a reduction in their confidence. As one participant noted: 
\begin{quote}
    \textbf{N11}:
    In the process of debunking, I realized that it is quite difficult to change people's inherent beliefs. Some people do not care much about whether the source of information is true or false, and this is also a social phenomenon that exists.
\end{quote}
Another challenge reported was the overwhelming amount of information presented in the game. In each round, players had to process news stories, the reactions of five personas and their changes, as well as their opponent’s output. After absorbing this information, participants were required to devise strategies and craft tailored responses. The volume of information, particularly by the end of the game, left some participants feeling exhausted, which may have impacted their performance.
In addition, the game’s mechanics required players to have a basic level of media knowledge to effectively take on the roles of “influencer” and “debunker.” Some participants also noted that the competitive nature of the game could lead to an unbalanced experience if one player was significantly stronger than the other.
As one participant noted:
\begin{quote}
    \textbf{N40}:
    The game was very fun, then I felt a bit nervous, and by the time I got to the end, it was a bit exhausting. I felt nervous because there was a lot of information at the beginning, and I was competitive with Player 2. There was much writing involved, and I felt uncertain because I’m not very good at writing, and I knew Player 2 was very skilled. So I felt a little nervous.
\end{quote}

%\subsection{Players' Perception and Understanding Throughout the Gameplays
%}
%Participants reported an increased awareness and understanding of misinformation through various aspects of the gameplay. This included elements such as game instructions, the news presented in each round, role-playing within the game, the reactions from the LLM, and the outputs from other players.

%\subsubsection{Perception of Misinformation Through In-Game News}
%The news presented in the game served as a clear example of misinformation for many players. Due to the confrontational mechanics of the game, both players were tasked with identifying the "false" components of each other's messages and responding accordingly. This experience heightened participants' awareness of misinformation and helped them recognize certain characteristics that signal suspicious information. As one participant noted: “\textit{In the process, I was able to see first-hand some of the flaws in the information (posted by others) and some of the claims made in an attempt to deceive people. And then it's also more accurate for me to judge the dis/misinformation afterwards (out of the game)}.” (N20) Out of 47 participants, X mentioned that the news in the game reflected real-world situations and enhanced their awareness of misinformation's features. One characteristic that emerged was the complexity of news, which is often not entirely true or false but rather a mixture of both. Players observed that this mixture made misinformation more concealed and difficult to recognize.“The information provided in each session, I think it's more in line with the wording of the current news, and then, we know that the news nowadays is more mixed with positive and negative information, so this game, I think, is more in line with a real-life situation”. N23 “The background news in each round feels quite realistic. What makes it feel real? Because I think our source of information should be modified from some real information, right? Just like I said, half-truths are the most misleading” N20.

%\textbf{Awareness of Source Credibility} Participants also became more attuned to the varying credibility of different sources, recognizing that misinformation often intentionally references authorities to bolster its credibility. As one participant reflected:\textit{ “As this round progresses, that news of his gives us more and more of a bit of news... It's the feeling that when I see another story after that, I'll realise consciously. Eh, he's mentioned an authority here. It's possible that it was mentioned on purpose, or there's some (information) that my suspicions about him might be more grounded.” (N40)}
%need more...
%N13 reported that through the game, the player realize that authorative information is not always true, and it it important to compare the authenticity of information through multiple sources. "I used to think that if information came from authoritative sources or reputable publications, it seemed more believable. But after playing the game, I realized that even those might not be true because my opponents in the game used a lot of fake evidence from so-called authority."(N13 N14 0819 Group B Interview, Line 23) "有,就是之前看虚假信息就是有的,感觉没这么讲,就是以前看下西西,感觉带上那种比较权威的那种人或者刊物什么的,感觉会感觉有点真,但是完了之后发现也都是也也有可能都是假的。 (N13 N14 0819 Group B Interview, 位置23)" m(寫的很好)
  

%\textbf{Critical Thinking and Awareness of Motivations} Another significant impact on players was the realization that information is often complicated and can represent different perspectives. This realization encouraged them to think more critically about the motivations behind the messages they encountered, with X participants explicitly mentioning this. For example: \textit{“One of the most direct ways is that (the game) lets me know that what I'm reading is very likely to be pre-determined. It's like the rules of the game itself, which is that I'm playing as someone in camp A, and I'm across from someone in camp B, and we're both posting messages for the benefit of our camps. Those messages may take on various styles or appearances, but they're all ultimately very purposeful. This, I think, is a strong point to learn.” (N9)} \textit{“It feels like one of the more educational aspects of the game is that [through this game] it's like I can think about what their ultimate goal is from a reverse mindset, and then look at a lot of information in life with that mindset.” (N42)}

%\subsubsection{The Role-Playing Experience and Its Influence on Learning}
%The role-playing element of the game also influenced participants' learning. For those who played the role of the misinformation distributor, some participants realized how low the cost of creating misinformation could be, making them more cautious about so-called "influencers." One participant observed:
%\textit{“Just the fact that I'm player 1, I just thought that having played it would show me that the cost of rumour creation is so low. It's that if I'm this kind of online celebrity or something, as this kind of person who can have a certain kind of ability to influence public opinion, and then my job is not a journalist kind of thing, I may not need to be very responsible for spreading this kind of rumour.” (N45)} Others reported that playing the role of a misinformation distributor gave them deeper insights into misinformation itself: \textit{“This information I know to be false, and then my task (in the game) is also to go and turn this false information into a real situation to carry out a persuasion. Under this influence, then, I actually correspond to a deeper understanding of this false information (...) What I'm seeing now is that this disinformation, it may not be that it's completely false, like when I'm writing content, I'm being half-truthful, and that's the hardest thing for people to discern in this situation (.....) And then in this case it's actually improved my ability to judge this, um, disinformation, and in that I've learnt more about the logic of disinformation.” (N20)}
%need more description about P2's role 

%\textbf{Player 2's Role and Its Influence on Learning}Players who took on the role of Player 2 primarily utilized logical reasoning to refute Player 1’s arguments. They identified flaws in the logic of Player 1’s information, pointed out instances of emotional manipulation, questioned the lack of transparency in financial relationships behind the messages, and highlighted the absence of evidence. These strategies effectively revealed the flaws in Player 1's messages and reduced their credibility. To strengthen their arguments, Player 2 often referred to authentic information from reputable sources or authorities, encouraging the public to remain calm and critically assess the information presented.
%N2(P2)通过在游戏中扮演debunker的角色,总结出自己在游戏中遇到P1使用了大量的情绪煽动性词汇,这让他更加明确自己当前角色的策略:即阅读一个信息的时候,将信息的事实和他所表达的情绪分开独立来看:关注更多的精力在事实上面。因为煽动性言论很容易让人迷失在真实事实。(“就是比如说在阅读一篇报道的时候,要放更多的精力在事实层面的事情,而不是作者本身观点和情绪上面的事情。比如说他打的一些感情牌,或者说是一些宣,就是煽动性的言论,这些事跟事实应该是分开的两部分。可能以后我在摄取信息的时候,会将更多精力放在到底哪些是真实发生的事情上面,而不是哪些是这个作者提出的感情方面的事情。 (N1 N2 0818 Group B Interview, 位置101)”)

%In addition to the debunking techniques that Player 2 learned, they also perceived how their opponents generate the misinformation. N22(P2)提到通过观察游戏中P1是怎么传播虚假信息的而学到的(“我在玩儿的同时我错了的话,我就知道了它这个是怎么说?这个虚假信息是怎么传播的?比方说我想发一个虚假信息,诶,我就知道了。嗯,就知道我这个虚假信息是怎么发出来的,下次我遇到这样的信息了,我就知道这个是假的了。 (N21 N22 0819 Group G Interview, 位置152)”)



%This in-game experience also increased participants' confidence in debunking misinformation in real-life scenarios. X out of 47 participants reported that their success in the game bolstered their confidence to challenge misinformation in reality. For example, N11 (P2) noted that the game provided positive feedback after successfully debunking misinformation, which led to higher scores. This validation of their debunking strategies through in-game feedback contributed to their increased confidence.

%\textbf{Adapting Debunking Strategies Based on Audience Characteristics}
%Participants also learned that different audiences require tailored strategies. For instance, in one scenario, a language model (LLM) simulated a young female student who was highly active on social media, trusted social media influencers and celebrities, and easily changed her opinion. After a few rounds, players recognized her personal characteristics and began referencing popular singers and celebrities' endorsements to persuade her. As one participant shared:"Quotation". Similarly, N12 (P1) reported that the personas helped them gain a more precise understanding of certain groups, both in the game and in real life. This understanding allowed them to adjust their strategies to better persuade these groups during the game.

%\textbf{Learning Tactics for Creating and Debunking Misinformation} 
%Participants also learned various tactics for both creating and debunking misinformation through the game’s instructions and their in-game experiences. This learning made them more cautious about information containing these features. Interestingly, many players (X/47) employed emotional manipulation strategies, followed by using hard science/facts as references and finally questioning the motivations behind messages. One explanation for this is that the LLM simulates emotional reactions in the populace, making players more aware of emotional manipulation. Consequently, players were driven to use emotionally manipulative strategies, or they noticed the LLM’s simulation of critical thinking, which encouraged them to seek evidence for support. When using emotional manipulation, most players aimed to stimulate anxiety and fear, with some also attempting to create feelings of hope. These strategies align with the characteristics of health misinformation [REF].  In the follow-up interviews, players mentioned that they became more aware of the emotions embedded in messages, which made them more suspicious of such content: \textit{“After going through this game, the kind of statements that are overly positive, overly exaggerated, or overly one-sided about an overly positive point of view just don't seem so true to me.” (N23)}


%\textbf{Variations in Debunking Confidence Among Participants}
%However, the debunking experience varied among participants. While some participants gained confidence, others reported a decrease in their confidence to debunk misinformation (X/47). These participants found that, similar to real life, some individuals held very strong pre-existing beliefs that were difficult to influence. The game reinforced this challenge, leading to a decrease in their confidence. As one participant explained:“In the process of debunking, I realized that it is quite difficult to change people's inherent beliefs. Some people do not care much about whether the source of information is true or false, and this is also a social phenomenon that exists.” (N11)

%\textbf{Impact of the Game on Future Debunking Actions}
%A few participants shared that the game increased their likelihood of taking action against misinformation in the future. This change was often related to their personal experiences, such as friends encountering disinformation or a heightened awareness of the negative consequences of misinformation on the general population. This increased awareness made them more willing to invest time and effort into distinguishing and debunking misinformation. As one participant expressed:“Quotation”. However, most participants mentioned that they might not actively debunk misinformation on social media after playing the game. They cited a dislike for online debates and the perception that it is a waste of time. These findings are consistent with previous research (REF).

%N2(P2) reflected on Future Debunking Actions


%one reason is that emotion can seem as a tool of misinformation 

%我感覺可能因為LLM的回覆中展現出來的也是emotion, 或者說他是一個critical thinking的人,他就想要尋求證據的支持,這種反饋會進一步驅使讀者使用這兩種策略。玩家實際上也是這樣做的。

%“我会留意那5個人对于对手的信息的reaction,然后我再根据他们的 reaction,比如他们觉得为什么会因为对手发了这个东西,他们会改变这个情绪?然后我再根据这一点去增加一些我下一条写的内容 (8.20 L組 N25N26, Pos. 456)”

%“应该就是那 5 个人的反应吧?就是他们的分数反应啥的,就是在这方面他们就很非常直接直观的告诉你,他们为什么会相信这个信息或不相信这个信息,那从他们的反应里面就可以得到一个下一轮的一个策略。 (8.20 L組 N25N26, Pos. 468)”

%“我觉得最有意思的就是他们的发表言论的变化。就是他们会随着我们说的不同的观点,然后也发表自己的意见,从各个方面。说话人 1 01:15:23所以您觉得这一点比较有趣,是吗?说话人 2 01:15:27对,我觉得这样还挺有意思,就每一次都不太一样啊。说话人 1 01:15:32好的,那为什么你会觉得这个比较有趣呢?这只是一种感受吗?还是有什么特别的。说话人 2 01:15:38原因为一开始我可能没有觉得他们说的对我有什么影响,但是后来的话我会根据他们说的东西,然后就是进行后期的想法。嗯,可以利用他们来控分。N23/N24.

%**LLM的reaction**也會讓玩家意識到,對不同的人有不同的說服策略,有的人是真的很難改變的(所以會導致玩家信心的下降)

%“嗯,比如说在第一轮游戏结束之后,我能看到有几个分数低的是那个退休的护士和一个高中生。然后我看到他们的评价,一个是说需要有更多的证据,然后另一个是可能是医学需要一个证据。嗯,然后所以之后在第二轮还是第三轮,我就有说到那个,我其实就做了一些虚假的新闻,我说他也做一些研究,然后发了一些 paper。 嗯,然后我也还好像还有提到,因为我看到有一个是年龄比较大的老人,嗯,然后因为我发现好像第一轮我用了那个提示,我发现一些网络的言论也会影响到这五个用户。 有一个是高中生,好像是高中生,他好像手机,他看中什么網絡信息,然后所以我也有用到一些像说其他一些死亡或者是老人这种关键词,我要去尝试去影响他们的 (8.22R组-N39N40, p. 25)

%最後,玩家體驗完在遊戲裡贏了之後,比如說服了LLM模擬的民意,看見這個人倒向他們的時候,其實增加了他的信心。

%**在遊戲結束之後**,黨玩家被問到以後的行為時,

%對於distinguish 虛假信息

%比較多的玩家說過:識到信息不是單純的對錯,有些是有立場的,以後可能會更加去關注信息背後的動機。

%從主觀描述來講,玩家對於虛假信息的意識是增強了,

%“因为感觉自己好像可以从逆向思维思考他们最终的目的是什么,然后带着这种思维去看生活当中很多信息”N42

%“嗯,怎么说呢?嗯,我感觉可能会有强一点,因为其实这个是有个人性格的原因,我是一个比较冲动的人,就是我有时候会看到信息会引起我情绪的变化,所以我在现实中就有可能会被一些东西给带着走。但是做了这个实验之后,你就会从一个比较全面的角度去思考,然后这个时候就有可能你在现实生活中之后,也会把这个思维方式带入到现实生活中,就是像去思考一道题一样,思考一下它的整个结构,它整个的框架有可能会有哪些情况,然后可能就可能在情绪上的变动就不会那么大。 (8.20 M組 N27N28, Pos. 208)”

%少數玩家說以後會多去看官方的信息,尋找來源。

%有的玩家會意識到信息中的情緒操控“通過這個遊戲之後,對於过于肯定的观点,过于夸张或过于一边倒的那种说法,在我看来就不是那么真实。”N23

%有的玩家進一步意識到了現實中虛假信息對普通人對負面影響,從而更加願意投入時間和精力去進行判斷

%“我感觉是增强了这个点的话,可能不是我的能力增强,而对于虚假信息的痛恨增强。因为实际上,在现实中虚假信息是特别多的,尤其是营销号、传媒号,甚至是那些半官方性的账号,所以在这种影响下会导致整体的风向或者说舆论变化,然后这样的话肯定是会对一些普通人造成一些影响的。所以说现在我对于这些虚假信息更痛苦,也就是变相的让我更加清醒的去判断。”N41

%**對於Debunking**

%有的人說改變不大,因為他本身就不是會在網上跟人起爭執的人

%有的人說信心會下降,因為他意識到了人是很難改變的,他本身就有這個意識,現在體驗過四輪遊戲之後發現更加是這樣了。

%部分玩家認為遊戲中發生的事情會讓他們聯想到現實中的事情,因此激起他們的反應。

%“然后如果说这个信息这损害到人的一些基本的利益,或者说生存,或者说一些金钱这一块,我可能会直接进行一个反驳,或者说直接进行一个举报等情况。对,这是我在今天,嗯,看到一个最直观的一个感受,因为像什么看到那五个人的一个状态,我尤其想到了什么,就是出高中的时候,就是朋友被骗的那些经历。对,然后看到以后就会有这种想法。对。 (0819_E组-N19 N20, Pos. 507)”

%遊戲難度:

%每個人在每一輪都獲得了很多信息,在思考和閱讀上有難度,但是可能不太適用於所有人,比如小朋友或者老年人沒有辦法使用。

%對以後的建議:

%在遊戲裡讓玩家有更多的主動權去驗證信息,因為目前“也可能是因为这是一个游戏。嗯,我先,所以我也没有别的信息来源去验证。嗯,就对话说的到底是真的还是假的。 (8.20 M組 N27N28, Pos. 323)“ 而且這樣有個問題是,兩個玩家其實全部都是在說自己的觀點,有時候就很難反駁。


%\subsection{Two}

%\subsubsection{Perception of Misinformation}

% To 5.2.1 Confrontational Game Mechanics
%As the game progressed, fluctuations in response times were evident. Player 1’s time to generate misinformation increased, particularly in Rounds 3 and 4, as their strategy shifted from simple emotional appeals to more elaborate narratives incorporating economic details. This shift suggests that Player 1 aimed to enhance the credibility of their misinformation by adding complexity and depth (N4). However, in some cases, Player 1’s response time decreased as they adopted to more inflammatory and emotional manipulation to persuade the LLM personas (N1). The behavioral patterns in N6 and N12 indicate a growing reliance on emotional manipulation over factual engagement when challenged. Additionally, Player 1 demonstrated a deeper understanding of misinformation techniques, increasingly employing multiple emotional manipulation strategies (N12). Furthermore, Player N33’s game content and interview responses suggest that in-game instructions played a significant role in helping them in generating misinformation.
%We observed that Player 2 employed a rational, evidence-based approach to counter misinformation, relying heavily on scientific facts and reasoning. For example, in Round 2, Player 2 (N2) effectively challenged Player 1’s misleading claims by referencing medical expert opinions and providing scientific explanations for patient deaths, underscoring a strong evidence-driven approach. Similarly, Player 2(N3) questioned the lack of clinical data supporting the R medicine and pointed out the potential commercial motivations. As the emotional complexity of the misinformation increased, Player 2's information became more structured and precise, with deeper critiques grounded in both scientific and ethical considerations (N3). In addition, Player 2 (N7) incorporated underlying factual evidence, such as economic motivations, to further strengthen their counterarguments. In some instances, Player 2 (N10) shifted focus from purely scientific critiques to address regulatory frameworks, leveraging an official perspective to challenge the spread of misinformation.

% To 5.2.6 Tailored Debunking Strategies
%Based on the trust level score, we observed that three out of five personas demonstrated strong trust in the misinformation due to emotional manipulation, while the other two did not. Player 2 effectively counters Player 1’s emotional manipulation claims through logical analysis and evidence, especially capturing the attention of more rational personas like John and Alex (N35). Player 2’s rational approach (N22) proved insufficient to fully shift the perspectives of those with entrenched beliefs, as indicated by the trust level scores. Incorporating stronger emotional appeals may improve Player 2's influence on personas who are more receptive to emotionally narratives.

%\subsubsection{Time to Generate Response}
%As the game progressed, we observed the fluctuations in response times. In Round 3 and 4, Player 1 took more time to generate misinformation, where their strategy shifted from emotional appeals to more elaborate narratives that included economic details. This change suggests that Player 1 wanted to make their misinformation more believable by making it more detailed (N4). However, there were instances where Player 1’s response time decreased as they shifted to more inflammatory and emotional manipulation tactics, aiming to convince LLM personas (N1). N6 and N12’s behavioral shift indicates a preference for emotional rather than engaging with factual rebuttals when challenged. Additionally, Player 1 increasingly used multiple emotional manipulative techniques, reflecting a deeper understanding of how to generate misinformation (N12).

%Similarly, the response time of Player 2’s rebuttals increased as Player 1’s information became more intricate. To counter the increasingly sophisticated misinformation, Player 2 needed more time to analyze, verify, and construct detailed rebuttals (N5). Early in the game, Player 2 relied on straightforward scientific evidence, but in Round 4, their responses became more structured and multi-layered. They employed rigorous fact-checking and logical analysis to dismantle Player 1’s conspiracy theories. This demonstrated Player 2’s growing capacity to maintain a calm, rational approach in the face of emotionally charged misinformation (N11).

%\subsubsection{Changes in Behavior Patterns}
% To 5.2.1 Confrontational Game Mechanics
%During the game, some players encountered challenging narrative contexts. In Round 3 where Player 1 was at disadvantage, they responded by adopting a positive storytelling strategy, portraying characters in an optimistic and proactive manner to build trust in the misinformation. Instead of using fake scientific evidence, Player 1 used emotional manipulation, emphasizing values such as "cultural tradition" and "ancestral wisdom" to downplay the role of science and overcome the unfavorable situation.
%Player 2 encountered similar difficulties when anecdotal evidence was used to spread misinformation, particularly in Rounds 2 and 4. In these cases, Player 2 had to maintain logical reasoning despite the strong emotional appeals. Some players countered misinformation by clarifying facts and directly addressing conspiracy theories, while others employed a more cautious approach, appealing public to wait for more evidence to prove. This strategy allowed Player 2 to weaken Player 1’s influence.


%Full width figures.
% \begin{figure*}[htbp]
%   \centering
%   \includegraphics[width=\textwidth]{figs/data.png}
%   \caption{Caption}
%   \label{fig:data}
%   \Description{Caption}
% \end{figure*}

%(18/29) reported increased confidence in identifying misinformation.
%N36(P1), N35(P2), N30(P2), N29(P1), N22(P2), N18(P2), N17(P1), N16(P1), N15(P2), N14(P2), N13(P1), N11(P2), N8(P1), N7(P2), N6(P1), N3(P2), N4(P1), N1(P1).
%\begin{enumerate}
 %\item N11(P2) reported that the game provided positive feedback after successfully debunking misinformation, leading to higher scores, which verify their prior debunking approaches. This process of positive feedback contributed to increased confidence.
 %\item However, N11(P2) reported that it will be more challenging to debunk misinformation in real life because it's difficult to know the cognitive level of the public you're addressing, which is different from the in-game personas, making it hard to effectively convey ideas.
 %\end{enumerate}

%(5/29) reported decreased confidence in identifying misinformation after learning these strategies.
%N47(P2), N31(P1), N33(P1), N10(P2), N9(P1)
%\begin{enumerate}
    %\item N10(P2)发现在游戏里面虚假信息竟然还可以直接这么编,刷新了自己的认知(“我觉得我以前大概认为我自己是能够识别带立场的信息和不带立场的信息的,就是但是我这次游戏玩完之后,我还意识到就是你是可以编的吗哈?就是那个所谓的虚假,对吧?就,就是就边界扩展了,底线降低了。嗯,我觉得大概是这么一种感受,嗯。 (N9 N10 0818 Group T Interview, 位置159)”)
%\end{enumerate}

%(25/29) reported that they learned how to distinguish the misinformation from the game
%N47(P2), N36(P1), N35(P2), N34(P2), N33(P1), N31(P1), N30(P2), N29(P1), N22(P2), N18(P2), N17(P1), N16(P1), N15(P2), N14(P2), N13(P1), N11(P2), N9(P1), N8(P1), N7(P2), N6(P1), N5(P2), N3(P2), N4(P1), N2(P2), N1(P1).
%\b%egin{enumerate}
    %\item N29(P1)提到在游戏中有这种体验之后,那么以后在现实生活中遇到了就会发现自己其实就是这么写的
    %\item N22(P2)提到通过观察游戏中P1是怎么传播虚假信息的而学到的(“我在玩儿的同时我错了的话,我就知道了它这个是怎么说?这个虚假信息是怎么传播的?比方说我想发一个虚假信息,诶,我就知道了。嗯,就知道我这个虚假信息是怎么发出来的,下次我遇到这样的信息了,我就知道这个是假的了。 (N21 N22 0819 Group G Interview, 位置152)”)
    %\item N2(P2)玩完游戏后声称自己会在阅读一个信息的时候,将信息的事实和他所表达的情绪分开独立来看:关注更多的精力在事实上面。因为煽动性言论很容易让人迷失在真实事实。(“就是比如说在阅读一篇报道的时候,要放更多的精力在事实层面的事情,而不是作者本身观点和情绪上面的事情。比如说他打的一些感情牌,或者说是一些宣,就是煽动性的言论,这些事跟事实应该是分开的两部分。可能以后我在摄取信息的时候,会将更多精力放在到底哪些是真实发生的事情上面,而不是哪些是这个作者提出的感情方面的事情。 (N1 N2 0818 Group B Interview, 位置101)”)
    %\item N2(P2) reported that when facing Player 1’s provocative strategies, they sometimes restrained their own statements because their information was truthful, and using provocative language might undermine the credibility of their truthful statements. This is something they will need to consider in real life as well.(“就是我会担心自己的语言过于的具有煽动性,而丧失了那种,对,丧失了本身的一个立场,包括比如说主要是有些修辞,比如说我会用一些排比,或者是一些强烈的情感方面的事情,但是这些又没有什么客观的事实依据,就是会让我感觉到比较矛盾。 (N1 N2 0818 Group B Interview, 位置93)”)
%\end{enumerate}

%(15/29) reported a connection between the game and real-world scenarios
%N34(P2), N33(P1), N31(P1), N21(P1), N18(P2), N8(P1), N14(P2), N12(P1), N11(P2), N10(P2), N7(P2), N6(P1), N5(P2), N2(P2), N1(P1). + N27,
%\begin{enumerate}
    %\item N11(P2) reported that the game deepened their understanding of misinformation.
    %\item N10(P2) reported that by playing the debunker role in the game, they refreshed their understanding of misinformation.
%\end{enumerate}

%(9/29) found the public opinion simulated by AI in the game interesting
%N15(P2), N14(P2), N13(P1), N11(P2), N10(P2), N9(P1), N4(P1), N2(P2), N1(P1).
%\begin{enumerate}
    %\item N9(P1) reported that due to AI model’s strengths, when they fabricated evidence in the game, the personas responded to this evidence, creating an immersive experience through a timely and specific feedback mechanism.(“然后印象比较深刻的就是,嗯,应该是第三轮的时候就是谈那个富豪 Jack 的事情,然后我就瞎写了一下,他有这个,就是把比较多的内容放在了关于这位富豪的个人经历上,然后我可以在市民的反馈中就是读到,嗯,大家对这一点的会就是说,比如说他们关于个人经历的部分不太相信,或者他被个人经历所打动,因为我相信这一点的内容是我纯编造,并且纯主观,就是纯粹是个人的行为,然后他可以在接收到一个特别及时而且特别具体的反馈。我觉得这一点还挺好玩的。对,而且他也造成了我的得就是失分还有失败,所以给我留下了。 (N9 N10 0818 Group T Interview, 位置6)”)
%\end{enumerate}

%(10/29) reported the common strategies they use in the game is to persuade specific persona to get a better score
%N17(P1), N16(P1), N15(P2), N14(P2), N13(P1), N12(P1), N11(P2), N10(P2), N5(P2), N3(P2), N4(P1), N2(P2).

%(5/29) reported the in-game instructions very helpful can learn a lot.
%N35(P2), N34(P2), N33(P1), N15(P2), N9(P1).
%N9(P1) reported that the instruction would serve as a reference in the future real life.

%LLM and persona reactions
%\begin{enumerate}
   %\item N12(P1) reported that the personas helped them gain a precise understanding of certain groups both in game and in the real life, allowing them to adjust their strategies in the game to persuade these groups.
    %\item N11(P2)提到在反驳的过程中会意识到,改变他们的固有观念是挺难的,有些人没有那么在意消息的来源是否真假,这也是存在的一种社会现象。同时不同人物画像的特色设置也增强玩家的认知,了解到这些画像更多,能感受到他们是缺乏很多判断真伪的技能的。
    %\item 对于这一点,N2(P2)玩家也提了建议,如果游戏中把这些persona的丰富特点外露出来更多,将会更有利于玩家们做游戏内的决策。(“在游戏里叫如何能够给玩家一种弱引导,让我知道如何攻克对点人物?比如说这个假如说我就是思维发散的来讲,比如说这个家庭主妇她有一个孩子,那是不是我在这个辩论中提到这个病毒对孩子的侵害,就跟他打感情牌是不是更有可能获得他的认可?但是这种引导我感觉在这个游戏中是比较弱的,就是然后包括那个护士,有一个退休男护士,好像是我觉得这个机制,我猜测背后机制可能就是我提到更多的事实或者医学方面的事情,会更容易获得他的信服。嗯,我不知道这个后面会不会有这样的算法,但我猜测可能是有的。对的,但是可能只有这个角色我感受到了,剩下的角色这方面的引导,我并没有一个很明确的感受,包括像学生中就是年轻学生这种的,我并不知道他喜欢怎样的说法或者材料,这方面的信息我并没有太思考到。 (N1 N2 0818 Group B Interview, 位置149)”)
%\end{enumerate}
\subsection{General Discussion}
Our
study provided empirical evidence that the intervention improved some aspects of users' general media literacy, their behavioral intentions to use different strategies for checking misinformation and their performance in discriminating between fake and real news. However, it did not substantially enhance players’ prosuming skills, despite the mechanics of the game being tailored to support content generation. One possible explanation of this result would be that pre-survey data indicated that participants %already 
entered the study with high levels of confidence in their prosuming abilities. This "ceiling effect" likely constrained measurable gains, as participants had or perceived less room for improvement.
Another explanation is that high-level skills of critical prosuming require collaborative efforts and collective intelligence between participants \cite{lin2013understanding}, which was not possible to achieve in the PvP model of the game. Finally, the game only provided four rounds of content creation for each person, which could be sufficient to apply a critical perspective on the content, but not enough to
train creation effectiveness.

Another important finding is that, similarly to the previous studies of another game-based intervention \cite{leder2024feedback} we did not find significant effects from the intervention on the person's self-confidence in tackling misinformation.
Interestingly, game log data indicated that players' proficiency improved during gameplay; by the third and fourth rounds, they typically produced longer and more comprehensive messages compared to the initial rounds. Social Cognitive Theory \cite{bandura1997self}. may explain this discrepancy. According to it, enactive mastery experiences — ie., successful task completion— are the most influential sources of self-efficacy\cite{bandura1997self}. Positive experiences bolster self-efficacy, while repeated failures undermine it. In our game, success was determined by evaluations from LLM-simulated public opinion. The qualitative results revealed that the players found that no single strategy was effective for all characters, requiring frequent adjustments based on the unique characteristics of each character. This unpredictability made it difficult to achieve consistent success. While players developed greater proficiency during the game, the difficulty in achieving consistent success may have limited their perceived self-efficacy. However, this outcom can also be considered through the lens of the educational effects of the game. Previous studies about differential challenges of misinformation showed that young adults %might lack the critical thinking skills needed to assess information effectively and 
often overestimate their ability to assess information effectively\cite{papapicco2022adolescents,porat2018measuring}; in this context, the reaction of our participants, most of whom were young adults could be a positive signal that they became aware of the complexity of misinformation and the absence of one-size-fit-all solutions. 
%I deleted this explanation
%Furthermore, as character evaluations responded to the actions of both players, scores fluctuated frequently, with the gains of one player often neutralized by the moves of their opponent. This dynamic likely hindered the confidence of the players in their ability to discern misinformation successfully, even though their actual performance improved, as evidenced by significant gains in discriminative test results.Thus, while players developed greater proficiency during the game, the lack of sustained clear success may have limited their perceived self-efficacy. 

%For example, in our qualitative findings, participants reported scepticism toward information that purely relied on authoritative sources because they found their opponents used fabricated evidence from these sources to gain LLM-simulated characters’ trust and increase scores. This gameplay experience reminded them of real-life situations, where misinformation often exploits trust by citing credible authorities. As a result, players learned to examine the intent behind messages instead of automatically trusting authoritative sources.
%\textcolor{blue}{This evolution in how participants evaluated information aligns with the Elaboration Likelihood Model (ELM)\cite{petty1984source}, which describes how individuals process persuasive messages via central(critical evaluation of content) or peripheral routes(reliance on heuristic cues). In the gameplay, participants appeared to shift from peripheral processing (trusting authority as a shortcut) to central processing (evaluating the intent and content of the message) when exposed to the misuse of authority in gameplay. Notably, many participants suggested cross-checking sources as a practical solution for verifying information, demonstrating their enhanced critical thinking skills}.

Lastly, In our qualitative findings, participants reported skepticism toward information that purely relied on authoritative sources because they found their opponents used fabricated evidence from these sources to gain the trust of the LLM-simulated characters, with a view to increasing their game score. This gameplay experience reminded them of real-life situations, where misinformation often exploits trust by citing credible authorities. As a result, players learned to examine the intent behind messages instead of automatically trusting authoritative sources. This evolution in how participants evaluated information aligns with the Elaboration Likelihood Model \cite{petty1984source}, which describes how individuals process persuasive messages via central (critical evaluation of content) or peripheral routes (reliance on heuristic cues). In the gameplay, participants appeared to shift from peripheral processing (trusting authority as a shortcut) to central processing (evaluating the intent and content of the message) when exposed to the misuse of authority in gameplay.
Furthermore, our findings align with studies showing that gamified inoculation techniques for pre-bunking misinformation can trigger skepticism related to both false and real news\cite{hameleers2023intended,modirrousta2023gamified}. While this induced skepticism might seem limiting —potentially thus undermining trust in high-credibility sources— we believe it supports the goal of fostering critical media literacy. By encouraging players to evaluate the trustfulness of content, source and intent,  the game develops essential skills for navigating today's complex information landscape. Rather than promoting cynicism, this skepticism cultivates constructive inquiry that helps individuals better discern reliable information. Notably, many participants pointed to the need to cross-check sources as a practical solution for verifying information, demonstrating their enhanced critical thinking skills.  
%Additionally, recent game interventions have explored ways to improve individuals' ability to assess both real and false information. Strategies include teaching players to identify credible versus low-credibility media \cite{micallef2021fakey}, rewarding accurate judgments \cite{barzilai2023misinformation}, and incorporating feedback tests after gameplay \cite{leder2024feedback}. }

\subsection{Outcomes of the Game Mechanics}

\subsubsection{PvP model for Media-Literacy Game}
Unlike prior misinformation education game which broadly employ single-player mechanics \cite{roozenbeek2019fake,camCambridgeGame,harmonysquare,jeon2021chamberbreaker}; "Breaking the News" applies PvP mechanics. While in general, previous studies in serious games showed that PvP games are more engaging and motivating \cite{cagiltay2015effect}; at the same time in competitive environments, the motivation to “win” may overshadow educational goals.  In addition, in recent studies of another misinformation game it showed that some students may not find %gamification or
competition enjoyable or motivating\cite{axelsson2024bad}. 
%our results also indicate players learnd from the opponents. 
Our results demonstrated that, in general, our participants were highly motivated to play one against another. Moreover, they indicated that they learned from each other's strategies, and it helped them better understand the dynamic of misinformation. Therefore, we can conclude that, in our case, this approach was beneficial to fulfilling the intended purpose of the game. 
However, our study was conducted on East Asian participants, who are part of a collectivist-oriented rather than competitive culture \cite{chung1999social}. Previous studies have shown that cultural factors play an important role in the degree of competitiveness in gamified interventions \cite{oyibo2017investigation}. Therefore, it is also possible that in other cultural settings, the game's incentives can trigger more intense competition, which can negatively affect educational results. 


%These games frame the narrative as either misinformation creator or misinformation debunker and asks the player to adopt one of the roles, tasked by discerning misinformation using choice-based interaction. 


%\subsubsection{Free-form Input Generation}
%\textcolor{blue}{One notable feature of Breaking the News is its open world narrative and free-form response format, which contrasts with the linear choice-based formats commonly used in prior game intervention [REF]. 
%Previous works in the field of role-playing games showed that this approach makes game interaction more natural and increases user engagement \cite{csepregi2021effect,ashby2023personalized}.


%This format positively influences players’ learning outcomes by enhancing engagement and replayability. 

%In choice-based formats,interactions are typically brief, requiring players to passively select predetermined options. This structure can leads to "guesswork," where learners may choose correct answers without fully understanding the underlying concepts. In contrast, free-form responses force players to actively reason and articulate original ideas, fostering deeper engagement and critical thinking. 

%Additionally, by allowing players to develop their own narratives and gameplay experiences, each playthrough feels unique, encouraging them to return to the game.

%This variability not only boosts intrinsic motivation but also contributes to the game's educational value. When players repeatedly engage with core mechanics through varied narratives, they continue practicing essential skills, thereby reinforcing and enhancing their learning outcomes\cite{kucklich2004play}}.

%\textcolor{blue}{An additional strength of this format lies in the autonomy it provides. Players can craft personalized narratives that align with their individual learning styles, a factor proven to result in better learning outcomes in serious game design\cite{hwang2012development}. Furthermore, the freedom to explore and shape their own gameplay experiences not only deepens engagement but also excites players, acting as a powerful motivator for repeated play\cite{ravyse2017success}. By promoting intrinsic motivation and repeated practice, the free-form and open-world design demonstrate their potential to enrich both player experience and educational effectiveness. Given that inoculation interventions often show diminishing effectiveness over time, developing an engaging, replayable game that can consistently reinforce players' resistance to misinformation.}

%Free form rewriting: 
\subsubsection{Free-form Input Generation}
One notable feature of Breaking the News is its free-form response format, which contrasts with the linear choice-based formats commonly used in prior game intervention\cite{roozenbeek2019fake,camCambridgeGame,harmonysquare,jeon2021chamberbreaker,micallef2021fakey}. 
Previous works in the field of role-playing games showed that this approach makes game interaction more natural and increases user engagement \cite{csepregi2021effect,ashby2023personalized}.
Based on our data, we can say that this format positively influences players’ learning outcomes by enhancing engagement and replayability. 
By allowing players to develop narratives themselves, each playthrough feels unique, encouraging players to return to the game. This variability not only boosts intrinsic motivation but also contributes to the game's educational value. When players repeatedly engage with core mechanics through varied narratives, they continue practicing essential skills, thereby reinforcing and enhancing their learning outcomes\cite{kucklich2004play}. 

In an educational setting, previous works showed that in choice-based formats, interactions are typically brief, requiring students to select predetermined options. This structure could lead to "guesswork," where learners may choose correct answers without fully understanding the underlying concepts. In contrast, free-form responses force students to actively reason and articulate original ideas, fostering deeper engagement and critical thinking \cite{bryfczynski2012besocratic}. This autonomy allows players to craft responses based on their understanding of their roles, making this another motivating factor for returning to the game\cite{ravyse2017success}.
 
%\textcolor{blue}{An additional strength of this format lies in the autonomy it provides, allowing them to craft responses based on their own understanding of their roles. For example, during gameplay,  we observed that one player in the journalist-debunker role chose to mimic social media comments rather than rely on scientific strategies, crafting responses that resembled typical user comments. The freedom to explore and shape their own gameplay experiences not only increase engagement but also motivates players to return\cite{ravyse2017success}. Given the memory-strengthening effects of repeated learning, requiring people to recall what they have learned helps them relearn and reinforce these lessons\cite{nader2009single}.Therefore, developing an engaging, replayable game can consistently reinforce players' resistance to misinformation.}
%Players can craft personalized narratives that align with their individual learning styles, a factor proven to result in better learning outcomes in serious game design\cite{hwang2012development}. Furthermore, 


\subsubsection{LLM-Powered Feedback}
%\textcolor{blue}{Another innovation of Breaking the News lies in its interaction and feedback mechanisms. The game incorporates LLM-based role-play as evaluators, creating an interactive feedback for players. LLMs demonstrate the ability to simulate human behavior and reactions, consistent with findings from prior research[REF]. To the best of our knowledge, this is the first attempt to integrate LLM role play into a misinformation education game, opening new possibilities for interactive learning experiences in media literacy game interventions.} 

%\textcolor{blue}{In the game, the LLMs mimic human reactions and behaviors by leveraging demographic, personality, behavioral, and psychological features towards misinformation. This approach provides a personalized gameplay experience, fostering greater engagement and enhanced learning outcomes compared to binary feedback mechanisms (e.g., true or false responses). Players interact with LLM personas by crafting strategies based on their perceived characteristics. When players’ actions get the expected results(eg, persona's reaction towards on the side), they experience a sense of achievement. Conversely, when the outcomes deviate from expectations, players adapt by employing alternative persuasive strategies tailored to the persona. This dynamic interaction encourages active learning, problem solving, and critical thinking. Notably, players exhibit higher levels of engagement with personas they found relatable due to personal experiences, focusing more attentively on their feedback. This level of engagement is difficult to achieve with traditional binary feedback mechanisms, which provide limited insights beyond correctness.}

%\textcolor{blue}{Feedback plays an important role in shaping game-based learning outcomes(REF). In our case, users observed that personas react more obviously to emotional elements. This is because persona’s feedback begin with emotional feelings (such as panic), followed by behaviors (like sharing news with friends or taking no action). These emotional responses also reflect their trust level scores. To get better game performance, players used various tactics to trigger emotions or address these emotions in gameplay. As a results, most participants reported that they learned to detect emotional manipulation techniques. To further enhance learning outcomes, future games could adjust LLM outputs to teach diverse strategies more effectively. }
%Another innovation of Breaking the News is the interaction and feedback component. Compared to choice-based format where player present predetermined options and get binary and concise feedback(eg, true or false)Players tasks actively craft their responses and then receive evaluation by five different LLM-stimulated personas. These personas are based on their characteristics, offering unique feedback toward the player’s output. By engaging with these nuanced evaluations, players gain insights into why certain arguments might appear trustworthy or untrustworthy, encouraging them to refine their strategies dynamically. Furthermore, Since the game is last multiple rounds, players will get to know personas and adjust their misinformation narratives or debunking approaches based on the personas’ reactions. Thus, they learn and practice various tactics to identify and debunk misinformation.
%innovation in LLM as evaluator in game and other settings. 
%feedback in serious game's advantage and disadvantage

Another innovation of Breaking the News lies in incorporating LLM-based "characters" as evaluators, creating interactive feedback for players. Although there are other attempts to incorporate LLMs into helping users
learn about misinformation \cite{danry2023don, hsu2024enhancing} (including gamified attempts\cite{tang2024mystery}), in these approaches, AI was mostly the source of correcting information. While this approach has benefits, it is can also be criticised due to  the possibility of LLMs creating incorrect by plausible text \cite{kim2024can,agarwal2024faithfulness}. In this case, it is possible that the intervention will disinform people to an even greater extent. In contrast, in our study, we incorporate AI not as a source of information but as non-playable characters with their own opinions. Therefore, the educational part of the game becomes more robust to resist the negative effects of erroneous generations, as they can only affect the opinion of the "character" but not the main narrative of the game. In general, LLMs demonstrate the ability to simulate human behavior and reactions, consistent with findings from prior research\cite{1park2023generative}; we also find, that the system is capable of emulating the opinion of 5 different characters at the same time and still keeping them consistently different during the game.
 
One advantage of dynamic LLM feedback is that it achieves greater engagement compared to binary feedback (e.g., true or false). Based on our observation, players adapted by employing alternative persuasive strategies tailored to the character and concentrated on the character's feedback. Notably, players exhibit higher levels of engagement with characters they found relatable due to personal experiences, focusing more attentively on their feedback. This level of engagement is difficult to achieve with traditional binary feedback mechanisms, which provide limited insights beyond checking their correctness.%\textcolor{blue}{Feedback plays an important role in shaping game-based learning outcomes(REF). In our case, users observed that characters react more obviously to emotional elements. This is because character’s feedback begin with emotional feelings (such as panic), followed by behaviors (like sharing news with friends or taking no action). These emotional responses also reflect their trust level scores. To get better game performance, players used various tactics to trigger emotions or address these emotions in gameplay. As a results, most participants reported that they learned to detect emotional manipulation techniques.} %To further enhance learning outcomes, future games could adjust LLM outputs to teach diverse strategies more effectively.

%In the game, the LLMs mimic human reactions and behaviours by leveraging demographic, personality, behavioural, and psychological features towards misinformation. Based on our observation, players were able to catch these different clues and crafted strategies based on these characteristics; players adapted by employing alternative persuasive strategies tailored to the character and concentrated on the character's feedback. This level of engagement is difficult to achieve with traditional binary feedback mechanisms, which provide limited insights beyond correctness.

\subsection{Design Implications for the Further Development of Serious Games with LLM-components}
Based on our experience designing and testing the game, we summarize the following recommendations to practitioners working on similar projects.

\subsubsection {Balance Between Freedom and Guidance}
In our game, we sought to challenge the existing choice-based approach in misinformation education games by providing users with free-form input. We found this approach triggers reflection, which helps to build hands-on experience and make the game more enjoyable. Yet, we also found that it relies on players’ existing knowledge of misinformation. For example, players might incorporate unverified information they’ve encountered on social media into the game, which is specifically problematic for the debunker role. While we provided the players with comprehensible instructions to guide their role's actions (how to act as a debunker or an evil influencer), it would be better to incorporate more context-specific tips in each stage of the game to help users explore different ways of winning the game and deepening their learning. We suggest using the approach used in the free-input educational interventions (e.g. \cite{bryfczynski2012besocratic}) to build clear, understandable criteria for free-form answers. These will not stop creativity but help people tailor their answers to the context of the game. We also suggest adding a preliminary stage to the game in the form of a simple choice-based questionnaire, helping people understand the role and the guidelines for the role. For example, we can introduce to the individual playing the debunker role the Debunking Ethics Standards  \cite{edmo2024,afp2024,eeas2024}, and ask them a series of questions about understanding the content of the guidelines before starting the main game. In addition, we can incorporate an additional AI-based mechanic to check the quality of answers, not only in terms of its influentiality, but also of its coherence with debunking practices guidelines. Similarly, we can demonstrate a "Score of Manipulativeness" to the influencer role. This better shows us how the ideas players implement into the message can be judged, from the point of view of dealing with misinformation.
\subsubsection {Replayability and Feedback}
%LLM 
One of the critical challenges in serious games is maintaining replayability, as this is important to facilitate the learning process \cite{adetunji2024unlocking} and making interventions more sustainable \cite{silveira2016open}. Moreover, a lack of replayability in educational games can limit both educational and behavioral change \cite{epstein2021tabletop}. To address this, we incorporated elements designed to ensure the game can be played multiple times, such as offering two distinct player roles and providing a free-form input mechanism that allows participants to explore a broader decision space. However, we recommend that future interventions consider additional methods to further enhance replayability.

One such approach involves leveraging opportunities to introduce various characters to represent public opinion. In our game, we observed that players finds characters' feedback more engaging when this resonates with them personally. By introducing more characters, or by allowing players to customize characters to better reflect their own experience, the game could encourage players to return and interact with new characters. The result would be a more engaging experience. For instance, research has shown that debunking misinformation often occurs within families and can sometimes lead to conflict \cite{scott2023figured}. In such a scenario, players could customize a character based on their previous experiences with family members, thereby practicing their own debunking strategies in a risk-free environment.
%Character feedback can be designed to elicit deeper cognitive engagement and reflective thinking. For example, when a player attempts to spread rumors intended to evoke fear and anxiety, the character might respond with questions such as: “What evidence supports this claim?” “Why might it be spreading so rapidly?” or “What would happen if I believe this information?” Studies have shown that AI-framed questioning can stimulate user reasoning and enable independent information evaluation(REF).This approach can also motivate players to return to the game equipped with new strategies. Over multiple playthroughs, as players become more skilled in critical reasoning, they gain access to increasingly sophisticated dialogues and insights from the characters, further enhancing replayability.

While in-game feedback guides player behavior during play, post-game debriefing sessions help consolidate learning and improve future performance. Research has found that debriefing is a crucial opportunity for players to process and integrate their learning experiences\cite{crookall2014engaging,leder2024feedback,barzilai2024learning}. After gameplay, we suggest arranging debriefing sessions that allow players to review their strategies, assess their effectiveness, and receive constructive feedback, potentially improving learning outcomes. For instance, after a session focused on combating misinformation, a post-game review might present an ideal debunking response or a well-supported counterargument. Such structured reflection enables players to internalize lessons and increases the likelihood that they will re-enter the game with newly gained insights, thereby reinforcing both learning and replayability.
%For example, some characters aligned closely with players’ prior experiences, prompting players to focus on that particular feedback. In other cases, a character proved difficult to persuade, sparking the player’s interest in challenging that character and crafting more tailored responses, sometimes at the expense of others.

\subsection{Limitations and Future Work}
We acknowledge the following limitations. 
% I still think it's need cultural context in limiation
Our participants were mainly from an East Asian country, which can limit the generalizability of our findings. Previous studies have shown that cultural contexts influence individuals' perceptions of misinformation and interventions \cite{noman2024designing}. For example, study revealed that Mexican and Spanish users were more likely to trust -related misinformation compared to users from Ireland, the UK and the USA \cite{roozenbeek2020susceptibility}. At the same time, non-western participants demonstrated a significantly higher willingness to challenge misinformation. The effect of proposed misinformation design interventions was significantly greater than that seen in analysis of the behaviors and attitudes in the UK population\cite{noman2024designing}. 
Previous studies also suggest that certain populations may face greater challenges in being able to critically evaluate information. For example, a large-scale study observed that Asian individuals encounter more difficulties in assessing health information from social media compared to other populations \cite{chandrasekaran2024racial}. 
%This increased difficulty could be attributed to various factors, such as differing educational approaches[REF], access to resources[REF], or cultural norms surrounding information consumption. 
In addition, individuals from an Asian background are more likely to incorporate social media information into their health-related decisions, potentially increasing their susceptibility to misinformation \cite{chandrasekaran2024racial}. Therefore, future work should explore the cross-cultural applicability of our findings in different cultural contexts. However, as the cultural background of our participants can potentially make them more vulnerable to misinformation than other populations, we believe that our intervention is valuable as providing insight into the outlook, activity and the relevance of certain interventions for these populations.
%Serious games have the potential to address this challenge by geographic and cultural boundaries. By tailoring narrative elements and characters to resonate with specific cultural contexts, these games can enhance both engagement and learning outcomes. In our game, we incorporated traditional medicine into the narrative(News in the game), knowing that it is a topic that resonates strongly with East Asian participants. For example, 48.4\% of Hong Kong residents reported using traditional medicine before the COVID-19 pandemic\cite{lam2021public}. Similarly, a national survey in South Korea found a 74.8\% prevalence of traditional medicine use overall\cite{ock2009use}.In China, traditional medicine is formally integrated into the healthcare system as one of the most widely practiced modalities\cite{chung2023implementation}.This cultural relevance provided a familiar backdrop for the game's content, allowing players to engage more intuitively and craft responses based on their lived experiences. 

%Since the game's news scenarios are easily modifiable, future adaptations could incorporate culturally specific settings to enhance relatability and impact for other audiences. By customizing content to align with different cultural contexts, the game can potentially improve its effectiveness in media literacy and reducing susceptibility to misinformation across diverse populations.}

%In a large-scale study of US populations, Chandrasekaran et al. \cite{chandrasekaran2024racial} found that people of Asian origin are more likely to incorporate social media information into their health-related decisions, potentially increasing their susceptibility to misinformation\cite{chandrasekaran2024racial}. At the same time, previous work showed that non-western participants demonstrated a significantly higher willingness to challenge misinformation and that the effect of proposed misinformation design interventions was significantly greater compared to the UK population.
%Taking together, the further studies required to determine if the intervention will be equally efficient on other ethnical groups, however, the current results showed that the game can be used to help the populations potentially more vulnerable to misinformation.
 
Our sample was relatively homogeneous in age. A recent meta-analysis of articles about different intervention approaches showed that neither age nor gender significantly impacts the effectiveness of media literacy interventions \cite{lu2024can}. However, previous work has suggested that media literacy interventions designed for certain age groups (e.g., older adults and adolescents) achieved greater effects\cite{moore2022digital,hartwig2024adolescents}. Future work should determine if our approach is efficient in other age groups of users and, if necessary, tailor scenarios to suit the various needs of different age groups.

Thirdly, our study provided only a one-time intervention and observed immediate learning effects; 
%we are unsure how long these effects will persist.
Previous work showed that even a one-time interaction with an educational game can provide long-term improvement in misinformation recognition. For instance, Maertens et al. tested the game "Bad News" and found that inoculation effects lasted for at least 13 weeks. This suggests the potential for the long-term effectiveness of active inoculation interventions with regular assessment\cite{maertens2021long}. Still, future research should include multiple time points to assess the long-term effectiveness of our game intervention. There should also be comparisons between one-time and multiple play sessions, with explorations of the impacts of players assuming different roles within the games.
%Suppose we had incorporated a control group or compared different types of training. In that case, we might have identified elements that could be better addressed through lectures, potentially reducing gameplay time and preventing participant exhaustion.
%two players learning diff things: spread vs control, diff play flow so how can you assume they played the same game?
In this study, each player was limited to a single role, either a misinformation creator or a debunker.  This resulted in different learning experiences depending on their assigned role. %While participants reported learning from their opponents, the educational outcomes between roles may have varied. 
The primary reason for not having role-switching in our study was the length of the game and its cognitive demands, which we feared would lead to player exhaustion if roles were switched mid-game. In future iterations, we aim to improve the design by allowing players to save their progress and switch roles during subsequent sessions. This could offer a more immersive experience, as players would gain perspectives from both the misinformation creator and debunker roles. Additionally, we aim to introduce new modes, such as a family mode, where players interact with two LLM-simulated characters. This option would reduce cognitive load while maintaining engagement. 
%how is this representing range of possible misinformation? survey is not comprehensive, should be finding that they learned about one subtopic but not another that was not addressed.

The current game is also limited in its sole focus on a pandemic scenario. In reality, misinformation spans multiple domains, with health-related misinformation able to influence political events such as elections. Our game only addressed text-based misinformation, while visual and video-based misinformation pose even greater challenges and are harder to detect. Future work could include multimedia content, such as images and videos, to more accurately simulate the diverse forms of misinformation that exist in the real world.

The current game approach may unintentionally foster skepticism toward both true and false news, a common issue in misinformation pre-bunking interventions\cite{hameleers2023intended,modirrousta2023gamified}. %Recent efforts aim to improve individuals' ability to assess information credibility. These include teaching players to distinguish between credible and low-credibility media \cite{micallef2021fakey}, rewarding accurate judgments \cite{barzilai2023misinformation}, and using feedback tests after gameplay \cite{leder2024feedback}. 
While we believe that the benefits of promoting critical thinking towards sources are very important in prebunking interventions,
%discuss feedback mechanisms in the design implications, 
we further recommend incorporating features that clearly differentiate high- and low-credibility sources during gameplay.

While LLMs like ChatGPT-4o were used to simulate human reactions in the game, these models do not fully replicate the complexities of human behavior. Human reactions are often nuanced and influenced by multiple factors, including culture, history, and personal experience. LLM-generated characters may oversimplify human emotions and fail to grasp the full context of certain situations. Despite us using an advanced ChatGPT-4o model, it may have missed these subtleties, leading to interactions that feel artificial or incomplete. For example, in the gameplay, players employ strong emotional manipulation strategies to provoke specific responses. However, the LLM-generated characters do not react as expected, where participants reported feeling frustrated when their emotional manipulation strategies did not yield the anticipated reactions. This can limit the diversity and depth of the simulated interactions and detract from the realism and fairness of the experience for certain audiences.

LLMs are trained on large datasets that may contain intrinsic biases, which can manifest themselves in unintended ways during gameplay\cite{kasneci2023chatgpt}. Studies have shown that LLMs are prone to so-called "hallucinations" and can reflect stereotypes or skewed perspectives, which could affect how certain characters respond in the game\cite{xie2024can}. For instance, if biased training data influence a simulated character’s reaction, it may inadvertently reinforce player stereotypes about certain groups of people. 

%Finally, the game was primarily designed in English, which may have affected immersion and comprehension for participants who were not fluent in the language.

%how is this representing range of possible misinformation? survey is not comprehensive, should be finding that they learned about one subtopic but not another that was not addressed.

%LLM do not truly represent the public opinion, biased, misrepresented.

%two players learning diff things: spread vs control, diff play flow so how can you assume they played the same game?


%\subsection{Future Work}
%There are several directions for future work. One potential expansion is to incorporate more diverse topics of misinformation beyond pandemics and health, such as political misinformation or climate change, to enhance replayability. Although misinformation tactics share common features across different domains, each topic presents unique characteristics and challenges. We have already demonstrated that games are an effective medium for confronting misinformation, and the mechanics of \textit{Breaking the News} are designed to simulate the complexity of misinformation in the real world. By changing the in-game news topics, we could tailor the educational impact for different audiences relatively easily.

%Another direction is to allow players to switch roles, experiencing both the misinformation creator and debunker perspectives. This could provide a fuller, more immersive experience. To further enhance the educational value and replayability, we also plan to incorporate more tools to help players investigate evidence. For instance, as in MathE, which designed artificial search engines for its game, we aim to introduce investigative techniques like lateral reading and source evaluation to support both player roles.

%Additionally, future work could explore different formats of misinformation within the game, including multimedia content like images and videos. This would address the limitation of focusing mainly on text-based misinformation and would reflect the growing challenges posed by visual misinformation in real life. Allowing players to create or debunk multimedia misinformation would make the game more relevant to the evolving nature of misinformation.
Game-based approaches have shown great promise as tools for inoculating individuals against the tactics commonly used to spread misinformation. Most existing games in this domain are single-player games which offer players limited, predefined choices. While this design reduces cognitive load, it often results in interactions which feel less natural and engaging. In response, we designed a two-player, PvP game that pits a misinformation creator against a misinformation stopper. By integrating LLM-powered personas to evaluate player outputs and provide real-time feedback, we created a more open-ended and immersive experience.
We found that the game we developed effectively improved players’ media literacy. Participants demonstrated an enhanced ability to evaluate and analyze media content, identify unreliable or misleading information, and employ effective counter-misinformation strategies. Moreover, the game's engaging mechanics, combined with the competitive element, motivated players to learn from both their own strategies and those of their opponents.
These findings suggest that integrating dynamic feedback systems and competitive gameplay elements into misinformation education games offers a compelling method to deepen users' engagement, while also improving their critical media skills. Future research can build on these insights to explore other forms of interactive learning environments, focusing on diverse player experiences and varying misinformation challenges.
%% if specified like this the section will be omitted in review mode
\acknowledgments{%
	This work was partly supported by the state of Bavaria through Bayerische Forschungsstiftung (BFS) under Grant AZ-1592-23-ForNeRo. The authors would like to thank all NARVIS and IFL Lab members for their valuable help and feedback.%
}


\bibliographystyle{abbrv-doi-hyperref}
%\bibliographystyle{abbrv-doi-hyperref-narrow}
%\bibliographystyle{abbrv-doi}
%\bibliographystyle{abbrv-doi-narrow}

\bibliography{09_Bibliography}



\end{document}

