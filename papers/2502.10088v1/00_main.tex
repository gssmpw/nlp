\documentclass[journal]{vgtc}                     % final (journal style)
%\documentclass[journal,hideappendix]{vgtc}        % final (journal style) without appendices
%\documentclass[review,journal]{vgtc}              % review (journal style)
%\documentclass[review,journal,hideappendix]{vgtc} % review (journal style)
%\documentclass[widereview]{vgtc}                  % wide-spaced review
%\documentclass[preprint,journal]{vgtc}            % preprint (journal style)






%% Uncomment one of the lines above depending on where your paper is
%% in the conference process. ``review'' and ``widereview'' are for review
%% submission, ``preprint'' is for pre-publication in an open access repository,
%% and the final version doesn't use a specific qualifier.

%% If you are submitting a paper to a conference for review with a double
%% blind reviewing process, please use one of the ``review'' options and replace the value ``0'' below with your
%% OnlineID. Otherwise, you may safely leave it at ``0''.
\onlineid{1049}

%% In preprint mode you may define your own headline. If not, the default IEEE copyright message will appear in preprint mode.
%\preprinttext{To appear in IEEE Transactions on Visualization and Computer Graphics.}

%% In preprint mode, this adds a link to the version of the paper on IEEEXplore
%% Uncomment this line when you produce a preprint version of the article 
%% after the article receives a DOI for the paper from IEEE
%\ieeedoi{xx.xxxx/TVCG.201x.xxxxxxx}

%% declare the category of your paper, only shown in review mode
\vgtccategory{Research}

%% please declare the paper type of your paper to help reviewers, only shown in review mode
%% choices:
%% * algorithm/technique
%% * application/design study
%% * evaluation
%% * system
%% * theory/model
\vgtcpapertype{please specify}

%% Paper title.
\title{Enhancing Patient Acceptance of Robotic Ultrasound \\through Conversational Virtual Agent and Immersive Visualizations}

%% Author ORCID IDs should be specified using \authororcid like below inside
%% of the \author command. ORCID IDs can be registered at https://orcid.org/.
%% Include only the 16-digit dashed ID.
\author{%
  \authororcid{Tianyu Song}{0000-0002-8428-9651},
  Felix Pabst, 
  \authororcid{Ulrich Eck}{0000-0002-5322-4724}, and 
  \authororcid{Nassir Navab}{0000-0002-6032-5611}
}

\authorfooter{
  %% insert punctuation at end of each item
  \item
  	Tianyu Song, Felix Pabst, Ulrich Eck and Nassir Navab are with Technical University of Munich.
  	
  	E-mail: \{tianyu.song, felix.pabst, ulrich.eck, nassir.navab\}@tum.de.

}

%% Abstract section.
\abstract{%
    %% We recommend that you link to your supplemental material here in the abstract, as well
  %% as in the Supplemental Materials section at the end.
  Robotic ultrasound systems have the potential to improve medical diagnostics, but patient acceptance remains a key challenge. 
To address this, we propose a novel system that combines an AI-based virtual agent, powered by a large language model (LLM), with three mixed reality visualizations aimed at enhancing patient comfort and trust. The LLM enables the virtual assistant to engage in natural, conversational dialogue with patients, answering questions in any format and offering real-time reassurance, creating a more intelligent and reliable interaction. The virtual assistant is animated as controlling the ultrasound probe, giving the impression that the robot is guided by the assistant.
The first visualization employs augmented reality (AR), allowing patients to see the real world and the robot with the virtual avatar superimposed. 
The second visualization is \replaced{a virtual reality (VR) environment with a passthrough window}{an augmented virtuality (AV) environment}, where the real-world body part being scanned is visible, while a 3D Gaussian Splatting reconstruction of the room, excluding the robot, forms the virtual environment. 
The third is a fully immersive virtual reality (VR) experience, featuring the same 3D reconstruction but entirely virtual, where the patient sees a virtual representation of their body being scanned in a robot-free environment. In this case, the virtual ultrasound probe, mirrors the movement of the probe controlled by the robot, creating a synchronized experience as it touches and moves over the patient’s virtual body. 
We conducted a comprehensive agent-guided robotic ultrasound study with all participants, comparing these visualizations against a standard robotic ultrasound procedure. Results showed significant improvements in patient trust, acceptance, and comfort. Based on these findings, we offer insights into designing future mixed reality visualizations and virtual agents to further enhance patient comfort and acceptance in autonomous medical procedures.
}

%% Keywords that describe your work. Will show as 'Index Terms' in journal
%% please capitalize first letter and insert punctuation after last keyword
\keywords{Mixed Reality, Virtual Agent, Robotic Ultrasound, Trust and Acceptance}

%% A teaser figure can be included as follows
\teaser{
  \centering
  \includegraphics[width=\linewidth]{figures/teaser.png}
  \caption{\textbf{Proposed Visualizations with a Conversational Virtual Agent Guiding the Robotic Ultrasound Procedure.} The figure shows three immersive visualizations as seen through the head-mounted display (from left to right): augmented reality, \replaced{virtual reality with a passthrough window}{augmented virtuality}, and fully immersive virtual reality. In both \add{augmented virtuality and }virtual reality visualizations, a realistic 3D Gaussian Splatting model of the actual room, excluding the robot, was used as the virtual environment. In each mode, the virtual human assistant communicates with the patient, offering explanations and reassurance to reduce stress and improve the patient’s experience during the autonomous ultrasound procedure.}
  \label{fig:teaser}
}
%% Uncomment below to disable the manuscript note
%\renewcommand{\manuscriptnotetxt}{}

%% Copyright space is enabled by default as required by guidelines.
%% It is disabled by the 'review' option or via the following command:
%\nocopyrightspace


%%%%%%%%%%%%%%%%%%%%%%%%%%%%%%%%%%%%%%%%%%%%%%%%%%%%%%%%%%%%%%%%
%%%%%%%%%%%%%%%%%%%%%% LOAD PACKAGES %%%%%%%%%%%%%%%%%%%%%%%%%%%
%%%%%%%%%%%%%%%%%%%%%%%%%%%%%%%%%%%%%%%%%%%%%%%%%%%%%%%%%%%%%%%%

%% Tell graphicx where to find files for figures when calling \includegraphics.
%% Note that due to the \DeclareGraphicsExtensions{} call it is no longer necessary
%% to provide the the path and extension of a graphics file:
%% \includegraphics{diamondrule} is completely sufficient.
\graphicspath{{figs/}{figures/}{pictures/}{images/}{./}} % where to search for the images

%% Only used in the template examples. You can remove these lines.
\usepackage{tabu}                      % only used for the table example
\usepackage{booktabs}                  % only used for the table example
\usepackage{lipsum}                    % used to generate placeholder text
\usepackage{mwe}                       % used to generate placeholder figures

%% We encourage the use of mathptmx for consistent usage of times font
%% throughout the proceedings. However, if you encounter conflicts
%% with other math-related packages, you may want to disable it.
\usepackage{mathptmx}                  % use matching math font
\usepackage{soul}
%\newcommand{\todo}[1]{\textcolor{red}{TODO: #1}}
\newcommand{\revised}[1]{\textcolor{black}{#1}}
\newcommand{\add}[1]{\textcolor{black}{#1}}
\newcommand{\deleted}[1]{\textcolor{red}{\st{#1}}}
%
%% Define the \replaced command
\newcommand{\replaced}[2]{%
%  \textcolor{red}{\st{#1}}%
  \textcolor{black}{#2}%
}


\begin{document}

%%%%%%%%%%%%%%%%%%%%%%%%%%%%%%%%%%%%%%%%%%%%%%%%%%%%%%%%%%%%%%%%
%%%%%%%%%%%%%%%%%%%%%% START OF THE PAPER %%%%%%%%%%%%%%%%%%%%%%
%%%%%%%%%%%%%%%%%%%%%%%%%%%%%%%%%%%%%%%%%%%%%%%%%%%%%%%%%%%%%%%%

%% The ``\maketitle'' command must be the first command after the
%% ``\begin{document}'' command. It prepares and prints the title block.
%% the only exception to this rule is the \firstsection command
\firstsection{Introduction}

\maketitle

\section{Introduction}

% \textcolor{red}{Still on working}

% \textcolor{red}{add label for each section}


Robot learning relies on diverse and high-quality data to learn complex behaviors \cite{aldaco2024aloha, wang2024dexcap}.
Recent studies highlight that models trained on datasets with greater complexity and variation in the domain tend to generalize more effectively across broader scenarios \cite{mann2020language, radford2021learning, gao2024efficient}.
% However, creating such diverse datasets in the real world presents significant challenges.
% Modifying physical environments and adjusting robot hardware settings require considerable time, effort, and financial resources.
% In contrast, simulation environments offer a flexible and efficient alternative.
% Simulations allow for the creation and modification of digital environments with a wide range of object shapes, weights, materials, lighting, textures, friction coefficients, and so on to incorporate domain randomization,
% which helps improve the robustness of models when deployed in real-world conditions.
% These environments can be easily adjusted and reset, enabling faster iterations and data collection.
% Additionally, simulations provide the ability to consistently reproduce scenarios, which is essential for benchmarking and model evaluation.
% Another advantage of simulations is their flexibility in sensor integration. Sensors such as cameras, LiDARs, and tactile sensors can be added or repositioned without the physical limitations present in real-world setups. Simulations also eliminate the risk of damaging expensive hardware during edge-case experiments, making them an ideal platform for testing rare or dangerous scenarios that are impractical to explore in real life.
By leveraging immersive perspectives and interactions, Extended Reality\footnote{Extended Reality is an umbrella term to refer to Augmented Reality, Mixed Reality, and Virtual Reality \cite{wikipediaExtendedReality}}
(XR)
is a promising candidate for efficient and intuitive large scale data collection \cite{jiang2024comprehensive, arcade}
% With the demand for collecting data, XR provides a promising approach for humans to teach robots by offering users an immersive experience.
in simulation \cite{jiang2024comprehensive, arcade, dexhub-park} and real-world scenarios \cite{openteach, opentelevision}.
However, reusing and reproducing current XR approaches for robot data collection for new settings and scenarios is complicated and requires significant effort.
% are difficult to reuse and reproduce system makes it hard to reuse and reproduce in another data collection pipeline.
This bottleneck arises from three main limitations of current XR data collection and interaction frameworks: \textit{asset limitation}, \textit{simulator limitation}, and \textit{device limitation}.
% \textcolor{red}{ASSIGN THESE CITATION PROPERLY:}
% \textcolor{red}{list them by time order???}
% of collecting data by using XR have three main limitations.
Current approaches suffering from \textit{asset limitation} \cite{arclfd, jiang2024comprehensive, arcade, george2025openvr, vicarios}
% Firstly, recent works \cite{jiang2024comprehensive, arcade, dexhub-park}
can only use predefined robot models and task scenes. Configuring new tasks requires significant effort, since each new object or model must be specifically integrated into the XR application.
% and it takes too much effort to configure new tasks in their systems since they cannot spawn arbitrary models in the XR application.
The vast majority of application are developed for specific simulators or real-world scenarios. This \textit{simulator limitation} \cite{mosbach2022accelerating, lipton2017baxter, dexhub-park, arcade}
% Secondly, existing systems are limited to a single simulation platform or real-world scenarios.
significantly reduces reusability and makes adaptation to new simulation platforms challenging.
Additionally, most current XR frameworks are designed for a specific version of a single XR headset, leading to a \textit{device limitation} 
\cite{lipton2017baxter, armada, openteach, meng2023virtual}.
% and there is no work working on the extendability of transferring to a new headsets as far as we know.
To the best of our knowledge, no existing work has explored the extensibility or transferability of their framework to different headsets.
These limitations hamper reproducibility and broader contributions of XR based data collection and interaction to the research community.
% as each research group typically has its own data collection pipeline.
% In addition to these main limitations, existing XR systems are not well suited for managing multiple robot systems,
% as they are often designed for single-operator use.

In addition to these main limitations, existing XR systems are often designed for single-operator use, prohibiting collaborative data collection.
At the same time, controlling multiple robots at once can be very difficult for a single operator,
making data collection in multi-robot scenarios particularly challenging \cite{orun2019effect}.
Although there are some works using collaborative data collection in the context of tele-operation \cite{tung2021learning, Qin2023AnyTeleopAG},
there is no XR-based data collection system supporting collaborative data collection.
This limitation highlights the need for more advanced XR solutions that can better support multi-robot and multi-user scenarios.
% \textcolor{red}{more papers about collaborative data collection}

To address all of these issues, we propose \textbf{IRIS},
an \textbf{I}mmersive \textbf{R}obot \textbf{I}nteraction \textbf{S}ystem.
This general system supports various simulators, benchmarks and real-world scenarios.
It is easily extensible to new simulators and XR headsets.
IRIS achieves generalization across six dimensions:
% \begin{itemize}
%     \item \textit{Cross-scene} : diverse object models;
%     \item \textit{Cross-embodiment}: diverse robot models;
%     \item \textit{Cross-simulator}: 
%     \item \textit{Cross-reality}: fd
%     \item \textit{Cross-platform}: fd
%     \item \textit{Cross-users}: fd
% \end{itemize}
\textbf{Cross-Scene}, \textbf{Cross-Embodiment}, \textbf{Cross-Simulator}, \textbf{Cross-Reality}, \textbf{Cross-Platform}, and \textbf{Cross-User}.

\textbf{Cross-Scene} and \textbf{Cross-Embodiment} allow the system to handle arbitrary objects and robots in the simulation,
eliminating restrictions about predefined models in XR applications.
IRIS achieves these generalizations by introducing a unified scene specification, representing all objects,
including robots, as data structures with meshes, materials, and textures.
The unified scene specification is transmitted to the XR application to create and visualize an identical scene.
By treating robots as standard objects, the system simplifies XR integration,
allowing researchers to work with various robots without special robot-specific configurations.
\textbf{Cross-Simulator} ensures compatibility with various simulation engines.
IRIS simplifies adaptation by parsing simulated scenes into the unified scene specification, eliminating the need for XR application modifications when switching simulators.
New simulators can be integrated by creating a parser to convert their scenes into the unified format.
This flexibility is demonstrated by IRIS’ support for Mujoco \cite{todorov2012mujoco}, IsaacSim \cite{mittal2023orbit}, CoppeliaSim \cite{coppeliaSim}, and even the recent Genesis \cite{Genesis} simulator.
\textbf{Cross-Reality} enables the system to function seamlessly in both virtual simulations and real-world applications.
IRIS enables real-world data collection through camera-based point cloud visualization.
\textbf{Cross-Platform} allows for compatibility across various XR devices.
Since XR device APIs differ significantly, making a single codebase impractical, IRIS XR application decouples its modules to maximize code reuse.
This application, developed by Unity \cite{unity3dUnityManual}, separates scene visualization and interaction, allowing developers to integrate new headsets by reusing the visualization code and only implementing input handling for hand, head, and motion controller tracking.
IRIS provides an implementation of the XR application in the Unity framework, allowing for a straightforward deployment to any device that supports Unity. 
So far, IRIS was successfully deployed to the Meta Quest 3 and HoloLens 2.
Finally, the \textbf{Cross-User} ability allows multiple users to interact within a shared scene.
IRIS achieves this ability by introducing a protocol to establish the communication between multiple XR headsets and the simulation or real-world scenarios.
Additionally, IRIS leverages spatial anchors to support the alignment of virtual scenes from all deployed XR headsets.
% To make an seamless user experience for robot learning data collection,
% IRIS also tested in three different robot control interface
% Furthermore, to demonstrate the extensibility of our approach, we have implemented a robot-world pipeline for real robot data collection, ensuring that the system can be used in both simulated and real-world environments.
The Immersive Robot Interaction System makes the following contributions\\
\textbf{(1) A unified scene specification} that is compatible with multiple robot simulators. It enables various XR headsets to visualize and interact with simulated objects and robots, providing an immersive experience while ensuring straightforward reusability and reproducibility.\\
\textbf{(2) A collaborative data collection framework} designed for XR environments. The framework facilitates enhanced robot data acquisition.\\
\textbf{(3) A user study} demonstrating that IRIS significantly improves data collection efficiency and intuitiveness compared to the LIBERO baseline.

% \begin{table*}[t]
%     \centering
%     \begin{tabular}{lccccccc}
%         \toprule
%         & \makecell{Physical\\Interaction}
%         & \makecell{XR\\Enabled}
%         & \makecell{Free\\View}
%         & \makecell{Multiple\\Robots}
%         & \makecell{Robot\\Control}
%         % Force Feedback???
%         & \makecell{Soft Object\\Supported}
%         & \makecell{Collaborative\\Data} \\
%         \midrule
%         ARC-LfD \cite{arclfd}                              & Real        & \cmark & \xmark & \xmark & Joint              & \xmark & \xmark \\
%         DART \cite{dexhub-park}                            & Sim         & \cmark & \cmark & \cmark & Cartesian          & \xmark & \xmark \\
%         \citet{jiang2024comprehensive}                     & Sim         & \cmark & \xmark & \xmark & Joint \& Cartesian & \xmark & \xmark \\
%         \citet{mosbach2022accelerating}                    & Sim         & \cmark & \cmark & \xmark & Cartesian          & \xmark & \xmark \\
%         ARCADE \cite{arcade}                               & Real        & \cmark & \cmark & \xmark & Cartesian          & \xmark & \xmark \\
%         Holo-Dex \cite{holodex}                            & Real        & \cmark & \xmark & \cmark & Cartesian          & \cmark & \xmark \\
%         ARMADA \cite{armada}                               & Real        & \cmark & \xmark & \cmark & Cartesian          & \cmark & \xmark \\
%         Open-TeleVision \cite{opentelevision}              & Real        & \cmark & \cmark & \cmark & Cartesian          & \cmark & \xmark \\
%         OPEN TEACH \cite{openteach}                        & Real        & \cmark & \xmark & \cmark & Cartesian          & \cmark & \cmark \\
%         GELLO \cite{wu2023gello}                           & Real        & \xmark & \cmark & \cmark & Joint              & \cmark & \xmark \\
%         DexCap \cite{wang2024dexcap}                       & Real        & \xmark & \cmark & \xmark & Cartesian          & \cmark & \xmark \\
%         AnyTeleop \cite{Qin2023AnyTeleopAG}                & Real        & \xmark & \xmark & \cmark & Cartesian          & \cmark & \cmark \\
%         Vicarios \cite{vicarios}                           & Real        & \cmark & \xmark & \xmark & Cartesian          & \cmark & \xmark \\     
%         Augmented Visual Cues \cite{augmentedvisualcues}   & Real        & \cmark & \cmark & \xmark & Cartesian          & \xmark & \xmark \\ 
%         \citet{wang2024robotic}                            & Real        & \cmark & \cmark & \xmark & Cartesian          & \cmark & \xmark \\
%         Bunny-VisionPro \cite{bunnyvisionpro}              & Real        & \cmark & \cmark & \cmark & Cartesian          & \cmark & \xmark \\
%         IMMERTWIN \cite{immertwin}                         & Real        & \cmark & \cmark & \cmark & Cartesian          & \xmark & \xmark \\
%         \citet{meng2023virtual}                            & Sim \& Real & \cmark & \cmark & \xmark & Cartesian          & \xmark & \xmark \\
%         Shared Control Framework \cite{sharedctlframework} & Real        & \cmark & \cmark & \cmark & Cartesian          & \xmark & \xmark \\
%         OpenVR \cite{openvr}                               & Real        & \cmark & \cmark & \xmark & Cartesian          & \xmark & \xmark \\
%         \citet{digitaltwinmr}                              & Real        & \cmark & \cmark & \xmark & Cartesian          & \cmark & \xmark \\
        
%         \midrule
%         \textbf{Ours} & Sim \& Real & \cmark & \cmark & \cmark & Joint \& Cartesian  & \cmark & \cmark \\
%         \bottomrule
%     \end{tabular}
%     \caption{This is a cross-column table with automatic line breaking.}
%     \label{tab:cross-column}
% \end{table*}

% \begin{table*}[t]
%     \centering
%     \begin{tabular}{lccccccc}
%         \toprule
%         & \makecell{Cross-Embodiment}
%         & \makecell{Cross-Scene}
%         & \makecell{Cross-Simulator}
%         & \makecell{Cross-Reality}
%         & \makecell{Cross-Platform}
%         & \makecell{Cross-User} \\
%         \midrule
%         ARC-LfD \cite{arclfd}                              & \xmark & \xmark & \xmark & \xmark & \xmark & \xmark \\
%         DART \cite{dexhub-park}                            & \cmark & \cmark & \xmark & \xmark & \xmark & \xmark \\
%         \citet{jiang2024comprehensive}                     & \xmark & \cmark & \xmark & \xmark & \xmark & \xmark \\
%         \citet{mosbach2022accelerating}                    & \xmark & \cmark & \xmark & \xmark & \xmark & \xmark \\
%         ARCADE \cite{arcade}                               & \xmark & \xmark & \xmark & \xmark & \xmark & \xmark \\
%         Holo-Dex \cite{holodex}                            & \cmark & \xmark & \xmark & \xmark & \xmark & \xmark \\
%         ARMADA \cite{armada}                               & \cmark & \xmark & \xmark & \xmark & \xmark & \xmark \\
%         Open-TeleVision \cite{opentelevision}              & \cmark & \xmark & \xmark & \xmark & \cmark & \xmark \\
%         OPEN TEACH \cite{openteach}                        & \cmark & \xmark & \xmark & \xmark & \xmark & \cmark \\
%         GELLO \cite{wu2023gello}                           & \cmark & \xmark & \xmark & \xmark & \xmark & \xmark \\
%         DexCap \cite{wang2024dexcap}                       & \xmark & \xmark & \xmark & \xmark & \xmark & \xmark \\
%         AnyTeleop \cite{Qin2023AnyTeleopAG}                & \cmark & \cmark & \cmark & \cmark & \xmark & \cmark \\
%         Vicarios \cite{vicarios}                           & \xmark & \xmark & \xmark & \xmark & \xmark & \xmark \\     
%         Augmented Visual Cues \cite{augmentedvisualcues}   & \xmark & \xmark & \xmark & \xmark & \xmark & \xmark \\ 
%         \citet{wang2024robotic}                            & \xmark & \xmark & \xmark & \xmark & \xmark & \xmark \\
%         Bunny-VisionPro \cite{bunnyvisionpro}              & \cmark & \xmark & \xmark & \xmark & \xmark & \xmark \\
%         IMMERTWIN \cite{immertwin}                         & \cmark & \xmark & \xmark & \xmark & \xmark & \xmark \\
%         \citet{meng2023virtual}                            & \xmark & \cmark & \xmark & \cmark & \xmark & \xmark \\
%         \citet{sharedctlframework}                         & \cmark & \xmark & \xmark & \xmark & \xmark & \xmark \\
%         OpenVR \cite{george2025openvr}                               & \xmark & \xmark & \xmark & \xmark & \xmark & \xmark \\
%         \citet{digitaltwinmr}                              & \xmark & \xmark & \xmark & \xmark & \xmark & \xmark \\
        
%         \midrule
%         \textbf{Ours} & \cmark & \cmark & \cmark & \cmark & \cmark & \cmark \\
%         \bottomrule
%     \end{tabular}
%     \caption{This is a cross-column table with automatic line breaking.}
% \end{table*}

% \begin{table*}[t]
%     \centering
%     \begin{tabular}{lccccccc}
%         \toprule
%         & \makecell{Cross-Scene}
%         & \makecell{Cross-Embodiment}
%         & \makecell{Cross-Simulator}
%         & \makecell{Cross-Reality}
%         & \makecell{Cross-Platform}
%         & \makecell{Cross-User}
%         & \makecell{Control Space} \\
%         \midrule
%         % Vicarios \cite{vicarios}                           & \xmark & \xmark & \xmark & \xmark & \xmark & \xmark \\     
%         % Augmented Visual Cues \cite{augmentedvisualcues}   & \xmark & \xmark & \xmark & \xmark & \xmark & \xmark \\ 
%         % OpenVR \cite{george2025openvr}                     & \xmark & \xmark & \xmark & \xmark & \xmark & \xmark \\
%         \citet{digitaltwinmr}                              & \xmark & \xmark & \xmark & \xmark & \xmark & \xmark &  \\
%         ARC-LfD \cite{arclfd}                              & \xmark & \xmark & \xmark & \xmark & \xmark & \xmark &  \\
%         \citet{sharedctlframework}                         & \cmark & \xmark & \xmark & \xmark & \xmark & \xmark &  \\
%         \citet{jiang2024comprehensive}                     & \cmark & \xmark & \xmark & \xmark & \xmark & \xmark &  \\
%         \citet{mosbach2022accelerating}                    & \cmark & \xmark & \xmark & \xmark & \xmark & \xmark & \\
%         Holo-Dex \cite{holodex}                            & \cmark & \xmark & \xmark & \xmark & \xmark & \xmark & \\
%         ARCADE \cite{arcade}                               & \cmark & \cmark & \xmark & \xmark & \xmark & \xmark & \\
%         DART \cite{dexhub-park}                            & Limited & Limited & Mujoco & Sim & Vision Pro & \xmark &  Cartesian\\
%         ARMADA \cite{armada}                               & \cmark & \cmark & \xmark & \xmark & \xmark & \xmark & \\
%         \citet{meng2023virtual}                            & \cmark & \cmark & \xmark & \cmark & \xmark & \xmark & \\
%         % GELLO \cite{wu2023gello}                           & \cmark & \xmark & \xmark & \xmark & \xmark & \xmark \\
%         % DexCap \cite{wang2024dexcap}                       & \xmark & \xmark & \xmark & \xmark & \xmark & \xmark \\
%         % AnyTeleop \cite{Qin2023AnyTeleopAG}                & \cmark & \cmark & \cmark & \cmark & \xmark & \cmark \\
%         % \citet{wang2024robotic}                            & \xmark & \xmark & \xmark & \xmark & \xmark & \xmark \\
%         Bunny-VisionPro \cite{bunnyvisionpro}              & \cmark & \cmark & \xmark & \xmark & \xmark & \xmark & \\
%         IMMERTWIN \cite{immertwin}                         & \cmark & \cmark & \xmark & \xmark & \xmark & \xmark & \\
%         Open-TeleVision \cite{opentelevision}              & \cmark & \cmark & \xmark & \xmark & \cmark & \xmark & \\
%         \citet{szczurek2023multimodal}                     & \xmark & \xmark & \xmark & Real & \xmark & \cmark & \\
%         OPEN TEACH \cite{openteach}                        & \cmark & \cmark & \xmark & \xmark & \xmark & \cmark & \\
%         \midrule
%         \textbf{Ours} & \cmark & \cmark & \cmark & \cmark & \cmark & \cmark \\
%         \bottomrule
%     \end{tabular}
%     \caption{TODO, Bruce: this table can be further optimized.}
% \end{table*}

\definecolor{goodgreen}{HTML}{228833}
\definecolor{goodred}{HTML}{EE6677}
\definecolor{goodgray}{HTML}{BBBBBB}

\begin{table*}[t]
    \centering
    \begin{adjustbox}{max width=\textwidth}
    \renewcommand{\arraystretch}{1.2}    
    \begin{tabular}{lccccccc}
        \toprule
        & \makecell{Cross-Scene}
        & \makecell{Cross-Embodiment}
        & \makecell{Cross-Simulator}
        & \makecell{Cross-Reality}
        & \makecell{Cross-Platform}
        & \makecell{Cross-User}
        & \makecell{Control Space} \\
        \midrule
        % Vicarios \cite{vicarios}                           & \xmark & \xmark & \xmark & \xmark & \xmark & \xmark \\     
        % Augmented Visual Cues \cite{augmentedvisualcues}   & \xmark & \xmark & \xmark & \xmark & \xmark & \xmark \\ 
        % OpenVR \cite{george2025openvr}                     & \xmark & \xmark & \xmark & \xmark & \xmark & \xmark \\
        \citet{digitaltwinmr}                              & \textcolor{goodred}{Limited}     & \textcolor{goodred}{Single Robot} & \textcolor{goodred}{Unity}    & \textcolor{goodred}{Real}          & \textcolor{goodred}{Meta Quest 2} & \textcolor{goodgray}{N/A} & \textcolor{goodred}{Cartesian} \\
        ARC-LfD \cite{arclfd}                              & \textcolor{goodgray}{N/A}        & \textcolor{goodred}{Single Robot} & \textcolor{goodgray}{N/A}     & \textcolor{goodred}{Real}          & \textcolor{goodred}{HoloLens}     & \textcolor{goodgray}{N/A} & \textcolor{goodred}{Cartesian} \\
        \citet{sharedctlframework}                         & \textcolor{goodred}{Limited}     & \textcolor{goodred}{Single Robot} & \textcolor{goodgray}{N/A}     & \textcolor{goodred}{Real}          & \textcolor{goodred}{HTC Vive Pro} & \textcolor{goodgray}{N/A} & \textcolor{goodred}{Cartesian} \\
        \citet{jiang2024comprehensive}                     & \textcolor{goodred}{Limited}     & \textcolor{goodred}{Single Robot} & \textcolor{goodgray}{N/A}     & \textcolor{goodred}{Real}          & \textcolor{goodred}{HoloLens 2}   & \textcolor{goodgray}{N/A} & \textcolor{goodgreen}{Joint \& Cartesian} \\
        \citet{mosbach2022accelerating}                    & \textcolor{goodgreen}{Available} & \textcolor{goodred}{Single Robot} & \textcolor{goodred}{IsaacGym} & \textcolor{goodred}{Sim}           & \textcolor{goodred}{Vive}         & \textcolor{goodgray}{N/A} & \textcolor{goodgreen}{Joint \& Cartesian} \\
        Holo-Dex \cite{holodex}                            & \textcolor{goodgray}{N/A}        & \textcolor{goodred}{Single Robot} & \textcolor{goodgray}{N/A}     & \textcolor{goodred}{Real}          & \textcolor{goodred}{Meta Quest 2} & \textcolor{goodgray}{N/A} & \textcolor{goodred}{Joint} \\
        ARCADE \cite{arcade}                               & \textcolor{goodgray}{N/A}        & \textcolor{goodred}{Single Robot} & \textcolor{goodgray}{N/A}     & \textcolor{goodred}{Real}          & \textcolor{goodred}{HoloLens 2}   & \textcolor{goodgray}{N/A} & \textcolor{goodred}{Cartesian} \\
        DART \cite{dexhub-park}                            & \textcolor{goodred}{Limited}     & \textcolor{goodred}{Limited}      & \textcolor{goodred}{Mujoco}   & \textcolor{goodred}{Sim}           & \textcolor{goodred}{Vision Pro}   & \textcolor{goodgray}{N/A} & \textcolor{goodred}{Cartesian} \\
        ARMADA \cite{armada}                               & \textcolor{goodgray}{N/A}        & \textcolor{goodred}{Limited}      & \textcolor{goodgray}{N/A}     & \textcolor{goodred}{Real}          & \textcolor{goodred}{Vision Pro}   & \textcolor{goodgray}{N/A} & \textcolor{goodred}{Cartesian} \\
        \citet{meng2023virtual}                            & \textcolor{goodred}{Limited}     & \textcolor{goodred}{Single Robot} & \textcolor{goodred}{PhysX}   & \textcolor{goodgreen}{Sim \& Real} & \textcolor{goodred}{HoloLens 2}   & \textcolor{goodgray}{N/A} & \textcolor{goodred}{Cartesian} \\
        % GELLO \cite{wu2023gello}                           & \cmark & \xmark & \xmark & \xmark & \xmark & \xmark \\
        % DexCap \cite{wang2024dexcap}                       & \xmark & \xmark & \xmark & \xmark & \xmark & \xmark \\
        % AnyTeleop \cite{Qin2023AnyTeleopAG}                & \cmark & \cmark & \cmark & \cmark & \xmark & \cmark \\
        % \citet{wang2024robotic}                            & \xmark & \xmark & \xmark & \xmark & \xmark & \xmark \\
        Bunny-VisionPro \cite{bunnyvisionpro}              & \textcolor{goodgray}{N/A}        & \textcolor{goodred}{Single Robot} & \textcolor{goodgray}{N/A}     & \textcolor{goodred}{Real}          & \textcolor{goodred}{Vision Pro}   & \textcolor{goodgray}{N/A} & \textcolor{goodred}{Cartesian} \\
        IMMERTWIN \cite{immertwin}                         & \textcolor{goodgray}{N/A}        & \textcolor{goodred}{Limited}      & \textcolor{goodgray}{N/A}     & \textcolor{goodred}{Real}          & \textcolor{goodred}{HTC Vive}     & \textcolor{goodgray}{N/A} & \textcolor{goodred}{Cartesian} \\
        Open-TeleVision \cite{opentelevision}              & \textcolor{goodgray}{N/A}        & \textcolor{goodred}{Limited}      & \textcolor{goodgray}{N/A}     & \textcolor{goodred}{Real}          & \textcolor{goodgreen}{Meta Quest, Vision Pro} & \textcolor{goodgray}{N/A} & \textcolor{goodred}{Cartesian} \\
        \citet{szczurek2023multimodal}                     & \textcolor{goodgray}{N/A}        & \textcolor{goodred}{Limited}      & \textcolor{goodgray}{N/A}     & \textcolor{goodred}{Real}          & \textcolor{goodred}{HoloLens 2}   & \textcolor{goodgreen}{Available} & \textcolor{goodred}{Joint \& Cartesian} \\
        OPEN TEACH \cite{openteach}                        & \textcolor{goodgray}{N/A}        & \textcolor{goodgreen}{Available}  & \textcolor{goodgray}{N/A}     & \textcolor{goodred}{Real}          & \textcolor{goodred}{Meta Quest 3} & \textcolor{goodred}{N/A} & \textcolor{goodgreen}{Joint \& Cartesian} \\
        \midrule
        \textbf{Ours}                                      & \textcolor{goodgreen}{Available} & \textcolor{goodgreen}{Available}  & \textcolor{goodgreen}{Mujoco, CoppeliaSim, IsaacSim} & \textcolor{goodgreen}{Sim \& Real} & \textcolor{goodgreen}{Meta Quest 3, HoloLens 2} & \textcolor{goodgreen}{Available} & \textcolor{goodgreen}{Joint \& Cartesian} \\
        \bottomrule
        \end{tabular}
    \end{adjustbox}
    \caption{Comparison of XR-based system for robots. IRIS is compared with related works in different dimensions.}
\end{table*}


%
We position CST with respect to current frameworks for discrimination testing along the goals of actionability and meaningfulness.
Later in Section~\ref{sec:CausalKnowledge} we discuss the role of causality for conceiving discrimination.
For a broader, multidisciplinary view on discrimination testing, we refer to the survey by~\textcite{Romei2014MultiSurveyDiscrimination}. 
For a recent survey of the fair ML testing literature, see \textcite{DBLP:journals/tosem/ChenZHHS24}.

Regarding actionability, it is important when proving discrimination to insure that the framework accounts for sources of randomness in the decision-making process. Popular non-algorithmic frameworks---such as natural \parencite{Godin2000Orchestra} and field \parencite{Bertrand2017_FieldExperimentDiscrimination} experiments, audit \parencite{Fix&Struyk1993_ClearConvincingEvidence} and correspondence \parencite{Bertrand2004_EmilyAndGreg, Rooth2021} studies---address this issue by using multiple observations to build inferential statistics. Similar statistics are asked in court for proving discrimination \parencite[Section 6.3]{EU2018_NonDiscriminationLaw}. 
Few algorithmic frameworks address this issue due to model complexity preventing formal inference \parencite{Athey2019MachineLearningForEconomists}. An exception are data mining frameworks for discrimination discovery \parencite{DBLP:conf/kdd/PedreschiRT08, DBLP:journals/tkdd/RuggieriPT10} that operationalize the non-algorithmic notions, including situation testing \parencite{Thanh_KnnSituationTesting2011, Zhang_CausalSituationTesting_2016}.
These frameworks \parencite{TR-DBLP:conf/sigsoft/GalhotraBM17, TR-DBLP:journals/corr/abs-1809-03260, DBLP:journals/jiis/QureshiKKRP20} keep the focus on comparing multiple control-test instances for making individual claims, providing evidence similar to that produced by the quantitative tools used in court.
It remains unclear if the same can be said about existing causal fair machine learning methods
% \parencite{DBLP:journals/jlap/MakhloufZP24} 
as these have yet to be used beyond academic circles.
The suitability of algorithmic fairness methods for testing discrimination, be it or not ADM, remains an ongoing discussion \parencite{DBLP:conf/fat/WeertsXTOP23}.

Regarding meaningfulness, situation testing and the other methods previously mentioned have been criticized for their handling of the counterfactual question behind the causal model of discrimination \parencite{Kohler2018CausalEddie, Hu_facct_sex_20, Kasirzadeh2021UseMisuse}. In particular, these actionable methods take for granted the influence of the protected attribute on all other attributes. This point can be seen, e.g.,~in how situation testing constructs the test group, which is equivalent to changing the protected attribute while keeping everything else equal. Such an approach goes against how most social scientists interpret the protected attribute and its role as a social construct when proving discrimination \parencite{Bonilla1997_RethinkingRace, rose_constructivist_2022, Sen2016_RaceABundle, Hanna2020_CriticalRace}. 
It is in that regard where structural causal models \parencite{PearlCausality2009} and their ability to generate counterfactuals via the abduction, action, and prediction steps (e.g.,~\textcite{Chiappa2019_PathCF, Yang2021_CausalIntersectionality}), including counterfactual fairness \parencite{Kusner2017CF}, have an advantage.
This advantage is overlooked by critics of counterfactual reasoning \parencite{Kasirzadeh2021UseMisuse, Hu_facct_sex_20}: generating counterfactuals, as long as the structural causal model is properly specified, accounts for the effects of changing the protected attribute on all other attributes. 
Hence, a framework like counterfactual fairness, relative to situation testing and other discrimination discovery methods, is more meaningful in its handling of protected attributes. 

CST bridges these two lines of work, borrowing the actionability aspects from frameworks like situation testing and meaningful aspects from frameworks like counterfactual fairness. 
% Intuitively, 
Counterfactual generation allows to create a comparator for the complainant that accounts for the influence of the protected attribute on the other attributes, departing from the idealized comparison.
It is not far, conceptually, from the broader ML problem of learning fair representations \parencite{Zemel2013LearningFairRepresentations} since we wish to learn (read, map) a new representation of the complainant that reflects where it would have been had it belonged to the non-protected group. 
It is a normative claim on what a non-protected instance similar to the complainant looks like.
Beyond counterfactual fairness and derivatives (e.g., \textcite{Chiappa2019_PathCF}), other works address this problem of deriving such a pair for the complainant.
For instance, \textcite{Plevcko2020FairDataAdaptation} use a quantile regression approach while \textcite{DBLP:journals/corr/abs-2307-12797} use a residual-based approach for generating the pair.
Both works rely on having access to a structural causal model, but do not exploit the abduction, action, and prediction steps for generating counterfactual distributions.
\textcite{BlackYF20_FlipTest}, instead, propose the FlipTest, a non-causal approach that uses an optimal transport mapping to derive the pair for the complainant.
These three works exemplify ML methods that use counterfactual reasoning to operationalize different interpretations of individual similarity. 
With CST we align with these and similar efforts to propose an alternative to the idealized comparison often used in discrimination testing.
% \sr{Isn't propensity score another form of representation learning?}

%
% EOS
%

\section{Methods}\label{sec:Methods}

\begin{figure*}
    \centering
    \includegraphics[width=\textwidth]{figures/Pipeline2.png}
    \caption{\textbf{Overview of the Proposed Pipeline.} The pipeline adds a visualization module, including augmented reality and virtual reality modalities, enhanced by a virtual human assistant. The system architecture facilitates real-time interaction and synchronization between the robotic ultrasound system, the virtual agent, and the patient, providing a more engaging and patient-centered experience.}
    \label{fig:overview}
\end{figure*}


In this work, we propose a pipeline, as illustrated in Fig.~\ref{fig:overview}, that introduces a visualization module into the existing robotic ultrasound system (RUS), enhancing the patient experience through conversational virtual agent and immersive visualizations. 
In the following subsections, we detail the components and theoretical frameworks behind the conversational virtual agent, the different visualization modalities (AR, \revised{AV}, and fully immersive VR), as well as the registration process that ensures the seamless integration of virtual and real-world elements.

\subsection{Force-Compliant Robotic Ultrasound System}

Our robotic ultrasound system employs a force-compliant control approach based on impedance control~\cite{jiang2020automatic,jiang2021autonomous,jiang2021deformation}, which ensures safe and accurate probe positioning during the ultrasound procedure by regulating both contact force and probe orientation. This control system operates as a spring-like mechanism with predefined stiffness values, ensuring that if an obstacle, such as the patient’s body, is encountered, the robot will either bypass with a reduced force or stop at the location to avoid applying excessive force.

Impedance control modulates the interaction forces between the robotic arm and the patient’s tissue by adjusting the robot’s stiffness, damping, and inertia parameters. The control law is defined as:
\begin{equation}
\tau = J^T \left[ F_d + K_m e + D \dot{e} + M \ddot{e} \right]
\end{equation}
where $\tau$ represents the computed joint torques, $J^T$ is the transposed Jacobian matrix, and $e$ is the pose error between the current and target positions. The stiffness, damping, and inertia matrices are denoted by $K_m$, $D$, and $M$, respectively. The desired force $F_d$ controls the applied contact force. To ensure stable force along the probe’s centerline, the robot uses a 1-DOF compliant controller for force control and a 5-DOF position controller for precise positioning.

This force-compliant control \revised{follows the same formulation used in Hennersperger et al.\cite{hennersperger2017towards} and Jiang et al.\cite{jiang2021autonomous}, which} allows the system to adapt to soft tissue deformations and variations in probe position, preventing excessive force that could cause discomfort or harm. \add{To enable communication between the RUS and the visualization module, messages are serialized and passed between the two. This ensures that the visualization module is always aware of the robot’s state and the current phase of the procedure. Additionally, commands produced through user interactions with the conversational virtual agent in the visualization module can be sent back to the RUS via serialized messages. This bidirectional communication enables real-time synchronization and seamless integration of robotic control and immersive visualizations.}


\subsection{Conversational Virtual Agent}

The conversational virtual agent serves as a central component of the proposed visualizations, aiming to enhance patient interaction by natural communication and reduce the sense of isolation commonly associated with robotic procedures~\cite{almeras2019operating}, therefore creating a more humanized and engaging experience.

At the core of the agent’s functionality is a speech-to-text (STT) system that transcribes the patient’s spoken words into textual input. The transcribed text is then processed by a large language model (LLM), which generates contextually appropriate responses based on pre-configured prompt. The use of an LLM ensures that the agent provides responses that are both relevant to the patient’s situation and emotionally supportive, fostering a sense of trust and ease during the procedure.
Once the response is generated, it is converted back into audible speech through a text-to-speech (TTS) engine, which provides natural output. The TTS system matches the assistant’s appearance and personality, ensuring consistent and realistic verbal interactions. \add{Similar technical pipelines of the conversational intelligent agent have been used in domains such as virtual museums~\cite{garcia2024speaking} and educational VR applications~\cite{yang2024effects}.}

In addition to the agent’s verbal responses, various animations are employed to enhance its realism and human-like presence~\cite{yu2021avatars}. These include eye blinking, mouth movements synchronized with speech generated by the TTS system, and a subtle idle breathing animation to give the avatar a more natural appearance. Beyond these baseline animations, inverse kinematics (IK) is used to further enhance realism by controlling the assistant’s head and hand movement. The IK algorithm calculates natural, human-like gestures in response to the patient’s position and the state of the robot. \revised{Studies have shown that non-verbal cues such as gestures, head movements, and subtle body language significantly enhance user immersion and realism in virtual environments~\cite{etienne2023perception, beck2012emotional}. Adapting these findings to our robotic ultrasound use case, we implemented a set of non-verbal behaviors to increase the agent’s presence and engagement.} For example, the assistant’s head turns toward the patient when they speak, and its hand animates to hold the ultrasound probe when the robot moves it within the agent’s arm reach, as though the virtual assistant were guiding the procedure. This visual and behavioral consistency strengthens the impression of the virtual agent being physically present and actively engaged with the patient.

Together, these elements form the foundation of the conversational virtual agent framework. The integration of real-time speech, intelligent language processing, and naturalistic physical interaction ensures that the agent serves as a comforting presence, enhancing the overall patient experience.

\subsection{Augmented Reality Visualization}

The AR visualization, illustrated in Fig.~\ref{fig:teaser}a, is designed to blend virtual elements with the patient’s real-world environment, allowing for seamless integration of the guidance from virtual human agent within a familiar, physical space. This approach maintains situational awareness by allowing the patient to see both their surroundings and the robotic arm at work while interacting with the assistant. 
In this visualization, the avatar appears seated next to the patient, engaging with them through real-time conversation. Through the HMD, the patient can see both the virtual assistant and the ultrasound probe, which the avatar is holding and guiding over the patient’s body. As the robotic arm moves the ultrasound probe in the physical world, the avatar’s virtual hand mimics these motions synchronously. This visual configuration provides the patient with the comforting illusion that the human-like avatar is controlling the robotic procedure, reducing the sense of detachment associated with the automated process.


\subsection{\revised{Augmented Virtuality} Visualization}

The \replaced{VR Passthrough}{AV} visualization, as demonstrated in Fig.~\ref{fig:teaser}b, combines an immersive virtual environment with selective real-world visibility through a passthrough window. This allows the patient to remain visually engaged with the area of their body being scanned while still interacting with the virtual human assistant in the virtual world. The virtual environment mirrors the physical room, but the robotic ultrasound system itself is not visible; only the ultrasound probe becomes visible when it enters the passthrough window, allowing the patient to see it during the scan.
Within the virtual environment, the patient can also see a virtual representation of the ultrasound probe, which mirrors the position of the real probe. As the robotic arm drives the real ultrasound probe toward the patient, the virtual probe moves correspondingly, providing a visual cue that the probe is approaching. When the probe transitions into the passthrough window, the virtual probe seamlessly aligns with and transitions into the real probe, ensuring continuity and reassuring the patient that the procedure is progressing as expected.

In this visualization, the virtual human assistant interacts with the patient in the same way as in the AR setting. The assistant is positioned to give the impression of guiding the robotic ultrasound. The patient’s view of the scan area remains in the passthrough window.
This visualization preserves the patient’s sense of presence and control by allowing them to observe their body during the scan. At the same time, the virtual environment and virtual human assistant help to reduce anxiety with guidance and interaction throughout the procedure.

\subsection{Fully Immersive Virtual Reality Visualization}

The fully immersive VR visualization, as shown in Fig.~\ref{fig:teaser}c, places the patient entirely in a virtual environment, removing any direct visual connection to the real world. In this setting, the virtual environment mirrors the physical room, but the robotic ultrasound system is absent as well, reinforcing the illusion that the virtual assistant is in full control of the procedure. The patient perceives the virtual agent as the sole operator, guiding the ultrasound probe, which enhances the feeling of human involvement and control.

In this visualization, the patient sees a virtual replica of their own body. As the robot moves the ultrasound probe on the patient’s actual body, the virtual probe in the VR environment moves in sync on the virtual body. This synchronization ensures that the patient feels a consistent tactile and visual connection between what they see in the virtual space and what they feel in the real world.
This immersive environment is designed to reduce anxiety by eliminating the often sterile, impersonal atmosphere of traditional robotic procedures, replacing it with a comforting virtual experience where the patient feels human involvement and control throughout. By maintaining real-time synchronization with their physical body and providing a strong sense of presence through the virtual human assistant, the visualization offers a fully immersive alternative that fosters comfort and trust in the procedure.

\subsection{Registration of Virtual and Real Components}

\begin{figure}
    \centering
    \includegraphics[width=0.85\columnwidth]{figures/registration.png}
    \caption{\add{\textbf{Registration of Virtual and Real Robot Using Predefined Points.} Predefined points on the virtual robot are shown in green. We marked the corresponding points on the real robot with the HMD, shown in orange. The dashed lines represent the transformation matrix $W_T$ to be solved, which aligns the two point sets.}}
    \label{fig:registration}
\end{figure}


Accurate registration between the virtual and real-world components is crucial for maintaining spatial coherence across all visualizations. To achieve this, we utilize a set of predefined, ordered points on both the virtual and physical elements to perform 3D-to-3D registration, similar to the approaches used in previous works~\cite{song2022happy,yu2022duplicated}. For instance, several key points $P_{virtual}(i)$ are selected on the virtual robot, which correspond to matching points marked on the real robotic system $P_{real}(i)$. Similarly, for the virtual environment, corner points of the physical room are used to align the virtual representation of the room with the real one. The transformation matrix  $W_T$  between these point sets is computed to minimize the difference between their positions. This can be expressed mathematically as:

\begin{equation}
	W_T = \arg\min_{\widehat{W}_T} d\left(\widehat{W}_T P_{virtual}(i), P_{real}(i)\right)
\end{equation}
where $W_T$ is the estimated transformation matrix between the real and virtual spaces, and $d$ represents the distance between corresponding points in the two spaces. \add{The registration process is visually illustrated in Fig.~\ref{fig:registration}.} We further decompose this transformation into a rotation component $W_R$ and a translation component $W_t$, solving each sequentially using the Kabsch algorithm~\cite{kabsch1993automatic}.

After the initial alignment is achieved, we can leverage the spatial anchor feature offered by modern HMDs to maintain this alignment between sessions. This means the calibration procedure only needs to be performed once, unless fine-tuning or refinements are required in future sessions. By using spatial anchors, the system ensures persistent alignment, enhancing the immersive experience for the patient without the need for repeated calibrations.

\vspace{-3mm}
\section{Experiments}
\vspace{-1mm}

We evaluate our proposed methods using both synthetic and real-world datasets to address three key empirical questions: \\[-2em]
\begin{itemize}[label=$\diamond$,leftmargin=*]
    \item How fast is our customized PAVA algorithm in evaluating $\prox_g$ compared to existing solvers?
    \item How fast is our proposed FISTA method in calculating the lower bounds compared to existing solvers?
    \item How fast is our customized BnB algorithm compared to existing solvers?
\end{itemize}
We implement our algorithms in python.

For baselines, we compare with the following state-of-the-art commercial and open-source SOCP solvers: Gurobi~\citep{gurobi}, MOSEK~\citep{mosek}, SCS~\citep{scs}, and Clarabel~\cite{Clarabel}, with the python package cvxpy~\cite{cvxpy} as the interface to these solvers.

\vspace{-2mm}
\subsection{How Fast Can We Evaluate $\text{prox}_{\rho^{-1} g}(\cdot)$?}

\begin{figure*}[!htb]
    % \vspace{-0.5em}
    \centering
    \includegraphics[width=0.85\textwidth]{sections/Plots/prox_comparison/prox_comparison.png}
    \vspace{-1em}
    \caption{Running time comparison of evaluating the proximal operators, for both $g$ (left) and $g^*$ (right).
    The baselines evaluate the proximal operators by directly solving the corresponding second-order conic problems (SOCP), respectively.}
    \label{fig:prox_comparison}
    \vspace{-3mm}
\end{figure*}

\begin{figure*}[!htb]
    \centering
    \includegraphics[width=0.85\textwidth]{sections/Plots/big_M_perturbation/convex_relaxation_comparison_n_p_ratio_1.0_M_2.0.png}
    \vspace{-1em}
    \caption{Running time comparison of solving Problem~\eqref{obj:original_sparse_problem_perspective_formulation_convex_relaxation}, the perspective relaxation of the original MIP problem.
    We set $M=2.0$, $\lambda_2=1.0$, and $n$-to-$p$ ratio to be 1. Gurobi cannot solve the cardinality constrained logistic regression problem.}
    \label{fig:solve_convex_relaxation_main_paper}
    \vspace{-5mm}
\end{figure*}

In this subsection, we demonstrate the computational efficiency of using our PAVA algorithm for evaluating the proximal operators.
We conduct the comparisons in two ways --- evaluating both a) the proximal operator of the original function $g$ and b) the proximal operator of its conjugate $g^*$.
Detailed experimental configurations, including parameter specifications and synthetic data generation process, are provided in Appendix~\ref{appendix:setup_for_evaluating_proximal_operators}.

The results shown in Figure~\ref{fig:prox_comparison} highlight the superiority of our method.
Our algorithm achieves a computational speedup of  approximately two orders of magnitude compared to conventional SOCP solvers.
This performance gain is largely due to our customized PAVA implementation in Algorithm~\ref{alg:PAVA_algorithm}.
For instance, in high-dimensional settings ($p=10^5$), baseline methods require several seconds to minutes to evaluate the proximal operators, whereas our approach completes the same task in 0.01 seconds.
Additionally, our method guarantees \textit{exact} solutions to the optimization problem, in contrast to the approximate solutions returned by the baselines.
This combination of precision and efficiency constitutes a critical advantage for our first-order optimization framework over generic conic programming solvers, as demonstrated in subsequent sections.

\vspace{-2mm}
\subsection{How Fast Can We Calculate the Lower Bound?}
\vspace{-1mm}

\begin{figure*}[!htb]
    \centering
    \includegraphics[width=0.95\textwidth]{sections/Plots/RestartedFISTA_linear_convergence_rate/convergence_comparison.png}
    \vspace{-2mm}
    \caption{Empirical convergence rate of our restarted FISTA (compared with PGD, the proximal gradient method, and FISTA) on solving the perspective relaxation in Problem~\eqref{obj:original_sparse_problem_perspective_formulation_convex_relaxation} with the logistic loss, $n=16000, p=16000, k=10, \rho=0.5, \lambda_2=1.0, \text{ and } M=2.0$. }
    \label{fig:RestartedFISTA_linear_convergence_rate}
    \vspace{-3mm}
\end{figure*}


% Please add the following required packages to your document preamble:
% \usepackage{multirow}
% \usepackage{graphicx}
\begin{table}[]
\centering
\vspace{-2mm}
\caption{GPU acceleration of our method on the linear regression task. Top and bottom rows correspond to the mean and standard deviation of running times (seconds).}
\vspace{2mm}
\label{tab:GPU_acceleration}
\resizebox{\columnwidth}{!}{%
\begin{tabular}{cccccc}
\toprule
p & 1k & 2k & 4k & 8k & 16k \\ \hline
\multirow{2}{*}{ours CPU} & 0.19 & 0.48 & 1.54 & 4.80 & 19.52 \\
 & (0.01) & (0.05) & (0.21) & (0.57) & (1.27) \\ \hline
\multirow{2}{*}{ours GPU} & 0.29 & 0.19 & 0.26 & 0.59 & 2.09 \\
 & (0.04) & (0.02) & (0.02) & (0.08) & (0.11)\\
 \bottomrule
\end{tabular}%
}
\vspace{-5mm}
\end{table}

% % Please add the following required packages to your document preamble:
% % \usepackage{graphicx}
% \begin{table*}[!ht]
% \centering
% \caption{}
% \label{tab:my-table}
% \resizebox{\textwidth}{!}{%
% \begin{tabular}{lcccccc}
% \toprule
%  & \multicolumn{2}{c}{ours} & \multicolumn{2}{c}{Gurobi} & \multicolumn{2}{c}{MOSEK} \\
%  & \multicolumn{1}{l}{time (s)} & \multicolumn{1}{l}{optimality gap (\%)} & \multicolumn{1}{l}{time (s)} & \multicolumn{1}{l}{optimality gap (\%)} & \multicolumn{1}{l}{time (s)} & \multicolumn{1}{l}{optimality gap (\%)} \\ \hline
% \begin{tabular}[c]{@{}l@{}}Linear Regression\\ Synthetic \\ (n=16000, p=16000)\end{tabular} & 57 & 0.0 & 3351 & - & 2148 & - \\ \hline
% \begin{tabular}[c]{@{}l@{}}Linear Regression\\ Cancer Drug Response\\ (n=822, p=2300)\end{tabular} & 47 & 0.0 & 1800 & 0.31 & 212 & 0.0 \\ \hline
% \begin{tabular}[c]{@{}l@{}}Logistic Regression\\ Synthetic\\ (n=16000, p=16000)\end{tabular} & 271 & 0.0 & N/A & N/A & 1800 & - \\ \hline
% \begin{tabular}[c]{@{}l@{}}Logistic Regression\\ Dorothea\\ (n=1150, p=91598)\end{tabular} & 62 & 0.0 & N/A & N/A & 600 & 0.0 \\
% \bottomrule
% \end{tabular}%
% }
% \end{table*}

% Please add the following required packages to your document preamble:
% \usepackage{multirow}
% \usepackage{graphicx}
\begin{table*}[]
\centering
\caption{Certifying optimality on large-scale and real-world datasets.}
\vspace{2mm}
\label{tab:my-table}
\resizebox{\textwidth}{!}{%
\begin{tabular}{llcccccc}
\toprule
 &  & \multicolumn{2}{c}{ours} & \multicolumn{2}{c}{Gurobi} & \multicolumn{2}{c}{MOSEK} \\
 &  & time (s) & opt. gap (\%) & time (s) & opt. gap (\%) & time (s) & opt. gap (\%) \\ \hline
\multirow{2}{*}{Linear Regression} & \begin{tabular}[c]{@{}l@{}}synthetic ($k=10, M=2$)\\ (n=16k, p=16k, seed=0)\end{tabular} & 79 & 0.0 & 1800 & - & 1915 & - \\ \cline{2-8}
 & \begin{tabular}[c]{@{}l@{}}Cancer Drug Response ($k=5, M=5$)\\ (n=822, p=2300)\end{tabular} & 41 & 0.0 & 1800 & 0.89 & 188 & 0.0 \\ \hline
\multirow{2}{*}{Logistic Regression} & \begin{tabular}[c]{@{}l@{}}Synthetic ($k=10, M=2$)\\ (n=16k, p=16k, seed=0)\end{tabular} & 626 & 0.0 & N/A & N/A & 2446 & - \\ \cline{2-8}
 & \begin{tabular}[c]{@{}l@{}}DOROTHEA ($k=15, M=2$)\\ (n=1150, p=91598)\end{tabular} & 91 & 0.0 & N/A & N/A & 634 & 0.0 \\
 \bottomrule
\end{tabular}%
}
% \vspace{-3mm}
\end{table*}
We next benchmark the computational speed and scalability of our method against the state-of-the-art solvers (Gurobi, MOSEK, SCS, and Clarabel) for solving the perspective relaxation of the original MIP problem.
Evaluations are performed on both linear and logistic regression tasks.


Experimental configurations are detailed in Appendix~\ref{appendix:setup_for_solving_the_perspective_relaxation}.
Additional perturbation studies, such as on the sample-to-feature ($n$-to-$p$) ratio, box constraint $M$, and $\ell_2$ regularization coefficient $\lambda_2$, are provided in Appendix~\ref{appendix:numerical_solve_convex_relaxation}.
All solvers are terminated upon achieving an optimality gap tolerance of $\epsilon=10^{-6}$ or exceeding a runtime limit of 1800 seconds.

The results, shown in Figure~\ref{fig:solve_convex_relaxation_main_paper}, demonstrates that our method outperforms the fastest conic solver (MOSEK) by over one order of magnitude.
For the largest tested instances ($n=16000$ and $p=16000$), our approach attains the target tolerance ($10^{-6}$) in under 100 seconds across regression and classification datasets, whereas most baselines fail to converge within the 1800-second threshold.

There are two factors driving this speedup.
First, our efficient proximal operator evaluation reduces per-iteration complexity.
Second, our efficient method to compute $g(\bbeta)$ (in Algorithm~\ref{alg:compute_g_value_algorithm}) exactly enables integration of the value-based restart technique within FISTA, significantly improving convergence.
Figure~\ref{fig:RestartedFISTA_linear_convergence_rate} illustrate this enhancement: while the proximal gradient algorithm (PGD) and FISTA exhibit sublinear convergence rates, FISTA with restarts achieves linear convergence on both dual loss and primal-dual gap metrics.
To the best of our knowledge, this marks the first empirical demonstration of linear convergence for a first-order method applied to solving the convex relaxation of this MIP class.

Finally, our method permits GPU acceleration because our most computationally intensive component is matrix-vector multiplications.
As shown in Table~\ref{tab:GPU_acceleration}, GPU implementation reduces runtime by an additional order of magnitude on high-dimensional instances.



\vspace{-0mm}
\subsection{How Fast Can We Certify Optimality?}
Finally, we demonstrate how our method's ability to compute tight lower bounds enables efficient optimality certification for large-scale datasets, outperforming state-of-the-art commericial MIP solvers.
Integrating our lower-bound computation into a minimalist branch-and-bound (BnB) framework, we prioritize node pruning via lower bound calculations while intentionally omitting advanced MIP heuristics (e.g., cutting planes, presolve routines) to evaluate the impact of our method.
Experimental configurations, including dataset descriptions and BnB implementation details, are provided in Appendix~\ref{appendix:setup_for_certifying_optimality}.
We benchmark our approach against Gurobi and MOSEK, reporting both runtime and final optimality gaps.


Results in Table~\ref{tab:my-table} show that our method certifies optimality for three of the four tested datasets within 2 minutes and the fourth within 10 minutes.
In contrast, Gurobi and MOSEK either exceed the time limit (1800 seconds) during the presolve stage or require significantly longer runtimes to achieve zero or small gaps.
Crucially, this efficiency stems from our efficient lower-bound computations and dynamic early termination criteria.
Specifically, we avoid waiting for full convergence by leveraging two key rules: 
(1) if the primal loss falls below the incumbent solution’s loss, we terminate early and proceed to branching; 
(2) if the dual loss exceeds the incumbent’s loss, we halt computation and prune the node immediately. This adaptive approach eliminates unnecessary iterations while ensuring we prune the search space effectively.






\section{Results}\label{Sec:Results}

The system’s performance was evaluated across key metrics, including latency, frame rate, and resolution. Latency was measured for the key components of the virtual human assistant interaction: STT exhibited a latency of $46 \pm 5$ ms, the LLM processing took $552 \pm 187$ ms, and the TTS synthesis had a latency of $1281 \pm 188$ ms.
The visual output was rendered at a resolution of $4128 \times 2208$ on the HMD, with frame rate recorded to assess the visual fluidity of each visualization modality. The \textbf{AR-VG} visualization maintained a consistent average frame rate of 72 FPS. Both \textbf{\revised{AV}-VG} and \textbf{FV-VG} operated at an average frame rate of 36 FPS.


\subsection{Stress Level}

\begin{figure}
    \centering
    \includegraphics[width=\columnwidth]{figures/hrv.png}
    \caption{\textbf{HRV during the Resting and Execution Phases.} In \textbf{RUS}, the RMSSD shows the steepest drop between the two phases, indicating a higher stress level compared to the proposed visualizations. \textbf{AR-VG} and \textbf{\revised{AV}-VG} perform similarly, while \textbf{FV-VG} exhibits highest RMSSD value and the smallest change between phases, suggesting that less stress is induced during the execution.}
    \label{fig:hrv}
\end{figure}

To assess stress levels during the robotic ultrasound procedure, we derived HRV from the ECG sensor data, focusing on the Root Mean Square of the Successive Differences (RMSSD), a commonly used measure of stress~\cite{shaffer2017overview}. Lower RMSSD values generally indicate higher stress levels. The analysis was performed using the HeartPy~\cite{van2019heartpy} Python package.
We analyzed HRV during two phases of the procedure: the resting phase, where the robot remained stationary and participants were free to interact with the virtual agent, and the execution phase, during which the robot performed the ultrasound scan. The HRV data for these phases are shown in Fig.~\ref{fig:hrv}.

Given the non-normal distribution of the data observed by the Shapiro-Wilk test, we used the Wilcoxon Signed-Rank Test for within-condition comparisons, assessing differences between the resting and execution phases for each visualization method. Although we observed a trend of lower RMSSD values during the execution phase compared to the resting phase, in \textbf{RUS} ($z = 25.0, p = 0.846\add{, d = 0.291}$), \textbf{AR-VG} ($z = 13.0, p = 0.547\add{, d = 0.141}$), \textbf{\revised{AV}-VG} ($z = 38.0, p = 0.970\add{, d = 0.180}$), and \textbf{FV-VG} ($z = 28.0, p = 0.700\add{, d = 0.032}$), the results did not indicate significance.
To compare HRV across the different visualization methods during both the resting and execution phases, we employed the Kruskal-Wallis Test. The analysis for the resting phase showed no significant difference in HRV across the visualization methods ($H = 0.485, p = 0.922\add{, \eta^2 = 0.012}$). During the execution phase, the test also yielded no significant difference between methods ($H = 3.430, p = 0.330\add{, \eta^2 = 0.086}$).



\subsection{Subjective Ratings}

\begin{figure*}[t]
    \centering
    \begin{subfigure}[b]{0.325\textwidth}
        \centering
        \includegraphics[width=\textwidth]{figures/hri.png}
        \caption{Trust in Human Robot Interaction}
        \label{fig:hri}
    \end{subfigure}
    \hfill % optional; add some horizontal spacing
    \begin{subfigure}[b]{0.325\textwidth}
        \centering
        \includegraphics[width=\textwidth]{figures/sus.png}
        \caption{System Usability Score}
        \label{fig:sus}
    \end{subfigure}
    \hfill % optional; add some horizontal spacing
    \begin{subfigure}[b]{0.325\textwidth}
        \centering
        \includegraphics[width=\textwidth]{figures/tlx.png}
        \caption{Perceived Workload}
        \label{fig:tlx}
    \end{subfigure}
    \caption{\textbf{Subjective Measurements for Trust Score, Usability, and Workload.} All proposed immersive visualizations with the conversational agent significantly increase the HRI trust score compared to \textbf{RUS}. \textbf{AR-VG} receives the highest trust score, the best usability, and the lowest workload among all methods. Statistical significance is indicated as $\star \left( p<0.05 \right)$, $\star \star \left( p<0.01 \right)$, and $\star \star \star \left( p<0.001 \right)$.}
    \label{fig:subjective}
\end{figure*}

HRI Trust scores under each condition for the robotic ultrasound were as follows: \textbf{RUS} ($M = 3.12, SD = 0.62$), \textbf{AR-VG} ($M = 4.33, SD = 0.42$), \textbf{\revised{AV}-VG} ($M = 4.29, SD = 0.38$), and \textbf{FV-VG} ($M = 4.06, SD = 0.68$). The results are visualized in Fig.~\ref{fig:hri}.
Statistical analysis using the Friedman test revealed a significant difference in trust scores across the visualization methods ($\chi^2(3) = 26.95, p = 6.02 \times 10^{-6}$). Post-hoc Dunn-Sid{\'a}k  pairwise comparisons further emphasized these differences. Significant differences were observed between \textbf{RUS} and \textbf{AR-VG} ($p = 0.000316\add{, d = 2.272}$), \textbf{RUS} and \textbf{\revised{AV}-VG} ($p = 0.00035\add{, d = 2.272}$), and \textbf{RUS} and \textbf{FV-VG} ($p = 0.012\add{, d = 1.428}$).

The SUS scores for each condition\revised{, normalized to a 0-1 scale,} are shown in Fig.~\ref{fig:sus}. A Friedman test revealed a significant difference in usability across the visualization methods ($\chi^2(3) = 16.60, p = 0.000854$). Post-hoc Dunn-Sid{\'a}k  pairwise comparisons indicated that the significant difference lies between \textbf{RUS} and \textbf{AR-VG} ($p = 0.037\add{, d = 1.343}$).

The NASA-TLX scores, \add{normalized to a 0-1 range}, are presented in Fig.~\ref{fig:tlx}. A Friedman test revealed a significant difference in task load across the visualization methods($\chi^2(3) = 9.03, p = 0.02$).
Although there was a tendency for both \textbf{AR-VG} and \textbf{\revised{AV}-VG} to show lower task load scores compared to \textbf{RUS}, Dunn-Sid{\'a}k  pairwise comparisons did not reveal any statistically significant differences between the visualization methods.


\subsection{User Preference and Feedback}

%\begin{figure}
%    \centering
%    \includegraphics[width=\columnwidth]{figures/rank.png}
%    \caption{\textbf{Preference Ranking.} \textbf{AR-VG} was the most preferred, followed by \textbf{\revised{AV}-VG}, \textbf{FV-VG}, and \textbf{RUS}.}
%    \label{fig:rank}
%\end{figure}


The results showed that \textbf{AR-VG} was the most preferred visualization, with 72$\%$ of participants ranking it as their top choice, 14$\%$ ranking it second, and 14$\%$ ranking it third. \textbf{\revised{AV}-VG} followed, with 21$\%$ of participants ranking it as the most preferred, 43$\%$ ranking it second, and 36$\%$ ranking it third. For \textbf{FV-VG}, 36$\%$ of participants ranked it in their top three choices. Finally, no participants ranking \textbf{RUS} as their first choice. However, 22$\%$ ranked it second, 42$\%$ ranked it third, and 36$\%$ ranked it as their least preferred visualization.

The qualitative feedback from participants provided further insight into their preferences. Participants in general appreciated the conversational abilities of the virtual assistant across \textbf{AR-VG}, \textbf{\revised{AV}-VG} and \textbf{FV-VG}. They noted that talking to the avatar felt natural and gave them more control over the procedure. In addition, several participants remarked that the hand animation of the virtual assistant taking control of the probe “made me trust the system more.” However, due to technical limitation, the avatar’s hand was not visible in the \textbf{\revised{AV}-VG} passthrough window, which led to some confusion about the interaction.
Concerns about the accuracy of VR visualizations were also raised. Participants noted that due to tracking error, sometimes misalignment between their real and virtual arms in \textbf{FV-VG} caused uncertainty about the success of the scan. Participants raised concerns about the robot’s actions, particularly when they could not see the real robot.
%, leading to uncertainty about the procedure.

Overall, the feedback indicated that participants favored the visualizations that offered a balance between immersion and real-world visibility and integrating a friendly, responsive avatar can improve patient trust and comfort in robotic ultrasound procedures.

\section{Discussion}\label{Sec:GeneralDiscussion}

\subsection{Hypotheses}
\textbf{\textit{H1.}} Our results demonstrate a significant increase in trust scores across all the proposed visualizations featuring the conversational virtual agent, compared to \textbf{RUS}. This finding strongly supports the hypothesis that the presence of the virtual agent contributes to reducing discomfort and increasing acceptance during the robotic ultrasound procedure. Participants were able to ask questions, receive feedback, and feel reassured by the agent’s presence, which appears to have played a key role in fostering trust. Notably, several participants highlighted the hand animation of the virtual assistant holding the ultrasound probe, describing it as a crucial factor in building trust. This subtle yet meaningful interaction gave participants the impression that the virtual agent was aware of the ongoing procedure, making the system appear more intelligent and responsive. By simulating the action of guiding the probe, the virtual assistant conveyed a sense of human control, reducing the perceived detachment often associated with autonomous systems. This visual synchronization between the agent’s actions and the real-world procedure helped humanize the experience, further enhancing confidence in the system’s accuracy and reliability.
Moreover, the usability of the system also improved across all the proposed methods featuring the virtual agent, although significant improvements in usability were only observed with \textbf{AR-VG}. The perceived workload was also reduced in both \textbf{AR-VG} and \textbf{\revised{AV}-VG}. The agent’s conversational abilities, particularly in offering explanations and responding to patient inputs, likely reduced the cognitive burden and made the system easier to navigate.

\textbf{\textit{H2.}} Our results provide partial support for the hypothesis that reducing the visibility of the robot will reduce stress and improve acceptance. When comparing the three immersive visualization methods, \textbf{FV-VG} showed the highest RMSSD values among all conditions in both the resting and execution phases, with the smallest change between these phases. This suggests that participants experienced the least increase in stress during the procedure in the fully immersive environment, potentially due to the absence of the robot’s visual presence, which could reduce feelings of intimidation or anxiety. \add{However, the lack of statistical significance across conditions indicates that the visualization methods may primarily influence psychological perceptions—such as reduced anxiety and improved comfort—rather than inducing measurable changes in physiological stress responses. Additionally, a larger sample size may increase the power of statistical analyses and reveal trends not observed in this study.}
Participant feedback also highlighted the varied reactions to the lack of robot visibility. One participant with no prior experience in robotic procedures noted that in \textbf{FV-VG}, the environment felt like “an animated world,” allowing them to focus less on the procedure itself. This participant expressed a sense of relief and detachment from the robotic aspect, commenting that “before you realize it, the procedure is done.” This suggests that for those unfamiliar with robotic systems, full immersion may help reduce anxiety by removing any focus on the technical aspects of the procedure.
However, several participants with more experience in robotics, especially those with development experience, expressed discomfort with not being able to see the robot’s movements. These participants indicated they would prefer to observe the robot, as they were concerned about the possibility of malfunction or errors. This feedback aligns with the lower trust scores for \textbf{FV-VG}, compared to \textbf{AR-VG} and \textbf{\revised{AV}-VG}, despite the reduced physiological stress. The misalignment between the real and virtual bodies in \textbf{FV-VG}, combined with the complete absence of visual cues from the robot, likely contributed to a lower sense of control and trust in the system.
In contrast, \textbf{AR-VG}, where the robot is visible alongside the virtual agent, had the highest trust scores. This suggests that for many participants, being able to observe the robot’s actions provided reassurance and increased their trust in the system. Similarly, \textbf{\revised{AV}-VG}, where the robot was hidden but the patient’s real arm was visible, performed well in terms of trust, though slightly lower than \textbf{AR-VG}. These findings indicate that while reducing the robot’s visibility may lower stress, maintaining some visual connection to the real world, whether through the robot or the patient’s body, is crucial for building trust.

\textbf{\textit{H3.}} Our results indicate support for the hypothesis that the level of immersion influences patient workload and usability. Among the three immersive visualization methods, \textbf{AR-VG} demonstrated the highest usability and the lowest perceived workload. This supports the hypothesis that AR, by maintaining a connection to the real world, allows for greater situational awareness, which makes the system easier to navigate and reduces cognitive effort. Participants could see their surroundings and the virtual agent, making the experience more intuitive and less mentally taxing. The blend of real-world context with virtual elements likely contributed to both the higher usability and the lower workload.
\textbf{\revised{AV}-VG} also performed well in terms of both usability and workload, though slightly below \textbf{AR-VG}. The passthrough window, which allowed participants to see their real arm during the procedure, offered a partial connection to the real world while still immersing them in a virtual environment. This balance between immersion and real-world visibility may have helped reduce mental load compared to \textbf{FV-VG}, as participants were reassured by seeing part of their real body. However, the higher level of immersion compared to \textbf{AR-VG} might have slightly increased cognitive effort, resulting in a moderate workload and usability score.
In contrast, \textbf{FV-VG} demonstrated the lowest usability and the highest perceived workload among the three methods. The fully immersive environment removed all real-world visual cues, requiring participants to rely entirely on the virtual environment and the virtual agent for orientation and guidance. This complete detachment from the real world may have contributed to a sense of disorientation, which in turn negatively impacted usability and increased increased cognitive demand, as participants had to adapt to the fully virtual setting.


\subsection{Insights}


\textbf{Context-aware Communication.}
The importance of context-aware communication from the virtual agent was a key finding in this study, and it aligns with broader research in human-robot interaction~\cite{chevalier2022context}. In medical settings, patients often feel anxious or disconnected from autonomous systems due to the perceived lack of transparency and control. By embedding a conversational agent that is aware of the procedure’s stages—beginning, execution, and ending—our system ensured that patients were continuously informed and reassured. This type of communication reduces uncertainty, which is crucial in maintaining trust and comfort, as seen in other works that emphasize the role of transparency in fostering trust in autonomous systems~\cite{ososky2014determinants,pynadath2018transparency}. Context-aware systems that adjust feedback based on the current state of the procedure, as we implemented, align with research suggesting that timely, relevant communication enhances user experience and trust~\cite{lisetti2015now}. Moreover, while automating feedback can reduce patient cognitive load, it is important to avoid over-automation, as excessive automation can lead to a loss of sense of agency (SoA)~\cite{haggard2012sense,ueda2021influence} and potentially increase stress, especially in medical contexts where patient involvement is critical.

 
\textbf{Balancing Immersion and Real-World Context.}
One of the key insights from our study is the delicate balance between immersion and real-world context in patient experience during robotic ultrasound. While participants generally preferred \textbf{AR-VG} and \textbf{\revised{AV}-VG}, the stress levels were actually lower in \textbf{FV-VG}. This suggests that while a highly immersive environment can reduce physiological stress, it may also disconnect patients from critical real-world cues, such as the robot’s actions, which are crucial for maintaining trust and confidence. Research has shown that users tend to feel more comfortable when they have some level of real-world feedback, particularly in medical settings, where understanding the procedure is important for reducing anxiety~\cite{burghardt2018non,weisfeld2021dealing}. To address this, a potential future design could combine the benefits of both approaches. For instance, in a fully immersive VR environment, or even a calm, relaxing virtual setting, abstract representations of the robot’s state could be introduced. This would allow patients to enjoy the calming benefits of the VR environment while still being aware of the robot’s movements, thus providing both stress reduction and a sense of control. Such a hybrid visualization approach could balance immersion with real-world awareness, enhancing both comfort and trust in autonomous medical procedures.

\textbf{Patient-Centered Design.}
\add{This study represents a first step toward integrating conversational virtual agents and immersive visualizations into robotic ultrasound systems.}
%As a foundational exploration, it highlights the potential of these technologies to humanize robotic medical procedures and improve patient trust and comfort.
A key takeaway from this study is the importance of designing immersive visualizations with the patient’s experience at the forefront, particularly in procedures where patients remain conscious. Our findings emphasize that immersive technologies should not merely serve as technical enhancements but must also be tailored to meet the emotional and psychological needs of patients. The inclusion of a conversational virtual agent, for example, not only humanized the procedure but also helped reduce feelings of isolation and discomfort by providing constant reassurance.
A patient-centered approach can extend beyond medical robotics to other fields where human interaction with autonomous systems is critical. For example, future designs should prioritize personalization~\cite{athanasiou2014towards}, allowing systems to adapt to individual patient preferences, whether through adjusting levels of immersion, offering more or less transparency during the procedure, or tailoring communication styles to the patient’s comfort level. Additionally, systems can be designed to remember previous interactions, enabling the virtual agent to build rapport by referencing past experiences. For instance, when a patient returns for a follow-up visit, the system could greet them and mention something from the previous session, helping to create a more familiar and personalized interaction. Ultimately, this approach ensures that patients remain active participants in their own care, which is essential for fostering long-term trust and acceptance of autonomous technologies.

\subsection{Limitations}
\revised{While the study demonstrates promising results as a proof-of-concept, several limitations and trade-offs should be addressed in future work.}

First, tracking inaccuracies in \textbf{FV-VG} affected user confidence, with some participants reporting misalignment between their real and virtual bodies. This issue stems from two factors: 1) inaccurate hand tracking from the HMD, and 2) the IK solver estimating the arm pose based solely on the hand and head positions. To mitigate this, adding additional sensors to the arm could improve tracking accuracy. However, this would increase the complexity of the setup, which could negatively impact usability. \add{Additionally, these inaccuracies may have introduced biases, placing \textbf{FV-VG} at a disadvantage compared to other conditions. Caution is warranted when interpreting its results, as differences may stem from technical issues rather than the visualization method itself. Future studies should refine tracking mechanisms to ensure fair comparisons and address this imbalance.}

Second, while participants appreciated the conversational abilities of the virtual agent, some reported delays in communication, leading to uncertainty about whether their input was received. To address this, incorporating a visual indicator, such as the avatar nodding its head, or audio feedback, like the avatar quickly responding with a verbal acknowledgment such as “uh-hum,” could help reassure users that their input has been recognized. Additionally, the LLM powering the virtual agent could be further enhanced by training it on more specific data. This would enable the agent to provide more professional and accurate answers during interactions, improving the overall user experience.

Third, in the \textbf{\revised{AV}-VG} implementation, due to technical limitations, virtual elements are not visible in the passthrough window. This is because the Unity Meta Quest SDK only allows the virtual layer to be rendered either above or below the real-world layer, but not mixed. This impacts usability and overall experience, as participants felt less confident without being able to see the avatar holding the ultrasound probe. Exploring other headsets or custom rendering engines that offer more flexibility in how virtual and real-world content is layered might provide a better user experience.

Furthermore, participants noted a depth perception issue in both AR and VR visualizations. The ultrasound image attached to the probe always appeared on top of the patient’s arm, causing patients to misjudge the probe’s position. This misjudgement led to doubts about the system’s accuracy. In the future, improving the visualization of the ultrasound image—such as by adjusting its transparency when it intersects with the arm—could help resolve this issue and provide a more realistic and reassuring experience.
\add{Finally, the participant pool was skewed toward participants with prior knowledge of robotics platforms. While this demographic provided valuable insights into the usability and technical aspects of the system, it may not fully represent the target population—patients with limited exposure to robotic systems. This may have influenced the trust and acceptance measures observed in the study. Additionally, participants’ acceptance of wearing an HMD during the procedure was not separately measured but was included in overall acceptance and trust ratings for the visualizations. Another limitation is the inability to separate the effects of the virtual avatar from those of the ultrasound probe visualization. For example, trust may have increased due to the avatar, the probe visualization, or their combined effects, while usability in \textbf{VP-VG} may have been impacted by physical-virtual alignment errors. Future studies should recruit a more diverse participant pool and explore alternative methods of delivering immersive visualizations to ensure broader applicability.}
%
In this work, we presented counterfactual situation testing (CST), a new actionable and meaningful framework for detecting individual discrimination in a dataset of classifier decisions.
We studied both single and multidimensional discrimination, focusing on the indirect setting.
For the latter kind, we compared its multiple and intersectional forms and provided the first evidence for the need to recognize intersectional discrimination as separate from multiple discrimination under non-discrimination law.
Compared to other methods, such as situation testing (ST) and counterfactual fairness (CF), CST uncovered more cases even when the classifier was counterfactually fair and after accounting for statistical significance.
For CF, in particular, we showed how CST equips it with confidence intervals, extending how we understand the robustness of this popular causal fairness definition. 

The decision-making settings tackled in this work are intended to showcase the CST framework and, importantly, to illustrate why it is necessary to draw a distinction between idealized and fairness given the difference comparisons when testing for individual discrimination. 
We hope the results motivate the adoption of the \textit{mutatis mutandis} manipulation over the \textit{ceteris paribus} manipulation.
We are aware that the experimental setting could be pushed further by considering higher dimensions or more complex causal structures. 
We leave this for future work.
%
Further,
extensions of CST should consider the impact of using different distance functions for measuring individual similarity \parencite{WilsonM97_HeteroDistanceFunctions}, and should explore a purely data-driven setup in which the running parameters and auxiliary causal knowledge are derived from the dataset \parencite{Cohen2013StatisticalPower, Peters2017_CausalInference}.
%
Furthermore,
extensions of CST should study settings in which the protected attribute goes beyond the binary, such as a high-cardinality categorical or an ordinal protected attribute \parencite{DBLP:journals/tkde/CerdaV22}. 
The setting in which the protected attribute is continuous is also of interest, though, in that case we could discretize it \parencite{DBLP:journals/tkde/GarciaLSLH13} and treat it as binary (the current setting) or as a high-cardinality categorical attribute.

Multidimensional discrimination testing is largely understudied \parencite{DBLP:conf/fat/0001HN23, WangRR22}. 
% Here, 
We have set a foundation for exploring the tension between multiple and intersectional discrimination, but future work should further study the problem of dealing with multiple protected attributes and their intersection.
It is of interest, for instance, formalizing the case in which one protected attribute dominates the others and the case in which the impact of each protected attribute varies based on individual characteristics.
% Formalizing the case in which one protected attribute dominates over the others as well as the case in which the effect of each protected attribute varies by individual characteristics are of interest.
While interaction terms and heterogeneous effects are understudied within SCM, both topics enjoy a well established literature in fields like economics \parencite{Wooldridge2015IntroductoryEconometrics}, which should enable future work.
% 
We hope these extensions and, overall, the fairness given the difference powering the CST framework motivate new work on algorithmic discrimination testing.

%
% EOS
%

%% if specified like this the section will be omitted in review mode
\acknowledgments{%
	This work was partly supported by the state of Bavaria through Bayerische Forschungsstiftung (BFS) under Grant AZ-1592-23-ForNeRo. The authors would like to thank all NARVIS and IFL Lab members for their valuable help and feedback.%
}


\bibliographystyle{abbrv-doi-hyperref}
%\bibliographystyle{abbrv-doi-hyperref-narrow}
%\bibliographystyle{abbrv-doi}
%\bibliographystyle{abbrv-doi-narrow}

\bibliography{09_Bibliography}



\end{document}

