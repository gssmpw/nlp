\section{Conclusion}\label{Conclusion}

In this paper, we presented a novel system aimed at enhancing patient acceptance of robotic ultrasound procedures through the integration of immersive mixed reality visualizations and an AI-based conversational virtual agent. Our system was evaluated across three different visualization modalities—\textbf{AR-VG}, \textbf{\revised{AV}-VG}, and \textbf{FV-VG}—each designed to improve patient trust, comfort, and usability compared to baseline robotic ultrasound.
The results of our study demonstrated that the inclusion of a conversational virtual agent significantly increased patient trust and reduced discomfort, with \textbf{AR-VG} emerging as the most preferred visualization method. While fully immersive VR reduced physiological stress, the participants expressed a stronger preference for visualizations that maintained some connection to the real world, as seen in \textbf{AR-VG} and \textbf{\revised{AV}-VG}. This balance between immersion and real-world context appears to be critical for maintaining both comfort and trust.
Overall, our findings provide valuable insights into how virtual agents and mixed reality visualizations can be leveraged to improve patient comfort and trust in autonomous medical procedures. By combining technological advancements with a focus on patient-centered design, future systems can further bridge the gap between automation and human empathy, paving the way for broader acceptance of robotic medical devices.
