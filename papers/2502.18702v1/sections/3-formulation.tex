\section{Task Formulation}
\label{sec:formulation-chapter5}
To explicitly capture contextual correlations during the entity extraction process, we reinterpret the original \ac{NER} task as two subtasks: recognizing named entities and identifying entity type-related features within the target sentence.

\noindent \textbf{Zero-Shot NER.}
In this paper, we focus on the \ac{NER} task in the strict zero-shot setting~\cite{DBLP:journals/corr/abs-2311-08921}. In this setting, no annotated data is available; instead, we only have access to an unlabeled corpus $\mathcal{D}_{u}$. Specifically, given an input sentence $\mathbf{x} = x_{1},\ldots,x_{n}$ with $n$ words from the test set $\mathcal{D}_{t}$, our aim is to recognize structured outputs $\mathbf{y}$ from $\mathbf{x}$, consisting of a set of $(e, t)$ pairs. Here, $e$ is an entity span, a sequence of tokens from $\mathbf{x}$, and $t$ is the corresponding entity type from a predefined set $\mathcal{T}$, such as persons, location, or miscellaneous.

\noindent \textbf{TRF extraction.}
\Acfp{TRF}, which are tokens strongly associated with entity types, are critical for the generalization of \ac{NER} models~\citep{DBLP:conf/emnlp/WangZCRRR23}.
Given an input sentence $\mathbf{x}\in \mathcal{D}_{t}$ including entity types $\mathcal{T}_{x}$, the goal of \ac{TRF} extraction is to identify all \acp{TRF} $\mathcal{R}$ that are related to the input sentence $\mathbf{x}$ for all entity types in $ \mathcal{T}_{x}$. Each \ac{TRF} $\mathbf{w}\in\mathcal{R}$ is an $m$-gram span. For instance, as illustrated in Figure~\ref{fig:exp-correlations}, ``member'' and ``teams'' are \acp{TRF} associated with the Person entity type, while ``music video games'' is recognized as a \ac{TRF} for the Miscellaneous type.