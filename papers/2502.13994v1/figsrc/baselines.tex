
\newcommand{\baselineBlock}[3]{
    \rotatebox{90}{\hspace*{#3}#2}&%
    \includegraphics[width=0.112\textwidth]{figures/baselines/#1_0.jpg}&%
    \includegraphics[width=0.112\textwidth]{figures/baselines/#1_1.jpg}&%
    \includegraphics[width=0.112\textwidth]{figures/baselines/#1_2.jpg}&%
    \includegraphics[width=0.112\textwidth]{figures/baselines/#1_3.jpg}\\%
}

\begin{figure}
  \centering%
  \setlength{\tabcolsep}{0.001\textwidth}%
  \renewcommand{\arraystretch}{0.5}%
  \footnotesize%
  \centering
  \hspace*{-1.2em}
  \begin{tabular}{c@{\hspace{0.25em}}|@{\hspace{0.25em}}c@{\hspace{0.5em}}cccc}
    \multirow{4}{*}{\rotatebox{90}{Image generators\hspace{5em}}}
      &\baselineBlock{spad}{SPAD}{3em}
      &\baselineBlock{diffhandles}{Diffusion Handles}{0.5em}
      &\baselineBlock{rgbx}{$\text{RGB}{\leftrightarrow}\text{X}$}{2em}
      &\baselineBlock{ours_diffusion}{Ours}{3em}
      \multicolumn{6}{c}{ }\\
      \multirow{3}{*}{\rotatebox{90}{Material generators\hspace{0.5em}}}
      &\baselineBlock{dreammat}{DreamMat}{2em}
      &\baselineBlock{texpainter}{TexPainter}{1em}
      &\baselineBlock{ours_reconstructed}{Ours}{3em}
  \end{tabular}
  \vspace{-2mm}      
  \caption{Comparison to prior work on the prompt \prompt{rusty kettle}. Image generators such as SPAD and Diffusion Handles are not designed to leverage known 3D geometry, which is available in our problem formulation, hence they struggle to accurately render the input from different viewpoints. $\text{RGB}{\leftrightarrow}\text{X}$ takes accurate scene intrinsics as input, but it is not equipped to ensure multi-view consistency (We evaluated the $\text{RGB}{\leftrightarrow}\text{X}$ method in a sequence of $\text{RGB}{\rightarrow}\text{X}{\rightarrow}\text{RGB}$, where inputs are our initial renderings, and outputs are edited renderings). Our problem is more related to material generation techniques such as DreamMat and TexPainter, but we focus on enhancing an existing material.   DreamMat uses a variant of SDS, which tends to output blurrier results.  TexPainter does not allow for view-dependent effects. Both techniques generate their output material from scratch instead of enhancing an input material (shown in \autoref{fig:hparams}).
  }
  \label{fig:baselines}
\end{figure}
