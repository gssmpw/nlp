\newcommand{\padthree}[1]{%
  \ifnum#1<10 00#1\else%
  \ifnum#1<100 0#1\else%
  #1%
  \fi\fi%
}



\newcommand{\newsceneblock}[6]{%
\begin{tabular}{cccccc}
    \multicolumn{6}{c}{#2} \\
    \multicolumn{2}{c}{(a) Conditioning renders} &
    \multicolumn{2}{c}{(b) Diffusion outputs} &
    \multicolumn{2}{c}{(c) Reconstructed material} \\
    \begin{overpic}[width=0.16\textwidth]{figures/results/#1/initial_rendering_\padthree{#3}.jpg}
        \put(2,2){\tiny\textcolor{white}{View #3}}
    \end{overpic}
    &
    \begin{overpic}[width=0.16\textwidth]{figures/results/#1/initial_rendering_\padthree{#4}.jpg}
        \put(2,2){\tiny\textcolor{white}{View #4}}
    \end{overpic}
    &
    \begin{overpic}[width=0.16\textwidth]{figures/results/#1/diffused_\padthree{#3}.jpg}
        \put(2,2){\tiny\textcolor{white}{View #3}}
    \end{overpic}
    &
    \begin{overpic}[width=0.16\textwidth]{figures/results/#1/diffused_\padthree{#4}.jpg}
        \put(2,2){\tiny\textcolor{white}{View #4}}
    \end{overpic}
    &
    \begin{overpic}[width=0.16\textwidth]{figures/results/#1/masked/recovered_default_\padthree{#3}.jpg}
        \put(2,2){\tiny\textcolor{white}{View #3}}
    \end{overpic}
    &
    \begin{overpic}[width=0.16\textwidth]{figures/results/#1/masked/recovered_default_\padthree{#4}.jpg}
        \put(2,2){\tiny\textcolor{white}{View #4}}
    \end{overpic}
    \\
    \begin{overpic}[width=0.16\textwidth]{figures/results/#1/initial_rendering_\padthree{#5}.jpg}
        \put(2,2){\tiny\textcolor{white}{View 4}}
    \end{overpic}
    &
    \begin{overpic}[width=0.16\textwidth]{figures/results/#1/initial_rendering_\padthree{#6}.jpg}
        \put(2,2){\tiny\textcolor{white}{View #6}}
    \end{overpic}
    &
    \begin{overpic}[width=0.16\textwidth]{figures/results/#1/diffused_\padthree{#5}.jpg}
        \put(2,2){\tiny\textcolor{white}{View #5}}
    \end{overpic}
    &
    \begin{overpic}[width=0.16\textwidth]{figures/results/#1/diffused_\padthree{#6}.jpg}
        \put(2,2){\tiny\textcolor{white}{View #6}}
    \end{overpic}
    &
    \begin{overpic}[width=0.16\textwidth]{figures/results/#1/masked/recovered_default_\padthree{#5}.jpg}
        \put(2,2){\tiny\textcolor{white}{View #5}}
    \end{overpic}
    &
    \begin{overpic}[width=0.16\textwidth]{figures/results/#1/masked/recovered_default_\padthree{#6}.jpg}
        \put(2,2){\tiny\textcolor{white}{View #6}}
    \end{overpic}
    \\
    \midrule
    (d) Initial albedo & (e) Reconstructed albedo & (f) Initial normal & (g) Reconstructed normal & (h) Initial roughness & (i) Reconstructed roughness \\
    \includegraphics[width=0.16\textwidth]{figures/results/#1/texture_albedo_initial.jpg} &
    \includegraphics[width=0.16\textwidth]{figures/results/#1/texture_albedo_edited.jpg} &
    \includegraphics[width=0.16\textwidth]{figures/results/#1/texture_normals_initial.jpg} &
    \includegraphics[width=0.16\textwidth]{figures/results/#1/texture_normals_edited.jpg} &
    \includegraphics[width=0.16\textwidth]{figures/results/#1/texture_roughness_initial.jpg} &
    \includegraphics[width=0.16\textwidth]{figures/results/#1/texture_roughness_edited.jpg}
    \vspace*{4mm}
\end{tabular}
}


\begin{figure*}
    \centering%
    \setlength{\tabcolsep}{0.002\textwidth}%
    \renewcommand{\arraystretch}{1}%
    \footnotesize%
    \newsceneblock{AC}{\small Air conditioner, prompt: \prompt{Overused, rusty, old air-conditioning unit}}{1}{2}{5}{6}
    \newsceneblock{statue}{\small Statue, prompt \prompt{Mossy stone statue}}{1}{2}{6}{7}
    \vspace*{-6mm}
    \caption{We use renderings of the original asset (two pairs of two adjacent views are shown in (a)) to condition the diffusion model to produce images with enhanced appearance (b), which is then back-propagated into the original material definition. In (c), we show the resulting asset on four frames from a turntable animation (we picked frames that correspond to the conditioning views); see the accompanying video for the full animation.
    Columns (d) to (i) show albedo, normal, and roughness textures of the initial and reconstructed material.
    }
    \label{fig:results}
\end{figure*}

