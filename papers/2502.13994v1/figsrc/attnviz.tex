\begin{figure*}[htbp]
    \centering%
    \setlength{\tabcolsep}{0.002\textwidth}%
    \renewcommand{\arraystretch}{1}%
    \footnotesize%
    \begin{tabular}{ccc}
        Latent pixel correspondences
        &
        Unmodified attention scores
        &
        Biased attention scores
        \\[0.8mm]
        \begin{overpic}[width=0.332\textwidth]{figures/attnviz/correspondences_crop.jpg}
            \begin{tikzpicture}[overlay, remember picture]
                \draw[-{Stealth[length=2mm]},green,thick,] (3.50,1.25) -- (3.12,0.78);
                \draw[-{Stealth[length=2mm]},red,thick,]   (1.45,1.30) -- (1.07,0.83);
                \draw[-{Stealth[length=2mm]},red,thick,]   (5.63,1.30) -- (5.25,0.83);
            \end{tikzpicture}
        \end{overpic}
        &
        \begin{overpic}[width=0.332\textwidth]{figures/attnviz/unbiased_crop.jpg}
        \end{overpic}
        &
        \begin{overpic}[width=0.332\textwidth]{figures/attnviz/biased_crop.jpg}
            \begin{tikzpicture}[overlay, remember picture]
                \draw[-{Stealth[length=2mm]},red,thick,]   (1.45,1.30) -- (1.07,0.83);
                \draw[-{Stealth[length=2mm]},red,thick,]   (5.63,1.30) -- (5.25,0.83);
            \end{tikzpicture}
        \end{overpic}
    \end{tabular}
    \caption{
        Left:
        For one latent pixel (green), we highlight the corresponding neighborhoods in the other conditioning views in red; we use ray tracing to check for occlusions.
        Middle:
        One attention score matrix \emph{row} related to that green latent pixel can be rearranged into an image showing how much it attends to all other latent pixels in one stage of the diffusion model.
        (We crop to three views of the $3 \times 3$ grid. The supplementary document includes an uncropped version.)
        Right:
        We bias the matrix entries in \emph{columns} that correspond to the identified red regions to promote attention---and hence consistency---between these latents.
    }
    \label{fig:attention_viz}
\end{figure*}
