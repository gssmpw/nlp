\section{Conclusion}
\label{sec:conclusion}

We presented a method for enhancing the detail of a classically authored material using a diffusion model.
Our method renders the provided material from multiple views, adds details to the renderings using a diffusion model, and then backpropagates the changes to the material using inverse rendering.
Inverse rendering requires detail to be consistent across views, and we achieve this with two technical contributions: noise correlation by projecting from a reference noise anchored in the UV space, and attention biasing using the known geometry of the object.
This requires no new datasets or expensive retraining and is largely built from off-the-shelf, pre-trained components.

The resulting method serves the important use case of \emph{human-in-the-loop} authoring: Rather than entirely replacing the artist and generating materials from scratch, we allow the artist to maintain creative control using traditional workflows, while reducing the time spent on tedious detailing of assets---analogous to ``auto-complete'' for material detail. Because the input and output of our method are traditional materials, our method can be used at any stage in the authoring process, and the produced enhacements arbitrarily post-processed, blended, and combined. We believe our work builds a solid foundation for future practical tools that will further improve the robustness and controllability of the generative process.

% \begin{table}[!t]
% \centering
% \scalebox{0.68}{
%     \begin{tabular}{ll cccc}
%       \toprule
%       & \multicolumn{4}{c}{\textbf{Intellipro Dataset}}\\
%       & \multicolumn{2}{c}{Rank Resume} & \multicolumn{2}{c}{Rank Job} \\
%       \cmidrule(lr){2-3} \cmidrule(lr){4-5} 
%       \textbf{Method}
%       &  Recall@100 & nDCG@100 & Recall@10 & nDCG@10 \\
%       \midrule
%       \confitold{}
%       & 71.28 &34.79 &76.50 &52.57 
%       \\
%       \cmidrule{2-5}
%       \confitsimple{}
%     & 82.53 &48.17
%        & 85.58 &64.91
     
%        \\
%        +\RunnerUpMiningShort{}
%     &85.43 &50.99 &91.38 &71.34 
%       \\
%       +\HyReShort
%         &- & -
%        &-&-\\
       
%       \bottomrule

%     \end{tabular}
%   }
% \caption{Ablation studies using Jina-v2-base as the encoder. ``\confitsimple{}'' refers using a simplified encoder architecture. \framework{} trains \confitsimple{} with \RunnerUpMiningShort{} and \HyReShort{}.}
% \label{tbl:ablation}
% \end{table}
\begin{table*}[!t]
\centering
\scalebox{0.75}{
    \begin{tabular}{l cccc cccc}
      \toprule
      & \multicolumn{4}{c}{\textbf{Recruiting Dataset}}
      & \multicolumn{4}{c}{\textbf{AliYun Dataset}}\\
      & \multicolumn{2}{c}{Rank Resume} & \multicolumn{2}{c}{Rank Job} 
      & \multicolumn{2}{c}{Rank Resume} & \multicolumn{2}{c}{Rank Job}\\
      \cmidrule(lr){2-3} \cmidrule(lr){4-5} 
      \cmidrule(lr){6-7} \cmidrule(lr){8-9} 
      \textbf{Method}
      & Recall@100 & nDCG@100 & Recall@10 & nDCG@10
      & Recall@100 & nDCG@100 & Recall@10 & nDCG@10\\
      \midrule
      \confitold{}
      & 71.28 & 34.79 & 76.50 & 52.57 
      & 87.81 & 65.06 & 72.39 & 56.12
      \\
      \cmidrule{2-9}
      \confitsimple{}
      & 82.53 & 48.17 & 85.58 & 64.91
      & 94.90&78.40 & 78.70& 65.45
       \\
      +\HyReShort{}
       &85.28 & 49.50
       &90.25 & 70.22
       & 96.62&81.99 & \textbf{81.16}& 67.63
       \\
      +\RunnerUpMiningShort{}
       % & 85.14& 49.82
       % &90.75&72.51
       & \textbf{86.13}&\textbf{51.90} & \textbf{94.25}&\textbf{73.32}
       & \textbf{97.07}&\textbf{83.11} & 80.49& \textbf{68.02}
       \\
   %     +\RunnerUpMiningShort{}
   %    & 85.43 & 50.99 & 91.38 & 71.34 
   %    & 96.24 & 82.95 & 80.12 & 66.96
   %    \\
   %    +\HyReShort{} old
   %     &85.28 & 49.50
   %     &90.25 & 70.22
   %     & 96.62&81.99 & 81.16& 67.63
   %     \\
   % +\HyReShort{} 
   %     % & 85.14& 49.82
   %     % &90.75&72.51
   %     & 86.83&51.77 &92.00 &72.04
   %     & 97.07&83.11 & 80.49& 68.02
   %     \\
      \bottomrule

    \end{tabular}
  }
\caption{\framework{} ablation studies. ``\confitsimple{}'' refers using a simplified encoder architecture. \framework{} trains \confitsimple{} with \RunnerUpMiningShort{} and \HyReShort{}. We use Jina-v2-base as the encoder due to its better performance.
}
\label{tbl:ablation}
\end{table*}

\begin{figure*}[ht!]
    \centering%
    \setlength{\tabcolsep}{0.001\textwidth}%
    \renewcommand{\arraystretch}{0.5}%
    \footnotesize%
    \centering
    \begin{tabular}{ccc@{\hspace{0.4em}}cc@{\hspace{0.4em}}cc}    
        &
        \multicolumn{2}{c}{(\textbf{a}) ControlNet tile \& normal}&
        \multicolumn{2}{c}{(\textbf{b}) + Multi-view visual prompting}&
        \multicolumn{2}{c}{(\textbf{c}) + View-correlated noise \& attention bias}\\[0.3em]
        \multirow{3}{*}{\rotatebox{90}{Diffusion-generated images\hspace*{-3em}}}&
        \includegraphics[width=0.161\textwidth, trim=0 50 0 0, clip]{figures/inv_rendering/greekvase/inconsistent/diffused_000.jpg}&
        \includegraphics[width=0.161\textwidth, trim=0 50 0 0, clip]{figures/inv_rendering/greekvase/inconsistent/diffused_001.jpg}&
        \includegraphics[width=0.161\textwidth, trim=0 50 0 0, clip]{figures/inv_rendering/greekvase/middle/diffused_000.jpg}&
        \includegraphics[width=0.161\textwidth, trim=0 50 0 0, clip]{figures/inv_rendering/greekvase/middle/diffused_001.jpg}&
        \includegraphics[width=0.161\textwidth, trim=0 50 0 0, clip]{figures/inv_rendering/greekvase/consistent/diffused_000.jpg}&
        \includegraphics[width=0.161\textwidth, trim=0 50 0 0, clip]{figures/inv_rendering/greekvase/consistent/diffused_001.jpg}\\
        &
        \includegraphics[width=0.161\textwidth]{figures/inv_rendering/greekvase/inconsistent/diffused_000_zoom1.jpg}&
        \includegraphics[width=0.161\textwidth]{figures/inv_rendering/greekvase/inconsistent/diffused_001_zoom1.jpg}&
        \includegraphics[width=0.161\textwidth]{figures/inv_rendering/greekvase/middle/diffused_000_zoom1.jpg}&
        \includegraphics[width=0.161\textwidth]{figures/inv_rendering/greekvase/middle/diffused_001_zoom1.jpg}&
        \includegraphics[width=0.161\textwidth]{figures/inv_rendering/greekvase/consistent/diffused_000_zoom1.jpg}&
        \includegraphics[width=0.161\textwidth]{figures/inv_rendering/greekvase/consistent/diffused_001_zoom1.jpg}\\
        &
        \includegraphics[width=0.161\textwidth, trim=0 40 0 0, clip]{figures/inv_rendering/greekvase/inconsistent/diffused_000_zoom2.jpg}&
        \includegraphics[width=0.161\textwidth, trim=0 40 0 0, clip]{figures/inv_rendering/greekvase/inconsistent/diffused_001_zoom2.jpg}&
        \includegraphics[width=0.161\textwidth, trim=0 40 0 0, clip]{figures/inv_rendering/greekvase/middle/diffused_000_zoom2.jpg}&
        \includegraphics[width=0.161\textwidth, trim=0 40 0 0, clip]{figures/inv_rendering/greekvase/middle/diffused_001_zoom2.jpg}&
        \includegraphics[width=0.161\textwidth, trim=0 40 0 0, clip]{figures/inv_rendering/greekvase/consistent/diffused_000_zoom2.jpg}&
        \includegraphics[width=0.161\textwidth, trim=0 40 0 0, clip]{figures/inv_rendering/greekvase/consistent/diffused_001_zoom2.jpg}\\[2pt]
        \multirow{3}{*}{\rotatebox{90}{Renderings with reconstructed material\hspace*{-8em}}}&
        \includegraphics[width=0.161\textwidth, trim=0 50 0 0, clip]{figures/inv_rendering/greekvase/inconsistent/recovered_default_000.jpg}&
        \includegraphics[width=0.161\textwidth, trim=0 50 0 0, clip]{figures/inv_rendering/greekvase/inconsistent/recovered_default_001.jpg}&
        \includegraphics[width=0.161\textwidth, trim=0 50 0 0, clip]{figures/inv_rendering/greekvase/middle/recovered_default_000.jpg}&
        \includegraphics[width=0.161\textwidth, trim=0 50 0 0, clip]{figures/inv_rendering/greekvase/middle/recovered_default_001.jpg}&
        \includegraphics[width=0.161\textwidth, trim=0 50 0 0, clip]{figures/inv_rendering/greekvase/consistent/recovered_default_000.jpg}&
        \includegraphics[width=0.161\textwidth, trim=0 50 0 0, clip]{figures/inv_rendering/greekvase/consistent/recovered_default_001.jpg}\\
        &
        \includegraphics[width=0.161\textwidth]{figures/inv_rendering/greekvase/inconsistent/recovered_default_000_zoom1.jpg}&
        \includegraphics[width=0.161\textwidth]{figures/inv_rendering/greekvase/inconsistent/recovered_default_001_zoom1.jpg}&
        \includegraphics[width=0.161\textwidth]{figures/inv_rendering/greekvase/middle/recovered_default_000_zoom1.jpg}&
        \includegraphics[width=0.161\textwidth]{figures/inv_rendering/greekvase/middle/recovered_default_001_zoom1.jpg}&
        \includegraphics[width=0.161\textwidth]{figures/inv_rendering/greekvase/consistent/recovered_default_000_zoom1.jpg}&
        \includegraphics[width=0.161\textwidth]{figures/inv_rendering/greekvase/consistent/recovered_default_001_zoom1.jpg}\\
        &
        \includegraphics[width=0.161\textwidth, trim=0 40 0 0, clip]{figures/inv_rendering/greekvase/inconsistent/recovered_default_000_zoom2.jpg}&
        \includegraphics[width=0.161\textwidth, trim=0 40 0 0, clip]{figures/inv_rendering/greekvase/inconsistent/recovered_default_001_zoom2.jpg}&
        \includegraphics[width=0.161\textwidth, trim=0 40 0 0, clip]{figures/inv_rendering/greekvase/middle/recovered_default_000_zoom2.jpg}&
        \includegraphics[width=0.161\textwidth, trim=0 40 0 0, clip]{figures/inv_rendering/greekvase/middle/recovered_default_001_zoom2.jpg}&
        \includegraphics[width=0.161\textwidth, trim=0 40 0 0, clip]{figures/inv_rendering/greekvase/consistent/recovered_default_000_zoom2.jpg}&
        \includegraphics[width=0.161\textwidth, trim=0 40 0 0, clip]{figures/inv_rendering/greekvase/consistent/recovered_default_001_zoom2.jpg}\\
        & View 1 & View 2 & View 1 & View 2 & View 1 & View 2
    \end{tabular}
    \vspace{-2mm}
    \caption{We compare results with our model, first without the multi-view visual prompting (i.e., assembling the conditioning images into a grid)~\cite{flashtex} (\textbf{a}), with it (\textbf{b}), and also with our two techniques for improving multi-view consistency.
    }
    \label{fig:invrender_inconsistent}
\end{figure*}





\begin{table*}[t]
\centering
\fontsize{11pt}{11pt}\selectfont
\begin{tabular}{lllllllllllll}
\toprule
\multicolumn{1}{c}{\textbf{task}} & \multicolumn{2}{c}{\textbf{Mir}} & \multicolumn{2}{c}{\textbf{Lai}} & \multicolumn{2}{c}{\textbf{Ziegen.}} & \multicolumn{2}{c}{\textbf{Cao}} & \multicolumn{2}{c}{\textbf{Alva-Man.}} & \multicolumn{1}{c}{\textbf{avg.}} & \textbf{\begin{tabular}[c]{@{}l@{}}avg.\\ rank\end{tabular}} \\
\multicolumn{1}{c}{\textbf{metrics}} & \multicolumn{1}{c}{\textbf{cor.}} & \multicolumn{1}{c}{\textbf{p-v.}} & \multicolumn{1}{c}{\textbf{cor.}} & \multicolumn{1}{c}{\textbf{p-v.}} & \multicolumn{1}{c}{\textbf{cor.}} & \multicolumn{1}{c}{\textbf{p-v.}} & \multicolumn{1}{c}{\textbf{cor.}} & \multicolumn{1}{c}{\textbf{p-v.}} & \multicolumn{1}{c}{\textbf{cor.}} & \multicolumn{1}{c}{\textbf{p-v.}} &  &  \\ \midrule
\textbf{S-Bleu} & 0.50 & 0.0 & 0.47 & 0.0 & 0.59 & 0.0 & 0.58 & 0.0 & 0.68 & 0.0 & 0.57 & 5.8 \\
\textbf{R-Bleu} & -- & -- & 0.27 & 0.0 & 0.30 & 0.0 & -- & -- & -- & -- & - &  \\
\textbf{S-Meteor} & 0.49 & 0.0 & 0.48 & 0.0 & 0.61 & 0.0 & 0.57 & 0.0 & 0.64 & 0.0 & 0.56 & 6.1 \\
\textbf{R-Meteor} & -- & -- & 0.34 & 0.0 & 0.26 & 0.0 & -- & -- & -- & -- & - &  \\
\textbf{S-Bertscore} & \textbf{0.53} & 0.0 & {\ul 0.80} & 0.0 & \textbf{0.70} & 0.0 & {\ul 0.66} & 0.0 & {\ul0.78} & 0.0 & \textbf{0.69} & \textbf{1.7} \\
\textbf{R-Bertscore} & -- & -- & 0.51 & 0.0 & 0.38 & 0.0 & -- & -- & -- & -- & - &  \\
\textbf{S-Bleurt} & {\ul 0.52} & 0.0 & {\ul 0.80} & 0.0 & 0.60 & 0.0 & \textbf{0.70} & 0.0 & \textbf{0.80} & 0.0 & {\ul 0.68} & {\ul 2.3} \\
\textbf{R-Bleurt} & -- & -- & 0.59 & 0.0 & -0.05 & 0.13 & -- & -- & -- & -- & - &  \\
\textbf{S-Cosine} & 0.51 & 0.0 & 0.69 & 0.0 & {\ul 0.62} & 0.0 & 0.61 & 0.0 & 0.65 & 0.0 & 0.62 & 4.4 \\
\textbf{R-Cosine} & -- & -- & 0.40 & 0.0 & 0.29 & 0.0 & -- & -- & -- & -- & - & \\ \midrule
\textbf{QuestEval} & 0.23 & 0.0 & 0.25 & 0.0 & 0.49 & 0.0 & 0.47 & 0.0 & 0.62 & 0.0 & 0.41 & 9.0 \\
\textbf{LLaMa3} & 0.36 & 0.0 & \textbf{0.84} & 0.0 & {\ul{0.62}} & 0.0 & 0.61 & 0.0 &  0.76 & 0.0 & 0.64 & 3.6 \\
\textbf{our (3b)} & 0.49 & 0.0 & 0.73 & 0.0 & 0.54 & 0.0 & 0.53 & 0.0 & 0.7 & 0.0 & 0.60 & 5.8 \\
\textbf{our (8b)} & 0.48 & 0.0 & 0.73 & 0.0 & 0.52 & 0.0 & 0.53 & 0.0 & 0.7 & 0.0 & 0.59 & 6.3 \\  \bottomrule
\end{tabular}
\caption{Pearson correlation on human evaluation on system output. `R-': reference-based. `S-': source-based.}
\label{tab:sys}
\end{table*}



\begin{table}%[]
\centering
\fontsize{11pt}{11pt}\selectfont
\begin{tabular}{llllll}
\toprule
\multicolumn{1}{c}{\textbf{task}} & \multicolumn{1}{c}{\textbf{Lai}} & \multicolumn{1}{c}{\textbf{Zei.}} & \multicolumn{1}{c}{\textbf{Scia.}} & \textbf{} & \textbf{} \\ 
\multicolumn{1}{c}{\textbf{metrics}} & \multicolumn{1}{c}{\textbf{cor.}} & \multicolumn{1}{c}{\textbf{cor.}} & \multicolumn{1}{c}{\textbf{cor.}} & \textbf{avg.} & \textbf{\begin{tabular}[c]{@{}l@{}}avg.\\ rank\end{tabular}} \\ \midrule
\textbf{S-Bleu} & 0.40 & 0.40 & 0.19* & 0.33 & 7.67 \\
\textbf{S-Meteor} & 0.41 & 0.42 & 0.16* & 0.33 & 7.33 \\
\textbf{S-BertS.} & {\ul0.58} & 0.47 & 0.31 & 0.45 & 3.67 \\
\textbf{S-Bleurt} & 0.45 & {\ul 0.54} & {\ul 0.37} & 0.45 & {\ul 3.33} \\
\textbf{S-Cosine} & 0.56 & 0.52 & 0.3 & {\ul 0.46} & {\ul 3.33} \\ \midrule
\textbf{QuestE.} & 0.27 & 0.35 & 0.06* & 0.23 & 9.00 \\
\textbf{LlaMA3} & \textbf{0.6} & \textbf{0.67} & \textbf{0.51} & \textbf{0.59} & \textbf{1.0} \\
\textbf{Our (3b)} & 0.51 & 0.49 & 0.23* & 0.39 & 4.83 \\
\textbf{Our (8b)} & 0.52 & 0.49 & 0.22* & 0.43 & 4.83 \\ \bottomrule
\end{tabular}
\caption{Pearson correlation on human ratings on reference output. *not significant; we cannot reject the null hypothesis of zero correlation}
\label{tab:ref}
\end{table}


\begin{table*}%[]
\centering
\fontsize{11pt}{11pt}\selectfont
\begin{tabular}{lllllllll}
\toprule
\textbf{task} & \multicolumn{1}{c}{\textbf{ALL}} & \multicolumn{1}{c}{\textbf{sentiment}} & \multicolumn{1}{c}{\textbf{detoxify}} & \multicolumn{1}{c}{\textbf{catchy}} & \multicolumn{1}{c}{\textbf{polite}} & \multicolumn{1}{c}{\textbf{persuasive}} & \multicolumn{1}{c}{\textbf{formal}} & \textbf{\begin{tabular}[c]{@{}l@{}}avg. \\ rank\end{tabular}} \\
\textbf{metrics} & \multicolumn{1}{c}{\textbf{cor.}} & \multicolumn{1}{c}{\textbf{cor.}} & \multicolumn{1}{c}{\textbf{cor.}} & \multicolumn{1}{c}{\textbf{cor.}} & \multicolumn{1}{c}{\textbf{cor.}} & \multicolumn{1}{c}{\textbf{cor.}} & \multicolumn{1}{c}{\textbf{cor.}} &  \\ \midrule
\textbf{S-Bleu} & -0.17 & -0.82 & -0.45 & -0.12* & -0.1* & -0.05 & -0.21 & 8.42 \\
\textbf{R-Bleu} & - & -0.5 & -0.45 &  &  &  &  &  \\
\textbf{S-Meteor} & -0.07* & -0.55 & -0.4 & -0.01* & 0.1* & -0.16 & -0.04* & 7.67 \\
\textbf{R-Meteor} & - & -0.17* & -0.39 & - & - & - & - & - \\
\textbf{S-BertScore} & 0.11 & -0.38 & -0.07* & -0.17* & 0.28 & 0.12 & 0.25 & 6.0 \\
\textbf{R-BertScore} & - & -0.02* & -0.21* & - & - & - & - & - \\
\textbf{S-Bleurt} & 0.29 & 0.05* & 0.45 & 0.06* & 0.29 & 0.23 & 0.46 & 4.2 \\
\textbf{R-Bleurt} & - &  0.21 & 0.38 & - & - & - & - & - \\
\textbf{S-Cosine} & 0.01* & -0.5 & -0.13* & -0.19* & 0.05* & -0.05* & 0.15* & 7.42 \\
\textbf{R-Cosine} & - & -0.11* & -0.16* & - & - & - & - & - \\ \midrule
\textbf{QuestEval} & 0.21 & {\ul{0.29}} & 0.23 & 0.37 & 0.19* & 0.35 & 0.14* & 4.67 \\
\textbf{LlaMA3} & \textbf{0.82} & \textbf{0.80} & \textbf{0.72} & \textbf{0.84} & \textbf{0.84} & \textbf{0.90} & \textbf{0.88} & \textbf{1.00} \\
\textbf{Our (3b)} & 0.47 & -0.11* & 0.37 & 0.61 & 0.53 & 0.54 & 0.66 & 3.5 \\
\textbf{Our (8b)} & {\ul{0.57}} & 0.09* & {\ul 0.49} & {\ul 0.72} & {\ul 0.64} & {\ul 0.62} & {\ul 0.67} & {\ul 2.17} \\ \bottomrule
\end{tabular}
\caption{Pearson correlation on human ratings on our constructed test set. 'R-': reference-based. 'S-': source-based. *not significant; we cannot reject the null hypothesis of zero correlation}
\label{tab:con}
\end{table*}

\section{Results}
We benchmark the different metrics on the different datasets using correlation to human judgement. For content preservation, we show results split on data with system output, reference output and our constructed test set: we show that the data source for evaluation leads to different conclusions on the metrics. In addition, we examine whether the metrics can rank style transfer systems similar to humans. On style strength, we likewise show correlations between human judgment and zero-shot evaluation approaches. When applicable, we summarize results by reporting the average correlation. And the average ranking of the metric per dataset (by ranking which metric obtains the highest correlation to human judgement per dataset). 

\subsection{Content preservation}
\paragraph{How do data sources affect the conclusion on best metric?}
The conclusions about the metrics' performance change radically depending on whether we use system output data, reference output, or our constructed test set. Ideally, a good metric correlates highly with humans on any data source. Ideally, for meta-evaluation, a metric should correlate consistently across all data sources, but the following shows that the correlations indicate different things, and the conclusion on the best metric should be drawn carefully.

Looking at the metrics correlations with humans on the data source with system output (Table~\ref{tab:sys}), we see a relatively high correlation for many of the metrics on many tasks. The overall best metrics are S-BertScore and S-BLEURT (avg+avg rank). We see no notable difference in our method of using the 3B or 8B model as the backbone.

Examining the average correlations based on data with reference output (Table~\ref{tab:ref}), now the zero-shoot prompting with LlaMA3 70B is the best-performing approach ($0.59$ avg). Tied for second place are source-based cosine embedding ($0.46$ avg), BLEURT ($0.45$ avg) and BertScore ($0.45$ avg). Our method follows on a 5. place: here, the 8b version (($0.43$ avg)) shows a bit stronger results than 3b ($0.39$ avg). The fact that the conclusions change, whether looking at reference or system output, confirms the observations made by \citet{scialom-etal-2021-questeval} on simplicity transfer.   

Now consider the results on our test set (Table~\ref{tab:con}): Several metrics show low or no correlation; we even see a significantly negative correlation for some metrics on ALL (BLEU) and for specific subparts of our test set for BLEU, Meteor, BertScore, Cosine. On the other end, LlaMA3 70B is again performing best, showing strong results ($0.82$ in ALL). The runner-up is now our 8B method, with a gap to the 3B version ($0.57$ vs $0.47$ in ALL). Note our method still shows zero correlation for the sentiment task. After, ranks BLEURT ($0.29$), QuestEval ($0.21$), BertScore ($0.11$), Cosine ($0.01$).  

On our test set, we find that some metrics that correlate relatively well on the other datasets, now exhibit low correlation. Hence, with our test set, we can now support the logical reasoning with data evidence: Evaluation of content preservation for style transfer needs to take the style shift into account. This conclusion could not be drawn using the existing data sources: We hypothesise that for the data with system-based output, successful output happens to be very similar to the source sentence and vice versa, and reference-based output might not contain server mistakes as they are gold references. Thus, none of the existing data sources tests the limits of the metrics.  


\paragraph{How do reference-based metrics compare to source-based ones?} Reference-based metrics show a lower correlation than the source-based counterpart for all metrics on both datasets with ratings on references (Table~\ref{tab:sys}). As discussed previously, reference-based metrics for style transfer have the drawback that many different good solutions on a rewrite might exist and not only one similar to a reference.


\paragraph{How well can the metrics rank the performance of style transfer methods?}
We compare the metrics' ability to judge the best style transfer methods w.r.t. the human annotations: Several of the data sources contain samples from different style transfer systems. In order to use metrics to assess the quality of the style transfer system, metrics should correctly find the best-performing system. Hence, we evaluate whether the metrics for content preservation provide the same system ranking as human evaluators. We take the mean of the score for every output on each system and the mean of the human annotations; we compare the systems using the Kendall's Tau correlation. 

We find only the evaluation using the dataset Mir, Lai, and Ziegen to result in significant correlations, probably because of sparsity in a number of system tests (App.~\ref{app:dataset}). Our method (8b) is the only metric providing a perfect ranking of the style transfer system on the Lai data, and Llama3 70B the only one on the Ziegen data. Results in App.~\ref{app:results}. 


\subsection{Style strength results}
%Evaluating style strengths is a challenging task. 
Llama3 70B shows better overall results than our method. However, our method scores higher than Llama3 70B on 2 out of 6 datasets, but it also exhibits zero correlation on one task (Table~\ref{tab:styleresults}).%More work i s needed on evaluating style strengths. 
 
\begin{table}%[]
\fontsize{11pt}{11pt}\selectfont
\begin{tabular}{lccc}
\toprule
\multicolumn{1}{c}{\textbf{}} & \textbf{LlaMA3} & \textbf{Our (3b)} & \textbf{Our (8b)} \\ \midrule
\textbf{Mir} & 0.46 & 0.54 & \textbf{0.57} \\
\textbf{Lai} & \textbf{0.57} & 0.18 & 0.19 \\
\textbf{Ziegen.} & 0.25 & 0.27 & \textbf{0.32} \\
\textbf{Alva-M.} & \textbf{0.59} & 0.03* & 0.02* \\
\textbf{Scialom} & \textbf{0.62} & 0.45 & 0.44 \\
\textbf{\begin{tabular}[c]{@{}l@{}}Our Test\end{tabular}} & \textbf{0.63} & 0.46 & 0.48 \\ \bottomrule
\end{tabular}
\caption{Style strength: Pearson correlation to human ratings. *not significant; we cannot reject the null hypothesis of zero corelation}
\label{tab:styleresults}
\end{table}

\subsection{Ablation}
We conduct several runs of the methods using LLMs with variations in instructions/prompts (App.~\ref{app:method}). We observe that the lower the correlation on a task, the higher the variation between the different runs. For our method, we only observe low variance between the runs.
None of the variations leads to different conclusions of the meta-evaluation. Results in App.~\ref{app:results}.