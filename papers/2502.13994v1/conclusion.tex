\section{Conclusion}
\label{sec:conclusion}

We presented a method for enhancing the detail of a classically authored material using a diffusion model.
Our method renders the provided material from multiple views, adds details to the renderings using a diffusion model, and then backpropagates the changes to the material using inverse rendering.
Inverse rendering requires detail to be consistent across views, and we achieve this with two technical contributions: noise correlation by projecting from a reference noise anchored in the UV space, and attention biasing using the known geometry of the object.
This requires no new datasets or expensive retraining and is largely built from off-the-shelf, pre-trained components.

The resulting method serves the important use case of \emph{human-in-the-loop} authoring: Rather than entirely replacing the artist and generating materials from scratch, we allow the artist to maintain creative control using traditional workflows, while reducing the time spent on tedious detailing of assets---analogous to ``auto-complete'' for material detail. Because the input and output of our method are traditional materials, our method can be used at any stage in the authoring process, and the produced enhacements arbitrarily post-processed, blended, and combined. We believe our work builds a solid foundation for future practical tools that will further improve the robustness and controllability of the generative process.

\begin{table*}
  [t]
  \centering
  \resizebox{\textwidth}{!}{%
  \begin{tabular}{cccccccccccc}
    \toprule \multicolumn{2}{c}{Components}                                                             & \multicolumn{5}{c}{Re-executability Rate (\%)} & \multicolumn{5}{c}{Readability (\#)} \\
    \cmidrule(lr){1-2} \cmidrule(lr){3-7} \cmidrule(lr){8-12}        \hspace{8pt}\labelemoji\hspace{8pt}                                                                & \hspace{8pt}\toolemoji\hspace{8pt}                                      & O0                                 & O1             & O2             & O3             & AVG            & O0             & O1             & O2             & O3             & AVG            \\
    \hline
    \rowcolor[rgb]{0.93,0.93,0.93}\multicolumn{12}{c}{\textbf{Initialize with LLM4Decompile-End-6.7B~\citep{llm4decompile}}}   \\
    \xmark                                                                                              & \xmark                                    & 69.51                              & 46.95          & 50.61          & 46.34          & 53.35          & 3.98 & 3.41 & 3.44 & 3.38 & 3.55 \\
    \cmark                                                                                              & \xmark                                    & 75.61                              & 50.61          & 50.00          & 50.00          & 56.55          & 4.01 & 3.44 & 3.39 & \textbf{3.49} & 3.58 \\
    \xmark                                                                                              & \cmark                                    & 83.54                     & \textbf{56.10}          & 51.22          & 50.61 & 60.37 & 4.05 & 3.51 & 3.51 & 3.42 & 3.62 \\
    \cmark                                                                                              & \cmark                                    & \textbf{85.37}                            & \textbf{56.10}                     & \textbf{51.83} & \textbf{52.43}          & \textbf{61.43} & \textbf{4.13} & \textbf{3.60} & \textbf{3.54} & \textbf{3.49} & \textbf{3.69} \\

    \rowcolor[rgb]{0.93,0.93,0.93}\multicolumn{12}{c}{\textbf{Initialize with Deepseek-Coder-6.7B-base~\citep{deepseekcoder}}} \\
    \xmark                                                                                              & \xmark                                    & 59.15                              & 35.98          & 39.02          & 37.80          & 42.99          & 3.71 & 3.05 & 3.16 & 3.05 & 3.24 \\
    \cmark                                                                                              & \xmark                                    & 66.46                              & 41.46          & 38.41          & 36.59          & 45.73          & 3.76 & 3.17 & \textbf{3.21} & 3.08 & 3.31 \\
    \xmark                                                                                              & \cmark                                    & 70.73                              & 39.63          & 39.02          & 40.24          & 47.41          & 3.90 & 3.17 & 3.08 & 3.11 & 3.31 \\
    \cmark                                                                                              & \cmark                                    & \textbf{79.88}                     & \textbf{45.73} & \textbf{43.90} & \textbf{42.68} & \textbf{53.05} & \textbf{3.96} & \textbf{3.21} & 3.18 & \textbf{3.19} & \textbf{3.38} \\
    \bottomrule
  \end{tabular}%
  }
  \caption{The ablation study of different methods across four optimization levels
  (O0, O1, O2, O3), as well as their average scores (AVG). The results in bold represent the optimal performance. The ~\labelemoji~ and ~\toolemoji~ means Relabedling and Function Call. \textbf{Bold} denotes the best performance.}
  \label{tab:ablation}
\end{table*}

\begin{figure*}[ht!]
    \centering%
    \setlength{\tabcolsep}{0.001\textwidth}%
    \renewcommand{\arraystretch}{0.5}%
    \footnotesize%
    \centering
    \begin{tabular}{ccc@{\hspace{0.4em}}cc@{\hspace{0.4em}}cc}    
        &
        \multicolumn{2}{c}{(\textbf{a}) ControlNet tile \& normal}&
        \multicolumn{2}{c}{(\textbf{b}) + Multi-view visual prompting}&
        \multicolumn{2}{c}{(\textbf{c}) + View-correlated noise \& attention bias}\\[0.3em]
        \multirow{3}{*}{\rotatebox{90}{Diffusion-generated images\hspace*{-3em}}}&
        \includegraphics[width=0.161\textwidth, trim=0 50 0 0, clip]{figures/inv_rendering/greekvase/inconsistent/diffused_000.jpg}&
        \includegraphics[width=0.161\textwidth, trim=0 50 0 0, clip]{figures/inv_rendering/greekvase/inconsistent/diffused_001.jpg}&
        \includegraphics[width=0.161\textwidth, trim=0 50 0 0, clip]{figures/inv_rendering/greekvase/middle/diffused_000.jpg}&
        \includegraphics[width=0.161\textwidth, trim=0 50 0 0, clip]{figures/inv_rendering/greekvase/middle/diffused_001.jpg}&
        \includegraphics[width=0.161\textwidth, trim=0 50 0 0, clip]{figures/inv_rendering/greekvase/consistent/diffused_000.jpg}&
        \includegraphics[width=0.161\textwidth, trim=0 50 0 0, clip]{figures/inv_rendering/greekvase/consistent/diffused_001.jpg}\\
        &
        \includegraphics[width=0.161\textwidth]{figures/inv_rendering/greekvase/inconsistent/diffused_000_zoom1.jpg}&
        \includegraphics[width=0.161\textwidth]{figures/inv_rendering/greekvase/inconsistent/diffused_001_zoom1.jpg}&
        \includegraphics[width=0.161\textwidth]{figures/inv_rendering/greekvase/middle/diffused_000_zoom1.jpg}&
        \includegraphics[width=0.161\textwidth]{figures/inv_rendering/greekvase/middle/diffused_001_zoom1.jpg}&
        \includegraphics[width=0.161\textwidth]{figures/inv_rendering/greekvase/consistent/diffused_000_zoom1.jpg}&
        \includegraphics[width=0.161\textwidth]{figures/inv_rendering/greekvase/consistent/diffused_001_zoom1.jpg}\\
        &
        \includegraphics[width=0.161\textwidth, trim=0 40 0 0, clip]{figures/inv_rendering/greekvase/inconsistent/diffused_000_zoom2.jpg}&
        \includegraphics[width=0.161\textwidth, trim=0 40 0 0, clip]{figures/inv_rendering/greekvase/inconsistent/diffused_001_zoom2.jpg}&
        \includegraphics[width=0.161\textwidth, trim=0 40 0 0, clip]{figures/inv_rendering/greekvase/middle/diffused_000_zoom2.jpg}&
        \includegraphics[width=0.161\textwidth, trim=0 40 0 0, clip]{figures/inv_rendering/greekvase/middle/diffused_001_zoom2.jpg}&
        \includegraphics[width=0.161\textwidth, trim=0 40 0 0, clip]{figures/inv_rendering/greekvase/consistent/diffused_000_zoom2.jpg}&
        \includegraphics[width=0.161\textwidth, trim=0 40 0 0, clip]{figures/inv_rendering/greekvase/consistent/diffused_001_zoom2.jpg}\\[2pt]
        \multirow{3}{*}{\rotatebox{90}{Renderings with reconstructed material\hspace*{-8em}}}&
        \includegraphics[width=0.161\textwidth, trim=0 50 0 0, clip]{figures/inv_rendering/greekvase/inconsistent/recovered_default_000.jpg}&
        \includegraphics[width=0.161\textwidth, trim=0 50 0 0, clip]{figures/inv_rendering/greekvase/inconsistent/recovered_default_001.jpg}&
        \includegraphics[width=0.161\textwidth, trim=0 50 0 0, clip]{figures/inv_rendering/greekvase/middle/recovered_default_000.jpg}&
        \includegraphics[width=0.161\textwidth, trim=0 50 0 0, clip]{figures/inv_rendering/greekvase/middle/recovered_default_001.jpg}&
        \includegraphics[width=0.161\textwidth, trim=0 50 0 0, clip]{figures/inv_rendering/greekvase/consistent/recovered_default_000.jpg}&
        \includegraphics[width=0.161\textwidth, trim=0 50 0 0, clip]{figures/inv_rendering/greekvase/consistent/recovered_default_001.jpg}\\
        &
        \includegraphics[width=0.161\textwidth]{figures/inv_rendering/greekvase/inconsistent/recovered_default_000_zoom1.jpg}&
        \includegraphics[width=0.161\textwidth]{figures/inv_rendering/greekvase/inconsistent/recovered_default_001_zoom1.jpg}&
        \includegraphics[width=0.161\textwidth]{figures/inv_rendering/greekvase/middle/recovered_default_000_zoom1.jpg}&
        \includegraphics[width=0.161\textwidth]{figures/inv_rendering/greekvase/middle/recovered_default_001_zoom1.jpg}&
        \includegraphics[width=0.161\textwidth]{figures/inv_rendering/greekvase/consistent/recovered_default_000_zoom1.jpg}&
        \includegraphics[width=0.161\textwidth]{figures/inv_rendering/greekvase/consistent/recovered_default_001_zoom1.jpg}\\
        &
        \includegraphics[width=0.161\textwidth, trim=0 40 0 0, clip]{figures/inv_rendering/greekvase/inconsistent/recovered_default_000_zoom2.jpg}&
        \includegraphics[width=0.161\textwidth, trim=0 40 0 0, clip]{figures/inv_rendering/greekvase/inconsistent/recovered_default_001_zoom2.jpg}&
        \includegraphics[width=0.161\textwidth, trim=0 40 0 0, clip]{figures/inv_rendering/greekvase/middle/recovered_default_000_zoom2.jpg}&
        \includegraphics[width=0.161\textwidth, trim=0 40 0 0, clip]{figures/inv_rendering/greekvase/middle/recovered_default_001_zoom2.jpg}&
        \includegraphics[width=0.161\textwidth, trim=0 40 0 0, clip]{figures/inv_rendering/greekvase/consistent/recovered_default_000_zoom2.jpg}&
        \includegraphics[width=0.161\textwidth, trim=0 40 0 0, clip]{figures/inv_rendering/greekvase/consistent/recovered_default_001_zoom2.jpg}\\
        & View 1 & View 2 & View 1 & View 2 & View 1 & View 2
    \end{tabular}
    \vspace{-2mm}
    \caption{We compare results with our model, first without the multi-view visual prompting (i.e., assembling the conditioning images into a grid)~\cite{flashtex} (\textbf{a}), with it (\textbf{b}), and also with our two techniques for improving multi-view consistency.
    }
    \label{fig:invrender_inconsistent}
\end{figure*}




\begin{table}[ht!]
\centering
\caption{\textbf{Super Resolution Performance Results.} Our proposed WGAN EEG Spatial Upsampling method significantly outperforms a baseline of Bicubic Interpolation commonly used in EEG upsampling pipelines.}
\label{tab:results}
\resizebox{0.8\linewidth}{!}{%
\begin{tabular}{@{}cccccc@{}}
\toprule
\multirow{2}{*}{\textbf{Dataset}} & \multirow{2}{*}{\textbf{Scale}} & \multicolumn{2}{c}{\textbf{Bicubic}} & \multicolumn{2}{c}{\textbf{WGAN}} \\ \cmidrule(l){3-6} 
                      &   & \textbf{MSE} & \textbf{MAE} & \textbf{MSE}    & \textbf{MAE}   \\
\toprule
\multirow{2}{*}{Val}  & 2 & 3.71E7       & 3.89E3       & \textbf{2.01E3} & \textbf{24.38} \\
                      & 4 & 7.23E7       & 6.42E3       & \textbf{8.53E3} & \textbf{63.83} \\
\midrule
\multirow{2}{*}{Test} & 2 & 3.75E7       & 3.91E3       & \textbf{2.06E3} & \textbf{24.66} \\
                      & 4 & 7.30E7       & 6.45E3       & \textbf{8.68E3} & \textbf{64.39} \\
\bottomrule
\end{tabular}%
}
\end{table}