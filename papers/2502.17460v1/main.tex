\documentclass[letterpaper, 10 pt, conference]{ieeeconf}  
\IEEEoverridecommandlockouts                                                                 
\overrideIEEEmargins                                      
\usepackage{multirow}
\usepackage{booktabs}
\usepackage{array}
\usepackage{amsmath,amssymb,amsfonts}
\usepackage{algorithmic}
\usepackage{graphicx}
\usepackage{textcomp}
\usepackage{xcolor}
\usepackage{verbatim}
\def\BibTeX{{\rm B\kern-.05em{\sc i\kern-.025em b}\kern-.08em
    T\kern-.1667em\lower.7ex\hbox{E}\kern-.125emX}}
%
%
%
%
%
%
%
%
%
%
%
%
%
%
%
%
%
%
%
%
%
%
%
%
%
%
%
%
%
%
%
%
%
%

%


\newcommand{\blue}{\color{blue}}
\newcommand{\red}{\color{red}}
\newcommand{\black}{\color{black}}

\newcommand{\clip}[1]{\mathrm{clip}\left(#1\right)}
\newcommand{\erf}{\mathrm{erf}}
\newcommand{\barh}{\bar{h}}
\newcommand{\barf}{\bar{f}}
%

\newcommand{\cmark}{$\checkmark$}
\newcommand{\xmark}{$\times$}
%
%

\newcommand{\overbar}[1]{\mkern 1.5mu\overline{\mkern-1.5mu#1\mkern-1.5mu}\mkern 1.5mu}
 \providecommand{\ttheta}{\bm{\theta}}
 \renewcommand{\aa}{\mathbf{a}}
  \providecommand{\bb}{\mathbf{b}}
  \providecommand{\md}{\mathrm{d}}
  \providecommand{\cc}{\mathbf{c}}
  \providecommand{\dd}{\mathbf{d}}
  \providecommand{\ee}{\mathbf{e}}
  \providecommand{\ff}{\mathbf{f}}
  \let\ggg\gg
  \renewcommand{\gg}{\boldsymbol{g}}
  \providecommand{\hh}{\mathbf{h}}
  \providecommand{\ii}{\mathbf{i}}
  \providecommand{\jj}{\mathbf{j}}
  \providecommand{\kk}{\mathbf{k}}
  %
  %
  \providecommand{\mm}{\mathbf{m}}
  \providecommand{\nn}{\mathbf{n}}
  \providecommand{\oo}{\mathbf{o}}
  \providecommand{\pp}{\mathbf{p}}
  \providecommand{\qq}{\mathbf{q}}
  \providecommand{\rr}{\mathbf{r}}
  \renewcommand{\ss}{\mathbf{s}}
  \providecommand{\tt}{\mathbf{t}}
  \providecommand{\uu}{\mathbf{u}}
  \providecommand{\vv}{\mathbf{v}}
  \providecommand{\ww}{\mathbf{w}}
  \providecommand{\xx}{\mathbf{x}}
  \providecommand{\yy}{\mathbf{y}}
  \providecommand{\zz}{\mathbf{z}}
   \providecommand{\AA}{\mathbf{A}}
  \providecommand{\aalpha}{\boldsymbol{\alpha}}
  \providecommand{\ssigma}{\boldsymbol{\sigma}}
  \providecommand{\llambda}{\boldsymbol{\lambda}}
 \providecommand{\GGamma}{\boldsymbol{\Gamma}}

  %
  \providecommand{\e}{\mathrm{e}}
  \providecommand{\mA}{\mathbf{A}}
  \providecommand{\mB}{\mathbf{B}}
  \providecommand{\mC}{\mathbf{C}}
  \providecommand{\mD}{\mathbf{D}}
  \providecommand{\mE}{\mathbf{E}}
  \providecommand{\mF}{\mathbf{F}}
  \providecommand{\mG}{\mathbf{G}}
  \providecommand{\mH}{\mathbf{H}}
  \providecommand{\mI}{\mathbf{I}}
  \providecommand{\mJ}{\mathbf{J}}
  \providecommand{\mK}{\mathbf{K}}
  \providecommand{\mL}{\mathbf{L}}
  \providecommand{\mM}{\mathbf{M}}
  \providecommand{\mN}{\mathbf{N}}
  \providecommand{\mO}{\mathbf{O}}
  \providecommand{\mP}{\mathbf{P}}
  \providecommand{\mQ}{\mathbf{Q}}
  \providecommand{\mR}{\mathbf{R}}
  \providecommand{\mS}{\mathbf{S}}
  \providecommand{\mT}{\mathbf{T}}
  \providecommand{\mU}{\mathbf{U}}
  \providecommand{\mV}{\mathbf{V}}
  \providecommand{\mW}{\mathbf{W}}
  \providecommand{\mX}{\mathbf{X}}
  \providecommand{\mY}{\mathbf{Y}}
  \providecommand{\mZ}{\mathbf{Z}}
  \providecommand{\mLambda}{\mathbf{\Lambda}}

  %
  \providecommand{\cA}{\mathcal{A}}
  \providecommand{\cB}{\mathcal{B}}
  \providecommand{\cC}{\mathcal{C}}
  \providecommand{\cD}{\mathcal{D}}
  \providecommand{\cE}{\mathcal{E}}
  \providecommand{\cF}{\mathcal{F}}
  \providecommand{\cG}{\mathcal{G}}
  \providecommand{\cH}{\mathcal{H}}
  \providecommand{\cJ}{\mathcal{J}}
  \providecommand{\cK}{\mathcal{K}}
  \providecommand{\cL}{\mathcal{L}}
  \providecommand{\cM}{\mathcal{M}}
  \providecommand{\cN}{\mathcal{N}}
  \providecommand{\cO}{\mathcal{O}}
  \providecommand{\cP}{\mathcal{P}}
  \providecommand{\cQ}{\mathcal{Q}}
  \providecommand{\cR}{\mathcal{R}}
  \providecommand{\cS}{\mathcal{S}}
  \providecommand{\cT}{\mathcal{T}}
  \providecommand{\cU}{\mathcal{U}}
  \providecommand{\cV}{\mathcal{V}}
  \providecommand{\cX}{\mathcal{X}}
  \providecommand{\cY}{\mathcal{Y}}
  \providecommand{\cW}{\mathcal{W}}
  \providecommand{\cZ}{\mathcal{Z}}
\providecommand{\bone}{\mathbbm{1}}
\def\leref#1{Lemma~\ref{#1}}
\def\figref#1{Fig.~\ref{#1}}
%
%
%
%
%
%
%
%
%
    %
    %
    %
    %
    %
    %
    %
    %
    %
    %
%


%
%
%

\newcommand{\zeman}[1]{{\color{cyan}[zeman: #1]}}
\newcommand{\meisam}[1]{{\textbf{\color{red}meisam: #1}}}
\newcommand{\yuan}[1]{{\textbf{\color{orange}yuan: #1}}}
\newtheorem{claim}{Claim}




%
%
\newcommand{\T}{\scriptscriptstyle T}
\def\maximize{\mathop{\text{maximize}}}
\def\minimize{\mathop{\text{minimize}}}
\def\deref#1{Definition~\ref{#1}}
\def\secref#1{Section~\ref{#1}}
\def\leref#1{Lemma~\ref{#1}}
\def\conref#1{Condition~\ref{#1}}
\def\thref#1{Theorem~\ref{#1}}
\def\remref#1{Remark~\ref{#1}}
\def\coref#1{Corollary~\ref{#1}}
\def\figref#1{Figure~\ref{#1}}
\def\figtab#1{Table~\ref{#1}}
\def\algref#1{Algorithm~\ref{#1}}
\def\appref#1{Appendix~\ref{#1}}
\def\asref#1{Assumption~\ref{#1}}
\def\bydef{\triangleq}
\def\pref#1{P\ref{#1}}

\DeclareMathOperator{\topk}{top}
\DeclareMathOperator{\randk}{rand}
%
\newcommand{\abs}[1]{\left\lvert #1\right\rvert}
\newcommand{\norm}[1]{\left\lVert #1\right\rVert}
\newcommand{\sqn}[1]{\left\lVert #1\right\rVert^2}
%
%
\newcommand{\lin}[1]{\ensuremath \left\langle #1 \right\rangle}

\newcommand{\ubar}[1]{\underaccent{\bar}{#1}}
\def\remark{\addtocounter{remark}{1}\def\@currentlabel{\theremark}%
\emph{Remark~\theremark}. } \makeatother
\newcounter{remark}
\newtheorem{condition}{Condition}
%
%
%
\newtheorem{property}{P}
 \usepackage{color-edits}
 \addauthor{mh}{red}
 \addauthor{xw}{magenta}
 \addauthor{ne}{brown}
 \newcommand{\brown}{\color{brown}}
\setlength{\textfloatsep}{5pt}
\setlength{\intextsep}{5pt}
\setlength{\abovedisplayskip}{3pt}
\setlength{\belowdisplayskip}{3pt}
\newcommand{\ols}[1]{\mskip.5\thinmuskip\overline{\mskip-.5\thinmuskip {#1} \mskip-.5\thinmuskip}\mskip.5\thinmuskip} %
\newcommand{\olsi}[1]{\,\overline{\!{#1}}} %
\newcommand{\CALL}[1]{\textbf{#1}}



\acrodef{aHPD}[\textit{a}HPD]{\textit{adaptive} HPD}
\acrodef{AMT}[AMT]{Amazon Mechanical Turk}
\acrodef{AQP}[AQP]{Approximate Query Processing}
\acrodef{BET}[BET]{Belief Evaluation Task}
\acrodef{CEM}[CEM]{Cluster Error Model}
\acrodef{CDF}[CDF]{Cumulative Distribution Function}
\acrodef{CI}[CI]{Confidence Interval}
\acrodef{CLT}[CLT]{Central Limit Theorem}
\acrodef{CrI}[CrI]{Credible Interval}
\acrodef{CWA}[CWA]{Closed World Assumption}
\acrodef{DB}[DB]{Database}
\acrodef{DeFacto}[DeFacto]{Deep Fact Validation}
\acrodef{EI}[EI]{Entity Identification}
\acrodef{EL}[EL]{Entity Linking}
\acrodef{EPSEM}[EPSEM]{Equal Probability of Selection Method}
\acrodef{ER}[ER]{Entity Resolution}
\acrodef{ET}[ET]{Equal-Tailed}
\acrodef{FPC}[FPC]{Finite Population Correction}
\acrodef{FV}[FV]{Fact Verification}
\acrodef{GDA}[GDA]{Gene-Disease Association}
\acrodef{HPD}[HPD]{Highest Posterior Density}
\acrodef{ICC}[ICC]{Intracluster Correlation Coefficient}
\acrodef{IE}[IE]{Information Extraction}
\acrodef{IG}[IG]{Inference Graph}
\acrodef{iid}[i.i.d.]{independent and identically distributed}
\acrodef{IR}[IR]{Information Retrieval}
\acrodef{KGC}[KGC]{Knowledge Graph Construction}
\acrodef{KG}[KG]{Knowledge Graph}
\acrodef{LCWA}[LCWA]{Local Closed World Assumption}
\acrodef{LD}[LD]{Linked Data}
\acrodef{MC}[MC]{Monte Carlo}
\acrodef{MCTS}[MCTS]{Monte Carlo Tree Search}
\acrodef{MLSS}[MLSS]{Multi-Level Splitting Sampling}
\acrodef{MoE}[MoE]{Margin of Error}
\acrodef{NELL}[NELL]{Never Ending Language Learning}
\acrodef{OWA}[OWA]{Open World Assumption}
\acrodef{PDF}[PDF]{Probability Density Function}
\acrodef{QA}[QA]{Question Answering}
\acrodef{RCS}[RCS]{Random Cluster Sampling}
\acrodef{RE}[RE]{Relation Extraction}
\acrodef{RecSys}[RecSys]{Recommender Systems}
\acrodef{RS}[RS]{Reservoir Sampling}
\acrodef{SEM}[SEM]{Score Error Model}
\acrodef{SLSQP}[SLSQP]{Sequential Least SQuares Programming}
\acrodef{SRS}[SRS]{Simple Random Sampling}
\acrodef{SS}[SS]{Stratified Sampling}
\acrodef{STWCS}[STWCS]{Stratified TWCS}
\acrodef{TAA}[TAA]{Triples Accuracy Assessment}
\acrodef{TEM}[TEM]{Triple Error Model}
\acrodef{TWCS}[TWCS]{Two-stage Weighted Cluster Sampling}
\acrodef{WCS}[WCS]{Weighted Cluster Sampling}

\title{\LARGE \bf
Finetuning and Quantization of EEG-Based Foundational BioSignal Models on ECG and PPG Data for Blood Pressure Estimation
}

\author{
Bálint Tóth$^{1}$, Dominik Senti$^{1}$, Thorir Mar Ingolfsson$^{1}$, \\
Jeffrey Zweidler$^{1}$, Alexandre Elsig$^{1}$, Luca Benini$^{1,2}$, Yawei Li$^{1}$
\thanks{$^{1}$ETH Z{\"u}rich, Z{\"u}rich, Switzerland.}
\thanks{$^{2}$DEI, University of Bologna, Bologna, Italy.}
\thanks{Bálint Tóth, Dominik Senti, and Thorir Mar Ingolfsson are co-first authors.}
}


\begin{document}

\maketitle
\thispagestyle{empty}
\pagestyle{empty}

%%%%%%%%%%%%%%%%%%%%%%%%%%%%%%%%%%%%%%%%%%%%%%%%%%%%%%%%%%%%%%%%%%%%
\begin{abstract}
Blood pressure (BP) is a key indicator of cardiovascular health. As hypertension remains a global cause of morbidity and mortality, accurate, continuous, and non-invasive BP monitoring is therefore of paramount importance. Photoplethysmography  (PPG) and electrocardiography (ECG) can potentially enable continuous BP monitoring, yet training accurate and robust machine learning (ML) models remains challenging due to variability in data quality and patient-specific factors. Recently, multiple research groups explored Electroencephalographic (EEG)--based foundation models and demonstrated their exceptional ability to learn rich temporal resolution. Considering the morphological similarities between different biosignals, the question arises of whether a model pre-trained on one modality can effectively be exploited to improve the accuracy of a different signal type. In this work, we take an initial step towards generalized biosignal foundation models by investigating whether model representations learned from abundant EEG data can effectively be transferred to ECG/PPG data solely with fine-tuning, without the need for large-scale additional pre-training, for the BP estimation task. Evaluations on the MIMIC-III and VitalDB datasets demonstrate that our approach achieves near state-of-the-art accuracy for diastolic BP (mean absolute error of 1.57 mmHg) and surpasses by 1.5$\times$ the accuracy of prior works for systolic BP (mean absolute error 2.72 mmHg). Additionally, we perform dynamic INT8 quantization, reducing the smallest model size by over $3.5\times$ (from $13.73$ MB down to $3.83$ MB) while preserving performance, thereby enabling unobtrusive, real-time BP monitoring on resource-constrained wearable devices.
\newline
\indent \textit{Clinical relevance}—Our cuffless BP estimation framework addresses the limitations of traditional, intermittent cuff-based devices by exploiting common ECG and PPG signals, enabling more frequent and unobtrusive measurement. With EEG-based pre-training and hardware-friendly quantization, this approach holds promise for low-power, continuous BP monitoring in both clinical and home environments, potentially improving early detection of hypertension and related cardiovascular risks.
\end{abstract}

%%%%%%%%%%%%%%%%%%%%%%%%%%%%%%%%%%%%%%%%%%%%%%%%%%%%%%%%%%%%%%%%%%%%%%%%%%%%%%%%

% 
% 
The widespread integration of communication networks and smart devices in modern control systems has increased the vulnerability of industrial systems to online cyber-attacks, e.g., Industroyer, Blackenergy, etc \citep{osti_1505628}.
% Modern control systems have seen a large push to include communication networks and smart devices to increase performance, made possible by improvements in communication device cost and energy consumption. This trend has been coupled with the usage of open-standard communication protocols among industrial control systems, making them vulnerable to online cyber-attacks such as Industroyer, Blackenergy, etc \citep{osti_1505628}. 
To counter this, methods have been developed to improve security by achieving attack detection, mitigation, and monitoring, among others \citep{sandberg2022secure}. This paper focuses on active attack diagnosis to mitigate stealthy attacks. 
%
%\subsection{Literature review}

Active diagnosis techniques rely on the inclusion of additional moduli to control systems
% inclusion within the control system of additional moduli 
to alter the behavior of the system compared to information known by the attacker. 
For instance, the concept of additive watermarking was introduced in \cite{mo2015physical}, where noise signals of known mean and variance are added at the plant and compensated for it at the controller. 
This compensation, however, is not exact, causing some performance degradation. Thus, trade-offs between performance and detectability  are necessary \citep{zhu2023detection}.
% A later work \citep{zhu2023detection} designs the watermark signal by trading performance for detection. Thus, although additive watermarking serves as a good detection scheme, they endure performance losses even in the nominal case. 

In encrypted control \citep{darup2021encrypted}, the sensor data is encrypted, sent to the controller, and then operated on directly. Encrypted input signals are sent back to the plant for decryption. Although encryption is widespread in IT security, in control systems it presents some concerns, such as the introduction of time delays \citep{stabile2024verifiable}, while it may present inherent weaknesses \citep{alisic2023model}.
% they are not preferred as they introduce time delays \citep{stabile2024verifiable} which can cause instability, and some encryption schemes can be very weak  \citep{alisic2023model}. 

In moving target defense \citep{griffioen2020moving}, the plant is augmented with fictitious dynamics, known to the controller. The plant output is transmitted to the controller along with the fictitious states over a network under attack. 
The additional measurements then aide in the detection of attacks. 
This comes at the cost of higher communication bandwidth needs, which increases rapidly with the dimension of the augmented systems.
% Since the dynamics of the fictitious dynamics are exactly known to the controller, the attack is detected easily. However, when the scale of the system increases, the communication bandwidth used by moving the target defense approach increases rapidly. 

Other recently proposed works include two-way coding \citep{fang2019two}, a weak encryuption technique, and dynamic masking \citep{abdalmoaty2023privacy}, which enhances privacy as well as security, have been shown to be effective against zero-dynamics attacks.
% Two-way coding \citep{fang2019two} and dynamic masking \citep{abdalmoaty2023privacy} are other recently proposed approaches. Two-way coding is another form of weak encryption technique whilst dynamic masking proposes an architecture that enhances both privacy and security. These schemes are shown to be effective against zero dynamics attacks but remain to be studied for other classes of attacks. 
% Recent extensions include \citep{mukherjee2021secure,ramos2024privacy}.
% Some other works which are related are \citep{mukherjee2021secure}, an extension of \cite{fang2019two}. The work \citep{ramos2024privacy} is an extension of moving target defense for multi-agent systems. 
Furthermore, filtering techniques for attack detection are proposed by \cite{murguia2020security,hashemi2022codesign,escudero2023safety}, while not focusing on stealthy attacks.
% The works \citep{murguia2020security,hashemi2022codesign,escudero2023safety} develop filtering techniques to guarantee safety, without being focused on stealthy covert attacks.

Multiplicative watermarking (mWM) has been proposed by the authors as a diagnosis technique \citep{ferrari2020switching}. mWM consists of a pair of filters on each communication channel between the plant and its controller; the scheme is affine to weak encryption, whereby ``encoding'' and ``decoding'' are done by changing signals' dynamic characteristics through inverse pairs of filters. This enables original signals to be recovered exactly, and thus does not lead to performance degradation.
% A multiplicative watermark is an affine to a weak encryption technique, through which the signal is ``encoded'' by a filter, changing its dynamic behavior. The use of inverse pairs means that the original signal can be recovered, through ``decoding'' via an inverse filter. As such, differently to techniques based on additive watermarking, no performance is lost due to the injection of noise, and there are no bandwidth limitations.

%\subsection{Contributions}
One of the critical features of multiplicative watermarking is that to detect stealthy attacks, the mWM filter parameters must be switched over time. In this paper, an algorithm to optimally design the mWM parameters after a switching event is presented, enhancing detection performance, without changing the switching time.
% This is done without changing the switching time, which is taken as given.

\textcolor{black}{
To formalize the filter design problem, we suppose the defender is interested in optimal performance against adversaries injecting covert attacks with matched system parameters \citep{smith2015covert}, including the mWM parameters prior to the switch. This scenario represents a worst case where malicious agents can take full control of the system while remaining undetected.
Thus, the attack strategy is explicitly included within the formulation of the closed-loop system, and the mWM filters are chosen by solving an optimization problem minimizing the attack-energy-constrained output-to-output gain (AEC-OOG) \citep{anand2023risk}, a variation of the output-to-output gain proposed in  \cite{teixeira2015strategic}.
}
The main contributions of this paper are:
% We consider an adversary injecting a covert attack with matched system parameters \citep{smith2015covert}, i.e., an attacker with full knowledge of the control system parameters, including those of the mWM filters before the switch. This scenario is taken as a worst case, as it has been shown that this class of attacks can be made stealthy. To quantitatively define a cost, the output-to-output gain (OOG) \citep{teixeira2015strategic} is leveraged,
% a metric introduced to evaluate the impact of an additive attack in a control system. %Specifically, OOG evaluates the worst-case performance loss that an attacker injecting an undetectable attack can obtain. 
% Here, the maximum performance loss caused by a stealthy adversary with limited energy is taken, the attack-energy-constrained OOG (AEC-OOG) \citep{anand2023risk}. The main contributions of this paper are:
\begin{enumerate}
%[label=\alph*.]
\item The problem of optimally designing the switching mWM filters is formulated as an optimization problem, with the AEC-OOG is taken as the objective;%where the AEC-OOG is taken as the impact metric; 
\item The worst-case scenario of a covert attack with exact knowledge of plant and mWM filter parameters is embedded within the design problem;
% The optimization problem is defined to incorporate the worst-case scenario of a covert attack with exact knowledge of plant and mWM filter parameters;
\item The feasibility of the optimization problem is shown to be dependent only on stability conditions; 
\item A solution scheme is proposed to promote randomization of the mWM filter parameters such that an eavesdropping adversary cannot remain stealthy.
\end{enumerate} 

This builds on the results of \cite{ferrari2020switching}, where the focus was on the design of the switching protocols, rather than the parameters themselves.
Compared to previous work \citep{gallo2021design}, this paper introduces an optimization problem which is always feasible (thanks to the use of AEC-OOG in the objective), while also considering a more sophisticated class of covert attacks, where the presence of watermark is known to the adversary. 
Moreover, this paper poses a different objective than \citep{zhang2023hybrid}; indeed, while \citep{zhang2023hybrid} provided a design strategy to ensure certain privacy properties, in this paper we address the problem of optimal parameter design following a switching event.


%\subsection{Organization}
The rest of the paper is organized as follows. 
After formulating the problem in Section~\ref{sec:PF}, we propose our design algorithm in Section~\ref{sec:main}, and analyze its properties. It is then evaluated through a numerical example in Section~\ref{sec:NE}, and concluding remarks are given Section~\ref{sec:Con}.
% We provide the problem background in Section~\ref{sec:PF}. We formulate the design problem in Section~\ref{sec:main}, together with an analysis of its properties. The proposed algorithm is evaluated through a numerical example in Section \ref{sec:NE}. Concluding remarks are offered in Section \ref{sec:Con}.
\section{Mobile Networks Powered by \glspl{LLM}}
\label{sec:LLM_enabled_MNs}
\begin{figure*}[t!]
\centering
\includegraphics[width=.99\textwidth]{Fig1.eps}
    \caption{Possible architectural designs for integrated \gls{LLM} and \gls{MNO} ecosystem.}
    \label{fig:LLM_possible_architectures}
\end{figure*}
The historical data of the \gls{MNO}, archived over years of expertise, constitutes a solid foundation for training the \gls{LLM} using structured and unstructured multi-modal inputs (as illustrated in Fig.~\ref{fig:LLM_possible_architectures}a) such as user intents, network logs, alarm descriptions, trouble tickets, \gls{PCAP} files (e.g. from Wireshark or tcpdump), dashboard screenshots, audio recordings (e.g. from \gls{IVR} systems), video feeds (e.g. from infrastructure surveillance), and \gls{NWDAF} analytics. To this end, a separate collection framework aggregates data from various sources into a centralized repository, and extracts most informative features such as warnings, error codes, timestamps, and user/gNB/session/bearer/\gls{QoS} flow/slice IDs. The extracted features are then converted into unified embeddings that are combined into a common vector space with suitable metadata (e.g. to differentiate data formats). The resulting vector store is used to fine-tune the \gls{LLM} to deeply internalize \gls{MNO}-specific knowledge \cite{Bariah2023understanding}. This allows the \gls{LLM} to learn patterns, sequences, and deviations that correlate with normal or faulty network operations. This is made possible using a timestamp-based cross-referencing to link different entries from several data sources, allowing detailed description and context for each flagged event as well as the resolution workflow for the spotted anomalies.

In live mobile networks, fresh multi-modal data is continuously fed into the \gls{LLM}, either uploaded in batches or streamed in real-time. The \gls{LLM} analyzes this data and identifies potential anomalous behaviors in light of its accumulated learning. In case of new anomalies not covered during the fine-tuning stage, the \gls{LLM} can rely on clustering techniques to group similar patterns and flag outliers as suspected behaviors. The \gls{LLM} is also capable of using \gls{RAG}-enabled external knowledge databases such as \gls{3GPP} documents \cite{Said2024instruct}, \gls{IEEE} standards, \gls{IETF} RFCs and vendors documentation \cite{soman2023observations} to compare the actual network behavior with the expected one to identify misconfigurations and spot unusual trends in protocols and communication flows. Well-crafted prompts, on the other hand, can guide the \gls{LLM} responses to provide focused solutions. Paradigms such as the \gls{CoT} reasoning can be used to break down the \gls{LLM} insights into a series of simplified and actionable sub-tasks. It can be extended by the \gls{ToT} technique to explore different reasoning paths and identify the most optimal solution. The \gls{LLM} can naturally produce stepwise reasoning if datasets used for fine-tuning contain \gls{CoT} and \gls{ToT} examples, or through creative prompting \cite{Zhou2024survey}. In parallel, \gls{NOC} engineers can intervene to confirm, guide or reject the \gls{LLM} findings, if needed, e.g. using its intuitive conversational interface. Through continuous self-learning, the \gls{LLM} will dynamically adapt to evolving network conditions, optimizing its performance over time \cite{Chaparadza2023optimization}.

%For instance, when a network experiences latency issues, the \gls{LLM} seamlessly analyze multi-modal information from diverse origins to identify the root cause, e.g. overloaded \gls{UPF} due to insufficient capacity, and then suggest a solution, e.g. step-by-step instructions including suitable code scripts for the involved \glspl{NF} to autonomously reroute traffic or modify policies. Conventional 5G networks can only alert about anomalies using suitable \gls{NWDAF} analytics that track the violated thresholds and notify the \gls{OAM} center to display the details on complex dashboards.

By incorporating \glspl{LLM} (e.g. as \glspl{NF}) into upcoming 6G networks, expected to be designed with \gls{SbD} principles \cite{Khaloopour2024Resilience}, \glspl{LLM} will naturally inherit the same built-in security safeguards rather than adding them as an afterthought. This design-driven approach focuses on proactive threat management, end-to-end encryption, authentication, network slicing isolation, \gls{AI}-driven threat detection with automated reactions, and stateless designs, fostering a resilient \gls{LLM}.
%The design-driven security in 5G and upcoming 6G networks ensures that security is natively integrated at every layer of the architecture rather than added as an afterthought. This approach focuses on proactive threat management, end-to-end encryption, authentication, network slicing, and \gls{AI}-driven threat detection and automated reactions to counter evolving cyber threats.



We based our analyses on the labeled data created in previous work~\cite{sanei2023characterizing}. The dataset distinguished 305 usability issues from five popular OSS projects (Jupyter Lab,
Google Colab, CoCalc, VSCode, and Atom) and identified their posters. In this paper, we focus on individuals who have ever posted a usability issue in that dataset. 

\subsection{Discovering the Role of Issue Posters}\label{sec: Discovering_role}

To detect the background of the usability issue posters in the dataset, we checked each user's \textit{Profile page} on GitHub, examining their bios, shared personal websites, LinkedIn pages, and/or shared resumes. If they have not shared these information, we searched for their LinkedIn profiles using their full names to extract details on their backgrounds and expertise. We considered their job titles posted in the information acquired this way and categorized them into (1) UX professionals, (2) managers, (3) data scientists, and (4) developers. UX professionals were defined as those indicating positions such as \textit{UX designer} and \textit{user interface and user experience designer}.

Among the 224 usability issue posters in the dataset, we were able to identify the role of 180 users. Within those 180 users, 121 (67.2\%) were developers, 34 (18.9\%) identified as data scientists, 21 (11.7\%) held managerial positions, and only four (2.2\%) were UX professionals. The UX professionals included one male contributed to \textit{VSCode}, another male contributed to \textit{Atom}, and two involved in \textit{Jupyter Lab} project, one male and one female. Notably, there were no UX professionals involved in \textit{CoCalc} and \textit{Google Colab} projects in our data sample. For easier referencing, in the following we call the UX professionals of VSCode as \textit{VSCode\_pro}, Atom \textit{Atom\_pro}, male of Jupyter Lab as \textit{Jupyter\_pro\_M} and female as \textit{Jupyter\_pro\_F}.

\subsection{Characteristics of Issues Posted by UX Professionals (RQ1)}

Once we identified the roles of the usability issue posters, we extracted all the issues posted by the four UX professionals across the five OSS projects. Next, we analyzed the extracted issues by adopting the following steps. First, following the approach outlined in \cite{sanei2023characterizing}, we labeled each issue with either usability or non-usability; and for each usability issue, we identified the main \textit{usability dimension} touched by the issue using the ten Nielsen heuristics~\cite{nielsen2005ten}. Then, similar to \cite{sanei2021impacts}, we identified the specific \textit{sentiment} and \textit{tone} expressed by the UX professionals when posting the usability issues. In our study, the sentiment captures the valence of the emotion that includes three categories (positive, negative, and neutral), while the tone describes emotion with seven affective factors (excited, frustrated, impolite, polite, sad, satisfied, and sympathetic). Subsequently, we analyzed the \textit{argument structure} of the usability issues to better understand the discursive device that the issue posters adopted to convince other discussion participants. We particularly identified whether a \textit{claim} and a \textit{premise} appeared in a usability issue post, using criteria proposed in prior work~\cite{skitalinskaya_learning_2021, wachsmuth_argumentation_2017, dowden1993logical}. Statements were considered as claims if they explicitly indicate the position or stance of the issue posters to the discussed usability issues; and premise means that a statement contains reasoning, evidence, or examples that support a stance. We compared how the above characteristics (i.e., usability dimensions, sentiments, tones, and argument structures) differed in issues posted by UX professionals and those without UX expertise.

\subsection{UX Professionals' Purpose Following Up on Issues (RQ2)}

% After investigating how UX professionals posted the usability issues, we recognized the importance of understanding their participation afterwards, particularly in following up on the discussion threads of the issues they posted. 
Thus, we first isolated comments made by the UX professionals posted to the usability issues they created within the datasets. Then, we employed an inductive content analysis~\cite{wamboldt1992content, Hsieh2005} and categorized the various purposes behind their contributions in posting each comment. For our analysis, the \textit{purpose} specifies the distinct goal that a particular comment serves within the context of the discussion thread. The purpose of a comment may vary based on its content and the immediate objective of the issue posters to write in the discussion to address one specific comment posted by another contributor. We grouped the identified purposes into themes through an iterative approach conducted by the two authors.


% \begin{figure*}[htpb!]
% \label{}
% \centering

%     {{\label{ROCIowaCedar} \includegraphics[width=\textwidth/3]{figures/IowaCedar_roc.png}}}%
%     \qquad
%     {{\label{ROCIowaDesMoines} \includegraphics[width=\textwidth/3]{figures/IowaDesMoines_roc.png} }%
%   \captionsetup{justification=centering}
%   \caption{\Acf{ROC} curves for \acf{RW} Iowa (CR) and  \acf{RW} Iowa (DM) dataset. Dummy model here represents a model whose output is solely a ``no Flood'' for all pixels.}
%   \label{fig:RW_ROC_Curves}%
% \end{figure*}



\section{Results and Discussions}
\label{sec:Results}

In this section, we aim to answer three main questions. First, we want to validate our hypothesis that \ac{SYN} data is a viable proxy for \ac{RW} data when training ML models for downscaling. Secondly, we seek to assess how much more skillful ML-based downscaling is compared to classical, non-data-driven techniques, such as our baseline methods, \textit{i.e.}, thresholded bicubic and Lanczos interpolation. Finally, we would like to appraise the extent to which data-driven models like ours are transferable (in terms of usefulness) to other regions without major performance degradations.  
To assess the quality of the models, we conduct a multiple comparison test --namely the Holm-Bonferroni procedure \cite{HolmBonferroni1979} -- that is designed to control the \ac{FWER}. We notice that, with a \ac{FWER} of $10^{-3}$, all the differences in model performance are significant. The only exception to this trend was observed in \ac{RW}-GH for whom the pairwise differences between \ac{RCAN} and \ac{ESRT}, Lanczos and Bicubic were not significant with the aforementioned \ac{FWER}. 

%Finally, we aim to find out the factors influencing the transferability of our models from one region to another.

\subsection{Potential of using SYN Data for RW downscaling}

In order to evaluate the utility of synthetic data for training, we compare performances of our candidate models on both \ac{SYN} and \ac{RW} Iowa data whose results are presented in Table \ref{tab:IowaResults}. We notice that 
\textbf{(i)} For the Iowa datasets, there is a drop in performance of all the models when going from \ac{SYN} to \ac{RW} datasets, 
\textbf{(ii)} for the \ac{RW}-IA (CR) as well as \ac{RW}-IA (DM) datasets, both bicubic and Lanczos interpolation have accuracies and MCC up to 70.89\% and 0.42 respectively while the deep learning models have accuracies and MCC up to 73.34\% and 0.46 respectively, 
\textbf{(iii)} There is a roughly 6\% accuracy improvement for the \ac{SYN} data for the deep learning models compared to the bicubic and lanczos models and this improvement drops to about 3\% for \ac{RW} data,  
\textbf{(iv)} the performance of all the models remain consistent across both \ac{RW}-IA datasets and \textbf{(v)} in \figref{fig:RW_ROC_Curves}, we observe that there is a high degree of overlap among the \ac{ROC} curves for the data-driven models.

From (i) and (iv) we can conclude that \ac{SYN} data is more intricate than \ac{RW} data. This implies that the benefits yielded by training with \ac{SYN} dataset, while significant, is not as prominent in the \ac{RW} Iowa datasets. 
% This may be due to sensor noise prevalent in the \ac{RW} Landsat-8 data that can be harder to reproduce in the synthetically generated examples. 
(i), (iii) and (v) implies that while \ac{SYN} data is not an exact replacement for \ac{RW} data, it provides a rather significant edge, which is all the more important when there is insufficient \ac{RW} for training. From (ii) we can conclude that the three proposed data driven models outperform classical super-resolution techniques such as bicubic and lanczos, conclusion supported by the \ac{ROC} curves in Figure \ref{fig:RW_ROC_Curves} for whom the data-driven models, in general, lie above the non-data-driven alternatives. Observation (iv) shows that  for the climatically similar \ac{RW}-Iowa(CR) and \ac{RW}-Iowa(DM) regions, training on \ac{SYN} Iowa data does indeed provide an edge. 

% have similar climate. 

\begin{figure*}[t!]
    \centering
    \begin{subfigure}[t]{0.5\textwidth}
        \centering
        \includegraphics[width=\textwidth/2]{figures/IowaCedar_roc.png}
        \caption{}
    \end{subfigure}%
    ~ 
    \begin{subfigure}[t]{0.5\textwidth}
        \centering
        \includegraphics[width=\textwidth/2]{figures/IowaDesMoines_roc.png}
        \caption{}
    \end{subfigure}
    \vspace*{0.5cm}
    \caption{    \label{fig:RW_ROC_Curves} \Acf{ROC} curves for (a) RW-IA (CR) and (b) RW-IA (DM) dataset. Na\"ive model here represents a model whose output is solely a ``no Flood'' for all pixels. Star here represents the pixel-wise classifier with a threshold of 0.5.}
\end{figure*}


\subsection{Effectiveness of data-driven approaches}

In order to evaluate the effectiveness of ML models in the downscaling task, we compare performances of our candidate models to Lanczos and bicubic interpolation methods by looking at figures of some sample predictions from Iowa (Figure \ref{fig:RWIowaDesMoines}), performance comparison in the region of Iowa in Table \ref{tab:IowaResults} and the ROC curves in Figure \ref{fig:RW_ROC_Curves} for \ac{RW} data. We notice that 
\textbf{(vi)} For RW-IA (DM) samples, the deep learning models maintain a higher degree of spatial continuity in the predicted \ac{FIM}, 
\textbf{(vii)} We observe that  bicubic and Lanczos interpolation produces over-smoothed \ac{FIM} reconstructions, while the plain \ac{RDN}, \ac{RCAN} and \ac{ESRT} models are more detail-inclusive. Similar conclusions can be drawn upon inspecting the \ac{ROC} curves in Figure \ref{fig:RW_ROC_Curves} and 
\textbf{(viii)} For RW-IA (CR), the ML models show a performance improvement of 3.06\% when comparing the best ML model and non-data-driven method and, while for RW-IA (DM) there is a performance improvement of 2.45\%.


Figures \ref{fig:EUSamples} and \ref{fig:RWIowaDesMoines} show the spatial disparity among the models whose details are often obscured in aggregated metrics such as accuracy. (vi) This implies that these data-driven models are better are recognizing an underlying stream network geometry than the classical methods. However, when it comes to narrow river streams, all the models struggle capturing the nuances of the \ac{FIM} resultant from localized high elevation features such as small islands within rivers or man-made structures. (vii) shows a clear advantage of our data-driven approaches over the non-data-driven alternatives. (viii) indicates the benefits of the data-driven models when evaluated over Iowa. 



\subsection{Applicability of our models to external regions}

To evaluate how transferable our models are, we draw conclusions from figures of the sample predictions from Western Europe (Figure \ref{fig:EUSamples}) and Ghana (Figure \ref{fig:GhanaSamples}) as well as the performance comparison in Table \ref{tab:ExternalResults}. We notice that 
\textbf{(ix)} for Ghana all of the models fail to adequately inundate the pixels over separated areas on account of several disconnected regions of inundation in the chosen area,
\textbf{(x)} the ML models outperform non-data driven methods for RW-EU, 
\textbf{(xi)} for the RW-EU dataset, there is an improvement of 4.89\% when comparing the accuracy of the best data- and non-data-driven methods, 
\textbf{(xii)} For RW-RR and RW-GH, there is marginal improvement (up to 0.77\% in accuracy) of the ML methods over the non-data driven methods and 
\textbf{(xiii)} For RW-EU, we notice that the ML models produce more connected streams over the non-data-driven models. 

(x) and (xi) implies that the models are transferable when considering hydroclimaticalogically similar regions since Iowa and the Meuse river in Europe lie within mid temperate zones. Similar to the observation (vi) for RW-IA (DM), (xiii) implies that the benefits of the ML model in identifying underlying network streams is also transferable to hydroclimatologically similar regions. In contrast, (xii) and (ix) both imply that the trained ML models struggle to generalize to RW-RR \& RW-GH. We speculate that this may be due to the significant differences in geography and climate when compared to Iowa. 

% More specifically, we notice that Ghana has a lot of disconnected regions when compared to Iowa and Western Europe, possibly indicating a geomorphological dissimilarity. Additionally, in the case of Red River and Ghana, we also speculate that they include drivers to flood inundation that are different from Iowa and Western Europe, which lie within mild temperate zones. Ghana on the other hand has a tropical (dry and hot) climate.

Our study directly implies that good quality synthetic data can be useful surrogates for downscaling low-resolution \acp{WFM} to high-resolution \acp{FIM} in regions, where such data are hard to come by, even when downscaling by a factor of 10. We noticed that such models were readily transferable to climatically similar regions as the region of training. However, Such derived ML models did not feature significantly different transferability when evaluated over hydroclimatologically dissimilar regions, which we attribute to different flood inundation characteristics, primarily at finer scales. A possible avenue to circumvent such issues is to explore additional training approaches that fall under the general area of domain adaptation. Nevertheless, data-driven models are still advantageous (and, hence, preferable) over non-data-driven alternatives in transfer scenarios like the one we considered here. 


%%%%%%%%%%%%%%%%%%%%%%%%%%%%%%% unused text %%%%%%%%%%%%%%%%%%%%%%%%%%%%%%%%%%%%%%%



% \tabref{tab:AccuracyResults} depicts test accuracies obtained by our models on both \ac{SYN} and \ac{RW} data. For Iowan floods, a comparison of \ac{SYN} and \ac{RW} results shows \textbf{(i)} bicubic and Lanczos interpolations remarkably gaining about $3\%$ in accuracy, as well as \textbf{(ii)} \ac{RDN} and \ac{RCAN} remaining relatively stable, while \textbf{(iii)} topography-aware models loosing $2.7\%$ in performance. From (i) one can conclude that \ac{SYN} data are morphologically slightly more intricate than \ac{RW} data. Also, (i) and (ii) likely imply that \ac{SYN} data, excluding topography, can serve as satisfactory surrogates of \ac{RW} data. However, as implied by (iii), our topography-dependent models seems to be particularly sensitive to distributional shifts of their combined inputs (\acp{WFM} and topographic features). More specifically, the topography-informed models' performance edge, while still statistically significant, is extremely marginal, even when compared to our non-data-driven approaches. Next, when comparing results between the cases of Iowan and Ghanaian \ac{RW} data, one observes that \textbf{(iv)} the accuracy of bicubic and Lanczos interpolations drops by almost $5\%$ due to over-smoothing. This may imply that Ghanaian \acp{FIM} bare a more complex morphology, when compared to Iowan \acp{FIM}. Also, \textbf{(v)} our topography-agnostic, data-driven models' performance degrades more gracefully (by about $2\%$), while \textbf{(vi)} our topography-aware models perform, virtually, as bad as our non-data-driven approaches. Hence, the differences in the data populations of the two regions we considered are significant enough to render our topography-dependent models noncompetitive. 




\section{Conclusion}
In this work, we demonstrated that a large transformer-based foundation model, originally trained on EEG, can be successfully adapted to estimate blood pressure from ECG and PPG with near-clinical accuracy. Using the CEReBrO model~\cite{CEReBrO} as our base, we achieved MAE as low as 1.57~mmHg (DBP) and 2.72~mmHg (SBP) on the MIMIC-III dataset, and 1.92~mmHg (DBP) and 3.14~mmHg (SBP) on VitalDB. These results fulfill critical benchmarks such as the BHS Grade~A and the AAMI guidelines, reinforcing the clinical viability of cross-biosignal knowledge transfer.

Beyond raw accuracy, we showed that post-training quantization substantially reduces the model footprint with minimal performance degradation. Dynamic quantization compressed the largest model from over $300$~MB to approximately $80$~MB—an almost $3.8\times$ reduction—while preserving MAE and \(R^2\) scores. This compression is vital for on-device, real-time BP monitoring in energy-constrained wearable systems, enabling more frequent and unobtrusive cardiovascular health assessments compared to traditional cuff-based devices.

Our findings underscore the promise of foundation models in biosignal processing and illustrate how quantization fosters practical deployment in low-resource or portable settings. Importantly, this work lays the groundwork for future research wherein the EEG-based foundation model not only serves as an effective tool for cross-biosignal transfer but also as a basis for developing dedicated ECG/PPG foundation models. Future work will explore additional compression methods such as pruning and distillation, integration of demographic factors for improved fairness, and further validation on embedded platforms to handle real-world artifacts and motion conditions. By combining efficient cross-biosignal transfer and hardware-friendly optimizations, this approach offers a path toward continuous, real-time blood pressure monitoring that can enhance patient care and clinical decision-making.

\section*{Acknowledgment}
\vspace{-0.1cm}
This project is supported by the Swiss National Science Foundation under the grant number 193813 (PEDESITE project) and by the ETH Future Computing Laboratory (EFCL), financed by a donation from Huawei Technologies.
\bibliographystyle{IEEEtran}
\bibliography{citations}
\end{document}
