\section{Implementation}

\subsection{Scanner and Analyzer Implementation}
We implement Sailor in Python. 
We use the Python lexical analysis library (plex) \cite{plex} to implement a parser for the Sail language, which is used by our Scanner.
Plex enables defining tokens, such as instruction definitions or CSR read/write expressions, using regular expressions, and associating each token with an action, like capturing CSR reads in an instruction's ISA-state footprint.
In Sail, each instruction has a definition, that further calls functions defined in the Sail model.
There are two approaches to the Scanner implementation: bottom-up (\ie{} find all the pieces of code in the Sail model where an ISA-state is modified and then trace back to find all the places where those pieces of code are called and so on) or top-down (scan one instruction definition at a time and track all the function calls it makes, while discovering the ISA-state being read/written). 
We implement the Scanner in a top-down fashion to limit recursion, focusing on computing the footprint of an instruction at a time, progressing with the natural execution flow. 

The Scanner first discovers all ISA-state definitions, function definitions and instruction definitions. 
Then it scans the function definitions and captures two things: the ISA-state footprint of the function (ISA-state read/written) and the function call footprint of the function (all the functions that are called inside the definition of this function).
Once all the functions have been scanned, the ISA-state footprint of each of the function is updated to add the ISA-state footprint of all the functions it calls. 
The function ISA-state footprints are cached to avoid redundant scanning. 
Then the Scanner parses all the instruction definitions, again capturing the ISA-state footprint and function call footprint for each instruction.
Similarly to functions, the ISA-state footprint of each instruction is updated to add the ISA-state footprint of all the functions it calls. 
The Scanner outputs the results in the CSV format.
The top-down approach and caching mechanism avoids recursion in our implementation of the Scanner. 
Python is also used to implement the Analyzer (\autoref{fig:algo}). 
\subsection{Validator Implementation}

As \autoref{fig:validator} shows, the Isla footprint tool generates the instruction execution trace per instruction. 
We automate the process of generating traces from Isla for many instructions.
We get the stream of instructions from the official instruction set listings in the RISC-V specification manual~\cite{riscv-specs} in the AsciiDoc format. 
We construct instructions from the listings which provides concrete values for the 7-bit opcodes and the associated operations (funct3 and funct7). 
For the source and destination registers, we use two concrete values from the GPRs in RISC-V: \texttt{zero} (x0) and any other GPR (\eg x10). 
We do this instead of using symbolic values here to limit the amount of path exploration in Isla. 
However, capturing the behaviour of the instruction with the read-only \texttt{zero} register in RISC-V ensures that the traces we capture covers this special case. 
For the immediate operands, we use symbolic values. 

We add a new option in Isla footprint to take as input the asciidoc file with the instruction listings and produce the traces for all the instructions. 
It does so in an annotated SMT-LIB2 format as \autoref{lst:mret-isla} shows. 
The Validator (written in Python) parses the instruction traces to extract the ISA-state footprint from Isla's output and compare those with the Scanner's output. 

\autoref{tab:loc} shows the code size of Sailor.\footnote{We plan to open-source Sailor.} 

\begin{table}[t]
\begin{center}
\begin{tabular}{ | c | c | } 
  \hline
  Module & Code Size (LoC)  \\ 
  \hline
  Scanner & 441 \\ 
  \hline
  Scanner RISC-V & 480 \\ 
  \hline 
  Analyzer & 181  \\ 
  \hline 
  Validator & 566 \\
  \hline 
\end{tabular}
\caption{Sailor code size in lines of Python code (LoC).
}
\label{tab:loc}
\end{center}
\end{table}


\subsection{Architecture Independent Implementation} 
\label{sec:arch-indep}
We apply Sailor primarily on the RISC-V Sail model \cite{sail-riscv}. 
However, Sailor can also be used with the Sail models of other architectures. 
Most of the Scanner implementation is architecture-independent. 
The Sail parser we implement using Plex is reusable across Sail models for other architectures such as Arm. 
We have an ISA-specific backend in the Scanner that is used to discover ISA-state \eg{} look for register definitions and find bitfields in the registers if they exist, etc. (see Scanner RISC-V in \autoref{tab:loc}). 
This is the only part that would need to be reimplemented to extend Sailor to other ISAs. 
The Analyzer implementation is completely architecture independent and can be used as is to analyze results from the Scanner for any ISA. 
The Validator implementation that compares that Scanner's output with that of Isla's traces is also architecture-independent. 
Isla can generate traces for Arm and RISC-V, so the validator can work on either of these.


