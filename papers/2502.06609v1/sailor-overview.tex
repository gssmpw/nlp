\section{Sailor Design}
\label{sec:sailor} 

\begin{figure}
    \includegraphics[width=\columnwidth]{sailor-overview-updated.pdf}
    \caption{Sailor Overview}
    \label{fig:sailor}
\end{figure}


\autoref{fig:sailor} shows the high-level overview of Sailor. It consists of three components: (A) Scanner, (B) Analyzer, and (C) Validator. The ultimate goal of the tool is to enable secure context switching of ISA-state by software that intends to enforce isolation between different security domains running on the system. 

The Scanner comprises of our Sail parser and an ISA-specific backend. 
The Scanner parses the Sail model of the ISA. 
This involves discovering the ISA, \ie{} privilege modes, registers and instructions. 
Then the Scanner dives deeper into the Sail model by parsing the register and instruction definitions, to extract their semantics and the inter-dependencies between different parts of the ISA.

The Analyzer takes as input the ISA-level insights extracted by the Scanner. 
The \texttt{Classifier Algorithm} (\autoref{sec:analyzer}) processes this input to classify ISA-state as security-sensitive, that must be swapped during context switches.

The quickest approach to leverage the algorithm's output is to use utilities like grep to find out whether existing software swaps the security-sensitive ISA-state. 
In this scenario, Sailor serves as a compass to bring the developer's attention to where the code manages the security-sensitive ISA-state, if it exists, and check that it does so properly.
Software developers can also use it as a guide for writing privileged software. 
For ISAs like Arm that could have hundreds of CSRs in the security-sensitive ISA-state, this information can be used to automatically generate tests. 
The tests can be simple assembly code that checks the security-sensitive ISA-state and rereads it back after a context switch to check if it changed. 
The algorithm's output can also be leveraged by ISA architects that design new ISA extensions as a guideline to ensure that the proposed extension does not affect the previously determined security-sensitivity of the ISA-state. We discuss this further in Section \ref{sec:isa-ext-discussion}.

Finally, the Validator corroborates the output of the Scanner using an existing tool for symbolic execution of Sail code, Isla~\cite{isla}.




