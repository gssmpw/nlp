\section{Introduction}
\label{sec:intro}
The increasing complexity of architectures, and the nitty-gritty details of the Instruction set architectures (ISA) being scattered throughout hundreds or thousands of pages have put system security at risk. 
A simple overlook of a single register in security-critical code can inadvertently expose the system to security vulnerabilities. 
Low-level software developers need to do the tedious job of manually navigating through the dense ISA specifications to ensure they do not introduce any security vulnerabilities in their code. 
Moreover, developers need to stay up-to-date with the evolving ISA specifications and the hardware platforms that support them.

ISAs have traditionally been specified using hand-written semantics in natural language, occasionally accompanied by pseudo-code, in bulky and comprehensive documents. 
For example, the Arm A-profile architecture reference manual~\cite{arm-ref-manual} and the privileged and unprivileged RISC-V ISA manuals~\cite{riscv-isa} consist of $~$15,000 and $~$900 pages, respectively. 
These ISA manuals are highly technical and are becoming increasingly dense, with new features and instructions added in every version. 
For instance, the Arm v7-A, v8-A and v9-A consist of $~$2700, $~$5200, and $~$15,000 pages, respectively~\cite{arm-v7-a, arm-v8-a, arm-ref-manual}. 
Similarly, on RISC-V, the non-ISA specifications (interrupt architecture, IOMMU, etc.) further add 767 pages\cite{riscv-specs}, and new ISA extensions are ratified every year~\cite{riscv-ratified}. 
Prior work shows a similar trend for the x86 ISA~\cite{hotos-baumann}. 

\begin{figure}
    \centering
    \includegraphics[width=160pt]{confidential_computing.pdf}
    \caption{Security domains isolated by a privileged software that mediates the context switch between the domains. 
    The security domains access the ISA-state through the Application Binary Interface (ABI). The privileged software must swap all security-sensitive ISA-state to prevent attacks.} 
    \label{fig:conf-comp}
\end{figure}


New features introduce new ISA-state, adding more complexity to the already convoluted ISA-state, especially when it comes to isolating software. \autoref{fig:conf-comp} shows two different security domains that execute in a system.
A privileged software mediates the context switch between the two security domains.
To the security domains, it should appear as if they are the only software that executes in the system~\cite{enclave-isolation, certikos}. 
There are two layers at which this guarantee needs to be enforced: the application programming interface (API) and the application binary interface (ABI). 
At the API-layer, which enables interaction between the different security domains, inputs from untrusted sources requires sanitization to prevent Iago-style attacks \cite{iago}. 
Through the ABI-layer, the security domains can access a significant part of the ISA-state. 
This involves 1000+ and 140+ registers on Arm v9-A~\cite{sail-arm} and RISC-V~\cite{sail-riscv}, respectively. 
The privileged software must ensure that all security-sensitive data of the first security domain has been wiped clean from the ISA-state, before passing control to the second security domain. 



In this paper, we focus on the ABI-layer, improper sanitization at which can introduce various security vulnerabilities. 
Prior work has discovered security vulnerabilities at the ABI-layer in industry-standard software as well as academic research projects. 
These involve improper flags sanitization \cite{totw} in Intel SGX-SDK~\cite{sgx-sdk} and Microsoft's open enclave~\cite{ms-open-enclave} runtimes,
register leakage~\cite{guard-dilemma} and ROP~\cite{rop-sgx} on Intel SGX-LKL~\cite{sgx-lkl} (another SGX runtime), interrupt hijacking~\cite{intel-tdx-sec} on Intel TDX~\cite{intel-tdx}, side-channels and computational integrity attacks through the floating point unit on Intel SGX and RISC-V Keystone \cite{dtrap-fpu, keystone}. 
Small oversights (of even a single bitfield in a register) in the context switch implementation introduce high-impact vulnerabilities that lead to simple and cost-effective attacks that even unprivileged adversaries can mount. 
The matter gets worse when dealing with ISAs such as RISC-V that promote adding new extensions and the privileged software inadequately tracks which ISA extensions are enabled by the platform vs. supported by the software \cite{dtrap-fpu}. 







Essentially, the attack surface introduced by improper sanitization of ISA-state (specifically, general purpose registers (GPRs) and control and status registers (CSRs)) when switching across security domains is often overlooked. 
To approach this problem, we propose leveraging the machine-readable ISA specification written in the Sail language~\cite{sail} to capture ISA-level insights and compute the set of security-sensitive ISA-state. 
With that, we aim to equip privileged software developers with conveniently accessible insights to the ISA that eliminates the need for developers to be overwhelmed with hundreds or thousands of pages of manuals. 

Sail enables specifying the semantics of all instructions in the ISA, capturing the effects of executing instructions in a complex system with multiple privilege levels on the ISA-state. 
Sail has already been used for writing specifications of multiple architectures (x86, Arm, RISC-V, MIPS, CHERI, and IBM Power).   
Sail has also resulted in tools such as Isla \cite{isla} for symbolic execution of Sail code using the Z3 SMT Solver, and verification of machine code using Islaris \cite{islaris}. 




We introduce Sailor, a novel tool that computes the set of security-sensitive ISA-state by automatically analyzing the ISA specification in Sail~\cite{sail}.
Sailor eliminates the tedious task of navigating through complex ISA specifications by leveraging Sail models of ISAs
to automatically compute the set of ISA-state that should be swapped during context switches across security domains (security-sensitive ISA-state). 
We apply Sailor on the Sail model for the RISC-V architecture to identify security-sensitive ISA-state \cite{sail-riscv}. 
Sailor parses the Sail model to automatically extract ISA-level insights that reflect the effects of executing instructions on the ISA-state as well as the effects of the ISA-state on instruction execution. 
For example, exceptions that occur during floating-point instruction execution on RISC-V update the \texttt{fflags} as a side-effect. 
Prior work demonstrated a side-channel attack where an adversary infers the victim's floating-point operations and exceptions by observing the \texttt{fflags} CSR~\cite{dtrap-fpu}. 
While prior work relied on manual ISA inspection, Sailor automates the discovery of such ISA-level insights.
We leverage these insights to compute the set of security-sensitive ISA-state and use that to perform security analysis of \textbf{four open-source confidential computing systems} i.e. Keystone \cite{keystone}, Komodo~\cite{komodo}, IBM ACE \cite{ibm-ace}, and Rivos Salus \cite{rivos-salus}. 
We find \textbf{three different classes of mishandled ISA-state} in the context switch implementations
of these systems.
We further reveal five different types of vulnerabilities that can be exploited across the systems by leveraging the mishandled ISA-state: computational integrity attacks, misconfigured emulated I/O fences, side-channels~\footnote{Side-channels in the context of this paper refers to exploiting ISA-state to unintentionally leak information. It does not refer to micro-architectural side-channels.}, covert-channels, and timing-channels (\autoref{tab:vuln}).~\footnote{Prior work previously reported two of these vulnerabilities for the Keystone framework~\cite{dtrap-fpu}.}

Our main contributions are: 
\begin{enumerate}
    \item We design and implement a new tool called Sailor that performs automatic analysis of the ISA specification written in Sail to compute the set of ISA-state that must be swapped during context switches.
    \item We evaluate four open-source confidential computing frameworks using the results of the algorithm. We report three different classes of mishandled ISA-state %
    in the context switch code of the frameworks (\autoref{tab:vuln}). We identify five different types of security vulnerabilities that arise from exploiting these mishandled ISA-state.  
    \item We validate the ISA-level insights generated from Sailor using an existing symbolic execution tool for Sail, Isla \cite{isla}. We found one bug in our tool during this process. 
\end{enumerate}

\textbf{Responsible disclosure:}  
We have disclosed all of the bugs and insights discovered through this work to each of the open-source frameworks (which are all research projects and are not used in production) via appropriate communication channels. 

