\section{Motivation \& Background}
\label{sec:motivation}
\subsection{ISAs are Complex}




ISA manuals are increasingly dense and ISA-state (architectural context), increasingly convoluted. 
It primarily comprises of general purpose registers (GPRs), control and status registers (CSRs), and memory. 
There is a plethora of research on keeping memory of different security domains confidential using various access control mechanisms \cite{vtx, riscv-isa, arm-mpu, sgx, trustzone}.
Yet, little focus has been put towards preventing security vulnerabilities that arise from improper sanitization of ISA-state when switching across security domains~\cite{totw}.  
However, high-impact vulnerabilities can be trivially introduced in context switch implementations while swapping ISA-state, that can further be exploited by an unprivileged adversary with minimal effort. 
The reason for this is two-fold.

First, control and status registers (CSRs in RISC-V), special-purpose registers (Arm) or model specific registers (MSRs in x86) are immensely complex - up to the bit-level. 
Each bit in such registers can reconfigure the processor's behavior, thus influencing the execution of the following instructions.
It is trivial to introduce bugs when setting/clearing some of the bits in a register due to possible inter-dependencies with bits of another register. 
For instance, when any floating point (FP) operations are executed on RISC-V platforms, a Dirty bit is set in the \texttt{sstatus} register as an optimization to check whether the floating point register state needs be saved. 
However, not clearing this bit during a context switch can leak information about floating point unit use inside a security domain. 
Similarly, prior work reported that Intel SGX's SDK fails to clear two bits during the context switch that harm the computational integrity of the operations executed inside the enclave and also lead to memory corruption~\cite{totw}. 

Second, there is a higher risk of introducing vulnerabilities at the ISA-level for ISAs that promote modularity and customization, such as RISC-V.
This is especially true if the privileged software does not properly handle discovery of the ISA extensions supported by the platform on which it runs. 
Prior work shows that incorrect sanitization of floating point state on the x86-64 and RISC-V architectures leads to unintentional side-channels that can be exploited by unprivileged adversaries~\cite{dtrap-fpu}. 
Moreover, one of the three classes of mishandled ISA-state we find also stem from ISA-extensions. 

\subsection{Overlooking security at the ABI-layer}

Existing strategies to avoid vulnerabilities focus on preventing software bugs at the application level (e.g. application binary scanning, symbolic execution of user code, etc.) or introduce patches to prevent sophisticated micro-architectural attacks (speculative execution, cache side-channels)~\cite{spectre, meltdown,flush-reload}. 
These techniques do not target bugs at the ABI-layer that can breach the security guarantees of security domains even when the application code is secure and caches are partitioned. 

Further, formal verification of privileged software (that targets to avoid attacks at the ABI-layer too) relies on self-written specifications (rather than official ISA specifications) that usually model a subset of the entire machine \cite{serval, komodo, certikos}.
This is important to make the verification efforts feasible, by limiting and pruning the paths that need to be explored. 
These partial machine models are used to prove functional correctness of the security monitors \ie{} whether the context switch is saving and restoring the registers (to and from memory) as expected.
They are also used to prove the non-interference security property \ie{} on switches from security domain A to security domain B and then back to security domain A, the security domain A must not be able to distinguish the system state before vs. after the execution of security domain B. 
However, the verification only proves properties that hold for the ISA-state modeled by the self-written specifications, instead of the entire ISA-state. 
In such cases, it's difficult to extend the proven properties to the ISA-state that was not modeled.
It is also difficult to identify security-sensitive ISA-state in the first place, which could be useful to determine the exact subset of the ISA that the verification frameworks must model. 
However, that requires studying the ISA specifications extensively and meticulously. 

\subsection{Secure Context Switches}
A naive approach to implement secure context switches is to consider all ISA-state to be intrinsically security-sensitive.
Thus, a context switch implementation in privileged software would swap the entire ISA-state when switching across security domains. 
While in theory this approach seems intuitive and appealing, there are two problems with this approach.

First, due to the convoluted nature of architectures, with hundreds or thousands of registers~\cite{riscv-isa, arm-ref-manual}, it is trivial to overlook ISA-state that requires to be swapped during context switches across security domains. 
This is especially true in the case of ISA-extensions. 
It is because of such oversights that security vulnerabilities get a chance to surface-up and be exploited by unprivileged adversaries with minimal efforts~\cite{dtrap-fpu, totw}. 

Second, context switches are expensive~\cite{context-switch-cost, context-switch-cost-hotos}.
Context switching across security domains doesn't just involve reading and writing of registers, but also saving the current context onto memory and reading the previously stored context from memory. 
Further, depending on the ISA-state that is exposed to a security domain (executing in a particular privilege mode) and the ISA-state that the execution of the security domain is dependent on, not all ISA-state might be security-sensitive. 
For example, there are three privilege modes on RISC-V \ie{} User (U-mode), Supervisor (S-mode) and Machine (M-mode).
If two security domains execute in U-mode, then the S-mode CSR \texttt{satp} would be considered security-sensitive as it contains the page table root address, that would impact the execution of the security domains in U-mode. 
However, the S-mode CSR \texttt{sepc} that contains the exception program counter on traps from U-mode to S-mode, would not be considered security-sensitive since it neither impacts execution of the security domains in U-mode, nor is accessible in U-mode.  
It can be beneficial especially for performance-critical systems to dig deeper in the ISA and find out exactly which ISA-state needs to be swapped during a context switch. 


