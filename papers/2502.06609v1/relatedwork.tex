\section{Related Work}
Prior work includes manual analysis of the ISA specification to find exploitable bugs in context switch implementations~\cite{guard-dilemma, rop-sgx, dtrap-fpu, totw, enclave-isolation}. 
Manual analysis of the entire ISA specification can trivially become erroneous, similar to introducing bugs in context switch implementations in the first place. 
These prior works typically focus on particular systems to analyze or finding bugs related to particular ISA extensions (\eg{} Floating Point Unit~\cite{dtrap-fpu}) to limit the scope of the required manual ISA-analysis. 

Some privileged software are formally verified for functionality and security properties, such as non-interference, often requiring multiple years of effort for extensive verification~\cite{sel4, certikos, komodo}. 
However, even that is not always enough to guarantee strict isolation at the ABI layer as these works write their own system specifications which are usually a subset of the actual ISA. 
This is done in order to make the verification feasible, since the work tends to be strenuous and if the state explored is not limited, then the verifiers or provers can quickly face path explosions. 
For instance, prior work verifies security monitor software for RISC-V and x86-32 architectures, but model only the integer (I), multiplication (M) and control and status register (Zicsr) instruction sets for RISC-V, and only GPRs for x86-32~\cite{serval}. 

Sail has an Ocaml parser that can be used with any of the different backends also written in Ocaml. 
The backends are for compiling Sail into the C/Ocaml emulators, or producing theorem prover definitions for Coq, Hol4, Isabelle, etc. 
In the existing implementation, the Sail parser cannot directly produce the type of ISA insights we extract with the Scanner. 
There are existing efforts to create a new backend that works with the Ocaml parser to extract ISA-related information from Sail \cite{thinkopenly}. 
The type of ISA-related information extracted includes name and operands of instructions, their description and register names, descriptions and sizes. 
The backend is meant to be simple and serve as a stepping stone for researchers to be able to extend that into more sophisticated use-cases. 
While Sailor is focused towards extracting more specific information from Sail towards the pre-defined goal of secure context switch implementations, compared to the work described above, both the works are aligned in making use of Sail to automatically retrieve ISA-related information and replace manual efforts of browsing through ISA specifications. 

Prior work involves attacks at the API-layer (as \autoref{fig:conf-comp} shows) which includes exploiting missing pointer checks, buffer overflows, etc. that can cause memory leaks or breach integrity of security domains~\cite{iago, totw}. 
Our work specifically focuses on the ABI-layer and not the API-layer, and thus requires mitigation strategies to prevent attacks at the API-layer to work in tandem. 
Similarly, our work does not target preventing micro-architectural attacks (cache side-channels, speculative execution attacks~\cite{spectre, meltdown, flush-reload}) and requires separately applying mitigation strategies for those as well.