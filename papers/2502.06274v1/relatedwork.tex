\section{Related Work}
\label{sec:related}
\paragraph{Existing Datasets.} Several specialized medical datasets have been developed for studying drug-side effect relationships. SIDER~\cite{kuhn2010side,kuhn2016sider} contains drug-side effect associations extracted from drug labels and public documents, serving as a foundational dataset of single drug documentation. Building upon this, OFFSIDES~\cite{kumar2024predicting} incorporates data from clinical trials, offering more rigorous evidence. TWOSIDES~\cite{tatonetti2012data} further advances the data complexity by providing comprehensive information about drug combinations and their associated side effects derived from FAERS~\cite{fda_faers,faers_technical}, capturing real-world multi-drug interaction cases. These datasets are further augmented by protein-protein interaction networks from databases like HPRD~\cite{keshava2009human} and BioGRID~\cite{stark2006biogrid}, and drug-protein target interactions from STITCH~\cite{kuhn2007stitch}, enabling researchers to construct multi-modal graphs for side effect prediction in both single-drug and combination therapy scenarios \cite{zitnik2018modeling, wang2009pubchem, ryu2018deep, vilar2017role}. While these datasets have achieved some success in characterizing drug-side effect relationships, they face significant limitations in capturing higher-order interactions in polypharmacy scenarios. The challenges stem from three main aspects: \ding{202} Insufficient coverage of drug combinations, which limits our understanding of complex drug interactions; \ding{203} Inconsistent data quality across different sources~\cite{ismail2022fda}, which affects the reliability of predictions; and \ding{204} Data sparsity issues in polypharmacy settings, where the number of possible drug combinations grows exponentially while available data remains limited~\cite{tekin2017measuring,tekin2018prevalence}.

% However, these datasets face significant limitations in their ability to capture higher-order relationships in drug-side effect interactions~\cite{tekin2017measuring,tekin2018prevalence}, due to insufficient coverage of drug combinations, inconsistent data quality across different sources~\cite{ismail2022fda}, and data sparsity issues, particularly under polypharmacal background.

% \paragraph{Existing Methods.} Traditional machine learning methods have been widely applied in drug-side effect relationship studies through data-driven approaches \cite{sachdev2020comprehensive}. Early researchers employed Naive Bayes \cite{jansen2003bayesian, burger2008accurate}, Support Vector Machines (SVMs) \cite{ferdousi2017computational} and network analysis \cite{ye2014construction} to process chemical structures \cite{staszak2022machine}, molecular fingerprints \cite{rogers2010extended}, and protein interaction networks \cite{chowdhary2009bayesian, bradford2006insights}. 

% Deep learning frameworks \cite{chowdhary2009bayesian, bradford2006insights}, including Convolutional Neural Networks (CNNs) and Long Short-Term Memory networks (LSTMs) \cite{kipf2016semi} have been combined to effectively model complex drug-side effect relationships in the context of polypharmacy. The integration of CNNs and LSTMs enables the extraction of spatial features from molecular structures and the capture of temporal dependencies in drug interactions, respectively. This combined approach has demonstrated significant success in predicting and understanding the side effects of caused by drug combinations. The Decagon model~\cite{zitnik2018modeling} further extended this capability to handle multi-modal graphs with diverse edge types representing different side effects, further enhancing the ability to model complex drug-drug interactions and their associated side effects.

% Matrix factorization approaches such as Non-negative Matrix Factorization (NMF)~\cite{lee1999learning} and Probabilistic Matrix Factorization (PMF)~\cite{jain2023graph} decomposed drug interaction matrices for predictions, though faced challenges with high computational complexity and interpretation of latent matrices. Recent developments introduced attention mechanisms through Graph Attention Networks (GAT) \cite{mohamedtrivec}. Researchers leveraged existing benchmark datasets from DrugBank\cite{wishart2018drugbank} and TWOSIDES~\cite{tatonetti2012data} to study drug-drug interactions~\cite{zhang2023emerging}, though these methods face significant limitations due to the scarcity of specialized datasets capturing higher-order drug-side effect relationships~\cite{lukavcivsin2019emergent}, particularly in polypharmacy scenarios where complex multi-drug interactions need to be modeled \cite{fatima2022comprehensive}.


%%%%%%%%%%%%%%%%%%%%%%%%%%%
\paragraph{Existing Methods.} 
Early research primarily relied on traditional machine learning methods, including Naive Bayes \cite{jansen2003bayesian, burger2008accurate}, Support Vector Machines (SVMs) \cite{ferdousi2017computational}, and Network Analysis \cite{ye2014construction}, which were used to analyze chemical structures \cite{staszak2022machine}, molecular fingerprints \cite{rogers2010extended}, and protein interaction networks \cite{chowdhary2009bayesian, bradford2006insights}. Subsequently, matrix factorization approaches such as NMF \cite{lee1999learning} and PMF \cite{jain2023graph} were introduced to decompose drug interaction matrices, though challenged by computational complexity and interpretability. With the rise of deep learning, CNNs and LSTMs were combined for the extraction of spatial features from molecular structures and the capture of temporal dependencies in drug interactions, effectively modeling complex drug-drug interaction relationships, particularly in polypharmacy scenarios \cite{xu2018leveraging, karim2019drug, wen2023multimodal, peng2024effective}. Due to the inherent advantages on leveraging the established biomedical relations, graph-based methods have been predominately investigated. For instance, the Decagon model \cite{zitnik2018modeling} leverages graphs to leverage both the drug-drug and protein-protein interactions to connect drug to side effects, and Graph Attention Networks (GAT) \cite{mohamedtrivec} were leveraged attention mechanisms to improve performance. Despite these advancements, challenges remain due to the scarcity of specialized datasets capturing higher-order drug-drug interaction relationships \cite{lukavcivsin2019emergent}, particularly in polypharmacy scenarios where complex multi-drug interactions need to be modeled \cite{fatima2022comprehensive}. While researchers have achieved some success using benchmark datasets like DrugBank \cite{wishart2018drugbank} and TWOSIDES \cite{tatonetti2012data, zhang2023emerging}, the current model performance remains limited by the lack of datasets containing higher-order drug-adverse effect relationships.


%Researchers have leveraged benchmark datasets like \textbf{DrugBank} \cite{wishart2018drugbank} and \textbf{TWOSIDES} \cite{tatonetti2012data} to study drug-drug interactions \cite{zhang2023emerging}, but further progress is limited by data availability and the complexity of modeling higher-order interactions.

%%%%%%%%%%%%%%%%%%%%%%%%%%