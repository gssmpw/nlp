%%%%%%%% ICML 2025 EXAMPLE LATEX SUBMISSION FILE %%%%%%%%%%%%%%%%%

\documentclass{article}
% \documentclass[twocolumn]{article}
\usepackage{abstract}


% 1. Font and encoding
\usepackage[T1]{fontenc}       % Font encoding
\usepackage{lmodern}           % Latin Modern font (improves font rendering)

% 2. Math-related
\usepackage{amsmath}           % Core math functionality
\usepackage{amssymb}           % Additional math symbols
\usepackage{bm}                % Bold math
\usepackage{mathtools}         % Extended math features
\usepackage{newtxmath}         % Text and math fonts (redefines \mathbf)
\let\openbox\relax             % Avoid conflicts with `amsthm`

% 3. Graphics and colors
\usepackage{graphicx}          % For including graphics
\usepackage{subfigure}         % For subfigures
\usepackage{xcolor}            % For colors
\definecolor{darkblue}{rgb}{0,0.08,0.45} % Custom color
\usepackage{wrapfig} % 在导言区添加 wrapfig 包



% 4. Hyperlinks and URLs
\usepackage{xurl}              % Break long URLs
\usepackage{hyperref}          % Hyperlinks in the document
\hypersetup{
    breaklinks=true,
    colorlinks=true,
    urlcolor=darkblue,
    linkcolor=darkblue,
    citecolor=darkblue
}

% 5. Tables and layout
\usepackage{booktabs}          % Professional table formatting
\usepackage{multirow}          % Multirow in tables
\usepackage{makecell}          % Improved table cell handling
\usepackage{float} % 用于控制表格位置
\usepackage{natbib} % 用于管理参考文献
\usepackage{lipsum} % 用于生成示例文本
\usepackage{marginnote}
\usepackage{textpos}
\usepackage{tabularx} % 引入 tabularx 包
% Table coloring scheme
% \documentclass{article}
\usepackage{booktabs}
\usepackage{xcolor}
\usepackage{colortbl}
\usepackage{graphicx}

% 6. Lists and enumeration
\usepackage{enumerate}         % Customizable enumeration
\usepackage{enumitem}          % Better list customization

% 7. Algorithms
% \usepackage[ruled,vlined]{algorithm2e}  % Algorithm2e package
% \SetAlgoNlRelativeSize{-1}             % Adjust line number size
% \SetInd{0em}{0em}                      % Indentation
% \SetAlgoVlined                         % Vlined blocks
% \usepackage{algorithmic}               % Algorithmic package
% \newcommand{\theHalgorithm}{\arabic{algorithm}}

% Patch Algorithm2e settings
% \usepackage{etoolbox} % For patching commands
% \makeatletter
% \patchcmd{\algocf@makecaption@ruled}{\hsize}{\textwidth}{}{}
% \patchcmd{\@algocf@start}{-1.5em}{0em}{}{}
% \makeatother
% \SetAlCapHSkip{0pt}
% \setlength{\algomargin}{0em}
% \usepackage[ruled,vlined]{algorithm2e}
% \usepackage{algpseudocode}

\newcommand{\jw}[1]{\textcolor{blue}{$\langle$JunWen: #1$\rangle$}}

\usepackage{mdframed}
\usepackage{tcolorbox}

% 8. ICML template
%\usepackage[nohyperref]{icml2025}  % Use ICML-specific template
% Uncomment if accepted:
\usepackage[accepted]{icml2025}

% 9. Theorems
\usepackage{amsthm}            % For theorem environments

% 10. Captions
\usepackage{caption}           % Customize captions
\usepackage[capitalize,noabbrev]{cleveref}

% footnotes
% \documentclass[twocolumn]{article}
\usepackage{footmisc}
\usepackage{booktabs}
\usepackage{graphicx}
\usepackage{stfloats} % 用于处理双栏模式下的脚注

% % Recommended, but optional, packages for figures and better typesetting:
% \usepackage{microtype}
\usepackage{graphicx}
\usepackage{placeins}
% \usepackage{subfigure}
% \usepackage{booktabs} % for professional tables
% \usepackage[nohyperref]{icml2025}
% % hyperref makes hyperlinks in the resulting PDF.
% % If your build breaks (sometimes temporarily if a hyperlink spans a page)
% % please comment out the following usepackage line and replace
% % \usepackage{icml2025} with \usepackage[nohyperref]{icml2025} above.
% % \usepackage{hyperref}
% % \usepackage{hyperref}
% % \hypersetup{
% %     breaklinks=true,
% %     colorlinks=true,
% %     urlcolor=blue
% % }
% % \usepackage{url}

% \usepackage{icml2025} % ICML template
% \usepackage{xurl} % For breaking URLs
% \usepackage{xcolor} % Already in your document
% \definecolor{darkblue}{rgb}{0,0.08,0.45} % Adjust RGB values as needed
% \usepackage{hyperref} % For clickable links
% \hypersetup{
%     breaklinks=true,
%     colorlinks=true,
%     urlcolor=darkblue,
%     linkcolor=darkblue,
%     citecolor=darkblue
% }
% % \bibliographystyle{plainnat}
% % \setcitestyle{authoryear, round, citesep={;}, aysep={,}, yysep={;}}


% \usepackage[ruled,vlined]{algorithm2e}
% \usepackage{bm}
% \usepackage[T1]{fontenc}
% \usepackage{lmodern}

% % \usepackage{hyperref}
% \usepackage{url}

% \usepackage{enumerate}
% \usepackage{enumitem}
% \usepackage{multirow}
% \usepackage{fix-cm}
% \usepackage{rotating} % For rotating the dataset names
% \usepackage{makecell} % Add this to the preamble



% % \usepackage{algorithm}
% \SetAlgoNlRelativeSize{-1}   % Adjust the size of line numbers
% \SetInd{0em}{0em}        % Adjust the indentation levels (first-level and nested)
% \SetAlgoVlined                % Ensure algorithms are vlined (vertical lines for blocks)

% \usepackage{etoolbox} % For \patchcmd
% \makeatletter
% \patchcmd{\algocf@makecaption@ruled}{\hsize}{\textwidth}{}{} % Caption to stretch full text width
% \patchcmd{\@algocf@start}{-1.5em}{0em}{}{} % For // to right margin
% \makeatother
% \SetAlCapHSkip{0pt} % Reset left skip of caption
% \setlength{\algomargin}{0em}


% \usepackage{algorithmic}

% % Attempt to make hyperref and algorithmic work together better:
% \newcommand{\theHalgorithm}{\arabic{algorithm}}

% % Use the following line for the initial blind version submitted for review:
% % \usepackage{icml2025}

% % If accepted, instead use the following line for the camera-ready submission:
%\usepackage[accepted]{icml2025}
% \usepackage{icml2025}

% % For theorems and such
% \usepackage{newtxmath}
% % \renewcommand{\mathbf}[1]{\textbf{#1}}
% \usepackage{amsmath}
% % \usepackage{bm}
% \usepackage{caption}
% \usepackage{amssymb}
% \usepackage{mathtools}
% \let\openbox\relax
% \usepackage{amsthm}

% % if you use cleveref..
% \usepackage[capitalize,noabbrev]{cleveref}

%%%%%%%%%%%%%%%%%%%%%%%%%%%%%%%%
% THEOREMS
%%%%%%%%%%%%%%%%%%%%%%%%%%%%%%%%
\theoremstyle{plain}
\newtheorem{theorem}{Theorem}[section]
\newtheorem{proposition}[theorem]{Proposition}
\newtheorem{lemma}[theorem]{Lemma}
\newtheorem{corollary}[theorem]{Corollary}
\theoremstyle{definition}
\newtheorem{definition}[theorem]{Definition}
\newtheorem{assumption}[theorem]{Assumption}
\theoremstyle{remark}
\newtheorem{remark}[theorem]{Remark}

% Todonotes is useful during development; simply uncomment the next line
%    and comment out the line below the next line to turn off comments
%\usepackage[disable,textsize=tiny]{todonotes}
\usepackage[textsize=tiny]{todonotes}

\usepackage{pifont}

% The \icmltitle you define below is probably too long as a header.
% Therefore, a short form for the running title is supplied here:
% \icmltitlerunning{HODDI: The First Higher-Order Drug-Side Effect Interaction Dataset for Advanced Healthcare Discovery}
\icmltitlerunning{HODDI}

\begin{document}

\twocolumn[
%\icmltitle{HODDI: The First Higher-Order Drug-Side Effect Interaction Dataset for Advanced Healthcare Discovery}
\icmltitle{HODDI: A Dataset of High-Order Drug-Drug Interactions for Computational Pharmacovigilance}

% It is OKAY to include author information, even for blind
% submissions: the style file will automatically remove it for you
% unless you've provided the [accepted] option to the icml2025
% package.

% List of affiliations: The first argument should be a (short)
% identifier you will use later to specify author affiliations
% Academic affiliations should list Department, University, City, Region, Country
% Industry affiliations should list Company, City, Region, Country

% You can specify symbols, otherwise they are numbered in order.
% Ideally, you should not use this facility. Affiliations will be numbered
% in order of appearance and this is the preferred way.
\icmlsetsymbol{equal}{*}

\begin{icmlauthorlist}
\icmlauthor{Zhaoying Wang}{iit}
\icmlauthor{Yingdan Shi}{iit}
\icmlauthor{Xiang Liu}{iit}
\icmlauthor{Can Chen}{unc}
\icmlauthor{Jun Wen}{harv}
\icmlauthor{Ren Wang}{iit}
% \icmlauthor{Firstname3 Lastname3}{comp}
% \icmlauthor{Firstname4 Lastname4}{sch}
% \icmlauthor{Firstname5 Lastname5}{yyy}
% \icmlauthor{Firstname6 Lastname6}{sch,yyy,comp}
% \icmlauthor{Firstname7 Lastname7}{comp}
%\icmlauthor{}{sch}
% \icmlauthor{Firstname8 Lastname8}{sch}
% \icmlauthor{Firstname8 Lastname8}{yyy,comp}
%\icmlauthor{}{sch}
%\icmlauthor{}{sch}
\end{icmlauthorlist}

\icmlaffiliation{iit}{Illinois Institute of Technology}
\icmlaffiliation{unc}{The University of North Carolina at Chapel Hill}
\icmlaffiliation{harv}{Harvard Medical School}

\icmlcorrespondingauthor{Can Chen}{canc@unc.edu}
\icmlcorrespondingauthor{Jun Wen}{Jun$\_$Wen@hms.harvard.edu}
\icmlcorrespondingauthor{Ren Wang}{rwang74@iit.edu}
% You may provide any keywords that you
% find helpful for describing your paper; these are used to populate
% the "keywords" metadata in the PDF but will not be shown in the document
% \icmlkeywords{Machine Learning, ICML}
\vskip 0.3in
]

% this must go after the closing bracket ] following \twocolumn[ ...

% This command actually creates the footnote in the first column
% listing the affiliations and the copyright notice.
% The command takes one argument, which is text to display at the start of the footnote.
% The \icmlEqualContribution command is standard text for equal contribution.
% Remove it (just {}) if you do not need this facility.

\printAffiliationsAndNotice{}  % leave blank if no need to mention equal contribution
% \printAffiliationsAndNotice{\icmlEqualContribution} % otherwise use the standard text.


\newcommand{\authnote}[2]{{\bf \textcolor{blue}{#1}: \em \textcolor{red}{#2}}}
\newcommand{\Ren}[1]{\authnote{Ren}{#1}}
\newcommand{\Yingdan}[1]{\authnote{Yingdan}{#1}}
\newcommand{\Zhaoying}[1]{\authnote{Zhaoying}{#1}}

\newcommand{\nodes}{\mathcal{V}}
\newcommand{\edges}{\mathcal{E}}
\newcommand{\node}{\mathcal{v}}


\newcommand{\trn}{G^{tr}}
\newcommand{\Dclf}{G^{clf}}
\newcommand{\Dwmk}{G^{wmk}}
\newcommand{\val}{G^{val}}
\newcommand{\tst}{G^{test}}


\newcommand{\Xtrn}{\textbf{X}^{tr}}
\newcommand{\Xclf}{\textbf{X}^{clf}}
\newcommand{\Xwmk}{\textbf{X}^{wmk}}
\newcommand{\Xval}{\textbf{X}^{val}}
\newcommand{\Xtst}{\textbf{X}^{test}}
\newcommand{\ytrn}{{\bf y}^{tr}}
\newcommand{\yclf}{{\bf y}^{clf}}
\newcommand{\ytr}{{\bf y}^{clf}}
\newcommand{\ywmk}{{\bf y}^{wmk}}
\newcommand{\yval}{{\bf y}^{val}}
\newcommand{\ytst}{{\bf y}^{test}}
\newcommand{\ytrnhat}{{\bf p}^{tr}}
\newcommand{\yclfhat}{{\bf p}^{clf}}
\newcommand{\ywmkhat}{{\bf p}^{wmk}}
\newcommand{\yvalhat}{{\bf p}^{val}}
\newcommand{\ytsthat}{{\bf p}^{test}}
\newcommand{\edgestrn}{\edges^{tr}}
\newcommand{\edgesclf}{\edges^{clf}}
\newcommand{\edgeswmk}{\edges^{wmk}}
\newcommand{\edgesval}{\edges^{val}}
\newcommand{\edgestst}{\edges^{test}}
\newcommand{\nodestrn}{\nodes^{tr}}
\newcommand{\nodesclf}{\nodes^{clf}}
\newcommand{\nodeswmk}{\nodes^{wmk}}
\newcommand{\nodesval}{\nodes^{val}}
\newcommand{\nodestst}{\nodes^{test}}
\newcommand{\e}{\textbf{e}}
\newcommand{\Ehat}{\hat{\e}}
\newcommand{\Ewmk}{\e^{wmk}}
\newcommand{\idxwmk}{\textbf{idx}}
\newcommand{\numsub}{T}
\newcommand{\Ewmkoneindexed}{\Ewmk_1[\idxwmk]}
\newcommand{\Ewmkiindexed}{\Ewmk_i[\idxwmk]}
\newcommand{\Ewmktindexed}{\Ewmk_{\numsub}[\idxwmk]}
\newcommand{\Ewmkhat}{\Ehat^{wmk}}
\newcommand{\Ewmkhatoneindexed}{\Ewmkhat[\idxwmk]}
\newcommand{\Ewmkhatiindexed}{\Ewmkhat_i[\idxwmk]}
\newcommand{\Ewmkhatkindexed}{\Ewmkhat_k[\idxwmk]}
\newcommand{\Ecdt}{\e^{cdt}}
\newcommand{\Ecdtoneindexed}{\Ecdt_1[\idxwmk]}
\newcommand{\Ecdtiindexed}{\Ecdt_i[\idxwmk]}
\newcommand{\Ecdttindexed}{\Ecdt_{\numsub}[\idxwmk]}
\newcommand{\Ecdthat}{\Ehat^{cdt}}
\newcommand{\Ecdthatoneindexed}{\Ecdthat[\idxwmk]}
\newcommand{\Ecdthatiindexed}{\Ecdthat_i[\idxwmk]}
\newcommand{\Ecdthatkindexed}{\Ecdthat_k[\idxwmk]}
\newcommand{\Lwmkf}{\mathcal{L}_{wmk}}
\newcommand{\Lclff}{\mathcal{L}_{clf}}
\newcommand{\Lwmk}{\mathcal{L}_{wmk}}
\newcommand{\Lclf}{\mathcal{L}_{clf}}
\newcommand{\Lce}{\mathcal{L}_{CE}}
\newcommand{\bigH}{\textit{\textbf{H}}}
\newcommand{\bigI}{\textit{\textbf{I}}}
\newcommand{\bigK}{\textit{\textbf{K}}}
\newcommand{\barK}{\bar{\bigK}}
\newcommand{\bigL}{\textit{\textbf{L}}}
\newcommand{\barL}{\bar{\bigL}}
\newcommand{\tilK}{\tilde{\bigK}}
\newcommand{\tilL}{\tilde{\bigL}}
\newcommand{\HKH}{\bigH\bigK^{(k)}\bigH}
\newcommand{\HLH}{\bigH\bigL\bigH}
\newcommand{\Xuk}{\textbf{x}^{(k)}_u}
\newcommand{\Xvk}{\textbf{x}^{(k)}_v}
\newcommand{\wmk}{\textbf{w}}
% \newcommand{\numsub}{T}
% \newcommand{\numsub}{\numsub^{\prime}}
\newcommand{\MIcdt}{\text{MI}^{cdt}}
\newcommand{\MItgt}{\text{MI}^{tgt}}
\newcommand{\MIlb}{\text{MI}^{LB}}
\newcommand{\atgt}{\alpha_{tgt}}
\newcommand{\alb}{\alpha_{LB}}
\newcommand{\av}{\alpha_{v}}
\newcommand{\signate}{\sigma_{nat_e}}
\newcommand{\signatp}{\sigma_{nat_p}}
\newcommand{\munate}{\mu_{nat_e}}
\newcommand{\munatp}{\mu_{nat_p}}
\newcommand{\zlb}{z_{LB}}
\newcommand{\zv}{z_{v}}
\newcommand{\nsub}{n_{sub}}
\newcommand{\explFn}{explain}
\newcommand{\BW}[1]{\textcolor{red}{[AW: ~#1]}}
\newcommand{\JD}[1]{\textcolor{blue}{[JD: ~#1]}}
\newcommand{\wmksubgraph}{G^{wmk}}
\newcommand{\wmksubgraphs}{\{\wmksubgraph_1 ,\dots, \wmksubgraph_{\numsub}\}}
\newcommand{\fix}{\marginpar{FIX}}
\newcommand{\new}{\marginpar{NEW}}

\begin{abstract}
Drug-side effect research is vital for understanding adverse reactions arising in complex multi-drug therapies. However, the scarcity of higher-order datasets that capture the combinatorial effects of multiple drugs severely limits progress in this field. Existing resources such as TWOSIDES primarily focus on pairwise interactions. To fill this critical gap, we introduce \textbf{HODDI}, the first \underline{H}igher-\underline{O}rder \underline{D}rug-\underline{D}rug \underline{I}nteraction Dataset, constructed from U.S. Food and Drug Administration (FDA) Adverse Event Reporting System (FAERS) records spanning the past decade, to advance computational pharmacovigilance. % \Ren{I have updated the dataset name to HODDI. Please replace all HODDI with HODDI, and update any mentions of "higher-order drug-side effect interaction" to "higher-order drug-drug interaction."} %To fill this critical gap, we present \textbf{\textbf{HODDI}}, the first \textbf{\underline{H}igher-\underline{O}rder \underline{D}rug-\underline{S}ide \underline{E}ffect \underline{I}nteraction \underline{D}ataset}, constructed from U.S. Food and Drug Administration (FDA) Adverse Event Reporting System (FAERS) records spanning the latest decade. 
HODDI contains 109,744 records involving 2,506 unique drugs and 4,569 unique side effects, specifically curated to capture multi-drug interactions and their collective impact on adverse effects. Comprehensive statistical analyses demonstrate HODDI's extensive coverage and robust analytical metrics, making it a valuable resource for studying higher-order drug relationships. Evaluating HODDI with multiple models, we found that simple Multi-Layer Perceptron (MLP) can outperform graph models, while hypergraph models demonstrate superior performance in capturing complex multi-drug interactions, further validating HODDI's effectiveness. %Through extensive evaluation using diverse graph-based and non-graph-based models, we demonstrate that hypergraph neural network (HGNN) models significantly outperform conventional graph neural network (GNN) and multi-layer perceptron (MLP) approaches, validating HODDI's effectiveness in capturing complex drug interaction patterns. 
Our findings highlight the inherent value of higher-order information in drug-side effect prediction and position HODDI as a benchmark dataset for advancing research in pharmacovigilance, drug safety, and personalized medicine. The dataset and codes are available at \url{https://github.com/TIML-Group/HODDI}.
% The dataset and codes are available at {\small \url{https://github.com/zwang-repo/HODDI}}.
\end{abstract}

\section{Introduction}
\label{intro}

Clinical observations have revealed numerous cases in which drug combinations lead to unexpected adverse effects \cite{tekin2017measuring,tekin2018prevalence}. For instance, the concurrent use of multiple antibiotics with anticoagulants can significantly increase bleeding risks \cite{baillargeon2012concurrent}, while the combination of various antidepressants may elevate the likelihood of serotonin syndrome \cite{quinn2009linezolid}. These scenarios underscore the urgent need to investigate higher-order drug-drug interactions \cite{tekin2018prevalence}, as understanding the intricate relationships between multiple drugs and their side effects is pivotal in pharmacology, forming the foundation for pharmacovigilance and drug safety studies \cite{tatonetti2012data, salas2022use, huang2021moltrans}. By elucidating these interactions, researchers can identify potential adverse effects early in the drug development process \cite{montastruc2006pharmacovigilance, kompa2022artificial}, optimize drug design \cite{lipinski2012experimental}, mitigate polypharmacy risks \cite{lukavcivsin2019emergent}, and advance personalized medicine \cite{molina2017personalized} by helping clinicians tailor treatments to individual patient characteristics \cite{ryu2018deep}. Nevertheless, current research faces two major challenges: the scarcity of datasets capturing complex, high-order drug-drug interactions \cite{zitnik2018modeling, wang2017pubchem}, and the limitations of existing computational methods in effectively modeling these high-order relationships \cite{masumshah2021neural, sachdev2020comprehensive, zakharov2014computational}.

%Research on the complex drug-side effect relationships plays a pivotal role in the pharmacological domain, laying the foundation for pharmacovigilance and drug safety research. With this understanding, researchers can identify potential adverse effects early in drug development, optimize drug design, and mitigate polypharmacy risks \cite{lukavcivsin2019emergent}, while also advancing personalized medicine by helping clinicians tailor treatments to individual patient characteristics \cite{ryu2018deep}. Clinical observations have revealed numerous cases where drug combinations lead to unexpected adverse effects \cite{tekin2017measuring,tekin2018prevalence}. For instance, the concurrent use of multiple antibiotics with anticoagulants can significantly increase bleeding risks, while the combination of various antidepressants may elevate the risk of serotonin syndrome\cite{baillargeon2012concurrent,quinn2009linezolid} - these scenarios highlight the critical importance of understanding high-order drug interactions \cite{tekin2018prevalence}. However, current research faces two major challenges: first, there is a significant lack of datasets that capture these complex, high-order drug-side effect interactions; second, existing computational methods are limited in their ability to model these high-order relationships effectively \cite{masumshah2021neural}.

Several existing datasets have been developed for the research of drug-side effect relationships. While these datasets are valuable resources, they have inherent limitations in capturing high-order drug interactions.  SIDER~\cite{kuhn2010side,kuhn2016sider} and OFFSIDES~\cite{kumar2024predicting} focus primarily on single-drug effects, while TWOSIDES~\cite{tatonetti2012data} advances to pairwise drug combinations. Despite being augmented by biological interaction networks  (e.g., HPRD~\cite{keshava2009human}, BioGRID~\cite{stark2006biogrid}, STITCH~\cite{kuhn2007stitch}), these datasets predominantly focus on single-drug or two-drug interactions, facing significant challenges in polypharmacy scenarios due to insufficient coverage of multi-drug combinations~\cite{tekin2017measuring}, inconsistent data quality~\cite{ismail2022fda}, and the exponential growth of possible drug combinations relative to available data~\cite{tekin2018prevalence}.% \Ren{No need to highlight other datasets or methods. I highlighted HODESID because that is the data proposed by us.}

% Several existing datasets have been developed to study drug-side effect relationships, though they are limited in capturing high-order drug interactions. SIDER, a widely-used database, primarily focuses on single drug-side effect associations, containing comprehensive information about marketed medicines and their recorded adverse reactions \cite{kuhn2010side}. TWOSIDES represents a significant advance by specifically documenting two-drug interactions and their associated side effects, enabling the study of pairwise drug combinations \cite{tatonetti2012data}. However, these existing datasets predominantly focus on single-drug or two-drug interactions, leaving a significant gap in understanding higher-order drug interactions that commonly occur in clinical settings \cite{tekin2017measuring}.

% There are also various computational methods that have been developed for drug-side effect relationship prediction. Matrix factorization approaches (NMF, PMF) and traditional MLPs struggled with high computational complexity and modeling non-linear relationships. Deep learning frameworks and CNNs showed limitations in feature extraction from molecular structures. While graph-based methods like Decagon and GCNs demonstrated promise by leveraging protein interaction networks, they were mainly restricted to pairwise relationships. Most existing methods rely on datasets containing only pairwise drug interactions, fundamentally limiting their capability to model complex polypharmacy scenarios involving multiple drugs. \Ren{add refs}


% Various computational methods have been developed for drug-side effect relationship prediction. Matrix factorization approaches (NMF, PMF) and traditional MLPs struggled with high computational complexity and modeling non-linear relationships \cite{jain2023graph}. Deep learning frameworks and CNNs showed limitations in feature extraction from molecular structures \cite{karim2019drug, peng2024effective}. While graph-based methods like Decagon and GCNs demonstrated promise by leveraging protein interaction networks \cite{zitnik2018modeling}, they were mainly restricted to pairwise relationships \cite{zitnik2018modeling, zhang2023emerging}. Most existing methods rely on datasets containing only pairwise drug interactions, fundamentally limiting their capability to model complex polypharmacy scenarios involving multiple drugs \cite{tekin2017measuring, tekin2018prevalence}.

% Various computational methods have been developed for drug-side effect relationship prediction. Matrix factorization approaches (NMF, PMF) and traditional machine learning methods faced challenges in computational complexity and modeling non-linear relationships \cite{jain2023graph}. 

Various computational methods have been developed for drug-side effect relationship prediction. Traditional matrix factorization approaches, such as Non-negative Matrix Factorization (NMF) \cite{lee1999learning} and Probabilistic Matrix Factorization (PMF) \cite{jain2023graph}, face challenges in computational complexity due to high-dimensional matrix operations \cite{jain2023graph}, while traditional machine learning methods struggled with modeling complex non-linear relationships in drug-adverse effect interactions \cite{lee1999learning}. Deep learning frameworks, such as the combination of Convolutional Neural Networks (CNNs) and Long Short-Term Memory networks (LSTMs), while improving spatial and temporal feature extraction capabilities, exhibit limitations in molecular structure representation \cite{karim2019drug, peng2024effective, xu2018leveraging}. Graph-based methods like Decagon and Graph Convolutional Network (GCN) demonstrated promise by leveraging protein interaction networks \cite{zitnik2018modeling}, but were primarily restricted to modeling pairwise drug interactions \cite{zitnik2018modeling, zhang2023emerging}. Due to the lack of specialized datasets capturing higher-order drug-adverse effect relationships, existing methods largely rely on datasets containing only pairwise drug interactions, fundamentally limiting their capability to model complex polypharmacy scenarios \cite{tekin2017measuring, tekin2018prevalence}.

To address this gap, we introduce the \textbf{H}igher-\textbf{O}rder \textbf{D}rug-\textbf{D}rug \textbf{I}nteraction (\textbf{HODDI}) dataset, the first resource specifically designed to capture higher-order drug-drug interactions and their associated side effects. HODDI is constructed from the FAERS database \cite{fda_faers, faers_technical} and consists of 109,744 records spanning the past 10 years, covering 2,506 unique drugs and 4,569 distinct side effects. Through rigorous data cleaning and conditional filtering, we focused on cases of co-administered drugs, enabling the study of their combinational impacts on adverse effects. Comprehensive statistical analyses characterize the dataset’s key properties, highlighting its potential for advancing polypharmacy research.



To assess HODDI's utility, we created evaluation subsets and tested its performance using multiple models, with detailed methods described in Sections~\ref{sec:subsets} and \ref{Benchmark Methods}. Our experiments reveal two key findings about leveraging higher-order information in drug-side effect prediction: \ding{202} Even simple architectures like Multi-Layer Perceptron (MLP) can achieve strong performance when utilizing higher-order features from our dataset, sometimes outperforming more sophisticated models like Graph Attention Network (GAT). This suggests the inherent value of higher-order drug interaction data; \ding{203} Models that explicitly incorporate hypergraph structures (e.g., HyGNN \cite{saifuddin2023hygnn}) can further enhance prediction accuracy by better capturing complex multi-drug relationships.

We summarize our contributions as follows:
\begin{itemize}[leftmargin=*]
\vspace{-2mm}
    \item We constructed \textbf{HODDI}, the first dataset focusing on higher-order drug-drug interactions from the FAERS database.
    \item We comprehensively analyzed HODDI's statistical characteristics across multiple time scales.
    \item Our experiments highlight the importance of higher-order information in drug-side effect prediction and demonstrate the effectiveness of the proposed hypergraph model.

    %     \begin{enumerate}[label=(\alph*),leftmargin=2em]
    %     \item Both graph-based and non-graph models show improved performance when leveraging higher-order features from HODDI.
    %     \item Hypergraph-based models further enhance prediction accuracy by explicitly modeling multi-drug interactions.
    %     \item High-confidence subsets better capture higher-order relationships, highlighting the importance of rigorous data preprocessing.
    % \end{enumerate}
\end{itemize}
\section{Related Work}
\label{sec:related}
\paragraph{Existing Datasets.} Several specialized medical datasets have been developed for studying drug-side effect relationships. SIDER~\cite{kuhn2010side,kuhn2016sider} contains drug-side effect associations extracted from drug labels and public documents, serving as a foundational dataset of single drug documentation. Building upon this, OFFSIDES~\cite{kumar2024predicting} incorporates data from clinical trials, offering more rigorous evidence. TWOSIDES~\cite{tatonetti2012data} further advances the data complexity by providing comprehensive information about drug combinations and their associated side effects derived from FAERS~\cite{fda_faers,faers_technical}, capturing real-world multi-drug interaction cases. These datasets are further augmented by protein-protein interaction networks from databases like HPRD~\cite{keshava2009human} and BioGRID~\cite{stark2006biogrid}, and drug-protein target interactions from STITCH~\cite{kuhn2007stitch}, enabling researchers to construct multi-modal graphs for side effect prediction in both single-drug and combination therapy scenarios \cite{zitnik2018modeling, wang2009pubchem, ryu2018deep, vilar2017role}. While these datasets have achieved some success in characterizing drug-side effect relationships, they face significant limitations in capturing higher-order interactions in polypharmacy scenarios. The challenges stem from three main aspects: \ding{202} Insufficient coverage of drug combinations, which limits our understanding of complex drug interactions; \ding{203} Inconsistent data quality across different sources~\cite{ismail2022fda}, which affects the reliability of predictions; and \ding{204} Data sparsity issues in polypharmacy settings, where the number of possible drug combinations grows exponentially while available data remains limited~\cite{tekin2017measuring,tekin2018prevalence}.

% However, these datasets face significant limitations in their ability to capture higher-order relationships in drug-side effect interactions~\cite{tekin2017measuring,tekin2018prevalence}, due to insufficient coverage of drug combinations, inconsistent data quality across different sources~\cite{ismail2022fda}, and data sparsity issues, particularly under polypharmacal background.

% \paragraph{Existing Methods.} Traditional machine learning methods have been widely applied in drug-side effect relationship studies through data-driven approaches \cite{sachdev2020comprehensive}. Early researchers employed Naive Bayes \cite{jansen2003bayesian, burger2008accurate}, Support Vector Machines (SVMs) \cite{ferdousi2017computational} and network analysis \cite{ye2014construction} to process chemical structures \cite{staszak2022machine}, molecular fingerprints \cite{rogers2010extended}, and protein interaction networks \cite{chowdhary2009bayesian, bradford2006insights}. 

% Deep learning frameworks \cite{chowdhary2009bayesian, bradford2006insights}, including Convolutional Neural Networks (CNNs) and Long Short-Term Memory networks (LSTMs) \cite{kipf2016semi} have been combined to effectively model complex drug-side effect relationships in the context of polypharmacy. The integration of CNNs and LSTMs enables the extraction of spatial features from molecular structures and the capture of temporal dependencies in drug interactions, respectively. This combined approach has demonstrated significant success in predicting and understanding the side effects of caused by drug combinations. The Decagon model~\cite{zitnik2018modeling} further extended this capability to handle multi-modal graphs with diverse edge types representing different side effects, further enhancing the ability to model complex drug-drug interactions and their associated side effects.

% Matrix factorization approaches such as Non-negative Matrix Factorization (NMF)~\cite{lee1999learning} and Probabilistic Matrix Factorization (PMF)~\cite{jain2023graph} decomposed drug interaction matrices for predictions, though faced challenges with high computational complexity and interpretation of latent matrices. Recent developments introduced attention mechanisms through Graph Attention Networks (GAT) \cite{mohamedtrivec}. Researchers leveraged existing benchmark datasets from DrugBank\cite{wishart2018drugbank} and TWOSIDES~\cite{tatonetti2012data} to study drug-drug interactions~\cite{zhang2023emerging}, though these methods face significant limitations due to the scarcity of specialized datasets capturing higher-order drug-side effect relationships~\cite{lukavcivsin2019emergent}, particularly in polypharmacy scenarios where complex multi-drug interactions need to be modeled \cite{fatima2022comprehensive}.


%%%%%%%%%%%%%%%%%%%%%%%%%%%
\paragraph{Existing Methods.} 
Early research primarily relied on traditional machine learning methods, including Naive Bayes \cite{jansen2003bayesian, burger2008accurate}, Support Vector Machines (SVMs) \cite{ferdousi2017computational}, and Network Analysis \cite{ye2014construction}, which were used to analyze chemical structures \cite{staszak2022machine}, molecular fingerprints \cite{rogers2010extended}, and protein interaction networks \cite{chowdhary2009bayesian, bradford2006insights}. Subsequently, matrix factorization approaches such as NMF \cite{lee1999learning} and PMF \cite{jain2023graph} were introduced to decompose drug interaction matrices, though challenged by computational complexity and interpretability. With the rise of deep learning, CNNs and LSTMs were combined for the extraction of spatial features from molecular structures and the capture of temporal dependencies in drug interactions, effectively modeling complex drug-drug interaction relationships, particularly in polypharmacy scenarios \cite{xu2018leveraging, karim2019drug, wen2023multimodal, peng2024effective}. Due to the inherent advantages on leveraging the established biomedical relations, graph-based methods have been predominately investigated. For instance, the Decagon model \cite{zitnik2018modeling} leverages graphs to leverage both the drug-drug and protein-protein interactions to connect drug to side effects, and Graph Attention Networks (GAT) \cite{mohamedtrivec} were leveraged attention mechanisms to improve performance. Despite these advancements, challenges remain due to the scarcity of specialized datasets capturing higher-order drug-drug interaction relationships \cite{lukavcivsin2019emergent}, particularly in polypharmacy scenarios where complex multi-drug interactions need to be modeled \cite{fatima2022comprehensive}. While researchers have achieved some success using benchmark datasets like DrugBank \cite{wishart2018drugbank} and TWOSIDES \cite{tatonetti2012data, zhang2023emerging}, the current model performance remains limited by the lack of datasets containing higher-order drug-adverse effect relationships.


%Researchers have leveraged benchmark datasets like \textbf{DrugBank} \cite{wishart2018drugbank} and \textbf{TWOSIDES} \cite{tatonetti2012data} to study drug-drug interactions \cite{zhang2023emerging}, but further progress is limited by data availability and the complexity of modeling higher-order interactions.

%%%%%%%%%%%%%%%%%%%%%%%%%%

\section{HODDI Dataset Construction}
\label{dataset_construction}

We utilized the FAERS database~\cite{fda_faers,faers_technical} as our primary data source, which contains comprehensive adverse event reports, medication error reports, and product quality complaints submitted to the U.S. Food and Drug Administration (FDA) by healthcare professionals, consumers, and manufacturers. We downloaded the quarterly datasets spanning from 2014Q3 to 2024Q3, covering the latest 41 quarters in total.\footnote{Q3 denotes the third quarter of the year.} Each quarterly dataset is a compressed zip file containing individual raw XML files. 


\paragraph{Key Components Extraction.}
After decompressing the downloaded zip files and parsing the XML data, we extracted key components from each record in the FAERS database, including the report ID, the side effect names, and the standardized drug names with their drug roles\protect\footnotemark\kern0.2em. 


\begin{table}[h]
    \centering
    \caption{An example of the extracted components from a FAERS record: Three medications—Hydroxychloroquine, Prednisone, and Terbinafine—were administered simultaneously, all classified under Drug Role 3 (concomitant medications).}
    \label{extracted_component}
    \resizebox{0.45\textwidth}{!}{
    \begin{tabular}{ccc}
    \toprule
    \rowcolor{gray!20}
    \textbf{Report ID} & \textbf{Side Effect} & \textbf{Drug (Role)} \\
    \midrule
    24135951 & \begin{tabular}[c]{@{}c@{}}Psoriasis\\Drug interaction\end{tabular} & \begin{tabular}[c]{@{}c@{}}Hydroxychloroquine (3)\\Prednisone (3)\\Terbinafine (3)\end{tabular} \\
    \bottomrule
    \end{tabular}}
\end{table}

\begin{@twocolumnfalse}
\footnotetext{Drug roles are categorized into three types in the FAERS database: \textbf{a) Drug Role 1:} primary suspect drugs that are primarily suspected of causing the adverse event; \textbf{b) Drug Role 2:} secondary suspect drugs that may have contributed to the event; \textbf{c) Drug Role 3:} concomitant medications that were being taken at the same time of the event.}
\end{@twocolumnfalse}


Table~\ref{extracted_component} illustrates an example of a simultaneous drug administration during an adverse event period. According to the clinical report (ID: 24135951), three concomitant medications (Hydroxychloroquine, Prednisone, and Terbinafine), classified as Drug Role 3, were administered concurrently during the treatment. This combination of drug therapy represents a complex pharmacological intervention with potentially risky drug-drug interactions.

\paragraph{Conditional Filtering.}

After extraction of key components from each record in the FAERS database, the resulting records from the latest 41 quarters (2014Q4-2024Q3) were filtered based on the following three conditions to capture the potential suspect drug interactions:


\textbf{a) Condition 1:} At least two concomitant drugs \textit{(Drug Role 3)} were reported in a record, with no primary suspect drugs \textit{(Drug Role 1)} or secondary suspect drugs \textit{(Drug Role 2)}; 

\textbf{b) Condition 2:} At least one primary suspect drug \textit{(Drug Role 1)} combined with at least one concomitant drug \textit{(Drug Role 3)}, with no secondary suspect drugs \textit{(Drug Role 2)}; 

\textbf{c) Condition 3: }Concurrent presence of all three drug roles - primary suspect drug \textit{(Drug Role 1)}, secondary suspect drug \textit{(Drug Role 2)}, and concomitant drug \textit{(Drug Role 3)}. 

These conditions enable comprehensive identification of potential adverse drug interactions, including cases where concomitant medications may contribute to adverse events.

\paragraph{Side Effects Processing.}

Side effect names were first extracted from the original FAERS records and then encoded into 768-dimensional embedding vectors using Self-aligning pretrained Bidirectional Encoder Representations from Transformers (SapBERT)~\cite{liu-etal-2021-self}, a model pre-trained on PubMed texts for biomedical entity normalization~\cite{lim2022sapbert}. 
The mapping between the standardized adverse event terminology (MedDRA terms) and their corresponding Unified Medical Language System (UMLS) Concept Unique Identifiers (CUIs) were obtained from the UMLS Metathesaurus~\cite{umls}.
The cosine similarities were calculated between all embedding pairs of the side effect names from the original FAERS records and the standardized MedDRA terms, with their embeddings generated using SapBERT.
For each side effect name in the records, its recommended CUI was determined by identifying the standardized MedDRA term with the highest cosine similarity to its embedding.
The cosine similarity thresholds of 0.8 and 0.9 were used to stratify the side effects based on their confidence levels, with the higher threshold value indicating the higher confidence of their classifications as recommended CUIs. 
The recommended CUIs served as the labels of side effect names, while the 768-dimensional embedding vectors of the corresponding standardized MedDRA terms served as the feature representations across all models. 


\paragraph{Drug Names Processing.}

% The standardized drug names were then mapped to their corresponding DrugBank IDs by querying from DrugBank's comprehensive database (\textbf{DrugBankFullDataBase.xml} file from \url{https://go.drugbank.com/releases/latest})~\cite{wishart2006drugbank}
The standardized drug names were extracted from the FAERS records and normalized by converting all standardized drug names to uppercase, removing salt form suffixes (e.g., hydrochloride/HCL), and handling compound drug names.
Then the normalized drug names were mapped to their corresponding DrugBank IDs using DrugBank's comprehensive database (DrugBankFullDatabase.xml)~\cite{wishart2006drugbank}, which contains the detailed drug information including synonyms and alternative names.





\paragraph{Positive Sample Construction.}

In the context of polypharmacy, positive samples refer to drug combinations that have been confirmed to cause specific side effects in real-world medical records. For HODDI dataset construction, positive samples were constructed from records with high-confidence side effects with high cosine similarities above 0.9 for side effect embeddings generated by SapBERT. To eliminate data redundancy, we removed duplicate records and checked for superset relationships. For each record with a higher number of co-administered drugs, we examined whether its drug combination and associated side effects completely overlapped with those in records containing fewer drugs. If such complete overlap was found, we removed the superset record to maintain the most parsimonious representation of drug-side effect associations across the dataset spanning 41 quarters (2014Q3-2024Q3). This reduction strategy ensures each unique drug combination is represented only by its minimal complete set, preserving the most fundamental drug interactions while eliminating redundant higher-order drug combinations, improving computational efficiency and reducing noises in the HODDI dataset.

\paragraph{Negative Sample Construction.}
Negative samples represent drug combinations that have not been reported to cause the specific adverse events in clinical records or are unlikely to trigger such adverse events based on their known pharmacological properties.
In this work, negative samples were artificially constructed cases by replacing both a drug and a side effect from positive samples with random alternatives, based on the assumption that it is extremely unlikely for a randomly assembled drug combination to cause a specific side effect while ensuring the generated samples don't exist among positive samples. An equal number of positive and negative samples were generated, which helps maintain the class balance in the HODDI dataset and prevents model bias during training, ensuring that the model would not favor predicting one class over another, and improving its generalization capability and prediction accuracy.
The obtained positive and negative samples were partitioned chronologically into 41 quarterly intervals from 2014Q3 to 2024Q3, resulting in 41 pairs of positive and negative sample files (82 CSV files in total).



%%%%%%%%%%%%%%%%%%%%%%%%%%%%%%%



\begin{table}[t] 
   \centering
    \caption{HODDI dataset statistics, including the number of records (\#Record), unique standardized drug names per record in positive/negative samples (\#Drug+/-), and unique CUIs in positive/negative samples (\#CUI+/-).}
   \setlength{\tabcolsep}{4pt}
   \centering
   \resizebox{0.48\textwidth}{!}{
   \begin{tabular}{@{}cccccc@{}}
       \toprule
       \rowcolor{gray!20}
       \textbf{\large Time} & \textbf{\large \#Record} & \textbf{\large \#Drug+} & \textbf{\large \#Drug-} & \textbf{\large \#CUI+} & \textbf{\large \#CUI-} \\ 
       \midrule
       % drug2-8\_SE5-50 & 21,503 & 1,779 & 10,460 & 1,420 & 4,540 \\
       % drug2-16\_SE5-50 & 24,777 & 2,050 & 10,917 & 1,580 & 4,566 \\
       % drug2-16\_SE5-100 & 38,430 & 2,156 & 11,835 & 1,778 & 4,579 \\
       \textbf{2014Q3-2024Q3} & 109,744 & 2,506 & 12,293 & 3,950 & 4,581 \\
       \bottomrule
   \end{tabular}
   }
   \vspace{-4mm} 

   \label{dataset_statistics_41quarters}
   \vspace{-2mm}
\end{table}

%%%%%%%%%%%%%%%%%%%%%%%%%%%%%%%%%%
\vspace{-2mm}
\section{Data Analysis}
\label{data_analysis}

\paragraph{Dataset Structure.}
The HODDI dataset comprises two distinct sets of CSV files: The first set contains the merged positive and negative samples consolidated into two CSV files (\textit{pos.csv} and \textit{neg.csv}). The second set encompasses quarterly data spanning from 2014Q3 to 2024Q3, containing 82 CSV files (41 quarters \(\times\) 2 files per quarter). For each quarter \(Q\), there are two corresponding files denoted as \(Q\_pos.csv\) and \(Q\_neg.csv\), where \(Q\) represents the quarter identifier (e.g., 2014Q3), and the suffixes \(pos\) and \(neg\) indicate positive and negative samples, respectively.
This dataset structure follows the established pharmacovigilance data processing methods~\cite{sakaeda2013data,poluzzi2012data}. Each CSV file contains five essential columns that capture key information about drug-side effect reports: report identifier, recommended CUI, DrugBank ID, hyperedge labels (1 for positive samples, -1 for negative samples), and temporal information (year and quarter of each record), shown in Table~\ref{hgnn_dataset_structure}. 
The HODDI dataset structure facilitates the analysis of higher-order relationships between drug combinations and their adverse effects, enabling comprehensive data mining of multi-drug interaction pattern.

\paragraph{Dataset Composition.}
As shown in Table~\ref{dataset_statistics_41quarters}, the HODDI dataset spanning from 2014Q3 to 2024Q3 comprises 109,744 drug-side effect association records (balanced between positive and negative samples). In the complete dataset, the positive samples contain 2,506 unique standardized drug names and 3,950 unique CUIs, while the negative samples include 12,293 unique drugs and 4,581 unique CUIs. The increased number of unique entities in the negative samples is attributed to the resampling process during negative sample generation~\cite{ismail2022fda}.

% Figure 1-3: distribution of drugs and se across all quarters

\begin{figure*}[t]
  \centering
  \includegraphics[trim=1 150 2 70, clip, width=\textwidth]{images/Figure123_overall_statistics4.pdf}
  \vspace{-8mm}
  \caption{Distribution analysis of medication records and adverse event frequencies in the HODDI dataset (2014Q3-2024Q3). (a) Distribution of drug counts per record (\#Drug/Record), showing a concentration between 2-8 drugs per record; (b) Frequency distribution of adverse event occurrences (\#Occurrence) in positive samples, with most events occurring 1-50 times; (c) Distribution of adverse event occurrences (\#Occurrence) in negative samples, displaying a normal-like distribution centered around 20-30 occurrences. The vertical dashed lines in (a) and (b) mark the intervals of 2-8 \#Drug/Record and 5-50 \#Occurrence, respectively. These intervals were selected as the filtering criteria for our evaluation set due to their high record counts and the most representative higher-order relationships.}
  \label{fig:overall_statistics_drug_se}
\end{figure*}

% \paragraph{Drug-Adverse Event Distributions.} % 原来的段落标题
% \paragraph{Comparative Analysis of Drug Counts and Adverse Event Frequencies.} % 候选标题1,太长了
% \paragraph{Distribution Analysis of Drugs and Adverse Events in HODDI.} % 候选标题2,太长了
\paragraph{Drug and Adverse Event Distribution.}
% \Ren{check the title of the paragraph}
We performed a distribution analysis of the medication records and the adverse event frequencies in the HODDI dataset (2014Q3-2024Q3), with results shown in Figure~\ref{fig:overall_statistics_drug_se}.
The drug count per record exhibits identical distribution for positive and negative samples, whereas the side effect occurrences show distinct distributions between positive and negative samples, as contrasted in Figure \ref{fig:overall_statistics_drug_se}(b) and (c).
This distributional distinction arises from our negative sample generation procedure, which employs random resampling of side effects for negative sample generation.
The drug counts per record in both positive and negative samples show left-skewed distributions with a typical long tail, as shown in Figure \ref{fig:overall_statistics_drug_se}(a). Most records (82\%) contain 2-10 drugs, with very few records (< 1\%) containing more than 30 drugs (maximum: 100).
For side effect occurrences, positive samples show a long-tail distribution spanning from 1 to 5000, with the majority (63.3\%) occurring 1-10 times, as shown in Figure \ref{fig:overall_statistics_drug_se}(b). In contrast, negative samples have a more concentrated and near-normal distribution, with 68.7\% of side effects occurring for 21-30 times within a narrower range of 11-50, as shown in Figure \ref{fig:overall_statistics_drug_se}(c).
Stratified statistics of the medication records and the adverse event frequencies are detailed in Appendix~\ref{Appendix_A1}.

% 表3,每一个季度,condition 1/2/3, cosine similarity 0.8/0.9
\paragraph{Drug Interaction Trends.}
Monitoring the temporal trends in suspected drug interactions and validating adverse effect recommendations is essential for pharmacovigilance, enabling tracking of reporting patterns, identification of safety signals, and evaluation of standardization methodologies~\cite{ji2016functional, noren2010temporal}.
To gain these insights, we conducted a longitudinal analysis of the HODDI dataset from 2014Q3 to 2024Q3, examining temporal trends in drug interaction records and the reliability of side effect ULMS CUI recommendations.
% Table~\ref{combined_stats_all_quarters} presents 
Figure~\ref{quarterly_trend} illustrates the quarterly distribution of records across three drug role conditions and side effect cosine similarity ranges between the actual and recommended standardized side effect embeddings.
The raw data is presented in Table~\ref{combined_stats_all_quarters}.
Over the 41 quarters analyzed, the HODDI dataset contains a total of 92,064 drug interaction records, with a mean of 2,246 records per quarter with a standard deviation of 1,412. Condition 3 was the most prevalent drug condition overall, accounting for 46\% of total records, followed by Condition 2 (41\%) and Condition 1 (13\%). However, the relative proportions of the conditions shifted substantially in late 2021.
Prior to 2021Q4, Conditions 2 and 3 were dominant, each averaging approximately 1,500 records per quarter. Starting in 2021Q4, Condition 1 became the most prevalent, consistently recording between 800-1,000 cases per quarter, while Conditions 2 and 3 decreased to around 300-400 records quarterly. This marked shift suggests evolving patterns in drug interaction reporting over time.
The cosine similarity analysis revealed generally high reliability in side effect recommendations, with 90\% of the 470,144 total CUI matches having similarities greater than 0.9. However, we also note that the proportion of low-confidence matches (similarity < 0.8) gradually increased from around 3-4\% in early periods to 8-10\% in recent quarters. This trend points to growing complexity in standardizing side effect terminology.
Our findings highlight the typical temporal trends in the HODDI dataset, with notable changes in the distribution of drug role conditions and a slight decline in recommendation reliability over the study period. Further research is warranted to understand the factors driving these shifts and their implications for drug interaction surveillance.
% We examined the temporal trends in the suspected drug interactions and the reliability of standardized side effect recommendations from 2014Q3 to 2024Q3~\cite{tanaka2024onsides}. Table~\ref{combined_stats_all_quarters} reveals distinct trends across three conditions for drug role classifications, with notable shifts occurring in late 2021. Prior to 2021Q4, Conditions 2 and 3 dominated, averaging around 1,500 records quarterly. However, post-2021Q4, Condition 1 became prevalent, consistently showing 800-1,000 records per quarter while Conditions 2 and 3 decreased to 300-400 records.
% The cosine similarity analysis between actual and recommended standardized side effect embeddings shows high confidence in CUI recommendations~\cite{lim2022sapbert}, with the majority (>80\%) having high cosine similarities above 0.9. Throughout all quarters under investigation, low-confidence matches (similarity <0.8) gradually increased from around 200 to 700-1,000 cases quarterly, suggesting growing complexity in side effect standardization.
\begin{figure}[h]
\setlength{\abovecaptionskip}{0pt}
\setlength{\belowcaptionskip}{-19pt}
    \centering
    \includegraphics[width=0.9\linewidth, trim=8 5 5 5, clip]{images/stacked_bar_graphs.pdf}
    
    \vspace{2mm}
    \caption{Temporal trends of the record number of drug conditions (1-3) and cosine similarity ranges (< 0.8, 0.8-0.9, > 0.9) in the HODDI dataset from 2014Q3 to 2024Q3. The horizontal axis shows the number of records (× 10³), with drug conditions displayed on the left and cosine similarity ranges on the right. The vertical axis represents the quarterly time periods of medication records ranging from 2014Q3 to 2024Q3. The stacked bars demonstrate the distribution of records across different categories over quarterly periods.}
    % \label{robustness_main}
    % \label{prune_fine_tune_fig}
    \label{quarterly_trend}
\end{figure}



% 表4,2024Q3 作为quarterly example:
% 在每一条记录中,药物总数/role1药物总数/role2药物总数/role3药物总数的最大值/最小值/中位数等
% 展示了 药物总数/role1药物总数/role2药物总数/role3药物总数 的变化幅度和基本分布
% \paragraph{Drug Distribution Variability.}
% Table~\ref{example_2024Q3_statistics} provides a detailed distribution analysis of drugs per record for 2024Q3, revealing substantial heterogeneity across different drug roles~\cite{tekin2018prevalence}. The maximum number of drugs varies considerably among roles, with Role 2 showing the highest (200), followed by Role 1 (187), and Role 3 showing notably fewer (77). The maximum of 203 drugs per record, summed across all drug roles, indicates instances of extensive polypharmacy. However, the central tendency measures suggest that most records contain relatively few drugs (median=1.0, mean=3.23). Examining the drug role-specific distributions, we can conclude that Role 1 shows the highest mean (1.70) and consistent presence (25th-75th percentile: 1-1) among all medication records, while Role 2 exhibits moderate usage (mean=1.51, 25th-75th percentile: 0-1), and Role 3 appears rarely (mean=0.02, median=0). The general interquartile range (1.0-3.0) indicates a left-skewed distribution, with most records involving minimal drug combinations but few extreme cases of extensive polypharmacy. \Ren{what can we get from this example? Are the conclusions applicable for all the 41Q?}

\paragraph{Drug Distribution Variability Across Quarters.}
We analyzed the drug distribution patterns across 41 quarters (2014Q1-2024Q1). Taking 2024Q3 as an illustrative example, we observed consistent patterns in drug role distributions, as shown in Table~\ref{example_2024Q3_statistics}. Typically, Role 1 medications show the highest presence (mean=1.70 in 2024Q3) and consistent usage patterns (25th-75th percentile: 1-1), while Role 2 exhibits moderate usage (mean=1.51), and Role 3 appears rarely (mean=0.02). The maximum number of drugs per record varies between quarters but consistently shows Role 2 having the highest counts, followed by Role 1, with Role 3 showing notably fewer drugs. While specific values fluctuate across quarters, the general pattern remains: most records contain few drugs (median typically around 1.0) with rare cases of extensive polypharmacy, resulting in a consistent left-skewed distribution pattern~\cite{tekin2018prevalence}.


\renewcommand{\arraystretch}{1.1}
\begin{table}[h]
    \centering
    \caption{An example of quarterly statistics of drug count per record in 2024Q3 HODDI dataset, showing the number of co-administered drug count per record, where Role 1/2/3 indicate different drug roles, and Total represents the sum of drugs across all drug roles (1, 2, and 3) within a single record. This table illustrates the distribution of drugs across different roles and the total drug count within each record.}
    \resizebox{0.4\textwidth}{!}{
    \begin{tabular}{@{}ccccc@{}}
        \toprule
        \rowcolor{gray!20}
        \textbf{\large Metric} & \textbf{\large Role 1} & \textbf{\large Role 2} & \textbf{\large Role 3} & \textbf{\large Total} \\ 
        \midrule
        \textbf{Maximum} & 187 & 200 & 77 & 203 \\
        \textbf{Minimum} & 0 & 0 & 0 & 1 \\
        \textbf{Mean} & 1.70 & 1.51 & 0.02 & 3.23 \\
        \textbf{Median} & 1 & 0 & 0 & 1 \\
        \textbf{25\textsuperscript{th} Percentile} & 1 & 0 & 0 & 1 \\
        \textbf{75\textsuperscript{th} Percentile} & 1 & 1 & 0 & 3 \\
        \bottomrule
    \end{tabular}}
    \label{example_2024Q3_statistics}
\end{table}
\renewcommand{\arraystretch}{1}


\section{Evaluation Subsets Generation}\label{sec:subsets}

To facilitate efficient model evaluation, we constructed three evaluation subsets from the HODDI dataset using the following criteria (excluding low-frequency cases), with their statistical details summarized in Table ~\ref{dataset_statistics_for_3_evaluation_subsets}:

\begin{itemize}[leftmargin=*]
\vspace{-2mm}
    \item \textbf{Evaluation Subset 1:} 21,503 sample pairs where each record contains 2-8 drugs, and each side effect appears for 5-50 times;

    \item \textbf{Evaluation Subset 2:} 24,777 sample pairs where each record contains 2-16 drugs, and each side effect appears for 5-50 times;

    \item \textbf{Evaluation Subset 3:} 38,430 sample pairs where each record contains 2-16 drugs, and each side effect appears for 5-100 times.
\end{itemize}
\vspace{-2mm}
For simplicity, we selected Evaluation Subset 1 as our {\textbf{evaluation set}} for HODDI dataset utility evaluation across graph- and non-graph-based models. One example record from the evaluation set is illustrated in Table~\ref{tab:dataset_example}.
This table presents a detailed example of a single record from the evaluation set, demonstrating the structured format used for capturing adverse drug event information, including unique identifiers for reports and drugs, temporal information, and the binary classification label for potential drug interactions.




% Evaluation Set 举例:其中的 1 条记录

% Version 1.1:表头在第1列
% 为了与 Table 6 一致,删掉了次要信息:最后2列

\renewcommand{\arraystretch}{1.5} 
\begin{table}[h]
    \centering
    \caption{An example record from evaluation set, showing key attributes including the report identifier (report\_id), the high-confidence UMLS CUI with its cosine similarity above 0.9 (SE\_above\_0.9), the standardized DrugBank identifiers (DrugBankID), the hyperedge label (hyperedge\_label), and the temporal information (time).}
    \label{tab:dataset_example}  
    \resizebox{0.48\textwidth}{!}{
    \begin{tabular}{>{\cellcolor{gray!10}\centering\bfseries}p{1.3cm} >{\centering}p{1.3cm} >{\centering}p{1.3cm} >{\centering}p{1.5cm} >{\centering}p{3cm} >{\centering\arraybackslash}p{0.8cm}}
    \toprule
    \textbf{Attribute} & \textbf{report\_id} & \textbf{SE > 0.9} & \textbf{DrugBankID} & \textbf{hyperedge\_label} & \textbf{time} \\
    \midrule
    \textbf{Value} & 10367822 & C0438167 & \makecell{DB00582 \\ DB08901} & 1 & 2014Q3 \\
    \bottomrule
    \end{tabular}}
\end{table}
\renewcommand{\arraystretch}{1}

% Version 1:表头在第1列 
% 问题:双栏,字太小。单栏,也不好看。
% \renewcommand{\arraystretch}{1.5} 
% \begin{table}[h]
%    \centering
%    \caption{An example record from evaluation set, showing key attributes including the report identifier (report\_id), the high-confidence UMLS CUI with its cosine similarity above 0.9 (SE\_above\_0.9), the standardized DrugBank identifiers (DrugBankID), the hyperedge label (hyperedge\_label), the temporal information (time), the number of co-administered drugs (DrugBankID\_list\_length), and the row index in the merged dataset spanning 2014Q3-2024Q3 (row\_index).}
%    \label{tab:dataset_example}  
%    \resizebox{0.495\textwidth}{!}{
%    \begin{tabular}{>{\cellcolor{gray!10}\centering\bfseries}p{2cm} >{\centering}p{2cm} >{\centering}p{2cm} >{\centering}p{2cm} >{\centering}p{2cm} >{\centering}p{2cm} >{\centering}p{2cm} >{\centering\arraybackslash}p{2cm}}
%    \toprule
%    \textbf{Attribute} & \textbf{report\_id} & \textbf{SE\_above\_0.9} & \textbf{DrugBankID} & \textbf{hyperedge\_label} & \textbf{time} & \textbf{\makecell{DrugBankID\\\_list\_length}} & \textbf{row\_index} \\
%    \midrule
%    \textbf{Value} & 10367822 & C0438167 & \makecell{'DB00582', \\ 'DB08901'} & 1 & 2014Q3 & 2 & 958 \\
%    \bottomrule
%    \end{tabular}}
% \end{table}
% \renewcommand{\arraystretch}{1} 



% Version 2:表头在第1行
% \begin{table}[h]
%     \centering
%     \caption{An example record from evaluation set, showing key attributes including the report identifier (report\_id), the high-confidence UMLS CUI with its cosine similarity above 0.9 (SE\_above\_0.9), the standardized DrugBank identifiers (DrugBankID), the hyperedge label (hyperedge\_label), the temporal information (time), the number of co-administered drugs (DrugBankID\_list\_length), and the row index in the merged dataset spanning 2014Q3-2024Q3 (row\_index).}
%     \label{tab:dataset_example} 
%     \resizebox{0.45\textwidth}{!}{
%     \begin{tabular}{cc}
%     \toprule
%     \rowcolor{gray!20}
%     \textbf{Attribute} & \textbf{Value} \\
%     \midrule
%     report\_id & 10367822 \\
%     SE\_above\_0.9 & C0438167 \\
%     DrugBankID & ['DB00582', 'DB08901'] \\
%     hyperedge\_label & 1 \\
%     time & 2014Q3 \\
%     DrugBankID\_list\_length & 2 \\
%     row\_index & 958 \\
%     \bottomrule
%     \end{tabular}}
% \end{table}

\begin{table}[t] 
   \centering
    \caption{HODDI dataset statistics across three evaluation subsets, showing the number of records (\#Records), unique standardized drug names per record in positive/negative samples (\#Drugs+/-), and unique CUIs in positive/negative samples (\#CUIs+/-) }
   \setlength{\tabcolsep}{4pt}
   \centering
   \resizebox{0.47\textwidth}{!}{
   \begin{tabular}{@{}cccccc@{}}
       \toprule
       \rowcolor{gray!20}
       \textbf{\large Time} & \textbf{\large \#Records} & \textbf{\large \#Drugs+} & \textbf{\large \#Drugs-} & \textbf{\large \#CUIs+} & \textbf{\large \#CUIs-} \\ 
       \midrule
       \textbf{Evaluation Subset 1} & 21,503 & 1,779 & 10,460 & 1,420 & 4,540 \\
       \textbf{Evaluation Subset 2} & 24,777 & 2,050 & 10,917 & 1,580 & 4,566 \\
       \textbf{Evaluation Subset 3} & 38,430 & 2,156 & 11,835 & 1,778 & 4,579 \\
       % 2014Q3-2024Q3 & 109,744 & 2,506 & 12,293 & 3,950 & 4,581 \\
       \bottomrule
   \end{tabular}
   }
   \vspace{2mm} 

   \label{dataset_statistics_for_3_evaluation_subsets}
   % \vspace{2mm}
\end{table}


\paragraph{Evaluation Set Conversion.}

To evaluate the HODDI dataset's  applicability to GNN-based models, we implemented the clique expansion method for the evaluation set conversion, due to its superior ability to capture the drug interaction patterns, despite higher computational demands \cite{klamt2009hypergraphs, zhou2006learning}. During evaluation set conversion, each hyperedge was converted into a complete subgraph where every pair of nodes within the hyperedge is connected by an edge, inheriting the hyperedge's properties (report ID, side effect names with recommended ULMS CUIs, and the record time). 
An example record from the \textbf{converted evaluation set} is illustrated in Table~\ref{tab:dataset_example2}.
This example record illustrates the conversion outcome where two drugs (identified as DB00582 and DB08901) form a binary interaction edge, documented in 2014Q3 with a report ID of 10367822. This record exemplifies the preservation of essential hyperedge attributes during their transformation into pairwise drug connections.
A comparative analysis of Tables~\ref{tab:dataset_example} and ~\ref{tab:dataset_example2} demonstrates the transformation process from the original evaluation set to its converted form, revealing that the evaluation set conversion reduces data complexity while sacrificing higher-order drug interaction information.

% Converted Evaluation Set 举例: 其中的 1 条记录
\begin{table}[h]
    \centering
    \caption{An example record from the converted evaluation set, showing key attributes including the report identifier (report\_id), the connected drugs in graph (source and target), edge label (edge\_label), and the temporal information (time).}
    \label{tab:dataset_example2}  
    \resizebox{0.48\textwidth}{!}{
    \begin{tabular}{>{\cellcolor{gray!10}\centering\bfseries}p{1.5cm} >{\centering}p{1.5cm} >{\centering}p{1.5cm} >{\centering}p{1.5cm} >{\centering}p{1.5cm} >{\centering\arraybackslash}p{1.5cm}}
    \toprule
    \textbf{Attribute} & \textbf{report\_id} & \textbf{source} & \textbf{target} & \textbf{edge\_label} & \textbf{time} \\
    \midrule
    \textbf{Value} & 10367822 & DB00582 & DB08901 & 1 & 2014Q3 \\
    \bottomrule
    \end{tabular}}
\end{table}

\vspace{2mm}

\section{Benchmark Methods}
\label{Benchmark Methods}

To validate the applicability and generalizability of the HODDI dataset, we tested the evaluation set (comprising 21,503 positive and negative samples each) across different representative data-driven deep-learning architectures. Three types of models were employed for this validation:

\begin{enumerate}

    \item \textbf{Multi-Layer Perceptron (MLP)}. As a fundamental neural network architecture, MLP serves as an important baseline for comparison, typical of its simple structure capable of processing flattened input features and easy implementation.
    
    \item \textbf{Traditional Graph Neural Networks (GCN, GAT)}. While limited to pairwise relationships, these models provide effective baseline performance for capturing drug-drug interactions. GCN leverages spectral graph convolutions to aggregate neighborhood information \cite{kipf2016semi}, while GAT employs attention mechanisms to weigh the importance of different node connections \cite{velivckovic2017graph}.



    \item \textbf{Hypergraph Architectures.} The hypergraph architectures excel at capturing higher-order relationships in drug and side effect interactions through the hypergraph structure \cite{10163497}. In this work, we first leverage the HyGNN model, which is a deep learning framework designed to model complex, high-order relationships in data \cite{saifuddin2023hygnn}. In addition, we introduce the \underline{H}yper\underline{g}raph \underline{N}eural \underline{N}etwork with \underline{S}MILES Embedding Features and \underline{A}ttention Mechanism (HGNN-SA), a novel architecture specifically crafted to capture high-order relationships in chemical data efficiently. By combining hypergraph convolution, multi-head attention mechanisms, and molecular embeddings derived from SMILES representations, our approach significantly enhances feature representation and boosts predictive accuracy. 
    

\end{enumerate}

This diverse model selection enables a comprehensive evaluation of how different architectures handle the complex, higher-order relationships present in drug-drug interactions in the HODDI dataset.


\paragraph{Feature Construction.}
For \textit{the features of drugs}, we encoded the SMILES strings of drugs into 768-dimensional embedding vectors using a pretrained SMILES2Vec model \cite{goh2017smiles2vec}. 
% For drugs with missing SMILES, we employed zero vectors as node embeddings to prevent noise introduction, allowing these embeddings to develop meaningful semantic information through information propagation (\jw{makes no sense: either remove these drugs or find their corresponding SMILS; }). 
Drugs with missing SMILES were excluded from our analysis to ensure data quality and consistency in molecular representation.
For \textit{the features of side effects}, we utilized the SapBERT pretrained model \cite{lim2022sapbert} to encode side effect descriptions into vectors of the same dimensionality.

\paragraph{Data Structures for Benchmarks.}
For \textit{multi-layer perceptron model}, the average drug feature is concatenated with the feature vector of the corresponding side effect of each record. Each concatenated vector with 1536 dimensions forms the input data for the MLP model.
For \textit{traditional graph neural networks}, we constructed heterogeneous graphs consisting of two types of edges: drug-drug interaction edges between drug nodes, and drug-side effect causal relationship edges connecting drug nodes to side effect nodes.
For \textit{hypergraph neural networks}, instead of building pairwise edges, we constructed hyperedges where each hyperedge connects multiple drug nodes that are associated with the same side effect, effectively capturing the drug co-occurrence relationships. Each hyperedge is labeled with its corresponding side effect, maintaining the semantic connection between drug combinations and their side effect.


% \begin{figure}[t]
%     \centering        
%     \includegraphics[width=1\linewidth, trim=2 6 4 2, clip]{images/model_comparison.png}
%     \vspace{-4mm}
%     \caption{Performance comparison of different models on \textcolor{red}{(a) validation set (should be deleted)} and (b) test set. The bar plot shows mean values with error bars indicating standard deviations for four metrics: Precision, F1, AUC, and PRAUC.}
%     \label{fig_model_comparison}
%     \vspace{-4mm}
% \end{figure}



\section{Results and Analysis}\label{sec:results}


\begin{table}[h]
    \centering
    \caption{Model performance metrics on the evaluation set from the HODDI dataset. Values in parentheses denote the standard deviations. \textbf{Bold} represents the best value in each column. Even basic architectures such as MLP can deliver robust performance when leveraging higher-order features from our dataset, surpassing more complex models like GAT. The hypergraph architectures, such as HyGNN \cite{saifuddin2023hygnn} and the HGNN-SA we designed, take this a step further by enhancing prediction accuracy through its ability to better model intricate multi-drug relationships.}

    \resizebox{0.48\textwidth}{!}{ 
    \begin{tabular}{@{}ccccc@{}}
        \toprule
        \rowcolor{gray!20}
        \textbf{\large Model} & \textbf{\large Precision} & \textbf{\large F1} & \textbf{\large AUC} & \textbf{\large PRAUC} \\
        \midrule
        \bf{HGNN-SA} & \bf{0.906} \textcolor{blue}{(0.002)} & \bf{0.933} \textcolor{blue}{(0.001)} & \bf{0.957} \textcolor{blue}{(0.003)} & \bf{0.939} \textcolor{blue}{(0.008)} \\
        HyGNN & 0.903 \textcolor{blue}{(0.004)} & 0.932 \textcolor{blue}{(0.002)} & 0.954 \textcolor{blue}{(0.002)} & 0.935 \textcolor{blue}{(0.005)} \\
        GCN & 0.745 \textcolor{blue}{(0.016)} & 0.778 \textcolor{blue}{(0.011)} & 0.829 \textcolor{blue}{(0.009)} & 0.805 \textcolor{blue}{(0.008)} \\
        GAT & 0.743 \textcolor{blue}{(0.033)} & 0.809 \textcolor{blue}{(0.013)} & 0.851 \textcolor{blue}{(0.029)} & 0.789 \textcolor{blue}{(0.046)} \\
        MLP & 0.805 \textcolor{blue}{(0.013)} & 0.819 \textcolor{blue}{(0.010)} & 0.897 \textcolor{blue}{(0.005)} & 0.872 \textcolor{blue}{(0.007)} \\
        \bottomrule
    \end{tabular}
    }
    \label{tab:test_results}
\end{table}

We conducted a comprehensive evaluation on our evaluation set  across different architectural paradigms. 
We randomly selected 29, 6, and 6 quarters as the training, validation, and test data, respectively, with an approximate ratio of 70:15:15.
Detailed experimental configurations, including hyperparameter settings, training protocols, and hardware specifications, are provided in Appendix \ref{Appendix_Experimental Setup}. 
The comparative results of the model performance are illustrated in Table~\ref{tab:test_results}.
Multiple evaluation metrics were employed to assess our model performance, including Precision, F1 score, Area Under the Receiver Operating Characteristic curve (AUC), and Area Under the Precision-Recall curve (PRAUC).

% We use Precision, F1, AUC, and PRAUC as our evaluation metrics.

Compared with GCN and GAT, MLP achieves higher overall performance across all metrics, with a precision of 80.5\%, an F1 score of 81.9\%, an AUC of 89.7\%, and a PRAUC of 87.2\%. This suggests that even a simple feedforward architecture can effectively leverage the higher-order feature representations provided by HODDI, highlighting the dataset's well-structured and informative nature. Among the GNN models, GAT and GCN demonstrate comparable performance. While GAT achieves a higher F1 score (80.9\% vs. 77.8\%) and AUC (85.1\% vs. 82.9\%) compared to GCN, the latter performs better in terms of F1 score and PRAUC. That means that HODDI is well-suited for both the MLP model and GNN models.

By leveraging hypergraph architectures, HyGNN and HGNN-SA, which effectively capture higher-order relationships in drug-side effect interactions, the results can be significantly improved. The HGNN-SA outperforms all other approaches, achieving a precision of 90.6\%, an F1 score of 93.3\%, an AUC of 95.7\%, and a PRAUC of 93.9\%. 
These results demonstrate that HGNN-based models which explicitly integrate hypergraph structures can effectively leverage the higher-order drug-drug interaction relationships captured in our HODDI dataset.

% significantly improve prediction accuracy by effectively capturing intricate multi-drug relationships.

Additionally, the low standard deviations across all models (ranging from 0.1\% to 4.6\%) indicate robust and stable learning behavior, further validating the dataset's reliability for capturing the interaction between drugs and side effects. The strong performance of both GNN models and the MLP model suggests that HODDI provides rich and high-quality feature representations, making it a valuable resource for advancing machine learning approaches in drug-drug interaction studies.



\section{Conclusions}
In this paper, we introduced HODDI, the first comprehensive dataset capturing higher-order drug-drug interactions. Our HODDI dataset comprises 109,744 records with 2,506 unique drugs and 4,569 unique side effects. Through extensive evaluation, we demonstrated that higher-order features significantly enhance prediction performance, with hypergraph-based model outperforming traditional approaches by effectively capturing multi-drug relationships. These findings underscore the importance of higher-order data in advancing pharmacovigilance, drug safety, and personalized medicine. 
Future research opportunities based on HODDI include the development of advanced heterogeneous hypergraph neural networks that leverage attention mechanisms and novel message-passing schemes, the construction of dynamic hypergraph datasets with temporal information, and the integration of large language models with multimodal data. Our work reveals that HODDI establishes a robust foundation for investigating polypharmacy adverse effects, contributing to safer and more effective pharmacological interventions.

\section*{Impact Statement}
This paper introduces a benchmark dataset for higher-order drug-drug interactions, addressing a critical need in healthcare and pharmaceutical research. The dataset enables systematic study of complex drug interactions and their side effects, potentially improving drug safety assessment and patient care. This data resource supports application-driven machine learning research in pharmacology while facilitating development of advanced models for predicting adverse drug reactions.


\bibliography{example_paper}
\bibliographystyle{icml2025}



\clearpage


\appendix
\onecolumn
% \maketitle
\section{Appendix}
% \begin{tcolorbox}
\subsection{HODDI Dataset Construction: Processing Details}
\label{Appendix_A0-1}
\FloatBarrier

\label{HODDI_construction_algo}
\begin{itemize}
    \item \textbf{Data Collection and Preprocessing}
    \begin{itemize}
        \item Download and extract XML files from FAERS quarterly datasets (2014Q3-2024Q3)
        \item Extract key components from FAERS records: report ID, drug information (standardized drug name and drug role), and side effect
        \item Filter records based on \textit{Drug Role} counts:
           \begin{itemize}
               \item Condition 1: Count (\textit{Drug Role 3}) $\geq$ 2, Count (\textit{Drug Role1}) = 0, Count (\textit{Drug Role 2}) = 0
               \item Condition 2: Count (\textit{Drug Role 1}) $\geq$ 1, Count (\textit{Drug Role 3}) $\geq$ 1, Count (\textit{Drug Role 2}) = 0
               \item Condition 3: Count (\textit{Drug Role 1}) $\geq$ 1, Count (\textit{Drug Role 2}) $\geq$ 1, Count (\textit{Drug Role 3}) $\geq$ 1
           \end{itemize}
    \end{itemize}

    \item \textbf{Standardized Drug Names Processing}
       \begin{itemize}
           \item Normalization of Standardized drug names:
               \begin{itemize}
                   \item Convert to uppercase
                   \item Remove salt form suffixes
                   \item Handle compound names
               \end{itemize}
           \item Map normalized drug names to DrugBank IDs
    \end{itemize}
    
    \item \textbf{Side Effects Processing}
       \begin{itemize}
           \item Generate 768-dimensional embeddings for FAERS side effects and MedDRA terms using SapBERT
           \item Calculate cosine similarities between SapBERT-based embeddings of FAERS side effects and MedDRA terms
           \item Map FAERS side effects to recommended MedDRA terms by highest similarity
           \item Query recommended UMLS CUIs from UMLS Metathesaurus using MedDRA terms
           \item Stratify recommended UMLS CUIs by the cosine similarity threshold (e.g., 0.9)
       \end{itemize}


    \item \textbf{Dataset Construction}
    \begin{itemize}
        \item Generate positive samples:
            \begin{itemize}
                \item Select records based on the predefined cosine similarity threshold
                \item Remove duplicate records and drug combination supersets
            \end{itemize}
        \item Generate negative samples:
            \begin{itemize}
                \item For each positive sample:
                    \begin{itemize}
                        \item Randomly replace one drug and one side effect from their respective complement sets
                        \item Verify that the generated negative sample is absent in all positive samples
                    \end{itemize}
            \end{itemize}
        \item Create evaluation subsets for benchmark model evaluation:
           \begin{itemize}
               \item Evaluation Subset 1: A set of records where each record contains 2-8 drugs and their associated side effects occur with a frequency of 5-50 times across all records
               \item Evaluation Subset 2: A set of records where each record contains 2-16 drugs and their associated side effects occur with a frequency of 5-50 times across all records
               \item Evaluation Subset 3: A set of records where each record contains 2-16 drugs and their associated side effects occur with a frequency of 5-100 times across all records
           \end{itemize}
    \end{itemize}
\end{itemize}
% \end{tcolorbox}
\clearpage



%%%%%%%%%%%%%%%%%%%%%%%%%%%%%%%%



\clearpage
\subsection{HODDI Dataset Structure}
\label{Appendix_A0-2}


\vspace{2mm}

Table~\ref{hgnn_dataset_structure} outlines the evaluation dataset structure, which intrinsically captures the higher-order drug-drug interactions relationships. Each record contains a FAERS report identifier, a recommended UMLS CUI code for adverse effect (with cosine similarity $\geq$ 0.9), and a list of DrugBank identifiers representing co-administered drugs. The hyperedge label (-1/1) indicates whether the drug combination leads to the specified side effect. Additional metadata includes the record's temporal information and the number of drugs involved in each record.

\vspace{2mm}

Table~\ref{gnn_dataset_structure} outlines the converted evaluation dataset structure, which examines the pairwise drug interactions. Each record consists of a FAERS report identifier, a source and target drug pair represented by their DrugBank IDs, and a recommended UMLS CUI representing the observed side effect caused by the drug pair. The edge label (-1/1) denotes whether this drug pair causes the specified adverse effect.

\vspace{2mm}

% 修改前:
\renewcommand{\arraystretch}{1.2} 
\begin{table}[!ht]
   \centering
   \small 
   \caption{Data structure of the evaluation set from HODDI dataset.}
   \resizebox{0.95\textwidth}{!}{
   \begin{tabular}{@{}lll@{}}
   % \begin{tabular}{@{}ccc@{}} 
       \toprule
       \rowcolor{gray!20}
       \multicolumn{1}{c}{\textbf{Variable}} & \multicolumn{1}{c}{\textbf{Column Name}} & \multicolumn{1}{c}{\textbf{Description}} \\ 
       \midrule
       Report ID & report\_id & FAERS report identifier (negative samples end with "n") \\
       Recommend UMLS CUI & SE\_above\_0.9 & High-confidence UMLS CUIs identified using a cosine similarity threshold of 0.9 \\
       DrugBank ID & DrugBankID & List of standardized DrugBank identifiers \\
       Hyperedge Label & hyperedge\_label & Binary label for hyperedge existence (1: positive samples; -1: negative samples) \\
       Record Time & time & Year and quarter of the record (e.g., 2014Q3) \\
       Drug Count & DrugBankID\_list\_length & Number of co-administered drugs \\
       Row Index & row\_index & Row index in merged dataset (2014Q3-2024Q3) \\
       \bottomrule
   \end{tabular}}
   \label{hgnn_dataset_structure}
\end{table}
\renewcommand{\arraystretch}{1} 
\vspace{2mm}


% \renewcommand{\arraystretch}{1.2} 
% \begin{table}[!ht]
%    \centering
%    \caption{Data structure of the converted evaluation set from HODDI dataset.}
%    \resizebox{0.95\textwidth}{!}{
%    \begin{tabular}{@{}lll@{}}
%        \toprule
%        \rowcolor{gray!20}
%        \multicolumn{1}{c}{\textbf{\footnotesize Variable}} & \multicolumn{1}{c}{\textbf{\footnotesize Column Name}} & \multicolumn{1}{c}{\textbf{\footnotesize Description}} \\ 
%        \midrule
%        \footnotesize Report ID & \footnotesize report\_id & \footnotesize FAERS report identifier (negative samples end with "n") \\
%        \footnotesize Source Drug & \footnotesize source & \footnotesize Drug node in undirected graph \\
%        \footnotesize Target Drug & target & \footnotesize Drug node in undirected graph \\
%        \footnotesize Side Effect CUI & \footnotesize SE\_label & \footnotesize High-confidence UMLS CUIs identified using a cosine similarity threshold of 0.9 \\
%        \footnotesize Edge Label & \footnotesize edge\_label & \footnotesize Binary label for edge existence (1: positive samples; -1: negative samples) \\
%        \bottomrule
%    \end{tabular}}
%    \label{gnn_dataset_structure}
% \end{table}
% \renewcommand{\arraystretch}{1} 

% 修改后:

% \renewcommand{\arraystretch}{1.2} 
% \begin{table}[!ht]
%    \centering
%    \small 
%    \caption{Data structure of the evaluation set from HODDI dataset.}
%    \begin{tabularx}{0.95\textwidth}{@{}p{3.1cm}p{3.1cm}X@{}} % 使用 tabularx,列宽自动调整
%        \toprule
%        \rowcolor{gray!20}
%        \multicolumn{1}{c}{\textbf{Variable}} & \multicolumn{1}{c}{\textbf{Column Name}} & \multicolumn{1}{c}{\textbf{Description}} \\ 
%        \midrule
%        Report ID & report\_id & FAERS report identifier (negative samples end with "n") \\
%        Recommend UMLS CUI & SE\_above\_0.9 & High-confidence UMLS CUIs identified using a cosine similarity threshold of 0.9 \\
%        DrugBank ID & DrugBankID & List of standardized DrugBank identifiers \\
%        Hyperedge Label & hyperedge\_label & Binary label for hyperedge existence (1: positive samples; -1: negative samples) \\
%        Record Time & time & Year and quarter of the record (e.g., 2014Q3) \\
%        Drug Count & DrugBankID\_list\_length & Number of co-administered drugs \\
%        Row Index & row\_index & Row index in merged dataset (2014Q3-2024Q3) \\
%        \bottomrule
%    \end{tabularx}
%    \label{hgnn_dataset_structure}
% \end{table}
% \renewcommand{\arraystretch}{1} 
% \vspace{2mm}

\renewcommand{\arraystretch}{1.2} 
\begin{table}[!ht]
   \centering
   \small 
   \caption{Data structure of the converted evaluation set from HODDI dataset.}
   \begin{tabularx}{0.95\textwidth}{@{}p{2.6cm}p{2.6cm}X@{}} % 使用 tabularx,列宽自动调整
       \toprule
       \rowcolor{gray!20}
       \multicolumn{1}{c}{\textbf{\footnotesize Variable}} & \multicolumn{1}{c}{\textbf{\footnotesize Column Name}} & \multicolumn{1}{c}{\textbf{\footnotesize Description}} \\ 
       \midrule
       Report ID & report\_id & FAERS report identifier (negative samples end with "n") \\
       Source Drug & source & Drug node in undirected graph \\
       Target Drug & target & Drug node in undirected graph \\
       Side Effect CUI & SE\_label & High-confidence UMLS CUIs identified using a cosine similarity threshold of 0.9 \\
       Edge Label & edge\_label & Binary label for edge existence (1: positive samples; -1: negative samples) \\
       \bottomrule
   \end{tabularx}
   \label{gnn_dataset_structure}
\end{table}
\renewcommand{\arraystretch}{1}

%%%%%%%%%%%%%%%%%%%%%%%%%%%%%%%%%%%%%%%%%%%%%%%%%
\clearpage
\subsection{HODDI Dataset Statistics}
% \paragraph{2014Q3-2024Q3}
\label{Appendix_A1}

In the HODDI dataset spanning from 2014Q3 to 2024Q3, we analyzed the distribution patterns of drug combination sizes and occurrence frequency of side effects.
As shown in Table~\ref{drugs_per_record}, 80.48\% of records contain between 2-10 drugs, with 54,304 (48.89\%) records containing 2-5 drugs and 35,087 (31.59\%) records containing 6-10 drugs.
The frequency decreases significantly for records with higher drug counts, with only 1,234 records (1.11\% of the total) containing no less than 30 drugs.

Table~\ref{se_positive} presents the distribution of side effect occurrences in positive samples from 2014Q3 to 2024Q3. The data shows a highly left-skewed and long-tailed distribution where 2,501 side effects (63.32\%) appear only 1-10 times, while only 3 side effects occur more than 1,000 times. The majority of side effects (82.63\%) occur less than 30 times, indicating that most side effects are relatively uncommon in the HODDI dataset.

Table~\ref{se_negative} presents the distribution of side effect occurrences in negative samples from 2014Q3 to 2024Q3. After resampling, the data exhibits a near-Gaussian distribution with a slight left skew, where 3,146 side effects (68.69\%) occur 21-30 times. The distribution is more concentrated compared to positive samples, with no occurrences below 10 times and few (15 cases, 0.33\%) above 40 times.




\begin{table}[h]
   \centering
   \caption{Distribution of drug count per record in positive and negative samples (2014Q3-2024Q3). (\#Drug/Record: drug count per record; \#Record: number of records.)}
   \resizebox{0.25\textwidth}{!}{ % 调整表格整体宽度
   \begin{tabular}{@{}>{\centering\arraybackslash}p{3cm}>{\centering\arraybackslash}p{2cm}@{}}
       \toprule
       \rowcolor{gray!20}
       \textbf{\large \#Drug/Record} & \textbf{\large \#Record} \\ 
       \midrule
       1 & 474 \\
       2-5 & 54,304 \\
       6-10 & 35,087 \\
       11-15 & 12,730 \\
       16-20 & 4,661 \\
       21-30 & 2,582 \\
       31-40 & 798 \\
       41-50 & 176 \\
       51-100 & 260 \\
       101+ & 0 \\
       \bottomrule
   \end{tabular}}
   
   \label{drugs_per_record}
\end{table}






\begin{table}[!h]
   \centering
   \caption{Distribution of side effect occurrences in positive samples (2014Q3-2024Q3). (\#Occurrence: number of side effect occurrences in positive samples; \#Side Effect: number of side effects in positive samples.)}
   \resizebox{0.25\textwidth}{!}{ % 调整表格整体宽度
   \begin{tabular}{@{}>{\centering\arraybackslash}p{3cm}>{\centering\arraybackslash}p{3cm}@{}}
       \toprule
       \rowcolor{gray!20}
       \textbf{\large  \#Occurrence} & \textbf{\large 
 \#Side Effect} \\ 
       \midrule
       1-10 & 2,501 \\
       11-20 & 535 \\
       21-30 & 228 \\
       31-40 & 126 \\
       41-50 & 91 \\
       51-100 & 221 \\
       101-200 & 137 \\
       201-500 & 85 \\
       501-1000 & 23 \\
       1001-5000 & 3 \\
       \bottomrule
   \end{tabular}}
   
   \label{se_positive}
\end{table}




\begin{table}[!h]
   \centering
   \caption{Distribution of side effect occurrences in negative samples (2014Q3-2024Q3). (\#Occurrence: number of side effect occurrences in negative samples; \#Side Effect: number of side effects in negative samples.)}
   \resizebox{0.25\textwidth}{!}{ % 调整表格整体宽度
   \begin{tabular}{@{}>{\centering\arraybackslash}p{3cm}>{\centering\arraybackslash}p{3cm}@{}}
       \toprule
       \rowcolor{gray!20}
       \textbf{\large \#Occurrence} & \textbf{\large \#Side Effect} \\ 
       \midrule
       1-10 & 0 \\
       11-20 & 773 \\
       21-30 & 3,146 \\
       31-40 & 646 \\
       41-50 & 15 \\
       \bottomrule
   \end{tabular}}
   
   \label{se_negative}
\end{table}

\clearpage
% 这个表格包含了 stacked bar graph 的原始数据
% 由于stacked bar graph已经包含了表格中的信息,所以我在正文中注释掉这个表格,并把它移动到Appendix
\begin{table*}[!ht]
    \centering
    \caption{Quarterly distribution of drug conditions and cosine similarity ranges in HODDI dataset (2014Q3-2024Q3). Black numbers show record counts for each drug condition and cosine similarity range, while blue numbers in parentheses indicate their proportions of total records. Total, Mean, and Std Dev rows present aggregate statistics for each column.}
    \resizebox{0.9\textwidth}{!}{
    \renewcommand{\arraystretch}{1.2}
    \begin{tabular}{@{}l|ccc|c|ccc|c@{}}
        \toprule
        \rowcolor{gray!20}
        & \multicolumn{4}{c|}{\textbf{\large Drug Condition}} & \multicolumn{4}{c}{\textbf{\large Cosine similarity}} \\
        \cmidrule(r){2-5} \cmidrule(l){6-9}
        \rowcolor{gray!20}
        \textbf{\large Time} & \textbf{1} & \textbf{2} & \textbf{3} & \textbf{Total} & \textbf{< 0.8} & \textbf{0.8 - 0.9} & \textbf{> 0.9} & \textbf{Total} \\
        \midrule
        \rowcolor{white}
        \textbf{2014Q3}  & 33 \textcolor{blue}{(0.02)} & 598 \textcolor{blue}{(0.44)} & 719 \textcolor{blue}{(0.53)} & 1350 & 193 \textcolor{blue}{(0.03)} & 171 \textcolor{blue}{(0.03)} & 5906 \textcolor{blue}{(0.94)} & 6270 \\
        \rowcolor{gray!8}
        \textbf{2014Q4}  & 27 \textcolor{blue}{(0.02)} & 515 \textcolor{blue}{(0.44)} & 641 \textcolor{blue}{(0.54)} & 1183 & 214 \textcolor{blue}{(0.04)} & 129 \textcolor{blue}{(0.02)} & 4847 \textcolor{blue}{(0.93)} & 5190 \\
        \rowcolor{white}
        \textbf{2015Q1}  & 30 \textcolor{blue}{(0.02)} & 561 \textcolor{blue}{(0.43)} & 718 \textcolor{blue}{(0.55)} & 1309 & 200 \textcolor{blue}{(0.03)} & 174 \textcolor{blue}{(0.03)} & 5526 \textcolor{blue}{(0.94)} & 5900 \\
        \rowcolor{gray!8}
        \textbf{2015Q2}  & 32 \textcolor{blue}{(0.02)} & 778 \textcolor{blue}{(0.45)} & 936 \textcolor{blue}{(0.54)} & 1746 & 308 \textcolor{blue}{(0.04)} & 235 \textcolor{blue}{(0.03)} & 7822 \textcolor{blue}{(0.94)} & 8365 \\
        \rowcolor{white}
        \textbf{2015Q3}  & 49 \textcolor{blue}{(0.03)} & 773 \textcolor{blue}{(0.40)} & 1092 \textcolor{blue}{(0.57)} & 1914 & 281 \textcolor{blue}{(0.03)} & 317 \textcolor{blue}{(0.04)} & 8196 \textcolor{blue}{(0.93)} & 8794 \\
        \rowcolor{gray!8}
        \textbf{2015Q4}  & 33 \textcolor{blue}{(0.02)} & 711 \textcolor{blue}{(0.38)} & 1138 \textcolor{blue}{(0.60)} & 1882 & 318 \textcolor{blue}{(0.04)} & 244 \textcolor{blue}{(0.03)} & 7963 \textcolor{blue}{(0.93)} & 8525 \\
        \rowcolor{white}
        \textbf{2016Q1}  & 39 \textcolor{blue}{(0.02)} & 717 \textcolor{blue}{(0.38)} & 1122 \textcolor{blue}{(0.60)} & 1878 & 385 \textcolor{blue}{(0.04)} & 269 \textcolor{blue}{(0.03)} & 7960 \textcolor{blue}{(0.92)} & 8614 \\
        \rowcolor{gray!8}
        \textbf{2016Q2}  & 55 \textcolor{blue}{(0.03)} & 869 \textcolor{blue}{(0.43)} & 1120 \textcolor{blue}{(0.55)} & 2044 & 437 \textcolor{blue}{(0.05)} & 300 \textcolor{blue}{(0.03)} & 8695 \textcolor{blue}{(0.92)} & 9432 \\
        \rowcolor{white}
        \textbf{2016Q3}  & 53 \textcolor{blue}{(0.02)} & 1154 \textcolor{blue}{(0.49)} & 1132 \textcolor{blue}{(0.48)} & 2339 & 432 \textcolor{blue}{(0.04)} & 310 \textcolor{blue}{(0.03)} & 9363 \textcolor{blue}{(0.93)} & 10105 \\
        \rowcolor{gray!8}
        \textbf{2016Q4}  & 40 \textcolor{blue}{(0.02)} & 830 \textcolor{blue}{(0.47)} & 879 \textcolor{blue}{(0.50)} & 1749 & 401 \textcolor{blue}{(0.05)} & 255 \textcolor{blue}{(0.03)} & 7503 \textcolor{blue}{(0.92)} & 8159 \\
        \rowcolor{white}
        \textbf{2017Q1}  & 44 \textcolor{blue}{(0.02)} & 821 \textcolor{blue}{(0.45)} & 953 \textcolor{blue}{(0.52)} & 1818 & 372 \textcolor{blue}{(0.04)} & 282 \textcolor{blue}{(0.03)} & 7642 \textcolor{blue}{(0.92)} & 8296 \\
        \rowcolor{gray!8}
        \textbf{2017Q2}  & 39 \textcolor{blue}{(0.02)} & 900 \textcolor{blue}{(0.48)} & 937 \textcolor{blue}{(0.50)} & 1876 & 404 \textcolor{blue}{(0.05)} & 299 \textcolor{blue}{(0.04)} & 7490 \textcolor{blue}{(0.91)} & 8193 \\
        \rowcolor{white}
        \textbf{2017Q3}  & 41 \textcolor{blue}{(0.02)} & 855 \textcolor{blue}{(0.44)} & 1042 \textcolor{blue}{(0.54)} & 1938 & 478 \textcolor{blue}{(0.05)} & 301 \textcolor{blue}{(0.03)} & 8659 \textcolor{blue}{(0.92)} & 9438 \\
        \rowcolor{gray!8}
        \textbf{2018Q1}  & 60 \textcolor{blue}{(0.02)} & 1280 \textcolor{blue}{(0.47)} & 1386 \textcolor{blue}{(0.51)} & 2726 & 599 \textcolor{blue}{(0.05)} & 516 \textcolor{blue}{(0.04)} & 11285 \textcolor{blue}{(0.91)} & 12400 \\
        \rowcolor{white}
        \textbf{2018Q2}  & 54 \textcolor{blue}{(0.02)} & 1643 \textcolor{blue}{(0.48)} & 1691 \textcolor{blue}{(0.50)} & 3388 & 855 \textcolor{blue}{(0.05)} & 733 \textcolor{blue}{(0.04)} & 16909 \textcolor{blue}{(0.91)} & 18497 \\
        \rowcolor{gray!8}
        \textbf{2018Q3}  & 57 \textcolor{blue}{(0.02)} & 1744 \textcolor{blue}{(0.51)} & 1641 \textcolor{blue}{(0.48)} & 3442 & 924 \textcolor{blue}{(0.05)} & 636 \textcolor{blue}{(0.04)} & 16286 \textcolor{blue}{(0.91)} & 17846 \\
        \rowcolor{white}
        \textbf{2018Q4}  & 76 \textcolor{blue}{(0.02)} & 1557 \textcolor{blue}{(0.47)} & 1654 \textcolor{blue}{(0.50)} & 3287 & 814 \textcolor{blue}{(0.05)} & 641 \textcolor{blue}{(0.04)} & 14740 \textcolor{blue}{(0.91)} & 16195 \\
        \rowcolor{gray!8}
        \textbf{2019Q1}  & 67 \textcolor{blue}{(0.02)} & 1609 \textcolor{blue}{(0.47)} & 1745 \textcolor{blue}{(0.51)} & 3421 & 845 \textcolor{blue}{(0.05)} & 731 \textcolor{blue}{(0.04)} & 15448 \textcolor{blue}{(0.91)} & 17024 \\
        \rowcolor{white}
        \textbf{2019Q2}  & 60 \textcolor{blue}{(0.02)} & 1524 \textcolor{blue}{(0.47)} & 1681 \textcolor{blue}{(0.51)} & 3265 & 878 \textcolor{blue}{(0.05)} & 679 \textcolor{blue}{(0.04)} & 14651 \textcolor{blue}{(0.90)} & 16208 \\
        \rowcolor{gray!8}
        \textbf{2019Q3}  & 71 \textcolor{blue}{(0.02)} & 1726 \textcolor{blue}{(0.49)} & 1749 \textcolor{blue}{(0.49)} & 3546 & 901 \textcolor{blue}{(0.05)} & 699 \textcolor{blue}{(0.04)} & 15704 \textcolor{blue}{(0.91)} & 17304 \\
        \rowcolor{white}
        \textbf{2019Q4}  & 72 \textcolor{blue}{(0.02)} & 1473 \textcolor{blue}{(0.46)} & 1639 \textcolor{blue}{(0.51)} & 3184 & 877 \textcolor{blue}{(0.06)} & 612 \textcolor{blue}{(0.04)} & 14311 \textcolor{blue}{(0.91)} & 15800 \\
        \rowcolor{gray!8}
        \textbf{2020Q1}  & 62 \textcolor{blue}{(0.02)} & 1586 \textcolor{blue}{(0.44)} & 1935 \textcolor{blue}{(0.54)} & 3583 & 897 \textcolor{blue}{(0.05)} & 609 \textcolor{blue}{(0.04)} & 15719 \textcolor{blue}{(0.91)} & 17225 \\
        \rowcolor{white}
        \textbf{2020Q2}  & 59 \textcolor{blue}{(0.02)} & 1570 \textcolor{blue}{(0.46)} & 1810 \textcolor{blue}{(0.53)} & 3439 & 1014 \textcolor{blue}{(0.06)} & 617 \textcolor{blue}{(0.04)} & 14650 \textcolor{blue}{(0.90)} & 16281 \\
        \rowcolor{gray!8}
        \textbf{2020Q3}  & 81 \textcolor{blue}{(0.02)} & 1630 \textcolor{blue}{(0.47)} & 1730 \textcolor{blue}{(0.50)} & 3441 & 1417 \textcolor{blue}{(0.08)} & 772 \textcolor{blue}{(0.04)} & 16243 \textcolor{blue}{(0.88)} & 18432 \\
        \rowcolor{white}
        \textbf{2020Q4}  & 59 \textcolor{blue}{(0.02)} & 1393 \textcolor{blue}{(0.50)} & 1334 \textcolor{blue}{(0.48)} & 2786 & 862 \textcolor{blue}{(0.06)} & 521 \textcolor{blue}{(0.04)} & 12734 \textcolor{blue}{(0.90)} & 14117 \\
        \rowcolor{gray!8}
        \textbf{2021Q1}  & 72 \textcolor{blue}{(0.03)} & 1407 \textcolor{blue}{(0.53)} & 1185 \textcolor{blue}{(0.44)} & 2664 & 933 \textcolor{blue}{(0.07)} & 563 \textcolor{blue}{(0.04)} & 12562 \textcolor{blue}{(0.89)} & 14058 \\
        \rowcolor{white}
        \textbf{2021Q2}  & 82 \textcolor{blue}{(0.03)} & 1455 \textcolor{blue}{(0.52)} & 1271 \textcolor{blue}{(0.45)} & 2808 & 1078 \textcolor{blue}{(0.07)} & 554 \textcolor{blue}{(0.04)} & 13083 \textcolor{blue}{(0.89)} & 14715 \\
        \rowcolor{gray!8}
        \textbf{2021Q3}  & 63 \textcolor{blue}{(0.02)} & 1289 \textcolor{blue}{(0.49)} & 1270 \textcolor{blue}{(0.48)} & 2622 & 1178 \textcolor{blue}{(0.08)} & 653 \textcolor{blue}{(0.04)} & 13341 \textcolor{blue}{(0.88)} & 15172 \\
        \rowcolor{white}
        \textbf{2021Q4}  & 801 \textcolor{blue}{(0.53)} & 322 \textcolor{blue}{(0.21)} & 377 \textcolor{blue}{(0.25)} & 1500 & 640 \textcolor{blue}{(0.07)} & 381 \textcolor{blue}{(0.04)} & 7756 \textcolor{blue}{(0.88)} & 8777 \\
        \rowcolor{gray!8}
        \textbf{2022Q1}  & 856 \textcolor{blue}{(0.51)} & 362 \textcolor{blue}{(0.22)} & 453 \textcolor{blue}{(0.27)} & 1671 & 852 \textcolor{blue}{(0.08)} & 555 \textcolor{blue}{(0.05)} & 9595 \textcolor{blue}{(0.87)} & 11002 \\
        \rowcolor{white}
        \textbf{2022Q2}  & 906 \textcolor{blue}{(0.59)} & 262 \textcolor{blue}{(0.17)} & 377 \textcolor{blue}{(0.24)} & 1545 & 828 \textcolor{blue}{(0.08)} & 494 \textcolor{blue}{(0.05)} & 8810 \textcolor{blue}{(0.87)} & 10132 \\
        \rowcolor{gray!8}
        \textbf{2022Q3}  & 906 \textcolor{blue}{(0.54)} & 328 \textcolor{blue}{(0.20)} & 443 \textcolor{blue}{(0.26)} & 1677 & 837 \textcolor{blue}{(0.08)} & 442 \textcolor{blue}{(0.04)} & 9014 \textcolor{blue}{(0.88)} & 10293 \\
        \rowcolor{white}
        \textbf{2022Q4}  & 849 \textcolor{blue}{(0.52)} & 287 \textcolor{blue}{(0.17)} & 512 \textcolor{blue}{(0.31)} & 1648 & 850 \textcolor{blue}{(0.08)} & 451 \textcolor{blue}{(0.04)} & 9537 \textcolor{blue}{(0.88)} & 10838 \\
        \rowcolor{gray!8}
        \textbf{2023Q1}  & 1007 \textcolor{blue}{(0.59)} & 333 \textcolor{blue}{(0.19)} & 374 \textcolor{blue}{(0.22)} & 1714 & 762 \textcolor{blue}{(0.08)} & 469 \textcolor{blue}{(0.05)} & 8468 \textcolor{blue}{(0.87)} & 9699 \\
        \rowcolor{white}
        \textbf{2023Q2}  & 1031 \textcolor{blue}{(0.58)} & 350 \textcolor{blue}{(0.20)} & 386 \textcolor{blue}{(0.22)} & 1767 & 770 \textcolor{blue}{(0.08)} & 398 \textcolor{blue}{(0.04)} & 8540 \textcolor{blue}{(0.88)} & 9708 \\
        \rowcolor{gray!8}
        \textbf{2023Q3}  & 954 \textcolor{blue}{(0.58)} & 354 \textcolor{blue}{(0.22)} & 331 \textcolor{blue}{(0.20)} & 1639 & 729 \textcolor{blue}{(0.08)} & 359 \textcolor{blue}{(0.04)} & 7857 \textcolor{blue}{(0.88)} & 8945 \\
        \rowcolor{white}
        \textbf{2023Q4}  & 900 \textcolor{blue}{(0.55)} & 365 \textcolor{blue}{(0.22)} & 361 \textcolor{blue}{(0.22)} & 1626 & 834 \textcolor{blue}{(0.08)} & 388 \textcolor{blue}{(0.04)} & 8631 \textcolor{blue}{(0.88)} & 9853 \\
        \rowcolor{gray!8}
        \textbf{2024Q1}  & 820 \textcolor{blue}{(0.51)} & 353 \textcolor{blue}{(0.22)} & 425 \textcolor{blue}{(0.27)} & 1598 & 787 \textcolor{blue}{(0.09)} & 406 \textcolor{blue}{(0.04)} & 7872 \textcolor{blue}{(0.87)} & 9065 \\
        \rowcolor{white}
        \textbf{2024Q2}  & 780 \textcolor{blue}{(0.51)} & 348 \textcolor{blue}{(0.23)} & 406 \textcolor{blue}{(0.26)} & 1534 & 801 \textcolor{blue}{(0.10)} & 360 \textcolor{blue}{(0.04)} & 7051 \textcolor{blue}{(0.86)} & 8212 \\
        \rowcolor{gray!8}
        \textbf{2024Q3}  & 821 \textcolor{blue}{(0.54)} & 326 \textcolor{blue}{(0.22)} & 362 \textcolor{blue}{(0.24)} & 1509 & 710 \textcolor{blue}{(0.09)} & 334 \textcolor{blue}{(0.04)} & 6992 \textcolor{blue}{(0.87)} & 8036 \\
        \midrule
        \rowcolor{gray!20}
        \textbf{Total} & 12172 \textcolor{blue}{(0.13)} & 37861 \textcolor{blue}{(0.41)} & 42031 \textcolor{blue}{(0.46)} & 92064 & 28331 \textcolor{blue}{(0.06)} & 18170 \textcolor{blue}{(0.04)} & 423643 \textcolor{blue}{(0.90)} & 470144 \\
        \rowcolor{gray!20}
        \textbf{Mean} & 296.88 \textcolor{blue}{(0.13)} & 923.44 \textcolor{blue}{(0.41)} & 1025.15 \textcolor{blue}{(0.46)} & 2246 & 6910 \textcolor{blue}{(0.39)} & 443.17 \textcolor{blue}{(0.03)} & 10332.76 \textcolor{blue}{(0.58)} & 17686 \\
        \rowcolor{gray!20}
        \textbf{Std Dev} & 386.15 \textcolor{blue}{(0.27)} & 513.34 \textcolor{blue}{(0.36)} & 512.72 \textcolor{blue}{(0.36)} & 1412 & 287.83 \textcolor{blue}{(0.05)} & 1752 \textcolor{blue}{(0.32)} & 3474.49 \textcolor{blue}{(0.63)} & 5514 \\
        \bottomrule
    \end{tabular}}
    \label{combined_stats_all_quarters}
\end{table*}

\clearpage
\subsection{Experimental Setup}
\label{Appendix_Experimental Setup}

\subsubsection{Training Details}

\paragraph{MLP.}
The drug and side effect representations are each projected from 768 to 64 dimensions through separate reduction layers. The resulting embeddings are concatenated and passed through a classifier with hidden layers of 128 and 32 dimensions, followed by a final output layer with a single neuron, using ReLU activation. The model is trained using the Adam optimizer with a learning rate of 0.001 and Binary Cross Entropy Loss. Training runs for a maximum of 500 epochs, with early stopping applied if the validation loss does not improve by at least 0.001 for 20 consecutive epochs. 

\paragraph{GCN and GAT.}
The GCN and GAT models share the same architecture, with the only difference being the use of GCN layers in the GCN model and GAT layers in the GAT model. The architecture consists of an input layer, followed by a hidden layer with 128 channels, a second layer reducing the dimensionality to 64 channels, and a final classifier layer with 2 output neurons. Both models are trained with a learning rate of 0.0005, weight decay of 0.001, and use AdamW as the optimizer. The loss function is Cross Entropy Loss with class weights, and early stopping is applied if the validation loss does not improve by 0.001 for 20 consecutive epochs. A learning rate scheduler, ReduceLROnPlateau, reduces the learning rate by a factor of 0.1 if the validation loss plateaus for 20 epochs. The training data is split into 29 quarters for training, 6 quarters for validation, and 6 quarters for testing, with a loss weight balance of 1.0, ensuring equal weighting between drug-drug and drug-side effect losses. Training is conducted for a maximum of 500 epochs.




\paragraph{HGNN-Based Method.} We introduce HGNN with SMILES Embedding Features and Attention Mechanism, a novel architecture specifically crafted to capture high-order relationships in chemical data efficiently. By combining hypergraph convolution, multi-head attention mechanisms, and molecular embeddings derived from SMILES representations, our approach significantly enhances feature representation and boosts predictive accuracy. Unlike conventional graph neural networks (GNNs), this model utilizes a hypergraph incidence matrix to model intricate interactions, where a single hyperedge can link multiple drug entities. Node features are derived from SMILES (Simplified Molecular Input Line Entry System) representations and converted into numerical embeddings via a SMILES-to-vector algorithm. These features undergo linear transformation and feature fusion before being processed through hypergraph convolution layers. To improve the model's expressive power, a multi-head attention mechanism is integrated into the hypergraph convolution process, enabling the dynamic assignment of importance to different hyperedges and facilitating more adaptable feature representations. The model architecture comprises an initial feature encoder, a series of hypergraph convolution layers (HypergraphConv), and a fully connected output layer for classification. The number of hypergraph convolution layers, denoted as $L$, is adjustable, allowing the model to capture both local and global dependencies in hypergraph-structured drug-side effect datasets. During forward propagation, drug node features are first encoded through a ReLU-activated transformation and combined with external SMILES-based embeddings. The hypergraph convolution operation is applied iteratively across $L$ layers, progressively refining node embeddings based on hypergraph connectivity. Finally, hyperedge features are computed by aggregating node representations, followed by a softmax layer for binary classification.
%The HGNN model utilizes spectral convolution on a hypergraph to learn node representations. The architecture consists of an encoder with a feature dimension of 256, followed by a graph convolutional layer with a feature dimension of 128. The model employs a Chebyshev filter of size 3. Training is performed with a learning rate of 0.01, a dropout rate of 0.1, and a weight decay of 0.01. The activation function is ReLU.

% \paragraph{Graph Neural Networks (GCN and GAT)}
% The GCN model shares the same architecture and training conditions as GAT, with the only difference being the use of GCN layers instead of GAT layers. The architecture uses:
% \begin{itemize}
%     \item Architecture:
%     \begin{itemize}
%         \item First layer: Input $\rightarrow$ hidden\_channels (128)
%         \item Second layer: $128 \rightarrow 64$ (hidden\_channels/2)
%         \item Classifier: $128 \rightarrow 2$
%     \end{itemize}
%     \item Learning Rate: 0.0005
%     \item Weight Decay: 0.001
%     \item Early Stopping:
%     \begin{itemize}
%         \item Patience: 20 epochs
%         \item Minimum delta: 0.001
%     \end{itemize}
%     \item Loss Weight Balance ($\alpha$): 1.0 (equal weighting between drug-drug and drug-side effect losses)
% \end{itemize}

% \paragraph{Multi-Layer Perceptron (MLP)}
% The MLP architecture was designed to reduce high-dimensional embeddings while maintaining important features:
% \begin{itemize}
%     \item Architecture:
%     \begin{itemize}
%         \item Drug reduction layer: $768 \rightarrow 64$ dimensions
%         \item Side effect reduction layer: $768 \rightarrow 64$ dimensions
%         \item Classifier: $128 \rightarrow 32 \rightarrow 1$ with ReLU activation
%     \end{itemize}
%     \item Learning Rate: 0.001
%     \item Early Stopping:
%     \begin{itemize}
%         \item Patience: 20 epochs
%         \item Minimum delta: 0.001
%     \end{itemize}
% \end{itemize}

% \paragraph{Hypergraph Neural Networks}
% \begin{itemize}
%     \item Maximum Epochs: 500
%     \item Early Stopping: 20 epochs patience
%     \item Optimizer: Adam
%     \item Loss Function: Binary cross-entropy for HGNN and HyGNN
%     \item Data Split: Based on quarterly periods
%     \begin{itemize}
%         \item Training: 29 quarters
%         \item Validation: 6 quarters
%         \item Testing: 6 quarters
%     \end{itemize}
% \end{itemize}

% \paragraph{Graph Neural Networks}
% \begin{itemize}
%     \item Maximum Epochs: 500
%     \item Learning Rate Scheduler: ReduceLROnPlateau
%     \begin{itemize}
%         \item Mode: max
%         \item Patience: 20
%         \item Factor: 0.1
%     \end{itemize}
%     \item Optimizer: AdamW
%     \item Loss Function: Cross Entropy Loss with class weights
%     \item Data Split: Based on quarterly periods
%     \begin{itemize}
%         \item Training: 29 quarters
%         \item Validation: 6 quarters
%         \item Testing: 6 quarters
%     \end{itemize}
% \end{itemize}



% \subparagraph{HyGNN}
% The HyGNN architecture employs a dual-attention mechanism with hyperedge-level and node-level attention:
% \begin{itemize}
%     \item Architecture:
%     \begin{itemize}
%         \item Drug reduction layer: $768 \rightarrow 64$ dimensions
%         \item Two levels of attention: hyperedge-level and node-level
%         \item Decoder: MLP ($128 \rightarrow 1$) or dot product
%     \end{itemize}
%     \item Learning Rate: 0.001
%     \item Early Stopping:
%     \begin{itemize}
%         \item Patience: 20 epochs
%         \item Minimum delta: based on validation loss
%     \end{itemize}
% \end{itemize}

% \subparagraph{CHESHIRE}
% CHESHIRE utilizes a Chebyshev spectral graph convolutional network with dual-attention:
% \begin{itemize}
%     \item Architecture:
%     \begin{itemize}
%         \item Encoder: one-layer neural network for feature initialization
%         \item CSGCN with K filters
%         \item Dual pooling: maximum minimum-based and Frobenius norm-based
%     \end{itemize}
%     \item Learning Rate: Optimized via grid search in \{1e-2, 5e-2, 1e-3, 5e-3\}
%     \item Hidden Units: Chosen from \{32, 64, 128\}
%     \item Dropout: \{0.1, 0.5\}
%     \item Weight Decay: \{1e-2, 1e-3\}
% \end{itemize}

\subsubsection{Hardware Configuration}
\begin{itemize}
    \item Computing Device: CUDA-enabled GPU (when available) or CPU
    \item Memory Management:
    \begin{itemize}
        \item SMILES processing batch size: 16
        \item Maximum sequence length: 256 for SMILES strings
    \end{itemize}
    % \item Random Seeds: Multiple fixed seeds for reproducibility \Yingdan{Random or fixed seeds? If fixed, then provide specific seed values.}
    \item Random Seeds: Fixed seed values of 42, 3407, 54321, and 123456 were used for reproducibility
\end{itemize}

%%%%%%%%%%%%%%%%%%%%%%
% \clearpage
% \subsection{Additional Experimental Results}
% \label{Appendix_Model Comparison}

% Test Set Results


% \begin{table}[t]
%     \centering
%     \caption{Model performance metrics. Values in parentheses denote standard deviations.}

%     \resizebox{0.45\textwidth}{!}{
%     \begin{tabular}{@{}lcccc@{}}
%         \toprule
%         \textbf{Model} & \textbf{Precision} & \textbf{F1} & \textbf{AUC} & \textbf{PRAUC} \\
%         \midrule
%         GCN & 0.745 (0.016) & 0.778 (0.011) & 0.829 (0.009) & 0.805 (0.008) \\
%         GAT & 0.743 (0.033) & 0.809 (0.013) & 0.851 (0.029) & 0.789 (0.046) \\
%         MLP & 0.805 (0.013) & 0.819 (0.010) & 0.897 (0.005) & 0.872 (0.007) \\
%         CHESHIRE & 0.906 (0.002) & 0.933 (0.001) & 0.957 (0.003) & 0.939 (0.008) \\
%         \bottomrule
%     \end{tabular}
%     }
%     \label{tab:test_results}
% \end{table}

% In the unusual situation where you want a paper to appear in the
% references without citing it in the main text, use \nocite
% \nocite{langley00}

% \bibliography{example_paper}
% \bibliographystyle{icml2025}


% %%%%%%%%%%%%%%%%%%%%%%%%%%%%%%%%%%%%%%%%%%%%%%%%%%%%%%%%%%%%%%%%%%%%%%%%%%%%%%%
% %%%%%%%%%%%%%%%%%%%%%%%%%%%%%%%%%%%%%%%%%%%%%%%%%%%%%%%%%%%%%%%%%%%%%%%%%%%%%%%
% % APPENDIX
% %%%%%%%%%%%%%%%%%%%%%%%%%%%%%%%%%%%%%%%%%%%%%%%%%%%%%%%%%%%%%%%%%%%%%%%%%%%%%%%
% %%%%%%%%%%%%%%%%%%%%%%%%%%%%%%%%%%%%%%%%%%%%%%%%%%%%%%%%%%%%%%%%%%%%%%%%%%%%%%%
% \newpage
% \appendix
% \onecolumn
% \section{You \emph{can} have an appendix here.}

% You can have as much text here as you want. The main body must be at most $8$ pages long.
% For the final version, one more page can be added.
% If you want, you can use an appendix like this one.  

% The $\mathtt{\backslash onecolumn}$ command above can be kept in place if you prefer a one-column appendix, or can be removed if you prefer a two-column appendix.  Apart from this possible change, the style (font size, spacing, margins, page numbering, etc.) should be kept the same as the main body.
% %%%%%%%%%%%%%%%%%%%%%%%%%%%%%%%%%%%%%%%%%%%%%%%%%%%%%%%%%%%%%%%%%%%%%%%%%%%%%%%
% %%%%%%%%%%%%%%%%%%%%%%%%%%%%%%%%%%%%%%%%%%%%%%%%%%%%%%%%%%%%%%%%%%%%%%%%%%%%%%%




% % This document was modified from the file originally made available by
% % Pat Langley and Andrea Danyluk for ICML-2K. This version was created
% % by Iain Murray in 2018, and modified by Alexandre Bouchard in
% % 2019 and 2021 and by Csaba Szepesvari, Gang Niu and Sivan Sabato in 2022.
% % Modified again in 2023 and 2024 by Sivan Sabato and Jonathan Scarlett.
% % Previous contributors include Dan Roy, Lise Getoor and Tobias
% % Scheffer, which was slightly modified from the 2010 version by
% % Thorsten Joachims & Johannes Fuernkranz, slightly modified from the
% % 2009 version by Kiri Wagstaff and Sam Roweis's 2008 version, which is
% % slightly modified from Prasad Tadepalli's 2007 version which is a
% % lightly changed version of the previous year's version by Andrew
% % Moore, which was in turn edited from those of Kristian Kersting and
% % Codrina Lauth. Alex Smola contributed to the algorithmic style files.
\end{document}