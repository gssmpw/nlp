%%%%%%%% ICML 2025 EXAMPLE LATEX SUBMISSION FILE %%%%%%%%%%%%%%%%%

\documentclass{article}

% Recommended, but optional, packages for figures and better typesetting:
\usepackage{microtype}
\usepackage{graphicx}
\usepackage{subfigure}
\usepackage{booktabs} % for professional tables

\usepackage{float}
\usepackage{caption}
\usepackage{makecell}
\usepackage{tabularx}
\usepackage{multirow}
\usepackage{bm}

% hyperref makes hyperlinks in the resulting PDF.
% If your build breaks (sometimes temporarily if a hyperlink spans a page)
% please comment out the following usepackage line and replace
% \usepackage{icml2025} with \usepackage[nohyperref]{icml2025} above.
\usepackage{hyperref}

\usepackage{lipsum}  


% Attempt to make hyperref and algorithmic work together better:
\newcommand{\theHalgorithm}{\arabic{algorithm}}

% Use the following line for the initial blind version submitted for review:
% \usepackage{icml2025}

% If accepted, instead use the following line for the camera-ready submission:
\usepackage[accepted]{icml2025}

% For theorems and such
\usepackage{amsmath}
\usepackage{amssymb}
\usepackage{mathtools}
\usepackage{amsthm}

% if you use cleveref..
\usepackage[capitalize,noabbrev]{cleveref}

%%%%%%%%%%%%%%%%%%%%%%%%%%%%%%%%
% THEOREMS
%%%%%%%%%%%%%%%%%%%%%%%%%%%%%%%%
\theoremstyle{plain}
\newtheorem{theorem}{Theorem}[section]
\newtheorem{proposition}[theorem]{Proposition}
\newtheorem{lemma}[theorem]{Lemma}
\newtheorem{corollary}[theorem]{Corollary}
\theoremstyle{definition}
\newtheorem{definition}[theorem]{Definition}
\newtheorem{assumption}[theorem]{Assumption}
\theoremstyle{remark}
\newtheorem{remark}[theorem]{Remark}


% Todonotes is useful during development; simply uncomment the next line
%    and comment out the line below the next line to turn off comments
%\usepackage[disable,textsize=tiny]{todonotes}
\usepackage[textsize=tiny]{todonotes}


% The \icmltitle you define below is probably too long as a header.
% Therefore, a short form for the running title is supplied here:
\icmltitlerunning{Screener: Self-supervised Pathology Segmentation Model for 3D Medical Images}

\begin{document}

\twocolumn[
\icmltitle{Screener: Self-supervised Pathology Segmentation Model for 3D Medical Images}

% It is OKAY to include author information, even for blind
% submissions: the style file will automatically remove it for you
% unless you've provided the [accepted] option to the icml2025
% package.

% List of affiliations: The first argument should be a (short)
% identifier you will use later to specify author affiliations
% Academic affiliations should list Department, University, City, Region, Country
% Industry affiliations should list Company, City, Region, Country

% You can specify symbols, otherwise they are numbered in order.
% Ideally, you should not use this facility. Affiliations will be numbered
% in order of appearance and this is the preferred way.
\icmlsetsymbol{equal}{*}

\begin{icmlauthorlist}
\icmlauthor{Mikhail Goncharov}{equal,skoltech,airi}
\icmlauthor{Eugenia Soboleva}{equal,ira-labs}
\icmlauthor{Mariia Donskova}{iitp,ira-labs}
\icmlauthor{Ivan Oseledets}{airi}
\icmlauthor{Marina Munkhoeva}{airi}
\icmlauthor{Maxim Panov}{mbzuai}
\end{icmlauthorlist}

\icmlaffiliation{skoltech}{Skolkovo Institute of Science and Technology (Skoltech), Moscow, Russia}
\icmlaffiliation{airi}{AIRI, Moscow, Russia}
\icmlaffiliation{ira-labs}{IRA-Labs, Moscow, Russia}
\icmlaffiliation{iitp}{Institute for Information Transmission Problems (IITP), Moscow, Russia}
\icmlaffiliation{mbzuai}{Mohamed bin Zayed University of Artificial Intelligence (MBZUAI), Abu Dhabi, UAE}

\icmlcorrespondingauthor{Mikhail Goncharov}{Mikhail.Goncharov2@skoltech.ru}
% \icmlcorrespondingauthor{Firstname2 Lastname2}{first2.last2@www.uk}

% You may provide any keywords that you
% find helpful for describing your paper; these are used to populate
% the "keywords" metadata in the PDF but will not be shown in the document
\icmlkeywords{Unsupervised Visual Anomaly Segmentation, Self-supervised learning, Density estimation, Computed Tomography}

\vskip 0.3in
]

% this must go after the closing bracket ] following \twocolumn[ ...

% This command actually creates the footnote in the first column
% listing the affiliations and the copyright notice.
% The command takes one argument, which is text to display at the start of the footnote.
% The \icmlEqualContribution command is standard text for equal contribution.
% Remove it (just {}) if you do not need this facility.

%\printAffiliationsAndNotice{}  % leave blank if no need to mention equal contribution
\printAffiliationsAndNotice{\icmlEqualContribution} % otherwise use the standard text.

\begin{abstract}


The choice of representation for geographic location significantly impacts the accuracy of models for a broad range of geospatial tasks, including fine-grained species classification, population density estimation, and biome classification. Recent works like SatCLIP and GeoCLIP learn such representations by contrastively aligning geolocation with co-located images. While these methods work exceptionally well, in this paper, we posit that the current training strategies fail to fully capture the important visual features. We provide an information theoretic perspective on why the resulting embeddings from these methods discard crucial visual information that is important for many downstream tasks. To solve this problem, we propose a novel retrieval-augmented strategy called RANGE. We build our method on the intuition that the visual features of a location can be estimated by combining the visual features from multiple similar-looking locations. We evaluate our method across a wide variety of tasks. Our results show that RANGE outperforms the existing state-of-the-art models with significant margins in most tasks. We show gains of up to 13.1\% on classification tasks and 0.145 $R^2$ on regression tasks. All our code and models will be made available at: \href{https://github.com/mvrl/RANGE}{https://github.com/mvrl/RANGE}.

\end{abstract}


\section{Introduction}
Backdoor attacks pose a concealed yet profound security risk to machine learning (ML) models, for which the adversaries can inject a stealth backdoor into the model during training, enabling them to illicitly control the model's output upon encountering predefined inputs. These attacks can even occur without the knowledge of developers or end-users, thereby undermining the trust in ML systems. As ML becomes more deeply embedded in critical sectors like finance, healthcare, and autonomous driving \citep{he2016deep, liu2020computing, tournier2019mrtrix3, adjabi2020past}, the potential damage from backdoor attacks grows, underscoring the emergency for developing robust defense mechanisms against backdoor attacks.

To address the threat of backdoor attacks, researchers have developed a variety of strategies \cite{liu2018fine,wu2021adversarial,wang2019neural,zeng2022adversarial,zhu2023neural,Zhu_2023_ICCV, wei2024shared,wei2024d3}, aimed at purifying backdoors within victim models. These methods are designed to integrate with current deployment workflows seamlessly and have demonstrated significant success in mitigating the effects of backdoor triggers \cite{wubackdoorbench, wu2023defenses, wu2024backdoorbench,dunnett2024countering}.  However, most state-of-the-art (SOTA) backdoor purification methods operate under the assumption that a small clean dataset, often referred to as \textbf{auxiliary dataset}, is available for purification. Such an assumption poses practical challenges, especially in scenarios where data is scarce. To tackle this challenge, efforts have been made to reduce the size of the required auxiliary dataset~\cite{chai2022oneshot,li2023reconstructive, Zhu_2023_ICCV} and even explore dataset-free purification techniques~\cite{zheng2022data,hong2023revisiting,lin2024fusing}. Although these approaches offer some improvements, recent evaluations \cite{dunnett2024countering, wu2024backdoorbench} continue to highlight the importance of sufficient auxiliary data for achieving robust defenses against backdoor attacks.

While significant progress has been made in reducing the size of auxiliary datasets, an equally critical yet underexplored question remains: \emph{how does the nature of the auxiliary dataset affect purification effectiveness?} In  real-world  applications, auxiliary datasets can vary widely, encompassing in-distribution data, synthetic data, or external data from different sources. Understanding how each type of auxiliary dataset influences the purification effectiveness is vital for selecting or constructing the most suitable auxiliary dataset and the corresponding technique. For instance, when multiple datasets are available, understanding how different datasets contribute to purification can guide defenders in selecting or crafting the most appropriate dataset. Conversely, when only limited auxiliary data is accessible, knowing which purification technique works best under those constraints is critical. Therefore, there is an urgent need for a thorough investigation into the impact of auxiliary datasets on purification effectiveness to guide defenders in  enhancing the security of ML systems. 

In this paper, we systematically investigate the critical role of auxiliary datasets in backdoor purification, aiming to bridge the gap between idealized and practical purification scenarios.  Specifically, we first construct a diverse set of auxiliary datasets to emulate real-world conditions, as summarized in Table~\ref{overall}. These datasets include in-distribution data, synthetic data, and external data from other sources. Through an evaluation of SOTA backdoor purification methods across these datasets, we uncover several critical insights: \textbf{1)} In-distribution datasets, particularly those carefully filtered from the original training data of the victim model, effectively preserve the model’s utility for its intended tasks but may fall short in eliminating backdoors. \textbf{2)} Incorporating OOD datasets can help the model forget backdoors but also bring the risk of forgetting critical learned knowledge, significantly degrading its overall performance. Building on these findings, we propose Guided Input Calibration (GIC), a novel technique that enhances backdoor purification by adaptively transforming auxiliary data to better align with the victim model’s learned representations. By leveraging the victim model itself to guide this transformation, GIC optimizes the purification process, striking a balance between preserving model utility and mitigating backdoor threats. Extensive experiments demonstrate that GIC significantly improves the effectiveness of backdoor purification across diverse auxiliary datasets, providing a practical and robust defense solution.

Our main contributions are threefold:
\textbf{1) Impact analysis of auxiliary datasets:} We take the \textbf{first step}  in systematically investigating how different types of auxiliary datasets influence backdoor purification effectiveness. Our findings provide novel insights and serve as a foundation for future research on optimizing dataset selection and construction for enhanced backdoor defense.
%
\textbf{2) Compilation and evaluation of diverse auxiliary datasets:}  We have compiled and rigorously evaluated a diverse set of auxiliary datasets using SOTA purification methods, making our datasets and code publicly available to facilitate and support future research on practical backdoor defense strategies.
%
\textbf{3) Introduction of GIC:} We introduce GIC, the \textbf{first} dedicated solution designed to align auxiliary datasets with the model’s learned representations, significantly enhancing backdoor mitigation across various dataset types. Our approach sets a new benchmark for practical and effective backdoor defense.



\section{Background} \label{section:LLM}

% \subsection{Large Language Model (LLM)}   

Figure~\ref{fig:LLaMA_model}(a) shows that a decoder-only LLM initially processes a user prompt in the “prefill” stage and subsequently generates tokens sequentially during the “decoding” stage.
Both stages contain an input embedding layer, multiple decoder transformer blocks, an output embedding layer, and a sampling layer.
Figure~\ref{fig:LLaMA_model}(b) demonstrates that the decoder transformer blocks consist of a self attention and a feed-forward network (FFN) layer, each paired with residual connection and normalization layers. 

% Differentiate between encoder/decoder, explain why operation intensity is low, explain the different parts of a transformer block. Discuss Table II here. 

% Explain the architecture with Llama2-70B.

% \begin{table}[thb]
% \renewcommand\arraystretch{1.05}
% \centering
% % \vspace{-5mm}
%     \caption{ML Model Parameter Size and Operational Intensity}
%     \vspace{-2mm}
%     \small
%     \label{tab:ML Model Parameter Size and Operational Intensity}    
%     \scalebox{0.95}{
%         \begin{tabular}{|c|c|c|c|c|}
%             \hline
%             & Llama2 & BLOOM & BERT & ResNet \\
%             Model & (70B) & (176B) & & 152 \\
%             \hline
%             Parameter Size (GB) & 140 & 352 & 0.17 & 0.16 \\
%             \hline
%             Op Intensity (Ops/Byte) & 1 & 1 & 282 & 346 \\
%             \hline
%           \end{tabular}
%     }
% \vspace{-3mm}
% \end{table}

% {\fontsize{8pt}{11pt}\selectfont 8pt font size test Memory Requirement}

\begin{figure}[t]
    \centering
    \includegraphics[width=8cm]{Figure/LLaMA_model_new_new.pdf}
    \caption{(a) Prefill stage encodes prompt tokens in parallel. Decoding stage generates output tokens sequentially.
    (b) LLM contains N$\times$ decoder transformer blocks. 
    (c) Llama2 model architecture.}
    \label{fig:LLaMA_model}
\end{figure}

Figure~\ref{fig:LLaMA_model}(c) demonstrates the Llama2~\cite{touvron2023llama} model architecture as a representative LLM.
% The self attention layer requires three GEMVs\footnote{GEMVs in multi-head attention~\cite{attention}, narrow GEMMs in grouped-query attention~\cite{gqa}.} to generate query, key and value vectors.
In the self-attention layer, query, key and value vectors are generated by multiplying input vector to corresponding weight matrices.
These matrices are segmented into multiple heads, representing different semantic dimensions.
The query and key vectors go though Rotary Positional Embedding (RoPE) to encode the relative positional information~\cite{rope-paper}.
Within each head, the generated key and value vectors are appended to their caches.
The query vector is multiplied by the key cache to produce a score vector.
After the Softmax operation, the score vector is multiplied by the value cache to yield the output vector.
The output vectors from all heads are concatenated and multiplied by output weight matrix, resulting in a vector that undergoes residual connection and Root Mean Square layer Normalization (RMSNorm)~\cite{rmsnorm-paper}.
The residual connection adds up the input and output vectors of a layer to avoid vanishing gradient~\cite{he2016deep}.
The FFN layer begins with two parallel fully connections, followed by a Sigmoid Linear Unit (SiLU), and ends with another fully connection.
\section{Study Design}
% robot: aliengo 
% We used the Unitree AlienGo quadruped robot. 
% See Appendix 1 in AlienGo Software Guide PDF
% Weight = 25kg, size (L,W,H) = (0.55, 0.35, 06) m when standing, (0.55, 0.35, 0.31) m when walking
% Handle is 0.4 m or 0.5 m. I'll need to check it to see which type it is.
We gathered input from primary stakeholders of the robot dog guide, divided into three subgroups: BVI individuals who have owned a dog guide, BVI individuals who were not dog guide owners, and sighted individuals with generally low degrees of familiarity with dog guides. While the main focus of this study was on the BVI participants, we elected to include survey responses from sighted participants given the importance of social acceptance of the robot by the general public, which could reflect upon the BVI users themselves and affect their interactions with the general population \cite{kayukawa2022perceive}. 

The need-finding processes consisted of two stages. During Stage 1, we conducted in-depth interviews with BVI participants, querying their experiences in using conventional assistive technologies and dog guides. During Stage 2, a large-scale survey was distributed to both BVI and sighted participants. 

This study was approved by the University’s Institutional Review Board (IRB), and all processes were conducted after obtaining the participants' consent.

\subsection{Stage 1: Interviews}
We recruited nine BVI participants (\textbf{Table}~\ref{tab:bvi-info}) for in-depth interviews, which lasted 45-90 minutes for current or former dog guide owners (DO) and 30-60 minutes for participants without dog guides (NDO). Group DO consisted of five participants, while Group NDO consisted of four participants.
% The interview participants were divided into two groups. Group DO (Dog guide Owner) consisted of five participants who were current or former dog guide owners and Group NDO (Non Dog guide Owner) consisted of three participants who were not dog guide owners. 
All participants were familiar with using white canes as a mobility aid. 

We recruited participants in both groups, DO and NDO, to gather data from those with substantial experience with dog guides, offering potentially more practical insights, and from those without prior experience, providing a perspective that may be less constrained and more open to novel approaches. 

We asked about the participants' overall impressions of a robot dog guide, expectations regarding its potential benefits and challenges compared to a conventional dog guide, their desired methods of giving commands and communicating with the robot dog guide, essential functionalities that the robot dog guide should offer, and their preferences for various aspects of the robot dog guide's form factors. 
For Group DO, we also included questions that asked about the participants' experiences with conventional dog guides. 

% We obtained permission to record the conversations for our records while simultaneously taking notes during the interviews. The interviews lasted 30-60 minutes for NDO participants and 45-90 minutes for DO participants. 

\subsection{Stage 2: Large-Scale Surveys} 
After gathering sufficient initial results from the interviews, we created an online survey for distributing to a larger pool of participants. The survey platform used was Qualtrics. 

\subsubsection{Survey Participants}
The survey had 100 participants divided into two primary groups. Group BVI consisted of 42 blind or visually impaired participants, and Group ST consisted of 58 sighted participants. \textbf{Table}~\ref{tab:survey-demographics} shows the demographic information of the survey participants. 

\subsubsection{Question Differentiation} 
Based on their responses to initial qualifying questions, survey participants were sorted into three subgroups: DO, NDO, and ST. Each participant was assigned one of three different versions of the survey. The surveys for BVI participants mirrored the interview categories (overall impressions, communication methods, functionalities, and form factors), but with a more quantitative approach rather than the open-ended questions used in interviews. The DO version included additional questions pertaining to their prior experience with dog guides. The ST version revolved around the participants' prior interactions with and feelings toward dog guides and dogs in general, their thoughts on a robot dog guide, and broad opinions on the aesthetic component of the robot's design. 

\section{Experiments}

\subsection{Setups}
\subsubsection{Implementation Details}
We apply our FDS method to two types of 3DGS: 
the original 3DGS, and 2DGS~\citep{huang20242d}. 
%
The number of iterations in our optimization 
process is 35,000.
We follow the default training configuration 
and apply our FDS method after 15,000 iterations,
then we add normal consistency loss for both
3DGS and 2DGS after 25000 iterations.
%
The weight for FDS, $\lambda_{fds}$, is set to 0.015,
the $\sigma$ is set to 23,
and the weight for normal consistency is set to 0.15
for all experiments. 
We removed the depth distortion loss in 2DGS 
because we found that it degrades its results in indoor scenes.
%
The Gaussian point cloud is initialized using Colmap
for all datasets.
%
%
We tested the impact of 
using Sea Raft~\citep{wang2025sea} and 
Raft\citep{teed2020raft} on FDS performance.
%
Due to the blurriness of the ScanNet dataset, 
additional prior constraints are required.
Thus, we incorporate normal prior supervision 
on the rendered normals 
in ScanNet (V2) dataset by default.
The normal prior is predicted by the Stable Normal 
model~\citep{ye2024stablenormal}
across all types of 3DGS.
%
The entire framework is implemented in 
PyTorch~\citep{paszke2019pytorch}, 
and all experiments are conducted on 
a single NVIDIA 4090D GPU.

\begin{figure}[t] \centering
    \makebox[0.16\textwidth]{\scriptsize Input}
    \makebox[0.16\textwidth]{\scriptsize 3DGS}
    \makebox[0.16\textwidth]{\scriptsize 2DGS}
    \makebox[0.16\textwidth]{\scriptsize 3DGS + FDS}
    \makebox[0.16\textwidth]{\scriptsize 2DGS + FDS}
    \makebox[0.16\textwidth]{\scriptsize GT (Depth)}

    \includegraphics[width=0.16\textwidth]{figure/fig3_img/compare3/gt_rgb/frame_00522.jpg}
    \includegraphics[width=0.16\textwidth]{figure/fig3_img/compare3/3DGS/frame_00522.jpg}
    \includegraphics[width=0.16\textwidth]{figure/fig3_img/compare3/2DGS/frame_00522.jpg}
    \includegraphics[width=0.16\textwidth]{figure/fig3_img/compare3/3DGS+FDS/frame_00522.jpg}
    \includegraphics[width=0.16\textwidth]{figure/fig3_img/compare3/2DGS+FDS/frame_00522.jpg}
    \includegraphics[width=0.16\textwidth]{figure/fig3_img/compare3/gt_depth/frame_00522.jpg} \\

    % \includegraphics[width=0.16\textwidth]{figure/fig3_img/compare1/gt_rgb/frame_00137.jpg}
    % \includegraphics[width=0.16\textwidth]{figure/fig3_img/compare1/3DGS/frame_00137.jpg}
    % \includegraphics[width=0.16\textwidth]{figure/fig3_img/compare1/2DGS/frame_00137.jpg}
    % \includegraphics[width=0.16\textwidth]{figure/fig3_img/compare1/3DGS+FDS/frame_00137.jpg}
    % \includegraphics[width=0.16\textwidth]{figure/fig3_img/compare1/2DGS+FDS/frame_00137.jpg}
    % \includegraphics[width=0.16\textwidth]{figure/fig3_img/compare1/gt_depth/frame_00137.jpg} \\

     \includegraphics[width=0.16\textwidth]{figure/fig3_img/compare2/gt_rgb/frame_00262.jpg}
    \includegraphics[width=0.16\textwidth]{figure/fig3_img/compare2/3DGS/frame_00262.jpg}
    \includegraphics[width=0.16\textwidth]{figure/fig3_img/compare2/2DGS/frame_00262.jpg}
    \includegraphics[width=0.16\textwidth]{figure/fig3_img/compare2/3DGS+FDS/frame_00262.jpg}
    \includegraphics[width=0.16\textwidth]{figure/fig3_img/compare2/2DGS+FDS/frame_00262.jpg}
    \includegraphics[width=0.16\textwidth]{figure/fig3_img/compare2/gt_depth/frame_00262.jpg} \\

    \includegraphics[width=0.16\textwidth]{figure/fig3_img/compare4/gt_rgb/frame00000.png}
    \includegraphics[width=0.16\textwidth]{figure/fig3_img/compare4/3DGS/frame00000.png}
    \includegraphics[width=0.16\textwidth]{figure/fig3_img/compare4/2DGS/frame00000.png}
    \includegraphics[width=0.16\textwidth]{figure/fig3_img/compare4/3DGS+FDS/frame00000.png}
    \includegraphics[width=0.16\textwidth]{figure/fig3_img/compare4/2DGS+FDS/frame00000.png}
    \includegraphics[width=0.16\textwidth]{figure/fig3_img/compare4/gt_depth/frame00000.png} \\

    \includegraphics[width=0.16\textwidth]{figure/fig3_img/compare5/gt_rgb/frame00080.png}
    \includegraphics[width=0.16\textwidth]{figure/fig3_img/compare5/3DGS/frame00080.png}
    \includegraphics[width=0.16\textwidth]{figure/fig3_img/compare5/2DGS/frame00080.png}
    \includegraphics[width=0.16\textwidth]{figure/fig3_img/compare5/3DGS+FDS/frame00080.png}
    \includegraphics[width=0.16\textwidth]{figure/fig3_img/compare5/2DGS+FDS/frame00080.png}
    \includegraphics[width=0.16\textwidth]{figure/fig3_img/compare5/gt_depth/frame00080.png} \\



    \caption{\textbf{Comparison of depth reconstruction on Mushroom and ScanNet datasets.} The original
    3DGS or 2DGS model equipped with FDS can remove unwanted floaters and reconstruct
    geometry more preciously.}
    \label{fig:compare}
\end{figure}


\subsubsection{Datasets and Metrics}

We evaluate our method for 3D reconstruction 
and novel view synthesis tasks on
\textbf{Mushroom}~\citep{ren2024mushroom},
\textbf{ScanNet (v2)}~\citep{dai2017scannet}, and 
\textbf{Replica}~\citep{replica19arxiv}
datasets,
which feature challenging indoor scenes with both 
sparse and dense image sampling.
%
The Mushroom dataset is an indoor dataset 
with sparse image sampling and two distinct 
camera trajectories. 
%
We train our model on the training split of 
the long capture sequence and evaluate 
novel view synthesis on the test split 
of the long capture sequences.
%
Five scenes are selected to evaluate our FDS, 
including "coffee room", "honka", "kokko", 
"sauna", and "vr room". 
%
ScanNet(V2)~\citep{dai2017scannet}  consists of 1,613 indoor scenes
with annotated camera poses and depth maps. 
%
We select 5 scenes from the ScanNet (V2) dataset, 
uniformly sampling one-tenth of the views,
following the approach in ~\citep{guo2022manhattan}.
To further improve the geometry rendering quality of 3DGS, 
%
Replica~\citep{replica19arxiv} contains small-scale 
real-world indoor scans. 
We evaluate our FDS on five scenes from 
Replica: office0, office1, office2, office3 and office4,
selecting one-tenth of the views for training.
%
The results for Replica are provided in the 
supplementary materials.
To evaluate the rendering quality and geometry 
of 3DGS, we report PSNR, SSIM, and LPIPS for 
rendering quality, along with Absolute Relative Distance 
(Abs Rel) as a depth quality metrics.
%
Additionally, for mesh evaluation, 
we use metrics including Accuracy, Completion, 
Chamfer-L1 distance, Normal Consistency, 
and F-scores.




\subsection{Results}
\subsubsection{Depth rendering and novel view synthesis}
The comparison results on Mushroom and 
ScanNet are presented in \tabref{tab:mushroom} 
and \tabref{tab:scannet}, respectively. 
%
Due to the sparsity of sampling 
in the Mushroom dataset,
challenges are posed for both GOF~\citep{yu2024gaussian} 
and PGSR~\citep{chen2024pgsr}, 
leading to their relative poor performance 
on the Mushroom dataset.
%
Our approach achieves the best performance 
with the FDS method applied during the training process.
The FDS significantly enhances the 
geometric quality of 3DGS on the Mushroom dataset, 
improving the "abs rel" metric by more than 50\%.
%
We found that Sea Raft~\citep{wang2025sea}
outperforms Raft~\citep{teed2020raft} on FDS, 
indicating that a better optical flow model 
can lead to more significant improvements.
%
Additionally, the render quality of RGB 
images shows a slight improvement, 
by 0.58 in 3DGS and 0.50 in 2DGS, 
benefiting from the incorporation of cross-view consistency in FDS. 
%
On the Mushroom
dataset, adding the FDS loss increases 
the training time by half an hour, which maintains the same
level as baseline.
%
Similarly, our method shows a notable improvement on the ScanNet dataset as well using Sea Raft~\citep{wang2025sea} Model. The "abs rel" metric in 2DGS is improved nearly 50\%. This demonstrates the robustness and effectiveness of the FDS method across different datasets.
%


% \begin{wraptable}{r}{0.6\linewidth} \centering
% \caption{\textbf{Ablation study on geometry priors.}} 
%         \label{tab:analysis_prior}
%         \resizebox{\textwidth}{!}{
\begin{tabular}{c| c c c c c | c c c c}

    \hline
     Method &  Acc$\downarrow$ & Comp $\downarrow$ & C-L1 $\downarrow$ & NC $\uparrow$ & F-Score $\uparrow$ &  Abs Rel $\downarrow$ &  PSNR $\uparrow$  & SSIM  $\uparrow$ & LPIPS $\downarrow$ \\ \hline
    2DGS&   0.1078&  0.0850&  0.0964&  0.7835&  0.5170&  0.1002&  23.56&  0.8166& 0.2730\\
    2DGS+Depth&   0.0862&  0.0702&  0.0782&  0.8153&  0.5965&  0.0672&  23.92&  0.8227& 0.2619 \\
    2DGS+MVDepth&   0.2065&  0.0917&  0.1491&  0.7832&  0.3178&  0.0792&  23.74&  0.8193& 0.2692 \\
    2DGS+Normal&   0.0939&  0.0637&  0.0788&  \textbf{0.8359}&  0.5782&  0.0768&  23.78&  0.8197& 0.2676 \\
    2DGS+FDS &  \textbf{0.0615} & \textbf{ 0.0534}& \textbf{0.0574}& 0.8151& \textbf{0.6974}&  \textbf{0.0561}&  \textbf{24.06}&  \textbf{0.8271}&\textbf{0.2610} \\ \hline
    2DGS+Depth+FDS &  0.0561 &  0.0519& 0.0540& 0.8295& 0.7282&  0.0454&  \textbf{24.22}& \textbf{0.8291}&\textbf{0.2570} \\
    2DGS+Normal+FDS &  \textbf{0.0529} & \textbf{ 0.0450}& \textbf{0.0490}& \textbf{0.8477}& \textbf{0.7430}&  \textbf{0.0443}&  24.10&  0.8283& 0.2590 \\
    2DGS+Depth+Normal &  0.0695 & 0.0513& 0.0604& 0.8540&0.6723&  0.0523&  24.09&  0.8264&0.2575\\ \hline
    2DGS+Depth+Normal+FDS &  \textbf{0.0506} & \textbf{0.0423}& \textbf{0.0464}& \textbf{0.8598}&\textbf{0.7613}&  \textbf{0.0403}&  \textbf{24.22}& 
    \textbf{0.8300}&\textbf{0.0403}\\
    
\bottomrule
\end{tabular}
}
% \end{wraptable}



The qualitative comparisons on the Mushroom and ScanNet dataset 
are illustrated in \figref{fig:compare}. 
%
%
As seen in the first row of \figref{fig:compare}, 
both the original 3DGS and 2DGS suffer from overfitting, 
leading to corrupted geometry generation. 
%
Our FDS effectively mitigates this issue by 
supervising the matching relationship between 
the input and sampled views, 
helping to recover the geometry.
%
FDS also improves the refinement of geometric details, 
as shown in other rows. 
By incorporating the matching prior through FDS, 
the quality of the rendered depth is significantly improved.
%

\begin{table}[t] \centering
\begin{minipage}[t]{0.96\linewidth}
        \captionof{table}{\textbf{3D Reconstruction 
        and novel view synthesis results on Mushroom dataset. * 
        Represents that FDS uses the Raft model.
        }}
        \label{tab:mushroom}
        \resizebox{\textwidth}{!}{
\begin{tabular}{c| c c c c c | c c c c c}
    \hline
     Method &  Acc$\downarrow$ & Comp $\downarrow$ & C-L1 $\downarrow$ & NC $\uparrow$ & F-Score $\uparrow$ &  Abs Rel $\downarrow$ &  PSNR $\uparrow$  & SSIM  $\uparrow$ & LPIPS $\downarrow$ & Time  $\downarrow$ \\ \hline

    % DN-splatter &   &  &  &  &  &  &  &  & \\
    GOF &  0.1812 & 0.1093 & 0.1453 & 0.6292 & 0.3665 & 0.2380  & 21.37  &  0.7762  & 0.3132  & $\approx$1.4h\\ 
    PGSR &  0.0971 & 0.1420 & 0.1196 & 0.7193 & 0.5105 & 0.1723  & 22.13  & 0.7773  & 0.2918  & $\approx$1.2h \\ \hline
    3DGS &   0.1167 &  0.1033&  0.1100&  0.7954&  0.3739&  0.1214&  24.18&  0.8392& 0.2511 &$\approx$0.8h \\
    3DGS + FDS$^*$ & 0.0569  & 0.0676 & 0.0623 & 0.8105 & 0.6573 & 0.0603 & 24.72  & 0.8489 & 0.2379 &$\approx$1.3h \\
    3DGS + FDS & \textbf{0.0527}  & \textbf{0.0565} & \textbf{0.0546} & \textbf{0.8178} & \textbf{0.6958} & \textbf{0.0568} & \textbf{24.76}  & \textbf{0.8486} & \textbf{0.2381} &$\approx$1.3h \\ \hline
    2DGS&   0.1078&  0.0850&  0.0964&  0.7835&  0.5170&  0.1002&  23.56&  0.8166& 0.2730 &$\approx$0.8h\\
    2DGS + FDS$^*$ &  0.0689 &  0.0646& 0.0667& 0.8042& 0.6582& 0.0589& 23.98&  0.8255&0.2621 &$\approx$1.3h\\
    2DGS + FDS &  \textbf{0.0615} & \textbf{ 0.0534}& \textbf{0.0574}& \textbf{0.8151}& \textbf{0.6974}&  \textbf{0.0561}&  \textbf{24.06}&  \textbf{0.8271}&\textbf{0.2610} &$\approx$1.3h \\ \hline
\end{tabular}
}
\end{minipage}\hfill
\end{table}

\begin{table}[t] \centering
\begin{minipage}[t]{0.96\linewidth}
        \captionof{table}{\textbf{3D Reconstruction 
        and novel view synthesis results on ScanNet dataset.}}
        \label{tab:scannet}
        \resizebox{\textwidth}{!}{
\begin{tabular}{c| c c c c c | c c c c }
    \hline
     Method &  Acc $\downarrow$ & Comp $\downarrow$ & C-L1 $\downarrow$ & NC $\uparrow$ & F-Score $\uparrow$ &  Abs Rel $\downarrow$ &  PSNR $\uparrow$  & SSIM  $\uparrow$ & LPIPS $\downarrow$ \\ \hline
    GOF & 1.8671  & 0.0805 & 0.9738 & 0.5622 & 0.2526 & 0.1597  & 21.55  & 0.7575  & 0.3881 \\
    PGSR &  0.2928 & 0.5103 & 0.4015 & 0.5567 & 0.1926 & 0.1661  & 21.71 & 0.7699  & 0.3899 \\ \hline

    3DGS &  0.4867 & 0.1211 & 0.3039 & 0.7342& 0.3059 & 0.1227 & 22.19& 0.7837 & 0.3907\\
    3DGS + FDS &  \textbf{0.2458} & \textbf{0.0787} & \textbf{0.1622} & \textbf{0.7831} & 
    \textbf{0.4482} & \textbf{0.0573} & \textbf{22.83} & \textbf{0.7911} & \textbf{0.3826} \\ \hline
    2DGS &  0.2658 & 0.0845 & 0.1752 & 0.7504& 0.4464 & 0.0831 & 22.59& 0.7881 & 0.3854\\
    2DGS + FDS &  \textbf{0.1457} & \textbf{0.0679} & \textbf{0.1068} & \textbf{0.7883} & 
    \textbf{0.5459} & \textbf{0.0432} & \textbf{22.91} & \textbf{0.7928} & \textbf{0.3800} \\ \hline
\end{tabular}
}
\end{minipage}\hfill
\end{table}


\begin{table}[t] \centering
\begin{minipage}[t]{0.96\linewidth}
        \captionof{table}{\textbf{Ablation study on geometry priors.}}
        \label{tab:analysis_prior}
        \resizebox{\textwidth}{!}{
\begin{tabular}{c| c c c c c | c c c c}

    \hline
     Method &  Acc$\downarrow$ & Comp $\downarrow$ & C-L1 $\downarrow$ & NC $\uparrow$ & F-Score $\uparrow$ &  Abs Rel $\downarrow$ &  PSNR $\uparrow$  & SSIM  $\uparrow$ & LPIPS $\downarrow$ \\ \hline
    2DGS&   0.1078&  0.0850&  0.0964&  0.7835&  0.5170&  0.1002&  23.56&  0.8166& 0.2730\\
    2DGS+Depth&   0.0862&  0.0702&  0.0782&  0.8153&  0.5965&  0.0672&  23.92&  0.8227& 0.2619 \\
    2DGS+MVDepth&   0.2065&  0.0917&  0.1491&  0.7832&  0.3178&  0.0792&  23.74&  0.8193& 0.2692 \\
    2DGS+Normal&   0.0939&  0.0637&  0.0788&  \textbf{0.8359}&  0.5782&  0.0768&  23.78&  0.8197& 0.2676 \\
    2DGS+FDS &  \textbf{0.0615} & \textbf{ 0.0534}& \textbf{0.0574}& 0.8151& \textbf{0.6974}&  \textbf{0.0561}&  \textbf{24.06}&  \textbf{0.8271}&\textbf{0.2610} \\ \hline
    2DGS+Depth+FDS &  0.0561 &  0.0519& 0.0540& 0.8295& 0.7282&  0.0454&  \textbf{24.22}& \textbf{0.8291}&\textbf{0.2570} \\
    2DGS+Normal+FDS &  \textbf{0.0529} & \textbf{ 0.0450}& \textbf{0.0490}& \textbf{0.8477}& \textbf{0.7430}&  \textbf{0.0443}&  24.10&  0.8283& 0.2590 \\
    2DGS+Depth+Normal &  0.0695 & 0.0513& 0.0604& 0.8540&0.6723&  0.0523&  24.09&  0.8264&0.2575\\ \hline
    2DGS+Depth+Normal+FDS &  \textbf{0.0506} & \textbf{0.0423}& \textbf{0.0464}& \textbf{0.8598}&\textbf{0.7613}&  \textbf{0.0403}&  \textbf{24.22}& 
    \textbf{0.8300}&\textbf{0.0403}\\
    
\bottomrule
\end{tabular}
}
\end{minipage}\hfill
\end{table}




\subsubsection{Mesh extraction}
To further demonstrate the improvement in geometry quality, 
we applied methods used in ~\citep{turkulainen2024dnsplatter} 
to extract meshes from the input views of optimized 3DGS. 
The comparison results are presented  
in \tabref{tab:mushroom}. 
With the integration of FDS, the mesh quality is significantly enhanced compared to the baseline, featuring fewer floaters and more well-defined shapes.
 %
% Following the incorporation of FDS, the reconstruction 
% results exhibit fewer floaters and more well-defined 
% shapes in the meshes. 
% Visualized comparisons
% are provided in the supplementary material.

% \begin{figure}[t] \centering
%     \makebox[0.19\textwidth]{\scriptsize GT}
%     \makebox[0.19\textwidth]{\scriptsize 3DGS}
%     \makebox[0.19\textwidth]{\scriptsize 3DGS+FDS}
%     \makebox[0.19\textwidth]{\scriptsize 2DGS}
%     \makebox[0.19\textwidth]{\scriptsize 2DGS+FDS} \\

%     \includegraphics[width=0.19\textwidth]{figure/fig4_img/compare1/gt02.png}
%     \includegraphics[width=0.19\textwidth]{figure/fig4_img/compare1/baseline06.png}
%     \includegraphics[width=0.19\textwidth]{figure/fig4_img/compare1/baseline_fds05.png}
%     \includegraphics[width=0.19\textwidth]{figure/fig4_img/compare1/2dgs04.png}
%     \includegraphics[width=0.19\textwidth]{figure/fig4_img/compare1/2dgs_fds03.png} \\

%     \includegraphics[width=0.19\textwidth]{figure/fig4_img/compare2/gt00.png}
%     \includegraphics[width=0.19\textwidth]{figure/fig4_img/compare2/baseline02.png}
%     \includegraphics[width=0.19\textwidth]{figure/fig4_img/compare2/baseline_fds01.png}
%     \includegraphics[width=0.19\textwidth]{figure/fig4_img/compare2/2dgs04.png}
%     \includegraphics[width=0.19\textwidth]{figure/fig4_img/compare2/2dgs_fds03.png} \\
      
%     \includegraphics[width=0.19\textwidth]{figure/fig4_img/compare3/gt05.png}
%     \includegraphics[width=0.19\textwidth]{figure/fig4_img/compare3/3dgs03.png}
%     \includegraphics[width=0.19\textwidth]{figure/fig4_img/compare3/3dgs_fds04.png}
%     \includegraphics[width=0.19\textwidth]{figure/fig4_img/compare3/2dgs02.png}
%     \includegraphics[width=0.19\textwidth]{figure/fig4_img/compare3/2dgs_fds01.png} \\

%     \caption{\textbf{Qualitative comparison of extracted mesh 
%     on Mushroom and ScanNet datasets.}}
%     \label{fig:mesh}
% \end{figure}












\subsection{Ablation study}


\textbf{Ablation study on geometry priors:} 
To highlight the advantage of incorporating matching priors, 
we incorporated various types of priors generated by different 
models into 2DGS. These include a monocular depth estimation
model (Depth Anything v2)~\citep{yang2024depth}, a two-view depth estimation 
model (Unimatch)~\citep{xu2023unifying}, 
and a monocular normal estimation model (DSINE)~\citep{bae2024rethinking}.
We adapt the scale and shift-invariant loss in Midas~\citep{birkl2023midas} for
monocular depth supervision and L1 loss for two-view depth supervison.
%
We use Sea Raft~\citep{wang2025sea} as our default optical flow model.
%
The comparison results on Mushroom dataset 
are shown in ~\tabref{tab:analysis_prior}.
We observe that the normal prior provides accurate shape information, 
enhancing the geometric quality of the radiance field. 
%
% In contrast, the monocular depth prior slightly increases 
% the 'Abs Rel' due to its ambiguous scale and inaccurate depth ordering.
% Moreover, the performance of monocular depth estimation 
% in the sauna scene is particularly poor, 
% primarily due to the presence of numerous reflective 
% surfaces and textureless walls, which limits the accuracy of monocular depth estimation.
%
The multi-view depth prior, hindered by the limited feature overlap 
between input views, fails to offer reliable geometric 
information. We test average "Abs Rel" of multi-view depth prior
, and the result is 0.19, which performs worse than the "Abs Rel" results 
rendered by original 2DGS.
From the results, it can be seen that depth order information provided by monocular depth improves
reconstruction accuracy. Meanwhile, our FDS achieves the best performance among all the priors, 
and by integrating all
three components, we obtained the optimal results.
%
%
\begin{figure}[t] \centering
    \makebox[0.16\textwidth]{\scriptsize RF (16000 iters)}
    \makebox[0.16\textwidth]{\scriptsize RF* (20000 iters)}
    \makebox[0.16\textwidth]{\scriptsize RF (20000 iters)  }
    \makebox[0.16\textwidth]{\scriptsize PF (16000 iters)}
    \makebox[0.16\textwidth]{\scriptsize PF (20000 iters)}


    % \includegraphics[width=0.16\textwidth]{figure/fig5_img/compare1/16000.png}
    % \includegraphics[width=0.16\textwidth]{figure/fig5_img/compare1/20000_wo_flow_loss.png}
    % \includegraphics[width=0.16\textwidth]{figure/fig5_img/compare1/20000.png}
    % \includegraphics[width=0.16\textwidth]{figure/fig5_img/compare1/16000_prior.png}
    % \includegraphics[width=0.16\textwidth]{figure/fig5_img/compare1/20000_prior.png}\\

    % \includegraphics[width=0.16\textwidth]{figure/fig5_img/compare2/16000.png}
    % \includegraphics[width=0.16\textwidth]{figure/fig5_img/compare2/20000_wo_flow_loss.png}
    % \includegraphics[width=0.16\textwidth]{figure/fig5_img/compare2/20000.png}
    % \includegraphics[width=0.16\textwidth]{figure/fig5_img/compare2/16000_prior.png}
    % \includegraphics[width=0.16\textwidth]{figure/fig5_img/compare2/20000_prior.png}\\

    \includegraphics[width=0.16\textwidth]{figure/fig5_img/compare3/16000.png}
    \includegraphics[width=0.16\textwidth]{figure/fig5_img/compare3/20000_wo_flow_loss.png}
    \includegraphics[width=0.16\textwidth]{figure/fig5_img/compare3/20000.png}
    \includegraphics[width=0.16\textwidth]{figure/fig5_img/compare3/16000_prior.png}
    \includegraphics[width=0.16\textwidth]{figure/fig5_img/compare3/20000_prior.png}\\
    
    \includegraphics[width=0.16\textwidth]{figure/fig5_img/compare4/16000.png}
    \includegraphics[width=0.16\textwidth]{figure/fig5_img/compare4/20000_wo_flow_loss.png}
    \includegraphics[width=0.16\textwidth]{figure/fig5_img/compare4/20000.png}
    \includegraphics[width=0.16\textwidth]{figure/fig5_img/compare4/16000_prior.png}
    \includegraphics[width=0.16\textwidth]{figure/fig5_img/compare4/20000_prior.png}\\

    \includegraphics[width=0.30\textwidth]{figure/fig5_img/bar.png}

    \caption{\textbf{The error map of Radiance Flow and Prior Flow.} RF: Radiance Flow, PF: Prior Flow, * means that there is no FDS loss supervision during optimization.}
    \label{fig:error_map}
\end{figure}




\textbf{Ablation study on FDS: }
In this section, we present the design of our FDS 
method through an ablation study on the 
Mushroom dataset to validate its effectiveness.
%
The optional configurations of FDS are outlined in ~\tabref{tab:ablation_fds}.
Our base model is the 2DGS equipped with FDS,
and its results are shown 
in the first row. The goal of this analysis 
is to evaluate the impact 
of various strategies on FDS sampling and loss design.
%
We observe that when we 
replace $I_i$ in \eqref{equ:mflow} with $C_i$, 
as shown in the second row, the geometric quality 
of 2DGS deteriorates. Using $I_i$ instead of $C_i$ 
help us to remove the floaters in $\bm{C^s}$, which are also 
remained in $\bm{C^i}$.
We also experiment with modifying the FDS loss. For example, 
in the third row, we use the neighbor 
input view as the sampling view, and replace the 
render result of neighbor view with ground truth image of its input view.
%
Due to the significant movement between images, the Prior Flow fails to accurately 
match the pixel between them, leading to a further degradation in geometric quality.
%
Finally, we attempt to fix the sampling view 
and found that this severely damaged the geometric quality, 
indicating that random sampling is essential for the stability 
of the mean error in the Prior flow.



\begin{table}[t] \centering

\begin{minipage}[t]{1.0\linewidth}
        \captionof{table}{\textbf{Ablation study on FDS strategies.}}
        \label{tab:ablation_fds}
        \resizebox{\textwidth}{!}{
\begin{tabular}{c|c|c|c|c|c|c|c}
    \hline
    \multicolumn{2}{c|}{$\mathcal{M}_{\theta}(X, \bm{C^s})$} & \multicolumn{3}{c|}{Loss} & \multicolumn{3}{c}{Metric}  \\
    \hline
    $X=C^i$ & $X=I^i$  & Input view & Sampled view     & Fixed Sampled view        & Abs Rel $\downarrow$ & F-score $\uparrow$ & NC $\uparrow$ \\
    \hline
    & \ding{51} &     &\ding{51}    &    &    \textbf{0.0561}        &  \textbf{0.6974}         & \textbf{0.8151}\\
    \hline
     \ding{51} &           &     &\ding{51}    &    &    0.0839        &  0.6242         &0.8030\\
     &  \ding{51} &   \ding{51}  &    &    &    0.0877       & 0.6091        & 0.7614 \\
      &  \ding{51} &    &    & \ding{51}    &    0.0724           & 0.6312          & 0.8015 \\
\bottomrule
\end{tabular}
}
\end{minipage}
\end{table}




\begin{figure}[htbp] \centering
    \makebox[0.22\textwidth]{}
    \makebox[0.22\textwidth]{}
    \makebox[0.22\textwidth]{}
    \makebox[0.22\textwidth]{}
    \\

    \includegraphics[width=0.22\textwidth]{figure/fig6_img/l1/rgb/frame00096.png}
    \includegraphics[width=0.22\textwidth]{figure/fig6_img/l1/render_rgb/frame00096.png}
    \includegraphics[width=0.22\textwidth]{figure/fig6_img/l1/render_depth/frame00096.png}
    \includegraphics[width=0.22\textwidth]{figure/fig6_img/l1/depth/frame00096.png}

    % \includegraphics[width=0.22\textwidth]{figure/fig6_img/l2/rgb/frame00112.png}
    % \includegraphics[width=0.22\textwidth]{figure/fig6_img/l2/render_rgb/frame00112.png}
    % \includegraphics[width=0.22\textwidth]{figure/fig6_img/l2/render_depth/frame00112.png}
    % \includegraphics[width=0.22\textwidth]{figure/fig6_img/l2/depth/frame00112.png}

    \caption{\textbf{Limitation of FDS.} }
    \label{fig:limitation}
\end{figure}


% \begin{figure}[t] \centering
%     \makebox[0.48\textwidth]{}
%     \makebox[0.48\textwidth]{}
%     \\
%     \includegraphics[width=0.48\textwidth]{figure/loss_Ignatius.pdf}
%     \includegraphics[width=0.48\textwidth]{figure/loss_family.pdf}
%     \caption{\textbf{Comparison the photometric error of Radiance Flow and Prior Flow:} 
%     We add FDS method after 2k iteration during training.
%     The results show
%     that:  1) The Prior Flow is more precise and 
%     robust than Radiance Flow during the radiance 
%     optimization; 2) After adding the FDS loss 
%     which utilize Prior 
%     flow to supervise the Radiance Flow at 2k iterations, 
%     both flow are more accurate, which lead to
%     a mutually reinforcing effects.(TODO fix it)} 
%     \label{fig:flowcompare}
% \end{figure}






\textbf{Interpretive Experiments: }
To demonstrate the mutual refinement of two flows in our FDS, 
For each view, we sample the unobserved 
views multiple times to compute the mean error 
of both Radiance Flow and Prior Flow. 
We use Raft~\citep{teed2020raft} as our default optical flow model
for visualization.
The ground truth flow is calculated based on 
~\eref{equ:flow_pose} and ~\eref{equ:flow} 
utilizing ground truth depth in dataset.
We introduce our FDS loss after 16000 iterations during 
optimization of 2DGS.
The error maps are shown in ~\figref{fig:error_map}.
Our analysis reveals that Radiance Flow tends to 
exhibit significant geometric errors, 
whereas Prior Flow can more accurately estimate the geometry,
effectively disregarding errors introduced by floating Gaussian points. 

%





\subsection{Limitation and further work}

Firstly, our FDS faces challenges in scenes with 
significant lighting variations between different 
views, as shown in the lamp of first row in ~\figref{fig:limitation}. 
%
Incorporating exposure compensation into FDS could help address this issue. 
%
 Additionally, our method struggles with 
 reflective surfaces and motion blur,
 leading to incorrect matching. 
 %
 In the future, we plan to explore the potential 
 of FDS in monocular video reconstruction tasks, 
 using only a single input image at each time step.
 


\section{Conclusions}
In this paper, we propose Flow Distillation Sampling (FDS), which
leverages the matching prior between input views and 
sampled unobserved views from the pretrained optical flow model, to improve the geometry quality
of Gaussian radiance field. 
Our method can be applied to different approaches (3DGS and 2DGS) to enhance the geometric rendering quality of the corresponding neural radiance fields.
We apply our method to the 3DGS-based framework, 
and the geometry is enhanced on the Mushroom, ScanNet, and Replica datasets.

\section*{Acknowledgements} This work was supported by 
National Key R\&D Program of China (2023YFB3209702), 
the National Natural Science Foundation of 
China (62441204, 62472213), and Gusu 
Innovation \& Entrepreneurship Leading Talents Program (ZXL2024361)
% !TEX root = ../main.tex

\section{Related work}
\label{sec:related_work}

\subsection{Visual unsupervised anomaly localization}

% In recent years the creation of the MVTec AD benchmark~\cite{mvtec} has given impetus to the development of new methods for visual unsupervised anomaly detection and localization. We review several main approaches which have representatives among top-5 methods on the localization track of the MVTec AD leaderboard
% The MVTec AD benchmark~\cite{mvtec}, developed in recent years, has been instrumental in propelling research towards new methods in visual unsupervised anomaly detection and localization.
In this section, we review several key approaches, each represented among the top five methods on the localization track of the MVTec AD benchmark~\cite{mvtec}, developed to stir progress in visual unsupervised anomaly detection and localization. 
% \footnote{\url{https://paperswithcode.com/sota/anomaly-detection-on-mvtec-ad}}.
% \paragraph{Synthetic anomalies} In unsupervised setting, real anomalies are either not present or not labeled in the training images. Some methods~\cite{memseg,mood_top1}, however, propose synthetic procedures that corrupt random regions in the images and train a segmentation model to predict the corrupted regions' masks.

\paragraph{Synthetic anomalies.} In unsupervised settings, real anomalies are typically absent or unlabeled in training images. To simulate anomalies, researchers synthetically corrupt random regions by replacing them with noise, random patterns from a special set~\cite{memseg}, or parts of other training images~\cite{mood_top1}. A segmentation model is trained to predict binary masks of corrupted regions, providing well-calibrated anomaly scores for individual pixels. While straightforward to train, these models may overfit to synthetic anomalies and struggle with real ones.
% . Unlabeled real anomalies in training images cannot be included in the binary masks, leading the model to predict zero scores for these regions and resulting in false negatives.

% One limitation of this approach is that the models may overfit to synthetic anomalies and generalize poorly to real anomalies. Another limitation is that training images may contain real anomalies which are unlabeled and cannot be included in the training binary masks. Thus, segmentation model is trained to predict zero scores for these regions which leads to false negatives.

% \paragraph{Reconstruction-based} Reconstruction-based methods build a generative model that takes an image $x$ as input and generates its normal (anomaly-free) version $\hat{x}$. Then anomaly scores are obtained as pixel-wise reconstruction errors between $x$ and $\hat{x}$. SotA methods from this family, e.g. DRAEM~\cite{draem}, DiffusionAD~\cite{diffusionad}, POUTA~\cite{pouta}, present a combination of reconstruction-based and synthetic-based approaches. First, they train a generative model to reconstruct synthetically corrupted image regions. Then, they train a segmentation model that takes a corrupted image and its reconstructed version as input and predicts the mask of the corrupted regions.

\paragraph{Reconstruction-based.} 
% In reconstruction-based methods, anomaly scores are obtained as reconstruction errors between the input image $x$ and generated normal (anomaly-free) counterpart $\hat{x}$.
% Reconstruction-based methods build a generative model that takes an image $x$ as input and generates its normal (anomaly-free) version $\hat{x}$. Then anomaly scores are obtained as reconstruction errors between $x$ and $\hat{x}$.
Trained solely on normal images, reconstruction-based approaches~\cite{autoencoder, vae, fanogan}, poorly reconstruct anomalous regions, allowing pixel-wise or feature-wise discrepancies to serve as anomaly scores. Later generative approaches~\cite{draem, diffusionad, pouta} integrate synthetic anomalies. The limitation stemming from anomaly-free train set assumption still persists -- if anomalous images are present, the model may learn to reconstruct anomalies as well as normal regions, undermining the ability to detect anomalies through differences between $x$ and $\hat{x}$.
% Early approaches, such as Autoencoders~\cite{autoencoder} and Variational Autoencoders~\cite{vae}, are trained solely on normal images. During inference, these models poorly reconstruct anomalous regions, allowing pixel-wise squared errors ${(x - \hat{x})^2}$ to serve as anomaly scores. Methods like f-AnoGAN~\cite{fanogan} enhance this by training W-GAN~\cite{wgan} $g$ to generate normal images and an encoder $f$ to map images to the GAN's latent space, ensuring ${\hat{x} = g(f(x)) \approx x}$. Anomalies are detected using a weighted average of reconstruction errors in pixel space and discrepancies in feature maps from GAN discriminator.

% State-of-the-art methods such as DRAEM~\cite{draem}, DiffusionAD~\cite{diffusionad}, and POUTA~\cite{pouta} integrate synthetic anomalies into the reconstruction process. They first train a generative model (autoencoder / diffusion model) to reconstruct synthetically corrupted regions. Then, they train a segmentation model that takes both the corrupted image and its reconstruction as input to predict masks of the corrupted regions.

% A major limitation of reconstruction-based methods is the assumption that the training set contains only normal images. If anomalous images are present, the generative model may learn to reconstruct anomalies as well as normal regions, undermining the ability to detect anomalies through differences between $x$ and $\hat{x}$.

% The earliest methods from this family are based on Autoencoder~\cite{autoencoder} or Variational Autoencoder~\cite{vae}, which are trained on anomaly-free images. At the inference stage, when it takes an image $x$ with anomalies it is intended to badly reconstruct the anomalous regions in $\hat{x}$, so that pixel-wise squared errors $(x - \hat{x})^2$ can be used as anomaly scores.

% Another method, f-AnoGAN~\cite{fanogan} at the first step trains W-GAN~\cite{wgan}, consisting of generator $g$ and discriminator $d$, to generate anomaly-free images $x \sim g(z)$ from latent variables $z \sim \mathcal{N}(0, I)$. Then, at the second step, it trains encoder $f$ to map anomaly-free images $x$ to the GAN's latent space, s.t. $\hat{x} = g(f(x)) \approx x$. At the inference stage, when $x$ is anomalous image, generator is assumed to generate its anomaly-free version $\hat{x}$, as it is trained only on normal images. Anomaly score are then obtained as a weighted average of reconstruction errors $(x - \hat{x})^2$ in pixel space and squared differences $(\varphi_d(x) - \varphi_d(x'))^2$ between feature maps $\varphi_d(x)$ and $\varphi_d(x')$ taken intermediate layers of GAN discriminator $d$.

% The SotA reconstruction-based methods, e.g. DRAEM~\cite{draem}, DiffusionAD~\cite{diffusionad}, POUTA~\cite{pouta}, present a combination with the approach based on synthetic anomalies. First, they train a generative model, e.g. autoencoder~\cite{draem,pouta} or diffusion model~\cite{diffusionad}, to reconstruct synthetically corrupted image regions. Then, they train a segmentation model that takes a corrupted image and its reconstructed version as input and predicts the mask of the corrupted regions.

% The main limitation of reconstruction-based methods is that they assume that training set does not contain anomalous images. Otherwise, generative model may learn to reconstruct anomalous regions as well as normal ones, which does not allow to detect anomalies by comparison of $x$ and $\hat{x}$.

\paragraph{Density-based.} Density-based methods for anomaly detection model the distribution of the training image patterns. As modeling of the joint distribution of raw pixel values is infeasible, these methods usually model the marginal or conditional distribution of pixel-wise deep feature vectors.

Some methods~\cite{ttr, pni} perform a non-parametric density estimation using memory banks. More scalable flow-based methods~\cite{fastflow,cflow,msflow}, leverage normalizing flows to assign low likelihoods to anomalies. From this family, we selected MSFlow as a representative baseline, because it is simpler than PNI, and yields similar top-5 results on the MVTec AD. 


\subsection{Medical unsupervised anomaly localization}
While there's no standard benchmark for pathology localization on CT images, MOOD~\cite{mood} offers a relevant benchmark with synthetic target anomalies. Unfortunately, at the time of preparing this work, the benchmark is closed for submissions, preventing us from evaluating our method on it. We include the top-performing method from MOOD~\cite{mood_top1} in our comparison, that relies on synthetic anomalies.

Other recognized methods for anomaly localization in medical images are reconstruction-based: variants of AE / VAE~\cite{autoencoder, dylov}, f-AnoGAN~\cite{fanogan}, and diffusion-based~\cite{latent_diffusion}. These approaches highly rely on the fact that the the training set consists of normal images only. However, it is challenging and costly to collect a large dataset of CT images of normal patients. While these methods work acceptable in the domain of 2D medical images and MRI, the capabilities of the methods have not been fully explored in a more complex CT data domain. We have adapted these methods to 3D.

\paragraph{Summary}
Our findings provide significant insights into the influence of correctness, explanations, and refinement on evaluation accuracy and user trust in AI-based planners. 
In particular, the findings are three-fold: 
(1) The \textbf{correctness} of the generated plans is the most significant factor that impacts the evaluation accuracy and user trust in the planners. As the PDDL solver is more capable of generating correct plans, it achieves the highest evaluation accuracy and trust. 
(2) The \textbf{explanation} component of the LLM planner improves evaluation accuracy, as LLM+Expl achieves higher accuracy than LLM alone. Despite this improvement, LLM+Expl minimally impacts user trust. However, alternative explanation methods may influence user trust differently from the manually generated explanations used in our approach.
% On the other hand, explanations may help refine the trust of the planner to a more appropriate level by indicating planner shortcomings.
(3) The \textbf{refinement} procedure in the LLM planner does not lead to a significant improvement in evaluation accuracy; however, it exhibits a positive influence on user trust that may indicate an overtrust in some situations.
% This finding is aligned with prior works showing that iterative refinements based on user feedback would increase user trust~\cite{kunkel2019let, sebo2019don}.
Finally, the propensity-to-trust analysis identifies correctness as the primary determinant of user trust, whereas explanations provided limited improvement in scenarios where the planner's accuracy is diminished.

% In conclusion, our results indicate that the planner's correctness is the dominant factor for both evaluation accuracy and user trust. Therefore, selecting high-quality training data and optimizing the training procedure of AI-based planners to improve planning correctness is the top priority. Once the AI planner achieves a similar correctness level to traditional graph-search planners, strengthening its capability to explain and refine plans will further improve user trust compared to traditional planners.

\paragraph{Future Research} Future steps in this research include expanding user studies with larger sample sizes to improve generalizability and including additional planning problems per session for a more comprehensive evaluation. Next, we will explore alternative methods for generating plan explanations beyond manual creation to identify approaches that more effectively enhance user trust. 
Additionally, we will examine user trust by employing multiple LLM-based planners with varying levels of planning accuracy to better understand the interplay between planning correctness and user trust. 
Furthermore, we aim to enable real-time user-planner interaction, allowing users to provide feedback and refine plans collaboratively, thereby fostering a more dynamic and user-centric planning process.

% \section*{Acknowledgments}
We thank an anonymous Prolific user whose answer to \autoref{xhw_study::question::perception} inspired our paper's title.

Work on this paper was funded by the Deutsche Forschungsgemeinschaft (DFG, German Research Foundation) under Germany's Excellence Strategy---\href{https://casa.rub.de}{EXC 2092 CASA}---390781972, through the DFG grant 389792660 as part of \href{https://perspicuous-computing.science}{TRR~248}, and by the Volkswagen Foundation grants AZ~9B830, AZ~98509, and AZ~98514 \href{https://explainable-intelligent.systems}{\enquote{Explainable Intelligent Systems}} (EIS). 

The Volkswagen Foundation and the DFG had no role in preparation, review, or approval of the manuscript; or the decision to submit the manuscript for publication. 
The authors declare no other financial interests.
\section*{Impact Statement}

This paper presents work whose goal is to advance the field of Machine Learning. There are many potential societal consequences of our work, none which we feel must be specifically highlighted here.

\bibliography{main}
\bibliographystyle{icml2025}

\newpage
\appendix
\onecolumn
% This is samplepaper.tex, a sample chapter demonstrating the
% LLNCS macro package for Springer Computer Science proceedings;
% Version 2.21 of 2022/01/12
%
\documentclass[runningheads]{llncs}
%
\usepackage[T1]{fontenc}
% T1 fonts will be used to generate the final print and online PDFs,
% so please use T1 fonts in your manuscript whenever possible.
% Other font encondings may result in incorrect characters.
%
\usepackage{graphicx}
\usepackage{longtable, makecell, multirow}
\usepackage{booktabs}
\usepackage{float}
% Used for displaying a sample figure. If possible, figure files should
% be included in EPS format.
%
% If you use the hyperref package, please uncomment the following two lines
% to display URLs in blue roman font according to Springer's eBook style:
%\usepackage{color}
%\renewcommand\UrlFont{\color{blue}\rmfamily}
%
\begin{document}
%
\title{Self-Supervised Learning for Pre-training Capsule Networks: Overcoming Medical Imaging Dataset Challenges}
%
\titlerunning{Self-Supervised Learning for Pre-training Capsule Networks}
% If the paper title is too long for the running head, you can set
% an abbreviated paper title here
%
\author{Heba El-Shimy\inst{1}\thanks{\email{he12@hw.ac.uk}} \and
Hind Zantout\inst{1} \and
Michael A. Lones\inst{2} \and
Neamat El Gayar\inst{1}}
%
%\authorrunning{F. Author et al.}
% First names are abbreviated in the running head.
% If there are more than two authors, 'et al.' is used.
%
\institute{Heriot-Watt University, Dubai, UAE \and
Heriot-Watt University, Edinburgh, Scotland, UK\\}
%
\maketitle              % typeset the header of the contribution
%
\begin{abstract}
Deep learning techniques are increasingly being adopted in diagnostic medical imaging. However, the limited availability of high-quality, large-scale medical datasets presents a significant challenge, often necessitating the use of transfer learning approaches. This study investigates self-supervised learning methods for pre-training capsule networks in polyp diagnostics for colon cancer. We used the PICCOLO dataset, comprising 3,433 samples, which exemplifies typical challenges in medical datasets: small size, class imbalance, and distribution shifts between data splits. Capsule networks offer inherent interpretability due to their architecture and inter-layer information routing mechanism. However, their limited native implementation in mainstream deep learning frameworks and the lack of pre-trained versions pose a significant challenge. This is particularly true if aiming to train them on small medical datasets, where leveraging pre-trained weights as initial parameters would be beneficial. We explored two auxiliary self-supervised learning tasks---colourisation and contrastive learning---for capsule network pre-training. We compared self-supervised pre-trained models against alternative initialisation strategies. Our findings suggest that contrastive learning and in-painting techniques are suitable auxiliary tasks for self-supervised learning in the medical domain. These techniques helped guide the model to capture important visual features that are beneficial for the downstream task of polyp classification, increasing its accuracy by 5.26\% compared to other weight initialisation methods.

\keywords{Self-supervised Learning \and Deep Learning \and Diagnostic Medical Imaging \and Computer-aided Detection \and Computer-aided Diagnosis \and Capsule Networks \and Colorectal Cancer}
\end{abstract}
%
%
%
\section{Introduction}
The use of deep learning techniques in diagnostic medical imaging (DMI) is increasing, due to its superior performance compared to human physicians~\cite{khalifa-2024}. These models, which consist of billions of parameters, can capture minute details in images that may go unnoticed by the human eye. However, achieving such high levels of performance requires deep learning models to train on tens to hundreds of thousands of labelled images~\cite{shen-2017}. In the medical field, one common challenge to the wide adoption of deep learning models is the availability of such high-quality expert-annotated data, prompting the adoption of transfer learning approaches. Fine-tuning models that are pre-trained on a large, complex dataset usually offers a good initialisation for model parameters, facilitating learning from smaller datasets by subtly adjusting the layer weights to accommodate new data features. A frequently used dataset for pre-training is ImageNet~\cite{imagenet_cvpr09}, a visual concepts database comprising more than 14 million images with human-labelled annotations of objects in at least one million images. Although belonging to a different domain, ImageNet is widely used for pre-training deep learning models, often resulting in better weight initialisation compared to other strategies such as random, Xavier/Glorot~\cite{glorot-2010} or Kaiming/He~\cite{He2015DelvingDI} initialisations. Various studies have used ImageNet pre-trained models in DMI, achieving notable results irrespective of the downstream task, as evidenced by the work in~\cite{krenzer-2023}, among others. Many deep learning frameworks provide ImageNet pre-trained variants of popular architectures such as Convolutional Neural Networks (CNNs) or Visual Transformers (ViTs). However, attempting to leverage ImageNet pre-training advantages for DMI with less popular, or possibly entirely new architectures, poses a challenging, potentially costly task. For these models, pre-training on ImageNet is necessary prior to using them for the intended downstream task. 

This study investigates the use of self-supervised learning (SSL) for pre-training capsule neural network models (CapsNets). CapsNets are valued for their inherent interpretability owing to their architecture~\cite{lalonde-2020a,sabour-2017}. While other interpretable approaches exist, the unique ability of CapsNets to encode object instantiation parameters and spatial relationships is particularly valuable for DMI. However, CapsNets are under-represented as an architecture, lacking native implementations in most deep learning frameworks. Researchers utilising CapsNets often face the challenge of developing the architecture from scratch. Moreover, the quest for optimal weight initialisation through pre-training on a huge dataset like ImageNet is further complicated by CapsNets' computationally expensive training requirements. This can be an unfeasible pre-requisite, especially for researchers working with limited resources. Our contributions in this study include: (1) developing a novel CapsNet architecture for the classification of polyps in colon cancer; (2) developing two datasets of polyp frames that support different SSL auxiliary tasks, namely, colourisation and contrastive learning; and (3) evaluating the performance of our CapsNet architecture on polyp diagnosis under three experimental conditions: training from scratch, pre-training with SSL, and utilising initial layers from a ResNet pre-trained on ImageNet.

The structure of this paper is as follows: Section~\ref{sec2} reviews related work on CapsNets and SSL. Section~\ref{sec3} describes our methodology and experimental setup. Section~\ref{sec4} discusses our results. Section~\ref{sec5} provides conclusions and future work.

%
%
%
\section{Related Work}\label{sec2}

\subsection{Capsule Networks}
Capsule networks (CapsNets), as introduced by~\cite{sabour-2017}, are an advancement of CNNs, taking advantage of their powerful feature extraction capabilities by applying multiple convolutional filters to inputs. Recognising CNNs' pooling layer limitations in routing information between layers, CapsNets incorporate inverse graphics and dynamic routing to capture part-whole relationships and route information between layers. CapsNets replace individual neurons in CNNs with small groups of neurons that can encode various instantiation parameters such as position, size, orientation, albedo, hue, and texture. Primary capsules are tasked with learning these instantiation parameters for objects depicted in the inputs, effectively reversing the graphics rendering process, which typically uses such parameters for generating images from objects. Furthermore, the dynamic routing algorithm iteratively learns part-whole relationships by computing the similarity between each primary capsule and the higher-layer capsules. Capsules from lower layers that are parts of a larger whole will exhibit stronger connections to the corresponding higher-layer capsule representing that whole entity. Due to their use of inverse graphics and dynamic routing, CapsNets are considered inherently interpretable and capable of disentangling overlapping objects, as demonstrated in~\cite{sabour-2017}. In addition, they offer better explainability for their predictions, as discussed in~\cite{lalonde-2020a}.

CapsNets use margin loss, which ensures that a capsule of class $k$ is allowed to have a lengthy instantiation vector if and only if the entities associated with that class exist in the image. The total margin loss is the sum of all final layer capsule losses:
\begin{equation}
L_k = T_k \max(0, m^{+} - \|\mathbf{v}_k\|)^2 + \lambda (1 - T_k) \max(0, \|\mathbf{v}_k\| - m^{-})^2
\label{eq_margin_loss}
\end{equation}
where hyperparameters $T_k$, $m^+$, and $m^-$ are assigned the values 1, 0.9, and 0.1, respectively, in accordance with~\cite{sabour-2017}.

CapsNets may optionally incorporate a decoder that uses the output vector of the predicted class capsule to reconstruct the original input. The decoder uses mean squared error (MSE) loss, which, combined with the margin loss, form the total loss for the network. The decoder loss is usually weighted to prevent it from dominating the total loss. Additionally, it has demonstrated a regularisation effect that prevents the network from overfitting.

Several studies have successfully implemented CapsNets for DMI as summarised in~\cite{ElShimy2022ARO}. Research indicates that CapsNet surpass CNNs in computer-aided diagnostics~\cite{deepika-2022} and can be trained on small, imbalanced datasets~\cite{Afshar20203DMCNA3}. Furthermore, novel architectures have adapted CapsNet to address its limitations by improving scalability and enhancing efficiency~\cite{deepika-2022}.


\subsection{Self-Supervised Learning}
Self-supervised learning (SSL) is a subset of unsupervised learning where models are trained without labelled data. SSL employs auxiliary prediction (pre-text) tasks, which guide the model to autonomously generate task-relevant labels, effectively serving as the model's own supervision signal~\cite{krenzer-2023}. The quality of the features learnt is contingent upon the auxiliary task, with recent years witnessing several experiments refining these tasks. Among the predominant approaches is SimCLR, introduced by~\cite{chen-2020}. This technique presents the model with two pairs of the same image subjected to different augmentations. By employing contrastive loss the model learns to identify the matching pairs despite the augmentation:
\begin{equation}
\mathcal{L} = -\log \frac{\exp(sim(z_i, z_j)/\tau)}{\sum_{k \neq i}^{2N} \exp(sim(z_i, z_k)/\tau)}
\end{equation}
where $z_i$ and $z_j$ denote the encoded representations of the two augmented views of an image, $sim(z_i,z_j)$ refers to the cosine similarity between vectors $z_i$ and $z_j$, and $\tau$ is a temperature parameter. The values of $k \neq i$ exclude instances where $k=i$ from the summation to avoid comparing an embedding with itself. With $N$ representing the batch size, the expression $2N$ accounts for the doubled number of images per batch. SimCLRv2, introduced by~\cite{Chen2020BigSM} was pre-trained on ImageNet in a self-supervised manner, achieving remarkable results when fine-tuned with a few labelled examples. SimCLRv2 is commonly used in benchmarking techniques as in~\cite{caron-2021}.

Context prediction is another form of auxiliary tasks for SSL introduced in~\cite{Doersch2015UnsupervisedVR}. The model is given a pair of patches from the same image and is trained by learning to predict which of eight possible spatial relationships exist between the patches (e.g., south-east, west, north). In-painting is another auxiliary task described in~\cite{Pathak-2016}, where small patches of a few pixels are removed from images and the model learns to fill these patches by predicting the missing pixel values. Colourisation can also be used as a pretext task for SSL; the model receives greyscale images as input and learns to predict the pixel values corresponding to their coloured counterparts~\cite{Larsson2017ColorizationAA}.

SSL has proven to be an effective method for pre-training models, achieving top-1 accuracy on ImageNet (predicting the correct class with the highest confidence). Using SSL for pre-training allows models to learn rich, generalisable, and robust representations from the data, capturing important visual features. In some instances, emerging properties such as semantic segmentation of images could be learnt with SSL, as demonstrated in~\cite{caron-2021}. SSL pre-training serves as an initial step to fine-tune models for downstream tasks, employing few-shot (1-5 samples) and low-shot (10-100 samples) learning techniques~\cite{krenzer-2023}. This method could be particularly effective in the medical domain, where data availability is constrained. 

%
%
%
\section{Self-Supervised Learning as a Pre-training Technique for Capsule Networks}\label{sec3}

\subsection{Dataset}
We used the PICCOLO dataset, which consists of 3,433 frames of 74 distinct polyps collected from 40 different patients at the Hospital Universitario Basurto in Bilbao, Spain~\cite{sanchez-peralta-2020}. The dataset includes high-resolution frames captured in white light (WL) and narrow band imaging (NBI) modalities. Associated with the data set is a metadata file that contains detailed annotations for each polyp, including its initial diagnosis by a gastroenterologist, the confirmed diagnosis from a pathologist, the polyp size, classifications according to the NICE~\cite{nice-2014} and Paris~\cite{paris-update} schemes, and the histological stratification. The dataset is pre-divided into a training set with 2,203 frames, a validation set with 898 frames, and a test set with 333 frames. A notable problem with this dataset is its small size, which poses challenges to deep learning models to accurately capture complex polyp features. In addition, the dataset presents three diagnosis classes with high intraclass variability and low interclass variability, adding to the complexity. There is a substantial imbalance within the dataset: the NICE Type 2 (adenoma) class constitutes over 72\% of the training data, while NICE Type 1 (hyperplasia) and NICE Type 3 (adenocarcinoma) make up 20\% and 8\%, respectively. The validation set exhibits a different distribution, but the dominant class remains NICE Type 1 at 66\%, while Types 1 and 3 make up 15\% and 19\%, respectively---contrasting with the training set. The challenge of class imbalance is further compounded by the shift in the data distribution of the test set. Type 3 becomes the dominant class in the test set, at 38\%, followed by Type 1 at 34\%, and Type 2, previously dominant in training and validation, now the minority class at merely 28\%.


\subsection{Dataset Pre-processing and Augmentation}
We developed a pre-processing pipeline that was uniformly applied across all experiments. Initially, we removed non-informative black borders, accounting for 15-20\% of each frame's area. Any residual black pixels at the edges were replaced with the average value of all non-black pixels within the frame. Next, we resized all frames to $224\times224$ pixels. Then, a mild Gaussian blur was applied for noise reduction, followed by contrast-limited adaptive histogram equalization (CLAHE)~\cite{pisano-1998} which enhanced contrast and texture to improve the model's feature detection capability. Given that the frames include a mix of WL and NBI modalities with differing histogram characteristics, and considering the strength of NBI in highlighting critical visual features of pre-cancerous and cancerous polyps, we developed an algorithm to automatically identify WL images and perform histogram matching using the closest NBI frame of the same polyp, ensuring dataset consistency. Finally, pixel values were rescaled to values between $0$ and $1$.

We applied a set of random augmentations solely on the training set. These augmentations encompassed: 1) random cropping which retains the image size at $224\times224$ and ensures no less than 75\% of a frame is cropped, thus preserving the visual features and context of polyps; 2) random rotation up to ($\pm25^\circ$); 3) random affine transformations with a maximum of $3^\circ$ and a shear value of 7; 4) random perspective change with a distortion scale of 0.15 and probability of 0.25; 5) random vertical and horizontal flips each with a probability of 0.35; and 6) random colour jitter where only the brightness, contrast, and saturation are modified by a value of 0.2, while the hue remains unchanged to avoid the loss of crucial polyp features.


\subsection{Model Architecture}
The original CapsNet architecture proposed in~\cite{sabour-2017} represents a simple architecture intended to recognise handwritten digits from low-resolution images of $28\times28$ pixels. The hyperparameters were specifically tailored for this task; for instance, the kernel size was relatively high and the stride was large. The decoder consisted of three fully-connected layers of relatively small number of neurons. However, when applied to higher-resolution images, these hyperparameters tend to skip the finer details during feature extraction. Additionally, using the original architecture with high-resolution images becomes computationally expensive because the number of primary capsules increases quadratically, influenced by the spatial dimensions of the feature maps.

We developed a modified CapsNet architecture, shown in Fig.~\ref{fig:ModCapsNet}. This architecture incorporates five additional convolutional layers before the primary capsules layer, using smaller kernel sizes and strides. This setup enables the model to capture the fine details of polyps. In the decoder, we used several transposed convolutions instead of fully connected layers in the original architecture. We added skip connections between the encoder and decoder, but down-weighted their contribution to the decoder feature maps. This approach helps the network improve its reconstruction capability instead of simply replicating features from the encoder.

\begin{figure}[h]
    \centering
    \includegraphics[width=1\linewidth]{SSL/ModCapsNet.png}
    \caption{Modified capsule network}
    \label{fig:ModCapsNet}
\end{figure}

In our classification experiments, we substituted the margin loss in~\cite{sabour-2017} with the spread loss in a subsequent CapsNet architecture presented in~\cite{hinton-2018}:
\begin{equation}
L_i = (\max(0, m - (a_t - a_i)))^2, \quad L = \sum_{i\neq t} L_i
\end{equation}
where $m$ is the margin, $a_t$ is the activation for the target class, and $a_i$ is the activation of a wrong class.
Spread loss aims to increase the gap between the correct class and the incorrect ones while incorporating an adaptive margin mechanism. Spread loss was found to be more suitable for multi-class classification tasks with more stable training and less sensitivity to weight initialisation and hyperparameter values.

\subsection{SSL Pretext Tasks}
We decided to investigate two SSL auxiliary tasks that are suitable for the downstream task of polyp classification. These included experiments on colourisation and contrastive learning as in SimCLR~\cite{chen-2020} combined with an in-painting task. Polyps are usually categorised on the basis of visual characteristics such as colour, surface patterns, and texture. The choice of colourisation and contrastive learning with in-painting tasks was based on the intuition that these tasks could guide the model to learn the colour and texture of polyps in images, capturing useful information that would support the downstream task.

\paragraph{Colourisation}:
For this task, the model is fed a batch of greyscale images along with their corresponding coloured versions as ground truth. The objective of the model is to learn the mapping of pixel values required to produce the coloured version. We used a staged-learning approach that incorporates multiple losses throughout the training inspired by~\cite{wang-2019}. Two loss functions were selected to form the reconstruction loss: L1 loss and perceptual loss. L1 loss, or mean absolute error (MAE) was chosen over MSE used in~\cite{sabour-2017} due to its lower sensitivity to outliers. Perceptual loss computes the MSE between feature maps extracted from different layers of a small, pre-trained CNN (VGG16) for both the reconstructed images and the ground truth.

Based on promising results from initial experiments, training was carried out in two stages. For the first 30\% of epochs, we gradually introduced perceptual loss by increasing its weight with every epoch:
\begin{equation}
\lambda_{p} = \min(0.1, \frac{epoch}{30} \times 0.1)
\end{equation}
\begin{equation}
\mathcal{L}_{reconstruction} = \lambda_{p}\mathcal{L}_{perceptual} + \mathcal{L}_{1}
\end{equation}
For the remaining epochs, the losses are weighted as follows:
\begin{equation}
\mathcal{L}_{reconstruction} = 0.1 \mathcal{L}_{perceptual} + \mathcal{L}_{1}
\end{equation}

Through this colourisation task, the model learns important visual features for polyp classification, such as vessel patterns and colour characteristics~\cite{nice-2014}.

\paragraph{Contrastive learning with in-painting}:
The model processes batches containing pairs of images subjected to distinct augmentations and masked/black patches. The objective is learning to identify the corresponding image pairs within each batch through a contrastive loss function, specifically the normalised temperature-scaled cross entropy (NT-Xent) proposed by~\cite{chen-2020}. This loss function operates on image feature embeddings, minimising the distance between semantically similar images while simultaneously maximising the distance between dissimilar ones as below:

\begin{equation}
\ell_{i,j} = -\log\frac{\exp({z_i^T z_j/\tau})}{\sum_{k=1}^{2N}1_{[k\neq i]}\exp({z_i^T z_k/\tau})}
\end{equation}
where the cosine similarity between two normalised embeddings $z_i$ and $z_j$ is defined as: 
\begin{equation}
z_iT z_j = \frac{z_i^T z_j}{||z_i|| \cdot ||z_j||}
\end{equation}
The aggregate contrastive loss for all positive pairs in a batch is computed as:
\begin{equation}
\mathcal{L}_{contrastive} = \frac{1}{2N}\sum_{k=1}^N[\ell_{2k-1,2k} + \ell_{2k,2k-1}]
\end{equation}
To enhance the capability of the model of in-painting the missing regions, we incorporate the staged reconstruction loss approach used in the colourisation task. The final loss function combines both losses:
\begin{equation}
\mathcal{L}_{total} = \mathcal{L}_{contrastive} + \mathcal{L}_{reconstruction}
\end{equation}

This auxiliary task encourages the model to learn visual features such as texture, colour, and surface patterns, which are crucial for polyp classification~\cite{nice-2014}.


\subsection{Experimental Setup}
We used the AdamW optimiser~\cite{Loshchilov2017DecoupledWD} in conjunction with a cosine annealing learning rate scheduler with warm restarts~\cite{Loshchilov2016SGDRSG}. The learning rate follows a cyclical pattern, initialising at $10^{-3}$ and gradually decreasing to a minimum of $10^{-6}$ before resetting, with the period of each subsequent cycle doubling. Due to computational constraints, we utilised a batch size of 8, which is suboptimal for CapsNets to learn effectively. To address this limitation, we implemented gradient accumulation over 8 consecutive batches, effectively simulating a batch size of 64. To mitigate class imbalance, we developed a custom weighted sampler that ensures uniform class distribution within each batch during the training phase. The SSL models were trained for 200 epochs, while the supervised classification models were trained for a minimum of 50 epochs. We implemented an early stopping strategy for the supervised models based on the validation loss to prevent overfitting. All experiments were conducted on dual NVIDIA A6000 GPUs with 48GB of VRAM each. Hyperparameters were selected through extensive and systematic experimentation, optimising model performance on a validation subset of the PICCOLO dataset.

%
%
%
\section{Results and Discussion}\label{sec4}
Our CapsNet architecture was trained for the supervised classification of polyps using four weight initialisation strategies: 1) Kaiming/He initialisation for the convolutional layers with ReLU activation combined with Glorot/Xavier initialisation for both the primary capsule layer and routing weights; 2) ImageNet pre-trained ResNet weights applied to the first four convolutional layers only; 3) SSL colourise pre-trained weights applied to all layers; and 4) SSL contrastive pre-trained weights applied to all layers.

As illustrated in Fig.~\ref{fig:contrastive}, CapsNet demonstrated enhanced feature extraction capabilities when trained using SSL contrastive learning combined with the in-painting auxiliary task. While the network successfully reconstructed most missing patches, it was unable to fully capture the full vibrancy of the colours in the original image, suggesting that extended training time may be beneficial. Furthermore, CapsNet effectively learned to identify corresponding image pairs, as evidenced by the reduced distances between the matching data points in the manifold space shown in Fig.~\ref{fig:contrastive_plot}.

\begin{figure}[ht]
    \centering
    \includegraphics[width=0.45\linewidth]{SSL/contrastive.png}
    \caption{Outputs from our modified CapsNet architecture trained by combining SSL contrastive and in-painting auxiliary tasks. The top and middle rows show pairs of differently augmented versions of the same images. The bottom row demonstrates the network's in-painting capability after 200 epochs of training.}
    \label{fig:contrastive}
\end{figure}

\begin{figure}[!h]
    \centering
    \includegraphics[width=0.45\linewidth]{SSL/contrastive_plot.png}
    \caption{t-SNE visualisation of feature embeddings from our modified CapsNet model after SSL contrastive learning and in-painting auxiliary task. Each dot represents an image projected into 2D space, with matching numeric IDs and colours indicating pairs of the same image.}
    \label{fig:contrastive_plot}
\end{figure}

In our experiments with the colourisation auxiliary task, the network successfully captured and reconstructed structural features but struggled to accurately colourise the greyscale images, as shown in Fig.~\ref{fig:colourise}. This suggests that contrastive learning and in-painting are more effective as SSL auxiliary tasks. The difficulty likely stems from the complexity of simultaneously modifying pixel values across all three colour channels (red, green, blue) for relatively high-resolution $224\times224$ images. 

\begin{figure}[!h]
    \centering
    \includegraphics[width=0.45\linewidth]{SSL/colourise.png}
    \caption{Outputs of our modified CapsNet after 200 epochs of training on SSL colourise task}
    \label{fig:colourise}
\end{figure}

To evaluate the performance of our models, we used multiple metrics including accuracy, macro-averaged F1 score, specificity, and area under the receiver operating characteristic curve (AUROC). However, due to dataset imbalance, we primarily focus on Matthew's correlation coefficient (MCC) and balanced accuracy as the key evaluation metrics. Table~\ref{tab:test-performance} summarises the performance of the models on the test set.

\begin{table}
  \centering
  \caption{Performance comparison of CapsNets using different initialisation strategies, evaluated on the PICCOLO test set.}
  \begin{tabular}{>{\raggedright\arraybackslash}m{1.85cm} >{\raggedright\arraybackslash}m{1.4cm} >{\raggedright\arraybackslash}m{1.3cm} >{\raggedright\arraybackslash}m{1.8cm} >{\raggedright\arraybackslash}m{1.5cm} >{\raggedright\arraybackslash}m{1.5cm} >{\raggedright\arraybackslash}m{1.8cm}}
    \toprule
    Model & Accuracy & MCC & Balanced Accuracy & AUROC & F1-score (Macro Avg.) & Specificity \\
    \midrule
    Kaiming+ & 0.38 & 0.16 & 0.12 & 0.59 & 0.34 & 0.69 \\
    Xavier & & & & & & \\
    \addlinespace[0.6em]
    ImageNet & 0.38 & 0.25 & 0.15 & 0.71 & 0.33 & 0.69 \\
    \addlinespace[0.6em]
    SSL contrastive pre-trained & 0.40 & 0.22 & 0.16 & 0.62 & 0.34 & 0.70 \\
    \addlinespace[0.6em]
    SSL colourisation pre-trained & 0.38 & 0.00 & 0.00 & 0.52 & 0.18 & 0.69 \\
    \bottomrule
  \end{tabular}
  \label{tab:test-performance}
\end{table}

From the results, the ImageNet-pretrained and SSL-contrastive pre-trained models perform comparably. The SSL-contrastive pre-trained model has higher accuracy (0.40 vs 0.38), fractionally higher balanced accuracy (0.16 vs 0.15), and marginally higher specificity (0.70 vs 0.69) indicating good detection of negative samples. The ImageNet-pretrained model achieved higher MCC (0.25 vs 0.22) indicating stronger correlation between predictions and labels, and notably higher AUROC (0.71 vs 0.62) suggesting better classification capability. The SSL-colourisation pre-trained model is the worst performing, with results approaching random prediction (AUROC: 0.52, MCC: 0.00). Notably, ImageNet pre-trained models' performance on PICCOLO is not necessarily much higher than training from scratch. The comparable performance between SSL-contrastive and ImageNet-pretrained models is promising, considering that SSL pre-training was performed on a significantly smaller dataset than ImageNet, suggesting this technique warrants further exploration.

During our investigation of CapsNet training dynamics, we observed significant sensitivity to parameter initialisation and slower convergence rates compared to traditional architectures. Monitoring weight updates revealed minimal changes in both primary capsules and routing weights across consecutive epochs, with variations typically ranging from $10^{-6}$ to $10^{-5}$. This effect was particularly pronounced in deeper architectures, where the gradient signal weakened considerably before reaching the class capsules.

The challenge was further exacerbated during SSL, where the absence of direct classification loss signals resulted in a reduced engagement of class capsules during training. While class capsules exhibited responses to the rest of the network's learning of auxiliary tasks, these adaptations were notably subtle. This could be addressed by extending the SSL training time for CapsNets from 200 to 500 epochs.

Further analysis revealed that class capsules implicitly learnt classification-relevant features during SSL training, as evidenced by network accuracy and MCC tracking. However, we observed that the capsules also learnt the dataset imbalances, which proved detrimental to subsequent training with transfer learning. When using weights pre-trained on imbalanced data, the network required an additional 5 to 10 epochs to overcome these inherent biases. This performance degradation was more severe compared to other initialisation strategies.

%
%
%
\section{Conclusions}\label{sec5}
Our investigation of SSL for pre-training CapsNet, particularly for DMI applications, has revealed several important considerations for effective pre-training. The results indicate that CapsNets require careful optimisation strategies for SSL auxiliary tasks, due to their unique training dynamics. However, our findings demonstrate that CapsNets can slowly and gradually learn important visual features when trained using a suitable SSL auxiliary task, particularly through contrastive learning combined with an in-painting task. Our SSL-contrastive pre-training approach, despite using a significantly smaller dataset, achieved comparable results to ImageNet pre-training, even surpassing it with increases of 5.26\% in accuracy, 6.67\% in balanced accuracy and 1.45\% in specificity. The results indicate that SSL pre-training is a viable technique that could be investigated further.

Future work will explore longer training periods and implement layer-specific learning rate adjustments to better accommodate the slow learning nature of CapsNets. We plan to investigate the impact of dataset balancing during SSL pre-training to address the challenges posed by implicit learning of data imbalances. The proposed experiments aim to establish more effective SSL strategies for CapsNet, which may open up new opportunities for transfer learning in DMI applications. Ultimately, this research contributes towards improving the performance of deep learning models on limited and challenging medical datasets.


%
% ---- Bibliography ----
%
% BibTeX users should specify bibliography style 'splncs04'.
% References will then be sorted and formatted in the correct style.
%
\nocite{*}
\bibliographystyle{splncs04}
\bibliography{ssl}
\end{document}

\section{Baseline condition models}
\label{appendix:condition_models}

\paragraph{Sin-cos positional encodings.} The existing density-based UVAS methods~\cite{cflow,msflow} for natural images use standard sin-cos positional encodings for conditioning. We also employ them as an option for condition model in our framework. However, let us clarify what we mean by sin-cos positional embeddings in CT images. Note that we never apply descriptor, condition or density models to the whole CT images due to memory constraints. Instead, at all the training stages and at the inference stage of our framework we always apply them to image crops of size $H \times W \times S$, as described in Sections~\ref{subsec:descriptor_model}, \ref{subsec:density_models}. When we say that we apply sin-cos positional embeddings condition model to an image crop, we mean that compute sin-cos encodings of absolute positions of its pixels w.r.t. to the whole CT image.

\paragraph{Anatomical positional embeddings.} To implement the idea of learning the conditional distribution of image patterns at each certain anatomical region, we need a condition model producing conditions $c[p]$ that encode which anatomical region is present in the image at every position $p$. Supervised model for organs' semantic segmentation would be an ideal condition model for this purpose. However, to our best knowledge, there is no supervised models that are able to segment all organs in CT images. That is why, we decided to try the self-supervised APE~\cite{ape} model which produces continuous embeddings of anatomical position of CT image pixels.
% !TEX root = ../main.tex

\section{Density Models}
\label{appendix:density_models}

Below, we describe simple Gaussian density model and more expressive learnable Normalizing Flow model.

\textbf{Gaussian} marginal density model is written as
\begin{equation}
    -\log q_{\theta^{\text{dens}}}(y) = \frac{1}{2}(y - \mu)^\top \Sigma^{-1} (y - \mu) + \frac{1}{2}\log \det \Sigma + \text{const},
\end{equation}
where the trainable parameters $\theta^{\text{dens}}$ are mean vector $\mu$ and diagonal covariance matrix $\Sigma$.

Conditional gaussian density model is written as
\begin{equation}
    -\log q_{\theta^{\text{dens}}}(y \mid c) = \frac{1}{2}(y - \mu_{\theta^{\text{dens}}}(c))^\top \left(\Sigma_{\theta^{\text{dens}}}(c)\right)^{-1}(y - \mu_{\theta^{\text{dens}}}(c)) + \frac{1}{2}\log \det \Sigma_{\theta^{\text{dens}}}(c) + \text{const},
\end{equation}
where $\mu_{\theta^{\text{dens}}}$ and $\Sigma_{\theta^{\text{dens}}}$ are MLP nets which take condition $c \in \mathbb{R}^{d^{\text{cond}}}$ as input and predict a conditional mean vector $\mu_{\theta^{\text{dens}}}(c) \in \mathbb{R}^{d^{\text{desc}}}$ and a vector of conditional variances which is used to construct the diagonal covariance matrix $\Sigma_{\theta^{\text{dens}}}(c) \in \mathbb{R}^{d^{\text{desc}} \times d^{\text{desc}}}$. 

As described in Section~\ref{subsec:density_models}, at both training and inference stages, we need to obtain dense negative log-density maps. Dense prediction by MLP nets $\mu_{\theta^{\text{dens}}}(c)$ and $\Sigma_{\theta^{\text{dens}}}(c)$ can be implemented using convolutional layers with kernel size $1 \times 1 \times 1$. In practice, we increase this kernel size to $3 \times 3 \times 3$, which can be equivalently formulated as conditioning on locally aggregated conditions.

\textbf{Normalizing flow} model of descriptors' marginal distribution is written as:
\begin{equation}
    -\log p_{\theta^{\text{dens}}}(y) = \frac{1}{2}\|f_{\theta^{\text{dens}}}(y)\|^2 - \log \left| \det \dfrac{\partial f_{\theta^{\text{dens}}}(y)}{\partial y} \right| + \text{const},
\end{equation}
where neural net $f_\theta$ must be invertible and has a tractable jacobian determinant.

Conditional normalizing flow model of descriptors' conditional distribution is given by:
\begin{equation}
    -\log p_{\theta^{\text{dens}}}(y \mid c) = \frac{1}{2}\|f_{\theta^{\text{dens}}}(y, c)\|^2 - \log \left| \det \dfrac{\partial f_{\theta^{\text{dens}}}(y, c)}{\partial y} \right| + \text{const},
\end{equation}
where neural net $f_\theta\colon \mathbb{R}^{d^{\text{desc}}} \times \mathbb{R}^{d^{\text{cond}}} \to \mathbb{R}^{d^{\text{desc}}}$ must be invertible w.r.t. the first argument, and the second term should be tractable.

We construct $f_\theta$ by stacking Glow layers~\cite{glow}: act-norms, invertible linear transforms and affine coupling layers. Note that at both training and inference stages we apply $f_\theta$ to descriptor maps $\mathbf{y} \in \mathbb{R}^{h \times w \times s \times d^{\text{desc}}}$ in a pixel-wise manner to obtain dense negative log-density maps. In conditional model, we apply conditioning in affine coupling layers similar to~\cite{cflow} and also in each act-norm layer by predicting maps of rescaling parameters based on condition maps.
\section{Detailed Method}\label{sec:details}

\subsection{Nested lattice codebook}

In this section, we describe the construction for a Vector Quantization (VQ) codebook of size $q^d$ for quantizing an $d$-dimensional vector, where $q$ is an integer parameter. To quantize a vector, we find the closest codebook element by Euclidean norm. We describe efficient encoding and decoding algorithms to a quantized representation in $\Z_q^d$.

Let $\Lambda$ be a lattice in $\RR^d$ with generator matrix $G$. We define the coordinates of a point $x \in \Lambda$ to be an integer vector $v$ such that $x = Gv$. Each point $P \in \Lambda$ has a corresponding Voronoi region $\m{V}_\Lambda(P)$, for which $P$ is the closest point in $\Lambda$ with respect to $L^2$ metric. To define the codebook, we consider the scaled lattice $q\Lambda$. Then:

\begin{definition}
    $x \in \Lambda$ belongs to codebook $C$ iff $x \in \m{V}_{q\Lambda}(0)$. Let $v$ be the coordinates of $x$. Then, the quantized representation of $x$ is $\mathcal{Q}(x) := v \mmod q$. Note that $\mathcal{Q}$ is a bijection between $C$ and $\Z_q^d$
\end{definition}

Using this representation, we can describe the encoding and decoding functions, assuming the point $x$ we are quantizing is in $\m{V}_{q\Lambda}(0)$. We will also need an oracle $Q_{\Lambda}(x)$, which maps $x$ to the closest point in $\Lambda$ to $x$.

\begin{algorithm}[h]
   \caption{Encode}
   \label{alg:encode}
\begin{algorithmic}
   \State {\bfseries Input:} $x \in V_{q\Lambda}(0)$, $Q_{\Lambda}$
   \State $p \leftarrow Q_{\Lambda}(x)$
   \State $v \leftarrow G^{-1}p$ \Comment{coordinates of $p$}
   \State {\bfseries return} {$v \mmod q$} \Comment{quantized representation of $p$}
\end{algorithmic}
\end{algorithm}




\begin{algorithm}[h]
\caption{Decode}
\label{decode-algo}
\begin{algorithmic}
   \State {\bfseries Input:} $c \in \Z_q^d$, $Q_{\Lambda}$
   \State $p \leftarrow Gc$ \Comment{equivalent to answer modulo $q\Lambda$}
   \State {\bfseries return} $p - q\,Q_{\Lambda}\!\bigl(\tfrac{p}{q}\bigr)$
\end{algorithmic}
\end{algorithm}

In practice, we will be using the Gosset ($E_8$) lattice as $\Lambda$ with $d = 8$. This lattice is a union of $D_8$ and $D_8 + \frac{1}{2}$, where $D_8$ contains elements of $\Z^8$ with even sum of coordinates. There is a simple algorithm for finding the closest point in the Gosset lattice, first described in \cite{1056484}. We provide the pseudocode for this algorithm together with the estimation of its runtime in Appendix \ref{sec:oracle}.

\subsection{Matrix quantization}

\label{matrix-quant}

When quantizing a matrix, we normalize its rows, and quantize each block of $d$ entries using the codebook. The algorithm \ref{alg:nestquant} describes the quantization procedure for each row of the matrix.

\begin{algorithm}[h]
\caption{NestQuant}
\label{alg:nestquant}
\begin{algorithmic}
   \State {\bfseries Input:} $A$ --- a vector of size $n = db$, $q$, array of $\beta$
   \State $QA$ --- $n$ integers \Comment{quantized representation}
   \State $B$ --- $b$ integers \Comment{scaling coefficient indices}
   \State \label{norm_nestquant} $s \leftarrow \lVert A_i\rVert_2$ \Comment{normalization coefficient}
   \State $A \leftarrow \frac{A\sqrt{n}}{s}$
   \For{$j = 0$ {\bfseries to} $b-1$}
        \State $err = \infty$
        \For{$p = 1$ {\bfseries to} $k$}
            \State $v \leftarrow A[dj+1..dj+d]$
            \State $enc \leftarrow \text{Encode}\left(\frac{v}{\beta_p}\right)$
            \State $recon \leftarrow \text{Decode}(enc) \cdot \beta_p$
            \If{$err > |recon - v|_2^2$}
                \State $err \leftarrow |recon - v|_2^2$
                \State $QA[dj+1..dj+d] \leftarrow enc$
                \State $B_{j} \leftarrow p$
            \EndIf
        \EndFor
   \EndFor
   \State {\bfseries Output:} $QA$, $B$, $s$
\end{algorithmic}
\end{algorithm}

We can take dot products of quantized vectors without complete dequantization using algorithm \ref{alg:dotproduct}. We use it in the generation stage on linear layers and for querying the KV cache.

\begin{algorithm}[h]
\caption{Dot product}
\label{alg:dotproduct}
\begin{algorithmic}
   \State {\bfseries Input:} $QA_1$, $B_1$, $s_1$ and $QA_2$, $B_2$, $s_2$ --- representations of two vectors of size $n = db$ from Algorithm \ref{alg:nestquant}, array $\beta$
   \State $ans \leftarrow 0$
   \For{$j = 0$ {\bfseries to} $b-1$}
        \State $p_1 \leftarrow \text{Decode}(QA_1[dj+1..dj+d])$
        \State $p_2 \leftarrow \text{Decode}(QA_2[dj+1..dj+d])$
        \State $ans \leftarrow ans + (p_1 \cdot p_2)\beta_{B_1[j]}\beta_{B_2[j]}$
   \EndFor
   \State {\bfseries return} $ans$
\end{algorithmic}
\end{algorithm}

\subsection{LLM quantization}

\label{subsec:llm-quant}

\ifisicml
\begin{figure}
    \centering
    \includegraphics[width=\linewidth]{figures/kv.pdf}
    \caption{The quantization scheme of multi-head attention. $H$ is Hadamard rotation described in \ref{subsec:llm-quant}. $\mathcal{Q}$ is the quantization function described in \ref{matrix-quant}}
    \label{fig:scheme}
\end{figure}

\else
\begin{figure}[h]
    \centering
    \includegraphics[width=0.5\linewidth]{figures/kv.pdf}
    \caption{The quantization scheme of multi-head attention. $H$ is Hadamard rotation described in \ref{subsec:llm-quant}. $\mathcal{Q}$ is the quantization function described in \ref{matrix-quant}}
    \label{fig:scheme}
\end{figure}

\fi

Recall that we apply a rotation matrix $H$ to every weight-activation pair of a linear layer without changing the output of the network. Let $n$ be the number of input features to the layer.

\begin{itemize}
    \item If $n = 2^k$, we set $H$ to be Hadamard matrix obtained by Sylvester's construction
    \item Otherwise, we decompose $n = 2^km$, such that $m$ is small and there exists a Hadamard matrix $H_1$ of size $m$. We construct Hadamard matrix $H_2$ of size $2^k$ using Sylvester's construction, and set $U = H_1 \otimes H_2$.
\end{itemize}

Note that it's possible to multiply an $r \times n$ matrix by $H$ in $O(rn \log n)$ in the first case and $O(rn(\log n + m))$ in the second case, which is negligible to other computational costs and can be done online.

In NestQuant, we quantize all weights, activations, keys, and values using Algorithm \ref{alg:nestquant}. We merge the Hadamard rotation with the weights and quantize them. We also apply the Hadamard rotation and quantization to the activations before linear layers. We also apply rotation to keys and queries, because it will not change the attention scores, and we quantize keys and values before putting them in the KV cache. Figure \ref{fig:scheme} illustrates the procedure for multi-head attention layers.

When quantizing a weight, we modify the NestQuant algorithm by introducing corrections to unquantized weights when a certain vector piece is quantized. We refer the reader to section 4.1 of \cite{tseng2024} for a more detailed description.

\subsection{Optimal scaling coefficients}

One of the important parts of the algorithm is finding the optimal set of $\beta_i$. Given the distribution of 8-vectors that are quantized via a codebook, it is possible to find an optimal set of given size exactly using a dynamic programming approach, which is described in Appendix \ref{dp-section}.

\subsection{Algorithm summary}
\label{algo-summary}

Here we describe the main steps of NestQuant.

\begin{enumerate}
    \item Collect the statistics for LDLQ. For each linear layer with in-dimension $d$, we compute a $d \times d$ matrix $H$.
    \item We choose an initial set of scaling coefficients $\hat{\beta}$, and for each weight we simulate LDLQ quantization with these coefficients, getting a set of 8-dimensional vectors to quantize.
    \item We run a dynamic programming algorithm described in Appendix \ref{dp-section} on the 8-vectors to find the optimal $\beta$-values for each weight matrix.
    \item We also run the dynamic programming algorithm for activations, keys, and values for each layer. To get the distribution of 8-vectors, we run the model on a small set of examples.
    \item We quantize the weights using LDLQ and precomputed $\beta$.
    \item During inference, we quantize any activation before it's passed to the linear layer, and any KV cache entry before it is saved.
\end{enumerate}
Note the complete lack of fine-tuning needed to make our method work.
\newpage
\begin{figure*}[!ht]
% \vskip 0.1in
% \vskip 0.2in
\begin{center}
\centerline{\includegraphics[width=\textwidth]{figures/_reconstructions.png}}
\caption{The figure shows 2D slices of CT images (first column) alongside reconstructions and anomaly maps generated by two methods: an Autoencoder~\cite{autoencoder} (second and third columns) and f-AnoGAN~\cite{fanogan} (last two columns). Autoencoder overfits for pixel reconstruction, so it generates pathologies and fails to segment them. Also Autoencoder produces blurry generations, leading to inaccurate reconstructions of fine details and high anomaly scores on these details (e.g., vessels in the lungs). f-AnoGAN, on the other hand, avoids generating pathologies, but the generation quality still is insufficient for precise segmentation of only pathological voxels. GANs are known to be unstable and sensitive to hyperparameters, necessitating careful tuning and experimentation to achieve optimal results.}
\label{fig:reconstructions}
\end{center}
\vskip -0.2in
\end{figure*}

\end{document}
