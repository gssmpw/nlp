\section{Baseline condition models}
\label{appendix:condition_models}

\paragraph{Sin-cos positional encodings.} The existing density-based UVAS methods~\cite{cflow,msflow} for natural images use standard sin-cos positional encodings for conditioning. We also employ them as an option for condition model in our framework. However, let us clarify what we mean by sin-cos positional embeddings in CT images. Note that we never apply descriptor, condition or density models to the whole CT images due to memory constraints. Instead, at all the training stages and at the inference stage of our framework we always apply them to image crops of size $H \times W \times S$, as described in Sections~\ref{subsec:descriptor_model}, \ref{subsec:density_models}. When we say that we apply sin-cos positional embeddings condition model to an image crop, we mean that compute sin-cos encodings of absolute positions of its pixels w.r.t. to the whole CT image.

\paragraph{Anatomical positional embeddings.} To implement the idea of learning the conditional distribution of image patterns at each certain anatomical region, we need a condition model producing conditions $c[p]$ that encode which anatomical region is present in the image at every position $p$. Supervised model for organs' semantic segmentation would be an ideal condition model for this purpose. However, to our best knowledge, there is no supervised models that are able to segment all organs in CT images. That is why, we decided to try the self-supervised APE~\cite{ape} model which produces continuous embeddings of anatomical position of CT image pixels.