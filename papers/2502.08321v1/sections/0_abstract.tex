\begin{abstract}

Accurate segmentation of all pathological findings in 3D medical images remains a significant challenge, as supervised models are limited to detecting only the few pathology classes annotated in existing datasets. To address this, we frame pathology segmentation as an unsupervised visual anomaly segmentation (UVAS) problem, leveraging the inherent rarity of pathological patterns compared to healthy ones. We enhance the existing density-based UVAS framework with two key innovations: (1) dense self-supervised learning (SSL) for feature extraction, eliminating the need for supervised pre-training, and (2) learned, masking-invariant dense features as conditioning variables, replacing hand-crafted positional encodings. Trained on over 30,000 unlabeled 3D CT volumes, our model, Screener, outperforms existing UVAS methods on four large-scale test datasets comprising 1,820 scans with diverse pathologies. Code and pre-trained models will be made publicly available.

\end{abstract}