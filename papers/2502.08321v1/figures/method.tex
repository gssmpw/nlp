\begin{figure*}[!ht]
\vskip 0.1in
% \vskip 0.2in
\begin{center}
\centerline{\includegraphics[width=0.85\textwidth]{figures/_method.png}}
\caption{Illustration of Screener. First, we pre-train a self-supervised descriptor model to produce informative feature maps which are invariant to image crops and color jitter. Second, we train a self-supervised condition model in the same way as the descriptor model, but also enforcing invariance to masking of random image blocks. Thus, condition model feature maps are ignorant about anomalies and contain only the information that can be always inferred from the unmasked context. Third, density model learns the conditional distribution $p_{Y \mid C}(y \mid c)$ of feature vectors $Y = y[p]$ and $C = c[p]$ produced by descriptor and condition models at random image position $p$. To obtain a map of anomaly scores we apply density model in a pixel-wise manner, which can be efficiently implemented using $1 \times 1 \times 1$ convolutions.}
\label{fig:method}
\end{center}
\vskip -0.2in
\end{figure*}