%%
%% This is file `sample-sigconf-authordraft.tex',
%% generated with the docstrip utility.
%%
%% The original source files were:
%%
%% samples.dtx  (with options: `all,proceedings,bibtex,authordraft')
%% 
%% IMPORTANT NOTICE:
%% 
%% For the copyright see the source file.
%% 
%% Any modified versions of this file must be renamed
%% with new filenames distinct from sample-sigconf-authordraft.tex.
%% 
%% For distribution of the original source see the terms
%% for copying and modification in the file samples.dtx.
%% 
%% This generated file may be distributed as long as the
%% original source files, as listed above, are part of the
%% same distribution. (The sources need not necessarily be
%% in the same archive or directory.)
%%
%%
%% Commands for TeXCount
%TC:macro \cite [option:text,text]
%TC:macro \citep [option:text,text]
%TC:macro \citet [option:text,text]
%TC:envir table 0 1
%TC:envir table* 0 1
%TC:envir tabular [ignore] word
%TC:envir displaymath 0 word
%TC:envir math 0 word
%TC:envir comment 0 0
%%
%%
%% The first command in your LaTeX source must be the \documentclass
%% command.
%%
%% For submission and review of your manuscript please change the
%% command to \documentclass[manuscript, screen, review]{acmart}.
%%
%% When submitting camera ready or to TAPS, please change the command
%% to \documentclass[sigconf]{acmart} or whichever template is required
%% for your publication.
%%
%%
\documentclass[sigconf]{acmart}

%%
%% \BibTeX command to typeset BibTeX logo in the docs
% \AtBeginDocument{%
%   \providecommand\BibTeX{{%
%     Bib\TeX}}}

\AtBeginDocument{%
  \providecommand\BibTeX{{%
    \normalfont B\kern-0.5em{\scshape i\kern-0.25em b}\kern-0.8em\TeX}}}

%% Rights management information.  This information is sent to you
%% when you complete the rights form.  These commands have SAMPLE
%% values in them; it is your responsibility as an author to replace
%% the commands and values with those provided to you when you
%% complete the rights form.
\copyrightyear{2025}
\acmYear{2025}
\setcopyright{cc}
\setcctype{by}
\acmConference[CHI '25]{CHI Conference on Human Factors in Computing Systems}{April 26-May 1, 2025}{Yokohama, Japan}
\acmBooktitle{CHI Conference on Human Factors in Computing Systems (CHI '25), April 26-May 1, 2025, Yokohama, Japan}\acmDOI{10.1145/3706598.3714319}
\acmISBN{979-8-4007-1394-1/25/04}

%% These commands are for a PROCEEDINGS abstract or paper.
\begin{CCSXML}
<ccs2012>
   <concept>
       <concept_id>10003120.10003121.10011748</concept_id>
       <concept_desc>Human-centered computing~Empirical studies in HCI</concept_desc>
       <concept_significance>500</concept_significance>
       </concept>
   <concept>
       <concept_id>10003120.10003121.10003122.10003334</concept_id>
       <concept_desc>Human-centered computing~User studies</concept_desc>
       <concept_significance>500</concept_significance>
       </concept>
   <concept>
       <concept_id>10003120.10003121.10003129.10010885</concept_id>
       <concept_desc>Human-centered computing~User interface management systems</concept_desc>
       <concept_significance>500</concept_significance>
       </concept>
   <concept>
       <concept_id>10010405.10010497.10010510.10010513</concept_id>
       <concept_desc>Applied computing~Annotation</concept_desc>
       <concept_significance>500</concept_significance>
       </concept>
   <concept>
       <concept_id>10010147.10010178.10010179</concept_id>
       <concept_desc>Computing methodologies~Natural language processing</concept_desc>
       <concept_significance>500</concept_significance>
       </concept>
   <concept>
       <concept_id>10003120.10003121.10003129.10011756</concept_id>
       <concept_desc>Human-centered computing~User interface programming</concept_desc>
       <concept_significance>500</concept_significance>
       </concept>
 </ccs2012>
\end{CCSXML}

\ccsdesc[500]{Human-centered computing~Empirical studies in HCI}
\ccsdesc[500]{Human-centered computing~User studies}
\ccsdesc[500]{Human-centered computing~User interface management systems}
\ccsdesc[500]{Applied computing~Annotation}
\ccsdesc[500]{Computing methodologies~Natural language processing}
\ccsdesc[500]{Human-centered computing~User interface programming}

% end user programming
% keyword: Data Annotation, Language Model, Iterative labeling, data labeling
\keywords{Data Annotation; Data Labeling; Large Language Model; Iterative labeling; End-User Programming}
%%
%%  Uncomment \acmBooktitle if the title of the proceedings is different
%%  from ``Proceedings of ...''!
%%
%%\acmBooktitle{Woodstock '18: ACM Symposium on Neural Gaze Detection,
%%  June 03--05, 2018, Woodstock, NY}


%%
%% Submission ID.
%% Use this when submitting an article to a sponsored event. You'll
%% receive a unique submission ID from the organizers
%% of the event, and this ID should be used as the parameter to this command.
% \acmSubmissionID{4469}

%%
%% For managing citations, it is recommended to use bibliography
%% files in BibTeX format.
%%
%% You can then either use BibTeX with the ACM-Reference-Format style,
%% or BibLaTeX with the acmnumeric or acmauthoryear sytles, that include
%% support for advanced citation of software artefact from the
%% biblatex-software package, also separately available on CTAN.
%%
%% Look at the sample-*-biblatex.tex files for templates showcasing
%% the biblatex styles.
%%

%%
%% The majority of ACM publications use numbered citations and
%% references.  The command \citestyle{authoryear} switches to the
%% "author year" style.
%%
%% If you are preparing content for an event
%% sponsored by ACM SIGGRAPH, you must use the "author year" style of
%% citations and references.
%% Uncommenting
%% the next command will enable that style.
%%\citestyle{acmauthoryear}


%%
%% end of the preamble, start of the body of the document source.
%%%%%%%%%%%%%%%%%%%%%%%%%%%%%%%%%%%%%%%%%%%%%%%%%%%%%%%
%%%%%%%%%%%%%%%    theorems %%%%%%%%%%%%%%%%%%%%%%%%%%%
%%%%%%%%%%%%%%%%%%%%%%%%%%%%%%%%%%%%%%%%%%%%%%%%%%%%%%%
% \usepackage{mdframed}
\usepackage{kantlipsum}

%%%%%%%%%%%%%%%%%%%%%%%%%%%%%%%%%%%%%%%%%%%%%%%%%%%%%%%
%%%%%%%%%%%%%%%    theorems %%%%%%%%%%%%%%%%%%%%%%%%%%%
%%%%%%%%%%%%%%%%%%%%%%%%%%%%%%%%%%%%%%%%%%%%%%%%%%%%%%%
\theoremstyle{plain}
\newtheorem{theorem}{Theorem}[section]
\newtheorem{proposition}[theorem]{Proposition}
\newtheorem{lemma}[theorem]{Lemma}
\newtheorem{example}[theorem]{Example}
\newtheorem{corollary}[theorem]{Corollary}
\theoremstyle{definition}
\newtheorem{definition}[theorem]{Definition}
\newtheorem{assumption}[theorem]{Assumption}
\theoremstyle{remark}
\newtheorem{remark}[theorem]{Remark}


% \titleformat{\subsection}[runin]% runin puts it in the same paragraph
%        {\normalfont\bfseries}% formatting commands to apply to the whole heading
%        {\thesubsection}% the label and number
%        {0.5em}% space between label/number and subsection title
%        {}% formatting commands applied just to subsection title
%        [.]% punctuation or other commands following subsection title


%%%%%%%%%%%%%%%%%%%%%%%%%%%%%%%%%%%%%%%%%%%%%%%%%%%%%%%
%%%%%%%%%%%%%%%  mathematical notations%%%%%%%%%%%%%%%%
% \usepackage[english]{babel}
% \usepackage[utf8]{inputenc}
% \usepackage[T1]{fontenc}

%% Figures, tables and lists
\usepackage[dvipsnames]{xcolor}
\usepackage{paralist}
\usepackage{graphicx}
\usepackage{subcaption}
\usepackage{longtable} 
\usepackage{multirow}
\usepackage{listings}
\usepackage{makecell}
\usepackage{array}
\usepackage{float}
\usepackage{dsfont}
\usepackage{rotating}
\usepackage{booktabs}
\usepackage{enumerate}
\usepackage{tikz}
\usepackage{pgf}
\usepackage{enumitem}
\usepackage{lipsum} % for generating filler text
\usepackage{titlesec}

%% Math
% \usepackage{amssymb, amsthm,bbm}
\usepackage{mathtools}
\usepackage{mathrsfs}
%% References and author info 
\mathtoolsset{showonlyrefs}
\usepackage{natbib}
\usepackage{authblk}
\usepackage{todonotes}
\usepackage{xr-hyper}


%%%%%%%%%%%%%%%%%%%%%%%%%%%%%%%%%%%%%%%%%%%%%%%%%%%%%%%
\newcommand{\R}{\mathbb R}
\newcommand{\EE}{\mathbb{E}}

\DeclareMathOperator{\Tr}{Tr}
\DeclareMathOperator*{\argmin}{argmin}
\DeclareMathOperator*{\argmax}{argmax}

\newcommand{\bs}[1]{\ensuremath{\boldsymbol{#1}}}
\newcommand{\mc}{\mathcal}
\newcommand{\opt}{^\star}


\newcommand{\diff}{\textnormal{d}}


\def \iid {\stackrel{\textnormal{i.i.d.}}{\sim}}
\def \iidtext {\textnormal{i.i.d.}}





%%%%%%%%%%%%%%%%%%%%%%%%%%%%%%%%%%%%%%%%%%%%%%%%%%%%%%%
%%%%%%%%%%%%%%%%%%%%% colors     %%%%%%%%%%%%%%%%%%%%%%
%%%%%%%%%%%%%%%%%%%%%%%%%%%%%%%%%%%%%%%%%%%%%%%%%%%%%%%
\definecolor{myblue}{rgb}{.8, .8, 1}
\definecolor{mathblue}{rgb}{0.2472, 0.24, 0.6} % mathematica's Color[1, 1--3]
\definecolor{mathred}{rgb}{0.6, 0.24, 0.442893}
\definecolor{mathyellow}{rgb}{0.6, 0.547014, 0.24}


% May add more in future.






\begin{document}

%%
%% The "title" command has an optional parameter,
%% allowing the author to define a "short title" to be used in page headers.

%%
%% The "author" command and its associated commands are used to define
%% the authors and their affiliations.
%% Of note is the shared affiliation of the first two authors, and the
%% "authornote" and "authornotemark" commands
%% used to denote shared contribution to the research.

%%
%% By default, the full list of authors will be used in the page
%% headers. Often, this list is too long, and will overlap
%% other information printed in the page headers. This command allows
%% the author to define a more concise list
%% of authors' names for this purpose.
\author{Zeyu He}
%\authornote{Both authors contributed equally to this research.}
\affiliation{%
  \institution{The Pennsylvania State University}
  %\streetaddress{P.O. Box 1212}
  \city{University Park}
  \state{PA}
  \country{USA}
  %\postcode{43017-6221}
}
\email{zmh5268@psu.edu}

\author{Saniya Naphade}
%\authornote{Both authors contributed equally to this research.}
\affiliation{%
  \institution{GumGum Inc.}
  %\streetaddress{P.O. Box 1212}
  \city{Tempe}
  \state{AZ}
  \country{USA}
  %\postcode{43017-6221}
}
\email{saniya.naphade@gumgum.com}

\author{Ting-Hao `Kenneth' Huang}
\affiliation{%
  \institution{The Pennsylvania State University}
  %\streetaddress{P.O. Box 1212}
  \city{University Park}
  \state{PA}
  \country{USA}
  %\postcode{43017-6221}
}
\email{txh710@psu.edu}

\renewcommand{\shortauthors}{He et al.}

%%
%% The abstract is a short summary of the work to be presented in the
%% article.
\begin{abstract}
Out-of-distribution (OOD) detection and OOD generalization are widely studied in Deep Neural Networks (DNNs), yet their relationship remains poorly understood. We empirically show that the degree of Neural Collapse (NC) in a network layer is inversely related with these objectives: stronger NC improves OOD detection but degrades generalization, while weaker NC enhances generalization at the cost of detection. This trade-off suggests that a single feature space cannot simultaneously achieve both tasks. To address this, we develop a theoretical framework linking NC to OOD detection and generalization. We show that entropy regularization mitigates NC to improve generalization, while a fixed Simplex Equiangular Tight Frame (ETF) projector enforces NC for better detection. Based on these insights, we propose a method to control NC at different DNN layers. In experiments, our method excels at both tasks across OOD datasets and DNN architectures. 

\begin{comment}   

Out-of-distribution (OOD) detection and OOD generalization are critical for deploying machine learning models in real-world scenarios. While substantial progress has been made in addressing these problems independently, few works have attempted to tackle them jointly. However, existing methods often rely on auxiliary OOD training data and primarily focus on covariate-shifted OOD data that share labels with in-distribution (ID) data. In contrast, we tackle the more realistic and challenging task of jointly detecting and generalizing to semantic OOD data with disjoint labels from the ID data, without auxiliary OOD training data.
Achieving both objectives simultaneously is inherently difficult due to a fundamental conflict — OOD generalization requires enhanced transferability, while OOD detection necessitates the inhibition of transfer.
To address this, we leverage insights from neural collapse (NC) — a phenomenon in deep networks where top-layer representations suppress feature variability and adopt a Simplex Equiangular Tight Frame (ETF) structure, impairing transferability. By controlling NC, we unify OOD detection and generalization: preventing NC enhances OOD transfer while inducing NC improves OOD detection.
Our proposed method excels at both tasks across various OOD datasets and architectures. 

\end{comment}


\end{abstract}

%%
%% The code below is generated by the tool at http://dl.acm.org/ccs.cfm.
%% Please copy and paste the code instead of the example below.
%%


%%
%% This command processes the author and affiliation and title
%% information and builds the first part of the formatted document.
\maketitle

\section{Introduction}\label{sec:intro}
\section{Introduction}

% State of the world (robots for creative activites)
The term ``robot,'' originally signifying `forced labor,' has long been associated with labor and work. Robots have demonstrated their utility in various automated productive and social contexts, where the primary goals are improving productivity, safety, and fostering social interactions with humans~\cite{simoes2022designing, weidemann2021role, honig2018understanding}. However, an increasing number of cases feature using of robots in creative settings. Unlike productive contexts, where the focus is on efficiency and task completion~\cite{arents2022smart}, or social contexts, where communication and trust are prioritized~\cite{nam2020trust, saunderson2019robots}, creative environments prioritize artistic innovation and expression~\cite{hsueh2024counts}. This shift fundamentally alters the dynamics of human-robot interaction, redefining the roles and expectations for both humans and robots.

For instance, robots’ social behaviors are leveraged to support the generation and expression of creative ideas~\cite{hu2021exploring, sandoval2022human, alves2020creativity}, and programmable robotic movements and trajectories are employed to inspire artistic activities such as sketching~\cite{lin2020your}. These studies often engage participants from creative fields who possess limited prior experience with robotics, and are typically conducted in short-term, experimental settings. Consequently, the findings from these studies remain constrained since much can be learned from professional practitioners' experiences to inform system design such as digital fabrication~\cite{hirsch2023nothing}. There is a notable gap in research examining the long-term, active, and practical experience of integrating robotic systems into the creative processes. As a result, the deeper insights into how robots facilitate and shape creative processes, beyond simply augmenting human creativity, remain underexplored. In this study, we aim to better understand the impacts of robots on creative processes and outcomes.

As early as Leonardo da Vinci's 16th century ``Automaton,'' artists have explored the creative affordances of robotic systems~\cite{shanken2002cybernetics, pagliarini2009development, jeon2017robotic}. The artistic creation process typically encompasses various stages, including the exploration of materials and techniques, ongoing experimentation and iteration, and the continual refinement of the artists' insights into their creative subjects~\cite{lewis2023art, sturdee2022state}. Therefore, investigating the artistic process involving robots offers an opportunity to gain deeper insights into robots' creative potential. Robotic art, in particular, provides a compelling case for this exploration.

We define robotic art as artworks that utilize robotic or automated machines to create artistic experiences and tangible artifacts. One example is robotic installation art, in which robots are programmed to follow specific rules that embody the artist’s expression (\autoref{fig:teaser} (a)). Another example is responsive art, in which robots react to their environment, with behaviors that change over time or in response to spectators (\autoref{fig:teaser} (b)). Additionally, there are robotic creators, which possess a degree of agency, allowing them to collaborate with human artists and produce works that extend beyond mere replication of human-created art (\autoref{fig:teaser} (c) and (d)). As such, robotic art becomes a rich case for exploring human-machine interactions in creative contexts. Gaining a deeper understanding of how robots facilitate artistic expression can provide insights for designing computing systems to support creative activities~\cite{gomez2021robot}.

% Therefore, we did...
We draw on semi-structured, in-depth interviews with renowned professional robotic artists to investigate the use of robots in artistic practice. Specifically, our goal is to understand how artistic exploration of robotic systems challenges conventional assumptions about the functions of robots, such as their roles in automating repetitive tasks or serving human needs. We also explore the implications of robots in the artistic process and examine how creativity may emerge within robotic art. To address these interrelated inquiries, our study focuses on the practice of robotic art, posing the research question: \textit{How do robotic artists utilize robots in their artistic practice?} We approach this inquiry through the perspectives and experiences of robotic artists, who creatively design, modify, and repurpose robotic systems for artistic expression and exploration.

% The key findings are...
Our findings highlight the social, material, and temporal dimensions of artists' practices that shape their creativity and artistic outcomes. The creation of robotic art is largely a social process, as artists receive both explicit and implicit feedback through the audience's reactions and reception of their work. Simultaneously, the embodiment and malfunctions inherent to robotic systems drive artistic experimentation. The temporal processes of creation and exhibition, beyond just the final product, further enhance the creative value. Our empirical analysis presents how creativity emerges through the interplay of social, material, and temporal interactions among artists, robots, audiences, and the environment.

% The contributions of this work are...
We make two main contributions to HCI in this study. 
First, we elucidate the interactive mechanisms among key actors---human creators, machines, audiences, and environments---within the practice of robotic art, a topic that remains underexplored in HCI. Our findings reveal the significance of sociality (e.g., interactions between artists and audiences), materiality (e.g., the embodiment and malfunctions of robots), and temporality (e.g., the processes of creation and exhibition) in shaping creative values. We propose that these three facets are central to the creative process and facilitate the emergence of creativity in robotic art.
Second, drawing from the findings, we offer implications for \textit{socially informed}, \textit{material-attentive}, and \textit{process-oriented} creation with computing systems. We suggest leveraging these three aspects to enhance creativity and the creative experience. Specifically, we discuss the value of incorporating implicit audience feedback, designing with technical malfunctions, and focusing on the post-creation process to foster alternative creative experiences with machines~\cite{alter2010designing, juarez2022glitch}.




\section{Related Work}
%\subsection{End-User Programming}

%\kenneth{The way I like to think about Related Work is that this section should (sometimes subtly, not explicitly, but effectively!) answer some underlying questions that reviewers might want to ask. So, here we go:}\steven{sounds good!}

\subsection{Ways of Optimizing Prompts for LLMs}
%\subsection{Prompt Engineering and How Good Humans Are at It}
Prompts are the primary means by which users interact with, utilize, and instruct LLMs. 
Since the emergence of these models, researchers and developers have invested significant effort into understanding how to craft better prompts for more effective use. 

\paragraph{Automatic Prompt Optimization.}
Much of the prior work has focused on automatically optimizing prompts. 
A common theme across these studies is the use of gold-standard labels to guide the optimization process.
For example, \citet{pryzant2023automatic} introduced an automatic prompt optimization method inspired by gradient descent; 
\citet{manas2024improving} presented an approach that begins with a user prompt and iteratively generates revised prompts to maximize consistency between the generated image and prompt, without human intervention; 
\citet{wan2024teach} explored two types of prompt optimization, instruction and exemplar, and suggested that combining both can yield optimal results; 
\citet{sun2023autohint} combined zero-shot and few-shot learning to optimize prompts automatically; %eliminating the need for manual prompt engineering; 
and \citet{levi2024intent} improved prompt optimization through synthetic data generation and iterative refinement, focusing on aligning prompts with user intent by creating challenging boundary cases for iterative prompt refinement.
While these studies were interesting and relevant, they generally assumed the availability of gold-standard labels and did not address situations where labels are absent or where standards are constantly evolving.

\paragraph{User-Driven Prompt Optimization.}
In addition to automatic prompt optimization, some research has focused on human capabilities in optimizing prompts. 
\citet{zhou2023revisiting} found that manual prompting often outperforms automated methods in various scenarios; 
\citet{10.1145/3544548.3581388} discovered that people tend to design prompts opportunistically rather than systematically, which often leads to lower success rates. 
To the best of our knowledge, the most relevant prior work is by \citet{wang2024end}, who developed an iterative refinement system that enables users to prompt LLMs to build a personalized classifier for social media content. 
Their study explored three user strategies for improving prompts and measured their effectiveness. 
While conceptually related to our work, their focus was not on how users evolve their understanding and expectations when interacting with LLMs. 
Instead, participants labeled ground truth at the beginning of the study, prior to using the system.



%--------------------- dead kitten --------------
\begin{comment}
 





The most relevant prior work is by \citet{wang2024end}, who developed an iterative refinement system allowing users to prompt LLMs to build a personalized classifier for social media content.
While their work is closely related to ours in concept, their study did not focus on how users evolve their understanding and expectations while working with LLMs. 
Instead, participants labeled ground truth at the outset before using the system.


\kenneth{The key question for our paper is this: Did prior work try to measure users' prompt engineering performance *over multiple iterations*? (What do we know about human performance in prompt engineering?) I think you can maybe find some papers, especially papers for automatic prompt optimization like DSPy, measuring users' individual prompt's output accuracy (or MSE) or performance (e.g., BLEU in generation task), but it might be hard to find papers capture and measure *multiple iterations* from the same user for the same prompt.--This is the main argument for our paper: we did something that was hard and thus has not been done.}

\kenneth{Take a look at this survey paper:~\cite{chen2023unleashing}}



\steven{iterative tool involve human}
PromptIDE is an interactive tool that helps the experts to iteratively refine tools by providing various prompts, visualizing their performance on small validation datasets, and iterative optimizing them based on quantitative feedback~\cite{strobelt2022interactive}. \steven{gold label exists}

PromptAID is a visual analytics system that helps non-experts iteratively improve prompts through exploration, perturbation, testing, and refinement. It supports prompts through keyword adjustment, paraphrasing, and adding few-shot examples. \steven{has test dataset, it is a complex system}

\steven{automate prompting}
\citet{pryzant2023automatic} introduces an automatic prompt optimization prompt inspired by gradient descent. \steven{this fell into software designing, involve gold labels}

The study starts from a user prompt and iteratively generates revised prompts with the goal of maximizing a consistency score between the generated image and prompt without a human in the loop\cite{manas2024improving}\steven{without human involvement in the loop, gold labels}

\citet{zhou2023revisiting} found that manual prompting often performed better than automated methods in various steps. 

\cite{wan2024teach} explores the distinction between two types of prompt optimization: instruction optimizer and exemplar. This study suggested combining both approaches could lead to optimal results.

\cite{sun2023autohint} combines zero-shot and few-shot learning to optimize prompts automatically, without manual efforts in prompt engineering.

\cite{levi2024intent} improve prompt engineering optimization by synthetic data generation and iterative refinement, focusing on aligning prompts with user intent by generating challenging boundary cases and using these to refine the prompt iteratively.





\paragraph{Prompt Engineering Tools.}
\kenneth{After making the first point, we can have a follow-up paragraph to say that many tools were created to help people do prompt engineering (list a few and name their focuses), but again, they did not focus on measuring how good humans are in prompt engineering--- Of course, there could be an argument that suggests: no matter how good you are, you will always need some tool. It is true---for example, ChainForge basically create a easy-to-use UI that make things easier, which is not really about accuracy---But for annotation tasks, performance is still critical and it is always good to know how well human did, almost like many AI leaderboard has various "human" performance for comparison.}
PromptMaker, a platform for rapidly prototyping new ML models using prompt-based programming, was difficult to evaluate their prompts systematically~\cite{10.1145/3491101.3503564}.

\cite{arawjo2024chainforge}  is an Open-source visual toolkit for prompt engineering and on-demand hypothesis testing of text-generation LLMs.

 promptfoo is test-driven LLM development, not trial-and-error, producing matrix views that let you quickly evaluate outputs across many prompts~\cite{webster2023promptfoo}.

\cite{madaan2024self} introduces a method that LLM iterative improve their output by using their own feedback, without external supervision. 

\saniya{austin etal points:
1. used only COPRO, evaluation criteria utilized a custom LLM-as-a-judge metric. The paper showed that their automated prompt optimizer worked better tha DSPy }
   
\end{comment}


\subsection{Tools for Prompt Engineering}
With the advances in LLMs, numerous tools have been developed to assist with prompt engineering. 
Most of these tools follow a software-engineering paradigm, where testing (such as unit tests or integration tests) is a central concept, and thus often assume the existence of gold-standard labels.
For example, PromptIDE is an interactive tool that helps experts iteratively refine prompts by providing various inputs, visualizing their performance on small validation datasets, and optimizing them based on quantitative feedback~\cite{strobelt2022interactive}; 
PromptAid is a visual analytics system for interactively creating, refining, testing, and iterating prompts while tracking accuracy changes~\cite{mishra2023promptaid};
%It allows users to adjust prompts through keyword modifications, paraphrasing, and adding few-shot examples; 
ChainForge is an open-source visual toolkit for prompt engineering and on-demand hypothesis testing of text-generation LLMs~\cite{arawjo2024chainforge};
and, promptfoo applies a test-driven approach to LLM development, producing matrix views that enable quick evaluation of outputs across multiple prompts~\cite{webster2023promptfoo}.
While these tools are inspiring and valuable, the scenarios we focus on do not rely on the constant availability of gold labels.

%\cite{mishra2023promptaid}


\begin{comment}






\kenneth{In here, we want to answer this questions: Why do we need to built \system? Can't we just use some existing tools??? The underlying answer could be: all the tools, including the one we mentioned in previous subsection, were not really aiming for ``general users'' and only thing general users can reliably use is probably chat interface come with ChatGPT etc.}

\citet{10.1145/3544548.3581388} mentioned that people tended to design prompts opportunistically, not systematically, which resulted in less success. \system provides a systematic process for composing and refining prompts, allowing non-expert users to adapt to the prompt creation process effortlessly.

\saniya{Amy Zhang points:
\newline 1. Accuracy didnot improve; reported improvements in recall
\newline 2. Observed that humans are pretty bad at being consistent
\newline 3. Quoted  Miles Turpin, Julian Michael, Ethan Perez, and Samuel Bowman. 2024. Language models don't always say what they think: unfaithful explanations
in chain-of-thought prompting. Advances in Neural Information Processing Systems 36 (2024).
Han Wang, Ming Shan Hee, Md Rabiul Awal, Kenny Tsu Wei Choo, and Roy Ka-Wei Lee. 2023. Evaluating GPT-3 Generated Explanations for
Hateful Content Moderation. arXiv:2305.17680 [cs.CL] for not using LLM prompt explanations
\newline 4. They had a bigger training set of around 700 examples: paper excerpt: "This process resulted in a balanced dataset of 800 comments. We randomly divided our dataset into a training dataset and a test dataset of 100 examples for each participant. The training dataset was used to help participants create their classiiers, whereas the test dataset was labeled by participants and used to evaluate their created classiiers."
}
    
\end{comment}

\subsection{Human-LLM Collaborative Data Annotation}
%Another relevant area of research involves using LLMs for data annotation. 
Beyond simply treating LLMs as automatic labelers---common in countless NLP projects~\cite{tan2024large}---a growing body of work explores how to combine human and LLM efforts to achieve better annotation outcomes, such as improved accuracy or speed.
Even as LLMs outperform humans in many labeling tasks, human-AI collaboration often produces better results than either alone~\cite{vaccaro2024combinations}.
For example, \citet{kim2024meganno+} introduced a human-LLM collaborative annotation system where LLMs handle bulk annotation tasks, while humans selectively verify and refine the annotations. 
%\steven{However, this system was limited to deployment within Jupyter Notebook, lacking an end-to-end solution. This design imposed significant barriers, as it required users to possess technical expertise for system setup before using the tool, limiting accessibility and scalability in non-technical domains.}
\citet{goel2023llms} proposed an approach that combines LLMs with human expertise to efficiently generate ground truth labels for medical text annotation.
Additionally, \citet{10.1145/3613904.3642834} demonstrated how aggregating crowd workers' labels with GPT-4's output can achieve higher labeling accuracy than either source alone.
These studies generally aim to split the workflow of data labeling between humans and LLMs in a smart way, making the task more effective or efficient. 

In contrast, our work does not focus on dividing or combining the workload, but on how humans can teach LLMs---through prompt refinement---to better label the specific type of data.
Few prior studies have emphasized iterative prompt refinement in human-LLM collaborative data annotation.
For example, \citet{liu2024harnessing} developed a workflow for video content analysis, refining prompts to improve LLM-generated annotations and align them with human judgment.
Additionally, \citet{zhang2023llmaaa} proposed LLMAAA, which uses LLMs as annotators in a feedback loop to label data efficiently.
Their study shows that poorly designed prompts result in subpar performance, especially in complex tasks. %while incorporating demonstrations and aligning label descriptions with natural language significantly enhances accuracy and reliability.
Our work advances this relatively understudied area of human-LLM collaborative annotation research.

%----------------------------- dead kitten --------------------------------

\begin{comment}








\steven{\citet{vaccaro2024combinations} emphaized that designing innovative processes for integrating humans and AI is as critical as developing advanced AI technologies. This aligns with the need for LLM-powered systems that iteratively guide AI outputs to meet user-specific standards, prioritizing effective collaboration between users and AI systems.}

\steven{\citet{liyanage2024gpt} found that GPT-4, using few-shot, zero-shot, and Chain-of-Thoughts (CoT) prompting techniques, could not outperform models fine-tuned on human-labeled data. Among these, the few-shot approach exhibited the highest degree of similarity to human annotations. However, in scenarios where gold labels are unavailable, fine-tuning is not applicable, and alternative methods must be explored.}

\steven{\citet{liu2024harnessing} developed a workflow for video content analysis, iteratively crafting prompts to enhance LLMs' ability to generate structured annotations and comprehensive explanations that aligned with human judgment. }

\steven{\citet{zamfirescu2023herding} found that while prompts can effectively address most UX goals, they struggle with nuanced, edge-case, or spontaneous interactions. The study highlights that the effectiveness of each instruction in the prompt is highly sensitive to its phrasing and location. Additionally, highly prescriptive prompts, though reliable, limited the spontaneity and flexibility of GPT responses.
In our system, users are only required to provide task information—such as task descriptions, rules, and examples—to construct instructions, allowing for greater flexibility in accommodating diverse task requirements..}

\steven{\citet{guyre2024prompt} illustrates how prompt engineering can empower non-experts to design tailored conversational agents by iteratively refining prompts and infusing domain-specific knowledge. Their study emphasizes democratizing chatbot development, allowing users to align AI behavior with their specific goals and values.}

\steven{\citet{zhang2023llmaaa} proposes LLMAAA that leverages LLMs as Active Annotators in a feedback loop to efficiently annotate data. The study highlights that poorly designed prompts lead to suboptimal performance by LLM annotators, particularly in complex or domain-specific tasks. However, incorporating demonstrations and aligning label descriptions with natural language significantly enhances annotation accuracy and reliability.}

%\kenneth{Here, we then answer this question: Did people create ANYTHING to support LLM-powered data annotation? There are two parts of the answer to this: 1) Many or even most papers, including our CHI paper last year, focus on the labeling performance of LLMs, for example, as compared to crowdsourcing. They did not focus on the UI aspect of it. 2) Some prompt chaining tools, like ChainForge, can support workflow like this, but (a) hey do not focus on data annotation in particular so some functions are missing, like data resampling, and (b) more importantly, they do not aim to support general users. Most of them expect you to know some programming, e.g., ChainForge clearly say it's a visual programming tool. They're not really aiming for generic users.}


\cite{kim2024meganno+} introduced a human-LLM collaborative annotation system that allows LLM to handle bulk annotation tasks while humans verify selectively to refine annotation. 

\cite{goel2023llms} introduced an approach that combines LLM wth human expertise to create an efficient method for generating
ground truth labels for medical text annotation.


\cite{shankar2024validates} introduced a tool, EvalGen, to address the challenge of validating LLM. 
EvalGen helps users design evaluation criteria for LLM outputs and align that evaluation with human preferences through a mixed-initiative system.
A key finding is the concept of criteria drift, where users modify their evaluation standards while grading outputs. 


\cite{brade2023promptify} Promptify utilizes an LLM-powered suggestion engine to help users quickly explore and craft diverse prompts for text-to-image generation tasks.

    
\end{comment}


%\subsection{Survey Study in Data Annotation}
%\steven{
We conducted a survey study to investigate how individuals interact with LLMs and utilize gold-standard labels in the data annotation process. 
The participants primarily represent roles in research, machine learning engineering, and software development. \\
\textbf{Workflows: }Participants described diverse workflows for integrating LLMs into data annotation process, highlighting a common iterative and human-in-the-loop approach. \textbf{Most workflows begin with manual annotation of a small subset of data to establish a baseline.} Participants then employ prompt engineering, iteratively refining LLM prompts by evaluating their performance against the manually annotated subset. \\
Once refined, the prompts are used to label larger datasets, with participants using tools or manual checks to review the LLM's annotations and identify any invalid labels. The process is typically concluded with a thorough manual verification of the dataset. \\
One participant mentioned they manually tabulate data points along with their descriptions. \\
\textbf{Initialize Prompting: }Most participants use their pre-defined prompts to initialized the annotation on their known tasks. 
For new tasks, one participant mentioned that they initialize the annotation process with LLMs by starting with a clear problem definition and iteratively refining a classification-based approach. For less familiar tasks, some participants may seek suggestions from the LLM to guide the initial setup.
\textbf{Revising Prompt: } Participants use a small dataset to finetune the prompt. They address issues by adding rules or context examples to tackle failure cases. When inconsistencies or error arise, they revisit and recheck the manually tagged dataset to improve performance. Some participants also engage the LLM by asking questions about data points and their descriptions, retraining to against inconsistencies to minimize hallucinations and enhance annotation reliability.
}

\subsection{Gold-Standard Labels in Annotation Tasks}\label{sec:related-work-gold-label}
Decades of research have shown that gold-standard labels play a critical role in quality control for data annotation pipelines~\cite{han2020crowd,gadiraju2015training,le2010ensuring,doroudi2016toward,hettiachchi2021challenge}.
Embedding items with known labels into the data annotation process allows requesters to reliably capture quality signals, 
such as workers' level of expertise~\cite{abraham2016many, abassi2019worker, yang2018improving} %\kenneth{TODO: Add refs about using gold labels to decide workers' expertise level}\steven{added}
or attentiveness to tasks~\cite{hettiachchi2021challenge, oleson2011programmatic}. %\kenneth{TODO: Add refs about using gold labels to do attention checks for workers}\steven{added}
These insights enable requesters to take appropriate actions, such as 
retraining annotators~\cite{le2010ensuring, doroudi2016toward,hettiachchi2021challenge}, %\kenneth{TODO: Add refs about retraining workers}\steven{added}
removing low-performing workers~\cite{10.1145/3613904.3642834, snow2008cheap,downs2010your,le2010ensuring}, %\kenneth{TODO: Add refs about removing or blocking low-performing workers}\steven{added}
or identifying potential issues in the annotation interfaces~\cite{toomim2011utility,10.1145/3613904.3642834, rahmanian2014user, komarov2013crowdsourcing}. %\kenneth{TODO: Add refs for crowd worker interfaces. At least cite: Toomim, M., Kriplean, T., Pörtner, C., \& Landay, J. (2011, May). Utility of human-computer interactions: Toward a science of preference measurement. In Proceedings of the SIGCHI Conference on Human Factors in Computing Systems (pp. 2275-2284).}\steven{added}
Gold labels are also beneficial for requesters during post-annotation data processing. 
They can be used to weight labels from different workers in label aggregation~\cite{abassi2017gold,abassi2019worker}, %\kenneth{TODO: Add label aggregation methods that use gold labels particularly to weight different workers}\steven{added}
improve label aggregation strategies~\cite{khattak2011quality, snow2008cheap},  %\kenneth{TODO: Add label aggregation methods that learn whatever from gold labels}\steven{added}
or 
exclude unreliable workers' outputs entirely~\cite{abassi2019worker}. %\kenneth{TODO: Cite ref using gold labels to remove workers from label aggregation}\steven{added}
Beyond requesters, gold labels are also beneficial for data labelers like crowd workers. 
Gold labels can be used to train workers~\cite{doroudi2016toward, le2010ensuring, gadiraju2015training,han2020crowd}, %\kenneth{TODO: Cite ref that uses gold labels for worker training}\steven{added}
provide real-time feedback to help them recalibrate their understanding of the task~\cite{le2010ensuring,hettiachchi2021challenge}, %\kenneth{TODO: Cite the visible gold paper from Amazon}\steven{added}
or remind them to pay more attention~\cite{ hettiachchi2021challenge,oleson2011programmatic}. %\kenneth{TODO: Cite attention check papers}\steven{amazon paper also warn workers in real time}

While gold labels are useful for quality control, as stated in the Introduction (Section~\ref{sec:intro}), %\kenneth{TODO: Update references}\steven{done}
they are not always available in real-world scenarios due to constraints such as data privacy or the cost of gathering gold labels~\cite{liu2019deep, yang2019evaluating, oikarinen2021detecting, slote2024unlocking}.
To address these challenges, researchers have developed methods to generate (approximations of) quality signals without gold labels. 
In the realm of LLM-powered data annotation, for instance, CoPrompter evaluates how well an LLM's output aligns with user-specified requirements as a feedback mechanism~\cite{joshi2024coprompter}. %\kenneth{TODO: Cite: Joshi, I., Shahid, S., Venneti, S., Vasu, M., Zheng, Y., Li, Y., ... \& Chan, G. Y. Y. (2024). CoPrompter: User-Centric Evaluation of LLM Instruction Alignment for Improved Prompt Engineering. arXiv preprint arXiv:2411.06099.}\steven{added}
Other studies also leverage the stability~\cite{chiang2023can} %\kenneth{TODO: Add ref}\steven{added}
%chiang2023can found LLM evaluation are stable over different formatting
or confidence~\cite{gligoric2024can} %\kenneth{TODO: Add ref}\steven{added}
%gligoric2024can introduce CONFIDENCEDRIVEN INFERENCE: a method that combines LLM annotations and LLM confidence indicators to strategically select which human annotations should be collected
of LLM outputs to infer quality signals.
%Our research investigates how effectively humans can iteratively refine prompts to guide LLMs in labeling data when gold-standard labels are unavailable, providing alternative quality signals.
Our research examines how effectively humans can refine prompts to guide LLMs in labeling data without gold-standard labels, providing insights into human prompting capabilities in the absence of reliable guidance signals.










%------------- dead kitten -------------
\begin{comment}




\kenneth{------------------------KENNETH IS WORKING HERE----------------------}



Gold-standard labels are widely used for quality control and crowd worker training~\cite{doroudi2016toward, gadiraju2015training,le2010ensuring,hettiachchi2021challenge}. For example, \citet{hettiachchi2021challenge} demonstrated that incorporating visible gold questions -- where annotators receive periodic feedback based on pre-labeled gold-standard examples -- could improve their work quality. 
Similarly, \citet{doroudi2016toward} found that providing expert examples was the most effective method of training for crowd workers and can help workers avoid specific types of incorrect responses. 
Additionally, \citet{le2010ensuring} employed dynamic learning systems that leveraged gold-standard labels to deliver real-time feedback and improve worker outcomes.
These studies, however, predominantly address the annotators' perspective -- workers who adhere to predefined guidelines and follow established standards.
While annotators are crucial components of the task pipeline, our study shifts focus to the requesters' perspective, those responsible for task design and pipeline management.
For requesters, gold-standard labels serve as signals to assess worker performance and refine training processes, thereby improving the overall quality of the entire pipeline.
Critically, the aforementioned studies assume the availability of gold-standard labels, typically under controlled experimental settings. 
In real-world scenarios, this assumption often does not hold due to constraints such as data privacy, security concerns, or the absence of labeled data~\cite{liu2019deep, yang2019evaluating, oikarinen2021detecting, slote2024unlocking}. 
To address this gap, our research explores settings where predefined gold-standard labels are unavailable. 
We designed a novel framework for requesters to iteratively develop and evolve their labeling standards through interactions with LLMs. 
By bridging the divide between controlled experiments and real-world challenges, our work highlights the potential of adaptive, LLM-driven approaches for dynamic task management without reliance on predefined gold-standard labels.

\steven{\citet{hettiachchi2021challenge} demonstrated that incorporating visible gold questions -- where annotators receive periodic feedback based on pre-labeled gold-standard examples -- could improve their work quality. 
Their study leveraged gold-standard labels to train crowd workers to align with pre-defined standards, effectively guiding annotators thorugh examples and feedback. 
While this approach focues on improving labeling quality at the annotator level, our work shifts the focus to requester and researcher perspective. Instead solely training labelers to meet pre-existing standards, we emphasize the broader implications of designing system in the entire labeling process, particularly in context involving dynamic or subjective tasks. \citet{gadiraju2015training} showed that training workers with gold labels can enhance accuracy and decrease response times. \citet{han2020crowd} used gold standard labels to guide crowd workers in revising incorrect judgments to align with predefined standards. 
}

\steven{
\citet{doroudi2016toward} found that providing expert examples was the most effective method of training for crowd workers. In our study, however, each participant was treated as an individual researcher rather than a crowd worker. While this finding underscores the value of providing gold labels to improve language model performance, it does not directly highlight their significance for researchers. Furthermore, \citet{doroudi2016toward} observed that gold standard labels help workers avoid specific types of incorrect responses. 
In contrast, our task is subjective, with participants’ standards potentially shifting across iterations. Introducing pre-set gold standard labels could enforce a uniform standard across each participant, which might not align with the iterative and subjective nature of our study
}

\steven{\citet{gadiraju2015training} showed that training workers with gold labels can enhance accuracy and decrease response times. [They were still focusing on crowd worker level.] }

\steven{\citet{han2020crowd} used gold standard labels for quality control and to guide crowd workers in revising incorrect judgments to align with predefined standards.}

\steven{\citet{le2010ensuring} employed gold standard labels within a dynamic learning environment that provided real-time feedback to train workers. However, the selection of specific examples for training could influence worker responses, potentially introducing bias in their judgments. [This is why we implemented a random sample in our system]}


\steven{\citet{liu2019deep} developed a HITL system that kept model upgrading with progressively collected data without having a pre-labeled data. [\textbf{they used 30 samples per iteration.} -add to justification for 10 and 50 instances.]}

\steven{\citet{wall2019using} found that end-users could build models without using expert patterns that have comparable performance to those who built by expert. This approach was required more effort and more mental demand than those who received guidance.}

\kenneth{TODO: Add references to every part of this paragraph.}
Decades of research have established that gold-standard labels are highly effective for quality control in data annotation~\cite{han2020crowd,gadiraju2015training,le2010ensuring,doroudi2016toward,hettiachchi2021challenge}. 
Embedding items with known labels into the annotation process enables requesters to monitor annotator or data quality and take actions such as retraining annotators, removing them from the pipeline, or reducing their weight in label aggregation. 
Beyond requesters, gold labels also allow for real-time feedback to workers, helping them recalibrate their understanding of the task or focus more carefully.
While gold labels are widely recognized as useful for quality control, most research assumes their availability.
However, as discussed in our Introduction (Section X), this assumption does not necessarily hold in real-world scenarios due to constraints such as data privacy or the cost of gathering gold labels~\cite{liu2019deep, yang2019evaluating, oikarinen2021detecting, slote2024unlocking}. 
To address these challenges, researchers have developed systems to provide proxy quality signals without gold labels. 
For instance, CoPrompter evaluates how well an LLM's output aligns with user-specified requirements as a feedback mechanism. 
Other studies leverage the stability or confidence of LLM outputs to infer quality signals.
Our research investigates how effectively humans can refine prompts to guide LLMs when gold-standard labels are unavailable.
    
\end{comment}

%\subsection{Explanations in AI-Assisted Tools}


%\subsection{Variables in System}
%There are lots of variables in a system could impact user's performance. 
\citet{kulesza2012tell} suggested that the more users understand the underlying system, the more effectively they can control it. 

\steven{\citet{lee2024clarify} introduces a system that allows non-expert users to train and correct models by directly interact with model using natural languages. In each iteration, the system will use similarity score between user description and image and display images above a threshold. The system will also provide 0-1 score indicating how well description separates the error cases from the correct prediction. Basically using metrics to guide user.
It does not mentioned about the sample size selection.}

\steven{[Data Instance:] In active learning, the goal is to minimize the amount of interaction required by users by querying the most important information~\cite{bernard2018vial}. [This can be used to justify why we increase to 50, to ensure the diversity. We cannot deploy algorithms to find most representative data sample because of the technical limitation of Google App Script]}

\steven{[Data Instance:] \citet{vermetten2022analyzing} investigated how the number of sample size affects the reliability of algorithm comparisons in iterative optimization. The study found that small sample sizes lead to high variability in performance estimates and larger sample sizes could decrease the impact of outliers. The performance could loss due to small samples and increasing sample size consistently improves reliability. }

\steven{\citet{purohit2018ranking} suggested capping the maximum number of annotation tasks assigned per unit of time to manage workload effectively to mitigate annotator burnout.}

\steven{\citet{pandey2022modeling} mentioned annotator can develop a mental representation of a concept by seeing a sufficient number of examples.}

\steven{\citet{wang2016human} limited users to verify the top-50 in each round, where users did binary classification on whether image was match or not.}

\steven{[explanation]\citet{kulesza2015principles} presents a system that explains the reason behind each prediction for users to better understand the system's logic to tailor the system toward their needs. In the system, users will modify feature weights within the model. n our LLM-powered system, users need to use natural language to guide the system. However, this can be more challenging because large models are less responsive to prompt variations compared to smaller models~\cite{zhuo2024prosa}.}

\steven{The more users understand the underlying system, the more effectively they can control it~\cite{kulesza2012tell}.}

\steven{\citet{teso2023leveraging} discusses a general framework for incorporating explanations into interactive machine learning. Users can get a better understanding of the machine's logic by observing the machine's explanations. [In LLM system, the explanation is the supporting argument for selecting a label.] Once understanding the bugs and limitation, users could modify the algorithm to correct flaws~\cite{kulesza2015principles}. [In our case, user cannot directly modify LLMs but only provide natural language to guide them. Also, subjective tasks does not have universal correct answers, where users need to provide their own standards to steer LLMs. ] }

\steven{[Task Difficulty:] 
A task being too difficult can frustrate users~\cite{zheng2022virtual}, particularly when exceeding their skill level, and a task being to easy can lead to boredom~\cite{zhang2021personalized}.
  These study focused on the impacts of difficulty on users' performance on a pre-defined task. However, in our study, our work prioritizes the dynamics of human-LLM interaction, emphasizing how effective humans could guide LLMs to align with their standard. In this context, the difficulty level of the task itself is less critical, as our primary objective is to assess the effectiveness of human guidance, regardless of the inherent complexity of the task.}


\steven{[task type:]\citet{cayir2016study} found the complexity and definition of a task significantly influence user performance. }

\steven{[task type:] \citet{hettiachchi2022survey} discusses different task assignment methods, including the modeling of worker performance and the impact of task heterogeneity on assignment strategies.
\citet{zhen2021crowdsourcing} provides a detailed exploration of task assignment challenges, task types, and their effects on worker performance and task outcomes. 
}
% \steven{ending of related word}We wanted to design a system to bridge the gap of xxxx: a graphical interface implemented on Google Sheet add-on, generalizing to single-class data annotation tasks, without requiring extensive knowledge of programming and system configuration. By combining the widespread familiarity and advanced features of Google Sheets with large-scale data annotation and iteration tracking, we aimed to make it easier for people to experiment with and benefit from LLMs.

\section{\system: A Spreadsheet-Based End-User Prompt Engineering Tool}


%\kenneth{Note that Figure 1 is a bit simplified, e.g., label verification, keep items for next iteration, is dismissed to make sure clear communication.}
%The \system is a Google Spreadsheet add-on with a user interface for guiding LLMs by providing rules or exemplary data to improve LLM performance. The system could be adapted to various single-class data annotation tasks. 
%Users can easily operate the system with minimal learning time and without the need for complex environment setup, unlike other applications.~\steven{find some other annotation systems}
In this paper, we present \system, a Google Sheets add-on that allows users to load a dataset into a Dataset spreadsheet, manually compose each part of a prompt within Task Context, Labeling Rules, and Shots sheets, and use the composed prompt to instruct an LLM to annotate data---all within the same Google Sheets document.
Motivated by the need to enable general users to prompt LLMs without installing and configuring professional tools like integrated development environments (IDEs) or Jupyter Notebooks, we decided to build a tool based on spreadsheets, which most computer users are already familiar with.
This section overviews \system's design and workflows.
%\kenneth{TODO: Maybe cite other tools and name what setup or configs they require}

\subsection{Design Goals}\label{sec:3-system-design}
%\kenneth{(1) Support PITD so we don't have gold label and don't calculate accuracy etc becasyue we think they're not reliable and evovling, and (2) using spreadsheet because it's just very easy to use and practically provide many flexibilities to users.}

%Prompting in the Dark: 
\paragraph{Adapting to Evolving Labeling Goals}
%\paragraph{Prompting in the Dark and Users' Needs for Evolving, Open-Ended Labeling Schemes.}
The goal of \system is to enable general users to create and refine prompts iteratively for LLM-powered data labeling, particularly in situations where they start without any labeled gold data or manual labeling, \ie, the ``prompting in the dark'' scenario. 
In these cases, users' understanding of the data and desired labeling scheme evolves through their interactions with the LLM, based on its predicted labels and explanations, rather than through their own manual efforts. 
The lack of gold labels (or sufficient labeled data) introduces the core challenge of the prompt-in-the-dark process: aside from users' observations and judgments about the labeling results, there is no concrete way to provide quick and comprehensive feedback on the progress of prompting.
We view this as a trade-off between two user needs in prompt engineering: 
{\em (i)} allowing users' understanding of the data and labeling goals to evolve, and 
{\em (ii)} providing clear guidance and reliable feedback to assess progress toward a defined annotation goal. 
Previously, supervised learning-based classifiers required labeled data, so users focused heavily on the second need, as manual labeling was always needed and assumed to finalize the coding scheme. 
The rise of LLMs has reduced the need for pre-labeled data, allowing users to put more focus on their first needs. 
The growing popularity of the prompting-in-the-dark approach reflects users' need for evolving and dynamic labeling practices~\cite{austin2024grad,zhang-etal-2024-glape,wang2024human}.
%zhang2023labelvizier to facilitate the validation and relabeling of large-scale technical text annotations. Its interactive, visual analytic interface allows users to detect and correct three main types of labeling errors: duplicate, wrong, and missing labels. 
%\kenneth{TODO: Do we have any reference to support it's a popular practice now?}\steven{those are three study that have no gold label, they iteratively use user-defined criteria to evaluate and refine.}\kenneth{Oh and did they manage to improve the accuracy over time??}\steven{Yeah, they got a higher rating and accuracy}\kenneth{Hmmmmmm so what's the deal? How are their systems and approaches different?} %, rather than simply saving time on manual labeling. 
% dudley2018review describe the interative machine leanring paradigm that user iterative build and refine model. The model refinement is driven by user input. It more focused on the human input to refine. kim2024evallm is a prompt refining tool by evaluating outputs on user-defined criteria. It is not a labeling task
Our design goal for \system is to offer users greater flexibility and freedom in defining how their data should be labeled.


%\kenneth{I revised this following paragraph to tailor it more close to our target users. Please take a look.}
\paragraph{Supporting a Wide Range of ``Newcomers'' Brought in By LLMs.}
%\paragraph{End-User Prompting Tools.}
%Our second design goal for \system is to create a tool for general users, including people with limited or no programming skills.
Our second design goal for \system is to develop a tool for users with \textbf{little to no experience in large-scale text data labeling}, including but not limited to those with limited or no programming skills.
The rationale behind this goal is two-fold.
On a practical level, 
people new to large-scale data annotation---empowered by LLMs to undertake such tasks with greater ease---are more likely to adopt approaches that diverge from conventional practices. 
In the crowdsourcing literature, many papers emphasize best practices for data annotation~\cite{hsueh2009data, sabou2014corpus, vondrick2013efficiently, drutsa2019practice, wang2013perspectives}, %\kenneth{TODO: Add ref on crowdsourcing best practices}\steven{added}
such as ethical pay rates~\cite{fort2011amazon,shmueli2021beyond}, %\kenneth{TODO: Add refs on crowd workers' ethical pay}\steven{added}
usable worker interfaces~\cite{toomim2011utility,10.1145/3613904.3642834, rahmanian2014user, komarov2013crowdsourcing}, %\kenneth{TODO: Add refs to crowd interface's impact on crowdsourced data quality--- Maybe cite our own CHI paper too}\steven{added}
and 
gold labels for quality control~\cite{han2020crowd,gadiraju2015training,le2010ensuring,doroudi2016toward,hettiachchi2021challenge}. %\kenneth{TODO: Again, add ref IN CROWDSOURCING for gold labls}\steven{added}
However, these practices are often neglected in real-world scenarios. 
For instance, many tasks on MTurk still offer very low pay~\cite{AI_workers_low_wages} %\kenneth{Add ref: A data-driven analysis of workers' earnings on Amazon Mechanical Turk}\steven{added}
or rely on poorly designed interfaces~\cite{fowler2023frustration}. %\kenneth{Add ref: Frustration and ennui among Amazon MTurk workers}\steven{added}
Newcomers to large-scale data annotation are even less likely to be familiar with these best practices, including carefully establishing gold labels before prompting LLMs.
%users without programming experience are more likely to prompt in the dark, struggling to interact effectively with LLMs. 
%In contrast, those with software engineering backgrounds are familiar with using established frameworks and tools, where automatic testing---such as unit and integration testing---is standard. 
The bigger picture is that LLMs are adding many ``new members'' to the world of programming and data science.
%---people with little or no coding experience. 
This group brings new practices, user needs, challenges, and research questions to HCI, requiring more focused attention.

%\bigskip
\sloppy
Based on these two design goals, we decided to build \system based on spreadsheets, a format that most computer users are already familiar with.
We distinguish our goals from existing efforts in two significant ways. 
First, while projects like LangChain or ChainForge focus on developers or those with programming backgrounds, requiring software installations or configurations, we aim to focus on general users who do not necessarily have such expertise. 
Second, some projects explore new interactions enabled by LLMs~\cite{10.1145/3586183.3606833}, but our project is concerned with understanding how effectively users can use familiar interfaces, such as spreadsheets, to interact with LLMs.

%--------------- dead kitten ----------

\begin{comment}
  


At the practical level, users with little or no programming background might be more likely to prompt in the dark. 
In contrast, individuals familiar with software engineering practices are accustomed to using existing frameworks or tools, where automatic testing---such as unit and integration testing---is standard.
These users have more experience as well as technological support for creating gold labels for testing.
At a deeper level, what is happening is that LLMs bring many people with limited or no programming skills into the realm of data science or programming tasks.
This group of people brings new and interesting challenges and research questions to HCI and thus deserves more attention.

%we believe that the greatest value of LLMs lies in the new possibilities they offer to general users. 
%For people without coding skills, LLMs enable tasks such as building websites from scratch, creating classifiers for automating email filtering, and labeling data to extract insights---activities that were previously within the realm of programmers. 

  
\end{comment}

\begin{figure*}[t]
    \centering
    \includegraphics[width=0.9\linewidth]{Figures/ui-overview.jpg}\Description{This is the user interface layout and pre-defined spreadsheet tab explanation. Each predefined sheet has a set of predefined columns. PromptSheet allows users to load a dataset into the Dataset tab, which is the starting tab of the system. By clicking each tab at the bottom of the Google Sheet, users can navigate to Task Context tab, Rule Book tab, Shots tab, Working Data Sample tab, task dashboard tab and task results tabs. The task result tabs will be generated after each new annotation round and store all new annotated results.}
    \caption{The user interface and all the predefined sheets of \system, where each sheet has a set of pre-defined columns. 
    \system allows users to load a dataset into a Dataset (A) sheet, manually compose each part of a prompt within Task Context (B), Labeling Rules (C), and Shots (D) sheets, and use the composed prompt to instruct an LLM to annotate data and store the labeling results in a separate task sheet (G). All functions are presented as manuals and buttons within the sidebar on the right.}
    \label{fig:system-interface-ui-kenneth}
\end{figure*}

\subsection{User Interface and Pre-Defined Sheets}

\system is a Google Sheets add-on that enables users to load data, sample a subset for labeling, compose and edit prompts, use these prompts to request LLMs for data labeling, and iteratively revise the prompts. 
Figure~\ref{fig:system-interface-ui-kenneth} shows the interface of \system.

\paragraph{Sidebar.}
Following Google Sheets' design constraints, all functions are presented as manuals and buttons within the sidebar on the right. 
The sidebar remains consistent across all sheets, regardless of which sheet is in use. 
At the top of the sidebar, \system provides a real-time notification that keeps users informed about its ongoing processes, such as ``Data Indexing,'' ``Data Sampling,'' ``Generating the Instructional Prompt,'' or ``Annotating.''


\paragraph{Pre-Defined Sheets.}
\system includes a set of predefined spreadsheets, each with a set of pre-defined columns. 
At the bottom of the interface, a series of tabs allows users to switch between sheets, with each sheet dedicated to a different part of the data labeling process. 
The following describes each sheet in detail. 
(To help readers easily identify which sheet we are referring to, we indexed each sheet as A, B, C, ..., and G in all the figures. 
These indexes were not present in the actual system to users.)

%\subsubsection{All the Sheets and What They Do}

%\hyperref[fig:system-interface-1]{Figure 1} overviews the \system's user interface. 
%The system consists of seven main components:

\begin{itemize}

\item \textbf{Dataset (Sheet A)}:
%The ``Dataset'' includes three columns: Data ID, Group ID, and Data Instance. 
%Each data instance is uniquely identified by a corresponding Data ID.
%A single Group ID may encompass one or multiple Data Instances.
This spreadsheet stores the full dataset.
Users can copy and paste the dataset into this sheet or use any supported Google Sheets import method (in Step 0).
The sheet includes three key predefined columns: (1) Data ID, (2) Group ID, and (3) Data Instance. 
Each data instance is uniquely indexed by its corresponding Data ID, which users can generate by clicking the ``Index Data ID'' function in the sidebar. 
The Group ID is used for annotating sequential data, such as when each sentence in an article is treated as a separate data instance, but all sentences from the same article share the same Group ID.
In our design, this sheet is intended to serve as a static data source, and we anticipate that users will not modify it after loading the data.

%\kenneth{TODO: Maybe add words to mention we don't expect people touch it after Step 0 and basicallyy serve as a database.}

\item \textbf{Task Context (Sheet B)}:
%The ``Context'' tab provides information to help describe the annotation task users are working on. It addresses questions related to the purpose and application of the data annotation task and the origin and size of each data instance. The LLM will use the information provided in this tab to generate an instructional prompt for a later step.
This spreadsheet stores the meta-information and context for the labeling task, which will later be incorporated into the prompt. 
The sheet includes predefined questions that characterize the task, such as the purpose of the data labeling, how the labels will be used, the source of the data, and the size of each data instance.
Table~\ref{tab:task-sheet-questions} in Appendix~\ref{sec:context-question-appendix} shows all the questions.
%\kenneth{TODO: Maybe add all the questions to Appendix.}
Users provide answers to these questions (in Step 1 or 4), and \system automatically incorporates both the questions and their answers into the prompt used for LLMs to label the data.


    
\item \textbf{Rule Book (Sheet C)}: 
%The ``Rule book'' tab is where users define the criteria and definitions for each label used during the annotation process.
This spreadsheet contains the labeling rules that the LLM will follow.
It includes two key predefined columns: (1) Label Name and (2) Rules for the Label. 
Users manually define the criteria and descriptions for each label in free text (in Step 1 or 4), detailing the guidelines for the annotation process. 
Multiple rules can be added for a single label, providing flexibility in defining the labeling criteria.


    
\item \textbf{Shots (Sheet D)}: 
%In the ``Shots'' tab, users can enter gold standard labels from their iterations or manually provide them as reference points.
This spreadsheet stores all the high-quality examples, including data instances and their corresponding labels, which will be included in the prompt to guide the LLM in labeling the data. 
These examples, commonly referred to as ``shots'' in prompts, follow the same predefined column structure, with an additional ``Gold-Standard Label'' column. 
Users can add these examples manually (in Step 1) or use \system's function to do so (in Step 4).


    
\item \textbf{Working Data Sample (Sheet E)}:
%Users can sample the data from the ``Dataset'' tab to the ``Working Data Sample'' tab. In the annotation process, \textit{only} data instance in the ``Working Data Sample'' tab will be annotated.
% by the LLMs using the instructional prompt, provided rules, and gold shots.
This spreadsheet stores the current subset of data selected from the full dataset, ready for the LLM to label.
Users can sample data from the Dataset sheet by clicking the corresponding buttons in the sidebar; users can choose between random sampling or selecting a specific range (Step 2). 
During the annotation process, only the data instances in the Working Data Sample sheet will be labeled. 
\system will copy the entire data sample from the Working Data Sample sheet to create a new sheet to label (Step 3).
    
    
\item \textbf{Task Dashboard (Sheet F)}:
%The ``Task Dashboard'' tab records all iteration task details such as task number, timestamp, used prompt, and total costs. 
This spreadsheet tracks all labeling tasks performed so far.
When the user clicks the ``Start Annotation'' button in the sidebar (in Step 3), \system creates a new sheet for the task (e.g., Task 1 sheet) and adds a new row in the Task Dashboard to record the labeling activities.
Task Dashboard sheet (Figure~\ref{fig:task-dashboard-new})
logs task details such as task number, timestamp, the prompt used, and total costs.

\item \textbf{Task 1 (Sheet G), Task 2, ..., Task N}:
%After each annotation, the annotation results will be saved in a new tab (e.g., Task\_1, Task\_2, etc) corresponding to that specific iteration. 
Each of these sheets stores the annotation results for each labeling request, including data samples, LLM-generated labels, and LLM explanations (optional).
These sheets also include columns that allow users to validate or correct the LLM labels and optionally add them to the Shots sheet (in Step 4).
When the user clicks the ``Start Annotation'' button in the sidebar (in Step 3), \system generates a new task sheet to handle the specific labeling task.

    
    
\end{itemize}

%\kenneth{Users are allowed to add new columns.}

Notably, while users must follow our guidelines for using the predefined columns in each sheet and inputting data correctly, they are free to add more columns or even additional sheets, just as they would in a regular Google Sheets document. 
For instance, when pasting a dataset into the Dataset sheet, it is common for the dataset to include its own IDs or additional information for each data entry. 
Users can easily store this extra information by creating new columns within the Dataset sheet.


%---------- dead kitten ----------

\begin{comment}



\subsubsection{Other Features} \steven{todo: add figures in the appendix. screenshots for different notification messages. interface screenshots for removal and clearing.}
\begin{itemize}
    \item \textbf{Real-Time System Notification: }\system provides a notification feature that informs users of its current processes, such as ``Data Indexing'', ``Data Sampling'', ``Generating the Instructional Prompt'', ``Annotating'', etc.
    \item \textbf{Remove Unselected Data Instance (Figure \ref{fig:remove-clear}): }This function will remove data instances that do not have the ``Keep it in the next data sample'' checked in the ``Working Data Sample'' tab.
    \item \textbf{Clear Data Instance (Figure \ref{fig:remove-clear}): }This function will clear all data instances in the ``Working Data Sample'' tab.
\end{itemize}

    
\end{comment}



\subsection{User Workflow}
\begin{figure*}[t]
    \centering
    \includegraphics[width=0.99\linewidth]{Figures/step-1.jpg}\Description{This is Step 1 described in Figure 1. Users can provide data annotation context in the Context Tab, provide their rule and definition in the Rule Book tab, and add gold standard labels in the Shots tab. These tabs will compose prompts for later GPT to use.}
    \caption{The overview of step 1 of the data labeling process, compose or refine the prompt.
This is the most critical step, where the user composes and refines prompts for the LLM to label the data. In \system, the prompt consists of three parts, each corresponding to a separate sheet: Context (B), Rule Book (C), and Shots (D). 
At the beginning of this labeling process, the user has only a vague idea of what they want to label and will continuously refine that idea. 
Each time the prompt is revised, it reflects an evolution of their understanding and approach to the labeling task.}
    \label{fig:step-1}
\end{figure*}

\begin{figure*}[t]
    \centering
    \includegraphics[width=0.99\linewidth]{Figures/step-2.jpg}\Description{This is Step 2 described in Figure 1. Users can either random or sequential sample data from the Dataset tab to the Working Data Sample tab. In the Working Data Sample, users can check “Keep it in the next data sample” for data instances that users want to remain in the Working Data Sample tab during sampling.}
    \caption{The overview of step 2 of the data labeling process, sample or resample a subset.
The full dataset is often too large for the user to thoroughly review, so sampling a subset is necessary.
%Labeling only a subset, rather than the entire dataset, is necessary because 
%Additionally, labeling the entire dataset iteratively would be prohibitively expensive. 
In this step, the user can (1) randomly or (2) sequentially sample data from the Dataset (A) sheet.
}
    \label{fig:system-interface-step-2}
\end{figure*}

\begin{figure*}[t]
    \centering
    \includegraphics[width=0.99\linewidth]{Figures/step-3.jpg}\Description{This is Step 3 and 4 described in Figure 1. After clicking Start Annotation, the results including LLM label and LLM explanation will be stored in a new tab. Users will review the data instances and LLM labels, by checking agree or providing their own labels. They also select the Gold Shot data instance to be added to Shots tab for later GPT to learn from. After verification, they can refine their prompt as Step 1 mentioned.}
    \caption{The overview of steps 3 and 4 of the data labeling process. After finalizing the prompt (Step 1) and sampling data instances (Step 2), in Step 3, the user clicks the ``Start Annotation'' button in the sidebar to annotate all instances in the Working Data Sample sheet. \system creates a new sheet, Task 1 (G), to store the data and labels of this labeling task, and also creates a new row for Task 1 in the Task Dashboard sheet. Then, in Step 4, the user can review the outcomes and refine the prompt accordingly (Step 1).
}
    \label{fig:system-interface-step-3-4-kenneth}
\end{figure*}

Users interact with \system to craft a prompt, use it to instruct the LLM in labeling data, review the results, and then revise the prompt through an iterative process. 
To demonstrate the users' workflow, we present a scenario where a user wants to employ \system to label a collection of tweets related to COVID with a 5-point sentiment scale, ranging from Very Negative (1) to Very Positive (5).
The goal is to analyze the sentiment of Twitter (now X) users toward COVID, with an emphasis on ensuring that the classification of each tweet reflects the user's own judgment.
In this case, the LLM's labels should align with the user's assessment of what is positive or negative, as well as the intensity of sentiment, rather than following an ``objective'' standard.
%In other words, the LLM's labels should align with the user's personal perception of the topic rather than adhering to an ``objective'' standard.\kenneth{This is not very accurate hmmm. Might need to edit later.}

%\begin{enumerate}
%    \item 
%\end{enumerate}

\begin{itemize}
   
\item 
\textbf{Step 0: Load and Index the Dataset.}
To begin using \system, the user opens a new Google Sheets document and activates the \system add-on. 
The system automatically sets up the necessary tabs, and the add-on interface appears on the right side of the spreadsheet (Figure~\ref{fig:system-interface-ui-kenneth}). 
The user then imports their data instances into the Dataset sheet, with the text of each tweet placed in the Data Instance column. 
The user must specify a Group ID for each instance. 
If the data are not sequential or grouped, they can assign unique Group IDs using Google Sheets' automatic numbering function.\footnote{\system is designed to accommodate single and grouped data instances within a Group ID. For tasks like sentiment analysis, each data instance is treated separately under its unique Group ID. For tasks that require contextual information, such as annotating text segments in an academic abstract (\eg, CODA-19~\cite{huang-etal-2020-coda}), \system can combine all data instances under the same Group ID into a single request to the LLM model. This flexibility allows the system to support different data instance formats based on user requirements.} 
Once the data is entered, the user clicks the ``Index Data ID'' button in the sidebar, and \system automatically assigns unique data IDs to each instance in the ``Data ID'' column.

\item 
\textbf{Step 1: Compose/Refine the Prompt (Figure~\ref{fig:step-1}).}
%Step 1: compose the promot using things. 
%Uses know a vague idea what they want and will keep revise that idea. But you need to write something down. When it comes to load data, spreadsheet is great!
This is the most critical step, where the user composes and refines prompts for the LLM to label the data. 
In \system, the prompt consists of three parts, each corresponding to a separate sheet: (1) Context, (2) Rule Book, and (3) Shots. 
Figure~\ref{fig:step-1} provides an overview of each sheet.
\begin{enumerate}

\item 
In the \textbf{Context} sheet, the user answers questions that describe the context of the data annotation task, such as the purpose of the annotation and the source of the data, to provide task-specific context for the LLM.

\item
In the \textbf{Rule Book} sheet, the user adds annotation labels along with their definitions. Providing content for both the Context and Rule Book sheets is mandatory, as the LLM requires this information in the prompt to function effectively.

\item
In the \textbf{Shots} sheet, the user adds data instances along with their corresponding gold labels, which serve as examples to help the LLM learn. While adding examples to the Shots sheet is optional during the first iteration---since the user may not yet have a well-defined gold standard for labeling---more examples can be identified as the user reviews data. These examples can be manually added or generated using \system's function (see Step 4).
\end{enumerate}
It is important to note that at the beginning of this labeling process, the user has only a vague idea of what they want to label and will continuously refine that idea. 
Each time the prompt is revised, it reflects an evolution of their understanding and approach to the labeling task.


\item 
\textbf{Step 2: Sample/Resample a Subset (Figure~\ref{fig:system-interface-step-2}).}
Next, the user employs \system to sample a subset of data for labeling. 
Labeling only a subset, rather than the entire dataset, is necessary because the full dataset is too large for the user to thoroughly review the LLM's results. 
Additionally, labeling the entire dataset iteratively would be prohibitively expensive. 
In this step, the user can (1) randomly or (2) sequentially sample data from the Dataset sheet into the Working Data Sample sheet:

\begin{itemize}

\item 
For a \textbf{Random Sample}, the user enters any whole number between 1 and the total number of group IDs in the dataset.
\system will then randomly select that number of groups and copy them into the Working Data Sample sheet. 

\item
In \textbf{Sequential Sample}, the user specifies a range of group IDs from the Dataset sheet, and \system will import the data instances from the selected range into the Working Data Sample sheet.
This feature allows users to process their data instances sequentially in batches, which is especially useful when the data instances have a sequential relationship, such as sentences within the same document.


%The purpose of this feature is to enable users to process their data instances sequentially in batches, making their work more manageable and easier to track.
%\kenneth{Mayeb say a few words on why we need this.}\steven{done.}

\end{itemize}

Once sampling begins, all previously existing data in the Working Data Sample sheet will be removed, except for instances marked as ``Keep it in the next data sample'' (Figure~\ref{fig:system-interface-step-2}). 
Only the data in the Working Data Sample sheet will be labeled by the LLM when the ``Start Annotation'' button is clicked in Step 3.

\item 
\textbf{Step 3: Use the Prompt to Instruct the LLM to Label the Data Sample (Figure~\ref{fig:system-interface-step-3-4-kenneth}).}
After finalizing the three prompt sheets---Context, Rule Book, and Shots---in Step 1 and sampling data instances in Step 2, the user clicks the ``Start Annotation'' button in the sidebar to annotate all instances in the Working Data Sample sheet (Figure~\ref{fig:system-interface-step-3-4-kenneth}).
\system creates a new sheet, Task 1 (Figure~\ref{fig:system-interface-step-3-4-kenneth}), to store the data and labels of this labeling task, and also creates a new row for Task 1 in the Task Dashboard sheet.

In the background, \system first combines the information in Context, Rule Book, and Shots sheets into a prompt (see Section~\ref{sec:implementation} for details).
%gathers the questions and answers from the Context sheet and feeds them into GPT-4 to generate an instruction prompt. 
%This prompt is then combined with the rules and provided gold shots to create the final annotation prompt. 
For each data group (\ie, data instances with the same Group ID), \system sends a request to the LLM using this prompt for annotation.
After receiving the LLM's output, the system parses the results and updates the Task 1 sheet with the annotated outcome for each instance. 
In our implementation, the LLM is always asked to provide explanations for its labels, though the user can decide whether to display these explanations in the annotation results.
%In our user study, we also explored the impact of showing the LLM's explanations to users.



%Step 3: Send it to LLM to label. System creat a new tab; you can navigate tasks using dashboard. Then you re

\item 
\textbf{Step 4: Observe, then Revise the Prompt (Figure~\ref{fig:system-interface-step-3-4-kenneth}).}
The labeling results are saved to the Task 1 sheet (Figure~\ref{fig:system-interface-step-3-4-kenneth}), where the user can manually verify the LLM's labels.
The user can review as many or as few data instances as they wish to develop a better understanding of the labeling task and the dataset. 
Based on this evolving understanding, they can refine the prompt by modifying the Context, Rule Book, and Shots sheets accordingly.

If the user disagrees with any of the labels, they can assign a new label to the data instance under the ``Human Label'' column. 
If the user identifies good examples, they can check the ``Gold Shot'' checkboxes. 
After selecting enough good examples, the user can click the ``Add Shots'' button in the sidebar to add these examples to the Shots sheet (Figure~\ref{fig:system-interface-step-3-4-kenneth}).
Like in other sheets, if the user wants certain data instances to be re-annotated in the next round, they can check the ``Keep it in the next data sample'' checkboxes. 
This will ensure that those instances are not removed during the next sampling process, allowing the user to observe whether the LLM's behavior changes over iterations.


\end{itemize}

When using \system, the user moves through Steps 1, 2, 3, and 4, and then returns to step 1 in an iterative process until they are satisfied with the LLM's labels.








%\subsubsection{Step 0: Initial Setups}


%\subsubsection{Step 1: Compose/Refine the Prompt}

%\subsubsection{Step 2: Sample/Resample a random subset}

%\subsubsection{Step 3: Use the Prompt to Instruct the LLM to Label the Data Sample}

%\subsubsection{Step 4: Observe, and Revise the Prompt}


\subsection{Implementation Details\label{sec:implementation}}
%\kenneth{TODO: Here we mention (1) what framework you used to implement Google Sheets add-on, (2) how do you convert Spreadsheet's content into a prompt, and (3) what LLM (which version exactly) you used and how did you send request (batch? or each data instance is one request?)--- Maybe talk about latency issue here a bit.}


\paragraph{Developing Google Sheets Add-On.}
%\kenneth{How do people built Google Sheets add-ons? Did we use an web server? Where do we store our data?}\steven{done}
\sloppy
\system utilized Google Sheets as its main platform, leveraging the convenience and functionality of its spreadsheet capability. The Google Sheets add-on was implemented in Google App Script, with Google Cloud Service serving as a back-end to store all action logging files. User-specific data, such as OpenAI information, was securely stored in user properties tied to individual email accounts, ensuring privacy protection. 

\paragraph{Converting a Spreadsheet's Content into a Prompt.}
Once users click on the ``Start Annotation'' button (Figure~\ref{fig:system-interface-step-3-4-kenneth}), \system will first collect all questions and answers from the ``Context'' tab and send a request to GPT-4o to generate an instructional prompt (Table~\ref{tab:instruction-prompt}). Next, \system will merge this generated prompt with rules and definitions from ``Rule Book'' and available gold standard labels from the ``Shots'' tab to create an annotation prompt (Table~\ref{tab:main-prompt} and Table~\ref{tab:main-multi-prompt}). Finally, \system will use this prompt to annotate all data instances. 

\paragraph{Interacting with the LLM through an API}
In this paper, we utilized OpenAI's \texttt{gpt-4o-2024-05-13} model for our study~\cite{openai2024gpt4o}.
%\kenneth{TODO: Add citation} \steven{done}
Technically, this LLM can be replaced by any other model that offers an API compatible with the ChatGPT-4 specification. 
In our implementation, we group all data instances with the same Group ID and send them in a single API request.
%In our current implementation, we did not batch requests; instead, we sent an individual API request for data instances with the same group ID.
%\kenneth{Is this accurate?}\steven{we sent by group ID}
%Future versions of \system could potentially benefit from batching to reduce latency.













%\subsection{System Design}


%\subsection{System Workflow (\hyperref[fig:system-workflow-fig]{Figure~\ref{fig:system-workflow-fig-v2}})}




% \begin{figure}
%     \centering
%     \includegraphics[width=0.85\linewidth]{Figures/Workflow/workflow-v2.jpeg}
%     \caption{System Workflow}
%     \label{fig:system-workflow-fig}
% \end{figure}












% \subsubsection{Main Procedure}
% The procedure consists of the following steps:
% \begin{itemize}
%     \item \textbf{Step 1 (Import Data): }Users can import their data instances with group ID into the `Dataset' tab and click ``Index Data ID'' to index all data instances.
%     \item \textbf{Step 2 (Answer Task Questions): }Users need to navigate to the `Context' tab to answer questions about the data annotation task they are working on.
%     \item \textbf{Step 3 (Define Labels and Rules): }Navigating to the `Rule Book' tab, users \textbf{have to} add annotation labels with corresponding definitions. 
%     \item \textbf{Step 4 (Add Gold Shots): }Users can add instances with their gold labels if applicable. 
%     \item \textbf{Step 5 (Sample Data): }Users can randomly or sequentially sample data into the `Working Data Sample' tab.
%     \item \textbf{Step 6 (Data Annotation): }After the ``Context'', ``Rule Book'', and ``Shot'' (if applicable) tabs are all settled, users can click on ``Start Annotation'' to annotate all instances sampled in the ``Working Data Sample'' tab. 
%     \item \textbf{Step 7 (Verification): }The annotation results will be saved to a new task tab. Users can start verifying LLM labels. Users who disagree with an LLM label can assign a new human-generated label to the data instance. If users find good examples that can be used for later LLM learning, they can check the ``Gold Shot'' checkboxes. They can also check the ``Keep it in the next data sample'' checkboxes if they want to re-annotate data instances.
%     \item \textbf{Iteration Procedure: }If users are not satisfied with the annotated results, they can modify the answer in the `Context' Tab \textbf{(Step 2)}; modify rules (add/adjust/delete labels or definitions) in the `Rule book' Tab \textbf{(Step 3)}; click ``Add Shots'' to add the selected ``Gold Shots'' to the `Shots' Tab \textbf{(Step 4)}; click ``Add Back'' to add instances back to the `Working Data Sample' Tab for re-annotating in the next iteration. Then, they can re-sample or re-use instances in the `Working Data Sample' tab for the next iteration \textbf{(Step 5)}. In the end, they repeat \textbf{Step 6}.
%     \item \textbf{Completion:} If users are satisfied with the annotation results, they may choose to conclude the task.
% \end{itemize}





% \paragraph{Two Data Instance Formats Handling}
% \system is designed to accommodate single and grouped data instances within a Group ID. For tasks like single-class sentiment analysis, each data instance is treated separately under its unique Group ID. However, for tasks that require contextual information, such as annotating text segments in an academic abstract (e.g., CODA-19~\cite{huang-etal-2020-coda}), \system can combine all data instances under the same Group ID into a single request to the LLM model. This flexibility allows the system to support different data instance formats based on user requirements.


\section{User Study\label{sec:user-study}}
%\section{Comparative Study Procedure}
Our goal is to investigate how effective people are at prompt engineering when gold labels are absent, namely, ``prompting in the dark''. 
To study this, we conducted an in-lab user study in which participants used \system to perform a 5-point sentiment labeling task on a tweet dataset. 
This section overviews the details of this study.
This study has been approved by the IRB office of the authors' institute. 


\subsection{Study Procedure}


%\subsection{In-lab Study}

%We conducted a 90-minute in-lab user study with participants using \system for an annotation task.

\subsubsection{Pilot Study\label{sec:pilot-study}}
Three participants were recruited through the authors' network for the pilot study. 
In the first pilot, we used CODA-19~\cite{huang-etal-2020-coda} as the data annotation task, where participants labeled text segments from academic abstracts into categories such as background, purpose, and findings.
We observed that the participant consistently agreed with nearly all the labels and did not suggest further refinements. 
This may have been due to the highly specialized nature of the abstracts, which made it difficult for a broader audience to fully understand and evaluate the labels. 
As a result, we decided to switch to a Twitter Sentiment task for the second pilot.
In this second pilot, we found that our guidelines were too flexible, leading to participant confusion and uncertainty about how to proceed. 
We made adjustments to provide more structure, such as requiring participants to complete at least four iterations, with each iteration involving the annotation of 10 out of 50 instances. 
After verification, participants were instructed to refine their rules and add gold standard labels.
Based on the results of the two pilot studies, we extended the study duration from 60 to 90 minutes to give participants enough time to learn the system and complete the tasks. Compensation was also adjusted to \$20. 
We tested these revised settings with the third participant and confirmed that the procedure worked effectively.
%Two participants were recruited via the authors' network for the pilot study. 
%\kenneth{TODO: say a few words about pilot study? what did we change after pilot study?}
%In the first pilot study, we used CODA-19~\cite{huang-etal-2020-coda} as the data annotation task, where participants annotated text segments from an academic abstract into background, purpose, method, finding, and others. We observed that the participant consistently agreed with almost all labels and did not suggest further refinement during the verification process. Additionally, while the LLM labels were different than the dataset gold labels, LLM labels and LLM explanations were logically sound. 
%Based on these findings, we decided to switch the annotation task to a Twitter Sentiment task for the second pilot. In this study, we found that our guidelines were too flexible, leading to participant confusion and uncertainty about how to proceed. To address that, we simplified and standardized the user study procedures. For example, we required them to do at least four iterations, with each iteration involving the annotation of 10 out of 50 instances. After verification, participants were instructed to refine their rules and add gold standard labels.

%Based on the results of two pilot studies, we extended the study duration from 60 minutes to 90 minutes to allow participants to have sufficient time to learn the system and complete the annotation tasks. The compensation was also adjusted to \$20.



\subsubsection{Participants Recruitment, Backgrounds, and Grouping}
%Recruitment
For our main study, we focused on recruiting individuals with reasonable familiarity with LLMs but relatively new to large-scale text data annotation. 
While \system is designed as an end-user prompting tool, in this study, we prioritized participants likely to represent the first wave of ``newcomers'' (as noted in our Design Goals in Section~\ref{sec:3-system-design}) entering LLM-powered data annotation. %\kenneth{TODO: Update the section ID} \steven{added}
This focus allowed us to avoid the need to teach participants the basics of LLMs, prompting, or text data annotation.
%For the main study, 
We recruited 20 participants from diverse educational backgrounds through the authors' networks, social media posts, and mailing lists within the authors' institute. 
The group included 1 Post-doctoral Researcher, 9 Ph.D. students, 9 Master's students, and 1 Undergraduate student.  
As part of the recruitment process, we specifically sought participants who met the criteria of possessing prior experience using LLMs. 
%\steven{added recruitment part}
%\steven{one participant was dropped because he did not attend the makeup session, should we mention that?}
Participants were compensated \$20 for their participation, and in our analysis, they are denoted as P1 to P20.
%\steven{Our system is designed for requesters or researchers who understand their task requirements. However, due to recruitment constraints, we could not involve participants with extensive data annotation experience. To address this, we chose a subjective task like Twitter sentiment analysis, where prior annotation expertise is less critical. This approach allows participants to rely on their own knowledge and judgment, simulating real-world scenarios where individuals guide LLMs in tasks that naturally depend on personal interpretation and expertise.}

%Backgrounds
%\kenneth{------------------KENNETH IS WORKING HERE---------------------------}

%\kenneth{------------------KENNETH IS WORKING HERE---------------------------}

%Grouping
Participants were randomly and evenly assigned to four groups based on two variables: 
% (1) whether or not they could view the LLM's explanations for its labels\footnote{\steven{Since two participants who had access to LLM explanations chose to turn them off, we grouped them with the no LLM explanation participants, resulting in 8 participants with access to LLM explanations and 12 participants without access.}}, and 
% (2) whether they had access to 50 instances per iteration or 10 instances per iteration.
(1) whether they had access to 50 instances per iteration or 10 instances per iteration, and
(2) whether or not they could view the LLM's explanations for its labels\footnote{Since one participant chose to disable the LLM explanations after the first iteration and another participant decided not to use the LLM explanations throughout the entire study, we grouped them with the no LLM explanation participants, resulting in 8 participants with access to LLM explanations and 12 participants without access.}.
Further details are provided in the study procedure section (Section~\ref{sec:study-procedure}). 

\paragraph{Survey on Participants' LLM Familiarity and Usage.}
To assess participants' familiarity with using LLMs, we conducted an optional post-study survey, offering an additional \$5 compensation for completion. 
(The full set of survey questions is provided in Table~\ref{tab:participants-llm-background-survey} in Appendix~\ref{sec:appendix=participant-background}.) %\kenneth{UPDATE REF}\steven{updated}
All participants responded. %\kenneth{UPDATE NUMBER}\steven{updated}
Most participants reported being familiar with LLMs, with an average familiarity score of 4.20 (SD=0.77) on a 5-point scale. %\kenneth{UPDATE NUMBER}\steven{updated}
15 participants had over one year of experience using LLMs, while 4 reported more than four months of experience, and 1 reported between one and three months. %\kenneth{UPDATE NUMBER} \steven{updated}
In terms of usage frequency, 16 participants used LLMs daily, 3 used them weekly, and 1 used them monthly. %\kenneth{UPDATE NUMBER} \steven{updated}
%Interaction durations varied: eight participants engaged with LLMs for more than 30 minutes, four for less than 5 minutes, three for 5-15 minutes, and four for 15-30 minutes. 
While most participants used LLMs for general tasks such as Q\&A, research, writing assistance, and programming/debugging, only 5 participants had experience using LLMs for data labeling. %\kenneth{UPDATE NUMBER}\steven{updated}
Participants rated their confidence in crafting prompts and their proficiency in interacting with LLMs similarly, with average scores of 3.75 (SD=0.85) and 3.85 (SD=0.88), respectively. %\kenneth{UPDATE NUMBER}\steven{updated}
%Many employed prompt engineering techniques, ranging from simple input adjustments to advanced methods like iterative refinement, system message editing, in-context learning, and chain-of-thought prompting to enhance LLM performance.
Overall, the participants represented individuals familiar with LLMs but relatively inexperienced with large-scale data annotation.



\subsubsection{Labeling Task, Scheme, and Data}
We selected the Coronavirus Tweet NLP Text Classification task, which categorizes tweets into five sentiment categories: Extremely Positive, Positive, Neutral, Negative, and Extremely Negative, using the dataset hosted on Kaggle.\footnote{Coronavirus tweets NLP - Text Classification: https://www.kaggle.com/datasets/datatattle/covid-19-nlp-text-classification/}
The dataset contains tweets from December 30, 2019, to September 7, 2020. 
For our study, we randomly sampled 1,060 tweets: 10 tweets were used for the tutorial task, 1,000 for the main study, and 50 for the final evaluation set (see Section~\ref{sec:study-procedure}).

This task was chosen, partially informed by our pilot study (Section~\ref{sec:pilot-study}), for several reasons. 
First, it strikes a balance in difficulty, being challenging enough to require iterative prompting efforts from LLMs, as a 5-class sentiment task is more complex than typical 2-class (positive, negative) or 3-class (positive, negative, neutral) sentiment classification tasks. 
Second, it avoids requiring specialized knowledge, ensuring a broad pool of potential participants. 
Tasks demanding domain-specific expertise would have significantly restricted recruitment; sentiment labeling for general COVID-related tweets is sufficiently accessible for this purpose. 
Finally, the task incorporates a subjective element, as it lacks universally agreed-upon gold labels. 
This aligns with our focus on ``prompting in the dark,'' where participants' understanding of the data, as well as labeling goals, evolve through iterations.
The subjective nature of the task allows participants to arrive at differing gold standards by the end of the process.
Considering these factors, we selected this task for our study.




%We chose this task during our pilot study (Section~\ref{sec:pilot-study}), primarily for its accessibility and its somewhat subjective nature: 
%sentiment analysis does not require specialized expertise; 
%the evaluation often depends on personal judgment.
%\kenneth{TODO Kenneth: This is not very accurate. need to revise later.}
%\steven{In contrast to tasks with strictly defined objectives, such as CODA-19~\cite{huang-etal-2020-coda}, which require extensive domain knowledge, subjective tasks like Twitter sentiment analysis do not have universally correct answers, relying instead on personal judgment for evaluation. 
%In this study, we encouraged participants to guide the LLM to align with their individual standards rather than steering it toward a fixed standard grounded in domain expertise}
%This \steven{task} allowed participants to guide the LLM according to their own interpretation of sentiment, particularly in deciding what qualifies as ``extremely'' positive or negative.





\subsubsection{Study Procedure\label{sec:study-procedure}}
For our main user study, most sessions were conducted remotely via Zoom or Microsoft Teams, with each session lasting between 87 and 127 minutes. 
Participants who attended in person used one of the author's laptops, while remote participants used their own computers. 
Since \system was a Google Add-on in a development version, it was installed on one of the author's laptops. 
Remote participants were given control of this laptop to conduct the experiment. 
Each session was recorded, capturing the screen, audio, and video for further analysis.

The study followed these steps:

\begin{enumerate}

\item 
\textbf{Onboarding:}
Participants were first introduced to the study's objectives and procedures, and their informed consent was obtained.

\item 
\textbf{Tutorial Task:}
Participants were then presented with a tutorial on the system's workflow and features, either delivered by one of the authors or via a prerecorded video, depending on their preference. 
Afterward, they completed a short tutorial task, identical to the main study task but involving only 10 data instances, to ensure their understanding of the system.

\item 
\textbf{Main Study:}
Participants were then asked to use \system to iteratively compose a prompt to label the sentiment of COVID-related tweets in alignment with their personal judgment of sentiment scores. 
Each participant was asked to complete at least four iterations (\ie, going through Steps 1 to 4 four or more times).
Participants were free to do additional iterations beyond the required four; on average, participants completed 4.75 iterations.
%\kenneth{update numbers}\steven{done}

Depending on their assigned group, participants used \system to annotate either 50 or 10 instances per iteration and then review the results.
For participants working with 50 instances per iteration, we advised that it was not necessary to manually verify all labels, as that would take too much time. 
% Participants in the group with access to LLM explanations could manually turn off the explanations if they felt that reading them was too time-consuming; only a few participants chose to do so.
Participants in the group with access to LLM explanations were explicitly informed that they
could manually turn on the explanations if they felt that they needed reasoning for each label. 
Most participants turned on the LLM explanations in the first iteration;
however, one participant chose to disable the LLM explanations after the first iteration and another participant decided not to use the LLM explanations throughout the entire study.
% , but a few participants chose to disable them after the first iteration. 
% One participant opted not to use the LLM explanations throughout the entire study.
%\kenneth{Is this true? How does this work?}\steven{the default is off, they can choose to turn on or keep it off.}\kenneth{hmmm how did we test the effect in such case then? Did they turn it on very often??? Did we always ask LLM to provide explanations?}\steven{All participants turned on the LLM explanation in the first iteration and use LLM explanation. and three participants decided to turn off the LLM explanation. }\kenneth{how about implementation? Did we always ask for LLMs to give us explanations in our prompt, just some participants do not have acees to it?}\steven{yes, the LLM explanation always in the raw output. Our LLM explanation checkbox is used to display or not display the LLM explanation. The label outputs remain consistent for all participants. }\steven{my bad, one participant did not turn on the LLM explanation the whole time.}

%After reviewing the LLM-generated labels, participants were encouraged to refine their Context, Rule Book, and Shots sheets if they gained new insights into the task, or to add gold shots.
%All participants were required to complete at least four iterations (i.e., going through Steps 1 to 4 four times). 

\item 
\textbf{Manual Labeling of the Gold Set:}
Upon completing the iterative process, participants manually labeled 50 tweets based on their own sentiment judgments. 
These labels reflected the participants' understanding of the data and the final labeling they aimed to achieve by the end of the study session. 
These manually labeled tweets were used as an evaluation dataset to assess the performance of the participants' prompts; they were not used to train or fine-tune any AI models.









%Upon completing the iterative process, participants manually labeled 50 tweets based on their own sentiment judgments. 
%These manually labeled tweets served as the evaluation dataset to assess the performance of their prompt.
%\kenneth{TODO Kenneth: (1) Say it captured at last understanding of users. (2) WE do not use it for any training or fine-tuning!!!}

\item 
\textbf{Post-Study Survey and Feedback Collection:}
At the end of the session, participants completed a questionnaire to rate the system's effectiveness, performance, and accessibility. They were also asked the following questions:
(1) Without this tool, how would you typically approach prompt engineering?
(2) How does your prompt engineering process compare before and after using this tool?
(3) Did the system help you complete the tasks more efficiently? If yes, please explain how.
(4) What features did you find most useful?
(5) Would you be interested in using this annotation system in your regular work or study? If no, please explain why.
(6) Do you have any suggestions for making the system more suitable for your needs?

Upon completing the questionnaire, we verbally asked participants to provide some last comments about the workflow, labeling task, and our system.



\end{enumerate}



%--------------- dead kitten -----------

\begin{comment}

\steven{
This setup allowed us to examine the effects of different sample distributions on annotation outcomes.
The choice of 10 instances reflects the limited time available to participants, allowing them to complete detailed annotations effectively.
In contrast, a previous HITL study without pre-set gold labels utilized 30 sample candidates per iteration for human selection~\cite{liu2019deep}. 
To balance this approach, the 50-instance group was introduced to explore a broader data distribution, allowing us to investigate the trade-offs between accessing essential and diverse samples in a human-in-the-loop system, as emphasized in ~\cite{wu2022survey, le2010ensuring}. Furthermore, considering that our sentiment task comprised 5 labels and involved random sampling, selecting 50 instances could significantly increase the chance of participants observing and engaging with all 5 labels.}


%https://docs.google.com/spreadsheets/d/1wcIDftEfVoDAAwXBnXoMgAhMeG-o-T-fB4CRuXlhy1g/edit?gid=0#gid=0
\subsubsection{Participants LLMs Background}\steven{participants background on going}
We distributed an additional questionnaire (Table~\ref{tab:participants-llm-background-survey}) to gather background information on LLMs from the participants who attended, offering a \$5 compensation. Out of 20 participants, 18 responded.\steven{fill in number later}

Most participants reported being familiar with using LLMs, with an average familiarity score of 4.28 (SD=0.75) on a 5-point scale. The majority displayed advanced knowledge of LLMs, while others showed an understanding of general principles. Only one participant reported having a basic level of understanding.
Near all participants had over one year of experience using LLMs, with only four reporting more than four months of experience and one reporting between one and three months of experience.
Almost all participants reported daily use of LLMs, except three reported weekly and one reported monthly.
Participants reported varying durations of interaction with the LLM. Eight participants interacted with the LLM for more than 30 minutes, four participants reported interactions lasting less than 5 minutes, three participants interacted for 5-15 minutes, and four participants reported interactions lasting 15-30 minutes.
Majority of participants indicated using LLMs for academic research, professional tasks, and personal projects. Two participants mentioned using LLMs for entertainment, while one reported using them for communication and another for translation and writing polishing. 
Participants employed LLMs for a wide range of tasks. 
Only four participants reported using LLMs for data labeling.
The majority engaged LLMs for general Q\&A, research, writing assistance, and programming or debugging. Half of the participants used LLMs for data analysis and visualization, while five employed them for creative tasks. Additionally, one participant sought conceptual explanations from LLMs. 
Participants rated their confidence in crafting prompts and their proficiency in interacting with LLMs to generate desired answers similarly, with an average score of 3.83 (SD=0.86) and 3.89 (SD=0.90) separately. 

Several participants use OpenAI's API for tasks such as dataset generation, research, virtual assistant creation, and integration into automation systems.
Many participants engage in prompt engineering, ranging from simple input adjustments to advanced techniques like iterative refinement, system message editing, in-context learning, and chain-of-thought (CoT) prompting for better LLM performance.
Participants use tools like OpenAI Playground, customized assistants, and external resources like Reddit for task-specific prompts.
Some participants either do not currently use GPT API but remain open to exploring them in the future.


While our participants did not have extensive data annotation experience, their familiarity with LLMs and general understanding of the Twitter Sentiment task requirements allowed them to engage meaningfully with the system. 
This aligns with our goal of simulating real-world scenarios where requesters -- whether researchers or practitioners -- guiding LLMs to align with their own standards.\steven{TODO: discussion their suitability}






\kenneth{--------------------------KENNETH IS WORKING HERE------------------------}


We finalized our experimental procedures after two user studies. 
The user evaluation dataset session was moved from the beginning to the end of the main experiment, as participants were likely to form a more consistent standard after reviewing numerous data instances
To preserve the integrity of the study, we kindly requested the first two participants (P1 and P2) to rejoin a 15-minute user study to re-annotate the evaluation dataset. After they completed the makeup session, we compensated them with \$5.\steven{we asked users for the makeup session}



\paragraph{Setup System Environment} During the system setup process, users should open the setting window by clicking the ``Setting'' icon located at the top right of the add-on interface. 
In our user study, we labeled the first top field as ``Participant ID'' to easily distinguish between participants. For real deployment, we will rename it to ``Data Annotation Task Name'' to allow users to track their iterative guidelines for LLMs in different tasks.
\steven {This participant ID was used during the user study, identifying each participant. I think we can change it to a data annotation name, which can be used to track different annotation tasks in the future. }\steven{I will add interface screenshot for both}
After users enter their participant ID/Annotation Task Name, they can click the ``Save Participant/Task Name'' button to save the information.
More importantly, users have to input their OpenAI API Keys and click the ``Save API'' button to save the key.





\paragraph{Pre-task interview} The steps of the study were as follows: We first introduced participants to the study objectives and procedure, securing their consent. Then, we presented the workflow and feature tutorials. Afterward, participants were asked to complete a short trial task to ensure their understanding of the system.


% The in-lab user study lasted approximately 90 minutes. The session consisted of a pre-task interview with a trial task (30 minutes), a main experiment (50 minutes), and a post-questionnaire (10 minutes). Both the trial task and main experiment used \textbf{COVID Twitter Sentiment Task} as the annotation task.

% In the pre-task interview, we presented an orientation to explain the purpose of the study. Then, each participant would listen to an informed consent form and provide verbal consent. After that, we provided participants with an instruction on our system and the annotation task. In the end, we asked participants to complete a short trial task to ensure their understanding of the system. 

\paragraph{Main experiment} Participants were instructed to complete at least four iterations. 
Participants used \system to annotate either 50 or 10 instances per iteration. For those who access 50 instances, we advised them to quickly glance at instances and labels during the verification process to gather as many insights as possible. 
For participants reviewing 10 instances per round, the number of instances was adjusted according to their verification speed due to time constraints. 
After the annotation process was completed, participants started to review each instance and LLM-generated labels. Those assigned to the group with access to LLM Explanation had the option to choose whether or not to display the LLM Explanation. 

Following the verification of LLM labels, participants were encouraged to refine their rule books if they discovered new insights into the task or to add gold shots if they identified a data instance and its corresponding label as a valuable reference point.
If time remained in the main experiment, participants were asked to perform additional refinement iterations. 

% During the refinement iteration, we observed that participants tended to review all the LLM explanations when provided with the options. This could potentially influence participants to align their judgments more closely with those of LLMs. 
% To mitigate potential bias, we split all participants randomly into two groups: one group worked with explanations, while the other worked without them. 

Upon completing all refinement processes, participants were asked to annotate 50 tweets based on their judgments for the sentiment task. These annotated tweets would serve as the evaluation dataset to assess the performance of LLM guided by the participants.

\paragraph{Post-questionnaire session} Participants completed a questionnaire, rating the effectiveness, performance, and accessibility of the system. They were also given questions: 
\textbf{(1) Without this tool, how would you typically approach prompt engineering?
(2) How would you compare your prompt engineering process before and after using this tool?
(3) Did the system help you complete the tasks more efficiently? If yes, please explain how.
(4) What features did you find most useful?
(5) Would you be interested in using this annotation system in your regular work or study? If no, please explain why.
(6) Do you have any suggestions for making the system more suitable for your needs?}
We recorded audio, captured screens, and recorded all system actions for each user study. 




% At the beginning of the user study, we structured our in-lab study session to last 60 minutes, focusing on using our system to refine prompts for the CODA-19 data annotation scheme~\cite{huang-etal-2020-coda}. The session consisted of a pre-task interview with a trial task (15 minutes), a main experiment (35 minutes), and a post-questionnaire (10 minutes). 

% After the first pilot study, we observed that the LLM performance on the CODA-19 was too good for the first participant to provide nuanced insights to improve prompts further. 
% Thus, we modified our annotation task to a Coronavirus tweet NLP Text Classification task~\footnote{https://www.kaggle.com/datasets/datatattle/covid-19-nlp-text-classification/}, which we will refer to as the COVID Twitter Sentiment Task in subsequent sections.
% However, after conducting the first two pilot studies, we found that the learning phrase on the system took more time than anticipated, and the main experiment annotation also required more time. To improve accessibility and track the refinement process, we extended our session to 90 minutes, increasing the pre-task part to 30 minutes and the main experiment to 50 minutes. 



% \subsection{Deployment Study}


\subsubsection{In-lab Session Procedure}
Most sessions were conducted remotely through Zoom or Microsoft Meetings. Each session typically spanned 87 minutes to 127 minutes. Participants who attended in person used one of the author's laptops, while those who joined remotely used their computers. Since the system was a Google Add-on development version, it was installed on one of the author's computers. Hence, we granted remote participants control of the author's laptop to experiment. 

\subsubsection{Annotation Scheme}
For our in-lab study, we chose to use a Coronavirus tweet NLP Text Classification task~\footnote{https://www.kaggle.com/datasets/datatattle/covid-19-nlp-text-classification/}, which categorized tweets into one of five categories, \textit{i.e.}, Extremely Positive, Positive, Neutral, Negative, and Extremely Negative. In subsequent sections, we will refer to this task as the \textbf{COVID Twitter Sentiment Task}.

This task was picked for its accessibility and flexible standards:
it did not require expertise and knowledge in analyzing tweet sentiments; evaluation of sentiment analysis often relied on personal judgment, allowing different participants to guide the LLM according to their personalized classification standards.

\subsubsection{Dataset}
The COVID Twitter Sentiment dataset contains tweets from December 30, 2019, to September 7, 2020 from Twitter. In this study, we randomly sampled 1,060 tweets from the dataset. We used 10 tweets for the trial task, 1,000 for the main experiment, and 50 for the evaluation set.


For the main study, the participants were recruited via authors' networks, social media posts, and mailing lists in the authors' institute. 
We recruited 20 participants from diverse educational backgrounds (1 Post-doctoral Researcher, 9 Ph.D. students, 9 Master's students, and 1 Undergraduate student). 
All participants had experience with using LLMs. 
Participants were compensated with \$20 for their participation. 
In our analysis presented in the paper, we denote participants as P1 to P20.

Participants were randomly and evenly assigned to whether accessing to LLM Explanation or not and were further assigned to whether access to 50 instances per round or not. 
\end{comment}

%\subsection{DSPy Fine-tuning}



%--------------- dead kitten ----------
\begin{comment}


\kenneth{--------------------------KENNETH IS WORKING HERE-----------------------------------}


We aimed to track accuracy trends in data annotation tasks throughout human prompt refinement processes. 
This section detailed the procedure. 
Previous study indicates that LLM explanations enhance understanding of the context~\cite{ma2023insightpilot,singh2024rethinking}. 
To explore this, we implemented two distinct conditions in our study: one group was given access to \textbf{LLM Explanation}, while the other group was not.
Due to time constraints, participants were only able to work with a limited number of data instances per iteration. We established two additional conditions: in one, participants worked on approximately 10 instances per iteration, allowing them to review each instance and its labels in detail. In the other, participants were presented with \textbf{50 instances} and instructed to quickly skim through them to gather insights.\steven{TODO: find papers. } \steven{two variables: LLM Explanation and 50 instances}

% and as individuals explore more data instances, they are able to uncover deeper insights within the data\steven{find papers}; 
% Thus, 

    
\end{comment}

\section{Findings}
\section{Results and Discussion}
\label{sec05}

In this section, we present the results, discuss them, and make some conclusions about the experiments.

With a slightly realistic scenario, the experiments present some interesting results. Figures \ref{fig:hist_score} and \ref{fig:hist_gen} show respectively histograms of (a) the final score after the full training process and (b) the number of generations the process took. Notice that most runs just stopped at 20 generations (maximum) and could not improve further, as Figure \ref{fig:hist_gen} suggests. Despite that, as can be seen in Fig. \ref{fig:hist_score}, more than 80\% of the runs ended with a score of 2 or less, meaning at most two wrong device activations on 260 interactions. 

\begin{figure*}
        \centering
        \begin{subfigure}[b]{0.475\textwidth}
            \centering
            \includegraphics[width=0.8\textwidth]{imgs/results/results_2/histogram_of_score_.png}
            \caption[]%
            {{\small Histogram of \textit{score} achieved on experiment runs. Notice that the lesser, the better.}}    
            \label{fig:hist_score}
        \end{subfigure}
        \hfill
        \begin{subfigure}[b]{0.475\textwidth}  
            \centering 
            \includegraphics[width=0.8\textwidth]{imgs/results/results_2/histogram_of_generations_.png}
            \caption[]%
            {{\small Histogram of \textit{generations} needed to achieve the lower score. Here, most runs needed the maximum number of generations}}    
            \label{fig:hist_gen}
        \end{subfigure}
        \caption[]
        {\small Histograms of the lowest score and generations needed to achieve that on each experiment run.} 
        \label{fig:hist_metrics}
\end{figure*}

Figures \ref{fig:hist_beh} and \ref{fig:hist_per} reflect the number of Behavioral and Perceptual Codelets respectively to achieve the best result in each run. As we can see in Fig. \ref{fig:hist_beh}, most runs needed 13 Behavioral Codelets, one for each Motor Codelet/actuation device. Fig \ref{fig:corr_behavior} shows the correlation between the number of behavioral codelets and score. The system response is better (lower score) as more Behavioral Codelets are used.
The number of Perceptual Codelets, on the other hand, shows approximate normal distributions with a slight bias to the right, meaning that the embedding may vary and the output still be good. This bias is reflected in the slight negative correlation between the number of Perceptual Codelets and Score (smaller than Behavioral).

\begin{figure*}
        \centering
        \begin{subfigure}[b]{0.475\textwidth}
            \centering
            \includegraphics[width=0.8\textwidth]{imgs/results/results_2/histogram_of_behaviorals_x.png}
            \caption[]%
            {{\small Histogram of the number of Behavioral Codelets needed to achieve the lowest score. Most runs needed 13, the same number of Motor Codelets (and actuators).}}    
            \label{fig:hist_beh}
        \end{subfigure}
        \hfill
        \begin{subfigure}[b]{0.475\textwidth}  
            \centering 
            \includegraphics[width=0.8\textwidth]{imgs/results/results_2/histogram_of_perceptuals_x.png}
            \caption[]%
            {{\small Histogram of the number of Perceptual Codelets needed to achieve the lowest score.}}    
            \label{fig:hist_per}
        \end{subfigure}
        \caption[]
        {\small Histogram of the number of ''internal'' Codelets needed to achieve the lowest score on each run.} 
        \label{fig:needed_codelets_2}
\end{figure*}


\begin{figure*}
        \centering
        \begin{subfigure}[b]{0.475\textwidth}
            \centering
            \includegraphics[width=0.8\textwidth]{imgs/results/results_2/correlation_correlation_of_score_with_behaviorals.png}
            \caption[]%
            {{\small Correlation between number of Behavioral Codelets and Score}}    
            \label{fig:corr_behavior}
        \end{subfigure}
        \hfill
        \begin{subfigure}[b]{0.475\textwidth}  
            \centering 
            \includegraphics[width=0.8\textwidth]{imgs/results/results_2/correlation_correlation_of_score_with_perceptuals.png}
            \caption[]%
            {{\small Correlation between number of Perceptual Codelets and Score}}    
            \label{fig:corr_per}
        \end{subfigure}
        \caption[]
        {\small Correlation between the number of ''internal'' Codelets and Score} 
        \label{fig:corr}
\end{figure*}



Table \ref{table:exp2} shows some statistics taken from the experiment. Notice that, while the individual number of Perceptual and Behavioral Codelets may go as low as 2, the combined ''Internal'' Codelets need a higher number to present satisfactory results.

\begin{table}[]
\centering
\caption{metrics on Experiments}
\label{table:exp2}
\begin{tabular}{@{}llllll@{}}
\toprule
              & mean    & median & std   & min & max \\ \midrule
score         & 1.438   & 1.0    & 1.894 & 0   & 26  \\
generations   & 13.6784 & 20.0   & 8.905 & 0   & 20  \\
n perceptuals & 9.5326  & 10.0   & 2.213 & 2   & 15  \\
n behaviorals & 11.2674 & 13.0   & 2.478 & 2   & 13  \\
n internals   & 20.8    & 22.0   & 3.998 & 7   & 28  \\ \bottomrule
\end{tabular}
\end{table}

\subsection{Conclusion and Future Works}

This paper has presented a pioneering approach to creating a Cognitive Twin by leveraging a distributed cognitive system in conjunction with an evolution strategy. Our work stands as a significant contribution to the field of cognitive computing by demonstrating the feasibility of orchestrating a multitude of simple physical and virtual devices to mimic a person's interaction behaviors. This achievement not only offers a practical application of distributed cognitive systems but also introduces a novel methodology for cognitive twin development, emphasizing the role of evolution strategies in optimizing system topology for more accurate behavior emulation.

In revisiting the themes introduced at the outset, our research seamlessly integrates the foundational principles of cognitive systems, Cyber-Physical Systems (CPS), and Systems of Systems (SoS) with contemporary advancements in artificial intelligence. By doing so, we have illustrated a comprehensive framework that not only addresses the complexities of human behavior simulation but also opens new avenues for automation, human-like agent creation, and in-depth behavioral analysis.

Comparatively, our approach distinguishes itself from established cognitive architectures such as ACT-R and SOAR, and the Standard Model of Mind, by emphasizing distributed processing and adaptability. While ACT-R and SOAR offer rich insights into cognitive processes through detailed psychological models, our model excels in harnessing distributed, interconnected devices to capture the multifaceted nature of human cognition. Similarly, the Standard Model of Mind provides a foundational framework for understanding cognitive functions. Yet, our work extends this understanding into the practical domain of CPS and distributed systems, offering a unique perspective on cognitive replication and interaction dynamics.

In conclusion, our research not only underscores the potential of distributed cognitive systems in creating sophisticated cognitive twins but also highlights the importance of evolutionary strategies in refining these systems. By drawing parallels and distinguishing our work from established cognitive architectures like ACT-R, SOAR, and the Standard Model of Mind, we contribute a novel perspective to the ongoing discourse on cognitive modeling and simulation. 



Future work will focus on further refining the distributed cognitive system and exploring its integration with other AI paradigms and models. This research sets the stage for developing more sophisticated Cognitive Twins capable of performing complex tasks with minimal human intervention. By continuing to build on this foundation, future studies can enhance the fidelity and applicability of Cognitive Twins, making them tools in the field of cognitive computing.
    


\section{User Feedback}
In addition to addressing the main research questions, a post-study survey (Appendex~\ref{sec:post-question-survey}) consisted of twenty-two questions, including seven Likert scale ratings and fifteen free-text responses from participants provided valuable insights on both ``prompting in the dark'' practices and our system.
We summarize these insights in this section.

%\kenneth{I re-organized this section. Please take a look.}

\subsection{Two Variables Impacting Participant Ratings\label{sec:two-var-on-rating}}
% \alan{Two variables impacting participant ratings?}
Figure~\ref{fig:user-rating} displayed the seven Likert scale rating responses by participants. The seven survey questions can be categorized into seven different categories. 
Appendix~\ref{app:two-var-on-rating} shows
the survey questions and the accompanying categories were rated on a seven-point Likert scale. 

\begin{comment}


listed below:

\begin{itemize}
    \item 
    \textbf{(Q1) Understandable}: The annotation task was easy to understand.

    \item
    \textbf{(Q2) Ease of Use}: The annotation tool is easy to use.

    \item
    \textbf{(Q4) Intuitive System}: The interface of the annotation system is intuitive.

    \item
    \textbf{(Q5) Performance Satisfaction}: How satisfied are you with the performance of the system?

    \item
    \textbf{(Q6) Prompt Improvement}: This tool was helpful in improving my prompt. 
    
    \item
    \textbf{(Q7) Process Efficiency}: Using this tool made the process of prompt engineering more efficient.

    \item
    \textbf{(Q19) Task Completion}: I completed the annotation tasks efficiently.

\end{itemize}
    
\end{comment}

%Figure~\ref{fig:user-rating} shows the participant's rating across different conditions.
\subsubsection{Participants reviewing 10 instances reported higher satisfaction ratings.}
Figure~\ref{fig:sample-size-rating} compares participants who reviewed 10 instances per iteration with those who reviewed 50.
Both groups provided similar ratings for system ease of use, system intuitiveness, and efficiency in processing prompt engineering, with comparable variation.
However, participants who reviewed 10 instances found the annotation tasks more difficult to understand compared to those who reviewed 50. Comparatively, participants who reviewed 10 instances reported higher levels of satisfaction with their performance, a stronger sense of prompt improvement, and better task completion rating. This could be attributed to their minimal modifications to the rule.
% We performed a KS test on all rating categories and no significant difference was found between two groups, indicating that the observed difference did not reach statistical significance.
It is noteworthy that we performed a Kolmogorov-Smirnov (KS) test on all rating categories, and no significant difference was found between the two groups, indicating that the observed difference did not reach statistical significance.

% \steven{Only the performance ratings from participants showed a significant difference between two instances groups based on the t-test (p-value=0.031)}

\subsubsection{Participants without LLM explanations rated the system as more intuitive, effective, and satisfying.}
Figure~\ref{fig:explanation-rating} shows the comparison of ratings between participants with and without LLM explanations.
Participants with LLM explanations found the annotation tasks more challenging, rated the system as less intuitive and harder to use, and viewed it as less effective in improving prompts, also with greater variation in their ratings. In contrast, participants without LLM explanations expressed higher level of satisfaction with the system performance, believing the tool improved prompt engineering efficiency and task completion effectiveness.
% We conducted a KS test on all ratings from participants and no significant difference was found between the two groups, suggesting that the observed difference were not statistically significant.
Notably, we conducted a Kolmogorov-Smirnov (KS) test on all participants' ratings, and no significant difference was found between the two groups, suggesting that the observed differences were not statistically significant.


\begin{figure*}
    \centering
    \begin{subfigure}[t]{0.48\textwidth}
        \includegraphics[width=\linewidth]{Figures/Post-Survey/user_rating_bar_chart_instance.png}\Description{This subplot is for the data sample group, based on participants’ post-survey responses. Each subplot features bar charts comparing two settings: 50 instances vs. 10 instances. Each bar is accompanied by a confidence interval displayed at the top.}
        \caption{Participants' ratings of the system and the annotation task, comparing those who accessed 10 instances per iteration to those who accessed 50 instances.}
        \label{fig:sample-size-rating}
    \end{subfigure}
    \hfill
    \begin{subfigure}[t]{0.48\textwidth}
        \includegraphics[width=\linewidth]{Figures/Post-Survey/user_rating_bar_chart_explanation.png}\Description{This subplot is for the explanation group, based on participants’ post-survey responses. Each subplot features bar charts comparing two settings: no explanation vs. explanation. Each bar is accompanied by a confidence interval displayed at the top.}
        \caption{Participants' ratings of the system and the annotation task, comparing those who utilized LLM explanations to those who did not.}
        \label{fig:explanation-rating}
    \end{subfigure}
    \caption{Participants' rating of the system and the annotation task. Each rating category refers to one question in the post-study survey. 
    \textbf{(Q1) Understandable}: The annotation task was easy to understand; 
    \textbf{(Q2) Ease of Use}: The annotation tool is easy to use;
    \textbf{(Q4) Intuitive System}: The interface of the annotation system is intuitive;
    \textbf{(Q5) Performance Satisfaction}: How satisfied are you with the performance of the system? 
    \textbf{(Q6) Prompt Improvement}: This tool was helpful in improving my prompt; 
    \textbf{(Q7) Process Efficiency}: Using this tool made the process of prompt engineering more efficient; 
    \textbf{(Q19) Task Completion}: I completed the annotation tasks efficiently.}
    \label{fig:user-rating}
\end{figure*}

%\steven{added user rating.}\kenneth{(1) Font for axis titles and the title of the figure are too big, (2) Make the figure wider so that the x-axis labels do not need to rotate (use newline for x-labels if possible), (3) this is kinda extra: can we break it down in two ways (a) with/without explanations and (b) smaller/bigger sample size.}\steven{done}

\subsection{Is \system Useful?}
\subsubsection{Participants considered \system helpful and efficient.}
In the post-study survey, we asked participants to rate the (Q5) performance satisfaction, (Q6) helpfulness of the tool, and (Q7) its efficiency on a seven-point Likert scale.
% The detailed questions asked are shown in Appendex~\ref{sec:post-question-survey}.
%\kenneth{TODO Steven: Update refernece}
Participants expressed high satisfaction, with an average rating of 6.350 (SD=0.745), and found the system helpful for improving prompts (6.400, SD=0.883) and making prompt engineering more efficient (6.600, SD=0.598).
%\kenneth{TODO Steven: Add numbers---- Are there really high???}
P4 noted, ``\textit{I really like this tool instead of traditional prompt engineering on ChatGPT and Copilot.}''
%\kenneth{TODO Steven: Add a few more examples for other participants.}
P15 mentioned, ``\textit{I would be interested in using this annotation system in my regular work or study, because I really like the idea [of] improving annotation performance by considering iteration annotation process between human and the GPT.}''
% A participant (P18) wrote in the questionnaire, ``I like how the system's UI has been designed and programmed.''



%\kenneth{Not sure how to phrase this...}
\subsubsection{\system is easy to use but less intuitive and with a steep learning curve.}
We asked participants to rate whether ``\textit{(Q1) The annotation task was easy to understand},'' ``\textit{(Q2) The annotation tool is easy to use},'' and ``\textit{(Q4) The interface of the annotation system is intuitive}'' on a seven-point Likert scale from ``Strongly Disagree'' (1) to ``Strongly Agree''(7). 
The average score for the ``easy to understand annotation task'' was 5.812, for the ``easy to use annotation tool'' was 5.375, and for the ``system is intuitive'' was 5.250. Suggesting that while the task and tool itself are not hard to understand and use, learning to properly use the tool can be harder for participants and required some learning.
For example, it was noted that the need to switch between tabs during the task can cause confusion.
% One participant mentioned that the system workflow was unclear, as he had to switch between different tabs, causing confusion and disruption.



\subsubsection{Participants found the Shots and Rule Book useful.}
%\paragraph{Having the flexibility to structure tasks freely and make adjustments on the go—whether modifying the Gold Shots or the Rule Book—reduces the burden of the traditional, iterative labeling process.}
We asked participants, in a free-text format, ``\textit{(Q10) What features did you find most useful?}'' Fourteen participants specifically mentioned that `Gold Shots' were particularly valuable, as they provided explicit examples to guide LLMs. 
Additionally, six participants highlighted the usefulness of the Rule Book. 
These two features stood out among the responses, demonstrating their importance in enhancing the user experience.
Participants noted that the flexibility to structure tasks freely and make on-the-fly adjustments---such as modifying the Gold Shots or Rule Book---eases the burden of the traditional iterative labeling process.







% We asked participants, ``What features did you find most useful?'' in free text form and 14 participants mentioned that ``Gold Shots'' were useful and could explicitly provide examples to guide LLMs.
% % 14 participants (P1-P6, P9, P11, P13, P14, P15, P16, P18, P20) thought the ``Gold Shot'' feature was useful as it could explicitly provide examples to guide LLMs. 
% On the other hand, availability of the Rule book was considered useful for 6 participants. 
%\steven{there are some participant like both.}

%\kenneth{Not sure about this...}




\subsubsection{Dilemma of showing LLM explanations.}
%\paragraph{Participants' desire to include LLM explanations.} 
Although our study found that providing LLM explanations can sometimes lead participants to generate labels more aligned with those produced by the LLM, participants still expressed a strong desire to have them included. P8 explicitly recommended incorporating LLM explanations, noting that participants were interested in understanding the reasoning behind potential discrepancies between their own labels and those generated by the LLM. P15 also emphasized the value of these explanations, stating, ``\textit{The explanation from GPT gave me some insights to modify my rules,''} and, ``\textit{I think GPT's explanation of the tweets is very helpful and it may help me to improve the accuracy of human annotation.''}

\begin{comment}
 

\subsubsection{What can be improved in \system?}
%\paragraph{Workflow can be confusing and interface not user friendly}
Some participants faced challenges when learning the system, and it can take them a long time to get comfortable with it. P1 said, ``\textit{The system is not logically clear for me because the system needs to jump in between different tabs, which is different than a normal workflow.}'' 
Some participants became confused with the system, even after several iterations. For instance, a few participants forgot that to proceed to the next iteration, they needed to sample the data and click ‘Start Annotation.’ Research team members need to remind participants of the workflow continuously. Additionally, the nature of the task requires moving from one tab to another, which adds difficulty for participants to navigate the interface and causes confusion.

   
\end{comment}

\subsection{About ``Prompting in the Dark''}

%\kenneth{Not sure about this too}
\subsubsection{Prompting in the dark without any tool is common.}
%\alan{Iterative, trial and error prompting is common without the help of tools.}}
We also asked participants, ``\textit{Without this tool, how would you typically approach prompt engineering?}''
We found that many participants commonly rely on iterative, trial and error strategies. Specifically, they start with prompts from scratch, test them on data points, adjust based on incorrect labels, and re-test until they are satisfied with the results.
% Responses varied: P0 emphasized giving LLMs as much context as possible\steven{this is P0, which is not included in our actual user study}, while others (P1, P3, P4) described a general refining process---starting prompts from scratch, testing on data points, adjusting based on incorrect labels, and re-testing until satisfied.
%\kenneth{TODO Steven: Add a few more examples for other participants.}
%\alan{trial and error, iterative process, personify}

P1 said, `\textit{I need to start with a prompt from scratch; then I will test it on real data points; I will observe those wrongly labeled data points and adjust my prompt accordingly. After the adjustment, I will rerun the testing on the real data points. The whole process is trail-and-error, which is really time-consuming and labor-consuming.}''
P3 stated, ``\textit{Give an initial prompt, if the answer is not meeting expectation, then change the prompt.}''
% P5[Keep trying different prompts to see if the responses I get satisfies my expectations.]
P6 reflected, ``\textit{Normally, if I do not get the desired output from the LLM, I will try to give more  specific instruction maybe some examples.}''
% P14[Try prompting with ChatGPT, if ChatGPT cannot provide a good answer, just rephrase the prompt and ask the question again.]
P17 said, ``\textit{I re-write my prompts several times (3-5 times) until I got an output that I like.}''

\begin{comment}
 

We also want to emphasize that, although not common, some participants employ a personification method by asking LLMs to take on a specific role or personality and make decisions based on that role. For example, P12 stated, ``\textit{I will first assign a role to GPT like ``Supposing you are an expert in coding, ...''. Then I will ask it to follow my instructions.}''
% Personify, P103 [Make llm assume that it is not an AI and act as a specific person who is involved in that specific activity. By Giving as much context as possible.], P12 [I will first assign a role to GPT like ``Supposing you are an expert in coding, ...''. Then I will ask it to follow my instructions.]
P13 said, ``\textit{I would narrate the incident or situation environment and then give prompt asking specific and questions clearly.}''

   
\end{comment}

% \paragraph{\system gave participants new insights into prompt engineering.}
% Interestingly,
% our survey also showed that using \system gave participants new insights into prompt engineering. 
% For example, P1 observed that ``adding or deleting a few words can change the overall output,'' and P2 noted, ``I can compare how many labels are correctly labeled before and after modifying prompts.''
%\kenneth{TODO Steven: Add a few more examples for other participants.--- This one is interesting. Say a few more if possible?}




\subsubsection{Prompting in the dark is hard, as participants lacked confidence in their labels.}
%\paragraph{Participants lacked confidence in labeling}
Without a comprehensive understanding of the entire dataset, participants found it challenging to generate suitable labels.
P19 mentioned, ``\textit{I am not confident about the label}''.
P12 pointed out that, ``\textit{When I need to express sentiment, I tend to be more reserved and avoid extremes. So, when labeling data, I usually prefer to choose negative/[positive] rather than extremely negative/[extremely positive]}''

%We also found that,
%\paragraph{Difficulty in capturing the full picture with limited samples}
%with participants engaging with only around 10 instances per round, they can overanalyze the limited tweets, resulting in a narrow refinement of their rule books. 
%This limitation also affected the gold standards labels for LLM learning. 
%P19 suggested, ``\textit{I didn't see any `Extremely Positive' in each verification round, but I saw five for now in the evaluation set.}''.

% \paragraph{Hard to revise Rule Book}
% Some participants preferred to write detailed label definitions at the start, which resulted in fewer rule revisions in the later rounds. Alternatively, they tended to add gold shots than editing rules.




% Some participants had difficulty understanding how to respond to context questions regarding the annotation task. \alan{clarify}They mistakenly answered the question by describing the system's workflow rather than focusing on the Twitter task. P3 and P8 suggested having more context tutorials for participants to understand the annotation task. 
% P9 pointed out, ``the `Start Annotation' button is hard to find.''.
%\kenneth{this is probably, in part, due to the nature of the problem}

%\subsection{Challenges and Pain-points}
%\kenneth{The following basically said we found the problems at the beginning of our study and what we did to mitigate them.}
% \paragraph{Difficulty of navigating the interface}
% The participants’ experience was not smooth, with most challenges centered around the workflow, such as moving from tab to tab and the complexity of the system features. 
% The most frequent issues participants encountered involved confusion with the system’s annotation procedure, which sometimes indicated users getting used to the interface.\alan{clarify} The issue often disappeared once users became more familiar with the system.
% Some participants (P2 and P7) struggled to respond to context questions\alan{?}, which were intended to gather information related to the Twitter sentiment analysis, such as the purpose and usage of the annotated data. However, they mistakenly focused on describing the system’s workflow rather than focusing on the annotation task.
% Reflecting on user feedback, we improved the tutorial by switching from a video format to a step-by-step manual walkthrough, ensuring participants could fully understand each tab and its purpose while also allowing them to ask questions at any point.\alan{does this actually help?}
% To help participants better understand the Twitter Sentiment task, we provided template answers that they could use as a reference to create their own responses or adopt directly.


%\subsubsection{Participants with 50 instances per iteration}

%\subsubsection{Participants with 10 instances per iteration}
 





%\kenneth{An important context here is: Our study results showed that (1) NOT showing LLM explanations is better, and (2) showing LLM explanations make users' labels more similar with each other, i.e., LLMs tailor users to be more like LLMs.}

% P10 liked ``The LLM explanations for every tweet after the levels were filled out '' the most.

% P15 mentioned, ``The explanation from GPT gave my some insights to modify my rules.'' and ``I think GPT's explanation of the tweets is very helpful and it may help me to improve the accuracy of human annotation.''

% \paragraph{Participants influenced by LLM explanations.} 
% For the first four participants (P1-P4), the system offered an LLM explanation option during the annotation process. We observed that each participant reviewed these explanations thoroughly, even though they were instructed to verify the LLM labels using their own judgment. 

% %\subsubsection{Participants with LLM Explanations}
% %\subsubsection{Participants tended to be impacted by LLM Explanations} 
% Participants tended to review LLM explanations thoroughly during the verifying process, even though they were instructed to verify the labels using their own judgment. 

%\paragraph{A few participants turned off the LLM Explanation option}

%\subsubsection{Participants without LLM Explanations}



\subsection{Users' Suggested Features}

\subsubsection{More automated supports for rule creation.}
Participants expressed concerns about creating rules that effectively suit the labeling task at hand. As a result, support for Rule Book creation is a welcome addition.
P2 remarked, ``\textit{It was hard to set the right rules,}'' while P3 suggested providing initialized instructions for labels and rules to ease the process. Additionally, participants (P1, P3) proposed that new rules could be automatically generated based on Gold Shots, existing rule books, and human explanations, streamlining the rule creation process.

%\kenneth{Can we make subsubsection title a complete sentence?}
%\alan{something like this?}
\subsubsection{Shorter LLM explanations for easier consumption.}

Although LLM-generated explanations received positive feedback from participants, there was concern about the length of these explanations. Many felt that the explanations were too long and could be difficult to consume. P1 recommended, ``\textit{It would be better if the LLM explanation could be shorter.}'' emphasizing the need for more concise outputs to improve user experience.





% P9 pointed out, ``the `Start Annotation' button is hard to find.''.


% \paragraph{Simplified Interfaces and Workflows.}
% Some participants suggested hiding uncommon features (P17) or displaying the function descriptions only when hovering over them (P20).

%\paragraph{Other suggestions}\steven{Not related to the annotation system}
% (P1)If there's a graphical UI, the system will be more accessible to laypersons without the HCI background. Currently, the system is too flexible to get lost in the interactions.
%(P3) More tutorials on the data annotation task.



%\subsection{Participant In-Lab Annotation Process Analysis}



%\subsection{Participant Self-Reported Response Analysis}



\section{Discussion}
\section{Discussion}
\label{sec:discussion}

In this section, we first summarize the conclusion and share some key observations. Then, we reflect on the usability of our method and propose potential applications. In the end, we discuss the limitations and future work.

\subsection{Effectiveness of \name{}}
\label{sec:discuss_effectiveness}
Firstly, based on the results from Section~\ref{sec:experiment}, we can draw the following conclusions:
\begin{itemize}
    \item It is efficient to detect unknown words by combining linguistic characteristics provided by the pre-trained language model (PLM) and gaze trajectory.
    \item The prediction is mainly based on the linguistic features from the textual context captured by PLM.
    \item Gaze locates the region of interest in a timely manner, which is necessary for real-time applications. Gaze also helps improve the model performance, but its contribution is limited compared to PLM.
\end{itemize}

Additionally, it is interesting that while we typically assume that the gaze modality should contribute significantly to the task of unknown word detection, the experimental results show that the contribution of gaze to the model’s improvement is small with the existence of PLM. Based on the previous analysis of line spacing and eye tracker accuracy, a possible reason for this is that under normal reading conditions (single-line spacing, line height 3-5 mm), the eye tracker’s accuracy is insufficient to precisely detect which line the gaze belongs to, thus failing to accurately locate the gaze on the words. Furthermore, changes in user posture during long reading sessions further reduce the accuracy of the eye tracker. In our system, PLM compensates for this issue by providing linguistic information based on the text.

From another perspective, the low contribution of gaze is not necessarily a disadvantage. Our method’s reduced reliance on gaze makes it more tolerant of noise. The model’s good performance on data collected by webcams further supports this conclusion. The reduced dependency on gaze data allows our model to be applied on more affordable and accessible devices, such as webcams.

\subsection{Usability of \name{}}
\label{sec:discuss_usability}
The results from the user evaluation (Section~\ref{sec:user_evaluation}) show that our reading assistance prototype helps users read more fluently and they are more willing to use it compared to traditional click-to-translate methods. In addition to providing real-time translation and explanations during reading, our system can also benefit ESL for long-term learning. For example, based on the unknown word detected by our system, we can generate a vocabulary list for memorizing and offer memory curve tracking. Furthermore, these unknown words can also be used to generate personalized summaries and notes.

The potential issue of generalizability across users, texts and devices can be addressed through fine-tuning and reinforcement learning methods. During the initial phases of usage, the system collects both gaze and text data for fine-tuning and lets users provide feedback on the model's predictions. This allows the model to continuously learn the user's unique gaze patterns and infer their vocabulary proficiency and domain expertise from textual content, thereby improving prediction accuracy.

\subsection{Limitation and Future Works}
\label{sec:discuss_limitation}
The quality of gaze data hinders the improvement model performance. The accuracy of the eye tracker is not enough for word-level detection. Common formatting, such as single-line spacing and 10-point font, results in a line height of approximately 3-5 mm when viewed using the PDF viewer with a sidebar on a 14-inch laptop. This requires an accuracy of about $0.3-0.6^\circ$ at a reading distance of 50-60 cm. However, most eye trackers have a gaze accuracy ranging from $0.2-1.1^\circ$~\cite{gaze_survey_2024}. Combined with additional errors caused by head and upper body movements, this level of accuracy is insufficient for real-world reading scenarios. During data collection and evaluation, some participants reported that even after calibration, the error could span 1-3 lines. This makes it difficult to determine the specific word the user is focusing on based solely on gaze coordinates, explaining why gaze-based baselines performed poorly on our data.

\change{The inaccuracy of the gaze data could also lead to the inaccuracy of data labeling. To mitigate the impact of mouse clicks on gaze behavior, we asked users to label unknown words during their second pass. However, this widely adopted labeling method inherently requires "guessing" which words correspond to a given gaze trajectory. Previous works mapped each gaze coordinate directly to a specific word to establish word-gaze pairs. This method is infeasible for text with normal line spacing, so we establish gaze-word pairs by defining a bounding box based on a segment of gaze to identify the corresponding words instead. While this approach improves robustness, it may also introduce mismatches between gaze and words and thus introduce noise to the dataset. To further improve model performance, more precise labeling methods are needed.}

Additionally, reading time can be longer than several minutes in daily scenarios, so gaze drift can significantly affect data quality. In our experiments, we observed that it is difficult for participants to maintain a fixed posture after calibration, though we required them to do so. The posture shift further increases errors. Therefore, in practical applications, real-time calibration of gaze data based on user posture is crucial to ensure data quality. If the existing eye-tracking technology can combined with user posture detection~\cite{faceori}, it is possible to reduce the impact of user posture on gaze data, thereby improving the quality of gaze data.




\section{Conclusion and Future Work}
\section{Conclusion}
\label{sec:Conclusion}
This work evaluates proprietary and open-weight models in agentic frameworks for handling ambiguity in software engineering. In code generation, to effectively integrate new information into the solution, an agent must detect ambiguity and ask targeted questions. Our key findings are:
\begin{itemize}[itemsep=0pt, topsep=0pt]
    \item Given an underspecified input, Claude Sonnet 3.5 and Claude Haiku 3.5 with interaction can achieve 80\% of their performance with a well-specified input. In contrast, open-weight models struggle: Deepseek relies on navigational cues to locate relevant files, while Llama 3.1 70B extracts limited information from the user.
    \item LLMs do not interact unless explicitly prompted, and their ambiguity detection is highly sensitive to prompt variations. Only Claude Sonnet 3.5 achieves a higher accuracy of 84\% in distinguishing between well-specified and underspecified input.

    \item Claude Sonnet 3.5, Haiku 3.5, and Deepseek effectively extract new, detailed user information, whereas Llama 3.1 struggles to ask the right questions.
    
\end{itemize}
Despite these advances, a gap remains between resolve rates for underspecified vs. fully specified issues. Open-weight models need better interaction strategies to improve resolution, while proprietary models, particularly Claude Haiku 3.5, require stronger prompting to engage interactively. This work establishes the current state-of-the-art in handling ambiguity through interaction, breaking the resolution process into multiple steps.




\begin{acks}
\input{Sections/8.5-Acknowledgement}
\end{acks}

\bibliographystyle{Style/ACM-Reference-Format}
\bibliography{Bibtex/sample-base,Bibtex/software,Bibtex/main}

%TC:ignore
\appendix
\newpage
\appendix
\onecolumn

\part{}
\section*{\centering \LARGE{Appendix}}
\mtcsettitle{parttoc}{Contents}
\parttoc

\clearpage

\section{Related Work}
\label{sec:relatedwork}
% \paragraph{Tool Usage and Toolchain Management} Research in this area focuses on how intelligent agents design and optimize tool networks to effectively execute complex tasks, particularly by dynamically generating, selecting, and combining tools based on task requirements.This includes methods for automated tool generation and optimization, emphasizing systems that can adaptively choose and adjust tool combinations according to different task needs.

% \paragraph{Multi-Agent Systems and Collaboration} Research in Multi-Agent Systems has explored how multiple intelligent agents can collaboratively solve complex tasks in dynamic environments. One significant contribution is the development of decentralized algorithms that allow agents to autonomously form beneficial collaborations and adapt to changing tasks without the need for a central server (DeLAMA) ~\citep{tang2024decentralizedlifelongadaptivemultiagentcollaborative}. Another key area of study focuses on collaboration among heterogeneous agents, where different agents with varied capabilities work together on complex tasks, such as cleaning large spaces, using hierarchical decision models to allocate sub-tasks effectively~\citep{liu2023heterogeneousembodiedmultiagentcollaboration}. Additionally, collaborative learning approaches like Collaborative Q-learning (CollaQ) enhance agent teamwork by decomposing the Q-function and introducing reward attribution techniques to improve performance in multi-agent environments, such as the StarCraft challenge ~\citep{zhang2020multiagentcollaborationrewardattribution}. Finally, research has also examined how multi-agent collaboration can enhance the performance of large language models (LLMs) in tasks like simulations and software development, highlighting the potential of intelligent agent collaboration to improve task outcomes~\citep{talebirad2023multiagentcollaborationharnessingpower}.

\paragraph{Code Generation and Task Solving with LLMs} Large Language Models (LLMs) have demonstrated remarkable potential in generating code to solve complex tasks. Prior studies highlight their effectiveness in mathematical computation ~\citep{zhou2023solving, wang2023mathcoder, gou2023tora}, tabular reasoning ~\citep{chen2022program, lyu2023faithful, lu2024chameleon}, and visual understanding ~\citep{suris2023vipergpt, choudhury2023zero, gupta2023visual}. Frameworks such as AutoGen ~\citep{wu2023autogen} and CodeActAgent~\citep{wang2024executable} extend this capability to agent-based tasks by interpreting executable code as actions. These models dynamically invoke basic tools based on environmental feedback, significantly expanding their utility. Despite their successes, these approaches often treat program generation processes independently, failing to model shared task features and limiting the reusability of functional modules across tasks.

\paragraph{Reusable Tool Creation and Abstraction} To address the limitations of single-use program generation, recent efforts have focused on creating reusable tools. CREATOR ~\citep{qian2023creator} separates the processes of planning (tool creation) and execution, while LATM ~\citep{cai2023large} and CRAFT ~\citep{yuan2023craft} pre-build tools using training and validation sets for task solving. However, these methods often generate a large number of tools, presenting challenges for their efficient reuse. Furthermore, while abstraction-based approaches like REGAL ~\citep{stengel2024regal} focus on extracting reusable tools from primitive programs, they primarily construct simple tools with limited functional complexity. Similarly, Trove ~\citep{wang2024trove} adopts a training-free approach by dynamically composing high-level tools during testing, but its reliance on self-consistency can lead to hallucinated knowledge, reducing accuracy in complex tasks.

\paragraph{Tool Selection for Complex Task Solving} Currently, research on tool selection and retrieval methods primarily focuses on selecting appropriate tools through retrieval mechanisms and LLM-based approaches. ToolRerank ~\citep{zheng2024toolrerank} uses adaptive truncation and hierarchy-aware reranking to improve retrieval results, while Re-Invoke ~\citep{chen2024reinvoketool} introduces an unsupervised framework with synthetic queries and multi-view ranking, enhancing both single-tool and multi-tool retrieval. COLT ~\citep{Qu_2024COLT} combines semantic matching with graph-based collaborative learning to capture relationships among tools, outperforming larger models in some cases. AvaTaR~\citep{wu2024avataroptimizingllmagents} automates the optimization of LLM prompts for better tool utilization, and DRAFT~\citep{qu2024DAFT} refines tool documentation through iterative feedback and exploration, helping LLMs better understand external tools. Despite progress, existing methods generally overlook cost-effectiveness and scalability in tool selection, and often struggle to efficiently adapt to new tools and task requirements in dynamic environments, leading to performance and efficiency bottlenecks. In contrast, our approach dynamically prioritizes tools by combining their relevance and structural importance, ensuring computational efficiency and scalability, thus enabling more effective solutions for complex tasks.
\section{Experimental Details}
\label{app:apexp}
\subsection{Open-ended Task}
\label{subsec:open}
\paragraph{Benchmark} We employed the benchmark proposed by Voyager~\citep{wang2023voyager}, using Minecraft as the experimental platform. Minecraft provides a sandbox environment where players gather resources and craft tools to achieve various goals. The simulation is built on MineDojo~\citep{fan2022minedojo} and uses Mineflayer~\citep{PrismarineJS2013} JavaScript APIs for motor control. 

\paragraph{Baselines}
We conducted a comprehensive comparison with four baselines. Except for Voyager, these methods were originally designed for NLP tasks without embodiment. Therefore, we had to reinterpret and adapt them for execution within the MineDojo environment, ensuring compatibility with the specific requirements of our experimental setup.
\begin{itemize}
    \item \textbf{ReAct:} ReAct~\citep{yao2022react} uses chain-of-thought prompting [46] by generating both reasoning traces and action
plans with LLMs. We provide it with our environment feedback and the agent states as observations.
    \item \textbf{Reflexion:} Reflexion~\citep{shinn2023reflexion} is built on top of ReAct~\citep{yao2022react}with self-reflection to infer more intuitive future actions.
    \item \textbf{AutoGPT:} AutoGPT~\citep{richardssignificant} is a popular software tool that automates NLP tasks by decomposing a high-level
goal into multiple subgoals and executing them in a ReAct-style loop. We re-implement AutoGPT by using GPT-4O to do task decomposition and provide it with the agent states, environment feedback,
and execution errors as observations for subgoal execution
We provide it with execution errors and our self-verification module.
    \item \textbf{Voyager:} Voyager~\citep{wang2023voyager} is a system that integrates an automated curriculum, a scalable skill library, and an iterative prompting framework based on environmental feedback to explore, store, and accumulate skill library within the Minecraft environment.
\end{itemize}


\paragraph{Metric}
The evaluation metric is based on the number of iterations required to progress through tool upgrades, from wooden to stone, iron, and finally diamond tools. Each execution of code is considered one iteration.

\paragraph{Model}
We leverage GPT-4o for text completion, along with the text-embedding-ada-002 API for text embedding. We set all temperatures to
0 except for the automatic curriculum, which uses temperature = 0.1 to encourage task diversity. 

\paragraph{Setting}
We set the maximum number of iterations to 160. For both \ours\ and Voyager, all agents are controlled by GPT-4o, with the number of tools retrieved per iteration set to 5. To ensure a fairer comparison, we removed the Tool Requirement Stage and bug-free checks in \ours\ , and allowed a maximum of 3 self-checks per iteration.

\paragraph{Item Types and Levels}
In the Minecraft task, there are different types and levels of items. Diamond tools are the highest level, and rare items such as golden apples also exist. High-level tools require some lower-level items to craft. Table \ref{tab:toollist} lists the key items in the Minecraft task.
\begingroup
\begin{table}[H]
\caption{List of item types and levels in the Minecraft task.}
\label{tab:toollist}
\vskip -0.1in
\setlength{\tabcolsep}{10pt} % 调整列间距
\begin{center}
\begin{small}
\begin{sc}
\begin{tabular}{l|c|c}
\toprule
\textnormal{\textbf{Category}} & \textnormal{\textbf{level}} & \textnormal{\textbf{Items}} \\
\midrule         
\midrule
\multirow{4}{*}{\multirow{3}{*}{\normalfont Tools}} 
              & \normalfont Wooden Tools & \normalfont Wooden\_Shovel,Wooden\_Pickaxe,Wooden\_Axe,Wooden\_Hoe,Wooden\_Sword \\
              \cmidrule{2-3}
              & \normalfont Stone Tools &\normalfont stone\_pickaxe, stone\_shovel,Stone\_Axe,Stone\_Hoe,Stone\_Sword   \\
              \cmidrule{2-3}
              & \normalfont Iron Tools &\normalfont iron\_pickaxe, iron\_axe, iron\_sword, iron\_shovel, iron\_hoe    \\
              \cmidrule{2-3}
              & \normalfont Diamond Tools &\normalfont diamond\_pickaxe, diamond\_sword, diamond\_axe, diamond\_shovel    \\
             
\midrule
\multirow{2}{*}{\multirow{1}{*}{\normalfont  Armor}} 
              & \normalfont Iron Armor &\normalfont iron\_chestplate, iron\_helmet, iron\_leggings  \\
              \cmidrule{2-3}
              & \normalfont Diamond Armor &\normalfont diamond\_chestplate, diamond\_helmet, diamond\_leggings, diamond\_boots     \\

\midrule
\multirow{3}{*}{\multirow{2}{*}{\normalfont  Food}} 
              & \normalfont Raw Food &\normalfont chicken, mutton, porkchop, rabbit, raw\_rabbit, spider\_eye, bone  \\
              \cmidrule{2-3}
              & \normalfont Cooked Food &\normalfont cooked\_beef, cooked\_chicken, cooked\_mutton, cooked\_porkchop, cooked rabbit  \\
              \cmidrule{2-3}
              & \normalfont Advanced Food &\normalfont golden apple    \\

\bottomrule
\end{tabular}
\end{sc}
\end{small}
\end{center}
\vskip -0.1in
\end{table}
\endgroup


\subsection{Agent Task}
\label{subsec:agent}
\paragraph{Benchmark}
We conducted experiments on two types of agent tasks, demonstrating {\ours}'s capabilities in both game-related and data science tasks.
\begin{itemize}
     \item \textbf{TextCraft:} We evaluate {\ours} on the TextCraft dataset~\citep{futuyma1988evolution}, which challenges agents to craft Minecraft items in a text-only environment~\citep{cote2019textworld}. Each task instance provides a goal and a sequence of crafting commands, which include distractors. We use depth-2 splits for testing and reserve a subset of depth-1 recipes for development, resulting in a 99/77 train/test split.
    \item \textbf{InfiAgent-DABench:} We also test {\ours} on the InfiAgent-DABench benchmark~\citep{hu2024infiagent}, which evaluates LLM-based agents on end-to-end data analysis tasks. This benchmark consists of 257 questions across 52 CSV files, with each question corresponding to a unique CSV file. Agents are required to generate code to analyze data and produce the specified output format. We randomly selected 20 CSV files and their associated question-answer pairs as training data, resulting in a train/test split of 98/159 instances.
\end{itemize}

\paragraph{Baselines}
We compare \ours\ with three methods described below.
\begin{itemize}
     \item \textbf{ReAct:} In this setting, we employ the executor to interact iteratively with the environment, adopting the think-act-observe prompting style from ReAct~\citep{yao2022react}.
     \item \textbf{Plan-Execution:} In contrast, the Plan-and-Execute approach~\citep{shridhar2023art, yang2023intercode} generates a plan upfront and assigns each sub-task to the executor. To ensure each step is executable without further decomposition, we provide new prompts with more detailed planning instructions.
    \item \textbf{Reflexion:} In the Reflection setting~\citep{shinn2023reflexion}, the agent engages in self-reflection after each step, drawing on environmental feedback and exploration history. 
\end{itemize}

\paragraph{Metric} 
The most practically important aspect of the solutions is correctness. For Textcraft, we verify whether the agent’s inventory contains the goal item. For DABench, we check if the agent’s final answer matches the ground truth.

\paragraph{Model}
During training, we use GPT-4o to construct the tool library with a temperature setting of 0. In the testing phase, we conduct a comprehensive comparison of various open-source and closed-source models. The open-source models include \textit{Qwen2.5-7B-Instruct, Qwen-Coder-7B-Instruct, Qwen2.5-14B-Instruct, Deepseeker-Coder-6.7B-Instruct, and Deepseeker-Coder-33B-Instruct}, while the closed-source models primarily include \textit{gpt-3.5-turbo-1106} and \textit{Claude-3-haiku}. During testing, the temperature is set to 0.3, and each experiment is repeated 3 times, with the average result reported.

\paragraph{Setting} 
For ReAct, Reflexion, and \ours\ , the maximum number of steps is set to 20. For Plan-Execution, the maximum number of steps for each sub-task is set to 8. In \ours\ , the number of tools retrieved during testing is limited to 3.



\subsection{Single-turn Code Task}
\label{subsec:code}
\paragraph{Benchmark}
To further explore {\ours}'s potential, we evaluated it on single-turn code generation tasks spanning mathematical reasoning, date comprehension, and tabular reasoning:
 \begin{itemize}
     \item \textbf{MATH:} We used a subset of the MATH dataset~\citep{hendrycks2021measuring}, focusing on 405 level-4 and level-5 algebra problems (MATH contains 5 levels of difficulty) that require textual understanding and advanced reasoning. We randomly selected 200 examples from the test set of the MATH dataset to construct the tool network, resulting in a train/test split of 200/405.
     \item \textbf{Date:} We use the date understanding task from BigBenchHard~\citep{srivastava2022beyond}, which consists of short word problems requiring date understanding. We follow the data splits provided by REGAL\citep{stengel2024regal}, resulting in a train/test split of 66/180.
     \item \textbf{TabMWP:} We further extend our general experiments on MATH by testing on TabMWP~\citep{grand2023learning}, a tabular reasoning dataset consisting of math word problems about tabular data. Based on the CRAFT~\citep{yuan2023craft} splits, we selected 470 problems from levels 7 and 8 (TabMWP contains 8 levels) from the 1,000 test examples. Additionally, we randomly selected 200 examples from the TabMWP training set, resulting in a train/test split of 200/470.
\end{itemize}

\paragraph{Baselines}
For these tasks, we use Programs of Thoughts (PoT)~\citep{chen2022program} and other existing tool-making methods as baselines for comparison.

\begin{itemize}
    \item \textbf{PoT:} The LLM utilizes a program to reason through the problem step by step~\citep{chen2022program}.
   \item \textbf{LATM:} LATM~\citep{cai2023large} samples 3 examples from the training set and create a tool for the task, which is further verified by 3 samples from the validation set. The created tool is then applied to all test cases.
    \item \textbf{CREATOR:} CREATOR~\citep{qian2023creator} disentangle planning (tool making) from execution, enabling Large Language Models (LLMs) to autonomously create a specific tool for each test case during inference.
     \item \textbf{CRAFT:} CRAFT~\citep{yuan2023craft} constructs task-specific toolsets by generating a tool for each training example. During testing, it utilizes a tool retrieval module and a reasoning process akin to CREATOR, generating a function first and then producing the corresponding invocation code. 
      % \item \textbf{Trove:} Trove~\citep{wang2024trove} introduces a training-free method based on self-consistency, where LMs interact with the toolbox through three modes—IMPORT, SKIP, and CREATE. Each mode is executed K times, and from the 3K responses, the function from the most consistent and optimal response is added to the toolbox.
      \item \textbf{REGAL:} During training, REGAL~\citep{stengel2024regal} refines primitive programs by extracting functions. In the testing phase, it retrieves both tools and refactored programs—comprising original and refactored versions—to generate a program that effectively solves the task. 
\end{itemize}
\paragraph{Metric}
We use correctness as the evaluation metric, measuring whether the execution outcome of the solution program exactly matches the ground-truth answer(s).
\paragraph{Model}
The models for the single-turn code generation task are the same as those used for the Agent Task, as presented in Section \ref{subsec:agent}.
\paragraph{Setting}
To ensure a fair comparison, we make slight adjustments to each method. For all methods, we allow up to 3 times for format checking and correction, as small models may not always follow the required output format. For PoT, we use 6 fixed examples of basic tool usage as few-shot. CREATOR employs the rectifying process, while for CRAFT, we use the same training set as our method and construct the tool library with GPT-4o, retrieving 3 tools during testing. For Regal, we use PoT along with GPT-4o to obtain ground-truth code, select the correct program, and have GPT-4o reconstruct it. To maintain fairness in tool generation quality, we standardize the few-shot examples of basic tools and retrieve 3 tools, along with 3 usage examples from the current tool library, avoiding errors from pruned tools. For our method, we train with GPT-4o, retrieving 3 tools and their corresponding usage examples during testing, while fixing the basic tool few-shot examples to 3, ensuring consistency with PoT’s total few-shot count.
\section{More Results}
\label{app:apresults}
\subsection{Open-ended Task}
\label{subsec:open-results}
\paragraph{More complex tools} 
Our hierarchical graph architecture offers significant advantages in handling complex tasks and large-scale systems. As shown in Figure \ref{fig:toolnet1}, Trial 1 starts with five nodes occupying three layers, and evolves into a five-layer network, with an increasing number of inter-tool calls. As shown in Figure \ref{fig:toolnet2}, Trial 2 starts with four nodes occupying four layers, and evolves into a five-layer network with more inter-tool calls. As shown in Figure \ref{fig:toolnet3}, Trial 3 starts with four nodes occupying three layers, and evolves into a six-layer network structure, with a growing number of inter-tool calls. Our tool graph becomes progressively more complex, flexibly expanding and optimizing its components. These results demonstrate that our method can generate tools that call each other, and combine them into more complex tools. This not only enhances scalability but also facilitates the creation of more sophisticated tools, enabling the solution of increasingly complex problems.


\paragraph{More types of inventory} Our method is able to generate more inventory types than Voyager. As shown in Table \ref{tab:Number}, we can see that {\ours} produces more inventory types in all three trials compared to Voyager.

The inventory collected by {\ours} in each trial is

\begin{itemize}
    \item \textbf{Trial 1:}  \textit{oak\_log, birch\_log, oak\_planks, birch\_planks, crafting\_table, stick, wooden\_pickaxe, dirt, cobblestone, coal, stone\_pickaxe, raw\_copper, furnace, copper\_ingot, andesite, raw\_iron, granite, iron\_ingot, iron\_pickaxe, shield, diorite, raw\_gold, lapis\_lazuli, redstone, diamond, diamond\_pickaxe, bucket, gold\_ingot, iron\_chestplate, arrow, iron\_sword, iron\_helmet, diamond\_sword, diamond\_helmet, lightning\_rod, chest, iron\_axe, iron\_leggings, sandstone, dandelion, spider\_eye, string, iron\_shovel, copper\_block, iron\_door, iron\_hoe, kelp, bow, dried\_kelp, torch, cooked\_beef, gray\_wool, cobbled\_deepslate, tuff, diamond\_leggings, bone, diamond\_chestplate, chicken, white\_banner, cooked\_chicken, egg, feather, oak\_sapling, apple, acacia\_log, golden\_apple, diamond\_axe}

    \item \textbf{Trial 2:}  \textit{oak\_sapling, oak\_log, stick, oak\_planks, crafting\_table, wooden\_pickaxe, dirt, cobblestone, stone\_pickaxe, diorite, raw\_iron, coal, lapis\_lazuli, gravel, furnace, iron\_ingot, raw\_copper, sandstone, granite, iron\_pickaxe, andesite, raw\_gold, gold\_ingot, diamond, diamond\_pickaxe, redstone, cobbled\_deepslate, bucket, iron\_sword, arrow, bow, bone, birch\_log, chest, amethyst\_block, calcite, smooth\_basalt, iron\_chestplate, diamond\_sword, diamond\_helmet, iron\_leggings, diamond\_boots, water\_bucket, string, orange\_tulip, mutton, white\_wool, porkchop, dandelion, cooked\_porkchop, cooked\_mutton}

    \item \textbf{Trial 3:}  \textit{jungle\_log, stick, oak\_sapling, jungle\_planks, crafting\_table, dirt, wooden\_pickaxe, cobblestone, stone\_pickaxe, raw\_iron, raw\_copper, furnace, iron\_ingot, iron\_pickaxe, coal, diorite, lapis\_lazuli, andesite, moss\_block, clay\_ball, redstone, raw\_gold, cobbled\_deepslate, granite, diamond, diamond\_pickaxe, copper\_ingot, gunpowder, bucket, gravel, gold\_ingot, oak\_log, iron\_sword, iron\_chestplate, chest, diamond\_sword, spruce\_sapling, rotten\_flesh, bone, rose\_bush, water\_bucket, string, oak\_planks, grass\_block, diamond\_helmet, iron\_leggings, emerald, snowball, rabbit\_hide, rabbit, spruce\_log, cooked\_rabbit, diamond\_boots}
\end{itemize}


The inventory collected by Voyager in each trial is
\begin{itemize}
    \item \textbf{Trial 1:}  \textit{oak\_log, birch\_log, oak\_sapling, birch\_sapling, oak\_planks, stick, crafting\_table, wooden\_pickaxe, dirt, cobblestone, stone\_pickaxe, raw\_copper, white\_tulip, coal, furnace, copper\_ingot, granite, raw\_iron, iron\_ingot, lightning\_rod, iron\_pickaxe, pink\_tulip, orange\_tulip, sandstone, shears, shield, diorite, cobbled\_deepslate, iron\_block, chest, tuff, lapis\_lazuli, redstone, diamond, raw\_gold, gold\_ingot, diamond\_pickaxe, diamond\_helmet, diamond\_sword, sand, andesite, arrow, bone, iron\_chestplate, beef, leather, oak\_leaves, porkchop, cooked\_beef, leather\_leggings}

    \item \textbf{Trial 2:}  \textit{dirt, oak\_log, oak\_planks, crafting\_table, stick, oak\_sapling, wooden\_pickaxe, cobblestone, coal, stone\_pickaxe, raw\_iron, granite, lapis\_lazuli, raw\_copper, furnace, iron\_ingot, copper\_ingot, iron\_helmet, iron\_pickaxe, diorite, andesite, salmon, ink\_sac, iron\_chestplate, lightning\_rod, cooked\_salmon, stone, stonecutter, rotten\_flesh, gravel, flint, chest, iron\_leggings, copper\_block, cobbled\_deepslate, tuff, diamond, diamond\_pickaxe, raw\_gold, gold\_ingot, redstone, diamond\_sword, egg, diamond\_boots, diamond\_axe}

    \item \textbf{Trial 3:}  \textit{jungle\_log, jungle\_planks, oak\_sapling, oak\_log, crafting\_table, stick, wooden\_pickaxe, dirt, cobblestone, coal, stone\_pickaxe, raw\_copper, furnace, copper\_ingot, magma\_block, lightning\_rod, stone\_axe, jungle\_boat, kelp, sand, sandstone, glass, raw\_iron, granite, lapis\_lazuli, diorite, iron\_ingot, bucket, iron\_pickaxe, chest, andesite, redstone, dried\_kelp, iron\_chestplate, wooden\_sword, shield, iron\_sword}
\end{itemize}

\vskip -0.2in
\begin{table}[H]
\caption{Number of different inventory types produced by each trial}
\label{tab:Number}
% \vskip 0.1in
\setlength{\tabcolsep}{12pt} % 调整列间距
\renewcommand{\arraystretch}{1.0} % 调整行间距
\begin{center}
% \resizebox{\textwidth}{!}{ % 自动调整表格宽度以适应页面
\begin{small}
\begin{sc}
\begin{tabular}{lccc} % 确保列数与标题一致
\toprule
\textnormal{\textbf{Method}} & \textnormal{\textbf{Trial 1}} & \textnormal{\textbf{Trial 2}} & \textnormal{\textbf{Trial 3}}  \\
\midrule
\normalfont Voyager     & 50  & 45  & 37    \\
\normalfont AETG(Ours)  & 67  & 51  & 53    \\
\bottomrule
\end{tabular}
\end{sc}
\end{small}
% }
\end{center}
\vskip -0.1in
\end{table}


\paragraph{Longer exploration path} To better demonstrate the exploration capabilities of the agent, we compared the exploration trajectories and their lengths. As shown in Figure \ref{fig:linermap}, our agent exhibits longer and more persistent exploration capabilities than Voyager. In Table \ref{tab:length}, the trajectory lengths of our agent are consistently much greater than those of Voyager. {\ours}is able to traverse across multiple terrains, with an average distance 2.66 times longer than Voyager. Additionally, {\ours} can explore across different continental plates, while Voyager remains confined to a single plate, highlighting the exceptional exploration capability of {\ours}.

% \vskip -0.2in
\begin{table}[H]
\caption{Exploration trajectory length in each trial, where \textit{Performance Gain} = $\textit{ours}/\textit{voyager}$.}
\label{tab:length}
% \vskip 0.1in
\setlength{\tabcolsep}{12pt} % 调整列间距
% \renewcommand{\arraystretch}{1.0} % 调整行间距
\begin{center}
% \resizebox{\textwidth}{!}{ % 自动调整表格宽度以适应页面
\begin{small}
\begin{sc}
\begin{tabular}{lcccc} % 确保列数与标题一致
\toprule
\textnormal{\textbf{Method}} & \textnormal{\textbf{Trial 1}} & \textnormal{\textbf{Trial 2}} & \textnormal{\textbf{Trial 3}} & \textnormal{\textbf{\textit{Avg}}}\\
\midrule
\normalfont Voyager     & 1925.74  & 4102.99  & 902.13  & 2310.29   \\
\normalfont {\ours}(Ours)  & 5665.75  & 8908.57  & 3895.06 & 6156.46  \\
\midrule
\normalfont \textit{Performance Gain} & 2.94  & 2.17   & 4.32    & 2.66 \\
\bottomrule
\end{tabular}
\end{sc}
\end{small}
% }
\end{center}
\vskip -0.1in
\end{table}


\vskip -0.2in
\begin{figure}[H]
\vskip 0.2in
\begin{center}
\centerline{\includegraphics[width=1\linewidth]{trial-map.png}}
% \vskip -0.2in
\caption{Map coverage: Three bird’s eye views of Minecraft maps. The trajectories are plotted based on the position coordinates where each agent interacts.}
\label{fig:trialmap}
\end{center}
\vskip -0.3in
\end{figure}


\vskip -0.2in
\begin{figure}[H]
\vskip 0.2in
\begin{center}
\centerline{\includegraphics[width=1\linewidth]{liner-map.png}}
% \vskip -0.2in
\caption{Movement trajectory Map: Three bird’s eye views of Minecraft maps. The trajectories are plotted based on the position coordinates where each agent interacts.}
\label{fig:linermap}
\end{center}
\vskip -0.3in
\end{figure}



\paragraph{Efficient Zero-Shot Generalization to Unseen Tasks} Based on the results presented in Table \ref{tab:newtechtree} and Figure \ref{fig:diamon and compass}, we can clearly observe the significant advantages of {\ours} in the open-ended task. Table \ref{tab:newtechtree} shows the number of iterations required for different methods to complete various tasks (Gold Sword, Compass, Diamond Hoe, Lava Bucket), where fewer iterations indicate higher efficiency. Compared to Voyager and {\ours} (w/o toolnet), {\ours} consistently requires significantly fewer iterations across all tasks, demonstrating substantial improvements in efficiency. Notably, in the Gold Sword task, {\ours} (ours) completes the task in just 14.00±1.73 iterations, whereas Voyager requires 46.33±14.57 iterations, showcasing its superior performance.

Figure \ref{fig:diamon and compass} further visualizes the intermediate progress of different methods on the "Craft a Compass" and "Craft a Diamond Hoe" tasks. It is evident that {\ours} learns and masters the necessary skills for crafting items more quickly. As the number of prompting iterations increases, {\ours} reaches the task objectives significantly earlier than the other methods. Additionally, while {\ours}(w/o Tool Graph) performs better than Voyager, it still lags behind {\ours}, indicating that the ToolNet component plays a crucial role in enhancing the model's capability.

Overall, these experimental results demonstrate that {\ours} not only learns new skills and crafting techniques more efficiently but also that its key module, Tool Graph, is essential for overall performance improvement. This further validates the effectiveness of our approach in self-driven exploration and task generalization.


\begingroup
\begin{table}[H]
\caption{The mastery of the tech tree in the Open-ended Task. The number indicates the number of iterations. The fewer the iterations, the more efficient the method. "N/A" indicates that the number of iterations for obtaining the current type of tool is not available.}
\label{tab:newtechtree}
\vskip 0.1in
\setlength{\tabcolsep}{12pt} % 调整列间距
% \renewcommand{\arraystretch}{1.0} % 调整行间距
\begin{center}
% \resizebox{\textwidth}{!}{ % 自动调整表格宽度以适应页面
\begin{small}
\begin{sc}
\begin{tabular}{lccccc} % 确保列数与标题一致
\toprule
\textnormal{\textbf{Method}} & \textnormal{\textbf{Trial}} & \textnormal{\textbf{Gold Sword}} & \textnormal{\textbf{Compass}} & \textnormal{\textbf{Diamond Pickaxe}} & \textnormal{\textbf{Lava Bucket}} \\
\midrule
\multirow{4}{*}{\multirow{2}{*}{\normalfont Voyager}} 
              & \normalfont Trial 1 & 48 & 16 &  24 & N/A         \\
              & \normalfont Trial 2 & 31 & 17 &  25 & 39         \\
              & \normalfont Trial 3 & 60 & 20 & 18  & N/A         \\
              \cmidrule{2-6}
              & \textit{Average} & 46.33$\pm$14.57 & 17.67$\pm$2.08 & 22.33$\pm$3.79 & 39.00$\pm$0.00 \\
\midrule
\multirow{4}{*}{\multirow{2}{*}{\normalfont {\ours}\textit{\small(w/o toolnet)}}} 
               & \normalfont Trial 1 & 26 & 27 & 23  & N/A         \\
              & \normalfont Trial 2 & 18 & 22 & 18  & N/A        \\
              & \normalfont Trial 3 & 56 & 15 & 30  & N/A          \\
              \cmidrule{2-6}
              & \textit{Average} & 33.33$\pm$20.03 & 21.33$\pm$6.03 & 23.67$\pm$6.03 & N/A$\pm$N/A \\
\midrule
\multirow{4}{*}{\multirow{2}{*}{\normalfont {\ours}\textit{\small(ours)}}} 
              & \normalfont Trial 1 & 13 & 28 & 16  & 19       \\
              & \normalfont Trial 2 & 13 & 10 & 14  & 27       \\
              & \normalfont Trial 3 & 16 & 13  & 13  & 18      \\
              \cmidrule{2-6}
              & \textit{Average} & \textbf{14.00$\pm$1.73} & \textbf{17.00$\pm$9.64} & \textbf{14.33$\pm$1.53} & \textbf{21.33$\pm$4.93} \\
             

\bottomrule
\end{tabular}
\end{sc}
\end{small}
% }
\end{center}
\vskip -0.1in
\end{table}
\endgroup



\begin{figure}[H]
\vskip 0.2in
\begin{center}
\centerline{\includegraphics[width=1\linewidth]{compass_and_diamond.png}}
% \vskip -0.2in
\caption{Zero-shot generalization to unseen tasks. Here, we visualize the intermediate progress of each method on the tasks "Craft a Compass" and "Craft a Diamond Hoe."}
\label{fig:diamon and compass}
\end{center}
\vskip -0.3in
\end{figure}



\subsection{Agent Task}
\label{subsec:agent-results}

Figures \ref{fig:toolnet-dabench} and \ref{fig:toolnet-textcraft} present the tool network evolution diagrams of DA-Bench and TextCraft, which visually reflect the call relationships between different tool functions. In these diagrams, each node represents a specific tool function, edges indicate the call dependencies between tools, and the shading of the nodes reflects the frequency of tool calls—darker colors indicate higher call frequency. From Figure \ref{fig:toolnet-dabench}, it can be observed that in DA-Bench, the tool network expands progressively as the task advances, forming multiple core nodes with higher call frequencies. This suggests that certain key tools are frequently called during the task execution, playing a central role. Additionally, the tool call relationships exhibit a hierarchical and well-organized structure, reflecting DA-Bench's efficiency in tool dependency management.

In contrast, Figure \ref{fig:toolnet-textcraft} illustrates the tool network evolution of TextCraft, which also shows a similar expansion trend overall. However, compared to DA-Bench, the tool call frequency in TextCraft is more evenly distributed across multiple nodes, meaning that the system calls a wider variety of tools during task execution, rather than relying on a few core tools. This distribution pattern may suggest that TextCraft adopts a more diverse tool usage strategy in task execution.

A comparative analysis of the two figures reveals that, although both DA-Bench and TextCraft exhibit certain hierarchical and expansive characteristics in their tool call patterns, DA-Bench relies more heavily on a few core tools, whereas TextCraft displays a more dispersed tool call pattern. This contrast not only highlights the differences in tool usage between the two, but also emphasizes the importance and effectiveness of ToolNet.





\subsection{Single-turn Code Task}
\label{subsec:code-results}

As shown in the Figure\ref{fig:toolnet-math} \ref{fig:toolnet-tabmwp}, this illustrates the evolution of the tool graph for the Math and TabMWP tasks. It is evident that the tool graph gradually becomes more complex, creating multiple layers of tools, making the tool graph more intricate. Since the Date task can be solved with fewer tools, there is no evolution of the tool graph. However, the generated tools can still effectively solve the task, while there exists a multi-level calling relationship.


\section{More Ablations}
\label{app:apablation}
\subsection{Open-ended Task}
\label{subsec:open-ablation}

As shown in Figure \ref{fig:ablation}, AETG significantly outperforms methods that lack certain functional modules in discovering new Minecraft items and skills. It can be observed that the performance is worst when "w/o retrieval" is used, indicating that the absence of retrieval has the greatest impact on overall functionality and plays a crucial role, thereby validating the effectiveness of our retrieval method. The performance with "w/o duplication" is slightly better, indicating its importance is weaker than that of "w/o retrieval." The performance of "w/o check" and "w/o pruning" is better, but still far behind AETG, which further demonstrates the importance and effectiveness of each functional component.

\vskip -0.1in
\begin{figure}[H]
% \vskip 0.2in
\begin{center}
\centerline{\includegraphics[width=0.6\linewidth]{toolnumber-ablation.png}}
% \vskip -0.2in
\caption{Ablation study of the iterative prompting mechanism. AETN surpasses all other options, highlighting the essential significance of each functional module in the iterative prompting mechanism.}
\label{fig:ablation}
\end{center}
\vskip -0.3in
\end{figure}


\subsection{Closed-Ended Task}
\label{subsec:closed-ended}
For the Closed-Ended Task, we select Textcraft from the Agent Task and Date from the Single-turn Code Task to evaluate the effectiveness of several components in our method. The results are shown in the Table \ref{tab:closed-toolnumber}.

\begingroup
\begin{table}[H]
\caption{The number of tools in Close-Ended Task.}
\label{tab:closed-toolnumber}
\vskip -0.1in
\setlength{\tabcolsep}{10pt} % 调整列间距
\begin{center}
\begin{small}
\begin{sc}
\begin{tabular}{l|cc}
\toprule
\textnormal{\textbf{Method}} & \textnormal{\textbf{TextCraft}}  & \textnormal{\textbf{Date}} \\
\midrule         

\normalfont W/o Self-Check & 42 & 9 \\
\midrule  
\normalfont W/o Merging & 49 & 11\\
\midrule  
\normalfont W/o pruning & 46 & 9 \\
\midrule  
\normalfont GATE & 44 & 4 \\


\bottomrule
\end{tabular}
\end{sc}
\end{small}
\end{center}
\vskip -0.1in
\end{table}
\endgroup

\section{Tool Making}
\label{app:toolgarph}
\subsection{Basic Tools}
\label{subsec:basic-tools}
As shown in the Table \ref{tab:basictool} , the basic tools generated by each method are displayed.

\begingroup
\begin{table}[H]
\caption{Basic tools in various methods.}
\label{tab:basictool}
\vskip -0.1in
\setlength{\tabcolsep}{10pt} % 调整列间距
\begin{center}
\begin{small}
\begin{sc}
\begin{tabular}{l|p{12cm}}
\toprule
\textnormal{\textbf{Tasks}} & \textnormal{\textbf{Basic Tools}}  \\
\midrule         

\normalfont Other Tasks & \normalfont ToolRequest, NotebookBlock, Terminate, CreateTool, EditTool, Python, Feedback, SendAPI, Feedback, Retrieval \\
\midrule  
\normalfont Minecraft & \normalfont smeltItem, killMob, waitForMobRemoved, givePlacedItemBack, useChest, exploreUntil, craftItem, mineBlock, shoot, placeItem, craftHelper, smeltItem, mineflayer, killMob, useChest, exploreUntil, craftItem, mineBlock, placeItem \\

\bottomrule
\end{tabular}
\end{sc}
\end{small}
\end{center}
\vskip -0.1in
\end{table}
\endgroup


\subsection{Tool construction Lists}
\label{subsec:tool construction}

\paragraph{CREATOR:}
\begin{itemize}[noitemsep, topsep=0pt]
    \item \textbf{MATH:}  \textit{sum of areas, find largest won matches, find K, total distance after bounces, find common ratio sum, count lattice points with distance squared, find c for radius, find circle equation and constants, polynomial degree product, calculate cells, find fiftieth term, find non domain values, inverse function product, find m and n, sum of fractions from roots, find roots of quadratic, main, find coefficients, compute expression, prime factors, find x y, find second largest angle, find y coordinate, find constants, evaluate expression, find b for one solution, find c, find minimum value, find possible s, solve expression, find cone height, solve abc, find minimum expression, \dots, time to hit ground, sum of reciprocals of roots, solve x floor x product, sum of possible x, find constant a, sum of squares of solutions, find cost per extra hour, is triangular number, find smallest b greater than 2011, solve exponential equation, solve club suit equation, find degree of h, f, find vertical asymptotes, domain width, maximize revenue, future value, total savings, find min interest rate, equation, find integers, sum of x coordinates squared, find integer values of a, smallest c for real domain, smallest integer c, find m, required investment, simplify expression, g, distance between midpoints, compute x and power, greatest possible a, find continued fraction value, find a b, solve mnp, compute sum, sum of integers in range,
    }

    \item \textbf{Date:}  \textit{get us thanksgiving date, get date one week from first monday of 2019, calculate anniversary date, calculate yesterday from last day of january, calculate one week ago from first monday, get first monday of 2019, calculate yesterday, calculate yesterday from rescheduled meeting, calculate date a month ago from rescheduled meeting, calculate yesterday from first monday of 2019, get date 10 days before us thanksgiving, calculate one week ago from egg runout, calculate one week ago from end of first quarter, calculate date 24 hours later, calculate date a month ago, calculate date 24 hours after anniversary, calculate one week from today from rescheduled meeting, \dots, get tomorrow from us thanksgiving, calculate yesterday from day before yesterday, calculate yesterday from anniversary, calculate date 10 days ago, calculate one year ago from egg run out date, calculate tomorrow from yesterday, calculate one week from last day of january, calculate one week from anniversary, calculate yesterday from eggs run out, calculate tomorrow from today, calculate tomorrow from day before yesterday, calculate one week ago from today, calculate one week ago, calculate date one month ago from anniversary, calculate one year ago from given date, calculate one week from given date}

    \item \textbf{TabMWP:}  \textit{calculate total cost, smallest points, price difference, cost of river rafts, calculate median, calculate range, calculate total spent, rate of change, cost difference, cost for rides, rate of change vacation days, total participants, calculate mean glasses, find mode of states visited, rate of change straight A students, calculate median basketball hoops, count bins with toys in range, people with at least 3 trips, count teams with fewer than 80 swimmers, calculate median clubs, count exact pushups, children with less than 2 necklaces, people played exactly 3 times, count people with fewer than 80 pullups, range of states visited, find spent amount, \dots, calculate median miles, people with fewer than 3 seashells, calculate median glasses, cost to buy cockatiels, largest broken lights, calculate spent, calculate ice cream cost, range of soccer fields, patrons with at least 2 books, count bushes with 20 roses, total people played golf, range of articles, count shipments with exactly 60 broken plates, total cost for lip balms, rate of change scholarships, count teams with fewer than 50 members, count tests with 34 problems, find mode of soccer fields, rate of change hockey games, find lowest score, count pizzas with exactly 48 pepperoni, count people with at least 30 points, cost of wooden benches, rate of change students, patients with fewer than 2 trips, find mode, total cost for hazelnuts, calculate mean fan letters, readers with at least 4 hats, count classrooms with 41 desks}
\end{itemize}

\paragraph{CRAFT:}
\begin{itemize}[noitemsep, topsep=0pt]
    \item  \textbf{MATH:}  \textit{find pack size, count distinct solutions, calculate points, find tank capacity, solve exponential log equation, total energy equilateral triangle, inverse square law force, find max value, total logs in stack, sum of multiples of 13, calculate exponential growth, gravitational force, find x for piecewise composition, positive difference, specific piecewise func, day exceeds 200 cents, find lattice points, count integer parameters for integer solutions, count zeros in square of power of ten minus one, energy stored, sum of squares of roots, sum odd integers, find d minus e squared, compute complex series sum, total energy configuration, sum of areas, \dots, max item price, solve two variable system, inverse variation power, total distance hopped, is prime, total distance, find constant term of polynomial, total distance moved, find perpendicular slope, calculate inverse proportionality, find value of A, count integer a, find min items for higher score, apply r n times, find min x, day exceeds threshold, calculate area in square yards, solve log equation, total items produced, find variable for distance condition, solve time at speeds, find largest solution, find weight of object, calculate proportional value, calculate material cost, solve for variable, total elements in arithmetic sequence, transformed domain, find day for algae coverage, calculate energy stored, least value of y, solve bowling ball weight, find min froods}

    \item \textbf{Date:} \textit{get today date, calculate one week ago, calculate n days from future date, calculate n days from date in format, calculate date days ago, calculate n months from date, calculate one week from today, calculate date after event, find palindrome day, calculate date a month ago, calculate date after days and months, calculate relative date, calculate n days from reference, calculate one year ago from today, calculate n hours from date, calculate date n days from, get date today, calculate date 10 days ago from deadline, calculate n weeks from date, \dots, calculate n units from date, calculate n years from date, calculate n weeks from first weekday of year, calculate today from tomorrow, find special day, calculate date 10 days ago from future, calculate n days after event, calculate date from days passed, calculate one week from christmas eve, calculate one year ago, calculate date 24 hours later, calculate n weeks from anniversary, calculate tomorrow from uk format date, calculate n days from date, is palindrome, calculate one week from first monday of year, calculate one week ago from anniversary}

    \item \textbf{TabMWP:} \textit{get frequency, calculate volleyballs in lockers, calculate total cost from package prices, calculate total items from group counts, calculate mode, calculate donation difference for person, count bags with 20 to 40 broken cookies, calculate total items from groups and items per group, count commutes of 50 minutes, get received amount, calculate total items for groups, find probability, calculate vacation cost, calculate rate of change, find received amount for transaction, calculate vote difference between two items for group, count customers, find minimum value in stem leaf, calculate metric wrenches, find smallest number, count books with 30 to 50 characters, \dots, count people with 67 pullups, calculate difference in donations for person, calculate total cost from unit price and weight, calculate total items from ratio, calculate total cost from unit weight prices and weight, calculate donation difference between causes, calculate difference, calculate net income, calculate grasshoppers on twigs, count total members in group, calculate expenses on date, find lightest child, calculate difference in amounts, count votes for item from groups, calculate probability from count table, get table cell value, calculate jeans in hampers, count instances with specific value in stem leaf, calculate donation difference for person and causes, calculate total from frequency and additional count, calculate range, calculate total reviews}
\end{itemize}


\paragraph{REGAL:}
\begin{itemize}[noitemsep, topsep=0pt]
    \item \textbf{MATH:}  \textit{solve for largest side, apply function sequence, solve rational equation, calculate expression sum, max sum of products, find b for perpendicular bisector, vertex of quadratic, calculate work days, calculate c for zero coefficient, simplify and rationalize sympy, find a for binomial square, compound interest, calculate inverse variation, expand expression, calculate average speed, calculate rs, sum sequence, solve for p, max consecutive integers, find x intercept, day exceeding threshold, find smallest sum, solve for ac pair, constant function, sum of distances, evaluate expression, sum finite geometric series, factor expression, find common difference, total coins pirates, calculate geometric first term, calculate closest whole number, calculate x minus y squared, solve letter values, find circle center v2, evaluate expression with sqrt, calculate sum of equations, \dots, calculate x3 plus y3, find negative intervals, calculate floor and abs, solve quadratic and find min, calculate y, solve for a, check equations, rationalize and simplify, calculate xyz, calculate distance, solve for x in simplified equation, calculate expression, calculate exponent, sum arithmetic series, complete square form, calculate x2 plus y2
    }

    \item \textbf{Date:}  \textit{subtract weeks from date, add weeks to date, format date, add days to date, subtract months from date, subtract days from date, subtract years from date, calculate date, calculate days between weekdays}

    \item \textbf{TabMWP:}  \textit{count range, find mode, total participants, count bushes with fewer roses, find max frequency, total items, count in range, calculate total items, count below threshold, count teams with minimum size, calculate total, calculate range, calculate fraction, sum frequencies below threshold, sum frequencies, calculate difference, calculate median, total outcomes, count specific height, count numbers in range, difference between groups, access frequency, calculate proportionality constant, count values below threshold, find median, calculate probability, calculate mode, get frequency, convert stem leaf to numbers, find minimum, get total items, count scores above, rate of change, calculate mean}
\end{itemize}



\subsection{The tool graph evolution diagrams of {\ours} for various tasks.}
\label{subsec:tool-graph}
Below are the tool graph evolution diagrams for various tasks. The Date task does not have a tool network evolution diagram, as date reasoning does not heavily rely on tool diversity.


\begin{figure}[H]
\vskip 0.2in
\begin{center}
\centerline{\includegraphics[width=1\linewidth]{toolnet-trial1.png}}
% \vskip -0.2in
\caption{
The tool graph evolution diagram for Minecraft Trial 1. In this diagram, each node represents a tool function, and the edges represent the invocation relationships between tools. The darker the color, the more frequently the tool is invoked. The network consists of a total of 6 layers, with layers 2 to 6 shown here from top to bottom.}
\label{fig:toolnet1}
\end{center}
\vskip -0.3in
\end{figure}

\vskip -0.2in
\begin{figure}[H]
\vskip 0.2in
\begin{center}
\centerline{\includegraphics[width=1\linewidth]{toolnet-trial2.png}}
% \vskip -0.2in
\caption{The tool graph evolution diagram for Minecraft Trial 2. In this diagram, each node represents a tool function, and the edges represent the invocation relationships between tools. The darker the color, the more frequently the tool is invoked. The network consists of a total of 6 layers, with layers 2 to 6 shown here from top to bottom.}
\label{fig:toolnet2}
\end{center}
\vskip -0.3in
\end{figure}

\vskip -0.2in
\begin{figure}[H]
\vskip 0.2in
\begin{center}
\centerline{\includegraphics[width=1\linewidth]{toolnet-trial3.png}}
% \vskip -0.2in
\caption{The tool graph evolution diagram for Minecraft Trial 3. In this diagram, each node represents a tool function, and the edges represent the invocation relationships between tools. The darker the color, the more frequently the tool is invoked. The network consists of a total of 6 layers, with layers 2 to 7 shown here from top to bottom.}
\label{fig:toolnet3}
\end{center}
\vskip -0.3in
\end{figure}


\begin{figure}[H]
\vskip 0.2in
\begin{center}
\centerline{\includegraphics[width=1\linewidth]{toolnet-dabench.png}}
% \vskip -0.2in
\caption{The tool graph evolution diagram of DA-Bench. In this diagram, each node represents a tool function, and the edges represent the invocation relationships between tools. The darker the color, the more frequently the tool is invoked.}
\label{fig:toolnet-dabench}
\end{center}
\vskip -0.3in
\end{figure}

\begin{figure}[H]
\vskip 0.2in
\begin{center}
\centerline{\includegraphics[width=1\linewidth]{toolnet-textcraft.png}}
% \vskip -0.2in
\caption{The tool graph evolution diagram of TextCraft. In this diagram, each node represents a tool function, and the edges represent the invocation relationships between tools. The darker the color, the more frequently the tool is invoked.}
\label{fig:toolnet-textcraft}
\end{center}
\vskip -0.3in
\end{figure}


\begin{figure}[H]
\vskip 0.2in
\begin{center}
\centerline{\includegraphics[width=1\linewidth]{toolnet-math.png}}
% \vskip -0.2in
\caption{The tool graph evolution diagram of MATH. In this diagram, each node represents a tool function, and the edges represent the invocation relationships between tools. The darker the color, the more frequently the tool is invoked.}
\label{fig:toolnet-math}
\end{center}
\vskip -0.3in
\end{figure}

\begin{figure}[H]
\vskip 0.2in
\begin{center}
\centerline{\includegraphics[width=1\linewidth]{toolnet-tabmwp.png}}
% \vskip -0.2in
\caption{The tool graph evolution diagram of TabMWP. In this diagram, each node represents a tool function, and the edges represent the invocation relationships between tools. The darker the color, the more frequently the tool is invoked.}
\label{fig:toolnet-tabmwp}
\end{center}
\vskip -0.3in
\end{figure}
\section{Prompt Template}
\label{app:prompt}
In this section, we provide the prompt templates of different types used throughout our experiment. These prompts were carefully crafted to ensure that the model's output aligns with the specific objectives of each task.

\subsection{Construction Stage}
In open-ended task online training, we made slight modifications to their prompts based on Voyager~\citep{wang2023voyager}. For close-ended tasks, the prompts used during the construction process are as follows:
\begin{tcolorbox}[title=Task Solver's Prompt, breakable, width=\textwidth,top=0mm]
\begin{Verbatim}[breaklines, fontsize=\footnotesize]
# Instruction #
You are the Task Solver in a collaborative team, specializing in reasoning and Python programming. Your role is to analyze tasks, collaborate with the Tool Manager, and solve problems step by step.
Directly solving tasks without tool analysis is not allowed. Request necessary tools before proceeding when needed, based on the task analysis.

# WORKFLOW #
You can decide which step to take based on the environment and current situation, adapting dynamically as the task progresses.
Stage 1. Tool Requests:
    Requesting tool is mandatory. Request generalized and reusable tools to solve the task. Focus on abstract functionality rather than task-specific details to enhance flexibility and adaptability.
Stage 2. Code and Interact: 
    Write notebook blocks incrementally, executing and interacting with the environment step by step. Avoid bundling all steps into a single block; instead, adjust dynamically based on feedback after each interaction.
Stage 3: Validate and Conclude: 
    When confident in the solution, review your work, validate the results, and conclude the task.

# Custom Library #
===api===

# NOTICE #
1. You must fully understand the action space and its parameters before using it.
2. If code execution fails, you should analyze the error and try to resolve it. If you find that the error is caused by the API, please promptly report the error information to the Tool Manager.
3. Regardless of how simple the issue may seem, you should always aim to summarize and refine the tool requirements.


# Tool Request Guidelines #
1. Keep It Simple: Design tools with single and simple functionality to ensure they are easy to implement, understand, and use. Avoid unnecessary complexity.
2. Define Purpose: Clearly outline the tool’s role within broader workflows. Focus on creating reusable tools that solve abstract problems rather than task-specific ones.
3. Specify Input and Output: Define the required input and expected output formats, prioritizing generic structures (e.g., dictionaries or lists) to enhance flexibility and adaptability.
4. Generalize Functionality: Ensure the tool is not tied to a specific task. Abstract its functionality to make it applicable to similar problems in other contexts.


# ACTION SPACE #
You should Only take One action below in one RESPONSE:
## NotebookBlock Action
* Signature: 
NotebookBlock():
```python
executable python script
```
* Description: The NotebookBlock action allows you to create and execute a Jupyter Notebook cell. The action will add a code block to the notebook with the content wrapped inside the paired ``` symbols. If the block already exists, it can be overwritten based on the specified conditions (e.g., execution errors). Once added or replaced, the block will be executed immediately.
* Restrictions: Only one notebook block can be managed or executed per action.
* Example
- Example1: 
NotebookBlock():
```python
# Calculate the area of a circle with a radius of 5
radius = 5
area = 3.1416 * radius ** 2
print(area)
```

## Tool_request Action
* Signature:
{
    "action_name": "tool_request",
    "argument": {
         "request": [
             ...
         ]
    }
}
* Description: The Tool Request Action allows you to send tool requirements to the Tool Manager and request it to create appropriate tools. You need to provide the action in a JSON format, where the argument field contains a request parameter that accepts a list. Each element in the list is a string describing the desired tool.
* Note:
* Examples:
- Example 1:
{
    "action_name": "tool_request",
    "argument": {
        "request": [
            "I need a tool that calculates the average value of a specified column in a dataset. The input should include the column name."
        ]
    }
}
- Example 2:
{
    "action_name": "tool_request",
    "argument": {
        "request": [
            "I need a tool that filters rows in a dataset based on a specified condition. The input should include the column name and the condition to filter by."
        ]
    }
}


## Terminate Action
* Signature: Terminate(result=the result of the task)
* Description: The Terminate action ends the process and provides the task result. The `result` argument contains the outcome or status of task completion.
* Examples:
  - Example1: Terminate(result="A")
  - Example2: Terminate(result="1.23")

# RESPONSE FORMAT #
For each task input, your response should contain:
1. One RESPONSE should only contain One Stage, One Thought and One Action.
2. An current phase of task completion, outlining the steps from planning to review, ensuring progress and adherence to the workflow.  (prefix "Stage: ").
3. An analysis of the task and the current environment, including reasoning to determine the next action based on your role as a SolvingAgent. (prefix "Thought: ").
4. An action from the **ACTION SPACE** (prefix "Action: "). Specify the action and its parameters for this step.

# RESPONSE EXAMPLE #
Observation: ...(the output of last actions, as provided by the environment and the code output, you don't need to generate it)

Stage:...(One Stage from `WORKFLOW`)
Thought: ...
Action: ...(Use an action from the ACTION SPACE no more than once per response.)

# TASK #
===task===
\end{Verbatim}
\end{tcolorbox}

\begin{tcolorbox}[title=Tool Manager's Prompt, breakable, width=\textwidth,top=0mm]
\begin{Verbatim}[breaklines, fontsize=\footnotesize]
# Instruction #
You are a Tool Manager in a collaborative team, specializing in assembling existing APIs to construct hierarchical and reusable abstract tools based on predefined criteria.
You will be provided with a custom library, similar to Python’s built-in modules, containing various functions related to date reasoning. For each task, you will receive:
1. Tool request: The specific goal or functionality the new tool must achieve.
2. Existing tools: A list of available functions from the custom library that you can utilize.
Your task is to analyze the given request and create a reusable tool by effectively leveraging the relevant functions from the existing tools or utilizing basic tools to achieve the desired functionality. 
If an existing tool from the provided library already fully satisfies the requirements, simply return that tool instead of duplicating functionality. Ensure all responses align with reusability and efficiency principles.

# Custom Library #
===api===

# Creation Criteria #
- **Reusability**: The function could be resued for more complex function.
- **Innovation**: Tools should offer innovation, not merely wrap or replicate existing APIs. Simply re-calling an API without significant enhancements does not qualify as innovation.
- **Completeness**: The function should handle potential edge cases to ensure completeness.
- **Leveraging Existing Functions**: The function should effectively utilize existing functions to enhance efficiency and avoid redundancy.
- **Functionality**: Ensure the tool runs successfully and is bug-free, guaranteeing full functionality.

# ACTION SPACE #
You should Only take One action below in one RESPONSE:
## Create tool Action
* Description: The Create Tool action allows you to develop a new tool and temporarily store it in a private repository accessible only to you. Each invocation creates a single tool at a time. You can repeatedly use this action to build smaller components, which can later be assembled into the final tool.
* Signature: 
Create_tool(tool_name=The name of the tool you want to create):
```python
The source code of tool
```
* Example:
Create_tool(tool_name=“calculate_column_statistics”):
```python
def calculate_column_statistics(dataset: pd.DataFrame, column_name: str) -> Dict[str, float]:
    """
    Calculates basic statistics (mean, median, standard deviation) for a specified column in a dataset.
    Parameters:
    - dataset: A pandas DataFrame containing the data.
    - column_name: The name of the column to calculate statistics for.
    Returns:
    - A dictionary containing the mean, median, and standard deviation of the column.
    """
    if column_name not in dataset.columns:
        raise ValueError(f"Column '{column_name}' not found in the dataset.")
    
    column_data = dataset[column_name]
    stats = {
        "mean": column_data.mean(),
        "median": column_data.median(),
        "std_dev": column_data.std()
    }
    return stats
```
## Edit tool Action
* Description: The Edit Tool action allows you to modify an existing tool and temporarily store it in a private repository that only you can access. You must provide the name of the tool to be updated along with the complete, revised code. Please note that only one tool can be edited at a time.
* Signature: 
Edit_tool(tool_name=The name of the tool you want to create):
```python
The edited source code of tool
```
* Examples:
Edit_tool(tool_name="filter_rows_by_condition"):
```python
def filter_rows_by_condition(dataset: pd.DataFrame, column_name: str, condition: str) -> pd.DataFrame:
    """
    Filters rows in a dataset based on a specified condition for a given column.
    Parameters:
    - dataset: A pandas DataFrame containing the data.
    - column_name: The name of the column to apply the condition to.
    - condition: A string representing the condition, e.g., 'value > 10'.
    Returns:
    - A filtered DataFrame containing only the rows that satisfy the condition.
    """
    if column_name not in dataset.columns:
        raise ValueError(f"Column '{column_name}' not found in the dataset.")
    
    try:
        filtered_dataset = dataset.query(f"{column_name} {condition}")
    except Exception as e:
        raise ValueError(f"Invalid condition: {condition}. Error: {e}")
    
    return filtered_dataset
```

# RESPONSE FORMAT #
For each task input, your response should contain:
1. Each response should contain only one "Thought," and one "Action."
2. Determine how to construct your tool to meet tool request and function creation criteria. Check if any functions in the Existing Tool can be invoked to assist in the tool’s development and ensure alignment with the criteria.(prefix "Thought: ").
3. An action dict from the **ACTION SPACE** (prefix "Action: "). Specify the action and its parameters for this step. 

# RESPONSE EXAMPLE  #
1. If you determine that the tool request cannot be solved using existing tools, choose this mode to provide a clear and complete code solution.

Thought: ...
Action: ...

2. If you determine that the tool request is already satisfied by existing tools, choose this mode to directly reference and return the relevant tool without creating additional solutions.
Thought: ...
Tool: {  
    "tool_name": "Name of Existing tools"
}

# NOTICE #
1. You can directly call and use the tool in the custom library in your code or tool without importing it.
2. You can only create or edit one tool per response, so take it one step at a time.

# TASK #
===task===
\end{Verbatim}
\end{tcolorbox}


\begin{tcolorbox}[title=Prompt of Self-Check Step 1, breakable, width=\textwidth,top=0mm]
\begin{Verbatim}[breaklines, fontsize=\footnotesize]
# Instruction #
You are evaluating whether the tools provided by the Tool Manager meet the required standards. 
You follow a defined workflow, take actions from the ACTION SPACE, and apply the evaluation criteria. 

# Evaluation Criteria #
- **Reusability**: The function should be designed for reuse in more complex scenarios. For instance, in the case of the `craft_wooden_sword()` tool, it would be more versatile if it could accept a quantity as an input parameter.
- **Innovation**: Tools should offer innovation, not merely wrap or replicate existing APIs. Simply re-calling an API without significant enhancements does not qualify as innovation. If an existing tool from the provided library already fully satisfies the requirements, simply return that tool instead of duplicating functionality. Ensure all responses align with reusability and efficiency principles.
- **Completeness**: The function should handle potential edge cases to ensure completeness.
- **Leveraging Existing Functions**: Check if any function in "Existing Function" is helpful for completing the task. If such functions exist but are not invoked in the provided code, relevant feedback should be given.

## Tool Abstraction ##
Tool abstraction is essential for enabling tools to adapt to diverse tasks. Key principles include:
- Design generic functions to handle queries of the same type, based on shared reasoning steps, avoiding specific object names or terms.
- Name functions and write docstrings to reflect the core reasoning pattern and data organization, without referencing specific objects.
- Use general variable names and pass all column names as arguments to enhance adaptability.

# ACTION SPACE #
You should Only take One action below in one RESPONSE:
# Feedback Action
* Signature: {
    "action_name": "Feedback",
    "argument": {
        "feedback": ...
        "passed": true/false
    }
}
* Description: The Feedback Action is represented as a JSON string that provides feedback to the Tool Manager or SolvingAgent. The feedback field contains comments or suggestions, while pass indicates whether the tool meets the requirements (true for approval, false for rejection). Feedback should be concise, constructive, and relevant. If pass is true, the feedback can be left empty; otherwise, it must be provided.
* Example:
- Example1:
{
    "action_name": "Feedback",
    "argument": {
        "feedback": "",
        "passed": true
    }
}
- Example2:
{
    "action_name": "Feedback",
    "argument": {
        "feedback": "The tool correctly solves the equation for small numbers, but fails when the coefficients are very large. Consider optimizing the algorithm for handling larger values and improving computational efficiency.",
        "passed": false
    }
}

# RESPONSE FORMAT #
For each task input, your response should contain:
1. One RESPONSE should ONLY contain One Thought and One Action.
2. An comprehensive analysis of the tool code based on the evaluation criteria.(prefix "Thought: ").
3. An action from the **ACTION SPACE** (prefix "Action: "). 

# EXAMPLE RESPONSE #
Observation: ...(output from the last action, provided by the environment and task input, no need for you to generate it)

Thought: 1. Reusability: ...
2. Innovation: ...
3. Completeness: ...
4. Leveraging Existing Functions: ...

Action: ...(Use an action from the ACTION SPACE once per response.)

# Custom Library #
===api===

# TASK #
===task===
\end{Verbatim}
\end{tcolorbox}

\begin{tcolorbox}[title=Prompt of Self-Check Step 2, breakable, width=\textwidth,top=0mm]
\begin{Verbatim}[breaklines, fontsize=\footnotesize]
# Instruction #
You are verifying whether the tools provided by the Tool Manager execute without runtime errors.
You will use a custom library, similar to the built-in library, which provides everything necessary for the tasks. Your task is only to execute the provided tool code and check for runtime errors, not to evaluate the tool’s functionality or correctness.

# Stage and Workflow #
1. **Ensure Bug-Free Tool Operation**:
	- Execute the tool to ensure it runs without any runtime bugs.
	- You don’t need to verify the function’s functionality; simply call it to check for any runtime errors.
	- If the tool is a retrieved API, skip this step and proceed.
2. **Send Feedback**:
	- After executing the code, provide feedback based on the output, indicating whether the operation was successful or not.

# Notice #
1. If any issues with the tool are found, promptly provide clear and critical feedback to the Tool Manager for resolution. 
2. You should not create or edit functions (tools) with the same name as the Existing Functions in the code.
3. You can directly call the APIs from the custom library without needing to import or declare any external libraries.
4. You don’t need to verify the function’s functionality or set up its standard output; simply call it to check for any errors.

# ACTION SPACE #
You should Only take One action below in one RESPONSE:
## Python Action
* Signature: 
Python(file_path=python_file):
```python
executable_python_code
```
* Description: The Python action will create a python file in the field `file_path` with the content wrapped by paired ``` symbols. If the file already exists, it will be overwritten. After creating the file, the python file will be executed. Remember You can only create one python file.
* Examples:
- Example1
Python(file_path="solution.py"):
```python
# Calculate the area of a circle with a radius of 5
radius = 5
area = 3.1416 * radius ** 2
print(f"The area of the circle is {area} square units.")
```
- Example2
Python(file_path="solution.py"):
```python
# Calculate the perimeter of a rectangle with length 8 and width 3
length = 8
width = 3
perimeter = 2 * (length + width)
print(f"The perimeter of the rectangle is {perimeter} units.")
```

# Feedback Action
* Signature: {
    "action_name": "Feedback",
    "argument": {
        "feedback": ...
        "passed": true/false
    }
}
* Description: The Feedback Action is used to provide feedback to the Tool Manager. The feedback field contains detailed comments or suggestions. If the tool encounters an error, you should set passed to false and provide a detailed feedback. If the tool runs without errors, you can set passed to true and leave feedback as an empty string.
* Examples:
- Example 1:
{
    "action_name": "Feedback",
    "argument": {
        "feedback": ""
        "passed": true
    }
}
- Example 2:
{
    "action_name": "Feedback",
    "argument": {
        "feedback": "The tool encountered an error while executing. The variable 'height' is missing in the function call. Please ensure that all required parameters are provided.",
        "passed": false
    }
}

# RESPONSE FORMAT #
For each task input, your response should contain:
1. One RESPONSE should ONLY contain One Thought and One Action.
2. An analysis of the task and current environment, reasoning through the next evaluation step based on your role as CheckingAgent.(prefix "Thought: ").
3. An action from the **ACTION SPACE** (prefix "Action: "). Specify the action and its parameters for this step.

# EXAMPLE RESPONSE #
Observation: ...(output from the last action, provided by the environment and task input, no need for you to generate it)

Thought: ...
Action: ...(Use an action from the ACTION SPACE once per response.)

# Custom Library #
You can use pandas, sklearn, or other Python libraries as part of the custom library.

* Note: You can directly call these tools without importing or redefining them in your code.

Let's think step by step.
# TASK #
===task===
\end{Verbatim}
\end{tcolorbox}

\subsection{Test Stage}
\label{appsub:test_prompt}
During the test stage, the prompts used for different datasets are as follows:
\begin{tcolorbox}[title=Prompt on DABench, breakable, width=\textwidth,top=0mm]
\begin{Verbatim}[breaklines, fontsize=\footnotesize]
# Instruction #
You are a helpful assistant, skilled in data science tasks.
You will be provided with a task description and related files. 
You should complete tasks by writing notebook code to interact with the environment containing the task files.
Additionally, you must strictly adhere to the task constraints. 
Once the task is completed, you need to format the answer as specified in the answer format and invoke the Terminate action to conclude.
You should use actions from the ACTION SPACE, follow the Response Format, and complete the task within 20 steps.

You may also leverage the following helper functions if needed.
===api===


===example===


# Response Format #
Your each response should contain:
1. One RESPONSE should only contain ONLY One Thought and ONLY One Action.
2. Only an analysis of the task and the current environment, including reasoning to determine the next action. (prefix "Thought: ").
3. Only an action from the **ACTION SPACE** (prefix "Action: "). Specify the action and its parameters for this step.

Observation: ...(Provided by the environment, no need for you to generate it.))

Thought: ...
Action: ...

# ACTION SPACE #
## NotebookBlock Action
* Signature: 
NotebookBlock():
```python
executable python script
```
* Description: The NotebookBlock action allows you to create and execute a Jupyter Notebook cell. The action will add a code block to the notebook with the content wrapped inside the paired ``` symbols. If the block already exists, it can be overwritten based on the specified conditions (e.g., execution errors). Once added or replaced, the block will be executed immediately.
* Restrictions: Each response must contain only one notebook block.
* Note: In a single block, you may call multiple tools or single.
* Example:
Action: NotebookBlock():
```python
# Calculate the area of a circle with a radius of 5
radius = 5
area = 3.1416 * radius ** 2
print(area)
```

# Terminate Action
* Signature: Terminate(result="the result of the task")
* Description: The Terminate action marks the completion of a task and presents the final result. It is a formatting guideline, not an executable Python function. The result parameter must contain a clear, specific answer that strictly complies with the task’s output format, with all required values explicitly provided.
Tips:
    - Ensure the result parameter provides a definite and concrete final answer.
    - Do not include unresolved Python expressions, placeholders, or variables (e.g., @value[{x + y}] or @result[{variable_name}] or "@result[{variable_name}]".format(variable_name)).
    - The output must adhere precisely to the task’s formatting specifications, ensuring clarity and consistency.
* Examples:
- Example 1: 
Answer Format: @shapiro_wilk_statistic[test_statistic] @shapiro_wilk_p_value[p_value]
Action: Terminate(result="@shapiro_wilk_statistic[0.56] @shapiro_wilk_p_value[0.0002]")
- Example 2: 
Answer Format: @total_votes_outliers_num[outlier_num]
where "outlier_num" is an integer representing the number of values considered outliers in the 'total_votes' column.
Action: Terminate(result="@total_votes_outliers[10]")
- Example3:
Action: Terminate(result="@normality_test_result[Not Normal] @p_value[0.000]")

## Response Example
Here are four examples of responses.
## Example1
Thought: The dataset has been loaded successfully and it contains the "Close Price" column. Now, we need to calculate the mean of the "Close Price" column using pandas.
Action: NotebookBlock():
```python
# Calculate the mean of the "Close Price" column
mean_close_price = data["Close Price"].mean()
# Round the result to two decimal places
mean_close_price_rounded = round(mean_close_price, 2)
print(mean_close_price_rounded)
```
## Example2
Thought: We need to filter the dataset to only include rows where the “Volume” is greater than 100,000. This will help focus on high-volume trades.
Action: NotebookBlock():
```python
# Filter rows where "Volume" is greater than 100,000
filtered_data = data[data["Volume"] > 100000]
# Display the filtered dataset
print(filtered_data)
```
## Example3
Thought: To analyze the correlation between “Open Price” and “Close Price,” we will calculate the Pearson correlation coefficient using pandas.
Action: NotebookBlock():
```python
# Calculate the correlation between "Open Price" and "Close Price"
correlation = data["Open Price"].corr(data["Close Price"])
# Print the correlation result
print(correlation)
```
## Example4
Thought: To check for missing values in the dataset, we need to check for null values in each column using pandas.
Action: NotebookBlock():
```python
# Check for missing values in each column
missing_values = data.isnull().sum()
# Display the result
print(missing_values)
```

# Begin #
Let's Begin.
## Task 
===task===
\end{Verbatim}
\end{tcolorbox}


\begin{tcolorbox}[title=Prompt on TextCraft, breakable, width=\textwidth,top=0mm]
\begin{Verbatim}[breaklines, fontsize=\footnotesize]
# Instruction #
You are provided with a set of useful crafting recipes to create items in Minecraft.
Crafting commands follow the format: "craft [target object] using [input ingredients]".
You can either "fetch" an object (ingredient) from the inventory or the environment or "craft" the target object using the provided crafting commands.
You are allowed to use only the crafting commands provided; do not invent or use your own crafting commands.
If a crafting command specifies a generic ingredient, such as "planks", you can substitute it with a specific type of that ingredient (e.g., “dark oak planks”).
To complete the crafting tasks, you will write notebook code utilizing tools from the "Custom Library". You should carefully read and understand the tool’s docstrings and code to fully grasp their functionality and usage.
The tools should be invoked by outputting a block of Python code. Additionally, you may incorporate Python constructs such as for-loops, if-statements, and other logic where necessary.
Please always use actions from the ACTION SPACE and follow the Response Format.


# ACTION SPACE #
## NotebookBlock Action
* Signature: 
NotebookBlock():
```python
executable python script
```
* Description: The NotebookBlock action creates and executes a Jupyter Notebook cell. It adds a code block wrapped in ``` symbols, overwriting existing blocks if specified (e.g., after execution errors). The block is executed immediately after being added or replaced.
* Note: In a single block, you may call multiple tools.

## Terminate Action
* Signature: Terminate(result=the result of the task)
* Description: The Terminate action ends the process and provides the task result. The `result` argument contains the outcome or status of task completion. Only the CheckingAgent has the authority to decide whether a task is finished.
* Examples:
  - Example1: Action: Terminate(result="3")
  - Example2: Action: Terminate(result="Successfully craft 2 oak planks")
  - Example3: Action: Terminate(result="Successfully craft 1 milk")


# Response Format #
For each task input, your response should contain:
1. One RESPONSE should only contain ONLY One Thought and ONLY One Action.
2. An analysis of the task and the current environment, including reasoning to determine the next action. (prefix "Thought: ").
3. An action from the **ACTION SPACE** (prefix "Action: "). Specify the action and its parameters for this step.

## Response Example
### Example1
Thought: To craft a crimson fence, I need to use the crafting command: “craft 3 crimson fence using 2 stick, 4 crimson planks.” This means I need two ingredients: 2 sticks and 4 crimson planks. To obtain 4 crimson planks, I can use the command “craft 4 crimson planks using 1 crimson stems,” so I need to fetch 1 crimson stem first. For the sticks, I can use the command “craft 4 stick using 2 planks.” Since the crafting of sticks requires planks, I can directly use the crimson planks I craft earlier for this purpose.
Steps: 
1. Fetch 1 crimson stems.
2. Craft 4 crimson planks using 1 crimson stems.
3. Craft 4 sticks using 2 crimson planks.
4. Craft 3 crimson fences using 2 sticks and 4 crimson planks.
Action: NotebookBlock()
```python
get_object('1 crimson stems')
```
### Example2
Thought: The inventory is currently empty, which means I need to retrieve the necessary spruce logs from the environment. Since we need spruce planks to craft the spruce trapdoor, and each spruce trapdoor requires 6 spruce planks, I should start by fetching an adequate number of spruce logs. Since each spruce log can craft 4 spruce planks, we need at least 2 spruce logs to ensure we can craft 8 spruce planks (more than enough for 2 spruce trapdoors if needed in the future).
Action: NotebookBlock()
```python
# Step 2: Fetch 2 spruce logs from the environment
get_object('2 spruce logs')
```
### Example3
Thought: We have successfully retrieved 2 spruce logs from the environment. 
Action: Terminate(result="Successfully crafted 8 spruce planks")


# Custom Library #
### Tool `check_inventory`
Source Code:
```python
def check_inventory() -> str:
    """
    Retrieves the current inventory state from the environment.
    The function sends an 'inventory' command to the environment
    and processes the observation to return a string representation
    of the inventory, listing items and their quantities.
    Returns:
        str: A string describing the inventory in the format:
             "Inventory: [item_name] (quantity)"
    """
    obs, _ = step('inventory')
    return obs
```
Usage Example:
```python
check_inventory() 
# If the environment has no items, Output: Inventory: You are not carrying anything.
# If the environment contains 2 oak planks, Output: Inventory: [oak planks] (2)
```
### Tool `get_object`
Source Code:
```python
def get_object(target: str) -> None:
    """
    Retrieves an item from the environment.

    The function prints the response message from the environment, 
    indicating whether the retrieval was successful or not.

    Args:
        target (str): The name of the item to be retrieved.

    Returns:
        None
    """
    obs, _ = step("get " + target)
    print(obs)
```
Usage Example:
Craft Command:
craft 2 yellow dye using 1 sunflower
craft 8 yellow carpet using 8 white carpet, 1 yellow dye
```python
get_object("1 sunflower") # Ouput: Got 1 sunflower
get_object("2 sunflower") # Ouput: Got 2 sunflower
# Note: You cannot retrieve yellow dye directly from the environment; it must first be crafted using sunflowers.
get_object("1 yellow dye") # Output: Could not find yellow dye
```
### Tool `craft_object`
Source Code:
```python
def craft_object(target: str, ingredients: List[str]) -> None:
    """
    Crafts a specified item using the given ingredients.

    This function's `target` and `ingredients` parameters correspond to the craft command: 
    "Craft 'target' using [ingredients]".
    
    **Note:** The `ingredients` must exactly match the command format. For example, if the command requires 
    '1 oak logs', providing '1 oak log' instead will not be recognized.

    Prints the environment's response to indicate whether the crafting operation was successful or not.

    Args:
        target (str): The item to craft along with its quantity (e.g., '4 oak planks').
        ingredients (List[str]): A list of required ingredients with their respective quantities 
                                (e.g., ['1 oak logs']).

    Returns:
        None

    """
    obs, _ = step("craft " + target + " using " + ", ".join(ingredients))
    print(obs)
```
Usage Example:
- Example1
Query: Craft 1 black dye using 1 wither rose
Craft Command:
craft 1 black dye using 1 wither rose
craft 1 black dye using 1 ink sac

```python
get_object('1 wither rose')
craft_object('1 black dye', ['1 wither rose'])
# Ouput:
# Got 1 wither rose
# Crafted 1 minecraft:black dye
```
- Example2
Craft Commands: 4 oak planks using 1 oak logs
```python
get_object("1 oak logs") # Got 1 oak logs
check_inventory() # Inventory: ["oak logs"] (1)
craft_object('4 oak planks', ['1 oak logs'])
# Crafted 4 minecraft:oak planks
```

===api===


===example===

# Example #
Here is a demo of how to fetch and craft objects.
Crafting commands:
craft 3 dark oak sign using 6 dark oak planks, 1 stick
craft 4 dark oak planks using 1 dark oak logs
craft 1 stick using 1 planks
craft 4 stick using 2 bamboo
craft 4 oak planks using 1 oak logs
craft 1 dark oak fence using 2 stick, 4 dark oak planks
craft 1 warped stairs using 6 warped planks
craft 3 oak sign using 6 oak planks, 1 stick

craft 2 oak planks

Thought: To craft 2 oak planks, I need to first check if I can fetch them directly from the environment or my inventory. If they are not available, I will craft them using the appropriate crafting recipe. Based on the crafting commands, I know that I can craft 4 oak planks using 1 oak logs. My plan is as follows:
1. Check if 2 oak planks are already in my inventory or can be fetched from the environment.
2. If not, verify if I have the required ingredient (1 oak logs).
3. Gather the missing ingredient, if necessary.
4. Craft 4 oak planks and ensure 2 are available for the task.
Action: NotebookBlock():
```python
check_inventory()
```

Observation: Inventory: [stick] (1) 

Thought: I could not find 2 oak planks or the required ingredient (1 oak logs) in my inventory. I need to fetch 1 oak logs first.
Action: NotebookBlock():
```
get_object("1 oak logs")
```

Observation: Got 1 oak logs

Thought: I now have the required ingredient to craft oak planks. I will use the crafting command: craft 4 oak planks using 1 oak logs.
Action: NotebookBlock():
```
craft_object("4 oak planks", ["1 oak logs"])
check_inventory()
```

Observation: Crafted 4 minecraft:oak_planks
Inventory: [stick] (1) [oak_planks] (4)

Thought: I now have 2 oak planks in my inventory. Task Completed!
Action: Terminate(result=“Successfully craft 2 oak planks”)


# Begin #
Let's Begin. Please follow the Instruction and focus on solve the task.
# Current items in the inventory #
These are the items currently in your inventory. You cannot use items that are not in your inventory to craft new items.
===inventory===

# Task #
===task===
\end{Verbatim}
\end{tcolorbox}

\begin{tcolorbox}[title=Prompt on MATH, breakable, width=\textwidth,top=0mm]
\begin{Verbatim}[breaklines, fontsize=\footnotesize]
Your task is to solve math competition problems by writing Python programs.

You may also leverage the following helper functions, but must avoid fabricating and calling undefined function names.
```python
===api===
```

Examples: 

Examples: 
Query: Point $P$ lies on the line $x= -3$ and is 10 units from the point $(5,2)$. Find the product of all possible $y$-coordinates that satisfy the given conditions.
Program: 
```python
from sympy import symbols, Eq, solve
# Define symbolic variable for y-coordinate of point P
y = symbols('y')
# Step 1: Given conditions
x1 = -3  # Point P lies on the vertical line x = -3
x2, y2 = 5, 2  # Coordinates of the given point (5, 2)
d = 10  # Distance between point P and (5,2)
# Step 2: Apply the distance formula
# Distance formula: sqrt((x2 - x1)^2 + (y - y2)^2) = d
# Squaring both sides to eliminate the square root:
# (x2 - x1)^2 + (y - y2)^2 = d^2
distance_equation = Eq((x2 - x1)**2 + (y - y2)**2, d**2)
# Step 3: Solve for possible values of y
y_solutions = solve(distance_equation, y)
# Step 4: Compute the product of all possible y-values
product = y_solutions[0] * y_solutions[1]
# Step 5: Output the final result
print("Final Answer:", product)
```

Query: If $3p+4q=8$ and $4p+3q=13$, what is $q$ equal to?
Program:
```python
from sympy import symbols, Eq, solve
# Define symbolic variables for the unknowns p and q
p, q = symbols('p q')
# Step 1: Define the given system of equations
eq1 = Eq(3 * p + 4 * q, 8)  # Equation 1: 3p + 4q = 8
eq2 = Eq(4 * p + 3 * q, 13)  # Equation 2: 4p + 3q = 13
# Step 2: Solve the system of equations for p and q
solution = solve((eq1, eq2), (p, q))
# Step 3: Extract and output the value of q
print("Final Answer:", solution[q])
```

Query: Simplify $\frac{3^4+3^2}{3^3-3}$. Express your answer as a common fraction.
Program:
```python
from sympy import symbols, simplify
# Define the variable
x = symbols('x')
# Define the expression
numerator = 3**4 + 3**2
denominator = 3**3 - 3
fraction = numerator / denominator
# Simplify the fraction
simplified_fraction = simplify(fraction)
# Print the result
print("Final Answer:", simplified_fraction)
```

===example===

## Begin !
Please generate ONLY the code wrapped in ```python...``` to solve the query below.

Query: ===task===
Program:
\end{Verbatim}
\end{tcolorbox}



\begin{tcolorbox}[title=Prompt on Date, breakable, width=\textwidth,top=0mm]
\begin{Verbatim}[breaklines, fontsize=\footnotesize]
Your task is to solve simple word problems by creating Python programs.

You may also leverage the following helper functions, but must avoid fabricating and calling undefined function names, such as `calculate_date_by_years`.
```python
===api===
```

Examples:

Query: In the US, Thanksgiving is on the fourth Thursday of November. Today is the US Thanksgiving of 2001. What is the date one week from today in MM/DD/YYYY?
Program:
```python
# import relevant packages
from datetime import date, time, datetime
from dateutil.relativedelta import relativedelta
import calendar
# 1. What is the date of the first Thursday of November? (independent, support: [])
date_1st_thu = date(2001,11,1)
while date_1st_thu.weekday() != calendar.THURSDAY:
    date_1st_thu += relativedelta(days=1)
# 2. How many days are there in a week? (independent, support: ["External knowledge: There are 7 days in a week."])
n_days_of_a_week = 7
# 3. What is the date today? (depends on 1 and 2, support: ["Today is the US Thanksgiving of 2001", "Thanksgiving is on the fourth Thursday of November"])
days_from_1st_to_4th_thu = (4-1) * n_days_of_a_week
date_today = date_1st_thu + relativedelta(days=days_from_1st_to_4th_thu)
# 4. What is the date one week from today? (depends on 3, support: [])
date_1week_from_today = date_today + relativedelta(weeks=1)
# 5. Final Answer: What is the date one week from today in MM/DD/YYYY? (depends on 4, support: [])
answer = date_1week_from_today.strftime("%m/%d/%Y")
# print the answer
print(answer)
```

Query: Yesterday was 12/31/1929. Today could not be 12/32/1929 because December has only 31 days. What is the date tomorrow in MM/DD/YYYY?
Program:
```python
# import relevant packages
from datetime import date, time, datetime
from dateutil.relativedelta import relativedelta
# 1. What is the date yesterday? (independent, support: ["Yesterday was 12/31/1929"])
date_yesterday = date(1929,12,31)
# 2. What is the date today? (depends on 1, support: ["Today could not be 12/32/1929 because December has only 31 days"])
date_today = date_yesterday + relativedelta(days=1)
# 3. What is the date tomorrow? (depends on 2, support: [])
date_tomorrow = date_today + relativedelta(days=1)
# 4. Final Answer: What is the date tomorrow in MM/DD/YYYY? (depends on 3, support: [])
answer = date_tomorrow.strftime("%m/%d/%Y")
# print the answer
print(answer)
```

Query: The day before yesterday was 11/23/1933. What is the date one week from today in MM/DD/YYYY?
Program:
```python
# import relevant packages
from datetime import date, time, datetime
from dateutil.relativedelta import relativedelta
# 1. What is the date the day before yesterday? (independent, support: ["The day before yesterday was 11/23/1933"])
date_day_before_yesterday = date(1933,11,23)
# 2. What is the date today? (depends on 1, support: [])
date_today = date_day_before_yesterday + relativedelta(days=2)
# 3. What is the date one week from today? (depends on 2, support: [])
date_1week_from_today = date_today + relativedelta(weeks=1)
# 4. Final Answer: What is the date one week from today in MM/DD/YYYY? (depends on 3, support: [])
answer = date_1week_from_today.strftime("%m/%d/%Y")
# print the answer
print(answer)
```

===example===

## Begin !
Please generate ONLY the code wrapped in ```python...``` to solve the query below.

Query: ===task===
Program:
\end{Verbatim}
\end{tcolorbox}



\begin{tcolorbox}[title=Prompt on TabMWP, breakable, width=\textwidth,top=0mm]
\begin{Verbatim}[breaklines, fontsize=\footnotesize]
Your task is to solve table-reasoning problems by writing Python programs.
You are given a table. The first row is the name for each column. Each column is seperated by "|" and each row is seperated by "\n".
Pay attention to the format of the table, and what the question asks.

You may also leverage the following helper functions, but must avoid fabricating and calling undefined function names.
```python
===api===
```


Examples: 
### Table
Name: None
Unit: $
Content:
Date | Description | Received | Expenses | Available Funds
 | Balance: end of July | | | $260.85
8/15 | tote bag | | $6.50 | $254.35
8/16 | farmers market | | $23.40 | $230.95
8/22 | paycheck | $58.65 | | $289.60
### Question
This is Akira's complete financial record for August. How much money did Akira receive on August 22?
### Solution code
```python
records = {
    "7/31": {"Description": "Balance: end of July", "Received": "", "Expenses": "", "Available Funds": 260.85},
    "8/15": {"Description": "tote bag", "Received": "", "Expenses": 6.5, "Available Funds": ""},
    "8/16": {"Description": "farmers market", "Received": "", "Expenses": 23.4, "Available Funds": ""},
    "8/22": {"Description": "paycheck", "Received": 58.65, "Expenses": "", "Available Funds": ""}
}
# Access the amount received on August 22
received_aug_22 = records["8/22"]["Received"]
print("Final Answer: ", received_aug_22)
```

### Table
Name: Orange candies per bag
Unit: bags
Content:
Stem | Leaf 
2 | 2, 3, 9
3 | 
4 | 
5 | 0, 6, 7, 9
6 | 0
7 | 1, 3, 9
8 | 5
### Question
A candy dispenser put various numbers of orange candies into bags. How many bags had at least 32 orange candies?
### Solution code
```python
data = {
    2: [2, 3, 9],
    3: [],
    4: [],
    5: [0, 6, 7, 9],
    6: [0],
    7: [1, 3, 9],
    8: [5]
}
# Initialize the count to zero
count = 0
# Iterate over the keys in the dictionary
for key in data:
    # Combine tenth digit and unit digit
    if key * 10 + data[key] >= 32:
        # Increment the count
        count += 1

# Output the result
print("Final Answer: ", count)
```

### Table
Name: Monthly Savings  
Unit: $  
Content:  
Date  | Description       | Received | Expenses | Available Funds |
       | Balance: end of May |   |   | $500.00 |
6/10  | groceries        |   | $45.75 | $454.25 |
6/15  | gas refill       |   | $30.20 | $424.05 |
6/25  | salary           | $1200.00 |   | $1624.05 |
### Question
How much money did Akira receive on June 25?
### Solution code
```python
import pandas as pd
records = {
    "5/31": {"Description": "Balance: end of May", "Received": "", "Expenses": "", "Available Funds": 500.00},
    "6/10": {"Description": "groceries", "Received": "", "Expenses": 45.75, "Available Funds": ""},
    "6/15": {"Description": "gas refill", "Received": "", "Expenses": 30.2, "Available Funds": ""},
    "6/25": {"Description": "salary", "Received": 1200.00, "Expenses": "", "Available Funds": ""}
}
# Access the amount received on June 25
received_june_25 = records["6/25"]["Received"]
print("Final Answer: ", received_june_25)
```

===example===

## Begin!
Please solve the task below and enclose your code within a single code block using ```python```  format.

===task===
### Solution code
\end{Verbatim}
\end{tcolorbox}









\section{Examples}
\label{app:example}
\subsection{Generated Tools}

\textbf{The tools generated for the Open-ended Tasks are as follows:}
\begin{tcolorbox}[title=CraftDiamondHelmet, width=\textwidth,top=0mm,  breakable]
\begin{Verbatim}[breaklines=true, breakanywhere=true, fontsize=\footnotesize]
async function craftDiamondHelmet(bot) {
  const mcData = require('minecraft-data')(bot.version);
  const Vec3 = require('vec3').Vec3;

  // Check inventory for the number of diamonds
  const diamondCount = bot.inventory.count(mcData.itemsByName["diamond"].id);
  const requiredDiamonds = 5;

  // If not enough diamonds, mine diamond ores
  if (diamondCount < requiredDiamonds) {
    const diamondsToMine = requiredDiamonds - diamondCount;
    bot.chat(`Need ${diamondsToMine} more diamond(s). Mining...`);
    await mineBlock(bot, "diamond_ore", diamondsToMine);
  }

  // Ensure a crafting table is placed nearby
  let craftingTable = bot.findBlock({
    matching: mcData.blocksByName.crafting_table.id,
    maxDistance: 32
  });
  if (!craftingTable) {
    bot.chat("Placing crafting table...");
    await placeItem(bot, "crafting_table", bot.entity.position.offset(1, 0, 0));
    craftingTable = bot.findBlock({
      matching: mcData.blocksByName.crafting_table.id,
      maxDistance: 32
    });
  }

  // Craft the diamond helmet
  bot.chat("Crafting diamond helmet...");
  await craftItem(bot, "diamond_helmet", 1);
  bot.chat("Diamond helmet crafted successfully.");
}
\end{Verbatim}
\end{tcolorbox}


\begin{tcolorbox}[title=CraftItemWithMaterials, width=\textwidth,top=0mm,  breakable]
\begin{Verbatim}[breaklines=true, breakanywhere=true, fontsize=\footnotesize]
async function craftItemWithMaterials(bot, itemName, requiredMaterials) {
  const mcData = require('minecraft-data')(bot.version);
  const Vec3 = require('vec3').Vec3;

  // Check inventory for required materials
  for (const material of requiredMaterials) {
    let itemCount = bot.inventory.count(mcData.itemsByName[material.name].id);
    if (itemCount < material.count) {
      const requiredCount = material.count - itemCount;
      bot.chat(`Need ${requiredCount} more ${material.name}(s).`);
      if (material.name === "diamond") {
        let diamondOre = await bot.findBlock({
          matching: mcData.blocksByName["diamond_ore"].id,
          maxDistance: 32
        });
        if (!diamondOre) {
          bot.chat("No diamond ore found nearby. Exploring...");
          diamondOre = await exploreUntil(bot, new Vec3(1, 0, 1), 60, () => {
            return bot.findBlock({
              matching: mcData.blocksByName["diamond_ore"].id,
              maxDistance: 32
            });
          });
        }
        if (diamondOre) {
          await mineBlock(bot, "diamond_ore", requiredCount);
        } else {
          bot.chat("Failed to find diamond ore after exploring.");
          return;
        }
      } else if (material.name === "stick") {
        const woodenPlanksCount = bot.inventory.count(mcData.itemsByName["oak_planks"].id) + bot.inventory.count(mcData.itemsByName["birch_planks"].id);
        if (woodenPlanksCount < 2) {
          const requiredLogs = Math.ceil((2 - woodenPlanksCount) / 4);
          bot.chat(`Need more wooden planks. Gathering ${requiredLogs} logs...`);
          await obtainWoodLogs(bot, requiredLogs);
          await craftItem(bot, "oak_planks", requiredLogs);
        }
        bot.chat("Crafting sticks...");
        await craftItem(bot, "stick", 1);
      }
    }
  }

  // Ensure a crafting table is placed nearby
  let craftingTable = bot.findBlock({
    matching: mcData.blocksByName.crafting_table.id,
    maxDistance: 32
  });
  if (!craftingTable) {
    bot.chat("Placing crafting table...");
    await placeItem(bot, "crafting_table", bot.entity.position.offset(1, 0, 0));
    craftingTable = bot.findBlock({
      matching: mcData.blocksByName.crafting_table.id,
      maxDistance: 32
    });
  }

  // Craft the item
  bot.chat(`Crafting ${itemName}...`);
  await craftItem(bot, itemName, 1);
  bot.chat(`${itemName} crafted successfully.`);
}

async function craftDiamondAxe(bot) {
  const requiredMaterials = [{
    name: "diamond",
    count: 3
  }, {
    name: "stick",
    count: 2
  }];
  await craftItemWithMaterials(bot, "diamond_axe", requiredMaterials);
}
\end{Verbatim}
\end{tcolorbox}


\textbf{The tools generated for the Agent Tasks are as follows:}
Here, we can clearly see the call relationships between functions, thus forming more complex tools.
\begin{tcolorbox}[title=Tools for DA-Bench, width=\textwidth,top=0mm,  breakable]
\begin{Verbatim}[breaklines=true, breakanywhere=true, fontsize=\footnotesize]
def filter_rows_by_non_null(df: pd.DataFrame, column_name: str) -> pd.DataFrame:
    """
    Filters rows in a dataset based on non-null values in a specified column.
    
    Parameters:
    - df (pd.DataFrame): The input DataFrame.
    - column_name (str): The name of the column to filter by non-null values.
    
    Returns:
    - pd.DataFrame: A DataFrame with rows containing non-null values in the specified column.
    
    Raises:
    - ValueError: If the specified column is not found in the DataFrame.
    """
    # Check if the column exists in the DataFrame
    if column_name not in df.columns:
        raise ValueError(f"Column '{column_name}' not found in the DataFrame.")
    
    # Filter rows based on non-null values in the specified column
    filtered_df = df.dropna(subset=[column_name])
    
    return filtered_df

def convert_column_to_numeric(df: pd.DataFrame, column_name: str) -> pd.DataFrame:
    """
    Converts a specified column in a DataFrame to numeric values, handling non-numeric values appropriately.
    
    Parameters:
    - df (pd.DataFrame): The input DataFrame.
    - column_name (str): The name of the column to convert to numeric values.
    
    Returns:
    - pd.DataFrame: The DataFrame with the specified column converted to numeric values.
    
    Raises:
    - ValueError: If the specified column is not found in the DataFrame.
    """
    # Check if the column exists in the DataFrame
    if column_name not in df.columns:
        raise ValueError(f"Column '{column_name}' not found in the DataFrame.")
    
    # Convert the specified column to numeric values, setting non-numeric values to NaN
    df[column_name] = pd.to_numeric(df[column_name], errors='coerce')
    
    # Filter out rows with non-numeric values in the specified column using the existing tool
    df = filter_rows_by_non_null(df, column_name)
    
    return df

def create_sum_feature(df: pd.DataFrame, new_column_name: str, columns_to_sum: list) -> pd.DataFrame:
    """
    Creates a new feature by summing specified columns in a DataFrame.
    
    Parameters:
    - df (pd.DataFrame): The input DataFrame.
    - new_column_name (str): The name of the new column to be created.
    - columns_to_sum (list): A list of column names to sum.
    
    Returns:
    - pd.DataFrame: The DataFrame with the new feature added.
    
    Raises:
    - ValueError: If any of the specified columns are not found in the DataFrame.
    """
    # Check if all specified columns exist in the DataFrame
    for column in columns_to_sum:
        if column not in df.columns:
            raise ValueError(f"Column '{column}' not found in the DataFrame.")
    
    # Convert specified columns to numeric values
    for column in columns_to_sum:
        df = convert_column_to_numeric(df, column)
    
    # Create the new feature by summing the specified columns
    df[new_column_name] = df[columns_to_sum].sum(axis=1)
    
    return df
\end{Verbatim}
\end{tcolorbox}


\begin{tcolorbox}[title=Tools for TextCraft, width=\textwidth,top=0mm, breakable]
\begin{Verbatim}[breaklines=true, breakanywhere=true, fontsize=\footnotesize]
def gather_materials_for_dye(required_materials: dict) -> bool:
    """
    Gathers the required materials for crafting any dye.
    
    Parameters:
    - required_materials (dict): A dictionary where keys are material names and values are the required quantities.
    
    The tool checks the inventory for these materials and gathers them if they are missing.
    
    Returns:
    - bool: True if all materials were successfully gathered, False otherwise.
    """
    # Gather the required materials
    if not gather_materials(required_materials):
        return False
    
    # Check if we have white dye, if not gather bone meal or lily of the valley to craft it
    inventory = check_inventory()
    if "white dye" in required_materials and "white dye" not in inventory:
        if not gather_materials({"bone meal": 1}) and not gather_materials({"lily of the valley": 1}):
            return False
        # Craft white dye using bone meal or lily of the valley
        if "bone meal" in inventory:
            craft_object("1 white dye", ["1 bone meal"])
        elif "lily of the valley" in inventory:
            craft_object("1 white dye", ["1 lily of the valley"])
    
    # Recheck the inventory to ensure all materials are gathered
    missing_items = check_missing_items([f"{qty} {item}" for item, qty in required_materials.items()])
    if missing_items:
        print(f"Missing items: {missing_items}")
        return False
    
    # Successfully gathered all materials
    return True

def craft_orange_dye(quantity: int) -> bool:
    """
    Crafts the specified quantity of orange dye.
    
    Parameters:
    - quantity (int): The number of orange dye to craft.
    
    Returns:
    - bool: True if the orange dye was successfully crafted, False otherwise.
    """
    # Define the required materials for crafting orange dye
    required_materials = {"orange tulip": quantity, "red dye": quantity, "yellow dye": quantity}
    
    # Gather the required materials using the existing gather_materials_for_dye function
    if not gather_materials_for_dye(required_materials):
        return False
    
    # Check the inventory for available materials
    inventory = check_inventory()
    
    # Craft orange dye using orange tulip if available
    if "orange tulip" in inventory:
        craft_object(f"{quantity} orange dye", [f"{quantity} orange tulip"])
        print(f"Crafted {quantity} orange dye using {quantity} orange tulip")
        return True
    
    # Craft orange dye using red dye and yellow dye if available
    if "red dye" in inventory and "yellow dye" in inventory:
        craft_object(f"{quantity} orange dye", [f"{quantity} red dye", f"{quantity} yellow dye"])
        print(f"Crafted {quantity} orange dye using {quantity} red dye and {quantity} yellow dye")
        return True
    
    # If neither method was successful, return False
    print("Failed to craft orange dye.")
    return False
\end{Verbatim}
\end{tcolorbox}


\textbf{The tools generated for the Single-turn Code Task are as follows:}
\begin{tcolorbox}[title=Tools for MATH, width=\textwidth,top=0mm, breakable]
\begin{Verbatim}[breaklines=true, breakanywhere=true, fontsize=\footnotesize]
def find_integer_satisfying_condition(condition):
    """
    Find the smallest positive integer that satisfies the given condition.

    Parameters:
        condition (function): A lambda function representing the condition to be checked.

    Returns:
        int: The smallest positive integer that satisfies the condition.
    """
    x = 1
    while True:
        if condition(x):
            return x
        x += 1

def calculate_min_correct_answers(total_problems, passing_percentage):
    """
    Calculate the minimum number of correct answers required to pass a test based on the total number of problems and the passing percentage.

    Parameters:
        total_problems (int): The total number of problems on the test.
        passing_percentage (float): The passing percentage required to pass the test.

    Returns:
        int: The minimum number of correct answers required to pass the test.
    """
    if total_problems <= 0:
        return "Total number of problems must be greater than zero."
    if not (0 <= passing_percentage <= 100):
        return "Passing percentage must be between 0 and 100."

    required_correct_answers = (passing_percentage / 100) * total_problems

    # Use find_integer_satisfying_condition to find the minimum integer satisfying the condition
    min_correct_answers = find_integer_satisfying_condition(lambda x: x >= required_correct_answers)
    
    return min_correct_answers
\end{Verbatim}
\end{tcolorbox}

\begin{tcolorbox}[title=Tools for Date, width=\textwidth,top=0mm, breakable]
\begin{Verbatim}[breaklines=true, breakanywhere=true, fontsize=\footnotesize]
def calculate_date_by_days(start_date_str: str, days_to_add: int, date_format="%m/%d/%Y") -> str:
    """
    Calculates the date a specified number of days before or after a given date.

    Parameters:
    - start_date_str (str): The starting date as a string in the format MM/DD/YYYY.
    - days_to_add (int): The number of days to add (positive) or subtract (negative) from the start date.
    - date_format (str): The format of the input and output date string. Default is 'MM/DD/YYYY'.

    Returns:
    - str: The resulting date in the format MM/DD/YYYY.
    
    Raises:
    - ValueError: If the input date string does not match the specified format.
    - OverflowError: If the resulting date is out of the valid range for dates.
    """
    from datetime import datetime, timedelta

    try:
        # Parse the input date string into a date object using the provided format
        start_date = datetime.strptime(start_date_str, date_format).date()

        # Calculate the new date by adding the specified number of days
        new_date = start_date + timedelta(days=days_to_add)

        # Format the new date back into the desired string format
        result_date_str = new_date.strftime(date_format)

        return result_date_str
    except ValueError as e:
        raise ValueError("Incorrect date format. Please ensure the date string matches the provided format.") from e
    except OverflowError as e:
        raise OverflowError("The resulting date is out of the valid range for dates.") from e

def calculate_date_by_days_uk_format(start_date_str: str, days_to_add: int) -> str:
    """
    Calculates the date a specified number of days before or after a given date in UK format (DD/MM/YYYY).

    Parameters:
    - start_date_str (str): The starting date as a string in the format DD/MM/YYYY.
    - days_to_add (int): The number of days to add (positive) or subtract (negative) from the start date.

    Returns:
    - str: The resulting date in the format MM/DD/YYYY.
    
    Raises:
    - ValueError: If the input date string does not match the specified format.
    """
    from datetime import datetime

    try:
        # Convert the input date from DD/MM/YYYY to MM/DD/YYYY
        start_date = datetime.strptime(start_date_str, "%d/%m/%Y")
        
        # Use the existing tool to calculate the new date
        result_date_str = calculate_date_by_days(start_date.strftime("%m/%d/%Y"), days_to_add, "%m/%d/%Y")
        
        return result_date_str
    except ValueError as e:
        raise ValueError("Incorrect date format. Please ensure the date string matches the provided format.") from e
\end{Verbatim}
\end{tcolorbox}


\begin{tcolorbox}[title=Tools for TabMWP, width=\textwidth,top=0mm, breakable]
\begin{Verbatim}[breaklines=true, breakanywhere=true, fontsize=\footnotesize]
import pandas as pd

def stem_and_leaf_to_dataframe(stem_leaf_dict: dict) -> pd.DataFrame:
    """
    Converts a stem-and-leaf plot into a DataFrame.

    Parameters:
    - stem_leaf_dict (dict): A dictionary where keys are the stems and values are lists of leaves.

    Returns:
    - pd.DataFrame: A DataFrame with a single column containing the combined values of stems and leaves.
    """
    # Initialize an empty list to store the combined values
    combined_values = []

    # Iterate through the dictionary to combine stems and leaves
    for stem, leaves in stem_leaf_dict.items():
        for leaf in leaves:
            combined_value = int(f"{stem}{leaf}")
            combined_values.append(combined_value)

    # Create a DataFrame from the combined values
    df = pd.DataFrame(combined_values, columns=["Values"])
    
    return df

import pandas as pd

def count_value_occurrences(stem_leaf_dict: dict, value) -> int:
    """
    Counts the occurrences of a specific value in a DataFrame column created from a stem-and-leaf plot.

    Parameters:
    - stem_leaf_dict (dict): A dictionary where keys are the stems and values are lists of leaves.
    - value: The value to count in the DataFrame.

    Returns:
    - int: The count of the specified value in the DataFrame.
    """
    # Convert the stem-and-leaf plot to a DataFrame using the existing tool
    df = stem_and_leaf_to_dataframe(stem_leaf_dict)
    
    # Count the occurrences of the specified value in the DataFrame
    count = df["Values"].value_counts().get(value, 0)
    
    return count
\end{Verbatim}
\end{tcolorbox}
%TC:endignore

\end{document}
\endinput
%%
%% End of file `sample-sigconf-authordraft.tex'.
