%%
%% This is file `sample-sigconf-authordraft.tex',
%% generated with the docstrip utility.
%%
%% The original source files were:
%%
%% samples.dtx  (with options: `all,proceedings,bibtex,authordraft')
%% 
%% IMPORTANT NOTICE:
%% 
%% For the copyright see the source file.
%% 
%% Any modified versions of this file must be renamed
%% with new filenames distinct from sample-sigconf-authordraft.tex.
%% 
%% For distribution of the original source see the terms
%% for copying and modification in the file samples.dtx.
%% 
%% This generated file may be distributed as long as the
%% original source files, as listed above, are part of the
%% same distribution. (The sources need not necessarily be
%% in the same archive or directory.)
%%
%%
%% Commands for TeXCount
%TC:macro \cite [option:text,text]
%TC:macro \citep [option:text,text]
%TC:macro \citet [option:text,text]
%TC:envir table 0 1
%TC:envir table* 0 1
%TC:envir tabular [ignore] word
%TC:envir displaymath 0 word
%TC:envir math 0 word
%TC:envir comment 0 0
%%
%%
%% The first command in your LaTeX source must be the \documentclass
%% command.
%%
%% For submission and review of your manuscript please change the
%% command to \documentclass[manuscript, screen, review]{acmart}.
%%
%% When submitting camera ready or to TAPS, please change the command
%% to \documentclass[sigconf]{acmart} or whichever template is required
%% for your publication.
%%
%%
\documentclass[sigconf]{acmart}

%%
%% \BibTeX command to typeset BibTeX logo in the docs
% \AtBeginDocument{%
%   \providecommand\BibTeX{{%
%     Bib\TeX}}}

\AtBeginDocument{%
  \providecommand\BibTeX{{%
    \normalfont B\kern-0.5em{\scshape i\kern-0.25em b}\kern-0.8em\TeX}}}

%% Rights management information.  This information is sent to you
%% when you complete the rights form.  These commands have SAMPLE
%% values in them; it is your responsibility as an author to replace
%% the commands and values with those provided to you when you
%% complete the rights form.
\copyrightyear{2025}
\acmYear{2025}
\setcopyright{cc}
\setcctype{by}
\acmConference[CHI '25]{CHI Conference on Human Factors in Computing Systems}{April 26-May 1, 2025}{Yokohama, Japan}
\acmBooktitle{CHI Conference on Human Factors in Computing Systems (CHI '25), April 26-May 1, 2025, Yokohama, Japan}\acmDOI{10.1145/3706598.3714319}
\acmISBN{979-8-4007-1394-1/25/04}

%% These commands are for a PROCEEDINGS abstract or paper.
\begin{CCSXML}
<ccs2012>
   <concept>
       <concept_id>10003120.10003121.10011748</concept_id>
       <concept_desc>Human-centered computing~Empirical studies in HCI</concept_desc>
       <concept_significance>500</concept_significance>
       </concept>
   <concept>
       <concept_id>10003120.10003121.10003122.10003334</concept_id>
       <concept_desc>Human-centered computing~User studies</concept_desc>
       <concept_significance>500</concept_significance>
       </concept>
   <concept>
       <concept_id>10003120.10003121.10003129.10010885</concept_id>
       <concept_desc>Human-centered computing~User interface management systems</concept_desc>
       <concept_significance>500</concept_significance>
       </concept>
   <concept>
       <concept_id>10010405.10010497.10010510.10010513</concept_id>
       <concept_desc>Applied computing~Annotation</concept_desc>
       <concept_significance>500</concept_significance>
       </concept>
   <concept>
       <concept_id>10010147.10010178.10010179</concept_id>
       <concept_desc>Computing methodologies~Natural language processing</concept_desc>
       <concept_significance>500</concept_significance>
       </concept>
   <concept>
       <concept_id>10003120.10003121.10003129.10011756</concept_id>
       <concept_desc>Human-centered computing~User interface programming</concept_desc>
       <concept_significance>500</concept_significance>
       </concept>
 </ccs2012>
\end{CCSXML}

\ccsdesc[500]{Human-centered computing~Empirical studies in HCI}
\ccsdesc[500]{Human-centered computing~User studies}
\ccsdesc[500]{Human-centered computing~User interface management systems}
\ccsdesc[500]{Applied computing~Annotation}
\ccsdesc[500]{Computing methodologies~Natural language processing}
\ccsdesc[500]{Human-centered computing~User interface programming}

% end user programming
% keyword: Data Annotation, Language Model, Iterative labeling, data labeling
\keywords{Data Annotation; Data Labeling; Large Language Model; Iterative labeling; End-User Programming}
%%
%%  Uncomment \acmBooktitle if the title of the proceedings is different
%%  from ``Proceedings of ...''!
%%
%%\acmBooktitle{Woodstock '18: ACM Symposium on Neural Gaze Detection,
%%  June 03--05, 2018, Woodstock, NY}


%%
%% Submission ID.
%% Use this when submitting an article to a sponsored event. You'll
%% receive a unique submission ID from the organizers
%% of the event, and this ID should be used as the parameter to this command.
% \acmSubmissionID{4469}

%%
%% For managing citations, it is recommended to use bibliography
%% files in BibTeX format.
%%
%% You can then either use BibTeX with the ACM-Reference-Format style,
%% or BibLaTeX with the acmnumeric or acmauthoryear sytles, that include
%% support for advanced citation of software artefact from the
%% biblatex-software package, also separately available on CTAN.
%%
%% Look at the sample-*-biblatex.tex files for templates showcasing
%% the biblatex styles.
%%

%%
%% The majority of ACM publications use numbered citations and
%% references.  The command \citestyle{authoryear} switches to the
%% "author year" style.
%%
%% If you are preparing content for an event
%% sponsored by ACM SIGGRAPH, you must use the "author year" style of
%% citations and references.
%% Uncommenting
%% the next command will enable that style.
%%\citestyle{acmauthoryear}


%%
%% end of the preamble, start of the body of the document source.
\def\method{\text MixMin~}
\def\methodnospace{\text MixMin}
\def\genmethod{$\mathbb{R}$\text Min~}
\def\genmethodnospace{ $\mathbb{R}$\text Min}

\begin{document}

%%
%% The "title" command has an optional parameter,
%% allowing the author to define a "short title" to be used in page headers.

%%
%% The "author" command and its associated commands are used to define
%% the authors and their affiliations.
%% Of note is the shared affiliation of the first two authors, and the
%% "authornote" and "authornotemark" commands
%% used to denote shared contribution to the research.

%%
%% By default, the full list of authors will be used in the page
%% headers. Often, this list is too long, and will overlap
%% other information printed in the page headers. This command allows
%% the author to define a more concise list
%% of authors' names for this purpose.
\author{Zeyu He}
%\authornote{Both authors contributed equally to this research.}
\affiliation{%
  \institution{The Pennsylvania State University}
  %\streetaddress{P.O. Box 1212}
  \city{University Park}
  \state{PA}
  \country{USA}
  %\postcode{43017-6221}
}
\email{zmh5268@psu.edu}

\author{Saniya Naphade}
%\authornote{Both authors contributed equally to this research.}
\affiliation{%
  \institution{GumGum Inc.}
  %\streetaddress{P.O. Box 1212}
  \city{Tempe}
  \state{AZ}
  \country{USA}
  %\postcode{43017-6221}
}
\email{saniya.naphade@gumgum.com}

\author{Ting-Hao `Kenneth' Huang}
\affiliation{%
  \institution{The Pennsylvania State University}
  %\streetaddress{P.O. Box 1212}
  \city{University Park}
  \state{PA}
  \country{USA}
  %\postcode{43017-6221}
}
\email{txh710@psu.edu}

\renewcommand{\shortauthors}{He et al.}

%%
%% The abstract is a short summary of the work to be presented in the
%% article.
Large language model (LLM)-based agents have shown promise in tackling complex tasks by interacting dynamically with the environment. 
Existing work primarily focuses on behavior cloning from expert demonstrations and preference learning through exploratory trajectory sampling. However, these methods often struggle in long-horizon tasks, where suboptimal actions accumulate step by step, causing agents to deviate from correct task trajectories.
To address this, we highlight the importance of \textit{timely calibration} and the need to automatically construct calibration trajectories for training agents. We propose \textbf{S}tep-Level \textbf{T}raj\textbf{e}ctory \textbf{Ca}libration (\textbf{\model}), a novel framework for LLM agent learning. 
Specifically, \model identifies suboptimal actions through a step-level reward comparison during exploration. It constructs calibrated trajectories using LLM-driven reflection, enabling agents to learn from improved decision-making processes. These calibrated trajectories, together with successful trajectory data, are utilized for reinforced training.
Extensive experiments demonstrate that \model significantly outperforms existing methods. Further analysis highlights that step-level calibration enables agents to complete tasks with greater robustness. 
Our code and data are available at \url{https://github.com/WangHanLinHenry/STeCa}.

%%
%% The code below is generated by the tool at http://dl.acm.org/ccs.cfm.
%% Please copy and paste the code instead of the example below.
%%


%%
%% This command processes the author and affiliation and title
%% information and builds the first part of the formatted document.
\maketitle

\section{Introduction}\label{sec:intro}
\section{Introduction}

Despite the remarkable capabilities of large language models (LLMs)~\cite{DBLP:conf/emnlp/QinZ0CYY23,DBLP:journals/corr/abs-2307-09288}, they often inevitably exhibit hallucinations due to incorrect or outdated knowledge embedded in their parameters~\cite{DBLP:journals/corr/abs-2309-01219, DBLP:journals/corr/abs-2302-12813, DBLP:journals/csur/JiLFYSXIBMF23}.
Given the significant time and expense required to retrain LLMs, there has been growing interest in \emph{model editing} (a.k.a., \emph{knowledge editing})~\cite{DBLP:conf/iclr/SinitsinPPPB20, DBLP:journals/corr/abs-2012-00363, DBLP:conf/acl/DaiDHSCW22, DBLP:conf/icml/MitchellLBMF22, DBLP:conf/nips/MengBAB22, DBLP:conf/iclr/MengSABB23, DBLP:conf/emnlp/YaoWT0LDC023, DBLP:conf/emnlp/ZhongWMPC23, DBLP:conf/icml/MaL0G24, DBLP:journals/corr/abs-2401-04700}, 
which aims to update the knowledge of LLMs cost-effectively.
Some existing methods of model editing achieve this by modifying model parameters, which can be generally divided into two categories~\cite{DBLP:journals/corr/abs-2308-07269, DBLP:conf/emnlp/YaoWT0LDC023}.
Specifically, one type is based on \emph{Meta-Learning}~\cite{DBLP:conf/emnlp/CaoAT21, DBLP:conf/acl/DaiDHSCW22}, while the other is based on \emph{Locate-then-Edit}~\cite{DBLP:conf/acl/DaiDHSCW22, DBLP:conf/nips/MengBAB22, DBLP:conf/iclr/MengSABB23}. This paper primarily focuses on the latter.

\begin{figure}[t]
  \centering
  \includegraphics[width=0.48\textwidth]{figures/demonstration.pdf}
  \vspace{-4mm}
  \caption{(a) Comparison of regular model editing and EAC. EAC compresses the editing information into the dimensions where the editing anchors are located. Here, we utilize the gradients generated during training and the magnitude of the updated knowledge vector to identify anchors. (b) Comparison of general downstream task performance before editing, after regular editing, and after constrained editing by EAC.}
  \vspace{-3mm}
  \label{demo}
\end{figure}

\emph{Sequential} model editing~\cite{DBLP:conf/emnlp/YaoWT0LDC023} can expedite the continual learning of LLMs where a series of consecutive edits are conducted.
This is very important in real-world scenarios because new knowledge continually appears, requiring the model to retain previous knowledge while conducting new edits. 
Some studies have experimentally revealed that in sequential editing, existing methods lead to a decrease in the general abilities of the model across downstream tasks~\cite{DBLP:journals/corr/abs-2401-04700, DBLP:conf/acl/GuptaRA24, DBLP:conf/acl/Yang0MLYC24, DBLP:conf/acl/HuC00024}. 
Besides, \citet{ma2024perturbation} have performed a theoretical analysis to elucidate the bottleneck of the general abilities during sequential editing.
However, previous work has not introduced an effective method that maintains editing performance while preserving general abilities in sequential editing.
This impacts model scalability and presents major challenges for continuous learning in LLMs.

In this paper, a statistical analysis is first conducted to help understand how the model is affected during sequential editing using two popular editing methods, including ROME~\cite{DBLP:conf/nips/MengBAB22} and MEMIT~\cite{DBLP:conf/iclr/MengSABB23}.
Matrix norms, particularly the L1 norm, have been shown to be effective indicators of matrix properties such as sparsity, stability, and conditioning, as evidenced by several theoretical works~\cite{kahan2013tutorial}. In our analysis of matrix norms, we observe significant deviations in the parameter matrix after sequential editing.
Besides, the semantic differences between the facts before and after editing are also visualized, and we find that the differences become larger as the deviation of the parameter matrix after editing increases.
Therefore, we assume that each edit during sequential editing not only updates the editing fact as expected but also unintentionally introduces non-trivial noise that can cause the edited model to deviate from its original semantics space.
Furthermore, the accumulation of non-trivial noise can amplify the negative impact on the general abilities of LLMs.

Inspired by these findings, a framework termed \textbf{E}diting \textbf{A}nchor \textbf{C}ompression (EAC) is proposed to constrain the deviation of the parameter matrix during sequential editing by reducing the norm of the update matrix at each step. 
As shown in Figure~\ref{demo}, EAC first selects a subset of dimension with a high product of gradient and magnitude values, namely editing anchors, that are considered crucial for encoding the new relation through a weighted gradient saliency map.
Retraining is then performed on the dimensions where these important editing anchors are located, effectively compressing the editing information.
By compressing information only in certain dimensions and leaving other dimensions unmodified, the deviation of the parameter matrix after editing is constrained. 
To further regulate changes in the L1 norm of the edited matrix to constrain the deviation, we incorporate a scored elastic net ~\cite{zou2005regularization} into the retraining process, optimizing the previously selected editing anchors.

To validate the effectiveness of the proposed EAC, experiments of applying EAC to \textbf{two popular editing methods} including ROME and MEMIT are conducted.
In addition, \textbf{three LLMs of varying sizes} including GPT2-XL~\cite{radford2019language}, LLaMA-3 (8B)~\cite{llama3} and LLaMA-2 (13B)~\cite{DBLP:journals/corr/abs-2307-09288} and \textbf{four representative tasks} including 
natural language inference~\cite{DBLP:conf/mlcw/DaganGM05}, 
summarization~\cite{gliwa-etal-2019-samsum},
open-domain question-answering~\cite{DBLP:journals/tacl/KwiatkowskiPRCP19},  
and sentiment analysis~\cite{DBLP:conf/emnlp/SocherPWCMNP13} are selected to extensively demonstrate the impact of model editing on the general abilities of LLMs. 
Experimental results demonstrate that in sequential editing, EAC can effectively preserve over 70\% of the general abilities of the model across downstream tasks and better retain the edited knowledge.

In summary, our contributions to this paper are three-fold:
(1) This paper statistically elucidates how deviations in the parameter matrix after editing are responsible for the decreased general abilities of the model across downstream tasks after sequential editing.
(2) A framework termed EAC is proposed, which ultimately aims to constrain the deviation of the parameter matrix after editing by compressing the editing information into editing anchors. 
(3) It is discovered that on models like GPT2-XL and LLaMA-3 (8B), EAC significantly preserves over 70\% of the general abilities across downstream tasks and retains the edited knowledge better.

\section{Related Work}


\section{Related Work}
Our work draws on and contributes to research in mobility aids and the built environment, online image-based survey for urban assessment, personalized routing applications and accessibility maps.

\subsection{Mobility Aids and the Built Environment}
People who use mobility aids (\textit{e.g.,} canes, walkers, mobility scooters, manual wheelchairs and motorized wheelchairs) face significant challenges navigating their communities.
Studies have repeatedly found that sidewalk conditions can significantly impede mobility among these users~\cite{bigonnesse_role_2018,fomiatti_experience_2014,f_bromley_city_2007,rosenberg_outdoor_2013, harris_physical_2015,korotchenko_power_2014}. 
In a review of the physical environment's role in mobility, \citet{bigonnesse_role_2018} summarized factors affecting mobility aid users, including uneven or narrow sidewalks (\textit{e.g.,}~\cite{fomiatti_experience_2014,f_bromley_city_2007}), rough pavements (\textit{e.g.,}~\cite{fomiatti_experience_2014,f_bromley_city_2007}), absent or poorly designed curb ramps (\textit{e.g.,}~\cite{rosenberg_outdoor_2013, f_bromley_city_2007, korotchenko_power_2014}), lack of crosswalks (\textit{e.g.,}~\cite{harris_physical_2015}), and various temporary obstacles (\textit{e.g.,}~\cite{harris_physical_2015}).

Though most research on mobility disability and the built environment has focused on wheelchair users~\cite{bigonnesse_role_2018}, mobility challenges are not experienced uniformly across different user populations~\cite{prescott_factors_2020, bigonnesse_role_2018}. 
For example, crutch users could overcome a specific physical barrier (such as two stairs down to a street), whereas motorized wheelchair users could not (without a ramp)~\cite{bigonnesse_role_2018}. 
Such variability demonstrates how person-environment interaction can differ based on mobility aids and environmental factors~\cite{sakakibara_rasch_2018,smith_review_2016}.
Further, mobility aids such as canes, crutches, or walkers are more commonly used than wheelchairs in the U.S.~\cite{taylor_americans_2014, firestine_travel_2024}: in 2022, approximately 4.7 million adults used a cane, crutches, or a walker, compared to 1.7 million who used a wheelchair~\cite{firestine_travel_2024}.
This underscores the importance of considering a diverse range of mobility aid users in urban accessibility research.
For example, \citet{prescott_factors_2020} explored the daily path areas of users of manual wheelchairs, motorized wheelchairs, scooters, walkers, canes, and crutches and found that the type of mobility device had a strong association with users' daily path area size.
Our study aims to further advance knowledge of how different mobility aid users perceive sidewalk barriers, with a more inclusive understanding of urban accessibility.

\begin{figure*}
    \centering
    \includegraphics[width=1\linewidth]{figures/figure-tutorial.png}
    \caption{Survey Part 2.1 showed all 52 images and asked participants to rate their passability based on their lived experience and use of their mobility aid. Above is the interactive tutorial we showed at the beginning of this part.}
    \Description{This figure shows a screenshot from the online survey. In survey part 2.1, participants were presented with 52 images and were asked to rate their passibility based on their lived experience and use of their mobility aid. The screenshot shows the interactive tutorial shown before this section.}
    \label{fig:survey-part2-instructions}
\end{figure*}

\subsection{Online Image-Based Survey for Urban Assessment}
Sidewalk barriers hinder individuals with mobility impairments not just by preventing particular travel paths but also by reducing confidence in self-navigating and decreasing one's willingness to travel to areas that might be physically challenging or unsafe~\cite{vasudevan_exploration_2016,clarke_mobility_2008}.
Prior work in this area traditionally uses three main study methods: in-person interviews (\textit{e.g}.~\cite{rosenberg_outdoor_2013,castrodale_mobilizing_2018}), GPS-based activity studies (\textit{e.g.,}~\cite{prescott_exploration_2021, prescott_factors_2020,rosenberg_outdoor_2013}), and online-questionnaires (\textit{e.g.,}~\cite{carlson_wheelchair_2002}). 
In-person interviews, while providing detailed and nuanced information, are limited by small sample sizes~\cite{rosenberg_outdoor_2013}. GPS-based activity studies involve tracking mobility aids user activity over a period of time, offering insights into movement patterns and activity space; however, these studies are constrained by geographical location~\cite{prescott_exploration_2021}. In contrast, online questionnaires can reach much larger populations and cover broader geographical regions, but they often yield high-level information that lacks the depth and nuance of the other approaches~\cite{carlson_wheelchair_2002}.
Our study aims to strike a balance between these approaches, capturing nuanced perspectives of mobility aid users about the built environment while maintaining a sufficiently large enough sample size for robust statistical analysis. 
Building on~\citet{bigonnesse_role_2018}'s work, we explore not only the types of factors considered to be barriers, but the \textit{intensity} of these barriers and their differential impacts.

Visual assessment of environmental features has long been employed by researchers across diverse fields, including human well-being~\cite{humpel_environmental_2002}, ecosystem sustainability~\cite{gobster_shared_2007}, and public policy~\cite{dobbie_public_2013}. 
These studies examine the relationship between images and the reactions they provoke in respondents or compare differences in reactions between groups.
Over the past decade, online visual preference surveys have gained popularity (\textit{e.g.,}~\cite{evans-cowley_streetseen_2014, salesses_collaborative_2013, goodspeed_research_2017}), where respondents are asked to make pairwise comparisons between randomly selected images.
Using this approach has two advantages: it adheres to the law of comparative judgment~\cite{thurstone_law_2017} by allowing respondents to make direct comparisons, and it prevents inter-rater inconsistency possible with scale ratings~\cite{goodspeed_research_2017}.
Additionally, online surveys generally offer advantages of increased sample sizes, reduced costs, and greater flexibility~\cite{wherrett_issues_1999}.
For people with disabilities, online surveys can be particularly beneficial. They help reach hidden or difficult-to-access populations~\cite{cook_challenges_2007,wright_researching_2005} and are believed to encourage more honest answers to sensitive questions~\cite{eckhardt_research_2007} by providing a higher level of anonymity and confidentiality~\cite{cook_challenges_2007, wright_researching_2005}.

\begin{figure*}
    \centering
    \includegraphics[width=1\linewidth]{figures/figure-comaprison-screenshot.png}
    \caption{In survey Part 2.2, participants were asked to perform a series of pairwise comparisons based on their 2.1 responses.}
    \Description{This figure shows a screenshot from the online survey. In Survey Part 2.2, participants were asked to perform a series of pairwise comparisons based on their 2.1 responses.}
    \label{fig:survey-part2b-pairwise}
\end{figure*}

\subsection{Personalized Routing Applications and Accessibility Maps}
Navigation challenges faced by mobility aid users can be mitigated through the provision of routes and directions that guide them to destinations safely, accurately, and efficiently~\cite{kasemsuppakorn_understanding_2015}. However, current commercial routing applications (\textit{e.g.}, \textit{Google Maps}) do not provide sufficient guidance for mobility aid users.
To address this gap, significant research has focused on routing systems for this population over the past two decades~\cite{barczyszyn_collaborative_2018, karimanzira_application_2006, matthews_modelling_2003, kasemsuppakorn_understanding_2015, volkel_routecheckr_2008, holone_people_2008, wheeler_personalized_2020, gharebaghi_user-specific_2021, ding_design_2007}.
One early, well-known prototype system is \textit{MAGUS}~\cite{matthews_modelling_2003}, which computes optimal routes for wheelchair users based on shortest distance, minimum barriers, fewest slopes, and limits on road crossings and challenging surfaces.
\textit{U-Access}~\cite{sobek_u-access_2006} provides the shortest route for people with three accessibility levels: unaided mobility, aided mobility (using crutch, cane, or walker), and wheelchair users.
However, U-Access only considers distance and ignores other
important factors for mobility aid users~\cite{barczyszyn_collaborative_2018}.
A series of projects by Kasemsuppakorn \textit{et al}.~\cite{kasemsuppakorn_personalised_2009, kasemsuppakorn_understanding_2015} attempted to create personalized routes for wheelchair users using fuzzy logic and \textit{Analytic Hierarchy Process} (AHP).

While influential, many personalized routing prototypes face limited adoption due to a scarcity of accessibility data for the built environment. 
Geo-crowdsourcing~\cite{karimi_personalized_2014}, a.k.a. volunteered geographic information (VGI)~\cite{goodchild_citizens_2007}, has emerged as an effective solution~\cite{karimi_personalized_2014, wheeler_personalized_2020}.
In this approach, users annotate maps with specific criteria or share personal experiences of locations, typically using web applications based on Google Maps or \textit{OpenStreetMap} (OSM)~\cite{karimi_personalized_2014}.
Examples include \textit{Wheelmap}~\cite{mobasheri_wheelmap_2017}, \textit{CAP4Access}~\cite{cap4access_cap4access_2014}, \textit{AXS Map}~\cite{axs_map_axs_2012}, and \textit{Project Sidewalk}~\cite{saha_project_2019}.
Recent research demonstrated the potential of using crowdsourced geodata for personalized routing~\cite{goldberg_interactive_2016, bolten_accessmap_2019,menkens_easywheel_2011, neis_measuring_2015}.
For example, \textit{EasyWheel}~\cite{menkens_easywheel_2011}, a mobile social navigation system based on OSM, provides wheelchair users with optimized routing, accessibility information for points of interest, and a social community for reporting barriers. 
\textit{AccessMap}~\cite{bolten_accessmap_2019} offers routing information tailored to users of canes, manual wheelchairs, or powered wheelchairs, calculating routes based on OSM data that includes slope, curbs, stairs and landmarks. 
Our work builds on the above by gathering perceptions of sidewalk obstacles from different mobility aid users to create generalizable profiles based on mobility aid type. We envision that these profiles can provide starting points in tools like Google Maps for personalized routing but can be further customized by the end user to specify additional needs (\textit{e.g.}, ability to navigate hills, \textit{etc.})

Beyond routing applications, our study data can contribute to modeling and visualizing higher-level abstractions of accessibility. 
Similar to \textit{AccessScore}~\cite{li_interactively_2018}, data from our survey can provide personalizable and interactive visual analytics of city-wide accessibility. By identifying both differences between mobility groups and common barriers within groups, we can develop analytical tools to prioritize barriers and assess the impact of their mitigation or removal, potentially benefiting the broadest range of mobility group users. Incorporating perceptions of passibility into urban planning processes provides a new dimension for urban planners' toolkits, which are often narrowly focused on compliance with ADA standards.





\section{\system: A Spreadsheet-Based End-User Prompt Engineering Tool}


%\kenneth{Note that Figure 1 is a bit simplified, e.g., label verification, keep items for next iteration, is dismissed to make sure clear communication.}
%The \system is a Google Spreadsheet add-on with a user interface for guiding LLMs by providing rules or exemplary data to improve LLM performance. The system could be adapted to various single-class data annotation tasks. 
%Users can easily operate the system with minimal learning time and without the need for complex environment setup, unlike other applications.~\steven{find some other annotation systems}
In this paper, we present \system, a Google Sheets add-on that allows users to load a dataset into a Dataset spreadsheet, manually compose each part of a prompt within Task Context, Labeling Rules, and Shots sheets, and use the composed prompt to instruct an LLM to annotate data---all within the same Google Sheets document.
Motivated by the need to enable general users to prompt LLMs without installing and configuring professional tools like integrated development environments (IDEs) or Jupyter Notebooks, we decided to build a tool based on spreadsheets, which most computer users are already familiar with.
This section overviews \system's design and workflows.
%\kenneth{TODO: Maybe cite other tools and name what setup or configs they require}

\subsection{Design Goals}\label{sec:3-system-design}
%\kenneth{(1) Support PITD so we don't have gold label and don't calculate accuracy etc becasyue we think they're not reliable and evovling, and (2) using spreadsheet because it's just very easy to use and practically provide many flexibilities to users.}

%Prompting in the Dark: 
\paragraph{Adapting to Evolving Labeling Goals}
%\paragraph{Prompting in the Dark and Users' Needs for Evolving, Open-Ended Labeling Schemes.}
The goal of \system is to enable general users to create and refine prompts iteratively for LLM-powered data labeling, particularly in situations where they start without any labeled gold data or manual labeling, \ie, the ``prompting in the dark'' scenario. 
In these cases, users' understanding of the data and desired labeling scheme evolves through their interactions with the LLM, based on its predicted labels and explanations, rather than through their own manual efforts. 
The lack of gold labels (or sufficient labeled data) introduces the core challenge of the prompt-in-the-dark process: aside from users' observations and judgments about the labeling results, there is no concrete way to provide quick and comprehensive feedback on the progress of prompting.
We view this as a trade-off between two user needs in prompt engineering: 
{\em (i)} allowing users' understanding of the data and labeling goals to evolve, and 
{\em (ii)} providing clear guidance and reliable feedback to assess progress toward a defined annotation goal. 
Previously, supervised learning-based classifiers required labeled data, so users focused heavily on the second need, as manual labeling was always needed and assumed to finalize the coding scheme. 
The rise of LLMs has reduced the need for pre-labeled data, allowing users to put more focus on their first needs. 
The growing popularity of the prompting-in-the-dark approach reflects users' need for evolving and dynamic labeling practices~\cite{austin2024grad,zhang-etal-2024-glape,wang2024human}.
%zhang2023labelvizier to facilitate the validation and relabeling of large-scale technical text annotations. Its interactive, visual analytic interface allows users to detect and correct three main types of labeling errors: duplicate, wrong, and missing labels. 
%\kenneth{TODO: Do we have any reference to support it's a popular practice now?}\steven{those are three study that have no gold label, they iteratively use user-defined criteria to evaluate and refine.}\kenneth{Oh and did they manage to improve the accuracy over time??}\steven{Yeah, they got a higher rating and accuracy}\kenneth{Hmmmmmm so what's the deal? How are their systems and approaches different?} %, rather than simply saving time on manual labeling. 
% dudley2018review describe the interative machine leanring paradigm that user iterative build and refine model. The model refinement is driven by user input. It more focused on the human input to refine. kim2024evallm is a prompt refining tool by evaluating outputs on user-defined criteria. It is not a labeling task
Our design goal for \system is to offer users greater flexibility and freedom in defining how their data should be labeled.


%\kenneth{I revised this following paragraph to tailor it more close to our target users. Please take a look.}
\paragraph{Supporting a Wide Range of ``Newcomers'' Brought in By LLMs.}
%\paragraph{End-User Prompting Tools.}
%Our second design goal for \system is to create a tool for general users, including people with limited or no programming skills.
Our second design goal for \system is to develop a tool for users with \textbf{little to no experience in large-scale text data labeling}, including but not limited to those with limited or no programming skills.
The rationale behind this goal is two-fold.
On a practical level, 
people new to large-scale data annotation---empowered by LLMs to undertake such tasks with greater ease---are more likely to adopt approaches that diverge from conventional practices. 
In the crowdsourcing literature, many papers emphasize best practices for data annotation~\cite{hsueh2009data, sabou2014corpus, vondrick2013efficiently, drutsa2019practice, wang2013perspectives}, %\kenneth{TODO: Add ref on crowdsourcing best practices}\steven{added}
such as ethical pay rates~\cite{fort2011amazon,shmueli2021beyond}, %\kenneth{TODO: Add refs on crowd workers' ethical pay}\steven{added}
usable worker interfaces~\cite{toomim2011utility,10.1145/3613904.3642834, rahmanian2014user, komarov2013crowdsourcing}, %\kenneth{TODO: Add refs to crowd interface's impact on crowdsourced data quality--- Maybe cite our own CHI paper too}\steven{added}
and 
gold labels for quality control~\cite{han2020crowd,gadiraju2015training,le2010ensuring,doroudi2016toward,hettiachchi2021challenge}. %\kenneth{TODO: Again, add ref IN CROWDSOURCING for gold labls}\steven{added}
However, these practices are often neglected in real-world scenarios. 
For instance, many tasks on MTurk still offer very low pay~\cite{AI_workers_low_wages} %\kenneth{Add ref: A data-driven analysis of workers' earnings on Amazon Mechanical Turk}\steven{added}
or rely on poorly designed interfaces~\cite{fowler2023frustration}. %\kenneth{Add ref: Frustration and ennui among Amazon MTurk workers}\steven{added}
Newcomers to large-scale data annotation are even less likely to be familiar with these best practices, including carefully establishing gold labels before prompting LLMs.
%users without programming experience are more likely to prompt in the dark, struggling to interact effectively with LLMs. 
%In contrast, those with software engineering backgrounds are familiar with using established frameworks and tools, where automatic testing---such as unit and integration testing---is standard. 
The bigger picture is that LLMs are adding many ``new members'' to the world of programming and data science.
%---people with little or no coding experience. 
This group brings new practices, user needs, challenges, and research questions to HCI, requiring more focused attention.

%\bigskip
\sloppy
Based on these two design goals, we decided to build \system based on spreadsheets, a format that most computer users are already familiar with.
We distinguish our goals from existing efforts in two significant ways. 
First, while projects like LangChain or ChainForge focus on developers or those with programming backgrounds, requiring software installations or configurations, we aim to focus on general users who do not necessarily have such expertise. 
Second, some projects explore new interactions enabled by LLMs~\cite{10.1145/3586183.3606833}, but our project is concerned with understanding how effectively users can use familiar interfaces, such as spreadsheets, to interact with LLMs.

%--------------- dead kitten ----------

\begin{comment}
  


At the practical level, users with little or no programming background might be more likely to prompt in the dark. 
In contrast, individuals familiar with software engineering practices are accustomed to using existing frameworks or tools, where automatic testing---such as unit and integration testing---is standard.
These users have more experience as well as technological support for creating gold labels for testing.
At a deeper level, what is happening is that LLMs bring many people with limited or no programming skills into the realm of data science or programming tasks.
This group of people brings new and interesting challenges and research questions to HCI and thus deserves more attention.

%we believe that the greatest value of LLMs lies in the new possibilities they offer to general users. 
%For people without coding skills, LLMs enable tasks such as building websites from scratch, creating classifiers for automating email filtering, and labeling data to extract insights---activities that were previously within the realm of programmers. 

  
\end{comment}

\begin{figure*}[t]
    \centering
    \includegraphics[width=0.9\linewidth]{Figures/ui-overview.jpg}\Description{This is the user interface layout and pre-defined spreadsheet tab explanation. Each predefined sheet has a set of predefined columns. PromptSheet allows users to load a dataset into the Dataset tab, which is the starting tab of the system. By clicking each tab at the bottom of the Google Sheet, users can navigate to Task Context tab, Rule Book tab, Shots tab, Working Data Sample tab, task dashboard tab and task results tabs. The task result tabs will be generated after each new annotation round and store all new annotated results.}
    \caption{The user interface and all the predefined sheets of \system, where each sheet has a set of pre-defined columns. 
    \system allows users to load a dataset into a Dataset (A) sheet, manually compose each part of a prompt within Task Context (B), Labeling Rules (C), and Shots (D) sheets, and use the composed prompt to instruct an LLM to annotate data and store the labeling results in a separate task sheet (G). All functions are presented as manuals and buttons within the sidebar on the right.}
    \label{fig:system-interface-ui-kenneth}
\end{figure*}

\subsection{User Interface and Pre-Defined Sheets}

\system is a Google Sheets add-on that enables users to load data, sample a subset for labeling, compose and edit prompts, use these prompts to request LLMs for data labeling, and iteratively revise the prompts. 
Figure~\ref{fig:system-interface-ui-kenneth} shows the interface of \system.

\paragraph{Sidebar.}
Following Google Sheets' design constraints, all functions are presented as manuals and buttons within the sidebar on the right. 
The sidebar remains consistent across all sheets, regardless of which sheet is in use. 
At the top of the sidebar, \system provides a real-time notification that keeps users informed about its ongoing processes, such as ``Data Indexing,'' ``Data Sampling,'' ``Generating the Instructional Prompt,'' or ``Annotating.''


\paragraph{Pre-Defined Sheets.}
\system includes a set of predefined spreadsheets, each with a set of pre-defined columns. 
At the bottom of the interface, a series of tabs allows users to switch between sheets, with each sheet dedicated to a different part of the data labeling process. 
The following describes each sheet in detail. 
(To help readers easily identify which sheet we are referring to, we indexed each sheet as A, B, C, ..., and G in all the figures. 
These indexes were not present in the actual system to users.)

%\subsubsection{All the Sheets and What They Do}

%\hyperref[fig:system-interface-1]{Figure 1} overviews the \system's user interface. 
%The system consists of seven main components:

\begin{itemize}

\item \textbf{Dataset (Sheet A)}:
%The ``Dataset'' includes three columns: Data ID, Group ID, and Data Instance. 
%Each data instance is uniquely identified by a corresponding Data ID.
%A single Group ID may encompass one or multiple Data Instances.
This spreadsheet stores the full dataset.
Users can copy and paste the dataset into this sheet or use any supported Google Sheets import method (in Step 0).
The sheet includes three key predefined columns: (1) Data ID, (2) Group ID, and (3) Data Instance. 
Each data instance is uniquely indexed by its corresponding Data ID, which users can generate by clicking the ``Index Data ID'' function in the sidebar. 
The Group ID is used for annotating sequential data, such as when each sentence in an article is treated as a separate data instance, but all sentences from the same article share the same Group ID.
In our design, this sheet is intended to serve as a static data source, and we anticipate that users will not modify it after loading the data.

%\kenneth{TODO: Maybe add words to mention we don't expect people touch it after Step 0 and basicallyy serve as a database.}

\item \textbf{Task Context (Sheet B)}:
%The ``Context'' tab provides information to help describe the annotation task users are working on. It addresses questions related to the purpose and application of the data annotation task and the origin and size of each data instance. The LLM will use the information provided in this tab to generate an instructional prompt for a later step.
This spreadsheet stores the meta-information and context for the labeling task, which will later be incorporated into the prompt. 
The sheet includes predefined questions that characterize the task, such as the purpose of the data labeling, how the labels will be used, the source of the data, and the size of each data instance.
Table~\ref{tab:task-sheet-questions} in Appendix~\ref{sec:context-question-appendix} shows all the questions.
%\kenneth{TODO: Maybe add all the questions to Appendix.}
Users provide answers to these questions (in Step 1 or 4), and \system automatically incorporates both the questions and their answers into the prompt used for LLMs to label the data.


    
\item \textbf{Rule Book (Sheet C)}: 
%The ``Rule book'' tab is where users define the criteria and definitions for each label used during the annotation process.
This spreadsheet contains the labeling rules that the LLM will follow.
It includes two key predefined columns: (1) Label Name and (2) Rules for the Label. 
Users manually define the criteria and descriptions for each label in free text (in Step 1 or 4), detailing the guidelines for the annotation process. 
Multiple rules can be added for a single label, providing flexibility in defining the labeling criteria.


    
\item \textbf{Shots (Sheet D)}: 
%In the ``Shots'' tab, users can enter gold standard labels from their iterations or manually provide them as reference points.
This spreadsheet stores all the high-quality examples, including data instances and their corresponding labels, which will be included in the prompt to guide the LLM in labeling the data. 
These examples, commonly referred to as ``shots'' in prompts, follow the same predefined column structure, with an additional ``Gold-Standard Label'' column. 
Users can add these examples manually (in Step 1) or use \system's function to do so (in Step 4).


    
\item \textbf{Working Data Sample (Sheet E)}:
%Users can sample the data from the ``Dataset'' tab to the ``Working Data Sample'' tab. In the annotation process, \textit{only} data instance in the ``Working Data Sample'' tab will be annotated.
% by the LLMs using the instructional prompt, provided rules, and gold shots.
This spreadsheet stores the current subset of data selected from the full dataset, ready for the LLM to label.
Users can sample data from the Dataset sheet by clicking the corresponding buttons in the sidebar; users can choose between random sampling or selecting a specific range (Step 2). 
During the annotation process, only the data instances in the Working Data Sample sheet will be labeled. 
\system will copy the entire data sample from the Working Data Sample sheet to create a new sheet to label (Step 3).
    
    
\item \textbf{Task Dashboard (Sheet F)}:
%The ``Task Dashboard'' tab records all iteration task details such as task number, timestamp, used prompt, and total costs. 
This spreadsheet tracks all labeling tasks performed so far.
When the user clicks the ``Start Annotation'' button in the sidebar (in Step 3), \system creates a new sheet for the task (e.g., Task 1 sheet) and adds a new row in the Task Dashboard to record the labeling activities.
Task Dashboard sheet (Figure~\ref{fig:task-dashboard-new})
logs task details such as task number, timestamp, the prompt used, and total costs.

\item \textbf{Task 1 (Sheet G), Task 2, ..., Task N}:
%After each annotation, the annotation results will be saved in a new tab (e.g., Task\_1, Task\_2, etc) corresponding to that specific iteration. 
Each of these sheets stores the annotation results for each labeling request, including data samples, LLM-generated labels, and LLM explanations (optional).
These sheets also include columns that allow users to validate or correct the LLM labels and optionally add them to the Shots sheet (in Step 4).
When the user clicks the ``Start Annotation'' button in the sidebar (in Step 3), \system generates a new task sheet to handle the specific labeling task.

    
    
\end{itemize}

%\kenneth{Users are allowed to add new columns.}

Notably, while users must follow our guidelines for using the predefined columns in each sheet and inputting data correctly, they are free to add more columns or even additional sheets, just as they would in a regular Google Sheets document. 
For instance, when pasting a dataset into the Dataset sheet, it is common for the dataset to include its own IDs or additional information for each data entry. 
Users can easily store this extra information by creating new columns within the Dataset sheet.


%---------- dead kitten ----------

\begin{comment}



\subsubsection{Other Features} \steven{todo: add figures in the appendix. screenshots for different notification messages. interface screenshots for removal and clearing.}
\begin{itemize}
    \item \textbf{Real-Time System Notification: }\system provides a notification feature that informs users of its current processes, such as ``Data Indexing'', ``Data Sampling'', ``Generating the Instructional Prompt'', ``Annotating'', etc.
    \item \textbf{Remove Unselected Data Instance (Figure \ref{fig:remove-clear}): }This function will remove data instances that do not have the ``Keep it in the next data sample'' checked in the ``Working Data Sample'' tab.
    \item \textbf{Clear Data Instance (Figure \ref{fig:remove-clear}): }This function will clear all data instances in the ``Working Data Sample'' tab.
\end{itemize}

    
\end{comment}



\subsection{User Workflow}
\begin{figure*}[t]
    \centering
    \includegraphics[width=0.99\linewidth]{Figures/step-1.jpg}\Description{This is Step 1 described in Figure 1. Users can provide data annotation context in the Context Tab, provide their rule and definition in the Rule Book tab, and add gold standard labels in the Shots tab. These tabs will compose prompts for later GPT to use.}
    \caption{The overview of step 1 of the data labeling process, compose or refine the prompt.
This is the most critical step, where the user composes and refines prompts for the LLM to label the data. In \system, the prompt consists of three parts, each corresponding to a separate sheet: Context (B), Rule Book (C), and Shots (D). 
At the beginning of this labeling process, the user has only a vague idea of what they want to label and will continuously refine that idea. 
Each time the prompt is revised, it reflects an evolution of their understanding and approach to the labeling task.}
    \label{fig:step-1}
\end{figure*}

\begin{figure*}[t]
    \centering
    \includegraphics[width=0.99\linewidth]{Figures/step-2.jpg}\Description{This is Step 2 described in Figure 1. Users can either random or sequential sample data from the Dataset tab to the Working Data Sample tab. In the Working Data Sample, users can check “Keep it in the next data sample” for data instances that users want to remain in the Working Data Sample tab during sampling.}
    \caption{The overview of step 2 of the data labeling process, sample or resample a subset.
The full dataset is often too large for the user to thoroughly review, so sampling a subset is necessary.
%Labeling only a subset, rather than the entire dataset, is necessary because 
%Additionally, labeling the entire dataset iteratively would be prohibitively expensive. 
In this step, the user can (1) randomly or (2) sequentially sample data from the Dataset (A) sheet.
}
    \label{fig:system-interface-step-2}
\end{figure*}

\begin{figure*}[t]
    \centering
    \includegraphics[width=0.99\linewidth]{Figures/step-3.jpg}\Description{This is Step 3 and 4 described in Figure 1. After clicking Start Annotation, the results including LLM label and LLM explanation will be stored in a new tab. Users will review the data instances and LLM labels, by checking agree or providing their own labels. They also select the Gold Shot data instance to be added to Shots tab for later GPT to learn from. After verification, they can refine their prompt as Step 1 mentioned.}
    \caption{The overview of steps 3 and 4 of the data labeling process. After finalizing the prompt (Step 1) and sampling data instances (Step 2), in Step 3, the user clicks the ``Start Annotation'' button in the sidebar to annotate all instances in the Working Data Sample sheet. \system creates a new sheet, Task 1 (G), to store the data and labels of this labeling task, and also creates a new row for Task 1 in the Task Dashboard sheet. Then, in Step 4, the user can review the outcomes and refine the prompt accordingly (Step 1).
}
    \label{fig:system-interface-step-3-4-kenneth}
\end{figure*}

Users interact with \system to craft a prompt, use it to instruct the LLM in labeling data, review the results, and then revise the prompt through an iterative process. 
To demonstrate the users' workflow, we present a scenario where a user wants to employ \system to label a collection of tweets related to COVID with a 5-point sentiment scale, ranging from Very Negative (1) to Very Positive (5).
The goal is to analyze the sentiment of Twitter (now X) users toward COVID, with an emphasis on ensuring that the classification of each tweet reflects the user's own judgment.
In this case, the LLM's labels should align with the user's assessment of what is positive or negative, as well as the intensity of sentiment, rather than following an ``objective'' standard.
%In other words, the LLM's labels should align with the user's personal perception of the topic rather than adhering to an ``objective'' standard.\kenneth{This is not very accurate hmmm. Might need to edit later.}

%\begin{enumerate}
%    \item 
%\end{enumerate}

\begin{itemize}
   
\item 
\textbf{Step 0: Load and Index the Dataset.}
To begin using \system, the user opens a new Google Sheets document and activates the \system add-on. 
The system automatically sets up the necessary tabs, and the add-on interface appears on the right side of the spreadsheet (Figure~\ref{fig:system-interface-ui-kenneth}). 
The user then imports their data instances into the Dataset sheet, with the text of each tweet placed in the Data Instance column. 
The user must specify a Group ID for each instance. 
If the data are not sequential or grouped, they can assign unique Group IDs using Google Sheets' automatic numbering function.\footnote{\system is designed to accommodate single and grouped data instances within a Group ID. For tasks like sentiment analysis, each data instance is treated separately under its unique Group ID. For tasks that require contextual information, such as annotating text segments in an academic abstract (\eg, CODA-19~\cite{huang-etal-2020-coda}), \system can combine all data instances under the same Group ID into a single request to the LLM model. This flexibility allows the system to support different data instance formats based on user requirements.} 
Once the data is entered, the user clicks the ``Index Data ID'' button in the sidebar, and \system automatically assigns unique data IDs to each instance in the ``Data ID'' column.

\item 
\textbf{Step 1: Compose/Refine the Prompt (Figure~\ref{fig:step-1}).}
%Step 1: compose the promot using things. 
%Uses know a vague idea what they want and will keep revise that idea. But you need to write something down. When it comes to load data, spreadsheet is great!
This is the most critical step, where the user composes and refines prompts for the LLM to label the data. 
In \system, the prompt consists of three parts, each corresponding to a separate sheet: (1) Context, (2) Rule Book, and (3) Shots. 
Figure~\ref{fig:step-1} provides an overview of each sheet.
\begin{enumerate}

\item 
In the \textbf{Context} sheet, the user answers questions that describe the context of the data annotation task, such as the purpose of the annotation and the source of the data, to provide task-specific context for the LLM.

\item
In the \textbf{Rule Book} sheet, the user adds annotation labels along with their definitions. Providing content for both the Context and Rule Book sheets is mandatory, as the LLM requires this information in the prompt to function effectively.

\item
In the \textbf{Shots} sheet, the user adds data instances along with their corresponding gold labels, which serve as examples to help the LLM learn. While adding examples to the Shots sheet is optional during the first iteration---since the user may not yet have a well-defined gold standard for labeling---more examples can be identified as the user reviews data. These examples can be manually added or generated using \system's function (see Step 4).
\end{enumerate}
It is important to note that at the beginning of this labeling process, the user has only a vague idea of what they want to label and will continuously refine that idea. 
Each time the prompt is revised, it reflects an evolution of their understanding and approach to the labeling task.


\item 
\textbf{Step 2: Sample/Resample a Subset (Figure~\ref{fig:system-interface-step-2}).}
Next, the user employs \system to sample a subset of data for labeling. 
Labeling only a subset, rather than the entire dataset, is necessary because the full dataset is too large for the user to thoroughly review the LLM's results. 
Additionally, labeling the entire dataset iteratively would be prohibitively expensive. 
In this step, the user can (1) randomly or (2) sequentially sample data from the Dataset sheet into the Working Data Sample sheet:

\begin{itemize}

\item 
For a \textbf{Random Sample}, the user enters any whole number between 1 and the total number of group IDs in the dataset.
\system will then randomly select that number of groups and copy them into the Working Data Sample sheet. 

\item
In \textbf{Sequential Sample}, the user specifies a range of group IDs from the Dataset sheet, and \system will import the data instances from the selected range into the Working Data Sample sheet.
This feature allows users to process their data instances sequentially in batches, which is especially useful when the data instances have a sequential relationship, such as sentences within the same document.


%The purpose of this feature is to enable users to process their data instances sequentially in batches, making their work more manageable and easier to track.
%\kenneth{Mayeb say a few words on why we need this.}\steven{done.}

\end{itemize}

Once sampling begins, all previously existing data in the Working Data Sample sheet will be removed, except for instances marked as ``Keep it in the next data sample'' (Figure~\ref{fig:system-interface-step-2}). 
Only the data in the Working Data Sample sheet will be labeled by the LLM when the ``Start Annotation'' button is clicked in Step 3.

\item 
\textbf{Step 3: Use the Prompt to Instruct the LLM to Label the Data Sample (Figure~\ref{fig:system-interface-step-3-4-kenneth}).}
After finalizing the three prompt sheets---Context, Rule Book, and Shots---in Step 1 and sampling data instances in Step 2, the user clicks the ``Start Annotation'' button in the sidebar to annotate all instances in the Working Data Sample sheet (Figure~\ref{fig:system-interface-step-3-4-kenneth}).
\system creates a new sheet, Task 1 (Figure~\ref{fig:system-interface-step-3-4-kenneth}), to store the data and labels of this labeling task, and also creates a new row for Task 1 in the Task Dashboard sheet.

In the background, \system first combines the information in Context, Rule Book, and Shots sheets into a prompt (see Section~\ref{sec:implementation} for details).
%gathers the questions and answers from the Context sheet and feeds them into GPT-4 to generate an instruction prompt. 
%This prompt is then combined with the rules and provided gold shots to create the final annotation prompt. 
For each data group (\ie, data instances with the same Group ID), \system sends a request to the LLM using this prompt for annotation.
After receiving the LLM's output, the system parses the results and updates the Task 1 sheet with the annotated outcome for each instance. 
In our implementation, the LLM is always asked to provide explanations for its labels, though the user can decide whether to display these explanations in the annotation results.
%In our user study, we also explored the impact of showing the LLM's explanations to users.



%Step 3: Send it to LLM to label. System creat a new tab; you can navigate tasks using dashboard. Then you re

\item 
\textbf{Step 4: Observe, then Revise the Prompt (Figure~\ref{fig:system-interface-step-3-4-kenneth}).}
The labeling results are saved to the Task 1 sheet (Figure~\ref{fig:system-interface-step-3-4-kenneth}), where the user can manually verify the LLM's labels.
The user can review as many or as few data instances as they wish to develop a better understanding of the labeling task and the dataset. 
Based on this evolving understanding, they can refine the prompt by modifying the Context, Rule Book, and Shots sheets accordingly.

If the user disagrees with any of the labels, they can assign a new label to the data instance under the ``Human Label'' column. 
If the user identifies good examples, they can check the ``Gold Shot'' checkboxes. 
After selecting enough good examples, the user can click the ``Add Shots'' button in the sidebar to add these examples to the Shots sheet (Figure~\ref{fig:system-interface-step-3-4-kenneth}).
Like in other sheets, if the user wants certain data instances to be re-annotated in the next round, they can check the ``Keep it in the next data sample'' checkboxes. 
This will ensure that those instances are not removed during the next sampling process, allowing the user to observe whether the LLM's behavior changes over iterations.


\end{itemize}

When using \system, the user moves through Steps 1, 2, 3, and 4, and then returns to step 1 in an iterative process until they are satisfied with the LLM's labels.








%\subsubsection{Step 0: Initial Setups}


%\subsubsection{Step 1: Compose/Refine the Prompt}

%\subsubsection{Step 2: Sample/Resample a random subset}

%\subsubsection{Step 3: Use the Prompt to Instruct the LLM to Label the Data Sample}

%\subsubsection{Step 4: Observe, and Revise the Prompt}


\subsection{Implementation Details\label{sec:implementation}}
%\kenneth{TODO: Here we mention (1) what framework you used to implement Google Sheets add-on, (2) how do you convert Spreadsheet's content into a prompt, and (3) what LLM (which version exactly) you used and how did you send request (batch? or each data instance is one request?)--- Maybe talk about latency issue here a bit.}


\paragraph{Developing Google Sheets Add-On.}
%\kenneth{How do people built Google Sheets add-ons? Did we use an web server? Where do we store our data?}\steven{done}
\sloppy
\system utilized Google Sheets as its main platform, leveraging the convenience and functionality of its spreadsheet capability. The Google Sheets add-on was implemented in Google App Script, with Google Cloud Service serving as a back-end to store all action logging files. User-specific data, such as OpenAI information, was securely stored in user properties tied to individual email accounts, ensuring privacy protection. 

\paragraph{Converting a Spreadsheet's Content into a Prompt.}
Once users click on the ``Start Annotation'' button (Figure~\ref{fig:system-interface-step-3-4-kenneth}), \system will first collect all questions and answers from the ``Context'' tab and send a request to GPT-4o to generate an instructional prompt (Table~\ref{tab:instruction-prompt}). Next, \system will merge this generated prompt with rules and definitions from ``Rule Book'' and available gold standard labels from the ``Shots'' tab to create an annotation prompt (Table~\ref{tab:main-prompt} and Table~\ref{tab:main-multi-prompt}). Finally, \system will use this prompt to annotate all data instances. 

\paragraph{Interacting with the LLM through an API}
In this paper, we utilized OpenAI's \texttt{gpt-4o-2024-05-13} model for our study~\cite{openai2024gpt4o}.
%\kenneth{TODO: Add citation} \steven{done}
Technically, this LLM can be replaced by any other model that offers an API compatible with the ChatGPT-4 specification. 
In our implementation, we group all data instances with the same Group ID and send them in a single API request.
%In our current implementation, we did not batch requests; instead, we sent an individual API request for data instances with the same group ID.
%\kenneth{Is this accurate?}\steven{we sent by group ID}
%Future versions of \system could potentially benefit from batching to reduce latency.













%\subsection{System Design}


%\subsection{System Workflow (\hyperref[fig:system-workflow-fig]{Figure~\ref{fig:system-workflow-fig-v2}})}




% \begin{figure}
%     \centering
%     \includegraphics[width=0.85\linewidth]{Figures/Workflow/workflow-v2.jpeg}
%     \caption{System Workflow}
%     \label{fig:system-workflow-fig}
% \end{figure}












% \subsubsection{Main Procedure}
% The procedure consists of the following steps:
% \begin{itemize}
%     \item \textbf{Step 1 (Import Data): }Users can import their data instances with group ID into the `Dataset' tab and click ``Index Data ID'' to index all data instances.
%     \item \textbf{Step 2 (Answer Task Questions): }Users need to navigate to the `Context' tab to answer questions about the data annotation task they are working on.
%     \item \textbf{Step 3 (Define Labels and Rules): }Navigating to the `Rule Book' tab, users \textbf{have to} add annotation labels with corresponding definitions. 
%     \item \textbf{Step 4 (Add Gold Shots): }Users can add instances with their gold labels if applicable. 
%     \item \textbf{Step 5 (Sample Data): }Users can randomly or sequentially sample data into the `Working Data Sample' tab.
%     \item \textbf{Step 6 (Data Annotation): }After the ``Context'', ``Rule Book'', and ``Shot'' (if applicable) tabs are all settled, users can click on ``Start Annotation'' to annotate all instances sampled in the ``Working Data Sample'' tab. 
%     \item \textbf{Step 7 (Verification): }The annotation results will be saved to a new task tab. Users can start verifying LLM labels. Users who disagree with an LLM label can assign a new human-generated label to the data instance. If users find good examples that can be used for later LLM learning, they can check the ``Gold Shot'' checkboxes. They can also check the ``Keep it in the next data sample'' checkboxes if they want to re-annotate data instances.
%     \item \textbf{Iteration Procedure: }If users are not satisfied with the annotated results, they can modify the answer in the `Context' Tab \textbf{(Step 2)}; modify rules (add/adjust/delete labels or definitions) in the `Rule book' Tab \textbf{(Step 3)}; click ``Add Shots'' to add the selected ``Gold Shots'' to the `Shots' Tab \textbf{(Step 4)}; click ``Add Back'' to add instances back to the `Working Data Sample' Tab for re-annotating in the next iteration. Then, they can re-sample or re-use instances in the `Working Data Sample' tab for the next iteration \textbf{(Step 5)}. In the end, they repeat \textbf{Step 6}.
%     \item \textbf{Completion:} If users are satisfied with the annotation results, they may choose to conclude the task.
% \end{itemize}





% \paragraph{Two Data Instance Formats Handling}
% \system is designed to accommodate single and grouped data instances within a Group ID. For tasks like single-class sentiment analysis, each data instance is treated separately under its unique Group ID. However, for tasks that require contextual information, such as annotating text segments in an academic abstract (e.g., CODA-19~\cite{huang-etal-2020-coda}), \system can combine all data instances under the same Group ID into a single request to the LLM model. This flexibility allows the system to support different data instance formats based on user requirements.


\section{User Study\label{sec:user-study}}
%\section{Comparative Study Procedure}
Our goal is to investigate how effective people are at prompt engineering when gold labels are absent, namely, ``prompting in the dark''. 
To study this, we conducted an in-lab user study in which participants used \system to perform a 5-point sentiment labeling task on a tweet dataset. 
This section overviews the details of this study.
This study has been approved by the IRB office of the authors' institute. 


\subsection{Study Procedure}


%\subsection{In-lab Study}

%We conducted a 90-minute in-lab user study with participants using \system for an annotation task.

\subsubsection{Pilot Study\label{sec:pilot-study}}
Three participants were recruited through the authors' network for the pilot study. 
In the first pilot, we used CODA-19~\cite{huang-etal-2020-coda} as the data annotation task, where participants labeled text segments from academic abstracts into categories such as background, purpose, and findings.
We observed that the participant consistently agreed with nearly all the labels and did not suggest further refinements. 
This may have been due to the highly specialized nature of the abstracts, which made it difficult for a broader audience to fully understand and evaluate the labels. 
As a result, we decided to switch to a Twitter Sentiment task for the second pilot.
In this second pilot, we found that our guidelines were too flexible, leading to participant confusion and uncertainty about how to proceed. 
We made adjustments to provide more structure, such as requiring participants to complete at least four iterations, with each iteration involving the annotation of 10 out of 50 instances. 
After verification, participants were instructed to refine their rules and add gold standard labels.
Based on the results of the two pilot studies, we extended the study duration from 60 to 90 minutes to give participants enough time to learn the system and complete the tasks. Compensation was also adjusted to \$20. 
We tested these revised settings with the third participant and confirmed that the procedure worked effectively.
%Two participants were recruited via the authors' network for the pilot study. 
%\kenneth{TODO: say a few words about pilot study? what did we change after pilot study?}
%In the first pilot study, we used CODA-19~\cite{huang-etal-2020-coda} as the data annotation task, where participants annotated text segments from an academic abstract into background, purpose, method, finding, and others. We observed that the participant consistently agreed with almost all labels and did not suggest further refinement during the verification process. Additionally, while the LLM labels were different than the dataset gold labels, LLM labels and LLM explanations were logically sound. 
%Based on these findings, we decided to switch the annotation task to a Twitter Sentiment task for the second pilot. In this study, we found that our guidelines were too flexible, leading to participant confusion and uncertainty about how to proceed. To address that, we simplified and standardized the user study procedures. For example, we required them to do at least four iterations, with each iteration involving the annotation of 10 out of 50 instances. After verification, participants were instructed to refine their rules and add gold standard labels.

%Based on the results of two pilot studies, we extended the study duration from 60 minutes to 90 minutes to allow participants to have sufficient time to learn the system and complete the annotation tasks. The compensation was also adjusted to \$20.



\subsubsection{Participants Recruitment, Backgrounds, and Grouping}
%Recruitment
For our main study, we focused on recruiting individuals with reasonable familiarity with LLMs but relatively new to large-scale text data annotation. 
While \system is designed as an end-user prompting tool, in this study, we prioritized participants likely to represent the first wave of ``newcomers'' (as noted in our Design Goals in Section~\ref{sec:3-system-design}) entering LLM-powered data annotation. %\kenneth{TODO: Update the section ID} \steven{added}
This focus allowed us to avoid the need to teach participants the basics of LLMs, prompting, or text data annotation.
%For the main study, 
We recruited 20 participants from diverse educational backgrounds through the authors' networks, social media posts, and mailing lists within the authors' institute. 
The group included 1 Post-doctoral Researcher, 9 Ph.D. students, 9 Master's students, and 1 Undergraduate student.  
As part of the recruitment process, we specifically sought participants who met the criteria of possessing prior experience using LLMs. 
%\steven{added recruitment part}
%\steven{one participant was dropped because he did not attend the makeup session, should we mention that?}
Participants were compensated \$20 for their participation, and in our analysis, they are denoted as P1 to P20.
%\steven{Our system is designed for requesters or researchers who understand their task requirements. However, due to recruitment constraints, we could not involve participants with extensive data annotation experience. To address this, we chose a subjective task like Twitter sentiment analysis, where prior annotation expertise is less critical. This approach allows participants to rely on their own knowledge and judgment, simulating real-world scenarios where individuals guide LLMs in tasks that naturally depend on personal interpretation and expertise.}

%Backgrounds
%\kenneth{------------------KENNETH IS WORKING HERE---------------------------}

%\kenneth{------------------KENNETH IS WORKING HERE---------------------------}

%Grouping
Participants were randomly and evenly assigned to four groups based on two variables: 
% (1) whether or not they could view the LLM's explanations for its labels\footnote{\steven{Since two participants who had access to LLM explanations chose to turn them off, we grouped them with the no LLM explanation participants, resulting in 8 participants with access to LLM explanations and 12 participants without access.}}, and 
% (2) whether they had access to 50 instances per iteration or 10 instances per iteration.
(1) whether they had access to 50 instances per iteration or 10 instances per iteration, and
(2) whether or not they could view the LLM's explanations for its labels\footnote{Since one participant chose to disable the LLM explanations after the first iteration and another participant decided not to use the LLM explanations throughout the entire study, we grouped them with the no LLM explanation participants, resulting in 8 participants with access to LLM explanations and 12 participants without access.}.
Further details are provided in the study procedure section (Section~\ref{sec:study-procedure}). 

\paragraph{Survey on Participants' LLM Familiarity and Usage.}
To assess participants' familiarity with using LLMs, we conducted an optional post-study survey, offering an additional \$5 compensation for completion. 
(The full set of survey questions is provided in Table~\ref{tab:participants-llm-background-survey} in Appendix~\ref{sec:appendix=participant-background}.) %\kenneth{UPDATE REF}\steven{updated}
All participants responded. %\kenneth{UPDATE NUMBER}\steven{updated}
Most participants reported being familiar with LLMs, with an average familiarity score of 4.20 (SD=0.77) on a 5-point scale. %\kenneth{UPDATE NUMBER}\steven{updated}
15 participants had over one year of experience using LLMs, while 4 reported more than four months of experience, and 1 reported between one and three months. %\kenneth{UPDATE NUMBER} \steven{updated}
In terms of usage frequency, 16 participants used LLMs daily, 3 used them weekly, and 1 used them monthly. %\kenneth{UPDATE NUMBER} \steven{updated}
%Interaction durations varied: eight participants engaged with LLMs for more than 30 minutes, four for less than 5 minutes, three for 5-15 minutes, and four for 15-30 minutes. 
While most participants used LLMs for general tasks such as Q\&A, research, writing assistance, and programming/debugging, only 5 participants had experience using LLMs for data labeling. %\kenneth{UPDATE NUMBER}\steven{updated}
Participants rated their confidence in crafting prompts and their proficiency in interacting with LLMs similarly, with average scores of 3.75 (SD=0.85) and 3.85 (SD=0.88), respectively. %\kenneth{UPDATE NUMBER}\steven{updated}
%Many employed prompt engineering techniques, ranging from simple input adjustments to advanced methods like iterative refinement, system message editing, in-context learning, and chain-of-thought prompting to enhance LLM performance.
Overall, the participants represented individuals familiar with LLMs but relatively inexperienced with large-scale data annotation.



\subsubsection{Labeling Task, Scheme, and Data}
We selected the Coronavirus Tweet NLP Text Classification task, which categorizes tweets into five sentiment categories: Extremely Positive, Positive, Neutral, Negative, and Extremely Negative, using the dataset hosted on Kaggle.\footnote{Coronavirus tweets NLP - Text Classification: https://www.kaggle.com/datasets/datatattle/covid-19-nlp-text-classification/}
The dataset contains tweets from December 30, 2019, to September 7, 2020. 
For our study, we randomly sampled 1,060 tweets: 10 tweets were used for the tutorial task, 1,000 for the main study, and 50 for the final evaluation set (see Section~\ref{sec:study-procedure}).

This task was chosen, partially informed by our pilot study (Section~\ref{sec:pilot-study}), for several reasons. 
First, it strikes a balance in difficulty, being challenging enough to require iterative prompting efforts from LLMs, as a 5-class sentiment task is more complex than typical 2-class (positive, negative) or 3-class (positive, negative, neutral) sentiment classification tasks. 
Second, it avoids requiring specialized knowledge, ensuring a broad pool of potential participants. 
Tasks demanding domain-specific expertise would have significantly restricted recruitment; sentiment labeling for general COVID-related tweets is sufficiently accessible for this purpose. 
Finally, the task incorporates a subjective element, as it lacks universally agreed-upon gold labels. 
This aligns with our focus on ``prompting in the dark,'' where participants' understanding of the data, as well as labeling goals, evolve through iterations.
The subjective nature of the task allows participants to arrive at differing gold standards by the end of the process.
Considering these factors, we selected this task for our study.




%We chose this task during our pilot study (Section~\ref{sec:pilot-study}), primarily for its accessibility and its somewhat subjective nature: 
%sentiment analysis does not require specialized expertise; 
%the evaluation often depends on personal judgment.
%\kenneth{TODO Kenneth: This is not very accurate. need to revise later.}
%\steven{In contrast to tasks with strictly defined objectives, such as CODA-19~\cite{huang-etal-2020-coda}, which require extensive domain knowledge, subjective tasks like Twitter sentiment analysis do not have universally correct answers, relying instead on personal judgment for evaluation. 
%In this study, we encouraged participants to guide the LLM to align with their individual standards rather than steering it toward a fixed standard grounded in domain expertise}
%This \steven{task} allowed participants to guide the LLM according to their own interpretation of sentiment, particularly in deciding what qualifies as ``extremely'' positive or negative.





\subsubsection{Study Procedure\label{sec:study-procedure}}
For our main user study, most sessions were conducted remotely via Zoom or Microsoft Teams, with each session lasting between 87 and 127 minutes. 
Participants who attended in person used one of the author's laptops, while remote participants used their own computers. 
Since \system was a Google Add-on in a development version, it was installed on one of the author's laptops. 
Remote participants were given control of this laptop to conduct the experiment. 
Each session was recorded, capturing the screen, audio, and video for further analysis.

The study followed these steps:

\begin{enumerate}

\item 
\textbf{Onboarding:}
Participants were first introduced to the study's objectives and procedures, and their informed consent was obtained.

\item 
\textbf{Tutorial Task:}
Participants were then presented with a tutorial on the system's workflow and features, either delivered by one of the authors or via a prerecorded video, depending on their preference. 
Afterward, they completed a short tutorial task, identical to the main study task but involving only 10 data instances, to ensure their understanding of the system.

\item 
\textbf{Main Study:}
Participants were then asked to use \system to iteratively compose a prompt to label the sentiment of COVID-related tweets in alignment with their personal judgment of sentiment scores. 
Each participant was asked to complete at least four iterations (\ie, going through Steps 1 to 4 four or more times).
Participants were free to do additional iterations beyond the required four; on average, participants completed 4.75 iterations.
%\kenneth{update numbers}\steven{done}

Depending on their assigned group, participants used \system to annotate either 50 or 10 instances per iteration and then review the results.
For participants working with 50 instances per iteration, we advised that it was not necessary to manually verify all labels, as that would take too much time. 
% Participants in the group with access to LLM explanations could manually turn off the explanations if they felt that reading them was too time-consuming; only a few participants chose to do so.
Participants in the group with access to LLM explanations were explicitly informed that they
could manually turn on the explanations if they felt that they needed reasoning for each label. 
Most participants turned on the LLM explanations in the first iteration;
however, one participant chose to disable the LLM explanations after the first iteration and another participant decided not to use the LLM explanations throughout the entire study.
% , but a few participants chose to disable them after the first iteration. 
% One participant opted not to use the LLM explanations throughout the entire study.
%\kenneth{Is this true? How does this work?}\steven{the default is off, they can choose to turn on or keep it off.}\kenneth{hmmm how did we test the effect in such case then? Did they turn it on very often??? Did we always ask LLM to provide explanations?}\steven{All participants turned on the LLM explanation in the first iteration and use LLM explanation. and three participants decided to turn off the LLM explanation. }\kenneth{how about implementation? Did we always ask for LLMs to give us explanations in our prompt, just some participants do not have acees to it?}\steven{yes, the LLM explanation always in the raw output. Our LLM explanation checkbox is used to display or not display the LLM explanation. The label outputs remain consistent for all participants. }\steven{my bad, one participant did not turn on the LLM explanation the whole time.}

%After reviewing the LLM-generated labels, participants were encouraged to refine their Context, Rule Book, and Shots sheets if they gained new insights into the task, or to add gold shots.
%All participants were required to complete at least four iterations (i.e., going through Steps 1 to 4 four times). 

\item 
\textbf{Manual Labeling of the Gold Set:}
Upon completing the iterative process, participants manually labeled 50 tweets based on their own sentiment judgments. 
These labels reflected the participants' understanding of the data and the final labeling they aimed to achieve by the end of the study session. 
These manually labeled tweets were used as an evaluation dataset to assess the performance of the participants' prompts; they were not used to train or fine-tune any AI models.









%Upon completing the iterative process, participants manually labeled 50 tweets based on their own sentiment judgments. 
%These manually labeled tweets served as the evaluation dataset to assess the performance of their prompt.
%\kenneth{TODO Kenneth: (1) Say it captured at last understanding of users. (2) WE do not use it for any training or fine-tuning!!!}

\item 
\textbf{Post-Study Survey and Feedback Collection:}
At the end of the session, participants completed a questionnaire to rate the system's effectiveness, performance, and accessibility. They were also asked the following questions:
(1) Without this tool, how would you typically approach prompt engineering?
(2) How does your prompt engineering process compare before and after using this tool?
(3) Did the system help you complete the tasks more efficiently? If yes, please explain how.
(4) What features did you find most useful?
(5) Would you be interested in using this annotation system in your regular work or study? If no, please explain why.
(6) Do you have any suggestions for making the system more suitable for your needs?

Upon completing the questionnaire, we verbally asked participants to provide some last comments about the workflow, labeling task, and our system.



\end{enumerate}



%--------------- dead kitten -----------

\begin{comment}

\steven{
This setup allowed us to examine the effects of different sample distributions on annotation outcomes.
The choice of 10 instances reflects the limited time available to participants, allowing them to complete detailed annotations effectively.
In contrast, a previous HITL study without pre-set gold labels utilized 30 sample candidates per iteration for human selection~\cite{liu2019deep}. 
To balance this approach, the 50-instance group was introduced to explore a broader data distribution, allowing us to investigate the trade-offs between accessing essential and diverse samples in a human-in-the-loop system, as emphasized in ~\cite{wu2022survey, le2010ensuring}. Furthermore, considering that our sentiment task comprised 5 labels and involved random sampling, selecting 50 instances could significantly increase the chance of participants observing and engaging with all 5 labels.}


%https://docs.google.com/spreadsheets/d/1wcIDftEfVoDAAwXBnXoMgAhMeG-o-T-fB4CRuXlhy1g/edit?gid=0#gid=0
\subsubsection{Participants LLMs Background}\steven{participants background on going}
We distributed an additional questionnaire (Table~\ref{tab:participants-llm-background-survey}) to gather background information on LLMs from the participants who attended, offering a \$5 compensation. Out of 20 participants, 18 responded.\steven{fill in number later}

Most participants reported being familiar with using LLMs, with an average familiarity score of 4.28 (SD=0.75) on a 5-point scale. The majority displayed advanced knowledge of LLMs, while others showed an understanding of general principles. Only one participant reported having a basic level of understanding.
Near all participants had over one year of experience using LLMs, with only four reporting more than four months of experience and one reporting between one and three months of experience.
Almost all participants reported daily use of LLMs, except three reported weekly and one reported monthly.
Participants reported varying durations of interaction with the LLM. Eight participants interacted with the LLM for more than 30 minutes, four participants reported interactions lasting less than 5 minutes, three participants interacted for 5-15 minutes, and four participants reported interactions lasting 15-30 minutes.
Majority of participants indicated using LLMs for academic research, professional tasks, and personal projects. Two participants mentioned using LLMs for entertainment, while one reported using them for communication and another for translation and writing polishing. 
Participants employed LLMs for a wide range of tasks. 
Only four participants reported using LLMs for data labeling.
The majority engaged LLMs for general Q\&A, research, writing assistance, and programming or debugging. Half of the participants used LLMs for data analysis and visualization, while five employed them for creative tasks. Additionally, one participant sought conceptual explanations from LLMs. 
Participants rated their confidence in crafting prompts and their proficiency in interacting with LLMs to generate desired answers similarly, with an average score of 3.83 (SD=0.86) and 3.89 (SD=0.90) separately. 

Several participants use OpenAI's API for tasks such as dataset generation, research, virtual assistant creation, and integration into automation systems.
Many participants engage in prompt engineering, ranging from simple input adjustments to advanced techniques like iterative refinement, system message editing, in-context learning, and chain-of-thought (CoT) prompting for better LLM performance.
Participants use tools like OpenAI Playground, customized assistants, and external resources like Reddit for task-specific prompts.
Some participants either do not currently use GPT API but remain open to exploring them in the future.


While our participants did not have extensive data annotation experience, their familiarity with LLMs and general understanding of the Twitter Sentiment task requirements allowed them to engage meaningfully with the system. 
This aligns with our goal of simulating real-world scenarios where requesters -- whether researchers or practitioners -- guiding LLMs to align with their own standards.\steven{TODO: discussion their suitability}






\kenneth{--------------------------KENNETH IS WORKING HERE------------------------}


We finalized our experimental procedures after two user studies. 
The user evaluation dataset session was moved from the beginning to the end of the main experiment, as participants were likely to form a more consistent standard after reviewing numerous data instances
To preserve the integrity of the study, we kindly requested the first two participants (P1 and P2) to rejoin a 15-minute user study to re-annotate the evaluation dataset. After they completed the makeup session, we compensated them with \$5.\steven{we asked users for the makeup session}



\paragraph{Setup System Environment} During the system setup process, users should open the setting window by clicking the ``Setting'' icon located at the top right of the add-on interface. 
In our user study, we labeled the first top field as ``Participant ID'' to easily distinguish between participants. For real deployment, we will rename it to ``Data Annotation Task Name'' to allow users to track their iterative guidelines for LLMs in different tasks.
\steven {This participant ID was used during the user study, identifying each participant. I think we can change it to a data annotation name, which can be used to track different annotation tasks in the future. }\steven{I will add interface screenshot for both}
After users enter their participant ID/Annotation Task Name, they can click the ``Save Participant/Task Name'' button to save the information.
More importantly, users have to input their OpenAI API Keys and click the ``Save API'' button to save the key.





\paragraph{Pre-task interview} The steps of the study were as follows: We first introduced participants to the study objectives and procedure, securing their consent. Then, we presented the workflow and feature tutorials. Afterward, participants were asked to complete a short trial task to ensure their understanding of the system.


% The in-lab user study lasted approximately 90 minutes. The session consisted of a pre-task interview with a trial task (30 minutes), a main experiment (50 minutes), and a post-questionnaire (10 minutes). Both the trial task and main experiment used \textbf{COVID Twitter Sentiment Task} as the annotation task.

% In the pre-task interview, we presented an orientation to explain the purpose of the study. Then, each participant would listen to an informed consent form and provide verbal consent. After that, we provided participants with an instruction on our system and the annotation task. In the end, we asked participants to complete a short trial task to ensure their understanding of the system. 

\paragraph{Main experiment} Participants were instructed to complete at least four iterations. 
Participants used \system to annotate either 50 or 10 instances per iteration. For those who access 50 instances, we advised them to quickly glance at instances and labels during the verification process to gather as many insights as possible. 
For participants reviewing 10 instances per round, the number of instances was adjusted according to their verification speed due to time constraints. 
After the annotation process was completed, participants started to review each instance and LLM-generated labels. Those assigned to the group with access to LLM Explanation had the option to choose whether or not to display the LLM Explanation. 

Following the verification of LLM labels, participants were encouraged to refine their rule books if they discovered new insights into the task or to add gold shots if they identified a data instance and its corresponding label as a valuable reference point.
If time remained in the main experiment, participants were asked to perform additional refinement iterations. 

% During the refinement iteration, we observed that participants tended to review all the LLM explanations when provided with the options. This could potentially influence participants to align their judgments more closely with those of LLMs. 
% To mitigate potential bias, we split all participants randomly into two groups: one group worked with explanations, while the other worked without them. 

Upon completing all refinement processes, participants were asked to annotate 50 tweets based on their judgments for the sentiment task. These annotated tweets would serve as the evaluation dataset to assess the performance of LLM guided by the participants.

\paragraph{Post-questionnaire session} Participants completed a questionnaire, rating the effectiveness, performance, and accessibility of the system. They were also given questions: 
\textbf{(1) Without this tool, how would you typically approach prompt engineering?
(2) How would you compare your prompt engineering process before and after using this tool?
(3) Did the system help you complete the tasks more efficiently? If yes, please explain how.
(4) What features did you find most useful?
(5) Would you be interested in using this annotation system in your regular work or study? If no, please explain why.
(6) Do you have any suggestions for making the system more suitable for your needs?}
We recorded audio, captured screens, and recorded all system actions for each user study. 




% At the beginning of the user study, we structured our in-lab study session to last 60 minutes, focusing on using our system to refine prompts for the CODA-19 data annotation scheme~\cite{huang-etal-2020-coda}. The session consisted of a pre-task interview with a trial task (15 minutes), a main experiment (35 minutes), and a post-questionnaire (10 minutes). 

% After the first pilot study, we observed that the LLM performance on the CODA-19 was too good for the first participant to provide nuanced insights to improve prompts further. 
% Thus, we modified our annotation task to a Coronavirus tweet NLP Text Classification task~\footnote{https://www.kaggle.com/datasets/datatattle/covid-19-nlp-text-classification/}, which we will refer to as the COVID Twitter Sentiment Task in subsequent sections.
% However, after conducting the first two pilot studies, we found that the learning phrase on the system took more time than anticipated, and the main experiment annotation also required more time. To improve accessibility and track the refinement process, we extended our session to 90 minutes, increasing the pre-task part to 30 minutes and the main experiment to 50 minutes. 



% \subsection{Deployment Study}


\subsubsection{In-lab Session Procedure}
Most sessions were conducted remotely through Zoom or Microsoft Meetings. Each session typically spanned 87 minutes to 127 minutes. Participants who attended in person used one of the author's laptops, while those who joined remotely used their computers. Since the system was a Google Add-on development version, it was installed on one of the author's computers. Hence, we granted remote participants control of the author's laptop to experiment. 

\subsubsection{Annotation Scheme}
For our in-lab study, we chose to use a Coronavirus tweet NLP Text Classification task~\footnote{https://www.kaggle.com/datasets/datatattle/covid-19-nlp-text-classification/}, which categorized tweets into one of five categories, \textit{i.e.}, Extremely Positive, Positive, Neutral, Negative, and Extremely Negative. In subsequent sections, we will refer to this task as the \textbf{COVID Twitter Sentiment Task}.

This task was picked for its accessibility and flexible standards:
it did not require expertise and knowledge in analyzing tweet sentiments; evaluation of sentiment analysis often relied on personal judgment, allowing different participants to guide the LLM according to their personalized classification standards.

\subsubsection{Dataset}
The COVID Twitter Sentiment dataset contains tweets from December 30, 2019, to September 7, 2020 from Twitter. In this study, we randomly sampled 1,060 tweets from the dataset. We used 10 tweets for the trial task, 1,000 for the main experiment, and 50 for the evaluation set.


For the main study, the participants were recruited via authors' networks, social media posts, and mailing lists in the authors' institute. 
We recruited 20 participants from diverse educational backgrounds (1 Post-doctoral Researcher, 9 Ph.D. students, 9 Master's students, and 1 Undergraduate student). 
All participants had experience with using LLMs. 
Participants were compensated with \$20 for their participation. 
In our analysis presented in the paper, we denote participants as P1 to P20.

Participants were randomly and evenly assigned to whether accessing to LLM Explanation or not and were further assigned to whether access to 50 instances per round or not. 
\end{comment}

%\subsection{DSPy Fine-tuning}



%--------------- dead kitten ----------
\begin{comment}


\kenneth{--------------------------KENNETH IS WORKING HERE-----------------------------------}


We aimed to track accuracy trends in data annotation tasks throughout human prompt refinement processes. 
This section detailed the procedure. 
Previous study indicates that LLM explanations enhance understanding of the context~\cite{ma2023insightpilot,singh2024rethinking}. 
To explore this, we implemented two distinct conditions in our study: one group was given access to \textbf{LLM Explanation}, while the other group was not.
Due to time constraints, participants were only able to work with a limited number of data instances per iteration. We established two additional conditions: in one, participants worked on approximately 10 instances per iteration, allowing them to review each instance and its labels in detail. In the other, participants were presented with \textbf{50 instances} and instructed to quickly skim through them to gather insights.\steven{TODO: find papers. } \steven{two variables: LLM Explanation and 50 instances}

% and as individuals explore more data instances, they are able to uncover deeper insights within the data\steven{find papers}; 
% Thus, 

    
\end{comment}

\section{Findings}
% \section{Discussion}

% \begin{figure}
  \centering
  \includegraphics[width=\linewidth]{figures/per_frame_boxplot.png}
  
  \caption{\label{fig:frame-boxplot} Comparison of the distribution of F1 scores across all frames for each model.}
\end{figure}
% \subsection{Model Performance}

% \subsubsection{Out-of-Domain Performance}


% \begin{table}
    \centering
    \begin{tabularx}{\linewidth}{Xcccc}
        \hline
        \textbf{Model} & \textbf{All} & \textbf{Amb} \\ 
        \hline
        % Qwen 2.5-7B     & 0.755 & 0.665 & 0.707 & 0.547 \\ % no candidates @ fe
        % Qwen 2.5-7B     & 0.668 & 0.665 & 0.666 & 0.500 \\ % cand @ fe 
        % Phi-4           & 0.798 & 0.717 & 0.756 & 0.607 \\ % no candidates @ fe
        % Phi-4           & 0.719 & 0.717 & 0.718 & 0.560 \\ % cand @ fe
        % Qwen 2.5-7B     & 91.76 & 90.95 \\ % cand @ fe 
        Phi-4                           & 0.375 & 0.262 \\ % Not finetuned
        % $\text{Phi-4}_{cand}$ w/o LF    & 0.927 & 0.918 \\ % Finetuned on candidates
        $\text{Phi-4}_{cand}$ w/o LF    & 0.882 & \textbf{0.862} \\ % Finetuned on candidates
        $\text{Phi-4}_{cand}$ w/ LF     & 0.894 & \textbf{0.862} \\ % Finetuned on candidates
        % $\text{Phi-4}_{cand}$ w/ LF     & \textbf{0.931} & \textbf{0.918} \\ % Finetuned on candidates
        \hline
        KAF-SPA             & 0.912 & 0.776 \\
        KGFI                & 0.924 & 0.844 \\
        CoFFTEA             & \textbf{0.926} & 0.850 \\
        \hline
    \end{tabularx}
    \caption{Results on frame identification using frame element predictions.}
    \label{tab:candidate_frame}
\end{table}
% \subsection{Frame Identification}
% Previous work~\cite{devasier-etal-2024-robust} explored the possibility of filtering candidate targets produced by matching potential lexical units using a frame identification model. To build upon this idea towards a single-step frame-semantic parsing method, we explore the potential of frame elements being used to filter out candidate targets. In this approach, no ground-truth frame inputs are given. This also removes the bias from the model assuming the input always has at least one frame element.

% We represent the LLM instructions using the JSON-exist representation as it performed the best in Table~\ref{tab:representation_performance}. We used Phi-4 for this experiment as it had a very high performance-to-size ratio, as shown in Table~\ref{tab:candidate_frame}. \todo{should run this on qwen-72b} We found that directly using the model performed poorly, likely due to bias in the model learning that each input contains the given frame. To address this, we fine-tuned the LLM using candidates from the training set and found a significant improvement in performance. \todo{add candidates examples}

% Performance on par with CoFFTEA, the previous-best frame identification system.
% Maybe qwen 72b will perform better.

\section{User Feedback}
In addition to addressing the main research questions, a post-study survey (Appendex~\ref{sec:post-question-survey}) consisted of twenty-two questions, including seven Likert scale ratings and fifteen free-text responses from participants provided valuable insights on both ``prompting in the dark'' practices and our system.
We summarize these insights in this section.

%\kenneth{I re-organized this section. Please take a look.}

\subsection{Two Variables Impacting Participant Ratings\label{sec:two-var-on-rating}}
% \alan{Two variables impacting participant ratings?}
Figure~\ref{fig:user-rating} displayed the seven Likert scale rating responses by participants. The seven survey questions can be categorized into seven different categories. 
Appendix~\ref{app:two-var-on-rating} shows
the survey questions and the accompanying categories were rated on a seven-point Likert scale. 

\begin{comment}


listed below:

\begin{itemize}
    \item 
    \textbf{(Q1) Understandable}: The annotation task was easy to understand.

    \item
    \textbf{(Q2) Ease of Use}: The annotation tool is easy to use.

    \item
    \textbf{(Q4) Intuitive System}: The interface of the annotation system is intuitive.

    \item
    \textbf{(Q5) Performance Satisfaction}: How satisfied are you with the performance of the system?

    \item
    \textbf{(Q6) Prompt Improvement}: This tool was helpful in improving my prompt. 
    
    \item
    \textbf{(Q7) Process Efficiency}: Using this tool made the process of prompt engineering more efficient.

    \item
    \textbf{(Q19) Task Completion}: I completed the annotation tasks efficiently.

\end{itemize}
    
\end{comment}

%Figure~\ref{fig:user-rating} shows the participant's rating across different conditions.
\subsubsection{Participants reviewing 10 instances reported higher satisfaction ratings.}
Figure~\ref{fig:sample-size-rating} compares participants who reviewed 10 instances per iteration with those who reviewed 50.
Both groups provided similar ratings for system ease of use, system intuitiveness, and efficiency in processing prompt engineering, with comparable variation.
However, participants who reviewed 10 instances found the annotation tasks more difficult to understand compared to those who reviewed 50. Comparatively, participants who reviewed 10 instances reported higher levels of satisfaction with their performance, a stronger sense of prompt improvement, and better task completion rating. This could be attributed to their minimal modifications to the rule.
% We performed a KS test on all rating categories and no significant difference was found between two groups, indicating that the observed difference did not reach statistical significance.
It is noteworthy that we performed a Kolmogorov-Smirnov (KS) test on all rating categories, and no significant difference was found between the two groups, indicating that the observed difference did not reach statistical significance.

% \steven{Only the performance ratings from participants showed a significant difference between two instances groups based on the t-test (p-value=0.031)}

\subsubsection{Participants without LLM explanations rated the system as more intuitive, effective, and satisfying.}
Figure~\ref{fig:explanation-rating} shows the comparison of ratings between participants with and without LLM explanations.
Participants with LLM explanations found the annotation tasks more challenging, rated the system as less intuitive and harder to use, and viewed it as less effective in improving prompts, also with greater variation in their ratings. In contrast, participants without LLM explanations expressed higher level of satisfaction with the system performance, believing the tool improved prompt engineering efficiency and task completion effectiveness.
% We conducted a KS test on all ratings from participants and no significant difference was found between the two groups, suggesting that the observed difference were not statistically significant.
Notably, we conducted a Kolmogorov-Smirnov (KS) test on all participants' ratings, and no significant difference was found between the two groups, suggesting that the observed differences were not statistically significant.


\begin{figure*}
    \centering
    \begin{subfigure}[t]{0.48\textwidth}
        \includegraphics[width=\linewidth]{Figures/Post-Survey/user_rating_bar_chart_instance.png}\Description{This subplot is for the data sample group, based on participants’ post-survey responses. Each subplot features bar charts comparing two settings: 50 instances vs. 10 instances. Each bar is accompanied by a confidence interval displayed at the top.}
        \caption{Participants' ratings of the system and the annotation task, comparing those who accessed 10 instances per iteration to those who accessed 50 instances.}
        \label{fig:sample-size-rating}
    \end{subfigure}
    \hfill
    \begin{subfigure}[t]{0.48\textwidth}
        \includegraphics[width=\linewidth]{Figures/Post-Survey/user_rating_bar_chart_explanation.png}\Description{This subplot is for the explanation group, based on participants’ post-survey responses. Each subplot features bar charts comparing two settings: no explanation vs. explanation. Each bar is accompanied by a confidence interval displayed at the top.}
        \caption{Participants' ratings of the system and the annotation task, comparing those who utilized LLM explanations to those who did not.}
        \label{fig:explanation-rating}
    \end{subfigure}
    \caption{Participants' rating of the system and the annotation task. Each rating category refers to one question in the post-study survey. 
    \textbf{(Q1) Understandable}: The annotation task was easy to understand; 
    \textbf{(Q2) Ease of Use}: The annotation tool is easy to use;
    \textbf{(Q4) Intuitive System}: The interface of the annotation system is intuitive;
    \textbf{(Q5) Performance Satisfaction}: How satisfied are you with the performance of the system? 
    \textbf{(Q6) Prompt Improvement}: This tool was helpful in improving my prompt; 
    \textbf{(Q7) Process Efficiency}: Using this tool made the process of prompt engineering more efficient; 
    \textbf{(Q19) Task Completion}: I completed the annotation tasks efficiently.}
    \label{fig:user-rating}
\end{figure*}

%\steven{added user rating.}\kenneth{(1) Font for axis titles and the title of the figure are too big, (2) Make the figure wider so that the x-axis labels do not need to rotate (use newline for x-labels if possible), (3) this is kinda extra: can we break it down in two ways (a) with/without explanations and (b) smaller/bigger sample size.}\steven{done}

\subsection{Is \system Useful?}
\subsubsection{Participants considered \system helpful and efficient.}
In the post-study survey, we asked participants to rate the (Q5) performance satisfaction, (Q6) helpfulness of the tool, and (Q7) its efficiency on a seven-point Likert scale.
% The detailed questions asked are shown in Appendex~\ref{sec:post-question-survey}.
%\kenneth{TODO Steven: Update refernece}
Participants expressed high satisfaction, with an average rating of 6.350 (SD=0.745), and found the system helpful for improving prompts (6.400, SD=0.883) and making prompt engineering more efficient (6.600, SD=0.598).
%\kenneth{TODO Steven: Add numbers---- Are there really high???}
P4 noted, ``\textit{I really like this tool instead of traditional prompt engineering on ChatGPT and Copilot.}''
%\kenneth{TODO Steven: Add a few more examples for other participants.}
P15 mentioned, ``\textit{I would be interested in using this annotation system in my regular work or study, because I really like the idea [of] improving annotation performance by considering iteration annotation process between human and the GPT.}''
% A participant (P18) wrote in the questionnaire, ``I like how the system's UI has been designed and programmed.''



%\kenneth{Not sure how to phrase this...}
\subsubsection{\system is easy to use but less intuitive and with a steep learning curve.}
We asked participants to rate whether ``\textit{(Q1) The annotation task was easy to understand},'' ``\textit{(Q2) The annotation tool is easy to use},'' and ``\textit{(Q4) The interface of the annotation system is intuitive}'' on a seven-point Likert scale from ``Strongly Disagree'' (1) to ``Strongly Agree''(7). 
The average score for the ``easy to understand annotation task'' was 5.812, for the ``easy to use annotation tool'' was 5.375, and for the ``system is intuitive'' was 5.250. Suggesting that while the task and tool itself are not hard to understand and use, learning to properly use the tool can be harder for participants and required some learning.
For example, it was noted that the need to switch between tabs during the task can cause confusion.
% One participant mentioned that the system workflow was unclear, as he had to switch between different tabs, causing confusion and disruption.



\subsubsection{Participants found the Shots and Rule Book useful.}
%\paragraph{Having the flexibility to structure tasks freely and make adjustments on the go—whether modifying the Gold Shots or the Rule Book—reduces the burden of the traditional, iterative labeling process.}
We asked participants, in a free-text format, ``\textit{(Q10) What features did you find most useful?}'' Fourteen participants specifically mentioned that `Gold Shots' were particularly valuable, as they provided explicit examples to guide LLMs. 
Additionally, six participants highlighted the usefulness of the Rule Book. 
These two features stood out among the responses, demonstrating their importance in enhancing the user experience.
Participants noted that the flexibility to structure tasks freely and make on-the-fly adjustments---such as modifying the Gold Shots or Rule Book---eases the burden of the traditional iterative labeling process.







% We asked participants, ``What features did you find most useful?'' in free text form and 14 participants mentioned that ``Gold Shots'' were useful and could explicitly provide examples to guide LLMs.
% % 14 participants (P1-P6, P9, P11, P13, P14, P15, P16, P18, P20) thought the ``Gold Shot'' feature was useful as it could explicitly provide examples to guide LLMs. 
% On the other hand, availability of the Rule book was considered useful for 6 participants. 
%\steven{there are some participant like both.}

%\kenneth{Not sure about this...}




\subsubsection{Dilemma of showing LLM explanations.}
%\paragraph{Participants' desire to include LLM explanations.} 
Although our study found that providing LLM explanations can sometimes lead participants to generate labels more aligned with those produced by the LLM, participants still expressed a strong desire to have them included. P8 explicitly recommended incorporating LLM explanations, noting that participants were interested in understanding the reasoning behind potential discrepancies between their own labels and those generated by the LLM. P15 also emphasized the value of these explanations, stating, ``\textit{The explanation from GPT gave me some insights to modify my rules,''} and, ``\textit{I think GPT's explanation of the tweets is very helpful and it may help me to improve the accuracy of human annotation.''}

\begin{comment}
 

\subsubsection{What can be improved in \system?}
%\paragraph{Workflow can be confusing and interface not user friendly}
Some participants faced challenges when learning the system, and it can take them a long time to get comfortable with it. P1 said, ``\textit{The system is not logically clear for me because the system needs to jump in between different tabs, which is different than a normal workflow.}'' 
Some participants became confused with the system, even after several iterations. For instance, a few participants forgot that to proceed to the next iteration, they needed to sample the data and click ‘Start Annotation.’ Research team members need to remind participants of the workflow continuously. Additionally, the nature of the task requires moving from one tab to another, which adds difficulty for participants to navigate the interface and causes confusion.

   
\end{comment}

\subsection{About ``Prompting in the Dark''}

%\kenneth{Not sure about this too}
\subsubsection{Prompting in the dark without any tool is common.}
%\alan{Iterative, trial and error prompting is common without the help of tools.}}
We also asked participants, ``\textit{Without this tool, how would you typically approach prompt engineering?}''
We found that many participants commonly rely on iterative, trial and error strategies. Specifically, they start with prompts from scratch, test them on data points, adjust based on incorrect labels, and re-test until they are satisfied with the results.
% Responses varied: P0 emphasized giving LLMs as much context as possible\steven{this is P0, which is not included in our actual user study}, while others (P1, P3, P4) described a general refining process---starting prompts from scratch, testing on data points, adjusting based on incorrect labels, and re-testing until satisfied.
%\kenneth{TODO Steven: Add a few more examples for other participants.}
%\alan{trial and error, iterative process, personify}

P1 said, `\textit{I need to start with a prompt from scratch; then I will test it on real data points; I will observe those wrongly labeled data points and adjust my prompt accordingly. After the adjustment, I will rerun the testing on the real data points. The whole process is trail-and-error, which is really time-consuming and labor-consuming.}''
P3 stated, ``\textit{Give an initial prompt, if the answer is not meeting expectation, then change the prompt.}''
% P5[Keep trying different prompts to see if the responses I get satisfies my expectations.]
P6 reflected, ``\textit{Normally, if I do not get the desired output from the LLM, I will try to give more  specific instruction maybe some examples.}''
% P14[Try prompting with ChatGPT, if ChatGPT cannot provide a good answer, just rephrase the prompt and ask the question again.]
P17 said, ``\textit{I re-write my prompts several times (3-5 times) until I got an output that I like.}''

\begin{comment}
 

We also want to emphasize that, although not common, some participants employ a personification method by asking LLMs to take on a specific role or personality and make decisions based on that role. For example, P12 stated, ``\textit{I will first assign a role to GPT like ``Supposing you are an expert in coding, ...''. Then I will ask it to follow my instructions.}''
% Personify, P103 [Make llm assume that it is not an AI and act as a specific person who is involved in that specific activity. By Giving as much context as possible.], P12 [I will first assign a role to GPT like ``Supposing you are an expert in coding, ...''. Then I will ask it to follow my instructions.]
P13 said, ``\textit{I would narrate the incident or situation environment and then give prompt asking specific and questions clearly.}''

   
\end{comment}

% \paragraph{\system gave participants new insights into prompt engineering.}
% Interestingly,
% our survey also showed that using \system gave participants new insights into prompt engineering. 
% For example, P1 observed that ``adding or deleting a few words can change the overall output,'' and P2 noted, ``I can compare how many labels are correctly labeled before and after modifying prompts.''
%\kenneth{TODO Steven: Add a few more examples for other participants.--- This one is interesting. Say a few more if possible?}




\subsubsection{Prompting in the dark is hard, as participants lacked confidence in their labels.}
%\paragraph{Participants lacked confidence in labeling}
Without a comprehensive understanding of the entire dataset, participants found it challenging to generate suitable labels.
P19 mentioned, ``\textit{I am not confident about the label}''.
P12 pointed out that, ``\textit{When I need to express sentiment, I tend to be more reserved and avoid extremes. So, when labeling data, I usually prefer to choose negative/[positive] rather than extremely negative/[extremely positive]}''

%We also found that,
%\paragraph{Difficulty in capturing the full picture with limited samples}
%with participants engaging with only around 10 instances per round, they can overanalyze the limited tweets, resulting in a narrow refinement of their rule books. 
%This limitation also affected the gold standards labels for LLM learning. 
%P19 suggested, ``\textit{I didn't see any `Extremely Positive' in each verification round, but I saw five for now in the evaluation set.}''.

% \paragraph{Hard to revise Rule Book}
% Some participants preferred to write detailed label definitions at the start, which resulted in fewer rule revisions in the later rounds. Alternatively, they tended to add gold shots than editing rules.




% Some participants had difficulty understanding how to respond to context questions regarding the annotation task. \alan{clarify}They mistakenly answered the question by describing the system's workflow rather than focusing on the Twitter task. P3 and P8 suggested having more context tutorials for participants to understand the annotation task. 
% P9 pointed out, ``the `Start Annotation' button is hard to find.''.
%\kenneth{this is probably, in part, due to the nature of the problem}

%\subsection{Challenges and Pain-points}
%\kenneth{The following basically said we found the problems at the beginning of our study and what we did to mitigate them.}
% \paragraph{Difficulty of navigating the interface}
% The participants’ experience was not smooth, with most challenges centered around the workflow, such as moving from tab to tab and the complexity of the system features. 
% The most frequent issues participants encountered involved confusion with the system’s annotation procedure, which sometimes indicated users getting used to the interface.\alan{clarify} The issue often disappeared once users became more familiar with the system.
% Some participants (P2 and P7) struggled to respond to context questions\alan{?}, which were intended to gather information related to the Twitter sentiment analysis, such as the purpose and usage of the annotated data. However, they mistakenly focused on describing the system’s workflow rather than focusing on the annotation task.
% Reflecting on user feedback, we improved the tutorial by switching from a video format to a step-by-step manual walkthrough, ensuring participants could fully understand each tab and its purpose while also allowing them to ask questions at any point.\alan{does this actually help?}
% To help participants better understand the Twitter Sentiment task, we provided template answers that they could use as a reference to create their own responses or adopt directly.


%\subsubsection{Participants with 50 instances per iteration}

%\subsubsection{Participants with 10 instances per iteration}
 





%\kenneth{An important context here is: Our study results showed that (1) NOT showing LLM explanations is better, and (2) showing LLM explanations make users' labels more similar with each other, i.e., LLMs tailor users to be more like LLMs.}

% P10 liked ``The LLM explanations for every tweet after the levels were filled out '' the most.

% P15 mentioned, ``The explanation from GPT gave my some insights to modify my rules.'' and ``I think GPT's explanation of the tweets is very helpful and it may help me to improve the accuracy of human annotation.''

% \paragraph{Participants influenced by LLM explanations.} 
% For the first four participants (P1-P4), the system offered an LLM explanation option during the annotation process. We observed that each participant reviewed these explanations thoroughly, even though they were instructed to verify the LLM labels using their own judgment. 

% %\subsubsection{Participants with LLM Explanations}
% %\subsubsection{Participants tended to be impacted by LLM Explanations} 
% Participants tended to review LLM explanations thoroughly during the verifying process, even though they were instructed to verify the labels using their own judgment. 

%\paragraph{A few participants turned off the LLM Explanation option}

%\subsubsection{Participants without LLM Explanations}



\subsection{Users' Suggested Features}

\subsubsection{More automated supports for rule creation.}
Participants expressed concerns about creating rules that effectively suit the labeling task at hand. As a result, support for Rule Book creation is a welcome addition.
P2 remarked, ``\textit{It was hard to set the right rules,}'' while P3 suggested providing initialized instructions for labels and rules to ease the process. Additionally, participants (P1, P3) proposed that new rules could be automatically generated based on Gold Shots, existing rule books, and human explanations, streamlining the rule creation process.

%\kenneth{Can we make subsubsection title a complete sentence?}
%\alan{something like this?}
\subsubsection{Shorter LLM explanations for easier consumption.}

Although LLM-generated explanations received positive feedback from participants, there was concern about the length of these explanations. Many felt that the explanations were too long and could be difficult to consume. P1 recommended, ``\textit{It would be better if the LLM explanation could be shorter.}'' emphasizing the need for more concise outputs to improve user experience.





% P9 pointed out, ``the `Start Annotation' button is hard to find.''.


% \paragraph{Simplified Interfaces and Workflows.}
% Some participants suggested hiding uncommon features (P17) or displaying the function descriptions only when hovering over them (P20).

%\paragraph{Other suggestions}\steven{Not related to the annotation system}
% (P1)If there's a graphical UI, the system will be more accessible to laypersons without the HCI background. Currently, the system is too flexible to get lost in the interactions.
%(P3) More tutorials on the data annotation task.



%\subsection{Participant In-Lab Annotation Process Analysis}



%\subsection{Participant Self-Reported Response Analysis}



\section{Discussion}
\section{Discussion}
The development of foundation models has increasingly relied on accessible data support to address complex tasks~\cite{zhang2024data}. Yet major challenges remain in collecting scalable clinical data in the healthcare system, such as data silos and privacy concerns. To overcome these challenges, MedForge integrates multi-center clinical knowledge sources into a cohesive medical foundation model via a collaborative scheme. MedForge offers a collaborative path to asynchronously integrate multi-center knowledge while maintaining strong flexibility for individual contributors.
This key design allows a cost-effective collaboration among clinical centers to build comprehensive medical models, enhancing private resource utilization across healthcare systems.

Inspired by collaborative open-source software development~\cite{raffel2023building, github}, our study allows individual clinical institutions to independently develop branch modules with their data locally. These branch modules are asynchronously integrated into a comprehensive model without the need to share original data, avoiding potential patient raw data leakage. Conceptually similar to the open-source collaborative system, iterative module merging development ensures the aggregation of model knowledge over time while incorporating diverse data insights from distributed institutions. In particular, this asynchronous scheme alleviates the demand for all users to synchronize module updates as required by conventional methods (e.g., LoRAHub~\cite{huang2023lorahub}).


MedForge's framework addresses multiple data challenges in the cycle of medical foundation model development, including data storage, transmission, and leakage. As the data collection process requires a large amount of distributed data, we show that dataset distillation contributes greatly to reducing data storage capacity. In MedForge, individual contributors can simply upload a lightweight version of the dataset to the central model developer. As a result, the remarkable reduction in data volume (e.g., 175 times less in LC25000) alleviates the burden of data transfer among multiple medical centers. For example, we distilled a 10,500 image training set into 60 representative distilled data while maintaining a strong model performance. We choose to use a lightweight dataset as a transformed representation of raw data to avoid the leakage of sensitive raw information.
Second, the asynchronous collaboration mode in MedForge allows flexible model merging, particularly for users from various local medical centers to participate in model knowledge integration. 
Third, MedForge reformulates the conventional top-down workflow of building foundational models by adopting a bottom-up approach. Instead of solely relying on upstream builders to predefine model functionalities, MedForge allows medical centers to actively contribute to model knowledge integration by providing plugin modules (i.e., LoRA) and distilled datasets. This approach supports flexible knowledge integration and allows models to be applicable to wide-ranging clinical tasks, addressing the key limitation of fixed functionalities in traditional workflows.

We demonstrate the strong capacity of MedForge via the asynchronous merging of three image classification tasks. MedForge offered an incremental merging strategy that is highly flexible compared to plain parameter average~\cite{wortsman2022model} and LoRAHub~\cite{huang2023lorahub}. Specifically, plain parameter averaging merges module parameters directly and ignores the contribution differences of each module. Although LoRAHub allows for flexible distribution of coefficients among modules, it lacks the ability to continuously update, limiting its capacity to incorporate new knowledge during the merging process. In contrast, MedForge shows its strong flexibility of continuous updates while considering the contribution differences among center modules. The robustness of MedForge has been demonstrated by shuffling merging order (Tab~\ref{tab:order}), which shows that merging new-coming modules will not hurt the model ability of previous tasks in various orders, mitigating the model catastrophic forgetting. 
MedForge also reveals a strong generality on various choices of component modules. Our experiments on dataset distillation settings (such as DC and without DSA technique) and PEFT techniques (such as DoRA) emphasize the extensible ability of MedForge's module settings. 

To fully exploit multi-scale clinical data, it will be necessary to include broader data modalities (e.g., electronic health records and radiological images). Managing these diverse data formats and standards among numerous contributors can be challenging due to the potential conflict between collaborators. 
Moreover, since MedForge integrates multiple clinical tasks that involve varying numbers of classification categories, conventional classifier heads with fixed class sizes are not applicable. However, the projection head of the CLIP model, designed to calculate similarities between image and text, is well-suited for this scenario. It allows MedForge to flexibly handle medical datasets with different category numbers, thus overcoming the challenge of multi-task classification. That said, this design choice also limits the variety of model architectures that can be utilized, as it depends specifically on the CLIP framework. Future investigations will explore extensive solutions to make the overall architecture more flexible. Additionally, incorporating more sophisticated data anonymization, such as synthetic data generation~\cite{ding2023large}, and encryption methods can also be considerable. To improve data privacy protection, test-time adaptation technique~\cite{wang2020tent, liang2024comprehensive} without substantial training data can be considered to alleviate the burden of data sharing in the healthcare system.



             


\section{Conclusion and Future Work}
\section{Conclusion}
We reveal a tradeoff in robust watermarks: Improved redundancy of watermark information enhances robustness, but increased redundancy raises the risk of watermark leakage. We propose DAPAO attack, a framework that requires only one image for watermark extraction, effectively achieving both watermark removal and spoofing attacks against cutting-edge robust watermarking methods. Our attack reaches an average success rate of 87\% in detection evasion (about 60\% higher than existing evasion attacks) and an average success rate of 85\% in forgery (approximately 51\% higher than current forgery studies). 

\begin{acks}
\input{Sections/8.5-Acknowledgement}
\end{acks}

\bibliographystyle{Style/ACM-Reference-Format}
\bibliography{Bibtex/sample-base,Bibtex/software,Bibtex/main}

%TC:ignore
\appendix
\newpage
\appendix
\section{Appendix}

\subsection{Conversational agent prompts for generating stable diffusion prompts in art-making phase}

\textbf{Role:} You will be able to capture the essence of the sessions and drawings in the recordings based on the art therapy session recordings I have given you and summarize them into a short sentence that will be used to guide the PROMPT for the Stable Diffusion model.

\vspace{0.5em} % 添加一些垂直间距

\textbf{Example input:}

\begin{itemize}[leftmargin=*]
    \item \textbf{USER:} [user-drawn] I drew the ocean. [canvas content] There is nothing on the canvas right now.
    \item \textbf{ASSISTANT:} What kind of ocean is this?
    \item \textbf{USER:} [user-drawn] I drew grass. [canvas content] Now there is an ocean on the canvas.
    \item \textbf{ASSISTANT:} What kind of grass is this?
    \item \textbf{USER:} [user-drawn] I drew the sky. [canvas content] Now there is grass and ocean on the canvas.
    \item \textbf{ASSISTANT:} What kind of sky is this?
    \item \textbf{USER:} [user-drawn] I drew mountains. [canvas content] Now there is sky, grass, and ocean on the canvas.
    \item \textbf{ASSISTANT:} What kind of mountain is this?
    \item \textbf{USER:} [user-drawn] I drew clouds. [canvas content] Now there is sky, mountain, grass, and ocean on the canvas.
    \item \textbf{ASSISTANT:} What kind of cloud is this?
    \item \textbf{USER:} [user dialogue] Colorful clouds, emerald green mountains and grass, choppy ocean
\end{itemize}

\vspace{0.5em} % 添加一些垂直间距

\textbf{Task:}

\begin{enumerate}[label=\textbf{Step \arabic*:}]
    \item \textbf{[Step 0]} Read the given transcript of the art therapy session, focusing on the content of \texttt{user: [user drawing]} and \texttt{user: [user dialog]}: Go to \textbf{[Step 1]}.
    \item \textbf{[Step 1]} Based on the input, find the last entry of user's input with \texttt{[canvas content]}, find the keywords of the screen elements that the canvas now contains (in the example input above, it is: sky, grass, sea), separate the keywords of each element with a comma, and add them to the generated result. Examples: [keyword1], [keyword2], [keyword3], \dots, [keyword n].
    \item \textbf{[Step 2]} Find whether there are more specific descriptions of the keywords of the painting elements in \texttt{[Step 1]} in \texttt{[User Dialog]} according to the input. If not, this step ends into \textbf{[Step 3]}; if there are, combine these descriptions and the keywords corresponding to the descriptions into a new descriptive phrase, and replace the previous keywords with the new phrases. Examples: [description of keyword 1] [keyword 1], [keyword 2 description of keyword 2], [description of keyword 3], \dots. Based on the above example input, the output is: rough sea, lush grass, blue sky.
    \item \textbf{[Step 3]} Based on the input, find out if there is a description of the painting style in the \texttt{[User Dialog]} in the dialog record, and if there is, add the style of the picture as a separate phrase after the corresponding phrase generated in \texttt{[Step 2]}, separated by commas. For example: [description of keyword 1] [keyword 1], [description of keyword 2] [keyword 2], \dots, [screen style phrase 1], [screen style phrase 2], [screen style phrase 3], \dots, [Picture Style Phrase n].
\end{enumerate}

\vspace{0.5em} % 添加一些垂直间距

\textbf{Output:} 

Only need to output the generated result of \textbf{[Step 3]}.

\vspace{0.5em} % 添加一些垂直间距

\textbf{Example output:} 

\emph{Rough sea, lush grass}

\subsection{Conversational agent prompts for discussion phase}

\textbf{Role:} <therapist\_name>, Professional Art Therapist

\textbf{Characteristics:} Flexible, empathetic, honest, respectful, trustworthy, non-judgmental.

\vspace{0.5em} % 添加垂直间距

\textbf{Task:} Based on the user's dialogic input, start sequentially from step [A], then step [B], to step [C], step [D], step [E] \dots Step [N] will be asked in a dialogical order, and after step [N], you can go to \textbf{Concluding Remarks}. You can select only one question to be asked at a time from the sample output display of step [N]! You have the flexibility to ask up to one round of extended dialog questions at step [N] based on the user's answers. Lead the user to deeper self-exploration and emotional expression, rather than simply asking questions.

\vspace{0.5em} % 添加垂直间距

\textbf{Operational Guidelines:}

\begin{enumerate}
    \item You must start with the first question and proceed sequentially through the steps in the conversational process (step [A], step [B], step [C], step [D], step [E], \dots, step [N]).
    \item Do not include references like step '[A]', step '[B]' directly in your reply text.
    \item You may include one round of extended dialog questions at any step [N] depending on the user's responses and situation. After that, move on to the next step.
    \item Always ensure empathy and respect are present in your responses, e.g., re-telling or summarizing the user's previous answer to show empathy and attention.
\end{enumerate}

\vspace{0.5em} % 添加垂直间距

\textbf{Therapist’s Configuration:}

\textbf{Principle 1:}  
\textit{Sample question:} How are you feeling about what you are creating in this moment?

\vspace{0.5em}

\textbf{Principle 2:}  
\textit{Sample question:} Can you share with me what this artwork represents to you personally? 

\vspace{0.5em}

\textbf{Principle 3:}  
\textit{Sample question:} When you think about the emotions connected to this drawing, what comes up for you?

\vspace{0.5em}

\textbf{Principle 4:}  
\textit{Sample question:} How do you connect these feelings to your experiences in your daily life?

\vspace{0.5em} % 添加垂直间距

\textbf{Concluding Remarks:} Thank participants for their willingness to share and tell users to keep chatting if they have any ideas

\vspace{1em} % 添加额外的间距

\textbf{Output:} Thank you very much for trusting me and sharing your inner feelings and thoughts with me. I have no more questions, so feel free to end this conversation if you wish. Or, if you wish, we can continue to talk.

\subsection{AI summary prompts}

\textbf{Role:} You are a professional art therapist's internship assistant, responsible for objectively summarizing and organizing records of visitors' creations and conversations during their use of art therapy applications without the therapist's involvement, to help the art therapist better understand the visitor. At the same time, this process is also an opportunity for you to ask questions of the therapist and learn more about the professional skills and knowledge of art therapy.

\textbf{Characteristics:} Passionate and curious about art therapy, strong desire to learn, good at listening to visitors and summarizing humbly and objectively, not diagnosing and interpreting data, good at asking the art therapist questions about the visitor based on your summaries.

\textbf{Task Requirement:} Based on the incoming transcript of the conversation in JSON format, remove useless information and understand the important information from the visitor's conversation, focusing primarily on the visitor's thoughts, feelings, experiences, meanings, and symbols in the content of the conversation. Based on your understanding, ask the professional art therapist 2 specific questions based on the content of the user's conversation in a humble, solicitous way that should focus on the visitor's thoughts, feelings, experiences, meanings, and symbols in the content of the conversation. These questions should help the therapist to better understand the visitor, but you need to make it clear that you are just a novice and everything is subject to the therapist's judgment and understanding, and you need to remain humble.

\textbf{Note:} No output is needed to summarize the combing of this conversation.



%TC:endignore

\end{document}
\endinput
%%
%% End of file `sample-sigconf-authordraft.tex'.
