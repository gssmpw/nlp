

%\kenneth{Note that Figure 1 is a bit simplified, e.g., label verification, keep items for next iteration, is dismissed to make sure clear communication.}
%The \system is a Google Spreadsheet add-on with a user interface for guiding LLMs by providing rules or exemplary data to improve LLM performance. The system could be adapted to various single-class data annotation tasks. 
%Users can easily operate the system with minimal learning time and without the need for complex environment setup, unlike other applications.~\steven{find some other annotation systems}
In this paper, we present \system, a Google Sheets add-on that allows users to load a dataset into a Dataset spreadsheet, manually compose each part of a prompt within Task Context, Labeling Rules, and Shots sheets, and use the composed prompt to instruct an LLM to annotate data---all within the same Google Sheets document.
Motivated by the need to enable general users to prompt LLMs without installing and configuring professional tools like integrated development environments (IDEs) or Jupyter Notebooks, we decided to build a tool based on spreadsheets, which most computer users are already familiar with.
This section overviews \system's design and workflows.
%\kenneth{TODO: Maybe cite other tools and name what setup or configs they require}

\subsection{Design Goals}\label{sec:3-system-design}
%\kenneth{(1) Support PITD so we don't have gold label and don't calculate accuracy etc becasyue we think they're not reliable and evovling, and (2) using spreadsheet because it's just very easy to use and practically provide many flexibilities to users.}

%Prompting in the Dark: 
\paragraph{Adapting to Evolving Labeling Goals}
%\paragraph{Prompting in the Dark and Users' Needs for Evolving, Open-Ended Labeling Schemes.}
The goal of \system is to enable general users to create and refine prompts iteratively for LLM-powered data labeling, particularly in situations where they start without any labeled gold data or manual labeling, \ie, the ``prompting in the dark'' scenario. 
In these cases, users' understanding of the data and desired labeling scheme evolves through their interactions with the LLM, based on its predicted labels and explanations, rather than through their own manual efforts. 
The lack of gold labels (or sufficient labeled data) introduces the core challenge of the prompt-in-the-dark process: aside from users' observations and judgments about the labeling results, there is no concrete way to provide quick and comprehensive feedback on the progress of prompting.
We view this as a trade-off between two user needs in prompt engineering: 
{\em (i)} allowing users' understanding of the data and labeling goals to evolve, and 
{\em (ii)} providing clear guidance and reliable feedback to assess progress toward a defined annotation goal. 
Previously, supervised learning-based classifiers required labeled data, so users focused heavily on the second need, as manual labeling was always needed and assumed to finalize the coding scheme. 
The rise of LLMs has reduced the need for pre-labeled data, allowing users to put more focus on their first needs. 
The growing popularity of the prompting-in-the-dark approach reflects users' need for evolving and dynamic labeling practices~\cite{austin2024grad,zhang-etal-2024-glape,wang2024human}.
%zhang2023labelvizier to facilitate the validation and relabeling of large-scale technical text annotations. Its interactive, visual analytic interface allows users to detect and correct three main types of labeling errors: duplicate, wrong, and missing labels. 
%\kenneth{TODO: Do we have any reference to support it's a popular practice now?}\steven{those are three study that have no gold label, they iteratively use user-defined criteria to evaluate and refine.}\kenneth{Oh and did they manage to improve the accuracy over time??}\steven{Yeah, they got a higher rating and accuracy}\kenneth{Hmmmmmm so what's the deal? How are their systems and approaches different?} %, rather than simply saving time on manual labeling. 
% dudley2018review describe the interative machine leanring paradigm that user iterative build and refine model. The model refinement is driven by user input. It more focused on the human input to refine. kim2024evallm is a prompt refining tool by evaluating outputs on user-defined criteria. It is not a labeling task
Our design goal for \system is to offer users greater flexibility and freedom in defining how their data should be labeled.


%\kenneth{I revised this following paragraph to tailor it more close to our target users. Please take a look.}
\paragraph{Supporting a Wide Range of ``Newcomers'' Brought in By LLMs.}
%\paragraph{End-User Prompting Tools.}
%Our second design goal for \system is to create a tool for general users, including people with limited or no programming skills.
Our second design goal for \system is to develop a tool for users with \textbf{little to no experience in large-scale text data labeling}, including but not limited to those with limited or no programming skills.
The rationale behind this goal is two-fold.
On a practical level, 
people new to large-scale data annotation---empowered by LLMs to undertake such tasks with greater ease---are more likely to adopt approaches that diverge from conventional practices. 
In the crowdsourcing literature, many papers emphasize best practices for data annotation~\cite{hsueh2009data, sabou2014corpus, vondrick2013efficiently, drutsa2019practice, wang2013perspectives}, %\kenneth{TODO: Add ref on crowdsourcing best practices}\steven{added}
such as ethical pay rates~\cite{fort2011amazon,shmueli2021beyond}, %\kenneth{TODO: Add refs on crowd workers' ethical pay}\steven{added}
usable worker interfaces~\cite{toomim2011utility,10.1145/3613904.3642834, rahmanian2014user, komarov2013crowdsourcing}, %\kenneth{TODO: Add refs to crowd interface's impact on crowdsourced data quality--- Maybe cite our own CHI paper too}\steven{added}
and 
gold labels for quality control~\cite{han2020crowd,gadiraju2015training,le2010ensuring,doroudi2016toward,hettiachchi2021challenge}. %\kenneth{TODO: Again, add ref IN CROWDSOURCING for gold labls}\steven{added}
However, these practices are often neglected in real-world scenarios. 
For instance, many tasks on MTurk still offer very low pay~\cite{AI_workers_low_wages} %\kenneth{Add ref: A data-driven analysis of workers' earnings on Amazon Mechanical Turk}\steven{added}
or rely on poorly designed interfaces~\cite{fowler2023frustration}. %\kenneth{Add ref: Frustration and ennui among Amazon MTurk workers}\steven{added}
Newcomers to large-scale data annotation are even less likely to be familiar with these best practices, including carefully establishing gold labels before prompting LLMs.
%users without programming experience are more likely to prompt in the dark, struggling to interact effectively with LLMs. 
%In contrast, those with software engineering backgrounds are familiar with using established frameworks and tools, where automatic testing---such as unit and integration testing---is standard. 
The bigger picture is that LLMs are adding many ``new members'' to the world of programming and data science.
%---people with little or no coding experience. 
This group brings new practices, user needs, challenges, and research questions to HCI, requiring more focused attention.

%\bigskip
\sloppy
Based on these two design goals, we decided to build \system based on spreadsheets, a format that most computer users are already familiar with.
We distinguish our goals from existing efforts in two significant ways. 
First, while projects like LangChain or ChainForge focus on developers or those with programming backgrounds, requiring software installations or configurations, we aim to focus on general users who do not necessarily have such expertise. 
Second, some projects explore new interactions enabled by LLMs~\cite{10.1145/3586183.3606833}, but our project is concerned with understanding how effectively users can use familiar interfaces, such as spreadsheets, to interact with LLMs.

%--------------- dead kitten ----------

\begin{comment}
  


At the practical level, users with little or no programming background might be more likely to prompt in the dark. 
In contrast, individuals familiar with software engineering practices are accustomed to using existing frameworks or tools, where automatic testing---such as unit and integration testing---is standard.
These users have more experience as well as technological support for creating gold labels for testing.
At a deeper level, what is happening is that LLMs bring many people with limited or no programming skills into the realm of data science or programming tasks.
This group of people brings new and interesting challenges and research questions to HCI and thus deserves more attention.

%we believe that the greatest value of LLMs lies in the new possibilities they offer to general users. 
%For people without coding skills, LLMs enable tasks such as building websites from scratch, creating classifiers for automating email filtering, and labeling data to extract insights---activities that were previously within the realm of programmers. 

  
\end{comment}

\begin{figure*}[t]
    \centering
    \includegraphics[width=0.9\linewidth]{Figures/ui-overview.jpg}\Description{This is the user interface layout and pre-defined spreadsheet tab explanation. Each predefined sheet has a set of predefined columns. PromptSheet allows users to load a dataset into the Dataset tab, which is the starting tab of the system. By clicking each tab at the bottom of the Google Sheet, users can navigate to Task Context tab, Rule Book tab, Shots tab, Working Data Sample tab, task dashboard tab and task results tabs. The task result tabs will be generated after each new annotation round and store all new annotated results.}
    \caption{The user interface and all the predefined sheets of \system, where each sheet has a set of pre-defined columns. 
    \system allows users to load a dataset into a Dataset (A) sheet, manually compose each part of a prompt within Task Context (B), Labeling Rules (C), and Shots (D) sheets, and use the composed prompt to instruct an LLM to annotate data and store the labeling results in a separate task sheet (G). All functions are presented as manuals and buttons within the sidebar on the right.}
    \label{fig:system-interface-ui-kenneth}
\end{figure*}

\subsection{User Interface and Pre-Defined Sheets}

\system is a Google Sheets add-on that enables users to load data, sample a subset for labeling, compose and edit prompts, use these prompts to request LLMs for data labeling, and iteratively revise the prompts. 
Figure~\ref{fig:system-interface-ui-kenneth} shows the interface of \system.

\paragraph{Sidebar.}
Following Google Sheets' design constraints, all functions are presented as manuals and buttons within the sidebar on the right. 
The sidebar remains consistent across all sheets, regardless of which sheet is in use. 
At the top of the sidebar, \system provides a real-time notification that keeps users informed about its ongoing processes, such as ``Data Indexing,'' ``Data Sampling,'' ``Generating the Instructional Prompt,'' or ``Annotating.''


\paragraph{Pre-Defined Sheets.}
\system includes a set of predefined spreadsheets, each with a set of pre-defined columns. 
At the bottom of the interface, a series of tabs allows users to switch between sheets, with each sheet dedicated to a different part of the data labeling process. 
The following describes each sheet in detail. 
(To help readers easily identify which sheet we are referring to, we indexed each sheet as A, B, C, ..., and G in all the figures. 
These indexes were not present in the actual system to users.)

%\subsubsection{All the Sheets and What They Do}

%\hyperref[fig:system-interface-1]{Figure 1} overviews the \system's user interface. 
%The system consists of seven main components:

\begin{itemize}

\item \textbf{Dataset (Sheet A)}:
%The ``Dataset'' includes three columns: Data ID, Group ID, and Data Instance. 
%Each data instance is uniquely identified by a corresponding Data ID.
%A single Group ID may encompass one or multiple Data Instances.
This spreadsheet stores the full dataset.
Users can copy and paste the dataset into this sheet or use any supported Google Sheets import method (in Step 0).
The sheet includes three key predefined columns: (1) Data ID, (2) Group ID, and (3) Data Instance. 
Each data instance is uniquely indexed by its corresponding Data ID, which users can generate by clicking the ``Index Data ID'' function in the sidebar. 
The Group ID is used for annotating sequential data, such as when each sentence in an article is treated as a separate data instance, but all sentences from the same article share the same Group ID.
In our design, this sheet is intended to serve as a static data source, and we anticipate that users will not modify it after loading the data.

%\kenneth{TODO: Maybe add words to mention we don't expect people touch it after Step 0 and basicallyy serve as a database.}

\item \textbf{Task Context (Sheet B)}:
%The ``Context'' tab provides information to help describe the annotation task users are working on. It addresses questions related to the purpose and application of the data annotation task and the origin and size of each data instance. The LLM will use the information provided in this tab to generate an instructional prompt for a later step.
This spreadsheet stores the meta-information and context for the labeling task, which will later be incorporated into the prompt. 
The sheet includes predefined questions that characterize the task, such as the purpose of the data labeling, how the labels will be used, the source of the data, and the size of each data instance.
Table~\ref{tab:task-sheet-questions} in Appendix~\ref{sec:context-question-appendix} shows all the questions.
%\kenneth{TODO: Maybe add all the questions to Appendix.}
Users provide answers to these questions (in Step 1 or 4), and \system automatically incorporates both the questions and their answers into the prompt used for LLMs to label the data.


    
\item \textbf{Rule Book (Sheet C)}: 
%The ``Rule book'' tab is where users define the criteria and definitions for each label used during the annotation process.
This spreadsheet contains the labeling rules that the LLM will follow.
It includes two key predefined columns: (1) Label Name and (2) Rules for the Label. 
Users manually define the criteria and descriptions for each label in free text (in Step 1 or 4), detailing the guidelines for the annotation process. 
Multiple rules can be added for a single label, providing flexibility in defining the labeling criteria.


    
\item \textbf{Shots (Sheet D)}: 
%In the ``Shots'' tab, users can enter gold standard labels from their iterations or manually provide them as reference points.
This spreadsheet stores all the high-quality examples, including data instances and their corresponding labels, which will be included in the prompt to guide the LLM in labeling the data. 
These examples, commonly referred to as ``shots'' in prompts, follow the same predefined column structure, with an additional ``Gold-Standard Label'' column. 
Users can add these examples manually (in Step 1) or use \system's function to do so (in Step 4).


    
\item \textbf{Working Data Sample (Sheet E)}:
%Users can sample the data from the ``Dataset'' tab to the ``Working Data Sample'' tab. In the annotation process, \textit{only} data instance in the ``Working Data Sample'' tab will be annotated.
% by the LLMs using the instructional prompt, provided rules, and gold shots.
This spreadsheet stores the current subset of data selected from the full dataset, ready for the LLM to label.
Users can sample data from the Dataset sheet by clicking the corresponding buttons in the sidebar; users can choose between random sampling or selecting a specific range (Step 2). 
During the annotation process, only the data instances in the Working Data Sample sheet will be labeled. 
\system will copy the entire data sample from the Working Data Sample sheet to create a new sheet to label (Step 3).
    
    
\item \textbf{Task Dashboard (Sheet F)}:
%The ``Task Dashboard'' tab records all iteration task details such as task number, timestamp, used prompt, and total costs. 
This spreadsheet tracks all labeling tasks performed so far.
When the user clicks the ``Start Annotation'' button in the sidebar (in Step 3), \system creates a new sheet for the task (e.g., Task 1 sheet) and adds a new row in the Task Dashboard to record the labeling activities.
Task Dashboard sheet (Figure~\ref{fig:task-dashboard-new})
logs task details such as task number, timestamp, the prompt used, and total costs.

\item \textbf{Task 1 (Sheet G), Task 2, ..., Task N}:
%After each annotation, the annotation results will be saved in a new tab (e.g., Task\_1, Task\_2, etc) corresponding to that specific iteration. 
Each of these sheets stores the annotation results for each labeling request, including data samples, LLM-generated labels, and LLM explanations (optional).
These sheets also include columns that allow users to validate or correct the LLM labels and optionally add them to the Shots sheet (in Step 4).
When the user clicks the ``Start Annotation'' button in the sidebar (in Step 3), \system generates a new task sheet to handle the specific labeling task.

    
    
\end{itemize}

%\kenneth{Users are allowed to add new columns.}

Notably, while users must follow our guidelines for using the predefined columns in each sheet and inputting data correctly, they are free to add more columns or even additional sheets, just as they would in a regular Google Sheets document. 
For instance, when pasting a dataset into the Dataset sheet, it is common for the dataset to include its own IDs or additional information for each data entry. 
Users can easily store this extra information by creating new columns within the Dataset sheet.


%---------- dead kitten ----------

\begin{comment}



\subsubsection{Other Features} \steven{todo: add figures in the appendix. screenshots for different notification messages. interface screenshots for removal and clearing.}
\begin{itemize}
    \item \textbf{Real-Time System Notification: }\system provides a notification feature that informs users of its current processes, such as ``Data Indexing'', ``Data Sampling'', ``Generating the Instructional Prompt'', ``Annotating'', etc.
    \item \textbf{Remove Unselected Data Instance (Figure \ref{fig:remove-clear}): }This function will remove data instances that do not have the ``Keep it in the next data sample'' checked in the ``Working Data Sample'' tab.
    \item \textbf{Clear Data Instance (Figure \ref{fig:remove-clear}): }This function will clear all data instances in the ``Working Data Sample'' tab.
\end{itemize}

    
\end{comment}



\subsection{User Workflow}
\begin{figure*}[t]
    \centering
    \includegraphics[width=0.99\linewidth]{Figures/step-1.jpg}\Description{This is Step 1 described in Figure 1. Users can provide data annotation context in the Context Tab, provide their rule and definition in the Rule Book tab, and add gold standard labels in the Shots tab. These tabs will compose prompts for later GPT to use.}
    \caption{The overview of step 1 of the data labeling process, compose or refine the prompt.
This is the most critical step, where the user composes and refines prompts for the LLM to label the data. In \system, the prompt consists of three parts, each corresponding to a separate sheet: Context (B), Rule Book (C), and Shots (D). 
At the beginning of this labeling process, the user has only a vague idea of what they want to label and will continuously refine that idea. 
Each time the prompt is revised, it reflects an evolution of their understanding and approach to the labeling task.}
    \label{fig:step-1}
\end{figure*}

\begin{figure*}[t]
    \centering
    \includegraphics[width=0.99\linewidth]{Figures/step-2.jpg}\Description{This is Step 2 described in Figure 1. Users can either random or sequential sample data from the Dataset tab to the Working Data Sample tab. In the Working Data Sample, users can check “Keep it in the next data sample” for data instances that users want to remain in the Working Data Sample tab during sampling.}
    \caption{The overview of step 2 of the data labeling process, sample or resample a subset.
The full dataset is often too large for the user to thoroughly review, so sampling a subset is necessary.
%Labeling only a subset, rather than the entire dataset, is necessary because 
%Additionally, labeling the entire dataset iteratively would be prohibitively expensive. 
In this step, the user can (1) randomly or (2) sequentially sample data from the Dataset (A) sheet.
}
    \label{fig:system-interface-step-2}
\end{figure*}

\begin{figure*}[t]
    \centering
    \includegraphics[width=0.99\linewidth]{Figures/step-3.jpg}\Description{This is Step 3 and 4 described in Figure 1. After clicking Start Annotation, the results including LLM label and LLM explanation will be stored in a new tab. Users will review the data instances and LLM labels, by checking agree or providing their own labels. They also select the Gold Shot data instance to be added to Shots tab for later GPT to learn from. After verification, they can refine their prompt as Step 1 mentioned.}
    \caption{The overview of steps 3 and 4 of the data labeling process. After finalizing the prompt (Step 1) and sampling data instances (Step 2), in Step 3, the user clicks the ``Start Annotation'' button in the sidebar to annotate all instances in the Working Data Sample sheet. \system creates a new sheet, Task 1 (G), to store the data and labels of this labeling task, and also creates a new row for Task 1 in the Task Dashboard sheet. Then, in Step 4, the user can review the outcomes and refine the prompt accordingly (Step 1).
}
    \label{fig:system-interface-step-3-4-kenneth}
\end{figure*}

Users interact with \system to craft a prompt, use it to instruct the LLM in labeling data, review the results, and then revise the prompt through an iterative process. 
To demonstrate the users' workflow, we present a scenario where a user wants to employ \system to label a collection of tweets related to COVID with a 5-point sentiment scale, ranging from Very Negative (1) to Very Positive (5).
The goal is to analyze the sentiment of Twitter (now X) users toward COVID, with an emphasis on ensuring that the classification of each tweet reflects the user's own judgment.
In this case, the LLM's labels should align with the user's assessment of what is positive or negative, as well as the intensity of sentiment, rather than following an ``objective'' standard.
%In other words, the LLM's labels should align with the user's personal perception of the topic rather than adhering to an ``objective'' standard.\kenneth{This is not very accurate hmmm. Might need to edit later.}

%\begin{enumerate}
%    \item 
%\end{enumerate}

\begin{itemize}
   
\item 
\textbf{Step 0: Load and Index the Dataset.}
To begin using \system, the user opens a new Google Sheets document and activates the \system add-on. 
The system automatically sets up the necessary tabs, and the add-on interface appears on the right side of the spreadsheet (Figure~\ref{fig:system-interface-ui-kenneth}). 
The user then imports their data instances into the Dataset sheet, with the text of each tweet placed in the Data Instance column. 
The user must specify a Group ID for each instance. 
If the data are not sequential or grouped, they can assign unique Group IDs using Google Sheets' automatic numbering function.\footnote{\system is designed to accommodate single and grouped data instances within a Group ID. For tasks like sentiment analysis, each data instance is treated separately under its unique Group ID. For tasks that require contextual information, such as annotating text segments in an academic abstract (\eg, CODA-19~\cite{huang-etal-2020-coda}), \system can combine all data instances under the same Group ID into a single request to the LLM model. This flexibility allows the system to support different data instance formats based on user requirements.} 
Once the data is entered, the user clicks the ``Index Data ID'' button in the sidebar, and \system automatically assigns unique data IDs to each instance in the ``Data ID'' column.

\item 
\textbf{Step 1: Compose/Refine the Prompt (Figure~\ref{fig:step-1}).}
%Step 1: compose the promot using things. 
%Uses know a vague idea what they want and will keep revise that idea. But you need to write something down. When it comes to load data, spreadsheet is great!
This is the most critical step, where the user composes and refines prompts for the LLM to label the data. 
In \system, the prompt consists of three parts, each corresponding to a separate sheet: (1) Context, (2) Rule Book, and (3) Shots. 
Figure~\ref{fig:step-1} provides an overview of each sheet.
\begin{enumerate}

\item 
In the \textbf{Context} sheet, the user answers questions that describe the context of the data annotation task, such as the purpose of the annotation and the source of the data, to provide task-specific context for the LLM.

\item
In the \textbf{Rule Book} sheet, the user adds annotation labels along with their definitions. Providing content for both the Context and Rule Book sheets is mandatory, as the LLM requires this information in the prompt to function effectively.

\item
In the \textbf{Shots} sheet, the user adds data instances along with their corresponding gold labels, which serve as examples to help the LLM learn. While adding examples to the Shots sheet is optional during the first iteration---since the user may not yet have a well-defined gold standard for labeling---more examples can be identified as the user reviews data. These examples can be manually added or generated using \system's function (see Step 4).
\end{enumerate}
It is important to note that at the beginning of this labeling process, the user has only a vague idea of what they want to label and will continuously refine that idea. 
Each time the prompt is revised, it reflects an evolution of their understanding and approach to the labeling task.


\item 
\textbf{Step 2: Sample/Resample a Subset (Figure~\ref{fig:system-interface-step-2}).}
Next, the user employs \system to sample a subset of data for labeling. 
Labeling only a subset, rather than the entire dataset, is necessary because the full dataset is too large for the user to thoroughly review the LLM's results. 
Additionally, labeling the entire dataset iteratively would be prohibitively expensive. 
In this step, the user can (1) randomly or (2) sequentially sample data from the Dataset sheet into the Working Data Sample sheet:

\begin{itemize}

\item 
For a \textbf{Random Sample}, the user enters any whole number between 1 and the total number of group IDs in the dataset.
\system will then randomly select that number of groups and copy them into the Working Data Sample sheet. 

\item
In \textbf{Sequential Sample}, the user specifies a range of group IDs from the Dataset sheet, and \system will import the data instances from the selected range into the Working Data Sample sheet.
This feature allows users to process their data instances sequentially in batches, which is especially useful when the data instances have a sequential relationship, such as sentences within the same document.


%The purpose of this feature is to enable users to process their data instances sequentially in batches, making their work more manageable and easier to track.
%\kenneth{Mayeb say a few words on why we need this.}\steven{done.}

\end{itemize}

Once sampling begins, all previously existing data in the Working Data Sample sheet will be removed, except for instances marked as ``Keep it in the next data sample'' (Figure~\ref{fig:system-interface-step-2}). 
Only the data in the Working Data Sample sheet will be labeled by the LLM when the ``Start Annotation'' button is clicked in Step 3.

\item 
\textbf{Step 3: Use the Prompt to Instruct the LLM to Label the Data Sample (Figure~\ref{fig:system-interface-step-3-4-kenneth}).}
After finalizing the three prompt sheets---Context, Rule Book, and Shots---in Step 1 and sampling data instances in Step 2, the user clicks the ``Start Annotation'' button in the sidebar to annotate all instances in the Working Data Sample sheet (Figure~\ref{fig:system-interface-step-3-4-kenneth}).
\system creates a new sheet, Task 1 (Figure~\ref{fig:system-interface-step-3-4-kenneth}), to store the data and labels of this labeling task, and also creates a new row for Task 1 in the Task Dashboard sheet.

In the background, \system first combines the information in Context, Rule Book, and Shots sheets into a prompt (see Section~\ref{sec:implementation} for details).
%gathers the questions and answers from the Context sheet and feeds them into GPT-4 to generate an instruction prompt. 
%This prompt is then combined with the rules and provided gold shots to create the final annotation prompt. 
For each data group (\ie, data instances with the same Group ID), \system sends a request to the LLM using this prompt for annotation.
After receiving the LLM's output, the system parses the results and updates the Task 1 sheet with the annotated outcome for each instance. 
In our implementation, the LLM is always asked to provide explanations for its labels, though the user can decide whether to display these explanations in the annotation results.
%In our user study, we also explored the impact of showing the LLM's explanations to users.



%Step 3: Send it to LLM to label. System creat a new tab; you can navigate tasks using dashboard. Then you re

\item 
\textbf{Step 4: Observe, then Revise the Prompt (Figure~\ref{fig:system-interface-step-3-4-kenneth}).}
The labeling results are saved to the Task 1 sheet (Figure~\ref{fig:system-interface-step-3-4-kenneth}), where the user can manually verify the LLM's labels.
The user can review as many or as few data instances as they wish to develop a better understanding of the labeling task and the dataset. 
Based on this evolving understanding, they can refine the prompt by modifying the Context, Rule Book, and Shots sheets accordingly.

If the user disagrees with any of the labels, they can assign a new label to the data instance under the ``Human Label'' column. 
If the user identifies good examples, they can check the ``Gold Shot'' checkboxes. 
After selecting enough good examples, the user can click the ``Add Shots'' button in the sidebar to add these examples to the Shots sheet (Figure~\ref{fig:system-interface-step-3-4-kenneth}).
Like in other sheets, if the user wants certain data instances to be re-annotated in the next round, they can check the ``Keep it in the next data sample'' checkboxes. 
This will ensure that those instances are not removed during the next sampling process, allowing the user to observe whether the LLM's behavior changes over iterations.


\end{itemize}

When using \system, the user moves through Steps 1, 2, 3, and 4, and then returns to step 1 in an iterative process until they are satisfied with the LLM's labels.








%\subsubsection{Step 0: Initial Setups}


%\subsubsection{Step 1: Compose/Refine the Prompt}

%\subsubsection{Step 2: Sample/Resample a random subset}

%\subsubsection{Step 3: Use the Prompt to Instruct the LLM to Label the Data Sample}

%\subsubsection{Step 4: Observe, and Revise the Prompt}


\subsection{Implementation Details\label{sec:implementation}}
%\kenneth{TODO: Here we mention (1) what framework you used to implement Google Sheets add-on, (2) how do you convert Spreadsheet's content into a prompt, and (3) what LLM (which version exactly) you used and how did you send request (batch? or each data instance is one request?)--- Maybe talk about latency issue here a bit.}


\paragraph{Developing Google Sheets Add-On.}
%\kenneth{How do people built Google Sheets add-ons? Did we use an web server? Where do we store our data?}\steven{done}
\sloppy
\system utilized Google Sheets as its main platform, leveraging the convenience and functionality of its spreadsheet capability. The Google Sheets add-on was implemented in Google App Script, with Google Cloud Service serving as a back-end to store all action logging files. User-specific data, such as OpenAI information, was securely stored in user properties tied to individual email accounts, ensuring privacy protection. 

\paragraph{Converting a Spreadsheet's Content into a Prompt.}
Once users click on the ``Start Annotation'' button (Figure~\ref{fig:system-interface-step-3-4-kenneth}), \system will first collect all questions and answers from the ``Context'' tab and send a request to GPT-4o to generate an instructional prompt (Table~\ref{tab:instruction-prompt}). Next, \system will merge this generated prompt with rules and definitions from ``Rule Book'' and available gold standard labels from the ``Shots'' tab to create an annotation prompt (Table~\ref{tab:main-prompt} and Table~\ref{tab:main-multi-prompt}). Finally, \system will use this prompt to annotate all data instances. 

\paragraph{Interacting with the LLM through an API}
In this paper, we utilized OpenAI's \texttt{gpt-4o-2024-05-13} model for our study~\cite{openai2024gpt4o}.
%\kenneth{TODO: Add citation} \steven{done}
Technically, this LLM can be replaced by any other model that offers an API compatible with the ChatGPT-4 specification. 
In our implementation, we group all data instances with the same Group ID and send them in a single API request.
%In our current implementation, we did not batch requests; instead, we sent an individual API request for data instances with the same group ID.
%\kenneth{Is this accurate?}\steven{we sent by group ID}
%Future versions of \system could potentially benefit from batching to reduce latency.













%\subsection{System Design}


%\subsection{System Workflow (\hyperref[fig:system-workflow-fig]{Figure~\ref{fig:system-workflow-fig-v2}})}




% \begin{figure}
%     \centering
%     \includegraphics[width=0.85\linewidth]{Figures/Workflow/workflow-v2.jpeg}
%     \caption{System Workflow}
%     \label{fig:system-workflow-fig}
% \end{figure}












% \subsubsection{Main Procedure}
% The procedure consists of the following steps:
% \begin{itemize}
%     \item \textbf{Step 1 (Import Data): }Users can import their data instances with group ID into the `Dataset' tab and click ``Index Data ID'' to index all data instances.
%     \item \textbf{Step 2 (Answer Task Questions): }Users need to navigate to the `Context' tab to answer questions about the data annotation task they are working on.
%     \item \textbf{Step 3 (Define Labels and Rules): }Navigating to the `Rule Book' tab, users \textbf{have to} add annotation labels with corresponding definitions. 
%     \item \textbf{Step 4 (Add Gold Shots): }Users can add instances with their gold labels if applicable. 
%     \item \textbf{Step 5 (Sample Data): }Users can randomly or sequentially sample data into the `Working Data Sample' tab.
%     \item \textbf{Step 6 (Data Annotation): }After the ``Context'', ``Rule Book'', and ``Shot'' (if applicable) tabs are all settled, users can click on ``Start Annotation'' to annotate all instances sampled in the ``Working Data Sample'' tab. 
%     \item \textbf{Step 7 (Verification): }The annotation results will be saved to a new task tab. Users can start verifying LLM labels. Users who disagree with an LLM label can assign a new human-generated label to the data instance. If users find good examples that can be used for later LLM learning, they can check the ``Gold Shot'' checkboxes. They can also check the ``Keep it in the next data sample'' checkboxes if they want to re-annotate data instances.
%     \item \textbf{Iteration Procedure: }If users are not satisfied with the annotated results, they can modify the answer in the `Context' Tab \textbf{(Step 2)}; modify rules (add/adjust/delete labels or definitions) in the `Rule book' Tab \textbf{(Step 3)}; click ``Add Shots'' to add the selected ``Gold Shots'' to the `Shots' Tab \textbf{(Step 4)}; click ``Add Back'' to add instances back to the `Working Data Sample' Tab for re-annotating in the next iteration. Then, they can re-sample or re-use instances in the `Working Data Sample' tab for the next iteration \textbf{(Step 5)}. In the end, they repeat \textbf{Step 6}.
%     \item \textbf{Completion:} If users are satisfied with the annotation results, they may choose to conclude the task.
% \end{itemize}





% \paragraph{Two Data Instance Formats Handling}
% \system is designed to accommodate single and grouped data instances within a Group ID. For tasks like single-class sentiment analysis, each data instance is treated separately under its unique Group ID. However, for tasks that require contextual information, such as annotating text segments in an academic abstract (e.g., CODA-19~\cite{huang-etal-2020-coda}), \system can combine all data instances under the same Group ID into a single request to the LLM model. This flexibility allows the system to support different data instance formats based on user requirements.
