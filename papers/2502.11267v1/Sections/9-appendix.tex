% \section{\system User Interface}
% \begin{figure}[tbh]
%     \centering
%     \includegraphics[width=0.85\linewidth]{Figures/Interfaces/System-Interface-5.jpeg}
%     \caption{The Task Dashboard tab systematically tracks all iteration records. This includes Task ID, a hyperlink to the specific task tab, timestamp, the prompt used in that iteration, and the total labeling cost.}
%     \label{fig:system-interface-5}
% \end{figure}

% \begin{figure}
%     \centering
%     \includegraphics[width=0.35\linewidth]{Figures/Interfaces/Remove-Clear-figure.jpg}
%     \caption{Remove Unselected Data Instance and Clear Data Instance Features}
%     \label{fig:remove-clear}
% \end{figure}


% \section{Participant Overall ACC and MSE}
% \begin{table}[ht]
    \centering
    \begin{tabular}{lcccc|cccc}
        \hline
        \multicolumn{1}{l} {Eval} & \multicolumn{1}{c}{\multirow{2}{*}{\textbf{ACC}}} & \multicolumn{3}{c|}{\textbf{T-Test over Avg. ACC}}& \multicolumn{1}{c}{\multirow{2}{*}{\textbf{MSE}}} & \multicolumn{3}{c}{\textbf{T-Test over Avg. MSE}}
        \\ \cmidrule(lr){3-5} \cmidrule(lr){7-9} 
        \multicolumn{1}{l} {Condition} & & LN & 50Y & 50N & & LN & 50Y & 50N\\
        \hline
        LY & \textbf{0.573} & .009**   & .007** & .937& \textbf{0.613} & <.001***   & .023* & .033*   \\ \hline
        LN  & 0.526 & -      & .752 & .029*  & 1.000 & -      & .020* & .502   \\ \hline
        50Y  & 0.532 & -      & -    & .033*  & 0.739 & -      & -    & .230   \\ \hline
        50N   & 0.572 & -      & -    & -    &   0.895 & -      & -    & -      \\ \hline
    \end{tabular}
    \caption{The result of single condition Accuracy and MSE. LY has significantly higher accuracy than LN and 50Y conditions; LY has significantly lower MSE than the rest of the conditions. The p-values were calculated by comparing accuracy or MSE across all participants in each condition (*:p<0.05; **:p<0.01; ***:p<0.001; independent t-test.) }
    \label{tab:single-condition-t-test}
\end{table}


% \begin{table}[ht]
    \centering
    \begin{tabular}{lcccc|cccc}
        \hline
        \multicolumn{1}{l} {Eval} & \multicolumn{1}{c}{\multirow{2}{*}{\textbf{ACC}}} & \multicolumn{3}{c|}{\textbf{T-Test over Avg. ACC}}& \multicolumn{1}{c}{\multirow{2}{*}{\textbf{MSE}}} & \multicolumn{3}{c}{\textbf{T-Test over Avg. MSE}}
        \\ \cmidrule(lr){3-5} \cmidrule(lr){7-9} 
        \multicolumn{1}{l} {Condition} & & LY-50N & LN-50Y & LN-50N & & LY-50N & LN-50Y & LN-50N\\
        \hline
        LY-50Y & 0.566 & .441   & .001** & .855      & \textbf{0.556} &  .135  & <.001*** & .025*   \\ \hline
        LY-50N  & \textbf{0.582} & -      & .001** & .473     & 0.682 & -      & .019* & .086   \\ \hline
        LN-50Y  & 0.498 & -      & -    & .033*       & 0.922 & -      & -    & .435   \\ \hline
        LN-50N   & 0.561 & -      & -    & -           & 1.105 & -      & -    & -      \\ \hline
    \end{tabular}
    \caption{The result of mixed condition Accuracy and MSE. LN-50Y has significantly lower accuracy than the rest of the combinations. LY-50Y has a significantly better MSE than participants who have LN conditions. }
    \label{tab:mixed-condition-t-test}
\end{table}



% \begin{table}[h]
    \centering
    \begin{tabular}{lcccc|cccc}
        \hline
        \multicolumn{1}{l} {DSPy} & \multicolumn{1}{c}{\multirow{2}{*}{\textbf{ACC$\uparrow$}}} & \multicolumn{3}{c|}{\textbf{T-Test over Avg. Changed ACC}}& \multicolumn{1}{c}{\multirow{2}{*}{\textbf{MSE$\downarrow$}}} & \multicolumn{3}{c}{\textbf{T-Test over Avg. Changed MSE}}
        \\ \cmidrule(lr){3-5} \cmidrule(lr){7-9} 
        \multicolumn{1}{l} {Changes} & & LY-50N & LN-50Y & LN-50N & & LY-50N & LN-50Y & LN-50N\\
        \hline
        \textbf{Bootstrap}\\
        \cmidrule(lr){1-9}
        LY-50Y & .004 & .335&.083&.011*      & .128 &  .903&.043*&.049*   \\ \hline
        LY-50N  & -.014 & -      & .025 & .109     & .119 & -      & .236&.217   \\ \hline
        LN-50Y  & \underline{.039} & -      & -    & <.001***       & .018 & -      & -    & .893   \\ \hline
        LN-50N   & -.055 & -      & -    & -           & .009 & -      & -    & -      \\ 
        \midrule
        \textbf{Simple Prompt}\\
        \cmidrule(lr){1-9}
        LY-50Y & -.003 & .564&.076&.002**      & .179 &  .045*&.034*&.002**   \\ \hline
        LY-50N  & .009 & -      & .129 & <.001***     & .035 & -      & .630&.219   \\ \hline
        LN-50Y  & \textbf{.041} & -      & -    & <.001***       & .073 & -      & -    & .070   \\ \hline
        LN-50N   & -.095 & -      & -    & -           & \underline{-.092} & -      & -    & -      \\ 
        \midrule
        \textbf{Mipro}\\
        \cmidrule(lr){1-9}
        LY-50Y & .029 & .040*&.826&.010**      & .094 &  .651&.025*&.261   \\ \hline
        LY-50N  & -.014 & -      & .079&.195     & .124 & -      & .026*&.188   \\ \hline
        LN-50Y  & .035 & -      & -    & .015*       & -.048 & -      & -    & .557   \\ \hline
        LN-50N   & -.056 & -      & -    & -           & .003 & -      & -    & -      \\ 
        \midrule
        \textbf{Copro}\\
        \cmidrule(lr){1-9}
        LY-50Y & .021 & .137&.651&.095      & .052 &  .436&.052&.024*   \\ \hline
        LY-50N  & -.006 & -      & .137&.774     & .098 & -      & .044*&.017*   \\ \hline
        LN-50Y  & .031 & -      & -    & .098       & -.046 & -      & -    & .443   \\ \hline
        LN-50N   & -.012 & -      & -    & -           & \textbf{-.104} & -      & -    & -      \\ 
        \hline
    \end{tabular}
    \caption{DSPy t-test over different conditions for each optimizer.\kenneth{Did we even mention Table 5 in the paper????? Maybe just remove it.}}
    \label{tab:dspy-all-conditions-t-test}
\end{table}



\section{\system Task Context Questions}\label{sec:context-question-appendix}
\begin{table*}[h]
\footnotesize
\centering
\begin{tabular}{@{}lll@{}}
\toprule
\textbf{\#} & \textbf{Questionnaire Question} \\ \midrule
Q1 &  \begin{tabular}[c]{@{}l@{}}What is the purpose of annotating this data? Common answers include gaining insight about something, wanting to compare something, \\wanting to create prompts, etc. Try to give us more details about your higher-level goal.\end{tabular} \\ \midrule
Q2 & \begin{tabular}[c]{@{}l@{}}How do you want to use the annotated data? Common answers include further analysis, training an AI model, presenting it to people, \\or using it in some downstream tasks inside some computer system. Try to give us more details about the use cases of the annotated data. \end{tabular} \\ \midrule
Q3 & \begin{tabular}[c]{@{}l@{}}What are these data? Please tell us more about the source and the characteristics of the data. For example, ``These are real-world product \\reviews written by Amazon users. We obtained this dataset by downloading it from Kaggle.'' or ``This is the transcript of interviews of our \\participants. Each interview is about 30 minutes long. The interview is about their experience in creative writing.'' or ``These are tweets \\posted on Twitter between Jan 2024 to March 2024.'' \end{tabular} \\ \midrule
Q4 & \begin{tabular}[c]{@{}l@{}}What is the size of each data instance (each row)? For example, ``Each instance is a tweet.'', ``Each instance is one Amazon product review.'', \\or ``Each instance is one sentence from the interview transcript.'' \end{tabular} \\ \midrule
Q5 & \begin{tabular}[c]{@{}l@{}}Is there anything particular you want us to mention in the prompt? We will add all the context you mentioned in this tab to the prompt \\for LLMs. Please mention anything you want the LLMs to be aware of. \end{tabular} 
\\ \bottomrule
\end{tabular}
\caption{Questions for \system Task Context.}
\label{tab:task-sheet-questions}
% \vspace{-5mm}
\end{table*}

% \begin{table*}
% \small
% \centering
% \begin{tabular}{@{}lll@{}}
% \toprule
% \textbf{\#} & \textbf{Aspect} & \textbf{Questionnaire Question} \\ \midrule
% Q1 & Reach-Out & \begin{tabular}[c]{@{}l@{}}If I have this question, I would reach out to other people, such as friends, family members, colleagues,\\ or experts, to get help. As compared to finding the answers by looking up information by myself with\\ a computer or a smartphone.\\(1) Strongly Disagree  (2) Disagree  (3) Neutral  (4) Agree  (5) Strongly Agree \end{tabular} \\ \midrule

% \\ \bottomrule
% \end{tabular}
% \caption{Questionnaire questions used for \system.}
% \label{tab:study-survey}
% \end{table*}

\section{Post-Study Survey for the user study}\label{sec:post-question-survey}
Table~\ref{tab:study-survey} shows the questions for the post-study survey.
\begin{table*}
\footnotesize
\centering
\begin{tabular}{@{}lll@{}}
\toprule
\textbf{\#} & \textbf{Aspect} & \textbf{Post-Study Survey Question} \\ \midrule
Q1 & Task Experience & \begin{tabular}[c]{@{}l@{}}The annotation task was easy to understand. \\(1) Strongly Disagree  (2) Disagree  (3) Somewhat Disagree (4) Neutral  (5) Somewhat Agree (6) Agree (7) Strongly Agree \end{tabular} \\ \midrule
Q2 & Task Experience & \begin{tabular}[c]{@{}l@{}}The annotation tool is easy to use.\\(1) Strongly Disagree  (2) Disagree  (3) Somewhat Disagree (4) Neutral  (5) Somewhat Agree (6) Agree (7) Strongly Agree \end{tabular} \\ \midrule
Q3 & Task Experience & \begin{tabular}[c]{@{}l@{}}Did you encounter any difficulties while using the system? If yes, please describe the difficulties. \end{tabular} \\ \midrule
Q4 & Task Experience & \begin{tabular}[c]{@{}l@{}}The interface of the annotation system was intuitive. \\(1) Strongly Disagree  (2) Disagree  (3) Somewhat Disagree (4) Neutral  (5) Somewhat Agree (6) Agree (7) Strongly Agree \end{tabular} \\ \midrule
Q5 & Task Experience & \begin{tabular}[c]{@{}l@{}}How satisfied are you with the performance of the system? \\(1) Strongly Disagree  (2) Disagree  (3) Somewhat Disagree (4) Neutral  (5) Somewhat Agree (6) Agree (7) Strongly Agree \end{tabular} \\ \midrule
Q6 & Practice Prompting & \begin{tabular}[c]{@{}l@{}}This tool was helpful in improving my prompt. \\(1) Strongly Disagree  (2) Disagree  (3) Somewhat Disagree (4) Neutral  (5) Somewhat Agree (6) Agree (7) Strongly Agree \end{tabular} \\ \midrule
Q7 & Practice Prompting & \begin{tabular}[c]{@{}l@{}}Using this tool made the process of prompt engineering more efficient. \\(1) Strongly Disagree  (2) Disagree  (3) Somewhat Disagree (4) Neutral  (5) Somewhat Agree (6) Agree (7) Strongly Agree \end{tabular} \\ \midrule
Q8 & Practice Prompting & \begin{tabular}[c]{@{}l@{}}Without this tool, how would you typically approach prompt engineering? \end{tabular} \\ \midrule
Q9 & Practice Prompting & \begin{tabular}[c]{@{}l@{}}How would you compare your prompt engineering process before and after using this tool? \end{tabular} \\ \midrule
Q10 & System Usability & \begin{tabular}[c]{@{}l@{}}What features did you find most useful? \end{tabular} \\ \midrule
Q11 & System Usability & \begin{tabular}[c]{@{}l@{}}What features did you find least useful or problematic? \end{tabular} \\ \midrule
Q12 & System Usability & \begin{tabular}[c]{@{}l@{}}Did you feel the need for any additional features or improvements? If yes, please describe them.\end{tabular} \\ \midrule
Q13 & Overall Feedback & \begin{tabular}[c]{@{}l@{}}What did you like most about the annotation system?\end{tabular} \\ \midrule
Q14 & Overall Feedback & \begin{tabular}[c]{@{}l@{}}What did you like least about the annotation system?\end{tabular} \\ \midrule
Q15 & Overall Feedback & \begin{tabular}[c]{@{}l@{}}Any additional comments or suggestions?\end{tabular} \\ \midrule
Q16 & User Interaction and Behavior & \begin{tabular}[c]{@{}l@{}}Did you find the system responsive to your actions? If yes, please describe them.\end{tabular} \\ \midrule
Q17 & User Interaction and Behavior & \begin{tabular}[c]{@{}l@{}}Were there any delays or lags while performing the tasks? If yes, please describe them\end{tabular} \\ \midrule
Q18 & User Interaction and Behavior & \begin{tabular}[c]{@{}l@{}}Did you use any help or support features provided by the system? If yes, were they helpful?\end{tabular} \\ \midrule
Q19 & Efficiency and Effectiveness & \begin{tabular}[c]{@{}l@{}}I completed the annotation tasks efficiently. \\(1) Strongly Disagree  (2) Disagree  (3) Somewhat Disagree (4) Neutral  (5) Somewhat Agree (6) Agree (7) Strongly Agree \end{tabular} \\ \midrule
Q20 & Efficiency and Effectiveness & \begin{tabular}[c]{@{}l@{}}Did the system help you complete the tasks more efficiently? If yes, please explain how.\end{tabular} \\ \midrule
Q21 & Future Use & \begin{tabular}[c]{@{}l@{}}Would you be interested in using this annotation system in your regular work or study? If no, please explain why.\end{tabular} \\ \midrule
Q22 & Future Use & \begin{tabular}[c]{@{}l@{}}Do you have any suggestions for making the system more suitable for your needs?\end{tabular}
\\ \bottomrule
\end{tabular}
\caption{Post-Study Survey questions used for \system. The survey is consisted of twenty-two questions, including seven Likert scale ratings and fifteen free-text responses.}
\label{tab:study-survey}
% \vspace{-5mm}
\end{table*}

% \begin{table*}
% \small
% \centering
% \begin{tabular}{@{}lll@{}}
% \toprule
% \textbf{\#} & \textbf{Aspect} & \textbf{Questionnaire Question} \\ \midrule
% Q1 & Reach-Out & \begin{tabular}[c]{@{}l@{}}If I have this question, I would reach out to other people, such as friends, family members, colleagues,\\ or experts, to get help. As compared to finding the answers by looking up information by myself with\\ a computer or a smartphone.\\(1) Strongly Disagree  (2) Disagree  (3) Neutral  (4) Agree  (5) Strongly Agree \end{tabular} \\ \midrule

% \\ \bottomrule
% \end{tabular}
% \caption{Questionnaire questions used for \system.}
% \label{tab:study-survey}
% \end{table*}

\section{Prompt}
Table~\ref{tab:instruction-prompt} shows the prompt \system used for generating instructional prompt.
\begin{table*}[h]
\centering
    \begin{tabular}{@{}p{\textwidth}@{}}
        \hline
        \textbf{Instructional Prompt Creation Prompt} \\
        Here are questions and corresponding answers for a task description.\\ \\

        ```\\
        \textbf{Question:} [What is the purpose of annotating this data? Common answers include gaining insight about something, wanting to compare something, wanting to create prompts, etc. Try to give us more details about your higher-level goal.] \textbf{Answer:} [Answer 1...] \\
        \textbf{Question:} [How do you want to use the annotated data? Common answers include further analysis, training an AI model, presenting it to people, or using it in some downstream tasks inside some computer system. Try to give us more details about the use cases of the annotated data.] \textbf{Answer:} [Answer 2...] \\
        \textbf{Question:} [What are these data? Please tell us more about the source and the characteristics of the data. For example, ``These are real-world product reviews written by Amazon users. We obtained this dataset by downloading it from Kaggle.'' or ``This is the transcript of interviews of our participants. Each interview is about 30 minutes long. The interview is about their experience in creative writing.'' or ``These are tweets posted on Twitter between Jan 2024 to March 2024.''] \textbf{Answer:} [Answer 3...] \\
        \textbf{Question:} [What is the size of each data instance (each row)? For example, ``Each instance is a tweet.'', ``Each instance is one Amazon product review.'', or ``Each instance is one sentence from the interview transcript.''] \textbf{Answer:} [Answer 4...] \\
        \textbf{Question:} [Is there anything particular you want us to mention in the prompt? We will add all the context you mentioned in this tab to the prompt for LLMs. Please mention anything you want the LLMs to be aware of.] \textbf{Answer:} [Answer 5...] \\
        \textbf{Question:} [Is it a single-class or multi-class labeling task? [required]] \textbf{Answer:} [single-class]
        ......\\
        '''\\
        \\
        Based on task questions and answers, help me generate a concrete DETAILED task instruction.\\
        Provide Instruction ONLY!\\
        DO NOT ADD ANY ADDITIONAL INFORMATION NOT INCLUDE IN THE PREVIOUS Q and A!!!\\
        This Instruction is generated for LLM!\\
        \hline
    \end{tabular}
    \caption{Prompt to generate instructional prompts in \system. Participants are required to answer the first five questions to provide context for the task. The last question and its answer were intentionally fixed, as our study focuses on a single-class labeling task. In the future, participants will be allowed to answer this question.}
    \label{tab:instruction-prompt}
\end{table*}

%\steven{added instructional prompt generation prompt}

Table~\ref{tab:main-prompt} and Table~\ref{tab:main-multi-prompt} show the prompts \system used for the annotation process.
\begin{table*}[h]
\centering
    \begin{tabular}{@{}p{\textwidth}@{}}
        \hline
        \textbf{Prompt for Annotation Process} \\
        \{\textbf{\$Instructional\_Prompt}\}\\ \\
        Please ensure each label adheres to its following rules and regulations.\\
        \\
        Below are the descriptions of various labels. Please assign the most appropriate label to each description provided.\\
        Label `\{\textbf{\$label\_1}\}': Assign this label if the tweet meets any of the following criteria:\\
        \{\textbf{\$rule\_1}\}\\ 
        Label `\{\textbf{\$label\_2}\}': Assign this label if the tweet meets any of the following criteria:\\
        \{\textbf{\$rule\_2}\}\\ 
        Label `\{\textbf{\$label\_3}\}': Assign this label if the tweet meets any of the following criteria:\\
        \{\textbf{\$rule\_3}\}\\ 
        Label `\{\textbf{\$label\_4}\}': Assign this label if the tweet meets any of the following criteria:\\
        \{\textbf{\$rule\_4}\}\\
        Label `\{\textbf{\$label\_5}\}': Assign this label if the tweet meets any of the following criteria:\\
        \{\textbf{\$rule\_5}\}\\
        ......\\ \\
        Please refer to the following Shots (Examples for LLMs to Learn) for annotation tasks, where each instance is corresponded with a label.\\
        Example:```{\textbf{\$instance\_1}}''' $=>$ Label:```{\textbf{\$label\_x}}'''\\
        Example:```{\textbf{\$instance\_2}}''' $=>$ Label:```{\textbf{\$label\_x}}'''\\
        Example:```{\textbf{\$instance\_3}}''' $=>$ Label:```{\textbf{\$label\_x}}'''\\
        ......\\ \\
        Output Format\\
        Your output should consist of two sections: ANSWER and EXPLANATION.\\
        ANSWER: Label: []\\
        EXPLANATION: Provide a brief explanation for your label choice.\\
        The following is the data instance need to be annotated:\\
        data-instance: \{\textbf{\$data\_instance}\}\\
        \hline
    \end{tabular}
    \caption{Prompt for the annotation process in \system for groups which have only one data instance.}
    \label{tab:main-prompt}
\end{table*}

%\steven{added prompt for single instance like Twitter sentiment task}

\begin{table*}[h]
\centering
    \begin{tabular}{@{}p{\textwidth}@{}}
        \hline
        \textbf{Prompt for Annotation Process} \\
        \{\textbf{\$Instructional\_Prompt}\}\\ \\
        Please ensure each label adheres to its following rules and regulations.\\
        \\
        Below are the descriptions of various labels. Please assign the most appropriate label to each description provided.\\
        Label `\{\textbf{\$label\_1}\}': Assign this label if the tweet meets any of the following criteria:\\
        \{\textbf{\$rule\_1}\}\\ 
        Label `\{\textbf{\$label\_2}\}': Assign this label if the tweet meets any of the following criteria:\\
        \{\textbf{\$rule\_2}\}\\ 
        Label `\{\textbf{\$label\_3}\}': Assign this label if the tweet meets any of the following criteria:\\
        \{\textbf{\$rule\_3}\}\\ 
        Label `\{\textbf{\$label\_4}\}': Assign this label if the tweet meets any of the following criteria:\\
        \{\textbf{\$rule\_4}\}\\
        Label `\{\textbf{\$label\_5}\}': Assign this label if the tweet meets any of the following criteria:\\
        \{\textbf{\$rule\_5}\}\\
        ......\\ \\
        Please refer to the following Shots (Examples for LLMs to Learn) for annotation tasks, where each instance is corresponded with a label.\\
        Example:```{\textbf{\$instance\_1}}''' $=>$ Label:```{\textbf{\$label\_x}}'''\\
        Example:```{\textbf{\$instance\_2}}''' $=>$ Label:```{\textbf{\$label\_x}}'''\\
        Example:```{\textbf{\$instance\_3}}''' $=>$ Label:```{\textbf{\$label\_x}}'''\\
        ......\\ \\
        Output Format\\
        For each labeled data instance, your output should consist of two sections: ANSWER and EXPLANATION, with the data instance id. Each label fragment should be divided by ``======''\\
        ANSWER: Label: []\\
        EXPLANATION: Provide a brief explanation for your label choice.\\
        The following are data instances from a group that need to be annotated:\\
        data-instance-1: \{\textbf{\$data\_instance\_1}\}\\
        data-instance-2: \{\textbf{\$data\_instance\_2}\}\\
        data-instance-3: \{\textbf{\$data\_instance\_3}\}\\
        ......\\
        \hline
    \end{tabular}
    \caption{Prompt for the annotation process in \system for groups which have multiple data instances.}
    \label{tab:main-multi-prompt}
\end{table*}

%\steven{added prompt for tasks require context like CODA-19}



\section{Supplemental System Figures}
Figure~\ref{fig:task-dashboard-new} shows the task dashboard layout in \system.
\begin{figure*}
    \centering
    \includegraphics[width=0.90\linewidth]{Figures/Interfaces/New/task-description.png}\Description{The task dashboard for PromptingSheet. This spreadsheet records key details, including the Task ID, Task Tab, Task Creation Time, Prompt Used for the Task, and Total Label Cost.}
    \caption{The Task Dashboard tab provides an overview of all labeling tasks. This spreadsheet records key details, including the Task ID, Task Tab, Task Creation Time, Prompt Used for the Task, and Total Label Cost. Users can click the hyperlink in the Task Tab column to access the corresponding task tab for more detailed information.}
    \label{fig:task-dashboard-new}
\end{figure*}

%\steven{response to task dashboard question}

\section{Supplemental ACC and MSE Figure}
Figure~\ref{fig:average-acc-mse-llm-instances-performance-new} and Figure~\ref{fig:average-acc-mse-llm-explanation-performance-new}  present the average ACC and MSE for participants under different conditions: comparing 10 vs. 50 instances per iteration and assessing the impact of LLM explanations.
\begin{figure*}
    \centering
    \begin{subfigure}{0.48\textwidth}
     \includegraphics[width=\linewidth]{Figures/Results/New-Pure-Participant/10-vs-50-instance/ACC_new_10_instance.png}\Description{This subplot shows the average ACC for participants reviewing 10 instances. The plot includes shaded regions representing the standard deviation and scatter points representing the data of individual participants.}
    \caption{ACC with 10 instances.}
    \label{fig:new-lineplot-acc-10}
  \end{subfigure}
  \hfill
  \begin{subfigure}{0.48\textwidth}
    \includegraphics[width=\linewidth]{Figures/Results/New-Pure-Participant/10-vs-50-instance/ACC_new_50_instance.png}\Description{This subplot shows the average ACC for participants reviewing 50 instances. The plot includes shaded regions representing the standard deviation and scatter points representing the data of individual participants.}
    \caption{ACC with 50 instances.}
    \label{fig:new-lineplot-acc-50}
  \end{subfigure}
  \begin{subfigure}{0.48\textwidth}
     \includegraphics[width=\linewidth]{Figures/Results/New-Pure-Participant/10-vs-50-instance/MSE_new_10_instance.png}\Description{This subplot shows the average MSE for participants reviewing 10 instances. The plot includes shaded regions representing the standard deviation and scatter points representing the data of individual participants.}
    \caption{MSE with 10 instances.}
    \label{fig:new-lineplot-mse-10}
  \end{subfigure}
  \hfill
  \begin{subfigure}{0.48\textwidth}
    \includegraphics[width=\linewidth]{Figures/Results/New-Pure-Participant/10-vs-50-instance/MSE_new_50_instance.png}\Description{This subplot shows the average MSE for participants reviewing 50 instances. The plot includes shaded regions representing the standard deviation and scatter points representing the data of individual participants.}
    \caption{MSE with 50 instances.}
    \label{fig:new-lineplot-mse-50}
  \end{subfigure}
  \caption{The average ACC and MSE for participants reviewing 10 or 50 instances per iteration are presented in four subfigures, comparing ACC and MSE between the two conditions.}
  \label{fig:average-acc-mse-llm-instances-performance-new}
\end{figure*}

%\steven{new: ACC comparison between 10 and 50 instances}

\begin{figure*}
    \centering
    \begin{subfigure}{0.48\textwidth}
     \includegraphics[width=\linewidth]{Figures/Results/New-Pure-Participant/no-explanation-vs-explanation/ACC_new_no_explanation.png}\Description{This subplot shows the average ACC for participants without access to LLM explanation. The plot includes shaded regions representing the standard deviation and scatter points representing the data of individual participants.}
    \caption{ACC without LLM Explanation.}
    \label{fig:new-lineplot-acc-no-exp}
  \end{subfigure}
  \hfill
  \begin{subfigure}{0.48\textwidth}
    \includegraphics[width=\linewidth]{Figures/Results/New-Pure-Participant/no-explanation-vs-explanation/ACC_new_explanation.png}\Description{This subplot shows the average ACC for participants with access to LLM explanation. The plot includes shaded regions representing the standard deviation and scatter points representing the data of individual participants.}
    \caption{ACC with LLM Explanation.}
    \label{fig:new-lineplot-acc-exp}
  \end{subfigure}
  \begin{subfigure}{0.48\textwidth}
     \includegraphics[width=\linewidth]{Figures/Results/New-Pure-Participant/no-explanation-vs-explanation/MSE_new_no_explanation.png}\Description{This subplot shows the average MSE for participants without access to LLM explanation. The plot includes shaded regions representing the standard deviation and scatter points representing the data of individual participants.}
    \caption{MSE without LLM Explanation.}
    \label{fig:new-lineplot-mse-no-exp}
  \end{subfigure}
  \hfill
  \begin{subfigure}{0.48\textwidth}
    \includegraphics[width=\linewidth]{Figures/Results/New-Pure-Participant/no-explanation-vs-explanation/MSE_new_explanation.png}\Description{This subplot shows the average MSE for participants with access to LLM explanation. The plot includes shaded regions representing the standard deviation and scatter points representing the data of individual participants.}
    \caption{MSE with LLM Explanation.}
    \label{fig:new-lineplot-mse-exp}
  \end{subfigure}
  \caption{The average ACC and MSE for participants with or without LLM Explanation are presented in four subfigures, comparing ACC and MSE between the two conditions.}
  \label{fig:average-acc-mse-llm-explanation-performance-new}
\end{figure*}

%\steven{new: ACC comparison between no explanation and explanation}

% \section{Survey Study for Understanding How Individuals Use LLMs in Data Annotation}
% \begin{table*}
\footnotesize
\centering
\begin{tabular}{@{}lll@{}}
\toprule
\textbf{\#} & \textbf{Aspect} & \textbf{Questionnaire Question} \\ \midrule
Q1 & Basic Information & \begin{tabular}[c]{@{}l@{}}What is the best description(s) for your role in the data labeling process? \\(1) Data Scientist (2) Researcher
(3) Project Manager
(4) Machine Learning Engineer
(5) Software Engineer
(6) Student\\
(7) Crowd Workers
(8) Other\end{tabular} \\ \midrule
Q2 & Basic Information & \begin{tabular}[c]{@{}l@{}}The annotation tool is easy to use.\\
(1) Technology/IT
(2) Finance
(3) Education
(4) Manufacturing
(5) Retail/E-commerce
(6) Government/Public Sector\\
(7) Research and Development
(8) Other
\end{tabular} \\ \midrule
Q3 & Basic Information & \begin{tabular}[c]{@{}l@{}}What is your years of experience in data labeling?\\
(1) Less than 1 year
(2) 1-3 years
(3) 4-6 years
(4) More than 6 years
 \end{tabular} \\ \midrule
Q4 & Basic Information & \begin{tabular}[c]{@{}l@{}}What is your years of experience in using LLMs to do data labeling?
 \\
(1) Less than a month
(2) 1-3 months
(3) 4 months to 1 year
(4) More than 1 year
 \end{tabular} \\ \midrule
Q5 & Basic Information & \begin{tabular}[c]{@{}l@{}}What types of data do you primarily work with in your role? \\
(1) Text
(2) Images
(3) Audio
(4) Video
(5) Sensor data (e.g., IoT)
(6) Multimodal data
(7) Other
\end{tabular} \\ \midrule
Q6 & Basic Information & \begin{tabular}[c]{@{}l@{}}How often do you use LLMs to assist any step in annotation tasks (e.g. for prompting, labeling, verification, etc)? \\
(1) Daily
(2) Weekly
(3) Bi-Weekly
(4) Monthly
(5) Never
\end{tabular} \\ \midrule
Q7 & Basic Information & \begin{tabular}[c]{@{}l@{}}What is the following best describe your process of using LLMs to data labeling?
 \\
(1) I use LLMs for an initial pass of labeling, then review and revise the labels myself.\\
(2) I manually label the data (or obtain labels from others) first and use LLMs to validate or review the labels.\\
(3) I consult LLMs for ambiguous cases or to resolve disagreements.\\
(4) I use LLMs for data pre-processing, such as transforming unstructured text into structured data.\\
(5) I use LLMs to generate task-based examples for creating additional data.\\
(6) I use LLMs to provide explanations for some labeled data.\\
(7) Other\\
 \end{tabular} \\ \midrule
Q8 & Creating Prompting & \begin{tabular}[c]{@{}l@{}}Please provide a brief description of your typical annotation process with an LLM.\\
\textit{Walk us through your steps, from initializing the task to reviewing and finalizing annotations. Include any strategies you use}\\ \textit{to ensure quality or consistency.}\end{tabular} \\ \midrule
Q9 & Creating Prompting & \begin{tabular}[c]{@{}l@{}}How do you typically initialize the annotation process with an LLM?\\
\textit{For example, do you start with predefined prompts, or do you ask the LLM for suggestions?}\end{tabular} \\ \midrule
Q10 & Creating Prompting & \begin{tabular}[c]{@{}l@{}}Please describe your process for revising prompts when using an LLM for annotation.\\
\textit{For example, how do you refine or adjust prompts to achieve the desired response? How do you handle bias or inconsistencies}\\ \textit{introduced by the LLM in your annotations?} \end{tabular} \\ \midrule
Q11 & Creating Prompting & \begin{tabular}[c]{@{}l@{}}How much time do you spend crafting and refining prompts for each annotation task?\\
(1) Less than 5 minutes
(2) 5-10 minutes
(3) 10-15 minutes
(4) More than 15 minutes
\end{tabular} \\ \midrule
Q12 & Process Prompting Response & \begin{tabular}[c]{@{}l@{}}Which of the following best describes your manual review efforts when working with LLM-generated labels?\\
(1) I rely on all LLM-generated labels.\\
(2) I only review labels for randomly sampled subset/ critical tasks/ specific use /ambiguous cases.\\
(3) I cross-check LLM-generated labels against established guidelines or standards.\\
(4) I compare LLM-generated labels with expert-provided labels for validation.\\
(5) I manually verify all LLM-generated labels for accuracy and consistency.\\
(6) I use LLM-generated labels as a starting point and fully revise them as needed.\\
(7) I collaborate with domain experts to evaluate and refine LLM-generated labels.\\
(8) Other\\
\end{tabular} \\ \midrule
Q13 & Process Prompting Response & \begin{tabular}[c]{@{}l@{}}When the label explanation is provided, please describe how you use LLM's explanation in your annotation tasks.
\end{tabular} \\ \midrule
Q14 & Process Prompting Response & \begin{tabular}[c]{@{}l@{}}As of November 2024, how reliable do you find LLMs for generating accurate annotations?\\
(1) Very reliable
(2) Somewhat Reliable
(3) Neutral
(4) Somewhat Unreliable
(5) Very unreliable
\end{tabular} \\ \midrule
Q15 & Process Prompting Response & \begin{tabular}[c]{@{}l@{}}How has using an LLM impacted your annotation speed?\\
(1) Greatly increased
(2) Slightly increased
(3) No change
(4) Slightly decreased
(5) Greatly decreased
\end{tabular} \\ \midrule
Q16 & Process Prompting Response & \begin{tabular}[c]{@{}l@{}}What impact has using an LLM had on the quality of your annotations?\\
(1) Greatly increased
(2) Slightly increased
(3) No change
(4) Slightly decreased
(5) Greatly decreased
\end{tabular} \\ \midrule
Q17 & Annotation Validation and Gold Labels & \begin{tabular}[c]{@{}l@{}}How do you validate LLM-assisted annotations?
\end{tabular} \\ \midrule
Q18 & Annotation Validation and Gold Labels& \begin{tabular}[c]{@{}l@{}}How often do you have gold-standard labels (either by you, other people, or experts) at the beginning of the annotation\\ process?\\
(1) 0–20\%
(2) 20–40\%
(3) 40–60\%
(4) 60–80\%
(5) 80–100\%
\end{tabular} \\ \midrule
Q19 & Annotation Validation and Gold Labels & \begin{tabular}[c]{@{}l@{}}What is the source of the gold-standard labels? If the gold-standard labels are not available or limited, how do you create\\ gold-standard labels? \end{tabular} \\ \midrule
Q20 & Annotation Validation and Gold Labels & \begin{tabular}[c]{@{}l@{}}How do you use gold-standard labels during the annotation process besides validation?\end{tabular} \\ 
\\ \bottomrule
\end{tabular}
\caption{Survey questions used for understanding how individuals use LLMs in data annotation.}
\label{tab:study-survey-annotation}
% \vspace{-5mm}
\end{table*}



\section{Questionnaires for Participants' LLM Background}\label{sec:appendix=participant-background}
\begin{table*}[h]
\footnotesize
\centering
\begin{tabular}{@{}lll@{}}
\toprule
\textbf{\#} & \textbf{Questionnaire Question} \\ \midrule
Q1 &  \begin{tabular}[c]{@{}l@{}}How would you rate your familiarity with using LLMs on a scale from 1 (not familiar) to 5 (very familiar)?\end{tabular} \\ \midrule
Q2 & \begin{tabular}[c]{@{}l@{}}Which of the following best describes your understanding of how LLMs work?\\ 
(1) I don’t understand at all.\\
(2) I have a basic understanding (e.g., they generate text based on input).\\
(3) I understand the general principles (e.g., machine learning, large datasets).\\
(4) I have a detailed understanding (e.g., specific architectures, training methods).
\end{tabular} \\ \midrule
Q3 & \begin{tabular}[c]{@{}l@{}}What are your years of experience in using LLMs? \\
(1) Less than a month\\
(2) 1-3 months\\
(3) 4 months to 1 year\\
(4) More than 1 year
\end{tabular} \\ \midrule
Q4 & \begin{tabular}[c]{@{}l@{}}How often do you use LLMs?\\
(1) Daily\\
(2) Weekly\\
(3) Bi-Weekly\\
(4) Monthly\\
(5) Never
\end{tabular} \\ \midrule
Q5 & \begin{tabular}[c]{@{}l@{}}How much time do you typically spend interacting with LLMs in a single session?\\
(1) Less than 5 minutes\\
(2) 5–15 minutes\\
(3) 15–30 minutes\\
(4) More than 30 minutes
\end{tabular} \\ \midrule
Q6 & \begin{tabular}[c]{@{}l@{}}In which contexts have you used LLMs? (Select all that apply)\\
(1) Academic research\\
(2) Professional work\\
(3) Personal projects\\
(4) Entertainment\\
(5) Other
\end{tabular} \\ \midrule
Q7 & \begin{tabular}[c]{@{}l@{}}For which tasks do you use LLMs? (Select all that apply)\\
(1) Data Labeling\\
(2) General Q\&A or research\\
(3) Writing assistance (e.g., drafting emails, reports)\\
(4) Programming or debugging\\
(5) Data analysis or visualization\\
(6) Creative tasks (e.g., storytelling, idea generation)\\
(7) Other
\end{tabular} \\ \midrule
Q8 & \begin{tabular}[c]{@{}l@{}}How confident are you in your ability to craft effective prompts for LLMs on a scale from 1 (beginner) to 5 (expert)?
\end{tabular} \\ \midrule
Q9 & \begin{tabular}[c]{@{}l@{}}How proficient do you feel in troubleshooting when an LLM generates incorrect or irrelevant responses on a scale from 1 (not proficient) to 5 \\(extremely proficient)?
\end{tabular} \\ \midrule
Q10 & \begin{tabular}[c]{@{}l@{}}Do you use any advanced techniques like prompt engineering or API integration with LLMs? If yes, please describe.
\end{tabular}
\\ \bottomrule
\end{tabular}
\caption{Questionnaire questions used for participants' LLM background.}
\label{tab:participants-llm-background-survey}
% \vspace{-5mm}
\end{table*}


\section{Post-Study Survey Questions with Likert-Scale Ratings\label{app:two-var-on-rating}}

The survey questions and the accompanying categories were rated on a seven-point Likert scale, 
as discussed in 
Section~\ref{sec:two-var-on-rating},
listed below:

\begin{itemize}
    \item 
    \textbf{(Q1) Understandable}: The annotation task was easy to understand.

    \item
    \textbf{(Q2) Ease of Use}: The annotation tool is easy to use.

    \item
    \textbf{(Q4) Intuitive System}: The interface of the annotation system is intuitive.

    \item
    \textbf{(Q5) Performance Satisfaction}: How satisfied are you with the performance of the system?

    \item
    \textbf{(Q6) Prompt Improvement}: This tool was helpful in improving my prompt. 
    
    \item
    \textbf{(Q7) Process Efficiency}: Using this tool made the process of prompt engineering more efficient.

    \item
    \textbf{(Q19) Task Completion}: I completed the annotation tasks efficiently.

\end{itemize}