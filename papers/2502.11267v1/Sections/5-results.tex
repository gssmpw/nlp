In this section, we organize our findings under each research question (RQs) mentioned in Section~\ref{sec:rq}.

\subsection{RQ 1-1: How effective are people at prompt engineering in the ``prompting in the dark'' scenario?}
\begin{figure*}[t]
    \centering
    \begin{subfigure}{0.48\textwidth}
    \includegraphics[width=\linewidth]{Figures/Results/New-Pure-Participant/avg_accuracy_plot_new.png}\Description{This is an average accuracy plots subfigure of the participants prompt performance. The subfigure includes shaded regions representing the standard deviation and scatter points indicating individual participants’ data.}
    \caption{The average ACC of the first prompt and the four revisions.}
    \label{fig:average-acc}
  \end{subfigure}
  \hfill
  \begin{subfigure}{0.48\textwidth}
    \includegraphics[width=\linewidth]{Figures/Results/New-Pure-Participant/avg_mse_plot_new.png}\Description{This is a MSE accuracy plots subfigure of the participants prompt performance. The subfigure includes shaded regions representing the standard deviation and scatter points indicating individual participants’ data.}
    \caption{The average MSE of the first prompt and the four revisions.}
    \label{fig:average-mse}
  \end{subfigure}
  \caption{The average ACC and MSE of the first prompt and the four subsequent revisions. These results show that prompting in the dark is not particularly effective. 
Average labeling accuracy only slightly improved after four iterations; the average MSE fluctuated, ultimately increasing only marginally by the fourth revision.}
  \label{fig:average-acc-mse-performance}
\end{figure*}

\begin{table*}[t]
\centering
%\small
\begin{tabular}{lcccc}
 & \multicolumn{2}{c}{\textbf{ACC$\uparrow$}} & \multicolumn{2}{c}{\textbf{MSE$\downarrow$}} \\ \hline
\multicolumn{1}{c}{} & \textbf{Avg.(SD)} & \textbf{\begin{tabular}[c]{@{}c@{}}\%Participant\\ Improved\\ Over Initial\end{tabular}} & \textbf{Avg.(SD)} & \textbf{\begin{tabular}[c]{@{}c@{}}\%Participant\\ Improved\\ Over Initial\end{tabular}} \\ \hline
\textbf{Initial Prompt} & .542 (.099) & - & .782 (.482) & - \\ \hline
\textbf{1st Revision} & .536 (.088) & 35\% & .795 (.546) & 40\% \\
\textbf{2nd Revision} & .542 (.073) & 40\% & .833 (.514) & 45\% \\
\textbf{3rd Revision} & \underline{\textbf{.549 (.082)}} & 50\% & .823 (.579) & 50\% \\
\textbf{4th Revision} & \underline{\textbf{.553 (.085)}} & 45\% & .810 (.566) & 40\% \\ \hline
\textbf{\begin{tabular}[c]{@{}l@{}}End of Session\\ (Avg \#Iter = 4.75)\end{tabular}} & \underline{\textbf{.546 (.084)}} & 45\% & .815 (.489) & 45\%
\end{tabular}
\caption{The average ACC and MSE of the first prompt, the four subsequent revisions, and at the end of the session. Improvements over the initial prompt are bolded and underlined. }
\label{tab:overall-ACC-MSE}
\end{table*}

\begin{figure*}[t]
    \centering
    \includegraphics[width=0.95\linewidth]{Figures/Results/New-Participant-DSPy/new_acc_plots.png}\Description{It contains 20 subfigures of the average ACC of each participant.}
    \caption{Accuracy Plots for all participants. The results show that the process is highly unreliable. 
Labeling accuracy improved for 9 participants after four iterations, declined for 10, and remained unchanged for 1. }
    \label{fig:individual-pure-acc}
\end{figure*}

\begin{figure*}[t]
    \centering
    \includegraphics[width=0.95\linewidth]{Figures/Results/New-Participant-DSPy/new_mse_plots.png}\Description{It contains 20 subfigures of the average MSE of each participant.}
    \caption{MSE Plots for all participants.
    The results show that the process is highly unreliable. 
MSE improved (\ie, decreased) for 8 participants, worsened for 10, and stayed the same for 2.}
    \label{fig:individual-pure-mse}
\end{figure*}



%\kenneth{NO ONE SHOULD DO THIS. THIS IS HARDDD.}

We evaluated each prompt iteration by comparing the labels generated by the LLM (GPT-4o) to the gold labels participants manually annotated for a set of 50 tweets at the end of their study session.
Performance was measured using accuracy (ACC) and Mean Square Error (MSE), two commonly used metrics in sentiment analysis~\cite{saxena2022introduction}.
%\kenneth{Why do we cite this here?} \steven{this describe that ACC and MSE are two evaluation matrices for analyzing sentiment task}
In the main study, each participant provided at least five prompts (with some offering more), and our evaluation focuses primarily on these five prompts, which were shared across all participants.
The ACC and MSE were calculated by comparing the annotation results from participant-provided prompts with the manually generated gold-standard labels. Each label was treated as a distinct category, and accuracy was then calculated accordingly.

\subsubsection{5-Point sentiment rating is a highly subjective task.}
%\kenneth{I kinda feel we don't need a table for kappa}

%The premise of our study is that users bring their own personal perspectives and judgments to even seemingly identical labeling tasks. 
%Before analyzing the results, we first validate this assumption in our study.
%We calculated both Cohen's kappa and Kendall's correlation between participants' manual ratings, collected at the end of our study and the ``gold-standard'' labels provided by the dataset, as well as the value between participants.
The premise of our study is that users bring their own personal perspectives and judgments to seemingly identical labeling tasks. 
To validate this assumption, we calculated both Cohen's kappa
%, as well as Spearman's and Kendall's correlations 
between participants' manual ratings---the labels collected at the end of the study---and the ``gold-standard'' labels from the dataset, as well as the agreement between participants themselves.
%\kenneth{TODO Steven: Kappa is pretty harsh. Can you additionally try Spearman or Spearman, Kendall correlation?}\steven{sure, Spearman: participant vs dataset is 0.343(SD=0.121), between participants is 0.530(SD=0.078). Kendall: participant vs dataset is 0.299(SD=0.107), and between participants is 0.474(SD=0.071) These two correlation scores are slightly better than Kappa but still not very good.}
%\paragraph{Inter-annotator agreement (Cohen's kappa)} 
Participants' labels show a poor alignment with the labels from the original dataset, with an average Kappa of 0.114 (SD=0.070). 
%Spearman's of 0.343 (SD=0.121), and 
%Kendall's of 0.299 (SD=0.107).
%Interestingly, 
The average Kappa value between participants was only slightly higher, with an average Kappa of 0.249 (SD=0.059).
%Spearman's of 0.530(SD=0.078), and 
%Kendall's of 0.474(SD=0.071).
%\kenneth{is this average between all the participant pairs?}\steven{yes}
%each individual participant and the rest of the participants was only slightly  higher at 0.249 (SD=0.059), suggesting participants were more consistent with each other than with the original dataset labels.
The low Kappa score indicated the Twitter Sentiment task we used in our study was a highly subjective task. 
Namely, each participant's gold labels were highly influenced by their personal interpretations and preferences.


\subsubsection{Prompting in the dark is ineffective}
Table~\ref{tab:overall-ACC-MSE} and Figure~\ref{fig:average-acc-mse-performance} show the average ACC and MSE for the first prompt and the four subsequent revisions.
%\kenneth{TODO: Add reference to the Table}\steven{done}
Our analysis, as captured in \system, indicates that prompting in the dark is not particularly effective. 
Among the 20 participants, average labeling accuracy (where higher is better) only slightly improved from 0.542 to 0.553 after four iterations (Figure~\ref{fig:average-acc}). 
Some participants went through more than four iterations. 
The average labeling accuracy at the end of their sessions---the final iteration for all participants---was improved to 0.546.
%\kenneth{Add one sentence about the end-of-session ACC performance}\steven{done}
Meanwhile, the average MSE (where lower is better) fluctuated, ultimately increasing from 0.782 to 0.810 by the fourth revision (Figure~\ref{fig:average-mse}). 
The average MSE for the end-of-session increased to 0.815.
%\kenneth{Add one sentence about the end-of-session MSE performance}\steven{done}
It is important to note that, since this is a 5-scale rating task, ACC is a more harsh metric, awarding credit only for exact matches, while MSE considers the distance between the predicted rating and the user-specified rating.



%\steven{significant increasing trend by using linear mixed-effect model}

\subsubsection{Prompting in the dark is unreliable.}
To further illustrate how the process unfolded for each participant, we present individual ACC and MSE charts in Figure~\ref{fig:individual-pure-acc} and Figure~\ref{fig:individual-pure-mse}. 
The results show that the process is highly unreliable. 
Labeling accuracy improved for 9 participants after four iterations, declined for 10, and remained unchanged for 1. 
Similarly, MSE improved (\ie, decreased) for 8 participants, worsened for 10, and stayed the same for 2. 
Overall, the practice of ``prompting in the dark''---iterating prompts without reference to gold labels---proved unreliable, with over half of the participants experiencing a decline in performance by the end of the study.








%--------------- dead kitten ---------------
\begin{comment}
  



The average labeling accuracy had a slight increase (Figure~\ref{fig:average-acc}), rising from 54.20\% to 55.30\% in the final refined prompt compared to the initial prompt. Meanwhile, the average MSE (Figure~\ref{fig:average-mse}) fluctuated and ultimately increased from 0.782 to 0.810 by the last prompt.

Figure~\ref{fig:individual-pure-acc} and Figure~\ref{fig:individual-pure-mse} illustrate the performance of participant-guided LLM for each iteration. 
Labeling accuracy improved for 9 participants after four iterations, decreased for 10 participants, and remained unchanged for 1 participant.
Labeling MSE improved (dropping in MSE) for 8 participants, decreased for 10 participants, and remained unchanged for 2 participants.

Overall, the prompt refined during the user study was \textbf{highly unreliable} as over half of the participants experienced a decline in their performance at the end of the study.



  
\end{comment}










\subsection{RQ 1-2: How does sample size affect human performance in prompt engineering?\label{sec:rq-1-2}}
% Please add the following required packages to your document preamble:
%\usepackage{booktabs}
\begin{table*}[t]
\centering
\footnotesize
\begin{tabular}{@{}lcccccccc@{}}
 & \multicolumn{4}{c}{\textbf{50 Samples/Round (N=10)}} & \multicolumn{4}{c}{\textbf{10 Samples/Round (N=10)}} \\ \cmidrule{2-9} 
 & \multicolumn{2}{c}{\textbf{ACC$\uparrow$}} & \multicolumn{2}{c}{\textbf{MSE$\downarrow$}} & \multicolumn{2}{c}{\textbf{ACC$\uparrow$}} & \multicolumn{2}{c}{\textbf{MSE$\downarrow$}} \\ \midrule
\multicolumn{1}{c}{} & \textbf{Avg.(SD)} & \textbf{\begin{tabular}[c]{@{}c@{}}\%Participant\\ Improved\\ Over Initial\end{tabular}} & \textbf{Avg.(SD)} & \textbf{\begin{tabular}[c]{@{}c@{}}\%Participant\\ Improved\\ Over Initial\end{tabular}} & \textbf{Avg.(SD)} & \textbf{\begin{tabular}[c]{@{}c@{}}\%Participant\\ Improved\\ Over Initial\end{tabular}} & \textbf{Avg.(SD)} & \textbf{\begin{tabular}[c]{@{}c@{}}\%Participant\\ Improved\\ Over Initial\end{tabular}} \\ \midrule
\textbf{Initial Prompt} & .520 (.075) & - & .742 (.242)  & - & .564 (.117) & - & .822 (.654) & - \\ \midrule
\textbf{1st Revision} & \underline{\textbf{.530 (.094)}} & 50\% & \underline{\textbf{.720 (.285)}} & 40\% & .542 (.086) & 20\% & .870 (.733) & 40\% \\
\textbf{2nd Revision} & \underline{\textbf{.528 (.050)}} & 50\% & .780 (.268) & 50\% & .556 (.091) & 30\% & .886 (.692) & 40\% \\
\textbf{3rd Revision} & \underline{\textbf{.546 (.083)}} & 70\% & \underline{\textbf{.722 (.308)}} & 60\% & .552 (.085) & 30\% & .924 (.768) & 40\% \\
\textbf{4th Revision} & \underline{\textbf{.536 (.095)}} & 60\% & \underline{\textbf{.730 (.310)}} & 40\% & \underline{\textbf{.570 (.074)}} & 30\% & .890 (.753) & 40\% \\ \midrule
\textbf{\begin{tabular}[c]{@{}l@{}}End of Session\\ (Avg \#Iter=4.75)\end{tabular}} & \underline{\textbf{.536 (.095)}} & 60\% & \underline{\textbf{.736 (.303)}} & 40\% & .556 (.075) & 30\% & .894 (.632) & 50\%
\end{tabular}
\caption{Comparison of participants who reviewed 50 instances per iteration versus those who reviewed 10 instances per iteration. Reviewing 50 instances per iteration resulted in more frequent and consistent improvements compared to reviewing 10 instances. Improvements over the initial prompt are bolded and underlined.}
\label{tab:table-50-example-kenneth}
\end{table*}

% \begin{table}[t]
    \centering
    \begin{threeparttable}
    \begin{tabular}{l|c|c|c|c|c|c|c|c}
        \hline
         & \multicolumn{4}{c|}{\textbf{w/ 50 Instances} \tiny{(n=10)}} & \multicolumn{4}{c}{\textbf{w/o 50 Instances} \tiny{(n=10)}} \\
         \cmidrule(lr){2-5} \cmidrule(lr){6-9}
         & \multicolumn{2}{c|}{ACC$\uparrow$} & \multicolumn{2}{c|}{MSE$\downarrow$}& \multicolumn{2}{c|}{ACC$\uparrow$} & \multicolumn{2}{c}{MSE$\downarrow$}\\
         \cmidrule(lr){2-3} \cmidrule(lr){4-5} \cmidrule(lr){6-7} \cmidrule(lr){8-9}
         & \multicolumn{1}{c|}{Avg} & \multicolumn{1}{c|}{\%} & \multicolumn{1}{c|}{Avg} & \multicolumn{1}{c|}{\%}& \multicolumn{1}{c|}{Avg} & \multicolumn{1}{c|}{\%}& \multicolumn{1}{c|}{Avg} & \multicolumn{1}{c}{\%}\\
         \hline
        \multicolumn{1}{l|}{\textbf{Start}} & \multicolumn{1}{c|}{.520} & \multicolumn{1}{c|}{-} & \multicolumn{1}{c|}{.742} & \multicolumn{1}{c|}{-} & \multicolumn{1}{c|}{.564} & \multicolumn{1}{c|}{-} & \multicolumn{1}{c|}{.822} & \multicolumn{1}{c|}{-}  \\
        \multicolumn{1}{l|}{\textbf{1st}}   & \multicolumn{1}{c|}{.530} & \multicolumn{1}{c|}{50} & \multicolumn{1}{c|}{.720} & \multicolumn{1}{c|}{40} & \multicolumn{1}{c|}{.542} & \multicolumn{1}{c|}{20} & \multicolumn{1}{c|}{.870} & \multicolumn{1}{c|}{40}  \\
        \multicolumn{1}{l|}{\textbf{2nd}}   & \multicolumn{1}{c|}{.528} & \multicolumn{1}{c|}{50} & \multicolumn{1}{c|}{.780} & \multicolumn{1}{c|}{50} & \multicolumn{1}{c|}{.556} & \multicolumn{1}{c|}{30} & \multicolumn{1}{c|}{.886} & \multicolumn{1}{c|}{40}  \\
        \multicolumn{1}{l|}{\textbf{3rd}}   & \multicolumn{1}{c|}{.546} & \multicolumn{1}{c|}{70} & \multicolumn{1}{c|}{.722} & \multicolumn{1}{c|}{60} & \multicolumn{1}{c|}{.552} & \multicolumn{1}{c|}{30} & \multicolumn{1}{c|}{.924} & \multicolumn{1}{c|}{40}  \\
        \multicolumn{1}{l|}{\textbf{4th}}   & \multicolumn{1}{c|}{.536} & \multicolumn{1}{c|}{60} & \multicolumn{1}{c|}{.730} & \multicolumn{1}{c|}{40} & \multicolumn{1}{c|}{.570} & \multicolumn{1}{c|}{30} & \multicolumn{1}{c|}{.890} & \multicolumn{1}{c|}{40}  \\
        \hline
        \multicolumn{1}{l|}{End of Session} & \multicolumn{1}{c}{\multirow{2}{*}{.536}}& \multicolumn{1}{c}{\multirow{2}{*}{60}}& \multicolumn{1}{c}{\multirow{2}{*}{.736}}& \multicolumn{1}{c}{\multirow{2}{*}{40}}& \multicolumn{1}{c}{\multirow{2}{*}{.556}}& \multicolumn{1}{c}{\multirow{2}{*}{30}}& \multicolumn{1}{c}{\multirow{2}{*}{.894}}& \multicolumn{1}{c}{\multirow{2}{*}{50}}\\
        \multicolumn{1}{l|}{Avg \#Iter (4.75)}\\
        \hline
    \end{tabular}
    \begin{tablenotes}
        \tiny
        \item Avg: the average value of either ACC or MSE.
        \item \%: number of user improved ACC/MSE over ``Start'' $/$number of total user
    \end{tablenotes}
    \end{threeparttable}
    \caption{Participants' performance per iteration with or without 50 instances}
    \label{tab:participant-explanation-detail}
\end{table}


        
        
        % \multicolumn{1}{l}{\textbf{End of Session}} \\



 % & \multicolumn{1}{c|}{}
 % & \multicolumn{1}{c|}{}
 % & \multicolumn{1}{c|}{}
 % & \multicolumn{1}{c|}{}
 % & \multicolumn{1}{c|}{}


% \begin{figure}[t]
%     \centering
%     \begin{subfigure}{0.48\textwidth}
%      \includegraphics[width=\linewidth]{Figures/Results/Explanation-Result/diff_bar_ACC_50_instance.png}
%     \caption{Difference in ACC (Higher is Better) Compared to Initial Prompt (Iteration 0).\kenneth{(1) The figure title should be ``Difference in ACC Compared to Initial Prompt'', (2) The subfigure caption should be ``Difference in ACC (Higher is Better) Compared to Initial Prompt (Iteration 0)'', (3) The y-axis title should be ``ACC Improvement (Higher is Better)'', (4) the number in x and y axis should use larger fonts, (5) For the improvement (positive) bars, let's use solid border lines and solid color for the bar. For the negative bars, let's use dotted border lines and maybe color with lower opacity and .....gridlines texture? Visually convey that which bars are decreased/negative ones.}}
%     \label{fig:acc-participants-50instance-or-not}
%   \end{subfigure}
%   \hfill
%   \begin{subfigure}{0.48\textwidth}
%     \includegraphics[width=\linewidth]{Figures/Results/Explanation-Result/diff_bar_MSE_50_instance.png}
%     \caption{The improvement (difference) in MSE.}
%     \label{fig:mse-participants-50instance-or-not}
%   \end{subfigure}
%   \caption{Comparison of improvement (difference) in ACC and MSE over the initial prompt between the 50-instance and 10-instance groups. Reviewing 50 instances per iteration results in more consistent improvements in ACC compared to reviewing 10 instances.}
% %.\kenneth{TODO Steven: (1) The fonts need to be bigger, especially the legend label. They are way too small. (2) The title should say ``50 Samples/Round'' vs. ``10 Samples/Round'' instead of w/ w/o. (3) MAYBE use Red for 50 and }
%   \label{fig:average-50-instance-acc-mse-performance}
% \end{figure}

\begin{figure*}[t]
        \centering
        \includegraphics[width=0.98\linewidth]{Figures/Results/new-diff-bar/new_diff_bar_MSE_50_instance_v2.png}\Description{It contains 2 subplots illustrating the effect of sample size on ACC and MSE across iterations. The bars represent performance changes relative to the initial prompt, with improvements highlighted by a thick red border.}
  \caption{Comparison of improvement (difference) in ACC and MSE over the initial prompt between the 50-instance and 10-instance groups. Bars with red borders indicate positive improvements over the initial prompt's outcome. Reviewing 50 instances per iteration results in more consistent improvements in ACC compared to reviewing 10 instances. }
%.\kenneth{TODO Steven: (1) The fonts need to be bigger, especially the legend label. They are way too small. (2) The title should say ``50 Samples/Round'' vs. ``10 Samples/Round'' instead of w/ w/o. (3) MAYBE use Red for 50 and }
  \label{fig:average-50-instance-acc-mse-performance}
\end{figure*}


\subsubsection{Reviewing 50 instances per iteration leads to more frequent and consistent improvements compared to reviewing 10 instances}

%\kenneth{TODO Steven: Can you talk to Zixin about how to do t-test (or any significance test) in this case? In particular (1) how to do it between iterations and (2) how to do it for the "differences" instead of for absolute value. I kinda feel like we care about the diff than the absolute value. 50 vs 10 group, the 10 group had a better starting acc/mse but that's not relevant to our system but rather just by chance. }

%Participants who reviewed 50 instances were designated as \textbf{50Y}, and those who reviewed 10 instances were labeled as \textbf{50N}.
%For 50Y participants, 6 out of 10 participants demonstrated an improvement in their labeling accuracy, whereas 3 out of 10 participants in the 50N group improved. 
%However, participants under 50Y demonstrated a slightly better change in MSE compared to 50N, as 50Y had 2 fewer individuals with a decline in MSE performance. 
Table~\ref{tab:table-50-example-kenneth} and Figure~\ref{fig:average-50-instance-acc-mse-performance} present a comparison of participants who reviewed 50 instances per iteration against those who reviewed 10 instances per iteration. The detailed breakdown is shown in Figure~\ref{fig:average-acc-mse-llm-instances-performance-new}.
%\kenneth{TODO Steven: Update the figure and table references.}\steven{done}
In terms of accuracy, at the end of the session (\ie, four or more revisions),
6 out of 10 participants in the 50-instance group showed improvement, while only 3 out of 10 participants in the 10-instance group improved.
%\kenneth{Is this AFTER 4 REVISIONs or AT THE END OF SESSION?}\steven{4 revision and at the end have the same number of improved partcipant}
On average, every iteration in the 50-instance group resulted in better accuracy compared to the initial prompt, though the improvement was not strictly increasing with each iteration.

For MSE, participants in the 50-instance group improved across three iterations, while those in the 10-instance group showed no improvement over the initial prompt in any round.

We also note that participants in the 10-instance group began with higher initial performance, but this was before they viewed the labeling results and occurred by chance, unrelated to the experimental conditions. 
Our analysis focuses on performance differences between iterations across both groups.

\paragraph{Significant Tests.}
We conducted eight linear mixed-effects models to examine the effect of iteration across four different conditions. 
The dependent variables were ACC and MSE, and participants were treated as random effects.
%Under conditions where participants \textbf{did not have access to LLM explanations,} we observed a significant increasing trend in accuracy with each iteration ($\beta$=0.013, p-value=0.009**). 
%Conversely, 
In the condition where participants \textbf{reviewed only 10 instances per iteration}, we found a significant increasing trend in MSE as the iterations progressed ($\beta$=0.019, p-value=0.043*). 




%\kenneth{Bigger is better!}

\subsection{RQ 1-3: How does displaying LLM explanations impact human performance in prompt engineering?\label{sec:llm-explanation-result}}
% Please add the following required packages to your document preamble:
% \usepackage{booktabs}
% \begin{table}[t]
% \centering
% \footnotesize
% \begin{tabular}{@{}lcccccccc@{}}
%  & \multicolumn{4}{c}{\textbf{LLM Explanations Shown}} & \multicolumn{4}{c}{\textbf{No LLM Explanations Shown}} \\ \cmidrule(l){2-9} 
%  & \multicolumn{2}{c}{\textbf{ACC$\uparrow$}} & \multicolumn{2}{c}{\textbf{MSE$\downarrow$}} & \multicolumn{2}{c}{\textbf{ACC$\uparrow$}} & \multicolumn{2}{c}{\textbf{MSE$\downarrow$}} \\ \midrule
% \multicolumn{1}{c}{} & \textbf{Avg. (SD)} & \textbf{\begin{tabular}[c]{@{}c@{}}\%Participant\\ Improved\\ Over Initial\end{tabular}} & \textbf{Avg. (SD)} & \textbf{\begin{tabular}[c]{@{}c@{}}\%Participant\\ Improved\\ Over Initial\end{tabular}} & \textbf{Avg. (SD)} & \textbf{\begin{tabular}[c]{@{}c@{}}\%Participant\\ Improved\\ Over Initial\end{tabular}} & \textbf{Avg.(SD)} & \textbf{\begin{tabular}[c]{@{}c@{}}\%Participant\\ Improved\\ Over Initial\end{tabular}} \\ \midrule
% \textbf{Initial Prompt} & .586 (.065) & - & .614 (.230) & - & .498 (.109) & - & .950 (.612) & - \\ \midrule
% \textbf{1st Revision} & .558 (.066) & 20\% & .624 (.261)  & 40\% & \underline{\textbf{.514 (.104)}} & 50\% & .966 (.705) & 40\% \\
% \textbf{2nd Revision} & .570 (.063)  & 20\% & .632 (.196) & 60\% & \underline{\textbf{.514 (.074)}} & 60\% & 1.034 (.655) & 30\% \\
% \textbf{3rd Revision} & .560 (.074) & 30\% & .646 (.310) & 50\% & \underline{\textbf{.538 (.092)}} & 70\% & 1.000 (.737)  & 50\% \\
% \textbf{4th Revision} & .556 (.076) & 20\% & .646 (.294) & 40\% & \underline{\textbf{.550 (.096)}}  & 70\% & .974 (.729) & 40\% \\ \midrule
% \textbf{\begin{tabular}[c]{@{}l@{}}End of Session\\ (Avg \#Iter=4.75)\end{tabular}} & .556 (.079) & 20\% & .674 (.275) & 40\% & \underline{\textbf{.536 (.091)}} & 70\% & .956 (.620) & 50\%
% \end{tabular}
% \caption{Comparison of ACC and MSE between participants with and without access to LLM explanations during the labeling process. Participants without access to LLM explanations showed improvement in accuracy over multiple revisions, while those with access did not exhibit the same level of improvement. Improvements over the initial prompt are bolded and underlined.}
% \label{tab:results-llm}
% \end{table}

% \begin{table}[t]
    \centering
    \begin{threeparttable}
    \begin{tabular}{l|c|c|c|c|c|c|c|c}
        \hline
         & \multicolumn{4}{c|}{\textbf{w/ Explanation} \tiny{(n=10)}} & \multicolumn{4}{c}{\textbf{w/o Explanation} \tiny{(n=10)}} \\
         % & \multicolumn{4}{c|}{\tiny{(n=10)}} & \multicolumn{4}{c}{\tiny{(n=10)}}\\ 
         \cmidrule(lr){2-5} \cmidrule(lr){6-9}
         & \multicolumn{2}{c|}{ACC$\uparrow$} & \multicolumn{2}{c|}{MSE$\downarrow$}& \multicolumn{2}{c|}{ACC$\uparrow$} & \multicolumn{2}{c}{MSE$\downarrow$}\\
         \cmidrule(lr){2-3} \cmidrule(lr){4-5} \cmidrule(lr){6-7} \cmidrule(lr){8-9}
         & \multicolumn{1}{c|}{Avg} & \multicolumn{1}{c|}{\%} & \multicolumn{1}{c|}{Avg} & \multicolumn{1}{c|}{\%}& \multicolumn{1}{c|}{Avg} & \multicolumn{1}{c|}{\%}& \multicolumn{1}{c|}{Avg} & \multicolumn{1}{c}{\%}\\
         \hline
        \multicolumn{1}{l|}{\textbf{Start}}  & \multicolumn{1}{c|}{.586} & \multicolumn{1}{c|}{-}  & \multicolumn{1}{c|}{.614} & \multicolumn{1}{c|}{-}& \multicolumn{1}{c|}{.498}& \multicolumn{1}{c|}{-} & \multicolumn{1}{c|}{.950}& \multicolumn{1}{c}{-}\\
        \multicolumn{1}{l|}{\textbf{1st}}    & \multicolumn{1}{c|}{.558} & \multicolumn{1}{c|}{20} & \multicolumn{1}{c|}{.624} & \multicolumn{1}{c|}{40}& \multicolumn{1}{c|}{.514}& \multicolumn{1}{c|}{50} & \multicolumn{1}{c|}{.966} & \multicolumn{1}{c}{40}\\
        \multicolumn{1}{l|}{\textbf{2nd}}    & \multicolumn{1}{c|}{.570} & \multicolumn{1}{c|}{20} & \multicolumn{1}{c|}{.632} & \multicolumn{1}{c|}{60}& \multicolumn{1}{c|}{.514}& \multicolumn{1}{c|}{60}& \multicolumn{1}{c|}{1.034} & \multicolumn{1}{c}{30}\\
        \multicolumn{1}{l|}{\textbf{3rd}}    & \multicolumn{1}{c|}{.560} & \multicolumn{1}{c|}{30} & \multicolumn{1}{c|}{.646} & \multicolumn{1}{c|}{50}& \multicolumn{1}{c|}{.538}& \multicolumn{1}{c|}{70}& \multicolumn{1}{c|}{1.000} & \multicolumn{1}{c}{50}\\
        \multicolumn{1}{l|}{\textbf{4th}}    & \multicolumn{1}{c|}{.556} & \multicolumn{1}{c|}{20} & \multicolumn{1}{c|}{.646} & \multicolumn{1}{c|}{40}& \multicolumn{1}{c|}{.55}& \multicolumn{1}{c|}{70}& \multicolumn{1}{c|}{.974} & \multicolumn{1}{c}{40}\\
        \hline
        \multicolumn{1}{l|}{End of Session} & \multicolumn{1}{c}{\multirow{2}{*}{.556}}& \multicolumn{1}{c}{\multirow{2}{*}{20}}& \multicolumn{1}{c}{\multirow{2}{*}{.674}} & \multicolumn{1}{c}{\multirow{2}{*}{40}} & \multicolumn{1}{c}{\multirow{2}{*}{.536}} & \multicolumn{1}{c}{\multirow{2}{*}{70}} & \multicolumn{1}{c}{\multirow{2}{*}{.956}}& \multicolumn{1}{c}{\multirow{2}{*}{50}}\\
        \multicolumn{1}{l|}{Avg \#Iter (4.75} \\
        \hline
    \end{tabular}
    \begin{tablenotes}
        \tiny
        \item Avg: the average value of either ACC or MSE.
        \item \%: number of user improved ACC/MSE over ``Start'' $/$number of total user
    \end{tablenotes}
    \end{threeparttable}
    \caption{Participants' performance per iteration with or without LLM Explanation.}
    \label{tab:participant-explanation-detail}
\end{table}


        
        
        % \multicolumn{1}{l}{\textbf{End of Session}} \\



 % & \multicolumn{1}{c|}{}
 % & \multicolumn{1}{c|}{}
 % & \multicolumn{1}{c|}{}
 % & \multicolumn{1}{c|}{}
 % & \multicolumn{1}{c|}{}
\begin{table*}[t]
\centering
\footnotesize
\begin{tabular}{@{}lcccccccc@{}}
 & \multicolumn{4}{c}{\textbf{LLM Explanations Shown (N=8)}} & \multicolumn{4}{c}{\textbf{No LLM Explanations Shown (N=12)}} \\ \cmidrule{2-9} 
 & \multicolumn{2}{c}{\textbf{ACC$\uparrow$}} & \multicolumn{2}{c}{\textbf{MSE$\downarrow$}} & \multicolumn{2}{c}{\textbf{ACC$\uparrow$}} & \multicolumn{2}{c}{\textbf{MSE$\downarrow$}} \\ \midrule
\multicolumn{1}{c}{} & \textbf{Avg. (SD)} & \textbf{\begin{tabular}[c]{@{}c@{}}\%Participant\\ Improved\\ Over Initial\end{tabular}} & \textbf{Avg. (SD)} & \textbf{\begin{tabular}[c]{@{}c@{}}\%Participant\\ Improved\\ Over Initial\end{tabular}} & \textbf{Avg. (SD)} & \textbf{\begin{tabular}[c]{@{}c@{}}\%Participant\\ Improved\\ Over Initial\end{tabular}} & \textbf{Avg.(SD)} & \textbf{\begin{tabular}[c]{@{}c@{}}\%Participant\\ Improved\\ Over Initial\end{tabular}} \\ \midrule
\textbf{Initial Prompt} & .590 (.063) & - & .608 (.259) & - & .510 (.107) & - & .898 (.566) & - \\ \midrule
\textbf{1st Revision} & .558 (.074) & 12.5\% & .633 (.292)  & 37.5\% & \underline{\textbf{.522 (.096)}} & 50\% & .903 (.655) & 41.7\% \\
\textbf{2nd Revision} & .568 (.070)  & 12.5\% & .640 (.221) & 50\% & \underline{\textbf{.525 (.072)}} & 58.3\% & .962 (.616) & 41.7\% \\
\textbf{3rd Revision} & .555 (.082) & 25\% & .660 (.349) & 50\% & \underline{\textbf{.545 (.085)}} & 66.7\% & .932 (.685)  & 50\% \\
\textbf{4th Revision} & .563 (.084) & 25\% & .645 (.333) & 50\% & \underline{\textbf{.547 (.088)}}  & 58.3\% & .920 (.671) & 33.3\% \\ \midrule
\textbf{\begin{tabular}[c]{@{}l@{}}End of Session\\ (Avg \#Iter=4.75)\end{tabular}} & .563 (.087) & 25\% & .680 (.311) & 50\% & \underline{\textbf{.535 (.083)}} & 58.3\% & .905 (.574) & 41.7\%
\end{tabular}
\caption{Comparison of ACC and MSE between participants with and without access to LLM explanations during the labeling process. Participants without access to LLM explanations showed improvement in accuracy over multiple revisions, while those with access did not exhibit the same level of improvement. Improvements over the initial prompt are bolded and underlined.}
\label{tab:results-llm-new}
\end{table*}



% \begin{figure}[t]
%     \centering
%     \begin{subfigure}{0.48\textwidth}
%      \includegraphics[width=\linewidth]{Figures/Results/Explanation-Result/diff_bar_ACC_explanation.png}
%     \caption{The improvement (difference) in ACC.}
%     \label{fig:acc-participants-explanation-or-not}
%   \end{subfigure}
%   \hfill
%   \begin{subfigure}{0.48\textwidth}
%     \includegraphics[width=\linewidth]{Figures/Results/Explanation-Result/diff_bar_MSE_explanation.png}
%     \caption{The improvement (difference) in MSE.}
%     \label{fig:mse-participants-explantation-or-not}
%   \end{subfigure}
%   \caption{Comparison of improvement (difference) in ACC and MSE over the initial prompt between the Explanation and No-Explanation groups. Participants without access to LLM explanations improved their accuracy over multiple revisions, while those with access did not.}
%   \label{fig:average-acc-mse-llm-explanation-performance}
% \end{figure}

% \begin{figure}[t]
%     \centering
%     \begin{subfigure}{0.48\textwidth}
%      \includegraphics[width=\linewidth]{Figures/Results/Explanation-Result/diff_bar_ACC_explanation_new.png}
%     \caption{The improvement (difference) in ACC.}
%     \label{fig:acc-participants-explanation-or-not-new}
%   \end{subfigure}
%   \hfill
%   \begin{subfigure}{0.48\textwidth}
%     \includegraphics[width=\linewidth]{Figures/Results/Explanation-Result/diff_bar_MSE_explanation_new.png}
%     \caption{The improvement (difference) in MSE.}
%     \label{fig:mse-participants-explantation-or-not-new}
%   \end{subfigure}
%   \caption{Comparison of improvement (difference) in ACC and MSE over the initial prompt between the Explanation and No-Explanation groups. Participants without access to LLM explanations improved their accuracy over multiple revisions, while those with access did not. \steven{removed the old version, this is newer version after moving participants to no llm group}}
%   \label{fig:average-acc-mse-llm-explanation-performance-new}
% \end{figure}

\begin{figure*}[t]
    \centering
    \includegraphics[width=0.98\linewidth]{Figures/Results/new-diff-bar/new_diff_bar_MSE_explanation_v2.png}\Description{It contains 2 subplots illustrating the effect of LLM explanation on ACC and MSE across iterations. The bars represent performance changes relative to the initial prompt, with improvements highlighted by a thick red border.}
  \caption{Comparison of improvement (difference) in ACC and MSE over the initial prompt between the Explanation and No-Explanation groups. Bars with red borders indicate positive improvements over the initial prompt's outcome. Participants without access to LLM explanations improved their accuracy over multiple revisions, while those with access did not.}
  \label{fig:diff-bar-average-acc-mse-llm-explanation-performance-new}
\end{figure*}



%new added explnation and 50 instance figures


%% end adding

\subsubsection{Participants without access to LLM explanations improved their accuracy over multiple revisions, while those with access did not.}
Table~\ref{tab:results-llm-new} and Figure~\ref{fig:diff-bar-average-acc-mse-llm-explanation-performance-new} show a comparison between participants with and without access to the LLM's explanations during the labeling process. The detailed breakdown is shown in Figure~\ref{fig:average-acc-mse-llm-explanation-performance-new}.
%\kenneth{TODO Steven: Update Table references} \steven{done}
%After four revisions, 
At the end of the session (\ie, four or more revisions),
7 out of 12 participants without access to explanations improved their accuracy, while only 2 out of 8 participants with access to explanations showed improvement.
%\kenneth{Is this after four revisions or at the end of the session?} \steven{both after 4th and at end session}
On average, the group without access saw improved accuracy over the initial prompt with each of the four revisions, whereas the group with access to explanations experienced a decline in accuracy across all revisions.

In terms of MSE, there was no significant difference between the two groups in the number of individuals who showed improvement. 
In fact, both groups saw an increase in MSE during the revision process.

As with the results related to data sample size (Section~\ref{sec:rq-1-2}), we are aware that participants with access to LLM explanations started with higher initial performance; this initial advantage occurred before the participants reviewed the labeling results and explanations.
Our analysis focuses on performance changes across iterations rather than the absolute performances.

\paragraph{Significant Tests.}
We conducted eight linear mixed-effects models to examine the effect of iteration across four different conditions. 
The dependent variables were ACC and MSE, and participants were treated as random effects.
Under conditions where participants \textbf{did not have access to LLM explanations,} we observed a significant increasing trend in accuracy with each iteration ($\beta$=0.010, p-value=0.027*). 


%Conversely, 
%In the condition where participants \textbf{reviewed only 10 instances per iteration}, we found a significant increasing trend in MSE as the iterations progressed ($\beta$=0.019, p-value=0.043*). 


%Participants with access to the LLM explanation were labeled as \textbf{LY}, while those without access were labeled as \textbf{LN}. 
%Table~\ref{fig:average-acc-mse-llm-performance} show the number of participants whose labeling accuracy and MSE improved, declined, or remained unchanged before and after using \system. Among 10 participants under LN, 7 participants improved their labeling accuracy, while only 2 participants under LY had improvement. 
%For labeling MSE, there was no difference between LY and LN participants in the number of individuals who showed improvement.


\subsubsection{Showing LLM explanations reduced labeling variation}
We observed that providing LLM explanations to participants led to more consistent labeling, as participants' labels became more similar to each other. 
Those with access to the LLM explanations had a higher Cohen's Kappa (0.333, SD=0.039), as well as higher Spearman (0.556, SD=0.053) and Kendall (0.504, SD=0.049) correlations compared to the group without access, whose Kappa was 0.193 (SD=0.047), Spearman 0.492 (SD=0.094), and Kendall 0.433 (SD=0.084).
%\kenneth{TODO Steven: Update the numbers.}
Additionally, Table~\ref{tab:results-llm-new} shows that the standard deviations (SD) of both ACC and MSE were systematically lower in the group with access to LLM explanations.
%\kenneth{TODO Steven: Update the reference.}\steven{done}

This suggests that \textbf{rather than users tailoring the LLM's behavior to their individual preferences, the LLM---through its explanations---encouraged users to align with its behavior}.


%\subsubsection{LLM explanations guides users, not users guide LLMs. (variation is bigger!)}
%We calculated Kappa under different conditions. 
%We found that participants who explored more data instances had the highest Kappa value of 0.135 (SD=0.073). 
%Participants who had access to the LLM explanation achieved the highest average inter-condition Kappa value of 0.333 (SD=0.039).





\begin{figure*}[t]
    \centering
    \begin{subfigure}[b]{0.49\linewidth}
        \centering
        \includegraphics[width=\linewidth]{Figures/Results/new-similarity/Rule-Simiarity-Metrics-data-sample.png}\Description{This is a subplot presenting rule similarity metrics for data sample and explanation group. Each subplot includes two density graphs representing normalized edit similarity and semantic similarity. For the data sample group, participants reviewing 50 instances per iteration tend to exhibit lower similarity within their own rules.}
        \caption{Data Sample Similarity Metrics}
        \label{fig:data-similarity}
    \end{subfigure}
    % \hfill
    \begin{subfigure}[b]{0.49\linewidth}
        \centering
        \includegraphics[width=\linewidth]{Figures/Results/new-similarity/Rule-Simiarity-Metrics-explanation.png}\Description{This is a subplot presenting rule similarity metrics for data sample and explanation group. Each subplot includes two density graphs representing normalized edit similarity and semantic similarity. In the explanation group, participants without access to LLM explanations generally show lower similarity within their own created rules.}
        \caption{Explanation Similarity Metrics}
        \label{fig:explanation-similarity}
    \end{subfigure}
    \caption{Comparison of Rule Similarity Metrics for Data Samples and Explanations. Overall, participants in the 50-instance group or those without access to LLM explanations were more likely to modify their rules.}
    \label{fig:rule-similarity-metrics}
\end{figure*}


\subsection{Additional Analysis: How Does Each Variable Influence Rule Editing?} %\steven{Rule Cos Sim}

Building on the impact of the two variables on prompting outcomes (RQ 1-2 and 1-3), a key follow-up question is \textit{why} these variables affect the outcomes differently. 
To explore this, we examined how the sample size per iteration and the presentation of LLM explanations influence how users edit the labeling rules---one of the main components of the final prompt---in \system.
For each prompt collected, we compiled all rules written by participants in the Rule Book sheet into a single string. 
We then calculated the sentence-level similarity between prompts from consecutive iterations for each session (\eg, between the initial prompt and iteration 1, iteration 1 and iteration 2, and so on).
Using \citet{huang2023ting}'s analysis method, we measured similarity in two ways:
(1) \textbf{Normalized Edit Similarity}: Calculated as $(1-Normalized~Edit~Distance)$,
where higher scores indicate greater similarity~\cite{yujian2007normalized, luozhouyang_python_similarity}.
(2) \textbf{Semantic Similarity}: Measured as the cosine similarity between semantic representations generated with Sentence-BERT~\cite{reimers-2019-sentence-bert}.
Each participant yielded eight similarity scores (2 similarity metrics $\times$ 4 pairs).
We used Kernel Density Estimation (KDE) to visualize the distribution of similarity between rules across consecutive iterations, as shown in Figure~\ref{fig:rule-similarity-metrics}. 
A similarity score of 1.0 indicates no changes, while 0.0 represents substantial modifications. 
Each chart compares participants in the 10-instance group to the 50-instance group or those with versus without access to LLM explanations.

\subsubsection{Larger rule changes were linked to better prompting outcomes.}
Our analysis revealed an interesting pattern: 
conditions that had better prompting outcomes---showing more data items, or not displaying LLM explanations---tended to have \textit{lower} similarity between labeling rules across consecutive iterations. 
In Figure~\ref{fig:rule-similarity-metrics}, the KDE curves for the 50-instance group (Figure~\ref{fig:rule-similarity-metrics}a) and the no-explanation group (Figure~\ref{fig:rule-similarity-metrics}b) skew further left compared to their counterpart conditions, regardless of the similarity metric. 
This indicates that participants in these settings---when seeing more data items, or when not having access to LLM explanations---made \textit{larger} changes to the rule books.
% Using the Kolmogorov-Smirnov (KS) test, only one significant difference was found in the normalized edit similarity between the group with LLM explanation access and the group without (p-value=0.031*). Specifically, participants without access to LLM explanations made significantly more frequent and substantial revisions to their rules than participants who had access to LLM explanations.
Using the Kolmogorov-Smirnov (KS) test, only one significant difference was found in the normalized edit similarity between the group with LLM explanation access and the group without (p-value=0.031*). Specifically, \textbf{participants without access to LLM explanations made significantly more frequent and substantial revisions to their rules} than participants who had access to LLM explanations.
This finding suggests an intriguing implication for human-LLM interaction: 
proactive and frequent revisions during iterative prompt refinement lead to better outcomes compared to making fewer revisions.
Encouraging meaningful revisions in prompting-in-the-dark scenarios---where no gold labels are available to guide or ``reward'' users---presents an interesting challenge for HCI research.

%when participants actively make more revisions during iterative prompt refinement, the outcomes improve. 
%In other words, humans have the ability to move closer to their intended outcomes in prompting-in-the-dark scenarios, though some conditions may not sufficiently encourage them to act on these abilities.




















%-------------- dead kitten ---------------

\begin{comment}
\paragraph{Rule Similarity Sample Size}
In the two charts in Figure~\ref{fig:data-similarity}, the curves for participants in the 50-instance group are flatter and skewed to the left, indicating that their rules tend to differ more from those in the previous iteration.
This suggests that users in this group are likely to make more edits to their rules.
Furthermore, participants in the 10-instance group exhibit a higher density near similarity values of 1.0 compared to those in the 50-instance group. This suggests that participants with access to fewer data samples per iteration are more likely to leave their rules unchanged. 

\paragraph{Rule Similarity Explanation} In the two charts shown Figure~\ref{fig:explanation-similarity}, the curves for participants did not have access to LLM explanations are flatter and skewed to the left.
This indicates that their rules have higher likelihood differ from those in the previous iteration, suggesting that these participants are more likely to make modifications to their rules.
Moreover, at similarity values near 1.0, participants with access to LLM explanations exhibit a higher density compared to those without access. This suggests that participants with LLM explanations are more likely to leave their rules unchanged.








We visualized the similarities of rule editing using Kernel Density Estimation (KDE), as shown in Figure~\ref{fig:rule-similarity-metrics}. A similarity score of 1.0 indicates that the rule remained unchanged from the previous iteration, while a score of 0.0 signifies significant modifications. Each chart compares either participants in the 10-instance group versus the 50-instance group or participants with access to LLM explanations versus those without.










Building on \citet{huang2023ting}'s analysis of creative writing editing, we employed two sentence-level similarity metrics to examine the rule editing performed by each participant: (1) \textbf{Normalized Edit Similarity}~\cite{yujian2007normalized, luozhouyang_python_similarity}, calculated as  1 - \textit{NormalizedEditDistance}, where higher scores indicate greater similarity; (2) \textbf{Semantic Similarity}, measured as the cosine similarity between semantic representations generated using Sentence-BERT~\cite{reimers-2019-sentence-bert}

We analyzed the rules defined by participants. 
Each participant iteratively reviewed and modified their rules across four rounds, leading to a total of five distinct versions. 
Before analyzing the rule difference, we aggregated  five task-specific label rules into a single rule for each version. \steven{is my description clear?}
To evaluate changes, we calculated the differences between rules from consecutive iterations (\eg iteration 0 to iteration 1 or iteration 2 to iteration 3), generating four unique similarity scores per participant. 
All comparisons were conducted within each participant's own set of rules.

We visualized the similarity levels of rule editing using Kernel Density Estimation (KDE), as shown in Figure~\ref{fig:rule-similarity-metrics}. A similarity score of 1.0 indicates that the rule remained unchanged from the previous iteration, while a score of 0.0 signifies significant modifications. Each chart compares either participants in the 10-instance group versus the 50-instance group or participants with access to LLM explanations versus those without.

    
\end{comment}

%\kenneth{Mixed results and somewhat negative... maybe not.}

\subsection{RQ 2: Can automatic prompt optimization tools like DSPy improve human performance in ``prompting in the dark'' scenarios?}

\begin{table*}[t]
\centering
\small
\begin{tabular}{lccccccccccc}
 & \multicolumn{3}{c}{\multirow{2}{*}{\textbf{Human}}} & \multicolumn{8}{c}{\textbf{DSPy}} \\ \cline{5-12} 
 & \multicolumn{3}{c}{} & \multicolumn{2}{c}{\textbf{\begin{tabular}[c]{@{}c@{}}Simple\\ Prompt\end{tabular}}} & \multicolumn{2}{c}{\textbf{\begin{tabular}[c]{@{}c@{}}Bootstrap-\\ FewShots\end{tabular}}} & \multicolumn{2}{c}{\textbf{COPRO}} & \multicolumn{2}{c}{\textbf{MIPRO}} \\ \hline
\multicolumn{1}{c}{} & \textbf{\begin{tabular}[c]{@{}c@{}}Avg.\\ \#Shot\end{tabular}} & \textbf{ACC$\uparrow$} & \textbf{MSE$\downarrow$} & \textbf{ACC$\uparrow$} & \textbf{MSE$\downarrow$} & \textbf{ACC$\uparrow$} & \textbf{MSE$\downarrow$} & \textbf{ACC$\uparrow$} & \textbf{MSE$\downarrow$} & \textbf{ACC$\uparrow$} & \textbf{MSE$\downarrow$} \\ \hline
\textbf{Initial} & 0.00 & .542 & .782 & - & - & - & - & - & - & - & - \\ \hline
\textbf{1st Revision} & 2.52 & .536 & .795 & .533 & .864 & .526 & .915 & \underline{\textbf{.565}} & .822 & .527  & .973 \\
\textbf{2nd Revision} & 4.80 & .542 & .833 & .533 & .864 & .534 & .934 & .538 & .873 & .536 & .862 \\
\textbf{3rd Revision} & 7.29 & \underline{\textbf{.549}} & .823 & .533 & .864 & \underline{\textbf{.550}} & .850 & \underline{\textbf{.544}} & .801 & \underline{\textbf{.547}} & .873 \\
\textbf{4th Revision} & 10.04 & \underline{\textbf{.553}} & .810 & .533 & .864 & .526 & .889 & \underline{\textbf{.554}} & .809 & .536 & .874 \\ \hline
\textbf{\begin{tabular}[c]{@{}l@{}}End of Session\\ (Avg \#Iter=4.75)\end{tabular}} & 11.14 & \underline{\textbf{.546}} & .815 & .533 & .864 & .535 & .873 & .528 & .817 & .528 & .868
\end{tabular}
\caption{Comparison of ACC and MSE between users' original prompts and prompts improved by four DSPy approaches. DSPy showed limited effectiveness in enhancing ACC or MSE, potentially due to the small number of gold shots. Improvements over the initial prompt are bolded and underlined.}
\label{tab:dspy-results}
\end{table*}


\begin{figure*}[t]
    \centering
    \begin{subfigure}{0.48\textwidth}
        \includegraphics[width=\linewidth]{Figures/Results/Participant_DSPy/participant_dspy_ACC_plot.png}\Description{This is an average accuracy plots subfigure of the participants prompt performance. There are scatter points for each DSPy algorithm from 1 to 4 iterations.}
        \caption{Average ACC of participants' prompts (Original) compared to prompts improved by DSPy's four approaches across each iteration.}
        \label{fig:dspy-plot-acc}
    \end{subfigure}
    \hfill
    \begin{subfigure}{0.48\textwidth}
        \includegraphics[width=\linewidth]{Figures/Results/Participant_DSPy/participant_dspy_MSE_plot.png}\Description{This is a MSE accuracy plots subfigure of the participants prompt performance. There are scatter points for each DSPy algorithm from 1 to 4 iterations.}
        \caption{Average MSE of participants' prompts (Original) compared to prompts improved by DSPy's four approaches across each iteration.}
        \label{fig:dspy-plot-mse}
    \end{subfigure}
    
    \caption{Average performance comparison between participants' prompts (Original) and those improved by DSPy's four approaches. DSPy was not effective in enhancing ACC or MSE, potentially due to the small number of gold shots.}
    %\kenneth{The space between these two subfigures should be wider. Captions stitch together....}\steven{done}
    \label{fig:dspy-four-settings-charts}
\end{figure*}

\begin{figure*}[t]
    \centering
    \includegraphics[width=0.92\linewidth]{Figures/Results/New-Participant-DSPy/new_acc_plots_dspy_v2.png}\Description{It contains 20 subfigures of average ACC of each participant plus DSPy Bootstrap scatter.}
    \caption{ACC of all participants compared to DSPy's BootstrapFewShots approach in each iteration. DSPy was not reliable in providing consistent improvements. (Some DSPy dots are missing because participants did not provide examples required for generating augmented samples in those iterations.)}
    \label{fig:acc-participant-plus-dspy}
\end{figure*}

%\steven{done. Make the plot a little transparent to ensure the DSPy dots covered by plot line more visible}

\begin{figure*}
    \centering
    \includegraphics[width=0.92\linewidth]{Figures/Results/New-Participant-DSPy/new_mse_plots_dspy_v2.png}\Description{It contains 20 subfigures of average MSE of each participant plus DSPy Bootstrap scatter.}
    \caption{MSE of all participants compared to DSPy's BootstrapFewShots approach in each iteration. DSPy was not reliable in providing consistent improvements. (Some DSPy dots are missing because participants did not provide examples required for generating augmented samples in those iterations.)}
    \label{fig:mse-participant-plus-dspy}
\end{figure*}

%\steven{done}\steven{todo: missing dots. done}

The previous section demonstrated that only a few settings were effective in helping participants improve prompt performance. 
This raises the question of how automatic prompt optimization tools might assist with refining prompts.
In our study, we explored DSPy~\cite{khattab2023dspy}, a framework designed to algorithmically optimize LLM prompts, to enhance the prompt at each stage of revision by participants. 
DSPy is particularly effective at working with small sets of labeled data and abstract, generic initial prompts, making it well-suited for the ``prompting in the dark'' scenario.

\paragraph{Study Setups.}

%Specifically, 
We experimented with the following four approaches offered by DSPy:

\begin{itemize}
\item 
\textbf{Simple Prompt (Baseline):}
This approach uses the abstract prompts constructed by DSPy and employs DSPy's simplest teleprompter, BootstrapFewShots, to generate optimized examples based on all the few-shot examples labeled by participants throughout the study session. 
It is a simple method that does not account for differences between iterations or the user's initial prompt, making it a baseline approach for using DSPy.
\item 
\textbf{BootstrapFewShots:}
This approach uses the task context (from Context sheet in \system), label definitions (Rule Book sheet), and few-shot examples (Shots sheet) provided by participants in each iteration, and applies DSPy's simplest teleprompter, BootstrapFewShots, for optimization. 
The BootstrapFewShots teleprompter automatically generates optimized examples to be included in the user-defined prompt based on the provided few-shot examples.
BootstrapFewShots is recommended when only a small amount of labeled data is available, such as 10 examples.\footnote{The recommended amount of data is based on DSPy's documentation: https://dspy-docs.vercel.app/docs/building-blocks/optimizers}
\item 
\textbf{COPRO:}
This approach is identical to the BootstrapFewShots setup but uses the COPRO teleprompter instead. 
The COPRO teleprompter focuses on optimizing the prompt instructions while keeping the few-shot examples constant. 
This enables the generation of more refined prompt instructions, even when labeled examples are limited or absent.
\item 
\textbf{MIPRO:}
This approach is identical to the BootstrapFewShots setup but uses the MIPRO teleprompter instead. 
MIPRO combines the features of both COPRO and BootstrapFewShots, refining the prompt instructions while also generating optimized examples using the provided few-shot data. 
MIPRO is recommended when a slightly larger amount of labeled data is available, such as 300 examples or more.
\end{itemize}

All the prompts were optimized with the goal of maximizing accuracy (ACC).

%\kenneth{Not really!}
%\subsubsection{Using DSPy}

\subsubsection{DSPy was not effective in improving ACC or MSE, possibly due to the small number of gold shots}
%Table~\ref{tab:dspy-results}
As shown in Table~\ref{tab:dspy-results} and Figure~\ref{fig:dspy-four-settings-charts}, none of the DSPy algorithms consistently improved ACC or MSE.
This may be attributed to the limited number of gold labels generated in our study, which was insufficient for DSPy to operate effectively, especially given the difficulty of a 5-class classification task. 
Table~\ref{tab:dspy-results} details the average number of gold shots used for DSPy in each revision.

\subsubsection{DSPy was not reliable in delivering consistent improvements}
Figure~\ref{fig:acc-participant-plus-dspy} and Figure~\ref{fig:mse-participant-plus-dspy} display the performance of DSPy's BootstrapFewShots (red dots) alongside human performance (blue line, which is the same as in Figure~\ref{fig:average-acc-mse-performance} for RQ1.)
These results indicate that DSPy's performance was unreliable across participants, as none of the DSPy algorithms consistently improved accuracy or MSE. 
We chose to plot the results for BootstrapFewShots because there was no significant difference across the four DSPy approaches we tested, and BootstrapFewShots was specifically recommended by DSPy's documentation when working with only 10 examples.
Some DSPy dots were missing because participants did not provide examples prior to those iterations, which was necessary for the DSPy algorithm to generate new augmented samples for prompt optimization. 
%\steven{respond to missing dots}










%-------------- dead kitten --------------
\begin{comment}


The previous section showed only certain settings could help participants on improving prompt performance. It would be interesting to explore the potential impacts of using an automatic prompt optimizing tool, such as DSPy~\cite{khattab2023dspy}, to assist with refinement. 


We then analyzed the DSPy fine-tuned prompts under different conditions in Table~\ref{tab:dspy-general-conditions-t-test}. LN-50Y demonstrated a significantly higher labeling accuracy improvement than other conditions, while LN-50N showed a significant drop in accuracy. In terms of MSE, both LY-50Y and LY-50N showed significant declines in MSE than LN-50Y and LN-50N. 

We further narrowed it down into different conditions for each DSPy algorithm in Table~\ref{tab:dspy-all-conditions-t-test}. For Bootstrap, LN-50N dropped the accuracy significantly than LY-50Y and LN-50Y. LY-50Y demonstrated a significant drop in MSE compared to LN-50Y and LN-50N. 
For Simple Prompt, LN-50N performed a significant dropping in labeling accuracy compared to other conditions, and LY-50Y had a significant drop in MSE compared to the rest of the conditions.  
For Mipro, LY-50Y showed a significant improvement in labeling accuracy compared to LY-50N and LN-50N. LN-50Y showed a notable improvement in accuracy over LN-50N and also achieved a significantly better MSE improvement compared to both LY-50Y and LY-50N.
In Copro, there was no significant difference between conditions for labeling accuracy. As for MSE, LY-50N exhibited a significant drop in conditions including LN, and LN-50N demonstrated a significant improvement in conditions containing LY.

In summary, DSPy algorithms demonstrated a greater improvement in accuracy and MSE for participants who \textbf{did not access the LLM explanation}. DSPy optimizers were more effective in helping participants with 50 instances per iteration enhance labeling accuracy, while participants reviewing fewer instances per round saw a more pronounced improvement in MSE. 

\begin{table}[ht]
    \centering
    \begin{tabular}{lcccc|cccc}
        \hline
        \multicolumn{1}{l} {DSPy} & \multicolumn{1}{c}{\multirow{2}{*}{\textbf{ACC$\uparrow$}}} & \multicolumn{3}{c|}{\textbf{T-Test over Avg. Changed ACC}}& \multicolumn{1}{c}{\multirow{2}{*}{\textbf{MSE$\downarrow$}}} & \multicolumn{3}{c}{\textbf{T-Test over Avg. Changed MSE}}
        \\ \cmidrule(lr){3-5} \cmidrule(lr){7-9} 
        \multicolumn{1}{l} {Changes} & & LY-50N & LN-50Y & LN-50N & & LY-50N & LN-50Y & LN-50N\\
        \hline
        LY-50Y & .012 & .060 & .033* & <.001***      & .114 &  .520  & <.001*** & <.001**   \\ \hline
        LY-50N  & -.006 & -      & <.001*** & <.001***     & .092 & -      & .020* & .002**   \\ \hline
        LN-50Y  & .036 & -      & -    & <.001***       & .003 & -      & -    & .215   \\ \hline
        LN-50N   & -.054 & -      & -    & -           & -.047 & -      & -    & -      \\ \hline
    \end{tabular}
    \caption{DSPy t-test over different conditions for all optimizers.}
    \label{tab:dspy-general-conditions-t-test}
\end{table}




% DSPy improved labeling accuracy on average for LN-50Y participants and consistently enhanced MSE when the Mipro and Copro optimizers were applied. 
% Interestingly, for participants under the LN-50N condition, DSPy worsened accuracy but contributed to the highest improvement in MSE.
% Overall, DSPy performed the best on prompting refinement on LN-50Y.
\begin{figure}
    \centering
    \begin{subfigure}{0.85\textwidth}
    \includegraphics[width=\linewidth]{Figures/Results/Participant_DSPy/avg_acc_diff_dspy.png}
    \caption{DSPy Average Improvement on ACC.}
    \label{fig:avg-acc-dspy-improvement}
  \end{subfigure}
  \hfill
  \begin{subfigure}{0.85\textwidth}
    \includegraphics[width=\linewidth]{Figures/Results/Participant_DSPy/avg_mse_diff_dspy.png}
    \caption{DSPy Average Improvement on MSE.}
    \label{fig:avg-mse-dspy-improvement}
  \end{subfigure}
  \caption{DSPy Average Improvement on ACC and MSE under different conditions.}
  \label{fig:average-acc-mse-dspy-improvement}
\end{figure}


\begin{table}[ht]
    \centering
    \begin{tabular}{l|cccccccccccc}
        \hline
         & \multicolumn{2}{c|}{\textbf{Human}} & \multicolumn{2}{c|}{\textbf{Human+DSPy}}& \multicolumn{2}{c|}{\textbf{Bootstrap}}& \multicolumn{2}{c|}{\textbf{Simple Prompt}}& \multicolumn{2}{c|}{\textbf{Mipro}}& \multicolumn{2}{c|}{\textbf{Copro}} \\
         \cmidrule(lr){2-3} \cmidrule(lr){4-5} \cmidrule(lr){6-7} \cmidrule(lr){8-9} \cmidrule(lr){10-11} \cmidrule(lr){12-13}  
         & \multicolumn{1}{c|}{ACC$\uparrow$} & \multicolumn{1}{c|}{MSE$\downarrow$} & \multicolumn{1}{c|}{ACC$\uparrow$} & \multicolumn{1}{c|}{MSE$\downarrow$}& \multicolumn{1}{c|}{ACC$\uparrow$} & \multicolumn{1}{c|}{MSE$\downarrow$}& \multicolumn{1}{c|}{ACC$\uparrow$} & \multicolumn{1}{c|}{MSE$\downarrow$}& \multicolumn{1}{c|}{ACC$\uparrow$} & \multicolumn{1}{c|}{MSE$\downarrow$}& \multicolumn{1}{c|}{ACC$\uparrow$} & \multicolumn{1}{c|}{MSE$\downarrow$}\\
         \hline
        \multicolumn{1}{l|}{\textbf{Start}} & \multicolumn{1}{c|}{0.542} & \multicolumn{1}{c|}{0.782} & \multicolumn{1}{c|}{-}& \multicolumn{1}{c|}{-}& \multicolumn{1}{c|}{-}& \multicolumn{1}{c|}{-}& \multicolumn{1}{c|}{-}& \multicolumn{1}{c|}{-}& \multicolumn{1}{c|}{-}& \multicolumn{1}{c|}{-}& \multicolumn{1}{c|}{-}& \multicolumn{1}{c|}{-}  \\
        \multicolumn{1}{l|}{\textbf{1st}}    & \multicolumn{1}{c|}{0.536}	 & \multicolumn{1}{c|}{0.795}	 & \multicolumn{1}{c|}{0.538}	 & \multicolumn{1}{c|}{0.889}	 & \multicolumn{1}{c|}{0.526}	 & \multicolumn{1}{c|}{0.915}	 & \multicolumn{1}{c|}{0.533}	 & \multicolumn{1}{c|}{0.864}	 & \multicolumn{1}{c|}{0.527}	 & \multicolumn{1}{c|}{0.973}	 & \multicolumn{1}{c|}{0.565}	 & \multicolumn{1}{c|}{0.822}   \\
        \multicolumn{1}{l|}{\textbf{2nd}}    & \multicolumn{1}{c|}{0.542}	 & \multicolumn{1}{c|}{0.833}	 & \multicolumn{1}{c|}{0.535}	 & \multicolumn{1}{c|}{0.883}	 & \multicolumn{1}{c|}{0.534}	 & \multicolumn{1}{c|}{0.934}	 & \multicolumn{1}{c|}{0.533}	 & \multicolumn{1}{c|}{0.864}	 & \multicolumn{1}{c|}{0.536}	 & \multicolumn{1}{c|}{0.862}	 & \multicolumn{1}{c|}{0.538}	 & \multicolumn{1}{c|}{0.873}   \\
        \multicolumn{1}{l|}{\textbf{3rd}}    & \multicolumn{1}{c|}{0.549}	 & \multicolumn{1}{c|}{0.823}	 & \multicolumn{1}{c|}{0.544}	 & \multicolumn{1}{c|}{0.847}	 & \multicolumn{1}{c|}{0.550}	 & \multicolumn{1}{c|}{0.850}	 & \multicolumn{1}{c|}{0.533}	 & \multicolumn{1}{c|}{0.864}	 & \multicolumn{1}{c|}{0.547}	 & \multicolumn{1}{c|}{0.873}	 & \multicolumn{1}{c|}{0.544}	 & \multicolumn{1}{c|}{0.801}  \\
        \multicolumn{1}{l|}{\textbf{4th}}    & \multicolumn{1}{c|}{0.553}	 & \multicolumn{1}{c|}{0.810}	 & \multicolumn{1}{c|}{0.537}	 & \multicolumn{1}{c|}{0.859}	 & \multicolumn{1}{c|}{0.526}	 & \multicolumn{1}{c|}{0.889}	 & \multicolumn{1}{c|}{0.533}	 & \multicolumn{1}{c|}{0.864}	 & \multicolumn{1}{c|}{0.536}	 & \multicolumn{1}{c|}{0.874}	 & \multicolumn{1}{c|}{0.554}	 & \multicolumn{1}{c|}{0.809}  \\
        \hline
        \multicolumn{1}{l|}{End of Session} & \multicolumn{1}{c}{\multirow{2}{*}{0.546}}& \multicolumn{1}{c}{\multirow{2}{*}{0.815}}& \multicolumn{1}{c}{\multirow{2}{*}{0.537}}& \multicolumn{1}{c}{\multirow{2}{*}{0.859}}& \multicolumn{1}{c}{\multirow{2}{*}{0.526}}& \multicolumn{1}{c}{\multirow{2}{*}{0.889}}& \multicolumn{1}{c}{\multirow{2}{*}{0.533}}& \multicolumn{1}{c}{\multirow{2}{*}{0.864}} & \multicolumn{1}{c}{\multirow{2}{*}{0.536}}& \multicolumn{1}{c}{\multirow{2}{*}{0.874}}& \multicolumn{1}{c}{\multirow{2}{*}{0.554}}& \multicolumn{1}{c}{\multirow{2}{*}{0.809}}\\
        \multicolumn{1}{l|}{Avg \#Iter (4.75)}\\
        \hline
    \end{tabular}
    \caption{Average Performance per iteration for Participants, Participant plus DSPy, and four different DSPy algorithms.}
    \label{tab:participant-explanation-detail}
\end{table}


        
        
        % \multicolumn{1}{l}{\textbf{End of Session}} \\



 % & \multicolumn{1}{c|}{}
 % & \multicolumn{1}{c|}{}
 % & \multicolumn{1}{c|}{}
 % & \multicolumn{1}{c|}{}
 % & \multicolumn{1}{c|}{}




Figure~\ref{fig:acc-participant-plus-dspy} and Figure~\ref{fig:mse-participant-plus-dspy} show that DSPy performance is unstable as none of the DSPy algorithms consistently improve accuracy or MSE.
When comparing the improvement in prompt labeling accuracy and MSE among all participants, there is \textbf{no significant difference} within each model. 










In our study, in addition to relying entirely on users' judgements, we also experimented with DSPy, a framework for algorithmically optimizing LLM prompts, to attempt to improve human performance of the prompt.
DSPy can work with small set of user labeled data and abstract, generci initial prompt, which fit in the ``prompting in the dark'' scenario.
In particular, we experimented the following four approaches provided by DSPy:

Althought


, explored a wide range of prompt refinement algorithms provided in .\kenneth{Add citations}
W


a system designed to optimize prompt engineering for large language models (LLMs) by providing various teleprompters that refine prompts and examples based on user input and task requirements. 
Our goal was to identify the best-performing prompt for each user iteration, with a focus on optimizing for accuracy.


    
\end{comment}


%\subsection{Additional Analysis}
%\kenneth{Not sure we need this but let me put it here for now.}
%When pairing LN and 50Y, 4 out of 5 participants improved their labeling accuracy. When combining LY and 50N, all participants' labeling accuracy dropped. In MSE, combinations including 50Y are slightly better than combinations with 50N. 

%In short, not having access to LLM explanations (LN) and exploring 50 instances per iteration (50Y) are the best strategy combinations in \system. 



%\kenneth{-----------------------KENNETH IS WORKING HERE------------------}

%In this section, we first overview the comparative results of human-refined and DSPy-refined prompts (Section XX) and then show the results of incorporating DSPy into human-refined prompts (Section XX). 

%\subsection{\system Results}


% \subsection{In-lab Human Refined Prompts}

