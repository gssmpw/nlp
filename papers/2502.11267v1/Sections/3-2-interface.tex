
\system is a Google Sheets add-on that enables users to load data, sample a subset for labeling, compose and edit prompts, use these prompts to request LLMs for data labeling, and iteratively revise the prompts. 
Figure~\ref{fig:system-interface-ui-kenneth} shows the interface of \system.

\paragraph{Sidebar.}
Following Google Sheets' design constraints, all functions are presented as manuals and buttons within the sidebar on the right. 
The sidebar remains consistent across all sheets, regardless of which sheet is in use. 
At the top of the sidebar, \system provides a real-time notification that keeps users informed about its ongoing processes, such as ``Data Indexing,'' ``Data Sampling,'' ``Generating the Instructional Prompt,'' or ``Annotating.''


\paragraph{Pre-Defined Sheets.}
\system includes a set of predefined spreadsheets, each with a set of pre-defined columns. 
At the bottom of the interface, a series of tabs allows users to switch between sheets, with each sheet dedicated to a different part of the data labeling process. 
The following describes each sheet in detail. 
(To help readers easily identify which sheet we are referring to, we indexed each sheet as A, B, C, ..., and G in all the figures. 
These indexes were not present in the actual system to users.)

%\subsubsection{All the Sheets and What They Do}

%\hyperref[fig:system-interface-1]{Figure 1} overviews the \system's user interface. 
%The system consists of seven main components:

\begin{itemize}

\item \textbf{Dataset (Sheet A)}:
%The ``Dataset'' includes three columns: Data ID, Group ID, and Data Instance. 
%Each data instance is uniquely identified by a corresponding Data ID.
%A single Group ID may encompass one or multiple Data Instances.
This spreadsheet stores the full dataset.
Users can copy and paste the dataset into this sheet or use any supported Google Sheets import method (in Step 0).
The sheet includes three key predefined columns: (1) Data ID, (2) Group ID, and (3) Data Instance. 
Each data instance is uniquely indexed by its corresponding Data ID, which users can generate by clicking the ``Index Data ID'' function in the sidebar. 
The Group ID is used for annotating sequential data, such as when each sentence in an article is treated as a separate data instance, but all sentences from the same article share the same Group ID.
In our design, this sheet is intended to serve as a static data source, and we anticipate that users will not modify it after loading the data.

%\kenneth{TODO: Maybe add words to mention we don't expect people touch it after Step 0 and basicallyy serve as a database.}

\item \textbf{Task Context (Sheet B)}:
%The ``Context'' tab provides information to help describe the annotation task users are working on. It addresses questions related to the purpose and application of the data annotation task and the origin and size of each data instance. The LLM will use the information provided in this tab to generate an instructional prompt for a later step.
This spreadsheet stores the meta-information and context for the labeling task, which will later be incorporated into the prompt. 
The sheet includes predefined questions that characterize the task, such as the purpose of the data labeling, how the labels will be used, the source of the data, and the size of each data instance.
Table~\ref{tab:task-sheet-questions} in Appendix~\ref{sec:context-question-appendix} shows all the questions.
%\kenneth{TODO: Maybe add all the questions to Appendix.}
Users provide answers to these questions (in Step 1 or 4), and \system automatically incorporates both the questions and their answers into the prompt used for LLMs to label the data.


    
\item \textbf{Rule Book (Sheet C)}: 
%The ``Rule book'' tab is where users define the criteria and definitions for each label used during the annotation process.
This spreadsheet contains the labeling rules that the LLM will follow.
It includes two key predefined columns: (1) Label Name and (2) Rules for the Label. 
Users manually define the criteria and descriptions for each label in free text (in Step 1 or 4), detailing the guidelines for the annotation process. 
Multiple rules can be added for a single label, providing flexibility in defining the labeling criteria.


    
\item \textbf{Shots (Sheet D)}: 
%In the ``Shots'' tab, users can enter gold standard labels from their iterations or manually provide them as reference points.
This spreadsheet stores all the high-quality examples, including data instances and their corresponding labels, which will be included in the prompt to guide the LLM in labeling the data. 
These examples, commonly referred to as ``shots'' in prompts, follow the same predefined column structure, with an additional ``Gold-Standard Label'' column. 
Users can add these examples manually (in Step 1) or use \system's function to do so (in Step 4).


    
\item \textbf{Working Data Sample (Sheet E)}:
%Users can sample the data from the ``Dataset'' tab to the ``Working Data Sample'' tab. In the annotation process, \textit{only} data instance in the ``Working Data Sample'' tab will be annotated.
% by the LLMs using the instructional prompt, provided rules, and gold shots.
This spreadsheet stores the current subset of data selected from the full dataset, ready for the LLM to label.
Users can sample data from the Dataset sheet by clicking the corresponding buttons in the sidebar; users can choose between random sampling or selecting a specific range (Step 2). 
During the annotation process, only the data instances in the Working Data Sample sheet will be labeled. 
\system will copy the entire data sample from the Working Data Sample sheet to create a new sheet to label (Step 3).
    
    
\item \textbf{Task Dashboard (Sheet F)}:
%The ``Task Dashboard'' tab records all iteration task details such as task number, timestamp, used prompt, and total costs. 
This spreadsheet tracks all labeling tasks performed so far.
When the user clicks the ``Start Annotation'' button in the sidebar (in Step 3), \system creates a new sheet for the task (e.g., Task 1 sheet) and adds a new row in the Task Dashboard to record the labeling activities.
Task Dashboard sheet (Figure~\ref{fig:task-dashboard-new})
logs task details such as task number, timestamp, the prompt used, and total costs.

\item \textbf{Task 1 (Sheet G), Task 2, ..., Task N}:
%After each annotation, the annotation results will be saved in a new tab (e.g., Task\_1, Task\_2, etc) corresponding to that specific iteration. 
Each of these sheets stores the annotation results for each labeling request, including data samples, LLM-generated labels, and LLM explanations (optional).
These sheets also include columns that allow users to validate or correct the LLM labels and optionally add them to the Shots sheet (in Step 4).
When the user clicks the ``Start Annotation'' button in the sidebar (in Step 3), \system generates a new task sheet to handle the specific labeling task.

    
    
\end{itemize}

%\kenneth{Users are allowed to add new columns.}

Notably, while users must follow our guidelines for using the predefined columns in each sheet and inputting data correctly, they are free to add more columns or even additional sheets, just as they would in a regular Google Sheets document. 
For instance, when pasting a dataset into the Dataset sheet, it is common for the dataset to include its own IDs or additional information for each data entry. 
Users can easily store this extra information by creating new columns within the Dataset sheet.


%---------- dead kitten ----------

\begin{comment}



\subsubsection{Other Features} \steven{todo: add figures in the appendix. screenshots for different notification messages. interface screenshots for removal and clearing.}
\begin{itemize}
    \item \textbf{Real-Time System Notification: }\system provides a notification feature that informs users of its current processes, such as ``Data Indexing'', ``Data Sampling'', ``Generating the Instructional Prompt'', ``Annotating'', etc.
    \item \textbf{Remove Unselected Data Instance (Figure \ref{fig:remove-clear}): }This function will remove data instances that do not have the ``Keep it in the next data sample'' checked in the ``Working Data Sample'' tab.
    \item \textbf{Clear Data Instance (Figure \ref{fig:remove-clear}): }This function will clear all data instances in the ``Working Data Sample'' tab.
\end{itemize}

    
\end{comment}