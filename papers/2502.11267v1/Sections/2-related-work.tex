%\subsection{End-User Programming}

%\kenneth{The way I like to think about Related Work is that this section should (sometimes subtly, not explicitly, but effectively!) answer some underlying questions that reviewers might want to ask. So, here we go:}\steven{sounds good!}

\subsection{Ways of Optimizing Prompts for LLMs}
%\subsection{Prompt Engineering and How Good Humans Are at It}
Prompts are the primary means by which users interact with, utilize, and instruct LLMs. 
Since the emergence of these models, researchers and developers have invested significant effort into understanding how to craft better prompts for more effective use. 

\paragraph{Automatic Prompt Optimization.}
Much of the prior work has focused on automatically optimizing prompts. 
A common theme across these studies is the use of gold-standard labels to guide the optimization process.
For example, \citet{pryzant2023automatic} introduced an automatic prompt optimization method inspired by gradient descent; 
\citet{manas2024improving} presented an approach that begins with a user prompt and iteratively generates revised prompts to maximize consistency between the generated image and prompt, without human intervention; 
\citet{wan2024teach} explored two types of prompt optimization, instruction and exemplar, and suggested that combining both can yield optimal results; 
\citet{sun2023autohint} combined zero-shot and few-shot learning to optimize prompts automatically; %eliminating the need for manual prompt engineering; 
and \citet{levi2024intent} improved prompt optimization through synthetic data generation and iterative refinement, focusing on aligning prompts with user intent by creating challenging boundary cases for iterative prompt refinement.
While these studies were interesting and relevant, they generally assumed the availability of gold-standard labels and did not address situations where labels are absent or where standards are constantly evolving.

\paragraph{User-Driven Prompt Optimization.}
In addition to automatic prompt optimization, some research has focused on human capabilities in optimizing prompts. 
\citet{zhou2023revisiting} found that manual prompting often outperforms automated methods in various scenarios; 
\citet{10.1145/3544548.3581388} discovered that people tend to design prompts opportunistically rather than systematically, which often leads to lower success rates. 
To the best of our knowledge, the most relevant prior work is by \citet{wang2024end}, who developed an iterative refinement system that enables users to prompt LLMs to build a personalized classifier for social media content. 
Their study explored three user strategies for improving prompts and measured their effectiveness. 
While conceptually related to our work, their focus was not on how users evolve their understanding and expectations when interacting with LLMs. 
Instead, participants labeled ground truth at the beginning of the study, prior to using the system.



%--------------------- dead kitten --------------
\begin{comment}
 





The most relevant prior work is by \citet{wang2024end}, who developed an iterative refinement system allowing users to prompt LLMs to build a personalized classifier for social media content.
While their work is closely related to ours in concept, their study did not focus on how users evolve their understanding and expectations while working with LLMs. 
Instead, participants labeled ground truth at the outset before using the system.


\kenneth{The key question for our paper is this: Did prior work try to measure users' prompt engineering performance *over multiple iterations*? (What do we know about human performance in prompt engineering?) I think you can maybe find some papers, especially papers for automatic prompt optimization like DSPy, measuring users' individual prompt's output accuracy (or MSE) or performance (e.g., BLEU in generation task), but it might be hard to find papers capture and measure *multiple iterations* from the same user for the same prompt.--This is the main argument for our paper: we did something that was hard and thus has not been done.}

\kenneth{Take a look at this survey paper:~\cite{chen2023unleashing}}



\steven{iterative tool involve human}
PromptIDE is an interactive tool that helps the experts to iteratively refine tools by providing various prompts, visualizing their performance on small validation datasets, and iterative optimizing them based on quantitative feedback~\cite{strobelt2022interactive}. \steven{gold label exists}

PromptAID is a visual analytics system that helps non-experts iteratively improve prompts through exploration, perturbation, testing, and refinement. It supports prompts through keyword adjustment, paraphrasing, and adding few-shot examples. \steven{has test dataset, it is a complex system}

\steven{automate prompting}
\citet{pryzant2023automatic} introduces an automatic prompt optimization prompt inspired by gradient descent. \steven{this fell into software designing, involve gold labels}

The study starts from a user prompt and iteratively generates revised prompts with the goal of maximizing a consistency score between the generated image and prompt without a human in the loop\cite{manas2024improving}\steven{without human involvement in the loop, gold labels}

\citet{zhou2023revisiting} found that manual prompting often performed better than automated methods in various steps. 

\cite{wan2024teach} explores the distinction between two types of prompt optimization: instruction optimizer and exemplar. This study suggested combining both approaches could lead to optimal results.

\cite{sun2023autohint} combines zero-shot and few-shot learning to optimize prompts automatically, without manual efforts in prompt engineering.

\cite{levi2024intent} improve prompt engineering optimization by synthetic data generation and iterative refinement, focusing on aligning prompts with user intent by generating challenging boundary cases and using these to refine the prompt iteratively.





\paragraph{Prompt Engineering Tools.}
\kenneth{After making the first point, we can have a follow-up paragraph to say that many tools were created to help people do prompt engineering (list a few and name their focuses), but again, they did not focus on measuring how good humans are in prompt engineering--- Of course, there could be an argument that suggests: no matter how good you are, you will always need some tool. It is true---for example, ChainForge basically create a easy-to-use UI that make things easier, which is not really about accuracy---But for annotation tasks, performance is still critical and it is always good to know how well human did, almost like many AI leaderboard has various "human" performance for comparison.}
PromptMaker, a platform for rapidly prototyping new ML models using prompt-based programming, was difficult to evaluate their prompts systematically~\cite{10.1145/3491101.3503564}.

\cite{arawjo2024chainforge}  is an Open-source visual toolkit for prompt engineering and on-demand hypothesis testing of text-generation LLMs.

 promptfoo is test-driven LLM development, not trial-and-error, producing matrix views that let you quickly evaluate outputs across many prompts~\cite{webster2023promptfoo}.

\cite{madaan2024self} introduces a method that LLM iterative improve their output by using their own feedback, without external supervision. 

\saniya{austin etal points:
1. used only COPRO, evaluation criteria utilized a custom LLM-as-a-judge metric. The paper showed that their automated prompt optimizer worked better tha DSPy }
   
\end{comment}


\subsection{Tools for Prompt Engineering}
With the advances in LLMs, numerous tools have been developed to assist with prompt engineering. 
Most of these tools follow a software-engineering paradigm, where testing (such as unit tests or integration tests) is a central concept, and thus often assume the existence of gold-standard labels.
For example, PromptIDE is an interactive tool that helps experts iteratively refine prompts by providing various inputs, visualizing their performance on small validation datasets, and optimizing them based on quantitative feedback~\cite{strobelt2022interactive}; 
PromptAid is a visual analytics system for interactively creating, refining, testing, and iterating prompts while tracking accuracy changes~\cite{mishra2023promptaid};
%It allows users to adjust prompts through keyword modifications, paraphrasing, and adding few-shot examples; 
ChainForge is an open-source visual toolkit for prompt engineering and on-demand hypothesis testing of text-generation LLMs~\cite{arawjo2024chainforge};
and, promptfoo applies a test-driven approach to LLM development, producing matrix views that enable quick evaluation of outputs across multiple prompts~\cite{webster2023promptfoo}.
While these tools are inspiring and valuable, the scenarios we focus on do not rely on the constant availability of gold labels.

%\cite{mishra2023promptaid}


\begin{comment}






\kenneth{In here, we want to answer this questions: Why do we need to built \system? Can't we just use some existing tools??? The underlying answer could be: all the tools, including the one we mentioned in previous subsection, were not really aiming for ``general users'' and only thing general users can reliably use is probably chat interface come with ChatGPT etc.}

\citet{10.1145/3544548.3581388} mentioned that people tended to design prompts opportunistically, not systematically, which resulted in less success. \system provides a systematic process for composing and refining prompts, allowing non-expert users to adapt to the prompt creation process effortlessly.

\saniya{Amy Zhang points:
\newline 1. Accuracy didnot improve; reported improvements in recall
\newline 2. Observed that humans are pretty bad at being consistent
\newline 3. Quoted  Miles Turpin, Julian Michael, Ethan Perez, and Samuel Bowman. 2024. Language models don't always say what they think: unfaithful explanations
in chain-of-thought prompting. Advances in Neural Information Processing Systems 36 (2024).
Han Wang, Ming Shan Hee, Md Rabiul Awal, Kenny Tsu Wei Choo, and Roy Ka-Wei Lee. 2023. Evaluating GPT-3 Generated Explanations for
Hateful Content Moderation. arXiv:2305.17680 [cs.CL] for not using LLM prompt explanations
\newline 4. They had a bigger training set of around 700 examples: paper excerpt: "This process resulted in a balanced dataset of 800 comments. We randomly divided our dataset into a training dataset and a test dataset of 100 examples for each participant. The training dataset was used to help participants create their classiiers, whereas the test dataset was labeled by participants and used to evaluate their created classiiers."
}
    
\end{comment}

\subsection{Human-LLM Collaborative Data Annotation}
%Another relevant area of research involves using LLMs for data annotation. 
Beyond simply treating LLMs as automatic labelers---common in countless NLP projects~\cite{tan2024large}---a growing body of work explores how to combine human and LLM efforts to achieve better annotation outcomes, such as improved accuracy or speed.
Even as LLMs outperform humans in many labeling tasks, human-AI collaboration often produces better results than either alone~\cite{vaccaro2024combinations}.
For example, \citet{kim2024meganno+} introduced a human-LLM collaborative annotation system where LLMs handle bulk annotation tasks, while humans selectively verify and refine the annotations. 
%\steven{However, this system was limited to deployment within Jupyter Notebook, lacking an end-to-end solution. This design imposed significant barriers, as it required users to possess technical expertise for system setup before using the tool, limiting accessibility and scalability in non-technical domains.}
\citet{goel2023llms} proposed an approach that combines LLMs with human expertise to efficiently generate ground truth labels for medical text annotation.
Additionally, \citet{10.1145/3613904.3642834} demonstrated how aggregating crowd workers' labels with GPT-4's output can achieve higher labeling accuracy than either source alone.
These studies generally aim to split the workflow of data labeling between humans and LLMs in a smart way, making the task more effective or efficient. 

In contrast, our work does not focus on dividing or combining the workload, but on how humans can teach LLMs---through prompt refinement---to better label the specific type of data.
Few prior studies have emphasized iterative prompt refinement in human-LLM collaborative data annotation.
For example, \citet{liu2024harnessing} developed a workflow for video content analysis, refining prompts to improve LLM-generated annotations and align them with human judgment.
Additionally, \citet{zhang2023llmaaa} proposed LLMAAA, which uses LLMs as annotators in a feedback loop to label data efficiently.
Their study shows that poorly designed prompts result in subpar performance, especially in complex tasks. %while incorporating demonstrations and aligning label descriptions with natural language significantly enhances accuracy and reliability.
Our work advances this relatively understudied area of human-LLM collaborative annotation research.

%----------------------------- dead kitten --------------------------------

\begin{comment}








\steven{\citet{vaccaro2024combinations} emphaized that designing innovative processes for integrating humans and AI is as critical as developing advanced AI technologies. This aligns with the need for LLM-powered systems that iteratively guide AI outputs to meet user-specific standards, prioritizing effective collaboration between users and AI systems.}

\steven{\citet{liyanage2024gpt} found that GPT-4, using few-shot, zero-shot, and Chain-of-Thoughts (CoT) prompting techniques, could not outperform models fine-tuned on human-labeled data. Among these, the few-shot approach exhibited the highest degree of similarity to human annotations. However, in scenarios where gold labels are unavailable, fine-tuning is not applicable, and alternative methods must be explored.}

\steven{\citet{liu2024harnessing} developed a workflow for video content analysis, iteratively crafting prompts to enhance LLMs' ability to generate structured annotations and comprehensive explanations that aligned with human judgment. }

\steven{\citet{zamfirescu2023herding} found that while prompts can effectively address most UX goals, they struggle with nuanced, edge-case, or spontaneous interactions. The study highlights that the effectiveness of each instruction in the prompt is highly sensitive to its phrasing and location. Additionally, highly prescriptive prompts, though reliable, limited the spontaneity and flexibility of GPT responses.
In our system, users are only required to provide task information—such as task descriptions, rules, and examples—to construct instructions, allowing for greater flexibility in accommodating diverse task requirements..}

\steven{\citet{guyre2024prompt} illustrates how prompt engineering can empower non-experts to design tailored conversational agents by iteratively refining prompts and infusing domain-specific knowledge. Their study emphasizes democratizing chatbot development, allowing users to align AI behavior with their specific goals and values.}

\steven{\citet{zhang2023llmaaa} proposes LLMAAA that leverages LLMs as Active Annotators in a feedback loop to efficiently annotate data. The study highlights that poorly designed prompts lead to suboptimal performance by LLM annotators, particularly in complex or domain-specific tasks. However, incorporating demonstrations and aligning label descriptions with natural language significantly enhances annotation accuracy and reliability.}

%\kenneth{Here, we then answer this question: Did people create ANYTHING to support LLM-powered data annotation? There are two parts of the answer to this: 1) Many or even most papers, including our CHI paper last year, focus on the labeling performance of LLMs, for example, as compared to crowdsourcing. They did not focus on the UI aspect of it. 2) Some prompt chaining tools, like ChainForge, can support workflow like this, but (a) hey do not focus on data annotation in particular so some functions are missing, like data resampling, and (b) more importantly, they do not aim to support general users. Most of them expect you to know some programming, e.g., ChainForge clearly say it's a visual programming tool. They're not really aiming for generic users.}


\cite{kim2024meganno+} introduced a human-LLM collaborative annotation system that allows LLM to handle bulk annotation tasks while humans verify selectively to refine annotation. 

\cite{goel2023llms} introduced an approach that combines LLM wth human expertise to create an efficient method for generating
ground truth labels for medical text annotation.


\cite{shankar2024validates} introduced a tool, EvalGen, to address the challenge of validating LLM. 
EvalGen helps users design evaluation criteria for LLM outputs and align that evaluation with human preferences through a mixed-initiative system.
A key finding is the concept of criteria drift, where users modify their evaluation standards while grading outputs. 


\cite{brade2023promptify} Promptify utilizes an LLM-powered suggestion engine to help users quickly explore and craft diverse prompts for text-to-image generation tasks.

    
\end{comment}


%\subsection{Survey Study in Data Annotation}
%\steven{
We conducted a survey study to investigate how individuals interact with LLMs and utilize gold-standard labels in the data annotation process. 
The participants primarily represent roles in research, machine learning engineering, and software development. \\
\textbf{Workflows: }Participants described diverse workflows for integrating LLMs into data annotation process, highlighting a common iterative and human-in-the-loop approach. \textbf{Most workflows begin with manual annotation of a small subset of data to establish a baseline.} Participants then employ prompt engineering, iteratively refining LLM prompts by evaluating their performance against the manually annotated subset. \\
Once refined, the prompts are used to label larger datasets, with participants using tools or manual checks to review the LLM's annotations and identify any invalid labels. The process is typically concluded with a thorough manual verification of the dataset. \\
One participant mentioned they manually tabulate data points along with their descriptions. \\
\textbf{Initialize Prompting: }Most participants use their pre-defined prompts to initialized the annotation on their known tasks. 
For new tasks, one participant mentioned that they initialize the annotation process with LLMs by starting with a clear problem definition and iteratively refining a classification-based approach. For less familiar tasks, some participants may seek suggestions from the LLM to guide the initial setup.
\textbf{Revising Prompt: } Participants use a small dataset to finetune the prompt. They address issues by adding rules or context examples to tackle failure cases. When inconsistencies or error arise, they revisit and recheck the manually tagged dataset to improve performance. Some participants also engage the LLM by asking questions about data points and their descriptions, retraining to against inconsistencies to minimize hallucinations and enhance annotation reliability.
}

\subsection{Gold-Standard Labels in Annotation Tasks}\label{sec:related-work-gold-label}
Decades of research have shown that gold-standard labels play a critical role in quality control for data annotation pipelines~\cite{han2020crowd,gadiraju2015training,le2010ensuring,doroudi2016toward,hettiachchi2021challenge}.
Embedding items with known labels into the data annotation process allows requesters to reliably capture quality signals, 
such as workers' level of expertise~\cite{abraham2016many, abassi2019worker, yang2018improving} %\kenneth{TODO: Add refs about using gold labels to decide workers' expertise level}\steven{added}
or attentiveness to tasks~\cite{hettiachchi2021challenge, oleson2011programmatic}. %\kenneth{TODO: Add refs about using gold labels to do attention checks for workers}\steven{added}
These insights enable requesters to take appropriate actions, such as 
retraining annotators~\cite{le2010ensuring, doroudi2016toward,hettiachchi2021challenge}, %\kenneth{TODO: Add refs about retraining workers}\steven{added}
removing low-performing workers~\cite{10.1145/3613904.3642834, snow2008cheap,downs2010your,le2010ensuring}, %\kenneth{TODO: Add refs about removing or blocking low-performing workers}\steven{added}
or identifying potential issues in the annotation interfaces~\cite{toomim2011utility,10.1145/3613904.3642834, rahmanian2014user, komarov2013crowdsourcing}. %\kenneth{TODO: Add refs for crowd worker interfaces. At least cite: Toomim, M., Kriplean, T., Pörtner, C., \& Landay, J. (2011, May). Utility of human-computer interactions: Toward a science of preference measurement. In Proceedings of the SIGCHI Conference on Human Factors in Computing Systems (pp. 2275-2284).}\steven{added}
Gold labels are also beneficial for requesters during post-annotation data processing. 
They can be used to weight labels from different workers in label aggregation~\cite{abassi2017gold,abassi2019worker}, %\kenneth{TODO: Add label aggregation methods that use gold labels particularly to weight different workers}\steven{added}
improve label aggregation strategies~\cite{khattak2011quality, snow2008cheap},  %\kenneth{TODO: Add label aggregation methods that learn whatever from gold labels}\steven{added}
or 
exclude unreliable workers' outputs entirely~\cite{abassi2019worker}. %\kenneth{TODO: Cite ref using gold labels to remove workers from label aggregation}\steven{added}
Beyond requesters, gold labels are also beneficial for data labelers like crowd workers. 
Gold labels can be used to train workers~\cite{doroudi2016toward, le2010ensuring, gadiraju2015training,han2020crowd}, %\kenneth{TODO: Cite ref that uses gold labels for worker training}\steven{added}
provide real-time feedback to help them recalibrate their understanding of the task~\cite{le2010ensuring,hettiachchi2021challenge}, %\kenneth{TODO: Cite the visible gold paper from Amazon}\steven{added}
or remind them to pay more attention~\cite{ hettiachchi2021challenge,oleson2011programmatic}. %\kenneth{TODO: Cite attention check papers}\steven{amazon paper also warn workers in real time}

While gold labels are useful for quality control, as stated in the Introduction (Section~\ref{sec:intro}), %\kenneth{TODO: Update references}\steven{done}
they are not always available in real-world scenarios due to constraints such as data privacy or the cost of gathering gold labels~\cite{liu2019deep, yang2019evaluating, oikarinen2021detecting, slote2024unlocking}.
To address these challenges, researchers have developed methods to generate (approximations of) quality signals without gold labels. 
In the realm of LLM-powered data annotation, for instance, CoPrompter evaluates how well an LLM's output aligns with user-specified requirements as a feedback mechanism~\cite{joshi2024coprompter}. %\kenneth{TODO: Cite: Joshi, I., Shahid, S., Venneti, S., Vasu, M., Zheng, Y., Li, Y., ... \& Chan, G. Y. Y. (2024). CoPrompter: User-Centric Evaluation of LLM Instruction Alignment for Improved Prompt Engineering. arXiv preprint arXiv:2411.06099.}\steven{added}
Other studies also leverage the stability~\cite{chiang2023can} %\kenneth{TODO: Add ref}\steven{added}
%chiang2023can found LLM evaluation are stable over different formatting
or confidence~\cite{gligoric2024can} %\kenneth{TODO: Add ref}\steven{added}
%gligoric2024can introduce CONFIDENCEDRIVEN INFERENCE: a method that combines LLM annotations and LLM confidence indicators to strategically select which human annotations should be collected
of LLM outputs to infer quality signals.
%Our research investigates how effectively humans can iteratively refine prompts to guide LLMs in labeling data when gold-standard labels are unavailable, providing alternative quality signals.
Our research examines how effectively humans can refine prompts to guide LLMs in labeling data without gold-standard labels, providing insights into human prompting capabilities in the absence of reliable guidance signals.










%------------- dead kitten -------------
\begin{comment}




\kenneth{------------------------KENNETH IS WORKING HERE----------------------}



Gold-standard labels are widely used for quality control and crowd worker training~\cite{doroudi2016toward, gadiraju2015training,le2010ensuring,hettiachchi2021challenge}. For example, \citet{hettiachchi2021challenge} demonstrated that incorporating visible gold questions -- where annotators receive periodic feedback based on pre-labeled gold-standard examples -- could improve their work quality. 
Similarly, \citet{doroudi2016toward} found that providing expert examples was the most effective method of training for crowd workers and can help workers avoid specific types of incorrect responses. 
Additionally, \citet{le2010ensuring} employed dynamic learning systems that leveraged gold-standard labels to deliver real-time feedback and improve worker outcomes.
These studies, however, predominantly address the annotators' perspective -- workers who adhere to predefined guidelines and follow established standards.
While annotators are crucial components of the task pipeline, our study shifts focus to the requesters' perspective, those responsible for task design and pipeline management.
For requesters, gold-standard labels serve as signals to assess worker performance and refine training processes, thereby improving the overall quality of the entire pipeline.
Critically, the aforementioned studies assume the availability of gold-standard labels, typically under controlled experimental settings. 
In real-world scenarios, this assumption often does not hold due to constraints such as data privacy, security concerns, or the absence of labeled data~\cite{liu2019deep, yang2019evaluating, oikarinen2021detecting, slote2024unlocking}. 
To address this gap, our research explores settings where predefined gold-standard labels are unavailable. 
We designed a novel framework for requesters to iteratively develop and evolve their labeling standards through interactions with LLMs. 
By bridging the divide between controlled experiments and real-world challenges, our work highlights the potential of adaptive, LLM-driven approaches for dynamic task management without reliance on predefined gold-standard labels.

\steven{\citet{hettiachchi2021challenge} demonstrated that incorporating visible gold questions -- where annotators receive periodic feedback based on pre-labeled gold-standard examples -- could improve their work quality. 
Their study leveraged gold-standard labels to train crowd workers to align with pre-defined standards, effectively guiding annotators thorugh examples and feedback. 
While this approach focues on improving labeling quality at the annotator level, our work shifts the focus to requester and researcher perspective. Instead solely training labelers to meet pre-existing standards, we emphasize the broader implications of designing system in the entire labeling process, particularly in context involving dynamic or subjective tasks. \citet{gadiraju2015training} showed that training workers with gold labels can enhance accuracy and decrease response times. \citet{han2020crowd} used gold standard labels to guide crowd workers in revising incorrect judgments to align with predefined standards. 
}

\steven{
\citet{doroudi2016toward} found that providing expert examples was the most effective method of training for crowd workers. In our study, however, each participant was treated as an individual researcher rather than a crowd worker. While this finding underscores the value of providing gold labels to improve language model performance, it does not directly highlight their significance for researchers. Furthermore, \citet{doroudi2016toward} observed that gold standard labels help workers avoid specific types of incorrect responses. 
In contrast, our task is subjective, with participants’ standards potentially shifting across iterations. Introducing pre-set gold standard labels could enforce a uniform standard across each participant, which might not align with the iterative and subjective nature of our study
}

\steven{\citet{gadiraju2015training} showed that training workers with gold labels can enhance accuracy and decrease response times. [They were still focusing on crowd worker level.] }

\steven{\citet{han2020crowd} used gold standard labels for quality control and to guide crowd workers in revising incorrect judgments to align with predefined standards.}

\steven{\citet{le2010ensuring} employed gold standard labels within a dynamic learning environment that provided real-time feedback to train workers. However, the selection of specific examples for training could influence worker responses, potentially introducing bias in their judgments. [This is why we implemented a random sample in our system]}


\steven{\citet{liu2019deep} developed a HITL system that kept model upgrading with progressively collected data without having a pre-labeled data. [\textbf{they used 30 samples per iteration.} -add to justification for 10 and 50 instances.]}

\steven{\citet{wall2019using} found that end-users could build models without using expert patterns that have comparable performance to those who built by expert. This approach was required more effort and more mental demand than those who received guidance.}

\kenneth{TODO: Add references to every part of this paragraph.}
Decades of research have established that gold-standard labels are highly effective for quality control in data annotation~\cite{han2020crowd,gadiraju2015training,le2010ensuring,doroudi2016toward,hettiachchi2021challenge}. 
Embedding items with known labels into the annotation process enables requesters to monitor annotator or data quality and take actions such as retraining annotators, removing them from the pipeline, or reducing their weight in label aggregation. 
Beyond requesters, gold labels also allow for real-time feedback to workers, helping them recalibrate their understanding of the task or focus more carefully.
While gold labels are widely recognized as useful for quality control, most research assumes their availability.
However, as discussed in our Introduction (Section X), this assumption does not necessarily hold in real-world scenarios due to constraints such as data privacy or the cost of gathering gold labels~\cite{liu2019deep, yang2019evaluating, oikarinen2021detecting, slote2024unlocking}. 
To address these challenges, researchers have developed systems to provide proxy quality signals without gold labels. 
For instance, CoPrompter evaluates how well an LLM's output aligns with user-specified requirements as a feedback mechanism. 
Other studies leverage the stability or confidence of LLM outputs to infer quality signals.
Our research investigates how effectively humans can refine prompts to guide LLMs when gold-standard labels are unavailable.
    
\end{comment}

%\subsection{Explanations in AI-Assisted Tools}


%\subsection{Variables in System}
%There are lots of variables in a system could impact user's performance. 
\citet{kulesza2012tell} suggested that the more users understand the underlying system, the more effectively they can control it. 

\steven{\citet{lee2024clarify} introduces a system that allows non-expert users to train and correct models by directly interact with model using natural languages. In each iteration, the system will use similarity score between user description and image and display images above a threshold. The system will also provide 0-1 score indicating how well description separates the error cases from the correct prediction. Basically using metrics to guide user.
It does not mentioned about the sample size selection.}

\steven{[Data Instance:] In active learning, the goal is to minimize the amount of interaction required by users by querying the most important information~\cite{bernard2018vial}. [This can be used to justify why we increase to 50, to ensure the diversity. We cannot deploy algorithms to find most representative data sample because of the technical limitation of Google App Script]}

\steven{[Data Instance:] \citet{vermetten2022analyzing} investigated how the number of sample size affects the reliability of algorithm comparisons in iterative optimization. The study found that small sample sizes lead to high variability in performance estimates and larger sample sizes could decrease the impact of outliers. The performance could loss due to small samples and increasing sample size consistently improves reliability. }

\steven{\citet{purohit2018ranking} suggested capping the maximum number of annotation tasks assigned per unit of time to manage workload effectively to mitigate annotator burnout.}

\steven{\citet{pandey2022modeling} mentioned annotator can develop a mental representation of a concept by seeing a sufficient number of examples.}

\steven{\citet{wang2016human} limited users to verify the top-50 in each round, where users did binary classification on whether image was match or not.}

\steven{[explanation]\citet{kulesza2015principles} presents a system that explains the reason behind each prediction for users to better understand the system's logic to tailor the system toward their needs. In the system, users will modify feature weights within the model. n our LLM-powered system, users need to use natural language to guide the system. However, this can be more challenging because large models are less responsive to prompt variations compared to smaller models~\cite{zhuo2024prosa}.}

\steven{The more users understand the underlying system, the more effectively they can control it~\cite{kulesza2012tell}.}

\steven{\citet{teso2023leveraging} discusses a general framework for incorporating explanations into interactive machine learning. Users can get a better understanding of the machine's logic by observing the machine's explanations. [In LLM system, the explanation is the supporting argument for selecting a label.] Once understanding the bugs and limitation, users could modify the algorithm to correct flaws~\cite{kulesza2015principles}. [In our case, user cannot directly modify LLMs but only provide natural language to guide them. Also, subjective tasks does not have universal correct answers, where users need to provide their own standards to steer LLMs. ] }

\steven{[Task Difficulty:] 
A task being too difficult can frustrate users~\cite{zheng2022virtual}, particularly when exceeding their skill level, and a task being to easy can lead to boredom~\cite{zhang2021personalized}.
  These study focused on the impacts of difficulty on users' performance on a pre-defined task. However, in our study, our work prioritizes the dynamics of human-LLM interaction, emphasizing how effective humans could guide LLMs to align with their standard. In this context, the difficulty level of the task itself is less critical, as our primary objective is to assess the effectiveness of human guidance, regardless of the inherent complexity of the task.}


\steven{[task type:]\citet{cayir2016study} found the complexity and definition of a task significantly influence user performance. }

\steven{[task type:] \citet{hettiachchi2022survey} discusses different task assignment methods, including the modeling of worker performance and the impact of task heterogeneity on assignment strategies.
\citet{zhen2021crowdsourcing} provides a detailed exploration of task assignment challenges, task types, and their effects on worker performance and task outcomes. 
}
% \steven{ending of related word}We wanted to design a system to bridge the gap of xxxx: a graphical interface implemented on Google Sheet add-on, generalizing to single-class data annotation tasks, without requiring extensive knowledge of programming and system configuration. By combining the widespread familiarity and advanced features of Google Sheets with large-scale data annotation and iteration tracking, we aimed to make it easier for people to experiment with and benefit from LLMs.