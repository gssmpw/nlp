% Please add the following required packages to your document preamble:
%\usepackage{booktabs}
\begin{table*}[t]
\centering
\footnotesize
\begin{tabular}{@{}lcccccccc@{}}
 & \multicolumn{4}{c}{\textbf{50 Samples/Round (N=10)}} & \multicolumn{4}{c}{\textbf{10 Samples/Round (N=10)}} \\ \cmidrule{2-9} 
 & \multicolumn{2}{c}{\textbf{ACC$\uparrow$}} & \multicolumn{2}{c}{\textbf{MSE$\downarrow$}} & \multicolumn{2}{c}{\textbf{ACC$\uparrow$}} & \multicolumn{2}{c}{\textbf{MSE$\downarrow$}} \\ \midrule
\multicolumn{1}{c}{} & \textbf{Avg.(SD)} & \textbf{\begin{tabular}[c]{@{}c@{}}\%Participant\\ Improved\\ Over Initial\end{tabular}} & \textbf{Avg.(SD)} & \textbf{\begin{tabular}[c]{@{}c@{}}\%Participant\\ Improved\\ Over Initial\end{tabular}} & \textbf{Avg.(SD)} & \textbf{\begin{tabular}[c]{@{}c@{}}\%Participant\\ Improved\\ Over Initial\end{tabular}} & \textbf{Avg.(SD)} & \textbf{\begin{tabular}[c]{@{}c@{}}\%Participant\\ Improved\\ Over Initial\end{tabular}} \\ \midrule
\textbf{Initial Prompt} & .520 (.075) & - & .742 (.242)  & - & .564 (.117) & - & .822 (.654) & - \\ \midrule
\textbf{1st Revision} & \underline{\textbf{.530 (.094)}} & 50\% & \underline{\textbf{.720 (.285)}} & 40\% & .542 (.086) & 20\% & .870 (.733) & 40\% \\
\textbf{2nd Revision} & \underline{\textbf{.528 (.050)}} & 50\% & .780 (.268) & 50\% & .556 (.091) & 30\% & .886 (.692) & 40\% \\
\textbf{3rd Revision} & \underline{\textbf{.546 (.083)}} & 70\% & \underline{\textbf{.722 (.308)}} & 60\% & .552 (.085) & 30\% & .924 (.768) & 40\% \\
\textbf{4th Revision} & \underline{\textbf{.536 (.095)}} & 60\% & \underline{\textbf{.730 (.310)}} & 40\% & \underline{\textbf{.570 (.074)}} & 30\% & .890 (.753) & 40\% \\ \midrule
\textbf{\begin{tabular}[c]{@{}l@{}}End of Session\\ (Avg \#Iter=4.75)\end{tabular}} & \underline{\textbf{.536 (.095)}} & 60\% & \underline{\textbf{.736 (.303)}} & 40\% & .556 (.075) & 30\% & .894 (.632) & 50\%
\end{tabular}
\caption{Comparison of participants who reviewed 50 instances per iteration versus those who reviewed 10 instances per iteration. Reviewing 50 instances per iteration resulted in more frequent and consistent improvements compared to reviewing 10 instances. Improvements over the initial prompt are bolded and underlined.}
\label{tab:table-50-example-kenneth}
\end{table*}

% \begin{table}[t]
    \centering
    \begin{threeparttable}
    \begin{tabular}{l|c|c|c|c|c|c|c|c}
        \hline
         & \multicolumn{4}{c|}{\textbf{w/ 50 Instances} \tiny{(n=10)}} & \multicolumn{4}{c}{\textbf{w/o 50 Instances} \tiny{(n=10)}} \\
         \cmidrule(lr){2-5} \cmidrule(lr){6-9}
         & \multicolumn{2}{c|}{ACC$\uparrow$} & \multicolumn{2}{c|}{MSE$\downarrow$}& \multicolumn{2}{c|}{ACC$\uparrow$} & \multicolumn{2}{c}{MSE$\downarrow$}\\
         \cmidrule(lr){2-3} \cmidrule(lr){4-5} \cmidrule(lr){6-7} \cmidrule(lr){8-9}
         & \multicolumn{1}{c|}{Avg} & \multicolumn{1}{c|}{\%} & \multicolumn{1}{c|}{Avg} & \multicolumn{1}{c|}{\%}& \multicolumn{1}{c|}{Avg} & \multicolumn{1}{c|}{\%}& \multicolumn{1}{c|}{Avg} & \multicolumn{1}{c}{\%}\\
         \hline
        \multicolumn{1}{l|}{\textbf{Start}} & \multicolumn{1}{c|}{.520} & \multicolumn{1}{c|}{-} & \multicolumn{1}{c|}{.742} & \multicolumn{1}{c|}{-} & \multicolumn{1}{c|}{.564} & \multicolumn{1}{c|}{-} & \multicolumn{1}{c|}{.822} & \multicolumn{1}{c|}{-}  \\
        \multicolumn{1}{l|}{\textbf{1st}}   & \multicolumn{1}{c|}{.530} & \multicolumn{1}{c|}{50} & \multicolumn{1}{c|}{.720} & \multicolumn{1}{c|}{40} & \multicolumn{1}{c|}{.542} & \multicolumn{1}{c|}{20} & \multicolumn{1}{c|}{.870} & \multicolumn{1}{c|}{40}  \\
        \multicolumn{1}{l|}{\textbf{2nd}}   & \multicolumn{1}{c|}{.528} & \multicolumn{1}{c|}{50} & \multicolumn{1}{c|}{.780} & \multicolumn{1}{c|}{50} & \multicolumn{1}{c|}{.556} & \multicolumn{1}{c|}{30} & \multicolumn{1}{c|}{.886} & \multicolumn{1}{c|}{40}  \\
        \multicolumn{1}{l|}{\textbf{3rd}}   & \multicolumn{1}{c|}{.546} & \multicolumn{1}{c|}{70} & \multicolumn{1}{c|}{.722} & \multicolumn{1}{c|}{60} & \multicolumn{1}{c|}{.552} & \multicolumn{1}{c|}{30} & \multicolumn{1}{c|}{.924} & \multicolumn{1}{c|}{40}  \\
        \multicolumn{1}{l|}{\textbf{4th}}   & \multicolumn{1}{c|}{.536} & \multicolumn{1}{c|}{60} & \multicolumn{1}{c|}{.730} & \multicolumn{1}{c|}{40} & \multicolumn{1}{c|}{.570} & \multicolumn{1}{c|}{30} & \multicolumn{1}{c|}{.890} & \multicolumn{1}{c|}{40}  \\
        \hline
        \multicolumn{1}{l|}{End of Session} & \multicolumn{1}{c}{\multirow{2}{*}{.536}}& \multicolumn{1}{c}{\multirow{2}{*}{60}}& \multicolumn{1}{c}{\multirow{2}{*}{.736}}& \multicolumn{1}{c}{\multirow{2}{*}{40}}& \multicolumn{1}{c}{\multirow{2}{*}{.556}}& \multicolumn{1}{c}{\multirow{2}{*}{30}}& \multicolumn{1}{c}{\multirow{2}{*}{.894}}& \multicolumn{1}{c}{\multirow{2}{*}{50}}\\
        \multicolumn{1}{l|}{Avg \#Iter (4.75)}\\
        \hline
    \end{tabular}
    \begin{tablenotes}
        \tiny
        \item Avg: the average value of either ACC or MSE.
        \item \%: number of user improved ACC/MSE over ``Start'' $/$number of total user
    \end{tablenotes}
    \end{threeparttable}
    \caption{Participants' performance per iteration with or without 50 instances}
    \label{tab:participant-explanation-detail}
\end{table}


        
        
        % \multicolumn{1}{l}{\textbf{End of Session}} \\



 % & \multicolumn{1}{c|}{}
 % & \multicolumn{1}{c|}{}
 % & \multicolumn{1}{c|}{}
 % & \multicolumn{1}{c|}{}
 % & \multicolumn{1}{c|}{}


% \begin{figure}[t]
%     \centering
%     \begin{subfigure}{0.48\textwidth}
%      \includegraphics[width=\linewidth]{Figures/Results/Explanation-Result/diff_bar_ACC_50_instance.png}
%     \caption{Difference in ACC (Higher is Better) Compared to Initial Prompt (Iteration 0).\kenneth{(1) The figure title should be ``Difference in ACC Compared to Initial Prompt'', (2) The subfigure caption should be ``Difference in ACC (Higher is Better) Compared to Initial Prompt (Iteration 0)'', (3) The y-axis title should be ``ACC Improvement (Higher is Better)'', (4) the number in x and y axis should use larger fonts, (5) For the improvement (positive) bars, let's use solid border lines and solid color for the bar. For the negative bars, let's use dotted border lines and maybe color with lower opacity and .....gridlines texture? Visually convey that which bars are decreased/negative ones.}}
%     \label{fig:acc-participants-50instance-or-not}
%   \end{subfigure}
%   \hfill
%   \begin{subfigure}{0.48\textwidth}
%     \includegraphics[width=\linewidth]{Figures/Results/Explanation-Result/diff_bar_MSE_50_instance.png}
%     \caption{The improvement (difference) in MSE.}
%     \label{fig:mse-participants-50instance-or-not}
%   \end{subfigure}
%   \caption{Comparison of improvement (difference) in ACC and MSE over the initial prompt between the 50-instance and 10-instance groups. Reviewing 50 instances per iteration results in more consistent improvements in ACC compared to reviewing 10 instances.}
% %.\kenneth{TODO Steven: (1) The fonts need to be bigger, especially the legend label. They are way too small. (2) The title should say ``50 Samples/Round'' vs. ``10 Samples/Round'' instead of w/ w/o. (3) MAYBE use Red for 50 and }
%   \label{fig:average-50-instance-acc-mse-performance}
% \end{figure}

\begin{figure*}[t]
        \centering
        \includegraphics[width=0.98\linewidth]{Figures/Results/new-diff-bar/new_diff_bar_MSE_50_instance_v2.png}\Description{It contains 2 subplots illustrating the effect of sample size on ACC and MSE across iterations. The bars represent performance changes relative to the initial prompt, with improvements highlighted by a thick red border.}
  \caption{Comparison of improvement (difference) in ACC and MSE over the initial prompt between the 50-instance and 10-instance groups. Bars with red borders indicate positive improvements over the initial prompt's outcome. Reviewing 50 instances per iteration results in more consistent improvements in ACC compared to reviewing 10 instances. }
%.\kenneth{TODO Steven: (1) The fonts need to be bigger, especially the legend label. They are way too small. (2) The title should say ``50 Samples/Round'' vs. ``10 Samples/Round'' instead of w/ w/o. (3) MAYBE use Red for 50 and }
  \label{fig:average-50-instance-acc-mse-performance}
\end{figure*}


\subsubsection{Reviewing 50 instances per iteration leads to more frequent and consistent improvements compared to reviewing 10 instances}

%\kenneth{TODO Steven: Can you talk to Zixin about how to do t-test (or any significance test) in this case? In particular (1) how to do it between iterations and (2) how to do it for the "differences" instead of for absolute value. I kinda feel like we care about the diff than the absolute value. 50 vs 10 group, the 10 group had a better starting acc/mse but that's not relevant to our system but rather just by chance. }

%Participants who reviewed 50 instances were designated as \textbf{50Y}, and those who reviewed 10 instances were labeled as \textbf{50N}.
%For 50Y participants, 6 out of 10 participants demonstrated an improvement in their labeling accuracy, whereas 3 out of 10 participants in the 50N group improved. 
%However, participants under 50Y demonstrated a slightly better change in MSE compared to 50N, as 50Y had 2 fewer individuals with a decline in MSE performance. 
Table~\ref{tab:table-50-example-kenneth} and Figure~\ref{fig:average-50-instance-acc-mse-performance} present a comparison of participants who reviewed 50 instances per iteration against those who reviewed 10 instances per iteration. The detailed breakdown is shown in Figure~\ref{fig:average-acc-mse-llm-instances-performance-new}.
%\kenneth{TODO Steven: Update the figure and table references.}\steven{done}
In terms of accuracy, at the end of the session (\ie, four or more revisions),
6 out of 10 participants in the 50-instance group showed improvement, while only 3 out of 10 participants in the 10-instance group improved.
%\kenneth{Is this AFTER 4 REVISIONs or AT THE END OF SESSION?}\steven{4 revision and at the end have the same number of improved partcipant}
On average, every iteration in the 50-instance group resulted in better accuracy compared to the initial prompt, though the improvement was not strictly increasing with each iteration.

For MSE, participants in the 50-instance group improved across three iterations, while those in the 10-instance group showed no improvement over the initial prompt in any round.

We also note that participants in the 10-instance group began with higher initial performance, but this was before they viewed the labeling results and occurred by chance, unrelated to the experimental conditions. 
Our analysis focuses on performance differences between iterations across both groups.

\paragraph{Significant Tests.}
We conducted eight linear mixed-effects models to examine the effect of iteration across four different conditions. 
The dependent variables were ACC and MSE, and participants were treated as random effects.
%Under conditions where participants \textbf{did not have access to LLM explanations,} we observed a significant increasing trend in accuracy with each iteration ($\beta$=0.013, p-value=0.009**). 
%Conversely, 
In the condition where participants \textbf{reviewed only 10 instances per iteration}, we found a significant increasing trend in MSE as the iterations progressed ($\beta$=0.019, p-value=0.043*). 


