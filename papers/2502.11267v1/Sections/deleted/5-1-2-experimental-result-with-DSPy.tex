The previous section showed only certain settings could help participants on improving prompt performance. It would be interesting to explore the potential impacts of using an automatic prompt optimizing tool, such as DSPy~\cite{khattab2023dspy}, to assist with refinement. 


We then analyzed the DSPy fine-tuned prompts under different conditions in Table~\ref{tab:dspy-general-conditions-t-test}. LN-50Y demonstrated a significantly higher labeling accuracy improvement than other conditions, while LN-50N showed a significant drop in accuracy. In terms of MSE, both LY-50Y and LY-50N showed significant declines in MSE than LN-50Y and LN-50N. 

We further narrowed it down into different conditions for each DSPy algorithm in Table~\ref{tab:dspy-all-conditions-t-test}. For Bootstrap, LN-50N dropped the accuracy significantly than LY-50Y and LN-50Y. LY-50Y demonstrated a significant drop in MSE compared to LN-50Y and LN-50N. 
For Simple Prompt, LN-50N performed a significant dropping in labeling accuracy compared to other conditions, and LY-50Y had a significant drop in MSE compared to the rest of the conditions.  
For Mipro, LY-50Y showed a significant improvement in labeling accuracy compared to LY-50N and LN-50N. LN-50Y showed a notable improvement in accuracy over LN-50N and also achieved a significantly better MSE improvement compared to both LY-50Y and LY-50N.
In Copro, there was no significant difference between conditions for labeling accuracy. As for MSE, LY-50N exhibited a significant drop in conditions including LN, and LN-50N demonstrated a significant improvement in conditions containing LY.

In summary, DSPy algorithms demonstrated a greater improvement in accuracy and MSE for participants who \textbf{did not access the LLM explanation}. DSPy optimizers were more effective in helping participants with 50 instances per iteration enhance labeling accuracy, while participants reviewing fewer instances per round saw a more pronounced improvement in MSE. 

\begin{table}[ht]
    \centering
    \begin{tabular}{lcccc|cccc}
        \hline
        \multicolumn{1}{l} {DSPy} & \multicolumn{1}{c}{\multirow{2}{*}{\textbf{ACC$\uparrow$}}} & \multicolumn{3}{c|}{\textbf{T-Test over Avg. Changed ACC}}& \multicolumn{1}{c}{\multirow{2}{*}{\textbf{MSE$\downarrow$}}} & \multicolumn{3}{c}{\textbf{T-Test over Avg. Changed MSE}}
        \\ \cmidrule(lr){3-5} \cmidrule(lr){7-9} 
        \multicolumn{1}{l} {Changes} & & LY-50N & LN-50Y & LN-50N & & LY-50N & LN-50Y & LN-50N\\
        \hline
        LY-50Y & .012 & .060 & .033* & <.001***      & .114 &  .520  & <.001*** & <.001**   \\ \hline
        LY-50N  & -.006 & -      & <.001*** & <.001***     & .092 & -      & .020* & .002**   \\ \hline
        LN-50Y  & .036 & -      & -    & <.001***       & .003 & -      & -    & .215   \\ \hline
        LN-50N   & -.054 & -      & -    & -           & -.047 & -      & -    & -      \\ \hline
    \end{tabular}
    \caption{DSPy t-test over different conditions for all optimizers.}
    \label{tab:dspy-general-conditions-t-test}
\end{table}




% DSPy improved labeling accuracy on average for LN-50Y participants and consistently enhanced MSE when the Mipro and Copro optimizers were applied. 
% Interestingly, for participants under the LN-50N condition, DSPy worsened accuracy but contributed to the highest improvement in MSE.
% Overall, DSPy performed the best on prompting refinement on LN-50Y.
\begin{figure}
    \centering
    \begin{subfigure}{0.85\textwidth}
    \includegraphics[width=\linewidth]{Figures/Results/Participant_DSPy/avg_acc_diff_dspy.png}
    \caption{DSPy Average Improvement on ACC.}
    \label{fig:avg-acc-dspy-improvement}
  \end{subfigure}
  \hfill
  \begin{subfigure}{0.85\textwidth}
    \includegraphics[width=\linewidth]{Figures/Results/Participant_DSPy/avg_mse_diff_dspy.png}
    \caption{DSPy Average Improvement on MSE.}
    \label{fig:avg-mse-dspy-improvement}
  \end{subfigure}
  \caption{DSPy Average Improvement on ACC and MSE under different conditions.}
  \label{fig:average-acc-mse-dspy-improvement}
\end{figure}


\begin{table}[ht]
    \centering
    \begin{tabular}{l|cccccccccccc}
        \hline
         & \multicolumn{2}{c|}{\textbf{Human}} & \multicolumn{2}{c|}{\textbf{Human+DSPy}}& \multicolumn{2}{c|}{\textbf{Bootstrap}}& \multicolumn{2}{c|}{\textbf{Simple Prompt}}& \multicolumn{2}{c|}{\textbf{Mipro}}& \multicolumn{2}{c|}{\textbf{Copro}} \\
         \cmidrule(lr){2-3} \cmidrule(lr){4-5} \cmidrule(lr){6-7} \cmidrule(lr){8-9} \cmidrule(lr){10-11} \cmidrule(lr){12-13}  
         & \multicolumn{1}{c|}{ACC$\uparrow$} & \multicolumn{1}{c|}{MSE$\downarrow$} & \multicolumn{1}{c|}{ACC$\uparrow$} & \multicolumn{1}{c|}{MSE$\downarrow$}& \multicolumn{1}{c|}{ACC$\uparrow$} & \multicolumn{1}{c|}{MSE$\downarrow$}& \multicolumn{1}{c|}{ACC$\uparrow$} & \multicolumn{1}{c|}{MSE$\downarrow$}& \multicolumn{1}{c|}{ACC$\uparrow$} & \multicolumn{1}{c|}{MSE$\downarrow$}& \multicolumn{1}{c|}{ACC$\uparrow$} & \multicolumn{1}{c|}{MSE$\downarrow$}\\
         \hline
        \multicolumn{1}{l|}{\textbf{Start}} & \multicolumn{1}{c|}{0.542} & \multicolumn{1}{c|}{0.782} & \multicolumn{1}{c|}{-}& \multicolumn{1}{c|}{-}& \multicolumn{1}{c|}{-}& \multicolumn{1}{c|}{-}& \multicolumn{1}{c|}{-}& \multicolumn{1}{c|}{-}& \multicolumn{1}{c|}{-}& \multicolumn{1}{c|}{-}& \multicolumn{1}{c|}{-}& \multicolumn{1}{c|}{-}  \\
        \multicolumn{1}{l|}{\textbf{1st}}    & \multicolumn{1}{c|}{0.536}	 & \multicolumn{1}{c|}{0.795}	 & \multicolumn{1}{c|}{0.538}	 & \multicolumn{1}{c|}{0.889}	 & \multicolumn{1}{c|}{0.526}	 & \multicolumn{1}{c|}{0.915}	 & \multicolumn{1}{c|}{0.533}	 & \multicolumn{1}{c|}{0.864}	 & \multicolumn{1}{c|}{0.527}	 & \multicolumn{1}{c|}{0.973}	 & \multicolumn{1}{c|}{0.565}	 & \multicolumn{1}{c|}{0.822}   \\
        \multicolumn{1}{l|}{\textbf{2nd}}    & \multicolumn{1}{c|}{0.542}	 & \multicolumn{1}{c|}{0.833}	 & \multicolumn{1}{c|}{0.535}	 & \multicolumn{1}{c|}{0.883}	 & \multicolumn{1}{c|}{0.534}	 & \multicolumn{1}{c|}{0.934}	 & \multicolumn{1}{c|}{0.533}	 & \multicolumn{1}{c|}{0.864}	 & \multicolumn{1}{c|}{0.536}	 & \multicolumn{1}{c|}{0.862}	 & \multicolumn{1}{c|}{0.538}	 & \multicolumn{1}{c|}{0.873}   \\
        \multicolumn{1}{l|}{\textbf{3rd}}    & \multicolumn{1}{c|}{0.549}	 & \multicolumn{1}{c|}{0.823}	 & \multicolumn{1}{c|}{0.544}	 & \multicolumn{1}{c|}{0.847}	 & \multicolumn{1}{c|}{0.550}	 & \multicolumn{1}{c|}{0.850}	 & \multicolumn{1}{c|}{0.533}	 & \multicolumn{1}{c|}{0.864}	 & \multicolumn{1}{c|}{0.547}	 & \multicolumn{1}{c|}{0.873}	 & \multicolumn{1}{c|}{0.544}	 & \multicolumn{1}{c|}{0.801}  \\
        \multicolumn{1}{l|}{\textbf{4th}}    & \multicolumn{1}{c|}{0.553}	 & \multicolumn{1}{c|}{0.810}	 & \multicolumn{1}{c|}{0.537}	 & \multicolumn{1}{c|}{0.859}	 & \multicolumn{1}{c|}{0.526}	 & \multicolumn{1}{c|}{0.889}	 & \multicolumn{1}{c|}{0.533}	 & \multicolumn{1}{c|}{0.864}	 & \multicolumn{1}{c|}{0.536}	 & \multicolumn{1}{c|}{0.874}	 & \multicolumn{1}{c|}{0.554}	 & \multicolumn{1}{c|}{0.809}  \\
        \hline
        \multicolumn{1}{l|}{End of Session} & \multicolumn{1}{c}{\multirow{2}{*}{0.546}}& \multicolumn{1}{c}{\multirow{2}{*}{0.815}}& \multicolumn{1}{c}{\multirow{2}{*}{0.537}}& \multicolumn{1}{c}{\multirow{2}{*}{0.859}}& \multicolumn{1}{c}{\multirow{2}{*}{0.526}}& \multicolumn{1}{c}{\multirow{2}{*}{0.889}}& \multicolumn{1}{c}{\multirow{2}{*}{0.533}}& \multicolumn{1}{c}{\multirow{2}{*}{0.864}} & \multicolumn{1}{c}{\multirow{2}{*}{0.536}}& \multicolumn{1}{c}{\multirow{2}{*}{0.874}}& \multicolumn{1}{c}{\multirow{2}{*}{0.554}}& \multicolumn{1}{c}{\multirow{2}{*}{0.809}}\\
        \multicolumn{1}{l|}{Avg \#Iter (4.75)}\\
        \hline
    \end{tabular}
    \caption{Average Performance per iteration for Participants, Participant plus DSPy, and four different DSPy algorithms.}
    \label{tab:participant-explanation-detail}
\end{table}


        
        
        % \multicolumn{1}{l}{\textbf{End of Session}} \\



 % & \multicolumn{1}{c|}{}
 % & \multicolumn{1}{c|}{}
 % & \multicolumn{1}{c|}{}
 % & \multicolumn{1}{c|}{}
 % & \multicolumn{1}{c|}{}

