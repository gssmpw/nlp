\kenneth{also mention explanation study}



%\subsubsection{Use the prompt to label the data sample}
After finalizing the prompt tabs (``Context'', ``Rule Book'', and ``Shots'') and sampling data instances, users can click on the ``Start Annotation'' button to annotate all instances in the ``Working Data Sample'' tab. 
On the backend, \system gathers all questions and answers from the ``Context'' tab and inputs them into GPT-4o to generate an instruction prompt.
The system then combines this prompt with rules and the provided gold shots to create a final annotation prompt.
For each data group, the system sends one request to GPT-4o using this final prompt for annotation.
After receiving the LLM's output, the system parses the results and updates the spreadsheet with the annotated information for each instance.

Before clicking the ``Start Annotation'' button, users can choose whether to display the LLM explanation in the annotation results. To evaluate the impact of LLM explanations on users' prompt performance, our user study participants were randomly divided into two equal groups: one group had the option to display the LLM explanation, and the other group did not have access to it.\steven{explanation study}