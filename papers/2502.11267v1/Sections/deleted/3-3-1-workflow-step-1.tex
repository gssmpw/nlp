

\kenneth{Three parts of the prompt}
%\subsubsection{Compose/Refine the prompt}
This step is the core element of the system, allowing users to create and modify prompts for their annotation tasks. In the ``Context'' tab, users need to answer questions about the data annotation task to provide task context to LLMs. Then, they will navigate to the ``Rule Book'' tab to add annotation labels with corresponding definitions. This procedure is mandatory as it supplies both label classes and rules that the LLMs will adhere to. Afterward, users can add instances with their gold labels, which will serve as examples from which the LLMs can learn. 
This step is optional in the first iteration, as users may not yet have a well-defined gold standard. As they review more data instances, they can identify additional instances that could serve as gold standard references for the LLMs.
Users can modify these three sections at any time in future iterations.