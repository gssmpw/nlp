%%%%%%%%%%%%%
%% MISE EN PAGE %%
%%%%%%%%%%%%%

%\numberwithin{equation}{subsection}
%\newcounter{fakecnt}[subsubsection]
%\def\thefakecnt{\arabic{subsubsection}}
%\renewcommand{\theequation}{\thefakecnt}



\DeclareFontFamily{U}{BOONDOX-calo}{\skewchar\font=45 }
\DeclareFontShape{U}{BOONDOX-calo}{m}{n}{
  <-> s*[1.05] BOONDOX-r-calo}{}
\DeclareFontShape{U}{BOONDOX-calo}{b}{n}{
  <-> s*[1.05] BOONDOX-b-calo}{}
\DeclareMathAlphabet{\mathcalb}{U}{BOONDOX-calo}{m}{n}
\SetMathAlphabet{\mathcalb}{bold}{U}{BOONDOX-calo}{b}{n}
%\DeclareMathAlphabet{\mathbcalb}{U}{BOONDOX-calo}{b}{n}
\newcommand{\caa}{\mathfrak} 


\newcommand{\conj}{\mathrel{\substack{\\[-.2em]\urcorner}}}
\newcommand{\comp}{\mathrel{\substack{\lrcorner \\[+.2em]}}}

\newcommand{\ltwocell}[3][0.5]{\ar@{}[#2] \ar@{=>}?(#1)+/r 0.175cm/;?(#1)+/l 0.175cm/^{#3}}
\newcommand{\ltwocello}[3][0.5]{\ar@{}[#2] \ar@{=>}?(#1)+/r 0.175cm/;?(#1)+/l 0.175cm/_{#3}}
\newcommand{\dtwocell}[3][0.5]{\ar@{}[#2] \ar@{=>}?(#1)+/u  0.175cm/;?(#1)+/d 0.175cm/^{#3}}


\newcommand{\dblcat}[2]{\mathbb{#1}\mathrm{#2}}
%\newcommand{\EnDblcat}[2]{\mathscr{#1}\mathrm{#2}}
%\newcommand{\lcbicat}[2]{\mathscr{#1}\mathcalb{#2}}

%\newcommand{\Mod}{\dblcat{M}{od}}
%\newcommand{\PsMod}{\dblcat{P}{sMod}}
%\newcommand{\TiPsMod}{\dblcat{T}{iPsMod}}
%\newcommand{\LoPsMod}{\dblcat{L}{oPsMod}}
\newcommand{\Hor}{\dblcat{H}{or}}
\renewcommand{\Vert}{\dblcat{V}{ert}}


%\newcommand{\Span}{\dblcat{S}{pan}}
%\newcommand{\Sq}{\dblcat{S}{q}}
\newcommand{\Quin}{\dblcat{Q}{uin}}

\newcounter{Assume}
\setcounter{Assume}{1}
\newenvironment{Assumption}[1][Assumption]{\medskip\noindent\textsc{#1 \theAssume :} }{\addtocounter{Assume}{1}\begin{flushright}\end{flushright}}


\newcommand{\cd}[2][]{\vcenter{\hbox{\xymatrix#1{#2}}}}
\newcommand{\ot}{\leftarrow}
\newcommand{\toto}{\rightrightarrows}
\newcommand{\twoto}{\Rightarrow}
\newcommand{\threeto}{\Rrightarrow}
\newcommand{\threeot}{\Lleftarrow}


\newcommand{\twocong}[2][0.5]{\ar@{}[#2] \save ?(#1)*{\cong}\restore}
\newcommand{\twoeq}[2][0.5]{\ar@{}[#2] \save ?(#1)*{=}\restore}
\newcommand{\threeeq}[2][0.5]{\ar@{}[#2] \save ?(#1)*{\equiv}\restore}



\newcommand{\horaru}[3][0.5]{\ar@{->}[#3]^(#1){#2}\ar@{}[#3]|(#1){\scriptstyle{\bullet}} }
\newcommand{\horard}[3][0.5]{\ar@{->}[#3]_(#1){#2}\ar@{}[#3]|(#1){\scriptstyle{\bullet}} }
\newcommand{\horarm}[3][0.5]{\ar@{->}[#3]|(#1){#2}\ar@{}[#3]|(0.3){\scriptstyle{\bullet}} }


\newcommand{\circaruh}[2]{\ar@{->}[#2]^{#1}\ar@{}[#2]|-{\scriptstyle{\ominus}} }
\newcommand{\circardh}[2]{\ar@{->}[#2]_{#1}\ar@{}[#2]|-{\scriptstyle{\ominus}} }
\newcommand{\circarmh}[2]{\ar@{->}[#2]|-{#1}\ar@{}[#2]|-{\scriptstyle{\ominus}} }

\newcommand{\horto}[2]{\SelectTips{eu}{10}
\xymatrix@C=.2in{#1\horaru{}{r} & #2}}
\newcommand{\circto}[2]{\SelectTips{eu}{10}
\xymatrix@C=.2in{#1\circaruh{}{r} & #2}}

%Hebrew letters

%\DeclareFontFamily{U}{rcjhbltx}{}
%\DeclareFontShape{U}{rcjhbltx}{m}{n}{<->rcjhbltx}{}
%\DeclareSymbolFont{hebrewletters}{U}{rcjhbltx}{m}{n}

%\DeclareMathSymbol{\gimell}{\mathord}{hebrewletters}{130}
%\DeclareMathSymbol{\dalet}{\mathord}{hebrewletters}{100}
%\DeclareMathSymbol{\dalett}{\mathord}{hebrewletters}{131}
%\DeclareMathSymbol{\he}{\mathord}{hebrewletters}{104}
%\DeclareMathSymbol{\vet}{\mathord}{hebrewletters}{107}
%\DeclareMathSymbol{\zayin}{\mathord}{hebrewletters}{135}
%\DeclareMathSymbol{\tet}{\mathord}{hebrewletters}{135}
%\DeclareMathSymbol{\bet}{\mathord}{hebrewletters}{129}
%\DeclareMathSymbol{\nun}{\mathord}{hebrewletters}{110}
%\DeclareMathSymbol{\nuns}{\mathord}{hebrewletters}{78}
%\DeclareMathSymbol{\resh}{\mathord}{hebrewletters}{114}
%\DeclareMathSymbol{\pe}{\mathord}{hebrewletters}{112}
%\DeclareMathSymbol{\fe}{\mathord}{hebrewletters}{112}
%\DeclareMathSymbol{\tav}{\mathord}{hebrewletters}{116}
%\DeclareMathSymbol{\het}{\mathord}{hebrewletters}{120}
%\DeclareMathSymbol{\yod}{\mathord}{hebrewletters}{121}
%\DeclareMathSymbol{\fes}{\mathord}{hebrewletters}{80}
%\DeclareMathSymbol{\lamed}{\mathord}{hebrewletters}{108}
%\DeclareMathSymbol{\mem}{\mathord}{hebrewletters}{109}\let\mim\mem
%\DeclareMathSymbol{\mems}{\mathord}{hebrewletters}{77}
%\DeclareMathSymbol{\ayin}{\mathord}{hebrewletters}{96}
%\DeclareMathSymbol{\tsadi}{\mathord}{hebrewletters}{118}
%\DeclareMathSymbol{\tsadis}{\mathord}{hebrewletters}{90}
%\DeclareMathSymbol{\qof}{\mathord}{hebrewletters}{113}
%\DeclareMathSymbol{\shin}{\mathord}{hebrewletters}{152}
%\DeclareMathSymbol{\xafs}{\mathord}{hebrewletters}{75}
%\DeclareMathSymbol{\shin}{\mathord}{hebrewletters}{152}
%\DeclareMathSymbol{\kaf}{\mathord}{hebrewletters}{138}
%\DeclareMathSymbol{\samek}{\mathord}{hebrewletters}{142}
%\DeclareMathSymbol{\samekh}{\mathord}{hebrewletters}{115}
%\DeclareMathSymbol{\pei}{\mathord}{hebrewletters}{144}

\newcommand{\Ff}{\mathfrak{F}}
\newcommand{\Gf}{\mathfrak{G}}
\newcommand{\Hf}{\mathfrak{H}}
%\newcommand{\If}{\mathfrak{I}}
\newcommand{\Jf}{\mathfrak{J}}
\newcommand{\Kf}{\mathfrak{K}}
\newcommand{\Lf}{\mathfrak{L}}
\newcommand{\Mf}{\mathfrak{M}}
\newcommand{\Of}{\mathfrak{O}}
\newcommand{\Pf}{\mathfrak{P}}
\newcommand{\Qf}{\mathfrak{Q}}
\newcommand{\Rf}{\mathfrak{R}}
\newcommand{\Tf}{\mathfrak{T}}
\newcommand{\Uf}{\mathfrak{U}}
\newcommand{\Vf}{\mathfrak{V}}
\newcommand{\Wf}{\mathfrak{W}}
\newcommand{\Xf}{\mathfrak{X}}
\newcommand{\Yf}{\mathfrak{Y}}
\newcommand{\Zf}{\mathfrak{Z}}


\newcommand{\critMD}{\blacktriangleleft}

\makeatletter
%\DeclareMathSymbol{\b@r}{\mathord}{symbols}{"2E}
\DeclareMathSymbol{\b@r}{\mathord}{largesymbols}{"0E}
\newcommand{\newc@ncel}[2]{%
  \ooalign{%
    \hfil$\vcenter{\moved@wn{#1}\hbox{\scalebox{1}[0.5]{$#1\b@r$}}}$\hfil\cr % the bar
    $#1#2$\cr % the symbol
  }%
}
\newcommand{\moved@wn}[1]{%
  \sbox\z@{$#1\mkern-3mu\nonscript\mkern3mu$}%
  \vskip\wd\z@
}
\newcommand{\nprecnapprox}{%
  \mathrel{\m@th\mathpalette\newc@ncel\precapprox}%
}
\newcommand{\nsuccnapprox}{%
  \mathrel{\m@th\mathpalette\newc@ncel\succapprox}%
}
\makeatother



%% Styles des sections
\newcommand{\periodafter}[1]{\ifstrempty{#1}{}{#1.}}
\titleformat{\section}[block]{\scshape\filcenter\LARGE\boldmath}{\thesection.}{.5em}{}
\titleformat{\subsection}[block]{\bfseries\filcenter\large\boldmath}{\thesubsection.}{.5em}{\medskip}
\titleformat{\subsubsection}[runin]{\bfseries\boldmath}{\thesubsubsection.}{.5em}{\periodafter}%{}[.]
\titlespacing{\subsubsection}{0pt}{\topsep}{.5em}

%% Styles des théorèmes
\newtheoremstyle{ntheorem}%
	{\topsep}{\topsep}{\itshape}{0pt}{\bfseries}{.}{.5em}%
	{\thmnumber{#2.\hspace{.5em}}\thmname{#1}\thmnote{ (#3)}}
	
\newtheoremstyle{ndefinition}%
	{\topsep}{\topsep}{\normalfont}{0pt}{\bfseries}{.}{.5em}%
	{\thmnumber{#2.\hspace{.5em}}\thmname{#1}\thmnote{ (#3)}}
	
\newtheoremstyle{nremark}%
	{\topsep}{\topsep}{\normalfont}{0pt}{\itshape}{.}{.5em}%
	{\thmnumber{}\thmname{#1}\thmnote{ (#3)}}
	
\newtheoremstyle{nfigure}%
	{\topsep}{\topsep}{\itshape}{0pt}{\bfseries}{.}{.5em}%
	{\thmnumber{#2.\hspace{.5em}}\thmname{#1}.\thmnote{ #3}}
	
\theoremstyle{ntheorem}
  	\newtheorem{theorem}[subsubsection]{Theorem}
  	\newtheorem{proposition}[subsubsection]{Proposition}
	\newtheorem{lemma}[subsubsection]{Lemma}
  	\newtheorem{corollary}[subsubsection]{Corollary}
  	\newtheorem{hyp}{Hypothesis}
	
\theoremstyle{nfigure}
	\newtheorem{myfig}[subsubsection]{Figure}

\theoremstyle{ndefinition}
	\newtheorem{definition}[subsubsection]{Definition}
	\newtheorem{notation}[subsubsection]{Notation}
	\newtheorem{example}[subsubsection]{Example}
	\newtheorem{remark}[subsubsection]{Remark}
	
%\makeatletter
%\def\@equationname{equation}
%\newenvironment{eqn}[1]{%
%\def\mymathenvironmenttouse{#1}%
%\ifx\mymathenvironmenttouse\@equationname%
%\refstepcounter{subsubsection}%
%\setcounter{equation}{\value{subsubsection}}% Keep the equation counter synchronized for hyperref reasons...
%\else
%\patchcmd{\@arrayparboxrestore}{equation}{subsubsection}{}{}% doesn't change output?
%\patchcmd{\print@eqnum}{equation}{subsubsection}{}{}%
%% Set counter equal to subsubsection, then increment subsubsection. The original command already increments equation.
%\pretocmd{\incr@eqnum}{\setcounter{equation}{\value{subsubsection}}\refstepcounter{subsubsection}}{}{}%
%% \def\print@eqnum{\tagform@\thematrix}% instead of etoolbox' \pathcmd
%% \def\incr@eqnum{\refstepcounter{matrix}\let\incr@eqnum\@empty}% instead of etoolbox' \pathcmd
%\fi
%\csname\mymathenvironmenttouse\endcsname%
%}{%
%\ifx\mymathenvironmenttouse\@equationname%
%\tag{\thesubsubsection}%
%\fi
%\csname end\mymathenvironmenttouse\endcsname%
%}
%\makeatother


\makeatletter
\def\@equationname{equation}
\newenvironment{eqn}[1]{%
    \def\mymathenvironmenttouse{#1}%
    \ifx\mymathenvironmenttouse\@equationname%
        \refstepcounter{subsubsection}%
    \else
        \patchcmd{\@arrayparboxrestore}{equation}{subsubsection}{}{}%          doesn't change output?
        \patchcmd{\print@eqnum}{equation}{subsubsection}{}{}%
        \patchcmd{\incr@eqnum}{equation}{subsubsection}{}{}%
%       \def\print@eqnum{\tagform@\thematrix}%                          instead of etoolbox' \pathcmd
%       \def\incr@eqnum{\refstepcounter{matrix}\let\incr@eqnum\@empty}% instead of etoolbox' \pathcmd
    \fi
    \csname\mymathenvironmenttouse\endcsname%
}{%
    \ifx\mymathenvironmenttouse\@equationname%
        \tag{\thesubsubsection}%
    \fi
    \csname end\mymathenvironmenttouse\endcsname%
}
\makeatother

%New variant of equation*
\makeatletter
\newenvironment{eq*}{%
 \incr@eqnum
  $
%  \mathdisplay{equation*}%
}{%
%  \endmathdisplay{equation*}%
  $
  \ignorespacesafterend
}
\makeatother
\newcommand{\tof}[1]{\quad\xrightarrow{#1} \quad}
%\newcommand{\toff}[2]{\xrightarrow{#1}_{#2}}
\newcommand{\toff}[2]{\underset{#2}{\xrightarrow{#1}}}
\newcommand{\otff}[2]{\underset{#2}{\hspace{-.2em}\xleftarrow{#1}\hspace{.3em}}}
\newcommand{\otfff}[2]{\underset{#2}{\hspace{-.2em}\xleftarrow{#1}\hspace{.3em}}}


\newcounter{DefEq}
\setcounter{DefEq}{0}
\newcommand{\deftag}{\addtocounter{DefEq}{1}\tag{D\theDefEq }}

\newcommand{\totag}{\tof{(\theequation)}}
\newcommand{\mtag}{(\theequation)}
\newcommand{\eqtag}{{(\theequation)}}

\newcommand{\ol}[2]{\overset{#1}{#2}}
\newcommand{\Eqtag}[1]{\ol{(\theequation)}{#1}}


%% Mise en page
\pagestyle{fancy}
\setlength{\oddsidemargin}{0cm}
\setlength{\evensidemargin}{0cm}
\setlength{\topmargin}{0cm} 
\setlength{\headheight}{1cm}
\setlength{\headsep}{1cm}
\setlength{\textwidth}{16cm}
\setlength{\marginparwidth}{0cm}
\setlength{\footskip}{2cm}
\setlength{\headwidth}{16cm}
\newcommand{\emptysectionmark}[1]{\markboth{\textbf{#1}}{\textbf{#1}}}
\renewcommand{\sectionmark}[1]{\markboth{\textbf{\thesection. #1}}{\textbf{\thesection. #1}}}
\renewcommand{\subsectionmark}[1]{\markright{\textbf{\thesubsection. #1}}}
\fancyhead{}\fancyfoot[LC,RC]{}
\fancyhead[LE]{\leftmark}
\fancyhead[RO]{\rightmark}
\fancyfoot[LE,RO]{$\thepage$}
\fancypagestyle{plain}{
\fancyhf{}\fancyfoot[LC,RC]{}
\fancyfoot[LE,RO]{$\thepage$}
\renewcommand{\headrulewidth}{0pt}
\renewcommand{\footrulewidth}{0pt}}
\newcommand{\headrulelength}{16cm}
\renewcommand{\headrulewidth}{1pt}
\renewcommand{\footrulewidth}{0pt}
\setlength{\arraycolsep}{1pt}
\renewcommand{\headrule}{
  {\hrule width\headwidth height\headrulewidth
   \vskip-\headrulewidth}
}

%%%%%%Macro texte%%%%%%
\newcommand{\pdf}[1]{\texorpdfstring{$#1$}{1}}

%%%%%%%%
%% XYPIC %%
%%%%%%%%

%% Options 
\renewcommand{\objectstyle}{\displaystyle}
\renewcommand{\labelstyle}{\displaystyle}
\UseTips
\SelectTips{eu}{11}

%% Nouvelle direction pour les monomorphismes (>->)
\newdir{ >}{{}*!/-10pt/@{>}}
\newdir{ -}{{}*!/-10pt/@{}}
\newdir{> }{{}*!/+10pt/@{>}}

%% Ajouts de la variante 4 pour les blancs, les lignes, les pointill�s, 
%% les zigouigouis (pas top) et les t�tes de fl�ches (presque finies).
%% Concr�tement, dans \xymatrix, la commande \ar @4 produit une 4-cellule.
\makeatletter

% Noms des nouvelles directions
\xyletcsnamecsname@{dir4{}}{dir{}}
\xydefcsname@{dir4{-}}{\line@ \quadruple@\xydashh@}
\xydefcsname@{dir4{.}}{\point@ \quadruple@\xydashh@}
\xydefcsname@{dir4{~}}{\squiggle@ \quadruple@\xybsqlh@}
\xydefcsname@{dir4{>}}{\Tttip@}
\xydefcsname@{dir4{<}}{\reverseDirection@\Tttip@}

% Commande quadruple
% Attention : la position des pourcents est importante !!
\xydef@\quadruple@#1{%
	\edef\Drop@@{%
		\dimen@=#1\relax
		\dimen@=.5\dimen@
		\A@=-\sinDirection\dimen@
		\B@=\cosDirection\dimen@
		\setboxz@h{%
			\setbox2=\hbox{\kern3\A@\raise3\B@\copy\z@}%
			\dp2=\z@ \ht2=\z@ \wd2=\z@ \box2
			\setbox2=\hbox{\kern\A@\raise\B@\copy\z@}%
			\dp2=\z@ \ht2=\z@ \wd2=\z@ \box2
			\setbox2=\hbox{\kern-\A@\raise-\B@\copy\z@}%
			\dp2=\z@ \ht2=\z@ \wd2=\z@ \box2
			\setbox2=\hbox{\kern-3\A@\raise-3\B@ \noexpand\boxz@}%
			\dp2=\z@ \ht2=\z@ \wd2=\z@ \box2
		}%
		\ht\z@=\z@ \dp\z@=\z@ \wd\z@=\z@ \noexpand\styledboxz@
	}%
}

% T�tes de quadruples fl�ches : positionnement ok, reste la jointure
% Les parties comment�es sont des essais � reprendre plus tard
\xydef@\Tttip@{\kern2pt \vrule height2pt depth2pt width\z@
	\Tttip@@ \kern2pt \egroup
	\U@c=0pt \D@c=0pt \L@c=0pt \R@c=0pt \Edge@c={\circleEdge}%
	\def\Leftness@{.5}\def\Upness@{.5}%
	\def\Drop@@{\styledboxz@}\def\Connect@@{\straight@{\dottedSpread@\jot}}}
	
\xydef@\Tttip@@{%
	\dimen@=.25\dimen@
%	\A@=-\sinDirection\dimen@
 	\B@=\cosDirection\dimen@
	\setboxz@h\bgroup\reverseDirection@\line@ \wdz@=\z@ \ht\z@=\z@ \dp\z@=\z@
%	\kern\A@ \raise\B@ \boxz@ \kern\L@c
%	\kern-\L@c \boxz@ \kern\L@c
	{\vDirection@(1,-1)\xydashl@ \xyatipfont\char\DirectionChar}%
	{\vDirection@(1,+1)\xydashl@ \xybtipfont\char\DirectionChar}%
}

% Red�finition de la commande \ar 
% Provoque un avertissement � la compilation
\xydef@\ar@form{
	\ifx \space@\next \expandafter\DN@\space{\xyFN@\ar@form}%
	\else\ifx ^\next \DN@ ^{\xyFN@\ar@style}\edef\arvariant@@{\string^}%
	\else\ifx _\next \DN@ _{\xyFN@\ar@style}\edef\arvariant@@{\string_}%
	\else\ifx 0\next \DN@ 0{\xyFN@\ar@style}\def\arvariant@@{0}%
	\else\ifx 1\next \DN@ 1{\xyFN@\ar@style}\def\arvariant@@{1}%
	\else\ifx 2\next \DN@ 2{\xyFN@\ar@style}\def\arvariant@@{2}%
	\else\ifx 3\next \DN@ 3{\xyFN@\ar@style}\def\arvariant@@{3}%
	\else\ifx 4\next \DN@ 4{\xyFN@\ar@style}\def\arvariant@@{4}%
	\else\ifx \bgroup\next \let\next@=\ar@style
	\else\ifx [\next \DN@[##1]{\ar@modifiers{[##1]}}%]
	\else\ifx *\next \DN@ *{\ar@modifiers}%
	\else\addLT@\ifx\next \let\next@=\ar@slide
	\else\ifx /\next \let\next@=\ar@curveslash
	\else\ifx (\next \let\next@=\ar@curveinout %)
	\else\addRQ@\ifx\next \addRQ@\DN@{\ar@curve@}%
	\else\addLQ@\ifx\next \addLQ@\DN@{\xyFN@\ar@curve}%
	\else\addDASH@\ifx\next \addDASH@\DN@{\defarstem@-\xyFN@\ar@}%
	\else\addEQ@\ifx\next \addEQ@\DN@{\def\arvariant@@{2}\defarstem@-\xyFN@\ar@}%
	\else\addDOT@\ifx\next \addDOT@\DN@{\defarstem@.\xyFN@\ar@}%
	\else\ifx :\next \DN@:{\def\arvariant@@{2}\defarstem@.\xyFN@\ar@}%
	\else\ifx ~\next \DN@~{\defarstem@~\xyFN@\ar@}%
	\else\ifx !\next \DN@!{\dasharstem@\xyFN@\ar@}%
	\else\ifx ?\next \DN@?{\ar@upsidedown\xyFN@\ar@}%
	\else \let\next@=\ar@error
	\fi\fi\fi\fi\fi\fi\fi\fi\fi\fi\fi\fi\fi\fi\fi\fi\fi\fi\fi\fi\fi\fi\fi \next@}

\makeatother


%%%%%%%%%%%%%%%%%
%% SYMBOLES %%%%%%%%
%%%%%%%%%%%%%%%%%
\newcommand{\ob}{\mathrm{ob}\ }

%% Flèches
\newcommand{\fl}{\rightarrow}
\newcommand{\fll}{\longrightarrow}
\newcommand{\ofl}[1]{\overset{\displaystyle #1}{\fll}}
\newcommand{\ifl}{\rightarrowtail}
\newcommand{\pfl}{\twoheadrightarrow}
\newcommand{\fllg}{\longleftarrow}
\newcommand{\oflg}[1]{\overset{\displaystyle #1}{\fllg}}

\newcommand{\dfl}{\Rightarrow}
\newcommand{\dfll}{\Longrightarrow}
\newcommand{\odfl}[1]{\overset{\displaystyle #1}{\dfll}}

\newcommand{\tfl}{\Rrightarrow}
\newcommand{\qfl}{\xymatrix@1@C=10pt{\ar@4 [r] &}}

%\newcommand{\multimapdotboth}{\xymatrix@1@C=10pt{\ar @{*-*} [r] &}}
\newcommand{\equivmap}{\xymatrix@1@C=10pt{\ar @{|-|} [r] &}}


%%%% One Cells

%%\parallelto{dr}{blue}{>_\Pr \pi}{V}

\newcommand{\parallelto}[4]{\ar @{} [#1] |(.6)*+<2pt>{}="w1" |(.9)*+<2pt>{}="w2" 
  \ar @[#2] @{-||} []; "w1" 
    \ar @[#2] []; "w2"  _-{\color{#2} #3 } ^-{\color{#2} #4}}
    
\newcommand{\parallelcurveto}[5]{\ar @{} @/#5/ [#1] |(.6)*+<2pt>{}="w1" |(.9)*+<2pt>{}="w2" 
  \ar @[#2] @{-||} []; "w1" 
    \ar @[#2]  @/#5/ []; "w2"  _-{\color{#2} #3 } ^-{\color{#2} #4}}    
    
    \newcommand{\paralleltom}[6]{\ar @{} [#1] |(.6)*+<2pt>{}="w1" |(.9)*+<2pt>{}="w2" 
  \ar @[#2]  @{-||} []; "w1" 
    \ar @[#2]  []; "w2"  _-{\color{#2} #3 } ^-{\color{#2} #4}  |(.8){\color{#5} #6}}
    
        \newcommand{\paralleldemitom}[6]{\ar @{} [#1] |(.3)*+<2pt>{}="w1" |(.48)*+<2pt>{}="w2"  
  \ar @[#2] @{-||} []; "w1" 
    \ar @[#2] []; "w2"  _-{\color{#2} #3 } ^-{\color{#2} #4}  |(.8){\color{#5} #6}}
     \newcommand{\parallelfromdemitom}[6]{\ar @{} [#1] |(.54)*+<2pt>{}="w0" |(.75)*+<2pt>{}="w1" |(.98)*+<2pt>{}="w2"  
  \ar @[#2] @{-||} "w0"; "w1" 
    \ar @[#2] "w0"; "w2"  _-{\color{#2} #3 } ^-{\color{#2} #4}  |(.8){\color{#5} #6}}

    \newcommand{\parallellongtom}[6]{\ar @{} [#1] |(.6)*+<2pt>{}="w1" |(.98)*+<2pt>{}="w2" 
  \ar @[#2]  @{-||} []; "w1" 
    \ar @[#2] []; "w2"  _-{\color{#2} #3 } ^-{\color{#2} #4}  |(.8){\color{#5} #6}}
        
    \newcommand{\parallelcurvetom}[7]{
    \ar @{->} @/_6ex/ [#2] _-{\color{#2} #3 } ^-{\color{#2} #4}  |(.8){\color{#5} #6}
  \ar   @[#2] @{-||} @/_2ex/ 
    }

    \newcommand{\paralleltomd}[6]{\ar @{} [#1] |(.6)*+<2pt>{}="w1" |(.9)*+<2pt>{}="w2" 
  \ar @[#2] @{-||} []; "w1" 
    \ar @[#2] @{->>} []; "w2"  _-{\color{#2} #3 } ^-{\color{#2} #4}  |(.8){\color{#5} #6}}
    
    \newcommand{\tom}[6]{\ar @{} [#1] |(.6)*+<2pt>{}="w1" |(.9)*+<2pt>{}="w2" 
    \ar @[#2] @{->} []; "w2"  _-{\color{#2} #3 } ^-{\color{#2} #4}  |(.8){\color{#5} #6}}

\newcommand{\affineto}[4]{\ar @{} [#1] |(.8)*+<2pt>{}="w1" |(.9)*+<2pt>{}="w2" 
  \ar @[#2] @{-}  "w1" ; "w2" _-{\color{#2} =}
    \ar @[#2] []; "w2"  _-{\color{#2} #3 } ^-{\color{#2} #4}}

\newcommand{\affinetom}[6]{\ar @{} [#1] |(.8)*+<2pt>{}="w1" |(.9)*+<2pt>{}="w2" 
  \ar @[#2] @{-}  "w1" ; "w2" _-{\color{#2} =}
    \ar @[#2] @{->>} []; "w2"  _-{\color{#2} #3 } ^-{\color{#2} #4}|(.8){\color{#5} #6}}
    
\newcommand{\positiveto}[4]{\ar @{} [#1] |(.8)*+<2pt>{}="w1" |(.9)*+<2pt>{}="w2" 
  \ar @[#2] @{-}  "w1" ; "w2" ^-{\color{#2} +}
    \ar @[#2] @{->>} []; "w2"  _-{\color{#2} #3 } ^-{\color{#2} #4}}

\newcommand{\positiveleftto}[4]{\ar @{} [#1] |(.8)*+<2pt>{}="w1" |(.9)*+<2pt>{}="w2" 
  \ar @[#2] @{-}  "w1" ; "w2" _-{\color{#2} +}
    \ar @[#2] @{->>} []; "w2"  _-{\color{#2} #3 } ^-{\color{#2} #4}}

\newcommand{\positivetom}[6]{\ar @{} [#1] |(.8)*+<2pt>{}="w1" |(.9)*+<2pt>{}="w2" 
  \ar @[#2] @{-}  "w1" ; "w2" ^-{\color{#2} +}
    \ar @[#2] []; "w2"  _-{\color{#2} #3 } ^-{\color{#2} #4}|(.8){\color{#5} #6}}


\newcommand{\affinelongtom}[6]{\ar @{} [#1] |(.8)*+<2pt>{}="w1" |(.98)*+<2pt>{}="w2" 
  \ar @[#2] @{-}  "w1" ; "w2" _-{\color{#2} =}
    \ar @[#2] []; "w2"  _-{\color{#2} #3 } ^-{\color{#2} #4}|(.7){\color{#5} #6}}

\newcommand{\affineshortto}[4]{\ar @{} [#1] |(.5)*+<2pt>{}="w1" |(.9)*+<2pt>{}="w2" 
  \ar @[#2] @{-}  "w1" ; "w2" _-{\color{#2} =}
    \ar @[#2] []; "w2"  _-{\color{#2} #3 } ^-{\color{#2} #4}}

\newcommand{\affineshorttom}[6]{\ar @{} [#1] |(.5)*+<2pt>{}="w1" |(.9)*+<2pt>{}="w2" 
  \ar @[#2] @{-}  "w1" ; "w2" _-{\color{#2} =}
    \ar @[#2] []; "w2"  _-{\color{#2} #3 } ^-{\color{#2} #4} |(.5){\color{#5} #6}}

  
\newcommand{\affineeqto}[4]{\ar @{} [#1] |(.7)*+<2pt>{}="w1" |(.9)*+<2pt>{}="w2" 
  \ar @[#2] @{-}  "w1" ; "w2" _-{\color{#2} =}
    \ar @[#2] @{|-|} []; "w2"  _-{\color{#2} #3 } ^-{\color{#2} #4}}
    

    
\newcommand{\paraffto}[4]{\ar @{} [#1] |(.6)*+<2pt>{}="w1" |(.8)*+<2pt>{}="w2" |(.9)*+<2pt>{}="w3" 
  \ar @[#2] @{-||} []; "w1"
   \ar @[#2] @{-}  "w2" ; "w3" _-{\color{#2} =} 
    \ar @[#2] []; "w3"  _-{\color{#2} #3 } ^-{\color{#2} #4}}
    
    
\newcommand{\parafftom}[6]{\ar @{} [#1] |(.6)*+<2pt>{}="w1" |(.8)*+<2pt>{}="w2" |(.9)*+<2pt>{}="w3" 
  \ar @[#2] @{-||} []; "w1"
   \ar @[#2] @{-}  "w2" ; "w3" _-{\color{#2} =} 
    \ar @[#2] []; "w3"  _-{\color{#2} #3 } ^-{\color{#2} #4}|(.8){\color{#5} #6}}
    
\newcommand{\parafflongtom}[6]{\ar @{} [#1] |(.6)*+<2pt>{}="w1" |(.8)*+<2pt>{}="w2" |(.98)*+<2pt>{}="w3" 
  \ar @[#2] @{-||} []; "w1"
   \ar @[#2] @{-}  "w2" ; "w3" _-{\color{#2} =} 
    \ar @[#2] []; "w3"  _-{\color{#2} #3 } ^-{\color{#2} #4}|(.7){\color{#5} #6}}
    
%\newcommand{\protwoard}[2]{\ar@{=>}[#2]_{#1}\ar@{}[#2]|-{\scriptstyle{\mid}} }
%\newcommand{\protwoarm}[2]{\ar@{=>}[#2]|-{#1}\ar@{}[#2]|-{\scriptstyle{\mid}} }


%% Parenthèses et crochets
\newcommand{\ens}[1]{\left\{ #1 \right\}}
\renewcommand{\angle}[1]{\langle #1 \rangle}
\newcommand{\pres}[2]{\langle #1 \;\vert\; #2 \rangle}
\newcommand{\bigpres}[2]{\big\langle\; #1 \;\;\big\vert\;\; #2 \;\big\rangle}
\newcommand{\cohpres}[3]{\langle #1 \;\vert\; #2 \;\vert\; #3 \rangle}
\newcommand{\bigcohpres}[3]{\big\langle\; #1 \;\;\big\vert\;\; #2 \;\;\big\vert\;\; #3 \;\big\rangle}

%% Accents
\newcommand{\po}[1]{{#1}^{\Box}}
\newcommand{\op}[1]{{#1}^{o}}
\newcommand{\hop}[2]{{#1}^{o_{#2}}}%higher opposite
\newcommand{\da}[1]{{#1}^{\dagger}}
\newcommand{\env}[1]{{#1}^{e}}
\newcommand{\cl}[1]{\overline{#1}}
\newcommand{\rep}[1]{\widehat{#1}}
\newcommand{\tild}[1]{\widetilde{#1}}
\newcommand{\tck}[1]{#1^{\top}}
\newcommand{\vect}[1]{\overrightarrow{#1}}

%% Opérateurs
\DeclareMathOperator{\Cons}{Con}
\DeclareMathOperator{\Con}{Con}
\DeclareMathOperator{\id}{Id}
\DeclareMathOperator{\Ker}{Ker}
\DeclareMathOperator{\coker}{coker}
\DeclareMathOperator{\Coker}{Coker}
\DeclareMathOperator{\Subst}{Subst}

\newcommand{\BigCop}{\bigsqcup}
\newcommand{\BigCoproduct}{\bigsqcup}
\newcommand{\Cop}{\sqcup}
\newcommand{\Coproduct}{\sqcup}

\DeclareMathOperator{\End}{End}
\DeclareMathOperator{\Aut}{Aut}
\DeclareMathOperator{\Lan}{Lan}

\DeclareMathOperator{\h}{h}
%\DeclareMathOperator{\crit}{c}
\DeclareMathOperator{\Hom}{\mathrm{Hom}}
\DeclareMathOperator{\Ext}{\mathrm{Ext}}
\DeclareMathOperator{\Tor}{\mathrm{Tor}}
\DeclareMathOperator{\Der}{\mathrm{Der}}

\DeclareMathOperator{\FDT}{FDT}
\DeclareMathOperator{\FDTAB}{FDT_{\mathrm{ab}}}
\DeclareMathOperator{\FP}{FP}
\DeclareMathOperator{\FHT}{FHT}
\DeclareMathOperator{\Ob}{Ob}
\DeclareMathOperator{\Var}{Var}


%% Opérations binaires=Products


\newcommand{\lmod}{\mathrel{|}\joinrel\Relbar}
\newcommand{\rmod}{\Relbar\joinrel\mathrel{|}}

\newcommand{\oleq}{\sqsubseteq }
\newcommand{\ogeq}{\sqsupseteq }
\newcommand{\tens}{\otimes}
\newcommand{\Ortho}{\mathord{\Perp}}
\newcommand{\ortho}{\mathord{\perp}}
\newcommand{\critB}{\mathrel{\natural}}
\newcommand{\succeqE}{\mathrel{\succeq_{\Er}}}
\newcommand{\succE}{\mathrel{\succ_{\Er}}}
\newcommand{\approxE}{\mathrel{\approx_{\Er}}}
\newcommand{\succeqH}{\mathrel{((\succeq_{\Er})_{lex})_H}}
\newcommand{\succH}{\mathrel{((\succ_{\Er})_{lex})_H}}
\newcommand{\succeqL}{\mathrel{(\succeq_{\Er})_{lex}}}
\newcommand{\succL}{\mathrel{(\succ_{\Er})_{lex}}}
%\newcommand{\ortho}{\Perp}
%\newcommand{\rperp}{\flat}
\newcommand{\rperp}{\rmod}
\newcommand{\imod}{\mathrel{|}\joinrel< }
\newcommand{\smod}{>\joinrel\mathrel{|}}
\newcommand{\leqmod}{\mathrel{|}\joinrel\leq} 
\newcommand{\geqmod}{\geq\joinrel\mathrel{|}}
\newcommand{\Sharp}{\sharp}
\newcommand{\crit}{\natural}
\newcommand{\conf}{\pitchfork}
\renewcommand\o{\otimes}
\def\ogray{\otimes_\catego{g}}
\def\cgray{\boxtimes_\catego{g}}


%\newcommand{\Perp}{\perp\mkern-9.5mu\perp}

%% Caractères 
%\renewcommand{\phi}{\varphi}
\renewcommand{\epsilon}{\varepsilon}

\newcommand{\Nb}{\mathbb{N}}
\newcommand{\Zb}{\mathbb{Z}}
\newcommand{\Gb}{\mathbb{G}}
\newcommand{\Rb}{\mathbb{R}}
\newcommand{\Sb}{\mathbb{S}}
\newcommand{\Tb}{\mathbb{T}}
\newcommand{\Ub}{\mathbb{U}}
\newcommand{\Vb}{\mathbb{V}}

\newcommand{\dr}{\partial}

\newcommand{\Dc}{\mathcalb{D}}
\newcommand{\Ec}{\mathcalb{E}}
\newcommand{\Fc}{\mathcalb{F}}
\newcommand{\Kc}{\mathcalb{K}}

%%\Ar automata
\newcommand{\Ar}{\mathbb{A}}
\newcommand{\Br}{\mathcal{B}}
\newcommand{\Cr}{\mathcal{C}}
\newcommand{\Dr}{\mathcal{D}}
\newcommand{\Er}{\mathcal{E}}
\newcommand{\Fr}{\mathcal{F}}
\newcommand{\Gr}{\mathcal{G}}
\newcommand{\Hr}{\mathcal{H}}
\newcommand{\Ir}{\mathcal{I}}
\newcommand{\Kr}{\mathcal{K}}
\newcommand{\Lr}{\mathcal{L}}
\newcommand{\Mr}{\mathcal{M}}
\newcommand{\Nr}{\mathcal{N}}
\newcommand{\Or}{\mathcal{O}}
\renewcommand{\Pr}{\mathcal{P}}
\newcommand{\Pa}{\mathscr{P}}
\newcommand{\Qr}{\mathcal{Q}}
\newcommand{\Sr}{\mathcal{S}}
\newcommand{\Rr}{\mathcal{R}}
\newcommand{\Tr}{\mathcal{T}}
\newcommand{\Ur}{\mathcal{U}}
\newcommand{\Vr}{\mathcal{V}}
\renewcommand{\Wr}{\mathcal{W}}
\newcommand{\Yr}{\mathcal{Y}}
\newcommand{\Xr}{\mathscr{X}}
\newcommand{\Zr}{\mathscr{Z}}

\newcommand{\Sf}{\mathfrak{S}}

\newcommand{\A}{\mathbf{A}}
\newcommand{\B}{\mathbf{B}}
\def\C{\mathbf{C}}
\newcommand{\D}{\mathbf{D}}
\newcommand{\F}{\mathbf{F}}
\newcommand{\GG}{\mathbf{G}}
\newcommand{\K}{\mathbb{K}}
\newcommand{\M}{\mathbf{M}}
\newcommand{\N}{\mathbf{N}}
\newcommand{\R}{\mathbf{R}}
\renewcommand{\S}{\mathbf{S}}
\newcommand{\V}{\mathbf{V}}
\newcommand{\W}{\mathbf{W}}
\newcommand{\T}{\mathbb{T}}


\newcommand{\Cl}{\mathsf{C}}
\newcommand{\Op}{\mathcal{O}}


%% Catégories
\def\catego#1{\mathsf{#1}}

%\newcommand{\cat}{\catego{Cat}}
\newcommand{\Ab}{\catego{Ab}}
\newcommand{\Gp}{\mathbf{Gp}}
\newcommand{\Cat}{\catego{Cat}}
\newcommand{\Pol}{\catego{Pol}}
\newcommand{\Grpd}{\catego{Grpd}}
\newcommand{\DbCat}{\catego{DbCat}}
\newcommand{\DbGrpd}{\catego{DbGrpd}}
\newcommand{\FldGrpd}{\catego{FoldGrpd}}
\newcommand{\DiPol}{\catego{DiPol}}
\newcommand{\Act}{\catego{Act}}
\newcommand{\Mod}{\catego{Mod}}
\newcommand{\Nat}{\catego{Nat}}
\newcommand{\Set}{\operatorname{Set}}
\newcommand{\Ord}{\mathbf{Ord}}
%\newcommand{\Rep}{\catego{Rep}}
\newcommand{\LinCat}{\catego{LinCat}}
\newcommand{\Vect}{\operatorname{Vect}}
\newcommand{\grVect}{\catego{grVect}}
\newcommand{\Vectg}{\catego{GrVect}}
\newcommand{\Alg}{\catego{Alg}}
\newcommand{\Algg}{\catego{GrAlg}}
\newcommand{\Ch}{\catego{Ch}}
\newcommand{\Chg}{\catego{GrCh}}
\newcommand{\Grph}{\catego{Grph}}
\newcommand{\OSet}{\operatorname{\Omega\text{-}\catego{Set}}}
\newcommand{\OVect}{\operatorname{\Omega\text{-}\catego{Vect}}}

\newcommand{\inOAlg}{\operatorname{\Omega_\infty\text{-}Alg}}
\newcommand{\OAlg}[1]{\Omega\text{-}\catego{Alg}_{#1}}
\newcommand{\Glob}{\catego{Glob}}
\newcommand{\Gpd}{\catego{Gpd}}
\newcommand{\Bimod}{\operatorname{Bimod}}
\newcommand{\Sem}{\operatorname{Sem}}
\newcommand{\Mon}{\operatorname{Mon}}
\newcommand{\OSem}{\operatorname{\Omega\text{-}Sem}}
\newcommand{\OMon}{\operatorname{\Omega\text{-}Mon}}
\newcommand{\Ind}{\catego{Ind}}
\newcommand{\nc}{\newcommand}
\newcommand{\Gph}{\catego{Gph}}

\newcommand{\Ct}[1]{\mathrm{Ct}(#1)}
\newcommand{\Wk}[1]{\mathrm{Wk}(#1)}

\newcommand{\As}{\mathbf{As}}
\newcommand{\Perm}{\mathbf{Perm}}
\newcommand{\Brd}{\mathbf{Braid}}

%%Opérades
\newcommand{\Com}{\mathbf{\mathcal{C}om}}
\newcommand{\Ass}{\mathbf{\mathcal{A}ss}}
%\newcommand{\A}{\mathbf{\mathcal{A}}}
\newcommand{\Dend}{\mathbf{\mathcal{D}end}}
\newcommand{\Dias}{\mathbf{\mathcal{D}ias}}
\newcommand{\nf}{\mathrm{nf}}
\newcommand{\bshuf}{\overline{\shuf}}
\newcommand{\rel}{\mathrm{rel}}
\def\Lie{\mathbf{\mathcal{L}ie}}
\def\Pois{\mathbf{\mathcal{P}ois}}
\def\NsOp{\mathbf{\mathcal{N}s\mathcal{O}p}}

%%Differential algebras
\newcommand{\RBA}[2]{\mathcal{R}\mathcal{B}_{#1}(#2)}
\newcommand{\DRBA}[2]{\mathcal{D}\mathcal{R}\mathcal{B}_{#1}(#2)}
\newcommand{\DA}[2]{\mathcal{D}_{#1}(#2)}

%%Automata
\newcommand{\auto}[1]{\mathbb{A}(#1)}

%% Cohomologie
\newcommand{\fnat}[2]{F_{#1}[#2]}
\newcommand{\ho}{\mathrm{H}}
\newcommand{\poldim}[1]{d_{\mathrm{pol}}(#1)}
\newcommand{\cohdim}[1]{\mathrm{cd}(#1)}
%

%%%%%%%%%%%%%%%%%%%% 
%% Added from the previous version (Zuan)
\DeclareMathOperator{\dom}{dom}
\DeclareMathOperator{\cell}{cell}
\DeclareMathOperator{\Red}{Red}
\DeclareMathOperator{\Nf}{Nf}
\DeclareMathOperator{\red}{Red_m}
\DeclareMathOperator{\supp}{supp}
\newcommand{\lins}[1]{\mathscr{A}(#1)}
\newcommand{\lin}[1]{\mathscr{A}_\Omega(#1)}
%%%%%%%%%%%%%%%%%%%%%%%% Statements
\newtheorem{thm}{Theorem}[section]
%\newtheorem{prop}{Proposition}[section]
\newtheorem{prop}[thm]{Proposition}
%\newtheorem{defn}{Definition}[section]
\newtheorem{lem}[thm]{Lemma}
\newtheorem{cor}[thm]{Corollary}
\newtheorem{prop-def}{Proposition-Definition}[section]
%\newtheorem{conj}[theorem]{Conjecture}
\theoremstyle{definition}
\newtheorem{defn}[thm]{Definition}
\newtheorem{claim}[thm]{Claim}
\newtheorem{exam}[thm]{Example}
%\newtheorem{propprop}{Proposed Proposition}[section]
\newtheorem{Ques}[thm]{Question}
\newtheorem{problem}{Problem}
\newtheorem{assumption}[thm]{Assumption}
\newtheorem{Conj}[thm]{Conjecture}

\newcommand{\tfll}{\xymatrix@1@C=20pt{\ar@3 [r] &}}
\newcommand{\otfl}[1]{\xymatrix@1@C=20pt{\ar@3 [r] ^-*+{#1} &}}


\renewcommand{\labelenumi}{{\rm(\roman{enumi})}}
\renewcommand{\theenumi}{\roman{enumi}}
\renewcommand{\labelenumii}{{\rm(\alph{enumii})}}
\renewcommand{\theenumii}{\alph{enumii}}

%%%%%%%%%%%%%%%%%%%%%%% symbols

\nc{\delete}[1]{{}}
\nc{\mmargin}[1]{}
%\nc{\Alg}{\mathrm{Alg}}
%\nc{\AvO}{\mathrm{AvO}}
%\nc{\AvA}{\mathrm{AvA}}
%\nc{\rmH}{\mathrm{H}}
%\delete{
\nc{\mlabel}[1]{\label{#1}}  % Use this to suppress names
\nc{\mcite}[1]{\cite{#1}}  % Use this to suppress names
\nc{\mref}[1]{\ref{#1}}  % Use this to suppress names
\nc{\mbibitem}[1]{\bibitem{#1}} % Use this to show number
%}

\delete{
	\nc{\mlabel}[1]{\label{#1}  % Use the next two lines to show names
		{\hfill \hspace{1cm}{\bf{{\ }\hfill(#1)}}}}
	\nc{\mcite}[1]{\cite{#1}{{\bf{{\ }(#1)}}}}  % Use this lines to show names
	\nc{\mref}[1]{\ref{#1}{{\bf{{\ }(#1)}}}}  % Use this lines to show names
	\nc{\mbibitem}[1]{\bibitem[\bf #1]{#1}} % Use this to show name
}


%%%%%%%%%%%%%%%%%  new commands
\newcommand{\sha}{{\mbox{\cyr X}}}
\newcommand{\shap}{{\mbox{\cyrs X}}}
\font\cyr=wncyr10 \font\cyrs=wncyr7

%\newcommand{\bcr}{{\mathfrak{B}\mathfrak{C}\mathfrak{R}}}
%\newcommand{\bca}{{\mathfrak{B}\mathfrak{C}\mathfrak{A}}}
%\newcommand{\bvs}{{\mathfrak{B}\mathfrak{V}\mathfrak{S}}}
%\newcommand{\bcd}{{\mathfrak{B}\mathfrak{C}\mathfrak{D}}}
%\newcommand{\brb}{{\mathfrak{B}\mathfrak{R}\mathfrak{B}}}
%\newcommand{\svs}{{\mathfrak{S}\mathfrak{V}\mathfrak{S}}}
%\newcommand{\sca}{{\mathfrak{S}\mathfrak{C}\mathfrak{A}}}
%\newcommand{\scr}{{\mathfrak{S}\mathfrak{C}\mathfrak{R}}}
%\newcommand{\ydh}{{_H^H\mathcal{Y}\mathcal{D}}}
%\newcommand{\dah}{{_H^H\mathcal{D}\mathcal{A}}}
%\newcommand{\ah}{{_H^H\mathcal{A}}}


%\newcommand{\sha}{{\mbox{\cyr X}}}
%\font\cyr=wncyr10 \font\cyrs=wncyr7

\newcommand{\bk}{{\mathbf{k}}}
%    Absolute value notation
\newcommand{\abs}[1]{\lvert#1\rvert}

%    Blank box placeholder for figures (to avoid requiring any
%    particular graphics capabilities for printing this document).
\newcommand{\blankbox}[2]{%
	\parbox{\columnwidth}{\centering
		%    Set fboxsep to 0 so that the actual size of the box will match the
		%    given measurements more closely.
		\setlength{\fboxsep}{0pt}%
		\fbox{\raisebox{0pt}[#2]{\hspace{#1}}}%
	}%
}

\nc{\vep}{\varepsilon}
\nc{\bin}[2]{ (_{\stackrel{\scs{#1}}{\scs{#2}}})}  %binomial coeff
\nc{\binc}[2]{(\!\! \begin{array}{c} \scs{#1}\\
		\scs{#2} \end{array}\!\!)}  %binomial coeff
\nc{\bincc}[2]{  ( {\scs{#1} \atop
		\vspace{-1cm}\scs{#2}} )}  %binomial coeff
\nc{\oline}[1]{\overline{#1}}
\nc{\mapm}[1]{\lfloor\!|{#1}|\!\rfloor}
\nc{\bs}{\bar{S}}
\nc{\la}{\longrightarrow}
\nc{\rar}{\rightarrow}
\nc{\lon }{\,\rightarrow\,}
\nc{\dar}{\downarrow}
\nc{\dap}[1]{\downarrow \rlap{$\scriptstyle{#1}$}}
\nc{\defeq}{\stackrel{\rm def}{=}}
\nc{\dis}[1]{\displaystyle{#1}}
\nc{\dotcup}{\ \displaystyle{\bigcup^\bullet}\ }
\nc{\hcm}{\ \hat{,}\ }
\nc{\hts}{\hat{\otimes}}
\nc{\hcirc}{\hat{\circ}}
\nc{\lleft}{[}
\nc{\lright}{]}
\nc{\curlyl}{\left \{ \begin{array}{c} {} \\ {} \end{array}
	\right .  \!\!\!\!\!\!\!}
\nc{\curlyr}{ \!\!\!\!\!\!\!
	\left . \begin{array}{c} {} \\ {} \end{array}
	\right \} }
\nc{\longmid}{\left | \begin{array}{c} {} \\ {} \end{array}
	\right . \!\!\!\!\!\!\!}
\nc{\ora}[1]{\stackrel{#1}{\rar}}
\nc{\ola}[1]{\stackrel{#1}{\la}}%${\Bbb Z}$
\nc{\scs}[1]{\scriptstyle{#1}} \nc{\mrm}[1]{{\rm #1}}
\nc{\dirlim}{\displaystyle{\lim_{\longrightarrow}}\,}
\nc{\invlim}{\displaystyle{\lim_{\longleftarrow}}\,}
\nc{\dislim}[1]{\displaystyle{\lim_{#1}}} \nc{\colim}{\mrm{colim}}
\nc{\mvp}{\vspace{0.3cm}} \nc{\tk}{^{(k)}} \nc{\tp}{^\prime}
\nc{\ttp}{^{\prime\prime}} \nc{\svp}{\vspace{2cm}}
\nc{\vp}{\vspace{8cm}}
%\nc{\proofend}{$\blacksquare$\vspace{0.3cm}}
\nc{\modg}[1]{\!<\!\!{#1}\!\!>}
%\nc{\intg}[1]{\lceil{#1}\rceil}  %old free int ring
\nc{\intg}[1]{F_C(#1)}
\nc{\lmodg}{\!<\!\!}
\nc{\rmodg}{\!\!>\!}
\nc{\cpi}{\widehat{\Pi}}
%\nc{\sha}{\scs{\mbox{\cyr X}}} %used to be \cyr
%\nc{\sha}{{\mbox{\cyr X}}}  %used to be \cyr
\nc{\ssha}{{\mbox{\cyrs X}}} %sha as product
\nc{\tsha}{{\mbox{\cyrt X}}}
%\nc{\shpr}{}
\nc{\shpr}{\diamond}    %Shuffle product
\nc{\labs}{\mid\!}
\nc{\rabs}{\!\mid}
 \nc{\zhx}{\text{-}}

%\font\cyr=wncyr10
%\font\cyrs=wncyr7
%\font\cyrt=wncyr5

%%%%%%%%%%%%%%%%%%%%% roman fonts, in alphabetic order
\nc{\ad}{\mrm{ad}}
\nc{\ann}{\mrm{ann}}
\nc{\Av}{\mrm{Av}}
\nc{\bim}{\mbox{-}\mathsf{Bimod}}
\nc{\br}{\mrm{bre}}
\nc{\can}{\mrm{can}}
\nc{\Cont}{\mrm{Cont}}
\nc{\rchar}{\mrm{char}}
\nc{\cok}{\mrm{coker}}
\nc{\db}{\mrm{db}}
\nc{\de}{\mrm{dep}}
\nc{\dgg}{\mrm{dgg}}
\nc{\dgp}{\mrm{dgp}}
\nc{\dgx}{\mrm{dgx}}
\nc{\Dif}{\mrm{Diff}}
\nc{\dtf}{{R-{\rm tf}}}
\nc{\dtor}{{R-{\rm tor}}}
\renewcommand{\det}{\mrm{det}}
\nc{\Div}{{\mrm Div}}
\nc{\Diff}{\mrm{DA}}
\nc{\Diffl}{\mathsf{DA}_\lambda}
\nc{\diffo}{{\mathsf{DO}_\lambda}}
\nc{\dl}{{\mathrm{PD}}}
\nc{\dRB}{{\mathrm{\Phi}_\mathsf{DRB}}}
\nc{\udRB}{{\mathrm{\Phi}_\mathsf{uDRB}}}
\nc{\OdRB}{{\mathrm{\Phi}_\mathsf{DRB}^0}}
%\nc{\OdRBp}{{\mathrm{\Phi}_\mathsf{dRB}^0'}}
 \nc{\OudRB}{{\mathrm{\Phi}_\mathsf{uDRB}^0}}
\nc{\inte}{{\mathrm{\Phi}_\mathsf{ID}}}
\nc{\uinte}{{\mathrm{\Phi}_\mathsf{uID}}}
\nc{\Ointe}{{\mathrm{\Phi}_\mathsf{ID}^0}}
 \nc{\Ouinte}{{\mathrm{\Phi}_\mathsf{uID}^0}}


\nc{\alg}{\mathsf{Alg}}
\nc{\Fil}{\mrm{Fil}}
\nc{\Frob}{\mrm{Frob}}
\nc{\Gal}{\mrm{Gal}}
\nc{\GL}{\mrm{GL}}
\nc{\Hoch}{\mrm{Hoch}}
\nc{\hsr}{\mrm{H}}
\nc{\hpol}{\mrm{HP}}
\nc{\im}{\mrm{im}}
\nc{\Id}{\mrm{Id}}
\nc{\ID}{\mrm{ID}}
\nc{\Irr}{\mrm{Irr}}
\nc{\incl}{\mrm{incl}}
\nc{\length}{\mrm{length}}
\nc{\NLSW}{\mrm{NLSW}}
\nc{\Nij}{\mrm{Nij}}
\nc{\mchar}{\rm char}
\nc{\mpart}{\mrm{part}}
\nc{\ql}{{\QQ_\ell}}
\nc{\qp}{{\QQ_p}}
\nc{\rank}{\mrm{rank}}
\nc{\rcot}{\mrm{cot}}
\nc{\rdef}{\mrm{def}}
\nc{\rdiv}{{\rm div}}
\nc{\Rey}{\mrm{Rey}}
\nc{\rtf}{{\rm tf}}
\nc{\rtor}{{\rm tor}}
\nc{\res}{\mrm{res}}
\nc{\SL}{\mrm{SL}}
\nc{\Spec}{\mrm{Spec}}
\nc{\tor}{\mrm{tor}}
\nc{\tr}{\mrm{tr}}
\nc{\wt}{\mrm{wt}}
\def\ot{\otimes}
\def\red{\color{red}}
\nc{\udl}{{\mathrm{udl}}}


%%%%%%%%%%%%%%%%%% bold face

\nc{\bfk}{{\bf k}}
\nc{\bfone}{{\bf 1}}
\nc{\bfzero}{{\bf 0}}
\nc{\detail}{\marginpar{\bf More detail}
	\noindent{\bf Need more detail!}
	\svp}
\nc{\gap}{\marginpar{\bf Incomplete}\noindent{\bf Incomplete!!}
	\svp}
\nc{\FMod}{\mathbf{FMod}}
\nc{\Int}{\mathbf{Int}}
%\nc{\Mon}{\mathbf{Mon}}
%\nc{\proof}{\noindent{\bf Proof: }}
%\nc{\remark}{\noindent{\bf Remark: }}
\nc{\remarks}{\noindent{\bf Remarks: }}
%\nc{\Rep}{\mathbf{Rep}}
%\nc{\Rings}{\mathbf{Rings}}
%\nc{\Sets}{\mathbf{Sets}}




%%%%%%%%%%%%%%%%%%%Bbb fonts
\nc{\BA}{{\mathbb A}}   \nc{\CC}{{\mathbb C}}
\nc{\DD}{{\mathbb D}}   \nc{\EE}{{\mathbb E}}
\nc{\FF}{{\mathbb F}}  
\nc{\HH}{{\mathbb H}}   \nc{\LL}{{\mathbb L}}
\nc{\NN}{{\mathbb N}}   \nc{\PP}{{\mathbb P}}
\nc{\QQ}{{\mathbb Q}}   \nc{\RR}{{\mathbb R}}
\nc{\TT}{{\mathbb T}}   \nc{\VV}{{\mathbb V}}
\nc{\ZZ}{{\mathbb Z}}   \nc{\TP}{\widetilde{P}}


%%%%%%%%%%%%%%%%%%% cal fonts

\nc{\cala}{{\mathcal A}}    \nc{\calc}{{\mathcal C}}
\nc{\cald}{\mathcal{D}}     \nc{\cale}{{\mathcal E}}
\nc{\calf}{{\mathcal F}}    \nc{\calg}{{\mathcal G}}
\nc{\calh}{{\mathcal H}}    \nc{\cali}{{\mathcal I}}
\nc{\call}{{\mathcal L}}    \nc{\calm}{{\mathcal M}}
\nc{\caln}{{\mathcal N}}    \nc{\calo}{{\mathcal O}}
\nc{\calp}{{\mathcal P}}    \nc{\calr}{{\mathcal R}}
\nc{\cals}{{\mathcal S}}    \nc{\calt}{{\Omega}}
\nc{\calv}{{\mathcal V}}    \nc{\calw}{{\mathcal W}}
\nc{\calx}{{\mathcal X}}    \nc{\calu}{{\mathcal U}}
\nc{\caly}{{\mathcal Y}}


\nc{\uOpAlg}{{\mathfrak{uOpAlg}}}

\nc{\OpAlg}{{\mathfrak{OpAlg}}}

\nc{\ComOpAlg}{{\mathfrak{ComOpAlg}}}

\nc{\OpVect}{{\mathfrak{OpVect}}}

\nc{\OpSet}{{\mathfrak{OpSet}}}

\nc{\OpMon}{{\mathfrak{OpMon}}}

\nc{\ComOpMon}{{\mathfrak{ComOpMon}}}

\nc{\OpSem}{{\mathfrak{OpSem}}}

\nc{\ComOpSem}{{\mathfrak{ComOpSem}}}

\nc{\uAlg}{{\mathfrak{uAlg}}}



\nc{\ComAlg}{{\mathfrak{ComAlg}}}







\nc{\ComMon}{{\mathfrak{ComMon}}}



%%新加

\nc{\mtOpSet}{{\Omega\zhx\mathfrak{Set}}}

\nc{\mtOpSem}{{\Omega\zhx\mathfrak{Sem}}}

\nc{\mtOpMon}{{\Omega\zhx\mathfrak{Mon}}}



\nc{\mtOpVect}{{\Omega\zhx\mathfrak{Vect}}}





\nc{\mtOpAlg}{{\Omega\zhx\mathfrak{Alg}}}

\nc{\mtuOpAlg}{{\Omega\zhx\mathfrak{uAlg}}}

\nc{\ComSem}{\mathfrak{ComSem}}
%%%%%%%%%%%%%%%%%%  frak fonts
\nc{\fraka}{{\mathfrak a}}
\nc{\frakb}{\mathfrak{b}}
\nc{\frakg}{{\frak g}}
\nc{\frakl}{{\frak l}}
\nc{\fraks}{{\frak s}}
\nc{\frakB}{{\frak B}}
\nc{\frakm}{{\frak m}}
\nc{\frakM}{{\frak M}}
\nc{\frakp}{{\frak p}}
\nc{\frakW}{{\frak W}}
\nc{\frakX}{{\frak X}}
\nc{\frakS}{{\frak{S}}}
\nc{\frakA}{{\frak A}}
\nc{\frakx}{{\frakx}}

\nc{\frakMstar}{{\mathfrak{M}_\Omega^\star}}
\nc{\frakSstar}{{\mathfrak{S}_\Omega^\star}}

\newcommand{\yn}[1]{\textcolor{blue}{#1 }}
\newcommand{\ynr}[1]{\textcolor{blue}{\underline{Yunnan:}#1 }}
\newcommand{\yh}[1]{\textcolor{purple}{#1 }}
\newcommand{\li}[1]{\textcolor{red}{#1 }}
\nc{\lir}[1]{\textcolor{red}{\underline{Li:}#1 }}
\newcommand{\gd}[1]{\textcolor{green}{#1 }}


\newcommand{\brca}[1]{\left\lfloor #1 \right\rfloor}%%operaors
\newcommand{\brck}[2]{\left\lfloor #1 \right\rfloor_{#2}}%%operaors

\DeclareMathOperator{\Sph}{Sph}

%%Zuan
\newcommand{\tg}{\triangle}
\newcommand{\LO}{$\Omega$}
\DeclareMathOperator{\lm}{lm}
\DeclareMathOperator{\lc}{lc}
\DeclareMathOperator{\lt}{lt}
