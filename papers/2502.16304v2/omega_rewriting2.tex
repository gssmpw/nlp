\documentclass[a4paper,twoside,11pt,english,intlimits]{article}

%%Packages
%% 
%%%%Encodage
\usepackage[utf8]{inputenc}
\usepackage[T2A,T1]{fontenc}

\usepackage[english]{babel}

%%%%Extensions
\usepackage{amsmath,amsthm,amssymb,stmaryrd}
\allowdisplaybreaks
\usepackage{mathtools}
%%Polices
\usepackage[ttscale=.875]{libertine}
\usepackage[libertine]{newtxmath}
 \usepackage{mathrsfs}%
\usepackage{bbm}

\usepackage{float}

%%%Mise en page
\usepackage{setspace}
\usepackage{enumitem}%\usepackage{enumerate}
%\usepackage{etex,pictex}
\usepackage{fancyhdr,titlesec,url}
\usepackage[all,knot,poly]{xy}
\usepackage{array}
\usepackage{hyperref}
\usepackage{graphicx}
\usepackage[numbers,sort&compress]{natbib}
\usepackage{multirow}
\usepackage{ifthen}
\usepackage{time}
\usepackage{algorithm2e}
\usepackage{caption}
\usepackage{subcaption}

\usepackage{etoolbox}

\usepackage{cancel}
\usepackage{tikz}
\usetikzlibrary{automata, positioning}

\usepackage[capitalize,noabbrev]{cleveref}



% Cyrillic characters
\DeclareSymbolFont{cyrillic}{T2A}{cmr}{m}{n}
\DeclareMathSymbol{\Sha}{\mathalpha}{cyrillic}{216}

\newcommand{\bigast}{\mathop{\scalebox{1.5}{\raisebox{-0.2ex}{$\ast$}}}}%

\DeclareFontFamily{U}{mathb}{\hyphenchar\font45}
\DeclareFontShape{U}{mathb}{m}{n}{
<-6> mathb5 <6-7> mathb6 <7-8> mathb7
<8-9> mathb8 <9-10> mathb9
<10-12> mathb10 <12-> mathb12
}{}


%%%%%%%%%%%%%
%% MISE EN PAGE %%
%%%%%%%%%%%%%

%\numberwithin{equation}{subsection}
%\newcounter{fakecnt}[subsubsection]
%\def\thefakecnt{\arabic{subsubsection}}
%\renewcommand{\theequation}{\thefakecnt}



\DeclareFontFamily{U}{BOONDOX-calo}{\skewchar\font=45 }
\DeclareFontShape{U}{BOONDOX-calo}{m}{n}{
  <-> s*[1.05] BOONDOX-r-calo}{}
\DeclareFontShape{U}{BOONDOX-calo}{b}{n}{
  <-> s*[1.05] BOONDOX-b-calo}{}
\DeclareMathAlphabet{\mathcalb}{U}{BOONDOX-calo}{m}{n}
\SetMathAlphabet{\mathcalb}{bold}{U}{BOONDOX-calo}{b}{n}
%\DeclareMathAlphabet{\mathbcalb}{U}{BOONDOX-calo}{b}{n}
\newcommand{\caa}{\mathfrak} 


\newcommand{\conj}{\mathrel{\substack{\\[-.2em]\urcorner}}}
\newcommand{\comp}{\mathrel{\substack{\lrcorner \\[+.2em]}}}

\newcommand{\ltwocell}[3][0.5]{\ar@{}[#2] \ar@{=>}?(#1)+/r 0.175cm/;?(#1)+/l 0.175cm/^{#3}}
\newcommand{\ltwocello}[3][0.5]{\ar@{}[#2] \ar@{=>}?(#1)+/r 0.175cm/;?(#1)+/l 0.175cm/_{#3}}
\newcommand{\dtwocell}[3][0.5]{\ar@{}[#2] \ar@{=>}?(#1)+/u  0.175cm/;?(#1)+/d 0.175cm/^{#3}}


\newcommand{\dblcat}[2]{\mathbb{#1}\mathrm{#2}}
%\newcommand{\EnDblcat}[2]{\mathscr{#1}\mathrm{#2}}
%\newcommand{\lcbicat}[2]{\mathscr{#1}\mathcalb{#2}}

%\newcommand{\Mod}{\dblcat{M}{od}}
%\newcommand{\PsMod}{\dblcat{P}{sMod}}
%\newcommand{\TiPsMod}{\dblcat{T}{iPsMod}}
%\newcommand{\LoPsMod}{\dblcat{L}{oPsMod}}
\newcommand{\Hor}{\dblcat{H}{or}}
\renewcommand{\Vert}{\dblcat{V}{ert}}


%\newcommand{\Span}{\dblcat{S}{pan}}
%\newcommand{\Sq}{\dblcat{S}{q}}
\newcommand{\Quin}{\dblcat{Q}{uin}}

\newcounter{Assume}
\setcounter{Assume}{1}
\newenvironment{Assumption}[1][Assumption]{\medskip\noindent\textsc{#1 \theAssume :} }{\addtocounter{Assume}{1}\begin{flushright}\end{flushright}}


\newcommand{\cd}[2][]{\vcenter{\hbox{\xymatrix#1{#2}}}}
\newcommand{\ot}{\leftarrow}
\newcommand{\toto}{\rightrightarrows}
\newcommand{\twoto}{\Rightarrow}
\newcommand{\threeto}{\Rrightarrow}
\newcommand{\threeot}{\Lleftarrow}


\newcommand{\twocong}[2][0.5]{\ar@{}[#2] \save ?(#1)*{\cong}\restore}
\newcommand{\twoeq}[2][0.5]{\ar@{}[#2] \save ?(#1)*{=}\restore}
\newcommand{\threeeq}[2][0.5]{\ar@{}[#2] \save ?(#1)*{\equiv}\restore}



\newcommand{\horaru}[3][0.5]{\ar@{->}[#3]^(#1){#2}\ar@{}[#3]|(#1){\scriptstyle{\bullet}} }
\newcommand{\horard}[3][0.5]{\ar@{->}[#3]_(#1){#2}\ar@{}[#3]|(#1){\scriptstyle{\bullet}} }
\newcommand{\horarm}[3][0.5]{\ar@{->}[#3]|(#1){#2}\ar@{}[#3]|(0.3){\scriptstyle{\bullet}} }


\newcommand{\circaruh}[2]{\ar@{->}[#2]^{#1}\ar@{}[#2]|-{\scriptstyle{\ominus}} }
\newcommand{\circardh}[2]{\ar@{->}[#2]_{#1}\ar@{}[#2]|-{\scriptstyle{\ominus}} }
\newcommand{\circarmh}[2]{\ar@{->}[#2]|-{#1}\ar@{}[#2]|-{\scriptstyle{\ominus}} }

\newcommand{\horto}[2]{\SelectTips{eu}{10}
\xymatrix@C=.2in{#1\horaru{}{r} & #2}}
\newcommand{\circto}[2]{\SelectTips{eu}{10}
\xymatrix@C=.2in{#1\circaruh{}{r} & #2}}

%Hebrew letters

%\DeclareFontFamily{U}{rcjhbltx}{}
%\DeclareFontShape{U}{rcjhbltx}{m}{n}{<->rcjhbltx}{}
%\DeclareSymbolFont{hebrewletters}{U}{rcjhbltx}{m}{n}

%\DeclareMathSymbol{\gimell}{\mathord}{hebrewletters}{130}
%\DeclareMathSymbol{\dalet}{\mathord}{hebrewletters}{100}
%\DeclareMathSymbol{\dalett}{\mathord}{hebrewletters}{131}
%\DeclareMathSymbol{\he}{\mathord}{hebrewletters}{104}
%\DeclareMathSymbol{\vet}{\mathord}{hebrewletters}{107}
%\DeclareMathSymbol{\zayin}{\mathord}{hebrewletters}{135}
%\DeclareMathSymbol{\tet}{\mathord}{hebrewletters}{135}
%\DeclareMathSymbol{\bet}{\mathord}{hebrewletters}{129}
%\DeclareMathSymbol{\nun}{\mathord}{hebrewletters}{110}
%\DeclareMathSymbol{\nuns}{\mathord}{hebrewletters}{78}
%\DeclareMathSymbol{\resh}{\mathord}{hebrewletters}{114}
%\DeclareMathSymbol{\pe}{\mathord}{hebrewletters}{112}
%\DeclareMathSymbol{\fe}{\mathord}{hebrewletters}{112}
%\DeclareMathSymbol{\tav}{\mathord}{hebrewletters}{116}
%\DeclareMathSymbol{\het}{\mathord}{hebrewletters}{120}
%\DeclareMathSymbol{\yod}{\mathord}{hebrewletters}{121}
%\DeclareMathSymbol{\fes}{\mathord}{hebrewletters}{80}
%\DeclareMathSymbol{\lamed}{\mathord}{hebrewletters}{108}
%\DeclareMathSymbol{\mem}{\mathord}{hebrewletters}{109}\let\mim\mem
%\DeclareMathSymbol{\mems}{\mathord}{hebrewletters}{77}
%\DeclareMathSymbol{\ayin}{\mathord}{hebrewletters}{96}
%\DeclareMathSymbol{\tsadi}{\mathord}{hebrewletters}{118}
%\DeclareMathSymbol{\tsadis}{\mathord}{hebrewletters}{90}
%\DeclareMathSymbol{\qof}{\mathord}{hebrewletters}{113}
%\DeclareMathSymbol{\shin}{\mathord}{hebrewletters}{152}
%\DeclareMathSymbol{\xafs}{\mathord}{hebrewletters}{75}
%\DeclareMathSymbol{\shin}{\mathord}{hebrewletters}{152}
%\DeclareMathSymbol{\kaf}{\mathord}{hebrewletters}{138}
%\DeclareMathSymbol{\samek}{\mathord}{hebrewletters}{142}
%\DeclareMathSymbol{\samekh}{\mathord}{hebrewletters}{115}
%\DeclareMathSymbol{\pei}{\mathord}{hebrewletters}{144}

\newcommand{\Ff}{\mathfrak{F}}
\newcommand{\Gf}{\mathfrak{G}}
\newcommand{\Hf}{\mathfrak{H}}
%\newcommand{\If}{\mathfrak{I}}
\newcommand{\Jf}{\mathfrak{J}}
\newcommand{\Kf}{\mathfrak{K}}
\newcommand{\Lf}{\mathfrak{L}}
\newcommand{\Mf}{\mathfrak{M}}
\newcommand{\Of}{\mathfrak{O}}
\newcommand{\Pf}{\mathfrak{P}}
\newcommand{\Qf}{\mathfrak{Q}}
\newcommand{\Rf}{\mathfrak{R}}
\newcommand{\Tf}{\mathfrak{T}}
\newcommand{\Uf}{\mathfrak{U}}
\newcommand{\Vf}{\mathfrak{V}}
\newcommand{\Wf}{\mathfrak{W}}
\newcommand{\Xf}{\mathfrak{X}}
\newcommand{\Yf}{\mathfrak{Y}}
\newcommand{\Zf}{\mathfrak{Z}}


\newcommand{\critMD}{\blacktriangleleft}

\makeatletter
%\DeclareMathSymbol{\b@r}{\mathord}{symbols}{"2E}
\DeclareMathSymbol{\b@r}{\mathord}{largesymbols}{"0E}
\newcommand{\newc@ncel}[2]{%
  \ooalign{%
    \hfil$\vcenter{\moved@wn{#1}\hbox{\scalebox{1}[0.5]{$#1\b@r$}}}$\hfil\cr % the bar
    $#1#2$\cr % the symbol
  }%
}
\newcommand{\moved@wn}[1]{%
  \sbox\z@{$#1\mkern-3mu\nonscript\mkern3mu$}%
  \vskip\wd\z@
}
\newcommand{\nprecnapprox}{%
  \mathrel{\m@th\mathpalette\newc@ncel\precapprox}%
}
\newcommand{\nsuccnapprox}{%
  \mathrel{\m@th\mathpalette\newc@ncel\succapprox}%
}
\makeatother



%% Styles des sections
\newcommand{\periodafter}[1]{\ifstrempty{#1}{}{#1.}}
\titleformat{\section}[block]{\scshape\filcenter\LARGE\boldmath}{\thesection.}{.5em}{}
\titleformat{\subsection}[block]{\bfseries\filcenter\large\boldmath}{\thesubsection.}{.5em}{\medskip}
\titleformat{\subsubsection}[runin]{\bfseries\boldmath}{\thesubsubsection.}{.5em}{\periodafter}%{}[.]
\titlespacing{\subsubsection}{0pt}{\topsep}{.5em}

%% Styles des théorèmes
\newtheoremstyle{ntheorem}%
	{\topsep}{\topsep}{\itshape}{0pt}{\bfseries}{.}{.5em}%
	{\thmnumber{#2.\hspace{.5em}}\thmname{#1}\thmnote{ (#3)}}
	
\newtheoremstyle{ndefinition}%
	{\topsep}{\topsep}{\normalfont}{0pt}{\bfseries}{.}{.5em}%
	{\thmnumber{#2.\hspace{.5em}}\thmname{#1}\thmnote{ (#3)}}
	
\newtheoremstyle{nremark}%
	{\topsep}{\topsep}{\normalfont}{0pt}{\itshape}{.}{.5em}%
	{\thmnumber{}\thmname{#1}\thmnote{ (#3)}}
	
\newtheoremstyle{nfigure}%
	{\topsep}{\topsep}{\itshape}{0pt}{\bfseries}{.}{.5em}%
	{\thmnumber{#2.\hspace{.5em}}\thmname{#1}.\thmnote{ #3}}
	
\theoremstyle{ntheorem}
  	\newtheorem{theorem}[subsubsection]{Theorem}
  	\newtheorem{proposition}[subsubsection]{Proposition}
	\newtheorem{lemma}[subsubsection]{Lemma}
  	\newtheorem{corollary}[subsubsection]{Corollary}
  	\newtheorem{hyp}{Hypothesis}
	
\theoremstyle{nfigure}
	\newtheorem{myfig}[subsubsection]{Figure}

\theoremstyle{ndefinition}
	\newtheorem{definition}[subsubsection]{Definition}
	\newtheorem{notation}[subsubsection]{Notation}
	\newtheorem{example}[subsubsection]{Example}
	\newtheorem{remark}[subsubsection]{Remark}
	
%\makeatletter
%\def\@equationname{equation}
%\newenvironment{eqn}[1]{%
%\def\mymathenvironmenttouse{#1}%
%\ifx\mymathenvironmenttouse\@equationname%
%\refstepcounter{subsubsection}%
%\setcounter{equation}{\value{subsubsection}}% Keep the equation counter synchronized for hyperref reasons...
%\else
%\patchcmd{\@arrayparboxrestore}{equation}{subsubsection}{}{}% doesn't change output?
%\patchcmd{\print@eqnum}{equation}{subsubsection}{}{}%
%% Set counter equal to subsubsection, then increment subsubsection. The original command already increments equation.
%\pretocmd{\incr@eqnum}{\setcounter{equation}{\value{subsubsection}}\refstepcounter{subsubsection}}{}{}%
%% \def\print@eqnum{\tagform@\thematrix}% instead of etoolbox' \pathcmd
%% \def\incr@eqnum{\refstepcounter{matrix}\let\incr@eqnum\@empty}% instead of etoolbox' \pathcmd
%\fi
%\csname\mymathenvironmenttouse\endcsname%
%}{%
%\ifx\mymathenvironmenttouse\@equationname%
%\tag{\thesubsubsection}%
%\fi
%\csname end\mymathenvironmenttouse\endcsname%
%}
%\makeatother


\makeatletter
\def\@equationname{equation}
\newenvironment{eqn}[1]{%
    \def\mymathenvironmenttouse{#1}%
    \ifx\mymathenvironmenttouse\@equationname%
        \refstepcounter{subsubsection}%
    \else
        \patchcmd{\@arrayparboxrestore}{equation}{subsubsection}{}{}%          doesn't change output?
        \patchcmd{\print@eqnum}{equation}{subsubsection}{}{}%
        \patchcmd{\incr@eqnum}{equation}{subsubsection}{}{}%
%       \def\print@eqnum{\tagform@\thematrix}%                          instead of etoolbox' \pathcmd
%       \def\incr@eqnum{\refstepcounter{matrix}\let\incr@eqnum\@empty}% instead of etoolbox' \pathcmd
    \fi
    \csname\mymathenvironmenttouse\endcsname%
}{%
    \ifx\mymathenvironmenttouse\@equationname%
        \tag{\thesubsubsection}%
    \fi
    \csname end\mymathenvironmenttouse\endcsname%
}
\makeatother

%New variant of equation*
\makeatletter
\newenvironment{eq*}{%
 \incr@eqnum
  $
%  \mathdisplay{equation*}%
}{%
%  \endmathdisplay{equation*}%
  $
  \ignorespacesafterend
}
\makeatother
\newcommand{\tof}[1]{\quad\xrightarrow{#1} \quad}
%\newcommand{\toff}[2]{\xrightarrow{#1}_{#2}}
\newcommand{\toff}[2]{\underset{#2}{\xrightarrow{#1}}}
\newcommand{\otff}[2]{\underset{#2}{\hspace{-.2em}\xleftarrow{#1}\hspace{.3em}}}
\newcommand{\otfff}[2]{\underset{#2}{\hspace{-.2em}\xleftarrow{#1}\hspace{.3em}}}


\newcounter{DefEq}
\setcounter{DefEq}{0}
\newcommand{\deftag}{\addtocounter{DefEq}{1}\tag{D\theDefEq }}

\newcommand{\totag}{\tof{(\theequation)}}
\newcommand{\mtag}{(\theequation)}
\newcommand{\eqtag}{{(\theequation)}}

\newcommand{\ol}[2]{\overset{#1}{#2}}
\newcommand{\Eqtag}[1]{\ol{(\theequation)}{#1}}


%% Mise en page
\pagestyle{fancy}
\setlength{\oddsidemargin}{0cm}
\setlength{\evensidemargin}{0cm}
\setlength{\topmargin}{0cm} 
\setlength{\headheight}{1cm}
\setlength{\headsep}{1cm}
\setlength{\textwidth}{16cm}
\setlength{\marginparwidth}{0cm}
\setlength{\footskip}{2cm}
\setlength{\headwidth}{16cm}
\newcommand{\emptysectionmark}[1]{\markboth{\textbf{#1}}{\textbf{#1}}}
\renewcommand{\sectionmark}[1]{\markboth{\textbf{\thesection. #1}}{\textbf{\thesection. #1}}}
\renewcommand{\subsectionmark}[1]{\markright{\textbf{\thesubsection. #1}}}
\fancyhead{}\fancyfoot[LC,RC]{}
\fancyhead[LE]{\leftmark}
\fancyhead[RO]{\rightmark}
\fancyfoot[LE,RO]{$\thepage$}
\fancypagestyle{plain}{
\fancyhf{}\fancyfoot[LC,RC]{}
\fancyfoot[LE,RO]{$\thepage$}
\renewcommand{\headrulewidth}{0pt}
\renewcommand{\footrulewidth}{0pt}}
\newcommand{\headrulelength}{16cm}
\renewcommand{\headrulewidth}{1pt}
\renewcommand{\footrulewidth}{0pt}
\setlength{\arraycolsep}{1pt}
\renewcommand{\headrule}{
  {\hrule width\headwidth height\headrulewidth
   \vskip-\headrulewidth}
}

%%%%%%Macro texte%%%%%%
\newcommand{\pdf}[1]{\texorpdfstring{$#1$}{1}}

%%%%%%%%
%% XYPIC %%
%%%%%%%%

%% Options 
\renewcommand{\objectstyle}{\displaystyle}
\renewcommand{\labelstyle}{\displaystyle}
\UseTips
\SelectTips{eu}{11}

%% Nouvelle direction pour les monomorphismes (>->)
\newdir{ >}{{}*!/-10pt/@{>}}
\newdir{ -}{{}*!/-10pt/@{}}
\newdir{> }{{}*!/+10pt/@{>}}

%% Ajouts de la variante 4 pour les blancs, les lignes, les pointill�s, 
%% les zigouigouis (pas top) et les t�tes de fl�ches (presque finies).
%% Concr�tement, dans \xymatrix, la commande \ar @4 produit une 4-cellule.
\makeatletter

% Noms des nouvelles directions
\xyletcsnamecsname@{dir4{}}{dir{}}
\xydefcsname@{dir4{-}}{\line@ \quadruple@\xydashh@}
\xydefcsname@{dir4{.}}{\point@ \quadruple@\xydashh@}
\xydefcsname@{dir4{~}}{\squiggle@ \quadruple@\xybsqlh@}
\xydefcsname@{dir4{>}}{\Tttip@}
\xydefcsname@{dir4{<}}{\reverseDirection@\Tttip@}

% Commande quadruple
% Attention : la position des pourcents est importante !!
\xydef@\quadruple@#1{%
	\edef\Drop@@{%
		\dimen@=#1\relax
		\dimen@=.5\dimen@
		\A@=-\sinDirection\dimen@
		\B@=\cosDirection\dimen@
		\setboxz@h{%
			\setbox2=\hbox{\kern3\A@\raise3\B@\copy\z@}%
			\dp2=\z@ \ht2=\z@ \wd2=\z@ \box2
			\setbox2=\hbox{\kern\A@\raise\B@\copy\z@}%
			\dp2=\z@ \ht2=\z@ \wd2=\z@ \box2
			\setbox2=\hbox{\kern-\A@\raise-\B@\copy\z@}%
			\dp2=\z@ \ht2=\z@ \wd2=\z@ \box2
			\setbox2=\hbox{\kern-3\A@\raise-3\B@ \noexpand\boxz@}%
			\dp2=\z@ \ht2=\z@ \wd2=\z@ \box2
		}%
		\ht\z@=\z@ \dp\z@=\z@ \wd\z@=\z@ \noexpand\styledboxz@
	}%
}

% T�tes de quadruples fl�ches : positionnement ok, reste la jointure
% Les parties comment�es sont des essais � reprendre plus tard
\xydef@\Tttip@{\kern2pt \vrule height2pt depth2pt width\z@
	\Tttip@@ \kern2pt \egroup
	\U@c=0pt \D@c=0pt \L@c=0pt \R@c=0pt \Edge@c={\circleEdge}%
	\def\Leftness@{.5}\def\Upness@{.5}%
	\def\Drop@@{\styledboxz@}\def\Connect@@{\straight@{\dottedSpread@\jot}}}
	
\xydef@\Tttip@@{%
	\dimen@=.25\dimen@
%	\A@=-\sinDirection\dimen@
 	\B@=\cosDirection\dimen@
	\setboxz@h\bgroup\reverseDirection@\line@ \wdz@=\z@ \ht\z@=\z@ \dp\z@=\z@
%	\kern\A@ \raise\B@ \boxz@ \kern\L@c
%	\kern-\L@c \boxz@ \kern\L@c
	{\vDirection@(1,-1)\xydashl@ \xyatipfont\char\DirectionChar}%
	{\vDirection@(1,+1)\xydashl@ \xybtipfont\char\DirectionChar}%
}

% Red�finition de la commande \ar 
% Provoque un avertissement � la compilation
\xydef@\ar@form{
	\ifx \space@\next \expandafter\DN@\space{\xyFN@\ar@form}%
	\else\ifx ^\next \DN@ ^{\xyFN@\ar@style}\edef\arvariant@@{\string^}%
	\else\ifx _\next \DN@ _{\xyFN@\ar@style}\edef\arvariant@@{\string_}%
	\else\ifx 0\next \DN@ 0{\xyFN@\ar@style}\def\arvariant@@{0}%
	\else\ifx 1\next \DN@ 1{\xyFN@\ar@style}\def\arvariant@@{1}%
	\else\ifx 2\next \DN@ 2{\xyFN@\ar@style}\def\arvariant@@{2}%
	\else\ifx 3\next \DN@ 3{\xyFN@\ar@style}\def\arvariant@@{3}%
	\else\ifx 4\next \DN@ 4{\xyFN@\ar@style}\def\arvariant@@{4}%
	\else\ifx \bgroup\next \let\next@=\ar@style
	\else\ifx [\next \DN@[##1]{\ar@modifiers{[##1]}}%]
	\else\ifx *\next \DN@ *{\ar@modifiers}%
	\else\addLT@\ifx\next \let\next@=\ar@slide
	\else\ifx /\next \let\next@=\ar@curveslash
	\else\ifx (\next \let\next@=\ar@curveinout %)
	\else\addRQ@\ifx\next \addRQ@\DN@{\ar@curve@}%
	\else\addLQ@\ifx\next \addLQ@\DN@{\xyFN@\ar@curve}%
	\else\addDASH@\ifx\next \addDASH@\DN@{\defarstem@-\xyFN@\ar@}%
	\else\addEQ@\ifx\next \addEQ@\DN@{\def\arvariant@@{2}\defarstem@-\xyFN@\ar@}%
	\else\addDOT@\ifx\next \addDOT@\DN@{\defarstem@.\xyFN@\ar@}%
	\else\ifx :\next \DN@:{\def\arvariant@@{2}\defarstem@.\xyFN@\ar@}%
	\else\ifx ~\next \DN@~{\defarstem@~\xyFN@\ar@}%
	\else\ifx !\next \DN@!{\dasharstem@\xyFN@\ar@}%
	\else\ifx ?\next \DN@?{\ar@upsidedown\xyFN@\ar@}%
	\else \let\next@=\ar@error
	\fi\fi\fi\fi\fi\fi\fi\fi\fi\fi\fi\fi\fi\fi\fi\fi\fi\fi\fi\fi\fi\fi\fi \next@}

\makeatother


%%%%%%%%%%%%%%%%%
%% SYMBOLES %%%%%%%%
%%%%%%%%%%%%%%%%%
\newcommand{\ob}{\mathrm{ob}\ }

%% Flèches
\newcommand{\fl}{\rightarrow}
\newcommand{\fll}{\longrightarrow}
\newcommand{\ofl}[1]{\overset{\displaystyle #1}{\fll}}
\newcommand{\ifl}{\rightarrowtail}
\newcommand{\pfl}{\twoheadrightarrow}
\newcommand{\fllg}{\longleftarrow}
\newcommand{\oflg}[1]{\overset{\displaystyle #1}{\fllg}}

\newcommand{\dfl}{\Rightarrow}
\newcommand{\dfll}{\Longrightarrow}
\newcommand{\odfl}[1]{\overset{\displaystyle #1}{\dfll}}

\newcommand{\tfl}{\Rrightarrow}
\newcommand{\qfl}{\xymatrix@1@C=10pt{\ar@4 [r] &}}

%\newcommand{\multimapdotboth}{\xymatrix@1@C=10pt{\ar @{*-*} [r] &}}
\newcommand{\equivmap}{\xymatrix@1@C=10pt{\ar @{|-|} [r] &}}


%%%% One Cells

%%\parallelto{dr}{blue}{>_\Pr \pi}{V}

\newcommand{\parallelto}[4]{\ar @{} [#1] |(.6)*+<2pt>{}="w1" |(.9)*+<2pt>{}="w2" 
  \ar @[#2] @{-||} []; "w1" 
    \ar @[#2] []; "w2"  _-{\color{#2} #3 } ^-{\color{#2} #4}}
    
\newcommand{\parallelcurveto}[5]{\ar @{} @/#5/ [#1] |(.6)*+<2pt>{}="w1" |(.9)*+<2pt>{}="w2" 
  \ar @[#2] @{-||} []; "w1" 
    \ar @[#2]  @/#5/ []; "w2"  _-{\color{#2} #3 } ^-{\color{#2} #4}}    
    
    \newcommand{\paralleltom}[6]{\ar @{} [#1] |(.6)*+<2pt>{}="w1" |(.9)*+<2pt>{}="w2" 
  \ar @[#2]  @{-||} []; "w1" 
    \ar @[#2]  []; "w2"  _-{\color{#2} #3 } ^-{\color{#2} #4}  |(.8){\color{#5} #6}}
    
        \newcommand{\paralleldemitom}[6]{\ar @{} [#1] |(.3)*+<2pt>{}="w1" |(.48)*+<2pt>{}="w2"  
  \ar @[#2] @{-||} []; "w1" 
    \ar @[#2] []; "w2"  _-{\color{#2} #3 } ^-{\color{#2} #4}  |(.8){\color{#5} #6}}
     \newcommand{\parallelfromdemitom}[6]{\ar @{} [#1] |(.54)*+<2pt>{}="w0" |(.75)*+<2pt>{}="w1" |(.98)*+<2pt>{}="w2"  
  \ar @[#2] @{-||} "w0"; "w1" 
    \ar @[#2] "w0"; "w2"  _-{\color{#2} #3 } ^-{\color{#2} #4}  |(.8){\color{#5} #6}}

    \newcommand{\parallellongtom}[6]{\ar @{} [#1] |(.6)*+<2pt>{}="w1" |(.98)*+<2pt>{}="w2" 
  \ar @[#2]  @{-||} []; "w1" 
    \ar @[#2] []; "w2"  _-{\color{#2} #3 } ^-{\color{#2} #4}  |(.8){\color{#5} #6}}
        
    \newcommand{\parallelcurvetom}[7]{
    \ar @{->} @/_6ex/ [#2] _-{\color{#2} #3 } ^-{\color{#2} #4}  |(.8){\color{#5} #6}
  \ar   @[#2] @{-||} @/_2ex/ 
    }

    \newcommand{\paralleltomd}[6]{\ar @{} [#1] |(.6)*+<2pt>{}="w1" |(.9)*+<2pt>{}="w2" 
  \ar @[#2] @{-||} []; "w1" 
    \ar @[#2] @{->>} []; "w2"  _-{\color{#2} #3 } ^-{\color{#2} #4}  |(.8){\color{#5} #6}}
    
    \newcommand{\tom}[6]{\ar @{} [#1] |(.6)*+<2pt>{}="w1" |(.9)*+<2pt>{}="w2" 
    \ar @[#2] @{->} []; "w2"  _-{\color{#2} #3 } ^-{\color{#2} #4}  |(.8){\color{#5} #6}}

\newcommand{\affineto}[4]{\ar @{} [#1] |(.8)*+<2pt>{}="w1" |(.9)*+<2pt>{}="w2" 
  \ar @[#2] @{-}  "w1" ; "w2" _-{\color{#2} =}
    \ar @[#2] []; "w2"  _-{\color{#2} #3 } ^-{\color{#2} #4}}

\newcommand{\affinetom}[6]{\ar @{} [#1] |(.8)*+<2pt>{}="w1" |(.9)*+<2pt>{}="w2" 
  \ar @[#2] @{-}  "w1" ; "w2" _-{\color{#2} =}
    \ar @[#2] @{->>} []; "w2"  _-{\color{#2} #3 } ^-{\color{#2} #4}|(.8){\color{#5} #6}}
    
\newcommand{\positiveto}[4]{\ar @{} [#1] |(.8)*+<2pt>{}="w1" |(.9)*+<2pt>{}="w2" 
  \ar @[#2] @{-}  "w1" ; "w2" ^-{\color{#2} +}
    \ar @[#2] @{->>} []; "w2"  _-{\color{#2} #3 } ^-{\color{#2} #4}}

\newcommand{\positiveleftto}[4]{\ar @{} [#1] |(.8)*+<2pt>{}="w1" |(.9)*+<2pt>{}="w2" 
  \ar @[#2] @{-}  "w1" ; "w2" _-{\color{#2} +}
    \ar @[#2] @{->>} []; "w2"  _-{\color{#2} #3 } ^-{\color{#2} #4}}

\newcommand{\positivetom}[6]{\ar @{} [#1] |(.8)*+<2pt>{}="w1" |(.9)*+<2pt>{}="w2" 
  \ar @[#2] @{-}  "w1" ; "w2" ^-{\color{#2} +}
    \ar @[#2] []; "w2"  _-{\color{#2} #3 } ^-{\color{#2} #4}|(.8){\color{#5} #6}}


\newcommand{\affinelongtom}[6]{\ar @{} [#1] |(.8)*+<2pt>{}="w1" |(.98)*+<2pt>{}="w2" 
  \ar @[#2] @{-}  "w1" ; "w2" _-{\color{#2} =}
    \ar @[#2] []; "w2"  _-{\color{#2} #3 } ^-{\color{#2} #4}|(.7){\color{#5} #6}}

\newcommand{\affineshortto}[4]{\ar @{} [#1] |(.5)*+<2pt>{}="w1" |(.9)*+<2pt>{}="w2" 
  \ar @[#2] @{-}  "w1" ; "w2" _-{\color{#2} =}
    \ar @[#2] []; "w2"  _-{\color{#2} #3 } ^-{\color{#2} #4}}

\newcommand{\affineshorttom}[6]{\ar @{} [#1] |(.5)*+<2pt>{}="w1" |(.9)*+<2pt>{}="w2" 
  \ar @[#2] @{-}  "w1" ; "w2" _-{\color{#2} =}
    \ar @[#2] []; "w2"  _-{\color{#2} #3 } ^-{\color{#2} #4} |(.5){\color{#5} #6}}

  
\newcommand{\affineeqto}[4]{\ar @{} [#1] |(.7)*+<2pt>{}="w1" |(.9)*+<2pt>{}="w2" 
  \ar @[#2] @{-}  "w1" ; "w2" _-{\color{#2} =}
    \ar @[#2] @{|-|} []; "w2"  _-{\color{#2} #3 } ^-{\color{#2} #4}}
    

    
\newcommand{\paraffto}[4]{\ar @{} [#1] |(.6)*+<2pt>{}="w1" |(.8)*+<2pt>{}="w2" |(.9)*+<2pt>{}="w3" 
  \ar @[#2] @{-||} []; "w1"
   \ar @[#2] @{-}  "w2" ; "w3" _-{\color{#2} =} 
    \ar @[#2] []; "w3"  _-{\color{#2} #3 } ^-{\color{#2} #4}}
    
    
\newcommand{\parafftom}[6]{\ar @{} [#1] |(.6)*+<2pt>{}="w1" |(.8)*+<2pt>{}="w2" |(.9)*+<2pt>{}="w3" 
  \ar @[#2] @{-||} []; "w1"
   \ar @[#2] @{-}  "w2" ; "w3" _-{\color{#2} =} 
    \ar @[#2] []; "w3"  _-{\color{#2} #3 } ^-{\color{#2} #4}|(.8){\color{#5} #6}}
    
\newcommand{\parafflongtom}[6]{\ar @{} [#1] |(.6)*+<2pt>{}="w1" |(.8)*+<2pt>{}="w2" |(.98)*+<2pt>{}="w3" 
  \ar @[#2] @{-||} []; "w1"
   \ar @[#2] @{-}  "w2" ; "w3" _-{\color{#2} =} 
    \ar @[#2] []; "w3"  _-{\color{#2} #3 } ^-{\color{#2} #4}|(.7){\color{#5} #6}}
    
%\newcommand{\protwoard}[2]{\ar@{=>}[#2]_{#1}\ar@{}[#2]|-{\scriptstyle{\mid}} }
%\newcommand{\protwoarm}[2]{\ar@{=>}[#2]|-{#1}\ar@{}[#2]|-{\scriptstyle{\mid}} }


%% Parenthèses et crochets
\newcommand{\ens}[1]{\left\{ #1 \right\}}
\renewcommand{\angle}[1]{\langle #1 \rangle}
\newcommand{\pres}[2]{\langle #1 \;\vert\; #2 \rangle}
\newcommand{\bigpres}[2]{\big\langle\; #1 \;\;\big\vert\;\; #2 \;\big\rangle}
\newcommand{\cohpres}[3]{\langle #1 \;\vert\; #2 \;\vert\; #3 \rangle}
\newcommand{\bigcohpres}[3]{\big\langle\; #1 \;\;\big\vert\;\; #2 \;\;\big\vert\;\; #3 \;\big\rangle}

%% Accents
\newcommand{\po}[1]{{#1}^{\Box}}
\newcommand{\op}[1]{{#1}^{o}}
\newcommand{\hop}[2]{{#1}^{o_{#2}}}%higher opposite
\newcommand{\da}[1]{{#1}^{\dagger}}
\newcommand{\env}[1]{{#1}^{e}}
\newcommand{\cl}[1]{\overline{#1}}
\newcommand{\rep}[1]{\widehat{#1}}
\newcommand{\tild}[1]{\widetilde{#1}}
\newcommand{\tck}[1]{#1^{\top}}
\newcommand{\vect}[1]{\overrightarrow{#1}}

%% Opérateurs
\DeclareMathOperator{\Cons}{Con}
\DeclareMathOperator{\Con}{Con}
\DeclareMathOperator{\id}{Id}
\DeclareMathOperator{\Ker}{Ker}
\DeclareMathOperator{\coker}{coker}
\DeclareMathOperator{\Coker}{Coker}
\DeclareMathOperator{\Subst}{Subst}

\newcommand{\BigCop}{\bigsqcup}
\newcommand{\BigCoproduct}{\bigsqcup}
\newcommand{\Cop}{\sqcup}
\newcommand{\Coproduct}{\sqcup}

\DeclareMathOperator{\End}{End}
\DeclareMathOperator{\Aut}{Aut}
\DeclareMathOperator{\Lan}{Lan}

\DeclareMathOperator{\h}{h}
%\DeclareMathOperator{\crit}{c}
\DeclareMathOperator{\Hom}{\mathrm{Hom}}
\DeclareMathOperator{\Ext}{\mathrm{Ext}}
\DeclareMathOperator{\Tor}{\mathrm{Tor}}
\DeclareMathOperator{\Der}{\mathrm{Der}}

\DeclareMathOperator{\FDT}{FDT}
\DeclareMathOperator{\FDTAB}{FDT_{\mathrm{ab}}}
\DeclareMathOperator{\FP}{FP}
\DeclareMathOperator{\FHT}{FHT}
\DeclareMathOperator{\Ob}{Ob}
\DeclareMathOperator{\Var}{Var}


%% Opérations binaires=Products


\newcommand{\lmod}{\mathrel{|}\joinrel\Relbar}
\newcommand{\rmod}{\Relbar\joinrel\mathrel{|}}

\newcommand{\oleq}{\sqsubseteq }
\newcommand{\ogeq}{\sqsupseteq }
\newcommand{\tens}{\otimes}
\newcommand{\Ortho}{\mathord{\Perp}}
\newcommand{\ortho}{\mathord{\perp}}
\newcommand{\critB}{\mathrel{\natural}}
\newcommand{\succeqE}{\mathrel{\succeq_{\Er}}}
\newcommand{\succE}{\mathrel{\succ_{\Er}}}
\newcommand{\approxE}{\mathrel{\approx_{\Er}}}
\newcommand{\succeqH}{\mathrel{((\succeq_{\Er})_{lex})_H}}
\newcommand{\succH}{\mathrel{((\succ_{\Er})_{lex})_H}}
\newcommand{\succeqL}{\mathrel{(\succeq_{\Er})_{lex}}}
\newcommand{\succL}{\mathrel{(\succ_{\Er})_{lex}}}
%\newcommand{\ortho}{\Perp}
%\newcommand{\rperp}{\flat}
\newcommand{\rperp}{\rmod}
\newcommand{\imod}{\mathrel{|}\joinrel< }
\newcommand{\smod}{>\joinrel\mathrel{|}}
\newcommand{\leqmod}{\mathrel{|}\joinrel\leq} 
\newcommand{\geqmod}{\geq\joinrel\mathrel{|}}
\newcommand{\Sharp}{\sharp}
\newcommand{\crit}{\natural}
\newcommand{\conf}{\pitchfork}
\renewcommand\o{\otimes}
\def\ogray{\otimes_\catego{g}}
\def\cgray{\boxtimes_\catego{g}}


%\newcommand{\Perp}{\perp\mkern-9.5mu\perp}

%% Caractères 
%\renewcommand{\phi}{\varphi}
\renewcommand{\epsilon}{\varepsilon}

\newcommand{\Nb}{\mathbb{N}}
\newcommand{\Zb}{\mathbb{Z}}
\newcommand{\Gb}{\mathbb{G}}
\newcommand{\Rb}{\mathbb{R}}
\newcommand{\Sb}{\mathbb{S}}
\newcommand{\Tb}{\mathbb{T}}
\newcommand{\Ub}{\mathbb{U}}
\newcommand{\Vb}{\mathbb{V}}

\newcommand{\dr}{\partial}

\newcommand{\Dc}{\mathcalb{D}}
\newcommand{\Ec}{\mathcalb{E}}
\newcommand{\Fc}{\mathcalb{F}}
\newcommand{\Kc}{\mathcalb{K}}

%%\Ar automata
\newcommand{\Ar}{\mathbb{A}}
\newcommand{\Br}{\mathcal{B}}
\newcommand{\Cr}{\mathcal{C}}
\newcommand{\Dr}{\mathcal{D}}
\newcommand{\Er}{\mathcal{E}}
\newcommand{\Fr}{\mathcal{F}}
\newcommand{\Gr}{\mathcal{G}}
\newcommand{\Hr}{\mathcal{H}}
\newcommand{\Ir}{\mathcal{I}}
\newcommand{\Kr}{\mathcal{K}}
\newcommand{\Lr}{\mathcal{L}}
\newcommand{\Mr}{\mathcal{M}}
\newcommand{\Nr}{\mathcal{N}}
\newcommand{\Or}{\mathcal{O}}
\renewcommand{\Pr}{\mathcal{P}}
\newcommand{\Pa}{\mathscr{P}}
\newcommand{\Qr}{\mathcal{Q}}
\newcommand{\Sr}{\mathcal{S}}
\newcommand{\Rr}{\mathcal{R}}
\newcommand{\Tr}{\mathcal{T}}
\newcommand{\Ur}{\mathcal{U}}
\newcommand{\Vr}{\mathcal{V}}
\renewcommand{\Wr}{\mathcal{W}}
\newcommand{\Yr}{\mathcal{Y}}
\newcommand{\Xr}{\mathscr{X}}
\newcommand{\Zr}{\mathscr{Z}}

\newcommand{\Sf}{\mathfrak{S}}

\newcommand{\A}{\mathbf{A}}
\newcommand{\B}{\mathbf{B}}
\def\C{\mathbf{C}}
\newcommand{\D}{\mathbf{D}}
\newcommand{\F}{\mathbf{F}}
\newcommand{\GG}{\mathbf{G}}
\newcommand{\K}{\mathbb{K}}
\newcommand{\M}{\mathbf{M}}
\newcommand{\N}{\mathbf{N}}
\newcommand{\R}{\mathbf{R}}
\renewcommand{\S}{\mathbf{S}}
\newcommand{\V}{\mathbf{V}}
\newcommand{\W}{\mathbf{W}}
\newcommand{\T}{\mathbb{T}}


\newcommand{\Cl}{\mathsf{C}}
\newcommand{\Op}{\mathcal{O}}


%% Catégories
\def\catego#1{\mathsf{#1}}

%\newcommand{\cat}{\catego{Cat}}
\newcommand{\Ab}{\catego{Ab}}
\newcommand{\Gp}{\mathbf{Gp}}
\newcommand{\Cat}{\catego{Cat}}
\newcommand{\Pol}{\catego{Pol}}
\newcommand{\Grpd}{\catego{Grpd}}
\newcommand{\DbCat}{\catego{DbCat}}
\newcommand{\DbGrpd}{\catego{DbGrpd}}
\newcommand{\FldGrpd}{\catego{FoldGrpd}}
\newcommand{\DiPol}{\catego{DiPol}}
\newcommand{\Act}{\catego{Act}}
\newcommand{\Mod}{\catego{Mod}}
\newcommand{\Nat}{\catego{Nat}}
\newcommand{\Set}{\operatorname{Set}}
\newcommand{\Ord}{\mathbf{Ord}}
%\newcommand{\Rep}{\catego{Rep}}
\newcommand{\LinCat}{\catego{LinCat}}
\newcommand{\Vect}{\operatorname{Vect}}
\newcommand{\grVect}{\catego{grVect}}
\newcommand{\Vectg}{\catego{GrVect}}
\newcommand{\Alg}{\catego{Alg}}
\newcommand{\Algg}{\catego{GrAlg}}
\newcommand{\Ch}{\catego{Ch}}
\newcommand{\Chg}{\catego{GrCh}}
\newcommand{\Grph}{\catego{Grph}}
\newcommand{\OSet}{\operatorname{\Omega\text{-}\catego{Set}}}
\newcommand{\OVect}{\operatorname{\Omega\text{-}\catego{Vect}}}

\newcommand{\inOAlg}{\operatorname{\Omega_\infty\text{-}Alg}}
\newcommand{\OAlg}[1]{\Omega\text{-}\catego{Alg}_{#1}}
\newcommand{\Glob}{\catego{Glob}}
\newcommand{\Gpd}{\catego{Gpd}}
\newcommand{\Bimod}{\operatorname{Bimod}}
\newcommand{\Sem}{\operatorname{Sem}}
\newcommand{\Mon}{\operatorname{Mon}}
\newcommand{\OSem}{\operatorname{\Omega\text{-}Sem}}
\newcommand{\OMon}{\operatorname{\Omega\text{-}Mon}}
\newcommand{\Ind}{\catego{Ind}}
\newcommand{\nc}{\newcommand}
\newcommand{\Gph}{\catego{Gph}}

\newcommand{\Ct}[1]{\mathrm{Ct}(#1)}
\newcommand{\Wk}[1]{\mathrm{Wk}(#1)}

\newcommand{\As}{\mathbf{As}}
\newcommand{\Perm}{\mathbf{Perm}}
\newcommand{\Brd}{\mathbf{Braid}}

%%Opérades
\newcommand{\Com}{\mathbf{\mathcal{C}om}}
\newcommand{\Ass}{\mathbf{\mathcal{A}ss}}
%\newcommand{\A}{\mathbf{\mathcal{A}}}
\newcommand{\Dend}{\mathbf{\mathcal{D}end}}
\newcommand{\Dias}{\mathbf{\mathcal{D}ias}}
\newcommand{\nf}{\mathrm{nf}}
\newcommand{\bshuf}{\overline{\shuf}}
\newcommand{\rel}{\mathrm{rel}}
\def\Lie{\mathbf{\mathcal{L}ie}}
\def\Pois{\mathbf{\mathcal{P}ois}}
\def\NsOp{\mathbf{\mathcal{N}s\mathcal{O}p}}

%%Differential algebras
\newcommand{\RBA}[2]{\mathcal{R}\mathcal{B}_{#1}(#2)}
\newcommand{\DRBA}[2]{\mathcal{D}\mathcal{R}\mathcal{B}_{#1}(#2)}
\newcommand{\DA}[2]{\mathcal{D}_{#1}(#2)}

%%Automata
\newcommand{\auto}[1]{\mathbb{A}(#1)}

%% Cohomologie
\newcommand{\fnat}[2]{F_{#1}[#2]}
\newcommand{\ho}{\mathrm{H}}
\newcommand{\poldim}[1]{d_{\mathrm{pol}}(#1)}
\newcommand{\cohdim}[1]{\mathrm{cd}(#1)}
%

%%%%%%%%%%%%%%%%%%%% 
%% Added from the previous version (Zuan)
\DeclareMathOperator{\dom}{dom}
\DeclareMathOperator{\cell}{cell}
\DeclareMathOperator{\Red}{Red}
\DeclareMathOperator{\Nf}{Nf}
\DeclareMathOperator{\red}{Red_m}
\DeclareMathOperator{\supp}{supp}
\newcommand{\lins}[1]{\mathscr{A}(#1)}
\newcommand{\lin}[1]{\mathscr{A}_\Omega(#1)}
%%%%%%%%%%%%%%%%%%%%%%%% Statements
\newtheorem{thm}{Theorem}[section]
%\newtheorem{prop}{Proposition}[section]
\newtheorem{prop}[thm]{Proposition}
%\newtheorem{defn}{Definition}[section]
\newtheorem{lem}[thm]{Lemma}
\newtheorem{cor}[thm]{Corollary}
\newtheorem{prop-def}{Proposition-Definition}[section]
%\newtheorem{conj}[theorem]{Conjecture}
\theoremstyle{definition}
\newtheorem{defn}[thm]{Definition}
\newtheorem{claim}[thm]{Claim}
\newtheorem{exam}[thm]{Example}
%\newtheorem{propprop}{Proposed Proposition}[section]
\newtheorem{Ques}[thm]{Question}
\newtheorem{problem}{Problem}
\newtheorem{assumption}[thm]{Assumption}
\newtheorem{Conj}[thm]{Conjecture}

\newcommand{\tfll}{\xymatrix@1@C=20pt{\ar@3 [r] &}}
\newcommand{\otfl}[1]{\xymatrix@1@C=20pt{\ar@3 [r] ^-*+{#1} &}}


\renewcommand{\labelenumi}{{\rm(\roman{enumi})}}
\renewcommand{\theenumi}{\roman{enumi}}
\renewcommand{\labelenumii}{{\rm(\alph{enumii})}}
\renewcommand{\theenumii}{\alph{enumii}}

%%%%%%%%%%%%%%%%%%%%%%% symbols

\nc{\delete}[1]{{}}
\nc{\mmargin}[1]{}
%\nc{\Alg}{\mathrm{Alg}}
%\nc{\AvO}{\mathrm{AvO}}
%\nc{\AvA}{\mathrm{AvA}}
%\nc{\rmH}{\mathrm{H}}
%\delete{
\nc{\mlabel}[1]{\label{#1}}  % Use this to suppress names
\nc{\mcite}[1]{\cite{#1}}  % Use this to suppress names
\nc{\mref}[1]{\ref{#1}}  % Use this to suppress names
\nc{\mbibitem}[1]{\bibitem{#1}} % Use this to show number
%}

\delete{
	\nc{\mlabel}[1]{\label{#1}  % Use the next two lines to show names
		{\hfill \hspace{1cm}{\bf{{\ }\hfill(#1)}}}}
	\nc{\mcite}[1]{\cite{#1}{{\bf{{\ }(#1)}}}}  % Use this lines to show names
	\nc{\mref}[1]{\ref{#1}{{\bf{{\ }(#1)}}}}  % Use this lines to show names
	\nc{\mbibitem}[1]{\bibitem[\bf #1]{#1}} % Use this to show name
}


%%%%%%%%%%%%%%%%%  new commands
\newcommand{\sha}{{\mbox{\cyr X}}}
\newcommand{\shap}{{\mbox{\cyrs X}}}
\font\cyr=wncyr10 \font\cyrs=wncyr7

%\newcommand{\bcr}{{\mathfrak{B}\mathfrak{C}\mathfrak{R}}}
%\newcommand{\bca}{{\mathfrak{B}\mathfrak{C}\mathfrak{A}}}
%\newcommand{\bvs}{{\mathfrak{B}\mathfrak{V}\mathfrak{S}}}
%\newcommand{\bcd}{{\mathfrak{B}\mathfrak{C}\mathfrak{D}}}
%\newcommand{\brb}{{\mathfrak{B}\mathfrak{R}\mathfrak{B}}}
%\newcommand{\svs}{{\mathfrak{S}\mathfrak{V}\mathfrak{S}}}
%\newcommand{\sca}{{\mathfrak{S}\mathfrak{C}\mathfrak{A}}}
%\newcommand{\scr}{{\mathfrak{S}\mathfrak{C}\mathfrak{R}}}
%\newcommand{\ydh}{{_H^H\mathcal{Y}\mathcal{D}}}
%\newcommand{\dah}{{_H^H\mathcal{D}\mathcal{A}}}
%\newcommand{\ah}{{_H^H\mathcal{A}}}


%\newcommand{\sha}{{\mbox{\cyr X}}}
%\font\cyr=wncyr10 \font\cyrs=wncyr7

\newcommand{\bk}{{\mathbf{k}}}
%    Absolute value notation
\newcommand{\abs}[1]{\lvert#1\rvert}

%    Blank box placeholder for figures (to avoid requiring any
%    particular graphics capabilities for printing this document).
\newcommand{\blankbox}[2]{%
	\parbox{\columnwidth}{\centering
		%    Set fboxsep to 0 so that the actual size of the box will match the
		%    given measurements more closely.
		\setlength{\fboxsep}{0pt}%
		\fbox{\raisebox{0pt}[#2]{\hspace{#1}}}%
	}%
}

\nc{\vep}{\varepsilon}
\nc{\bin}[2]{ (_{\stackrel{\scs{#1}}{\scs{#2}}})}  %binomial coeff
\nc{\binc}[2]{(\!\! \begin{array}{c} \scs{#1}\\
		\scs{#2} \end{array}\!\!)}  %binomial coeff
\nc{\bincc}[2]{  ( {\scs{#1} \atop
		\vspace{-1cm}\scs{#2}} )}  %binomial coeff
\nc{\oline}[1]{\overline{#1}}
\nc{\mapm}[1]{\lfloor\!|{#1}|\!\rfloor}
\nc{\bs}{\bar{S}}
\nc{\la}{\longrightarrow}
\nc{\rar}{\rightarrow}
\nc{\lon }{\,\rightarrow\,}
\nc{\dar}{\downarrow}
\nc{\dap}[1]{\downarrow \rlap{$\scriptstyle{#1}$}}
\nc{\defeq}{\stackrel{\rm def}{=}}
\nc{\dis}[1]{\displaystyle{#1}}
\nc{\dotcup}{\ \displaystyle{\bigcup^\bullet}\ }
\nc{\hcm}{\ \hat{,}\ }
\nc{\hts}{\hat{\otimes}}
\nc{\hcirc}{\hat{\circ}}
\nc{\lleft}{[}
\nc{\lright}{]}
\nc{\curlyl}{\left \{ \begin{array}{c} {} \\ {} \end{array}
	\right .  \!\!\!\!\!\!\!}
\nc{\curlyr}{ \!\!\!\!\!\!\!
	\left . \begin{array}{c} {} \\ {} \end{array}
	\right \} }
\nc{\longmid}{\left | \begin{array}{c} {} \\ {} \end{array}
	\right . \!\!\!\!\!\!\!}
\nc{\ora}[1]{\stackrel{#1}{\rar}}
\nc{\ola}[1]{\stackrel{#1}{\la}}%${\Bbb Z}$
\nc{\scs}[1]{\scriptstyle{#1}} \nc{\mrm}[1]{{\rm #1}}
\nc{\dirlim}{\displaystyle{\lim_{\longrightarrow}}\,}
\nc{\invlim}{\displaystyle{\lim_{\longleftarrow}}\,}
\nc{\dislim}[1]{\displaystyle{\lim_{#1}}} \nc{\colim}{\mrm{colim}}
\nc{\mvp}{\vspace{0.3cm}} \nc{\tk}{^{(k)}} \nc{\tp}{^\prime}
\nc{\ttp}{^{\prime\prime}} \nc{\svp}{\vspace{2cm}}
\nc{\vp}{\vspace{8cm}}
%\nc{\proofend}{$\blacksquare$\vspace{0.3cm}}
\nc{\modg}[1]{\!<\!\!{#1}\!\!>}
%\nc{\intg}[1]{\lceil{#1}\rceil}  %old free int ring
\nc{\intg}[1]{F_C(#1)}
\nc{\lmodg}{\!<\!\!}
\nc{\rmodg}{\!\!>\!}
\nc{\cpi}{\widehat{\Pi}}
%\nc{\sha}{\scs{\mbox{\cyr X}}} %used to be \cyr
%\nc{\sha}{{\mbox{\cyr X}}}  %used to be \cyr
\nc{\ssha}{{\mbox{\cyrs X}}} %sha as product
\nc{\tsha}{{\mbox{\cyrt X}}}
%\nc{\shpr}{}
\nc{\shpr}{\diamond}    %Shuffle product
\nc{\labs}{\mid\!}
\nc{\rabs}{\!\mid}
 \nc{\zhx}{\text{-}}

%\font\cyr=wncyr10
%\font\cyrs=wncyr7
%\font\cyrt=wncyr5

%%%%%%%%%%%%%%%%%%%%% roman fonts, in alphabetic order
\nc{\ad}{\mrm{ad}}
\nc{\ann}{\mrm{ann}}
\nc{\Av}{\mrm{Av}}
\nc{\bim}{\mbox{-}\mathsf{Bimod}}
\nc{\br}{\mrm{bre}}
\nc{\can}{\mrm{can}}
\nc{\Cont}{\mrm{Cont}}
\nc{\rchar}{\mrm{char}}
\nc{\cok}{\mrm{coker}}
\nc{\db}{\mrm{db}}
\nc{\de}{\mrm{dep}}
\nc{\dgg}{\mrm{dgg}}
\nc{\dgp}{\mrm{dgp}}
\nc{\dgx}{\mrm{dgx}}
\nc{\Dif}{\mrm{Diff}}
\nc{\dtf}{{R-{\rm tf}}}
\nc{\dtor}{{R-{\rm tor}}}
\renewcommand{\det}{\mrm{det}}
\nc{\Div}{{\mrm Div}}
\nc{\Diff}{\mrm{DA}}
\nc{\Diffl}{\mathsf{DA}_\lambda}
\nc{\diffo}{{\mathsf{DO}_\lambda}}
\nc{\dl}{{\mathrm{PD}}}
\nc{\dRB}{{\mathrm{\Phi}_\mathsf{DRB}}}
\nc{\udRB}{{\mathrm{\Phi}_\mathsf{uDRB}}}
\nc{\OdRB}{{\mathrm{\Phi}_\mathsf{DRB}^0}}
%\nc{\OdRBp}{{\mathrm{\Phi}_\mathsf{dRB}^0'}}
 \nc{\OudRB}{{\mathrm{\Phi}_\mathsf{uDRB}^0}}
\nc{\inte}{{\mathrm{\Phi}_\mathsf{ID}}}
\nc{\uinte}{{\mathrm{\Phi}_\mathsf{uID}}}
\nc{\Ointe}{{\mathrm{\Phi}_\mathsf{ID}^0}}
 \nc{\Ouinte}{{\mathrm{\Phi}_\mathsf{uID}^0}}


\nc{\alg}{\mathsf{Alg}}
\nc{\Fil}{\mrm{Fil}}
\nc{\Frob}{\mrm{Frob}}
\nc{\Gal}{\mrm{Gal}}
\nc{\GL}{\mrm{GL}}
\nc{\Hoch}{\mrm{Hoch}}
\nc{\hsr}{\mrm{H}}
\nc{\hpol}{\mrm{HP}}
\nc{\im}{\mrm{im}}
\nc{\Id}{\mrm{Id}}
\nc{\ID}{\mrm{ID}}
\nc{\Irr}{\mrm{Irr}}
\nc{\incl}{\mrm{incl}}
\nc{\length}{\mrm{length}}
\nc{\NLSW}{\mrm{NLSW}}
\nc{\Nij}{\mrm{Nij}}
\nc{\mchar}{\rm char}
\nc{\mpart}{\mrm{part}}
\nc{\ql}{{\QQ_\ell}}
\nc{\qp}{{\QQ_p}}
\nc{\rank}{\mrm{rank}}
\nc{\rcot}{\mrm{cot}}
\nc{\rdef}{\mrm{def}}
\nc{\rdiv}{{\rm div}}
\nc{\Rey}{\mrm{Rey}}
\nc{\rtf}{{\rm tf}}
\nc{\rtor}{{\rm tor}}
\nc{\res}{\mrm{res}}
\nc{\SL}{\mrm{SL}}
\nc{\Spec}{\mrm{Spec}}
\nc{\tor}{\mrm{tor}}
\nc{\tr}{\mrm{tr}}
\nc{\wt}{\mrm{wt}}
\def\ot{\otimes}
\def\red{\color{red}}
\nc{\udl}{{\mathrm{udl}}}


%%%%%%%%%%%%%%%%%% bold face

\nc{\bfk}{{\bf k}}
\nc{\bfone}{{\bf 1}}
\nc{\bfzero}{{\bf 0}}
\nc{\detail}{\marginpar{\bf More detail}
	\noindent{\bf Need more detail!}
	\svp}
\nc{\gap}{\marginpar{\bf Incomplete}\noindent{\bf Incomplete!!}
	\svp}
\nc{\FMod}{\mathbf{FMod}}
\nc{\Int}{\mathbf{Int}}
%\nc{\Mon}{\mathbf{Mon}}
%\nc{\proof}{\noindent{\bf Proof: }}
%\nc{\remark}{\noindent{\bf Remark: }}
\nc{\remarks}{\noindent{\bf Remarks: }}
%\nc{\Rep}{\mathbf{Rep}}
%\nc{\Rings}{\mathbf{Rings}}
%\nc{\Sets}{\mathbf{Sets}}




%%%%%%%%%%%%%%%%%%%Bbb fonts
\nc{\BA}{{\mathbb A}}   \nc{\CC}{{\mathbb C}}
\nc{\DD}{{\mathbb D}}   \nc{\EE}{{\mathbb E}}
\nc{\FF}{{\mathbb F}}  
\nc{\HH}{{\mathbb H}}   \nc{\LL}{{\mathbb L}}
\nc{\NN}{{\mathbb N}}   \nc{\PP}{{\mathbb P}}
\nc{\QQ}{{\mathbb Q}}   \nc{\RR}{{\mathbb R}}
\nc{\TT}{{\mathbb T}}   \nc{\VV}{{\mathbb V}}
\nc{\ZZ}{{\mathbb Z}}   \nc{\TP}{\widetilde{P}}


%%%%%%%%%%%%%%%%%%% cal fonts

\nc{\cala}{{\mathcal A}}    \nc{\calc}{{\mathcal C}}
\nc{\cald}{\mathcal{D}}     \nc{\cale}{{\mathcal E}}
\nc{\calf}{{\mathcal F}}    \nc{\calg}{{\mathcal G}}
\nc{\calh}{{\mathcal H}}    \nc{\cali}{{\mathcal I}}
\nc{\call}{{\mathcal L}}    \nc{\calm}{{\mathcal M}}
\nc{\caln}{{\mathcal N}}    \nc{\calo}{{\mathcal O}}
\nc{\calp}{{\mathcal P}}    \nc{\calr}{{\mathcal R}}
\nc{\cals}{{\mathcal S}}    \nc{\calt}{{\Omega}}
\nc{\calv}{{\mathcal V}}    \nc{\calw}{{\mathcal W}}
\nc{\calx}{{\mathcal X}}    \nc{\calu}{{\mathcal U}}
\nc{\caly}{{\mathcal Y}}


\nc{\uOpAlg}{{\mathfrak{uOpAlg}}}

\nc{\OpAlg}{{\mathfrak{OpAlg}}}

\nc{\ComOpAlg}{{\mathfrak{ComOpAlg}}}

\nc{\OpVect}{{\mathfrak{OpVect}}}

\nc{\OpSet}{{\mathfrak{OpSet}}}

\nc{\OpMon}{{\mathfrak{OpMon}}}

\nc{\ComOpMon}{{\mathfrak{ComOpMon}}}

\nc{\OpSem}{{\mathfrak{OpSem}}}

\nc{\ComOpSem}{{\mathfrak{ComOpSem}}}

\nc{\uAlg}{{\mathfrak{uAlg}}}



\nc{\ComAlg}{{\mathfrak{ComAlg}}}







\nc{\ComMon}{{\mathfrak{ComMon}}}



%%新加

\nc{\mtOpSet}{{\Omega\zhx\mathfrak{Set}}}

\nc{\mtOpSem}{{\Omega\zhx\mathfrak{Sem}}}

\nc{\mtOpMon}{{\Omega\zhx\mathfrak{Mon}}}



\nc{\mtOpVect}{{\Omega\zhx\mathfrak{Vect}}}





\nc{\mtOpAlg}{{\Omega\zhx\mathfrak{Alg}}}

\nc{\mtuOpAlg}{{\Omega\zhx\mathfrak{uAlg}}}

\nc{\ComSem}{\mathfrak{ComSem}}
%%%%%%%%%%%%%%%%%%  frak fonts
\nc{\fraka}{{\mathfrak a}}
\nc{\frakb}{\mathfrak{b}}
\nc{\frakg}{{\frak g}}
\nc{\frakl}{{\frak l}}
\nc{\fraks}{{\frak s}}
\nc{\frakB}{{\frak B}}
\nc{\frakm}{{\frak m}}
\nc{\frakM}{{\frak M}}
\nc{\frakp}{{\frak p}}
\nc{\frakW}{{\frak W}}
\nc{\frakX}{{\frak X}}
\nc{\frakS}{{\frak{S}}}
\nc{\frakA}{{\frak A}}
\nc{\frakx}{{\frakx}}

\nc{\frakMstar}{{\mathfrak{M}_\Omega^\star}}
\nc{\frakSstar}{{\mathfrak{S}_\Omega^\star}}

\newcommand{\yn}[1]{\textcolor{blue}{#1 }}
\newcommand{\ynr}[1]{\textcolor{blue}{\underline{Yunnan:}#1 }}
\newcommand{\yh}[1]{\textcolor{purple}{#1 }}
\newcommand{\li}[1]{\textcolor{red}{#1 }}
\nc{\lir}[1]{\textcolor{red}{\underline{Li:}#1 }}
\newcommand{\gd}[1]{\textcolor{green}{#1 }}


\newcommand{\brca}[1]{\left\lfloor #1 \right\rfloor}%%operaors
\newcommand{\brck}[2]{\left\lfloor #1 \right\rfloor_{#2}}%%operaors

\DeclareMathOperator{\Sph}{Sph}

%%Zuan
\newcommand{\tg}{\triangle}
\newcommand{\LO}{$\Omega$}
\DeclareMathOperator{\lm}{lm}
\DeclareMathOperator{\lc}{lc}
\DeclareMathOperator{\lt}{lt}

%% Listes
\newcommand{\notsharp}{%
  \mkern-1mu\not\mathrel{\mkern1mu\Sharp}\mkern1mu
}
\newcommand{\cross}{\pitchfork}
\newcommand{\minus}{\scalebox{0.75}[1.0]{$-$}}
\newcommand{\notminus}{%
  \mkern-1mu\not\mathrel{\mkern1mu\minus}\mkern1mu
}
\renewcommand{\labelitemi}{$-$}
\renewcommand{\theenumi}{\roman{enumi}}%(
\renewcommand{\labelenumi}{\upshape{\textbf{\theenumi)}}}


%% Options xy
\xyoption{rotate,color}

\renewcommand{\objectstyle}{\displaystyle}
%\renewcommand{\labelstyle}{\displaystyle}
\renewcommand{\labelstyle}{\scriptstyle}

% \myvdots from https://tex.stackexchange.com/a/528775/216213
\makeatletter
\DeclareRobustCommand
  \myvdots{\vbox{\baselineskip4\p@ \lineskiplimit\z@
    \hbox{.}\hbox{.}\hbox{.}}}
\makeatother

\newcommand{\sm}{\scriptstyle}

\newcommand{\pushoutcorner}[1][dr]{\save*!/#1+1.2pc/#1:(1,-1)@^{|-}\restore}
\newcommand{\pullbackcorner}[1][dr]{\save*!/#1-1.6pc/#1:(-1,1)@^{|-}\restore}

%%%%pour l'heure
\newcount\hh
\newcount\mm
\mm=\time
\hh=\time
\divide\hh by 60
\divide\mm by 60
\multiply\mm by 60
\mm=-\mm
\advance\mm by \time
\def\hhmm{\number\hh:\ifnum\mm<10{}0\fi\number\mm}
%%%%%%%%%%%%%

%%% Commentaires
\definecolor{vert}{rgb}{0,0.45,0}
\definecolor{rouge}{rgb}{0.89,0.04,0.36}
\definecolor{oorange}{RGB}{148,76,5}
\definecolor{MyGray}{gray}{0.6}
\definecolor{MyRed}{RGB}{212,42,42}
%\newcommand{\todo}[1]{{\color{orange}#1}}
%\newcommand{\todoo}[1]{{\color{oorange}#1}}
%\newcommand{\out}[1]{{\color{olive}#1}}
%\newcommand{\new}[1]{{\color{blue}#1}}
%\newcommand{\New}[1]{{\color{vert}#1}}
%\newcommand{\tempo}[1]{{\color{rouge}#1}}
%\newcommand{\quid}[1]{{\color{MyRed}#1}}
%%%%%%%%%%%%%%%%%%%%%%%%%%%%%%%%%%

%% Auteurs et coordonnées
\newcommand{\auteur}[3]{
\noindent
\begin{minipage}[t]{.45\textwidth}
\begin{flushright}
\textsc{#1} \\
{\footnotesize\textsf{#2}}
\end{flushright} 
\end{minipage}
\qquad
\begin{minipage}[t]{.45\textwidth}
#3
\end{minipage}
}


%%% compteur pour les listes
%\newcounter{tempcounter}
% Utiliser plutôt l'option [resume] offert par le package enumitem !

\begin{document}
\thispagestyle{empty}

\begin{center}

% Title
\begin{doublespace}
\begin{huge}
{\scshape Polygraphic resolutions for operated algebras}
\end{huge}

\vskip+2pt

\bigskip
\hrule height 1.5pt 
\bigskip

\vskip+5pt

% Auteurs
\begin{Large}
{\scshape Zuan Liu - Philippe Malbos}
\end{Large}
\end{doublespace}

%%%
%\vfill

%%%
\vskip+25pt

\begin{small}\begin{minipage}{14cm}
\noindent\textbf{Abstract --}
This paper introduces the structure of operated polygraphs as a categorical model for rewriting in operated algebras, generalizing Gröbner-Shirshov bases with non-monomial termination orders. We provide a combinatorial description of critical branchings of operated polygraphs using the structure of polyautomata that we introduce in this paper. Polyautomata extend linear polygraphs equipped with an operator structure formalized by a pushdown automaton. We show how to construct polygraphic resolutions of free operated algebras from their confluent and terminating presentations. Finally, we apply our constructions to several families of operated algebras, including Rota-Baxter algebras, differential algebras, and differential Rota-Baxter algebras.
\medskip

\smallskip\noindent\textbf{Keywords --} 
Operated algebras, higher-dimensional rewriting,  Gröbner-Shirshov bases, automata, polygraphic resolutions.

\smallskip\noindent\textbf{M.S.C. 2020 --} 
12H05, 16Z10, 18N30, 68Q42, 18G10.
\end{minipage}
\end{small}
\end{center}

\begin{center}
%%% Table des matieres
\begin{small}\begin{minipage}{12cm}
\renewcommand{\contentsname}{}
\setcounter{tocdepth}{1}
\tableofcontents
\end{minipage}
\end{small}
\end{center}

\section{Introduction}
An \emph{operated algebra}, also called \emph{$\Omega$-algebra}, is an associative algebra with linear operators. 
For example, \emph{differential algebras}, introduced by Ritt~\cite{Ritt1950,Ritt1932} in the theory of differential equations, are operated algebras, with an operator satisfying the Leibniz rule.
Differential algebras have been developed in many areas, such as differential Galois theory~\cite{Magid1994,MariusMichael} and differential algebraic group theory~\cite{Kolchin1973}.
\emph{Rota-Baxter algebras} form another important class of operated algebras, introduced by Baxter \cite{Baxter1960}, see also~\cite{Rota1969, Cartier1972, Atkinson1963}. 
They generalize the algebra of continuous functions through integral operators. 
These algebras have found applications across various fields of mathematics and physics, including renormalization in quantum field theory~\cite{Guo10}, the analysis of Volterra integral equations~\cite{GGL22}, Hopf algebras~\cite{ConnesKreimer2000}, Yang-Baxter equations~\cite{Bai2007}, and Rota-Baxter Lie algebras~\cite{GLS21}.
\emph{Differential Rota-Baxter algebras} are equipped with both differential and integral operators. 
Introduced by Guo and Keigher~\cite{GKeigher08}, they describe the relationship between such operators satisfying the first fundamental theorem of calculus. 
The homological study of operated algebras has been the subject of several works.
In particular, homological properties such as Koszul duality, minimal model, deformations, and cohomology theory have been explored in \cite{SongWangZhang,WangZhou24,ChenGuoWangZhou} for Rota-Baxter and differential algebras.
Additionally, these algebras have also been studied from a computational approach, using Gröbner-Shirshov bases theory \cite{BokutChenQiu,GuoSitZhang,LQQZ,GaoGuoSit,LiGuo}.

\bigskip

In this line of work, this paper introduces a new rewriting method to compute resolutions of operated algebras. 
Rewriting is a computational model widely used in algebra, logic and computer science, which consists of calculating in an equational theory by generating the set of equalities by means of a system of \emph{rewriting rules}, and sequentially applying the rules that replace the subterms of a formula by other terms.
In algebra, rewriting theory enables the calculation of linear bases~\cite{Buchberger65,Ufnarovski95} and resolutions of algebraic structures in abelian and categorical settings~\cite{Anick86,Kobayashi90,Kobayashi05,Groves90,Brown92,GuiraudMalbos12advances}. For operated algebras, \emph{Gröbner–Shirshov bases}, GS bases for short, introduced in \cite{Shirshov,Buchberger06},  are frequently used alongside rewriting methods for the standardization of ideal representations in free operated algebras, yielding linear bases and other related constructions \cite{BokutChenQiu,GGR15,GuoSitZhang,QQWZ21}. GS methods require a \emph{monomial order} compatible with the rewriting rules, ensuring the termination of calculations. Nevertheless, finding a monomial order for an operated algebra is more complicated  than in the associative case due to the nesting of operators in monomials~\cite{GGR15,LQQZ,BokutChenQiu}. 

The aim of this paper is twofold: first, it introduces a new termination method in operated algebras without relying on monomial orders, and second, it introduces the notion of \emph{polygraphic resolutions} for operated algebras. 
The linear rewriting strategies introduced in \cite{GuiraudHoffbeckMalbos19} allow one to compute resolutions of an associative algebra by extending its rewriting presentation into a polygraphic resolution generated by confluence diagrams of the higher overlapping of rules.
We extend this approach to operated algebras by introducing the structure of \emph{$\Omega$-polygraphs} as a rewriting setting for operated algebras. It can be used to derive their linear bases and resolutions.
This construction is part of a line of research that consists in constructing polygraphic resolutions of algebraic structures from convergent presentations of them, e.g. for monoids and categories \cite{GuiraudMalbos18,GuiraudMalbos12advances}, associative algebras \cite{GuiraudHoffbeckMalbos19} and operads \cite{MalbosRen23}. 


\bigskip

Now we present  the main results of this paper. 
The first part consists of setting up the polygraphic framework for $\Omega$-algebras.
In Section~\ref{S:HigherOperatedAlgebras}, we  introduce the structure of \emph{higher $\Omega$-algebras}, thereafter called \emph{$\omega$-algebras}, as internal $\omega$-categories in the category of $\Omega$-algebras.
Using the underlying linear and operator structures, Proposition~\ref{P:IsomorphismOmegaAlgebrasBimodules} characterizes $\omega$-algebras in terms of globular bimodules over $\Omega$-algebras.
Following this characterization, Subsection~\ref{SS:Polygraphs for operated algebras} introduces the notion of \emph{$\Omega$-polygraphs} as systems of generators and relations for presentations of $\omega$-algebras. This construction is an extension of the structure of linear polygraphs introduced for associative algebras in \cite{GuiraudHoffbeckMalbos19}.

In Section~\ref{S:OperatedRewriting}, we expand the structure of $\Omega$-$1$-polygraphs to rewriting systems for $\Omega$-algebras. 
We define the rewriting properties such as \emph{termination} and \emph{confluence} on $\Omega$-$1$-polygraphs. We show that this polygraphic approach allows us to define more general termination orders than those used in GS bases theory, see Remark~\ref{E:NoMonomialOrder}.
Proposition~\ref{P:DerivationTerminating} proves termination of an $\Omega$-$1$-polygraph using the method of \emph{derivations} introduced in \cite{Guiraud06jpaa,GuiraudMalbos09}. Proposition~\ref{P:LinearBasisConvergentPolygraphs} states that for \emph{convergent} $\Omega$-$1$-polygraphs, which are both confluent and terminating, the set of \emph{normal forms} is a linear basis for the presented algebra. 
Finally, we classify the \emph{local branching} of $\Omega$-$1$-polygraphs, and state the coherent critical branching lemma in Theorem~\ref{T:CriticalBranchingLemma}. In Subsection~\ref{SS:GSConvergence}, we make explicit the relationship between GS bases of $\Omega$-algebras and convergent \LO-1-polygraphs. 

Due to possible nesting of operators, the critical branchings of the \LO-$1$-polygraph cannot be described in terms of string overlaps, see Remark~\ref{R:DifferencePolygraphAnd OmegaPolygraph}. In order to relate the structures of \LO-$1$-polygraphs and linear $1$-polygraphs, in Section~\ref{S:Polyautomata}, we describe the operator structures using \emph{pushdown automata (PDA)}. 
Explicitly,  we construct a PDA $\Ar_\Omega$ accepting all monomials of the $\Omega$-algebras defined with respect to an indexed set $\Omega$. Given a linear 1-polygraph $\Sigma$ and a PDA $\Ar$, we define a \emph{$1$-polyautomaton} as a pair $(\Sigma, \Ar)$, where both $s(\alpha)$ and $t(\alpha)$ are linear combinations of monomials accepted by the PDA $\Ar$, for every $\alpha \in \Sigma_1$. 
Theorem~\ref{T:MainResultSection4} proves an equivalence between the category of \LO-1-polygraphs and the category of linear 1-polygraphs. Using this correspondence, Theorem~\ref{T:OmegaCBshape} provides an interpretation of the critical branchings in \LO-$1$-polygraphs in terms of the critical branchings of linear $1$-polygraphs.

Section~\ref{S:PolygraphicResolutions} presents the main result of this paper. 
Starting with an $\Omega$-algebra presented by a reduced convergent left-monomial \LO-1-polygraph~$X$, Theorem~\ref{T:Main conclusion} constructs its polygraphic resolution \emph{à la Squier} $\mathrm{Sq}(X)$.
The $0$-generators and $1$-generators of the $\omega$-polygraph $\mathrm{Sq}(X)$ consist of the variables and the rewriting rules defining the algebra, respectively. For $n\geq 2$, the $n$-generators are the sources of the critical $n$-branchings of the polygraph~$X$, as defined in~\eqref{SSS:Higher-critical branching}.  An associative algebra being an $\Omega$-algebra with $\Omega$ empty, our polygraphic resolution generalizes the corresponding construction for associative algebras~\cite{GuiraudHoffbeckMalbos19}.
In Section~\ref{S:ExampleOfResolutionOfOmegaAlgebras}, we apply Theorem~\ref{T:Main conclusion} to construct polygraphic resolutions for Rota–Baxter algebras, differential algebras, and differential Rota–Baxter algebras. To this end, we construct reduced and convergent \LO-1-polygraphs for these $\Omega$-algebras by defining  derivations to prove termination and applying the critical branchings theorem to prove confluence.

\subsubsection*{Organization of the paper}
Section~\ref{S:HigherOperatedAlgebras} recalls the main constructions of $\Omega$-algebras used in this work. 
It also introduces the structure of higher $\Omega$-algebras and \LO-polygraphs.
Section~\ref{S:OperatedRewriting} deals with the rewriting properties of \LO-1-polygraphs, including the construction of derivations for termination proofs, the operated version of the coherent critical branchings lemma, and comparisons with Gröbner–Shirshov bases.
Section~\ref{S:Polyautomata} introduces the structure of polyautomata, encoding both the associative linear structure and the operator structure to establish the equivalence between the categories of \LO-$1$-polygraphs and linear $1$-polygraphs.
This correspondence allows us to establish an interpretation of the critical branchings of $\Omega$-$1$-polygraphs in terms of those of polygraphs of associative algebras.
Section~\ref{S:PolygraphicResolutions} introduces polygraphic resolutions of $\Omega$-algebras. 
We characterize the property of an $\Omega$-$\omega$-polygraph of being a resolution by the existence of a contraction defined on its generators. We then show how to construct such a contraction starting with a reduced convergent $\Omega$-$1$-polygraph.
Finally, Section~\ref{S:ExampleOfResolutionOfOmegaAlgebras} presents constructions of polygraphic resolutions for some classical $\Omega$-algebras.

\subsubsection*{Conventions and notations}
Throughout this paper, we fix a field $\bk$ of characteristic zero and an element $\lambda$ in $\bk$.
Unless stated otherwise, all algebras in this paper are assumed to be associative and unital.
All operators are indexed by a set $\Omega$ and the set of variables is denoted by~$Z$.

\section{Higher operated algebras and operated polygraphs}
\label{S:HigherOperatedAlgebras}
This section introduces the notion of higher $\Omega$-algebras and the structure of \LO-polygraphs as systems of generators and relations for  higher $\Omega$-algebras. 
Operated algebraic structures were defined in \cite{QQWZ21}. 
We first recall the notion of $\Omega$-object in a category and the construction of free $\Omega$-algebras from \cite{BokutChenQiu, Guo09}.

\subsection{Operated algebras}

\subsubsection{Operated objects in a category}
\label{SSS:OmegaObjects}
An \emph{(internal) $\Omega$-object} in a category $\Cr$ is  an object $X$ of  $\Cr$ equipped with a family of maps $\Tr_\tau: X\rightarrow X$, called \emph{operators}, indexed by $\tau\in \Omega$.
A \emph{morphism of $\Omega$-objects from $(X, \Tr_\tau)$ to $(X^\prime, \Tr^\prime_\tau)$} is a morphism $f: X \rightarrow X^\prime$ in $\Cr$ such that $f\circ \Tr_\tau=\Tr^\prime_\tau \circ f$, for every $\tau \in \Omega$. 
The $\Omega$-objects and their morphisms form a category, denoted by $\operatorname{\Omega\text{-}\Cr}$. 
When $\Cr$ is $\bk$-linear, we require the operators to be $\bk$-linear.

In this paper, we will consider $\OSet$, $\OMon$, $\OVect$, and $\OAlg{}{}$ as the categories of $\Omega$-objects in the categories $\Set$, $\Mon$, $\Vect$, and $\Alg{}$ of sets, monoids, vector spaces, and algebras, respectively.
When no confusion is possible, we will write $(X, \Tr)$ or $X$ for short.
Note that if $\Omega$ is empty,  the categories $\Cr$ and $\Omega\text{-}\Cr$ are isomorphic. 

\subsubsection{Free $\Omega$-algebras}
\label{SSS:FreeOmegaMonoid}
The free $\Omega$-monoid $Z^\Omega$ on a set $Z$ is constructed by induction as follows.
We set $Z^\Omega_0\coloneq Z^\ast$, the free monoid on $Z$.
Denote by $\lfloor Z \rfloor_\tau$ the set of formal elements $\lfloor z \rfloor_\tau$, where $\tau \in \Omega$ and $z \in Z$, and define
\[
\lfloor Z \rfloor_\Omega \coloneq \bigsqcup\limits_{\tau \in \Omega} \left\lfloor Z \right\rfloor_\tau.
\]
We set $Z^\Omega_1\coloneq (Z\sqcup   \lfloor Z^\Omega_0 \rfloor_\Omega)^\ast$. 
The inclusion $Z \hookrightarrow Z\sqcup \lfloor Z^\Omega_0 \rfloor_\Omega$ induces an injective morphism
\[
i_{0,1}: Z^\Omega_0 \hookrightarrow Z^\Omega_1.
\]
For $1\leq k \leq n-1$, suppose $Z^\Omega_{k}$ constructed with injective morphisms $i_{k-1,k}: Z^\Omega_{k-1} \hookrightarrow Z^\Omega_{k}$.
We then set
\[
Z^\Omega_{n} \coloneq (Z\sqcup \lfloor Z^\Omega_{n-1} \rfloor_\Omega)^\ast.
\]
The inclusion $Z\sqcup \lfloor Z^\Omega_{n-2} \rfloor_\Omega\hookrightarrow Z\sqcup \lfloor Z^\Omega_{n-1} \rfloor_\Omega$  also induces an injective morphism
\[
i_{n-1,n}: Z^\Omega_{n-1} \hookrightarrow Z^\Omega_n.
\]
By construction, we have $Z^\Omega_i\subset Z^\Omega_{i+1}$ for $i\geq 0$, and we define $Z^\Omega\coloneq\varinjlim Z^\Omega_{n}$. The maps sending $u\in Z^\Omega_{n}$ to $\lfloor u \rfloor_{\tau} \in Z^\Omega_{n+1}$ induce a family of operators $\lfloor~\rfloor_\tau$ on $Z^\Omega$ indexed by $\tau \in \Omega$. 
As a result, $(Z^\Omega, \brck{~}{\tau})$ is the \emph{free $\Omega$-monoid} on~$Z$. For $u\in Z^\Omega$, we define $\lfloor u\rfloor_0 \coloneq u$, where $0$ is a new element not included in $\Omega$.

An \emph{$\Omega$-monomial} $u$ is a non-identity  element of $Z^\Omega$, which can be uniquely written as $u = u_1u_2\cdots u_n$, where $u_i\in Z \sqcup \brck{Z^\Omega}{\tau}$ and $n$ is the \emph{breadth} of $u$, denoted by $\operatorname{bre}(u) = n$.
The \emph{depth} of $u$ is defined by
\[
\operatorname{dep}(u) \coloneq \min \left\{n \mid u \in Z^\Omega_n \right\}.
\]

We denote by $(\lin{Z}, \brck{~}{\tau})$ the \emph{free $\Omega$-algebra} on $Z$, defined as the $\bk$-linear span of $\Omega$-monomials in $Z^\Omega$, and whose operators $\lfloor ~ \rfloor_\tau$ are induced by linearity.
This defines a functor $ \lin{-}: \catego{Set} \to \OAlg{} $, left adjoint to the forgetful functor $ \calu: \OAlg{} \to \catego{Set} $.

\subsubsection{Examples}
\label{SSS:ExamplesFreeOmegaAlgebras}
We present examples of free 
$\Omega$-algebras, which will be further studied in Section~\ref{S:ExampleOfResolutionOfOmegaAlgebras}.
\begin{enumerate}
\item The \emph{free differential algebra of weight $\lambda$} on a set $Z$ is the free $\Omega$-algebra, denoted by $\DA{\lambda}{Z}$, equipped with an operator $D$ such that
\begin{eqn}{equation}
\label{E:DifferentialRelation}
D(ab) = D(a)b + aD(b) + \lambda D(a)(b) \quad \text{and}\quad  D(1)=0, \quad \text{ for all } a, b \in \DA{\lambda}{Z}.
\end{eqn}
\item The \emph{free Rota-Baxter algebra of weight $\lambda$} on a set $Z$ is the free $\Omega$-algebra, denoted by $\RBA{\lambda}{Z}$, equipped with an operator $P$ such that 
\begin{eqn}{equation}
\label{E:RotaBaxterRelation}
P(a)P(b)= P(aP(b)) + P(P(a)b) + \lambda P(ab), \quad \text{ for all $a,b\in \RBA{\lambda}{Z}$.}
\end{eqn}
\item
The \emph{free differential Rota-Baxter algebra of weight $\lambda$} on a set $	Z$ is the free $\Omega$-algebra, denoted by  $\DRBA{\lambda}{Z}$,  equipped with two  operators  $D$ and $P$  satisfying \eqref{E:DifferentialRelation}, \eqref{E:RotaBaxterRelation} and the relation
\[
D(P(a))=a, \quad \text{ for all } a \in \DRBA{\lambda}{Z}.
\]
\end{enumerate}
We will use $D$ and $P$ to denote the \emph{differential operator} and \emph{Rota-Baxter operator}, respectively, instead of $\brck{~}{D}$ and $\brck{~}{P}$, following the conventions in \cite{QQWZ21, GGR15, BokutChenQiu}.

\subsubsection{Free operated bimodules}
\label{SSS:FreeOmegaBimodule}
Let $(A, \Tr)$ be an $\Omega$-algebra.
An \emph{$(A, \Tr)$-bimodule} is an $\Omega$-vector space $(M, \Tr^M)$, where $M$ is an $A$-bimodule.
A \emph{morphism of $(A, \Tr)$-bimodules} from $(M, \Tr^{M})$ to $(M', \Tr^{M'})$ is a morphism $f: M \to M'$ of both $\Omega$-vector spaces and $A$-bimodules.
We denote by $\Omega\text{-}\Bimod(A)$ the category of $(A, \Tr)$-bimodules and their morphisms.

The \emph{free $(A, \Tr)$-bimodule} on a set $Z$, denoted by $\mathcal{M}_{\Omega}(Z)$, is constructed as follows. 
We set $\mathcal{M}_0(Z) \coloneq A \otimes Z \otimes A$, and for each $n \geq 0$,
\[
\mathcal{M}_{n+1}(Z) \: \coloneq \: A\otimes\brck{\mathcal{M}_{n}(Z)}{\Omega \sqcup \{0\}}\otimes A.
\]
By construction, we have $\mathcal{M}_{i}(Z) \subset \mathcal{M}_{i+1}(Z)$ for $i \geq 0$, and we define $\mathcal{M}_{\omega}(Z) \coloneq \varinjlim \mathcal{M}_{n}(Z)$. The maps sending $m \in \mathcal{M}_{n}(Z)$ to $\lfloor m \rfloor_{\tau} \in \mathcal{M}_{n+1}(Z)$ induce a family of operators $\lfloor~\rfloor_\tau$ on $\mathcal{M}_{\omega}(Z)$ indexed by $\tau \in \Omega$.
Finally, we define $\mathcal{M}_{\Omega}(Z)$ as the linear span of the elements in $\mathcal{M}_{\omega}(Z)$.

\subsubsection{Operated contexts}
\label{SSS:OneHole}
Let $\square$ be a symbol not in $Z$.
The $\Omega$-monoid of \emph{(one hole) $\Omega$-contexts} is the subset of $(Z\sqcup\square)^\Omega$, denoted by $Z^\Omega[\square]$, consisting of $\Omega$-monomials with $\square$ occurring only once. 

For $q\in  Z^\Omega[\square]$ and $u\in Z^\Omega$, we define $q|_{u}\in Z^\Omega$ as the element obtained by replacing the symbol~$\square$ in $q$ with $u$. 
The \emph{composition} of contexts $p$ and $q$ in $Z^\Omega[\square]$ is defined by $q\circ p\coloneq p|_q$.
For $a=\sum_i\lambda_iu_i\in \lin{Z}$, with~$\lambda_i\in\bk$ and $u_i\in Z^\Omega$, we define
\[
q|_a\coloneq\sum_i\lambda_iq|_{u_i}.
\]
Similarly, we extend this structure by linearity into the bimodule of \emph{(one hole) $\Omega$-contexts}, denoted by~$\mathcal{M}_{\omega}(Z)[\square]$.
Any element of $\mathcal{M}_{\Omega}(Z)$ can be written as a linear combination
\[
D|_x \: = \: (C_1\circ C_2 \circ \cdots \circ C_n)|_x,
\]
where $x\in Z$ and $C_k=\left\lfloor a_k\otimes\square \otimes b_k \right\rfloor_{\tau_k}$, with $a_k,b_k\in A$, $\tau_k\in\Omega\sqcup\{0\}$ and  $1\leq k \leq n$. 

\subsubsection{Operated ideals}
An \emph{$\Omega$-ideal} of an $\Omega$-algebra $(A,\Tr)$ is an ideal of the algebra $A$ closed under the action of its operators.
We denote by $I_{\Omega}(S)$ the $\Omega$-ideal of $\lin{Z}$ generated by a subset $S$ of $\lin{Z}$, and by $\lin{Z}/I_{\Omega}(S)$ the quotient $\Omega$-algebra.

If $\Omega$ is empty, $\lin{Z}$ is the free associative algebra, and 
$I_{\Omega}(S)$ is made of all the linear combinations of monomials $\left.p\right|_{s}$,
where $s \in S$ and $p=u\square v \in Z^\Omega[\square]$ with $u, v \in Z^\Omega$.  
If $\Omega$ is not empty, an element in~$I_{\Omega}(S)$ is a linear combination 
\[
q|_s \: = \: (p_1\circ p_2 \circ \cdots \circ p_n)|_s,
\]
where $p_k = \left\lfloor u_k \square  v_k\right\rfloor_{\tau_k}$ with  $u_k, v_k \in Z^\Omega$ and $\tau_k \in \Omega\sqcup\{0\}$, for $1\leq k \leq n$.

\subsection{Higher operated algebras}
\label{SS:HigherOperatedAlgebras}
This subsection introduces  higher $\Omega$-algebras as a generalization of higher associative algebras introduced in \cite{GuiraudHoffbeckMalbos19}. 
We also make explicit their structure in terms of operated bimodules.

\subsubsection{Globular objects}
An \emph{(internal) globular object of a category~$\Cr$} is a sequence $X\coloneq(X_k)_{k\geq 0}$ of objects in $\mathcal{C}$, equipped with the following families of morphisms in~$\Cr$ 
\[
(s_k:X_{k+1}\fl X_k)_{k\geq 0},
\qquad
(t_k:X_{k+1}\fl X_k)_{k\geq 0},
\quad\text{and}\quad
(i_k:X_{k-1}\fl X_k)_{k\geq 1},
\]
satisfying the following globular and identity relations
\begin{eqn}{equation}
\label{E:GlobularRelations}
s_ks_{k+1} = s_kt_{k+1}, 
\qquad 
t_ks_{k+1} = t_kt_{k+1}
\qquad\text{and}\qquad
 s_ki_{k+1} = t_ki_{k+1} = \id_{X_k},
\end{eqn}
for every $k\geq 0$.
The elements of~$X_k$ are \emph{$k$-cells of~$X$}.
For $x\in X_k$ with $k\geq 1$, the $(k-1)$-cells~$s_{k-1}(x)$ and~$t_{k-1}(x)$ are the \emph{source} and \emph{target} of $x$, also denoted by $s(x)$ and $t(x)$.
A \emph{morphism of globular objects}~$f : X \fl Y$ is a family $(f_k : X_k \fl Y_k)_{k\geq 0}$ of morphisms in $\Cr$ that commutes with morphisms $s_k$,~$t_k$ and $i_k$.
We denote by~$\Glob(\Cr)$ the category of globular objects  of~$\Cr$ and their morphisms. 
For~$n\geq 0$, we denote by $\Glob_n(\Cr)$ the full subcategory of $\Glob(\Cr)$ consisting of globular objects indexed up to $n$, and called \emph{$n$-globular objects}. 

For a globular object $X$ and $0\leq m\leq k \leq n$, the \emph{$k$-source}, \emph{$k$-target}, and \emph{$k$-identity maps} are defined by iterated composition
\[
s_{n-1}\ldots s_k:X_n \fl X_k,
\quad 
t_{n-1}\ldots t_k:X_n \fl X_k
\quad \text{ and } \quad
i_{m+1}\ldots i_k:X_m \fl X_k,
\]
also respectively denoted by $s_k$, $t_k$ and $i_k$ for short.
We will write~$x$ instead of $i_k(x)$. 
For~$k \geq 0$, we denote by $X \star_k X$ the pullback of the morphisms $s_k,t_k : X \fl X_k$.
For $n\geq 1$, two $n$-cells $a$ and $b$ are  \emph{parallel} if $s(a)=s(b)$ and $t(a)=t(b)$.   
An \emph{$n$-sphere} of $X$ is a pair of parallel $n$-cells.

\subsubsection{Higher categories}
For $n \geq 0$,  an \emph{(internal) $n$-category} in $\Cr$ consists of an $n$-globular object $X$ of $\Cr$ and a \emph{$k$-composition} map $X_n \star_k X_n \to X_n$ for all $k < n$, along with $k$-source and $k$-target maps
\[
\xymatrix @!C @C=3em @W=2em {
X_k
& X_n,
\ar@<-0.8ex> [l] _-{s_k}
\ar@<0.8ex> [l] ^-{t_k}
}
\]
such that the $2$-globular object
\[
\xymatrix @!C @C=3em @W=2em {
X_j
& X_k
\ar@<-0.8ex> [l] _-{s_j}
\ar@<0.8ex> [l] ^-{t_j}
& X_l
\ar@<-0.8ex> [l] _-{s_k}
\ar@<0.8ex> [l] ^-{t_k}
}
\]
is a 2-category in $\Cr$ for all $j < k < l$.
We denote by $\Cat_n(\Cr)$ the category of $n$-categories in $\Cr$ and by~$\Cat_\omega(\Cr)$ the category of \emph{$\omega$-categories}, defined as the limit of the  sequence $\big(\Cat_i(\Cr) \leftarrow \Cat_{i+1}(\Cr)\big)_{i\geq 0}$ of forgetful functors.

\subsubsection{Higher $\Omega$-algebras}
For $n \in \mathbb{N} \sqcup \{\omega\}$, we denote by $\OAlg{n}$ the category $\Cat_n(\OAlg{})$, whose objects are called \emph{$\Omega$-$n$-algebras} or \emph{n-algebras} for short. 

Given an $n$-algebra $(A,\Tr_\tau)$, for each $\tau \in \Omega$, there is a corresponding operator $\Tr_{\tau,k}$ on $A_k$.
The source, target, and identity maps being morphisms of $\Omega$-algebras, the following commuting relations
\begin{eqn}{equation}
\label{E:OperatorsAndsti}
s(\Tr_{\tau,k}(a)) =\Tr_{\tau,k-1}(s(a)),
\quad
t(\Tr_{\tau,k}(a)) =\Tr_{\tau,k-1}(t(a))
\quad \text{ and } \quad
i(\Tr_{\tau,k}(a)) =\Tr_{\tau,k+1}(i(a)),
\end{eqn}
hold for all $a\in A_k$ and $\tau\in\Omega$. The third relation can also be written as $1_{\Tr_{\tau,k}(a)} = \Tr_{\tau,k+1}(1_a)$.
In the sequel, we will omit the $\tau$ notation in such formulas.
From this structure, we deduce the following $\Omega$-algebraic properties.

\begin{proposition}
\label{L:PropertiesOmegaAlgebras}
For an $\omega$-algebra $(A, \Tr)$, the following conditions hold
\begin{enumerate}
\item\label{1} \label{I:nVect-1} 
For all $0\leq k <n$ and $(a,b)$ in $A\star_k A$, we have $a \star_k b = a - t_k (a) + b$,
\item For all $n \geq 1$, every $n$-cell $a$ of $A$ is invertible with inverse $a^- = s(a) - a + t(a)$,
\item For all $0\leq k <n$ and $(a,b)$ in $A\star_k A$, we have  $\Tr_n(a^{-})=\Tr_n(a)^{-}$ and $\Tr_n(a \star_k b )=\Tr_n(a) \star_k \Tr_n(b)$.
\end{enumerate}
\end{proposition}
\begin{proof}
The proofs of {\bf i)} and {\bf ii)} are given in 
{\cite[Prop. 1.2.3]{GuiraudHoffbeckMalbos19}} using the underlying associative structure. 
Property {\bf iii)} follows directly from \eqref{E:OperatorsAndsti}.
\end{proof}

\subsubsection{Globular operated bimodules}
We denote by $\Glob(\Omega\text{-}\Bimod)$  the category of globular operated bimodules  whose objects are pairs $\big((A, \Tr), (M, \Tr')\big)$, where $(A, \Tr)$ is an $\Omega$-algebra and $(M, \Tr')$ is a globular $(A, \Tr)$-bimodule. 
Let $\Glob_{\text{sub}}(\Omega\text{-}\Bimod)$ denote its \emph{full subcategory}, consisting of those pairs for which $(M_0, \Tr'_0)$ is isomorphic to $(A, \Tr)$ with its canonical $(A, \Tr)$-bimodule structure, and satisfying the following relation, for all $a,b$ in $M_n$,
\begin{eqn}{equation}
\label{E:LinearExchangeRelation}
a s_0(b) + t_0(a) b - t_0(a) s_0(b) \: = \: s_0(a) b + a t_0(b) - s_0(a) t_0(b).
\end{eqn}

The following result extends the one known for  associative algebras {\cite[Thm. 1.3.3]{GuiraudHoffbeckMalbos19}}.

\begin{proposition}
\label{P:IsomorphismOmegaAlgebrasBimodules}
The categories $\OAlg{\omega}$ and $\Glob_{\text{sub}}(\Omega\text{-}\Bimod)$ are isomorphic.
\end{proposition}

\subsection{Polygraphs for operated algebras}
\label{SS:Polygraphs for operated algebras}
The structure of polygraphs was introduced by Street~\cite{Street76} and Burroni~\cite{Burroni93} as systems of generators and relations for strict $\omega$-categories. We refer to \cite{Polybook2025} for a comprehensive presentation of the structure of polygraphs in rewriting theory and higher category theory. 
More recently, polygraphs have been introduced for presentations of higher associative algebras~\cite{GuiraudHoffbeckMalbos19} and shuffle operads~\cite{MalbosRen23}.
In this subsection, we develop the structure of polygraphs for presentations of higher $\Omega$-algebras.

\subsubsection{Extended higher $\Omega$-algebras}
For $n \geq 0$, the category $\OAlg{n}^+$ of \emph{extended $n$-algebras} is defined as the pullback of forgetful functors
\[
\xymatrix @=1.75em{
\OAlg{n}^+
	\pullbackcorner
\ar[r]
\ar[d]
&
\Glob_{n+1}(\Set)
\ar[d]^-{\calu_n}
\\
\OAlg{n}
\ar[r] _-{{\calv_n^\Omega}}
&
\Glob_{n}(\Set)
}
\]
where the functor $\calu_n$ forgets the structure of $(n+1)$-cells and the functor $\calv_n^\Omega$ forgets the $\Omega$-algebraic structure. 
Explicitly, an object in~$\OAlg{n}^+$ is a pair~$\big((A,\Tr),X\big)$, where~$(A,\Tr)$ is an $n$-algebra and~$X$ is a \emph{cellular extension} of~$A_n$, that is a set $X$ equipped with two maps
\[
\xymatrix @=1.5em {
A_n 
& X
\ar@<-0.5ex> [l] _-{s}
\ar@<0.5ex> [l] ^-{t}
}
\]
such that, for every $x \in X$, the pair $(s(x), t(x))$ is an $n$-sphere of $A$. 
A morphism from $\big((A,\Tr), X\big)$ to $\big((B,\Tr'), Y\big)$ in~$\OAlg{n}^+$ consists of a pair $(f, g)$, where $f: (A,\Tr) \to (B,\Tr')$ is a morphism of $n$-algebras, and $g: X \to Y$ is a map that commutes with the source and target maps.

\subsubsection{Free higher $\Omega$-algebras}
For $n\geq 1$, we define $A[X]$ the \emph{free $n$-algebra} on an extended $(n-1)$-algebra $\big((A,\Tr),X\big)$ as follows. First, we construct an $(A_0,\Tr_0)$-bimodule $(M, \Tr')$, by setting
\[
M \: = \: \mathcal{M}_{\Omega}(X)\oplus A_{n-1},
\]
where $\mathcal{M}_{\Omega}(X)$ is the free $(A_0, \Tr_0)$-bimodule on $X$  as defined in \eqref{SSS:FreeOmegaBimodule}, and
\[
\Tr'_\tau(m+c) \: \coloneq \: \left\lfloor m\right \rfloor_\tau+\Tr_{\tau,n-1}(c),
\]
for all $\tau\in\Omega$, $m\in \mathcal{M}_{\Omega}(X)$ and $c\in A_{n-1}$. Following \eqref{SSS:OneHole}, the elements of $(M, \Tr')$ are the linear combinations of  $D|_x$ and $(n-1)$-cells $c$ of $(A, \Tr)$, where  $x \in X$ and $D\in \mathcal{M}_{\omega}(Z)[\square]$.
The source, target and identity
maps $s, t$ and $i$ in $(M, \Tr')$  are defined by 
\begin{eqn}{equation}
\label{E:mapsBimodules}
s(\left.D \right|_{x})  = \left.D \right|_{s(x)}, \quad t(\left.D \right|_{x})  = \left.D \right|_{t(x)}  \quad 
\text{and} \quad
s(c)=t(c) =i(c) =c.
\end{eqn}
We define the $(A_0, \Tr_0)$-bimodule $A[X]_n$ as the quotient of $(M, \Tr')$ by the $(A_0, \Tr_0)$-bimodule ideal generated by elements
\[
\big(a s_0(b)+t_0(a) b-t_0(a) s_0(b)\big)- \big(s_0(a) b+a t_0(b)-s_0(a) t_0(b)\big),
\]
for all $a, b$ in $\mathcal{M}_{\Omega}(X)$. 
The source, target and identity maps defined in \eqref{E:mapsBimodules} are compatible with this quotient. Hence, by Proposition~\ref{P:IsomorphismOmegaAlgebrasBimodules}, the $(A_0, \Tr_0)$-bimodule $A[X]_n$ extends $(A, \Tr)$ uniquely into the free $n$-algebra $A[X]$.

\subsubsection{Operated polygraphs}
Let us define the category $\Pol_n(\OAlg{})$ of \emph{\LO-$n$-polygraphs} by induction on $n$.
For $n = 0$, we define $\Pol_0(\OAlg{})$ as the category of sets. The \emph{free $0$-algebra functor} maps a set $Z$ to the free $\Omega$-algebra $\lin{Z}$.
For $n\geq1$,
assuming that the category $\Pol_{n-1}(\OAlg{})$ and the free $(n-1)$-algebra functor $\mathscr{A}_{n-1} : \Pol_{n-1}(\OAlg{}) \fl \OAlg{n-1}$ have been constructed, we define the category $\Pol_n(\OAlg{})$ as the pullback
\begin{eqn}{equation}
\label{E:PolygraphFunctor}
\vcenter{\xymatrix @=2.5em{
\Pol_n(\OAlg{})
	\pullbackcorner
\ar[r]
\ar[d] _-{\mathcal{V}_{n-1}}
&
\OAlg{n-1}^+
\ar[d]^{\calw_{n-1}}
\\
\Pol_{n-1}(\OAlg{})
\ar[r]_(.57){\mathscr{A}_{n-1}}
&
\OAlg{n-1}
}}
\end{eqn}
of the functor $\mathscr{A}_{n-1}$ and the functor ${\calw_{n-1}}$ that forgets the cellular extension.
The free $n$-algebra functor is defined as the composition
\[
\mathscr{A}_n : \Pol_n(\OAlg{}) \longrightarrow \OAlg{n-1}^+ \longrightarrow \OAlg{n},
\]
of the functor $\Pol_n(\OAlg{}) \to \OAlg{n-1}^+$ from \eqref{E:PolygraphFunctor}, followed by the functor mapping $(A, X)$ to $A[X]$.
The category $\Pol_\omega(\OAlg{})$ of \emph{$\Omega$-$\omega$-polygraphs} is defined as the limit of the sequence $(\mathcal{V}_n : \Pol_n(\OAlg{})\fl \Pol_{n-1}(\OAlg{}))_{n>0}$ of forgetful functors $\mathcal{V}_{n-1}$ defined by the pullback \eqref{E:PolygraphFunctor}.

Expanding this definition, an \LO-$n$-polygraph is a sequence $X = (Z, X_1, \ldots, X_n)$ made of a set $Z$ of~\emph{$0$-generators}, a cellular extension $X_1$ of $\lin{Z}$, and, for each $1 \leq k \leq n-1$, a cellular extension $X_{k+1}$ of the free $k$-algebra $\mathscr{A}_k(Z, X_1, \ldots, X_k)$, whose elements are called \emph{$(k+1)$-generators} of $X$.
We will denote the free $n$-algebra on $X$ by $\lin{X}$.

\subsubsection{Higher $\Omega$-monomials}
\label{SSS: HigherOmegaMonomials}
Let $X$ be an \LO-$\omega$-polygraph.
Every $0$-cell $a$ of $\lin{X}$ can be uniquely written as a linear combination
\[
a = \sum_{i=1}^p \lambda_i u_i
\]
of distinct $\Omega$-monomials $u_i$ with  nonzero scalars $\lambda_i$. 
The \emph{support} of $a$ is the set $\supp(a) \coloneq \{u_1,\dots,u_p\}$. 

For $n\geq1$, an \emph{$\Omega$-$n$-monomial}, or \emph{$n$-monomial} for short,
$\left.q\right|_{\alpha}$ of $\lin{X}$ is an $n$-cell of $\lin{X}$ where~$\alpha$ is an $n$-generator of $X$ and~$q\in Z^\Omega[\square]$. Following the construction of the free $n$-algebra, every $n$-cell $a$ of $\lin{X}$ can be written as a linear combination
\begin{eqn}{equation}
\label{E:OmegaMonomialDecomposition}
a = \sum_{i=1}^p \lambda_i v_i + 1_c
\end{eqn}
of distinct $\Omega$-$n$-monomials  $v_i$ and an $(n-1)$-cell $c$. This decomposition is unique up to the exchange relation \eqref{E:LinearExchangeRelation}.
The \emph{size of~$a$} is the minimum number of $\Omega$-$n$-monomials required to write~$a$ as in \eqref{E:OmegaMonomialDecomposition}.

\subsubsection{Linear polygraphs}
\label{R:OmegaEmpty}
\LO-polygraphs are natural generalizations of \emph{linear polygraphs} introduced in \cite{GuiraudHoffbeckMalbos19}.
Indeed, when $\Omega$ is empty, an \LO-$\omega$-polygraph is a linear $\omega$-polygraph, presenting an $\omega$-associative algebra.
We will denote by $\Pol_n(\Alg)$ the category of linear $n$-polygraphs and their morphisms. 

\section{Rewriting in operated algebras}
\label{S:OperatedRewriting}
This section presents the main rewriting properties of \LO-1-polygraphs and the coherent critical branching theorem in the operated setting. We compare the shape of critical branchings generated by \LO-1-polygraphs with those of linear 1-polygraphs, with further exploration in Section~\ref{S:Polyautomata}. Additionally, we relate convergent \LO-1-polygraphs to Gröbner–Shirshov theory for $\Omega$-algebras.

\subsection{Polygraphic presentations of operated algebras}
In this subsection, we present the rewriting properties of \LO-$1$-polygraphs. Inspired by~\cite{GuiraudMalbos09}, we introduce the notion of derivations to prove the termination of \LO-$1$-polygraphs.

\subsubsection{Operated polygraphic rewriting}
\label{SSS:1-polygraph}
 An \LO-$1$-polygraph is a pair $X=(Z,X_1)$, where $Z$ is a set and $X_1$ is a cellular extension
\[
\xymatrix @=1.5em {
\lin{Z} 
& X_1.
\ar@<-0.5ex> [l] _-{s}
\ar@<0.5ex> [l] ^-{t}
}
\]
A $1$-cell $f$ in the free $1$-algebra $\lin{X}$ can be written as
\[
f = \sum_{i=1}^p \lambda_i \left.q_i\right|_{\alpha_i} + 1_c,
\]
where $\alpha_i \in X_1$, $q_i \in Z^\Omega[\square]$, $c \in \lin{Z}$ and $\left.q_i\right|_{\alpha_i}$ are $1$-monomials.
%
An \LO-$1$-polygraph $X$ is  \emph{left-monomial} if, for every $\alpha \in X_1$, the source $s(\alpha)$ is an $\Omega$-monomial in~$\lin{Z}$ that does not belong to $\operatorname{Supp}\big(t(\alpha)\big)$. A \emph{rewriting step} in $X$ is a 1-cell $\lambda f + 1_c$ in $\lin{X}$, where $\lambda \neq 0$, $f$ is of size 1, and $s(f) \notin \operatorname{Supp}(c)$. A 1-cell in $\lin{X}$ is \emph{positive} if it is a (possibly empty) 0-composition of rewriting steps in $X$.
A positive cell is \emph{finite} if it admits such a finite composition, and \emph{infinite} otherwise. 

\subsubsection{Presentations of $\Omega$-algebras}
For $\alpha\in X_1$, we set $\partial(\alpha) \coloneq s(\alpha) - t(\alpha)$. 
The \emph{ideal $I_\Omega(X)$ of $X$} is the $\Omega$-ideal of $\lin{Z}$ generated by the set of $0$-cells
\[
\partial(X_1) \: \coloneq  \: \{ \partial(\alpha) \mid \alpha \in X_1 \}.
\]
It consists of linear combinations of the form $\left.q_i\right|_{\partial(\alpha_i)}$,
where $\alpha_i \in X_1$ and $q_i \in Z^\Omega[\square]$.
The \emph{$\Omega$-algebra presented by $X$} is the quotient $\Omega$-algebra
\[
\overline{X} \: \coloneq \: \lin{Z}/I_\Omega(X).
\]
A \emph{presentation} of an $\Omega$-algebra $A$ is an $\Omega$-$1$-polygraph $X$ such that $A$ is isomorphic to $\overline{X}$.
Two \LO-$1$-polygraphs are \emph{Tietze-equivalent} if they present isomorphic algebras. 
Note that any \LO-1-polygraph is Tietze equivalent to a left-monomial one. 
Hence, from now on, all \LO-$1$-polygraphs are left-monomial.

\subsubsection{Termination and monomial orders}
\label{SSS:Termination}
An \LO-$1$-polygraph $X$ is \emph{terminating} if the 1-algebra $\lin{X}$ does not contain any infinite positive cell. A \emph{monomial order} on $Z^\Omega$ is a well-founded total order $\prec$ on $Z^\Omega$ stable under products and operators, meaning that
\[
u\prec v \Rightarrow q|_u\prec q|_v\quad\text{for all }u,v\in Z^\Omega\text{ and } q\in Z^\Omega[\square].
\]
We say that $\prec$ is \emph{compatible} with $X_1$ if $v \prec s(\alpha)$ holds for all $\alpha\in X_1$ with $v$ in $\operatorname{Supp}(t(\alpha))$.
If there exists a monomial order $\prec$ on $Z^\Omega$ compatible with $X_1$, then $X$ is terminating.

A \emph{rewriting order} on $X$ is a relation $\preccurlyeq_X$ on $\lin{Z}$ such that $v \preccurlyeq_X u$ if there exists a positive $1$-cell  $u\fl b$ for $u, v \in Z^\Omega$ and $v\in\operatorname{Supp}(b)$, or $b \preccurlyeq_X a$ if, for any $v \in \operatorname{Supp}(b)\setminus\operatorname{Supp}(a)$, there exists a $u \in \operatorname{Supp}(a)\setminus\operatorname{Supp}(b)$ such that $v \preccurlyeq_X u$. 
Define $v \prec_X u$ if $v \preccurlyeq_X u$ but $v \ne u$.
Note that $\prec_X$ is well-founded if $X$ terminates.

\subsubsection{Normal forms}
For an \LO-$1$-polygraph $X$, a $0$-cell $a$ of $\lin{X}$ is a \emph{normal form} if there is no rewriting step with source $a$, else it is \emph{reducible}. 
A \emph{normal form} of a $0$-cell $a$, denoted by $\Nf(a,X)$ or~$\rep{a}$ for short, is a normal form such that there exists a positive $1$-cell from $a$ to $\Nf(a,X)$. 
We denote by $\Nf(X)$ the set of normal forms of $\lin{X}$.

\subsubsection{Derivations}
A \emph{derivation} of $\lin{Z}$ with values in an $\lin{Z}$-bimodule $N$ is a linear map $d: \lin{Z} \to N$ satisfying the following conditions:
\begin{eqn}{equation}
\label{E:Derivation}
d(ab)=d(a)  \cdot b+ a \cdot d(b), \quad d(1)=0, \quad\text{ and } \quad d(\brck{a}{\tau})=\brck{d(a)}{\tau},
\end{eqn}
for all $a, b \in \lin{Z}$.  
A well-founded partial order $\prec$ on $N$ is \emph{monotone} if the following hold 
\[
m_1+n+m_2\succ m_1+n'+m_2,\quad u\cdot n\cdot v\succ u\cdot n'\cdot v \quad\text{ and }\quad  \brck{n}{\tau}\succ\brck{n'}{\tau},
\]
for all $m_1, m_2, n, n' \in N$ and $u, v \in Z^\Omega$ such that $n \succ n'$.

\begin{proposition}
\label{P:DerivationTerminating}
Let $X$ be an \LO-1-polygraph.
If there exist a derivation $d: \lin{Z} \fl N$ and a monotone well-founded partial order $\prec$ on $d(Z^\Omega)$, such that $d(s(\alpha)) \succ d(v)$, for all $\alpha\in X_1$ and $v\in\operatorname{Supp}(t(\alpha))$, then $X$ is terminating.
\end{proposition}
\begin{proof}
Suppose that there exists such a derivation and prove that it is compatible with $\Omega$-contexts.
If $d(v) \succ d(v')$ holds for $v,v'\in Z^\Omega$, we have
\[
d(uvw)=d(u)\cdot vw + u\cdot d(v)\cdot w+uv \cdot d(w)\succ d(u)\cdot vw + u\cdot d(v')\cdot w+uv \cdot d(w)=d(uv'w),
\]
and
\[
d(\brck{v}{\tau})=\brck{d(v)}{\tau}\succ\brck{d(v')}{\tau}=d(\brck{v'}{\tau}),
\]
for all $u,w\in Z^\Omega$.
By induction, we deduce that  $d(\left.q\right|_v) \succ d(\left.q\right|_{v'})$, for every $q \in Z^\Omega[\square]$. 
Now, suppose that there exists an infinite positive cell
\[
a_1\fl a_2 \fl \cdots \fl a_n\fl \cdots,
\]
with $a_i\in \lin{Z}$. 
We deduce an infinite sequence $u_1\fl u_2\fl \cdots \fl u_n\fl \cdots$,
where $u_i=s(\left.q_i\right|_{\alpha_i})$ and $u_{i+1}\in \operatorname{Supp}(t(\left.q_i\right|_{\alpha_i}))$ for some $\alpha_i\in X_1$ and $q_i \in Z^\Omega[\square]$. By the compatibility of $\prec$, this implies a strictly decreasing chain
\[
d(u_1)\succ d(u_2)\succ\cdots\succ d(u_n)\succ\cdots, 
\]
which contradicts the well-foundedness of $\prec$.
Hence, the polygraph $X$ terminates.
\end{proof}

\subsection{Confluence and critical branchings of operated polygraphs}
This subsection deals with confluence properties of \LO-1-polygraphs. The classification of local branchings in \LO-1-polygraphs extends the case of linear 1-polygraphs in \cite{GuiraudHoffbeckMalbos19}. We highlight the differences in their critical  branchings, and state the coherent critical branching theorem in the operated setting.

\subsubsection{Branchings and confluence}
A \emph{branching} of an \LO-1-polygraph $X$ is a pair $(f, g)$ of positive 1-cells such that $s(f) = s(g)$. It is \emph{local} if both $f$ and $g$ are rewriting steps of $X$.
Given a cellular extension~$Y$ of $\lin{X}$, a branching~$(f,g)$ is \emph{$Y$-confluent} if there exist positive $1$-cells~$h$ and~$k$ in~$\lin{X}$ and a $2$-cell~$F$ in~$\lin{X}[Y]$ as follows
\[
\vcenter{\xymatrix @R=0.5em{
& b
\ar@/^/ [dr] ^-{h}
\ar@2 []!<0pt,-12.5pt>;[dd]!<0pt,12.5pt> ^-*+{F}
\\
a
\ar@/^/ [ur] ^-{f}
\ar@/_/ [dr] _-{g}
&& d
\\
& c
\ar@/_/ [ur] _-{k}
}}
\]
We say that~$X$ is \emph{$Y$-confluent} (resp.  \emph{locally $Y$-confluent) at a 0-cell $a$} if every branching (resp.\ local branching) of~$X$ with source~$a$ is $Y$-confluent, and that~$X$ is \emph{$Y$-confluent} (resp.\ \emph{locally $Y$-confluent}) if it is so at every $0$-cell of~$\lin{X}$. 
When $Y$ contains all $1$-spheres of $\lin{X}$, $Y$-confluence corresponds to the classical notion of \emph{confluence}. We say that $X$ is \emph{convergent} when it is both terminating and confluent.
Each $0$-cell~$a$ of $\lin{X}$ then has a unique normal form.

\subsubsection{Classification of local branchings}
\label{SSS:ClassificationOfLocalBranchings}
The local branchings of \LO-1-polygraphs fall into the following families
\begin{enumerate}
    \item \emph{Aspherical branchings}: $(\lambda f + c, \lambda f + c)$, where $f$ is a rewriting step of $X$, and $\lambda$ is a nonzero scalar.

    \item \emph{Additive branchings}: $(\lambda f + \mu v + c, \lambda u + \mu g + c)$, where $f: u \to a$ and $g: v \to b$ are 1-monomials in $\lin{X}$, $\lambda, \mu$ are nonzero scalars, and $c$ is a 0-cell in $\lin{X}$, satisfying $u \neq v$ and $u, v \notin \operatorname{Supp}(c)$.

    \item \emph{Peiffer branchings}: $(\lambda q|_{\lfloor fv \rfloor_{\tau}} + c, \lambda q|_{\lfloor ug \rfloor_{\tau}} + c)$,  
where $f: u \to a$ and $g: v \to b$ are 1-monomials in $\lin{X}$, $\lambda$ is a nonzero scalar, and $c$ is a 0-cell in $\lin{X}$, satisfying $q|_{\lfloor uv \rfloor_{\tau}} \notin \operatorname{Supp}(c)$.  
This case corresponds to the Peiffer branching in the associative setting, as defined in {\cite[Def. 3.2.2]{GuiraudHoffbeckMalbos19}}, when $q = \square$ and $\tau = 0$.

    \item \emph{Overlapping branchings}: $(\lambda f + c, \lambda g + c)$,  
where $f: u \to a$ and $g: u \to b$ are 1-monomials in $\lin{X}$ such that the pair $(f, g)$ is neither aspherical, additive, nor Peiffer.  
Here, $\lambda$ is a nonzero scalar, and $c$ is any 0-cell of $\lin{X}$, with $u \notin \operatorname{Supp}(c)$.
\end{enumerate}

\subsubsection{Critical branchings}
The \emph{critical branchings} of an $\Omega$-1-polygraph $X$ are the overlapping branchings in \eqref{SSS:ClassificationOfLocalBranchings} for which $\lambda = 1$ and $c = 0$, and which cannot be factored as $(f, g) = (\left.q\right|_{f'}, \left.q\right|_{g'})$, where~$q \in Z^\Omega[\square]$ and $q \neq \square$. Specifically, if we define a well-founded order $\prec$ on overlapping branchings as follows
\[
(\left.q\right|_{f'},\left.q\right|_{g'})\succ (f', g')   \text{ for } q\in Z^\Omega[\square] \text{ and } q\neq \square,
\]
then the critical branching is minimal with respect to this order. Explicitly, every overlapping branching has a unique decomposition  as $(\left.q\right|_{f}+c, \left.q\right|_{g}+c)$, where $(f, g)$ is the critical branching.
We denote the set of critical branchings of a polygraph $X$ by $\mathrm{CB}(X)$. 
In particular, there are the following two shapes of critical branchings, called \emph{intersection} $(fw,ug)$ and \emph{inclusion branching} $(\left.q\right|_{h},k)$ respectively,
\begin{eqn}{equation}
\label{E:OperatedCriticalBranchings}
\raisebox{0.7cm}{
\xymatrix @R=0.5em {
& aw
\\
uvw
\ar@/^/ [ur] ^-{fw}
\ar@/_/ [dr] _-{ug}
\\
& ub
}
\qquad\qquad\qquad
\xymatrix @R=0.5em {
& \left.q\right|_{a'}
\\
\left.q\right|_{v'}
\ar@/^/ [ur] ^-{\left.q\right|_{h}}
\ar@/_/ [dr] _-{k}
\\
& b'
}}
\end{eqn}
where $u,v,w,v' \in Z^{\Omega} \setminus \{1\}$, $q\neq \square$ and $f: uv\fl a$, $g: vw \fl b$, $h: v'\fl a'$, $k:\left.q\right|_{v'} \fl b'$ belong to $X_1$.

Let $Y$ be a cellular extension of $\lin{X}$. We say that~$X$ is \emph{critical $Y$-confluent at a 0-cell $a$} if every critical branching  of~$X$ with source~$a$ is $Y$-confluent, and that~$X$ is \emph{critical $Y$-confluent} if it is so at every~$0$-cell of~$\lin{X}$. 

\begin{remark}
\label{R:DifferencePolygraphAnd OmegaPolygraph}
When $\Omega = \emptyset$, in \eqref{E:OperatedCriticalBranchings}, $q$ takes the form $u' \square w'$, which corresponds to the inclusion branchings $(u'hw',k)$ in the associative setting. However, operators introduce additional complexity in the structure of critical branchings. For example, the rules $k:\left\lfloor xy \right\rfloor \to yx$ and $h:xy \to z$ give rise to an inclusion branching $(\left\lfloor h \right\rfloor,k)$ that cannot be expressed in the associative setting. In Section~\ref{S:Polyautomata}, we explain how to describe these critical branchings in terms of string overlaps.
\end{remark}

The following lemma is the operated analogue of {\cite[Lemmata 3.1.3 and 4.1.2]{GuiraudHoffbeckMalbos19}}:
\begin{lemma}
\label{SSS:LemmaOfCriticalBranchings}
Let~$X$ be an \LO-$1$-polygraph, and $Y$ be a cellular extension of~$\lin{X}$ such that $X$ is $Y$-confluent at every $0$-cell $b\prec_X a$  for some fixed $0$-cell~$a$ of~$\lin{X}$.
 Let~$f$ be a $1$-cell of~$\lin{X}$ that admits a decomposition
\[
a_0 \:\ofl{f_1}\: a_1 \:\ofl{f_2}\: \cdots \:\ofl{f_p}\: a_p
\]
into $1$-cells of size~$1$. If there exist positive $1$-cells $a\fl a_i$ for every~$0<i<p$, then there exist positive $1$-cells~$g$ and~$h$ in~$\lin{X}$ and a $2$-cell~$F$ in~$\lin{X}[Y]$ as in
\[
\xymatrix @R=1em {
& a_p
	\ar@/^/ [dr] ^-{h}
	\ar@2 []!<0pt,-12pt>;[d]!<0pt,3pt> ^-*+{F}
\\	
a_0 
	\ar@/^/ [ur] ^-{f} 
	\ar@/_/ [rr] _-{g}
&& a'
}
\]
\end{lemma}

\begin{theorem}
\label{T:CriticalBranchingLemma}
Let~$X$ be a terminating \LO-$1$-polygraph, and~$Y$ be a cellular extension of~$\lin{X}$. If $X$ is critically $Y$-confluent, then $X$ is  $Y$-confluent.
\end{theorem}
\begin{proof}
We prove this result by considering the four cases of local branchings in \eqref{SSS:ClassificationOfLocalBranchings}. 
The proof of the confluence of these branchings uses the method from {\cite[Thm. 4.2.1]{GuiraudHoffbeckMalbos19}} for associative algebras, except for the Peiffer branchings, whose source is given by $\lambda \left.q\right|_{\left\lfloor uv\right\rfloor_{\tau}} + c$, as in~\eqref{SSS:ClassificationOfLocalBranchings}. 
We prove the confluence of such a branching by induction on rewriting order $\prec_X$ defined in \eqref{SSS:Termination}.
Given a reducible 0-cell $a$ of $\lin{X}$, we assume that $X$ is locally $Y$-confluent at every $0$-cell $b\prec_X a$.
By applying the coherent version of Newman's Lemma, as stated in {\cite[Pro. 4.1.3]{GuiraudHoffbeckMalbos19}}, which also holds in the operated setting, we conclude that $X$ is $Y$-confluent at every $b$.
We then have a coherently confluent diagram as follows
\[
\xymatrix @R=2.5em @C=1.5em {
& \lambda \left.q\right|_{\left\lfloor av\right\rfloor_{\tau}}+ c
\ar@/^/ [rr] ^-{f'_1}
\ar@{.>} [dr] |-*+{\lambda\left.q\right|_{\left\lfloor ag\right\rfloor_{\tau}}+ c}
&& a'
\ar@/^/ [drr] ^-{f'_2}
\ar@2 []!<2.5pt,-37.5pt>;[dd]!<2.5pt,37.5pt> ^-*+{H}
\\
\lambda \left.q\right|_{\left\lfloor uv\right\rfloor_{\tau}} + c
\ar@/^2ex/ [ur] ^-*+{\lambda\left.q\right|_{\brck{fv}{\tau}}+ c}
\ar@/_2ex/ [dr] _-*+{\lambda \left.q\right|_{\left\lfloor ug\right\rfloor_{\tau}} + c}
\ar@{} [rr] |(0.45){\sm =}
&& \lambda\left.q\right|_{\left\lfloor ab\right\rfloor_{\tau}} + c
\ar [ur] |-*+{h}
\ar [dr] |-*+{k}
\ar@2 [u]!<0pt,-7.5pt>;[]!<0pt,27.5pt> ^-*+{F^-}
\ar@2 []!<0pt,-27.5pt>;[d]!<0pt,7.5pt> ^-*+{G}
&&& d
\\
& \lambda \left.q\right|_{\left\lfloor ub\right\rfloor_{\tau}} + c
\ar@{.>} [ur] |-*+{\lambda \left.q\right|_{\left\lfloor fb\right\rfloor_{\tau}}+ c}
\ar@/_/ [rr] _-{g'_1}
&& b'
\ar@/_/ [urr] _-{g'_2}
}
\]
Note that the dotted 1-cells $\lambda \left.q\right|_{\left\lfloor ag\right\rfloor_{\tau}}+ c$ and $\lambda \left.q\right|_{\left\lfloor fb\right\rfloor_{\tau}}+ c$ may not be positive 1-cells if either $\operatorname{supp}(\left.q\right|_{\left\lfloor av\right\rfloor_{\tau}}) \cap\operatorname{supp}(c)$ or $\operatorname{supp}(\left.q\right|_{\left\lfloor ub\right\rfloor_{\tau}}) \cap \operatorname{supp}(c)$ is not empty. Following Lemma~\ref{SSS:LemmaOfCriticalBranchings}, we derive positive 1-cells $f_1', g_1', h, k$ and 2-cells $F$ and $G$. Since $\lambda \left.q\right|_{\left\lfloor ab\right\rfloor_{\tau}} + c\prec_X\lambda \left.q\right|_{\left\lfloor uv\right\rfloor_{\tau}} + c$, we further obtain positive 1-cells $f_2'$, $g_2'$, and a 2-cell $H$ that ensures $(h,k)$ is $Y$-confluent by hypothesis.
Finally, $X$ is $Y$-confluent by applying Newman's Lemma.
\end{proof}

As in the case of associative algebras {\cite[Thm. 3.4.2]{GuiraudHoffbeckMalbos19}}, convergent polygraphs provide canonical linear bases, as stated by the following result.

\begin{proposition}
\label{P:LinearBasisConvergentPolygraphs}
When $X$ is a convergent \LO-1-polygraph, the set $\Nf(X)$ forms a linear basis of the $\Omega$-algebra~$\overline{X}$.
\end{proposition}

\subsubsection{Reduced convergent presentations}
\label{SSS:ReducedPolygraph}
An \LO-$1$-polygraph $X$ is \emph{left-reduced} if the only rewriting step in $X$ with source $s(\alpha)$ is $\alpha$ itself, for every $1$-generator~$\alpha$.
It is \emph{right-reduced} if, for every $1$-generator~$\alpha$, the 0-cell $t(\alpha)$ is a normal form.
The polygraph $X$ is \emph{reduced} if it is both left-reduced and right-reduced.
Following {\cite[Theorem~2.4]{Squier87}}, every (finite) convergent string rewriting is Tietze equivalent to a (finite) reduced convergent one. This result also holds for \LO-$1$-polygraphs.

\begin{example}
\label{E:ConvergentOfDifferentialAlgebra}
The free differential algebra $\DA{\lambda}{Z}$ is presented by the following \LO-$1$-polygraph
\begin{eqn}{equation}
\label{E:FreeDifferentialAlgebra}
\begin{aligned}
X^{D} \coloneq (Z,X_1^{D}),
\qquad
X_1^{D} \coloneq \big\{ \alpha[u,v]&: D(uv)\rightarrow D(u)v+u D(v)+\lambda D(u) D(v),\\
\varphi&:D(1)\fl0 \text{ } |\text{ }  u,v \in Z^\Omega\setminus \{1\} \big\}.
\end{aligned}
\end{eqn}
We set $N \coloneq \mathbb{Z}^3$ and define a derivation $d: Z^\Omega \to N$ by  
\[
d(u) = \bigg(\sum_{D|u} \max\{\operatorname{deg}_\Omega(D) + \operatorname{deg}_Z(D) - 1, 0\}, \operatorname{deg}_\Omega(u), \operatorname{deg}_Z(u) \bigg),
\]
for every $u \in Z^\Omega$, where $D|u$ denotes each occurrence of the operator $D$ in $u$. Here, $\operatorname{deg}_\Omega(D)$ and $\operatorname{deg}_Z(D)$ count the number of operators and $0$-generators inside the operator $D$, respectively, while $\operatorname{deg}_\Omega(u)$ and $\operatorname{deg}_Z(u)$ count the number of operators and $0$-generators in $u$.
For instance, we have $d(D(1))=(0,1,0)$, $d(D(xy))=(1,1,2)$, and $d(D(x)D(y))=(0,2,2)$ for $x,y\in Z$.
For $(m_1,m_2,m_3)\in N$ and $a\in \DA{\lambda}{Z}$, we define
\[
a\cdot(m_1,m_2,m_3)
=(m_1,m_2,m_3)\cdot a
=(m_1,m_2,m_3),
\quad
D\bigl((m_1,m_2,m_3)\bigr)
=(\operatorname{max}\{m_1+m_2+m_3-1,0\},m_2+1,m_3).
\]
By definition $d$ satisfies the conditions in \eqref{E:Derivation}. Next, we endow $d(Z^\Omega) \subseteq N$ with a monotone \emph{lexicographic order}$\prec$, comparing tuples $(m,n,l)\in d(Z^\Omega)$ lexicographically, where $d(1) = (0,0,0)$ is the minimal element. This ensures that $d(D(u)D(v))$, $d(D(u)v)$ and $d(uD(v))$ are all less than $d(D(uv))$ for any $u, v \in Z^\Omega\setminus \{1\}$, and $d(D(1)) \succ (0,0,0)$. Hence, $X^D$ is terminating.

If we write $\alpha[1,1] \coloneq \varphi $, then the polygraph $X^{D}$ has two families of critical branchings
\[
\xymatrix @R=0.5em@C=4em {
&a_1 \ar@/^/ [dr] 
\\
D(\left.q\right|_{D(uv)}w)
\ar@/^/ [ur] ^-{\alpha[\left.q\right|_{D(uv)},w]}
\ar@/_/ [dr] _-{D(\left.q\right|_{\alpha[u,v]}w)}
&& a_3
\\
& a_2 \ar@/_/ [ur] 
}
\qquad\qquad\qquad
\xymatrix @R=0.5em@C=4em {
&b_1 \ar@/^/ [dr] 
\\
D(w\left.q\right|_{D(uv)})
\ar@/^/ [ur] ^-{\alpha[w,\left.q\right|_{D(uv)}]}
\ar@/_/ [dr] _-{D(w\left.q\right|_{\alpha[u,v]})}
&& b_3
\\
&b_2 \ar@/_/ [ur] 
}
\]
indexed by $u,v,w\in Z^\Omega$ with $w\neq 1$, both of which are confluent. Here, we have
\[
\begin{aligned}
a_1 &= D(\left.q\right|_{D(uv)})w + \left.q\right|_{D(uv)} D(w) + \lambda D(\left.q\right|_{D(uv)}) D(w), \\
a_2 &= D(\left.q\right|_{D(u)v}w) + D(\left.q\right|_{uD(v)}w) + \lambda D(\left.q\right|_{D(u)D(v)}w), \\
a_3 &= D(\left.q\right|_{D(u)v})w + D(\left.q\right|_{uD(v)})w + \lambda D(\left.q\right|_{D(u)D(v)})w \\
&\quad + \left.q\right|_{D(u)v} D(w) + \left.q\right|_{uD(v)} D(w) + \lambda \left.q\right|_{D(u)D(v)} D(w) \\
&\quad + \lambda D(\left.q\right|_{D(u)v}) D(w) + \lambda D(\left.q\right|_{uD(v)}) D(w) + \lambda^2 D(\left.q\right|_{D(u)D(v)}) D(w),
\end{aligned}
\]
and $b_1, b_2$ and $b_3$ can be similarly written. Thus $X^D$ is convergent.
Let $D^\theta(Z) \coloneq \{ D^i(x) \mid x \in Z, i \geq 0 \}$, where $D^0(x) \coloneq x$. The set
\[
\Nf(X^D)=\left(D^\theta(Z)\right)^\ast
\]
forms a linear basis of $\DA{\lambda}{Z}$.
Note that, since $X^D$ contains inclusion branchings, it is not reduced. We will construct a reduced presentation of $\DA{\lambda}{Z}$ in Subsection~\ref{SS:ResolutionsOfFreeDifferentialAlgebras}.
\end{example}

\subsection{Gröbner-Shirshov bases and convergence}
\label{SS:GSConvergence}
In this subsection, we establish a relationship between Gröbner–Shirshov bases of operated algebras \cite{BokutChenQiu,GGR15,QQWZ21} and convergent \LO-1-polygraphs.

\subsubsection{Gröbner-Shirshov bases}
Let $X$ be an \LO-1-polygraph and $S$ be a nonzero subset of $\lin{Z}$.
For the critical branchings in \eqref{E:OperatedCriticalBranchings}, we set $r = uvw$ (resp. $r = \left.q\right|_{v'}$) and define the $0$-cells
\begin{eqn}{equation}
\label{EQ:TwoCompositions} 
(a,b)_r \: \coloneq \: aw - ub \quad \text{\big(resp. $(a',b')_r \: \coloneq \: \left.q\right|_{a'} - b'$\big)}. 
\end{eqn}
Given a monomial order $\prec$ on $Z^\Omega$ compatible with $X_1$ and a nonzero $0$-cell $c$ in $\lin{Z}$, we denote by $\lm_\prec(c)$ the maximal $\Omega$-monomial in $\operatorname{Supp}(c)$.
The cell $c$ is \emph{trivial modulo $(S,r)$} 
if there is a decomposition
\[
c=\Sigma_i \lambda_i \left.q_i\right|_{s_i} \text{ with } \left.q_i\right|_{\lm_\prec(s_i)}\prec r,
\]
where $\lambda_i \in \bk$, $q_i \in Z^\Omega[\square]$, and $s_i \in S$.

A nonzero subset $S$ is a \emph{Gröbner-Shirshov (GS) basis of $\lin{Z}$ with respect to $\prec$} if, for all critical branchings \eqref{E:OperatedCriticalBranchings}, the $0$-cells in \eqref{EQ:TwoCompositions} are trivial modulo $(S, r)$.

\begin{proposition}
\label{P:ConvergenceGrobnerShirshov}
Let $X$ be an \LO-$1$-polygraph.
If the set $\partial(X_1)$ forms a GS basis~of $\lin{Z}$ with respect to a monomial order $\prec$ compatible with $X_1$, then the polygraph $X$ is convergent.
\end{proposition}
\begin{proof}
The termination of $X$ follows from the compatibility of the rewriting rules with the monomial order $\prec$. We consider every intersection branching $(fw,ug)$ in \eqref{E:OperatedCriticalBranchings}. Since $\partial(X_1)$ forms a GS basis of $\lin{Z}$ with respect to $\prec$, there is a decomposition
\begin{eqn}{equation}
\label{E:GSDecomposition}
aw-ub=\Sigma_i \lambda_i \left.q_i\right|_{\partial(\alpha_i)}, 
\end{eqn}
where $\alpha_i\in X_1$ and $\left.q_i\right|_{s(\alpha_i)}\prec uvw$.
Since $\partial(\alpha_i)$ and $0$ have the same image in $\lin{Z}$, we have $\rep{\partial(\alpha_i)} = 0$.
Applying the normal form to both sides of \eqref{E:GSDecomposition}, we obtain $\rep{aw}=\rep{ub}$. Hence every critical branching 
$(fw,ug)$ is confluent. 
We apply the same reasoning to the inclusion branchings.
Taking the cellular extension $Y$ containing all 1-spheres of $\lin{X}$, the confluence of $X$ is deduced from Theorem~\ref{T:CriticalBranchingLemma}. 
\end{proof}

\begin{remark}
\label{E:NoMonomialOrder}
The converse of Proposition~\ref{P:ConvergenceGrobnerShirshov} does not hold in general. Indeed, an \mbox{$\Omega$-1-polygraph} can be terminating without admitting a compatible monomial order. 
For example, the \LO-1-polygraph
\[
X \:=\: \big\{x, y, z \mid x\brca{y}z \fl \brca{x}y\brca{z} + \brca{x}\brca{y}z + x\brca{y}\brca{z}\big\}
\]
is terminating since, for every $q \in Z^\Omega[\square]$, the $\Omega$-monomial $\left.q\right|_{x\brca{y}z}$ contains one more factor $x\brca{y}z$ than $\left.q\right|_{\brca{x}y\brca{z}}$, $\left.q\right|_{\brca{x}\brca{y}z}$, or $\left.q\right|_{x\brca{y}\brca{z}}$.
However, there does not exist a monomial order $\prec$ compatible with $X_1$. Such an order $\prec$  would imply $x\brca{y}z \succ \brca{x}\brca{y}z$ and $x\brca{y}z \succ x\brca{y}\brca{z}$, leading to $x \succ \brca{x}$, $z \succ \brca{z}$, and thus to
\[
x \succ \brca{x} \succ \brca{\brca{x}} \succ \ldots  \quad \text{and} \quad z \succ \brca{z} \succ \brca{\brca{z}} \succ \ldots,
\]
contradicting that $\prec$ is a well-founded total order.
\end{remark}

\subsubsection{Completion procedure}
\label{SSS:CompletionProcedure}
When $X$ is a non-confluent terminating \LO-1-polygraph, we can complete the set of $1$-generators of $X$, without changing the presented algebra, in order to reach confluence.
This \emph{completion procedure} is well known in rewriting theory, see 
\cite{Buchberger65} for commutative algebras and \cite{KnuthBendix70} for term rewriting. 
Starting with a monomial order $\prec$ compatible with $X_1$, the procedure examines each non-confluent critical branching $(f, g)$ in $X$, and reduces $t(f)$ and $t(g)$ to some normal forms~$\rep{t(f)}$ and~$\rep{t(g)}$. 
A new $1$-generator $\lm_\prec(a) \to \lambda^{-1}a - \lm_\prec(a)$ is then added to the polygraph. 
When it terminates, the procedure produces a terminating \LO-1-polygraph $Y$ such that $\overline{X} \cong \overline{Y}$.

\section{Polyautomata and operated polygraphs}
\label{S:Polyautomata}
In this section, we introduce the structure of polyautomata to encode the operator structure of \LO-1-polygraphs. We interpret their critical branchings in terms of string overlaps in Theorem~\ref{T:OmegaCBshape} and establish a categorical equivalence between \LO-1-polygraphs and linear 1-polygraphs in Theorem~\ref{T:MainResultSection4}.

\subsection{Pushdown automata}
The automaton structure is a model of computation based on the notion of state.
Among these, \emph{pushdown automata (PDA)} use a last-in-first-out stack to process \emph{context-free languages}. Their state transitions depend on both the input symbol and the top of the stack, enabling them to handle nested structures such as operators.

\subsubsection{}
\label{SSS:PDA}
Recall that a \emph{pushdown automaton (PDA) on $\Sigma_0$} is a tuple $\mathbb{A}=(Q, \Sigma_0, \Gamma, \delta, q_0, F)$, where:
\begin{enumerate}
    \item $Q$ is a finite set of internal states,
    \item $\Sigma_0$ is the input alphabet,
    \item $\Gamma$ is a finite set of symbols called the \emph{stack alphabet},
    \item $\delta: Q \times \{\Sigma_0\cup \epsilon\} \times \{\Gamma\cup \epsilon\cup \$\} \to Q \times \Gamma^*$ is the state transition function, where $\epsilon$ is an empty string. In a given state, the PDA reads both the input symbol and the top symbol of the stack, then transitions to a new state and updates the stack top,
    \item $q_0 \in Q$ is the initial state,
    \item $F \subseteq Q$ is the set of accepting states.
\end{enumerate}

A \emph{monomial accepted by $\Ar$} is a word $ w = a_1 \cdots a_n$ in $\Sigma_0^* $ such that there exists a finite sequence of valid transitions
\[
(q_0, \$)
\xrightarrow[\;a_1\;]{} (q_1, \Gamma_1)
\xrightarrow[\;a_2\;]{} \cdots 
\xrightarrow[\;a_n\;]{} (q_n, \Gamma_n),
\]
with $q_n \in F$.
We denote by \emph{$\Sigma_0^{\mathbb{A}}$} the set of monomials accepted by $\Ar$. Note that~$\Sigma_0^{\mathbb{A}}$ does not form a monoid in general. 
We further denote by $\bk \Sigma_0^\Ar$ the set of \emph{polynomials accepted by $\Ar$}, consisting of linear combinations of monomials in $\Sigma_0^\Ar$.

\subsubsection{Examples}
A PDA is \emph{trivial} if it accepts all monomials in $\Sigma_0^\ast$. 
It can be pictured as follows
\[
\begin{tikzpicture}[shorten >=1pt, node distance=2.8cm, on grid, auto]
   \node[state, initial] (q_0)   {$q_0$}; 
   \node[state] (q_1) [right=of q_0] {$q_1$}; 
   \node[state, accepting] (q_2) [right=of q_1] {$q_2$};

    \path[->] 
    (q_0) edge[below]              node {$\epsilon, \epsilon \fl \$ $} (q_1)
    (q_1) edge [loop above] node {$\Sigma_0, \epsilon \fl \epsilon $} ()
          edge              node[below] {$\epsilon, \$ \fl \epsilon $} (q_2);
\end{tikzpicture}
\]
where $\Sigma_0, \epsilon \fl \epsilon$ denotes the set of instruction $x, \epsilon \fl \epsilon$ for every $x\in \Sigma_0$. 
The accepting states $q_2$ is represented by a double circle.

As a nontrivial example, consider the PDA $\Ar=(\{q_0,q_1,q_2\},\{a,b\},\{\$,0\},\delta,q_0,q_2)$, with
\[
    \delta(q_0, \epsilon, \epsilon)  = (q_1, \$), \quad  \delta(q_1, a, \epsilon)  = (q_1, 0),  \quad \delta(q_1, b, 0)  = (q_1, \epsilon),  \quad
\delta(q_1,  \epsilon,\$)  = (q_2, \epsilon).
\]
Its transition diagram is given below
\[
\begin{tikzpicture}[shorten >=1pt, node distance=2.7cm, on grid, auto]
   \node[state, initial] (q_0)   {$q_0$}; 
   \node[state] (q_1) [right=of q_0] {$q_1$}; 
   \node[state, accepting] (q_2) [right=of q_1] {$q_2$};

    \path[->] 
    (q_0) edge[below]              node {$\epsilon, \epsilon \fl \$ $} (q_1)
    (q_1) edge [loop above] node {$a, \epsilon \fl 0 $; $b, 0 \fl \epsilon $} ()
          edge              node[below] {$\epsilon, \$ \fl \epsilon $} (q_2);
\end{tikzpicture}
\]
This PDA accepts monomials of the form $a^n b^n$ for $n \geq 0$. The key mechanism lies in its stack operations: each $a$ pushes a $0$ onto the stack, while each $b$ pops a $0$, ensuring a balanced number of $a$'s and $b$'s. For instance, For instance, when processing the word $aabb$, the automaton follows these transitions:
\[
\begin{aligned}
    \text{Input: } & aabb, \quad \text{Stack: } \epsilon\xrightarrow{\epsilon} \$ \xrightarrow{a} \$0 \xrightarrow{a} \$00 \xrightarrow{b} \$0 \xrightarrow{b} \$ \xrightarrow{\epsilon}\epsilon.
\end{aligned}
\]
The PDA reaches the accepting state $q_2$ once the stack is emptied and all transitions are completed.

\subsection{Polyautomatic formulation of rewriting in operated algebras}
\label{SS:PolyautomaticFormulation}
This subsection introduces the notion of polyautomata. 
We show how to make explicit the structure of an $\Omega$-algebra by a polyautomaton. 
We deduce an equivalence between the categories of $\Pol(\OAlg{})$ and $\Pol(\Alg)$.

\subsubsection{Polyautomata}
As mentioned in Remark~\ref{R:OmegaEmpty}, a linear $1$-polygraph $\Sigma=(\Sigma_0,\Sigma_1)$, as introduced in \cite{GuiraudHoffbeckMalbos19}, is an \LO-1-polygraph with an empty set $\Omega$.
A \emph{$1$-polyautomaton} is a pair $(\Sigma,\mathbb{A})$ consisting of a linear $1$-polygraph $\Sigma$ and a PDA $\Ar$ on $\Sigma_0$, such that for every $\alpha\in \Sigma_1$, both the source $s(\alpha)$ and the target~$t(\alpha)$ are polynomials accepted by $\Ar$.
We denote by $\Pol_1(\Ar)$ the full subcategory of $\Pol_1(\Alg)$  consisting of $1$-polyautomata on $\Ar$. When $\Ar$ is trivial, these two categories coincide. 

\subsubsection{Bracket polyautomaton}
\label{SSS:OperatedRewriting}
The \emph{bracket $1$-polyautomaton} is a data $(\Sigma,\Ar_\Omega)$ made of
\begin{enumerate}
\item  $\Sigma_0\coloneq Z\sqcup \mathrm{Brck}(\Omega)$, where the  \emph{bracket set}  $\mathrm{Brck}(\Omega)$ is defined as follows
\[
\mathrm{Brck}(\Omega)\coloneq\bigcup_{\tau_i\in\Omega}\{\mathsf{\ell}_{\tau_i},\mathsf{r}_{\tau_i}\}.
\]
For simplicity, we will write $\mathsf{\ell}_{\tau_i}$ and $\mathsf{r}_{\tau_i}$ as $\mathsf{\ell}_i$ and $\mathsf{r}_i$ when there is no ambiguity.
\item The PDA $\Ar_\Omega$ is illustrated by the following state transition diagram
\begin{eqn}{equation}
\label{EQ:PolyautomataForOmegaMonoids}
\raisebox{-1.5cm}{
\begin{tikzpicture}[shorten >=1pt, node distance=2.8cm, on grid, auto]
   \node[state, initial] (q_0)   {$q_0$}; 
   \node[state] (q_1) [right=of q_0] {$q_1$}; 
   \node[state, accepting] (q_2) [right=of q_1] {$q_2$};

    \path[->] 
    (q_0) edge[below]              node {$\epsilon, \epsilon \fl \$ $} (q_1)
    (q_1) edge [loop above] node {$\mathsf{\ell}_i, \epsilon \fl i; \;\mathsf{r}_j, j \fl \epsilon $} ()
          edge [loop below] node {$Z, \epsilon \fl \epsilon $} ()
          edge              node[below] {$\epsilon, \$ \fl \epsilon $} (q_2);
\end{tikzpicture}
}
\end{eqn}
\end{enumerate}

\begin{lemma}
\label{L:OmegaIsomorphic}
The set $\Sigma_0^{\Ar_\Omega}$ with concatenation operation forms a monoid, isomorphic  to the free $\Omega$-monoid $Z^\Omega$. Moreover, this  induces an isomorphism $\psi$ of algebras between $\lin{Z}$ and $\bk\Sigma_0^{\Ar_{\Omega}}$.
\end{lemma}
\begin{proof}
First, we define a map $\psi: \Sigma_0^{\Ar_\Omega} \rightarrow Z^\Omega$ as follows. For any $u \in \Sigma_0^{\Ar_\Omega}$ and $\tau_i \in \Omega$, $\psi(u)$ is obtained by replacing each $\mathsf{\ell}_i$ with the left bracket "$\lfloor$" and each $\mathsf{r}_i$ with the right bracket "$\rfloor_{\tau_i}$" of the bracket $\brck{~}{\tau_i}$. For example,
\[
\psi(\mathsf{\ell}_1 x \mathsf{r}_1 \mathsf{\ell}_2 y \mathsf{r}_2) = \brck{x}{\tau_1}\brck{y}{\tau_2}.
\]
In particular, we set $\psi(\mathsf{\ell}_i \mathsf{r}_i) = \brck{1}{\tau_i}$ and $\psi(\epsilon) = 1$.

The map $\psi$ is surjective. Indeed, as shown in \eqref{EQ:PolyautomataForOmegaMonoids}, state $q_0$ transitions to $q_1$ by reading the empty string $\epsilon$, initializing the stack with $\$$. To reach $q_2$ from $q_1$, the stack must remain unchanged as $\$$.
In particular, a direct transition from $q_1$ to $q_2$ without additional instructions results in an output of~$\epsilon$.
At~$q_1$, by repeatedly reading instructions of the form $x, \epsilon \mapsto \epsilon$ for all $x \in Z$, $\Ar_\Omega$ can output any monomial in $Z^\ast$ while keeping the stack unchanged as $\$$, and then transition to $q_2$ to stop the process. Thus, we have $\psi(Z^\ast) = Z_0^\Omega$, where $Z_0^\Omega$ is defined in \eqref{SSS:FreeOmegaMonoid}.
Alternatively, at $q_1$, $\Ar_\Omega$ may first read the instruction $\mathsf{\ell}_i, \epsilon \mapsto i$, output $\mathsf{\ell}_i$, and push $i$ onto the stack. Since the presence of $i$ in the stack does not interfere with the instructions $x, \epsilon \mapsto \epsilon$, $\Ar_\Omega$ can continue to output any monomial in $Z^\ast$. Finally, by reading $\mathsf{r}_i, i \mapsto \epsilon$, $\Ar_\Omega$ outputs a monomial of the form $\mathsf{\ell}_i Z^\ast \mathsf{r}_i$, remove $i$ from the stack, and reach $q_2$. 
Similarly, $\Ar_\Omega$ can accept monomials of the form
\[
(Z \sqcup \mathsf{\ell}_{i_n} \cdots \mathsf{\ell}_{i_1} Z^\ast \mathsf{r}_{i_1} \cdots \mathsf{r}_{i_n})^\ast
\]
Thus, we have $\psi\left((Z \sqcup \mathsf{\ell}_{i_n} \cdots \mathsf{\ell}_{i_1} Z^\ast \mathsf{r}_{i_1} \cdots \mathsf{r}_{i_n})^\ast\right) = Z_1^\Omega$. By similar reasoning, for any subset $Z_k^\Omega \subseteq Z^\Omega$, we can construct its preimage under $\psi$.
Since $Z^\Omega$ is defined by $Z^\Omega \coloneq \varinjlim Z_n^\Omega$, we conclude that $\psi$ is surjective. 

The injectivity of $\psi$ and its compatibility with products are straightforward. It follows that \linebreak $\psi : \Sigma_0^{\Ar_\Omega} \to Z^\Omega$  is a monoid isomorphism, that we extend by linearity into an isomorphism $\psi : \bk\Sigma_0^{\Ar_\Omega} \to \lin{Z}$ of algebras.
\end{proof}

\subsubsection{Operated rewriting system}
\label{SSS:PolygraphVsPolyautomaton}
We define a functor
\[
\Sigma(-) \: :\: \Pol_1(\OAlg{}) \longrightarrow \Pol_1(\Ar_\Omega),
\]
which maps an \LO-$1$-polygraph $X$ to the $1$-polyautomaton $\Sigma(X) = (\Sigma_0, \Sigma_1, \Ar_\Omega)$, where $\Sigma_0 := Z \sqcup \mathrm{Brck}(\Omega)$ and
\[
\Sigma_1 \: \coloneq \: \{\psi^{-1}(\alpha): \psi^{-1}(a) \fl\psi^{-1}(b), \text{ for all } \alpha: a \fl b \in X_1 \},
\]
with $\psi$ defined in Lemma~\ref{L:OmegaIsomorphic}. Since $\psi^{-1}(a), \psi^{-1}(b) \in \bk \Sigma_0^{\Ar_\Omega}$, the cellular extension $\Sigma_1$ is well-defined.
This functor establishes a one-to-one correspondence between $\Omega$-$1$-polygraphs and  $1$-polyautomata. Moreover, we have the algebraic isomorphism
\[
\bk\Sigma_0^{ \Ar_\Omega} / \Sigma_1 \cong \overline{X}.
\]
The 1-polyautomaton $(\Sigma_0,\Sigma_1,\Ar_\Omega)$ is called an \emph{operated rewriting system} or an \emph{$\Omega$-rewriting system}.

\subsubsection{Example}
\label{D:DiffAutomaton}
According to \eqref{E:ConvergentOfDifferentialAlgebra} and that $\left(D^\theta(Z)\right)^\ast$ is a linear basis of the free differential algebra~$\DA{\lambda}{Z}$, the following PDA, denoted by $\Ar^{\scriptscriptstyle D}$, accepts the normal forms of $\DA{\lambda}{Z}$
\begin{eqn}{equation}
\label{D:PDAOfDiff}
\raisebox{-1.6cm}{
\begin{tikzpicture}[shorten >=1pt, node distance=2.8cm, on grid, auto]
    \node[state, initial] (q0) {$q_0$};
    \node[state] (q1) [right=of q0] {$q_1$};
    \node[state] (q2) [below left=of q1] {$q_2$};
    \node[state] (q3) [below right=of q1] {$q_3$};
    \node[state, accepting] (q5) [right=of q1] {$q_5$};
    \path[->]  (q0) edge node [above] {$\varepsilon, \varepsilon \to \$$} (q1);
    \path[->] (q1) edge [loop above] node {$Z, \varepsilon \to \varepsilon$} (q1);
    \path[->] (q1) edge node [above] {$\varepsilon, \$ \to \varepsilon$} (q5);
    \path[->] (q1) edge node [left] {$\varepsilon, \varepsilon \to \varepsilon$} (q2);
    \path[->] (q3) edge node [right] {$\varepsilon, \varepsilon \to \varepsilon$} (q1);
    \path[->] (q2) edge [loop left] node {$\mathsf{\ell}_{\scriptscriptstyle D}
, \varepsilon \to D$} (q2);
    \path[->] (q3) edge [loop right] node {$\mathsf{r}_{\scriptscriptstyle D}
, D \to \varepsilon$} (q3);
    \path[->] (q2) edge node [above] {$Z, \varepsilon \to \varepsilon$} (q3);
\end{tikzpicture}
}
\end{eqn} 
A differential algebra $A$ can thus be presented by a $1$-polyautomaton $(\Sigma, \Ar^{\scriptscriptstyle D})$, where $\Sigma_0 \coloneq Z \sqcup \mathrm{Brck}(\Omega)$ and $\Sigma_1$ is its set of defining relations. 
The algebra $A$ is thus isomorphic to the quotient algebra~$\bk\Sigma_0^{\Ar^{\scriptscriptstyle D}} / \Sigma_1$. 
For instance, for commutative differential algebras as in \cite{ChenChenLi}, we set $\Sigma_1 \coloneq \{uv \to vu \mid u,v \in \Sigma_0^{\Ar^{\scriptscriptstyle D}}\}$.

\begin{theorem}
\label{T:MainResultSection4}
The categories $\OAlg{}$ and $\Alg$ (resp. $\Pol_1(\OAlg{})$ and $\Pol_1(\Alg)$) are equivalent.
\end{theorem}
\begin{proof} 
Consider $\mathrm{Std}(A)$ the standard $\Omega$-$1$-polygraph of an $\Omega$-algebra $(A,\Tr_\tau)$. 
Its $0$-generators are elements of $A$ and its $1$-generators are $u \otimes v \to uv$ and $\brck{u}{\tau} \to \Tr_\tau(u)$, for all $u, v \in A$ and $\tau \in \Omega$, where $u\otimes v$ denotes the product of $u$ and $v$ in the free algebra $\lin{A}$, and $uv$ as their product in $A$. 
We define the functor $F: \OAlg{} \to \Alg$ by setting $F(A)=\bk \Sigma_0^{\Ar_\Omega}/\Sigma_1$, where $\Sigma(\mathrm{Std}(A))$ is the $1$-polyautomaton on the $\Omega$-$1$-polygraph $\mathrm{Std}(A)$  as defined in \eqref{SSS:PolygraphVsPolyautomaton}. The action of $F$ on morphisms is defined naturally.
Conversely, we define the functor $G: \Alg \to \OAlg{}$ by regarding every algebra as an $\Omega$-algebra with an empty set $\Omega$.

Next, for an $\Omega$-algebra $A$ and an algebra $B$, we consider natural transformations $\eta_A:=\psi^{-1}: A \to GF(A)$ and $\eta'_B := \id_B: B \to FG(B)$.
By definition, the following diagrams commute
\[
\xymatrix @R=2em {
A_1
   \ar[r]^{\hspace{-0.4cm} \eta_{A_1}}
	\ar[d]_{\hspace{0.5cm} f}
& GF(A_1)
	\ar[d] ^-{GF(f)}
\\
A_2
	\ar[r] _-{\eta_{A_2}}
& GF(A_2)
}
\qquad \quad
\xymatrix @R=2em {
B_1
   \ar[r]^{\hspace{-0.4cm} \eta'_{B_1}}
	\ar[d]_{\hspace{0.5cm} g}
& FG(B_1)
	\ar[d] ^-{FG(g)}
\\
B_2
	\ar[r] _-{\eta'_{B_2}}
& FG(B_2)
}
\]
for all $\Omega$-algebras $A_1,A_2$ and algebras $B_1,B_2$.
Therefore, we establish an equivalence of categories between $\OAlg{}$ and $\Alg$.
By \eqref{SSS:PolygraphVsPolyautomaton}, the equivalence between the categories $\Pol_1(\OAlg{})$ and $\Pol_1(\Alg)$ follows from that between $\OAlg{}$ and $\Alg$.
\end{proof}

\begin{remark}
 One can gain insight into the construction of $\Ar_\Omega$ in \eqref{EQ:PolyautomataForOmegaMonoids} by alternatively defining a subset of $\Sigma_0^\ast$, consisting of monomials $u$ that satisfy the following three conditions.
 Let $\operatorname{deg}_{\mathsf{\ell}_i}(u)$ and $\operatorname{deg}_{\mathsf{r}_i}(u)$ denote the number of occurrences of $\mathsf{\ell}_i$ and $\mathsf{r}_i$ in $u$, respectively.
\begin{enumerate}
\item\label{I:Automata1} For each $i$, $\operatorname{deg}_{\mathsf{\ell}_i}(u) = \operatorname{deg}_{\mathsf{r}_i}(u) = n_i$.
\item \label{I:Automata2} Write  $u=u_0\mathsf{\ell}_i u_1 \mathsf{\ell}_i u_2 \cdots u_{n-1} \mathsf{\ell}_i u_{n_i}$, 
where $u_0, \ldots, u_{n_i} \in \{\Sigma_0 \setminus \mathsf{\ell}_i\}^\ast$. The following condition holds
\[
\operatorname{deg}_{\mathsf{r}_i}(u_0) + \operatorname{deg}_{\mathsf{r}_i}(u_1) + \cdots + \operatorname{deg}_{\mathsf{r}_i}(u_m) \leq m, \quad \text{ for } 0 \leq m \leq n_i.
\]
For each $\mathsf{\ell}_i$ located between $u_m$ and $u_{m+1}$, we first search for the first occurrence of $\mathsf{r}_i$ in $u_{m+1}$ from left to right. If no such $\mathsf{r}_i$ is found, we then search for the second occurrence of $\mathsf{r}_i$ in $u_{m+2}$, the third occurrence in $u_{m+3}$, and so on, until it is found. Such a pair $(\mathsf{\ell}_i, \mathsf{r}_i)$ is called an \emph{$\Omega$-pair}.

\item For each $\Omega$-pair $(\mathsf{\ell}_i, \mathsf{r}_i)$, in the monomial $u = w_1 \mathsf{\ell}_i v \mathsf{r}_i w_2$, the submonomial $v$  satisfies conditions {\bf i)} and {\bf ii)}.
\end{enumerate}
Monomials satisfying these conditions are also isomorphic to $\Omega$-monomials, similarly to those in $\Sigma_0^{\Ar_\Omega}$.
\end{remark}

\subsection{From operated to non-operated: critical branchings}

In this subsection, we show how to represent critical branchings of an $\Omega$-$1$-polygraph as critical branchings of the associated linear $1$-polygraph. In this subsection, $X$ stands for an $\Omega$-$1$-polygraph.

\begin{theorem}
\label{T:OmegaCBshape}
There exists a one-to-one correspondence between the sets of critical branchings  $\mathrm{CB}(X)$ and $\mathrm{CB}(\Sigma(X))$.
\end{theorem}
\begin{proof}
We consider a critical branching $(f,g) \in \mathrm{CB}(X)$, where $f: a \to b$ and $g: a \to c$. We map $(f,g)$ to $(f', g')$ in $\mathrm{CB}(\Sigma(X))$,  where $f': \psi^{-1}(a) \to \psi^{-1}(b)$ and $g': \psi^{-1}(a) \to \psi^{-1}(c)$.  
By Lemma~\ref{L:OmegaIsomorphic}, the bijective map $\psi^{-1}$ induces a correspondence between $\mathrm{CB}(X)$ and $\mathrm{CB}(\Sigma(X))$.
\end{proof}

\begin{remark}
\label{R:CriticalBranchingForLinearPolyautomaton}
We denote by $\dashrightarrow$ and $\fl$ the $0$-cells and $1$-cells of $\Sigma(X)$, respectively, and by $\overset{a}{\dashrightarrow} \overset{b}{\dashrightarrow}$ the 0-cells $ab$.
Following Theorem~\ref{T:OmegaCBshape}, the intersection and inclusion branchings of $X$ have the following shapes
\begin{eqn}{equation}
\label{D:TwoCriticalBranchingsShapes}
\xymatrix @C=4em {
\strut 
\ar@{..>} [r] _-{u}
\ar@{..>}@/^7ex/ [rr] ^-{}="aw"
& \strut
\ar@{..>} [r] ^-{v}
\ar@{..>}@/_7ex/ [rr] _-{}="ub"
& \strut
\ar@{..>}[r] ="w"^-{w}
& \strut
\ar@1 "1,2";"aw"!<0pt,3pt> ^-{\alpha}
\ar@1 "1,3";"ub"!<0pt,7.5pt> ^-{\beta}
}
\qquad
\xymatrix @C=4em {
\strut 
\ar@{..>} [r] _-{u'}
\ar@{..>}@/^7ex/ [rrr] ^-{}="aw"
& \strut
\ar@{..>} [r] |-{v'}="jd"
\ar@{..>}@/_7ex/ [r] _-{}="ub"
& \strut
\ar@{..>}[r] ="w"_-{w'}
& \strut
\ar@1 "jd"!<0pt,7.5pt>;"aw"!<0pt,-7.5pt> _-{\alpha}
\ar@1 "jd"!<0pt,7.5pt>;"ub"!<0pt,-7.5pt> ^-{\beta} 
}
\end{eqn}
respectively, with $\alpha,\beta$ in $\Sigma(X)_1$.
For the inclusion branchings in \eqref{D:TwoCriticalBranchingsShapes}, although $u', w' \in \Sigma_0^\ast$, they are not elements of $\Sigma_0^{\Ar_\Omega}$ in general.
For instance, consider the example from~\eqref{R:DifferencePolygraphAnd OmegaPolygraph}
\[
X\:\coloneq\{x,y,z \mid f:\left\lfloor xy \right\rfloor\fl yx, g:xy\fl z \}.
\]
We have the $1$-polyautomaton $\Sigma(X)=(\Sigma_0,\Sigma_1,\Ar_\Omega)$, where $\Sigma_0 = \{x, y, z, \mathsf{\ell},\mathsf{r}\}$ and $\Sigma_1 = \{\alpha: \mathsf{\ell} xy \mathsf{r} \fl yx, \beta: xy \fl z\}$. Then, the critical branching of  $X$ can be illustrated as follows
\[
\xymatrix @C=2.5em {
\ar@{..>} [r] _-{\mathsf{\ell}}="s"
\ar@{..>}@/^6ex/ [rrrr] ^-{yx}="aw"
&
\ar@{..>} [r] ^{x}="t"
\ar@{..>}@/_6ex/ [rr] _-{z}="ub"
& 
\ar@{..>}[r] ^-{y}="w"
& 
\ar@{..>}[r] _-{\mathsf{r}}="r"
& 
\ar@1 "1,3";"aw"!<0pt,3pt> ^-{\alpha }
\ar@1 "1,3";"ub"!<0pt,7.5pt> ^-{\beta}
}
\]
Here, we have $x, y, \mathsf{\ell}, \mathsf{r} \in \Sigma_0^\ast$, but $\mathsf{\ell}, \mathsf{r} \notin \Sigma_0^{\Ar_\Omega}$, as they are not accepted by the PDA $\Ar_\Omega$.
\end{remark}

\begin{lemma}
\label{L:IntersectionShapes}
Let $u, v, w \in \Sigma_0^\ast$ as in \eqref{SSS:OperatedRewriting}. If both $uv$ and $vw$ belong to $\Sigma_0^{\Ar_\Omega}$, then $u, v, w$ belong to $\Sigma_0^{\Ar_\Omega}$.
\end{lemma}
\begin{proof}
If either $uv$ or $vw$ is $\epsilon$, the result follows trivially.
Let $u = x_1 \cdots x_m$, $v = y_1 \cdots y_n$, and $w = z_1 \cdots z_k$, where $x_i, y_i, z_i \in \Sigma_0$.
Since $vw \in \Sigma_0^{\Ar_\Omega}$, the PDA~$\Ar_\Omega$ can accept the monomial $y_1 \cdots y_n z_1 \cdots z_k$. This implies that $\Ar_\Omega$ can output $y_1 \cdots y_n$ while remaining in state $q_1$ in \eqref{EQ:PolyautomataForOmegaMonoids}. At this point, the top of the stack may contain either the symbol $i$ or $\$$.

Since $uv \in \Sigma_0^{\Ar_\Omega}$, $\Ar_\Omega$ also accepts the monomial $x_1 \cdots x_m y_1 \cdots y_n$. If the top of the stack contains the symbol $i$ after outputting $y_1 \cdots y_n$, it will still contain $i$ after outputting $x_1 \cdots x_m y_1 \cdots y_n$. This prevents~$\Ar_\Omega$ from transitioning to $q_2$ and halting, which contradicts $uv \in \Sigma_0^{\Ar_\Omega}$.

Therefore, when $\Ar_\Omega$ outputs $y_1 \cdots y_n$, the top of the stack must be $\$$, ensuring that $\Ar_\Omega$ accepts $y_1 \cdots y_n$, transitions to state $q_2$, and halts. Hence, $v \in \Sigma_0^{\Ar_\Omega}$.
Finally, if either $u$ or $w$ were not in~$\Sigma_0^{\Ar_\Omega}$, it would contradict the assumption that both $uv$ and $vw$ belong to $\Sigma_0^{\Ar_\Omega}$. Thus, we conclude that $u, v, w \in~\Sigma_0^{\Ar_\Omega}$.
\end{proof}

\begin{corollary}
\label{C:IntersectionShape}
For the intersection branchings in \eqref{D:TwoCriticalBranchingsShapes},  we have $u,v,w\in \Sigma_0^{\Ar_\Omega}$.
\end{corollary}

\subsubsection{Higher critical branchings}
\label{SSS:Higher-critical branching}
For $n \geq 2$, a \emph{critical $n$-branching of $X$} is a tuple $(f_1, \ldots, f_n)$ of $1$-cells $f_i$ with the same source, such that each pair $(f_i, f_j)$ is a critical branching of $X$ for all $i \neq j$.

\begin{example}
\label{E:InvolutiveAlgebra}
Consider an \LO-1-polygraph $X^I$ with $X_1^I\coloneq \{\beta_u:\brca{\brca{u}}\fl u \text{ }|\text{ } u\in Z^\Omega \}$ .
We denote by $\brca{u}^k$ the $k$-fold bracketing of~$u$. Then, for $n\geq 2$, the $n$-tuple
\[
\left(\beta_{\brca{u}^{n-1}}, \brca{\beta_{\brca{u}^{n-2}}}, \brca{\beta_{\brca{u}^{n-3}}}^2,\ldots, \brca{\beta_{u}}^{n-1}\right)
\]
is a critical $n$-branching with source $\brca{u}^{n+1}$ for every $0$-cells $u$.
In particular, we have the $2$-critical branching $(\beta_{\brca{u}}, \brca{\beta_u})$ and the $3$-critical branching $(\beta_{\brca{u}^2}, \brca{\beta_{\brca{u}}}, \brca{\beta_u}^2)$ as follows
\[
\xymatrix@!R@!C@C=4em{
\brca{\brca{\brca{u}}}
\ar@/^5ex/ [r] ^-{\brca{\beta_u}}
\ar@/_5ex/ [r] _-{\beta_{\brca{u}}}
& \brca{u}
}
\qquad\quad
\xymatrix@!R@!C@C=6em{
\brca{\brca{\brca{\brca{u}}}}
\ar@/^5ex/ [r] ^-{\brca{\brca{\beta_u}}}
\ar@// [r] |-{\brca{\beta_{\brca{u}}}}
\ar@/_5ex/ [r] _-{\beta_{\brca{\brca{u}}}}
& \brca{\brca{u}}
}
\]
All critical branchings of $X^I$ are confluent. 
The termination of $X^I$ follows from the decrease in the number of operators under the application of the rule $\beta_u$. Consequently, $X^I$ is convergent.
\end{example}

\begin{remark}
\label{R:Reducedn-Branchings}
When the polygraph $X$ is reduced, all its critical branchings are intersection branchings as in \eqref{D:TwoCriticalBranchingsShapes}.
Indeed, the inclusion branchings in \eqref{D:TwoCriticalBranchingsShapes} imply that there exist two rewriting steps $\alpha$ and $u'\beta w'$ with source $u'v'w'$.
Therefore, for any critcial $n$-branching $(f_1, \dots, f_n)$ of $X$, each pair $(f_i, f_j)$ is an intersection branching of $X$ for all $i \neq j$. For instance, we illustrate critical $3$-branchings of $X$ as follows
\[
\xymatrix @C=3em {
\strut 
\ar@{..>} [r] _-{u_1}
\ar@{..>}@/^7ex/ [rr] ^-{}="a"
& \strut
\ar@{..>} [r] ^{u_2}
\ar@{..>}@/_7ex/ [rrr] _-{}="b"
& \strut
\ar@{..>}[r] ^-{u_3}="w"
& \strut
\ar@{..>}[r] ^-{u_4}="m"
\ar@{..>}@/^7ex/ [rr] ^-{}="c"
& \strut
\ar@{..>}[r] _-{u_5}="s"
& \strut
\ar@1 "1,2";"a"!<0pt,3pt> ^-{}
\ar@1 "w"!<0pt,7.5pt>;"b"!<0pt,-7.5pt> ^-{}
\ar@1 "1,5";"c"!<0pt,3pt> ^-{}
}
\quad
\xymatrix @C=3em {
\strut 
\ar@{..>} [r] _-{u_1}
\ar@{..>}@/^7ex/ [rrr] ^-{}="a"
& \strut
\ar@{..>} [r] _{u_2}="n"
\ar@{..>}@/_7ex/ [rrr] _-{}="b"
& \strut
\ar@{..>}[r] ^-{u_3}="w"
\ar@{..>}@/^7ex/ [rrr] ^-{}="c"
& \strut
\ar@{..>}[r] _-{u_4}="m"
& \strut
\ar@{..>}[r] _-{u_5}="s"
& \strut
\ar@1 "n"!<0pt,7.5pt>;"a"!<0pt,-7.5pt> ^-{}
\ar@1 "w"!<0pt,7.5pt>;"b"!<0pt,-7.5pt> ^-{}
\ar@1 "m"!<0pt,7.5pt>;"c"!<0pt,-7.5pt> ^-{}
}
\]
\end{remark}

\section{Polygraphic resolutions of operated algebras}
\label{S:PolygraphicResolutions}

In this section, we present the acyclic properties of an \LO-$\omega$-polygraph using homotopical contractions, as introduced in \cite{GuiraudHoffbeckMalbos19}. The main result of this paper, Theorem~\ref{T:Main conclusion}, constructs polygraphic resolutions for $\Omega$-algebras from convergent and reduced presentations, extending the constructions for associative algebras \cite{GuiraudHoffbeckMalbos19}, categories \cite{GuiraudMalbos12advances}, and operads \cite{MalbosRen23} to $\Omega$-algebras.

\subsection{Polygraphic resolutions and contractions}

A notion of homotopy on associative $\omega$-algebras were introduced in~\cite{GuiraudHoffbeckMalbos19}.
In this subsection, we extend this notion to $\Omega$-$\omega$-algebras. 
To account for the operator structure, we introduce the notion of bracket contraction to characterize acyclic $\Omega$-$\omega$-algebras.

\subsubsection{Polygraphic resolutions}
A cellular extension $Y$ of an $n$-algebra $A$ is \emph{acyclic} if, for every $n$-sphere $(f,g)$ in $A_n$, there exists an $(n+1)$-cell of the free $(n+1)$-algebra $A[Y]$ with source $f$ and target~$g$.
An \LO-$\omega$-polygraph $X$ is a \emph{polygraphic resolution} of an $\Omega$-algebra $A$ if its underlying \LO-1-polygraph $(Z, X_1)$ is a presentation of~$A$ and all the cellular extensions $X_n$ are acyclic.

\subsubsection{Homotopies}
Let $F,G:A\fl B$ be two morphisms of \LO-$\omega$-algebras. A \emph{homotopy from~$F$ to~$G$} is an \emph{indexed morphism} of \LO-$\omega$-algebras
\[
\eta \: : \: A \ofl{} B
\]
of degree~$1$, namely, a sequence $\eta = (\eta_k : A_k \fl B_{k+1})_{k \geq 0}$ of morphisms of \LO-algebras, satisfying the following conditions, where we write~$\eta_a\coloneq\eta_k(a)$ for every $a\in A_k$,
\begin{enumerate}
\item for every~$0$-cell~$a$ of~$A$, 
\[
s(\eta_a) = F(a)
\qquad\text{and}\qquad
t(\eta_a) = G(a),
\]
\item for $n\geq 1$ and every $n$-cell~$a$ of~$A$,
\begin{eqn}{align}
\label{E:SourceHomotopy}
s(\eta_a) &= F(a) \star_0 \eta_{t_0(a)} \star_1 \cdots \star_{n-1} \eta_{t_{n-1}(a)}, 
\\[0.5ex]
\label{E:TargetHomotopy}
t(\eta_a) &= \eta_{s_{n-1}(a)} \star_{n-1} \cdots \star_1 \eta_{s_0(a)} \star_0 G(a),
\end{eqn}
\item for $n\geq 0$ and every $n$-cell~$a$ of~$A$,
\[
\eta_{1_a} = 1_{\eta_a}.
\]
\end{enumerate}

Following {\cite[Def. 5.1.1]{GuiraudHoffbeckMalbos19}} in the associative setting, \eqref{E:SourceHomotopy} and \eqref{E:TargetHomotopy} are well-defined. 
The globularity of $\eta_a$, for every $n$-cell $a$ of $A$, follows from
\begin{align*}
& ss(\eta_a)
= s(F(a)) \star_0 \eta_{t_0(a)} \star_1 \cdots \star_{n-2} \eta_{t_{n-2}(a)} 
= s(\eta_{s(a)}) 
= st(\eta_a)
\\[0.5ex]
\text{and}\quad 
&
ts(\eta_a) 
= t(\eta_{t(a)})
= \eta_{s_{n-2}(a)} \star_{n-2} \cdots \star_1 \eta_{s_0(a)} \star_0 t(G(a))
= tt(\eta_a).
\end{align*}

\begin{remark}
From this definition, we prove by induction on $n$ that for every $a\in A_n$, the following relation holds 
\[
\eta_{\brca{a}} = \brca{\eta_{a}}.
\]
Indeed, for $a\in A_0$, since $\brca{~}$ commutes with the morphisms $F$ and $G$, we have $\eta_{\brca{a}} = \brca{\eta_{a}}$.
Assume $\eta_{\brca{a}} = \brca{\eta_{a}}$ holds for every $a\in A_k$ with $0 \leq k \leq n$.
Given $a\in A_{n+1}$, using the inductive hypothesis and the commutativity of $\brca{~}$ with $F$, $s$, and $\star_0$, we have
\[
s(\eta_{\brca{a}}) = \brca{F(a)} \star_0 \brca{\eta_{t_0(a)}} \star_1 \cdots \star_{n-1} \brca{\eta_{t_{n-1}(a)}} = \brca{s(\eta_{a})}.
\]
Similarly, we deduce
$t(\eta_{\brca{a}}) = \brca{t(\eta_{a})}$.
\end{remark}

\begin{example}
In low dimensions, the homotopy $\eta$ maps a 1-cell $f: a \fl a'$ of $A$ to a 2-cell
\[
\xymatrix @R=0.25em @C=1.5em {
& F(a')
\ar@/^/ [dr] ^-{\eta_{a'}}
\ar@2 []!<0pt,-15pt>;[dd]!<0pt,15pt> ^-*+{\eta_f}
\\
F(a)
\ar@/^/ [ur] ^-{F(f)}
\ar@/_/ [dr] _-{\eta_a}
&& G(a')
\\
& G(a)
\ar@/_/ [ur] _-{G(f)}
}
\]
of~$B$, and a $2$-cell $H:f\dfl f':a\fl a'$ of~$A$ to a $3$-cell 
\[
\vcenter{\xymatrix @R=0.25em @C=4em {
& F(a')
\ar@/^/ [dr] ^-{\eta_{a'}}
\ar@2 []!<5pt,-15pt>;[dd]!<5pt,15pt> ^-*+{\eta_{f'}}
\\
F(a)
\ar@/^5ex/ [ur] ^(0.2){F(f)} _{}="s"
\ar@/_/ [ur] |-*+{F(f')} ^{}="t"
\ar@2 "s"!<-6pt,-10pt>;"t"!<-9pt,10pt> ^-{F(H)}
\ar@/_/ [dr] _{\eta_a}
&& {G(a')} 
\\
& G(a)
\ar@/_/ [ur] _-{G(f')}
}}
\quad \tfl^{\eta_H}\quad
\vcenter{\xymatrix @R=0.25em @C=4em {
& F(a')
\ar@/^/ [dr] ^-{\eta_{a'}}
\ar@2 []!<-10pt,-15pt>;[dd]!<-10pt,15pt> ^-*+{\eta_{f}}
\\
F(a)
\ar@/^/ [ur] ^-{F(f)}
\ar@/_/ [dr] _{\eta_a}
&& {G(a')} 
\\
& G(a)
\ar@/^/ [ur] |-*+{G(f)} _-{}="s"
\ar@/_5ex/ [ur] _(0.8){G(f')} ^{}="t"
\ar@2 "s"!<-7pt,-12pt>;"t"!<-10pt,7pt> ^-{G(H)}
}}
\]
of $B$, and a 3-cell $\Gamma: H \tfl H': f \dfl f': a \fl a'$

\[
\xymatrix@!R@!C@C=9em{
a
\ar@/^5ex/ [r] ^-{f}="src"
\ar@/_5ex/ [r] _-{f'}="tgt"
\ar@2 "src"!<-18pt,-15pt>;"tgt"!<-18pt,15pt> _-*+{H} ^-{}="src2"
\ar@2 "src"!<18pt,-15pt>;"tgt"!<18pt,15pt> ^-*+{H'} _-{}="tgt2"
\ar@3 "src2"!<7.5pt,0pt>;"tgt2"!<-7.5pt,0pt> ^-*+{\Gamma}
& a'
}
\]
to a 4-cell of $B$ whose the source is 
\[
\xymatrix@!C@C=25em{
F(a)
\ar@/^6ex/ [r] ^-{F(f)\star_0\eta_{a'}}="src"
\ar@/_6ex/ [r] _-{\eta_{a}\star_0 G(f')}="tgt"
\ar@2 "src"!<-57pt,-18pt>;"tgt"!<-57pt,18pt> _-*+{\scriptscriptstyle F(H)\star_0 \eta_{a'} \star_1 \eta_{f'}} ^-{}="src2"
\ar@2 "src"!<0pt,-15pt>;"tgt"!<0pt,15pt> |(0.71){\scriptscriptstyle F(H')\star_0 \eta_{a'}\star_1 \eta_{f'}} _-{}="tgt1"
\ar@2 "src"!<57pt,-18pt>;"tgt"!<57pt,18pt> ^-*+{\scriptscriptstyle \eta_{f}\star_1\eta_{a}\star_0G(H')} _-{}="tgt2"
\ar@3 "src2"!<7.5pt,5pt>;"tgt1"!<-7.5pt,5pt> ^-*+{\scriptscriptstyle F(\Gamma)\star_0 \eta_{a'} \star_1 \eta_{f'}}
\ar@3 "tgt1"!<7.5pt,5pt>;"tgt2"!<-7.5pt,5pt> ^-*+{\scriptscriptstyle\eta_{H'}}
& G(a')
}
\]
and target is
\[
\xymatrix@!C@C=25em{
F(a)
\ar@/^6ex/ [r] ^-{F(f)\star_0\eta_{a'}}="src"
\ar@/_6ex/ [r] _-{\eta_{a}\star_0 G(f')}="tgt"
\ar@2 "src"!<-57pt,-18pt>;"tgt"!<-57pt,18pt> _-*+{\scriptscriptstyle F(H)\star_0 \eta_{a'} \star_1 \eta_{f'}} ^-{}="src2"
\ar@2 "src"!<0pt,-15pt>;"tgt"!<0pt,15pt> |(0.71){\scriptscriptstyle \eta_f \star_1 \eta_a \star_0 G(H)} _-{}="tgt1"
\ar@2 "src"!<57pt,-18pt>;"tgt"!<57pt,18pt> ^-*+{\scriptscriptstyle \eta_{f}\star_1\eta_{a}\star_0G(H')} _-{}="tgt2"
\ar@3 "src2"!<7.5pt,5pt>;"tgt1"!<-7.5pt,5pt> ^-*+{\scriptscriptstyle \eta_H}
\ar@3 "tgt1"!<7.5pt,5pt>;"tgt2"!<-7.5pt,5pt> ^-*+{\scriptscriptstyle\eta_f \star_1 \eta_a \star_0 G(\Gamma)}
& G(a')
}
\]
\end{example}

The following result is proved as in the case of linear-polygraphs  {\cite[Lem.~5.1.4]{GuiraudHoffbeckMalbos19}}.

\begin{lemma}
\label{L:LemmaOfHomotopy}
Let~$X$ be an \LO-$\omega$-polygraph, $(A, \Tr)$ an $\omega$-algebra, and~$F,G:\lin{X}\fl A$ morphisms of $\omega$-algebras. A homotopy~$\eta$ from~$F$ to~$G$ is uniquely and entirely determined by its values on the $n$-monomials of~$\lin{X}$, for~$n\geq 0$, provided the relation
\begin{eqn}{equation}
\label{E:HomotopyExchange}
\eta_{us_0(v)} + \eta_{t_0(u)v} - \eta_{t_0(u)s_0(v)} 
\: = \: \eta_{s_0(u)v} + \eta_{ut_0(v)} - \eta_{s_0(u)t_0(v)} 
\end{eqn}
is satisfied for all $n$-monomials~$u$ and~$v$ of~$\lin{X}$.
\end{lemma}

\subsubsection{Unital sections and contractions}
Let $X$ be an \LO-$\omega$-polygraph, the presented algebra $\cl{X}$ can be viewed as an $\omega$-algebra whose all $n$-cells are identities for $n \geq 1$.
An \emph{unital section} of~$X$ is a linear map of $\omega$-algebras  $\iota:\cl{X}\fl\lin{X}$ that is a section of the canonical projection $\pi:\lin{X}\pfl\cl{X}$ and that satisfies $\iota(1)=1$.
For any $n$-cell  $a$ of~$\lin{X}$, we write~$\rep{a}$ for $\iota\pi(a)$, and have $\rep{a}=1_{\rep{s_0(a)}}$ for~$n\geq 1$.

An \emph{$\iota$-contraction of~$X$} is a homotopy~$\sigma$ from~$\id_{\lin{X}}$ to $\iota\pi$ such that $\sigma_a = 1_a$
for every $n$-cell~$a$ of~$\lin{X}$ that belongs to the image of~$\iota$ or of~$\sigma$. The $\iota$-contraction $\sigma$ is \emph{right and bracketed} if, for every $n \geq 0$, for all $n$-cells $f, g, h$ in $\lin{X}$ with respective $0$-sources $a, b, c$, and for any $\Omega$-monomial $v$ in~$\lin{Z}$, the following conditions hold
\begin{eqn}{align}
\label{E:RightContraction}
\sigma_{fg} &= a\sigma_g \star_0 \sigma_{f\rep{b}},
\\[0.5ex]
\label{E:BracketedContraction}
\sigma_{\left\lfloor h\right\rfloor v} &= \left\lfloor \sigma_h\right\rfloor v \star_0 \sigma_{\left\lfloor \rep{c}\right\rfloor v}.
\end{eqn}
The right-hand side of \eqref{E:BracketedContraction} is well-defined, as $t_0(\left\lfloor \sigma_h \right\rfloor v)= \left\lfloor \rep{c} \right\rfloor v =s_0(\sigma_{\left\lfloor \rep{c} \right\rfloor v})$
holds.

\begin{example}
Let us explain the right and bracketed $\iota$-contraction $\sigma$ in low dimensions. For 0-cells $a,b \text{ and } c$, we have
\[
\xymatrix @R=0.75em {
& {a\rep{b}}
\ar @/^/ [dr] ^-{\sigma_{a\rep{b}}} 
\ar@{} [d] |(0.70){\sm =} 
\\
ab
\ar @/^/ [ur] ^-{a\sigma_{b}} 
\ar @/_/ [rr] _-{\sigma_{ab}} 
&& {\rep{ab}}
}
\quad\quad\quad\quad
\xymatrix @R=0.75em {
& \left\lfloor \rep{c}\right\rfloor v
\ar@/^/ [dr] ^-{\sigma_{\left\lfloor \rep{c}\right\rfloor v}}
\ar@{} [d] |(0.70){\sm =} 
\\
\left\lfloor c\right\rfloor v
\ar@/^/ [ur] ^-{\left\lfloor \sigma_c\right\rfloor v}
\ar@/_/ [rr] _{\sigma_{\left\lfloor c\right\rfloor v}}
&& {\rep{\left\lfloor c\right\rfloor v} }
}
\]
For 1-cells $f:a\fl b$, $g:c\fl d$ and a 0-cell $v$, we have
\[
\vcenter{\xymatrix @R=0.75em {
& bd
\ar@/^/ [dr] ^-{\sigma_{bd}}
\ar@2 []!<0pt,-10pt>;[d]!<0pt,2.5pt> ^-*+{\sigma_{fg}}
\\
ac
\ar@/^/ [ur] ^-{fg}
\ar@/_/ [rr] _-{\sigma_{ac}}
&& {\rep{ac}}
}}
\qquad = \qquad
\vcenter{\xymatrix @R=1.5em @C = 3.5em {
&& bd
\ar @/^5ex/ [ddrr] ^-{\sigma_{bd}} _-{}="B"
\ar [dr] |-*+{b\sigma_{d}}
\ar@{} [dd] |-{\sm =}
\\
& ad
\ar [ur] |-*+{fd}
\ar [dr] |-*+{f\sigma_{d}}
\ar@2 []!<-5pt,-12pt>;[d]!<-5pt,11pt> ^-*+{a\sigma_g}
&& {b\rep{c}}
\ar [dr] |-*+{\sigma_{b\rep{c}}}
\ar@2 []!<-5pt,-12pt>;[d]!<-5pt,11pt> ^-*+{\sigma_{f\rep{c}}}
\ar@{} [];"B" |(0.6){\sm =}
\\
ac 
\ar @/^5ex/ [uurr] ^-{fg} _-{}="A"
\ar@{} [ur];"A" |(0.5){\sm =}
\ar [ur] |-*+{ag}
\ar [rr] |-*+{a\sigma_c}
\ar @/_5ex/ [rrrr] _-{\sigma_{ac}} ^-{}="C"
&& {a\rep{c}}
\ar [ur] |-*+{f\rep{c}}
\ar [rr] |-*+{\sigma_{a\rep{c}}}
\ar@{} [];"C" |-{\sm =}
&& {\rep{ac}}
}}
\]
\[
\vcenter{\xymatrix @R=0.75em {
& \left\lfloor b\right\rfloor v
\ar@/^/ [dr] ^-{\sigma_{\left\lfloor b\right\rfloor v}}
\ar@2 []!<0pt,-10pt>;[d]!<0pt,2.5pt> ^-*+{\sigma_{\left\lfloor f\right\rfloor v}}
\\
\left\lfloor a\right\rfloor v
\ar@/^/ [ur] ^-{\left\lfloor f\right\rfloor v}
\ar@/_/ [rr] _-{\sigma_{\left\lfloor a\right\rfloor v}}
&& {\rep{\left\lfloor a\right\rfloor v}}
}}
= 
\vcenter{\xymatrix @R=3em @C=4em {
& \left\lfloor b\right\rfloor v
\ar [dr] |-*+{\scriptstyle \left\lfloor \sigma_b\right\rfloor v}
\ar@2 []!<-9pt,-12pt>;[d]!<-9pt,11pt> ^-*+{\scriptstyle \left\lfloor \sigma_f\right\rfloor v}
\ar @/^5ex/ [drrr] ^-{\scriptstyle \sigma_{\left\lfloor b\right\rfloor v}} _-{}="B"
& & &
\\
\left\lfloor a\right\rfloor v
\ar@/^/ [ur] |-*+{\scriptstyle \left\lfloor f\right\rfloor v}
\ar [rr] |-*+{\scriptstyle \left\lfloor \sigma_a\right\rfloor v}
\ar @/_5ex/ [rrrr] _-{\scriptstyle \sigma_{\left\lfloor a\right\rfloor v}} ^-{}="C"
& 
& \left\lfloor \rep{a}\right\rfloor v
\ar [rr] |-*+{\scriptstyle \sigma_{\left\lfloor \rep{a}\right\rfloor v}}
\ar@{} [];"C" |-{\sm =}
\ar@{} [ur] |(0.45){\sm =}
&& \rep{\left\lfloor a\right\rfloor v} 
}}
\] 
\end{example}

\subsection{Operated polygraphic resolutions from convergence}
This subsection presents the main result of this paper, Theorem~\ref{SSS:SquierPolygraphicResolution}, that constructs a polygraphic resolution for an $\Omega$-algebra from a convergent presentation.
Such a resolution \emph{à la Squier} is generated in each dimension $n\geq 2$ by the sources of the critical $n$-branchings of the presentation.

\subsubsection{Reduced and essential $\Omega$-monomials}\label{SSS:EssentialMonomial}
Let $\iota$ be an unital section of an \LO-$\omega$-polygraph $X$. An $\Omega$-monomial~$u$ of~$\lin{Z}$ is \emph{$\iota$-reduced} if $u=\rep{u}$. A non-$\iota$-reduced $\Omega$-monomial~$u$ is \emph{$\iota$-essential} if 
\begin{enumerate}
\item $u=xv$, where $x$ is a $0$-generator of~$X$ and $v$ is an $\iota$-reduced $\Omega$-monomial of~$\lin{Z}$,
\item $u=\left\lfloor w\right\rfloor  v$, where $w$ and $v$ are both $\iota$-reduced $\Omega$-monomials of~$\lin{Z}$.
\end{enumerate}

If~$\sigma$ is an $\iota$-contraction of~$X$, for~$n\geq 1$, an $n$-cell~$a$ of~$\lin{X}$ is \emph{$\sigma$-reduced} if it is an identity or in the image of~$\sigma$. If the $\iota$-contraction $\sigma$ is a right and bracketed, for~$n\geq 1$, a non-$\sigma$-reduced $n$-monomial~$a$ of~$\lin{X}$ is \emph{$\sigma$-essential} if 
\begin{enumerate}[resume]
\item $u=\alpha v$, where $\alpha$ is an $n$-generator of~$X$ and $v$ is a $\iota$-reduced $\Omega$-monomial of~$\lin{Z}$.
\end{enumerate}

\begin{lemma}
\label{L:RightAndBracketed}
Let $X$ be an \LO-$\omega$-polygraph and $\iota$ be a unital section of $X$. A right and bracketed $\iota$-contraction $\sigma$ of $X$ is
uniquely and entirely determined by its values on the $\iota$-essential $\Omega$-monomials of~$\lin{Z}$ and, for $n>1$, on the $\sigma$-essential $\Omega$-monomials of $\lin{X}$. 
\end{lemma}
\begin{proof}
By Lemma~\ref{L:LemmaOfHomotopy}, it suffices to check that the values of $\sigma$ on the $\iota$-essential $\Omega$-monomials and on $\sigma$-essential $\Omega$-monomials determine its values on other $\Omega$-monomials and $n$-monomials, and that equation \eqref{E:HomotopyExchange} holds.
If $u$ is a non-$\iota$-essential $\Omega$-monomial, we consider three cases
\begin{enumerate}
\item $u=1$. Then $\sigma_1 = 1$ is forced since $1$ is $\iota$-reduced.
\item $u=yv$, where $y$ is either a $0$-generator of $X$ or an $\iota$-reduced monomial $\lfloor w \rfloor$, and $v$ is a non-$\iota$-reduced $\Omega$-monomial of $\lin{Z}$. Then \eqref{E:RightContraction} imposes $\sigma_{yv} = y\sigma_v \star_0 \sigma_{y\rep{v}}$. We proceed by induction on the length of $v$ to define $\sigma_v$.
\item $u=\left\lfloor w\right\rfloor v$, where $w$ is a non-$\iota$-reduced $\Omega$-monomial of $\lin{X}$ and $v$ is an arbitrary $\Omega$-monomial. Then \eqref{E:BracketedContraction} imposes $\sigma_{\left\lfloor w\right\rfloor v} = \left\lfloor \sigma_w \right\rfloor v \star_0 \sigma_{\left\lfloor \rep{w}\right\rfloor v}$. When $v$ is $\iota$-reduced, $\left\lfloor \rep{w}\right\rfloor v$ is also $\iota$-reduced. We define~$\sigma_w$ by induction on the number of operators in $w$ and its length. When $v$ is not $\iota$-reduced, we define $\sigma_{\left\lfloor \rep{w}\right\rfloor v}$ based on the previous case.
\end{enumerate}
If $w$ is an $n$-monomial, then $w$ can be written as $(p_1\circ p_2 \circ \cdots \circ p_n)|_\alpha$,
where  $\alpha$ is a 1-generator with 0-source $a$ and $p_k\coloneq\left\lfloor u_k \square  v_k\right\rfloor_{\tau_k}\in Z^\Omega[\square]$ with  $u_k, v_k \in Z^\Omega$. We distinguish between two cases:
\begin{enumerate}
\item $w = u\alpha v$. Then \eqref{E:RightContraction} imposes $\sigma_{u\alpha v} = u a \sigma_v \star_0 u\sigma_{\alpha \rep{v}} \star_0 \sigma_{u \rep{a v}}$. We proceed by induction on the length of $u$ and $v$ to define $\sigma_{u \rep{a v}}$ and $\sigma_v$.
\item $w = u\left\lfloor \left.q\right|_{\alpha}\right\rfloor v$. Then \eqref{E:RightContraction} imposes $\sigma_{u \left\lfloor \left.q\right|_{\alpha}\right\rfloor v} = u \left\lfloor \left.q\right|_{a}\right\rfloor \sigma_v \star_0 u\sigma_{\left\lfloor \left.q\right|_{\alpha}\right\rfloor \rep{v}} \star_0 \sigma_{u \rep{\left\lfloor \left.q\right|_{a}\right\rfloor v}}$. We proceed by induction on the length of $u$ and $v$ to define $\sigma_v$ and $\sigma_{u \rep{\left\lfloor \left.q\right|_{a}\right\rfloor v}}$.
As for $\sigma_{\left\lfloor \left.q\right|_{\alpha}\right\rfloor \rep{v}}$,  \eqref{E:BracketedContraction} imposes $\sigma_{\left\lfloor \left.q\right|_{\alpha}\right\rfloor \rep{v}} = \left\lfloor \sigma_{\left.q\right|_{\alpha}}\right\rfloor \rep{v} \star_0 \sigma_{\left\lfloor \rep{\left.q\right|_{a}}\right\rfloor \rep{v}}$. We then proceed by induction on the number of operators and the length of $\left.q\right|_{\alpha}$ to define $\sigma_{\left.q\right|_{\alpha}}$.
\end{enumerate}
To verify \eqref{E:HomotopyExchange}, we refer to the case of associative algebras in {\cite[Lemma5.2.5]{GuiraudHoffbeckMalbos19}} and that of the shuffle operad in {\cite[Lemma5.1.6]{MalbosRen23}}, as their proofs do not differ significantly.
\end{proof}

\begin{proposition}
\label{T:ResolutionContraction}
Let $X$ be an \LO-$\omega$-polygraph with a fixed unital section $\iota$. Then $X$ is a polygraphic
resolution of the $\overline{X}$ if, and only if, it admits a right and bracketed $\iota$-contraction.
\end{proposition}
\begin{proof}
Assume that $X$ is a polygraphic resolution of $\overline{X}$. We define a right and bracketed $\iota$-contractions using Lemma~\ref{L:RightAndBracketed}. 
For an essential monomial $xv$, both $xv$ and $\rep{xv}$ map to the same element in $\lin{Z}$, so there exists a 1-cell $\sigma_{xv}: xv \to \rep{xv}$ in $\lin{X}$.
Similarly, for an essential monomial of the form $\left\lfloor w \right\rfloor v$, $\sigma_{\left\lfloor w \right\rfloor v}$ is defined analogously.
Assume that $\sigma$ is defined on all~$n$-cells of $\lin{X}$. We now extend it to the~$\sigma$-essential $n$-monomial $\alpha v$. By hypothesis, $s(\sigma_{\alpha v})$ and $t(\sigma_{\alpha v})$ are parallel, ensuring the existence of an $(n+1)$-cell $\sigma_{\alpha v}$.

Conversely, for $n \geq 1$, let $\sigma$ be a right and bracketed $\iota$-contraction, and let $f$ and $g$ be parallel $n$-cells in $\lin{X}$. We show that $t(\sigma_f) = \sigma_{s(f)} = \sigma_{s(g)} = t(\sigma_g)$, ensuring that the $(n+1)$-cell $\sigma_f \star_n \sigma_g^-$ is well-defined.
The fact that $t_k(f) = t_k(g)$ for all $0 \leq k < n$ implies that
\[
\sigma_f \star_n \sigma_g^- \star_{n-1} \sigma_{t_{n-1}(f)}^- \star_{n-2} \cdots \star_0 \sigma_{t_0(f)}^-
\]
is a well-defined $n$-cell of $\lin{X}$ with the source $f$ and target $g$. Hence, $X_{n+1}$ is acyclic.
\end{proof}

In the following, we assume that each 0-generator of $X$ is a normal form; if not, we reduce this polygraph to a smaller one using a collapsing mechanism from {\cite[Subsec.5.3]{GuiraudHoffbeckMalbos19}}.

\subsubsection{Operated polygraphic resolutions}
\label{SSS:SquierPolygraphicResolution}
Let $X$ be a reduced convergent \LO-1-polygraph. 
We define $\mathrm{Sq}(X)$ as the family of generators $\left(\mathrm{Sq}_n(X)\right)_{n \geqslant 0}$, where
\begin{enumerate}[label=\bf\roman*), ref=\theenumi)]
\setlist[enumerate,2]{label=\bf\alph*), ref=\theenumi-\theenumii)}
\item $\mathrm{Sq}_0(X) = Z$,
\item  $\mathrm{Sq}_1(X)$ is the set of tuples $(u_1,u_2)$, written $u_1 | u_2$, satisfying one of the following two cases
\begin{enumerate}
\item\label{I:Condition1OfSq1} $u_1\in Z$, or there exists $u_0\in Z^\Omega$ such that $u_1=\left\lfloor u_0 \right\rfloor \in \Nf(X)$, and $u_2\in \Nf(X)$,
\item \label{I:Condition1OfSq2}$u_1 = \epsilon$ and $u_2 = \left\lfloor u_0 \right\rfloor$, where $u_0\in \Nf(X)$,
\end{enumerate}
plus the following condition
\begin{enumerate}[resume]
\item\label{I:Condition1OfSq3} $u_1 u_2$ is reducible, and every proper left-factor of $u_1 u_2$ is a normal form,
\end{enumerate}
\item  For $n \geq 2$, $\mathrm{Sq}_n(X)$ is the set of tuples $(u_1, \ldots, u_{n+1})$, written $u_1 | \cdots | u_{n+1}$, such that the following conditions hold
\begin{enumerate}
\item $(u_1,u_2) \in \mathrm{Sq}_1(X)$ satisfying the cases \ref{I:Condition1OfSq1} and \ref{I:Condition1OfSq3},
\item $u_i\in \Nf(X)$, for every $i>2$,
\item For every $2 \leq i< n+1$, $u_i u_{i+1}$ is reducible, and every proper left-factor of $u_i u_{i+1}$ is a normal form.
\end{enumerate}
\end{enumerate}

\begin{theorem}
\label{T:Main conclusion}
Let $X$ be a reduced convergent \LO-$1$-polygraph. There exists a unique polygraphic structure on $\mathrm{Sq}(X)$ and a unique unital section $\iota$, as well as a right and bracketed $\iota$-contraction $\sigma$ of $\mathrm{Sq}(X)$, such that $\iota\pi(u) = \widehat{u}$ for every $u\in Z^\Omega$, and
\begin{eqn}{equation}
\label{E:ValueOfSigamOnEssentialMonomial}
\sigma_{\left(u_1|\cdots| u_n\right) u_{n+1}} = \begin{cases}
u_1|\cdots| u_{n+1} & \text{if } u_1|\cdots| u_{n+1} \in \mathrm{Sq}_n(X), \\
1_{\left(u_1|\cdots| u_n\right) u_{n+1}} & \text{if } u_nu_{n+1}\in \Nf(X),
\end{cases}
\end{eqn}
for all $(n-1)$-generators $u_1 | \cdots | u_n$ in $\mathrm{Sq}_{n-1}(X)$, and $u_{n+1} \in \Nf(X)$ with $n \geq 1$. When $n=1$, we allow writing $u_1 = \brca{u_0} \in \Nf(X)$. 
We also set
\begin{eqn}{equation}
\label{E:ExtraCases}
\sigma_{\brca{u_0}}=\epsilon|\brca{u_0} \quad \text{and} \quad \sigma_{(\epsilon|\brca{u_0})u_3}=1_{(\epsilon|\brca{u_0})u_3},
\end{eqn}
for all reducible $\Omega$-monomial $\brca{u_0}$ with $u_0\in \Nf(X)$, and  all 1-generators $\epsilon|\brca{u_0}$ with $u_3\in \Nf(X)$.
Then this structure makes $\mathrm{Sq}(X)$ a polygraphic resolution of $\lin{Z}$.
\end{theorem}
\begin{proof}
When  \eqref{E:ValueOfSigamOnEssentialMonomial} holds, the source and target maps of $\mathrm{Sq}(X)$, except for the 1-generators $\epsilon|\brca{u_0}$, are determined by the first case. Writing $\underline{u} = u_1 | \cdots | u_n$ for short, we obtain
\[
s(u_1|\cdots|u_{n+1})=s(\sigma_{\underline{u}u_{n+1}})=\underline{u}u_{n+1}\star_0 \sigma_{t_0(\underline{u})u_{n+1}}\star_1\cdots\star_{n-2} \sigma_{t_{n-2}(\underline{u})u_{n+1}},
\]
and
\[
t(u_1|\cdots|u_{n+1})=t(\sigma_{\underline{u}u_{n+1}})=
\begin{cases}
\rep{u_1u_2}
&\text{if~$n=1$,}
\\
\sigma_{s(\underline{u})u_{n+1}}
&\text{otherwise.}
\end{cases}
\]
We determine $s(\epsilon|\brca{u_0}) = \brca{u_0}$ and $t(\epsilon|\brca{u_0}) = \rep{\brca{u_0}}$ by \eqref{E:ExtraCases}.
Next, we define the values of $\sigma$ on $n$-cells of $\lin{\mathrm{Sq}(X)}$. According to Lemma~\ref{L:RightAndBracketed}, it suffices to define $\sigma$ on the $\iota$-essential $\Omega$-monomials and $\sigma$-essential $n$-monomials.

For the $\iota$-essential $\Omega$-monomial $\left\lfloor u_0\right\rfloor u_3$, where $u_0, u_3\in\Nf(X)$ but $\left\lfloor u_0\right\rfloor$ is reducible. If $u_3$ is identity, we have $\sigma_{\left\lfloor u_0\right\rfloor}=\epsilon|\left\lfloor u_0\right\rfloor$. 
Otherwise, \eqref{E:ExtraCases} reads $\sigma_{(\epsilon|\brca{u_0})u_3}=1_{(\epsilon|\brca{u_0})u_3}$, that is the source and target of~$\sigma_{(\epsilon|\brca{u_0})u_3}$ must be equal, giving the value of $\sigma$ on $\left\lfloor u_0\right\rfloor u_3$:
\[
\sigma_{\left\lfloor u_0\right\rfloor u_3}=t(\sigma_{(\epsilon|\brca{u_0})u_3})=s(\sigma_{(\epsilon|\brca{u_0})u_3})=(\epsilon|\brca{u_0})u_3 \star_0 \sigma_{\rep{\left\lfloor u_0\right\rfloor}u_3}.
\]

For the $\iota$-essential monomial $u_1u_2$, where $u_1\in Z$ or $u_1=\brca{u_0}\in\Nf(X)$, $u_2\in\Nf(X)$, and $u_1u_2$ is reducible. If $u_1|u_2\in\mathrm{Sq}_1(X)$, then \eqref{E:ValueOfSigamOnEssentialMonomial} imposes $\sigma_{u_1u_2}=u_1|u_2$. If not, there exists a proper factorization $u_2=v_2w_2$ such that $u_1|v_2\in\mathrm{Sq}_1(X)$. We have $\sigma_{(u_1|v_2)w_2}=1_{(u_1|v_2)w_2}$ by the second case in \eqref{E:ValueOfSigamOnEssentialMonomial}, that is 
\[
\sigma_{u_1u_2}
= t(\sigma_{(u_1| v_2)w_2})
= s(\sigma_{(u_1| v_2)w_2})
= (u_1| v_2)w_2 \star_0 \sigma_{\rep{u_1v_2}w_2}.
\]

Now, consider $n\geq 2$. For the $\sigma$-essential monomial $\underline{u}u_{n+1}$, where $\underline{u}$ is a $(n-1)$-generator of $\mathrm{Sq}(X)$, and $u_{n+1}\in\Nf(X)$. 
We distinguish four cases. First, for $n=1$, if $\underline{u}u_3=(\epsilon|\brca{u_0})u_3$, then \eqref{E:ExtraCases} reads $\sigma_{(\epsilon|\brca{u_0})u_3}=1_{(\epsilon|\brca{u_0})u_3}$.
Second, if $\underline{u}| u_{n+1}\in \mathrm{Sq}(X)$, then \eqref{E:ValueOfSigamOnEssentialMonomial} imposes $\sigma_{\underline{u}u_{n+1}}=\underline{u}| u_{n+1}$. Third, if~$u_n u_{n+1}\in\Nf(X)$, then \eqref{E:ValueOfSigamOnEssentialMonomial} imposes $\sigma_{\underline{u}u_{n+1}}=1_{\underline{u}u_{n+1}}$. Otherwise, there exists a proper factorization $u_{n+1}=v_{n+1}w_{n+1}$ such that $\underline{u}| v_{n+1}\in\mathrm{Sq}(X)$. In that case, \eqref{E:ValueOfSigamOnEssentialMonomial} implies that the source and the target of~$\sigma_{(\underline{u}| v_{n+1})w_{n+1}}$ are equal. On the one hand, we have
\[
s(\sigma_{(\underline{u}| v_{n+1})w_{n+1}})
= (\underline{u}| v_{n+1}) w_{n+1} \star_0 \sigma_{t_0(\underline{u}|
v_{n+1})w_{n+1}} \star_1 \cdots \star_{n-2} \sigma_{t_ {n-2}(\underline{u}| v_{n+1})
w_{n+1}},
\]
and, on the other hand, we obtain
\[
t(\sigma_{(\underline{u}| v_{n+1})w_{n+1}})
= \sigma_{s(\underline{u}| v_{n+1}) w_{n+1}} 
= \sigma_{s(\sigma_{\underline{u} v_{n+1}}) w_{n+1}} 
= \sigma_{\underline{u}u_{n+1} \star_0 \sigma_{t_0(\underline{u})v_{n+1}} w_{n+1}
\star_1 \cdots\star_{n-2} \sigma_{t_{n-2}(\underline{u})v_{n+1}} w_{n+1}}.
\]
Since $a \star_k b = a + b - t_k(a)$ and $\sigma$ commutes with $t$, we can rewrite the latter expression, by induction on~$n$, as a linear composition of $n$-cells containing $\sigma_{\underline{u}u_{n+1}}$, $\sigma_{\sigma_{t_{n-2}(\underline{u})v_{n+1}} w_{n+1}}$, and other lower-dimensional cells. Thus, $\sigma_{\underline{u}u_{n+1}}$ can be determined from the relation $s(\sigma_{(\underline{u}| v_{n+1})w_{n+1}}) = t(\sigma_{(\underline{u}| v_{n+1})w_{n+1}})$.

Finally,  by Proposition~\ref{T:ResolutionContraction}, we conclude that $\mathrm{Sq}(X)$ is a polygraphic resolution of $\overline{X}$.
\end{proof}

\begin{example}
Consider the \LO-1-polygraph $X^I$ from Example~\ref{E:InvolutiveAlgebra}. This polygraph is convergent, but not reduced.
In order to construct $\Nf(X^I)$, we consider
\[
\Phi_I = \bigcup_{n \geq 0} \Phi_n,
\quad\text{with}\quad
\Phi_0 = \left(Z\cup \brca{Z^\ast} \right)^\ast
\quad \text{and} \quad 
\Phi_n = \left(Z\cup\left\lfloor\left( \Phi_{n-1}\right)_{\geq 2}\right\rfloor\right)^\ast
\;\text{for $n \geq 1$,}
\]
where $\left( \Phi_{n-1}\right)_{\geq 2}$ denotes the set of all $\Omega$-monomials $u$ in $\Phi_{n-1}$ satisfying $\operatorname{bre}(u) \geq 2$.  
Note that the construction of $\Phi_I$ excludes all $\Omega$-monomials of the form $\left.q\right|_{\brca{\brca{u}}}$ in $Z^\Omega$, so we have $\Nf(X^I) = \Phi_I$.
We then present the reduced polygraph $\widetilde{X}^I = (Z, \widetilde{X}_1^I)$, which is Tietze equivalent to $X^I$, where
\[
\widetilde{X}_1^I\coloneq\{\beta_u:\brca{\brca{u}}\fl u \mid u\in(\Phi_I)_{n \geq 2} \text{ or } u\in Z\}.
\]
Thus, the $\Omega$-algebra presented by $\widetilde{X}^I$ has the polygraphic resolution $\mathrm{Sq}(\widetilde{X}^I)$, where
\begin{enumerate}
\item $\mathrm{Sq}_0(\widetilde{X}^I)=Z$,
\item $\mathrm{Sq}_1(\widetilde{X}^I)$ has the 1-generators $\epsilon |\brca{\brca{u}}:\brca{\brca{u}}\fl u, \text{ for all } u\in  (\Phi_I)_{n \geq 2} \text{ or } u\in $Z,
\item for every $n>1$, $\mathrm{Sq}_n(\widetilde{X}^I)$ is empty.
\end{enumerate}
\end{example}

\section{Examples of resolutions of operated algebras}
\label{S:ExampleOfResolutionOfOmegaAlgebras}

In this final section, we apply the above constructions to some classical free $\Omega$-algebras, including free Rota-Baxter algebras \cite{BokutChenQiu, KuruschGuo}, free differential algebras \cite{BokutChenQiu,LiGuo,
GuoSitZhang}, and free differential Rota-Baxter algebras \cite{GKeigher08,BokutChenQiu,LQQZ}.
 
\subsection{Polygraphic resolutions of free Rota-Baxter algebras}
\label{SS:RotaBaxterResolutions}
This subsection presents a polygraphic resolution of the free Rota-Baxter algebra $\RBA{\lambda}{Z}$ on $Z$.

\subsubsection{Normal forms} 
\label{SSS:PolygraphForRB}
The algebra $\RBA{\lambda}{Z}$ is presented by the  \LO-1-polygraph $X^P$ with 
\[
X_1^P \: \coloneq \: \big\{ \alpha[u,v]: P(u)P(v)\rightarrow P(P(u)v)+P(uP(v))+\lambda P(uv)
\; |\; u,v\in Z^\Omega\,\big\}.
\]  
We set $N=\mathbb{Z}^2$ and define a derivation $d : Z^\Omega \to N$ by  
\[
d(u) = \bigg(\operatorname{deg}_\Omega(u), \sum_{P|u} \big(\operatorname{deg}_\Omega(u)-\operatorname{deg}_\Omega(P)\big) \bigg),
\]
for every $u\in Z^\Omega$, where  $P|u$ denotes each occurrence of the operator $P$ in $u$, and $\operatorname{deg}_\Omega(P)$  counts the number of operators inside the operator $P$.
For instance, we have $d(P(1))=(1,1)$, $d (P(x)P(y) )=(2,4)$, and $d (P (P(x)y ) ) =(2,3)$ for $x,y\in Z$.
For every $(m_1,m_2)\in N$ and $v\in Z^\Omega$ with $\operatorname{deg}_\Omega(v)=n$, we~set
\[
v\cdot(m_1,m_2)=(m_1 ,m_2)\cdot v=(m_1,m_2+n m_1), 
\quad
P((m_1, m_2))
=(m_1+1, m_2+m_1+1).
\]
By definition $d$ satisfies \eqref{E:Derivation}. By equipping $d(Z^\Omega)$ with a monotone  lexicographic order, we ensure that $d(P(u)v)$, $d(uP(v))$ and $d(uv)$ are all less than $d(P(u)P(v))$ for any $u, v \in Z^\Omega$, with $d(1) = (0,0)$ as the minimal element.
Thus,  $X^P$ is terminating.
There are three families of critical branchings in~$X^P$:
\begin{enumerate}
\item $ (P(u)\alpha[v,w], \alpha[u,v]P(w) )$ with the source $P(u)P(v)P(w)$, 
\item $ (P ( \left.q\right |_{\alpha[u,v]} )P(w), \alpha [ \left.q\right |_{P(u)P(v)}, w ] )$ with the source  $P ( \left.q\right |_{P(u)P(v)} )P(w)$, 
\item $(P(u)P( \left.q\right |_{\alpha[v,w]}), \alpha [u,  \left.q\right |_{P(v)P(w)} ])$  with the source $P(u)P (\left.q\right |_{P(v)P(w)} )$,
\end{enumerate}
which are all confluent by a straightforward computation, similar to Example~\ref{E:ConvergentOfDifferentialAlgebra}. Hence, $X^P$ is convergent.

Now, we construct the set $\Nf(X^P)$. Define the \emph{alternating product} of objects $U$ and $V$ with the operator~$P$ (see also \cite{KuruschGuo}) as
\[
\Lambda_P(U, V) \coloneq \left(\bigcup_{r \geq 0}(U P(V))^r U\right) \cup \left(\bigcup_{r \geq 1}(U P(V))^r\right) \cup \left(\bigcup_{r \geq 0}(P(V) U)^r P(V)\right) \cup \left(\bigcup_{r \geq 1}(P(V) U)^r\right).
\]
We introduce the following notations
\[
 \Phi_0 \coloneq Z^\ast \setminus \{1\}, 
\quad
 \Phi_1 \coloneq \Lambda_P(\Phi_0, Z^\ast),
\quad
\text{ and } \quad\Phi_n \coloneq  \Lambda_P(\Phi_0, \Phi_{_{n-1}}),
\;\;\text{for $n \geq 2$},
\]
and define the set
\[
\Phi_P \coloneq \left(\bigcup_{n \geq 0} \Phi_n\right) \cup \{1\}.
\]
Thus, we have $\Nf(X^P)=\Phi_P$,
since there are no $\Omega$-monomials of the form $\left.q\right|_{P(u)P(v)}$ in $\Nf(X^P)$, for all~$u, v \in Z^\Omega$ and $q\in Z^\Omega[\square]$.

\begin{remark}
As in Example~\ref{D:DiffAutomaton}, we  construct a PDA $\Ar^{\scriptscriptstyle P}$
\begin{eqn}{equation}
\label{D:PDAOfRB}
\raisebox{-1.6cm}{
\begin{tikzpicture}[shorten >=1pt, node distance=2.8cm, on grid, auto]
   \node[state, initial] (q_0)   {$q_0$}; 
   \node[state] (q_1) [right=of q_0] {$q_1$}; 
   \node[state] (q_2) [right=of q_1] {$q_2$};
   \node[state,accepting] (q_3) [above=1.7cm of q1] {$q_3$};
   \node[state,accepting] (q_4) [above=1.7cm of q_2] {$q_4$};

    \path[->] 
    (q_0) edge[below]              node {$\epsilon, \epsilon \fl \$ $} (q_1)
    (q_1) edge [loop below] node {$\mathsf{\ell}_{\scriptscriptstyle P}
, \epsilon \fl P $} ()
          edge[bend right=15] node[below] {$\epsilon, \epsilon \fl \epsilon $} (q_2)
    (q_2) edge [loop below] node {$\mathsf{r}_{\scriptscriptstyle P}
, P \fl \epsilon $} ()
          edge[bend right=15] node[above] {$Z, \epsilon \fl \epsilon $} (q_1);

    \path[->] 
    (q_1) edge[left]              node {$\epsilon, \$ \fl \epsilon $} (q_3);

    \path[->] 
    (q_2)  edge[right]              node {$\epsilon, \$ \fl \epsilon $} (q_4);
\end{tikzpicture}
}
\end{eqn}
which accepts all monomials in $\Phi_P$.  We obtain $\Ar^{\scriptscriptstyle P}$ by modifying $\Ar_\Omega$ to exclude monomials containing the subword $\mathsf{r}{\scriptscriptstyle P} \mathsf{\ell}{\scriptscriptstyle P}$, since $\Phi_P = Z^\Omega \setminus \{\left.q\right|_{P(u)P(v)}\}$ for any $u, v \in Z^\Omega$.
\end{remark}
 
The polygraph $X^P$ is not reduced, as it contains two families of inclusion branchings with sources $P(q|_{P(u)P(v)})P(w)$ and $P(u)P(q|_{P(v)P(w)})$ for all $u, v, w \in Z^\Omega$.

\subsubsection{A reduced presentation} 
\label{SSS:LemmaForReducedRB}
We write $\rep{w}=\Nf(w,X^P)$, for every $w\in Z^\Omega$, and construct a reduced \LO-1-polygraph $\widetilde{X}^P$, which is Tietze equivalent to $X^P$, with 
\[
\widetilde{X_1}^P\coloneq \{ \alpha[u,v]: P(u)P(v)\rightarrow P(\rep{P(u)v})+P(\rep{uP(v)})+\lambda P(\rep{uv})  \mid u,v \in \Phi_P\}.
\]
It follows that $\Nf(X^P) = \Nf(\widetilde{X}^P)=\Phi_P$.
 Indeed, it suffices to prove that  $Q(X^P)=Q(\widetilde{X}^P)$, where
\[
Q(X^P)\coloneq\{\left.q\right|_{P(u)P(v)}|u,v\in Z^\Omega\} \quad \text{ and } \quad
Q(\widetilde{X}^P)\coloneq\{\left.q\right|_{P(u)P(v)}|u,v\in\Phi_P\}.
\]
The inclusion $Q(\widetilde{X}^P) \subset Q(X^P)$ is straightforward.
For any $\left.q_0\right|_{P(u_0)P(v_0)}$ in $Q(X^P)$, if $u_0,v_0\in\Phi_P$, then it belongs to $Q(\widetilde{X}^P)$. Suppose $u_0 \notin \Phi_P$, then there exists a decomposition $u_0 = \left.q_1\right|_{P(u_1)P(v_1)}$. Repeating this process, we eventually obtain $\left.q_0\right|_{P(u_0)P(v_0)}=\left.q_k\right|_{P(u_k)P(v_k)}$ for $u_k,v_k\in\Phi_P$. Thus we conclude that~$Q(X^P) \subset Q(\widetilde{X}^P)$.

\begin{theorem}
\label{T:ResolutionRB} 
The algebra $\RBA{\lambda}{Z}$ has the polygraphic resolution  $\mathrm{Sq}(\widetilde{X}^P)$, where
\begin{enumerate}
\item $\mathrm{Sq}_0(\widetilde{X}^P) :=Z$,
\item $\mathrm{Sq}_n(\widetilde{X}^P) := \{\, P(u_1)|P(u_2)|\cdots|P(u_{n+1}) \;\big|\; u_i\in \Phi_P\,\}$, for every $n\geq1$.
\end{enumerate}
\end{theorem}
\begin{proof}
The proof follows from the fact that $\widetilde{X}^P$ is reduced and convergent, along with \eqref{T:Main conclusion}.
\end{proof}

\subsubsection{Low-dimensional generators of $\mathrm{Sq}(\widetilde{X}^P)$}
\label{SSS:InterpretationForRBResolution}
The following diagrams correspond the $1$-generator $P(u_1) | P(u_2)$, the 2-generators $P(u_1) | P(u_2) | P(u_3)$, and the 3-generators $P(u_1) | P(u_2) | P(u_3) | P(u_4)$ to higher critical branchings
\[
\xymatrix @C=3.5em {
\strut 
\ar@{..>} [r] _-{P(u_1)}
\ar@{..>}@/^7ex/ [rr] ^-{m}="a"
& \strut
\ar@{..>} [r] _{P(u_2)}
& \strut
\ar@1 "1,2";"a"!<0pt,7.5pt> |-{P(u_1)|P(u_2)}
}
\quad
\xymatrix @C=3.7em {
\strut 
\ar@{..>} [r] _-{P(u_1)}
\ar@{..>}@/^7ex/ [rr] ^-{m}="aw"
& \strut
\ar@{..>} [r] ^{P(u_2)}
\ar@{..>}@/_7ex/ [rr] _-{n}="ub"
& \strut
\ar@{..>}[r] ^-{P(u_3)}="w"
& \strut
\ar@1 "1,2";"aw"!<0pt,3pt> |-{P(u_1)|P(u_2)}
\ar@1 "1,3";"ub"!<0pt,7.5pt> |-{P(u_2)|P(u_3)}
}
\quad
\xymatrix @C=4em {
\strut 
\ar@{..>} [r] _-{P(u_1)}
\ar@{..>}@/^7ex/ [rr] ^-{m}="a"
& \strut
\ar@{..>} [r] ^{P(u_2)}
\ar@{..>}@/_7ex/ [rr] _-{n}="b"
& \strut
\ar@{..>}[r] ^-{P(u_3)}="w"
\ar@{..>}@/^7ex/ [rr] ^-{o}="c"
& \strut
\ar@{..>}[r] _-{P(u_4)}="s"
& \strut
\ar@1 "1,2";"a"!<0pt,3pt> |-{P(u_1)|P(u_2)}
\ar@1 "1,3";"b"!<0pt,7.5pt> |-{P(u_2)|P(u_3)}
\ar@1 "1,4";"c"!<0pt,3pt> |-{P(u_3)|P(u_4)}
}
\]
respectively, where $u_1,u_2,u_3,u_4\in \Phi_P$ and 
\[
\begin{aligned}
m&=P(\rep{P(u_1)u_2})+P(\rep{u_1P(u_2)})+\lambda P(\rep{u_1u_2}),\\
n&=P(\rep{P(u_2)u_3})+P(\rep{u_2P(u_3)})+\lambda P(\rep{u_2u_3}),\\
o&=P(\rep{P(u_3)u_4})+P(\rep{u_3P(u_4)})+\lambda P(\rep{u_3u_4}).
\end{aligned}
\]
We also illustrate their shapes in low dimensions.
For the 1-generators $P(u_1) | P(u_2)$, we have
\[
P(u_1)|P(u_2):P(u_1)P(u_2)\fl \rep{P(u_1)P(u_2)}.
\]
By Theorem~\ref{T:Main conclusion}, the 2-generators $P(u_1) | P(u_2) | P(u_3)$ and the 3-generators $P(u_1) | P(u_2) | P(u_3) | P(u_4)$ have the following shapes
\[
\xymatrix @C=3em@R=2em {
& \rep{ab}c
\ar@/^/ [dr] ^-{\sigma_{\rep{ab}c}}
\ar@2 []!<0pt,-10pt>;[d]!<0pt,2.5pt> ^-*+{a|b|c}
\\
abc
\ar@/^/ [ur] ^-{(a|b)c}
\ar@/_/ [rr] _-{\sigma_{abc}}
&& {\rep{abc}}
}
\]
\[
\vcenter{\xymatrix @C=5em @R=4.5em {
{\rep{ab}cd}
	\ar@/^/ [r] ^-{\sigma_{\rep{ab}c}d}
	\ar@2 []!<10pt,-15pt>;[d]!<10pt,25pt> ^(0.1){(a| b| c)d}
& {\rep{abc}d}
	\ar@/^/ [d] ^-{\sigma_{\rep{abc}d}}
	\ar@2 []!<-10pt,-25pt>;[d]!<-10pt,15pt> _(0.9){\sigma_{\sigma_{abc}d}}
\\
abcd
	\ar@/^/ [u] ^-{(a| b)cd}
	\ar@//  [ur] |-*+{\sigma_{abc}d}
	\ar@/_/ [r] _-{\sigma_{abcd}}
& {\rep{abcd}}
}}
\!\!
\overset{\displaystyle a| b| c| d}{\Rrightarrow}
\!\!
\vcenter{\xymatrix @C=5em @R=4.5em {
{\rep{au}vw}
	\ar@/^/ [r] ^-{\sigma_{\rep{ab}c}d}
	\ar@// [dr] |-*+{\sigma_{\rep{ab}cd}}
	\ar@2 []!<10pt,-25pt>;[d]!<10pt,15pt> ^(0.9){\sigma_{(a| b)cd}}
& {\rep{abc}d}
	\ar@/^/ [d] ^-{\sigma_{\rep{abc}d}}
	\ar@2 []!<-10pt,-15pt>;[d]!<-10pt,25pt> _(0.1){\sigma_{\sigma_{\rep{ab}c}d}}
\\
abcd
	\ar@/^/ [u] ^-{(a| b)cd}
	\ar@/_/ [r] _-{\sigma_{abcd}}
& {\rep{abcd}}
}}
\]
where $a,b,c$ and $d$ correspond to $P(u_1)$, $P(u_2)$, $P(u_3)$, and $P(u_4)$, respectively.

\subsection{Polygraphic resolutions of free differential algebras}
\label{SS:ResolutionsOfFreeDifferentialAlgebras}
This subsection presents two polygraphic resolutions of the free differential algebra $\DA{\lambda}{Z}$ on $Z$.

\subsubsection{A reduced presentation} 
\label{SSS:ReducedPolygraphForDA}
The algebra $\DA{\lambda}{Z}$ is presented by the convergent polygraph $X^D$, as defined in \eqref{E:FreeDifferentialAlgebra}. 
We construct a reduced \LO-1-polygraph $\widetilde{X}^D$ with
\begin{equation*}
\begin{aligned}
\widetilde{X_1}^D\coloneq \big\{ \alpha[u_1,u_2,\ldots,u_n]&:D(u_1u_2\cdots u_n)\rightarrow \sum_{\substack{1\leq i_1<\cdots< i_k\leq n \\ 1\leq k\leq n}} \lambda^{k-1}D_{i_1,\ldots, i_k}\left(u_1,\ldots ,u_n\right), \\
\varphi&:D(1)\fl0 \mid u_1,\ldots,u_n \in D^\theta(Z)\setminus \{1\}, n\geq 2 \big\},
\end{aligned}
\end{equation*} 
where $D_{i_1,\ldots, i_k}\left(u_1,\ldots, u_n\right)\coloneq u_1\cdots D(u_{i_1})\cdots D(u_{i_k}) \cdots u_n$.

\begin{lemma} 
\label{L:TwoTietzeEquivalentPolyForDA}
The polygraphs $\widetilde{X}^D$ and $X^D$ are Tietze equivalent.
\end{lemma}
\begin{proof}
The polygraph $\widetilde{X}^D$ is  convergent since it contains no critical branchings. We prove this lemma in two steps.
First, we show that  $\Nf(\widetilde{X}^D)=\Nf(X^D)$. It suffices to prove that  $Q(X^{D})=Q(\widetilde{X}^D)$, where
\[
Q(X^D)\coloneq\{\left.q\right|_{D(uv)}|u,v\in Z^\Omega\setminus \{1\}\} \quad\text{ and }\quad Q(\widetilde{X}^D)\coloneq\{\left.q\right|_{D(u_1 \cdots u_n)}|u_i\in D^\theta(Z)\setminus \{1\},n\geq2\}.
\]
The inclusion $Q(\widetilde{X}^D)\subset Q(X^D)$ is straightforward.
For any $\left.q\right|_{D(u_1\cdots u_n)} \in Q(X^D)$ with $\operatorname{bre}(u_i) = 1$, if all $u_i\in D^\theta(Z)\setminus \{1\}$, then it belongs to $Q(\widetilde{X}^D)$. If not, suppose $u_1 = D(v_1\cdots v_m) \in Q(X^D)$ with $\operatorname{bre}(v_i) = 1$, and repeat this process for $D(v_1\cdots v_m)$. Eventually, we obtain $\left.q\right|_{D(u_1\cdots u_n)} = \left.q'\right|_{D(w_1 \cdots w_k)}$, where $w_1, \dots, w_k \in D^\theta(Z)\setminus \{1\}$, which shows that $ Q(X^D)\subset Q(\widetilde{X}^D)$. 

Next, we prove that $\Nf(w,X^D)=\Nf(w,\widetilde{X}^D)$, for every $w\in Z^\Omega$. 
We first consider the case $w=D(u)$, with $u\in Z^\Omega$. There exists a decomposition $D(u) = D(s_1 \cdots s_n)$, with $\operatorname{bre}(s_i) = 1$, and we proceed by induction on $\max(\operatorname{dep}(s_i))$. For the case $\max(\operatorname{dep}(s_i)) = 0$, meaning that all $s_i \in Z$, we have
\[
D(u)=D(s_1\cdots s_n)\fl \sum_{\substack{1\leq i_1<\cdots< i_k\leq n \\ 1\leq k\leq n}} \lambda^{k-1}D_{i_1,\ldots, i_k}(s_1,\ldots ,s_n)
,
\]
in $X^D$.
It follows that $\Nf(D(u), X^D) = \Nf(D(u), \widetilde{X}^D)$.
Now, we assume that it holds for $\max(\operatorname{dep}(s_i))\leq m$. If $\max(\operatorname{dep}(s_i))=m+1$, we have $s_j = D(v_j)$ with $\operatorname{dep}(v_j) = m$ for some $1\leq v_j\leq n$. By the induction hypothesis, $\Nf(v_j,X^D)=\Nf(v_j,\widetilde{X}^D)$ holds, implying that the normal forms of $D(u)$ are equal in both $X^D$ and $\widetilde{X}^D$.
So, for every $w = w_1 w_2 \cdots w_n \in Z^\Omega$ with $\operatorname{bre}(w_i) = 1$, we have $\Nf(w_i, X^D) = \Nf(w_i, \widetilde{X}^D)$ for every $i$, and thus $\Nf(w, X^D) = \Nf(w, \widetilde{X}^D)$.
Hence, $\widetilde{X}^D$ and $X^D$ are Tietze equivalent.
\end{proof}

\begin{theorem}
\label{T:FirstResolutionOfDA}
The algebra $\DA{\lambda}{Z}$ has the polygraphic resolution  $\mathrm{Sq}(\widetilde{X}^D)$, where
\begin{enumerate}
\item $\mathrm{Sq}_0(\widetilde{X}^D) := Z$,
\item $\mathrm{Sq}_1(\widetilde{X}^D) := \{\, \epsilon|D(1), \epsilon | D(uv) \;\big|\; u,v\in (D^\theta(Z))^\ast\setminus \{1\}\,\}$,
\item $\mathrm{Sq}_n(\widetilde{X}^D)$ is empty, for every $n \geq 2$.
\end{enumerate}
\end{theorem}

If $\lambda \neq 0$, we provide another polygraphic resolution of $\DA{\lambda}{Z}$, similar to $\mathrm{Sq}(\widetilde{X}^P)$.

\subsubsection{A reduced presentation for $\lambda \neq 0$}
When $\lambda\neq0$, we give another presentation $Y^{D}\coloneq\{Z,Y_1^{D}\}$ of $\DA{\lambda}{Z}$ with
\[
Y_1^{D}\coloneq \big\{ \alpha[u,v]: D(u)D(v)\fl \lambda^{-1}\big(D(uv)- D(u)v-u D(v)\big),\; \varphi:D(1)\fl 0 \mid u,v \in Z^\Omega\setminus \{1\}\big\}.
\]
The termination of $Y^D$ follows from the decrease of the number of operators when applying the rules $\alpha[u,v]$ and $\varphi$.
This polygraph is also confluent, as it has five families of critical branchings
\begin{enumerate}
\item $ (D(u)\alpha[v,w], \alpha[u,v]D(w) )$, with the source $D(u)D(v)D(w)$, 
\item $ (D ( \left.q \right |_{\alpha[u,v]} )D(w), \alpha [ \left.q \right |_{D(u)D(v)}, w ] )$, with the source  $D ( \left.q \right |_{D(u)D(v)} )D(w)$, 
\item $ (D(u)D ( \left.q \right |_{\alpha[v,w]} ), \alpha [u,  \left.q \right |_{D(v)D(w)} ] )$,  with the source $D(u)D ( \left.q \right |_{D(v)D(w)} )$,
\item  $ (D ( \left.q \right |_{\varphi} )D(w), \alpha [ \left.q \right |_{D(1)}, w ] )$, with the source  $D ( \left.q \right |_{D(1)} )D(w)$,
\item $ (D(u)D ( \left.q \right |_{\varphi} ), \alpha [u,  \left.q \right |_{D(1)} ] )$,  with the source $D(u)D ( \left.q \right |_{D(1)} )$,
\end{enumerate}
all of which can be verified as confluent through straightforward computation. Define the alternating product of $U$ and $V$ with operator $D$ as
\[
\Lambda_D(U, V) \coloneq \left(\bigcup_{r \geq 0}(U D(V))^r U\right) \cup \left(\bigcup_{r \geq 1}(U D(V))^r\right) \cup \left(\bigcup_{r \geq 0}(D(V) U)^r D(V)\right) \cup \left(\bigcup_{r \geq 1}(D(V) U)^r\right).
\]
We introduce the notations $\Phi_0 \coloneq Z^\ast \setminus \{1\}$ and $\Phi_n \coloneq \Lambda_D(\Phi_0, \Phi_{n-1})$ for $n \geq 1$, and define the set
\[
\Phi_D \coloneq \left(\bigcup_{n \geq 0} \Phi_n\right) \cup \{1\}.
\] 
Thus, we have $\Nf(Y^{D})=\Phi_D$.
Note that the construction of $\Phi_D$ differs from that of $\Phi_P$ in \eqref{SSS:PolygraphForRB}, as~$\Phi_1 = \Lambda_D(\Phi_0, Z^\ast \setminus \{1\})$, which implies $D(1) \notin \Phi_D$.
Next, we write $\rep{w}=\Nf(w,Y^D)$ for every $w\in Z^\Omega$ and present a reduced convergent polygraph $\widetilde{Y}^D$, which is Tietze equivalent to $Y^D$, with
\[
\widetilde{Y}_1^D \coloneq \big\{ \alpha[u,v]: D(u)D(v)\rightarrow \lambda^{-1}\big(\rep{D(u)v}- \rep{uD(v)}-D(\rep{uv})\big),\;\varphi:D(1)\fl 0 \mid u, v \in \Phi_D\setminus \{1\} \big\}.
\]
Similarly to the explanation of $\Nf(X^P)$ and $\Nf(\widetilde{X}^P)$ in \eqref{SSS:LemmaForReducedRB}, we have $\Nf(Y^{D}) = \Nf(\widetilde{Y}^D)=\Phi_D$.

\begin{theorem}
When $\lambda\neq0$, the algebra $\DA{\lambda}{Z}$ has the polygraphic resolution  $\mathrm{Sq}(\widetilde{Y}^D)$, where 
\begin{enumerate}
\item $\mathrm{Sq}_0(\widetilde{Y}^D) :=Z$,
\item $\mathrm{Sq}_1(\widetilde{Y}^D) := \{\, \epsilon|D(1), D(u_1)|D(u_2)\; \big| \; u_i\in \Phi_D\setminus \{1\}\,\}$,
\item $\mathrm{Sq}_n(\widetilde{Y}^D) := \{\, D(u_1)|D(u_2)|\cdots|D(u_{n+1})\; \big| \; u_i\in \Phi_D\setminus~\{1\}\,\}$, for every $n \geq 2$.
\end{enumerate}
\end{theorem}

\subsection{Polygraphic resolutions of free differential Rota-Baxter algebras}

In this subsection, we consider the case with multiple operators and construct a polygraphic resolution of the free differential Rota-Baxter algebra $\DRBA{\lambda}{Z}$ on $Z$, assuming that $\lambda \neq 0$.

\subsubsection{A presentation of $\DRBA{\lambda}{Z}$}
\label{SSS:DRBPresentation}
The algebra $\DRBA{\lambda}{Z}$ is presented by
the \LO-1-polygraph $X$ with
\begin{equation*}
\begin{aligned}
X_1 \coloneq \{  \alpha[u,v]&:P(u)P(v) \rightarrow P(P(u)v) + P(uP(v)) + \lambda P(uv), \\
\beta[w_1,w_2]&: D(w_1) D(w_2)\rightarrow \lambda^{-1} \big(D(w_1w_2)- D(w_1)w_2- w_1 D(w_2)\big), \\
 \gamma[u]&:D(P(u)) \rightarrow u, \\
\varphi&:D(1)\rightarrow 0\mid u,v,w_1,w_2 \in Z^\Omega \text{ and } w_1,w_2\neq1 \}.
\end{aligned}
\end{equation*} 
We set $N=\mathbb{Z}^2$ and define a derivation $d : Z^\Omega \longrightarrow N$ as in~\eqref{SSS:PolygraphForRB}, by setting
\[
d(u) = \bigg( \operatorname{deg}_\Omega(u), \sum_{P|u} \big(\operatorname{deg}_\Omega(u) - \operatorname{deg}_\Omega(P)\big) 
+ \sum_{D|u} \big(\operatorname{deg}_\Omega(u) - \operatorname{deg}_\Omega(D)\big) \bigg),
\]
for every $u \in Z^\Omega$.
Moreover, we define a bimodule structure by setting 
\begin{align*}
v\cdot(m_1,m_2)=(m_1 ,m_2)\cdot v &=(m_1,m_2+n m_1),\\
P((m_1, m_2))
=(m_1+1, m_2+m_1+1), &
\quad
D((m_1, m_2))
=(m_1+1, m_2+m_1+1),
\end{align*}
for all $(m_1,m_2)\in N$ and $v\in Z^\Omega$ with $\operatorname{deg}_\Omega(v)=n$.
It follows that $d$ satisfies \eqref{E:Derivation}. By equipping $d(Z^\Omega)$ with a monotone  lexicographic order, where $d(1) = (0,0)$ is the minimal element, we ensure that $X$ is terminating.

The critical branchings of $X$ are all confluent (see Appendix), except for the following two cases
\[
\xymatrix @R=0.5em {
& uD(v) \quad\quad\quad\quad \quad\quad\quad\quad\quad\quad\quad\quad\quad\quad\quad\quad
\\
D(P(u))D(v)
\ar@/^3ex/ [ur] ^-{\gamma[u] D(v)}
\ar@/_3ex/ [dr] _-{\beta[P(u),v]}
\\
& \lambda^{-1} D(P(u)v)-\lambda^{-1} D(P(u))v-\lambda^{-1} P(u)D(v)
}
\]
\[
\xymatrix @R=0.5em {
& D(v)w \quad\quad\quad\quad \quad\quad\quad\quad\quad\quad\quad\quad\quad\quad\quad\quad
\\
D(v)D(P(w))
\ar@/^3ex/ [ur] ^-{D(v)\gamma[w]}
\ar@/_3ex/ [dr] _-{\beta[v,P(w)]}
\\
& \lambda^{-1} D(vP(w))-\lambda^{-1} D(v)P(w)-\lambda^{-1} vD(P(w)).
}
\]
where $u,v,w\in Z^\Omega$ and $v\neq1$.
By the completion procedure in \eqref{SSS:CompletionProcedure} and the derivation $d$ above, we complete the polygraph $X$ into a convergent one, denoted $X^{PD}$, where
\[
\begin{aligned}
X_1^{PD} \coloneq \{  \alpha[u,v]&:P(u)P(v) \rightarrow P(P(u)v) + P(uP(v)) + \lambda P(uv), \\
\beta[w_1,w_2]&: D(w_1) D(w_2)\rightarrow \lambda^{-1} \big(D(w_1w_2)- D(w_1) w_2- w_1 D(w_2)\big), \\
 \gamma[u]&:D(P(u)) \rightarrow u, \\
 \delta_1[u,w_2]&:P(u)D(w_2) \rightarrow P(D(u)w_2) + uw_2 + \lambda uD(w_2),\\
 \delta_2[w_1,v]&:D(w_1)P(v) \rightarrow D(w_1P(v)) + w_1v + \lambda D(w_1)v,\\
\varphi&:D(1)\rightarrow 0\mid u,v,w_1,w_2 \in Z^\Omega \text{ and } w_1,w_2\neq1 \}.
\end{aligned}
\]
We list all critical branchings of $X^{PD}$ in the appendix, which are all confluent.

\subsubsection{Normal forms}
Let us denote $L^{ij}(u) \coloneq P^i\left(D^j(u)\right)$, for every $u \in Z^\Omega$. 
We define $P^0(1) \coloneq 1$ and 
\[
\begin{aligned} 
\Lambda_{PD}(U, V) \: \coloneq \: & \left(\bigcup_{i, j, k\geq0;  r \geq 0}P^k(1)\left(U L^{ij}\left(V\right)\right)^r U P^k(1)\right) \cup\left(\bigcup_{ i, j, k \geq 0; r \geq 1}P^k(1)\left(U L^{ij}\left(V\right)\right)^r\right) \\ & \cup\left(\bigcup_{i, j, k \geq 0; r \geq 1}\left(L^{ij}\left(V\right) U\right)^r L^{ij}\left(V\right)\right) \cup\left(\bigcup_{i, j, k\geq 0; r \geq 1 }\left(L^{ij}\left(V\right) U\right)^r P^k(1)\right)
\end{aligned}
\]
We introduce the notations $Z^+\coloneq Z^\ast \setminus \{1\}$, and we set
\[
\Phi_0\coloneq\bigcup_{k\geq 0; r \geq 0}\left(Z^+ P^k(1)\right)^r Z^+
\]
Inductively, we define $\Phi_1 \coloneq \Lambda_{PD}(\Phi_0, Z^+)$ and
 $\Phi_n \coloneq \Lambda_{PD}(\Phi_0, \Phi_{n-1})$ for $n > 1$. Finally, we  set
\[
\Phi\coloneq\bigcup_{n \geq 0} \Phi_n, \quad
\Phi_{PD}\coloneq L^{ij}(\Phi)\cup P^k(1)
\]
for all $i,j,k\geq 0$.
Thus, $\Nf(X^{PD}) = \Phi_{PD}$.
Here, the construction of $\Lambda_{PD}(U, V)$ differs from that of $\Lambda_{P}(U, V)$ and $\Lambda_{D}(U, V)$, as it is designed to exclude $\Omega$-monomials of the form $q|_{D(P(u))}$ from $\Phi_{PD}$.

\begin{remark}
As in \eqref{D:PDAOfDiff} and \eqref{D:PDAOfRB}, 
we construct a PDA $\Ar^{\scriptscriptstyle PD}$ that accepts the monomials in~$\Phi_{PD}$
\begin{eqn}{equation}
\label{D:PDAOfDRB}
\raisebox{-1.7cm}{
\begin{tikzpicture}[shorten >=1pt, node distance=3cm, on grid, auto]
   % States
   \node[state, initial] (q_0) {$q_0$}; 
   \node[state] (q_1) [right=3.1cm of q_0] {$q_1$}; 
   \node[state] (q_2) [right=of q_1] {$q_2$}; 
   \node[state] (q_3) [right=of q_2] {$q_3$};
   \node[state] (q_4) [below=2cm of q_3] {$q_4$};
   \node[state] (q_5) [below=2cm of q_1] {$q_5$};
   \node[state,accepting] (q_6) [left=3.1cm of q_5] {$q_6$};

   % Loops
   \path[->]
   (q_2) edge[loop above] node {$\mathsf{\ell}_{\scriptscriptstyle D}
$} ()
   (q_3) edge[loop above] node {$\mathsf{\ell}_{\scriptscriptstyle P}
$; Z} ()
   (q_4) edge[loop right] node {$\mathsf{r}_{\scriptscriptstyle P}
$} ()
   (q_5) edge[loop right] node[] {$\mathsf{r}_{\scriptscriptstyle P}
$; $\mathsf{r}_{\scriptscriptstyle D}
$} ()
   (q_1) edge[loop above] node {$\mathsf{\ell}_{\scriptscriptstyle P}
$} ();

   % Transitions
   \path[->]
   (q_0) edge[left] node[above] {$\epsilon, \epsilon \fl \$$} (q_1)
   (q_1) edge[bend left=15] node[above] {$\mathsf{\ell}_{\scriptscriptstyle D}
$} (q_2)
   (q_1) edge[bend left=15] node[] {$\epsilon$} (q_5)
   (q_5) edge[bend left=15] node[] {$Z$} (q_1)
   (q_2) edge[bend left=15] node[above] {$\mathsf{\ell}_{\scriptscriptstyle P}
$} (q_3)
   (q_2) edge[bend left=15] node[] {$Z$} (q_1)
   (q_3) edge[bend left=15] node[below] {$\mathsf{\ell}_{\scriptscriptstyle D}
$} (q_2)
   (q_3) edge[left] node[right] {$\mathsf{r}_{\scriptscriptstyle P}
$} (q_4)
   (q_4) edge[bend left=18] node[below] {Z} (q_1)
   (q_5) edge[left] node[below] {$\epsilon, \$ \fl \epsilon $} (q_6)
   (q_1) edge[left] node[] {$\epsilon, \$ \fl \epsilon $} (q_6);
\end{tikzpicture}
}
\end{eqn}
For simplicity, we display only the input symbols in certain instructions while omitting the corresponding stack operations
\[
\epsilon, \epsilon \fl \epsilon;\quad Z, \epsilon \fl \epsilon;\quad \mathsf{\ell}_{\scriptscriptstyle D}
, \epsilon \fl D; \quad\mathsf{\ell}_{\scriptscriptstyle P}
, \epsilon \fl P;\quad \mathsf{r}_{\scriptscriptstyle P}
, P \fl \epsilon;\quad \mathsf{r}_{\scriptscriptstyle D}
, D \fl \epsilon.
\]
From the construction of $X_1^{PD}$, it suffices to modify $\Ar_\Omega$ to exclude monomials containing the subwords $\mathsf{r}_{\scriptscriptstyle P}
\mathsf{\ell}_{\scriptscriptstyle P}
, \mathsf{r}_{\scriptscriptstyle D}
\mathsf{\ell}_{\scriptscriptstyle D}
, \mathsf{r}_{\scriptscriptstyle P}
\mathsf{\ell}_{\scriptscriptstyle D}
, \mathsf{r}_{\scriptscriptstyle D}
\mathsf{\ell}_{\scriptscriptstyle P}
, \mathsf{\ell}_{\scriptscriptstyle D}
\mathsf{r}_{\scriptscriptstyle D}
$, and $\mathsf{\ell}_{\scriptscriptstyle D}
\mathsf{\ell}_{\scriptscriptstyle P}
 u\mathsf{r}_{\scriptscriptstyle P}
\mathsf{r}_{\scriptscriptstyle D}
$, where $u\in\{\mathsf{\ell}_{\scriptscriptstyle P}
,\mathsf{r}_{\scriptscriptstyle P}
,\mathsf{\ell}_{\scriptscriptstyle D}
,\mathsf{r}_{\scriptscriptstyle D}
,Z\}^\ast$.
\end{remark}

\subsubsection{A reduced presentation}
\label{SSS:ReducedDR}
We write $\rep{w}=\Nf(w,X^{PD})$ for every $w\in Z^\Omega$ and construct a reduced presentation $\widetilde{X}^{PD} $, which is Tietze equivalent to $X^{PD}$, with
\[
\begin{aligned}
\widetilde{X}_1^{PD} \coloneq \{  \alpha[u,v]&:P(u)P(v) \rightarrow P(\rep{P(u)v}) + P(\rep{uP(v)}) + \lambda P(\rep{uv}), \\
\beta[u,v]&: D(w_1) D(w_2)\rightarrow \lambda^{-1} \big(\rep{D(w_1w_2)}- \rep{D(w_1) w_2}- \rep{w_1 D(w_2)}\big), \\
\gamma[u]&:D(P(u)) \rightarrow u, \\
 \delta_1[u,w_2]&:P(u)D(w_2) \rightarrow P(\rep{D(u)w_2}) + \rep{uw_2} + \lambda \rep{uD(w_2)},\\
 \delta_2[w_1,v]&:D(w_1)P(v) \rightarrow \rep{D(w_1P(v))} + \rep{w_1v} + \lambda \rep{D(w_1)v},\\
\varphi&:D(1)\rightarrow 0\mid u,v,w_1,w_2 \in \Phi_{PD} \text{ and } w_1,w_2\neq1 \}.
\end{aligned}
\]
It follows that $\Nf(X^{PD}) = \Nf(\widetilde{X}^{PD})=\Phi_{PD}$.

\begin{theorem}
When $\lambda\neq0$, the algebra $\DRBA{\lambda}{Z}$ has the polygraphic resolution  $\mathrm{Sq}(\widetilde{X}^{PD})$, where
\begin{enumerate}
\item $\mathrm{Sq}_0(\widetilde{X}^{PD}) := Z$,
\item $\mathrm{Sq}_1(\widetilde{X}^{PD}) := \{\,\epsilon|D(1),\; \epsilon|D(P(u_0)), \; R(u_1)|R(u_2) \;\big|\; R\in \{P,D\}, \; u_0, R(u_i)\in \Phi_{PD}\,\}$,
\item $\mathrm{Sq}_n(\widetilde{X}^{PD}) := \{\,R(u_1)|R(u_2)|\cdots|R(u_{n+1})\; \big|\; R(u_i)\in\Phi_{PD}\,\}$, for every $n \geq 2$.
\end{enumerate}
\end{theorem}

\subsubsection*{Acknowledgements} This work was supported by the China Scholarship Council (CSC) under Grant No. 202406140026. 

\begin{small}
\renewcommand{\refname}{\Large\textsc{References}}
\bibliographystyle{plain}
\bibliography{biblioCURRENT}
\end{small}

\clearpage
\section*{Appendix}
The sources of all critical branchings of the polygraph $X^{PD}$ in \eqref{SSS:DRBPresentation} are listed below. Here, we denote by $\alpha \wedge \beta$ the set of sources for critical branchings of the form $(\alpha[u, v], \beta[u, v])$, with similar conventions for other notations. This enumeration is similar to that presented in {\cite[Thm.3.7]{LQQZ}}, which studies the GS bases theory of free differential Rota–Baxter algebras. For all $u,v,w\in Z^\Omega$, $s,t,r\in Z^\Omega\setminus \{1\}$, and $q \in Z^\Omega[\square]$, we have
\begin{itemize}
	\item [$\alpha \wedge \alpha$] \quad $P(u)P(v)P(w)$, \quad $P\left(\left.q\right|_{P(u)P(v)} \right)P\left(w\right)$, \quad  $P\left(u\right) P\left(\left.q\right|_{P(v)P(w)}\right)$

	\item [$\alpha \wedge \beta$] \quad $P\left(\left.q\right|_{D(s)D(t)} \right)P\left(u\right)$, \quad $P\left(u\right) P\left(\left.q\right|_{D(s)D(t)}\right)$,

	\item [$\alpha \wedge \gamma$] \quad $P\left(\left.q\right|_{D(P(u))} \right)P\left(v\right)$, \quad $P\left(u\right) P\left(\left.q\right|_{D(P(v))}\right)$,

		\item [$\alpha \wedge \delta_1$] \quad $P(u)P(v)D(s)$,\quad $P\left(\left.q\right|_{P(u)D(s)} \right)P\left(v\right)$, \quad $P\left(u\right) P\left(\left.q\right|_{P(v)D(s)}\right)$,

		\item [$\alpha \wedge \delta_2$] \quad $P\left(\left.q\right|_{D(s)P(u)} \right)P\left(v\right)$, \quad $P\left(u\right) P\left(\left.q\right|_{D(s)P(v)}\right)$,

		\item [$\alpha \wedge \varphi$] \quad $P\left(\left.q\right|_{D(1)} \right)P\left(w\right)$, \quad $P\left(u\right) P\left(\left.q\right|_{D(1)}\right)$,

\item [$\beta \wedge \alpha$] \quad $D\left(\left.q\right|_{P(u)(P(v)}\right)D\left(s\right)$,\quad $D\left(s\right)D\left(\left.q\right|_{P(u)P(v)}\right)$,
	
	\item [$\beta \wedge \beta$] \quad$D(s)D(t)D(r)$ , \quad $D\left(\left.q\right|_{D(s)D(t)}\right)D\left(r\right)$,\quad $D\left(s\right)D\left(\left.q\right|_{D(t)D(r)}\right)$,

		\item [$\beta \wedge \gamma$] \quad $D(P(u))D(s)$, \quad $D(s)D(P(u))$,\quad $D\left(\left.q\right|_{D(P(u))}\right)D\left(s\right)$,\quad $D\left(s\right)D\left(\left.q\right|_{D(P(u))}\right)$,

		\item [$\beta \wedge \delta_1$] \quad  $D\left(\left.q\right|_{P(u)D(s)}\right)D\left(t\right)$,\quad $D\left(s\right)D\left(\left.q\right|_{P(u)D(t)}\right)$,

	
		\item [$\beta \wedge \delta_2$] \quad $D(s)D(t)P(u)$,\quad  $D\left(\left.q\right|_{D(s)P(u)}\right)D\left(t\right)$,\quad $D\left(s\right)D\left(\left.q\right|_{D(t)P(u)}\right)$,

		\item [$\beta \wedge \varphi_2$] \quad  $D\left(\left.q\right|_{D(1)}\right)D\left(s\right)$,\quad $D\left(s\right)D\left(\left.q\right|_{D(1)}\right)$,

		\item [$\gamma \wedge \alpha$] \quad $D\left(P\left(\left.q\right|_{P(u)P(v)}\right)\right)$,
	
		\item [$\gamma \wedge \beta$] \quad $D\left(P\left(\left.q\right|_{D(s)D(t)}\right)\right)$,

		\item [$\gamma \wedge \gamma$] \quad  $D\left(P\left(\left.q\right|_{D(P(u))}\right)\right)$,

		\item [$\gamma \wedge \delta_1$] \quad $D\left(P\left(\left.q\right|_{P(u)D(s)}\right)\right)$,

		\item [$\gamma \wedge \delta_2$] \quad $D\left(P\left(\left.q\right|_{D(s)P(u)}\right)\right)$,

		\item [$\gamma \wedge \varphi$] \quad $D\left(P\left(\left.q\right|_{D(1)}\right)\right)$,

		\item [$\delta_1 \wedge \alpha$] \quad $P\left(\left.q\right|_{P(u)P(v)}\right)D\left(s\right)$,\quad $P\left(u\right)D\left(\left.q\right|_{P(v)P(w)}\right)$,

\item [$\delta_1 \wedge \beta$] \quad $P(u)D(s)D(t)$, \quad $P\left(\left.q\right|_{D(s)D(t)}\right)D\left(u\right)$,\quad $P\left(u\right)D\left(\left.q\right|_{D(s)D(t)}\right)$,


    \item [$\delta_1 \wedge \gamma$] \quad $P(u)D(P(v))$, \quad $P\left(\left.q\right|_{D(P(u))}\right)D\left(s\right)$,\quad $P\left(u\right)D\left(\left.q\right|_{D(P(v))}\right)$,

        \item [$\delta_1 \wedge \delta_1$] \quad $P\left(\left.q\right|_{P(u)D(s)}\right)D\left(t\right)$,\quad $P\left(u\right)D\left(\left.q\right|_{P(v)D(s)}\right)$,

	  \item [$\delta_1 \wedge \delta_2$] \quad $P(u)D(s)P(v)$, \quad $P\left(\left.q\right|_{D(s)P(u)}\right)D\left(t\right)$,\quad $P\left(u\right)D\left(\left.q\right|_{D(s)P(v)}\right)$,

	\item [$\delta_1 \wedge \varphi$] \quad $P\left(\left.q\right|_{D(1)}\right)D\left(s\right)$,\quad $P\left(u\right)D\left(\left.q\right|_{D(1)}\right)$,

		\item [$\delta_2 \wedge \alpha$] \quad $D(s)P(u)P(v)$, \quad $D\left(\left.q\right|_{P(u)P(v)}\right)P\left(w\right)$,\quad $D\left(s\right)P\left(\left.q\right|_{P(u)P(v)}\right)$,



		\item [$\delta_2 \wedge \beta$] \quad $D\left(\left.q\right|_{D(s)D(t)}\right)P\left(u\right)$,\quad $D\left(s\right)P\left(\left.q\right|_{D(t)D(r)}\right)$,


		\item [$\delta_2 \wedge \gamma$] \quad $D(P(u))P(v)$,\quad $D\left(\left.q\right|_{D(P(u))}\right)P\left(v\right)$,\quad $D\left(s\right)P\left(\left.q\right|_{D(P(u))}\right)$,


\item [$\delta_2 \wedge \delta_1$] \quad $D(s)P(u)D(t)$,\quad $D\left(\left.q\right|_{P(u)D(s)}\right)P\left(v\right)$,\quad $D\left(s\right)P\left(\left.q\right|_{P(u)D(t)}\right)$,


\item [$\delta_2 \wedge \delta_2$] \quad $D\left(\left.q\right|_{D(s)P(u)}\right)P\left(v\right)$,\quad $D\left(s\right)P\left(\left.q\right|_{D(t)P(u)}\right)$,

\item [$\delta_2 \wedge \varphi$] \quad $D\left(\left.q\right|_{D(1)}\right)P\left(u\right)$,\quad $D\left(s\right)P\left(\left.q\right|_{D(1)}\right)$.
 \end{itemize}

\clearpage

\quad

\vfill

\begin{footnotesize}

\bigskip
\auteur{Zuan Liu}{zliu@math.univ-lyon1.fr}
{Université Claude Bernard Lyon 1\\
CNRS, Institut Camille Jordan, UMR5208\\
43 blvd. du 11 novembre 1918\\
F-69622 Villeurbanne cedex, France}

\bigskip
\auteur{Philippe Malbos}{malbos@math.univ-lyon1.fr}
{Université Claude Bernard Lyon 1\\
CNRS, Institut Camille Jordan, UMR5208\\
43 blvd. du 11 novembre 1918\\
F-69622 Villeurbanne cedex, France}
\end{footnotesize}

\vspace{1.5cm}

\begin{small}---\;\;\today\;\;-\;\;\hhmm\;\;---\end{small}
\end{document}
