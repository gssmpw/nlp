\section{Propositions}
\begin{lemma}
    \label{lemma:x-lnx}
    For any $a, x > 0$, $x \geq 2a\ln{a}$ implies $x \geq a\ln{x}$.
\end{lemma}
%
\begin{proof}
First we prove that for $\forall x > 0$, $x > 2\ln{x}$.

Consider the function $y = x - 2\ln{x}$. It is enough to prove that $y > 0$ is always true on its domain. we have that
%
\begin{align*}
    & \frac{dy}{dx} = 1 - \frac{2}{x}\\
    & \frac{d^2y}{d^2x} = \frac{2}{x^2} > 0
\end{align*}
%
So $y = x - \ln{x}$ is convex. Also by:
%
\begin{equation*}
    \frac{dy}{dx} = 1 - \frac{2}{x} = 0 \Rightarrow x = 2
\end{equation*}
%
The minimum of $y = x - \ln{x} = 2 - \ln{2} > 0$. Thus $x > 2\ln{x}, \forall x > 0$.

Back to the inequalities in the lemma, let $z = \frac xa$, then we need to prove that $z \geq 2\ln{a}$ implies $z \geq \ln{(za)}$.
%
\begin{itemize}
    \item If $a \geq z$, then:
    %
    \begin{equation*}
        \ln{za} = \ln{z} + \ln{a} \leq 2 \ln{a} \leq z
    \end{equation*}
    %
    Which is by the assumption of $z \geq 2\ln{a}$.
    \item  If $a < z$, then:
    %
    \begin{equation*}
        \ln{a} < \ln{z} \Rightarrow \ln{za} \leq 2\ln{z} < z
    \end{equation*}
    %
    which is by our proof in the beginning that $z \geq 2\ln{z}, \forall z > 0$. \qedhere
\end{itemize}
\end{proof}
%
\begin{lemma}
    \label{lemma:summation}
        Let $\Omega$ be an outcome space, and each of $(A_i)^n_{i = 1}$ and $(B_i)^n_{i = 1}$ be $n$ random variables on $\Omega$. It holds that:
        \begin{equation*}
            \left\{\sum_{i=1}^{n}A_i \geq \sum_{i=1}^{n}B_i \right\} \subseteq \left\{\bigcup^n_{i = 1}(A_i \geq B_i)\right\}
        \end{equation*}
    \end{lemma}
%
\begin{proof}
        Proof by contradiction.
        
        Suppose $\left\{\sum_{i=1}^{n}A_i \geq \sum_{i=1}^{n}B_i \right\} \supset \left\{\bigcup^n_{i = 1}(A_i \geq B_i)\right\}$, then it means there exists an $\omega \in \Omega$ such that $\sum_{i=1}^{n}A_i(\omega) \geq \sum_{i=1}^{n}B_i(\omega)$ but $A_1(\omega) < B_1(\omega), A_2(\omega) < B_2(\omega), \cdots A_n(\omega) < B_n(\omega)$ that results in $\sum_{i=1}^{n}A_i(\omega) < \sum_{i=1}^{n}B_i(\omega)$ which is a contradiction. \qedhere
    \end{proof}
%
\begin{corollary}
    \label{corollary:union_bound}
        Let $(A_i)^n_{i = 1}$ and $(B_i)^n_{i = 1}$ be $n$ random variables on probability space $(\Omega, \mathcal{F}, \mathbb{P})$. It holds that:
        \begin{equation*}
            \prob{\sum_{i=1}^{n}A_i \geq \sum_{i=1}^{n}B_i} \leq \sum_{i = 1}^{n}\prob{A_i \geq B_i}
        \end{equation*}
        \begin{proof}
            Using~\cref{lemma:summation} and due to monotonicity of measures we have:
            \begin{equation*}
            \prob{\sum_{i=1}^{n}A_i \geq \sum_{i=1}^{n}B_i } \leq \prob{\bigcup^n_{i = 1}(A_i \geq B_i)}
        \end{equation*}
        By applying the union bound the inequality is obtained.
        \end{proof}
    \end{corollary}
%
\begin{lemma}
\label{lemma:general_closeness}
    Let $M_1 = \angle{\c{S}, \c{A}, P_1, r_1, \mon{f}, \gamma}$ and $M_2 = \angle{\c{S}, \c{A}, P_2, r_2, \mon{f}, \gamma}$ be two Mon-MDPs where $r_1 = (\env{r}_1, \mon{r}_1)$, $R_2 = (\env{r}_2, \mon{r}_2)$. Assume the following bounds hold:
    %
    \begin{equation*}
        -\env{\rmax} \leq \env{r}_1, \env{r}_2 \leq \env{\rmax} \text{ and } -\mon{\rmax} \leq \mon{r}_1, \mon{r}_2 \leq \mon{\rmax}
    \end{equation*}
    %
    Additionally assume the following conditions hold for all joint state-action pairs $(s, a) \equiv (\env{s}, \mon{s}, \env{a}, \mon{a})$: 
    %
    \begin{align*}
    \abs{\env{r}_1(\env{s}, \env{a}) - \env{r}_2(\env{s}, \env{a})} & \leq \env{\varphi} \\
    \abs{\mon{r}_1(\mon{s}, \mon{a}) - \mon{r}_2(\mon{s}, \mon{a})} & \leq \mon{\varphi} \\
    \oneNorm{P_1(\cdot \mid s, a) - P_2(\cdot \mid s, a)} & \leq \varphi
    \end{align*}
    %
    then for any stationary deterministic policies $\pi$ it holds that:
    %
    \begin{equation*}
        \abs{Q^{M_1}_\pi(s, a) - Q^{M_2}_\pi(s, a)} \leq \frac{\env{\varphi} + \mon{\varphi} + 2\varphi\gamma(\env{\rmax} + \mon{\rmax})}{(1 - \gamma)^2}
    \end{equation*}
    %
\end{lemma}
%
\begin{proof}
Let \\
%
$\Delta \coloneqq \max_{(s, a)}\abs{Q^{M_1}_\pi(s, a) - Q^{M_2}_\pi(s, a)}, \\
%
\Delta_1 \coloneqq \env{r}_1(\env{s}, \env{a}) - \env{r}_2(\env{s}, \env{a}), \\
%
\Delta_2 \coloneqq \mon{r}_1(\mon{s}, \mon{a}) - \mon{r}_2(\mon{s}, \mon{a})$ and \\
%
$\Delta_3 \coloneqq \gamma\sum_{s'}P_1(s' \mid s, a)V^{M_1}_\pi(s') - \gamma\sum_{s'}P_2(s' \mid s, a)V^{M_2}_\pi(s')$.
%

Then:
%
    \begin{align*}
       & \abs{Q^{M_1}_\pi(s, a) - Q^{M_2}_\pi(s, a)} = \abs{\Delta_1 + \Delta_2 + \Delta_3} 
        \leq \abs{\Delta_1} + \abs{\Delta_2} + \abs{\Delta_3} \\
       %
       & \leq \env{\varphi} + \mon{\varphi} + \gamma \abs{\sum_{s'}\left(P_1(s' \mid s, a)V^{M_1}_\pi(s') - P_2(s' \mid s, a)V^{M_2}_\pi(s')\right)} \\
       %
       & = \env{\varphi} + \mon{\varphi} + \gamma \abs{\sum_{s'}\left(P_1(s' \mid s, a)V^{M_1}_\pi(s') - P_1(s' \mid s, a)V^{M_2}_\pi(s')+ P_1(s' \mid s, a)V^{M_2}_\pi(s') - P_2(s' \mid s, a)V^{M_2}_\pi(s')\right)} \\
       %
       & = \env{\varphi} + \mon{\varphi} + \gamma \abs{\sum_{s'}\left(P_1(s' \mid s, a)\left(V^{M_1}_\pi(s') - V^{M_2}_\pi(s')\right)+ \left(P_1(s' \mid s, a) - P_2(s' \mid s, a)\right)V^{M_2}_\pi(s')\right)} \\
       %
       & \leq \env{\varphi} + \mon{\varphi} + \gamma \abs{\sum_{s'}P_1(s' \mid s, a)\left(V^{M_1}_\pi(s') - V^{M_2}_\pi(s')\right) }+ \gamma\abs{\sum_{s'}\left(P_1(s' \mid s, a) - P_2(s' \mid s, a)\right)V^{M_2}_\pi(s')} \\
       %
       & \leq \env{\varphi} + \mon{\varphi} + \gamma \abs{\sum_{s'}P_1(s' \mid s, a)\left(V^{M_1}_\pi(s') - V^{M_2}_\pi(s')\right) } + \frac{2\gamma\varphi\left(\env{\rmax} + \mon{\rmax} \right)}{1 - \gamma}
       % &
       % \leq \env{\varphi} + \mon{\varphi} + \gamma \Delta + \frac{2\gamma\varphi\left(\env{\rmax} + \mon{\rmax} \right)}{1 - \gamma}
    \end{align*}
    %
    By taking the $\max_{(s, a)}$ from the both sides we have:
    %
    \begin{align*}
        \Delta & \leq \env{\varphi} + \mon{\varphi} + \gamma \Delta + \frac{2\gamma\varphi\left(\env{\rmax} + \mon{\rmax} \right)}{1 - \gamma} \\
        %
        (1 - \gamma) \Delta & \leq \env{\varphi} + \mon{\varphi} + \frac{2\gamma\varphi\left(\env{\rmax} + \mon{\rmax} \right)}{1 - \gamma} \\
        %
        \Delta & \leq \frac{\env{\varphi} + \mon{\varphi}}{1 - \gamma} + \frac{2\gamma\varphi\left(\env{\rmax} + \mon{\rmax} \right)}{(1 - \gamma)^2} \\
        %
        \Delta & \leq \frac{\env{\varphi} + \mon{\varphi} + 2\gamma\varphi(\env{\rmax} + \mon{\rmax})}{(1 - \gamma)^2}
    \end{align*}
    And since $\abs{Q^{M_1}_\pi(s, a) - Q^{M_2}_\pi(s, a)} \leq \Delta$, the proof is completed. \qedhere
\end{proof}
%
For the following lemma without loss of generality assume $\env{\rmax} = \mon{\rmax} = 1$.
%
\begin{lemma}
\label{lemma:tau_closeness}
    Let $M_1 = \angle{\c{S}, \c{A}, P_1, r_1, \mon{f}, \gamma}$ and $M_2 = \angle{\c{S}, \c{A}, P_2, r_2, \mon{f}, \gamma}$ be two Mon-MDPs where $r_1 = (\env{r}_1, \mon{r}_1)$, $r_2 = (\env{r}_2, \mon{r}_2)$. Assume the following bounds hold:
    %
    \begin{equation*}
        -\env{\rmax} \leq \env{r}_1, \env{r}_2 \leq \env{\rmax} \text{ and } -\mon{\rmax} \leq \mon{r}_1, \mon{r}_2 \leq \mon{\rmax}
    \end{equation*}
    %
    Suppose further that for all joint state-action pairs $(s, a) \equiv (\env{s}, \mon{s}, \env{a}, \mon{a})$ the following conditions are satisfied:
    %
    \begin{align*}
    \abs{\env{r}_1(\env{s}, \env{a}) - \env{r}_2(\env{s}, \env{a})} & \leq \env{\varphi} \\
    \abs{\mon{r}_1(\mon{s}, \mon{a}) - \mon{r}_2(\mon{s}, \mon{a})} & \leq \mon{\varphi} \\
    \oneNorm{P_1(\cdot \mid s, a) - P_2(\cdot \mid s, a)} & \leq \varphi
    \end{align*}
    %
    There exists a constant $C$ such that for any $0 \leq \varepsilon \leq \nicefrac{(\env{\rmax} + \mon{\rmax})}{1 - \gamma}$, and any stationary policy $\pi$, if $\env{\varphi} = \mon{\varphi} = \varphi = C\frac{\varepsilon(1 - \gamma)^2}{\env{\rmax} + \mon{\rmax}}$, then
    \begin{equation*}
        \abs{Q^{M_1}_\pi(s, a) - Q^{M_2}_\pi(s, a)} \leq \varepsilon
    \end{equation*}
\end{lemma}
%
\begin{proof}
    Using~\cref{lemma:general_closeness}, we should show that:
    %
    \begin{equation*}
        \frac{\env{\varphi} + \mon{\varphi} + 2\varphi\gamma(\env{\rmax} + \mon{\rmax})}{(1 - \gamma)^2} = 2\varphi\frac{1 + \gamma(\env{\rmax} + \mon{\rmax})}{(1 - \gamma)^2} \leq \varepsilon
    \end{equation*}
    %
    which yields:
    %
    \begin{equation*}
        \varphi \leq \frac{\varepsilon(1 - \gamma)^2}{2\left(1 + \gamma(\env{\rmax} + \mon{\rmax})\right)}
    \end{equation*}
    %
    And by our assumption that $\env{\rmax} = \mon{\rmax} = 1$:
    %
    \begin{equation*}
        \varphi \leq \frac{\varepsilon(1 - \gamma)^2}{6(\env{\rmax} + \mon{\rmax})}
    \end{equation*}
    %
    By choosing $C = 6$ the proof is completed. \qedhere
\end{proof}
%
\begin{lemma}[Chernoff-Hoeffding's inequality]
    For $(V_i)_{i = 1}^{n}$ independent samples on probability space $(\Omega, \mathcal{F}, \mathbb{P})$ where $V_i \in [a_i, b_i]$ for all $i$ and $\epsilon > 0$, we have:
    %
    \begin{equation*}
        \prob{\EV{V_1}{} - \frac 1n \sum_{i=1}^{n} V_i \geq \epsilon} \leq \exp{\left(- \frac{2n^2\epsilon^2}{\sum_{i = 1}^{n}(b_i - a_i)^2}\right)}
    \end{equation*}
    %
\end{lemma}
%
% \begin{remark}
%     \label{remark:iid}
%     We make use of Chernoff-Hoeffding's for sample mean of reward and transitions for any fixed state-action pair $(s, a)$ upon visiting it for the $v$th times; and, since we are in the episodic communicating setting, Markov property allows us to treat sample means as if they are mean of independent identically distributed random~\citep{auer2002finite, ucb2016abanditblog, csaba2022cmput653}.
% \end{remark}
% %
% \begin{remark}
%     Chernoff-Hoeffding's inequality assumes fixed number of samples in the sequence. In order in to apply the inequality when the number of samples is random, either a union bound over the possible values of the sequence length should be used~\citep{lattimore2020bandit, garivier2013informational, auer2002finite} or by using the law of iterated logarithm increase the logarithmic term in the confidence width~\citep{lattimore2020bandit, SueiWenChen2004concentration}.
% \end{remark}
% %
% \begin{remark}
%     \label{remark:azuma}
%     Azuma-Hoeffding's inequality asserts the the above inequality also holds for (sub/super-) martingale sequences.
% \end{remark}