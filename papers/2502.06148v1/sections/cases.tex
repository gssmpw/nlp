\newcolumntype{L}[1]{>{\raggedright\arraybackslash}p{#1}}

\begin{table*}[ht]
\setlength\abovecaptionskip{-0.3px}
\setlength\belowcaptionskip{-0.1px}
\caption{Examples of errors and corresponding percentages. 
Q, G, and P denote the question, golden answer, and prediction, respectively. C1 and C2 represent two candidate responses.}
\label{tab:error-case} 
\centering
\renewcommand{\arraystretch}{1.2} % Adjust row spacing
\setlength{\tabcolsep}{4pt} % Adjust column spacing
\footnotesize
\begin{tabular}{L{0.1\textwidth}|L{0.85\textwidth}}
\toprule
\multirow{6}{*}{\makecell[l]{Lack of Evidence \\ (51\%)}} 
& Q: The Express, The Telegraph, The Economist, The Times, The Star, and The Independent were all what in 1830's London? (Not Newspapers) \\
& G: \textcolor{blue}{Stagecoaches} \\
& C1: Explanation: The Express, The Telegraph...were not entities in London during the 1830s... Answer: \textcolor{red}{Unknown}\\
& C2: Explanation: ...the passages provided only discuss their histories and developments starting from the mid-1800s... Answer: \textcolor{red}{Unknown} \\
& P: Explanation: The Express, The Telegraph...were not entities in London during the 1830s... Answer: \textcolor{red}{Unknown} \\
\midrule

\multirow{6}{*}{\makecell[l]{Partial Matching \\ (20\%)}} 
& Q: In which Parliamentary constituency did Michael Portillo replace Alan Clark as MP in 1999? \\
& G: \textcolor{blue}{Kensington and Chelsea (borough)} \\
& C1: Explanation: ...Portillo is most famously known for defeating Clark...in the constituency of Wellingborough. Answer: \textcolor{red}{Wellingborough}.\\
& C2: Explanation: Michael Portillo replaced Alan Clark as MP in Kensington and Chelsea... Answer: \textcolor{blue}{Kensington and Chelsea}\textcolor{gray}{ (borough)} \\
& P: Explanation: Michael Portillo replaced Alan Clark as MP in Kensington and Chelsea... Answer: \textcolor{blue}{Kensington and Chelsea}\textcolor{gray}{ (borough)} \\
\midrule

\multirow{7}{*}{\makecell[l]{Reasoning Error \\ (14\%)}} 
& Q: Richard Nixon was Vice President to which US state? \\
& G: \textcolor{blue}{Dwight D Eisenhower} \\
& C1: Explanation: Richard Nixon served as the Vice President under \textcolor{blue}{Dwight D. Eisenhower} from 1953 to 1961... However, it's important to note that Nixon didn't directly represent Kansas as Vice President; instead, he assisted President Eisenhower. Answer: \textcolor{red}{Kansas (through Eisenhower)} \\
& C2: Explanation: Richard Nixon was the Vice President of the United States for the state of California. Answer: \textcolor{red}{California.}\\
& P: Explanation: Richard Nixon was the Vice President of the United States for the state of California. Answer: \textcolor{red}{California.} \\
\midrule

\multirow{6}{*}{\makecell[l]{Selection Error \\ (12\%)}} 
& Q: The Greek word praxis refers to what, contrasting with theory? \\
& G: \textcolor{blue}{Practice} \\
& C1: Explanation: In ancient Greek philosophy, the term "praxis" refers to the application or practice of... Answer: \textcolor{blue}{practice} \\
& C2: Explanation: The Greek word praxis refers to the process of enacting... Answer: \textcolor{red}{The enactment or realization of a theory} \\
& P: Explanation: The Greek word praxis refers to the process of enacting... Answer: \textcolor{red}{The enactment or realization of a theory} \\
\midrule

\multirow{6}{*}{\makecell[l]{Formatting Error \\ (3\%)}} 
& Q: What do 'Barb' charts show? \\
& G: \textcolor{blue}{Television Ratings} \\
& C1: Explanation: Bar charts are a type of chart that display... Answer: \textcolor{red}{Bar charts show categorical data...} \\
& C2: Explanation: ...and other research findings related to television audience measurement and ratings in the United Kingdom. Answer: \textcolor{red}{Barb charts show comparisons}... In the context of \textcolor{blue}{television ratings}... \\
& P: Research ... television audience measurement and ratings in the United Kingdom. In the context of \textcolor{blue}{television ratings}... \\
\bottomrule
\end{tabular}
\end{table*}