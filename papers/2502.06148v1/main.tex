%%
%% This is file `sample-sigconf-authordraft.tex',
%% generated with the docstrip utility.
%%
%% The original source files were:
%%
%% samples.dtx  (with options: `all,proceedings,bibtex,authordraft')
%% 
%% IMPORTANT NOTICE:
%% 
%% For the copyright see the source file.
%% 
%% Any modified versions of this file must be renamed
%% with new filenames distinct from sample-sigconf-authordraft.tex.
%% 
%% For distribution of the original source see the terms
%% for copying and modification in the file samples.dtx.
%% 
%% This generated file may be distributed as long as the
%% original source files, as listed above, are part of the
%% same distribution. (The sources need not necessarily be
%% in the same archive or directory.)
%%
%%
%% Commands for TeXCount
%TC:macro \cite [option:text,text]
%TC:macro \citep [option:text,text]
%TC:macro \citet [option:text,text]
%TC:envir table 0 1
%TC:envir table* 0 1
%TC:envir tabular [ignore] word
%TC:envir displaymath 0 word
%TC:envir math 0 word
%TC:envir comment 0 0
%%
%% The first command in your LaTeX source must be the \documentclass
%% command.
%%
%% For submission and review of your manuscript please change the
%% command to \documentclass[manuscript, screen, review]{acmart}.
%%
%% When submitting camera ready or to TAPS, please change the command
%% to \documentclass[sigconf]{acmart} or whichever template is required
%% for your publication.
%%
%%
%%

%\documentclass[sigconf,natbib=true,anonymous=true]{acmart}authordraft
\documentclass[sigconf,natbib=true]{acmart}
\usepackage{CJKutf8} % 引入CJKutf8宏包
\let\Bbbk\relax
\usepackage{multirow}
\usepackage{arydshln} 
\usepackage{amssymb,enumitem}
\usepackage{makecell}


\def\Figref#1{Figure~\ref{#1}}
\newcommand{\tabref}[1]{Table~\ref{#1}}

\newcommand{\fb}{\textcolor{blue}}
\newcommand{\framework}{\textbf{Self-Selection}}
\newcommand{\approach}{\textbf{Self-Selection-RGP}}


\newcommand{\headernodot}[1]{\vspace*{1mm}\noindent\textbf{#1}}
\newcommand{\header}[1]{\headernodot{#1.}}


%%
%% \BibTeX command to typeset BibTeX logo in the docs
\AtBeginDocument{%
  \providecommand\BibTeX{{%
    Bib\TeX}}}

%% Rights management information.  This information is sent to you
%% when you complete the rights form.  These commands have SAMPLE
%% values in them; it is your responsibility as an author to replace
%% the commands and values with those provided to you when you
%% complete the rights form.
\setcopyright{acmlicensed}
\copyrightyear{2018}
\acmYear{2018}
\acmDOI{XXXXXXX.XXXXXXX}
%% These commands are for a PROCEEDINGS abstract or paper.
\acmConference[Conference acronym 'XX]{Make sure to enter the correct
  conference title from your rights confirmation emai}{June 03--05,
  2018}{Woodstock, NY}
%%
%%  Uncomment \acmBooktitle if the title of the proceedings is different
%%  from ``Proceedings of ...''!
%%
%%\acmBooktitle{Woodstock '18: ACM Symposium on Neural Gaze Detection,
%%  June 03--05, 2018, Woodstock, NY}
\acmISBN{978-1-4503-XXXX-X/18/06}


%%
%% Submission ID.
%% Use this when submitting an article to a sponsored event. You'll
%% receive a unique submission ID from the organizers
%% of the event, and this ID should be used as the parameter to this command.
%%\acmSubmissionID{123-A56-BU3}

%%
%% For managing citations, it is recommended to use bibliography
%% files in BibTeX format.
%%
%% You can then either use BibTeX with the ACM-Reference-Format style,
%% or BibLaTeX with the acmnumeric or acmauthoryear sytles, that include
%% support for advanced citation of software artefact from the
%% biblatex-software package, also separately available on CTAN.
%%
%% Look at the sample-*-biblatex.tex files for templates showcasing
%% the biblatex styles.
%%

%%
%% The majority of ACM publications use numbered citations and
%% references.  The command \citestyle{authoryear} switches to the
%% "author year" style.
%%
%% If you are preparing content for an event
%% sponsored by ACM SIGGRAPH, you must use the "author year" style of
%% citations and references.
%% Uncommenting
%% the next command will enable that style.
%%\citestyle{acmauthoryear}


%%
%% end of the preamble, start of the body of the document source.
\begin{document}

%%
%% The "title" command has an optional parameter,
%% allowing the author to define a "short title" to be used in page headers.
\title{Optimizing Knowledge Integration in Retrieval-Augmented Generation with Self-Selection}

%%
%% The "author" command and its associated commands are used to define
%% the authors and their affiliations.
%% Of note is the shared affiliation of the first two authors, and the
%% "authornote" and "authornotemark" commands
%% used to denote shared contribution to the research.
% \author{Ben Trovato}
% \authornote{Both authors contributed equally to this research.}
% \email{trovato@corporation.com}
% \orcid{1234-5678-9012}
% \author{G.K.M. Tobin}
% \authornotemark[1]
% \email{webmaster@marysville-ohio.com}
% \affiliation{%
%   \institution{Institute for Clarity in Documentation}
%   \city{Dublin}
%   \state{Ohio}
%   \country{USA}
% }

% \author{Lars Th{\o}rv{\"a}ld}
% \affiliation{%
%   \institution{The Th{\o}rv{\"a}ld Group}
%   \city{Hekla}
%   \country{Iceland}}
% \email{larst@affiliation.org}

\author{Yan Weng}
\affiliation{%
  \institution{University of Science and Technology of China}
    \city{Hefei}
    \country{China}  
    }
\email{wengyan@mail.ustc.edu.cn}

\author{Fengbin Zhu}
\authornote{Corresponding authors.}
\affiliation{%
  \institution{National University of Singapore}
  \country{Singapore}
  }
\email{zhfengbin@gmail.com}

\author{Tong Ye}
\affiliation{%
  \institution{Institute of Dataspace, Hefei Comprehensive National Science Center}
    \city{Hefei}
    \country{China} 
    }
\email{tye9601@gmail.com}

\author{Haoyan Liu}
\affiliation{%
  \institution{University of Science and Technology of China}
    \city{Hefei}
    \country{China}  
    }
\email{liuhaoyan@ustc.edu.cn}

\author{Fuli Feng}
\authornotemark[1]
\affiliation{%
  \institution{University of Science and Technology of China}
    \city{Hefei}
    \country{China}  
    }
\email{fulifeng93@gmail.com}


\author{Tat-Seng Chua}
\affiliation{%
  \institution{National University of Singapore}
  \country{Singapore}
  }
\email{dcscts@nus.edu.sg}

%%
%% By default, the full list of authors will be used in the page
%% headers. Often, this list is too long, and will overlap
%% other information printed in the page headers. This command allows
%% the author to define a more concise list
%% of authors' names for this purpose.
\renewcommand{\shortauthors}{Trovato et al.}

%%
%% The abstract is a short summary of the work to be presented in the
%% article.
\begin{abstract}
Retrieval-Augmented Generation (RAG), which integrates external knowledge into Large Language Models (LLMs), has proven effective in enabling LLMs to produce more accurate and reliable responses.
However, it remains a significant challenge how to effectively integrate external retrieved knowledge with internal parametric knowledge in LLMs.
In this work, we propose a novel \framework~RAG framework, where the LLM is made to select from pairwise responses generated with internal parametric knowledge solely and with external retrieved knowledge together to achieve enhanced accuracy. 
To this end, we devise a \approach~method to enhance the capabilities of the LLM in both generating and selecting the correct answer, by training the LLM with Direct Preference Optimization (DPO) over a curated Retrieval-Generation Preference (RGP) dataset. 
Experimental results with two open-source LLMs (i.e., Llama2-13B-Chat and Mistral-7B) well demonstrate the superiority of our approach over other baseline methods on Natural Questions (NQ) and TrivialQA datasets. 
% Our work paves the way for more robust and reliable LLMs in RAG.

\end{abstract}

%%
%% The code below is generated by the tool at http://dl.acm.org/ccs.cfm.
%% Please copy and paste the code instead of the example below.
%%
\begin{CCSXML}
<ccs2012>
   <concept>
       <concept_id>10002951.10003317.10003338.10003341</concept_id>
       <concept_desc>Information systems~Language models</concept_desc>
       <concept_significance>500</concept_significance>
       </concept>
   <concept>
       <concept_id>10002951.10003317.10003347.10003348</concept_id>
       <concept_desc>Information systems~Question answering</concept_desc>
       <concept_significance>500</concept_significance>
       </concept>
 </ccs2012>
\end{CCSXML}

\ccsdesc[500]{Information systems~Language models}
\ccsdesc[500]{Information systems~Question answering}

%%
%% Keywords. The author(s) should pick words that accurately describe
%% the work being presented. Separate the keywords with commas.
\keywords{retrieval-augmented generation, large language models, question answering}
%% A "teaser" image appears between the author and affiliation
%% information and the body of the document, and typically spans the
%% page.

\received{20 February 2007}
\received[revised]{12 March 2009}
\received[accepted]{5 June 2009}

%%
%% This command processes the author and affiliation and title
%% information and builds the first part of the formatted document.
\maketitle


\begin{figure}[h]
  \centering
  \includegraphics[width=\linewidth]{figs/idea.pdf}
  \caption{An illustration of the proposed \framework~framework. For a given query, 
   an LLM is first requested to generate the answers and their respective explanations 
   with and without external knowledge. 
   Then, the LLM is adopted to take them as input and choose one from them as the final answer with its explanation. }
   \label{sample}
\end{figure}


\section{INTRODUCTION}
Large Language Models (LLMs) have demonstrated remarkable capabilities across various tasks~\cite{Brown2020Few-Shot,Touvron2023LLaMA,openai2024gpt4}.
However, their reliance on static parametric knowledge~\cite{Kasai2023RealTimeQA, mallen2023trust} often leads to inaccuracy or hallucination in responses~\cite{Welleck2020Neural, min2023factscore}.
Retrieval-Augmented Generation (RAG)~\cite{Lewis2020RAG,Guu2020REALM,ram2023context,asai2023retrieval} supplements LLMs with relevant knowledge retrieved from external sources, attracting increasing research interest.
One critical challenge for existing RAG systems is how to effectively integrate internal parametric knowledge with external retrieved knowledge to generate more accurate and reliable results.
% However, indiscriminately utilizing the retrieved knowledge may introduce irrelevant or off-topic information, compromising the quality of the response\cite{shi2023Irrelevant}.


In existing RAG approaches, LLMs depend either \emph{highly} or \emph{conditionally} upon external knowledge.
The former consistently uses the retrieved content as supporting evidence~\cite{Lewis2020RAG, Guu2020REALM, trivedi2023interleaving}, which often introduces irrelevant or noisy information and overlooks the valuable internal knowledge in LLMs, resulting in sub-optimal results.
In comparison, the latter integrates external knowledge into LLMs conditionally based on specific strategies, such as characteristics of input query \cite{mallen2023trust, jeong2024adaptive, wang2023skr}, probability of generated tokens~\cite{jiang2023active, su2024dragin}, or relevance of retrieved content~\cite{zhang2023merging, Xu2024Search, liu2024raisf}.
The query-based and token-based strategies generally utilize a fixed question set or a predefined threshold to decide whether to incorporate external knowledge, limiting their effectiveness due to incomplete information; 
the relevance-based strategy employs an additional validation module to assess the retrieved content, with its accuracy largely determining the quality of the final responses.
%with the successful integration of external knowledge heavily dependent on the accuracy of this additional module.
 

In this work, we explore leveraging the \textbf{LLM itself} to determine the correct result by \textbf{holistically} evaluating the outputs generated with and without external knowledge.
As illustrated in \Figref{sample}, given a query ``What does Ctrl+Shift+T do?'', we instruct the LLM to generate the \emph{LLM Answer} (i.e., ``New tab'') and the corresponding explanation (i.e., reasoning steps) with its internal parametric knowledge.
Meanwhile, we employ a retriever to obtain the relevant passages from external knowledge bases and feed the query and the retrieved passages to the LLM to produce the \emph{RAG Answer} (i.e., ``T'') and the corresponding explanation. 
Next, we instruct the LLM to take the query, \emph{LLM Answer} with its explanation and \emph{RAG Answer} with its explanation as input to choose the more accurate one (i.e., ``New tab'').
In this manner, the relevant internal and external knowledge to the query is comprehensively considered, facilitating the LLM in generating accurate responses, while the RAG framework maintains its simplicity by not requiring additional modules.

% selects the final answer from the responses generated using its internal parametric knowledge and external retrieved knowledge, thereby achieving improved accuracy.


Accordingly, we devise a novel \framework\ RAG framework that empowers the LLM to identify the more accurate answer to a query by evaluating both \emph{LLM Answer} and \emph{RAG Answer}, along with their respective explanations.
We validate the performance of the proposed \framework~framework with two open-sourced LLMs (see Section \ref{sec:main-result}) and find that it tends to fail in some scenarios, which we attribute to its limited capacity in distinguishing the correct answer from the incorrect one. 
To enhance the accuracy of the LLM selecting the right one among multiple responses generated from different knowledge sources, we develop a~\approach~method, leveraging Direct Preference Optimization (DPO)~\cite{Rafailov2023DPO} to fine-tune the LLM with a curated Retrieval-Generation Preference (\textbf{RGP}) dataset.
To construct this RGP dataset, we employ GPT-3.5~\cite{openai2024gpt4} to generate an \emph{LLM Answer} and an \emph{RAG Answer} for each query sampled from  WebQuestions~\cite{berant2013webq}, SQuAD 2.0~\cite{rajpurkar2018squad} and SciQ~\cite{welbl2017sciq}, and then retain only the pairs consisting of one correct answer and one incorrect answer, each accompanied by its corresponding explanation.
It consists of $3,756$ pairs of \emph{LLM Answer} and \emph{RAG answer} with their respective explanations, which we promise to release to the public to facilitate future research. 
%With the constructed dataset, we apply the advanced Direct Preference Optimization (DPO)\cite{Rafailov2023DPO} to enhance open-source LLMs in both selecting and generating accurate responses.


%第五段 实验结果 insight 

With this dataset, we train two different LLMs, including Mistral-7B~\cite{jiang2023mistral7b} and LLaMa-2-13B-Chat~\cite{touvron2023llama2openfoundation}, and evaluate them on two widely used datasets, i.e., Natural Questions (NQ)~\cite{kwiatkowski2019natural} and TrivialQA~\cite{joshi2017triviaqa}.
It is demonstrated that our \approach~method consistently achieves high effectiveness across various retrieval settings and different LLMs, enhancing the robustness and stability of RAG systems.
Moreover, additional experiments reveal that our \approach~method not only enhances LLMs' ability to distinguish valid answers from noisy ones but also improves their answer generation capabilities.
We further validate the rationale of each design in our method through ablation studies, and conduct error case analyses to offer deeper insights into the limitations of our proposed method.


In summary, the major contributions of our paper are three-fold: 
\begin{itemize} []
\item  We introduce a novel \framework~ RAG framework that leverages LLMs to determine the correct answer by evaluating a pair of responses generated with internal parametric knowledge solely and also with external retrieved knowledge.
\item  We propose a \approach~method that applies Direct Preference Optimization (DPO) to enhance LLMs in both identifying and generating the correct answers with a curated Retrieval-Generation Preference (RGP) dataset.
\item Extensive experiments with two open-sourced LLMs achieve superior performance on two widely-used datasets, demonstrating the effectiveness of our proposed Self-Selection framework and Self-Selection-RGP method.
\end{itemize}


\begin{figure*}[h]
  \centering
  \includegraphics[width=\textwidth]{figs/framework.pdf}
  \caption{An illustration of the proposed \approach~method.}
  \Description{}
  \label{method}
\end{figure*}

\section{\ourmethod}
\label{sec:method}

\begin{figure*}[t]
    \centering
    \includegraphics[width=0.95\linewidth]{figures/figure2.pdf} %pdf
    \caption{The framework of \ourmethod, including three modules: Router, Verification Agents, and Judger.
    }
    \label{fig:pipeline}
\end{figure*}


In this work, we empirically implement a reward agent, named \ourmethod, which integrates the base human preference reward model with verifiable correctness signals from two key aspects: factuality, which assesses the correctness of claimed facts, and instruction-following, which evaluates whether the response satisfies the hard constraints specified in the instruction~\citep{zhou2023instruction}. 
Both aspects significantly impact reliability and user experience in real-world applications and are challenging to evaluate effectively with existing reward models~\citep{liu2024rm}.
This section introduces the overall model architecture (\cref{sec:model_archi}) and the specific modules (\Cref{sec:router,sec:scoring_agents,sec:judger}) of \ourmethod.


\subsection{Model Architecture}
\label{sec:model_archi}
Following the concept in Euqation~\ref{eq:eq1}, the overall architecture of \ourmethod is illustrated in Figure~\ref{fig:pipeline}, which consists of three main modules: (1) \textit{Router}, which analyzes the instruction and determines which agents to invoke, corresponding to $A_x$ in Equation~\ref{eq:eq1}. As different instructions may require evaluations of different aspects of responses, dynamically selecting verification agents helps reduce inference costs and mitigate potential cumulative errors.
(2) \textit{Verification agents}, which evaluate different aspects of response correctness. In our implementation, we design two agents for assessing factuality and instruction-following, both based on LLMs augmented with additional tools. 
(3) \textit{Judger}, which integrates the scores from the verification agents and human preferences from the base reward model to produce a final reward, corresponding to determining $\lambda$ and $w_i$ in Equation~\ref{eq:eq1}.
We will provide detailed descriptions in the following sections.


\subsection{Router}
\label{sec:router}
Given an instruction, the router analyzes its requirements to the response to select the appropriate verification agents. 
The router is powered by an existing LLM backbone. Specifically, we first manually provide a concise description for each verification agent, 
explaining its functionality and specifying the conditions for its usage. Then, we input the instruction with all agent descriptions into the LLM, prompting it to select appropriate verification agents for correctness assessment. More implementation details are placed in appendix~\ref{sec:app_method}.


\subsection{Verification Agents}
\label{sec:scoring_agents}
\paragraph{Factuality}
Previous studies have proposed various methods to evaluate the factuality of responses, such as FactScore~\citep{min2023factscore}, which can be directly used as a verification agent. However, these methods typically require extensive search engine queries to verify the correctness of each atomic fact, which is costly and inefficient for reward scoring. Intuitively, pairwise scoring based on only the differences between two responses can effectively reduce search engine queries and time costs. Therefore, we propose a pairwise factuality verification agent for efficiently evaluating the factual correctness of response pairs.
The agent is illustrated in Figure~\ref{fig:pipeline}, which consists of four main components:
(1) Difference proposal, which identifies key differences in claimed facts between two given responses.
(2) Query generation, which constructs queries based on the identified differences to retrieve evidence for distinguishing these differences.
(3) Evidence generation, which uses the generated queries to retrieve supporting evidence using either external search engines or parametric knowledge in LLMs.
(4) Verification, which assigns an integer score from $0$ to $1$ to each response, using the collected evidence and original responses as inputs.
The verification agent can effectively capture subtle factuality differences~\citep{jiang2023llm} between responses while significantly reducing inference-time costs by verifying only their differences rather than all claimed facts.
All modules are implemented using an LLM backbone. The implementation details are placed in appendix~\ref{sec:app_method}.



% N >2的情况。

\paragraph{Instruction-Following}
The evaluation of the instruction following primarily assesses the adherence to hard constraints~\citep{zhou2023instruction} specified in the instruction, such as length constraints. 
Typically, instruction-following constraints can be categorized into soft and hard constraints, where the former focuses on semantic aspects, such as language style, while the latter focuses on surface-form constraints, such as format, which can be objectively evaluated. For instruction-following, our verification agent focuses on hard constraints, which are difficult to evaluate with existing reward models but can be efficiently verified using external tools, such as Python code scripts.
The agent is shown in Figure~\ref{fig:pipeline}, including three components:
(1) Constraint parsing, which extracts hard constraints from the instruction.
(2) Code generation and refinement, which generates Python scripts used to check the adherence to the extracted constraints. The generated code takes the response as input and returns either $0$ or $1$, where $1$ indicates that the constraint is satisfied, and $0$ otherwise.
We also incorporate a refinement step like \citet{madaan2024self} to correct invalid or syntactically incorrect code. Specifically, we execute the generated Python code using a Python interpreter, and if an error occurs, the error information and original code are fed back into the model to generate a refined code script.
(3) Verification, which executes the generated code in the Python interpreter to obtain a binary score ($0$ or $1$). The final score is the average of all hard constraint scores.
All the modules are implemented using LLMs. The specific prompts and implementation details are provided in appendix~\ref{sec:app_method}.



\subsection{Judger}
\label{sec:judger}
The judger integrates reward scores from verification agents and human preferences from base reward models. In our implementation, we use a weighted sum as the judger, where $\lambda$ and $w_i$ are all set to $1.0$, to compute the final reward score in Equation~\ref{eq:eq1}. One can also adopt different $\lambda$ and $w_i$ for better applicability in different scenarios. Additionally, the judger can dynamically adjust $\lambda$ and $w_i$ based on the instruction like gating network~\citep{wang2024interpretable}, we leave it as future work.





\section{EXPERIMENTS}
\label{sec:experiments}
In this section, we evaluate the proposed method with extensive experiments.
Note that we conduct our experiments with two variants based on our proposed \framework~framework,  i) \textbf{Self-Selection-Ori}, which refers to the RAG method that applies Self-Selection on vanilla LLMs. ii) \textbf{Self-Selection-RGP}, which denotes the RAG method that applies Self-Selection on the LLMs trained with our augmented RGP dataset.
For a comprehensive evaluation, we seek to address the following research questions:
\begin{itemize} 
\item \textbf{RQ1}: How does Self-Selection-RGP perform compared to other compared methods? 
\item \textbf{RQ2}: Whether Self-Selection-RGP is generalizable across different base LLMs and retrieval settings? 
\item \textbf{RQ3}: To what extent can Self-Selection-RGP affect the LLMs' inherent ability in answer generation? 
\item \textbf{RQ4}: What is the effect of each design in our proposed Self-Selection-RGP? 
\end{itemize}

\subsection{Experimental Setup}
\subsubsection{\textbf{Datasets}}
 We verify the proposed method with two open-domain QA datasets, Natural Question (NQ)~\cite{kwiatkowski2019natural} and TriviaQA~\cite{joshi2017triviaqa}.
\begin{itemize}[leftmargin=*,nosep]
    \item \textbf{Natural Questions}: Natural Questions is a widely used dataset for the evaluation of RAG systems. Each QA pair is annotated by human annotators based on a Wikipedia page.
    \item \textbf{TriviaQA}: TriviaQA is another popular dataset based on trivia questions, paired with independently collected evidence documents, which is designed to support challenging reading comprehension tasks.
\end{itemize}
Following prior works \cite{trivedi2023interleaving,jeong2024adaptive}, we use the same test split for each dataset with the same external corpus to evaluate RAG methods. 
We present the statistics in \tabref{tab:dataset}.  

\begin{table}[htbp]
  \centering
  \large
  \setlength\abovecaptionskip{-0.3pt}
  \setlength\belowcaptionskip{-0.3pt}
  \caption{Statistics of the test datasets.}
  \resizebox{\columnwidth}{!}{%
    \begin{tabular}{p{4cm}p{3cm}p{3cm}} % 调整列宽
    \toprule
    \bf Dataset & \bf \#Passages & \bf \#QA pairs \\
    \midrule
    Natural Questions (NQ) & 21,015,324 & 500 \\
    TriviaQA & 21,015,324 & 500 \\
    \bottomrule
    \end{tabular}%
  }
  \label{tab:dataset}%
\end{table}



\begin{table*}[t]
  \setlength\abovecaptionskip{-0.3pt}
  \setlength\belowcaptionskip{-0.3pt}
  \setlength{\tabcolsep}{1.9mm}
  \centering
  \caption{Main results of our proposed methods and all baseline methods. }
    \begin{tabular}{llrrrrrrrrrrrr}
    \toprule
          &       & \multicolumn{6}{c}{\textit{zero-shot}}        & \multicolumn{6}{c}{\textit{few-shot}} \\
          &       & \multicolumn{3}{c}{\textbf{NQ}} & \multicolumn{3}{c}{\textbf{TriviaQA}} & \multicolumn{3}{c}{\textbf{NQ}} & \multicolumn{3}{c}{\textbf{TriviaQA}} \\
    \multicolumn{1}{c}{Base LLM} & \multicolumn{1}{c}{Method} & \multicolumn{1}{c}{EM} & \multicolumn{1}{c}{F1} & \multicolumn{1}{c}{Acc} & \multicolumn{1}{c}{EM} & \multicolumn{1}{c}{F1} & \multicolumn{1}{c}{Acc} & \multicolumn{1}{c}{EM} & \multicolumn{1}{c}{F1} & \multicolumn{1}{c}{Acc} & \multicolumn{1}{c}{EM} & \multicolumn{1}{c}{F1} & \multicolumn{1}{c}{Acc} \\
    \midrule
    \multirow{6}[2]{*}{Mistral (7B)} & LLM Only & 21.8  & 35.5  & 34.0  & 41.2  & 53.4  & 52.6  & 20.8  & 34.9  & 32.2  & 43.8  & 54.1  & 53.4 \\
          & Standard RAG  & 35.8  & 51.2  & 51.0  & 45.8  & 58.1  & 59.8  & 37.8  & 51.6  & 50.6  & 47.2  & 59.4  & 60.4 \\
          & Self-RAG & -     & -     & -     & 29.0  & 43.2  & 60.6  & -     & -     & -     & 41.0  & 53.0  & 55.0 \\
          & SURE  & \textbf{39.0} & 52.4  & 47.6  & 48.6  & 59.7  & 61.6  & 38.8  & 51.1  & 47.6  & 49.2  & 60.1  & 61.4 \\
          & \textbf{Self-Selection-Ori} & 34.6  & 50.1  & 50.2  & 48.4  & 61.2  & 62.8  & 33.0  & 47.3  & 45.2  & 49.8  & 61.7  & 62.4 \\
          & \textbf{Self-Selection-RGP} & 37.8  & \textbf{52.5} & \textbf{53.6} & \textbf{54.4} & \textbf{66.2} & \textbf{67.0} & \textbf{40.2} & \textbf{53.8} & \textbf{53.2} & \textbf{54.4} & \textbf{66.2} & \textbf{65.4} \\
    \midrule
    \multirow{6}[2]{*}{Llama2-Chat (13B)} & LLM Only & 21.2  & 31.9  & 28.2  & 43.2  & 50.1  & 48.0  & 24.8  & 35.3  & 30.2  & 49.8  & 57.1  & 54.0 \\
          & Standard RAG  & 24.6  & 37.0  & 45.2  & 35.2  & 46.1  & 55.0  & 31.8  & 43.4  & 44.8  & 46.0  & 54.8  & 54.6 \\
          & Self-RAG & -     & -     & -     & 17.2  & 36.6  & 63.4  & -     & -     & -     & 39.0  & 52.2  & 59.0 \\
          & SURE  & \textbf{39.4} & \textbf{52.3} & \textbf{52.0} & 50.4  & 63.0  & 63.8  & \textbf{42.6} & \textbf{53.2} & 50.4  & 40.6  & 51.3  & 65.0 \\
          & \textbf{Self-Selection-Ori} & 31.6  & 43.8  & 45.2  & 43.0  & 53.5  & 56.6  & 33.2  & 44.6  & 43.2  & 49.2  & 59.9  & 60.0 \\
          & \textbf{Self-Selection-RGP} & 36.6  & 49.2  & 46.2  & \textbf{56.6} & \textbf{66.3} & \textbf{66.0} & 40.0  & 52.4  & \textbf{51.6} & \textbf{52.6} & \textbf{65.5} & \textbf{66.8} \\
    \bottomrule
    \end{tabular}%
  \label{tab:main}%
\end{table*}%
\subsubsection{\textbf{Baselines}}
We will compare our proposed methods with the following baseline methods:
\begin{itemize}[leftmargin=*,nosep]
    \item \textbf{LLM Only}: The response to each query is generated solely by LLMs. 
    \item \textbf{Standard RAG}: The response to each query is produced by LLMs after appending the retrieved passages to the input.
    \item \textbf{Self-RAG}~\cite{asai2024selfrag}: Specialized reflection tokens are utilized to enable LLMs to control retrieval and evaluate the relevance of the retrieved content during reasoning. In the experiments, we use the open-source models fine-tuned with the SELF-RAG framework, including the fine-tuned models based on Mistral 7B and Llama2-13B\cite{asai2024selfrag,SciPhi-AI2024}.
    \item \textbf{SURE}~\cite{kim2024sure}: LLMs first generate summaries of the retrieved passages for each candidate answer, and then identify the most plausible answer by evaluating each summary’s validity and ranking. We employ Mistral 7B  and Llama2-13B-Chat \cite{touvron2023llama2openfoundation} as the backbone models, consistent with our method.
\end{itemize}

\subsubsection{\textbf{Evaluation Metrics}}
In our experiments, we use the Exact Match (EM), F1 score and Accuracy (Acc) as our evaluation metrics, following the standard evaluation protocol~\cite{mallen2023trust,jeong2024adaptive,kim2024sure}. 
The EM measures whether the prediction and the gold answer are exactly the same; the F1 score measures the number of overlapping words between the prediction and the gold answer; the Accuracy metric measures whether the prediction contains the gold answer.
We normalize the predictions and gold answers (i.e., lowercasing and punctuation) to compute the metrics, following the implementation in previous works~\cite{rajpurkar-etal-2016-squad,kim2024sure}.


\subsubsection{\textbf{Implementation Details}}
In total, we sample $11,756$ QA pairs from WebQuestions~\cite{berant2013webq}, SQuAD2.0~\cite{rajpurkar2018squad}, and SciQ~\cite{welbl2017sciq}.
We use the official 2018 English Wikipedia as the external knowledge base similar to prior works~\cite{karpukhin-etal-2020-dense,jeong2024adaptive,asai2024selfrag}. 
We use pre-trained BGE (i.e., bge-large-en-v1.5)~\cite{Xiao2024cpack} as the retriever to obtain the relevant passages for each question.
For each query, we retrieve a variable number of passages, ranging from $1$ to $5$.
We adopt GPT-3.5 as our LLM for generating answers and explanations for each question, as well as evaluating the consistency between two given answers.
We utilize a Sentence transformer model (i.e., all-mpnet-base-v2)~\cite{reimers-2020-multilingual-sentence-bert} to identify the top-K similar questions for dataset augmentation.
After the data augmentation, we retain $21,928$ preference instances for model training.


For model training, we adopt the widely-used Mistral 7B (i.e., Mistral-7B-Instruct-v0.2)~\cite{jiang2023mistral7b} and Llama2-13B-Chat~\cite{touvron2023llama2openfoundation} as the base LLMs.
We apply DPO with Low-Rank Adapters (LoRA) \cite{hu2022lora} to train LLMs.
We conduct all our experiments on a GPU machine with 4 A800 NVIDIA RTX GPUs.
For model inference,  we use the same retriever BGE (i.e., bge-large-en-v1.5)~\cite{Xiao2024cpack} across all compared methods for a fair comparison.
We retrieve the top $5$ passages for each query from the external knowledge base.
Our experiments are conducted underboth zero-shot and few-shot settings. For the zero-shot setting, we follow the official settings of prior work\cite{asai2024selfrag,kim2024sure}.
For the few-shot setting, we include three examples in the prompts formatted according to the objective of each method.
Since the training data of Self-RAG includes the NQ dataset, we do not consider the results of Self-RAG on the NQ dataset in our experiments.

\begin{figure*}[t]
\setlength\abovecaptionskip{0px}
\setlength\belowcaptionskip{0px}
  \centering
  \includegraphics[width=\linewidth]{figs/bm25_full_2.pdf}
  \caption{An illustration of the effects of using a different retriever. Our \approach~and Standard RAG both use BM25 as the new retriever. We adopt Mistral-7B as the base LLM for Standard RAG (7B) and Self-Selection-RGP (7B), and Llama2-13B-Chat for Standard RAG (13B) and Self-Selection-RGP (13B). } 
  \label{fig:retriever}
\end{figure*}
\subsection{Main Results (RQ1)}
\label{sec:main-result}

To verify the effectiveness of the proposed \framework~ framework, we compare the performance with the baseline methods.
In Table~\ref{tab:main}, we present the main results on the NQ and TriviaQA datasets with Mistral-7B~\cite{jiang2023mistral7b} and Llama2-13B-Chat~\cite{Touvron2023LLaMA}, under both zero-shot and few-shot settings. 
From the results, we make several key observations: 

\begin{itemize}[leftmargin=*,nosep]
\item With Mistral-7B as the base LLM, our Self-Selection-RGP method consistently outperforms all compared methods on NQ and TriviaQA datasets under both zero-shot and few-shot settings.
On the TriviaQA dataset, our Self-Selection-RGP model achieves an accuracy of $67.0$ in the zero-shot setting and $65.4$ in the few-shot setting, making significant improvements of $5.4$ and $4.0$ points, respectively, compared to the best baseline method SURE which scores  $61.6$ and $61.4$.
In contrast, the performance gains of our Self-Selection-RGP method on the NQ dataset are relatively smaller.
Our method scores $53.6$ and $53.2$ in accuracy under zero-shot and few-shot settings, respectively, which makes an improvement of $2.6$ and $2.6$ over the best baseline method Standard RAG which scores $51.0$ and  $50.6$.
These substantial performance improvements over prior methods highlight the effectiveness of our proposed Self-Selection-RGP method.
\item With LLama2-13B-Chat as the base LLM, our Self-Selection-RGP method consistently delivers strong performance on both TriviaQA and NQ datasets. 
In particular, Self-Selection-RGP exhibits superior performance on the TriviaQA dataset, achieving an accuracy of $66.0$ and $66.8$ under zero-shot and few-shot settings, respectively. 
These results reflect improvements of $2.2$ and $1.8$ over the best baseline method SURE, which scores $63.8$ and $65.0$ in accuracy.
In comparison, our Self-Selection-RGP demonstrates competitive performance on the NQ dataset relative to the SURE method.
In the few-shot setting, Self-Selection-RGP attains an accuracy of $51.6$, resulting in a $1.2$ improvement over SURE’s accuracy of $50.4$. 
However, in the zero-shot setting, our Self-Selection-RGP scores $46.2$, which is $5.8$ lower than SURE’s accuracy of $52.0$.
This performance disparity may be attributed to the fact that the questions in the NQ dataset are inherently more challenging for LLMs compared to those in TriviaQA, as evidenced by the performance of the LLM-only approach in \tabref{tab:main}.
This complexity hampers our method's ability to effectively distinguish the correct answers from the incorrect ones.
\item In comparison, the Self-Selection-Ori method demonstrates superior performance on the TriviaQA dataset in both zero-shot and few-shot settings when utilizing Mistral-7B as the base LLM, surpassing all baseline methods.
However, the Self-Selection-Ori method encounters difficulties in producing accurate results on the NQ dataset or when using LLama2-13B-Chat as the base LLM. 
The potential reasons for the performance disparity are twofold: 1) Mistral-7B exhibits more advanced reasoning capabilities compared to the LLama2-13B series models, as supported by the comparative analysis presented in the Mistral-7B technical report~\cite{jiang2023mistral7b}; and 2) the questions in the NQ dataset are relatively more challenging than those in TriviaQA for LLMs, as aforementioned.
\end{itemize}



\subsection{Impact of Different Retrieval Settings (RQ2)}
\subsubsection{\textbf{ Effect of Different Retriever}}
We experiment to demonstrate the compatibility of the proposed \framework~framework with different retrieval methods. 
Specifically, beyond the BGE retriever considered in Table \ref{tab:main}, we also use BM25 as the retriever, and compare our \approach~method with Standard RAG to analyze whether our method can still maintain superiority in performance over Standard RAG with the new retriever.
We adopt Mistral-7B and Llama2-13B-Chat as the base language models, respectively.
This results in the following four RAG systems with BM25 as their retriever: Standard RAG (7B) and Self-Selection-RGP (7B), where the base LLM is Mistral-7B, as well as Standard RAG (13B) and Self-Selection-RGP (13B), where the base LLM is Llama2-13B-Chat.
We present the performance of these four systems on both the NQ and TriviaQA datasets under zero-shot and few-shot settings in \Figref{fig:retriever}.
From the figure, we make the following observations: 

\begin{itemize}[leftmargin=*,nosep]
\item  With BM25 as the retriever, our proposed Self-Selection-RGP method consistently outperforms the Standard RAG method in each setting as illustrated in \Figref{fig:retriever}, revealing the compatibility and generalizability of our method regarding new retrieval techniques like BM25.
To be more specific, as shown in \Figref{fig:retriever} (a) and (c), in the zero-shot setting, our Self-Selection-RGP (7B) and Self-Selection-RGP (13B) outperform the Standard RAG (7B) and Standard RAG (13B) models by substantial margins on all three metrics over both NQ and TriviaQA datasets. 
This highlights the significant effectiveness of our Self-Selection-RGP method in the zero-shot scenarios.
Comparably, as shown in \Figref{fig:retriever} (b) and (d), in the few-shot settings, the performance gains achieved by our Self-Selection-RGP method are relatively small. 
This may be because the samples available in the few-shot scenarios can provide sufficient knowledge to LLMs, diminishing the relative advantage of our proposed approach compared to the more standard RAG method.

\item Our Self-Selection-RGP method demonstrates enhanced robustness and stability compared to the Standard RAG method. 
This is evident in the comparisons shown in \Figref{fig:retriever}, where the performance variation between zero-shot and few-shot settings for Self-Selection-RGP is significantly smaller than that observed for the Standard RAG method, as illustrated in both (a) versus (b) and (c) versus (d). 
This advantage stems from the Self-Selection-RGP's lower sensitivity to input noise or perturbations, achieved through its preference alignment training.
This makes Self-Selection-RGP more adaptable to diverse retrieval techniques, even when the retrieved passages are not precisely relevant. 
Therefore, the proposed Self-Selection-RGP is capable of producing more reliable and accurate results.
This improved resilience to input variation allows our Self-Selection-RGP to maintain more consistent performance across a wide range of task conditions, leading to its enhanced robustness and stability compared to the Standard RAG method. 
\end{itemize}

\begin{figure}[t]
\setlength\abovecaptionskip{-0.3px}
\setlength\belowcaptionskip{-8px}
  \centering
  \includegraphics[width=\linewidth]{figs/number.pdf}
  \caption{An illustration of the effects of varying the number of retrieved passages. } 
  \label{fig:num_passage}
  \vspace{-0.2cm}
\end{figure}


\subsubsection{\textbf{Effect of Number of Retrieved Passages}}
Next, we investigate the impact of the number of retrieved passages on the performance.
Simply increasing the number of retrieved passages by providing a broader range of external retrieval information has been one of the most straightforward and effective methods to improve the performance of retrieve-and-read systems~\cite{karpukhin-etal-2020-dense}. 
However, the effectiveness and applicability of this approach may be limited, as LLMs are constrained by the length of the input context~\cite{liu-etal-2024-lost}.
In the following, we evaluate the effects on our proposed methods by varying the number of retrieved passages, including $1$, $3$, $5$, $8$ and $10$.
Specifically, we adopt BGE and BM25 as the retriever and Mistral-7B as the base LLM in this experiment. 
We conduct a comparative analysis of performance across three RAG methods, i.e. Standard RAG, Self-Selection, and Self-Selection-RGP, on both NQ and TriviaQA datasets.
We summarize the experimental results in \Figref{fig:num_passage}.
From these results, we can make the following observations:
\begin{itemize}[leftmargin=*,nosep]
    \item Increasing the number of retrieved passages initially leads to noticeable improvements in the accuracy of all three RAG methods. However, once the number of retrieved passages exceeds a certain threshold, adding more passages has only marginal effects on the performance, and in some cases even degrades the overall performance.
    This may be due to information overload within the constrained context window of LLMs, which impairs the model’s capacity to accurately generate the correct responses.
    \item On the TriviaQA dataset, our Self-Selection-RGP method consistently yields the highest performance across varying numbers of retrieved passages and different retrievers. The Self-Selection method ranks second, while the Standard RAG method performs the worst.
    The performance comparison reveals the effectiveness of our proposed Self-Selection framework.
    \item On the NQ dataset, our performance of the three methods are similar to that on TriviaQA, with Self-Selection-RGP consistently ranking first across different numbers of retrieved passages.
    When the retriever BGE is adopted, all the three methods exhibit competitive performance across varying numbers of retrieved passages. 
    Note that Self-Selection-RGP method achieves the best performance when the number of retrieved passages is set to $5$
    Compared to the LLM-only approach, Standard RAG significantly enhances the performance by integrating retrieved passages into the LLM input, thereby establishing a strong baseline, as shown in \tabref{tab:main}. 
    Our method can still achieve comparable performance to this strong baseline, well showcasing its remarkable effectiveness across a varying number of retrieved passages.
\end{itemize}
 
\subsection{Analysis of Answer Generation Capability of LLMs (RQ3)}
\label{sec:llm-improve}


According to the experimental results presented above, the LLMs trained with the preference dataset have demonstrated remarkable improvements in distinguishing the correct responses from the incorrect ones.
One question naturally arises: ``To what extent does this training process affect the LLMs’ inherent ability in answer generation with or without the dependence on external retrieved knowledge?''
To answer this question, we design an experiment to compare LLMs' answer generation performance before and after preference alignment training. 
We adopt Mistral-7B as the base LLM in this analysis.
We first apply \approach~to train Mistral-7B using the augmented RGP dataset, resulting in a trained LLM that is referred to as Self-Selection-RGP-7B.

\begin{figure}[t]
\setlength\abovecaptionskip{-0.4px}
\setlength\belowcaptionskip{-4px}
  \centering
  \includegraphics[width=\linewidth]{figs/ablation.pdf}
  \caption{Ablation study. ``Std RAG'' refers to Standard RAG; ``w/o Aug'' indicates the method without Dataset Augmentation; ``w/o Align'' denotes the method without Preference Alignment; and ``SS-RGP'' represents our proposed Self-Selection RAG method.} 
  \label{fig:ablation}
  \vspace{-0.4cm}
\end{figure}



\begin{figure*}[t]
\setlength\abovecaptionskip{-0.3px}
\setlength\belowcaptionskip{-0.4px}
  \centering
  \includegraphics[width=\linewidth]{figs/RQ3_2.pdf}
  \caption{Analysis of the answer generation capability before and after preference alignment training. ``Mistral-7B'' denotes the vanilla LLM before preference alignment training; ``Self-Selection-RGP-7B'' refers to the LLM obtained after training Mistral-7B with the preference alignment dataset.} % Add a caption if needed
  \label{fig:llm-ans}
\end{figure*}


To comprehensively compare the answer generation capabilities of \emph{Mistral-7B} and \emph{Self-Selection-RGP-7B}, we conduct extensive experiments on NQ and TriviaQA datasets, under both zero-shot and few-shot settings, with and without the use of external knowledge.
We present the experimental results in \Figref{fig:llm-ans}, from which we make the following observations:
\begin{itemize} [leftmargin=*,nosep]
    \item As shown \Figref{fig:llm-ans} (a) and (b), without depending on external knowledge, the Self-Selection-RGP-7B exhibits improved capabilities in answer generation on TriviaQA while showing slightly worse performance on NQ, compared to the Mistral-7B, in both zero-shot and few-shot settings.  
    \item From \Figref{fig:llm-ans} (c) and (d), the Self-Selection-RGP-7B consistently outperforms Mistral-7B on both datasets when involving the retrieved relevant passages in their inputs, showing its enhanced answer generation abilities.
\end{itemize}
 
 The empirical findings presented above demonstrate that the Self-Selection-RGP-7B model exhibits enhanced capabilities not only in answer selection, but also in the broader task of answer generation.
 This suggests that the proposed Self-Selection-RGP method which trains LLMs on the augmented RGP dataset has led to notable improvements in LLMs' ability to generate high-quality answers, highlighting its potential for enhancing the overall response generation performance of LLMs.


\subsection{Ablation Study (RQ4)}

To verify the effect of each design in our proposed method, we conduct an ablation study with Mistral-7B on both NQ and TriviaQA datasets.
We utilize three methods for ablation experiments in addition to our proposed \approach~method, including 
1) \textbf{Standard RAG}, which appends the retrieved passages to the input of the vanilla LLM;
2) \textbf{w/o Dataset Augmentation}, which removes the step of dataset augmentation from our proposed method and trains Mistral-7B with the original RGP dataset only; 
3)\textbf{w/o Preference Alignment}, which removes the step of preference alignment for the LLM and applies the \framework~on the vanilla LLM.
All the methods are evaluated under the zero-shot setting for a fair comparison.
We present the results in \Figref{fig:ablation}, from which we make the following observations: 

\begin{itemize}[leftmargin=*,nosep]
\item Removing either the step of dataset augmentation or preference alignment results in performance drop in all evaluation metrics on both NQ and TriviaQA datasets, highlighting the rationale of the design in our proposed \approach~method. 
\item Comparably, the removal of preference alignment results in a more substantial decrease in performance, revealing the importance of teaching LLMs how to choose the correct answer from multiple candidates.
\item Compared to the Standard RAG method, our method consistently yields superior performance on both datasets, demonstrating its great effectiveness in enhancing RAG systems. 

\end{itemize}

\subsection{Error Analysis}
We conduct an error analysis to investigate the limitations of our \approach~method. 
With the Mistral-7B as the base LLM, we sample $100$ from the errors made by our \approach~method on the TriviaQA dataset to conduct our analysis. 
We categorize the errors into five groups, as shown in ~\tabref{tab:error-case}, each with an example:
(1) Lack of Evidence (51\%): The LLM itself does not have sufficient internal knowledge to answer the question, and the retrieved passages also fail to provide enough information;
(2) Partial Matching (20\%): The final prediction captures part of the correct answer only;
(3) Reasoning Error (14\%): The generated explanation(s) contain the answer or the relevant information to infer the answer, but the LLM fails to predict the answer;
(4) Selection Error(12\%): One of the pairwise predictions is correct,  but the model fails to identify the correct one;
(5) Formatting Error(3\%): The correct answer is included in the prediction but the output format does not follow the instruction, leading to a failed interpretation.


From the table, we make the following observations:
\begin{itemize}
    \item  Over half of the errors are caused by the failure to obtain the knowledge that is necessary for predicting the answer, i.e., Lack of Evidence. This reflects the importance of developing advanced techniques to fetch the relevant information given a question from external knowledge bases to complement LLMs' internal knowledge for producing more accurate results.
    \item Approximately $39\%$ of errors, including Partial Matching, Reasoning Errors, and Formatting Errors, originate from LLMs' inadequacies in accurately interpreting the human instruction or the relevant knowledge required to infer the precise answer in the correct format. To address this issue, LLMs with enhanced reasoning capabilities are required.  
    \item $12\%$ errors arise from the LLMs' inability to effectively distinguish the correct answer from plausible ones, underscoring the demand for LLMs with enhanced reasoning capabilities. 
\end{itemize}














\section{Case studies}\label{sec:cases}
We now review the relationship between
merge-width and previously studied graph parameters.
First, it is not difficult to check that 
$\mw_\infty(G)$ is functionally equivalent to the clique-width of $G$. 
\begin{theorem}\label{thm:cw}
  A graph class $\CC$ has bounded clique-width if and only if $\mw_\infty(\CC)<\infty$.
\end{theorem}


    It is not difficult and quite instructive to prove the result directly,
    by translating between construction sequences and clique-width expressions, and vice-versa.
    Such a proof would provide explicit (linear) bounds in each direction.
    To be more succinct, we derive the statement without explicit bounds, by deriving it from results proved later in \Cref{sec:cases}.

\begin{proof}[Proof sketch]

    For the left-to-right implication,
    let $G$ be a graph of clique-width $k$.
    By \cite[Thm. 1]{tww6}, $G$ has a contraction sequence $\cal P_1,\ldots,\cal P_n$, such that at every 
    step $i\in\set{1,\ldots,n}$ of the sequence,
    every connected component of the red graph of $\cal P_i$ contains 
    at most $k'$ parts, for some constant $k'\in\N$ depending on $k$.
    By transforming the contraction sequence into a merge sequence as in the proof of \Cref{lem:tww} below, by \eqref{eq:tww-mw} we derive that the resulting construction sequence has radius-$\infty$ width at most $k'$. Thus, $\mw_\infty(G)\le k'$.
    This proves the left-to-right implication.

    Conversely, suppose that $\mw_\infty(\CC)<\infty$.
    It follows from \Cref{thm:fw-cases} below that $\fw_\infty(\CC)<\infty$.
    Therefore, $\CC$ has bounded clique-width, by \cite[Thm. II.6]{flip-width}.
\end{proof}






\noindent Thus, in a sense, the finite-radius merge-width parameters are local variants of clique-width.













\subsection{Merge-width and twin-width}\label{sec:tww}
We first discuss the relationship of merge-width and twin-width,
whose definition we recall now.
Recall that a \emph{contraction sequence} of a graph $G$ is a sequence of merge operations, which starts with the partition of $V(G)$ into singletons, and ends with the partition with one part.
Two sets $A,B\subset V(G)$ of vertices are \emph{homogeneous} if $AB\subset E(G)$ or $AB\cap E(G)=\emptyset$.
The \emph{red graph} of a partition $\cal P$
is the graph with vertices $\cal P$ and edges connecting pairs $\set{A,B}\in{\cal P\choose 2}$ 
which are not homogeneous. The twin-width of a graph $G$ is the minimum number $d$ 
such that $G$ admits a contraction sequence such that at every step, the red graph of the current partition has maximum degree at most $d$.


\twwintro*
\Cref{thm:tww} follows immediately from the next lemma.
\begin{lemma}\label{lem:tww}
  Fox every $r,d\in \N$ and graph $G$ of twin-width $d$, we have 
  $$\mw_r(G)\le 2+d+\ldots+d^r.$$
\end{lemma}
\begin{proof}
Fix a contraction sequence $\cal P_1,\ldots,\cal P_n$ of $G$, such that
the red graph of each partition $\cal P_t$ has maximum degree at most $d$.
For a vertex \(v\), denote by \(\cal P_t(v)\) the part of \(\cal P_t\) containing \(v\).
For $t\in[n]$, let $R_t$ consist of those pairs $ab\in {V\choose 2}$ such that $\cal P_t(a),\cal P_t(b)$ are either equal, or not homogeneous in $G$.
Then 
  $(\cal P_1,R_1),\ldots,(\cal P_n,R_n),$
  is a merge sequence of $G$. We bound its radius-$r$ width, for fixed $r\in\N$.

  For every $t\in[n]$, part $A\in\cal P_t$ and vertex $a\in A$, 
  we have that 
  \begin{align}\label{eq:tww-mw}
  B^r_{R_t}(a)\subset \bigcup B^r_{\cal P_t}(A),  
  \end{align}
  where $B^r_{R_t}(a)$ is the ball of radius $r$ around $a$ in the graph $(V,R_t)$, 
  and $B^r_{\cal P_t}(A)$ is the ball of radius $r$ around $A$ in the red graph of $\cal P_t$.
  In particular, $B^r_{R_t}(a)$ intersects at most $|B^r_{\cal P_t}(A)|$ parts of $\cal P_t$.
  As degree in the red graph is at most \(d\), the radius-$r$ width of $(\cal P_t,R_t)$ is at most $|B^r_{\cal P_t}(A)|\le 1+d+\ldots+d^r$.
  Since $\cal P_{t-1}$ is a refinement of $\cal P_t$ with exactly one more part,
  it follows that 
  the radius-$r$ width of $(\cal P_{t-1},R_t)$ is at most $2+d+\ldots+d^r$.
Consequently,
$\mw_r(G)\le 2+d+\ldots+d^r.$
\end{proof}


\subsection{Merge-width and Sparsity}\label{sec:sparsity}
The converse to \Cref{thm:tww} is false. 
For instance,  the class 
of graphs of maximum degree bounded by $d$ (a class which has unbounded twin-width already for $d=3$) 
has bounded merge-width.
Indeed, there is a trivial merge sequence for such a graph $G$: 
 $$\bigl(\textit{partition into singletons},\emptyset\bigr),\quad \bigl(\textit{partition with one part},E(G)\bigr).$$
The radius-$r$ width of this sequence is 
the maximum size of a radius-$r$ ball in $G$,
which is at most $1+d+\cdots+d^r$, 
so 
$\mw_r(G)\le 1+d+\cdots+d^r$.



If the above merge sequence for graphs of bounded degree seems uninsightful,
this reflects the fact that graphs of bounded degree are trivial from the perspective of Sparsity theory.
The next result is slightly more 
illuminating.
The proof is already given in \Cref{ex:degeneracy}.




\begin{theorem}\label{thm:deg}
  If $G$ is a $d$-degenerate graph then 
  $\mw_1(G)\le d+2.$
\end{theorem}


Next, we prove:
\beintro*

The simple construction from \Cref{thm:deg} does not trivially generalize to bound the radius-$r$ merge-width 
of graphs in classes of bounded expansion,
as it appears that to bound $\mw_r(G)$ for $r>1$, a more involved construction 
is needed, which we now present.

\medskip

We first recall the definition of classes of bounded expansion for completeness only, as we will not use the original definition, only the characterization given in the upcoming \Cref{fact:be}.
By definition,  a graph class $\CC$ has \emph{bounded expansion} if and only if for every $r\in\N$,
there is some $k\in\R$ such that for every graph $G$
which can be obtained from some graph in $\CC$ by first removing vertices and edges, and then contracting 
some vertex subsets of radius ${\le r}$ to single vertices, 
we have that $|E(G)|\le k\cdot |V(G)|$.
Classes of bounded expansion include every class of bounded maximum degree,
the class of planar graphs, and every graph class that excludes some graph  as a  minor, or as a topological minor.


Fix a graph $G$, number $r\in\N$, and total order $\le$ on $V(G)$. 
A vertex \(u\) is \emph{weakly \(r\)-reachable} from a vertex \(v\) if there is a path of length at most \(r\) from \(v\) 
to \(u\) and all vertices on that path are at least as large as \(u\).
Let \(\wreach_r(G,\le,v)\) be the set of vertices that are weakly \(r\)-reachable from \(v\) with respect to the ordering.
Then the weak \(r\)-coloring number of \(G\) is defined as
\[
\wcol_r(G) \coloneqq \min_{\le} \max_{v \in V(G)} |\wreach_r(G,\le,v)|,
\]
where the minimum ranges over all total orders $\le$ of $V$.

\begin{lemma}[\cite{zhu2009colouring}]\label{fact:be}
A graph class $\CC$ has bounded expansion if and only if $\wcol_r(\CC)<\infty$, for all~${r\in\N}$.
\end{lemma}













\Cref{thm:be} follows immediately from the next lemma,
which 
shows that the 
 radius-$r$ merge-width 
is bounded in terms of the weak $(r+1)$-coloring number. 


\begin{lemma}\label{lem:wcol}
  For every $k,r\in\N$ and graph $G$ with $\wcol_{r+1}(G)\le k$, we have
  $$\mw_r(G)\le 3\cdot 2^{k} .$$
\end{lemma}
In particular, we show that $\mw_2(G)$  is bounded in terms of $\wcol_3(G)$,
and we do not know whether it is bounded in terms of $\wcol_2(G)$.

\begin{proof}Fix $r\in\N$, and a graph $G$ with vertex set $V$.
Let $\le$ be an ordering of the vertex set $V$ of $G$ 
such that for every vertex \(v\), the weak $r$-reachability set $\wreach_{r+1}(G,\le,v)$ has size at most $k$.

For $t=1,\ldots,n$, define the following.
\begin{itemize}
  \item Let $L_t$ comprise the $t$ largest elements of $V$ with respect to $\le$, and let $S_t\coloneqq V-L_t$.
  \item Let  $\cal P_t$ be the partition of $V$ into \emph{atomic types}
  over the set $S_t$. That is, $\cal P_t$ partitions $S_t$ into singletons, and partitions vertices  $v\in L_t$ according to their neighborhood $N(v)\cap S_t$~in~$S_t$.
  \item Let $R_t \coloneqq\setof{ab\in E(G)}{a,b\in L_t}$ be the set of all edges in $G$ with both endpoints in $L_t$.
\end{itemize}
We verify that $(\cal P_1,R_1),\ldots,(\cal P_n,R_n)$ is a merge sequence of $G$. Clearly, $\cal P_1$ is the partition into singletons, 
and $\cal P_n$ has one part.
Observe that for $t\in[n]$ and any two parts $A,B$ of $\cal P_t$,
either $A,B\subset L_t$, and therefore  $E(G)\cap AB\subset R_t$,
or, otherwise, at least one of $A,B$ is contained in $S_t$, and then $A$ and $B$ are homogeneous,
as both $A$ and  $B$ are atomic types over $S_t$. 
In any case, $AB-R_t\subset E(G)$ or 
$AB-R_t\subset {V\choose 2}-E(G)$, so the conditions of a merge sequence are satisfied.


Fix $t\in[n]$. Fix two vertices $v,w\in L_t$, such that 
there is a path of length at most $r$ in $(V,R_t)$ from $v$ to $w$.
Then there is a path $\pi$ of length at most \(r\) from $v$ to $w$ in $G$
contained in $L_t$.
In particular, for every vertex $u\in N_G(w)\cap S_t$ 
there is a path from $v$ to $u$ of length at most $r+1$ in \(G\) (namely the path $\pi$ followed by the edge $wu$) such that every inner vertex on that path is in $L_t$.
In other words, $N_G(w)\cap S_t \subseteq \wreach_{r+1}(G,\le,v)$.
 As $|\wreach_{r+1}(G,\le,v)|\le k$ by assumption, and 
 the atomic type of $w$ over $S_t$ is uniquely 
determined by $N(w)\cap S_t$, it follows that 
there are at most $2^k$ parts of $\cal P_t$ 
that are reachable by a path of length at most $r$ from $v$.
Thus, the radius-$r$ width of $(\cal P_t,R_t)$ is at most
$2^k$.



To bound the radius-$r$ width of $(\cal P_{t-1},R_t)$, for $t>1$, notice that every part in $\cal P_{t}$ 
is a union of at most three parts in $\cal P_{t-1}$.
 Namely, if $v$ is the unique vertex with $v\in S_{t-1}-S_{t}$, then the atomic type $A$ of a vertex $u$ over $S_{t}$ 
 is uniquely determined by the atomic type of $u$ over $S_{t-1}$ and the atomic type of $u$ over $\set v$. 
 There are three 
 possible atomic types over $\set v$, corresponding 
 to
 whether $u=v$, or $(u\neq v)\land E(u,v)$, or 
$(u\neq v)\land \neg E(u,v)$. This proves that $A$ is a union of at most three parts in $\cal P_{t-1}$.

It follows that the radius-$r$ width of $(\cal P_{t-1},R_t)$ is at most $3\cdot 2^k$, and we have constructed a merge sequence of $G$ 
of radius-$r$ width at most $3\cdot 2^k$.
\end{proof}























 






\subsection{Merge-width and flip-width}\label{sec:fw}
We study the relationship between merge-width and flip-width, whose definition we now recall.

For a graph $G$ and 
two sets $A,B\subset V(G)$,
\emph{flipping} the pair $(A,B)$ in $G$ 
results in the graph $G'$ 
with edges $E(G')=E(G)\triangle AB$, where \(\triangle\) denotes the symmetric difference.
Given a partition $\cal P$ of $V(G)$, 
a \emph{$\cal P$-flip} of $G$ is a graph $G'$
obtained from $G$ by flipping arbitrary pairs $A,B\in\cal P$ (possibly with $A=B$).
Thus, a graph $G'$ is a $\cal P$-flip of $G$ if and only if for every pair $A,B\in\cal P$, 
either $E(G')\cap AB=E(G)\cap AB$ or $E(G')\cap AB=AB-E(G)$.
A \(k\)-flip is a \(\cal P\)-flip with \(|\cal P| \le k\).

We recall the definition of flip-width from \cite{flip-width}.
Fix $k,r\in\N$.
The \emph{flip-width game} of radius $r$ and width $k$ is played on a graph $G$ between two players, flipper and runner.
Initially, runner picks a vertex $v_0\in V(G)$.
In round $t=1,2,\ldots$,
    flipper announces a $k$-flip $G_t$ of $G$;
    then runner picks $v_t\in B^r_{G_{t-1}}(v_{t-1})$ -- that is, they traverse 
    a path of length at most $r$ in the \emph{previous} graph $G_{t-1}$.
    Flipper wins if $v_t$ is isolated in $G_t$.
The \emph{radius-$r$ flip-width} of $G$, denoted $\fw_r(G)$,
is the least number $k$ such that flipper has a winning strategy for the flip-width game of width $k$ and radius $r$ on $G$.
We prove:
\begin{theorem}\label{thm:fw-cases}
  Every class of bounded merge-width has bounded flip-width.
  More precisely, for all $r\in\N$ and every graph $G$,
  $$\fw_r(G)\le 4^{\mw_{2r-1}(G)}.$$
\end{theorem}

\Cref{thm:fw-cases} allows to deduce several results about classes of bounded merge-width
from the corresponding results for classes of bounded flip-width.
For instance, we obtain the two corollaries stated in the introduction:
\corbe*
The forward implication in \Cref{cor:be} follows from \Cref{thm:be} and the trivial fact that each class of bounded expansion excludes some biclique as a subgraph. The backwards direction follows from \Cref{thm:fw-cases} and from the analogous statement for classes of bounded flip-width, proved in \cite[Thm. VI.3]{flip-width}.


\cortww*
In \Cref{cor:tww-mw}, we use the notion of merge-width for binary structures, specifically for \emph{ordered graphs} --  graphs equipped with a total order on the vertex set. This is defined in \Cref{sec:computing}.  \Cref{thm:fw-cases} applies to classes binary structures, allowing to derive \Cref{cor:tww-mw} by a similar reasoning as for \Cref{cor:be} above:
The forward direction uses \Cref{thm:tww}and the fact that every graph can be extended with a total order, without increasing its twin-width \cite{tww4}, and the backwards direction uses \Cref{thm:fw-cases} and the analogue of \Cref{cor:tww-mw} proved in \cite[Thm. VII.3]{flip-width}.
For simplicity, below we present the proof of \Cref{thm:fw-cases} only for graph classes. The more general case is analogous, 
but the requires notion of flips and of flip-width for binary structures as defined in \cite[Sec. V.B]{flip-width}.

We state another corollary that follows from \Cref{thm:deg} and an analogous result from \cite[Thm. VI.1]{flip-width}.
\begin{corollary}\label{cor:deg}
  Let $\CC$ be a graph class. Then $\CC$ has bounded degeneracy if and only if $\mw_1(\CC)<\infty$ and $\CC$ excludes some biclique $K_{t,t}$ as a subgraph.
\end{corollary}




\medskip
To prove \Cref{thm:fw-cases},
we start with a simple observation that allows us to rephrase the main condition from the definition 
of a merge sequence in terms of flips, as follows.


\begin{lemma}\label{lem:flip}
    Let $G=(V,E)$ be a graph, let $\cal P$ be a partition of $V$, and let $R\subset {V\choose 2}$.
    The following conditions are equivalent:
    \begin{enumerate}
        \item for all $A,B\in\cal P$, either $AB-R\subset E$, or $AB-R\subset AB - E$,
        \item  there is a $\cal P$-flip $G'$ of $G$ with $E(G')\subset R$.
    \end{enumerate}
    \end{lemma}
    \begin{proof}
        (1$\rightarrow$2).
        The $\cal P$-flip $G'$
        is obtained from $G$ by flipping each pair $A,B\in\cal P$ such that $AB-R\subset E$. As a result $AB-R\subset AB-E(G')$, 
        and therefore $E(G')\cap AB\subset R$ for each $A,B\in\cal P_t$.
        Altogether, $E(G')\subset R$.
    
        (2$\rightarrow$1). Fix $A,B\in\cal P$. We have that
        $E(G')\cap AB\subset R\cap AB$,
        so $AB-R\subset AB-E(G')$.
        As $G'$ is a $\cal P$-flip of $G$,
        we have that either $E(G')\cap AB=E(G)\cap AB$ or $E(G')\cap AB=AB-E(G)$.
        Altogether, either $AB-R\subset E(G)\cap AB$, or  $AB-R\subset AB-E(G)$.
    \end{proof}
    
Using \Cref{lem:flip}, we may therefore redefine merge-width in terms of flips, as follows.




\begin{definition}Let $G$ be a graph.
A \emph{restrained flip sequence} for $G$ is a sequence
\begin{align}\label{eq:mono-fw}
    (\cal P_1,R_1,G_1),\ldots,(\cal P_m,R_m,G_m)    
\end{align}
such that:
\begin{itemize}
    \item $\cal P_1,\ldots,\cal P_m$ is a refining sequence of partitions, starting with the  partition $\cal P_1$ of $V(G)$ with one part, and ending with the partition $\cal P_m$ into singletons,
    \item the graph $G_t$ is a $\cal P_{t}$-flip of $G$, for $t\in [m]$,
    \item ${V\choose 2}= R_1\supseteq \cdots \supseteq R_m=\emptyset$, and
    \item  $E(G_t)\subset R_t$ for $t\in[m]$.
\end{itemize}
The radius-$r$ width of the restrained flip sequence is 
the maximum, over $t\in[m-1]$,
of the radius-$r$ width of $(\cal P_{t+1},R_{t})$.
\end{definition}

Note that in the reformulation above,
the partitions $\cal P_1,\ldots,\cal P_m$ are becoming finer with each step, instead of becoming coarser as in merge sequences. In each step of a restrained flip sequence, we provide a $\cal P_t$-flip $G_t$.
The sequence $R_1\supseteq \cdots\supseteq R_m$
is a descending sequence of subsets of $V\choose 2$ with $R_m=\emptyset$. The set $R_t$ is called the \emph{restraint} at time $t$, as we require that $E(G_s)\subset R_t$ for all $s\in[m]$ with $s\ge t$, thus restraining all the future graphs $G_t,G_{t+1},\ldots,G_m$.  Note that  $E(G_m)\subset R_m=\emptyset$, so $G_m$ is edgeless.

As $G_t$ is a $\cal P_t$-flip of $G$, 
$G_{t-1}$ is a $\cal P_{t-1}$-flip of $G$, and $\cal P_{t-1}$ is coarser than $\cal P_t$,
we may equivalently require that $G_t$ as a $\cal P_{t}$-flip of $G_{t-1}$, rather than of $G$.
Therefore, in the sequence $G_1,G_2,G_3,\ldots$ 
each subsequent graph is a flip of the previous graph.
This is similar to the setting of flipper games considered in \cite{flippers}.




\begin{lemma}\label{lem:mw-flip seq}
    For every graph $G$,
    there is a correspondence between merge sequences and restrained flip sequences for $G$,
    which preserves the length, and the radius-$r$ width of the sequences, for each $r\in\N\cup\set{\infty}$.
\end{lemma}




\begin{proof}
    Fix a merge sequence 
    $(\cal P_1,R_1),\ldots,(\cal P_m,R_m)$
    of $G$.
 By \Cref{lem:flip} (1$\rightarrow$2),
 for each $t\in[m]$,
    there is a  $\cal P_t$-flip $G_t$ of $G$ such that 
$E(G_t)\subset R_t$.
    The sequence 
    $(\cal P_m,R_m,G_m),\ldots,(\cal P_1,R_1,G_1)$
    is a restrained flip sequence for $G$.
    
    Conversely,
    given a restrained flip sequence for $G$ of the form $(\cal P_1,R_1,G_1),\ldots,(\cal P_m,R_m,G_m)$,
    by \Cref{lem:flip} (2$\rightarrow$1) we conclude that 
    the sequence $(\cal P_m,R_m),\ldots,(\cal P_1,R_1)$ is a merge sequence of $G$.
\end{proof}










The next lemma shows that radius-$r$ flip-width is bounded in terms of radius-$(2r-1)$ merge-width, thus proving \Cref{thm:fw-cases}.




\begin{lemma}\label{lem:fw}
    Fix $r,s\in\N$. 
    For every graph $G$, if $\mw_{2r-1}(G)\le s$ then  $\fw_r(G)\le 4^s$. 
\end{lemma}

\begin{proof}
    Let $G$ be a graph with $\mw_{2r-1}(G)\le s$. Then $G$ has a merge sequence of radius-$(2r-1)$ merge-width at most $s$.
    By \Cref{lem:mw-flip seq}, $G$ has a 
 restrained flip sequence of radius-$(2r+1)$ width at most \(s\) of the form
$$(R_1,\cal P_1,G_1),\ldots,(R_n,\cal P_n,G_n).$$

The following lemma will allow us to construct a strategy for flipper in the flip-width game.

\begin{lemma}\label{lem:strategy}
    Fix $t\in [2,n]$, and 
    let $s$ be the radius-$(2r-1)$ width of $(\cal P_t,R_{t-1})$.
    For every $v\in V(G)$ there is a $4^s$-flip $G_t'$ of $G$ such that:
$$B^r_{G_t'}(w)\subset B^r_{G_t}(w)\quad\text{for every $w\in B^r_{G_{t-1}}(v)$.}$$
\end{lemma}
Before proving \Cref{lem:strategy}, we first show how it yields a winning strategy
for flipper in the flip-width game of radius $r$
    and width $4^s$.
Flipper's strategy will be to announce graphs $G_1',G_2',\ldots$, ensuring  that following invariant holds after round $t\ge 1$ of the game:
    \begin{align}\label{eq:fw-invariant}
    B^r_{G_{t}'}(v_t)\subset B^r_{G_t}(v_t).    
    \end{align}
    
    In the first round, when $t=1$, flipper announces $G_1'=G_1$ and runner picks a vertex 
    $v_1$,
    and the invariant holds trivially.
    Suppose the invariant is satisfied after round $t-1$ of the game,
    that is, 
    $$B^r_{G_{t-1}'}(v_{t-1})\subset B^r_{G_{t-1}}(v_{t-1}).$$
    Now flipper announces the $4^s$-flip $G_t'$ of $G$
    given by \Cref{lem:strategy} for $v \coloneqq v_{t-1}$.
    Next, runner picks a vertex $v_t\in B^r_{G_{t-1}'}(v_{t-1})$.
In particular, $v_t\in B^r_{G_{t-1}}(v_{t-1})$, so 
 \eqref{eq:fw-invariant} holds by \Cref{lem:strategy}, and the invariant is fulfilled.

    Playing according to this strategy, flipper wins within $n$ rounds,
    since $E(G_n)=\emptyset$, and therefore $v_n$ is isolated in $G_n'$
    by \eqref{eq:fw-invariant}. This proves that $\fw_r(G)\le 4^s$, and thus \Cref{lem:fw}.
\end{proof}

\begin{remark}\label{rem:duration}
    Observe that the number of rounds needed by flipper to win in the flip-width game of radius $r$ and width $4^s$ can be bounded by $|V(G)|$.
    Namely, the proof of \Cref{lem:fw} above shows that 
    the number of rounds is (at most) equal to the length $n$ of a restrained flip sequence 
     of radius-$(2r+1)$ width at most \(s\) for $G$.
     By \Cref{lem:mw-flip seq}, this corresponds to the length of a radius-$(2r+1)$ merge sequence of width $s$. Any merge sequence of $G$ can be converted into one of length $n=|V(G)|$, while preserving its radius-$r$ width for each $r\in\N$, since if for two consecutive pairs $(\cal P_i, R_i),(\cal P_{i+1},R_{i+1})$ in a merge sequence we have $\cal P_i=\cal P_{i+1}$, then $(\cal P_{i+1},R_{i+1})$ can be dropped from the merge sequence.
\end{remark}

We now prove \Cref{lem:strategy}. 



    \begin{proof}[Proof of \Cref{lem:strategy}]
        Fix $v\in V(G)$. For $i=0,\ldots,2r-1$, let 
 $\cal Q_i\subset \cal P_t$ consist of all parts $A\in\cal P_t$ that can be reached by a path of length at most $i$ by from $v$ in the graph $(V,R_{t-1})$.
In particular, $|\cal Q_{2r-1}|\le s$.

Let $G_t'=(V,E')$ be the graph 
such that for any two parts $A,B\in\cal P_t$:
$$E'\cap AB=
\begin{cases}
    E(G)\cap AB&\text{if $A,B\notin\cal Q_{2r-1}$}\\
    E(G_t)\cap AB&\text{otherwise}.
\end{cases}$$

\begin{claim}\label{cl:k-flip}
    $G_t'$ is a $4^s$-flip of $G$.
\end{claim}
\begin{claimproof}
    As $G_t$ is a $\cal P_t$-flip of $G$, so is $G_t'$.
    Therefore, for any two parts $A,B\in\cal P_t$, 
    the set $E'\cap AB$ is either equal to $E\cap AB$,
    or to $AB-E$.

    For every part $A\in\cal Q_{2r-1}$, let $U(A)$ be the union of all parts 
    $B\in\cal P_t$ such that $E'\cap AB= AB-E$.
    Then for every $B$ with $B\subset U(A)$ or $B\subset V- U(A)$,
    we have that
    $E'\cap AB=E\cap AB$ or 
     $E'\cap AB=AB-E$.

    Consider the set family:
    $$\cal F\coloneqq \cal Q_{2r-1} \cup \setof{U(A)}{A\in \cal Q_{2r-1}}.$$ Then $|\cal F|\le 2|\cal Q_{2r-1}|\le 2s$.
    
    Let $\cal Q$ be the partition 
    of $V$ such that 
    two vertices $a,b$ are in the same part of $\cal Q$ if and only if $a$ and $b$ belong to the same sets in $\cal F$.
    Then $|\cal Q|\le 2^{|\cal F|}\le 4^{s}$.
    Moreover, for every $A,B\in \cal Q$,
    either $E'\cap AB=E\cap AB$,
    or $E'\cap AB=AB-E$.
    It follows that 
    $G_t'$ is a $4^s$-flip of $G$.
\end{claimproof}



\begin{claim}\label{cl:balls}
    For all $w\in B^r_{G_{t-1}}(v)$
 and $i=0,\ldots,r$ we have:
    $$B^i_{G_t'}(w)\subset B^i_{G_t}(w).$$
\end{claim}

\begin{claimproof}
    We induct on $i$. For $i=0$ the statement is trivial.
In the inductive step, fix $1\le i\le r$ and $u\in B^i_{G'}(w)$ with $u\neq w$.
Then there is some  $a\in B^{i-1}_{G'}(w)$ with $ua\in E(G')$.
By inductive assumption, $a\in B^{i-1}_{G_t}(w)$.
From  $i\le r$ and  $E(G_t)\subset R_{t-1}$
we get that $a\in B^{r-1}_{R_{t-1}}(w)$,
which together with $w\in B^{r}_{G_{t-1}}(v)\subset B^{r}_{R_{t-1}}(v)$ implies $a\in B^{2r-1}_{R_{t-1}}(v)$. In particular, $a\in \bigcup \cal Q_{2r-1}$.
By definition of $E'$, this implies $ua\in E(G_t)$. Together with 
$a\in B^{i-1}_{G_t}(w)$, this implies that $u\in B^{i}_{G_t}(w)$, as required.
\end{claimproof}
\Cref{cl:k-flip} and \Cref{cl:balls} together prove \Cref{lem:strategy}.
\end{proof}




\subsection{Almost bounded merge-width}\label{sec:abmw}
A graph class $\CC$ has \emph{almost bounded merge-width}
if for every $r\in\N$ and $\eps>0$,
we have that $\mw_r(G)\le O_{\CC,r,\eps}(|V(G)|^\eps)$, for all $n\in\N$ and all $n$-vertex graphs $G\in\CC$.

Clearly, classes of bounded merge-width have almost bounded merge-width.
The following result states that all nowhere dense classes have almost bounded merge-width. We will prove this later below, by inspecting the proof of \Cref{thm:be}.

\thmnwd*

Classes of \emph{almost bounded flip-width} are defined analogously as classes of almost bounded merge-width, with $\mw_r(G)$ replaced with $\fw_r(G)$.
The following is the main result of this section.

\thmabmw*

As every hereditary class of almost bounded flip-width is monadically dependent \cite[Thm. 2.12]{flip-breakability}, we obtain the following.
\begin{corollary}\label{cor:abmw-mNIP}
    Every hereditary class of almost bounded merge-width is monadically dependent.
\end{corollary}

\Cref{thm:nwd} and \Cref{cor:abmw-mNIP} together justify our \Cref{conj:almostboundedmergewidth}
that almost bounded merge-width coincides with monadic dependence on hereditary classes.










\subsection{Preliminaries}
Before proving \Cref{thm:nwd} and \Cref{thm:abmw}, 
we first recall some basic notions.

\medskip
A \emph{set system} is a pair $(X,\cal F)$ with $\cal F\subset 2^X$.
Its \emph{VC-dimension} is the maximal size of a subset $Y\subset X$ such that 
$\setof{Y\cap F}{F\in\cal F}=2^Y$.
We recall the fundamental Sauer-Shelah-Perles lemma~\cite{sauer,shelah-sauer-lemma}.

\begin{lemma}[Sauer-Shelah-Perles lemma]\label{lem:sauer-shelah-perles}
  Let $(X,\cal F)$ be a set system of VC-dimension~$d$.
  Then $|\cal F|\le O(|X|^d)$.
\end{lemma}

The \emph{VC-dimension} of a graph $G$, denoted $\VCdim(G)$, 
is defined as the VC-dimension of the set system $\bigl(V(G),\setof{N(v)}{v\in V(G)}\bigr)$.
More explicitly, $\VCdim(G)$ is the maximal size of a subset $X\subset V(G)$
such that $\setof{N(v)\cap X}{v\in V(G)}=2^X$.

Define the \emph{atomic complexity} of a graph $G$, as the function $\pi_G\from\N\to\N$ defined as
 $$\pi_G(s)\coloneqq\max\Bigl\{|S|+\bigl|\{N_G(v)\cap S\,:\,v\in V(G)-S\}\bigr|\,:\, S\subset V(G), |S|\le s\Bigr\} \quad\text{for all }s \in \N.$$
(Equivalently, $\pi_G(s)$ is the maximal number of atomic types over a set $S$ of size at most $s$ in $G$. This is essentially equal to the \emph{neighborhood complexity}, up to an additive factor of $s$.) 
In particular, $\pi_G(s)\le s+2^s$, for all graphs $G$ and all $s\in\N$.
From \Cref{lem:sauer-shelah-perles}, we get the following.
\begin{corollary}\label{cor:sauer-shelah}
    Let $G$ be a graph of VC-dimension $d$.
    Then $\pi_G(s)\le O(s^d)$, for all $s\in \N$.
\end{corollary}



\subsection{Proof of \Cref{thm:nwd}}


To prove \Cref{thm:nwd}, we use the following result.
\begin{fact}[\cite{zhu2009colouring,sparsity-book}]\label{fact:nd-wcol}
    Let $\CC$ be a nowhere dense graph class, and fix $r\in\N$ and $\eps>0$.
    Then for every graph $G\in\CC$
    $$\wcol_r(G)\le O_{\CC,r,\eps}(|V(G)|^{\eps}).$$
\end{fact}

\begin{lemma}\label{lem:nd-VC}
    Let $\CC$ be a nowhere dense graph class. 
    Then there is some $d\in \N$ such that all graphs $G\in \CC$ have VC-dimension less than $d$.
\end{lemma}
\begin{proof}
    As $\CC$ is nowhere dense, there is some $d\in\N$ such that 
    no graph $G\in \CC$ contains the $1$-subdivision of the clique $K_d$, as a subgraph.
    It follows that every graph $G\in \CC$ has VC-dimension less than $d$.
\end{proof}


\begin{lemma}\label{lem:wcol-poly}
    For every $r\in\N\cup\set{\infty}$ and graph $G$, we have
    $$\mw_r(G)\le 3\cdot\pi_G(\wcol_{r+1}(G)).$$
  \end{lemma}
  \begin{proof}[Proof sketch]
    We follow the proof of \Cref{lem:wcol}, and use the same notation. 
    Let \(k\coloneqq\wcol_{r+1}(G)\).
    Fix $t\in [n]$, and let $L_t,S_t,\cal P_t,R_t,\le$ be as defined in the proof.
    Fix $v\in L_t$.
    It was observed that for every $w\in V(G)$ which is reachable from $v$ by a path of length $r$
in the graph $(V,R_t)$, 
 the atomic type of $w$ over $S_t$ is uniquely 
determined by $N_G(w)\cap S_t \subseteq \wreach_{r+1}(G,\le,v)$,
and moreover, $|\wreach_{r+1}(G,\le,v)|\le k$.
Since $$\Big|\setof{N_G(w)\cap \wreach_{r+1}(G,\le,v)}{w\in V(G)}\Big|\le  \pi_G(k),$$
we conclude that are at most $\pi_G(k)$ parts of $\cal P_t$ 
that are reachable by a path of length at most $r+1$ from $v$.
Thus, the radius-$r$ width of $(\cal P_t,R_t)$ is at most
$\pi_G(k)$. The rest of the argument remains unchanged, yielding the final bound $\mw_r(G)\le 3\cdot \pi_G(k)\le 3\cdot \pi_G(\wcol_{r+1}(G))$.
  \end{proof}
  
  \Cref{thm:nwd} now follows by combining the previous insights.
  \begin{proof}[Proof of \Cref{thm:nwd}]
    By \Cref{lem:nd-VC}, there is some $d\in\N$ such that every $G\in\CC$ has VC-dimension at most $d$.
    Fix $r\in\N$ and $\eps>0$.
Then, for every graph $G\in\CC$, by \Cref{lem:wcol-poly} and \Cref{fact:nd-wcol}, we have 
$$\mw_r(G)\le 3\cdot \pi_G\bigl(\wcol_{r+1}(G)\bigr)\le  3\cdot \pi_G\bigl(O_{\CC,r,\eps}(|V(G)|^\eps)\bigr)\le O_{\CC,r,\eps}\bigl(|V(G)|^{\eps\cdot d}\bigr).$$
Since $\eps>0$ is arbitrary, this implies that $\CC$ has almost bounded merge-width.
  \end{proof}

  \subsection{Proof of \Cref{thm:abmw}}


We start by proving that hereditary classes of almost bounded merge-width have bounded VC-dimension. 
In fact, we prove a stronger result, expressed using the following notion.

\newcommand{\ntn}{\mathrm{ntn}}

A pair $u,v$ of distinct vertices in a graph $G$ is a pair of \emph{$k$-near-twins} if $|N_G(u)\triangle N_G(v)|\le k$.
 For a bipartite graph $G=(X,Y,E)$, let the near-twin number $\ntn(G)$ denote 
 the smallest number $k$ such that 
for all $X'\subset X,Y'\subset Y$ with $|X'|+|Y'|>2$,
the bipartite subgraph $G[X',Y']$ induced by $X'$ and $Y'$  in $G$
has a pair of $k$-near-twins contained either in $X'$, or in $Y'$.

\begin{lemma}\label{lem:twins-bip}
    Let $G=(X,Y,E)$ be a bipartite graph with $\mw_1(G)\le k$
    and with $|X|+|Y|>2$.
    Then $G$ contains a pair of $2k$-near-twins
     contained in a single part of $G$. In particular, $\ntn(G)\le 2k$.
\end{lemma}
\begin{proof}
    Fix a construction sequence of $G$ of radius-$1$ width $k$. Consider the first moment when some part $A$ of the current partition $\cal P$ contains two vertices $a,b$ that belong to the same part of the bipartition $\set{X,Y}$ of $G$. Suppose $a,b\in X$. Then $N_G(a)\triangle N_G(b)\subset (N_R(a)\cup N_R(b))\cap Y$,
    where $N_R(\cdot)$ denotes the neighborhood in the graph $(V,R)$  of currently resolved pairs.
    Note that $|N_R(a)\cap Y|\le k$,
    since no two vertices of $Y$ are in a single part of the current partition $\cal P$, and $N_R(a)$ is contained in at most $k$ parts of $\cal P$. Similarly, $|N_R(b)\cap Y|\le k$.
    It follows that $|N_G(a)\triangle N_G(b)|\le 2k$.
\end{proof}

The following lemma is implicit in \cite{flip-width}
(it follows from 
\cite[Lem. 5.25]{flip-width-arxiv}).
\begin{lemma}\label{lem:ntn}
    If $\CC$ is a hereditary class of bipartite graphs such that $\ntn(G)\le o(|V(G)|)$ for $G \in \CC$, then $\VCdim(\CC)<\infty$.
\end{lemma}





\begin{corollary}\label{lem:abmw-vc}
    If $\CC$ is a hereditary class of graphs such that $\mw_1(G)\le o(|V(G)|)$ for $G \in \CC$, then $\VCdim(\CC)<\infty$.
\end{corollary}
    \begin{proof}
For a graph $G$ define the bipartite graph $B(G)$, with two parts of size $V(G)$,
representing the binary relation $E(G)\subset V(G)\times V(G)$.
Then $\mw_1(B(G))\le O (\mw_1(G))$, as a construction sequence of $G$ can
be easily converted to a construction sequence for $B(G)$.
In particular, $\mw_1(B(G))\le o(|V(G)|)$ for $G\in \CC$.
Moreover, $\VCdim(G)=\VCdim(B(G))$.
The conclusion follows from \Cref{lem:twins-bip}
and \Cref{lem:ntn}, applied to the class $\setof{B(G)}{G\in\CC}$.
    \end{proof}




    Recall that by the Sauer-Shelah-Perles Lemma (see \Cref{lem:sauer-shelah-perles}), 
    if $G$ is a graph of VC-dimension at most $d$,
    then $\pi_G(s)\le O(s^d)$ for all $s\in\N$. The following variant of \Cref{lem:fw}
     therefore gives -- for graphs of fixed VC-dimension -- a polynomial bound on the flip-width parameters, in terms of the merge-width parameters.

    
    \begin{lemma}\label{lem:fw-poly}Fix $r\in\N$.
        For every graph $G$,  $$\fw_r(G)\le \pi_G(\mw_{2r}(G)).$$
    \end{lemma}
\Cref{lem:fw-poly} strengthens 
 \Cref{lem:fw} by replacing the upper bound $4^s$ by $\pi_G(s)$.
    However, \Cref{lem:fw-poly} gives a slightly better dependency on the radius, as the upper bound involves $\mw_{2r-1}(G)$, rather than $\mw_{2r}(G)$.
    
    \Cref{lem:fw-poly} follows from the lemma below, 
    exactly in the same way as \Cref{lem:fw} follows from \Cref{lem:strategy}.
    
    \begin{lemma}\label{lem:strategy-poly}
        Fix $t\in [2,n]$, and 
        let $s$ be the radius-$2r$ width of $(\cal P_t,R_{t-1})$.
        For every $v\in V(G)$ there is a $\pi_G(s)$-flip $G_t'$ of $G$ such that:
    $$B^r_{G_t'}(w)\subset B^r_{G_t}(w)\quad\text{for every $w\in B^r_{G_{t-1}}(v)$.}$$
    \end{lemma}
    
    
    
    
        \begin{proof}Denote $s\coloneqq \mw_{2r}(G)$ and $V\coloneqq V(G)$.
            Fix $v\in V$. 
            For $i=0,\ldots,2r$, let 
            $B_i$ denote the set of vertices  that can be reached by a path of length at most $i$ by from $v$ in the graph $(V,R_{t-1})$.
            Let $\cal Q_i=\setof{A\cap B_{2r}}{A\in\cal P_t,A\cap B_{2r}\subset B_i}$.
    In particular, $|\cal Q_{2r}|\le s$.
    
    
    Observe that
    there are no edges in $G_{t}$ between  $B_{2r-1}$ and $V-B_{2r}$,
    since an edge  $uw\in E(G_t)$ with  $u\in B_{2r-1}$  implies $w\in B_{2r}$,
    as $E(G_{t})\subset R_t\subset R_{t-1}$.
    Since each part of $\cal Q_{2r-1}$ is contained in $B_{2r-1}$ and in some part of $\cal P_t$,
     and $G_t$ is a $\cal P_t$-flip of $G$, 
    it follows that each part of $\cal Q_{2r-1}$ is homogeneous in $G$ towards 
    every vertex $w\in V(G)-B_{2r}$.
    
    
    Let $\cal R$ be the partition of $V(G)-B_{2r}$ which partitions vertices according to their neighborhood in $B_{2r-1}$, in $G$.
    Denote $\cal Q\coloneqq \cal Q_{2r}\cup\cal R$. Then $\cal Q$ is a partition of $V$.
    
    \begin{claim}\label{cl:small-q-poly}
        $|\cal Q|\le \pi_G(s).$
       \end{claim}
       \begin{claimproof}
           For every part $A\in \cal Q_{2r-1}$ pick a vertex $v_A\in B_{2r-1}\cap A$,
           and let $S=\setof{v_A}{A\in\cal Q_{2r-1}}$. In particular, $|S|\le s$.
           It follows from the above that, in the graph $G$, for every vertex $w\in V(G)-B_{2r}$,
           the neighborhood  of $w$ in $B_{2r-1}$ is uniquely determined by the neighborhood of $w$ in $S$; moreover, $V(G)-B_{2r}$ is disjoint from $S$.
           Therefore, $$|\cal R|\le  \pi_G(s)-s.$$
           Together with $|\cal Q_{2r}|\le s$, this yields the conclusion.
       \end{claimproof}
           
    
    Consider the graph $G'$ with vertex set $V$,
    such that for any two parts $A,B\in\cal Q$:
    $$E(G')\cap AB=
    \begin{cases}
        E(G)\cap AB&\text{if $A,B\notin \cal Q_{2r-1}$},\\
        E(G_t)\cap AB&\text{otherwise.}\\
    \end{cases}$$
    
    \begin{claim}\label{cl:k-flip-poly}
        $G'$ is a $\cal Q$-flip of $G$.
    \end{claim}
    \begin{claimproof}
        
        It is enough to show that for all   $A,B\in \cal Q$, 
        \begin{align}
            E(G')\cap AB=E(G)\cap AB\qquad\textit{or}\qquad
            E(G')\cap AB= AB-E(G).\label{eq:flip-poly}    
        \end{align}
        We consider several cases.
    \begin{itemize}
        \item If $A,B\notin \cal Q_{2r-1}$, this is clear by definition of $E(G')$.
        \item Suppose that $A\in\cal Q_{2r-1}$ and $B\in \cal Q_{2r}$.
        Then \eqref{eq:flip-poly} follows, as $E(G')\cap AB=E(G_t)\cap AB$ by definition,
        and $G_t$ is a $\cal P_t$-flip of $G$, and each of the parts $A,B$ is contained in some part of $\cal P_t$.
        \item Suppose that $A\in\cal Q_{2r-1}$ and $B\notin\cal Q_{2r}$. Then 
        $A\subset B_{2r-1}$ and 
        $B\subset V(G)-B_{2r}$, and $B\in\cal R$. By the previous discussion, $E(G_t)\cap AB=\emptyset$. Moreover, each vertex of $b\in B$ is homogeneous in $G$ towards $A$,
        that is, each $b$ is either complete, or anti-complete towards $A$ in $G$.
        Since all vertices of $B$ have equal neighborhoods in $A$
        (by definition of $\cal R$, as $B\in\cal R$ and $A\subset B_{2r-1}$),
        it follows that $B$ is homogeneous in $G$ towards $A$.
    Together with $E(G_t)\cap AB=\emptyset$, this yields \eqref{eq:flip-poly}.
    \end{itemize}
    Up to exchanging the roles of $A$ and $B$, this covers all the cases.
    \end{claimproof}
    
    
    
    
    \begin{claim}\label{cl:balls-poly}
        For all $w\in B^r_{G_{t-1}}(v)$
     and $i=0,\ldots,r$ we have:
        $$B^i_{G'}(w)\subset B^i_{G_t}(w).$$
    \end{claim}
    
    \begin{claimproof}
        We induct on $i$. For $i=0$ the statement is trivial.
    In the inductive step, fix $1\le i\le r$ and $u\in B^i_{G'}(w)$ with $u\neq w$.
    Then there is some  $a\in B^{i-1}_{G'}(w)$ with $ua\in E(G')$.
    By inductive assumption, $a\in B^{i-1}_{G_t}(w)$.
    From  $i\le r$ and  $E(G_t)\subset R_{t-1}$
    we get that $a\in B^{r-1}_{R_{t-1}}(w)$,
    which together with $w\in B^{r}_{G_{t-1}}(v)\subset B^{r}_{R_{t-1}}(v)$ implies $a\in B^{2r-1}_{R_{t-1}}(v)=B_{2r-1}$. In particular, $a\in \bigcup \cal Q_{2r-1}$.
    As $ua\in E(G')$, this implies $ua\in E(G_t)$, by definition of $E(G')$. Together with 
    $a\in B^{i-1}_{G_t}(w)$, this implies that $u\in B^{i}_{G_t}(w)$, as required.
    \end{claimproof}
    \Cref{cl:k-flip-poly,cl:small-q-poly,cl:balls-poly} together prove \Cref{lem:strategy-poly}.
    \end{proof}
   As mentioned, \Cref{lem:strategy-poly} implies \Cref{lem:fw-poly},
   analogously as in the proof of \Cref{lem:fw}.


\thmabmw*
\begin{proof}
    Let $\CC$ be a hereditary class of almost bounded merge-width.
    By \Cref{lem:abmw-vc}, $\CC$  has VC-dimension bounded by some $d\in\N$.
By \Cref{lem:sauer-shelah-perles}, we have that $\pi_G(s)\le O(s^d)$ for all $G\in\CC$ and $s\in\N$.

Fix $r\in\N$ and $\eps>0$.
Then, for every graph $G\in\CC$, we have 
$$\fw_r(G)\le \pi_G(\mw_{2r}(G))\le \pi_G(O_{\CC,r,\eps}(|V(G)|^\eps))\le O_{\CC,r,\eps}(|V(G)|^{\eps\cdot d}).$$
Since $\eps>0$ is arbitrary, this implies that $\CC$ has almost bounded flip-width.
\end{proof}

























    





\section{RELATED WORK}

\subsection{Retrieval-Augmented Generation}
Retrieval-Augmented Generation (RAG) \cite{Lewis2020RAG,Guu2020REALM} has been widely used for improving the performance of LLMs across various tasks by incorporating an Information Retriever (IR) module to leverage external knowledge. 
Most RAG systems~\cite{Lewis2020RAG,ram2023context,Izacard2023Atlas} integrate retrieved knowledge (e.g., passages) directly into the input, where the LLMs generate answers based on the external information obtained with the IR module.
There are also some methods utilizing Chain of Thought (CoT) \cite{wei2022cot, trivedi2023interleaving} or task decomposition \cite{Xu2024Search, wang2024selfdc, kim2024sure} to integrate external knowledge in intermediate reasoning steps or sub-tasks. 
Though effective, such indiscriminate use of external knowledge may introduce noise, degrading the quality of generated responses.
To address this problem, conditional use of external knowledge in RAG has been investigated.
Some works decide whether to retrieve and utilize external knowledge based on query characteristics, such as assessing query complexity through entity frequency \cite{mallen2023trust}, searching for similar questions \cite{wang2023skr}, or collecting question sets for training \cite{jeong2024adaptive}.
Some other works determine whether to integrate external knowledge according to the next token generation probability by LLMs, 
such as Self-RAG~\cite{asai2024selfrag}, FLARE\cite{jiang2023active}, DRAGIN~\cite{su2024dragin}, and Self-DC~\cite{wang2024selfdc}. 
In addition to these adaptive retrieval approaches, relevance-based methods~\cite{zhang2023merging,Xu2024Search,liu2024raisf} employ a relevance verification module to filter retrieved passages. 
For example, RA-ISF~\cite{liu2024raisf} assesses the relevance of retrieved passages by training a small LLM.
However, adaptive retrieval methods often rely solely on the input query or generated tokens, which limits their effectiveness as they may only acquire incomplete information. 
In comparison, relevance-based methods heavily rely on an additional verification module, leading to increased complexity of RAG systems and high sensitivity of the final response's quality to its verification accuracy. 
In this work, we propose a novel approach that leverages the LLM itself to holistically evaluate and reconcile responses from only its internal parametric knowledge and also from externally retrieved information, aiming to deliver more accurate responses.


\subsection{Preference Alignment for LLMs}
Preference alignment has emerged as an effective approach for improving the reliability of LLMs~\cite{Ouyang2022Training} by enabling them to evolve from their generated responses and environmental feedback.
Among existing preference alignment techniques, Reinforcement Learning from Human Feedback (RLHF) leverages human-provided feedback to train reward models, ensuring that LLMs produce responses aligned well with human preferences~\cite{Christiano2017RLHF,ziegler2019finetuning}.
RLHF has been shown to improve both the performance and the user-friendliness of LLMs in various Natural Language Processing (NLP) tasks, including summarization \cite{Stiennon2020summarize}, question answering \cite{Nakano2021WebGPT}, and instruction following \cite{Ouyang2022Training}.
However, RLHF requires extensive human annotation to train the reward model and involves a complex three-stage process, resulting in limited scalability and high training complexity.
To improve the scalability in preference alignment, Reinforcement Learning from AI Feedback (RLAIF)~\cite{Bai2022ConstitutionalAH,lee2024rlaif} utilizes the feedback from the LLM itself to train a reward model to optimize LLM performance through reinforcement learning. 
Direct Preference Optimization (DPO) \cite{Rafailov2023DPO} defines preference loss directly via a change of variables, which treats the LLM itself as its reward model.
By eliminating the need for an additional reward model, DPO substantially reduces the complexity involved in preference alignment training. 
In RAG systems, some studies utilize the signals generated by LLMs to optimize the retriever~\cite{Bonifacio2022InPars,shi2024replug} to retrieve LLM-preferred data, while other works align LLMs with specific domain knowledge and specific tasks through reinforcement learning~\cite{zhang2024knowledgeable,Yang2024IMRAG,Salemi2024Optimization,dong2024understandllmneedsdual,song2024measuring}.
In this work, we generate a preference dataset automatically and utilize it to strengthen LLMs' answer selection and generation capabilities in RAG systems via DPO.




\section{CONCLUSION}
In this work, we propose a novel \framework~framework to improve the accuracy and reliability of responses generated by LLMs in RAG systems.
Our method allows the LLM to select the more accurate one from a pair of responses generated based on internal parametric knowledge solely and by integrating external retrieved knowledge, to achieve enhanced performance. 
To strengthen the capabilities of the LLM in generating and selecting correct answers, we develop a \approach~method that trains the LLM with Direct Preference Optimization over a newly built Retrieval-Generation Preference (RGP) dataset. 
We conduct extensive experiments and analyses, which well validate the effectiveness of the proposed method. 
We hope this work paves the way for the development of more robust and reliable LLMs in RAG.

%%
%% The acknowledgments section is defined using the "acks" environment
%% (and NOT an unnumbered section). This ensures the proper
%% identification of the section in the article metadata, and the
%% consistent spelling of the heading.
%\begin{acks}
%To Robert, for the bagels and explaining CMYK and color spaces.
%\end{acks}

%%
%% The next two lines define the bibliography style to be used, and
%% the bibliography file.
\bibliographystyle{ACM-Reference-Format}
\bibliography{sample-base}


%%
%% If your work has an appendix, this is the place to put it.
%\appendix

%\section{Research Methods}


%\section{Online Resources}


\end{document}
\endinput
%%
%% End of file `sample-sigconf-authordraft.tex'.
