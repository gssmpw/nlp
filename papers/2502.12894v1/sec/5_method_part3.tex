

Ensuring that the reconstructed 3D scene is not only geometrically accurate but also physically plausible is essential for realistic scene generation. This section details the strategies employed to maintain physical consistency and enforce inter-object relationships within the 3D environment. By leveraging the relationship graph inferred during scene analysis and integrating physical simulation techniques, we ensure that each object interacts naturally with others, adhering to real-world physical constraints.



\subsection{Relationship Graph Integration}
The relationship graph, constructed during the scene analysis phase, serves as the foundation for enforcing physical interactions between objects. This graph encodes the spatial and functional relationships, such as support, hanging, and contact, that dictate how objects coexist within the scene.

\paragraph{Relationships constraints}
Objects categorized under 'Support' and 'Hanging' relationships require specific alignment and positioning to reflect realistic interactions. Contact relationships indicate mutual interactions between objects, such as touching, leaning, or stacking.

For objects identified as being supported by others, we ensure that their bases are precisely aligned with the supporting surfaces. This involves adjusting the vertical position and orientation of the supported objects to rest naturally on the supporting entities. By adhering to these constraints, we prevent unrealistic floating or sinking of objects, maintaining the structural integrity of the scene.

Objects classified as hanging are suspended from designated anchor points on other objects or structural elements within the scene. We calculate the appropriate suspension points and adjust the orientation of the hanging objects to reflect natural drooping or alignment based on gravity and object weight. This ensures that hanging objects appear realistically suspended, enhancing the overall coherence of the scene.

For objects in mutual contact, we enforce precise alignment to avoid unintended overlaps or penetrations. This involves calculating the exact points of contact and adjusting the positions and orientations of the involved objects to achieve seamless interaction. By meticulously aligning contact points, we ensure that objects appear to interact naturally, whether they are leaning against each other or stacked in a stable configuration. In cases where objects are subject to dynamic interactions, such as one object leaning against another under external forces, we simulate these interactions to adjust object placements accordingly. This dynamic handling ensures that the scene remains physically plausible even under varying interaction scenarios.

\subsection{Physical Simulation Integration}
To guarantee that the physical interactions among objects are realistic, we integrate physics simulation into the reconstruction pipeline. This step rectifies any residual placement inaccuracies and enhances the overall stability and realism of the scene.

\paragraph{Force Direction Analysis}
Understanding the direction and magnitude of forces acting between objects is crucial for realistic placement and interaction.

We categorize the directions of forces between objects as Vertical, Horizontal, or Both. This classification informs how objects influence each other's positions and orientations within the scene. Based on the classified force directions, we apply corresponding adjustments to object placements. For instance, vertically supported objects are adjusted to ensure they are neither floating nor sinking, while horizontally influenced objects are aligned to reflect natural leaning or balancing.

\paragraph{Stability Verification}
Ensuring the stability of the entire scene involves verifying that all objects are appropriately balanced and that their interactions do not lead to unrealistic configurations. We perform balance checks to ensure that supported and hanging objects maintain equilibrium within the scene. This involves analyzing the distribution of forces and ensuring that no object is subject to excessive or unbalanced forces that would cause instability. Physics simulations help detect and prevent collisions or penetrations between objects. By simulating physical interactions, we can identify and correct any residual placement issues, ensuring that objects coexist without unrealistic overlaps.

\paragraph{Penetration Correction}
Despite initial alignment efforts, minor penetrations between objects may still occur. We address these by adjusting the RTS parameters to eliminate overlaps.

Using collision detection algorithms, we identify regions where objects intersect unnaturally. We then adjust the Rotation, Transformation, and Scale (RTS) parameters of the involved objects to resolve these intersections, ensuring that each object maintains its integrity without infringing on others. This process is performed iteratively, with adjustments made incrementally to achieve a balanced and realistic scene configuration. Each iteration checks for remaining penetrations and applies necessary corrections until all overlaps are resolved.


\begin{figure*} [h]
  \centering
  \includegraphics[width=\textwidth]{figs/gallery.png}
  \captionof{figure}{
  Result gallery of randomly selected images, including diverse indoor and outdoor scenes, close-up photographs, and images generated using image generation techniques.
  %A result gallery for randomly selected images, which includes various indoor and outdoor scenes, close-up photographs, and images generated using image generation methods.
  } \label{fig_gallery}
\end{figure*}

