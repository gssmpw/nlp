% CVPR 2022 Paper Template
% based on the CVPR template provided by Ming-Ming Cheng (https://github.com/MCG-NKU/CVPR_Template)
% modified and extended by Stefan Roth (stefan.roth@NOSPAMtu-darmstadt.de)

\documentclass[10pt,twocolumn,letterpaper]{article}

%%%%%%%%% PAPER TYPE  - PLEASE UPDATE FOR FINAL VERSION
% \usepackage[review]{cvpr}      % To produce the REVIEW version
\usepackage{cvpr}              % To produce the CAMERA-READY version
%\usepackage[pagenumbers]{cvpr} % To force page numbers, e.g. for an arXiv version

% Include other packages here, before hyperref.
\usepackage{graphicx}
\usepackage{amsmath}
\usepackage{amssymb}
\usepackage{booktabs}


% It is strongly recommended to use hyperref, especially for the review version.
% hyperref with option pagebackref eases the reviewers' job.
% Please disable hyperref *only* if you encounter grave issues, e.g. with the
% file validation for the camera-ready version.
%
% If you comment hyperref and then uncomment it, you should delete
% ReviewTempalte.aux before re-running LaTeX.
% (Or just hit 'q' on the first LaTeX run, let it finish, and you
%  should be clear).
\usepackage[pagebackref,breaklinks,colorlinks]{hyperref}

% \usepackage{lipsum}
% \usepackage{xcolor}
% \usepackage{bbding}
\usepackage{bm}
% \usepackage{enumitem}
% \usepackage{multirow}
% \usepackage{xspace}
% \usepackage{natbib}
% \usepackage{authblk}
\newcommand{\methodname}{CAST\xspace}


% Support for easy cross-referencing
\usepackage[capitalize]{cleveref}
\crefname{section}{Sec.}{Secs.}
\Crefname{section}{Section}{Sections}
\Crefname{table}{Table}{Tables}
\crefname{table}{Tab.}{Tabs.}


%%%%%%%%% PAPER ID  - PLEASE UPDATE
\def\cvprPaperID{*****} % *** Enter the CVPR Paper ID here
\def\confName{CVPR}
\def\confYear{2022}


\begin{document}

%%%%%%%%% TITLE - PLEASE UPDATE
\title{CAST: Component-Aligned 3D Scene Reconstruction from an RGB Image}


\author{
% Kaixin Yao\\
% ShanghaiTech University\\
% Kaixin Yao$^{1,2\star}$ \quad Longwen Zhang$^{1,2\star}$ \quad Xinhao Yan$^{1,2}$ \quad Yan Zeng$^{1,2}$ \quad Qixuan Zhang$^{1,2\dagger}$ \\ Lan Xu$^{1\ddag}$ \quad Wei Yang$^{3\ddag}$ \quad Jiayuan Gu$^{1\ddag}$ \quad Jingyi Yu$^{1\ddag}$ \\
Kaixin Yao$^{1,2\star}$ \quad Longwen Zhang$^{1,2\star}$ \quad Xinhao Yan$^{1,2}$ \quad Yan Zeng$^{1,2}$ \quad Qixuan Zhang$^{1,2\dagger}$ \\ Lan Xu$^{1}$ \quad Wei Yang$^{3}$ \quad Jiayuan Gu$^{1}$ \quad Jingyi Yu$^{1}$ \\
  $^1$ShanghaiTech University\quad 
  $^2$Deemos Technology\quad \\
  $^3$Huazhong University of Science and Technology \\
{\tt\small \{yaokx2023,zhanglw2,yanxh,zengyan2024,zhangqx1,xulan1,gujy1,yujingyi\}@shanghaitech.edu.cn} \\
{\tt\small weiyangcs@hust.edu.cn} \\
{\tt \href{https://sites.google.com/view/cast4}{https://sites.google.com/view/cast4}}
% For a paper whose authors are  all at the same institution,
% omit the following lines up until the closing ``}''.
% Additional authors and addresses can be added with ``\and'',
% just like the second author.
% To save space, use either the email address or home page, not both
% \and
% Longwen Zhang\\
% ShanghaiTech University\\
% {\tt\small zhanglw2@shanghaitech.edu.cn}
% \and
% Longwen Zhang\\
% ShanghaiTech University\\
% {\tt\small zhanglw2@shanghaitech.edu.cn}
}
% \maketitle

% \begin{figure*}
% \centering
%   \includegraphics[width=\textwidth]{figs/teaser.png}
%   \caption{CAST brings diverse 3D scenes to life from a single image, where the relationships between objects shaped by their physical roles and interactions come together to form a cohesive and immersive virtual environment.}
%   \label{fig:teaser}
% \end{figure*}
\twocolumn[{%
\renewcommand\twocolumn[1][]{#1}%
\maketitle
% \begin{center}
%     {\Large \href{https://sites.google.com/view/cast4}{https://sites.google.com/view/cast4}}
% \end{center}
\includegraphics[width=\linewidth]{figs/teaser.png}
% \vspace{-2em}
\captionof{figure}{CAST brings diverse 3D scenes to life from a single image, where the relationships between objects shaped by their physical roles and interactions come together to form a cohesive and immersive virtual environment.\vspace{1em}}
\label{fig:teaser}
}]

% \def\thefootnote{}\footnotetext{$\star$ Equal contribution. $\dagger$ Project Leader. $\ddag$ Corresponding authors.}
\def\thefootnote{}\footnotetext{$\star$ Equal contribution. $\dagger$ Project Leader.} 
% \thefootnote{$\dagger$}\footnotetext{Project Leader.}

\begin{abstract}
\label{sec:abstract}
    Recovering high-quality 3D scenes from a single RGB image is a challenging task in computer graphics. Current methods often struggle with domain-specific limitations or low-quality object generation. To address these, we propose CAST (Component-Aligned 3D Scene Reconstruction from a Single RGB Image), a novel method for 3D scene reconstruction and recovery. CAST starts by extracting object-level 2D segmentation and relative depth information from the input image, followed by using a GPT-based model to analyze inter-object spatial relationships. This enables the understanding of how objects relate to each other within the scene, ensuring more coherent reconstruction. CAST then employs an occlusion-aware large-scale 3D generation model to independently generate each object's full geometry, using MAE and point cloud conditioning to mitigate the effects of occlusions and partial object information, ensuring accurate alignment with the source image's geometry and texture. To align each object with the scene, the alignment generation model computes the necessary transformations, allowing the generated meshes to be accurately placed and integrated into the scene's point cloud. Finally, CAST incorporates a physics-aware correction step that leverages a fine-grained relation graph to generate a constraint graph. This graph guides the optimization of object poses, ensuring physical consistency and spatial coherence. By utilizing Signed Distance Fields (SDF), the model effectively addresses issues such as occlusions, object penetration, and floating objects, ensuring that the generated scene accurately reflects real-world physical interactions. Experimental results demonstrate that CAST significantly improves the quality of single-image 3D scene reconstruction, offering enhanced realism and accuracy in scene recovery tasks. CAST has practical applications in virtual content creation, such as immersive game environments and film production, where real-world setups can be seamlessly integrated into virtual landscapes. Additionally, CAST can be leveraged in robotics, enabling efficient real-to-simulation workflows and providing realistic, scalable simulation environments for robotic systems.
\end{abstract}


\section{Introduction}
\label{sec:intro}
    \section{Introduction}
Backdoor attacks pose a concealed yet profound security risk to machine learning (ML) models, for which the adversaries can inject a stealth backdoor into the model during training, enabling them to illicitly control the model's output upon encountering predefined inputs. These attacks can even occur without the knowledge of developers or end-users, thereby undermining the trust in ML systems. As ML becomes more deeply embedded in critical sectors like finance, healthcare, and autonomous driving \citep{he2016deep, liu2020computing, tournier2019mrtrix3, adjabi2020past}, the potential damage from backdoor attacks grows, underscoring the emergency for developing robust defense mechanisms against backdoor attacks.

To address the threat of backdoor attacks, researchers have developed a variety of strategies \cite{liu2018fine,wu2021adversarial,wang2019neural,zeng2022adversarial,zhu2023neural,Zhu_2023_ICCV, wei2024shared,wei2024d3}, aimed at purifying backdoors within victim models. These methods are designed to integrate with current deployment workflows seamlessly and have demonstrated significant success in mitigating the effects of backdoor triggers \cite{wubackdoorbench, wu2023defenses, wu2024backdoorbench,dunnett2024countering}.  However, most state-of-the-art (SOTA) backdoor purification methods operate under the assumption that a small clean dataset, often referred to as \textbf{auxiliary dataset}, is available for purification. Such an assumption poses practical challenges, especially in scenarios where data is scarce. To tackle this challenge, efforts have been made to reduce the size of the required auxiliary dataset~\cite{chai2022oneshot,li2023reconstructive, Zhu_2023_ICCV} and even explore dataset-free purification techniques~\cite{zheng2022data,hong2023revisiting,lin2024fusing}. Although these approaches offer some improvements, recent evaluations \cite{dunnett2024countering, wu2024backdoorbench} continue to highlight the importance of sufficient auxiliary data for achieving robust defenses against backdoor attacks.

While significant progress has been made in reducing the size of auxiliary datasets, an equally critical yet underexplored question remains: \emph{how does the nature of the auxiliary dataset affect purification effectiveness?} In  real-world  applications, auxiliary datasets can vary widely, encompassing in-distribution data, synthetic data, or external data from different sources. Understanding how each type of auxiliary dataset influences the purification effectiveness is vital for selecting or constructing the most suitable auxiliary dataset and the corresponding technique. For instance, when multiple datasets are available, understanding how different datasets contribute to purification can guide defenders in selecting or crafting the most appropriate dataset. Conversely, when only limited auxiliary data is accessible, knowing which purification technique works best under those constraints is critical. Therefore, there is an urgent need for a thorough investigation into the impact of auxiliary datasets on purification effectiveness to guide defenders in  enhancing the security of ML systems. 

In this paper, we systematically investigate the critical role of auxiliary datasets in backdoor purification, aiming to bridge the gap between idealized and practical purification scenarios.  Specifically, we first construct a diverse set of auxiliary datasets to emulate real-world conditions, as summarized in Table~\ref{overall}. These datasets include in-distribution data, synthetic data, and external data from other sources. Through an evaluation of SOTA backdoor purification methods across these datasets, we uncover several critical insights: \textbf{1)} In-distribution datasets, particularly those carefully filtered from the original training data of the victim model, effectively preserve the model’s utility for its intended tasks but may fall short in eliminating backdoors. \textbf{2)} Incorporating OOD datasets can help the model forget backdoors but also bring the risk of forgetting critical learned knowledge, significantly degrading its overall performance. Building on these findings, we propose Guided Input Calibration (GIC), a novel technique that enhances backdoor purification by adaptively transforming auxiliary data to better align with the victim model’s learned representations. By leveraging the victim model itself to guide this transformation, GIC optimizes the purification process, striking a balance between preserving model utility and mitigating backdoor threats. Extensive experiments demonstrate that GIC significantly improves the effectiveness of backdoor purification across diverse auxiliary datasets, providing a practical and robust defense solution.

Our main contributions are threefold:
\textbf{1) Impact analysis of auxiliary datasets:} We take the \textbf{first step}  in systematically investigating how different types of auxiliary datasets influence backdoor purification effectiveness. Our findings provide novel insights and serve as a foundation for future research on optimizing dataset selection and construction for enhanced backdoor defense.
%
\textbf{2) Compilation and evaluation of diverse auxiliary datasets:}  We have compiled and rigorously evaluated a diverse set of auxiliary datasets using SOTA purification methods, making our datasets and code publicly available to facilitate and support future research on practical backdoor defense strategies.
%
\textbf{3) Introduction of GIC:} We introduce GIC, the \textbf{first} dedicated solution designed to align auxiliary datasets with the model’s learned representations, significantly enhancing backdoor mitigation across various dataset types. Our approach sets a new benchmark for practical and effective backdoor defense.




\section{Related Work}
\label{sec:related}
    \section{Related Work}
\label{sec:related-works}
\subsection{Novel View Synthesis}
Novel view synthesis is a foundational task in the computer vision and graphics, which aims to generate unseen views of a scene from a given set of images.
% Many methods have been designed to solve this problem by posing it as 3D geometry based rendering, where point clouds~\cite{point_differentiable,point_nfs}, mesh~\cite{worldsheet,FVS,SVS}, planes~\cite{automatci_photo_pop_up,tour_into_the_picture} and multi-plane images~\cite{MINE,single_view_mpi,stereo_magnification}, \etal
Numerous methods have been developed to address this problem by approaching it as 3D geometry-based rendering, such as using meshes~\cite{worldsheet,FVS,SVS}, MPI~\cite{MINE,single_view_mpi,stereo_magnification}, point clouds~\cite{point_differentiable,point_nfs}, etc.
% planes~\cite{automatci_photo_pop_up,tour_into_the_picture}, 


\begin{figure*}[!t]
    \centering
    \includegraphics[width=1.0\linewidth]{figures/overview-v7.png}
    %\caption{\textbf{Overview.} Given a set of images, our method obtains both camera intrinsics and extrinsics, as well as a 3DGS model. First, we obtain the initial camera parameters, global track points from image correspondences and monodepth with reprojection loss. Then we incorporate the global track information and select Gaussian kernels associated with track points. We jointly optimize the parameters $K$, $T_{cw}$, 3DGS through multi-view geometric consistency $L_{t2d}$, $L_{t3d}$, $L_{scale}$ and photometric consistency $L_1$, $L_{D-SSIM}$.}
    \caption{\textbf{Overview.} Given a set of images, our method obtains both camera intrinsics and extrinsics, as well as a 3DGS model. During the initialization, we extract the global tracks, and initialize camera parameters and Gaussians from image correspondences and monodepth with reprojection loss. We determine Gaussian kernels with recovered 3D track points, and then jointly optimize the parameters $K$, $T_{cw}$, 3DGS through the proposed global track constraints (i.e., $L_{t2d}$, $L_{t3d}$, and $L_{scale}$) and original photometric losses (i.e., $L_1$ and $L_{D-SSIM}$).}
    \label{fig:overview}
\end{figure*}

Recently, Neural Radiance Fields (NeRF)~\cite{2020NeRF} provide a novel solution to this problem by representing scenes as implicit radiance fields using neural networks, achieving photo-realistic rendering quality. Although having some works in improving efficiency~\cite{instant_nerf2022, lin2022enerf}, the time-consuming training and rendering still limit its practicality.
Alternatively, 3D Gaussian Splatting (3DGS)~\cite{3DGS2023} models the scene as explicit Gaussian kernels, with differentiable splatting for rendering. Its improved real-time rendering performance, lower storage and efficiency, quickly attract more attentions.
% Different from NeRF-based methods which need MLPs to model the scene and huge computational cost for rendering, 3DGS has stronger real-time performance, higher storage and computational efficiency, benefits from its explicit representation and gradient backpropagation.

\subsection{Optimizing Camera Poses in NeRFs and 3DGS}
Although NeRF and 3DGS can provide impressive scene representation, these methods all need accurate camera parameters (both intrinsic and extrinsic) as additional inputs, which are mostly obtained by COLMAP~\cite{colmap2016}.
% This strong reliance on COLMAP significantly limits their use in real-world applications, so optimizing the camera parameters during the scene training becomes crucial.
When the prior is inaccurate or unknown, accurately estimating camera parameters and scene representations becomes crucial.

% In early works, only photometric constraints are used for scene training and camera pose estimation. 
% iNeRF~\cite{iNerf2021} optimizes the camera poses based on a pre-trained NeRF model.
% NeRFmm~\cite{wang2021nerfmm} introduce a joint optimization process, which estimates the camera poses and trains NeRF model jointly.
% BARF~\cite{barf2021} and GARF~\cite{2022GARF} provide new positional encoding strategy to handle with the gradient inconsistency issue of positional embedding and yield promising results.
% However, they achieve satisfactory optimization results when only the pose initialization is quite closed to the ground-truth, as the photometric constrains can only improve the quality of camera estimation within a small range.
% Later, more prior information of geometry and correspondence, \ie monocular depth and feature matching, are introduced into joint optimisation to enhance the capability of camera poses estimation.
% SC-NeRF~\cite{SCNeRF2021} minimizes a projected ray distance loss based on correspondence of adjacent frames.
% NoPe-NeRF~\cite{bian2022nopenerf} chooses monocular depth maps as geometric priors, and defines undistorted depth loss and relative pose constraints for joint optimization.
In earlier studies, scene training and camera pose estimation relied solely on photometric constraints. iNeRF~\cite{iNerf2021} refines the camera poses using a pre-trained NeRF model. NeRFmm~\cite{wang2021nerfmm} introduces a joint optimization approach that simultaneously estimates camera poses and trains the NeRF model. BARF~\cite{barf2021} and GARF~\cite{2022GARF} propose a new positional encoding strategy to address the gradient inconsistency issues in positional embedding, achieving promising results. However, these methods only yield satisfactory optimization when the initial pose is very close to the ground truth, as photometric constraints alone can only enhance camera estimation quality within a limited range. Subsequently, 
% additional prior information on geometry and correspondence, such as monocular depth and feature matching, has been incorporated into joint optimization to improve the accuracy of camera pose estimation. 
SC-NeRF~\cite{SCNeRF2021} minimizes a projected ray distance loss based on correspondence between adjacent frames. NoPe-NeRF~\cite{bian2022nopenerf} utilizes monocular depth maps as geometric priors and defines undistorted depth loss and relative pose constraints.

% With regard to 3D Gaussian Splatting, CF-3DGS~\cite{CF-3DGS-2024} also leverages mono-depth information to constrain the optimization of local 3DGS for relative pose estimation and later learn a global 3DGS progressively in a sequential manner.
% InstantSplat~\cite{fan2024instantsplat} focus on sparse view scenes, first use DUSt3R~\cite{dust3r2024cvpr} to generate a set of densely covered and pixel-aligned points for 3D Gaussian initialization, then introduce a parallel grid partitioning strategy in joint optimization to speed up.
% % Jiang et al.~\cite{Jiang_2024sig} proposed to build the scene continuously and progressively, to next unregistered frame, they use registration and adjustment to adjust the previous registered camera poses and align unregistered monocular depths, later refine the joint model by matching detected correspondences in screen-space coordinates.
% \gjh{Jiang et al.~\cite{Jiang_2024sig} also implemented an incremental approach for reconstructing camera poses and scenes. Initially, they perform feature matching between the current image and the image rendered by a differentiable surface renderer. They then construct matching point errors, depth errors, and photometric errors to achieve the registration and adjustment of the current image. Finally, based on the depth map, the pixels of the current image are projected as new 3D Gaussians. However, this method still exhibits limitations when dealing with complex scenes and unordered images.}
% % CG-3DGS~\cite{sun2024correspondenceguidedsfmfree3dgaussian} follows CF-3DGS, first construct a coarse point cloud from mono-depth maps to train a 3DGS model, then progressively estimate camera poses based on this pre-trained model by constraining the correspondences between rendering view and ground-truth.
% \gjh{Similarly, CG-3DGS~\cite{sun2024correspondenceguidedsfmfree3dgaussian} first utilizes monocular depth estimation and the camera parameters from the first frame to initialize a set of 3D Gaussians. It then progressively estimates camera poses based on this pre-trained model by constraining the correspondences between the rendered views and the ground truth.}
% % Free-SurGS~\cite{freesurgs2024} matches the projection flow derived from 3D Gaussians with optical flow to estimate the poses, to compensate for the limitations of photometric loss.
% \gjh{Free-SurGS~\cite{freesurgs2024} introduces the first SfM-free 3DGS approach for surgical scene reconstruction. Due to the challenges posed by weak textures and photometric inconsistencies in surgical scenes, Free-SurGS achieves pose estimation by minimizing the flow loss between the projection flow and the optical flow. Subsequently, it keeps the camera pose fixed and optimizes the scene representation by minimizing the photometric loss, depth loss and flow loss.}
% \gjh{However, most current works assume camera intrinsics are known and primarily focus on optimizing camera poses. Additionally, these methods typically rely on sequentially ordered image inputs and incrementally optimize camera parameters and scene representation. This inevitably leads to drift errors, preventing the achievement of globally consistent results. Our work aims to address these issues.}

Regarding 3D Gaussian Splatting, CF-3DGS~\cite{CF-3DGS-2024} utilizes mono-depth information to refine the optimization of local 3DGS for relative pose estimation and subsequently learns a global 3DGS in a sequential manner. InstantSplat~\cite{fan2024instantsplat} targets sparse view scenes, initially employing DUSt3R~\cite{dust3r2024cvpr} to create a densely covered, pixel-aligned point set for initializing 3D Gaussian models, and then implements a parallel grid partitioning strategy to accelerate joint optimization. Jiang \etal~\cite{Jiang_2024sig} develops an incremental method for reconstructing camera poses and scenes, but it struggles with complex scenes and unordered images. 
% Similarly, CG-3DGS~\cite{sun2024correspondenceguidedsfmfree3dgaussian} progressively estimates camera poses using a pre-trained model by aligning the correspondences between rendered views and actual scenes. Free-SurGS~\cite{freesurgs2024} pioneers an SfM-free 3DGS method for reconstructing surgical scenes, overcoming challenges such as weak textures and photometric inconsistencies by minimizing the discrepancy between projection flow and optical flow.
%\pb{SF-3DGS-HT~\cite{ji2024sfmfree3dgaussiansplatting} introduced VFI into training as additional photometric constraints. They separated the whole scene into several local 3DGS models and then merged them hierarchically, which leads to a significant improvement on simple and dense view scenes.}
HT-3DGS~\cite{ji2024sfmfree3dgaussiansplatting} interpolates frames for training and splits the scene into local clips, using a hierarchical strategy to build 3DGS model. It works well for simple scenes, but fails with dramatic motions due to unstable interpolation and low efficiency.
% {While effective for simple scenes, it struggles with dramatic motion due to unstable view interpolation and suffers from low computational efficiency.}

However, most existing methods generally depend on sequentially ordered image inputs and incrementally optimize camera parameters and 3DGS, which often leads to drift errors and hinders achieving globally consistent results. Our work seeks to overcome these limitations.

    
\section{Overview}
% \section{Overview}


Scene-level reconstruction from a single image is a fundamental challenge in computer graphics, with broad applications in animation, virtual reality, and interactive gaming. Unlike object-level reconstruction, which focuses on isolated objects, scene-level reconstruction emphasizes the arrangement and relations of multiple entities under realistic (or stylized) physics. By capturing per-object structures, spatial relationships, and contextual cues, this holistic approach enables more immersive experiences, compelling narratives, and efficient workflows—benefits that surpass those of single-object reconstructions. Although previous methodologies have explored feed-forward pipelines or retrieval-based approaches using fixed 3D templates~\cite{liu2022towards,dai2024automated}, these methods often struggle to capture nuanced scene semantics and complex object relationships. To address these limitations, we propose a generation-driven scene ``reincarnation'' approach with emphasized object relations to construct high-fidelity, contextually consistent 3D environments from a single, unannotated RGB image whether sourced from real-world photography or synthetic data (see Fig.~\ref{fig_overview1}).

A key insight of our method is the thorough object relation analysis of scene contextual information. First, we perform object segmentation to identify and localize constituent objects within the image. We then obtain preliminary geometric information, i.e., point clouds, and explore semantic and spatial relationships among objects. This contextual backbone informs our subsequent object-wise generation pipeline, ensuring that each reconstructed object retains not only its geometric fidelity but also its correct placement in the broader scene. Finally, we synthesize a coherent 3D environment that respects physical plausibility—achieving structurally sound layouts and realistic interactions among scene elements.

%Our research is guided by two primary questions:

%How can generative models effectively capture complex inter-object relationships to produce realistic, scene-level reconstructions from a single image?
%What strategies for integrating geometric cues and contextual information yield the highest accuracy and plausibility in 3D reconstructions?

%By answering these questions, we demonstrate that generative methods offer a more flexible and robust alternative to feed-forward or retrieval-based techniques. They allow for fine-grained control over object-level detail and global scene composition, thereby streamlining content creation pipelines for animation, game development, and other areas that require accurate and visually compelling 3D models. This work not only illustrates the advantages of a generation-centric framework but also lays the groundwork for future advances in scene-level 3D reconstruction, highlighting the growing importance of context-driven approaches in bridging the gap between 2D imagery and rich, interactive virtual environments.

Our research focuses on two primary objectives: to explore how generative models can effectively capture complex inter-object relationships in order to produce realistic, scene-level reconstructions from a single image; and to identify strategies for integrating geometric cues and contextual information that maximize accuracy and plausibility in 3D reconstructions.
Through this investigation, we demonstrate that generative methods provide a more flexible and robust alternative to traditional feed-forward and retrieval-based techniques. These methods allow for fine-grained control over both object-level details and global scene composition, thus streamlining content creation pipelines for animation, game development, and other fields requiring accurate, visually compelling 3D models.
%
This work highlights the advantages of a generation-centric framework and lays the groundwork for future advancements in scene-level 3D reconstruction. It also underscores the growing importance of context-driven approaches in bridging the gap between 2D imagery and immersive, interactive virtual environments.

% \vspace{16pt}
%\paragraph{Scene decomposition} 
%
\paragraph{Preprocessing}
To facilitate comprehensive scene reconstruction from a single image, we first perform an extensive semantic extraction that provides a robust foundation for subsequent processing. Specifically, we employ Florence-2~\cite{xiao2024florence} to identify objects, generate their descriptions, and localize each object with bounding boxes. We then leverage GPT-4v~\cite{achiam2023gpt} to filter out spurious detections and isolate meaningful constituent objects, allowing for open-vocabulary object identification that is not constrained by predefined categories. Next, we use GroundedSAM-v2~\cite{ren2024grounded} to produce a refined segmentation mask $\{ \bm{M}_i \}$ for each labeled object $\{ \bm{o}_i \}$, thereby obtaining both precise object boundaries and corresponding occlusion masks, which play a crucial auxiliary role in the object generation stage.
Apart from semantic cues, we also integrate geometric information by extracting a scene-level point cloud. Using MoGe~\cite{wang2024moge}, we generate pixel-aligned point clouds $\{ \bm{q}_i \}$ for each object $\{ \bm{o}_i \}, i \in \{1,\dots,N\}$ and a global camera parameter in the scene coordinate system. This additional geometric data is subsequently matched to each object’s segmentation mask, providing a reliable structural reference for the final 3D scene reconstruction.
% Accurate object relationships are also crucial for preserving spatial consistency and depth perception in single-image 3D reconstruction. We use GPT-4v to infer a \red{comprehensive self-designed} scene graph (see Sec.~\ref{sec:physics-aware-correction}.) 
% including semantic labels, spatial positions, relationship types, and force directions.
% These relationships help reduce ambiguity while maintaining the functional and aesthetic coherence of the scene, thereby enabling high-fidelity results, preserving both the structural layout and the underlying physical attributes of the input image. 
% By combining semantic extraction, robust geometric data with accurate relationships, our approach ensures that each object is accurately represented in terms of both appearance, spatial configuration and the underlying physical attributes, creating a strong backbone for downstream scene-level generation and analysis.


    \label{sec:method3}



\begin{figure} []
  \centering
  \includegraphics[width=0.98\linewidth]{figs/pipeline2.png}
  \captionof{figure}{Network design of our alignment generation model (Sec.~\ref{sec:transformationgen}), occlusion-aware object generation model (Sec.~\ref{sec:objectgen}), and an illustrative figure of the texture generation model.} \label{fig_overview2}
\end{figure}






% \section{Pose-Aware 3D Instance Generation}
% \section{Joint Generation of 3D Object and Transformation}
\section{Perceptive 3D Instance Generation}
\label{sec:method2}

% 为什么我们要一个一个物品生成,为什么不直接生成一个整体场景?
%   (这不是废话吗,生成一个整体场景的mesh有屁用,拆开来一个一个生成才有更多的下游任务)
% 为什么生成单个物品的时候要在Canonical Space?
%   没有数据集先验,生成在世界坐标原地的,几何质量都很差。生成在Canonical Space除了质量更好,更重要的是,Canonical Space实际上是艺术家定义的坐标系和标准,生成在这个空间中就和艺术家一个标准,对于资产标准化、下游应用如编辑、渲染、动画、理解、处理都非常方便。不生成在Canonical Space简直就是反人类。
% 为什么这个section需要部分点云作为输入?
%   因为我们需要和原图更加一致。如果只使用图片而不使用点云,有可能会造成大体上看起来差不多,但是几何形状差很多,比如长度不一样,因此不能完美的放置回场景里。我们希望生成的物品能和图片上完全对应上,深度点云是最精确的控制方式
% 为什么我们不使用icp来把生成好的物品对齐回去?
%   icp没有使用到语义信息,而且效果很差。我们的变换生成器在大数据上训练过,有语义信息
% 
% canonical space 怎么定义的?

In the endeavor to reconstruct high-fidelity 3D scenes from single RGB images, a brute-force approach involves generating the entire scene mesh directly using techniques such as single-image depth estimation or diffusion priors. However, this method inherently struggles to manage occlusions, render invisible components, and accurately represent object relationships due to the complex and intertwined nature of real-world scenes. 
%The intricate configurations and interdependencies among multiple objects often lead to inaccuracies and incomplete reconstructions when relying solely on direct mesh generation. Consequently, these limitations underscore the need for more sophisticated methodologies that can effectively disentangle and model the various elements and their interactions within a scene.
%In the pursuit of reconstructing a high-fidelity 3D scene from a single RGB image, the brute-force way is to generate the entire scene mesh directly using techniques like single image depth or diffusion priors. However, such approach inevitablely fail to handle occlusions、invisible parts and object relations, due to the overly complicated configuration of a scene.
%In the pursuit of reconstructing a high-fidelity 3D scene from a single RGB image, it is imperative to generate and align each object meticulously. 
%Instead of generating an entire scene mesh, our approach focuses on individual object generation and precise transformation alignment. This strategy not only enhances the geometric quality of each asset but also facilitates a wide range of downstream applications such as editing, rendering, and simulation. Moreover, by operating within a canonical space, we ensure that the generated assets adhere to standardized orientations and scales, aligning seamlessly with artist-defined coordinate systems.
Instead of generating an entire scene mesh directly, our approach focuses on individual object generation and then arranges the objects via precise relational alignment, as illustrated in~\ref{fig_overview2}. This strategy offers several advantages: 1. focusing on individual objects ensures higher geometric fidelity and allows for detailed modeling, resulting in more accurate and visually appealing scene components.
%
2. operating within a canonical space ensures that generated assets adhere to standardized orientations and scales, seamlessly integrating with artist-defined coordinate systems and promoting consistency across digital content creation tools.
%
3. the modular approach supports various applications such as editing, rendering, and simulation, enabling independent manipulation of objects for greater flexibility and efficiency.
%
By decomposing scene reconstruction into object-wise generation and alignment, our method improves asset quality and manageability while enhancing the overall coherence and functionality of the 3D environment. This approach addresses challenges like geometric precision and efficient post-processing, advancing single-image 3D scene generation.

Object-wise generation presents significant challenges, primarily due to partial observations of objects within a scene caused by occlusions and limited sensor coverage. Additionally, existing generation methods often fail to coordinate multiple objects cohesively, resulting in inconsistent and unrealistic scenes. To overcome these limitations, we propose an Occlusion-Aware 3D Object Generation framework that integrates partial observations with comprehensive scene understanding. 
Specifically, given an image and its point cloud, our framework generates a high-quality 3D asset that not only resembles the input image but also aligns accurately with the partial point cloud represented in its corresponding canonical space.
Furthermore, we compute a transformation matrix that maps the generated object from its canonical space back to the original scene space, ensuring spatial consistency within the scene.
%This section delves into the core components of our generative pipeline: occlusion-Aware Object Generatio, transformation Generation and joint iterative integration.
% 
%However, object-wise generation also faces considerable challenges. First, we often have only partial observations of each object in the scene due to occlusions and xxx, and the generated objects are not coordinated to produce a consistent scene. To address, we propose an occlusion aware 3D object generation and xxx. Specifically, given the image and the partial point cloud in world space of an object in the scene, we generate a high-quality 3D asset that both resembles the image and aligns with the partial point cloud in canonical space, and a transformation that maps between canonical space and the original world space.
% 
%This section delves into the core components of our generative pipeline: occlusion aware object generation, transformation generation, and their integration through a joint, iterative process.




%\subsection{Object Centric 3D Generation}
%\label{sec:objectgen}
% 为什么生成单个物品的时候要在Canonical Space?
% 为什么需要部分点云作为输入?

%The first module of our pipeline is dedicated to the generation of individual 3D objects. Generating objects one by one, rather than constructing an entire scene mesh, offers significant advantages in terms of quality and usability. When objects are generated independently, each asset can achieve higher geometric fidelity and detail, which is often compromised in holistic scene generation approaches. Additionally, individual object generation aligns with the modular nature of many downstream tasks, allowing for more flexible editing and manipulation.

A critical aspect of our object generation process is the utilization of a large generative model to generate holistic and high-fidelity object meshes from partial image and point cloud observations. To do so, we first follow state-of-art native 3D generative models~\cite{xiang2024structured,zhang20233dshape2vecset,zhang2024clay} to pre-train a large-scale 3D generative model conditioning on textual and image inputs. %
%. By incorporating the partial point cloud, we ensure that the generated geometry is not only visually consistent with the input image but also accurately reflects the underlying depth information. This dual conditioning enables the creation of high-quality 3D assets that maintain structural integrity and geometric precision.
%
%Operating within a canonical space $[-1,1]^3$ is another cornerstone of our design. Without the constraints of a predefined coordinate system, generating objects directly in world space can lead to inconsistencies and subpar geometric quality due to the lack of dataset priors. By standardizing the generation process in a canonical space, we achieve superior asset quality and ensure that each object adheres to a consistent orientation and scale. This alignment with artist-defined standards is crucial for seamless integration into various applications, enhancing the overall usability and interoperability of the generated assets.
%\paragraph{3D Generative Base Model}
%A fundamental component of our pipeline is the generation of high-quality 3D objects that accurately reflect the geometry and appearance depicted in the input image. 
% DetailGen3d withdraw,....  others has cited very nearby?% [CLAY, 3DShape2VecSet, Dora, DetailGen3D]
We build upon existing generative frameworks featuring 3DShape2VecSet representation~\cite{zhang20233dshape2vecset,zhang2024clay}, which prioritize geometry generation by utilizing a Geometry Variational Autoencoder (VAE). This VAE framework encodes uniformly sampled surface point clouds into unordered latent codes and decodes these latent representations into Signed Distance Fields (SDFs). Formally, the VAE encoder $\mathcal{E}$ and decoder $\mathcal{D}$ are defined as:
\begin{equation}
    \bm{Z}=\mathcal{E}(\bm{X}),\ \mathcal{D}(\bm{Z},\bm{p}) = \text{SDF}(\bm{p}),
\end{equation}
where $\bm{X}$ represents the sampled surface point cloud of the geometry, $\bm{Z}$ is the corresponding latent code, and $\text{SDF}(\bm{p})$  denotes the operation of querying the SDF value at point $\bm{p}$ for subsequent mesh extraction via marching cubes.
% 
To effectively incorporate image information into the geometry generation process, we employ DINOv2~\cite{oquab2023dinov2} as our image encoder, following methodologies outlined in \cite{zhang20233dshape2vecset,zhang2024clay, xiang2024structured}, % citet
%  [CLAY, TRELLIS]. 
%DINOv2 excels in capturing rich semantic features from input images, which are crucial for conditioning the generative model. 
The geometry latent diffusion model (LDM) is then formulated as:
\begin{equation}
    \epsilon_\text{obj} (\bm{Z}_t; t,\bm{c}) \rightarrow \bm{Z},
\end{equation}
where $\epsilon$ represents the diffusion transformer model, $\bm{Z}_t$ is noisy geometry latent code at timestep $t$, and $\bm{c}$ denotes the encoded image features from DINOv2. We follow the pre-training process of prior works~\cite{zhang20233dshape2vecset,zhang2024clay} and pre-train the base model on Objaverse~\cite{deitke2023objaverse}. Upon training, our generation model $\epsilon$  is capable of generating detailed 3D geometry solely based on image features.

%\paragraph{Occlusion Adaption}
\subsection{Occlusion-aware 3D Object Generation}
\label{sec:objectgen}

Directly applying 3D generative-based models faces considerable challenges as
real-world scenarios often present challenges such as partial occlusions in the input images, which severely degrade the quality and accuracy of the generated object geometries. To address this issue, we leverage the Masked Auto Encoder (MAE) capabilities of DINOv2. Specifically, during inference, we provide an occlusion mask $\bm{M}$ alongside the input image $\bm{I}$, enabling the encoder to handle missing pixels by inferring latent features for the occluded regions. This is formalized as:
\begin{equation}
    \bm{c}_m=\mathcal{E}_\text{DINOv2}(\bm{I} \odot \bm{M}),    
\end{equation}
where $\bm{M}$ is a binary mask indicating which tokens should be masked and replaced with a [mask] token. During the pretraining phase, DINOv2 is trained with randomly set masks, allowing it to robustly infer missing parts based on the visible regions. Consequently, during inference, even if parts of the object image are occluded, the encoder can effectively reconstruct the necessary features, ensuring that the generative model maintains high-quality and accurate 3D reconstructions.
%
This integration of image conditioning and occlusion handling is pivotal for our pipeline, as it ensures that the generated 3D objects are both visually consistent with the input images and geometrically faithful to the underlying structure. 
%{\color{red} By conditioning the generative model on both robust image features and partial point cloud data, we achieve a harmonious balance between visual fidelity and geometric precision, laying a solid foundation for the subsequent transformation and alignment processes.}


%\paragraph{Partial Point Cloud Conditioning}

%To enhance the fidelity and accuracy of the generated 3D objects, our system incorporates partial point clouds alongside image-based cues. This dual conditioning ensures that the generated geometries not only align visually with the input images but also accurately reflect their underlying scale, shape, and depth. During the training phase, we simulate real-world partial scans or estimated depth map by rendering each 3D asset from multiple viewpoints, thereby obtaining corresponding RGB images, camera parameters, and ground-truth depth maps. These RGB images are then processed using advanced depth estimation techniques such as MoGe [xxx] and DepthAnything, to produce estimated depth map and then projected as partial point clouds. These point clouds are coarsely aligned with the original objects within the canonical $[-1,1]^3$  space to maintain consistency.


\paragraph{Canonical Point Cloud Conditioning} 
Though our object generation model produces visually plausible meshes from input object images, it is challenging to generate pixel-aligned geometry due to the high-level nature of the encoded image condition $\bm{c}$ and the absence of pixel-wise supervision. We address this issue by additionally conditioning our object generation model on observed partial point clouds in canonical coordinates.
%
This dual conditioning ensures that the generated geometries not only align visually with the input images but also accurately reflect their underlying scale, shape, and depth. 
During the conditioning training, we simulate real-world partial scans or estimated depth maps by rendering each 3D asset from multiple viewpoints, thereby obtaining corresponding RGB images, camera parameters, and ground-truth depth maps. These RGB images are then processed using advanced depth estimation techniques, including MoGe~\cite{wang2024moge} and Metric3D~\cite{yang2024depth}, to produce an estimated depth map and then projected as partial point clouds. These point clouds are coarsely aligned with the original objects within the canonical $[-1,1]^3$ space.


To bolster the model's robustness and its ability to generalize across diverse real-world scenarios, we employ a data augmentation strategy that interpolates between ground-truth partial point clouds $\bm{p}_\text{gt}$ (projected from ground truth depth map to simulate accurate depth) and noisier, estimated partial point clouds $\bm{p}_\text{est}$ (projected from estimated depth map and aligned to simulate estimated noisy depth from RGB). This interpolation is mathematically represented as:
%\begin{equation}
 $  \bm{p}_\text{disturb} = \alpha\cdot \bm{p}_\text{gt} +(1-\alpha)\cdot \bm{p}_\text{est}$,
%\end{equation}
where $\alpha\in[0,1]$ is a weighting factor which is sampled uniformly during training. Our object generator, named ``\textit{ObjectGen}'', with partial point cloud conditioning is formulated as
\begin{equation}
    \epsilon (\bm{Z}_t; t,\bm{c},\bm{p}_\text{disturb}) \rightarrow \bm{Z},
\end{equation}
where the conditioning adaptation scheme is based on attention mechanism similar to \cite{zhang20233dshape2vecset,zhang2024clay}. % citet
Additionally, to mimic real-world occlusions and missing data, we randomly mask sets of basic primitives—such as circles and rectangles—in the depth maps from various camera views. This results in partial point clouds with occluded and incomplete regions, further enhancing the model's ability to handle imperfect inputs.
%
A critical design choice in our approach is to maintain the alignment of partial point clouds with the geometry in our training data set. Unlike methods that apply random scaling, translation, or rotation to augmented point clouds, our aligned partial point clouds ensure that the generative model can more effectively conform to the input point clouds's inherent structure. This alignment restricts the model to adhere closely to the actual shapes and scales of objects, thereby facilitating more precise and coherent 3D reconstructions. By conditioning on these well-aligned partial point clouds, our model achieves superior alignment both in overall size and local geometric details, resulting in high-quality and reliable 3D geometry generation.
%\paragraph{Texture Generation}








\subsection{Generative Alignment}
\label{sec:transformationgen}


Each generated 3D object is within a normalized volume and assumes a canonical pose that may not be aligned with the image and scene space point cloud. 
This is because the image conditions use high-level features, such as DINOv2, to achieve better generalization.
Ensuring that each object is correctly transformed and scaled to align with its presentation in the scene is crucial for scene composition.
%
Though traditional alignment methods, such as Iterative Closest Point (ICP)~\cite{arun1987least,best1992method}, can be employed, they often fail to account for semantic context, leading to frequent misalignments and diminished accuracy (see Fig.~\ref{Ablation_icp_dr}). 
% maybe 
Instead, we introduce an alignment generative model conditioned on the scene-space partial point cloud $\bm{q} \in \mathbb{R}^{N\times 3}$ and the canonical-space geometry latent code $\bm{Z}$. Formally, we define our alignment generator ``\textit{AlignGen}'' as:
\begin{equation}
    \epsilon_\text{align} (\bm{p}_t; t, \bm{q}, \bm{Z}) \rightarrow \bm{p},
\end{equation}
where $\epsilon_\text{align}$ is a point cloud diffusion transformer, $\bm{p} \in \mathbb{R}^{N\times 3}$ is the transformed version of the scene-space partial point cloud to the canonical space, aligning with the generated object mesh. $\bm{Z}$ is the generated geometry latent of object corresponding to $\bm{p}$ from the object generation model. 
%
$\bm{p}_t$ is the noised version of $\bm{p}$ at timestep $t$.
%, $\bm{q} \in \mathbb{R}^{N\times 3}$ is the input partial point cloud (in world space) that may have a random transformation, and $\bm{z}$  is the geometry latent code from our object generation module. 
% 
In essence, the generation model maps the scene-space partial point cloud $\bm{q}$ to $\bm{p}$ in the canonical $[-1,1]^3$ space, aligning it with the generated object mesh. We can subsequently recover the similarity transformation (i.e., scaling, rotation, and translation) from $\bm{q}$ and $\bm{p}$ using the Umeyama algorithm~\cite{umeyama1991least} as they are point-wise corresponded. This final step is numerically more stable than directly predicting transformation parameters.% using ICP.%, as it exploits point-wise correspondences between $\bm{T}$ and $\bm{p}$.

% TODO: Add input normalization
In practice, we employ distinct conditioning strategies for the input point cloud $\bm{q}$ and the geometric latent $\bm{Z}$. For $\bm{q}$, we concatenate the input point cloud with the diffusion sample $\bm{p}_t$ along the feature channel dimension, enabling the transformer architecture to learn explicit correspondences between the noisy canonical-frame partial cloud and the world-space partial cloud.
%
For the geometric latent $\bm{Z}$, we apply a cross-attention mechanism to inject it into the point diffusion transformer. This approach ensures that the model effectively incorporates spatial and geometric relationships.
%
Additionally, due to symmetry and replicated geometrical shapes, multiple valid $\bm{p}$ may exist for a given $\bm{q}$ and $\bm{Z}$. Our diffusion model addresses this by sampling multiple noise realizations and aggregating the resulting transformations, to select the most confident and coherent representations.

%In practice, we use distinct conditioning strategies for $\bm{q}$ and $\bm{Z}$ respectively. We concatenate the point clouds $\bm{q}$ with diffusion latent $\bm{p}_t$ along feature channel dimension, allowing the transformer architecture to learn explicit correspondences between the noisy, canonical-frame partial cloud and the world-space partial cloud.
%While for the geometric latent $\bm{Z}$, we use cross attention mechanism to inject into the diffusion transformer.
%Moreover, given $\bm{q}$ and $\bm{Z}$, there may exists multiple valid $\bm{p}$ due to symmetry and replicated geometry shapes.
%Our diffusive model naturally handle these situations by simply sampling multiple noises and averaging more confident and coherent transformations. 
% 
%, reflecting variations in valid object orientation that still satisfy transformation priors and geometric constraints. 
%Hence, we train our transformation generator as a distribution esitmator.
%: 
%During training, for each training model we use VAE to encode it as $\bm{Z}$ and partially sample a set of surface points as $\bm{T}$, and then apply random transformation to synthesize $\bm{p}$.
% 
%By incorporating conditions from geometric latent, our transformation model can infer precise scaling, rotation, and translation parameters, providing better reliability and accuracy. 
% 
%Consequently, the transformation generative model forms a robust bridge between object geometry and real-world placement, paving the way for physically accurate and semantically consistent scene reconstructions.


%\paragraph{Transformation Disambiguity}
%A significant advantage of our transformation generation module is its ability to learn and represent a distribution of possible transformations through the diffusion process. By modeling transformations as a diffusion sampling, our system can generate multiple transformation samples for the same input partial point cloud, capturing the inherent variability in object orientations and scales. This capability is particularly beneficial for objects with symmetrical or non-unique rotational properties, such as circular fountains or square tables, where multiple valid rotations exist without altering the object's essential characteristics.
%
%For instance, consider a cylindrical water bottle that has a fixed up-axis (y-axis) but allows arbitrary rotation around this axis (x and z-axes). Sampling a single transformation in such cases would fail to capture the full range of valid orientations, potentially leading to misalignment or loss of geometric features. By generating multiple transformation samples, our approach ensures that the up-axis remains consistently aligned while allowing variations in the other axes. Averaging these multiple samples helps in accurately determining the dominant orientation and mitigating the impact of any outliers, thereby enhancing the overall alignment accuracy.
%
%To implement this, for each input partial point cloud, we generate up to 10 transformation samples. We then identify and remove outliers from these samples and compute the average transformation based on the remaining valid transformations. This process not only improves the reliability of the transformation estimation but also ensures that the generated 3D objects are accurately and consistently positioned within the scene, aligning with both visual and geometric cues from the input data.




% \subsection{Integration: Joint Iterative Generation}


% % {\color{red}{
% % The true strength of our approach lies in the seamless integration of object generation and transformation alignment through a joint, iterative process. Initially, the object generation module produces a geometry that closely resembles the input image but may not align perfectly with the input partial point cloud due to random transformations inherent in the point cloud data. To rectify this, the transformation generation module estimates the necessary transformations to align the generated geometry with the partial point cloud.

% % This alignment process is not a one-off adjustment but an iterative refinement. Once the initial transformation is applied, the updated partial point cloud and the input image provide a more accurate basis for generating the next iteration of the geometry. By repeatedly alternating between object generation and transformation estimation, the system progressively improves the alignment between the generated assets and the input data. This iterative approach ensures that the final geometry is not only visually consistent with the input image but also geometrically and spatially accurate within the scene.

% % The integration of these two modules addresses the limitations of relying solely on image or point cloud data. By leveraging both visual and depth information in a coordinated manner, our joint generation process achieves a harmonious balance between aesthetic fidelity and geometric precision. This results in 3D assets that are both high in quality and accurately positioned, laying the groundwork for constructing a physically correct and visually coherent scene.
% % }
% % }
% The true strength of our approach lies in the seamless integration of the object generation and transformation alignment modules through a joint, iterative process. This integration ensures that each generated 3D object is not only visually consistent with the input image but also accurately positioned and scaled within the scene. The iterative workflow can be summarized in three key steps:



% A key component of our pipeline is the joint iterative process, which tightly integrates the object generation module and the transformation generation module. This iterative approach progressively refines both the geometry and the alignment, ensuring accurate reconstructions. The process is outlined as follows:

% \paragraph{Step 1: Initial Geometry Generation} The object generation module $\text{ObjectGen}$ (Sec.~\ref{sec:objectgen}) produces a geometry latent code $\bm{z}^{(k)}$ at each iteration. This process is conditioned on the image features $\bm{c}$ (from DINOv2) and the current transformed partial point cloud $\bm{p}^{(k)}$ with conditioning scale factor $\beta^{(k)}$:
% \begin{equation}
%     \bm{z}^{(k)} = \text{ObjectGen}(\bm{c}, \bm{p}^{(k)} \otimes \beta^{(k)}).
% \end{equation}
% $\beta^{(k)}$ is the conditioning scale factor for controlling the impact of transformed partial point cloud conditioning. In the first iteration $k=0$, $\bm{p}^{(0)}$ has no valid value and is set to $q$ and set $ \beta^{(0)}=0$ which means do not use partial point cloud conditioning at the first iteration. $\{\beta^{(k)}\}$ is set to linearly grows from 0 to 1 during iteration.
% The latent code $\bm{z}^{(k)}$ is then decoded into a 3D geometry using the VAE decoder $\mathcal{D}$.

% \paragraph{Step 2: Transformation Estimation} Using the updated geometry latent code $\bm{z}^{(k)}$ and the world-space partial point cloud $\bm{q}$, the transformation generation module $\text{TransformationGen}$ (Sec.~\ref{sec:transformationgen}) predicts a transformed canonical-space partial point cloud $\bm{p}^{(k+1)}$:
% \begin{equation}
%     \bm{p}^{(k+1)} = \text{TransformationGen}(\bm{q}, \bm{z}^{(k)}) \odot \bm{q}.
% \end{equation}

% \paragraph{Step 3: Refinement of Geometry and Alignment} The rigid transformation is applied to update the partial point cloud, improving the alignment between the generated object and the observed scene data. This updated information is fed back into the iterative loop to refine the next generation of geometry and transformations.

% \paragraph{Convergence} This iterative process continues until the geometry and transformation alignments converge, resulting in a high-fidelity object reconstruction that is both visually accurate and spatially coherent.
% The iterative process continues until the change in transformation parameters falls below a predefined threshold or a maximum number of iterations is reached.


%\subsection{Integration: Joint Iterative Generation}
\subsection{Iterative Generation Procedure}

Recall that in our design, the object point cloud is unusable for object generation initially, as it is represented in the scene space, while our object generation model requires canonical-space point cloud for conditioning.
%
%The generated and aligned object xxx coarse, but xxx. Our design enables seamless integration of the object generation and alignment modules through a joint, iterative process. 
Solely depending on image cues for object generation often fails to produce pixel-aligned geometry, mainly because of the high-level semantic conditioning and biases inherent in 3D datasets.
Fortunately, our design enables seamless integration of the object generation and alignment modules through a joint, iterative process. 
This integration ensures that each generated 3D object is not only visually consistent with the input image but also accurately positioned and scaled within the scene. The iterative workflow with step index $k$ can be summarized in three key steps:

\noindent\paragraph{Step 1: Object Generation}
%
For an object image with mask, the \textbf{Object Generation} module (Sec.~\ref{sec:objectgen}) synthesizes the geometry latent code \(\bm{z}^{(k)}\) based on the image features \(\bm{c}\) derived from DINOv2 and the aligned point cloud \(\bm{p}^{(k)}\) in canonical coordinates. We set $\bm{p}^{(0)}$ to scene space point cloud $\bm{q}$ and set the point cloud conditioning scale factor \(\beta^{(k)}\) to progressively increase from 0 to 1 as iteration procedure goes on, allowing the partial point cloud to take influence over time. Formally, this process is represented as:
\begin{equation}
    \bm{z}^{(k)} = \text{ObjectGen}(\bm{c}, \bm{p}^{(k)} \otimes \beta^{(k)}).
\end{equation}
%In the initial iteration (\(k=0\)), \(\bm{p}^{(0)}\) is set to the input partial point cloud \(\bm{p}\) estimated from image with \(\beta^{(0)} = 0\). 
Hence our object generator solely relies on masked image conditioning in the first step. The latent code \(\bm{z}^{(k)}\) is then decoded into a 3D geometry using the VAE decoder \(\mathcal{D}\).

\noindent \paragraph{Step 2: Alignment}
%
Subsequently, the \textbf{Generative Alignment} module (Sec.~\ref{sec:transformationgen}) takes the newly generated geometry latent code \(\bm{z}^{(k)}\) and the partial point cloud \(\bm{q}\) in scene coordinates to predict a transformed canonical-space partial point cloud \(\bm{p}^{(k+1)}\):

\begin{equation}
\bm{p}^{(k+1)} = \text{AlignGen}(\bm{q}, \bm{z}^{(k)}).
\end{equation}

This transformed point cloud \(\bm{p}^{(k+1)}\) serves as an improved alignment reference for the next iteration. By leveraging the generative transformation model, the model ensures that the scaling, rotation, and translation adjustments are both precise and semantically informed.%, surpassing traditional methods like ICP in accuracy and stability.

\noindent \paragraph{Step 3: Refinement}
%
With the updated partial point cloud \(\bm{p}^{(k+1)}\), the system could estimate a new similarity transformation to refine the alignment of the generated geometry within the scene. This updated partial point cloud is then fed back into the \textbf{Object Generation} module for the next iteration, allowing for progressive enhancements in both geometry accuracy and spatial positioning.

%\paragraph{Convergence}
\medskip
This iterative loop—alternating between geometry generation and transformation estimation—continues until convergence criteria are met. Convergence is achieved when the changes in transformation parameters fall below a predefined threshold or when a maximum number of iterations is reached. The result is a high-fidelity 3D object that is both visually accurate and geometrically aligned with the input data.
%
%\medskip
By tightly integrating the \textbf{Object Generation} and \textbf{Alignment Generation} modules within an iterative framework, our approach effectively balances aesthetic fidelity with geometric precision. This joint generation process leverages both visual and depth information, ensuring that each 3D asset is of high quality and accurately positioned. Consequently, the pipeline lays a robust foundation for constructing physically correct and visually coherent 3D scenes, facilitating a wide range of downstream applications such as editing, rendering, and animation.

Once the object geometry is determined, we apply a state-of-the-art texture generation module to create photo-realistic surface details. Following established texture synthesis pipelines~\cite{zhang20233dshape2vecset,zhang2024clay}, we assign UV mappings and train a generative network to paint detailed textures onto the 3D meshes. This module is designed to robustly handle images under various augmentation, ensuring that the final textures match the input appearance even under occlusion or limited visibility.


% While the transformer-based RTS estimation provides initial placement parameters, achieving precise alignment within the 3D scene necessitates further refinement. We introduce an iterative refinement process that evaluates alignment quality and adjusts RTS parameters accordingly to ensure high-fidelity object placement. This process enhances the accuracy and quality of RTS parameter estimation, ensuring that each object's placement meets predefined alignment and quality criteria.


% \paragraph{Refinement Algorithm}
% The refinement process is fully automated, eliminating the need for manual intervention and ensuring consistent and reliable placement of objects within the 3D scene. The iterative loop operates as follows:IoU Computation:
% Calculate the IoU between the point cloud and the current mesh reconstruction to assess alignment quality.RTS Adjustment:If the IoU falls below the predefined threshold, the transformer model predicts a refined set of RTS parameters based on the current alignment status.Reconstruction Update:Apply the refined RTS parameters to the 3D mesh, updating its position and orientation within the scene.Iteration:Repeat the IoU computation and RTS adjustment steps until the IoU meets or exceeds the threshold or a maximum number of iterations is reached.

% To optimize computational efficiency, the iterative refinement loop incorporates adaptive learning rates and early termination criteria, preventing excessive iterations and reducing processing time. Additionally, parallel processing strategies are employed to handle multiple objects simultaneously, further enhancing the efficiency of the refinement process.

% The iterative refinement ensures that each object's placement is both geometrically accurate and contextually coherent, addressing common challenges such as object misalignment and geometric inconsistencies. By systematically refining RTS parameters based on quantitative quality metrics, we achieve high-quality and realistic 3D scene reconstructions that faithfully represent the original input image.
\label{sec:method4}

% \section{Physical Simulation and Consistency Enforcement}
% \section{Physical Consistency and Relationship Enforcement}


\begin{figure*}
  \centering
  \includegraphics[width=\textwidth]{figs/sec5.png}
  \caption{Physics-aware correction via constraint graph mapped from fine-grained relation graph. Top: Floating surfboard grounded on the van. Bottom: Penetrating guitar and cooler separated.} 
  \label{fig_simu_demonstrate}
\end{figure*}

\begin{figure*}
  \centering
  \includegraphics[width=\textwidth]{figs/gallery.png}
  \caption{Bringing the vibrant diversity of the real world into the virtual realm, this collection reimagines open-vocabulary scenes as immersive digital environments, capturing the richness and depth of each unique setting. For each scene, the images display as follows: the top-left shows the input image, the top-center displays the rendered geometry, and the right presents the rendered image with realistic textures.}
  \label{fig_gallery}
\end{figure*}

% 
% \begin{figure*} []
%   \centering
%   \includegraphics[width=\textwidth]{figs/sec5.png}
%   \captionof{figure}{Physics-aware correction via constraint graph mapped from fine-grained relation graph. Top: Floating surfboard grounded on the van. Bottom: Penetrating guitar and cooler separated.} \label{fig_simu_demonstrate}
% \end{figure*}

% \begin{figure*} []
%   \centering
%   \includegraphics[width=\textwidth]{figs/gallery.png}
%   \captionof{figure}{Bringing the vibrant diversity of the real world into the virtual realm, this collection reimagines open-vocabulary scenes as immersive digital environments, capturing the richness and depth of each unique setting. For each scene, the images display as follows: the top-left shows the input image, the top-center displays the rendered geometry, and the right presents the rendered image with realistic textures.}
%   \label{fig_gallery}
% \end{figure*}

\section{Physics-Aware Correction}
\label{sec:physics-aware-correction}

The pipeline detailed in Sec.~\ref{sec:method2} individually generates each 3D object instance and estimates its similarity transformation (scaling, rotation, and translation) based on a single input image. While our proposed modules achieve high accuracy, the resulting scenes are sometimes not physically plausible. For instance, as illustrated in Fig.~\ref{fig_simu_demonstrate}, one object (e.g., a guitar) may intersect with another (e.g., a cooler), or an object (e.g., a surfboard) may appear to float unnaturally without any support (e.g., from a van). 

To address these issues, we introduce a physics-aware correction process that optimizes the rotation and translation of objects, ensuring the scene adheres to physical constraints consistent with common sense.
The correction process is motivated by physical simulation (Sec.~\ref{sec:rigid-body-sim}) and formulated as an optimization problem (Sec.~\ref{sec:physical-constraints}) based on inter-object relationships represented by a scene graph (Sec.~\ref{sec:scene-relation-graph}) extracted from the image.

\subsection{A Quick Primer to Rigid-Body Simulation}
\label{sec:rigid-body-sim}

We introduce the fundamental principles of physical (rigid-body) simulation, which inspire our problem formulation and make our framework more accessible to downstream applications such as gaming and robotics. For a thorough survey, we refer the readers to \cite{BenderETC12}. % citet

In rigid-body simulations, the world is modeled as an ordinary differential equation (ODE) process. In each simulation step, it begins with the Newton-Euler (differential) equations, which describe the dynamic motion of rigid bodies in the absence of contact. \emph{Collision detection} is conducted to find the contact points between rigid bodies, which are needed to determine contact forces. For contact handling and collision resolution, there are usually several conditions: non-penetration constraints to prevent bodies from overlapping, a friction model ensuring contact forces remain within their friction cones, and complementarity constraints that enforce specific disjunctive relationships among variables. Solvers are used to resolve the system comprising equations and inequalities, subsequently updating the velocity and position of each rigid body.

A straightforward approach to enhance physical plausibility is to utilize an off-the-shelf rigid-body simulator to process the scene, starting from the initial state estimated by the pipeline previously described and obtaining the rest state after simulation. However, this method presents several challenges.
% \\1) \emph{Partial Scene}: Since our primary focus is scene generation from a single image rather than reconstruction, some objects may be missing. Simulating a partial scene under full physical rules can lead to suboptimal results (see Fig.~\ref{fig_simu_demonstrate}).
\\ 1) \emph{Partial Scene}: Some objects may be missing due to the limitations of 2D foundation models, and thus not reconstructed. Simulating a partial scene under full physical rules can lead to suboptimal results (see Fig.~\ref{Ablation_simu}). 
\\2) \emph{Imperfect Geometries}: While our 3D generative model produces high-quality geometries, minor imperfections may still occur. Rigid-body simulators typically require convex decomposition~\cite{mamou2009simple,mamou2016volumetric,wei2022approximate} of objects, which introduces additional complexity and hyperparameters. Overly fine-grained decomposition can result in non-flat, complex surfaces, causing objects to fall or move unexpectedly during simulation. Conversely, coarse decomposition may lead to visually floating objects due to discrepancies between the visual and collision geometries.
\\3) \emph{Initial Penetrations}: Despite the high accuracy of pose estimation, significant inter-object penetrations may exist in the initial state. These penetrations create instability for standard rigid-body solvers and, in some cases, lead to unsolvable scenarios if the solver is not customized for those cases.

Thus, we propose a customized and simplified ``physical simulation'' to optimize the object poses, ensuring that the scene adheres to common-sense physical principles derived from the single image.
\emph{Note that our approach does not model full dynamics. For example, an object may not remain stable in its current pose over time. However, it should be physically plausible at the current time step. We argue that our optimized results can serve as a reliable initialization for subsequent physical simulations.}




\subsection{Problem Formulation and Physical Constraints}
\label{sec:physical-constraints}

We formulate the physics-aware correction process as an optimization problem, aiming to minimize the total cost that represents pairwise constraints on objects.

\begin{equation}
\label{eq:physics-aware-objective}
    \min_{\mathcal{T}=\{T_1,T_2,\dots,T_N\}} \sum_{i,j}C(T_i,T_j;\bm{o}_i,\bm{o}_j)
\end{equation}
where $N$ is the number of objects, $T_i$ is the rigid transformation (rotation and translation) of the i-th object $\bm{o}_i$. $C$ is the cost function representing the  relationship between $\bm{o}_i$ and $\bm{o}_j$. Note that the cost function varies depending on the type of relationship.

Motivated by physical simulation, we categorize the relationships into two types: \emph{contact} and \emph{support}. The relationships are identified with the assistance of a VLM, as detailed in Sec.~\ref{sec:scene-relation-graph}.

1) \emph{Contact} describes whether two objects $\bm{o}_i$ and $\bm{o}_j$ are in contact. Let $D_i(p)$ denote the signed distance function induced by $\bm{o}_i$ at the point $p$, which is used to define the constraint. $D_i(p)=D_j(p)=0$ indicates that $p$ is a contact point of $\bm{o}_i$ and $\bm{o}_j$. When $D_i(p)=0$ ($p$ is a surface point of $\bm{o}_i$), $D_j(p)<0$ indicates inter-object penetration, while $D_j(p)>0$ means the objects are separated. Thus, the cost function can be defined as:

% \begin{equation}
% \label{eq:contact-constraint}
% \begin{split}
%     C(T_i,T_j) = \max(|\min_{p \in \partial \bm{o}_i} D_j(p(T_i))|, \sigma) + \max(|\min_{p \in \partial \bm{o}_j} D_i(p(T_j)|, \sigma)\\
%     \text{if } \bm{o}_i \text{ and } \bm{o}_j \text{ are in contact}
% \end{split}
% \end{equation}
\begin{equation}
\label{eq:contact-constraint}
\begin{split}
    C(T_i, T_j; \bm{o}_i \to \bm{o}_j) = 
    \frac{\sum_{p \in \partial \bm{o}_j} D_i(p(T_j)) \mathbb{I}(D_i(p(T_j))<0)}{\sum_{p \in \partial \bm{o}_j} \mathbb{I}(D_i(p(T_j))<0)} 
    \\
    + \max(\min_{p \in \partial \bm{o}_j}D_i(p(T_j)),0)
    \\
    C(T_i, T_j; \bm{o}_j \to \bm{o}_i) =\frac{\sum_{p \in \partial \bm{o}_i} D_j(p(T_i)) \mathbb{I}(D_j(p(T_i))<0)}{\sum_{p \in \partial \bm{o}_i} \mathbb{I}(D_j(p(T_i))<0)}
    \\
    + \max(\min_{p \in \partial \bm{o}_i}D_j(p(T_i)),0)
    \\
    C(T_i,T_j) = C(T_i, T_j; \bm{o}_i \to \bm{o}_j) + C(T_i, T_j; \bm{o}_j \to \bm{o}_i)
    \\
    \text{if } \bm{o}_i \text{ and } \bm{o}_j \text{ are in contact}
\end{split}
\end{equation}
where $\partial \bm{o}_i$ denotes the surface of $\bm{o}_i$, and $\mathbb{I}$ is the indicator function. The constraint ensures that there is no penetration and at least one contact point between the objects. Note that $p \in \partial \bm{o}_i$ is a function of $T_i$. The contact constraint defined here is bilateral, meaning it applies to both objects.

2) \emph{Support} is a unilateral constraint, which is a special case of \emph{Contact}. If $\bm{o}_i$ supports $\bm{o}_j$, it implies that the pose $T_j$ of $\bm{o}_j$ should be optimized while $\bm{o}_i$ is assumed to be static. This scenario typically occurs when multiple objects are stacked vertically. The cost function for this case is similar to the one in \emph{Contact}, but it only involves one direction:

\begin{equation}
\label{eq:support-constraint}
    C(T_i,T_j) = |\min_{p \in \partial \bm{o}_j} D_i(p(T_j)|,\ \text{if } \bm{o}_i \text{ supports } \bm{o}_j
\end{equation}

% In addition, we simulate the gravitational force acting on the supported object $\bm{o}_j$ to ensure it makes contact with the supporting object $\bm{o}_i$ in a physically realistic manner, minimizing the likelihood of it toppling over.
% \begin{equation}
% \label{eq:support-constraint2}
%     C(:, T_j) = \frac{\sum_{p \in \partial \bm{o}_j} p(T_j) \cdot \vec{g}}{|\partial \bm{o}_j|}
% \end{equation}
% where $g$ is the direction of gravity.

Furthermore, for flat supporting surfaces like the ground or walls, we regularize the SDF values near the contact region, to ensure that objects make close contact with these surfaces. This regularization handles scenarios where objects are partially reconstructed, such as a van with only two wheels, as illustrated in Fig.~\ref{fig_simu_demonstrate}.
\begin{equation}
\label{eq:support-constraint3}
    C(T_i,T_j) = \frac{\sum_{p \in \partial \bm{o}_j} D_i(p(T_j) \mathbb{I}(0<D_i(p)<\sigma)}{\sum_{p \in \partial \bm{o}_j} \mathbb{I}(0<D_i(p)<\sigma)}
\end{equation}
where $\mathbb{I}$ is the indicator function, and $\sigma$ is a threshold to decide whether a point is sufficiently close to the surface.

% For special supporting objects like the ground or a wall as $\bm{o}_i$, we additionally regularize the pose of $\bm{o}_j$, to ensure that the normals of $\bm{o}_j$ near the contact point align closely with the normal of $\bm{o}_j$ -- typically vertical for the ground and horizontal for walls.

% \begin{equation}
% \label{eq:support-constraint2}
%     C_{support}(T_i,T_j) = \frac{1}{|\mathcal{N}(p)|}\sum_{p \in \mathcal{N}(P_{c})} -\vec{N_i}(T_i) \cdot \vec{N_j}(p(T_j))
% \end{equation}
% where $P_{c}$ represents the contact point found in Eq.~\ref{eq:support-constraint}, which achieves the minimum, and $\mathcal{N}(P_{c})$ denotes the neighborhood of $P_{c}$.

\subsection{Scene Relation Graph}
\label{sec:scene-relation-graph}

Physical cues, particularly inter-object relationships, are visually present in the image.  We leverage the strong common-sense reasoning capabilities~\cite{rana2023sayplan,li2024llm,cheng2024navila} of visual-language models, specifically GPT-4v~\cite{achiam2023gpt}, to identify pairwise physical constraints as defined in Sec.~\ref{sec:physical-constraints}. Given an image, we employ the Set of Mark~\cite{yang2023setofmark} (SoM) technique to visually prompt GPT-4v to describe the inter-object relationships, and subsequently extract a \emph{scene relation graph} from the answers.
To address the sampling uncertainty inherent in VLMs, we adopt an ensemble strategy, combining results from multiple trials to produce a robust inferred graph. 
% \red{The detailed prompt design is provided in the Appendix}. 

Instead of directly asking GPT-4v to identify \emph{Support} and \emph{Contact} relationships, we first provide it with more fine-grained physical relationships, such as \emph{Stack} (Object 2 supports Object 1), \emph{Lean} (Object 1 leans against Object 2), and \emph{Hang} (Object 2 supports Object 1 from above). We then map these detailed relationships to the predefined categories of \emph{Support} and \emph{Contact} for further optimization. Specifically, if there are edges pointing toward each other between two nodes, the edge is categorized as \emph{Contact}; otherwise, it is categorized as \emph{Support}. Prompting GPT-4v with these nuanced relationships helps eliminate potential ambiguity in binary relationship classification and facilitates more accurate reasoning by GPT-4v.
An example of the resulting graph is illustrated in Fig.~\ref{fig_simu_demonstrate}.

% \red{Mention the mapping!}
The mapped scene constraint graph is a directed graph where nodes represent object instances and edges denote physical relationships between objects. A \emph{Contact} relationship is represented by a bidirectional edge, while a \emph{Support} relationship is depicted as a directed edge. This graph serves as the foundation for defining the cost functions used in Eq.~\ref{eq:physics-aware-objective}.




\subsection{Optimization with Physics-Aware Relation Graph}

Given the physical constraints defined by the inferred relation graph, we can instantiate our cost functions as described in Eq.~\ref{eq:physics-aware-objective}. The graph allows us to reduce the number of pairwise constraints that need to be optimized, in contrast to a full physical simulation.

For the implementation, we uniformly sample a fixed number of points from the surface of each object at its rest pose. These points are then transformed according to the current object’s pose parameters and used to query the SDF values with respect to another object (and its pose). SDF computation is handled by Open3D, and Pytorch is used to auto-differentiate the loss function.
% We disable the optimization for 

% \subsection{Verification}

% An interesting case is that sometimes the graph may introduce some conflict that can be detected by our optimization. For example, in Figure xxx, the graph shows that the surfboard is supported by both the blanket and the ground while the blanket is also supported by the ground. We know that the blanket is thin, but strictly speaking the blanket is supported by the blanket rather than the ground. Our final result after optimization indicates that the constraint between the blanket and the ground is not satisfied, which can be corrected by human if needed.



% separate object generation and 6D pose estimation -> physically realistic
% why not directly physical simulation? missing objects, severe penetrations (undefined behaviors), static objects (like wall)
% Thus, we introduce relationship graph for essential constraints but not comprehensive.

% The pipeline described in the previous section separately reconstructs each object instance from a single image, by generating its geometry and estimating its 6D pose as well as size. However, the resulting reconstructed scene may be physically unrealistic. For example, some objects may penetrate each other or be floating unnaturally. To address these issues, it is essential to refine the object poses to ensure they adhere to physical constraints grounded in common sense.
% In this section, we detail our physics-aware post-processing approach. First, we utilize VLMs to generate a relationship graph for the input image. This graph describes inter-object relationships, such as contact, with edges representing physical constraints that the scene should satisfy.
% Inspired by physical simulations~\cite{}, we incorporate these constraints into an optimization framework and refine object poses.

% \subsection{Relationship Graph and Constraints}

% The core of rigid-body simulation is handling collision and contact.
% Cite the classical method. However, collision detection can be hard as we need convex decomposition.
% Inspired, we introduce a more computationally friendly method based on SDF.
% Define contact -> for all x, 0 < min(SDF(x)) < thresh
% Define support -> and the gradient should be upward
% Define hanging -> and the gradient should be downward

% ``The central aspects of a rigid body simulation are contact and joint handling''~\cite{}. One of the most fundamental physical constraints is that objects must not penetrate each other. \citet{} introduced a widely adopted approach in rigid-body simulation that separately handles contact and collisions. For collisions, impulse velocities are calculated by solving a system of 15 equations, leveraging the principles of momentum conservation and Newton’s law of restitution. To mitigate interpenetration, a single virtual spring is introduced.

% Inspired by the penalty method~\cite{} in physical simulation, we apply a restorative force at the point
% of deepest penetration between two bodies to resolve penetration.

% Denote each geometry by $X_i$. The most important physical constraint is that the objects should not penetrate each other. Two objects $X_i$ and $X_j$ do not penetrate indicates that $\min\{SDF(x_i, x_j)\} > 0$. Besides, if two objects contact, it implies that $\min\{SDF(x_i, x_j)\} < \sigma$.

% In our design, the relationship graph defines 3 different types of relationship: \emph{Contact}, \emph{Support}, \emph{Hanging}. \emph{Contact} describes whether two objects contact, and it is a bi-directional relationship. It satisfies (a) no penetration and (b) has contact point. \emph{Support} describes whether the object A supports the object B. It is a directional relationship, so that we only optimize the pose of B later.

% Note that we should first resolve severe penetrations before putting the scene into the rigid-body simulation. 

% Use sampled points as proxy to approximate
% Highlight that the gradient is the direction to the closest point
%-------------------------------
\section{Physics-Aware Correction}
\label{sec:physics-aware-correction}

The pipeline detailed in Sec.~\ref{sec:method2} individually generates each 3D object instance and estimates its similarity transformation (scaling, rotation, and translation) based on a single input image. While our proposed modules achieve high accuracy, the resulting scenes are sometimes not physically plausible. For instance, as illustrated in Fig.~\ref{fig_simu_demonstrate}, one object (e.g., a guitar) may intersect with another (e.g., a cooler), or an object (e.g., a surfboard) may appear to float unnaturally without any support (e.g., from a van). 

To address these issues, we introduce a physics-aware correction process that optimizes the rotation and translation of objects, ensuring the scene adheres to physical constraints consistent with common sense.
The correction process is motivated by physical simulation (Sec.~\ref{sec:rigid-body-sim}) and formulated as an optimization problem (Sec.~\ref{sec:physical-constraints}) based on inter-object relationships represented by a scene graph (Sec.~\ref{sec:scene-relation-graph}) extracted from the image.

\subsection{A Quick Primer to Rigid-Body Simulation}
\label{sec:rigid-body-sim}

We introduce the fundamental principles of physical (rigid-body) simulation, which inspire our problem formulation and make our framework more accessible to downstream applications such as gaming and robotics. For a thorough survey, we refer the readers to \cite{BenderETC12}. % citet

In rigid-body simulations, the world is modeled as an ordinary differential equation (ODE) process. In each simulation step, it begins with the Newton-Euler (differential) equations, which describe the dynamic motion of rigid bodies in the absence of contact. \emph{Collision detection} is conducted to find the contact points between rigid bodies, which are needed to determine contact forces. For contact handling and collision resolution, there are usually several conditions: non-penetration constraints to prevent bodies from overlapping, a friction model ensuring contact forces remain within their friction cones, and complementarity constraints that enforce specific disjunctive relationships among variables. Solvers are used to resolve the system comprising equations and inequalities, subsequently updating the velocity and position of each rigid body.

A straightforward approach to enhance physical plausibility is to utilize an off-the-shelf rigid-body simulator to process the scene, starting from the initial state estimated by the pipeline previously described and obtaining the rest state after simulation. However, this method presents several challenges.
% \\1) \emph{Partial Scene}: Since our primary focus is scene generation from a single image rather than reconstruction, some objects may be missing. Simulating a partial scene under full physical rules can lead to suboptimal results (see Fig.~\ref{fig_simu_demonstrate}).
\\ 1) \emph{Partial Scene}: Some objects may be missing due to the limitations of 2D foundation models, and thus not reconstructed. Simulating a partial scene under full physical rules can lead to suboptimal results (see Fig.~\ref{Ablation_simu}). 
\\2) \emph{Imperfect Geometries}: While our 3D generative model produces high-quality geometries, minor imperfections may still occur. Rigid-body simulators typically require convex decomposition~\cite{mamou2009simple,mamou2016volumetric,wei2022approximate} of objects, which introduces additional complexity and hyperparameters. Overly fine-grained decomposition can result in non-flat, complex surfaces, causing objects to fall or move unexpectedly during simulation. Conversely, coarse decomposition may lead to visually floating objects due to discrepancies between the visual and collision geometries.
\\3) \emph{Initial Penetrations}: Despite the high accuracy of pose estimation, significant inter-object penetrations may exist in the initial state. These penetrations create instability for standard rigid-body solvers and, in some cases, lead to unsolvable scenarios if the solver is not customized for those cases.

Thus, we propose a customized and simplified ``physical simulation'' to optimize the object poses, ensuring that the scene adheres to common-sense physical principles derived from the single image.
\emph{Note that our approach does not model full dynamics. For example, an object may not remain stable in its current pose over time. However, it should be physically plausible at the current time step. We argue that our optimized results can serve as a reliable initialization for subsequent physical simulations.}




\subsection{Problem Formulation and Physical Constraints}
\label{sec:physical-constraints}

We formulate the physics-aware correction process as an optimization problem, aiming to minimize the total cost that represents pairwise constraints on objects.

\begin{equation}
\label{eq:physics-aware-objective}
    \min_{\mathcal{T}=\{T_1,T_2,\dots,T_N\}} \sum_{i,j}C(T_i,T_j;\bm{o}_i,\bm{o}_j)
\end{equation}
where $N$ is the number of objects, $T_i$ is the rigid transformation (rotation and translation) of the i-th object $\bm{o}_i$. $C$ is the cost function representing the  relationship between $\bm{o}_i$ and $\bm{o}_j$. Note that the cost function varies depending on the type of relationship.

Motivated by physical simulation, we categorize the relationships into two types: \emph{contact} and \emph{support}. The relationships are identified with the assistance of a VLM, as detailed in Sec.~\ref{sec:scene-relation-graph}.

1) \emph{Contact} describes whether two objects $\bm{o}_i$ and $\bm{o}_j$ are in contact. Let $D_i(p)$ denote the signed distance function induced by $\bm{o}_i$ at the point $p$, which is used to define the constraint. $D_i(p)=D_j(p)=0$ indicates that $p$ is a contact point of $\bm{o}_i$ and $\bm{o}_j$. When $D_i(p)=0$ ($p$ is a surface point of $\bm{o}_i$), $D_j(p)<0$ indicates inter-object penetration, while $D_j(p)>0$ means the objects are separated. Thus, the cost function can be defined as:

% \begin{equation}
% \label{eq:contact-constraint}
% \begin{split}
%     C(T_i,T_j) = \max(|\min_{p \in \partial \bm{o}_i} D_j(p(T_i))|, \sigma) + \max(|\min_{p \in \partial \bm{o}_j} D_i(p(T_j)|, \sigma)\\
%     \text{if } \bm{o}_i \text{ and } \bm{o}_j \text{ are in contact}
% \end{split}
% \end{equation}
\begin{equation}
\label{eq:contact-constraint}
\begin{split}
    C(T_i, T_j; \bm{o}_i \to \bm{o}_j) = 
    \frac{\sum_{p \in \partial \bm{o}_j} D_i(p(T_j)) \mathbb{I}(D_i(p(T_j))<0)}{\sum_{p \in \partial \bm{o}_j} \mathbb{I}(D_i(p(T_j))<0)} 
    \\
    + \max(\min_{p \in \partial \bm{o}_j}D_i(p(T_j)),0)
    \\
    C(T_i, T_j; \bm{o}_j \to \bm{o}_i) =\frac{\sum_{p \in \partial \bm{o}_i} D_j(p(T_i)) \mathbb{I}(D_j(p(T_i))<0)}{\sum_{p \in \partial \bm{o}_i} \mathbb{I}(D_j(p(T_i))<0)}
    \\
    + \max(\min_{p \in \partial \bm{o}_i}D_j(p(T_i)),0)
    \\
    C(T_i,T_j) = C(T_i, T_j; \bm{o}_i \to \bm{o}_j) + C(T_i, T_j; \bm{o}_j \to \bm{o}_i)
    \\
    \text{if } \bm{o}_i \text{ and } \bm{o}_j \text{ are in contact}
\end{split}
\end{equation}
where $\partial \bm{o}_i$ denotes the surface of $\bm{o}_i$, and $\mathbb{I}$ is the indicator function. The constraint ensures that there is no penetration and at least one contact point between the objects. Note that $p \in \partial \bm{o}_i$ is a function of $T_i$. The contact constraint defined here is bilateral, meaning it applies to both objects.

2) \emph{Support} is a unilateral constraint, which is a special case of \emph{Contact}. If $\bm{o}_i$ supports $\bm{o}_j$, it implies that the pose $T_j$ of $\bm{o}_j$ should be optimized while $\bm{o}_i$ is assumed to be static. This scenario typically occurs when multiple objects are stacked vertically. The cost function for this case is similar to the one in \emph{Contact}, but it only involves one direction:

\begin{equation}
\label{eq:support-constraint}
    C(T_i,T_j) = |\min_{p \in \partial \bm{o}_j} D_i(p(T_j)|,\ \text{if } \bm{o}_i \text{ supports } \bm{o}_j
\end{equation}

% In addition, we simulate the gravitational force acting on the supported object $\bm{o}_j$ to ensure it makes contact with the supporting object $\bm{o}_i$ in a physically realistic manner, minimizing the likelihood of it toppling over.
% \begin{equation}
% \label{eq:support-constraint2}
%     C(:, T_j) = \frac{\sum_{p \in \partial \bm{o}_j} p(T_j) \cdot \vec{g}}{|\partial \bm{o}_j|}
% \end{equation}
% where $g$ is the direction of gravity.

Furthermore, for flat supporting surfaces like the ground or walls, we regularize the SDF values near the contact region, to ensure that objects make close contact with these surfaces. This regularization handles scenarios where objects are partially reconstructed, such as a van with only two wheels, as illustrated in Fig.~\ref{fig_simu_demonstrate}.
\begin{equation}
\label{eq:support-constraint3}
    C(T_i,T_j) = \frac{\sum_{p \in \partial \bm{o}_j} D_i(p(T_j) \mathbb{I}(0<D_i(p)<\sigma)}{\sum_{p \in \partial \bm{o}_j} \mathbb{I}(0<D_i(p)<\sigma)}
\end{equation}
where $\mathbb{I}$ is the indicator function, and $\sigma$ is a threshold to decide whether a point is sufficiently close to the surface.

% For special supporting objects like the ground or a wall as $\bm{o}_i$, we additionally regularize the pose of $\bm{o}_j$, to ensure that the normals of $\bm{o}_j$ near the contact point align closely with the normal of $\bm{o}_j$ -- typically vertical for the ground and horizontal for walls.

% \begin{equation}
% \label{eq:support-constraint2}
%     C_{support}(T_i,T_j) = \frac{1}{|\mathcal{N}(p)|}\sum_{p \in \mathcal{N}(P_{c})} -\vec{N_i}(T_i) \cdot \vec{N_j}(p(T_j))
% \end{equation}
% where $P_{c}$ represents the contact point found in Eq.~\ref{eq:support-constraint}, which achieves the minimum, and $\mathcal{N}(P_{c})$ denotes the neighborhood of $P_{c}$.

\subsection{Scene Relation Graph}
\label{sec:scene-relation-graph}

Physical cues, particularly inter-object relationships, are visually present in the image.  We leverage the strong common-sense reasoning capabilities~\cite{rana2023sayplan,li2024llm,cheng2024navila} of visual-language models, specifically GPT-4v~\cite{achiam2023gpt}, to identify pairwise physical constraints as defined in Sec.~\ref{sec:physical-constraints}. Given an image, we employ the Set of Mark~\cite{yang2023setofmark} (SoM) technique to visually prompt GPT-4v to describe the inter-object relationships, and subsequently extract a \emph{scene relation graph} from the answers.
To address the sampling uncertainty inherent in VLMs, we adopt an ensemble strategy, combining results from multiple trials to produce a robust inferred graph. 
% \red{The detailed prompt design is provided in the Appendix}. 

Instead of directly asking GPT-4v to identify \emph{Support} and \emph{Contact} relationships, we first provide it with more fine-grained physical relationships, such as \emph{Stack} (Object 2 supports Object 1), \emph{Lean} (Object 1 leans against Object 2), and \emph{Hang} (Object 2 supports Object 1 from above). We then map these detailed relationships to the predefined categories of \emph{Support} and \emph{Contact} for further optimization. Specifically, if there are edges pointing toward each other between two nodes, the edge is categorized as \emph{Contact}; otherwise, it is categorized as \emph{Support}. Prompting GPT-4v with these nuanced relationships helps eliminate potential ambiguity in binary relationship classification and facilitates more accurate reasoning by GPT-4v.
An example of the resulting graph is illustrated in Fig.~\ref{fig_simu_demonstrate}.

% \red{Mention the mapping!}
The mapped scene constraint graph is a directed graph where nodes represent object instances and edges denote physical relationships between objects. A \emph{Contact} relationship is represented by a bidirectional edge, while a \emph{Support} relationship is depicted as a directed edge. This graph serves as the foundation for defining the cost functions used in Eq.~\ref{eq:physics-aware-objective}.




\subsection{Optimization with Physics-Aware Relation Graph}

Given the physical constraints defined by the inferred relation graph, we can instantiate our cost functions as described in Eq.~\ref{eq:physics-aware-objective}. The graph allows us to reduce the number of pairwise constraints that need to be optimized, in contrast to a full physical simulation.

For the implementation, we uniformly sample a fixed number of points from the surface of each object at its rest pose. These points are then transformed according to the current object’s pose parameters and used to query the SDF values with respect to another object (and its pose). SDF computation is handled by Open3D, and Pytorch is used to auto-differentiate the loss function.
%-------------------------------




\section{Experiments}\label{sec:experiments}


\begin{figure*}[!ht]
\centering
\includegraphics[width=0.32\textwidth]{eps_figs/eps_eps_diff_m_Random.pdf}
\includegraphics[width=0.32\textwidth]{eps_figs/eps_eps_diff_m_False_Negative.pdf}
\includegraphics[width=0.32\textwidth]{eps_figs/eps_eps_diff_m_False_Positive.pdf}
\caption{
Three kinds of error rates with different bit-array lengths $m$. We fix the number of inserted elements $|A|=10^5$, the number of hash functions $k = 3$, and $\delta = 0.01$ in $(\epsilon, \delta)$-DP. 
In the figure, $\log$ denotes $\log_2$. 
{\bf Left:} Total error denotes the case when we randomly choose queries from the universe $[n]$; 
{\bf Middle:} False negative denotes the case when we randomly choose queries from the set $S$, which represents the set of elements inserted into the DP Bloom filter; 
{\bf Right:} False positive denotes the case when we randomly choose queries from the set $\ov{S} = [n] \backslash S$.  
As $m$ increases, the total error rate and false positive error rate decrease accordingly, while false negative error rate remains constant. 
As $\epsilon$ approaches $0$, the DP Bloom filter gets closer to random guessing. In this case, the false positive error rate converges to $\frac{1}{2^k}$, and the false negative error rate converges to $1 - \frac{1}{2^k}$. This is consistent with our result in Lemma~\ref{lem:random_guess}
Our \textsc{DPBloomFilter} achieves practical utility when $\epsilon$ is small(e.g. $\epsilon < 10$).
}
\label{fig:eps_diff_m}
\end{figure*}


\begin{figure*}[!ht]
\centering
\includegraphics[width=0.32\textwidth]{eps_figs/eps_eps_diff_na_Random.pdf}
\includegraphics[width=0.32\textwidth]{eps_figs/eps_eps_diff_na_False_Negative.pdf}
\includegraphics[width=0.32\textwidth]{eps_figs/eps_eps_diff_na_False_Positive.pdf}
\caption{
Three kinds of error rates with different numbers of inserted elements $|A|$. We fix the length of bit-array $m=2^{19}$, the number of hash functions $k = 3$, and $\delta = 0.01$ in $(\epsilon, \delta)$-DP.
As $|A|$ increases, the Total Error Rate and false positive error rate increase accordingly, while the false negative error rate remains constant. 
}
\label{fig:eps_diff_na}
\end{figure*}

\begin{figure*}[!ht]
\centering
\includegraphics[width=0.32\textwidth]{eps_figs/eps_eps_diff_k_Random.pdf}
\includegraphics[width=0.32\textwidth]{eps_figs/eps_eps_diff_k_False_Negative.pdf}
\includegraphics[width=0.32\textwidth]{eps_figs/eps_eps_diff_k_False_Positive.pdf}
\caption{
Three kinds of error rates with different numbers of hash function $k$.  
We fix the length of bit-array $m=2^{19}$, the number of inserted elements $|A| = 10^5$, and $\delta = 0.01$ in $(\epsilon, \delta)$-DP.
As $k$ increases, the Total Error Rate and false positive error rate decrease accordingly, while the false negative error rate increases accordingly. 
}
\label{fig:eps_diff_k}
\end{figure*}

In this section, we introduce the simulation experiments conducted on the DPBloomfilter.
In Section~\ref{sec:exp:setup}, we introduce the basic setup of our experiments and restate basic definitions of three kinds of error.
In Section~\ref{sec:exp:main_result}, we discuss the results of our experiments, which align with our theoretical analysis. 

\subsection{Experiments Setup and Basic Notations} \label{sec:exp:setup}


Recall that we have the following notations. 
Let $m$ denote the length of the bit array in the DPBloomfilter.
Let $|A|$ denote the number of elements inserted into the DPBloomfilter. 
Let $k$ denote the number of hash functions used in the DPBloomfilter.
Let $\epsilon, \delta$ denote the differential privacy parameters of the DPBloomfilter. 
Let $N$ denotes the $1 - \delta$ quantile of $W$ (see Definition~\ref{def:W}), and the close-form of the distribution of $W$ is shown in Lemma~\ref{lem:distribution_of_W}. 
Let $\epsilon_0 = \epsilon / N$. By Theorem~\ref{thm:query_privacy:informal}, we choose $\epsilon_0$ in this way can guarantee to $(\epsilon, \delta)$-DP in the whole algorithm. 
Unless specified, we adopt $m = 2^{19}, |A| = 10^5, k=8, n = 2^{63} \approx 10^{19}$ in the following experiments. 
We choose this $n$ because this $n$ is the biggest integer that can be represented on our server.

Recall that $[n]$ denotes the universe. 
Let $S$ denote the elements inserted into the DPBloomfilter. 
Let $\ov{S} = [n] \backslash S$ denote the elements not inserted into the DPBloomfilter. Let $\wt{z} \in \{ 0, 1 \}$ denote the answer output by DPBloomfilter. 

We report three kinds of error rates in our experiments. They are the following: 
(1) {\bf total error}, where we randomly choose queries from the universe $[n]$ and report the error rate of our DPBloomfilter;
(2) {\bf false positive error}, where we random choose queries from $\ov{S}$. When the DPBloomfilter outputs $\wt{z} = 1$, this will cause a false positive error; 
(3) {\bf false negative error}, where we random choose queries from $S$. When the DPBloomfilter outputs $\wt{z} = 0$, this will cause a false negative error. 

\subsection{Experiment Results} \label{sec:exp:main_result}

In this section, we conduct experiments based on the setting mentioned in the previous section. Specifically, we run simulation experiments on different $m$, $|A|$, and $k$ to demonstrate the utility of our algorithm under differential privacy guarantees. 

In Figure~\ref{fig:eps_diff_m}, we conduct experiments on different $m$, whereas $m$ increases, the total error rate and false positive error rate decrease accordingly, while the false negative error rate remains constant. 

In Figure~\ref{fig:eps_diff_na}, we also conduct experiments on different $|A|$, whereas $|A|$ increases, the total error rate and false positive error rate increase accordingly. At the same time, the false negative error rate remains constant.
This phenomenon is consistent with our theoretical analysis of the utility of DPBloomfilter (Theorem~\ref{thm:dpbloom_true_accuracy:informal}). Recall that we have $\alpha = \Pr[z=0]$, denoting the probability of an arbitrary query $q \notin A$. 
Since $|A|$ increases, $\alpha$ decreases, the utility guarantee in Theorem~\ref{thm:dpbloom_true_accuracy:informal}, which is consistent with higher error rate in our experiment results. 


In Figure~\ref{fig:eps_diff_k}, we conduct experiments on different $k$ as well, whereas $k$ increases, the total error rate, and false positive error rate decrease, while the false negative error rate increases accordingly. 

Note that in Figure~\ref{fig:eps_diff_m}, Figure~\ref{fig:eps_diff_na}, and Figure~\ref{fig:eps_diff_k}, as $\epsilon$ approaches $0$, the DPBloomfilter gets closer to random guessing. In this case, the false positive error rate converges to $\frac{1}{2^k}$, and the false negative error rate converges to $1 - \frac{1}{2^k}$. This is consistent with our result in Lemma~\ref{lem:random_guess}. 
Also, as $\epsilon$ increases, the three types of error rates in the Bloom filter with differential privacy (DP) approach the error rates observed when DP is not applied. This is consistent with the intuition that when $\epsilon$ increases, there is less privacy. Therefore, the performance approaches the performance of a Bloom filter without any privacy guarantees. 



\section{Conclusions}
\label{sec:conclusions}
    \section{Discussion}\label{sec:discussion}



\subsection{From Interactive Prompting to Interactive Multi-modal Prompting}
The rapid advancements of large pre-trained generative models including large language models and text-to-image generation models, have inspired many HCI researchers to develop interactive tools to support users in crafting appropriate prompts.
% Studies on this topic in last two years' HCI conferences are predominantly focused on helping users refine single-modality textual prompts.
Many previous studies are focused on helping users refine single-modality textual prompts.
However, for many real-world applications concerning data beyond text modality, such as multi-modal AI and embodied intelligence, information from other modalities is essential in constructing sophisticated multi-modal prompts that fully convey users' instruction.
This demand inspires some researchers to develop multimodal prompting interactions to facilitate generation tasks ranging from visual modality image generation~\cite{wang2024promptcharm, promptpaint} to textual modality story generation~\cite{chung2022tale}.
% Some previous studies contributed relevant findings on this topic. 
Specifically, for the image generation task, recent studies have contributed some relevant findings on multi-modal prompting.
For example, PromptCharm~\cite{wang2024promptcharm} discovers the importance of multimodal feedback in refining initial text-based prompting in diffusion models.
However, the multi-modal interactions in PromptCharm are mainly focused on the feedback empowered the inpainting function, instead of supporting initial multimodal sketch-prompt control. 

\begin{figure*}[t]
    \centering
    \includegraphics[width=0.9\textwidth]{src/img/novice_expert.pdf}
    \vspace{-2mm}
    \caption{The comparison between novice and expert participants in painting reveals that experts produce more accurate and fine-grained sketches, resulting in closer alignment with reference images in close-ended tasks. Conversely, in open-ended tasks, expert fine-grained strokes fail to generate precise results due to \tool's lack of control at the thin stroke level.}
    \Description{The comparison between novice and expert participants in painting reveals that experts produce more accurate and fine-grained sketches, resulting in closer alignment with reference images in close-ended tasks. Novice users create rougher sketches with less accuracy in shape. Conversely, in open-ended tasks, expert fine-grained strokes fail to generate precise results due to \tool's lack of control at the thin stroke level, while novice users' broader strokes yield results more aligned with their sketches.}
    \label{fig:novice_expert}
    % \vspace{-3mm}
\end{figure*}


% In particular, in the initial control input, users are unable to explicitly specify multi-modal generation intents.
In another example, PromptPaint~\cite{promptpaint} stresses the importance of paint-medium-like interactions and introduces Prompt stencil functions that allow users to perform fine-grained controls with localized image generation. 
However, insufficient spatial control (\eg, PromptPaint only allows for single-object prompt stencil at a time) and unstable models can still leave some users feeling the uncertainty of AI and a varying degree of ownership of the generated artwork~\cite{promptpaint}.
% As a result, the gap between intuitive multi-modal or paint-medium-like control and the current prompting interface still exists, which requires further research on multi-modal prompting interactions.
From this perspective, our work seeks to further enhance multi-object spatial-semantic prompting control by users' natural sketching.
However, there are still some challenges to be resolved, such as consistent multi-object generation in multiple rounds to increase stability and improved understanding of user sketches.   


% \new{
% From this perspective, our work is a step forward in this direction by allowing multi-object spatial-semantic prompting control by users' natural sketching, which considers the interplay between multiple sketch regions.
% % To further advance the multi-modal prompting experience, there are some aspects we identify to be important.
% % One of the important aspects is enhancing the consistency and stability of multiple rounds of generation to reduce the uncertainty and loss of control on users' part.
% % For this purpose, we need to develop techniques to incorporate consistent generation~\cite{tewel2024training} into multi-modal prompting framework.}
% % Another important aspect is improving generative models' understanding of the implicit user intents \new{implied by the paint-medium-like or sketch-based input (\eg, sketch of two people with their hands slightly overlapping indicates holding hand without needing explicit prompt).
% % This can facilitate more natural control and alleviate users' effort in tuning the textual prompt.
% % In addition, it can increase users' sense of ownership as the generated results can be more aligned with their sketching intents.
% }
% For example, when users draw sketches of two people with their hands slightly overlapping, current region-based models cannot automatically infer users' implicit intention that the two people are holding hands.
% Instead, they still require users to explicitly specify in the prompt such relationship.
% \tool addresses this through sketch-aware prompt recommendation to fill in the necessary semantic information, alleviating users' workload.
% However, some users want the generative AI in the future to be able to directly infer this natural implicit intentions from the sketches without additional prompting since prompt recommendation can still be unstable sometimes.


% \new{
% Besides visual generation, 
% }
% For example, one of the important aspect is referring~\cite{he2024multi}, linking specific text semantics with specific spatial object, which is partly what we do in our sketch-aware prompt recommendation.
% Analogously, in natural communication between humans, text or audio alone often cannot suffice in expressing the speakers' intentions, and speakers often need to refer to an existing spatial object or draw out an illustration of her ideas for better explanation.
% Philosophically, we HCI researchers are mostly concerned about the human-end experience in human-AI communications.
% However, studies on prompting is unique in that we should not just care about the human-end interaction, but also make sure that AI can really get what the human means and produce intention-aligned output.
% Such consideration can drastically impact the design of prompting interactions in human-AI collaboration applications.
% On this note, although studies on multi-modal interactions is a well-established topic in HCI community, it remains a challenging problem what kind of multi-modal information is really effective in helping humans convey their ideas to current and next generation large AI models.




\subsection{Novice Performance vs. Expert Performance}\label{sec:nVe}
In this section we discuss the performance difference between novice and expert regarding experience in painting and prompting.
First, regarding painting skills, some participants with experience (4/12) preferred to draw accurate and fine-grained shapes at the beginning. 
All novice users (5/12) draw rough and less accurate shapes, while some participants with basic painting skills (3/12) also favored sketching rough areas of objects, as exemplified in Figure~\ref{fig:novice_expert}.
The experienced participants using fine-grained strokes (4/12, none of whom were experienced in prompting) achieved higher IoU scores (0.557) in the close-ended task (0.535) when using \tool. 
This is because their sketches were closer in shape and location to the reference, making the single object decomposition result more accurate.
Also, experienced participants are better at arranging spatial location and size of objects than novice participants.
However, some experienced participants (3/12) have mentioned that the fine-grained stroke sometimes makes them frustrated.
As P1's comment for his result in open-ended task: "\emph{It seems it cannot understand thin strokes; even if the shape is accurate, it can only generate content roughly around the area, especially when there is overlapping.}" 
This suggests that while \tool\ provides rough control to produce reasonably fine results from less accurate sketches for novice users, it may disappoint experienced users seeking more precise control through finer strokes. 
As shown in the last column in Figure~\ref{fig:novice_expert}, the dragon hovering in the sky was wrongly turned into a standing large dragon by \tool.

Second, regarding prompting skills, 3 out of 12 participants had one or more years of experience in T2I prompting. These participants used more modifiers than others during both T2I and R2I tasks.
Their performance in the T2I (0.335) and R2I (0.469) tasks showed higher scores than the average T2I (0.314) and R2I (0.418), but there was no performance improvement with \tool\ between their results (0.508) and the overall average score (0.528). 
This indicates that \tool\ can assist novice users in prompting, enabling them to produce satisfactory images similar to those created by users with prompting expertise.



\subsection{Applicability of \tool}
The feedback from user study highlighted several potential applications for our system. 
Three participants (P2, P6, P8) mentioned its possible use in commercial advertising design, emphasizing the importance of controllability for such work. 
They noted that the system's flexibility allows designers to quickly experiment with different settings.
Some participants (N = 3) also mentioned its potential for digital asset creation, particularly for game asset design. 
P7, a game mod developer, found the system highly useful for mod development. 
He explained: "\emph{Mods often require a series of images with a consistent theme and specific spatial requirements. 
For example, in a sacrifice scene, how the objects are arranged is closely tied to the mod's background. It would be difficult for a developer without professional skills, but with this system, it is possible to quickly construct such images}."
A few participants expressed similar thoughts regarding its use in scene construction, such as in film production. 
An interesting suggestion came from participant P4, who proposed its application in crime scene description. 
She pointed out that witnesses are often not skilled artists, and typically describe crime scenes verbally while someone else illustrates their account. 
With this system, witnesses could more easily express what they saw themselves, potentially producing depictions closer to the real events. "\emph{Details like object locations and distances from buildings can be easily conveyed using the system}," she added.

% \subsection{Model Understanding of Users' Implicit Intents}
% In region-sketch-based control of generative models, a significant gap between interaction design and actual implementation is the model's failure in understanding users' naturally expressed intentions.
% For example, when users draw sketches of two people with their hands slightly overlapping, current region-based models cannot automatically infer users' implicit intention that the two people are holding hands.
% Instead, they still require users to explicitly specify in the prompt such relationship.
% \tool addresses this through sketch-aware prompt recommendation to fill in the necessary semantic information, alleviating users' workload.
% However, some users want the generative AI in the future to be able to directly infer this natural implicit intentions from the sketches without additional prompting since prompt recommendation can still be unstable sometimes.
% This problem reflects a more general dilemma, which ubiquitously exists in all forms of conditioned control for generative models such as canny or scribble control.
% This is because all the control models are trained on pairs of explicit control signal and target image, which is lacking further interpretation or customization of the user intentions behind the seemingly straightforward input.
% For another example, the generative models cannot understand what abstraction level the user has in mind for her personal scribbles.
% Such problems leave more challenges to be addressed by future human-AI co-creation research.
% One possible direction is fine-tuning the conditioned models on individual user's conditioned control data to provide more customized interpretation. 

% \subsection{Balance between recommendation and autonomy}
% AIGC tools are a typical example of 
\subsection{Progressive Sketching}
Currently \tool is mainly aimed at novice users who are only capable of creating very rough sketches by themselves.
However, more accomplished painters or even professional artists typically have a coarse-to-fine creative process. 
Such a process is most evident in painting styles like traditional oil painting or digital impasto painting, where artists first quickly lay down large color patches to outline the most primitive proportion and structure of visual elements.
After that, the artists will progressively add layers of finer color strokes to the canvas to gradually refine the painting to an exquisite piece of artwork.
One participant in our user study (P1) , as a professional painter, has mentioned a similar point "\emph{
I think it is useful for laying out the big picture, give some inspirations for the initial drawing stage}."
Therefore, rough sketch also plays a part in the professional artists' creation process, yet it is more challenging to integrate AI into this more complex coarse-to-fine procedure.
Particularly, artists would like to preserve some of their finer strokes in later progression, not just the shape of the initial sketch.
In addition, instead of requiring the tool to generate a finished piece of artwork, some artists may prefer a model that can generate another more accurate sketch based on the initial one, and leave the final coloring and refining to the artists themselves.
To accommodate these diverse progressive sketching requirements, a more advanced sketch-based AI-assisted creation tool should be developed that can seamlessly enable artist intervention at any stage of the sketch and maximally preserve their creative intents to the finest level. 

\subsection{Ethical Issues}
Intellectual property and unethical misuse are two potential ethical concerns of AI-assisted creative tools, particularly those targeting novice users.
In terms of intellectual property, \tool hands over to novice users more control, giving them a higher sense of ownership of the creation.
However, the question still remains: how much contribution from the user's part constitutes full authorship of the artwork?
As \tool still relies on backbone generative models which may be trained on uncopyrighted data largely responsible for turning the sketch into finished artwork, we should design some mechanisms to circumvent this risk.
For example, we can allow artists to upload backbone models trained on their own artworks to integrate with our sketch control.
Regarding unethical misuse, \tool makes fine-grained spatial control more accessible to novice users, who may maliciously generate inappropriate content such as more realistic deepfake with specific postures they want or other explicit content.
To address this issue, we plan to incorporate a more sophisticated filtering mechanism that can detect and screen unethical content with more complex spatial-semantic conditions. 
% In the future, we plan to enable artists to upload their own style model

% \subsection{From interactive prompting to interactive spatial prompting}


\subsection{Limitations and Future work}

    \textbf{User Study Design}. Our open-ended task assesses the usability of \tool's system features in general use cases. To further examine aspects such as creativity and controllability across different methods, the open-ended task could be improved by incorporating baselines to provide more insightful comparative analysis. 
    Besides, in close-ended tasks, while the fixing order of tool usage prevents prior knowledge leakage, it might introduce learning effects. In our study, we include practice sessions for the three systems before the formal task to mitigate these effects. In the future, utilizing parallel tests (\textit{e.g.} different content with the same difficulty) or adding a control group could further reduce the learning effects.

    \textbf{Failure Cases}. There are certain failure cases with \tool that can limit its usability. 
    Firstly, when there are three or more objects with similar semantics, objects may still be missing despite prompt recommendations. 
    Secondly, if an object's stroke is thin, \tool may incorrectly interpret it as a full area, as demonstrated in the expert results of the open-ended task in Figure~\ref{fig:novice_expert}. 
    Finally, sometimes inclusion relationships (\textit{e.g.} inside) between objects cannot be generated correctly, partially due to biases in the base model that lack training samples with such relationship. 

    \textbf{More support for single object adjustment}.
    Participants (N=4) suggested that additional control features should be introduced, beyond just adjusting size and location. They noted that when objects overlap, they cannot freely control which object appears on top or which should be covered, and overlapping areas are currently not allowed.
    They proposed adding features such as layer control and depth control within the single-object mask manipulation. Currently, the system assigns layers based on color order, but future versions should allow users to adjust the layer of each object freely, while considering weighted prompts for overlapping areas.

    \textbf{More customized generation ability}.
    Our current system is built around a single model $ColorfulXL-Lightning$, which limits its ability to fully support the diverse creative needs of users. Feedback from participants has indicated a strong desire for more flexibility in style and personalization, such as integrating fine-tuned models that cater to specific artistic styles or individual preferences. 
    This limitation restricts the ability to adapt to varied creative intents across different users and contexts.
    In future iterations, we plan to address this by embedding a model selection feature, allowing users to choose from a variety of pre-trained or custom fine-tuned models that better align with their stylistic preferences. 
    
    \textbf{Integrate other model functions}.
    Our current system is compatible with many existing tools, such as Promptist~\cite{hao2024optimizing} and Magic Prompt, allowing users to iteratively generate prompts for single objects. However, the integration of these functions is somewhat limited in scope, and users may benefit from a broader range of interactive options, especially for more complex generation tasks. Additionally, for multimodal large models, users can currently explore using affordable or open-source models like Qwen2-VL~\cite{qwen} and InternVL2-Llama3~\cite{llama}, which have demonstrated solid inference performance in our tests. While GPT-4o remains a leading choice, alternative models also offer competitive results.
    Moving forward, we aim to integrate more multimodal large models into the system, giving users the flexibility to choose the models that best fit their needs. 
    


\section{Conclusion}\label{sec:conclusion}
In this paper, we present \tool, an interactive system designed to help novice users create high-quality, fine-grained images that align with their intentions based on rough sketches. 
The system first refines the user's initial prompt into a complete and coherent one that matches the rough sketch, ensuring the generated results are both stable, coherent and high quality.
To further support users in achieving fine-grained alignment between the generated image and their creative intent without requiring professional skills, we introduce a decompose-and-recompose strategy. 
This allows users to select desired, refined object shapes for individual decomposed objects and then recombine them, providing flexible mask manipulation for precise spatial control.
The framework operates through a coarse-to-fine process, enabling iterative and fine-grained control that is not possible with traditional end-to-end generation methods. 
Our user study demonstrates that \tool offers novice users enhanced flexibility in control and fine-grained alignment between their intentions and the generated images.

%-------------------------------------------------------------------------




%%%%%%%%% REFERENCES
{\small
\bibliographystyle{ieee_fullname}
\bibliography{main}
}

\end{document}
