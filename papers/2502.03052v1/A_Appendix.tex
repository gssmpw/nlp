\section{Algorithm}
\label{appendix:A}

The three-stage PiF algorithm is summarized in Algorithm~\ref{alg:1}.

\begin{algorithm}[h]
\caption{\emph{Perceived-importance Flatten Method}}
\label{alg:1}
{\bfseries Input}: Source Model $f_{\text{S}}$, Original Sentence $\mathbb{S} = [x_1, \ldots, x_\text{i}]$, Iteration ${T}$, temperature $\tau$, Replaced Candidate Top-$\mathcal{N}$, Replacement Candidate Top-$\mathcal{M}$, Comparison Token Top-$\mathcal{K}$, Sentence Similarity Threshold $\Theta$.\\
{\bfseries Output}: Jailbreaking Sentence $\mathbb{S}_\text{jail} = [x_1, \ldots, x_\text{i} ]$.\\
1: Initialization $\mathbb{S}_\text{jail} \leftarrow \mathbb{S}$;\\
\For{$\text{iter} \in {T}$}{ 
\textbf{$\triangleright$ Generate on the source MLM / CLM} $f_{\text{S}}$ \textbf{with temperature $\tau$}\\
{\textbf{\# Stage \uppercase\expandafter{\romannumeral1}}}\\
2: Compute the \emph{perceived-importance} for each token in sentence, $I_\text{i}$  $\forall$  $x_\text{i}$ $\in$ $\mathbb{S}_\text{jail}$, using the evaluation template;\\
3: Probabilistically sample the index ${n}$ as the final token to be replaced from the top-$\mathcal{N}$ tokens based on their inverse \emph{perceived-importance} $- I_\text{i}$;\\
{\textbf{\# Stage \uppercase\expandafter{\romannumeral2}}}\\
4: Predict the top-$\mathcal{M}$ tokens ${M}$ at the position of the token to be replaced, ${n}$, within the sentence $\mathbb{S}_{\text{jail}[1:\text{n}-1]}$\texttt{[MASK]}$\mathbb{S}_{\text{jail}[\text{n}+1:\text{i}]}$;\\
5: Construct replacement sentences $\mathbb{L}[m]$ =  $\mathbb{S}_{\text{jail}[1:\text{n}-1]}$[$m$]$\mathbb{S}_{\text{jail}[\text{n}+1:\text{i}]}$ for $m \in M$;\\
6: Select the top-$\mathcal{K}$ tokens ${k}$ in the original outputs $\mathcal{O}(\mathbb{S}_\text{jail})$ with the evaluation template;\\
7: Select the index ${m}$ for the final replacement token, which exhibits the maximum changes in the model's intent perception, $\|\mathcal{O}(\mathbb{S}_\text{jail})[k] - \mathcal{O}(\mathbb{L}[m])[k]\|_{2}^{2}$;\\
{\textbf{\# Stage \uppercase\expandafter{\romannumeral3}}}\\
8: Calculate sentence-level semantic similarity $\theta$ between $\mathbb{S}_\text{jail}$ and $\mathbb{L}[m]$;\\
9: Update $\mathbb{S}_\text{jail} \leftarrow \mathbb{L}[m]$, \textbf{if} $\theta$ $\geq$ $\Theta$;\\
\textbf{$\triangleright$ Attack the target CLM} $f_{\text{T}}$\\
10: Input jailbreaking sentence $\mathbb{S}_\text{jail}$ into $f_{\text{T}}$.
}
\end{algorithm}
