\section{Related Works
}
\label{sec:headings}
Stock market prediction algorithms are nothing new but we hope to find new patterns and innovate on the current state of algorithms with our own better and more accurate model. Our approach relies heavily on the hypothesis that great correlations exist between media outlets and sentiment. [1] used a model called snscrape and found that there are correlations worth measuring in relation to the effect they have on stocks and their closing prices for the day. [2] developed a model called FinReport which used news analysis and company financials along with LLMs to help generate reports designed to help deliver a reportable stock forecast. A sentiment lexicon we thought was a novel idea by [3] which would assign words a negative or positive meaning and additional weights based on their importance. [4] brings an excellent narrative and a solution when dealing with the stock market; the stock market is a chaotic data space and therefore they believe that temporal analysis or small time range analysis is thus more predictable. Our model relies heavily on predicting wall street's sentiment as earnings reports are easily influenced by the feelings on wall street. [5] explores how public sentiment, as expressed through Twitter, can predict stock market fluctuations.It highlights the value of combining sentiment data with financial metrics. [6] uses sentiment analysis and machine learning techniques to predict stock market movements by analyzing Twitter data. This study highlights a strong correlation between public sentiment on Twitter and stock price fluctuations, supporting our approach of using media sentiment to assess stock volatility [7] combines text from tweets and stock price data to predict stock movements using deep generative models. It's particularly useful for us in our aim to integrate media exposure with stock performance. [8] introduces a model that integrates financial data, social media sentiment, and correlations between stocks. It uses a graph neural network to predict stock movements, particularly focusing on the effects of social media discussions on platforms like Twitter and Reddit.[9] leverages advanced NLP techniques like BERTopic to analyze sentiment from stock market-related comments. Neural networks create a black box style environment and can make how the data is manipulated unknown and if the functions the data goes through are unknown then that can make target predictions less understood. [10] uses a neural network design and the graph style network of nodes offers great target accuracy when dealing with stocks but great uncertainty on how it arrives at that conclusion in some cases if not all. Our model focuses on data and finding hidden patterns through feature aggregation, models like [11] emphasize the importance LLMs have in querying data, the LLM is question here is Meta’s LLAMA which is able to decipher and interpolate underlying details from large text which can be used as features greatly enhancing our data points or creating aggregation for new features. ONe risk of using an LLM however is the black box nature of not fully understanding how it derives its results. [12]focuses on using neural networks to classify market states based on labels such as moving averages and other market features.[13]proposes a framework for using Twitter sentiment to predict stock movements. It employs sentiment analysis tools and correlates them with stock prices, making it highly relevant for your goal of integrating media sentiment and financial data.[14]explores how social media discussions can predict stock trends and volatility using NLP techniques. These NLP techniques are particularly relevant to sentiment analysis, supporting the integration of media sentiment data for improved stock volatility predictions.[15] Use a sentiment analysis dictionary to create a 70.59\% accuracy model to predict stock trends after news sentiment.[16] Using sentiment analysis and supervised machine learning principles to the tweets to determine if the those tweets are correlating to the rise and fall of stock prices. [17] Using semantic role labeling Pooling (SRLP) to create compact representations of news paragraphs and further incorporating stock factors to make final predictions.[19] Using Long Short-Term Memory and fully connected layers to train a Machine Learning model with historical stock values, indicators, and Twitter attributes for information. Then sentiment analysis is conducted with Valence Aware Dictionary and sentiment Reasoner showing adding more Twitter attributes raised the Mean Squared Error by 3\%.[20] A multilingual language model trained on Twitter data. Provides an evolution framework for sentiment analysis across multiple different languages. Demonstrates competitive performance with multilingual and cross-lingual sentiment.