\section{Related Work}
We first position our work w.r.t. the literature on insight generation. We then discuss Text-to-SQL and how %it's
it is relevant to our solution.

\vspace{1ex}\noindent
\textbf{Automatic Insights.}
%The domain of automatic insights generation is an open field of research, with many 
Several models have been proposed for insight generation from tabular data, such as InsightPilot~\citep{ma2023demonstration}, QuickInsights~\citep{ding2019quickinsights}, OpenAI Data Analysis~\citep{openai2024data}, or Langchain's Pandas Agent~\citep{langchain2024}.
%These 

These works propose different definitions of insights. Some approaches consider insights derived from predefined templates~\citep{DBLP:journals/corr/abs-2008-13060} populated with aggregate measures across subsets of tables.
The generation of these %type of 
insights is done either with a mining framework based on %calculated 
metrics~\citep{ding2019quickinsights} or using an LLM to guide the process~\citep{ma2023demonstration}. Although this approach offers several strengths, including providing visual data and ensuring factual correctness, it also presents certain limitations.
Other approaches~\citep{openai2024data, langchain2024} directly leverage LLMs to generate code to %generate
obtain insights. These methods are often specialized in single-step data analysis tasks (for example, calculating a correlation or doing some regression task) based on very concrete user instructions.
All % these 
previous approaches are based on one or more of the following assumptions and limitations:
\begin{itemize}
    \item Some approaches have only been tested on simple table structures and have not been evaluated on multi-table databases~\citep{ma2023demonstration, ding2019quickinsights, openai2024data, langchain2024}.
    \item Some approaches %use 
    require a user-defined goal (for example, ``Show me interesting trends in mathematics scores for students'') to guide the insight generation process~\citep{ma2023demonstration, ding2019quickinsights, openai2024data, langchain2024}. %(works like~\citep{sahu2024insightbenchevaluatingbusinessanalytics}    show that using generic goals instead of user defined ones greatly decreases the quality of generated insights, showing how challenging the task is).
    %\item Some approaches rely on dataframes for data storage and processing. These structures face limitations in terms of speed, in-memory storage, and scalability, becoming inefficient in multiple scenarios compared to relational databases~\citep{openai2024data, langchain2024, ding2019quickinsights}.
    %\item Some approaches focus on individual data analysis steps, lacking end-to-end generation of insights~\citep{openai2024data, langchain2024}.
    \item Some approaches are template-based and require clean data that has been filtered and descriptive naming in columns~\citep{ding2019quickinsights}. For example, the insight ``[Subject]=[Math] has an increasing trend over [Time]'' becomes meaningless if the placeholders lack descriptive names or if they are irrelevant to the context, e.g., finding trends on IDs or telephone numbers.
\end{itemize}

\vspace{1ex}\noindent
\textbf{Text-to-SQL.}
Text-to-SQL is one of the most popular approaches for enabling natural language interfaces (NL) to databases. % and still remains an open research problem. It's 
It  %main objective 
consists of converting %natural language 
NL questions into valid SQL queries, % for %relationaldatabases, 
where the result of the query is a set of tuples that answer the question.

The advent of %large language models (
LLMs %, coupled with their increasing affordability and improved performance, 
has facilitated the development of %more 
complex architectures, such as LLM chains and agentic flows, which have significantly enhanced the efficacy of Text-to-SQL systems~\citep{tai-etal-2023-exploring, li2024petsql, gao2023texttosql, pourreza2023dinsql}. Our %research 
work employs such architectures to facilitate the automatic generation of insights from databases via automatically generated SQL queries.
While there is evidence that Text-to-SQL systems struggle with proprietary, large schemas~\citep{abs-2409-02038,papicchio2023qatch}, there are promising solutions to tackle this problem with prompt compression~\citep{li2024snapkv,tacl_a_00716}.