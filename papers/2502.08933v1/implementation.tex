
\begin{figure*}[t!]
	\centering
\includegraphics[width=1\textwidth]{images/SocialRS_RLFramework_ImplementationNew.pdf}
	\caption{{\textbf{\autolike{} Implementation.} We implement \autolike{} for the TikTok Android app. It works in the following ways. Once the user gives the inputs, \autolike{} (1) opens TikTok to its ``For You'' page and extracts the TikTok URL of the current recommended content; (2) sends it to a backend server and the server visits the TikTok URL in a web browser; (3) extracts the text description and transcribes the video's audio into text, and concatenates both into one string; (5) classifies the text into $\langle$Topic, Sentiment$\rangle$; and (6) calculates the reward, and follows a policy to select and take an action (\eg{} liking the current TikTok); and (7) scrolls to the next TikTok. \autolike{} repeats 1--7 until some time horizon $T$. We omit the output of pathways for brevity, see Fig.~\ref{fig:autolike-framework}.}}
	\label{fig:autolike-framework-impl}
    \vspace{-5pt}
\end{figure*}

\section{\autolike{} Implementation for TikTok}
\label{sec:autolike-impl}


In this section, we detail the implementation of \autolike{} for TikTok on mobile Android devices, and shown in Fig.~\ref{fig:autolike-framework-impl}.
\autolike{} for TikTok runs in a live setting to capture how TikTok's \rs{} changes through interactions and over time. The implementation and its challenges may be unique to the chosen social platform, but the necessary components of \autolike{} stay the same, as described in Sec.~\ref{sec:autolike-framework}. %


\subsection{Agent: TikTok}
\label{sec:autolike-agent}

\descr{Actions and Action Space.}
Recall that the RL agent interacts with the \rs{} through the available actions. Our action space consists of skipping, watching, liking, bookmarking, and reposting. 
We utilize UIAutomator2~\cite{uiautomator2}, a Python wrapper of the standard Android UI testing framework~\cite{UIAutomator}, to programmatically automate the agent's interaction with TikTok. %
For skipping, we use the functionality for swiping up. For watching, we first extract the duration of the video using a Python library called PykTok~\cite{pyktok} (during Fig.~\ref{fig:autolike-framework-impl} step 3) and then use a Python sleep timer for that amount. The remaining actions utilize a selector-based methodology, \ie{} select a UI element using a string-based selector and take an action on it. To achieve this, we first go to the \fyp{} on our mobile device, then run ``dump\_hierarchy()'', which returns an XML representation of the \fyp{}. We then visually inspect the XML to find the like button using keywords that we see, such as ``Like''. Once identified, we create the selector for the button. In this case, the like button is ``description=Like'', which selects all buttons with the description ``Like'' and then clicks on it. %
We repeat this process for bookmarking and reposting. By the end, we have a selector for each action. Note that we must update this if the \fyp{} changes. However, we turned off auto-update for TikTok to prevent this from happening. 
We note that this selector methodology can be applied to other social media apps on Android.
Next, to determine whether an action is valid, we call ``exists()'' on the selector. This prevents us from choosing an invalid action during a particular time step.



\subsection{Environment: TikTok}
\label{sec:autolike-env}

The environment encapsulates TikTok, including the \rs{}/\fyp{}, its states, the user profile, and the reward calculation. In our case, we use a rooted Pixel 3 Android device for TikTok. 

\subsubsection{User Profiles}
\label{sec:user-profiles-impl}
Creating new users for TikTok is challenging, as they require unique mobile numbers or emails. In addition, they require an incubation period using TikTok for several days (2--4 days) for TikTok to enable certain features, such as liking and following. Currently, we manually conduct this process. We stop when TikTok displays ads, which indicates that TikTok treats the user as real. To reduce \autolike{'s} impact on TikTok's ad ecosystem, we intentionally skip all ads.
Once we have finished the creation and incubation process, we make sure the \fyp{} is fresh (\ie{} TikTok treats the user as a new user). First, we leverage TikTok's feature of refreshing the \fyp{} through its settings~\cite{tiktok-refresh-fyp}. Second, we refresh three identifiers: the advertising ID, the Android ID, and the device ID. The former is reset using Android's settings. The latter two are reset by modifying the device's ``settings\_secure.xml'' and ``settings\_ssaid.xml'' files. 




\subsubsection{Determine the State: TikTok}
\label{sec:states-impl}
A state is defined as $\langle$Topic, Sentiment$\rangle$ ($\in [0, 0.1, 0.2, ..., 1]$) from Sec.~\ref{sec:env-framework} and represents the current recommended content. %

Specifically, we devise and implement the steps described in Fig.~\ref{fig:autolike-framework-impl} (steps 1--5). 
\begin{itemize} 
    \item \textit{Steps 1-2: Identifying the TikTok URL.} We automate the interaction with TikTok using UIAutomator2~\cite{uiautomator2}. We use TikTok's Copy URL feature designed to share the content with others. This places the URL of the TikTok on the Android clipboard. However, Android prevents clipboard readings for security purposes. To bypass this, we automate the following: (1) click Home to go to the Android home screen; (2) click Search to focus on its default search bar; (3) paste the URL. This process now allows us to read from the clipboard. Note that there are other ways to retrieve this, such as monitoring and inspecting the network traffic of the TikTok app and extracting the TikTok URL~\cite{KaplanTikTok}.
    \item \textit{Step 3: Extracting the TikTok Data.} We rely on PykTok, a Python library~\cite{pyktok}. We give it the URL of the current TikTok to extract its metadata and video (as an MP4) using a web browser. To get the text from the video's audio, we use OpenAI's Whisper ``large-v2'' model~\cite{openaiwhisper}, a state-of-the-art general-purpose speech recognition model that can do audio-to-text given our TikTok MP4. We concatenate the two texts that we have: ``description $\ +\ $ text from audio''. In addition, we also clean the text to remove common hashtags, such as ``\verb|#|foryoupage'', and exact matches to our topic of interest, such as ``\verb|#|mentalhealth''. This both reduces the noise for classification later and prevents reliance on the exact matches of our topic of interest.
    \item \textit{Step 4: Classifying the TikTok.} We utilize Facebook's ``bart-large-mnli'' model~\cite{bart-large-mnli}, which employs natural language inference (NLI) to train a model for zero-shot text classification. Thus, we feed our text into the model for the topic and negative sentiment classification and receive confidence scores $\in [0, 1]$. We evaluate these models for our use case in Sec.~\ref{sec:autolike-prelim-results}. 
    \item \textit{Step 5: State.} We transform the confidence scores into a state, $\langle$Topic, Sentiment$\rangle$. For instance, a TikTok video may receive $\langle$0.06, 0.07$\rangle$, which will transform to the nearest valid state of $\langle$0.1, 0.1$\rangle$. %
\end{itemize}


