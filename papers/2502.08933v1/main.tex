\documentclass[manuscript,screen]{acmart}
\AtBeginDocument{%
  \providecommand\BibTeX{{%
    Bib\TeX}}}

\setcopyright{acmlicensed}
\copyrightyear{2025}
\acmYear{2025}
\acmDOI{XXXXXXX.XXXXXXX}
\acmConference[Conference 'XX]{}{June 03--05,
  2018}{Woodstock, NY}
\acmISBN{978-1-4503-XXXX-X/2018/06}






\usepackage[tight,footnotesize]{subfigure}
\usepackage{tikz}
\usepackage{color, pifont, comment}
\usepackage{xspace}
\usepackage{algorithm}
\usepackage{algpseudocode}
\usepackage{enumerate}
\usepackage{enumitem}

\usepackage[export]{adjustbox}
\usepackage{stmaryrd}
\usepackage{colortbl}
\usepackage{xurl}
\usepackage[compact]{titlesec}
\usepackage[subtle,tracking=normal]{savetrees}
\usepackage{fontawesome5}
\usepackage{eqparbox}

\usepackage{pifont, comment}
\usepackage{xspace}
\usepackage{upquote}
\usepackage{listings}
\usepackage{xcolor}


\usepackage{cases}

\usepackage{caption}
\usepackage{multirow}
\usepackage{tabularx}

\newcommand{\ie}{{\it i.e.,}\xspace}
\newcommand{\eg}{{\it e.g.,}\xspace}
\newcommand{\etc}{{\it etc.}\xspace}
\newcommand{\vs}{{\it vs.}\xspace}
\newcommand{\etal}{{\it et al.}\xspace}
\newcommand{\descr}[1]{\vspace{.05in} \noindent\textbf{#1}}
\newcommand{\tilda}{\raise.17ex\hbox{$\scriptstyle\mathtt{\sim}$}}
\newcommand{\tool}{{AutoLike}}
\newcommand{\autolike}{{AutoLike}}
\newcommand{\rs}{{RS}}
\newcommand{\recsys}{{recommendation system}}
\newcommand{\fyp}{{FYP}}
\newcommand{\si}{{$SI$}} %
\newcommand{\autolikedatasetone}{AUTOLIKE-D1} 
\newcommand{\autolikedatasettwo}{AUTOLIKE-D2}
\newcommand{\TODO}[1]{\textcolor{red}{#1}}

\begin{document}

\title{\autolike{}: Auditing Social Media Recommendations through User  Interactions}

\author{Hieu Le}
\authornote{This work was primarily completed  during the author's graduate studies at the University of California, Irvine.}
\email{levanhieu@gmail.com}
\affiliation{%
  \institution{University of California, Irvine}
  \city{Irvine}
  \state{California}
  \country{USA}
}

\author{Salma Elmalaki}
\email{salma.elmalaki@uci.edu}
\affiliation{%
  \institution{University of California, Irvine}
  \city{Irvine}
  \state{California}
  \country{USA}
  }


\author{Zubair Shafiq}
\email{zubair@ucdavis.edu}
\affiliation{%
  \institution{University of California, Davis}
  \city{Davis}
  \state{California}
  \country{USA}
}

\author{Athina Markopoulou}
\email{athina@uci.edu}
\affiliation{%
  \institution{University of California, Irvine}
  \city{Irvine}
  \state{California}
  \country{USA}
 }


\renewcommand{\shortauthors}{Le et al.}

\begin{abstract}
Modern social media platforms, such as TikTok, Facebook, and YouTube, rely on recommendation systems to personalize content for users based on user interactions with endless streams of content, such as ``For You'' pages. However, these complex algorithms can inadvertently deliver problematic content related to self-harm, mental health, and eating disorders. %
We introduce AutoLike, a framework to audit recommendation systems in social media platforms for topics of interest and their sentiments. To automate the process, we formulate the problem as a reinforcement learning problem. AutoLike drives the recommendation system to serve a particular type of content through interactions (e.g., liking).
We apply the AutoLike framework to the TikTok platform as a case study. We evaluate how well AutoLike identifies TikTok content automatically across nine topics of interest; and conduct eight experiments to demonstrate how well it drives TikTok's recommendation system towards particular topics and sentiments. AutoLike has the potential to assist regulators in auditing recommendation systems for problematic content. \textbf{(Warning: This paper contains qualitative examples that may be viewed as offensive or harmful.)}
\end{abstract}

\begin{CCSXML}
<ccs2012>
 <concept>
  <concept_id>00000000.0000000.0000000</concept_id>
  <concept_desc>Do Not Use This Code, Generate the Correct Terms for Your Paper</concept_desc>
  <concept_significance>500</concept_significance>
 </concept>
 <concept>
  <concept_id>00000000.00000000.00000000</concept_id>
  <concept_desc>Do Not Use This Code, Generate the Correct Terms for Your Paper</concept_desc>
  <concept_significance>300</concept_significance>
 </concept>
 <concept>
  <concept_id>00000000.00000000.00000000</concept_id>
  <concept_desc>Do Not Use This Code, Generate the Correct Terms for Your Paper</concept_desc>
  <concept_significance>100</concept_significance>
 </concept>
 <concept>
  <concept_id>00000000.00000000.00000000</concept_id>
  <concept_desc>Do Not Use This Code, Generate the Correct Terms for Your Paper</concept_desc>
  <concept_significance>100</concept_significance>
 </concept>
</ccs2012>
\end{CCSXML}


\keywords{Social Media, Recommendations, Algorithmic Auditing}


\maketitle

\documentclass[../main.tex]{subfiles}
\graphicspath{{../images/}}
\makeatletter
\def\input@path{{../images/}}
\makeatother
\begin{document}
\section{Introduction}
\begin{figure}
\centering
\begin{tikzpicture}
\node[inner sep=0pt] (ws) at (0, 0) {
\includegraphics[height=.4\textwidth, trim={10cm 0 10cm 0},clip]{world_space.png}};
\node[inner sep=0pt] (cs) at (6,0) {\includegraphics[height=.4\textwidth, trim={10cm 1cm 10cm 4cm},clip]{conf_space.png}};
\end{tikzpicture}
\vspace{-5pt}
\label{fig:pbrm_intro}
\caption{\textbf{Left}: Shows world space obstacles as grey spheres. Robots start and goal configuration is colored red and green, respectively. Configurations along the computed path are colored transparent blue. \textbf{Right:} Mapped world space scenario to configuration space. Obstacle region is the grey mesh. Red spheres are collision-free regions computed by the neural SCDF. The optimized shortest path in the convex corridor is the blue curve.}
\vspace{-25pt}
\end{figure}
Motion planning is the problem of finding a collision-free trajectory that connects a given start and goal configuration. The planning takes place in the configuration space of the robot. For single body robots, like mobile robots or drones, the configuration space and the world space are usually the same. This simplifies the planning, since explicit obstacle representations are available which enables geometrical tools like separating hyperplanes, smallest distance to obstacles etc., to be used when designing motion planning algorithms. For multi-body robots like manipulators, the situation is completely different. The world space obstacles are usually mapped to non-convex regions, and to make the problem even harder, the mapping is usually not known. Forming explicit representations of the obstacle region in the configuration space is usually too expensive or intractable. Despite all of this, sampling based planners are used with great success, which mainly is due to their use of implicit representations of the obstacle region. The basic idea is to construct a graph in the configuration space that covers and connects the collision-free region. From this graph, a path can be extracted that connects a given start and goal configuration. The approach is computationally expensive, since the graph is constructed with the smallest geometrical building block available, points, which represents a collision-check. Furthermore, the extracted paths from the graph are non-smooth and jagged due to the stochastic nature of the approach. This adds an additional post-processing step to the process, where the paths are shortcutted and smoothened, before the path can be used for tracking. Clearly a lot of time is invested to form this graph and produce smooth paths. Thus, if the obstacles start to move, then all of this work is done in no use, since all points that make up this graph need to be re-verified, which is simply too time consuming to be done in real time.
\\\\
In this work, we want to address the existing drawbacks of the sampling based planners. Our main contribution is an improved motion planner where each vertex in the graph covers a collision-free region in the form of a sphere instead of a point and where the edges are formed with neighboring intersecting spheres. This representation has the advantage of instead of returning piecewise linear paths, returning a sequence of overlapping spheres, i.e. a convex corridor, that connects a given start and goal configuration, illustrated in Figure \ref{fig:pbrm_intro}. This convex corridor allows us to use convex optimization to produce smooth trajectories, instead of computationally expensive post-processing methods. The representation further allows us to estimate the coverage of the collision-free space, which gives us awareness and feedback in the offline roadmap construction phase. Finally, our representation is simple to adapt to moving obstacles, simply requery for the new radii and recheck for intersections. 
\\\\
The spherical collision-free regions are formed using a signed distance function (SDF), which is a function that returns the smallest distance from an arbitrary point to the boundary of an obstacle. As the name implies, the distance is signed, thus if the point is inside the obstacle it is negative otherwise positive. If the distance is positive, a sphere with radius equal to the distance is guaranteed to cover a collision-free region. Using an SDF in motion planning is not new, but what is novel about our approach is that we express the distance in the configuration space instead of the world space and by doing so allows us to form these convex collision-free regions. We refer to the resulting SDF as a signed configuration distance function (SCDF). Computing an SCDF analytically is non-trivial, our approach is therefore to parameterize the SCDF with a deep neural network and learn the mapping by supervised learning. Our resulting neural SCDF can compute distances for different parameter values of obstacle shapes and we also show how multiple distances can be combined, thus making our approach flexible.
\section{Related work}
Motion planning algorithms can roughly be divided into three families, grid-based, sampling based and optimization based methods. Grid-based methods (GBM) discretize the planning space from which a graph is then compiled. A standard search method is A$^\star$ \citep{a_star}, which is classified as an \textit{informed} search method, since it employs a heuristic function to speed up the search. A$^\star$ guarantees to return an optimal path at the level of discretization used. GBMs usually discretize the planning space by a regular lattice and this limits the GBMs to problems with low dimensionality due to the curse of dimensionality. Thus, GBMs are usually limited to single-body robots where the degrees of freedom (DOF) are low. To overcome the inherent scaling problem with the GBMs, stochastic methods are usually used for multi-body robots. These methods are termed as sampling-based methods (SBM) and core members within this family are the rapidly-exploring random trees (RRT) \citep{rrt} and the probabilistic roadmap (PRM) \citep{prm}. RRT grows a tree from the start configuration and explores the collision-free region in a rapid way until it is able to connect to the goal region. RRT is usually improved by bi-directional planning \citep{rrt_connect}, i.e. an additional tree is grown from the goal configuration and the trees are tested for connection after any tree has been expanded. RRT is a single-query method, thus it searches for a path from scratch each time it is queried. Contrary to this, PRM is a multi-query method, which solves for multiple queries without starting from scratch. PRM does this by creating a roadmap (graph) that covers the collision-free space as an offline step. The graph is then used to solve for multiple queries. PRMs are used in cases where the environment does not change since the extra offline step is too computationally costly and needs to be re-done if the environment is changed. In our work, we address this inherent issue by using a different roadmap representation. Our vertices in the graph cover a collision-free region in the form of spheres and we form the edges by checking for intersecting spheres. If something in the environment changes, we recompute the spheres radii and recheck the intersections, without relying on collision detection. We use a trained neural network to compute the sphere radius, therefore querying for the radius can be done fast, hence our representation enables the PRM for dynamic environments.
\\\\
In the recent decades, optimization based methods (OBM) \citep{chomp, schulman, itomp, stomp} have been introduced as an alternative to SBM for multi-body robots. Like the SBM, the OBMs scale well to higher dimensional problems and produce smoother motion. It is common to use a SDF in the optimization since it is a smooth function, thus enabling gradient-based methods. However, the standard way of expressing the SDF is in world space. The distance therefore needs to be mapped to the configuration space by the forward kinematics. This mapping makes the optimization problem a non-linear program (NLP), which is computationally expensive to solve. Recently, a different approach has been proposed. In \cite{mp_gcs} motion planning is formulated as a convex optimization problem by using the graph of convex sets framework \citep{gcs}. The underlying idea is to decompose the collision-free space into intersecting convex sets from which a convex optimization problem is formulated. In cases where an explicit representation of the obstacles in the configuration space exists, like for single-body robots, creating collision-free convex regions can be done fast \citep{iris}. For multi-body robots, this is non-trivial. Existing work does this successfully \citep{iris_nlp, iris_c} by an optimization based approach, but the methods are still too time consuming to be used in the presence of moving obstacles. Our approach is instead to use deep learning to learn an SDF expressed in the configuration space. With this, we can query for shortest distances to the collision boundary, which allows us to expand spherical regions which are collision-free. Our approach is fast and therefore enables our suggested roadmap planner to be used in dynamic environments.
\\\\
Recent research has focused on learning collision detection \citep{fk_kernel_distance, diffco, graphdistnet} by predicting the signed distance between the robot links and the surrounding obstacles in the world space. The learned SDF is used in trajectory optimization but since the distance is expressed in the world space, the problem becomes an NLP and therefore takes a long time to solve. We take a novel approach and suggest to instead express the signed distance in the configuration space. This allows us to improve the PRM at the same time as it enables convex optimization for trajectory optimization, which runs faster and is more reliable than NLP solvers. In \cite{cspf} a learned signed distance function in the configuration space is proposed similar to our approach. However, their approach is restricted to point cloud representations, while we propose to represent the obstacles as parameterized geometric shapes, e.g. spheres. Furthermore, we also show how to use our learned SCDF to improve an existing roadmap planner.
\section{Problem formulation}
A robot is located in the world space, $\W \subset \R^3 $. The unique location of the robot is given by its configuration $\q \in \C$, where $\C$ is the configuration space. The set of points covered by the robots bodies at a certain configuration is expressed as $\B(\q) \subset \W$. The robot is surrounded by $\NrObst$ obstacles $\O = \bigcup_{i=1}^{\NrObst} \O_i$, where  $\O_i \subset \W$. The representation of the obstacle in the configuration space is the set $\C\O_i = \{\q \in \C \: |\: \B(\q) \cap \O_i \neq \emptyset \}$. The obstacle space is formed as $\Co = \bigcup_{i=1}^{\NrObst} \C \O_i$. The complement is referred to as the free space, $\Cf = \C \setminus \Co$. The path planning problem is a tuple, ($\Cf$, $\qStart$, $\qGoal$), where we want to connect a query pair, consisting of a start, $\qStart$, and goal configuration, $\qGoal$, with a geometric path, $\q(s): [0, 1] \mapsto \Cf$, such that $\q(0)=\qStart$ and $\q(1)=\qGoal$, or report correctly when such a path does not exist.
\end{document}

% TODO more details in methodology and data processing
% merge methodology and 
\section{Framework for Analyzing Emotion}
In this section, we present our framework for analyzing emotion. We first establish a basic understanding of emotion polarity by determining the sentiment valence of each root tweet and comment. We then use multi-label emotion detection to predict the emotion categories associated with each post. Based on this data, we explore the interactive nature of emotions, by identifying common patterns in emotion transition pairs between temporally-adjacent posts. Finally we investigate the emotional trajectory within threads to understand how emotional intensity and type shift over time, by aggregating the predicted labels for posts at each time stamp in a given thread. As part of this, we contrast rumour with non-rumour threads, to gain a holistic understanding of emotional expression in rumours and non-rumours on Twitter.

% elaborate a bit on why we choose EmoLLM, compared with other automatic emotion detection methods
\paragraph{Affective Computing: Automatic Emotion Detection}
Manually annotating emotions is both costly and time-consuming, so we use an LLM-based emotion detection model, EmoLLM~\citep{liu2024emollms}, which is specifically designed for sentiment analysis and emotion detection. The model was instruction-tuned on SemEval 2018 Task1 using a comprehensive emotion labeling scheme grounded in established theoretical frameworks. We prompt the model to perform Valence Ordinal Classification (V-oc), Emotion Classification (E-c), and Emotion Intensity regression (E-i). Detailed prompts are shown in \Cref{tab:emollm_ins}.

\paragraph{Categorical Emotion Labeling Scheme} \label{para:emotion_label}
Numerous emotion label sets  have been proposed~\citep{Ekman1992AnAF, Plutchik1980AGP, Russell1980ACM}. According to \citet{Ekman1992AnAF, Plutchik1980AGP}, certain emotions, such as joy, fear, and sadness, are considered more fundamental than others, both physiologically and cognitively. The Valence-Arousal-Dominance (VAD) model \citep{Russell1980ACM} categorizes emotions within a three-dimensional space of valence (positivity-negativity), arousal (active-passive), and dominance (dominant-submissive). Inspired by \citet{mohammad-etal-2018-semeval}, we incorporate elements from both basic emotion theories and the VAD model, and further ground EmoLLM emotion predictions to develop the following emotion label schemes: (1) \textit{neutral or no emotion}; (2) \textit{negative emotions}: anger (also includes annoyance and rage),  disgust (also includes disinterest, dislike, and loathing), fear (also includes apprehension, anxiety, and terror), pessimism (also includes cynicism, and no confidence), sadness (also includes pensiveness and grief); 3) \textit{positive emotions}: joy (also includes serenity and ecstasy), love (also includes affection), optimism (also includes hopefulness and confidence), anticipation (also includes interest and vigilance), surprise (also includes distraction and amazement) and trust (also includes acceptance, liking, and admiration). 


\paragraph{Emotion Polarity: Sentiment Valence} 
To understand the basic emotion polarity expressed in rumour and non-rumour content, we begin with sentiment valence analysis. Sentiment valence aims to capture the overall emotional tone conveyed by a post, in terms of how positive or negative it is~\citep{liu2024emosurvey}. We frame the sentiment valence task as ordinal regression~\citep{mohammad-etal-2018-semeval}. As shown in \Cref{tab:emollm_ins}, for a given tweet post, we classify it into one of seven ordinal levels of sentiment intensity, spanning varying degrees of positive and negative valence, that best represents the tweeter's mental state. The tweet posts within a thread can be divided into two categories: root tweets, which are posted by the publisher, and follow posts, which include all subsequent replies under the root post. We begin by conducting sentiment valence analysis on each post within the thread conversation. 
% TJB: confused by how comments can include all subsequent replies; we seem to be overloading the terminology, for comments to be both individual posts and series of posts
% RX: yes, I am unifiying all terms.
For each category, we compute the mean sentiment valence to enable further investigation into the specific emotions associated with different sentiment valences over a thread.
% TJB: clarify for comments whether the classification is done over the combined meta-document (i.e. the root + all comments to that point) or individually over the separate documents and then combined ... or over individual documents, in which case the statement about "all replies" needs clarification
% RX: we separate root and comments for each tweet conversation, the former is the root tweet posted by the publisher while the rest are comments. "all replies" mean all comments under root tweet, we aggregate them by computing the mean sentiment, and then average over each part.

\paragraph{Emotion Distribution} 
Following sentiment valence analysis, we then examine specific emotions and their distribution in rumour and non-rumour tweet posts.
Motivated by the fact that a certain tweet might exhibit more than one emotion, we frame the task as multi-label emotion detection problem. As shown as V-oc in \Cref{tab:emollm_ins}, given a tweet, we classify it into one of seven ordinal classes, corresponding to various levels of positive and negative sentiment intensity. To reduce noise from automatic emotion detectors, we take the top-three predicted emotions for each tweet. We then aggregate and plot the emotion distribution to provide an overview of dominant emotional trends across the rumour and non-rumour posts. Given that the follow posts make up the majority of the data compared to the root posts, we will focus on using follow posts in our next analysis.
% TJB: what is the basis of saying that the signal is richer? simply that there are more reply posts than root posts? clarify
% RX: yes, and we are more interested in interaction in comments.

\paragraph{Emotion Transitions} 
Emotions are contagious and highly interactive~\citep{Ferrara_2015}. When publishers write tweets that convey their emotions, readers are likely to respond with emotional reactions of their own~\citep{Ferrara_2015,emotion_dynamics}. In this part, we model this interactive nature of emotions in the form of emotion transition pairs, which are built from two chronologically-adjacent tweets. In each pair, the first element represents the emotion inferred from the initial content published at a given time, and the second element represents the emotion inferred from the reply content published immediately after. For example, if the first tweet exhibits \textit{joy} \textit{trust} and \textit{anticipation}, and the second tweet shows \textit{anger}, \textit{disgust} and \textit{surprise}, we form the pairs (\textit{joy}, \textit{anger}), (\textit{joy}, \textit{disgust}), (\textit{joy}, \textit{surprise}), (\textit{trust}, \textit{surprise}), (\textit{trust}, \textit{surprise}), (\textit{trust}, \textit{disgust}), (\textit{anticipation}, \textit{anger}), (\textit{anticipation}, \textit{surprise}) and (\textit{anticipation}, \textit{disgust}). We create transitions for all combinations of emotion pairs and explore the likelihood of emotion transition pairs occurring in rumour and non-rumour content. Exploring emotion transitions allows us to understand the emotional flow in social media conversations and uncover typical patterns of rumour and non-rumour content, and any differences between the two.

\paragraph{Emotion Trajectories} 
We explore the cumulative trajectory of emotion over time to observe how emotions evolve during the conversational thread. We collect all detected emotion labels for each tweet from both rumour and non-rumour content, then track cumulative emotion counts at each chronological step. Finally, we visualize these trends and apply regression models to analyze the growth of emotions over time. This temporal analysis reveals how emotions accumulate or intensify across time, offering insight into the trajectory of emotions in rumour and non-rumour content.

\begin{table*}[!h]
    \centering
    \small
    \begin{tabular}{cccccccccccc}
        \toprule
        \textbf{Setting} & \textbf{Ru} & \textbf{Non} & \textbf{p} & \textbf{\#Ru/Non} & \textbf{T} & \textbf{F} & \textbf{U} & \textbf{$p$ (U vs T)} & \textbf{$p$ (U vs F)} & \textbf{\#T/\#F/\#U} \\
        \midrule
        \textbf{PHEME root} & \textbf{$-$0.25} & $-$0.17 & 0.00 & 2602/2602 & $-$0.21 & $-$0.11 & \textbf{$-$0.39} & 7.75e-11 & 4.41e-11 & 629/629/629 \\
        \textbf{PHEME follow} & \textbf{$-$0.33} & $-$0.26 & 6.47e-09 & & $-$0.35 & $-$0.20 & \textbf{$-$0.39} & 0.03 & 8.38e-15 & \\
        \textbf{Twitter15 root} & \textbf{$-$0.26} & $-$0.01 & 3.51e-05 & 372/372 & $-$0.21 & $-$0.20 & \textbf{$-$0.34} & 0.01 & 0.01 & 359/359/359 \\
        \textbf{Twitter15 follow} & \textbf{$-$0.27} & $-$0.06 & 1.65e-09 & & $-$0.24 & $-$0.25 & \textbf{$-$0.30} & 0.16 & 0.21 & \\
        \textbf{Twitter16 root} & \textbf{$-$0.18} & \z0.07 & 0.00 & 205/205 & \z0.11 & $-$0.22 & \textbf{$-$0.30} & 1.35e-06 & 0.18 & 63/63/63 \\
        \textbf{Twitter16 follow} & \textbf{$-$0.31} & $-$0.12 & 9.19e-06 & & $-$0.30 & \textbf{$-$0.36} & $-$0.27 & 0.67 & 0.90 & \\
        % \textbf{CoAID root} & \textbf{$-$0.34} & $-$0.16 & 0.01 & 167/167 & - & - & - & - & - & - \\
        % \textbf{CoAID follow} & \textbf{$-$0.24} & $-$0.13 & 0.01 & & - & - & - & - & - & \\
        \bottomrule
    \end{tabular}
    \caption{Valence Ordinal Regression results for all datasets. root = root posts, follow = follow posts, Ru = rumour, Non = Non-rumour, T = True rumour, F = False rumour, U = Unverified rumour; $p$ values indicates significance of a one-tailed t-test.}
\label{tab:voc_results}
\end{table*}

\begin{algorithm}[ht!] 
\caption{PC Algorithm}
\label{pc}
\begin{algorithmic}[1] 
\State \textbf{Input:} Data $\mathbf{X}$, significance level $\alpha$
\State \textbf{Output:} Completed Partially Directed Acyclic Graph (CPDAG)

\State Initialize a complete undirected graph $G$ with all variables as nodes.

\State \textbf{Step 1: Skeleton Identification}
\For{each pair of variables $(X, Y)$ in $G$}
    \State Find the subset $S \subseteq \text{Adj}(X, G) \setminus \{Y\}$ such that 
    $X \indep Y \mid S$ with significance $\alpha$.
    \If{such a subset $S$ exists}
        \State Remove the edge $X - Y$ from $G$.
    \EndIf
\EndFor

\State \textbf{Step 2: Edge Orientation}
\For{each triple of variables $(X, Y, Z)$ in $G$ where $X - Z - Y$ and $X, Y$ are not adjacent}
    \If{$Z \notin S$ for all separating sets $S$ for $X$ and $Y$}
        \State Orient as $X \to Z \leftarrow Y$ (identify a collider).
    \EndIf
\EndFor

\While{possible}
    \For{each edge $(X - Y)$ in $G$}
        \If{there exists a directed path $X \to \dots \to Z$ such that $Z - Y$}
            \State Orient as $X \to Y$ (acyclicity rule).
        \ElsIf{orienting $X - Y$ as $X \to Y$ creates a new v-structure}
            \State Orient as $X \to Y$ (v-structure rule).
        \EndIf
    \EndFor
\EndWhile

\State \textbf{return} the CPDAG representing the equivalence class of causal graphs.

% how we frame the task, compute the emotion intensity, how to aggregate on conversation level

\end{algorithmic}
\end{algorithm}


\paragraph{Causal Relationship of Emotions in Rumour \& Non-Rumour Threads}
To gain a deeper insight into the relationship between rumours and the emotions underlying them, we extend our analysis beyond statistical correlation by conducting a causal analysis. Specifically, we apply the Peter-Clark (PC) algorithm \cite{Spirtes2000}, a classical constraint-based causal discovery algorithm on the three merged datasets. 

Uncovering causal relations between variables of interest is never an easy problem. Under the fundamental assumption of \textit{causal Markov condition} that a variable is conditionally independent of all its non-effects given its direct cause, \textit{faithfulness} ensures that the casual graph exactly encodes the independence and conditional independence relations among variables. These two assumptions allow us to infer causal relationships from observed statistical independencies, forming the cornerstone of constraint-based causal discovery methods. 

The PC algorithm identifies causal relationships among the variables of interest, represented as a directed acyclic graph (DAG), by numerating the independence and conditional independence relationships. The algorithm consists of two main steps: 
\begin{enumerate}
    \item \textbf{Skeleton Identification}: Starting with a complete undirected graph where all variables are connected, edges are iteratively removed based on conditional independence and independence relationships among variables, inferred by a conditional independence test. This step returns an undirected graph, which we call a skeleton. 
    \item \textbf{Edge Orientation}: After constructing the skeleton, edges are oriented by a set of predefined rules (Meek's Rule \cite{meek1997graphical}) to avoid cycles and orient collider structures.
\end{enumerate}

The complete PC algorithm is provided in algorithm \ref{pc}. It returns a  completed partially directed acyclic graph (CPDAG), which represents an equivalence class of causal graphs that are consistent with the observed data’s independence and conditional independence relations. In our implementation, we adopt the  Fisher-z test \cite{fisher_probable_1921} to infer the conditional independence relations.

%%% Local Variables:
%%% mode: latex
%%% TeX-master: "../main_anonymous"
%%% End:

\section{Implementation Environment}
\label{sec:implementation_environment}

Here we introduce the detailed implementation details and environment for reproducibility purpose. For our model, we choose hyperparameters based on the performance on validation set (Document classification task in the main paper explains how we split validation set). The results in the main paper are obtain by 5 independent runs. The standard deviations reported in the main paper are 1-sigma error bars and are obtained by calling its corresponding function in Excel library. All the experiments were done on Linux server with an NVIDIA A40 GPU with 46,068 MiB. Its operating system is CentOS Linux 7 (Core). We implemented our proposed model GTFormer using Python 3.10 as programming language and PyTorch 2.0.0 as deep learning library. Other frameworks include NumPy 1.23.1, sklearn 0.23.2, and scipy 1.5.2. We emphasize that the main focus of our model is effectiveness, instead of running efficiency. But for completeness, we still make a short comment on execution time. Our model is efficient, on the largest dataset Web, the training takes less than 40 hours to converge. We will release code and datasets upon publication.
\section{Evaluation}
We provide three sets of insights into this section, organised as \textit{findings (F*)}. We quantitatively study the effect of the adversarial and counterfactual perturbations on the performance of informal reasoners and autoformalisation methods. Then, we dive deeper into method variants. Finally, 
we analyse the nature of formalisation errors made by the models.

\subsection{Robustness Analysis}
\paragraph{\textbf{\emph{F1: Noise perturbations have a stronger effect on formalisation methods than informal \ac{LLM} reasoners.}}}
Table~\ref{tab:distraction_k4_formalisation} shows that, on average, the accuracy of both direct and \ac{CoT} informal reasoning remains between $73\%$ and $74\%$ in the face of added noise. While the autoformalisation method performs similarly to informal reasoners on the original dataset, its performance decreases between $4\%$ and $11\%$. The accuracy drops especially with logical (L) and tautological (T) distractions, whose logical language formats trick the \ac{LLM} into formalizing the noisy clauses. On the other hand, the linguistically complex and more natural sentences of encyclopedic distractions show a minor effect, suggesting that \acp{LLM} successfully avoids formalizing the more complicated sentences.

\paragraph{\textbf{\emph{F2: All \ac{LLM}-based reasoning methods suffer a drop for counterfactual perturbations.}}} % influence .}}}
Table~\ref{tab:distraction_k4_formalisation} shows that counterfactual statements cause a significant decrease in performance for both the informal reasoners and autoformalisation methods of between $12\%$ and $13\%$ on average. 
Moreover, this observation also holds for all tested models, i.e., none are robust towards counterfactual perturbations across every evaluated dimension. Even the strongest model, GPT 4o-mini, yields a performance of 63-68\%, which is relatively close to the random performance of 50\%. The high impact of counterfactual statements (the single ``not'' inserted) could be due to the inability of \acp{LLM} to overwrite prior knowledge with explicitly stated information or memorization of the answers. We study the error sources further in §\ref{subsec:errors}.  

\noindent \paragraph{\textbf{\emph{F3: Introducing multiple noise sentences has an effect only for logical distractions.}}}
We show the impact of introducing between one and four sentences for the two top-performing autoformalisation models in Figure~\ref{fig:length_distraction}. The figure shows similar trends with and without counterfactual perturbations.
As additional logical distractions are introduced, the model performance consistently decreases. Tautological (T) distractions lead to a decline in accuracy with a single disruptive sentence, yet adding more noise does not worsen the outcome. 
The tautological corpus introduces truth constants for all sentences as a persistent unseen logical construct. Given that this leads only to a decrease for a single occurrence, we can assume that a model can consistently handle the same unseen logical construct. In contrast, the logical corpus increases the chance of adding text, requiring new, previously unseen reasoning constructs for each added sentence. The impact of encyclopedic noise remains negligible, generalising F1 to $k$ sentences. Similarly, counterfactual perturbations remain much more effective for all settings, generalising F2.

\begin{table}[!t]
\small
\setlength{\modelspacing}{2pt}
\setlength{\tabcolsep}{1.7pt} % Default value: 6pt
\setlength{\belowrulesep}{4pt}
\begin{threeparttable}
    \centering
    \begin{tabular}{cc l r rrr @{\quad} rrrr}
\toprule
\multirow{2}{*}{} & \multirow{2}{*}{} & Reasoning & \multirow{2}{*}{O} & \multicolumn{3}{c}{Distraction} & \multicolumn{4}{c}{Counterfactual} \\
 & & Format & & E& L & T & $\text{O}_C$ & $\text{E}_C$& $\text{L}_C$ & $\text{T}_C$\\
\midrule
\multirow{6}{*}{\rotatebox{90}{Gemma-2}} & \multirow{3}{*}{\rotatebox{90}{9b}}
   & Informal (direct) & \textbf{0.78} & \textbf{0.80} & \textbf{0.79} & \textbf{0.77} & 0.58 & 0.52 & 0.50 & 0.59 \\
 & & Informal (CoT) & 0.72 & 0.78 & 0.73 & 0.76 & 0.61 & \textbf{0.57} & \textbf{0.60} & \textbf{0.66} \\
 & & Formal (FOL) & 0.62 & 0.58 & 0.52 & 0.53 & \textbf{0.63} & 0.52 & 0.46 & 0.46 \\[\modelspacing]
\cmidrule{2-11}
 & \multirow{3}{*}{\rotatebox{90}{27b}} 
   & Informal (direct) & 0.71 & 0.69 & \textbf{0.66} & \textbf{0.68} & 0.59 & 0.51 & 0.54 & 0.59 \\
 & & Informal (CoT) & 0.66 & 0.65 & 0.64 & 0.63 & 0.62 & 0.58 & \textbf{0.62} & \textbf{0.64} \\
 & & Formal (FOL) & \textbf{0.74} & \textbf{0.74} & 0.61 & 0.61 & \underline{\textbf{0.72}} & \underline{\textbf{0.67}} & 0.58 & 0.51 \\[\modelspacing]
\midrule
\multirow{6}{*}{\rotatebox{90}{Mistral}} & \multirow{3}{*}{\rotatebox{90}{7B}} 
   & Informal (direct) & 0.77 & \textbf{0.77} & 0.75 & \textbf{0.79} & \textbf{0.63} & \textbf{0.54} & \textbf{0.54} & \textbf{0.66} \\
 & & Informal (CoT) & \textbf{0.79} & 0.75 & \textbf{0.77} & 0.78 & 0.55 & 0.52 & \textbf{0.54} & 0.58 \\
 & & Formal (FOL) & 0.62 & 0.58 & 0.54 & 0.57 & 0.50 & \textbf{0.54} & 0.51 & 0.52 \\[\modelspacing]
\cmidrule{2-11}
 & \multirow{3}{*}{\rotatebox{90}{Small}} 
   & Informal (direct) & \textbf{0.77} & \textbf{0.76} & \textbf{0.76} & \textbf{0.75} & 0.61 & 0.51 & 0.56 & 0.59 \\
 & & Informal (CoT) & 0.72 & 0.72 & 0.72 & 0.71 & \textbf{0.62} & \textbf{0.59} & \textbf{0.62} & \textbf{0.68} \\
 & & Formal (FOL) & 0.68 & 0.59 & 0.53 & 0.64 & 0.54 & 0.55 & 0.49 & 0.51 \\[\modelspacing]
\midrule
\multirow{6}{*}{\rotatebox{90}{Llama-3.1}} & \multirow{3}{*}{\rotatebox{90}{8B}} 
   & Informal (direct) & 0.63 & 0.61 & 0.64 & 0.66 & 0.61 & \textbf{0.62} & 0.59 & 0.61 \\
 & & Informal (CoT) & 0.73 & \textbf{0.73} & \textbf{0.71} & \textbf{0.72} & \textbf{0.62} & 0.59 & \textbf{0.61} & \textbf{0.65} \\
 & & Formal (FOL) & \textbf{0.77} & 0.71 & 0.63 & 0.52 & 0.60 & 0.58 & 0.55 & 0.52 \\[\modelspacing]
\cmidrule{2-11}
 & \multirow{3}{*}{\rotatebox{90}{70B}} 
   & Informal (direct) & 0.77 & 0.74 & 0.74 & 0.73 & 0.62 & 0.53 & 0.56 & 0.64 \\
 & & Informal (CoT) & \textbf{0.78} & \textbf{0.75} & \textbf{0.76} & \textbf{0.76} & 0.64 & 0.61 & \textbf{0.66} & \underline{\textbf{0.73}} \\
 & & Formal (FOL) & 0.74 & 0.73 & 0.71 & 0.71 & \textbf{0.66} & \textbf{0.62} & 0.59 & 0.57 \\[\modelspacing]
 \midrule
\multirow{3}{*}{\rotatebox{90}{GPT}} & \multirow{3}{*}{\rotatebox{90}{4o-mini}} 
   & Informal (direct) & 0.78 & 0.77 & 0.79 & 0.79 & 0.64 & 0.61 & 0.61 & 0.63 \\
 & & Informal (CoT) & 0.80 & 0.80 & \underline{\textbf{0.81}} & \underline{\textbf{0.82}} & \textbf{0.68} & \textbf{0.63} & \underline{\textbf{0.68}} & \textbf{0.64} \\
 & & Formal (FOL) & \underline{\textbf{0.84}} & \underline{\textbf{0.82}} & 0.73 & 0.79 & 0.63 & 0.62 & 0.57 & 0.54 \\[\modelspacing]
 \midrule
\multicolumn{2}{c}{\multirow{3}{*}{\textbf{Avg}}} 
 & Informal (direct) & 0.74 & 0.73 & 0.73 & 0.73 & 0.61 & 0.55 & 0.56 & 0.62 \\
 & & Informal (CoT) & 0.74 & 0.74 & 0.73 & 0.74 & 0.62 & 0.58 & 0.62 & 0.65 \\
  & & Formal (FOL) & 0.72 & 0.68 &	0.61 & 0.62 & 0.61 & 0.59 & 0.54 & 0.52 \\
\bottomrule
\end{tabular}
\caption{Accuracies of informal and autoformalisation-based deductive reasoners. The best overall model per dataset is underlined; the best model version is marked in bold.}
\label{tab:distraction_k4_formalisation}
\end{threeparttable}
\end{table} 

\begin{figure}[!t]
    \centering
    \scriptsize
    \begin{tikzpicture}
        \begin{axis}[name=gpt,
            title={GPT-4o-mini},
            width=0.6\linewidth,
            height=0.6\linewidth,
            xlabel={\# Noise sentences},
            ylabel={Accuracy},
            xmin=-0.1, xmax=4.1,
            ymin=0.5, ymax=0.9,
            xtick={1,2,4},
            ytick={0.55, 0.6, 0.65, 0.75, 0.8, 0.85},
            title style={yshift=-0.6em},
            legend style={at={(1,-0.15)},
	           anchor=north,legend columns=-1},
            x label style={at={(axis description cs:1,-0.05)},anchor=north},
            y label style={at={(axis description cs:-0.15,0.5)},anchor=south},
            ymajorgrids=true,
            grid style=dashed,
        ]
            \addplot[color=blue, mark=square,]
                coordinates {
                (0,0.848076939582825)(1,0.823076903820038)(2,0.826923072338104)(4,0.821153819561005)
                };
            \addplot[color=red, mark=triangle,]
                coordinates {
                (0,0.848076939582825)(1,0.817307710647583)(2,0.801923096179962)(4,0.759615361690521)
                };
            \addplot[color=green, mark=diamond,] 
                coordinates {
                (0,0.848076939582825)(1,0.767307698726654)(2,0.769230782985687)(4,0.803846180438995)
                };
            \addplot[color=blue, mark=square*] 
                coordinates {
                (0,0.627777755260468)(1,0.622222244739533)(2,0.600000023841858)(4,0.633333325386047)
                };
            \addplot[color=red, mark=triangle*,] 
                coordinates {
                (0,0.627777755260468)(1,0.611111104488373)(2,0.611111104488373)(4,0.594444453716278)
                };
            \addplot[color=green, mark=diamond*,] 
                coordinates {
                (0,0.627777755260468)(1,0.572222232818604)(2,0.538888871669769)(4,0.555555582046509)
                };
                \legend{E,L,T,$\text{E}_C$, $\text{L}_C$ , $\text{T}_C$}
        \end{axis}

        \begin{axis}[name=llama, at={($(gpt.east)+(0.1cm,0)$)},anchor=west,
            title={Llama 3.1 70b},
            width=0.6\linewidth,
            height=0.6\linewidth,
            xmin=-0.1,, xmax=4.1,
            ymin=0.5, ymax=0.9,
            xtick={1,2,4},
            ytick={0.55, 0.6, 0.65, 0.75, 0.8, 0.85},
            title style={yshift=-0.6em},
            yticklabel=\empty,
            ymajorgrids=true,
            grid style=dashed,
        ]
            \addplot[color=blue, mark=square,]
                coordinates {
                (0,0.838461518287659)(1,0.817307710647583)(2,0.805769205093384)(4,0.817307710647583)
                };
            \addplot[color=red, mark=triangle,]
                coordinates {
                (0,0.838461518287659)(1,0.819230794906616)(2,0.803846180438995)(4,0.771153867244721)
                };
            \addplot[color=green, mark=diamond,]
                coordinates {
                (0,0.838461518287659)(1,0.803846180438995)(2,0.807692289352417)(4,0.805769205093384)
                };
            \addplot[color=blue, mark=square*]
                coordinates {
                (0,0.627777755260468)(1,0.622222244739533)(2,0.577777802944183)(4,0.594444453716278)
                };
            \addplot[color=red, mark=triangle*,]
                coordinates {
                (0,0.627777755260468)(1,0.583333313465118)(2,0.561111092567444)(4,0.577777802944183)
                };
            \addplot[color=green, mark=diamond*,]
                coordinates {
                (0,0.627777755260468)(1,0.627777755260468)(2,0.566666662693024)(4,0.577777802944183)
                };
        \end{axis}
    \end{tikzpicture}
    \caption{Influence of the number of noisy sentences for FOL.}
    \label{fig:length_distraction}
\end{figure}



\subsection{Impact of Method Design}
\paragraph{\textbf{\emph{F4: \ac{CoT} prompting is most impactful when both noise and counterfactual perturbations are applied.}}}
The accuracies for the individual \acp{LLM} in Table~\ref{tab:distraction_k4_formalisation} show that the impact of \ac{CoT} is negligible for noise-only datasets (first four columns). Meanwhile, the benefit from \ac{CoT} is most pronounced in the datasets that combine noise and counterfactual perturbations.
The better-performing informal prompting strategy for a model remains stable for all types of distractions. Still, the decline in performance due to counterfactuals leads to a less consistent preference for a specific prompting style.

\paragraph{\textbf{\emph{F5: The best-performing grammar differs per model and is unstable across data versions.}}}

The evaluation of different logical forms for formal \ac{LLM}-based reasoning in Table~\ref{tab:distraction_k4_logical_form} shows the preference of some models for specific syntactic formats.
Llama 3.1 70B has a considerable improvement of $12\%$ with TPTP syntax on the original set, while Llama 3.1 8B benefits from the R-FOL syntax. However, all grammars show a declining accuracy trend and increased syntax errors for noise perturbations, where the best grammar loses its advantage over the rest. 
When comparing the grammars on the counterfactual partitions, we observe that TPTP is consistently more robust than the standard first-order logic grammar. Here, GPT 4o-mini shows a reduction from $O$ to $O_C$ of $20\%$ for FOL and only $12\%$ for the TPTP grammar. Since this does not correlate with fewer syntax errors, the formalisation in TPTP prevents semantical errors for counterfactual premises. 
A positive reading of these results, especially the minor differences between FOL and R-FOL, is that autoformalisation \acp{LLM} can adapt to the grammar syntax prescribed in the prompt without further loss in performance.

\begin{table}[!t]
\small
\setlength{\modelspacing}{2pt}
\setlength{\tabcolsep}{1.7pt} % Default value: 6pt
\setlength{\belowrulesep}{4pt}
\begin{threeparttable}
    \centering
    \begin{tabular}{cc l r rrr @{\quad} rrrr}
\toprule
\multirow{2}{*}{} & \multirow{2}{*}{} & Grammar & \multirow{2}{*}{O} & \multicolumn{3}{c}{Distraction} & \multicolumn{4}{c}{Counterfactual} \\
 & & Syntax & & E& L & T & $\text{O}_C$ & $\text{E}_C$& $\text{L}_C$ & $\text{T}_C$\\
\midrule
\multirow{6}{*}{\rotatebox{90}{Llama-3.1}} & \multirow{3}{*}{\rotatebox{90}{8B}} 
   & FOL & 0.77 & \textbf{0.71} & 0.61 & \textbf{0.53} & 0.58 & \textbf{0.55} & 0.52 & \textbf{0.56} \\
 & & R-FOL & \textbf{0.78} & 0.69 & \textbf{0.62} & \textbf{0.53} & 0.58 & \textbf{0.55} & \textbf{0.54} & 0.52 \\
 & & TPTP & 0.73 & 0.67 & 0.55 & 0.51 & \textbf{0.68} & 0.54 & 0.46 & 0.51 \\[\modelspacing]
\cmidrule{2-11}
 & \multirow{3}{*}{\rotatebox{90}{70B}} 
   & FOL & 0.76 & 0.73 & 0.71 & \textbf{0.72} & 0.67 & 0.57 & 0.63 & 0.56 \\
 & & R-FOL & 0.76 & 0.73 & 0.67 & 0.71 & 0.64 & 0.57 & 0.53 & 0.64 \\
 & & TPTP & \underline{\textbf{0.88}} & \underline{\textbf{0.84}} & \underline{\textbf{0.81}} & \textbf{0.72} & \underline{\textbf{0.81}} & \underline{\textbf{0.68}} & \underline{\textbf{0.67}} & \underline{\textbf{0.68}} \\[\modelspacing]
\midrule
\multirow{3}{*}{\rotatebox{90}{GPT}} & \multirow{3}{*}{\rotatebox{90}{4o-mini}} 
   & FOL & \textbf{0.84} & \textbf{0.82} & \textbf{0.72} & \underline{\textbf{0.78}} & 0.64 & \textbf{0.63} & \textbf{0.61} & 0.51 \\
 & & R-FOL & \textbf{0.84} & 0.77 & 0.70 & \underline{\textbf{0.78}} & \textbf{0.72} & 0.56 & 0.54 & \textbf{0.63} \\
 & & TPTP & 0.83 & \textbf{0.82} & 0.71 & 0.71 & 0.69 & \textbf{0.63} & 0.57 & 0.57 \\
\bottomrule
\end{tabular}
\caption{Accuracies of different formalisation grammars for autoformalisation.}
\label{tab:distraction_k4_logical_form}
\end{threeparttable}
\end{table} 

\paragraph{\textbf{\emph{F6: Feedback does not help \acp{LLM} self-correct to mitigate robustness issues.}}}
\autoref{tab:distraction_k4_feedback} shows the results with different error recovery mechanisms. The results indicate that no feedback strategy emerges as a winner in the different datasets. 
All feedback variants reduce syntax errors for noise perturbations, but given the lack of a consistent increase in accuracy, the corrected formalisations are most likely to contain semantic errors still. 
The type of feedback message only has a minor influence on correcting syntax errors, whereas Llama 3.1 70b and GPT 4o-mini correct slightly more syntax errors with specific error messages. This finding aligns with \cite{huang2023large}, who also found that \acp{LLM} cannot consistently self-correct their reasoning after receiving relevant feedback.

\begin{table}[!ht]
\small
\setlength{\modelspacing}{2pt}
\setlength{\tabcolsep}{1.7pt} % Default value: 6pt
\setlength{\belowrulesep}{4pt}
\begin{threeparttable}
    \centering
    \begin{tabular}{cc l r rrr @{\quad} rrrr}
\toprule
\multirow{2}{*}{} & \multirow{2}{*}{} & \multirow{2}{*}{Feedback} & \multirow{2}{*}{O} & \multicolumn{3}{c}{Distraction} & \multicolumn{4}{c}{Counterfactual} \\
 & & & & E& L & T & $\text{O}_C$ & $\text{E}_C$& $\text{L}_C$ & $\text{T}_C$\\
\midrule
\multirow{8}{*}{\rotatebox{90}{Llama-3.1}} & \multirow{4}{*}{\rotatebox{90}{8B}} 
   & No recovery & 0.77 & \textbf{0.72} & 0.62 & 0.53 & 0.59 & 0.58 & 0.56 & \textbf{0.56} \\
 & & Error type & \textbf{0.79} & 0.71 & 0.63 & \textbf{0.56} & \textbf{0.66} & 0.54 & 0.52 & 0.51 \\
 & & Error message & 0.78 & 0.71 & \textbf{0.67} & 0.55 & 0.59 & 0.53 & \underline{\textbf{0.64}} & 0.49 \\
 & & Warning & 0.74 & 0.66 & 0.58 & 0.55 & 0.55 & \textbf{0.60} & 0.49 & 0.49 \\[\modelspacing]
\cmidrule{2-11}
 & \multirow{4}{*}{\rotatebox{90}{70B}} 
   & No recovery & \textbf{0.77} & \textbf{0.72} & \textbf{0.73} & 0.71 & \textbf{0.64} & 0.59 & \textbf{0.61} & 0.56 \\
 & & Error type & 0.72 & 0.70 & 0.72 & \textbf{0.73} & 0.62 & 0.56 & 0.60 & 0.58 \\
 & & Error message & 0.71 & 0.70 & \textbf{0.73} & 0.71 & \textbf{0.64} & 0.59 & 0.54 & \underline{\textbf{0.64}} \\
 & & Warning & 0.69 & \textbf{0.72} & 0.72 & 0.72 & 0.62 & \underline{\textbf{0.65}} & \textbf{0.61} & 0.63 \\[\modelspacing]
\midrule
\multirow{4}{*}{\rotatebox{90}{GPT}} & \multirow{4}{*}{\rotatebox{90}{4o-mini}} 
   & No recovery & \underline{\textbf{0.84}} & \underline{\textbf{0.82}} & 0.73 & 0.79 & 0.64 & \textbf{0.62} & 0.56 & \textbf{0.56} \\
 & & Error type & 0.83 & 0.79 & 0.74 & 0.76 & 0.67 & 0.57 & 0.56 & \textbf{0.56} \\
 & & Error message & \underline{\textbf{0.84}} & 0.78 & \underline{\textbf{0.77}} & \underline{\textbf{0.80}} & 0.62 & 0.59 & 0.56 & \textbf{0.56} \\
 & & Warning & \underline{\textbf{0.84}} & 0.75 & 0.73 & 0.76 & \underline{\textbf{0.70}} & 0.61 & \textbf{0.61} & 0.55 \\
 \bottomrule
\end{tabular}
\caption{Accuracies of error recovery strategies.}
\label{tab:distraction_k4_feedback}
\end{threeparttable}
\end{table} 

\subsection{Error Analysis}
\label{subsec:errors}
\paragraph{\textbf{\emph{F7: Autoformalisation increases syntax errors for noise perturbations.}}}
The low performance for noise perturbations correlates with more syntax errors for all models and distraction categories (cf. execution rates in Table~\ref{tab:appendix_k4_formalisation_exec}). The three worst-performing models (both Mistral models, Gemma-2 9b) generate, at best, for $37\%$  and, at worst, for only $4\%$ of the samples, a valid logical form.
Gemma-2 9b and Llama3.1 8b produce more syntax errors than the larger counterparts, suggesting that larger models are more robust towards noise perturbations. 
The accuracy of syntactically valid samples is higher than the informal reasoning methods for most distractions (Table~\ref{tab:appendix_k4_formalisation_vacc}), motivating informal reasoning as a backup strategy for formal reasoning. The error message feedback reveals two common syntax errors: 1) errors by models with an initial low execution rate exhibit issues with the template structure, including using incorrect keywords or adding conversational phrases;
2) perturbation-related errors, the most common of which is using undefined truth constants as part of tautological distractions. 

\paragraph{\textbf{\emph{F8: Autoformalisation increases semantic errors for counterfactuals.}}}
Unlike the introduced noise, counterfactual perturbations do not lead to more syntax errors. The execution rate in Table~\ref{tab:appendix_k4_formalisation_exec} is stable or improves for counterfactuals. However, we see a drop in accuracy for the counterfactual column $\text{O}_C$ in Table~\ref{tab:distraction_k4_formalisation} and can conclude that the number of logical forms with semantic errors has to increase. This suggests that the introduced negation is not correctly formalised. Looking at the warnings generated by the feedback mechanism, for GPT 4o-mini, $161$ warning messages are generated on the unperturbed data. $54$ of these were fixed with a single iteration. Not considering predicates and individuals as part of the context is the most frequent warning across all models. 
\section{Related Work}
Alongside a discussion of what is meant by LLM harmfulness,
this section covers two distinct strands of related work: measuring types of harm in LLMs, and LLMs for diverse annotation tasks. %First,

%Different kinds of 
Diverse undesirable LLM outputs, from toxic language to privacy invasion, have been discussed in the observed \cite{banko-etal-2020-unified}. Here we review the ones we include in our definition of ``harm.'' %definition. Plus, we review LLMs as judges. 
Toxic content can be elicited from both generative  \cite{deshpande2023toxicity} and masked LLMs \cite{ousidhoum-etal-2021-probing}. 
%Among ways 
To measure toxic or hateful language, some use APIs such as PerspectiveAPI \cite{lees2022new} or HateBERT \cite{caselli-etal-2021-hatebert}. \citet{openai2024gpt4technicalreport} report that GPT4 produces toxic content 0.78\% of the time, versus 6.48\% in GPT3.5.
%as opposed to GPT3.5 with 6.48\%. On the other hand,
\citet{dubey2024llama} report that llama3-70B produces harmful content 5\% of the time, %whereas the 405B model generates harm 3\% of the time. 
compared to 3\% in the 405B model.
Instead of %single value classifiers to measure harm, 
reporting an absolute rate, we focus on relative harmfulness of different LLMs. %, so we point to recent work on LLMs for annotation.

The first category of harm we consider is social stereotyping and bias. %discrimination. It has been shown that 
LLMs can perpetuate social bias based on gender, race, religion etc. \cite{lin-etal-2022-gendered,bender2021dangers,field-etal-2021-survey,gupta-etal-2024-sociodemographic,andriushchenko2024agentharm,mazeika2024harmbench}. This can marginalize these groups more, and results in less fair model performance. \citet{guo2024hey} designed a competition to elicit biased output from LLMs to assess the perception of bias from non-expert users. %The first part of our work is similar to this analysis, but 
We also intentionally elicit harmful output, going %we look at other types of harms besides bias.
beyond social bias.

%When the models become stronger, they become more robust to jailbreaking attacks to elicit harmful content. However, there are datasets that can still jailbreak models to produce harmful content \cite{andriushchenko2024agentharm,mazeika2024harmbench}.

Our second category of harm is offensiveness and toxicity, which %. As opposed to stereotyping or social discrimination, this harm 
%is more subjective and harder to define than the previous category, so there 
lacks an established definition due to its greater subjectivity \cite{dev-etal-2022-measures,korre-etal-2023-harmful}. We include hate speech (HS) and abusive language as toxic content. HS can be defined as expressions of offensive and discriminatory discourse towards a group or an individual based on characteristics such as race, religion, nationality, or other group characteristics \cite{john2000hate,jahan2023systematic,basile2019semeval,davidson2017automated}. It includes racism, negative stereotyping, and sexist language. On the other hand, abusive language is content with inappropriate words such as profanity or disrespectful terms. It also includes psychological threats such as humiliation. %or constant criticism. %Toxic content can be elicited from both generative models \cite{deshpande2023toxicity} and masked language models \cite{ousidhoum-etal-2021-probing}.

%In addition to obvious toxic content, LLMs can generate diverse implicit toxic outputs using reinforcement learning with favoring toxic content in the reward function \cite{wen-etal-2023-unveiling}.  Regarding the subjectivity of this task, \cite{korre-etal-2023-harmful} reannotate the existing datasets with different definitions of toxicity and show that broader definitions result in more robust annotations, but interannotator agreements are still lower than 0.5. \cite{dev-etal-2022-measures} also point out the lack of definition for bias and harm in general and propose a framework to guide researchers during the development of bias measures.

Harm can be implicit, such as privacy invasion
%We are also interested in privacy invasion,
where there is 
leakage of personal information. %leakage from the model. 
%LLMs can memorize details of the training data and then leak private information such as 
This includes social security numbers, phone numbers, or bank account information \cite{carlini2021extracting,brown2022does}. 
%There are several frameworks to test the privacy of LLMs \cite{li2024llm} and generate data for personal attribute inference \cite{yukhymenko2024synthetic,kim2024propile}.

%Our definition of harm includes hate speech (HS) as well. HS can be defined as \textcolor{red}{expressions of} hatred towards a social group, the humiliation of the members of a group, or %communication disparaging  extreme disparagement of a person or a group based on race, color, ethnicity, gender, sexual orientation, nationality, religion, or other group characteristics .

For data annotation, LLMs
%Besides text generation, 
%LLMs have been used to annotate data because they 
can %be comparable to 
replace humans for some tasks, %and make the annotation process faster and cheaper 
with gains in efficiency and economy \cite{tan2024large}. They have been used for sociological annotations such as for classification of stance, bots or humor  \cite{ziems2024can,zhu2023can}. For tasks such as topic and frame detection or sentence segmentation they can surpass crowd-workers
%Some works show that they can surpass crowd-workers for some tasks such as topic and frame detection or sentence segmentation %into research aspects 
\cite{he2024if,gilardi2023chatgpt}. Some have argued that human-LLM collaboration results in more reliable annotation \cite{he2024if,zhang2023llmaaa,kim2024meganno+}. In addition to more objective tasks,
%LLMs have been used to annotate data %even 
they have been applied to subjective annotations such as offensiveness and abusiveness \cite{pavlovic-poesio-2024-effectiveness,zhu2023can,he2023annollm}, %. For example, LLMs are used as judges to rank responses from different LLMs 
or to rank outputs from different LLMs based on helpfulness, accuracy, or relevance \cite{zheng2023judging,lin2024wildbench,dubois2024length}. These works tend to focus on human-large LLM interactions, whereas we focus on single-turn responses from smaller LLMs. We inspire from \citet{zheng2023judging} but we only measure harm instead of overall performance. Plus, we use 3 LLMs to evaluate smaller LLMs.
\section*{Conclusion}
This paper aims to enhance our understanding of the computational complexity of computing various Shapley value variants. We found that for various ML models --- including decision trees, regression tree ensembles, weighted automata, and linear regression --- both local and global interventional and baseline SHAP can be computed in polynomial time under HMM modeled distributions. This extends popular algorithms, such as TreeSHAP, beyond their empirical distributional scope. We also establish strict complexity gaps between the various SHAP variants (baseline, interventional, and conditional) and prove the intractability of computing SHAP for tree ensembles and neural networks in simplified scenarios. Overall, we present SHAP as a versatile framework whose complexity depends on four key factors: \begin{inparaenum}[(i)] \item model type, \item SHAP variant, \item distribution modeling approach, \item and local vs. global explanations\end{inparaenum}. We believe this perspective provides deeper insight into the computational complexity of SHAP, paving the way for future work.




%We believe that our framework provides a more intricate understanding of SHAP computation complexity across different models, distributions, and variants, paving the way for further research.

Our work opens promising directions for future research. First, expanding our computational analysis to other SHAP-related metrics, such as asymmetric SHAP~\citep{frye20} and SAGE~\citep{covert2020understanding}, would be valuable. Additionally, we aim to explore more expressive distribution classes and relaxed assumptions beyond those in Section \ref{sec:tractable} while maintaining tractable SHAP computation. Finally, when exact computation is intractable (Section \ref{sec:intractable}), investigating the approximability of SHAP metrics through approximation and parameterized complexity theory~\citep{downey2012parameterized} is an important direction.

%Our work opens several promising avenues for future research on the computational properties of explainable AI methods, with a particular focus on SHAP. First, it would be interesting to broaden the computational analysis conducted in this work to include other popular SHAP-related metrics in the literature, such as asymmetric SHAP \cite{frye20} and SAGE \cite{covert2020understanding}. Also, in the future, we aim to explore more expressive distribution classes and relaxed distributional assumptions—extending beyond those examined in Section \ref{sec:tractable} —that still yield tractable SHAP computation. Finally, when exact computation proves intractable (Section \ref{sec:intractable}), it is worthwhile to theoretically investigate the question of the approximability of computing the SHAP metrics across various configurations, through the lens of approximation and parametrized complexity theory \cite{arora2009computational}.

%This paper aims to deepen our understanding of the computational complexity involved in obtaining different Shapley value variants. We found that for a variety of ML models, including decision trees, tree ensembles for regression, weighted automata, and linear regression models — computing both local and global interventional and baseline SHAP can be done in polynomial time when distributions are modeled by HMMs. This extends the distributional scope of popular algorithms like TreeSHAP, which is limited to empirical distributions. Additionally, we demonstrate a strict complexity gap between SHAP variants, showing that interventional and baseline SHAP can be strictly easier to compute than conditional SHAP. Despite these positive results, we uncovered intractability for various SHAP variants in neural networks and tree ensembles. Finally, we provided generalized complexity relations across SHAP variants. We believe that our framework offers a deeper understanding of the complexity involved in computing SHAP across various variants, models, distributions, as well as in both local and global computations, laying the groundwork for future research.

\section*{Acknowledgments}
This work is supported in part by the National Science Foundation under award numbers 1900654, 1956393, 2339266, and a gift from the UC Noyce Foundation. We would like to thank Muhammad Haroon, Shraddha Hardikar, and Luca Baldesi, for their preliminary assistance and feedback on how \autolike{} can be implemented and evaluated. 
 



\bibliographystyle{ACM-Reference-Format}
\bibliography{master,online}


\appendix
\section{Ethical Considerations}
\label{app:ethics}
In this work, we discuss sensitive topics and issues, such as mental health, eating disorders, and self-harm. All of the experiments in Sec.~\ref{sec:autolike-prelim-results} were automated and did not involve any real human users. Manual evaluation of sensitive materials was completed by the authors, who provided  expressed consent and affirmed to the team that they were not adversely affected by the review process. Furthermore, our implementation and deployment of \autolike{} on TikTok aimed to reduce its impact on the platform. For instance, our implementation automatically skipped TikTok ads. In addition, the streamlined design of \autolike{} in Sec.~\ref{sec:tiktok-eval} reduced the amount of TikToks processed from potential tens of thousands down to 200 TikToks per experiment. Similarly, we designed the controlled experiment to be re-usable, which further minimized the impact on TikTok.

\end{document}
