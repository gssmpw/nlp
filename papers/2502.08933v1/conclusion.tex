\section{Conclusion \& Future Directions}
\label{sec:autolike-conclusion}

\descr{Summary.}
Recommendation systems within social platforms provide users with convenient ways to discover new and relevant content through personalization (``For You'' Pages). 
However, these algorithms can also spread misinformation and serve harmful content to users. We introduce \autolike{}, a reinforcement learning framework to audit social media recommendation systems across two dimensions, such as a topic of interest and sentiment (\eg{} ``Mental Health'', ``Sad''). It accomplishes this through automated user interactions (\eg{} liking, watching) with the content, while learning the most efficient way to drive the algorithm to serve content related to the content of interest. We apply \autolike{} on TikTok as a case study, providing a possible implementation of \autolike{} for Android devices and the mobile TikTok app. We evaluated the classification performance for TikToks and demonstrated that a streamlined version of \autolike{} can drive TikTok's algorithm to serve content related to the given topic of interest and sentiment. 



\descr{Future Directions.}
There are many directions for extensions and applications.
First, in the TikTok case study, we could evaluate additional actions beyond just liking and skipping. %
Second, we can apply the \autolike{} framework to other social media platforms and \fyp{} (\eg{} Instagram reels, YouTube shorts, \etc). We can also focus on special types of users, such as children and adolescents.
Third, we can extend the RL framework to: (1) add additional dimensions in the state, \eg{} truthfulness, intent, \etc; (2) combine the implicit expression of interest in a type of content via user interactions with explicitly declared interests and user attributes (\eg  users can declare interests through the setting menus). 
To enable such extensions, we plan to release the software of \autolike{} for TikTok and our collected datasets in Table~\ref{tab:tiktok-prelim-datasets}. We hope that this serves as a useful starting point on which the community can build and expand.



