\section{Discussion and Concluding Remarks}
\label{sec:discussion}

Using our obfuscation detection framework, we conducted a large-scale, eight-year investigation into code obfuscation practices in the Google Play Store, analyzing more than 500,000 APKs. To the best of our knowledge, this study is the first of its kind. Finally, we discuss the implications and limitations of our findings.

\subsection{Implications}

\noindent{{\bf Adoption of code obfuscation:}} Overall, code obfuscation is on an increasing trend in the Google Play Store. More specifically, we found that the average percentage of obfuscated apps between 2016-2018 was 53\%, and that increased to 62\% in 2021-2023. These results indicate more and more developers are aware of the associated intellectual property and security issues in Google Play and are taking actions to mitigate them. However, app store administrators may want to balance obfuscation and readability since excessive code obfuscation can hinder in-build security checks in app stores, e.g., Bouncer~\cite{nawaz2022evaluation, bacci2018impact}. As a result, understanding existing obfuscation practices will allow app store admins to build policies that balance obfuscation and code understandability. \\ \vspace{-3mm}


\noindent{{\bf Use of code obfuscation tools:}} We found that Proguard is the most commonly used obfuscation tool (40.92\%), likely because it is free and the default option in Android Studio. Surprisingly, a significant fraction of apps use the commercial obfuscator Allatori (36.64\%). Additionally, we found that 21.43\% of apps use unknown obfuscation tools, which presents a potential direction for future research. This trend may be driven by developers seeking advanced protection for their intellectual property, opting for more advanced tools like DexGuard~\cite{dexguard} over more commonly used options such as ProGuard. In addition, these empirical findings are important because often malware analysts conduct code reverse engineering and must be aware of available obfuscation tools and techniques~\cite{zhang2021android, wang2017changed, kuhnel2015fast}. Our results, including details of commonly used tools and the possible existence of non-mainstream/unknown obfuscators, will interest them. \\ \vspace{-3mm}


\noindent{{\bf Obfuscation techniques:}} Our results showed that Identifier Renaming is the most common obfuscation technique, used by 99.62\% of apps. We also found that 58.7\% of apps use all three main obfuscation techniques. The use of multiple obfuscation techniques, rather than a single technique, introduces additional complexity to the obfuscated app, thereby making reverse engineering more challenging. This highlights the need for more complex de-obfuscation solutions for app analysis as a necessary future research direction. Current de-obfuscation methods, such as those proposed in~\cite{bichsel2016statistical, baumann2017anti, you2022deoptfuscator}, tend to focus on individual obfuscation techniques, which limits their scope. Given the evolving industry trends, it is essential to address more complex combinations of obfuscation techniques in future. \\ \vspace{-3mm}


\noindent{{\bf App genres:}} Our app genre-wise obfuscation analysis found that Gaming and Casino apps use obfuscation more frequently than others. This is indeed not surprising. Due to financial transactions and gambling, Casino apps strive for obfuscation. The competitive nature and vulnerability to re-packing in gaming apps also explain their higher obfuscation usage. In future, we can expect other app categories that also handle important data and transactions, such as Finance, Health and Fitness, and Medical, to adopt more obfuscation. However, it is important to note that the developers must not use obfuscation as a security solution, as many examples in the past have shown that ``security by obscurity'' doesn't work. \\ \vspace{-3mm}

\noindent{{\bf Top developers and top apps:}} Finally, we found that top developers and apps use obfuscation more frequently, often preferring commercial obfuscators for added protection. This highlights the importance of obfuscation tools for app developers, encouraging their use to protect apps and IP~\cite{faruki2016android}. Furthermore, it educates small-scale developers on best practices from top developers and can help establish industry standards for obfuscation, especially in the era of GenAI, where code data are used to train AI models, sometimes without developer consent. To a certain extent, longitudinal data demonstrate an increasing adoption of obfuscation by smaller developers, which is a positive indicator of the overall health of the Android ecosystem. \\ \vspace{-7mm}


\subsection{Limitations and Future Work}
\label{sec:limitations}
\noindent{{\bf Limitations:}} In our large-scale analysis, we leveraged two separate datasets collected in 2018 and 2023. As a result, we didn't have sufficient representative samples from 2019 and 2020. While this is a limitation of our analysis, overall trends we observed are unlikely to change even including those data.

It is also important to note that the classifiers discussed in Section~\ref{sec:classifier_bank} could have prediction errors, which may propagate into our large-scale analysis, potentially influencing the overall findings. To mitigate this, we validated our classifiers' performance on unseen data ({\bf cf.} Section~\ref{sec:training_and_validation}), demonstrating their generalizability. However, there can still be error propagation, and with real-world APKs, there is no method for accurate ground-truth establishment. 
\\

\noindent{{\bf Future Work:}} In our obfuscation detection process, we classify APKs as a whole. However, obfuscation may originate from specific libraries within the APK, whether first-party or third-party. A potential future research direction is to examine which parts of the app are obfuscated. Conducting such an analysis on real-world data without ground truth is challenging, as it becomes difficult to identify the exact library once it has been obfuscated. This challenge is commonly encountered in third-party library detection research in Android~\cite{wang2018orlis, zhan2020automated, zhang2019libid, wu2023libscan}.

Additionally, our analysis focused on three commonly used and accessible tools. Some apps may employ DexGuard ({\bf cf.} Section~\ref{sec:commontools}), and our classifier should ideally categorize such apps as ‘Others.’ However, due to similarities between DexGuard and ProGuard, apps using DexGuard might be mistakenly classified as using ProGuard. A possible future direction is to extend this research by incorporating additional obfuscators, including commercially licensed tools like DexGuard, to improve the classifier’s capability and address this limitation. 
