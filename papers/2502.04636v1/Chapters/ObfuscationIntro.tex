\section{Background}
\label{sec:Obfuscation Intro}

\subsection{Common Obfuscation Techniques}
\label{sec:obfustexchniques}

Obfuscation systematically converts the source code of the program into a form that is beyond human readability. This transformation maintains the application's functionality unchanged while altering the program's code. Several previous works studied and categorized obfuscation techniques~\cite{zhang2021android,conti2022obfuscation,guo2022survey,dong2018understanding,wermke2018large}. In the following, we describe some of the well-known obfuscation techniques used in Android apps.

\subsubsection{Identifier Renaming (IR)}

In \textit{Identifier Renaming}, identifiers in the code (e.g., class names, method names, and field names) are substituted with random characters or strings. This aims to make the code less readable by obfuscating readable information without changing the program logic. 

\subsubsection{Control Flow Modification (CF)}
The primary concept behind Control Flow Modification is to change the sequence of program execution, making it more difficult to understand and analyze. This technique is commonly used to protect software from reverse engineering and tampering. In ~\cite{guo2022survey} and ~\cite{zhang2021android}, various methods for achieving CF are discussed, and we outlined the popular techniques below.

\begin{itemize}
    \item \textbf{Control flow flattening} incorporates a construct that may include an infinite or finite loop with a termination condition. Within this construct, individual basic blocks are encapsulated as cases of a switch statement. While the original basic block is executed during runtime, the process of decompiling the switch case statement and restoring the initial code is challenging, due to the convoluted `\texttt{if}' and `\texttt{goto}' statements.
    
    \item \textbf{Call indirection} involves creating a new method to invoke the original method. Within the course of code execution, each method call is shadowed by this intermediary method, which, in turn, invokes the original method. It introduces complexity in the process of code restoration and impairs readability~\cite{zhang2021android,bacci2018a}. 
    
    \item \textbf{Reflection} is a technique in Java to alter the runtime behaviour of a program dynamically. It primarily leverages the \texttt{Java.lang.reflect.*} API, an integral component of the native Java library, to access and manipulate methods during program execution. 
    
    \item \textbf{Other} obfuscation methods involve techniques that sometimes overlap with the earlier methods, and as such, delineating boundaries between these techniques is challenging. For example, ~\cite{li2019obfusifier} introduces adding \texttt{`nop'} instructions and unconditional jumps. This is known as \textit{junk code insertion}. Moreover, developers can employ \textit{opaque predicates}, such as conditional statements or branches, to create a simulated branch~\cite{guo2022survey}, which constitutes a \textit{bogus control flow}. This practice generates two branches yielding the same outcome, with one branch containing the original code and the other comprising unreachable junk instructions. 
\end{itemize}
  
\subsubsection{String Encryption (SE)} Storing sensitive information or identifiable prompts in plain text strings within the source code may render the application vulnerable to third-party examination and reverse engineering. To avoid that, \textit{String Encryption} transforms human-readable strings within the code into human-unreadable character sequences. 


\subsection{Commonly used Obfuscation Tools}
\label{sec:commontools}

Developers usually resort to tools to obfuscate code. Previous works~\cite{wang2017changed, dong2018understanding, mirzaei2019androdet, park2019framework} have reported multiple code obfuscation tools of various kinds, as we describe below.

\begin{itemize}

\item {\textbf{ProGuard}}~\cite{proguard} is an inbuilt and free obfuscator for Android Studio by GuardSquare. It can be easily activated by adding ProGuard rules in the \texttt{build.gradle} file. ProGuard can perform only Identifier Renaming and Code Optimization as specified in the user guide.


\item \textbf{Allatori}~\cite{allatori} is a commercial obfuscator by Smardec Inc. It is  offered as both paid and free educational versions with equal functionality. Integrating Allatori into an Android project is similar to ProGuard. However, it requires adding the Allatori \texttt{jar} file and configuration file in the \texttt{build.gradle}. It supports all three main obfuscation techniques discussed in Section~\ref{sec:obfustexchniques}.

    
\item \textbf{DashO}~\cite{dasho} is another commercial obfuscator by PreEmptive Inc. It is a paid tool, with the possibility of requesting a 7-day evaluation licence. Developers can use DashO UI to open source code files and enable necessary configurations. DashO UI will then add the required settings to the \texttt{build.gradle} file. It supports all three main obfuscation techniques described earlier.


\item \textbf{Obfuscapk}~\cite{obfuscapk} was initially developed as an open-source obfuscation tool for researchers to obfuscate Android applications. It implements all three techniques discussed in Section~\ref{sec:Obfuscation Intro}. As a validation dataset, we used the AndroOBFS dataset~\cite{androobfs}, which was obfuscated using ObfuscAPK. Further details on this dataset and its role in validating our method's performance will be discussed in Section~\ref{sec:building_dataset}.


\item \textbf{DexGuard}~\cite{dexguard} is an advanced paid version of ProGuard, also provided by GuardSquare. DexGuard implements all three techniques discussed earlier and also supports Runtime Application Self-Protection (RASP) for app hardening. We were unable to obtain a free or evaluation version. Therefore, we do not use it when building the training set later.

\end{itemize}

Table~\ref{tab:ob_tool_cap} summarizes the features of these obfuscation tools.


\begin{table}[h]
\caption{Summary of Android obfuscation tools}
\label{tab:ob_tool_cap}
\resizebox{\columnwidth}{!}{%
\begin{tabular}{lccc}
\hline
\textbf{Tool} & \textbf{\begin{tabular}[c]{@{}c@{}}Identifier\\ Renaming\end{tabular}} & \textbf{\begin{tabular}[c]{@{}c@{}}Control Flow\\ Modification\end{tabular}} & \textbf{\begin{tabular}[c]{@{}c@{}}String\\ Encryption\end{tabular}} \\ \hline
{ProGuard} & Yes & No & No \\
{DashO} & Yes & Yes & Yes \\
{Allatori} & Yes & Yes & Yes \\
{ObfuscAPK} & Yes & Yes & Yes \\
{DexGuard} & Yes & Yes & Yes \\ \hline
\end{tabular}%
}
\end{table}
