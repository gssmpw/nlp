\section{Related Work}
\label{Sec:Related Work}

\subsection{Obfuscation Detection}

Multiple works developed methods to detect obfuscation and identify the tools and techniques used. For instance, Kühnel et al.~\cite{kuhnel2015fast} introduced the IREA framework, employing a rule-based detection algorithm tailored to identify obfuscation techniques such as Identifier Renaming. Despite achieving high accuracy, this approach's reliance on specific rules limits its applicability across diverse scenarios. 

Similarly, Wermke et al.~\cite{wermke2018large} developed OBFUSCAN to emulate ProGuard's behaviour, focusing primarily on detection techniques aligned with ProGuard's functionalities. However, this specialization restricts its performance with other obfuscation tools. In contrast, Wang et al.~\cite{wang2017changed} presented a multi-class classifier utilizing SVM to identify obfuscation tools and configurations, demonstrating promising accuracy levels. Nonetheless, the limited dataset used for validation raises concerns regarding its generalizability. Dong et al.~\cite{dong2018understanding} and Park et al.~\cite{park2019framework} proposed machine learning-based methods to detect obfuscation techniques. AndrODet~\cite{mirzaei2019androdet} and AndrODet*~\cite{conti2022obfuscation} also had similar approaches.

In our work, we draw ideas from these machine learning-based obfuscation detectors and design a comprehensive framework that facilitates large-scale analysis of APKs. Also, in contrast to these works, we train and test our classifiers in a diverse set of datasets to increase the generalizability that is required for real-world settings.

\subsection{Empirical Studies of Obfuscation}
OBFUSCAN~\cite{wermke2018large} is the most related to our work which examined obfuscation usage in the Google Play Store and developer awareness through a survey with 1.7 million apps from 2010 to 2017, primarily detecting Identifier Renaming. 
Similarly, Dong et al.~\cite{dong2018understanding} introduced an obfuscation detection methodology and conducted a large-scale investigation using 26k Google Play Store apps and 65k 3rd Party Apps from 2016 to 2017. However, the authors mainly focus their study on obfuscation techniques.

Another study~\cite{wang2018software} introduced an NLP-based obfuscation detection tool focusing on symbol renaming and conducted a large-scale empirical study on the iPhone Operating System. Hammad et al.~\cite{hammad2018large} discussed the effects of code obfuscations in Android apps, evaluating commercial anti-malware products against various obfuscation tools and strategies. Additionally, Kargén et al.~\cite{kargen2023characterizing} used anomaly detection followed by manual inspection to investigate popular obfuscation techniques among Malware and Google Play Store APKs, mainly focusing on Control Flow obfuscation with Java Reflections using 13k apps released in 2020.

\textit{In contrast to these works, our research is the largest of its kind, covers a span of eight years, allowing us to observe trends, and covers a broader scope of code obfuscation practices simultaneously}.
