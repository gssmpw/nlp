\section{Introduction}
\label{Sec:Introduction}

\IEEEPARstart{A}{ndroid} plays a vital role in the smartphone market, holding more than 70\% market share~\cite{StatCounter_Global_Stats}. Within that, the Google Play Store serves as the primary app repository, offering over 1.7 million apps as of August 2024~\cite{a2024_google_number}. The Google Play Store is highly accessible, allowing developers to publish and monetize their apps with fewer barriers to entry.

Due to the relative ease of publishing apps, various malpractices are common in the Google Play Store~\cite{GARG2021102087}. For example, some malicious authors may repackage or counterfeit legitimate apps for nefarious purposes~\cite{zhou2012detecting, bhat2019survey}. Similarly, some malicious parties may steal intellectual property (IP) by reverse engineering apps~\cite{albakri2022survey}. These malpractices pose a significant threat to legitimate app developers and end users. As a countermeasure, app developers use code obfuscation to protect their apps and intellectual property (IP)~\cite{faruki2016android}. Equally, malware authors can also use obfuscation to evade anti-malware tools and to conceal functionality~\cite{elsersy2022rise, SIHAG2021100365}.

Although obfuscation provides security benefits, it also hinders reverse engineering, which is essential for static analysis by app investigators~\cite{molina2025light, beer2024tabbed, tan2023ptpdroid}. This poses challenges for malware analysts and app store administrators in enforcing security policies, as obfuscation can bypass anti-malware mechanisms and app store regulations~\cite{elsersy2022rise, SIHAG2021100365}. Additionally, obfuscation introduces performance degradations and limitations in Android research\cite{tan2023ptpdroid, gao2024comprehensive, pradeep2022not}. While some studies propose obfuscation-resilient research methods~\cite{liu2023enhancing, wang2022malwhiteout, wang2023uncovering}, others either overlook obfuscation or consider only basic techniques and tools\cite{li2024malcertain, liu2023no, chen2024attention}, possibly due to limited awareness of its prevalence. Given these challenges, it is important to examine the prevalence and trends of obfuscation in the Google Play Store. While some prior work has introduced methods for obfuscation detection~\cite{dong2018understanding, kuhnel2015fast, park2019framework, wang2017changed, wermke2018large}, no studies have fully examined the use of code obfuscation by app developers, the tools and techniques they employ, or the evolution of obfuscation adoption over time. 


To this end, in this paper, we investigate code obfuscation adoption and practices in the Google Play Store from different aspects to benefit various stakeholders of the app market ecosystem. For developers, our work highlights the importance of obfuscation tools and industry standards to protect apps and IP. For researchers, we provide insights into industry trends. For malware analysts and app store administrators, our work underscores the necessity of robust security measures and regulations. Our research involves developing a set of classifiers to detect obfuscated code using various tools and techniques and analyzing trends over time by conducting a longitudinal study using data from two snapshots of the Google Play Store, taken five years apart. More specifically, we make the following contributions.

\begin{itemize}
\IEEEpubidadjcol
    \item We propose a bank of classifiers to detect whether an app is obfuscated and, if so, identify the obfuscation tools and techniques used. On our test set, we achieved 97\% accuracy in detecting obfuscation, 99\% accuracy in identifying the tool used for obfuscation, and 88\% accuracy in identifying the obfuscation technique. 
   
    \item Using these classifiers, we conduct a longitudinal study spanning eight years, from 2016 to 2023, to understand how code obfuscation practices have evolved in the Google Play Store. To the best of our knowledge, this is the first large-scale study of its kind to analyse over half a million Android applications.
    
    \item We show that overall code obfuscation in the Google Play Store increased by nearly 13\% from 2016 to 2023. We also find that ProGuard~\cite{proguard} and Allatori~\cite{allatori} are the two most commonly used tools by developers. Gaming apps tend to use obfuscation more than non-gaming apps, with Casino games showing the highest prevalence, at 80\% of apps obfuscated, and over 85\% using multiple techniques.

    \item We report a 28\% increase in obfuscation among top developers from 2018 to 2023 and an 11.7\% increase among developers with only one app. Over 90\% of the top 1,000 apps are obfuscated, with higher-ranked apps using multiple obfuscation techniques more frequently than lower-ranked ones. We further report that ProGuard is the most commonly used tool among lower-ranked apps.

\end{itemize}

The rest of the paper is organized as follows. In Section~\ref{sec:Obfuscation Intro}, we present background information such as common obfuscation tools and techniques. Section~\ref{sec:Methodology} details our obfuscation detection framework, and Section~\ref{sec:large-scale-analysis} describes our large-scale dataset. We present our findings on obfuscation trends in the Google Play Store in Section~\ref{sec:Results}. Section~\ref{Sec:Related Work} reviews related work, while Section~\ref{sec:discussion} discusses the implications and the limitations of our work and concludes the paper.