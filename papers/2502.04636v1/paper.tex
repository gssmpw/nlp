\documentclass[lettersize,journal,anonymous]{IEEEtran}
\usepackage{amsmath,amsfonts}
\usepackage{algorithmic}
\usepackage{algorithm}
\usepackage{array}
\usepackage[caption=false,font=normalsize,labelfont=sf,textfont=sf]{subfig}
\usepackage{textcomp}
\usepackage{stfloats}
\usepackage{url}
\usepackage{verbatim}
\usepackage{cite}
\usepackage{graphicx}
\usepackage{multirow}
\usepackage{xcolor}
\usepackage{hyperref}
\hyphenation{op-tical net-works semi-conduc-tor IEEE-Xplore}

\begin{document}

\title{An Empirical Study of Code Obfuscation Practices in the Google Play Store\textsuperscript{\dag{}}}

\author{Akila Niroshan,  
        Suranga Seneviratne,  
        Aruna Seneviratne
        
\thanks{\dag{} This is an extension of our previous work~\cite{akila2025stateofobfuscation} published in the ACM/SIGAPP Symposium on Applied Computing (SAC), 2025.}

\thanks{Akila Niroshan and Aruna Seneviratne are with the University of New South Wales (UNSW), Sydney, Australia (e-mail: a.pothpitiyage\_don@unsw.edu.au, a.seneviratne@unsw.edu.au).}
\thanks{Suranga Seneviratne is with the University of Sydney, Sydney, Australia (e-mail: suranga.seneviratne@sydney.edu.au).}

}

\IEEEpubid{\begin{minipage}{\textwidth}\centering
This work has been submitted to the IEEE for possible publication. Copyright~\copyright~ may be transferred without notice, after which this version may no longer be accessible.
\end{minipage}}
\IEEEpubidadjcol

\maketitle

\begin{abstract}
The Android ecosystem is vulnerable to issues such as app repackaging, counterfeiting, and piracy, threatening both developers and users. To mitigate these risks, developers often employ code obfuscation techniques. However, while effective in protecting legitimate applications, obfuscation also hinders security investigations as it is often exploited for malicious purposes. As such, it is important to understand code obfuscation practices in Android apps. In this paper, we analyze over 500,000 Android APKs from Google Play, spanning an eight-year period, to investigate the evolution and prevalence of code obfuscation techniques. First, we propose a set of classifiers to detect obfuscated code, tools, and techniques and then conduct a longitudinal analysis to identify trends. Our results show a 13\% increase in obfuscation from 2016 to 2023, with ProGuard and Allatori as the most commonly used tools. We also show that obfuscation is more prevalent in top-ranked apps and gaming genres such as Casino apps. To our knowledge, this is the first large-scale study of obfuscation adoption in the Google Play Store, providing insights for developers and security analysts.
\end{abstract}



\begin{IEEEkeywords}
Android, Code Obfuscation, Mobile Apps, App Store Mining.
\end{IEEEkeywords}



% 
% 
The widespread integration of communication networks and smart devices in modern control systems has increased the vulnerability of industrial systems to online cyber-attacks, e.g., Industroyer, Blackenergy, etc \citep{osti_1505628}.
% Modern control systems have seen a large push to include communication networks and smart devices to increase performance, made possible by improvements in communication device cost and energy consumption. This trend has been coupled with the usage of open-standard communication protocols among industrial control systems, making them vulnerable to online cyber-attacks such as Industroyer, Blackenergy, etc \citep{osti_1505628}. 
To counter this, methods have been developed to improve security by achieving attack detection, mitigation, and monitoring, among others \citep{sandberg2022secure}. This paper focuses on active attack diagnosis to mitigate stealthy attacks. 
%
%\subsection{Literature review}

Active diagnosis techniques rely on the inclusion of additional moduli to control systems
% inclusion within the control system of additional moduli 
to alter the behavior of the system compared to information known by the attacker. 
For instance, the concept of additive watermarking was introduced in \cite{mo2015physical}, where noise signals of known mean and variance are added at the plant and compensated for it at the controller. 
This compensation, however, is not exact, causing some performance degradation. Thus, trade-offs between performance and detectability  are necessary \citep{zhu2023detection}.
% A later work \citep{zhu2023detection} designs the watermark signal by trading performance for detection. Thus, although additive watermarking serves as a good detection scheme, they endure performance losses even in the nominal case. 

In encrypted control \citep{darup2021encrypted}, the sensor data is encrypted, sent to the controller, and then operated on directly. Encrypted input signals are sent back to the plant for decryption. Although encryption is widespread in IT security, in control systems it presents some concerns, such as the introduction of time delays \citep{stabile2024verifiable}, while it may present inherent weaknesses \citep{alisic2023model}.
% they are not preferred as they introduce time delays \citep{stabile2024verifiable} which can cause instability, and some encryption schemes can be very weak  \citep{alisic2023model}. 

In moving target defense \citep{griffioen2020moving}, the plant is augmented with fictitious dynamics, known to the controller. The plant output is transmitted to the controller along with the fictitious states over a network under attack. 
The additional measurements then aide in the detection of attacks. 
This comes at the cost of higher communication bandwidth needs, which increases rapidly with the dimension of the augmented systems.
% Since the dynamics of the fictitious dynamics are exactly known to the controller, the attack is detected easily. However, when the scale of the system increases, the communication bandwidth used by moving the target defense approach increases rapidly. 

Other recently proposed works include two-way coding \citep{fang2019two}, a weak encryuption technique, and dynamic masking \citep{abdalmoaty2023privacy}, which enhances privacy as well as security, have been shown to be effective against zero-dynamics attacks.
% Two-way coding \citep{fang2019two} and dynamic masking \citep{abdalmoaty2023privacy} are other recently proposed approaches. Two-way coding is another form of weak encryption technique whilst dynamic masking proposes an architecture that enhances both privacy and security. These schemes are shown to be effective against zero dynamics attacks but remain to be studied for other classes of attacks. 
% Recent extensions include \citep{mukherjee2021secure,ramos2024privacy}.
% Some other works which are related are \citep{mukherjee2021secure}, an extension of \cite{fang2019two}. The work \citep{ramos2024privacy} is an extension of moving target defense for multi-agent systems. 
Furthermore, filtering techniques for attack detection are proposed by \cite{murguia2020security,hashemi2022codesign,escudero2023safety}, while not focusing on stealthy attacks.
% The works \citep{murguia2020security,hashemi2022codesign,escudero2023safety} develop filtering techniques to guarantee safety, without being focused on stealthy covert attacks.

Multiplicative watermarking (mWM) has been proposed by the authors as a diagnosis technique \citep{ferrari2020switching}. mWM consists of a pair of filters on each communication channel between the plant and its controller; the scheme is affine to weak encryption, whereby ``encoding'' and ``decoding'' are done by changing signals' dynamic characteristics through inverse pairs of filters. This enables original signals to be recovered exactly, and thus does not lead to performance degradation.
% A multiplicative watermark is an affine to a weak encryption technique, through which the signal is ``encoded'' by a filter, changing its dynamic behavior. The use of inverse pairs means that the original signal can be recovered, through ``decoding'' via an inverse filter. As such, differently to techniques based on additive watermarking, no performance is lost due to the injection of noise, and there are no bandwidth limitations.

%\subsection{Contributions}
One of the critical features of multiplicative watermarking is that to detect stealthy attacks, the mWM filter parameters must be switched over time. In this paper, an algorithm to optimally design the mWM parameters after a switching event is presented, enhancing detection performance, without changing the switching time.
% This is done without changing the switching time, which is taken as given.

\textcolor{black}{
To formalize the filter design problem, we suppose the defender is interested in optimal performance against adversaries injecting covert attacks with matched system parameters \citep{smith2015covert}, including the mWM parameters prior to the switch. This scenario represents a worst case where malicious agents can take full control of the system while remaining undetected.
Thus, the attack strategy is explicitly included within the formulation of the closed-loop system, and the mWM filters are chosen by solving an optimization problem minimizing the attack-energy-constrained output-to-output gain (AEC-OOG) \citep{anand2023risk}, a variation of the output-to-output gain proposed in  \cite{teixeira2015strategic}.
}
The main contributions of this paper are:
% We consider an adversary injecting a covert attack with matched system parameters \citep{smith2015covert}, i.e., an attacker with full knowledge of the control system parameters, including those of the mWM filters before the switch. This scenario is taken as a worst case, as it has been shown that this class of attacks can be made stealthy. To quantitatively define a cost, the output-to-output gain (OOG) \citep{teixeira2015strategic} is leveraged,
% a metric introduced to evaluate the impact of an additive attack in a control system. %Specifically, OOG evaluates the worst-case performance loss that an attacker injecting an undetectable attack can obtain. 
% Here, the maximum performance loss caused by a stealthy adversary with limited energy is taken, the attack-energy-constrained OOG (AEC-OOG) \citep{anand2023risk}. The main contributions of this paper are:
\begin{enumerate}
%[label=\alph*.]
\item The problem of optimally designing the switching mWM filters is formulated as an optimization problem, with the AEC-OOG is taken as the objective;%where the AEC-OOG is taken as the impact metric; 
\item The worst-case scenario of a covert attack with exact knowledge of plant and mWM filter parameters is embedded within the design problem;
% The optimization problem is defined to incorporate the worst-case scenario of a covert attack with exact knowledge of plant and mWM filter parameters;
\item The feasibility of the optimization problem is shown to be dependent only on stability conditions; 
\item A solution scheme is proposed to promote randomization of the mWM filter parameters such that an eavesdropping adversary cannot remain stealthy.
\end{enumerate} 

This builds on the results of \cite{ferrari2020switching}, where the focus was on the design of the switching protocols, rather than the parameters themselves.
Compared to previous work \citep{gallo2021design}, this paper introduces an optimization problem which is always feasible (thanks to the use of AEC-OOG in the objective), while also considering a more sophisticated class of covert attacks, where the presence of watermark is known to the adversary. 
Moreover, this paper poses a different objective than \citep{zhang2023hybrid}; indeed, while \citep{zhang2023hybrid} provided a design strategy to ensure certain privacy properties, in this paper we address the problem of optimal parameter design following a switching event.


%\subsection{Organization}
The rest of the paper is organized as follows. 
After formulating the problem in Section~\ref{sec:PF}, we propose our design algorithm in Section~\ref{sec:main}, and analyze its properties. It is then evaluated through a numerical example in Section~\ref{sec:NE}, and concluding remarks are given Section~\ref{sec:Con}.
% We provide the problem background in Section~\ref{sec:PF}. We formulate the design problem in Section~\ref{sec:main}, together with an analysis of its properties. The proposed algorithm is evaluated through a numerical example in Section \ref{sec:NE}. Concluding remarks are offered in Section \ref{sec:Con}.

\section{Background}
\label{sec:Obfuscation Intro}

\subsection{Common Obfuscation Techniques}
\label{sec:obfustexchniques}

Obfuscation systematically converts the source code of the program into a form that is beyond human readability. This transformation maintains the application's functionality unchanged while altering the program's code. Several previous works studied and categorized obfuscation techniques~\cite{zhang2021android,conti2022obfuscation,guo2022survey,dong2018understanding,wermke2018large}. In the following, we describe some of the well-known obfuscation techniques used in Android apps.

\subsubsection{Identifier Renaming (IR)}

In \textit{Identifier Renaming}, identifiers in the code (e.g., class names, method names, and field names) are substituted with random characters or strings. This aims to make the code less readable by obfuscating readable information without changing the program logic. 

\subsubsection{Control Flow Modification (CF)}
The primary concept behind Control Flow Modification is to change the sequence of program execution, making it more difficult to understand and analyze. This technique is commonly used to protect software from reverse engineering and tampering. In ~\cite{guo2022survey} and ~\cite{zhang2021android}, various methods for achieving CF are discussed, and we outlined the popular techniques below.

\begin{itemize}
    \item \textbf{Control flow flattening} incorporates a construct that may include an infinite or finite loop with a termination condition. Within this construct, individual basic blocks are encapsulated as cases of a switch statement. While the original basic block is executed during runtime, the process of decompiling the switch case statement and restoring the initial code is challenging, due to the convoluted `\texttt{if}' and `\texttt{goto}' statements.
    
    \item \textbf{Call indirection} involves creating a new method to invoke the original method. Within the course of code execution, each method call is shadowed by this intermediary method, which, in turn, invokes the original method. It introduces complexity in the process of code restoration and impairs readability~\cite{zhang2021android,bacci2018a}. 
    
    \item \textbf{Reflection} is a technique in Java to alter the runtime behaviour of a program dynamically. It primarily leverages the \texttt{Java.lang.reflect.*} API, an integral component of the native Java library, to access and manipulate methods during program execution. 
    
    \item \textbf{Other} obfuscation methods involve techniques that sometimes overlap with the earlier methods, and as such, delineating boundaries between these techniques is challenging. For example, ~\cite{li2019obfusifier} introduces adding \texttt{`nop'} instructions and unconditional jumps. This is known as \textit{junk code insertion}. Moreover, developers can employ \textit{opaque predicates}, such as conditional statements or branches, to create a simulated branch~\cite{guo2022survey}, which constitutes a \textit{bogus control flow}. This practice generates two branches yielding the same outcome, with one branch containing the original code and the other comprising unreachable junk instructions. 
\end{itemize}
  
\subsubsection{String Encryption (SE)} Storing sensitive information or identifiable prompts in plain text strings within the source code may render the application vulnerable to third-party examination and reverse engineering. To avoid that, \textit{String Encryption} transforms human-readable strings within the code into human-unreadable character sequences. 


\subsection{Commonly used Obfuscation Tools}
\label{sec:commontools}

Developers usually resort to tools to obfuscate code. Previous works~\cite{wang2017changed, dong2018understanding, mirzaei2019androdet, park2019framework} have reported multiple code obfuscation tools of various kinds, as we describe below.

\begin{itemize}

\item {\textbf{ProGuard}}~\cite{proguard} is an inbuilt and free obfuscator for Android Studio by GuardSquare. It can be easily activated by adding ProGuard rules in the \texttt{build.gradle} file. ProGuard can perform only Identifier Renaming and Code Optimization as specified in the user guide.


\item \textbf{Allatori}~\cite{allatori} is a commercial obfuscator by Smardec Inc. It is  offered as both paid and free educational versions with equal functionality. Integrating Allatori into an Android project is similar to ProGuard. However, it requires adding the Allatori \texttt{jar} file and configuration file in the \texttt{build.gradle}. It supports all three main obfuscation techniques discussed in Section~\ref{sec:obfustexchniques}.

    
\item \textbf{DashO}~\cite{dasho} is another commercial obfuscator by PreEmptive Inc. It is a paid tool, with the possibility of requesting a 7-day evaluation licence. Developers can use DashO UI to open source code files and enable necessary configurations. DashO UI will then add the required settings to the \texttt{build.gradle} file. It supports all three main obfuscation techniques described earlier.


\item \textbf{Obfuscapk}~\cite{obfuscapk} was initially developed as an open-source obfuscation tool for researchers to obfuscate Android applications. It implements all three techniques discussed in Section~\ref{sec:Obfuscation Intro}. As a validation dataset, we used the AndroOBFS dataset~\cite{androobfs}, which was obfuscated using ObfuscAPK. Further details on this dataset and its role in validating our method's performance will be discussed in Section~\ref{sec:building_dataset}.


\item \textbf{DexGuard}~\cite{dexguard} is an advanced paid version of ProGuard, also provided by GuardSquare. DexGuard implements all three techniques discussed earlier and also supports Runtime Application Self-Protection (RASP) for app hardening. We were unable to obtain a free or evaluation version. Therefore, we do not use it when building the training set later.

\end{itemize}

Table~\ref{tab:ob_tool_cap} summarizes the features of these obfuscation tools.


\begin{table}[h]
\caption{Summary of Android obfuscation tools}
\label{tab:ob_tool_cap}
\resizebox{\columnwidth}{!}{%
\begin{tabular}{lccc}
\hline
\textbf{Tool} & \textbf{\begin{tabular}[c]{@{}c@{}}Identifier\\ Renaming\end{tabular}} & \textbf{\begin{tabular}[c]{@{}c@{}}Control Flow\\ Modification\end{tabular}} & \textbf{\begin{tabular}[c]{@{}c@{}}String\\ Encryption\end{tabular}} \\ \hline
{ProGuard} & Yes & No & No \\
{DashO} & Yes & Yes & Yes \\
{Allatori} & Yes & Yes & Yes \\
{ObfuscAPK} & Yes & Yes & Yes \\
{DexGuard} & Yes & Yes & Yes \\ \hline
\end{tabular}%
}
\end{table}

\section{Methodology}
In this section, we outline the key research questions driving this study, followed by a detailed description of the methodology used to design and conduct the survey.
\subsection{Research Questions}
\begin{enumerate}
    \item[\textbf{RQ1:}] How do developers allocate their time during a typical workweek, and how does this compare to their perception of an \textbf{ideal workweek?}
    \item[\textbf{RQ2:}] How are developer's satisfaction and productivity affected by \textbf{deviations} from their ideal workweek?
     \item[\textbf{RQ3:}] For which tasks do developers prefer using \textbf{AI tools}, and how does the frequency of AI tool usage \textbf{influence} their satisfaction and productivity?
\end{enumerate}

\subsection{Survey Design}
% Describe how the survey was conducted, survey structure, sample size, which activities were selected and how, incentives, etc. 

To gain insights into the types of activities developers engage in during a typical work week, we conducted a series of exploratory interviews with 12 randomly selected participants. These semi-structured interviews provided a qualitative foundation, allowing us to iteratively develop a comprehensive list of higher-level activities that reflect both ideal and actual workweek allocations. The findings from these interviews were instrumental in refining our survey questions and design.

% - When was it distributed
% - How many people were invited
% - how was the survey advertised
% - incentive provided to participants
% - how many responses received (with response rates)
% - Board of ethics description \& instruments
% - Describe the main questions asked in the survey

The survey was distributed in \textcolor{blue}{May 2024} to software engineers working in Microsoft teams across India and the United States. A total of 6000 developers were invited to participate via email. Framed as a study aimed at boosting developer productivity by understanding how they allocate their time in a workday, the survey received 510 complete responses (responses rate of 8.5\%). After finishing the survey, the participants could enter a sweepstake to win one out of ten \$50 Amazon.com Gift Cards.
\textcolor{blue}{description of ethics}.

The main questions in the survey were as follows:
\begin{enumerate}
    \item Their roles and years of experience in the industry/team
    \item The hours spent on various activities in their typical workweek
    \item Ideally, the percentage of time they would want to allocate to each activity in a workweek
    \item How productive and satisfied were they by their past workweek
    \item Activities they find most cognitively challenging
    \item How often do they use AI tools to assist in their daily activities
    \item Two open-ended questions about the activities they would want to automate using AI tools, and advice for new hires to boost their productivity and satisfaction levels 
\end{enumerate}



\subsection{Data Analysis \& Exploration}
% Here, we could start with discussing the survey group:
% - demographic observations
% - distribution of participants (based on the years experience in the industry/team), 

From the exploratory interviews, we identified sixteen key activities, which were subsequently used to quantify the developers' time allocation across their work week. 

\subsection{Limitations}
\section{Large Scale Analysis}
\label{sec:large-scale-analysis}

\subsection{Dataset}
\label{subsec:dataset}

\begin{table*}[h]
\tiny
\caption{Number of APKs per year}
\label{tab:large-scale_dataset}
\resizebox{\linewidth}{!}{%
\begin{tabular}{c|c|c|c|c|c|c|c|c|c}
\hline
\textbf{Year} & 2016 & 2017 & 2018 & 2019 & 2020 & 2021 & 2022 & 2023 & \textbf{Total} \\ \hline
\textbf{\begin{tabular}[c]{@{}c@{}} Available APKs\end{tabular}} & 174,136 & 501,865 & 157,613 & 7,205 & 21,014 & 60,705 & 240,775 & 65,697 & 1,229,010 \\ \hline
\textbf{Analysed APKs} & 74,817 & 159,639 & 80,112 & 7,201 & 20,982 & 59,539 & 81,134 & 65,543 & 548,967 \\ \hline
\end{tabular}%
}
\end{table*}

Our large-scale analysis data is based on two large snapshots of the Google Play Store collected around 2018 and 2023. The 2018 dataset that was collected as a part of our previous work~\cite{karunanayake2020multi,rajasegaran2019multi} contains metadata of over 1.2 million apps that were collected between January and March 2018 and 1,023,521 APK files. The 2023 dataset contains metadata of over one million apps and was collected between January and November and 395,396 APK files.


Both datasets were collected in the same way using a Python-based crawler. First, the crawler discovered available apps on the Google Play Store. Then, it collected app metadata (e.g., app ID, app genre, developer name, number of downloads, rating details) and APK executables for free apps. There are several reasons behind the difference between the number of apps for which we crawled metadata and the number of apps for which we downloaded the APKs. First, the APK crawler is significantly slower than the metadata crawler. Second, we do not download the APKs of paid apps. Third, some apps do not support the Android device we simulated to download APKs.

One of the fields in app metadata is the ``\texttt{last update date}''. We use this field to categorise apps by year as summarised in Table~\ref{tab:large-scale_dataset}. As can be seen, the centre years of our two crawls, i.e., 2017 and 2022, have the highest number of apps. We have a notably smaller number of apps for 2019 and 2020 because they were only collected in 2022, and only a limited number of apps have the last update date in 2019 and 2020. These apps can bias our analysis as these represent apps that have been most likely abandoned by app developers. As a result, we do not consider 2019 and 2020 in our extended analysis. For each year, we analyse a random sample of apps as listed in Table~\ref{tab:large-scale_dataset}. The reason for not analysing all apps in all the years is the time, as APK decompilation takes time. 



\subsection{Process of APK Analysis} For each APK we analyse, we use Androguard~\cite{desnos2018androguard} and our pre-processing scripts to create the feature vector described in Section~\ref{sec:features}. Next, we make a prediction using our \textit{Obfuscation Detector}. If the app is predicted as not obfuscated, we record this and stop further analysis for that app. If the APK is obfuscated, we use the same feature vector with the \textit{Obfuscation Tool Detector Bank} and \textit{Obfuscation Technique Detector Bank} to identify the tool and technique(s) used. In the Tool Detector step, if all three classifiers give a probability of less than $0.5$, we categorize the APK as using an \textit{Other} tool. Otherwise, we use the highest probability to determine the tool. In the Technique Detector, the classifier identifies the obfuscation technique \textit{(IR, CF, SE)} if its probability exceeds $0.5$. This overall process is illustrated in Figure~\ref{fig:overview_1b}.

% \begin{figure*}[htpb!]
% \label{}
% \centering

%     {{\label{ROCIowaCedar} \includegraphics[width=\textwidth/3]{figures/IowaCedar_roc.png}}}%
%     \qquad
%     {{\label{ROCIowaDesMoines} \includegraphics[width=\textwidth/3]{figures/IowaDesMoines_roc.png} }%
%   \captionsetup{justification=centering}
%   \caption{\Acf{ROC} curves for \acf{RW} Iowa (CR) and  \acf{RW} Iowa (DM) dataset. Dummy model here represents a model whose output is solely a ``no Flood'' for all pixels.}
%   \label{fig:RW_ROC_Curves}%
% \end{figure*}



\section{Results and Discussions}
\label{sec:Results}

In this section, we aim to answer three main questions. First, we want to validate our hypothesis that \ac{SYN} data is a viable proxy for \ac{RW} data when training ML models for downscaling. Secondly, we seek to assess how much more skillful ML-based downscaling is compared to classical, non-data-driven techniques, such as our baseline methods, \textit{i.e.}, thresholded bicubic and Lanczos interpolation. Finally, we would like to appraise the extent to which data-driven models like ours are transferable (in terms of usefulness) to other regions without major performance degradations.  
To assess the quality of the models, we conduct a multiple comparison test --namely the Holm-Bonferroni procedure \cite{HolmBonferroni1979} -- that is designed to control the \ac{FWER}. We notice that, with a \ac{FWER} of $10^{-3}$, all the differences in model performance are significant. The only exception to this trend was observed in \ac{RW}-GH for whom the pairwise differences between \ac{RCAN} and \ac{ESRT}, Lanczos and Bicubic were not significant with the aforementioned \ac{FWER}. 

%Finally, we aim to find out the factors influencing the transferability of our models from one region to another.

\subsection{Potential of using SYN Data for RW downscaling}

In order to evaluate the utility of synthetic data for training, we compare performances of our candidate models on both \ac{SYN} and \ac{RW} Iowa data whose results are presented in Table \ref{tab:IowaResults}. We notice that 
\textbf{(i)} For the Iowa datasets, there is a drop in performance of all the models when going from \ac{SYN} to \ac{RW} datasets, 
\textbf{(ii)} for the \ac{RW}-IA (CR) as well as \ac{RW}-IA (DM) datasets, both bicubic and Lanczos interpolation have accuracies and MCC up to 70.89\% and 0.42 respectively while the deep learning models have accuracies and MCC up to 73.34\% and 0.46 respectively, 
\textbf{(iii)} There is a roughly 6\% accuracy improvement for the \ac{SYN} data for the deep learning models compared to the bicubic and lanczos models and this improvement drops to about 3\% for \ac{RW} data,  
\textbf{(iv)} the performance of all the models remain consistent across both \ac{RW}-IA datasets and \textbf{(v)} in \figref{fig:RW_ROC_Curves}, we observe that there is a high degree of overlap among the \ac{ROC} curves for the data-driven models.

From (i) and (iv) we can conclude that \ac{SYN} data is more intricate than \ac{RW} data. This implies that the benefits yielded by training with \ac{SYN} dataset, while significant, is not as prominent in the \ac{RW} Iowa datasets. 
% This may be due to sensor noise prevalent in the \ac{RW} Landsat-8 data that can be harder to reproduce in the synthetically generated examples. 
(i), (iii) and (v) implies that while \ac{SYN} data is not an exact replacement for \ac{RW} data, it provides a rather significant edge, which is all the more important when there is insufficient \ac{RW} for training. From (ii) we can conclude that the three proposed data driven models outperform classical super-resolution techniques such as bicubic and lanczos, conclusion supported by the \ac{ROC} curves in Figure \ref{fig:RW_ROC_Curves} for whom the data-driven models, in general, lie above the non-data-driven alternatives. Observation (iv) shows that  for the climatically similar \ac{RW}-Iowa(CR) and \ac{RW}-Iowa(DM) regions, training on \ac{SYN} Iowa data does indeed provide an edge. 

% have similar climate. 

\begin{figure*}[t!]
    \centering
    \begin{subfigure}[t]{0.5\textwidth}
        \centering
        \includegraphics[width=\textwidth/2]{figures/IowaCedar_roc.png}
        \caption{}
    \end{subfigure}%
    ~ 
    \begin{subfigure}[t]{0.5\textwidth}
        \centering
        \includegraphics[width=\textwidth/2]{figures/IowaDesMoines_roc.png}
        \caption{}
    \end{subfigure}
    \vspace*{0.5cm}
    \caption{    \label{fig:RW_ROC_Curves} \Acf{ROC} curves for (a) RW-IA (CR) and (b) RW-IA (DM) dataset. Na\"ive model here represents a model whose output is solely a ``no Flood'' for all pixels. Star here represents the pixel-wise classifier with a threshold of 0.5.}
\end{figure*}


\subsection{Effectiveness of data-driven approaches}

In order to evaluate the effectiveness of ML models in the downscaling task, we compare performances of our candidate models to Lanczos and bicubic interpolation methods by looking at figures of some sample predictions from Iowa (Figure \ref{fig:RWIowaDesMoines}), performance comparison in the region of Iowa in Table \ref{tab:IowaResults} and the ROC curves in Figure \ref{fig:RW_ROC_Curves} for \ac{RW} data. We notice that 
\textbf{(vi)} For RW-IA (DM) samples, the deep learning models maintain a higher degree of spatial continuity in the predicted \ac{FIM}, 
\textbf{(vii)} We observe that  bicubic and Lanczos interpolation produces over-smoothed \ac{FIM} reconstructions, while the plain \ac{RDN}, \ac{RCAN} and \ac{ESRT} models are more detail-inclusive. Similar conclusions can be drawn upon inspecting the \ac{ROC} curves in Figure \ref{fig:RW_ROC_Curves} and 
\textbf{(viii)} For RW-IA (CR), the ML models show a performance improvement of 3.06\% when comparing the best ML model and non-data-driven method and, while for RW-IA (DM) there is a performance improvement of 2.45\%.


Figures \ref{fig:EUSamples} and \ref{fig:RWIowaDesMoines} show the spatial disparity among the models whose details are often obscured in aggregated metrics such as accuracy. (vi) This implies that these data-driven models are better are recognizing an underlying stream network geometry than the classical methods. However, when it comes to narrow river streams, all the models struggle capturing the nuances of the \ac{FIM} resultant from localized high elevation features such as small islands within rivers or man-made structures. (vii) shows a clear advantage of our data-driven approaches over the non-data-driven alternatives. (viii) indicates the benefits of the data-driven models when evaluated over Iowa. 



\subsection{Applicability of our models to external regions}

To evaluate how transferable our models are, we draw conclusions from figures of the sample predictions from Western Europe (Figure \ref{fig:EUSamples}) and Ghana (Figure \ref{fig:GhanaSamples}) as well as the performance comparison in Table \ref{tab:ExternalResults}. We notice that 
\textbf{(ix)} for Ghana all of the models fail to adequately inundate the pixels over separated areas on account of several disconnected regions of inundation in the chosen area,
\textbf{(x)} the ML models outperform non-data driven methods for RW-EU, 
\textbf{(xi)} for the RW-EU dataset, there is an improvement of 4.89\% when comparing the accuracy of the best data- and non-data-driven methods, 
\textbf{(xii)} For RW-RR and RW-GH, there is marginal improvement (up to 0.77\% in accuracy) of the ML methods over the non-data driven methods and 
\textbf{(xiii)} For RW-EU, we notice that the ML models produce more connected streams over the non-data-driven models. 

(x) and (xi) implies that the models are transferable when considering hydroclimaticalogically similar regions since Iowa and the Meuse river in Europe lie within mid temperate zones. Similar to the observation (vi) for RW-IA (DM), (xiii) implies that the benefits of the ML model in identifying underlying network streams is also transferable to hydroclimatologically similar regions. In contrast, (xii) and (ix) both imply that the trained ML models struggle to generalize to RW-RR \& RW-GH. We speculate that this may be due to the significant differences in geography and climate when compared to Iowa. 

% More specifically, we notice that Ghana has a lot of disconnected regions when compared to Iowa and Western Europe, possibly indicating a geomorphological dissimilarity. Additionally, in the case of Red River and Ghana, we also speculate that they include drivers to flood inundation that are different from Iowa and Western Europe, which lie within mild temperate zones. Ghana on the other hand has a tropical (dry and hot) climate.

Our study directly implies that good quality synthetic data can be useful surrogates for downscaling low-resolution \acp{WFM} to high-resolution \acp{FIM} in regions, where such data are hard to come by, even when downscaling by a factor of 10. We noticed that such models were readily transferable to climatically similar regions as the region of training. However, Such derived ML models did not feature significantly different transferability when evaluated over hydroclimatologically dissimilar regions, which we attribute to different flood inundation characteristics, primarily at finer scales. A possible avenue to circumvent such issues is to explore additional training approaches that fall under the general area of domain adaptation. Nevertheless, data-driven models are still advantageous (and, hence, preferable) over non-data-driven alternatives in transfer scenarios like the one we considered here. 


%%%%%%%%%%%%%%%%%%%%%%%%%%%%%%% unused text %%%%%%%%%%%%%%%%%%%%%%%%%%%%%%%%%%%%%%%



% \tabref{tab:AccuracyResults} depicts test accuracies obtained by our models on both \ac{SYN} and \ac{RW} data. For Iowan floods, a comparison of \ac{SYN} and \ac{RW} results shows \textbf{(i)} bicubic and Lanczos interpolations remarkably gaining about $3\%$ in accuracy, as well as \textbf{(ii)} \ac{RDN} and \ac{RCAN} remaining relatively stable, while \textbf{(iii)} topography-aware models loosing $2.7\%$ in performance. From (i) one can conclude that \ac{SYN} data are morphologically slightly more intricate than \ac{RW} data. Also, (i) and (ii) likely imply that \ac{SYN} data, excluding topography, can serve as satisfactory surrogates of \ac{RW} data. However, as implied by (iii), our topography-dependent models seems to be particularly sensitive to distributional shifts of their combined inputs (\acp{WFM} and topographic features). More specifically, the topography-informed models' performance edge, while still statistically significant, is extremely marginal, even when compared to our non-data-driven approaches. Next, when comparing results between the cases of Iowan and Ghanaian \ac{RW} data, one observes that \textbf{(iv)} the accuracy of bicubic and Lanczos interpolations drops by almost $5\%$ due to over-smoothing. This may imply that Ghanaian \acp{FIM} bare a more complex morphology, when compared to Iowan \acp{FIM}. Also, \textbf{(v)} our topography-agnostic, data-driven models' performance degrades more gracefully (by about $2\%$), while \textbf{(vi)} our topography-aware models perform, virtually, as bad as our non-data-driven approaches. Hence, the differences in the data populations of the two regions we considered are significant enough to render our topography-dependent models noncompetitive. 



\section{Related Work}

\subsection{View-Dependent Control}
View-dependent representations have a long history in computer graphics.
In his pioneering work, Rademacher proposed interpolating between \textit{key viewpoints} and associated \textit{key deformations} to manipulate 3D models~\cite{rademacher1999view}.
Other researchers have extended the idea to create 3D animation systems~\cite{10.1111:j.1467-8659.2004.00772.x}, streamline the modeling process~\cite{DBLP:journals/corr/abs-2103-15472}, and integrate physical simulation\cite{koyama2013view}.
Of particular note, Rivers et al.~\cite{rivers25Dcartoonmodels} introduced \textit{2.5D Cartoon Models}, a combination of planar meshes transformed, based upon view angle, so as to appears three dimensional.
Our work draws upon these works but is, to our knowledge, the first work to attempt to use view-dependent techniques to retarget 3D motion onto 2D characters.   

\subsection{Animation from 2D Images}

% output is still 2D
Many researchers have proposed different methods for creating animations from 2D images. Hornung et al.~\cite{Hornung2007anim2Dpicmotion} presented a method to deform a character from a photograph given user-provided joint annotations.
\textit{Toonsynth}~\cite{Dvoroznak18-SIG} and \textit{Neural Puppet}~\cite{poursaeed2020neural} both present methods to create new images of hand-drawn characters from examples.
% output is 3D model
Other researchers have proposed methods of obtaining 3D geometry from 2D sketches~\cite{igarashi2006teddy, Dvoroznak20-SA} and images~\cite{ArtiSketch,weng2019photo}.
% focus on sketches specifically
A number of works have specifically focused on childlike drawings.
Lingens et al.~\cite{lingens2020towards} proposed an evolutionary algorithm for animating children's drawings. 
\textit{MagicToon}~\cite{feng2017magictoon} creates a 3D model from childlike drawings for AR applications.
Similar to our work, Smith et al.~\cite{SmithHodgins} proposed a method for animating childlike drawings using 3D skeletal motion. 
However, the resulting animations are only suitable for use in 2D applications and the type of motions it supports are limited.

Unlike these previous works, our solution can be used in 3D contexts but does not create a 3D model. We instead relying upon a view-dependent formulation of the animated character.
\section{Discussion and Future Work}\label{sec:discussion}
This paper pioneers the novel approach of selective response, showing that withholding responses can be a powerful tool for GenAI systems. By opting not to answer every query as accurately as it can---particularly when new or complex topics emerge---GenAI can encourage user participation on community-driven platforms and thereby generate more high-quality data for future training. This mechanism ultimately enhances GenAI's long-term performance and revenue. From a welfare perspective, our results indicate that such selective engagement can also benefit users, leading to better solutions and increased overall satisfaction. Since this work is the first to address selective response strategies for GenAI, numerous promising directions remain for future research; we highlight some of them below. 

First, from a technical standpoint, all of the results in this paper rely on Assumption~\ref{assumption: data lip}, involving the lipshitz condition of the accuracy function and the sensitivity parameter $\beta$. Future work could seek to relax this assumption. Furthermore, our constrained optimization approach in Subsection~\ref{sec: welfare constrained revenue maximization} could be extended to approximate the optimal (continuous) strategy instead of the optimal discrete strategy.

Second, our stylized model adopts the simplifying---though unrealistic---assumption that only a single GenAI platform exists. Admittedly, this makes it easier to focus on the idea of selective responses, and indeed, this assumption is pivotal in keeping our analysis tractable. Future research could explore scenarios with multiple GenAI platforms and human-centered forums. In such settings, one platform's selective response might redirect users not only to forums but also to competing GenAI platforms, leading to the tragedy of the commons \cite{hardin1968tragedy}: Although all GenAI platforms benefit from fresh data generation, none may choose to respond selectively if it means losing users to competitors. 

Third, we assumed Forum behaves non-strategically. In reality, human-centered platforms often monetize their data by selling it to GenAI platforms, adding a further layer of strategic interaction for GenAI. Moreover, data transfer between the platforms can form the basis for collaboration: GenAI could employ selective response to bolster Forum content creation, and Forum could, in turn, attribute that content to GenAI for subsequent use in retraining.


%Third, we make the (again) simplifying assumption that Forum is non-strategic. However, in practice, human-centered platforms can sell their data to GenAI platforms. This adds additional considerations for GenAI. Furthermore, data transmission between the platforms can also become the basis for collaboration: GenAI can use selective response to ensure enough content is generated in Forum, and Forum could provide the data attributed to this mechanism back to GenAI. 


%Second, this paper makes the simplifying yet unrealistic assumption of the existence of one GenAI platform. Indeed, this simplifies many aspects and allows us to analyze selective responses. Future work could address the data generation process with more than one GenAI platform and possibly several human-centered forums. In such a case, selective response of one GenAI platform can either drive users to forums or to other GenAI platforms; thus, we might face a tragedy of the commons situation~\ref{hardin1968tragedy}, where all GenAI platforms are interested in fresh data generation but none volunteer to selectively respond and lose users. 

%This paper examines the competition between a generative AI platform and human-based platforms, challenging the assumption that always providing answers is optimal. We analyzed the impact of withholding answers on GenAI's revenue and developed an efficient approximately optimal algorithm for this purpose. We further explored how withholding affects users, showing that it can lead to better outcomes compared to always answering. Specifically, we demonstrated that withholding can Pareto-dominate this strategy and derived the necessary and sufficient conditions for that. Finally, we proposed a second approximately optimal algorithm that maximizes GenAI's revenue while ensuring users are better off than when GenAI answers all queries.

%On a more conceptual level, our model assumes that GenAI’s data comes solely from the competing platform (Forum). Future research could explore a scenario where GenAI can purchase additional data from a third party. This extension could provide valuable insights into the interplay between withholding answers and data purchasing, and whether these two strategies can complement each other or must be traded off.

\section*{Acknowledgments}
This research was supported by the Australian Government through the Australian Research Council’s Discovery
Projects funding scheme (Project ID DP220102520).


\bibliographystyle{IEEEtran}
\bibliography{references}

\newpage
% biography section
% 
% If you have an EPS/PDF photo (graphicx package needed) extra braces are
% needed around the contents of the optional argument to biography to prevent
% the LaTeX parser from getting confused when it sees the complicated
% \includegraphics command within an optional argument. (You could create
% your own custom macro containing the \includegraphics command to make things
% simpler here.)
%\begin{IEEEbiography}[{\includegraphics[width=1in,height=1.25in,clip,keepaspectratio]{mshell}}]{Michael Shell}
% or if you just want to reserve a space for a photo:

\begin{IEEEbiographynophoto}{Akshay Aravamudan}
is a PhD student at the Florida Institute of Technology. His research areas of interest include machine learning, stochastic point processes for the study of information diffusion and influence characterization in social media, machine learning for hydrology and machine learning on the edge. Before joining FIT’s Center for Advanced Data Analytics \& Systems (CADAS), he obtained his M.S. in Computer Engineering at the Florida Institute of Technology.
\end{IEEEbiographynophoto}

% if you will not have a photo at all:
\begin{IEEEbiographynophoto}{Zimeena Rasheed}
is a PhD student at Rutgers University, New Brunswick. She graduated with a M.S. in Civil Engineering from Florida Institute of Technology in 2020. Her research broadly includes the study of stream flows, their prediction for anticipation of floods as well as satellite-based flood inundation. 
\end{IEEEbiographynophoto}

% if you will not have a photo at all:
\begin{IEEEbiographynophoto}{Xi Zhang}
is a PhD student at the Florida Institute of Technology. Her research interests include probablistic modeling, statistical methods, ML/AI with applications in social media, cyber-security and the earth sciences. She received her M.S. in Electrical Engineering at Florida Institute of Technology, and she is currently a member of FIT’s Center for Advanced Data Analytics \& Systems (CADAS).
\end{IEEEbiographynophoto}


% if you will not have a photo at all:
\begin{IEEEbiographynophoto}{Kira E. Scarpignato}
is expected to graduate with a B.S in Biomedical Engineering and a minor in Biology in May 2024 from Florida Institute of Technology. Her current research is focused on cardiovascular tissue engineering.
\end{IEEEbiographynophoto}

% if you will not have a photo at all:
\begin{IEEEbiographynophoto}{Efthymios I. Nikolopoulos}
is an Associate Professor of Civil \& Environmental Engineering at Rutgers University, New Brunswick. He received his Engineering Diploma from the Technical University of Crete, Greece, his M.Sc. degree from the University of Iowa, and his Ph.D. degree from the University of Connecticut, U.S. He has also worked as a postdoc at the University of Padova, Italy. His expertise is in the modeling and monitoring of hydrometeorological and hydrologic extremes (extreme precipitation, floods, droughts, debris flows). His main research goal is to improve the understanding and predictability of hydrologic extremes and develop methods to mitigate their impacts. 
\end{IEEEbiographynophoto}

% if you will not have a photo at all:
\begin{IEEEbiographynophoto}{Witold F. Krajewski}
received his Ph.D. from Warsaw University of Technology in 1980. He is the Rose \& Joseph Summers Chair in Water Resources Engineering and Professor in Civil and Environmental Engineering at the University of Iowa. He serves as Director of the Iowa Flood Center, an entity funded by the State of Iowa in the United States and housed at the University of Iowa. His research interests include all aspects of flood forecasting. He is a Fellow of the American Meteorological Society (AMS), the American Geophysical Union (AGU), and a member of the U.S. National Academy of Engineering.

\end{IEEEbiographynophoto}
% if you will not have a photo at all:
\begin{IEEEbiographynophoto}{Georgios C. Anagnostopoulos}
is an Associate Professor of Electrical \& Computer Engineering at the Florida Institute of Technology in Melbourne, Florida. He received is Engineering Diploma from the University of Patras in 1994 and his M.Sc. and Ph.D. degrees in Electrical Engineering from the University of Central Florida in 1997 and 2001 respectively. His areas of expertise are machine learning, modeling and optimization. He is a senior member of the IEEE.
\end{IEEEbiographynophoto}

% You can push biographies down or up by placing
% a \vfill before or after them. The appropriate
% use of \vfill depends on what kind of text is
% on the last page and whether or not the columns
% are being equalized.

%\vfill

% Can be used to pull up biographies so that the bottom of the last one
% is flush with the other column.
%\enlargethispage{-5in}


\vfill

\end{document}


