%%%% ijcai25.tex

\typeout{IJCAI--25 Instructions for Authors}

% These are the instructions for authors for IJCAI-25.

\documentclass{article}
\pdfpagewidth=8.5in
\pdfpageheight=11in

% The file ijcai25.sty is a copy from ijcai22.sty
% The file ijcai22.sty is NOT the same as previous years'
\usepackage{ijcai25}

\usepackage[table]{xcolor}

% Use the postscript times font!
\usepackage{times}
\usepackage{soul}
\usepackage{url}
% \usepackage[hidelinks]{hyperref}
\usepackage[breaklinks,colorlinks,citecolor=teal,urlcolor=teal]{hyperref}
% \usepackage[breaklinks,colorlinks]{hyperref}
\usepackage[utf8]{inputenc}
\usepackage[small]{caption}
\usepackage{graphicx}
\usepackage{amsmath}
\usepackage{amsthm}
\usepackage{booktabs}
\usepackage{algorithm}
\usepackage{algorithmic}
\usepackage[switch]{lineno}
\usepackage{amssymb,mathtools}
\usepackage{cleveref}
\usepackage{utfsym}
\usepackage{multirow}
\newcommand{\blue}[1]{\textcolor{blue}{#1}}
\usepackage{enumitem}
\usepackage{pifont}



% Comment out this line in the camera-ready submission
% \linenumbers

\urlstyle{same}

% the following package is optional:
%\usepackage{latexsym}

% See https://www.overleaf.com/learn/latex/theorems_and_proofs
% for a nice explanation of how to define new theorems, but keep
% in mind that the amsthm package is already included in this
% template and that you must *not* alter the styling.
\newtheorem{example}{Example}
\newtheorem{theorem}{Theorem}

% Following comment is from ijcai97-submit.tex:
% The preparation of these files was supported by Schlumberger Palo Alto
% Research, AT\&T Bell Laboratories, and Morgan Kaufmann Publishers.
% Shirley Jowell, of Morgan Kaufmann Publishers, and Peter F.
% Patel-Schneider, of AT\&T Bell Laboratories collaborated on their
% preparation.

% These instructions can be modified and used in other conferences as long
% as credit to the authors and supporting agencies is retained, this notice
% is not changed, and further modification or reuse is not restricted.
% Neither Shirley Jowell nor Peter F. Patel-Schneider can be listed as
% contacts for providing assistance without their prior permission.

% To use for other conferences, change references to files and the
% conference appropriate and use other authors, contacts, publishers, and
% organizations.
% Also change the deadline and address for returning papers and the length and
% page charge instructions.
% Put where the files are available in the appropriate places.


% PDF Info Is REQUIRED.

% Please leave this \pdfinfo block untouched both for the submission and
% Camera Ready Copy. Do not include Title and Author information in the pdfinfo section
\pdfinfo{
/TemplateVersion (IJCAI.2025.0)
}

\title{Toward Robust Non-Transferable Learning: A Survey and Benchmark}


% Single author syntax
\author{
  \vspace{-5mm}
Ziming Hong\thanks{}
\and 
Yongli Xiang$^*$
\And
Tongliang Liu$^{\dagger}$\\
\affiliations
Sydney AI Centre, The University of Sydney\\
\emails
hoongzm@gmail.com,
yxia0023@uni.sydney.edu.au,
tongliang.liu@sydney.edu.au
}



\begin{document}


\twocolumn[{%
\renewcommand\twocolumn[1][]{#1}%
\maketitle
\begin{center}
  \vspace{-15mm}
  \centering\includegraphics[width=\linewidth]{figs/motiv1.pdf}\\
  \vspace{2mm}
  \centering\includegraphics[width=1.0\linewidth]{figs/motiv2.pdf}
  \vspace{-7mm}
  \captionof{figure}{We systematically review non-transferable learning (NTL) and introduce \texttt{NTLBench}, an unified framework for benchmarking NTL. This figure compares 5 methods (\textcolor[HTML]{7884AC}{NTL}, \textcolor[HTML]{7884AC}{CUTI-domain}, \textcolor[HTML]{7884AC}{H-NTL}, \textcolor[HTML]{7884AC}{SOPHON}, \textcolor[HTML]{7884AC}{CUPI-domain}) on CIFAR \& STL with VGG-13, evaluating pre-training performance and robustness
  against 5 \textcolor[HTML]{E4785F}{source domain fine-tuning} attacks, 4 \textcolor[HTML]{8151BA}{target domain fine-tuning} attacks, and 6 \textcolor[HTML]{59A4B7}{source-free domain adaptation} attacks (higher value means better robustness). 
  \texttt{NTLBench} will be released soon at \url{https://github.com/tmllab/NTLBench}.
  } 
  \vspace{3mm}
\label{fig:opening}
\end{center}%
}]

\renewcommand{\thefootnote}{*}
\footnotetext{Equal contribution. $^{\dagger}$Corresponding author.}
\renewcommand{\thefootnote}{\arabic{footnote}}


\begin{abstract}
  \vspace{-0.2mm}
  Over the past decades, researchers have primarily focused on improving the generalization abilities of models, with limited attention given to regulating such generalization. However, the ability of models to generalize to unintended data (e.g., harmful or unauthorized data) can be exploited by malicious adversaries in unforeseen ways, potentially resulting in violations of model ethics. Non-transferable learning (NTL), a task aimed at reshaping the generalization abilities of deep learning models, was proposed to address these challenges. While numerous methods have been proposed in this field, a comprehensive review of existing progress and a thorough analysis of current limitations remain lacking. In this paper, we bridge this gap by presenting the first comprehensive survey on NTL and introducing \texttt{NTLBench}, the first benchmark to evaluate NTL performance and robustness within a unified framework. Specifically, we first introduce the task settings, general framework, and criteria of NTL, followed by a summary of NTL approaches. Furthermore, we emphasize the often-overlooked issue of robustness against various attacks that can destroy the non-transferable mechanism established by NTL. Experiments conducted via \texttt{NTLBench} verify the limitations of existing NTL methods in robustness. Finally, we discuss the practical applications of NTL, along with its future directions and associated challenges.
  \vspace{-0.5mm}
\end{abstract}




\section{Introduction}\label{sec:intro}

In computational finance, Monte Carlo simulations are used extensively to estimate the expected value of financial payoffs based on the solution of stochastic differential equations (SDEs) which model the evolution of stock prices, interest rates, exchange rates and other quantities \cite{glasserman04}.  Monte Carlo methods are very general and flexible, but for high accuracy it requires generating a large number of costly SDE path approximations, which has motivated research into a number of variance reduction or, equivalently, cost reduction techniques. One such method is
Multilevel Monte Carlo (MLMC), which was proposed in \cite{GILES2008} and was adapted for various applications that are summarised in \cite{Giles_overview17} and successfully combined with other methods such as quasi-Monte Carlo methods. The main idea of MLMC is to approximate the payoff using different time stepping resolutions when numerically solving the underlying SDE and to generate an optimal number of samples on each level, such that the overall computational cost is minimised subject to the desired bound on the variance. %, such that the total computational cost is minimised. 
The computational savings come from the fact that most samples are computed on the coarser levels and hence are less expensive while only a few samples from the finest levels are required \cite{GILES2008}.


Among the directions in which the computational cost 
of MLMC methods could further be reduced, an important avenue is the use of lower precision calculations, especially for the first Monte Carlo levels where the targeted accuracy is relatively low. 
 An overview of the research on mixed precision for the standard Monte Carlo (MC) framework is provided in \cite{ChowMixedPrecisionStandardMC} but only a few references study the potential of low precision computation in the MLMC framework \cite{Rounding_error_oliver}. To the best of our knowledge, the only MLMC framework with customised precision in the literature is \cite{brugger2014mixed}, but they use a uniform precision for all operations on each Monte Carlo level instead of optimising 
 the precision of each intermediary variable to reduce as much as possible the cost of path generation.
 
An important motivation for an MLMC framework with variable precision would be performing the low precision computations on reconfigurable hardware devices such as Field Programmable Gate Arrays (FPGAs). FPGAs contain customizable logic blocks and connectors that make it easy to adapt the digital circuit architecture for a specific application, leading to a highly parallel and optimised implementation. Therefore they are successfully exploited in applications that require high speed and have high computational workload, such as signal processing \cite{woods2008fpga}, and real time applications like high frequency trading \cite{HFT1,HFT2}. That is why a number of previous works in hardware architecture design implemented the MLMC algorithm to price financial options using FPGAs as accelerators, which resulted in improved speed and power efficiency compared to full CPU architectures \cite{Schryver2013AMM}. The paper \cite{lindsey2016domain} also proposed 
a Domain Specific Language to automate the configuration of FPGAs for this specific application. However, only \cite{brugger2014mixed} proposed a heuristic to reduce the precision in calculations.

In addition, all aforementioned works considered that the random number generation (RNG) is performed in single or double precision. Yet in most cases an important portion of the workload in the overall MLMC simulation comes from the RNG and in \cite{brugger2014mixed} this limited the total computational savings.
To reduce the cost of MLMC simulations in particular those based on the Geometric Brownian Motion (GBM), \cite{approximateICDF_Oliver, NestedOliver} have proposed to use approximate random numbers that are generated by applying an approximation of the inverse CDF to uniform random numbers. In \cite{NestedOliver}, the authors proposed a way to integrate these lower precision random variables into a \textit{nested} MLMC framework and completed a numerical analysis to bound the resulting error at each MC level by a product of the time step and the error in the random number approximation. The same authors show in \cite{approximateICDF_Oliver} that using approximate random variables reduces the cost of path generation by a factor 7.


In this paper we propose a nested MLMC framework that combines the use of approximate random normal variables and lower precision calculations to reduce the computational cost of MLMC even further than \cite{brugger2014mixed,NestedOliver}. We illustrate the efficiency of our framework in Matlab, after making several assumptions on the cost of operations and size of the errors that we carefully justify. We focus on the case of GBM and use the approximate RNG methods presented in \cite{approximateICDF_Oliver} as well as a new slightly modified method that combines CDF inversion and the central limit theorem. To choose the precision of the variables in the low precision path generation, we introduce a novel method to optimise the bit-widths. This optimisation is performed before the main path generation loop is executed and is based on a linear model of the payoff error  
due to rounding when computing in low precision. The error model relies on algorithmic differentiation in a similar manner to \cite{unifying-bwoptim,bitwidth-AD,ADAPT}. The bit-width optimisation procedure can be performed off-line, so this stage can be excluded from the on-line time complexity of our framework. The user specified desired accuracy is then enforced by calculating on-line the number of samples that need to be generated.

In terms of hardware design, we suggest implementing the low precision path generation on FPGAs and the full-precision ones on a CPU or GPU. 
The FPGA offers enough flexibility to define a separate bit-width for every variable in the low precision path generation, and can be reconfigured periodically to update the bit-widths when the market parameters have changed considerably. 


The paper is organized as follows : \Cref{sec:MLMC} introduces MLMC and nested MLMC to make clear the estimator that is implemented in our framework. Then in \Cref{sec:RNG} we detail the methods that could be used to obtain approximate random normally distributed numbers very cheaply for the low precision path generation. In \Cref{sec:error_model} and \Cref{sec:costModel} we propose an error model and a cost model (resp.) that we then use to formulate the optimisation problem that is solved to obtain the optimal bit-widths of fixed point variables in \Cref{sec:optimisation}. Finally we summarise our results and future directions in \Cref{sec:conclusion}.



\begin{figure*}[t]
\vskip 0.2in
\begin{center}
\centerline{\includegraphics[width=\textwidth]{Figures/pipeline-vlm-v4.pdf}}
\caption{Overview of our data-aware preference optimization. For each preference instance: (1) We first break the preferred and rejected response into sub-sentences by prompting a large language model (LLM); 
(2) Next, we estimate the similarity scores between each sub-sentence and the given image using the CLIP classifier, and then calculate the differences between the preferred and rejected response as the hardness of the data; 
(3) Finally, we incorporate the estimated hardness into the preference optimization process by modifying $\beta$ in Equ~\eqref{equ:dpo}, allowing the model to adjust based on the data hardness.}
\label{fig:pipleine-vlm}
\end{center}
\vskip -0.2in
\end{figure*}


\section{Preliminary}
\label{sec:preliminary}
In this section, we briefly review the MLLM preference learning procedure, which starts by sampling pairwise preference data with a supervised fine-turned (SFT) model, and then optimizes on such preference data. Specifically, we categorize this process into the following aspects:

\noindent \textbf{Supervised Fine-Tuning (SFT).}
Preference learning of an MLLM $\bm{\pi}$ begins with an SFT model $\bm{\pi}_{\text{SFT}}$. Concretely, the SFT process fine-tunes the pre-trained MLLM model with millions of multi-modal question-answer pairs to align LLM with multi-modal tasks. 
After this process, we construct preference data by sampling pair-wise preference responses from $\bm{\pi}_{\mathrm{SFT}}$, formalized as $(y_w, y_l) \sim \bm{\pi}_{\mathrm{SFT}}(y|x,\mathcal{I})$, where $(\mathcal{I}$ denotes the image and $x$ is the prompt question. 
Meanwhile, $(y_w, y_l)$ are labeled as preferred and less preferred responses by humans, formalized as $(y_w \succ  y_l | \mathcal{I}, x)$.

\noindent \textbf{RLHF with Reward Models.}
Given pair-wise preference data $(y_w, y_l) \sim \bm{\pi}_{\mathrm{SFT}}(y|x,\mathcal{I})$, the preference learning process can be described in 2 stages: reward modeling and preference optimization. 
Specifically, the reward model $r_{\bm{\theta}}(y|\mathcal{I}, x)$ is defined to rank the model responses by learning to distinguish $y_w$ from $y_l$, and the preference optimization aims to distill the preference knowledge into MLLM. 
To learn a reward model, pioneering work \cite{rlhf} employs the Bradley-Terry model \cite{BT_model} to model the pair-wise preference distribution as:
\begin{equation}
\resizebox{.9\hsize}{!}{
\begin{math}
\begin{aligned}
    \mathrm{P}(y_w \succ  y_l|\mathcal{I}, x) & =  \sigma(r^{*}(y_w|\mathcal{I}, x)- (r^{*}(y_l|\mathcal{I}, x)) \\
     & = \frac{\mathrm{exp}(r^{*}(y_w|\mathcal{I}, x))}{\mathrm{exp}(r^{*}(y_w|\mathcal{I}, x))+\mathrm{exp}(r^{*}(y_l|\mathcal{I}, x))}.
\end{aligned}
\end{math}
}
\end{equation}

Thus, the learning process can be achieved by minimizing the negative log-likelihood $-\mathrm{logP}(y_w \succ y_l|\mathcal{I}, x)$ over the preference data with the parametrized reward model $r_{\bm{\phi}}(y_w|\mathcal{I}, x)$ initialized as $\bm{\pi}_{\mathrm{SFT}}$ with a simple linear layer to produce reward prediction. 
With the well-optimized reward model $r_{\phi}^{*}(y|\mathcal{I}, x)$, prior work \cite{rlhf} proposes to employ policy optimization algorithms in RL such as PPO \cite{PPO} to maximize the learned reward with KL-penalty, which can be formalized as:
\begin{equation}
\label{equ:ppo}
\begin{aligned}
    \underset{\bm{\pi}_{\theta}}{\text{max}} & \  \mathbf{E}_{(\mathcal{I},x) \sim \mathcal{D}, y \sim \bm{\pi}_{\theta}(\cdot|\mathcal{I}, x)} [r_{\phi}^{*}(y|\mathcal{I}, x)] \\
    & -\beta \mathbb{D}_{\mathbf{KL}}[\bm{\pi}_{\theta}(y|\mathcal{I},x)||\bm{\pi}_{\text{ref}}(y|\mathcal{I},x)], 
\end{aligned}
\end{equation}
where the fixed reference model $\bm{\pi}_{\text{ref}}$ is parameterized as $\bm{\pi}_{\text{SFT}}$, and the hyper-parameter $\beta$ controls the deviation of $\bm{\pi}_{\theta}$ from $\bm{\pi}_{\text{ref}}$ during the optimization process.

\noindent \textbf{Direct Preference Optimization (DPO).}
To relieve the high computational complexity of reward training in RLHF, DPO \cite{DPO} is proposed, which provides a simple way to directly optimize $\bm{\pi}_{\theta}$ with the pair-wise preference data, without parametrized reward model. Specifically, the DPO loss can be described as:
\begin{equation}
\label{equ:dpo}
\begin{aligned}
    \mathcal{L}_{\mathrm{dpo}} = - \bm{\mathrm{E}}_{(\mathcal{I},x, y_{w}, y_{l})} [ {\log \sigma}( & \beta \log \frac{{\pi}_{\bm{\theta}}(y_{w}|\mathcal{I},x)}{{\pi}_{\mathrm{ref}}(y_{w}|\mathcal{I},x)} \\
    - & \beta \log \frac{{\pi}_{\bm{\theta}}(y_{l}|\mathcal{I},x)}{{\pi}_{\mathrm{ref}}(y_{l}|\mathcal{I},x)}) ].
\end{aligned}
\end{equation}


\section{Approaches for NTL}
\label{sec:immeNTL}

Target-specified NTL approaches contain fundamental solutions for NTL, and thus, we first review them in \Cref{sec:target-specified}. Then, in \Cref{sec:source-only}, we review how existing works implement source-only NTL in the absence of a target domain.


\subsection{Target-Specified NTL}
\label{sec:target-specified}

Briefly, in target-specified setting, the target domain is known and we aim to restrict the generalization of a deep learning model from the source domain toward the certain target domain. 
Existing methods perform target-domain regularization either on the feature space or the output space, as we summarized in \Cref{tab:summary} (\textbf{Non-Transferable Approach} column). For more details, we introduce existing strategies as follows:

\paragraph{Output space regularization.} Output-space regularizations directly manipulate the model logits on the target domain. More specifically, these operations can be categorized into \textit{untargeted regularization} and \textit{targeted regularization}. \textit{Untargeted regularization} \cite{wang2021non,wang2023model,zhou2024archlock,peng2024map} could usually be formalized as a maximizing optimization problem, where existing methods implement this regularization by maximizing the KL divergence between the model outputs and the real labels, thus disturbing the model predictions on the target domain. However, such untargeted regularizations may face convergence issues \cite{deng2024sophon}. \textit{Targeted regularization} \cite{wang2023domain,deng2024sophon} found a proxy task on the target domain (i.e., modify the labels), thus converting the maximization objective in untargeted regularization to a minimization optimization problem. 
DSO \cite{wang2023domain} transforms the correct labels to error labels without overlap (e.g., $y_{\text{err}} = y+1$) and uses error labels as the target-domain supervision. 
H-NTL \cite{hong2024improving} first disentangle the content and style factors via a variation inference framework~\cite{blei2017variational,yao2021instance,von2021self,lin2024cs,lin2025learning}, and then, they learn the NTL model by fitting the contents of the source domain and the style of the target domain. Due to the assumption that the style is approximately to be independent to the content representations, the non-transferability could be implemented.
SOPHON \cite{deng2024sophon} aims at both image classification and generation tasks. For classification, they propose to modify the cross-entropy (CE) loss to its inverse version (i.e., modify the label $y$ to $1-y$) or calculate the KL divergence between the model outputs and a uniform distribution. For generation, SOPHON proposes to use a Denial of Service (DoS) loss, i.e., let the diffusion model fit a zero matrix at each step. Compared to untargeted regularizations, targeted regularizations always have better convergence.

\paragraph{Feature space regularization.} Feature-space regularizations further reduce the similarity between feature representations from different domains, thus restricting the transferability directly on the feature space. Feature-space regularizations can also be categorized into \textit{untargeted} and \textit{targeted} strategies, depending on whether they directly enlarge the distribution gap through a maximization objective or convert it to a minimization problem by finding a proxy target. For \textit{untargeted regularization}, existing methods \cite{wang2021non,zeng2022unsupervised} propose to maximize the maximum mean discrepancy (MMD) loss between the feature representations from different domains, where MMD measures the distribution discrepancy. 
For \textit{targeted regularization}, UNTL \cite{zeng2022unsupervised} proposes to build an auxiliary domain classifier with feature representations from different domains as inputs. By minimizing the domain-classification loss, the domain classifier could help the NTL model learn domain-distinct representations. 
NTP \cite{ding2024non} aims to minimize the Fisher Discriminant Analysis (FDA) term \cite{shao2022not} in the target domain. Specifically, a smaller FDA value indicates a reduced difference in class means and increased feature variance within each class, which is associated with poorer target domain performance.



\subsection{Source-Only NTL}
\label{sec:source-only}

Under the assumption that only source domain data is available, existing works~\cite{wang2021non,wang2023model,wang2023domain,hong2024improving} take various data augmentation methods to obtain auxiliary domains from the source domain and see them as the target domain. 
Thus, the source-only NTL problem can be solved by target-specified NTL approaches. 
These augmentation methods can be split into the following three categories: 


\paragraph{Adversarial domain generation.} 
Wang \textit{et al.} \shortcite{wang2021non} use generative adversarial network (GAN) \cite{mirza2014conditional,chen2016infogan} 
to synthesize fake images from the source domain and see them as the target domain.
They train the GAN by controlling the distance and direction of the synthetic distributions to the real source domain, thus enhancing the diversity of synthetic samples and improving the degradation of any distribution with shifts to the real source domain. 
CUTI-domain \cite{wang2023model} and CUPI-domain \cite{wang2024say} add Gaussian noise to the GAN-based adaptive instance normalization (AdaIN) \cite{huang2017arbitrary} to obtain synthetic samples with random styles. They use both the synthetic samples from AdaIN and Wang \textit{et al.} \shortcite{wang2021non} as the target domain. 
MAP \cite{peng2024map} also follows the GAN framework.
 They additionally add a mutual information (MI) minimization term to enhance the variation between synthetic samples and the real source domain samples, ensuring more distinct style features.

\paragraph{Strong image augmentation.} H-NTL \cite{hong2024improving} conducts strong image augmentation \cite{sohn2020fixmatch,cubuk2020randaugment,huang2023harnessing} on real source domain data.
Strong image augmentations (e.g., blurring, sharpness, solarize) do not influence the contents but significantly change the image styles, thus imposing interventions~\cite{von2021self} on the style factors in images. Then, all augmented images are treated as the target domain for training source-only NTL.

\paragraph{Perturbation-based method.} DSO \cite{wang2023domain} proposes to minimize the worst-case risk on the uncertainty set~\cite{sagawa2019distributionally,huang2023robust,wang2023defending} over the source domain distribution, where the risk is empirically calculated through a classification loss between the model predictions and the error label.

% \vspace{-1mm}
\section{Post-Training Robustness of NTL}
\label{sec:robustness}

NTL models are expected to keep the non-transferability after malicious attacks.
However, not all existing works evaluate the robustness of their method, as we listed in \Cref{tab:summary} (the last column). 
In this section, we review the robustness of the source and target domains as considered in previous works.

\subsection{Robustness Against Source Domain Attack} 
\label{sec:robustness_src}

Earlier evaluations in \cite{wang2021non,wang2023model} show that NTL models are still resistant to state-of-art watermark removal attacks when up to 30\% source domain data are available for attack. Hong \textit{et al.} \shortcite{hong2024your} further investigate the robustness of NTL and propose TransNTL, demonstrating that non-transferability can be destroyed using less than 10\% of the source domain data. Specifically,
they find NTL \cite{wang2021non} and CUTI-domain \cite{wang2023model} inevitably result in significant generalization impairments on slightly perturbed source domains \cite{hendrycks2019benchmarking,cubuk2020randaugment}. Accordingly, they propose TransNTL to fine-tune the NTL model under an impairment repair self-distillation framework, where the source-domain predictions are used to teach the model itself how to predict on perturbed source domains. As a result, the fine-tuned model is just like a SL model without the non-transferability. Furthermore, they also propose a defense method to fix this loophole by pre-repairing the generalization impairments in perturbed source domains.
Specifically, they add a defense regularization term on existing NTL and CUT-domain training. Minimizing the defense regularization term enables the NTL model to exhibit source-domain consistent behaviors on perturbed source-domain data, thus enhancing the robustness against TransNTL attack.

\subsection{Robustness Against Target Domain Attack} 
\label{sec:robustness_tgt}

Fine-tuning the NTL models using target domain data is a more direct strategy to break the non-transferability if malicious attackers have access to some labeled target domain data. However, most existing methods \cite{wang2021non,wang2023model,zeng2022unsupervised,wang2023domain,hong2024improving,peng2024map} ignore the robustness of their methods against target-domain fine-tuning attacks. SOPHON \cite{deng2024sophon} formally proposes the problem of non-fine-tunable learning, which aims at ensuring the target-domain performance could still be poor after being fine-tuned using target domain data.
Their main idea is to involve the fine-tuning process in training stage. 
Specifically, they leverage model agnostic meta-learning (MAML) \cite{finn2017model} to simulate multiple-step fine-tuning for the current model on the target domain. Then, they add per-step risk of the target domain as the total target-domain risk. By maximizing the total target-domain risk, the robustness against target-domain attacks can be enhanced. 
ArchLock \cite{zhou2024archlock} aims to find the most non-transferable network architectures \cite{liu2018darts,real2019regularized}, where they also implicitly consider the robustness on the target domain. Specifically, they maximize the \textit{minimum risk} of an architecture on the target domain in searching the non-transferable architectures. The minimum risk is found by searching the optimal \textit{parameters} of the \textit{architecture} with the minimum task loss on the target domain.
 
However, labeled target domain data being available to malicious attackers is a strong assumption that may not always hold in practical scenarios. A more realistic scenario is that the attacker only has access to some unlabeled target domain data. Whether NTL can resist attacks driven by unlabeled target domain data has not yet been studied. 


\section{Benchmarking}
%1.5
\subsection{Datasets for Benchmarking}
\label{sec:dataset}
We benchmark the metrics on existing datasets containing human evaluations on style strength and/or content preservation. We use datasets for rewriting tasks on simplifying sentiment, formality, and appropriate arguments. The data consists of system-written output and/or human-written references. The human rating is obtained on different scales, such as a 5-point Likert scale or ratings of 1-100. When benchmarking the metrics, we consider the Pearson correlation to the mean human judgement. We benchmark on the following datasets with the abbreviated names in []: 
\begin{itemize}[noitemsep]
    \item \citet{lai-etal-2022-human} [Lai] on formal/informal,
    \item \citet{mir-etal-2019-evaluating} [Mir] on positive/negative,
    \item \citet{alva-manchego-etal-2020-asset} [Alva-M.] on simplifying, 
    \item \citet{scialom-etal-2021-questeval} [Scialom] on simplifying,
    \item \citet{ziegenbein-etal-2024-llm} [Ziegen.] on appropriated arguments,
    \item \citet{cao-etal-2020-expertise} [Cao] on rewriting for layman/expert.
\end{itemize}
See App.~\ref{app:dataset} for details about the datasets.


We analyse the intercorrelations of the dimensions using Pearson correlation between the mean annotations (Table~\ref{tab:intercor}) -- we mostly see a significant positive correlation between the dimensions, but also a case of negative correlation between style and meaning preservation on sentiment on Mir. We hypothesise there are two effects in \textbf{intercorrelations}; the data happens to be such that: 1) ``successful'' output is successful in all dimensions, which would yield a positive correlation, 2) large style change affects content preservation, which would yield a negative correlation.

\begin{table}[t]
\fontsize{11pt}{11pt}\selectfont
\begin{tabular}{lccc}
\toprule
\textbf{Data} & \textbf{Cor. S-C} & \textbf{Cor. S-F} & \textbf{Cor. F-C} \\ \midrule
Lai & 0.387 & 0.692 & 0.705 \\
Mir & -0.342 & - & - \\
Alva-M. & 0.709 & 0.828 & 0.763 \\
Scialom & 0.471 & 0.673 & 0.592 \\
Ziegen. & 0.103 & 0.673 & -0.015*\\ \bottomrule
\end{tabular}
\caption{Correlation on mean annotations between dimensions. *not significant}
\label{tab:intercor}
\end{table}

\subsection{Our Constructed Test Set for Style Transfer Evaluation}

\textbf{Methodology}. We create test data with high variation on content preservation by constructing output with deliberate content errors, and output with large style shifts.
For each source sentence, we construct two output sentences: 
\begin{itemize}[noitemsep]%,nolistsep
    \item [\textbf{1)}]one where we preserve the content and shift the style to a large degree with more variation in the rewrite;
    \item [\textbf{2)}] one where we shift the style to a lesser degree, staying closer to the wording in the source sentence but, in addition, producing an error in the content. 
\end{itemize}
\textbf{Ideally,} we aim for data where all output sentences succeed on the style transfer, but where, for each source sentence, one output succeeds on content preservation and one does not, creating variation in the level of content preservation. 

\textbf{The methodology for adding content errors} is inspired by \citet{devaraj-etal-2022-evaluating} -- they analyse data on the task of simplifying text by categorising errors into a taxonomy of substitution, deletion and insertion. We construct errors with these categories in mind -- we substitute or swap key information, we drop key information, or we fabricate additional information not supported by the source sentence.

\textbf{In total}, our constructed test data covers six style/attribute tasks and consists of 500 samples, 100 of which are manually created and the remaining synthetically generated. All samples are annotated by three workers on a 5-point Likert scale on style strength and content preservation. 

\textbf{Synthetically generated}. We instruct an LLM (GPT-4o-mini, from openai.com) in steps to provide us with the decided transfer samples. In the first step, we ask the LLM to generate source sentences. Secondly, we prompt the LLM to transfer the source samples \textit{a lot more} to a new style in order to provide us with the first set of rewritten sentences. For the second set of rewrites, we a) prompt to transfer the source sentences \textit{a bit more} to a new style and b) take the transferred sentence from point a) and introduce a content error. We conduct this on four rewriting tasks: i) a headline to be more catchy; ii) an impolite sentence from an email to be polite; iii) a persuasive request to be more persuasive; and iv) a sentence with informal language with grammatical mistakes and internet slang to be formal. Prompt details in App.~\ref{app:syn}.    

\textbf{Manually generated}. We construct 100 transfer samples starting from a source sentence and manually construct the output sentences by adding errors, as in steps 1 and 2. In addition, we supply each sample with a reference to enable a small evaluation of reference-based methods. We construct rewrites on two tasks: Task 1) on sentiment (positive/negative). In this case, we use as source sentences headlines from Wiki News (Wikinews.org)\footnote{\url{megarhyme.com/blog/wikinews-dataset/},  Creative Commons Attribution 2.5} with minor modifications. Task 2) on detoxifying, and here we use toxic sentences from \citet{logacheva-etal-2022-paradetox}. 


\paragraph{Samples}.
We show one synthetically generated  sample in Fig.~\ref{fig:sample}, and below one manually constructed sample (sentiment transfer task; positive):
\begin{itemize}[leftmargin=*,noitemsep,nolistsep]
    \item Source sentence: \textit{President of China lunches with Brazilian President.}
    \item Output 1: \textit{The Great Presidents of China and Brazilian strengthen important ties over lunch.}
    \item Output 2: \textit{The President of China enjoys lunches with the Brazilian first lady.}
\end{itemize}


\paragraph{Human Evaluation}
We obtain a high level of human agreement on content preservation on our test set with a Krippendorf Alpha of $0.768$ and a good level of success in style transfer, with an average score on the 5-point Likert scale of $4.27$. Details divided into subparts are outlined in Table~\ref{tab:con_data}: we report the percentage of samples that at least obtain an average rating of 3 (fair) to show that most sentences succeeded in style transfer.

Three workers annotate each sample in batches corresponding to each subpart. Five different workers participated in total, recruited on a crowdsourcing platform.
Annotators are asked to rate content preservation and style change on a 5-point Likert scale, as in previous work \cite{mir-etal-2019-evaluating,ziegenbein-etal-2024-llm}. We use the scale
\begin{lstlisting}[basicstyle=\ttfamily,
  breaklines=true]
1: Very poor, 2: Poor, 3: Fair, 4: Good, 5: Very Good.
\end{lstlisting}
Details on annotation guideline, setup and payment in Annotation guideline in App.~\ref{app:anno}.
 
\begin{table}[t]
\begin{tabular}{llrlr}
\toprule
\multicolumn{1}{l}{\textbf{Task}} & \multicolumn{1}{l}{\textbf{Const.}} & \multicolumn{1}{r}{\textbf{Supp.}} & \multicolumn{1}{l}{\textbf{\begin{tabular}[c]{@{}c@{}}Con. \\ IAA \end{tabular}}} & \multicolumn{1}{c}{\textbf{\begin{tabular}[c]{@{}c@{}}S>=3 \\(\%)\end{tabular}}} \\ \midrule
sentiment & m & 50 & 0.676 & 88 \\
detoxify & m & 50 & 0.758 & 96 \\
catchy & s & 100 & 0.806 & 90 \\
polite & s & 100 & 0.645 & 100 \\
persuasive & s & 100 & 0.80 & 88 \\
formal & s & 100 & 0.817 & 99 \\ \bottomrule
\end{tabular}
\caption{Stats on our constructed test set. Constructed m: manually, s: synthetic.}
\label{tab:con_data}
\end{table}

% \vspace{-1mm}
\section{Applications of NTL}
\label{sec:applications}


NTL supports different applications, depending on which data are used as source and target domain. We introduce two applications in model intellectual property (IP) protection and then the application of harmful fine-tuning defense. 

\paragraph{Ownership verification (OV).} OV is a passive IP protection manner, which aims to verify the ownership of a deep learning model \cite{cheng2021mid,lederer2023identifying}. NTL solves ownership verification by triggering misclassification on data with pre-defined triggers \cite{wang2021non,chen2024mark,guo2024zeromark}. For example, when training, we add a shallow trigger (only known by the model owner) on the original dataset data and see them as the target domain, while the original data without the trigger is regarded as the source domain. Then, target-specified NTL is used to train a model. Therefore, the ownership can be verified via observing the performance difference of a model on the original data and the data with the pre-defined trigger. For SL model, the shallow trigger has minor influence on the model performance, and thus, the model shows similar performance on original data and data with triggers. In contrast, the NTL model specific to this pre-defined trigger has high performance on the original data but random-guess-like performance on data with the trigger. This provides evidence for verifying the model's ownership.
% 
\paragraph{Applicability authorization (AA).} AA is an active IP protection approach that ensures models can only be effective on authorized data \cite{wang2021non,xu2024idea,si2024iclguard}. NTL solves AA by degrading the model generalization outside the authorized domain. Basic solution is to add a pre-defined trigger on original data (seen as source domain), and the original data without the correct triggers is regarded as the target domain. After training by NTL, the model will only perform well on authorized data (i.e., the data with the trigger). Any unauthorized data will be randomly predicted by the NTL model. Thus, AA can be achieved.




\paragraph{Safety alignment and harmful fine-tuning defense.} 
Fine-tuning large language models (LLMs) with user's own data for downstream tasks has recently become a popular online service \cite{huang2024harmful,openai2024finetune}. However, this practice raises concerns about compromising the safety alignment of LLMs \cite{qi2023fine,yang2023shadow,zhan2023removing}, as harmful data may be present in users' datasets, whether intentionally or unintentionally. To address the risks of harmful fine-tuning, various defensive solutions \cite{huang2024booster,rosati2024representation,huang2024vaccine} have been proposed to ensure that fine-tuned LLMs can effectively refuse harmful queries. Specifically, these defense methods aim to limit the transferability of LLMs from harmless queries to harmful ones, which techniques are variants of the objectives of NTL. 
Actually, all existing NTL approaches can be applied to this task by regarding the alignment data as the source domain and the harmful data as the target domain. Then, target-specified NTL can be conducted to defend agaginst harmful fine-tuning attacks.

\section{Related Work}
Alongside a discussion of what is meant by LLM harmfulness,
this section covers two distinct strands of related work: measuring types of harm in LLMs, and LLMs for diverse annotation tasks. %First,

%Different kinds of 
Diverse undesirable LLM outputs, from toxic language to privacy invasion, have been discussed in the observed \cite{banko-etal-2020-unified}. Here we review the ones we include in our definition of ``harm.'' %definition. Plus, we review LLMs as judges. 
Toxic content can be elicited from both generative  \cite{deshpande2023toxicity} and masked LLMs \cite{ousidhoum-etal-2021-probing}. 
%Among ways 
To measure toxic or hateful language, some use APIs such as PerspectiveAPI \cite{lees2022new} or HateBERT \cite{caselli-etal-2021-hatebert}. \citet{openai2024gpt4technicalreport} report that GPT4 produces toxic content 0.78\% of the time, versus 6.48\% in GPT3.5.
%as opposed to GPT3.5 with 6.48\%. On the other hand,
\citet{dubey2024llama} report that llama3-70B produces harmful content 5\% of the time, %whereas the 405B model generates harm 3\% of the time. 
compared to 3\% in the 405B model.
Instead of %single value classifiers to measure harm, 
reporting an absolute rate, we focus on relative harmfulness of different LLMs. %, so we point to recent work on LLMs for annotation.

The first category of harm we consider is social stereotyping and bias. %discrimination. It has been shown that 
LLMs can perpetuate social bias based on gender, race, religion etc. \cite{lin-etal-2022-gendered,bender2021dangers,field-etal-2021-survey,gupta-etal-2024-sociodemographic,andriushchenko2024agentharm,mazeika2024harmbench}. This can marginalize these groups more, and results in less fair model performance. \citet{guo2024hey} designed a competition to elicit biased output from LLMs to assess the perception of bias from non-expert users. %The first part of our work is similar to this analysis, but 
We also intentionally elicit harmful output, going %we look at other types of harms besides bias.
beyond social bias.

%When the models become stronger, they become more robust to jailbreaking attacks to elicit harmful content. However, there are datasets that can still jailbreak models to produce harmful content \cite{andriushchenko2024agentharm,mazeika2024harmbench}.

Our second category of harm is offensiveness and toxicity, which %. As opposed to stereotyping or social discrimination, this harm 
%is more subjective and harder to define than the previous category, so there 
lacks an established definition due to its greater subjectivity \cite{dev-etal-2022-measures,korre-etal-2023-harmful}. We include hate speech (HS) and abusive language as toxic content. HS can be defined as expressions of offensive and discriminatory discourse towards a group or an individual based on characteristics such as race, religion, nationality, or other group characteristics \cite{john2000hate,jahan2023systematic,basile2019semeval,davidson2017automated}. It includes racism, negative stereotyping, and sexist language. On the other hand, abusive language is content with inappropriate words such as profanity or disrespectful terms. It also includes psychological threats such as humiliation. %or constant criticism. %Toxic content can be elicited from both generative models \cite{deshpande2023toxicity} and masked language models \cite{ousidhoum-etal-2021-probing}.

%In addition to obvious toxic content, LLMs can generate diverse implicit toxic outputs using reinforcement learning with favoring toxic content in the reward function \cite{wen-etal-2023-unveiling}.  Regarding the subjectivity of this task, \cite{korre-etal-2023-harmful} reannotate the existing datasets with different definitions of toxicity and show that broader definitions result in more robust annotations, but interannotator agreements are still lower than 0.5. \cite{dev-etal-2022-measures} also point out the lack of definition for bias and harm in general and propose a framework to guide researchers during the development of bias measures.

Harm can be implicit, such as privacy invasion
%We are also interested in privacy invasion,
where there is 
leakage of personal information. %leakage from the model. 
%LLMs can memorize details of the training data and then leak private information such as 
This includes social security numbers, phone numbers, or bank account information \cite{carlini2021extracting,brown2022does}. 
%There are several frameworks to test the privacy of LLMs \cite{li2024llm} and generate data for personal attribute inference \cite{yukhymenko2024synthetic,kim2024propile}.

%Our definition of harm includes hate speech (HS) as well. HS can be defined as \textcolor{red}{expressions of} hatred towards a social group, the humiliation of the members of a group, or %communication disparaging  extreme disparagement of a person or a group based on race, color, ethnicity, gender, sexual orientation, nationality, religion, or other group characteristics .

For data annotation, LLMs
%Besides text generation, 
%LLMs have been used to annotate data because they 
can %be comparable to 
replace humans for some tasks, %and make the annotation process faster and cheaper 
with gains in efficiency and economy \cite{tan2024large}. They have been used for sociological annotations such as for classification of stance, bots or humor  \cite{ziems2024can,zhu2023can}. For tasks such as topic and frame detection or sentence segmentation they can surpass crowd-workers
%Some works show that they can surpass crowd-workers for some tasks such as topic and frame detection or sentence segmentation %into research aspects 
\cite{he2024if,gilardi2023chatgpt}. Some have argued that human-LLM collaboration results in more reliable annotation \cite{he2024if,zhang2023llmaaa,kim2024meganno+}. In addition to more objective tasks,
%LLMs have been used to annotate data %even 
they have been applied to subjective annotations such as offensiveness and abusiveness \cite{pavlovic-poesio-2024-effectiveness,zhu2023can,he2023annollm}, %. For example, LLMs are used as judges to rank responses from different LLMs 
or to rank outputs from different LLMs based on helpfulness, accuracy, or relevance \cite{zheng2023judging,lin2024wildbench,dubois2024length}. These works tend to focus on human-large LLM interactions, whereas we focus on single-turn responses from smaller LLMs. We inspire from \citet{zheng2023judging} but we only measure harm instead of overall performance. Plus, we use 3 LLMs to evaluate smaller LLMs.
This work identifies signal collapse as a critical bottleneck in one-shot neural network pruning. Performance loss in pruned networks is due to \textbf{signal collapse} in addition to the removal of critical parameters. We propose \textbf{REFLOW} (\textbf{Re}storing \textbf{F}low of \textbf{Low}-variance signals), a simple yet effective method that mitigates signal collapse without computationally expensive weight updates. By focusing on signal preservation, REFLOW highlights the importance of mitigating signal collapse in sparse networks and enables magnitude pruning to match or surpass state-of-the-art one-shot pruning methods such as CHITA, CBS, and WF.

REFLOW consistently achieves state-of-the-art accuracy across diverse architectures, restoring ResNeXt-101 from under 4.1\% to 78.9\% top-1 accuracy at 80\% sparsity on ImageNet. Its lightweight design makes it a practical solution for both research and deployment, delivering high-quality sparse models without the overhead of traditional approaches. These findings challenge the traditional emphasis on weight selection strategies and underscore the critical role of signal propagation for achieving high-quality sparse networks in the context of one-shot pruning.









%% The file named.bst is a bibliography style file for BibTeX 0.99c
\bibliographystyle{named}
\bibliography{ijcai25}

\end{document}

