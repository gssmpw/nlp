%%%% ijcai25.tex

\typeout{IJCAI--25 Instructions for Authors}

% These are the instructions for authors for IJCAI-25.

\documentclass{article}
\pdfpagewidth=8.5in
\pdfpageheight=11in

% The file ijcai25.sty is a copy from ijcai22.sty
% The file ijcai22.sty is NOT the same as previous years'
\usepackage{ijcai25}

\usepackage[table]{xcolor}

% Use the postscript times font!
\usepackage{times}
\usepackage{soul}
\usepackage{url}
% \usepackage[hidelinks]{hyperref}
\usepackage[breaklinks,colorlinks,citecolor=teal,urlcolor=teal]{hyperref}
% \usepackage[breaklinks,colorlinks]{hyperref}
\usepackage[utf8]{inputenc}
\usepackage[small]{caption}
\usepackage{graphicx}
\usepackage{amsmath}
\usepackage{amsthm}
\usepackage{booktabs}
\usepackage{algorithm}
\usepackage{algorithmic}
\usepackage[switch]{lineno}
\usepackage{amssymb,mathtools}
\usepackage{cleveref}
\usepackage{utfsym}
\usepackage{multirow}
\newcommand{\blue}[1]{\textcolor{blue}{#1}}
\usepackage{enumitem}
\usepackage{pifont}



% Comment out this line in the camera-ready submission
% \linenumbers

\urlstyle{same}

% the following package is optional:
%\usepackage{latexsym}

% See https://www.overleaf.com/learn/latex/theorems_and_proofs
% for a nice explanation of how to define new theorems, but keep
% in mind that the amsthm package is already included in this
% template and that you must *not* alter the styling.
\newtheorem{example}{Example}
\newtheorem{theorem}{Theorem}

% Following comment is from ijcai97-submit.tex:
% The preparation of these files was supported by Schlumberger Palo Alto
% Research, AT\&T Bell Laboratories, and Morgan Kaufmann Publishers.
% Shirley Jowell, of Morgan Kaufmann Publishers, and Peter F.
% Patel-Schneider, of AT\&T Bell Laboratories collaborated on their
% preparation.

% These instructions can be modified and used in other conferences as long
% as credit to the authors and supporting agencies is retained, this notice
% is not changed, and further modification or reuse is not restricted.
% Neither Shirley Jowell nor Peter F. Patel-Schneider can be listed as
% contacts for providing assistance without their prior permission.

% To use for other conferences, change references to files and the
% conference appropriate and use other authors, contacts, publishers, and
% organizations.
% Also change the deadline and address for returning papers and the length and
% page charge instructions.
% Put where the files are available in the appropriate places.


% PDF Info Is REQUIRED.

% Please leave this \pdfinfo block untouched both for the submission and
% Camera Ready Copy. Do not include Title and Author information in the pdfinfo section
\pdfinfo{
/TemplateVersion (IJCAI.2025.0)
}

\title{Toward Robust Non-Transferable Learning: A Survey and Benchmark}


% Single author syntax
\author{
  \vspace{-5mm}
Ziming Hong\thanks{}
\and 
Yongli Xiang$^*$
\And
Tongliang Liu$^{\dagger}$\\
\affiliations
Sydney AI Centre, The University of Sydney\\
\emails
hoongzm@gmail.com,
yxia0023@uni.sydney.edu.au,
tongliang.liu@sydney.edu.au
}



\begin{document}


\twocolumn[{%
\renewcommand\twocolumn[1][]{#1}%
\maketitle
\begin{center}
  \vspace{-15mm}
  \centering\includegraphics[width=\linewidth]{figs/motiv1.pdf}\\
  \vspace{2mm}
  \centering\includegraphics[width=1.0\linewidth]{figs/motiv2.pdf}
  \vspace{-7mm}
  \captionof{figure}{We systematically review non-transferable learning (NTL) and introduce \texttt{NTLBench}, an unified framework for benchmarking NTL. This figure compares 5 methods (\textcolor[HTML]{7884AC}{NTL}, \textcolor[HTML]{7884AC}{CUTI-domain}, \textcolor[HTML]{7884AC}{H-NTL}, \textcolor[HTML]{7884AC}{SOPHON}, \textcolor[HTML]{7884AC}{CUPI-domain}) on CIFAR \& STL with VGG-13, evaluating pre-training performance and robustness
  against 5 \textcolor[HTML]{E4785F}{source domain fine-tuning} attacks, 4 \textcolor[HTML]{8151BA}{target domain fine-tuning} attacks, and 6 \textcolor[HTML]{59A4B7}{source-free domain adaptation} attacks (higher value means better robustness). 
  \texttt{NTLBench} will be released soon at \url{https://github.com/tmllab/NTLBench}.
  } 
  \vspace{3mm}
\label{fig:opening}
\end{center}%
}]

\renewcommand{\thefootnote}{*}
\footnotetext{Equal contribution. $^{\dagger}$Corresponding author.}
\renewcommand{\thefootnote}{\arabic{footnote}}


\begin{abstract}
  \vspace{-0.2mm}
  Over the past decades, researchers have primarily focused on improving the generalization abilities of models, with limited attention given to regulating such generalization. However, the ability of models to generalize to unintended data (e.g., harmful or unauthorized data) can be exploited by malicious adversaries in unforeseen ways, potentially resulting in violations of model ethics. Non-transferable learning (NTL), a task aimed at reshaping the generalization abilities of deep learning models, was proposed to address these challenges. While numerous methods have been proposed in this field, a comprehensive review of existing progress and a thorough analysis of current limitations remain lacking. In this paper, we bridge this gap by presenting the first comprehensive survey on NTL and introducing \texttt{NTLBench}, the first benchmark to evaluate NTL performance and robustness within a unified framework. Specifically, we first introduce the task settings, general framework, and criteria of NTL, followed by a summary of NTL approaches. Furthermore, we emphasize the often-overlooked issue of robustness against various attacks that can destroy the non-transferable mechanism established by NTL. Experiments conducted via \texttt{NTLBench} verify the limitations of existing NTL methods in robustness. Finally, we discuss the practical applications of NTL, along with its future directions and associated challenges.
  \vspace{-0.5mm}
\end{abstract}




\section{Introduction}

Video generation has garnered significant attention owing to its transformative potential across a wide range of applications, such media content creation~\citep{polyak2024movie}, advertising~\citep{zhang2024virbo,bacher2021advert}, video games~\citep{yang2024playable,valevski2024diffusion, oasis2024}, and world model simulators~\citep{ha2018world, videoworldsimulators2024, agarwal2025cosmos}. Benefiting from advanced generative algorithms~\citep{goodfellow2014generative, ho2020denoising, liu2023flow, lipman2023flow}, scalable model architectures~\citep{vaswani2017attention, peebles2023scalable}, vast amounts of internet-sourced data~\citep{chen2024panda, nan2024openvid, ju2024miradata}, and ongoing expansion of computing capabilities~\citep{nvidia2022h100, nvidia2023dgxgh200, nvidia2024h200nvl}, remarkable advancements have been achieved in the field of video generation~\citep{ho2022video, ho2022imagen, singer2023makeavideo, blattmann2023align, videoworldsimulators2024, kuaishou2024klingai, yang2024cogvideox, jin2024pyramidal, polyak2024movie, kong2024hunyuanvideo, ji2024prompt}.


In this work, we present \textbf{\ours}, a family of rectified flow~\citep{lipman2023flow, liu2023flow} transformer models designed for joint image and video generation, establishing a pathway toward industry-grade performance. This report centers on four key components: data curation, model architecture design, flow formulation, and training infrastructure optimization—each rigorously refined to meet the demands of high-quality, large-scale video generation.


\begin{figure}[ht]
    \centering
    \begin{subfigure}[b]{0.82\linewidth}
        \centering
        \includegraphics[width=\linewidth]{figures/t2i_1024.pdf}
        \caption{Text-to-Image Samples}\label{fig:main-demo-t2i}
    \end{subfigure}
    \vfill
    \begin{subfigure}[b]{0.82\linewidth}
        \centering
        \includegraphics[width=\linewidth]{figures/t2v_samples.pdf}
        \caption{Text-to-Video Samples}\label{fig:main-demo-t2v}
    \end{subfigure}
\caption{\textbf{Generated samples from \ours.} Key components are highlighted in \textcolor{red}{\textbf{RED}}.}\label{fig:main-demo}
\end{figure}


First, we present a comprehensive data processing pipeline designed to construct large-scale, high-quality image and video-text datasets. The pipeline integrates multiple advanced techniques, including video and image filtering based on aesthetic scores, OCR-driven content analysis, and subjective evaluations, to ensure exceptional visual and contextual quality. Furthermore, we employ multimodal large language models~(MLLMs)~\citep{yuan2025tarsier2} to generate dense and contextually aligned captions, which are subsequently refined using an additional large language model~(LLM)~\citep{yang2024qwen2} to enhance their accuracy, fluency, and descriptive richness. As a result, we have curated a robust training dataset comprising approximately 36M video-text pairs and 160M image-text pairs, which are proven sufficient for training industry-level generative models.

Secondly, we take a pioneering step by applying rectified flow formulation~\citep{lipman2023flow} for joint image and video generation, implemented through the \ours model family, which comprises Transformer architectures with 2B and 8B parameters. At its core, the \ours framework employs a 3D joint image-video variational autoencoder (VAE) to compress image and video inputs into a shared latent space, facilitating unified representation. This shared latent space is coupled with a full-attention~\citep{vaswani2017attention} mechanism, enabling seamless joint training of image and video. This architecture delivers high-quality, coherent outputs across both images and videos, establishing a unified framework for visual generation tasks.


Furthermore, to support the training of \ours at scale, we have developed a robust infrastructure tailored for large-scale model training. Our approach incorporates advanced parallelism strategies~\citep{jacobs2023deepspeed, pytorch_fsdp} to manage memory efficiently during long-context training. Additionally, we employ ByteCheckpoint~\citep{wan2024bytecheckpoint} for high-performance checkpointing and integrate fault-tolerant mechanisms from MegaScale~\citep{jiang2024megascale} to ensure stability and scalability across large GPU clusters. These optimizations enable \ours to handle the computational and data challenges of generative modeling with exceptional efficiency and reliability.


We evaluate \ours on both text-to-image and text-to-video benchmarks to highlight its competitive advantages. For text-to-image generation, \ours-T2I demonstrates strong performance across multiple benchmarks, including T2I-CompBench~\citep{huang2023t2i-compbench}, GenEval~\citep{ghosh2024geneval}, and DPG-Bench~\citep{hu2024ella_dbgbench}, excelling in both visual quality and text-image alignment. In text-to-video benchmarks, \ours-T2V achieves state-of-the-art performance on the UCF-101~\citep{ucf101} zero-shot generation task. Additionally, \ours-T2V attains an impressive score of \textbf{84.85} on VBench~\citep{huang2024vbench}, securing the top position on the leaderboard (as of 2025-01-25) and surpassing several leading commercial text-to-video models. Qualitative results, illustrated in \Cref{fig:main-demo}, further demonstrate the superior quality of the generated media samples. These findings underscore \ours's effectiveness in multi-modal generation and its potential as a high-performing solution for both research and commercial applications.
\section{Preliminary} \label{sec:preliminary}
\paragraph{Random variable and distribution.} Let $\mathcal{X} = \mathcal{X}_v \times \mathcal{X}_t$ denote the input space, where $\mathcal{X}_v$ and $\mathcal{X}_t$ correspond to the visual and textual feature spaces, respectively. Similarly, let $\mathcal{Y}$ denote the response space. We define the random variables $\mathbf{X} = (X_v, X_t) \in \mathcal{X}$ and $Y \in \mathcal{Y}$, where $\mathbf{X}$ is the sequence of tokens that combine visual and text input queries, and $Y$ represents the associated response tokens. The joint population is denoted by $P_{\mathbf{X}Y}$, with marginals $P_{\mathbf{X}}$, $P_{Y}$, and the conditional distribution $P_{Y|\mathbf{X}}$. For subsequent sections, $P_{\mathbf{X}Y}$ refers to the instruction tuning data distribution which we consider as in-distribution (ID). 

\paragraph{MLLM and visual instruction tuning.} MLLM usually consists of three components: (1) a visual encoder, (2) a vision-to-language projector, and (3) an LLM that processes a multimodal input sequence to generate a valid textual output $y$ in response to an input query $\mathbf{x}$. An MLLM can be regarded as modeling a conditional distribution $P_{\theta}(y|\mathbf{x})$, where $\theta$ is the model parameters. To attain the multimodal conversation capability, MLLMs commonly undergo a phase so-called \textit{visual instruction tuning} \cite{liu2023visual, dai2023instructblip} with an autoregressive objective as follows:
{
\begin{align} \label{eq::1}
    % & \min_{\theta\in\Theta} \mathbb{E}_{\mathbf{x},y\sim P_{\mathbf{X}Y}} [-\log P_{\theta}(y|\mathbf{x})] \nonumber \\
     \min_{\theta\in\Theta} \mathbb{E}_{\mathbf{x},y\sim P_{\mathbf{X}Y}} [\sum_{l=0}^{L}-\log P_{\theta}(y_{l}|\mathbf{x},y_{<l})],
\end{align}}
where $L$ is a sequence length and $y=(y_{0},...,y_{L})$. After being trained by Eq. \eqref{eq::1}, MLLM produces a response given a query of any possible tasks represented by text.

\paragraph{Evaluation of open-ended generations.} 

(M)LLM-as-a-judge method \cite{zheng2023judging, kim2023prometheus} is commonly adopted to evaluate open-ended generation. In this paradigm, a judge model produces preference scores or rankings for the responses given a query, model responses, and a scoring rubric. Among the evaluation metrics, the \emph{win rate} (Eq. \eqref{eq:win_rate}) is one of the most widely used and representative.


\begin{definition}[\textbf{Win Rate}] Given a parametric reward function $r:\mathcal{X}\times \mathcal{Y}\rightarrow \mathbb{R}$, the 
win rate (WR) of model $P_{\theta}$ w.r.t. $P_{\mathbf{X}Y}$ are defined as follows: 
\begin{equation} \label{eq:win_rate}
\begin{split}
    &\text{WR}(P_{\mathbf{X}Y};\theta):=\mathbb{E}_{\begin{subarray}{l} \mathbf{x},y \sim P_{\mathbf{X}Y} \\ \hat{y} \sim P_{\theta}(\cdot|\mathbf{x}) \end{subarray}}[\mathbb{I}(r(\mathbf{x},\hat{y}) > r(\mathbf{x},y))],
\end{split}
\end{equation}
where $\mathbb{I}(\cdot)$ is  the indicator function.
\end{definition}
Here, the reward function $r(\cdot,\cdot)$, can be any possible (multimodal) LLMs such as GPT-4o \cite{hurst2024gpt}.




\section{Approaches for NTL}
\label{sec:immeNTL}

Target-specified NTL approaches contain fundamental solutions for NTL, and thus, we first review them in \Cref{sec:target-specified}. Then, in \Cref{sec:source-only}, we review how existing works implement source-only NTL in the absence of a target domain.


\subsection{Target-Specified NTL}
\label{sec:target-specified}

Briefly, in target-specified setting, the target domain is known and we aim to restrict the generalization of a deep learning model from the source domain toward the certain target domain. 
Existing methods perform target-domain regularization either on the feature space or the output space, as we summarized in \Cref{tab:summary} (\textbf{Non-Transferable Approach} column). For more details, we introduce existing strategies as follows:

\paragraph{Output space regularization.} Output-space regularizations directly manipulate the model logits on the target domain. More specifically, these operations can be categorized into \textit{untargeted regularization} and \textit{targeted regularization}. \textit{Untargeted regularization} \cite{wang2021non,wang2023model,zhou2024archlock,peng2024map} could usually be formalized as a maximizing optimization problem, where existing methods implement this regularization by maximizing the KL divergence between the model outputs and the real labels, thus disturbing the model predictions on the target domain. However, such untargeted regularizations may face convergence issues \cite{deng2024sophon}. \textit{Targeted regularization} \cite{wang2023domain,deng2024sophon} found a proxy task on the target domain (i.e., modify the labels), thus converting the maximization objective in untargeted regularization to a minimization optimization problem. 
DSO \cite{wang2023domain} transforms the correct labels to error labels without overlap (e.g., $y_{\text{err}} = y+1$) and uses error labels as the target-domain supervision. 
H-NTL \cite{hong2024improving} first disentangle the content and style factors via a variation inference framework~\cite{blei2017variational,yao2021instance,von2021self,lin2024cs,lin2025learning}, and then, they learn the NTL model by fitting the contents of the source domain and the style of the target domain. Due to the assumption that the style is approximately to be independent to the content representations, the non-transferability could be implemented.
SOPHON \cite{deng2024sophon} aims at both image classification and generation tasks. For classification, they propose to modify the cross-entropy (CE) loss to its inverse version (i.e., modify the label $y$ to $1-y$) or calculate the KL divergence between the model outputs and a uniform distribution. For generation, SOPHON proposes to use a Denial of Service (DoS) loss, i.e., let the diffusion model fit a zero matrix at each step. Compared to untargeted regularizations, targeted regularizations always have better convergence.

\paragraph{Feature space regularization.} Feature-space regularizations further reduce the similarity between feature representations from different domains, thus restricting the transferability directly on the feature space. Feature-space regularizations can also be categorized into \textit{untargeted} and \textit{targeted} strategies, depending on whether they directly enlarge the distribution gap through a maximization objective or convert it to a minimization problem by finding a proxy target. For \textit{untargeted regularization}, existing methods \cite{wang2021non,zeng2022unsupervised} propose to maximize the maximum mean discrepancy (MMD) loss between the feature representations from different domains, where MMD measures the distribution discrepancy. 
For \textit{targeted regularization}, UNTL \cite{zeng2022unsupervised} proposes to build an auxiliary domain classifier with feature representations from different domains as inputs. By minimizing the domain-classification loss, the domain classifier could help the NTL model learn domain-distinct representations. 
NTP \cite{ding2024non} aims to minimize the Fisher Discriminant Analysis (FDA) term \cite{shao2022not} in the target domain. Specifically, a smaller FDA value indicates a reduced difference in class means and increased feature variance within each class, which is associated with poorer target domain performance.



\subsection{Source-Only NTL}
\label{sec:source-only}

Under the assumption that only source domain data is available, existing works~\cite{wang2021non,wang2023model,wang2023domain,hong2024improving} take various data augmentation methods to obtain auxiliary domains from the source domain and see them as the target domain. 
Thus, the source-only NTL problem can be solved by target-specified NTL approaches. 
These augmentation methods can be split into the following three categories: 


\paragraph{Adversarial domain generation.} 
Wang \textit{et al.} \shortcite{wang2021non} use generative adversarial network (GAN) \cite{mirza2014conditional,chen2016infogan} 
to synthesize fake images from the source domain and see them as the target domain.
They train the GAN by controlling the distance and direction of the synthetic distributions to the real source domain, thus enhancing the diversity of synthetic samples and improving the degradation of any distribution with shifts to the real source domain. 
CUTI-domain \cite{wang2023model} and CUPI-domain \cite{wang2024say} add Gaussian noise to the GAN-based adaptive instance normalization (AdaIN) \cite{huang2017arbitrary} to obtain synthetic samples with random styles. They use both the synthetic samples from AdaIN and Wang \textit{et al.} \shortcite{wang2021non} as the target domain. 
MAP \cite{peng2024map} also follows the GAN framework.
 They additionally add a mutual information (MI) minimization term to enhance the variation between synthetic samples and the real source domain samples, ensuring more distinct style features.

\paragraph{Strong image augmentation.} H-NTL \cite{hong2024improving} conducts strong image augmentation \cite{sohn2020fixmatch,cubuk2020randaugment,huang2023harnessing} on real source domain data.
Strong image augmentations (e.g., blurring, sharpness, solarize) do not influence the contents but significantly change the image styles, thus imposing interventions~\cite{von2021self} on the style factors in images. Then, all augmented images are treated as the target domain for training source-only NTL.

\paragraph{Perturbation-based method.} DSO \cite{wang2023domain} proposes to minimize the worst-case risk on the uncertainty set~\cite{sagawa2019distributionally,huang2023robust,wang2023defending} over the source domain distribution, where the risk is empirically calculated through a classification loss between the model predictions and the error label.

% \vspace{-1mm}
\section{Post-Training Robustness of NTL}
\label{sec:robustness}

NTL models are expected to keep the non-transferability after malicious attacks.
However, not all existing works evaluate the robustness of their method, as we listed in \Cref{tab:summary} (the last column). 
In this section, we review the robustness of the source and target domains as considered in previous works.

\subsection{Robustness Against Source Domain Attack} 
\label{sec:robustness_src}

Earlier evaluations in \cite{wang2021non,wang2023model} show that NTL models are still resistant to state-of-art watermark removal attacks when up to 30\% source domain data are available for attack. Hong \textit{et al.} \shortcite{hong2024your} further investigate the robustness of NTL and propose TransNTL, demonstrating that non-transferability can be destroyed using less than 10\% of the source domain data. Specifically,
they find NTL \cite{wang2021non} and CUTI-domain \cite{wang2023model} inevitably result in significant generalization impairments on slightly perturbed source domains \cite{hendrycks2019benchmarking,cubuk2020randaugment}. Accordingly, they propose TransNTL to fine-tune the NTL model under an impairment repair self-distillation framework, where the source-domain predictions are used to teach the model itself how to predict on perturbed source domains. As a result, the fine-tuned model is just like a SL model without the non-transferability. Furthermore, they also propose a defense method to fix this loophole by pre-repairing the generalization impairments in perturbed source domains.
Specifically, they add a defense regularization term on existing NTL and CUT-domain training. Minimizing the defense regularization term enables the NTL model to exhibit source-domain consistent behaviors on perturbed source-domain data, thus enhancing the robustness against TransNTL attack.

\subsection{Robustness Against Target Domain Attack} 
\label{sec:robustness_tgt}

Fine-tuning the NTL models using target domain data is a more direct strategy to break the non-transferability if malicious attackers have access to some labeled target domain data. However, most existing methods \cite{wang2021non,wang2023model,zeng2022unsupervised,wang2023domain,hong2024improving,peng2024map} ignore the robustness of their methods against target-domain fine-tuning attacks. SOPHON \cite{deng2024sophon} formally proposes the problem of non-fine-tunable learning, which aims at ensuring the target-domain performance could still be poor after being fine-tuned using target domain data.
Their main idea is to involve the fine-tuning process in training stage. 
Specifically, they leverage model agnostic meta-learning (MAML) \cite{finn2017model} to simulate multiple-step fine-tuning for the current model on the target domain. Then, they add per-step risk of the target domain as the total target-domain risk. By maximizing the total target-domain risk, the robustness against target-domain attacks can be enhanced. 
ArchLock \cite{zhou2024archlock} aims to find the most non-transferable network architectures \cite{liu2018darts,real2019regularized}, where they also implicitly consider the robustness on the target domain. Specifically, they maximize the \textit{minimum risk} of an architecture on the target domain in searching the non-transferable architectures. The minimum risk is found by searching the optimal \textit{parameters} of the \textit{architecture} with the minimum task loss on the target domain.
 
However, labeled target domain data being available to malicious attackers is a strong assumption that may not always hold in practical scenarios. A more realistic scenario is that the attacker only has access to some unlabeled target domain data. Whether NTL can resist attacks driven by unlabeled target domain data has not yet been studied. 


\newcommand{\cods}[1]{\textcolor[RGB]{51,68,103}{#1}}
\newcommand{\codt}[1]{\textcolor[RGB]{156,77,93}{#1}}

\newcommand{\e}[2]{{#1}}
\newcommand{\tl}[2]{\begin{tabular}[c]{@{}c@{}} \cods{#1}\\\codt{#2}\end{tabular}}
\newcommand{\tln}[2]{\begin{tabular}[c]{@{}c@{}} #1\\#2\end{tabular}}

\newcommand{\tc}[2]{\cods{#1}&\codt{#2}}

\newcommand{\tlds}[2]{\begin{tabular}[c]{@{}c@{}} \cods{#1}\\\cods{(#2)}\end{tabular}}
\newcommand{\tldt}[2]{\begin{tabular}[c]{@{}c@{}} \codt{#1}\\\codt{(#2)}\end{tabular}}

\newcommand{\bestoo}[0]{\cellcolor[HTML]{FFEEEB}}
\newcommand{\besto}[0]{\cellcolor[HTML]{E7F2F5}}
\newcommand{\bestt}[0]{\cellcolor[HTML]{faf1d8}}


\setlength{\tabcolsep}{3pt}
\setlength{\fboxsep}{0pt}

\begin{table*}[ht!]
    % \scriptsize
    \tiny
    % \small
    \centering
    \begin{tabular}{c|cc|cc|cc|cc|cc|cc|cc|cc|cc||cc}
      \toprule
      & 
      \multicolumn{2}{c|}{\textbf{Digits}} &
      \multicolumn{2}{c|}{\textbf{RMNIST}} &
      \multicolumn{2}{c|}{\textbf{CIFAR \& STL}} &
      \multicolumn{2}{c|}{\textbf{VisDA}} &
      \multicolumn{2}{c|}{\textbf{Office-Home}} &
      \multicolumn{2}{c|}{\textbf{DomainNet}} &
      \multicolumn{2}{c|}{\textbf{VLCS}} &
      \multicolumn{2}{c|}{\textbf{PCAS}} &
      \multicolumn{2}{c||}{\textbf{TerraInc}} &
      \multicolumn{2}{c}{\textbf{Avg.}} 
      \\

      \cmidrule(lr){2-21}  

      &
      \textbf{SA} $\uparrow$ & \textbf{TA} $\downarrow$ & 
      \textbf{SA} $\uparrow$ & \textbf{TA} $\downarrow$ & 
      \textbf{SA} $\uparrow$ & \textbf{TA} $\downarrow$ & 
      \textbf{SA} $\uparrow$ & \textbf{TA} $\downarrow$ & 
      \textbf{SA} $\uparrow$ & \textbf{TA} $\downarrow$ & 
      \textbf{SA} $\uparrow$ & \textbf{TA} $\downarrow$ & 
      \textbf{SA} $\uparrow$ & \textbf{TA} $\downarrow$ & 
      \textbf{SA} $\uparrow$ & \textbf{TA} $\downarrow$ & 
      \textbf{SA} $\uparrow$ & \textbf{TA} $\downarrow$ & 
      \textbf{SA} $\uparrow$ & \textbf{TA} $\downarrow$ 
      \\
        
      \midrule
      \midrule
      SL & 
      \tc{\e{97.7}{?}}{\e{56.0}{?}} &
      \tc{\e{99.2}{?}}{\e{62.4}{?}} &
      \tc{\e{88.2}{?}}{\e{65.7}{?}} &
      \tc{\e{86.8}{?}}{\e{37.7}{?}} &
      \tc{\e{66.4}{?}}{\e{36.9}{?}} &
      \tc{\e{45.6}{?}}{\e{ 9.9}{?}} &
      \tc{\e{79.9}{?}}{\e{56.9}{?}} &
      \tc{\e{89.5}{?}}{\e{47.3}{?}} &
      \tc{\e{93.6}{?}}{\e{14.9}{?}} &
      \tc{\e{83.0}{?}}{\e{43.0}{?}} \\

      \cmidrule(lr){1-21}
      \tln{NTL}{\cite{wang2021non}} & 
      \tc{\tlds{95.6}{-2.1}}{\tldt{12.2}{-43.8}} &
      \tc{\tlds{98.7}{-0.5}}{\tldt{12.3}{-50.1}} &
      \tc{\tlds{83.9}{-4.4}}{\tldt{ 9.9}{-55.8}} &
      \tc{\tlds{82.0}{-4.8}}{\tldt{10.9}{-26.8}} &
      \tc{\tlds{64.8}{-1.6}}{\tldt{32.4}{-4.5}} &
      \tc{\tlds{ 7.6}{-38.0}}{\tldt{ 1.4}{-8.6}} &
      \tc{\tlds{78.0}{-1.9}}{\tldt{27.1}{-29.8}} &
      \tc{\tlds{85.8}{-3.7}}{\tldt{18.0}{-29.2}} &
      \tc{\tlds{90.0}{-3.6}}{\tldt{ 8.8}{-6.1}} &
      \tc{\tlds{76.3}{-6.7}}{\tldt{14.8}{-28.3}} \\

      \cmidrule(lr){2-21}
      \tln{CUTI-domain}{\cite{wang2023model}} & 
      \tc{\tlds{97.0}{-0.8}}{\tldt{ 9.5}{-46.5}} &
      \tc{\tlds{99.2}{-0.1}}{\tldt{15.5}{-46.9}} &
      \tc{\tlds{85.1}{-3.2}}{\tldt{10.7}{-55.0}} &
      \tc{\tlds{85.3}{-1.5}}{\tldt{ 8.9}{-28.8}} &
      \tc{\besto\tlds{56.7}{-9.7}}{\besto\tldt{17.8}{-19.1}} &
      \tc{\tlds{14.0}{-31.7}}{\tldt{ 2.0}{-7.9}} &
      \tc{\besto\tlds{78.3}{-1.6}}{\besto\tldt{26.7}{-30.1}} &
      \tc{\tlds{88.4}{-1.1}}{\tldt{18.3}{-28.9}} &
      \tc{\besto\tlds{87.9}{-5.7}}{\besto\tldt{ 0.8}{-14.1}} &
      \tc{\besto\tlds{76.9}{-6.1}}{\besto\tldt{12.2}{-30.8}} \\

      \cmidrule(lr){2-21}
      \tln{H-NTL}{\cite{hong2024improving}} & 
      \tc{\tlds{97.5}{-0.2}}{\tldt{ 9.6}{-46.4}} &
      \tc{\besto\tlds{99.0}{-0.2}}{\besto\tldt{10.8}{-51.5}} &
      \tc{\besto\tlds{87.2}{-1.0}}{\besto\tldt{ 9.9}{-55.8}} &
      \tc{\besto\tlds{86.5}{-0.3}}{\besto\tldt{ 8.6}{-29.0}} &
      \tc{\tlds{51.1}{-15.2}}{\tldt{17.0}{-19.8}} &
      \tc{\besto\tlds{33.3}{-12.3}}{\besto\tldt{ 2.1}{-7.8}} &
      \tc{\tlds{79.2}{-0.8}}{\tldt{42.7}{-14.2}} &
      \tc{\tlds{89.1}{-0.3}}{\tldt{22.1}{-25.1}} &
      \tc{\tlds{88.4}{-5.2}}{\tldt{14.6}{-0.2}} &
      \tc{\tlds{79.0}{-4.0}}{\tldt{15.3}{-27.8}} \\

      \cmidrule(lr){2-21}
      \tln{SOPHON}{\cite{deng2024sophon}} & 
      \tc{\tlds{95.2}{-2.5}}{\tldt{ 9.9}{-46.1}} &
      \tc{\tlds{96.6}{-2.6}}{\tldt{38.8}{-23.6}} &
      \tc{\tlds{69.5}{-18.7}}{\tldt{24.8}{-40.9}} &
      \tc{\tlds{77.3}{-9.5}}{\tldt{10.9}{-26.8}} &
      \tc{\tlds{45.9}{-20.4}}{\tldt{17.6}{-19.3}} &
      \tc{\tlds{30.1}{-15.6}}{\tldt{ 2.5}{-7.4}} &
      \tc{\tlds{79.4}{-0.6}}{\tldt{29.5}{-27.4}} &
      \tc{\tlds{86.7}{-2.8}}{\tldt{21.6}{-25.7}} &
      \tc{\tlds{88.8}{-4.8}}{\tldt{ 7.1}{-7.7}} &
      \tc{\tlds{74.4}{-8.6}}{\tldt{18.1}{-25.0}} \\

      \cmidrule(lr){2-21}
      \tln{CUPI-domain}{\cite{wang2024say}} &  
      \tc{\besto\tlds{96.7}{-1.0}}{\besto\tldt{ 8.8}{-47.2}} &
      \tc{\tlds{98.8}{-0.4}}{\tldt{21.0}{-41.3}} &
      \tc{\tlds{86.0}{-2.3}}{\tldt{11.3}{-54.4}} &
      \tc{\tlds{84.6}{-2.2}}{\tldt{ 8.2}{-29.5}} &
      \tc{\tlds{11.6}{-54.7}}{\tldt{ 2.3}{-34.6}} &
      \tc{\tlds{ 0.8}{-44.9}}{\tldt{ 0.3}{-9.7}} &
      \tc{\tlds{77.5}{-2.5}}{\tldt{29.5}{-27.4}} &
      \tc{\besto\tlds{87.8}{-1.7}}{\besto\tldt{11.5}{-35.8}} &
      \tc{\tlds{82.4}{-11.1}}{\tldt{ 1.3}{-13.6}} &
      \tc{\tlds{69.6}{-13.4}}{\tldt{10.4}{-32.6}} \\

      \bottomrule
    \end{tabular}
    \vspace{-3mm}
    \caption{Comparison of SL and 5 NTL methods on multiple datasets. We report the \cods{source-domain accuracy} (\textbf{SA}) (\%) in \cods{blue} and \codt{target-domain accuracy} (\textbf{TA}) (\%) in \codt{red}. The best results of overall performance (OA) are highlighted in \colorbox[HTML]{E7F2F5}{blue background}. The accuracy drop compared to the pre-trained model is shown in brackets. The average performance of 9 datasets are shown in the last column (\textbf{Avg.}).}
    \label{tab:tgt-spec}
    \vspace{-4mm}
  \end{table*}
  
  
  


% \vspace{-1mm}
\section{Benchmarking NTL}
\label{sec:exp}

The post-training robustness has not been well-evaluated in NTL, which motivates us to build a comprehensive benchmark.
In this section, we first demonstrate the framework of our \texttt{NTLBench} (\Cref{sec:ntlbench}). Then, we present main results by conducting our \texttt{NTLBench} (\Cref{sec:ntlbenchresults}), including the pretrained NTL performance on multiple datasets, and the robustness of NTL baselines against different attacks. 


\subsection{\texttt{NTLBench}}
\label{sec:ntlbench}

We propose the first NTL benchmark (\texttt{NTLBench}), which contains a standard and unified training and evaluation process. \texttt{NTLBench} supports 5 SOTA NTL methods, 9 datasets (more than 116 domain pairs), 5 network architectures families, and 15 post-training attacks from 3 attack settings, providing more than 40,000 experimental configurations. 

\paragraph{Datasets.} Our \texttt{NTLBench} is compatible with: Digits (5 domains)~\cite{deng2012mnist,hull1994database,netzer2011reading,ganin2016domain,roy2018effects}, RotatedMNIST (3 domains)~\cite{ghifary2015domain}, CIFAR and STL (2 domains)~\cite{krizhevsky2009learning,coates2011analysis}, VisDA (2 domains)~\cite{peng2017visda}, Office-Home (4 domains)~\cite{venkateswara2017deep}, DomainNet (6 domains)~\cite{peng2019moment}, VLCS (4 domains)~\cite{fang2013unbiased}, PCAS (4 domains)~\cite{li2017deeper}, and TerraInc (5 domains)~\cite{beery2018recognition}. Different domains in any dataset share the same label space, but have distribution shifts, thus being suitable for evaluating NTL methods.

\paragraph{NTL baselines.}
\texttt{NTLBench} involves all open-source NTL methods: NTL~\cite{wang2021non}, CUTI-domain~\cite{wang2023model}, H-NTL~\cite{hong2024improving}, SOPHON~\cite{deng2024sophon}, CUPI-domain~\cite{wang2024say}. Besides, we also add a vanilla supervised learning (SL) as a reference.

\paragraph{Network architecture.}
The proposed \texttt{NTLBench} is compatible with multiple network architectures, including but not limited to: 
VGG~\cite{simonyan2014very}, ResNet~\cite{he2016deep}, WideResNet~\cite{zagoruyko2016wide}, ViT~\cite{dosovitskiy2020image}, SwinT~\cite{liu2021swin}.

\paragraph{Threat I: source domain fine-tuning (SourceFT).} \textit{Attacking goal}: SourceFT tries to destroy the non-transferability by fine-tuning the NTL model using a small set of source domain data. \textit{Attacking method}: \texttt{NTLBench} involves 5 methods, including four basic fine-tuning strategies\footnote{\label{initfc}initFC: re-initialize the last full-connect (FC) layer. direct: no re-initialize. all: fine-tune the whole model. FC: fine-tune last FC.}: initFC-all, initFC-FC, direct-FC, direct-all~\cite{deng2024sophon} and the special designed attack for NTL: TransNTL~\cite{hong2024your}. 


\paragraph{Threat II: target domain fine-tuning (TargetFT).} \textit{Attacking goal}: TargetFT tries to directly use labeled target domain data to fine-tune the NTL model, thus recovering target domain performance. \textit{Attacking method}: \texttt{NTLBench} use 4 basic fine-tuning strategies\textsuperscript{\ref{initfc}} leveraged in~\cite{deng2024sophon} as attack methods: initFC-all, initFC-FC, direct-FC, direct-all.

\paragraph{Threat III: source-free domain adaptation (SFDA).} \textit{Attacking goal}: We introduce SFDA to test whether using unlabeled target domain data poses a threat to NTL. \textit{Attacking method}: \texttt{NTLBench} involves 6 SOTA SFDA methods: SHOT~\cite{liang2020we}, CoWA~\cite{lee2022confidence}, NRC~\cite{yang2021exploiting}, PLUE~\cite{litrico2023guiding}, AdaContrast~\cite{chen2022contrastive}, and DIFO~\cite{tang2024source}.

\paragraph{Evaluation metric.} For source domain, we use source domain accuracy (\textbf{SA}) to evaluate the performance. Higher SA means lower influence of non-transferability to the source domain utility.
For target domain, we use target domain accuracy (\textbf{TA}) to evaluate the performance. Lower TA means better performance of non-transferability.
Besides, we calculate the overall performance (denoted as \textbf{OA}) of an NTL method as: $\text{OA}=(\text{SA}+(100\%-\text{TA}))/2$, with higher OA representing better overall performance of an NTL method. These evaluation metrics are applicable for both non-transferability performance and robustness against different attacks.


\subsection{Main Results and Analysis.}
\label{sec:ntlbenchresults}

Due to the limited space, we present main results obtained from our \texttt{NTLBench}. 
We first show the key implementation details, and then we present and analyse of our results.

\paragraph{Implementation details.}
Briefly, in pre-training stage, we sequentially pair $i$-th and ($i$+1)-th domains within a dataset for training. Each domain is randomly split into 8:1:1 for training, validation, and testing. The results for each dataset are averaged across domain pairs. NTL methods and the reference SL method are pretrained by up to 50 epochs. We search suitable hyper-parameters for each method by setting 5 values around their original value and choose the best value according to the best OA on validation set. All the batch size, learning rate, and optimizer are follow their original implementations. Following the original NTL paper~\cite{wang2021non}, we use VGG-13 without batch-normalization. All input images are resize to 64$\times$64. 
In attack stage, we use 10\% amount of the training set to perform attack. All attack results we reported are run on CIFAR \& STL. Attack training is up to 50 epochs.
We run all experiments on RTX 4090 (24G).


\paragraph{Non-transferability performance.} The non-transferability performance are shown in \Cref{tab:tgt-spec}, where we compare 5 NTL methods and SL on 9 datasets. From the results, all NTL methods generally effectively degrade source-to-target generalization, leading to a significant drop in TA compared to SL. However, in more complex datasets such as Office-Home and DomainNet, existing NTL methods fail to achieve a satisfactory balance between maintaining SA and degrading TA, highlighting their limitations. From the \textbf{Avg.} column, CUTI-domain reaches the overall best performance.
% \label{sec:ntlbench1}

\paragraph{Post-training robustness.} 
For \textbf{SourceFT} attack (\Cref{tab:atk_src}), fine-tuning each NTL model by using basic fine-tuning strategies on 10\% source domain data cannot directly recover the source-to-target generalization. However, all NTL methods are fragile when facing the TransNTL attack. For \textbf{TargetFT} attack (\Cref{tab:atk_tgt_label}), all NTL methods cannot fully resist supervised fine-tuning attack by using target domain data. In particular, fine-tuning all parameters usually results in better attack effectiveness. For \textbf{SFDA} (\Cref{tab:atk_tgt_sfda}), although the target domain data are unlabeled, advanced source-free unsupervised domain adaptation, leveraging self-supervised strategies, can still partially recover target domain performance. All these results verify the fragility of existing NTL methods. 




\begin{table}[t!]
    % \scriptsize
    \tiny
    % \small
    \centering
    % \hspace{-2mm}
    \begin{tabular}{@{\hspace{4pt}}c@{\hspace{3pt}}|c@{\hspace{2pt}}c@{\hspace{3pt}}|@{\hspace{3pt}}c@{\hspace{2pt}}c@{\hspace{3pt}}|@{\hspace{3pt}}c@{\hspace{3pt}}c@{\hspace{3pt}}|@{\hspace{3pt}}c@{\hspace{3pt}}c@{\hspace{3pt}}|@{\hspace{3pt}}c@{\hspace{3pt}}c@{\hspace{3pt}}}

        
    \toprule
  
      & 
      \multicolumn{2}{c|@{\hspace{3pt}}}{\textbf{NTL}} &
      \multicolumn{2}{c|@{\hspace{3pt}}}{\textbf{CUTI}} &
      \multicolumn{2}{c|@{\hspace{3pt}}}{\textbf{H-NTL}} &
      \multicolumn{2}{c|@{\hspace{3pt}}}{\textbf{SOPHON}} &
      \multicolumn{2}{@{\hspace{3pt}}c}{\textbf{CUPI}}
      \\
  
      \cmidrule(lr){2-11}
  
      &
      \textbf{SA} $\uparrow$ & \textbf{TA} $\downarrow$ & 
      \textbf{SA} $\uparrow$ & \textbf{TA} $\downarrow$ & 
      \textbf{SA} $\uparrow$ & \textbf{TA} $\downarrow$ & 
      \textbf{SA} $\uparrow$ & \textbf{TA} $\downarrow$ & 
      \textbf{SA} $\uparrow$ & \textbf{TA} $\downarrow$ 
      
      \\
      \midrule
      \midrule
      Pre-train & 
      \tc{\e{83.9}{?}}{\e{ 9.9}{?}} &
      \tc{\e{85.1}{?}}{\e{10.6}{?}} &
      \tc{\e{87.2}{?}}{\e{ 9.9}{?}} &
      \tc{\e{69.5}{?}}{\e{24.8}{?}} &
      \tc{\e{86.0}{?}}{\e{11.3}{?}} \\
      \cmidrule(lr){1-11}
  
      initFC-all & 
      \tc{\tlds{84.0}{+0.2}}{\tldt{ 9.8}{-0.1}} &
      \tc{\tlds{84.2}{-0.9}}{\tldt{10.6}{+0.0}} &
      \tc{\tlds{87.8}{+0.6}}{\tldt{16.2}{+6.3}} &
      \tc{\tlds{82.2}{+12.7}}{\tldt{38.1}{+13.3}} &
      \tc{\tlds{85.3}{-0.7}}{\tldt{11.4}{+0.1}} \\
      \cmidrule(lr){2-11}
  
      initFC-FC & 
      \tc{\tlds{84.2}{+0.3}}{\tldt{10.0}{+0.1}} &
      \tc{\tlds{85.4}{+0.3}}{\tldt{10.6}{+0.0}} &
      \tc{\tlds{87.2}{-0.1}}{\tldt{10.2}{+0.3}} &
      \tc{\tlds{71.9}{+2.4}}{\tldt{23.3}{-1.6}} &
      \tc{\tlds{85.9}{-0.1}}{\tldt{11.3}{+0.0}} \\
      \cmidrule(lr){2-11}
  
      direct-FC & 
      \tc{\tlds{84.0}{+0.2}}{\tldt{ 9.9}{+0.0}} &
      \tc{\tlds{85.2}{+0.2}}{\tldt{10.6}{+0.0}} &
      \tc{\tlds{87.3}{+0.1}}{\tldt{ 9.9}{+0.0}} &
      \tc{\tlds{74.3}{+4.8}}{\tldt{23.8}{-1.1}} &
      \tc{\tlds{86.1}{+0.1}}{\tldt{11.3}{+0.0}} \\
      \cmidrule(lr){2-11}
  
      direct-all & 
      \tc{\tlds{84.7}{+0.8}}{\tldt{ 9.8}{-0.1}} &
      \tc{\tlds{85.3}{+0.3}}{\tldt{10.9}{+0.3}} &
      \tc{\tlds{88.0}{+0.8}}{\tldt{10.1}{+0.2}} &
      \tc{\tlds{83.4}{+13.9}}{\tldt{32.2}{+7.4}} &
      \tc{\tlds{85.5}{-0.5}}{\tldt{11.3}{+0.0}} \\
      \cmidrule(lr){2-11}
  
      TransNTL & 
      \tc{\bestoo \tlds{81.7}{-2.2}}{\bestoo \tldt{44.3}{+34.4}} &
      \tc{\bestoo \tlds{81.3}{-3.8}}{\bestoo \tldt{61.0}{+50.3}} &
      \tc{\bestoo \tlds{86.3}{-1.0}}{\bestoo \tldt{63.7}{+53.8}} &
      \tc{\bestoo \tlds{83.8}{+14.3}}{\bestoo \tldt{60.1}{+35.3}} &
      \tc{\bestoo \tlds{83.1}{-2.9}}{\bestoo \tldt{60.6}{+49.3}} \\
  
      \bottomrule
    \end{tabular}
    \vspace{-3mm}
    \caption{NTL robustness against source domain fine-tuning (Source-\\FT). We show \cods{source-domain accuracy} (\textbf{SA}) (\%) and \codt{target-domain accuracy} (\textbf{TA}) (\%). The most serious threat (worst OA) to each NTL is marked as\colorbox[HTML]{fee8e4}{ red.} Accuracy drop from the pre-trained model is in ($\cdot$).}
    \label{tab:atk_src}
    \vspace{1mm}
    \begin{tabular}{@{\hspace{4pt}}c@{\hspace{3pt}}|c@{\hspace{2pt}}c@{\hspace{3pt}}|@{\hspace{3pt}}c@{\hspace{2pt}}c@{\hspace{3pt}}|@{\hspace{3pt}}c@{\hspace{3pt}}c@{\hspace{3pt}}|@{\hspace{3pt}}c@{\hspace{3pt}}c@{\hspace{3pt}}|@{\hspace{3pt}}c@{\hspace{3pt}}c@{\hspace{3pt}}}
      \toprule
    
        & 
        \multicolumn{2}{c|@{\hspace{3pt}}}{\textbf{NTL}} &
        \multicolumn{2}{c|@{\hspace{3pt}}}{\textbf{CUTI}} &
        \multicolumn{2}{c|@{\hspace{3pt}}}{\textbf{H-NTL}} &
        \multicolumn{2}{c|@{\hspace{3pt}}}{\textbf{SOPHON}} &
        \multicolumn{2}{@{\hspace{3pt}}c}{\textbf{CUPI}}
        \\
    
        \cmidrule(lr){2-11}
    
        &
        \textbf{SA} $\uparrow$ & \textbf{TA} $\downarrow$ & 
        \textbf{SA} $\uparrow$ & \textbf{TA} $\downarrow$ & 
        \textbf{SA} $\uparrow$ & \textbf{TA} $\downarrow$ & 
        \textbf{SA} $\uparrow$ & \textbf{TA} $\downarrow$ & 
        \textbf{SA} $\uparrow$ & \textbf{TA} $\downarrow$ 
        
        \\
        \midrule
        \midrule
        Pre-train & 
        \tc{\e{83.9}{?}}{\e{ 9.9}{?}} &
        \tc{\e{85.1}{?}}{\e{10.7}{?}} &
        \tc{\e{87.2}{?}}{\e{ 9.9}{?}} &
        \tc{\e{69.5}{?}}{\e{24.8}{?}} &
        \tc{\e{86.0}{?}}{\e{11.3}{?}} \\
        \cmidrule(lr){1-11}
    
        initFC-all & 
        \tc{\bestoo \tlds{23.9}{-60.0}}{\bestoo \tldt{37.8}{+27.9}} &
        \tc{\bestoo \tlds{13.3}{-71.8}}{\bestoo \tldt{15.9}{+5.3}} &
        \tc{\tlds{19.0}{-68.3}}{\tldt{10.4}{+0.5}} &
        \tc{\tlds{59.0}{-10.5}}{\tldt{68.5}{+43.7}} &
        \tc{\tlds{41.2}{-44.8}}{\tldt{53.1}{+41.8}} \\
        \cmidrule(lr){2-11}
    
        initFC-FC & 
        \tc{\tlds{33.9}{-50.0}}{\tldt{ 9.6}{-0.4}} &
        \tc{\tlds{30.2}{-54.9}}{\tldt{ 9.7}{-1.0}} &
        \tc{\tlds{19.1}{-68.1}}{\tldt{ 9.7}{-0.2}} &
        \tc{\tlds{21.6}{-48.0}}{\tldt{16.8}{-8.1}} &
        \tc{\tlds{21.8}{-64.2}}{\tldt{12.1}{+0.8}} \\
        \cmidrule(lr){2-11}
    
        direct-FC & 
        \tc{\tlds{64.2}{-19.7}}{\tldt{10.2}{+0.3}} &
        \tc{\tlds{38.0}{-47.1}}{\tldt{10.6}{-0.1}} &
        \tc{\tlds{87.1}{-0.1}}{\tldt{10.0}{+0.1}} &
        \tc{\tlds{70.5}{+1.0}}{\tldt{24.5}{-0.4}} &
        \tc{\tlds{78.6}{-7.4}}{\tldt{11.0}{-0.4}} \\
        \cmidrule(lr){2-11}
    
        direct-all & 
        \tc{\tlds{13.9}{-70.0}}{\tldt{17.6}{+7.7}} &
        \tc{\tlds{10.1}{-75.0}}{\tldt{ 8.8}{-1.9}} &
        \tc{\bestoo \tlds{84.7}{-2.5}}{\bestoo \tldt{53.3}{+43.4}} &
        \tc{\bestoo \tlds{68.0}{-1.6}}{\bestoo \tldt{72.9}{+48.1}} &
        \tc{\bestoo \tlds{51.9}{-34.1}}{\bestoo \tldt{58.4}{+47.1}} \\
    
        \bottomrule
      \end{tabular}
      \vspace{-3mm}
      \caption{NTL robustness against target domain fine-tuning (Target-\\FT).  We report \cods{source-domain accuracy} (\textbf{SA}) (\%) and \codt{target-domain accuracy} (\textbf{TA}) (\%). The most serious threat (best TA) to each NTL is marked as\colorbox[HTML]{fee8e4}{ red.} Accuracy drop from the pre-trained model is in ($\cdot$).}
      \vspace{-3mm}
      \label{tab:atk_tgt_label}
  \end{table}
  
\paragraph{More results.} 

Additional results and analysis on: various architectures, attack using different data amount, cross-domain/task, and visualizations (e.g., feature activation, t-SNE \cite{van2008visualizing}, GradCAM \cite{selvaraju2017grad}) will be released soon at our online page.
  
  
  \begin{table}[t!]
    % \scriptsize
    \tiny
    % \small
    \centering
    % \hspace{-2mm}
    \begin{tabular}{@{\hspace{4pt}}c@{\hspace{3pt}}|c@{\hspace{2pt}}c@{\hspace{3pt}}|@{\hspace{3pt}}c@{\hspace{2pt}}c@{\hspace{3pt}}|@{\hspace{3pt}}c@{\hspace{3pt}}c@{\hspace{3pt}}|@{\hspace{3pt}}c@{\hspace{3pt}}c@{\hspace{3pt}}|@{\hspace{3pt}}c@{\hspace{3pt}}c@{\hspace{3pt}}}
    \toprule
  
      & 
      \multicolumn{2}{c|@{\hspace{3pt}}}{\textbf{NTL}} &
      \multicolumn{2}{c|@{\hspace{3pt}}}{\textbf{CUTI}} &
      \multicolumn{2}{c|@{\hspace{3pt}}}{\textbf{H-NTL}} &
      \multicolumn{2}{c|@{\hspace{3pt}}}{\textbf{SOPHON}} &
      \multicolumn{2}{@{\hspace{3pt}}c}{\textbf{CUPI}}
      \\
  
      \cmidrule(lr){2-11}
  
      &
      \textbf{SA} $\uparrow$ & \textbf{TA} $\downarrow$ & 
      \textbf{SA} $\uparrow$ & \textbf{TA} $\downarrow$ & 
      \textbf{SA} $\uparrow$ & \textbf{TA} $\downarrow$ & 
      \textbf{SA} $\uparrow$ & \textbf{TA} $\downarrow$ & 
      \textbf{SA} $\uparrow$ & \textbf{TA} $\downarrow$ 
      
      \\
      \midrule
      \midrule
      Pre-train & 
      \tc{\e{83.9}{?}}{\e{ 9.9}{?}} &
      \tc{\e{85.1}{?}}{\e{10.7}{?}} &
      \tc{\e{87.2}{?}}{\e{ 9.9}{?}} &
      \tc{\e{69.5}{?}}{\e{24.8}{?}} &
      \tc{\e{85.5}{?}}{\e{11.3}{?}} \\
      \cmidrule(lr){1-11}
  
      SHOT & 
      \tc{\tlds{63.0}{-20.9}}{\tldt{29.6}{+19.7}} &
      \tc{\tlds{35.3}{-49.8}}{\tldt{34.7}{+24.0}} &
      \tc{\tlds{86.6}{-0.6}}{\tldt{41.9}{+32.0}} &
      \tc{\bestoo \tlds{64.8}{-4.8}}{\bestoo \tldt{56.7}{+31.9}} &
      \tc{\tlds{85.8}{+0.3}}{\tldt{11.3}{+0.0}} \\
      \cmidrule(lr){2-11}
  
      CoWA & 
      \tc{\tlds{81.1}{-2.8}}{\tldt{12.4}{+2.5}} &
      \tc{\tlds{84.0}{-1.1}}{\tldt{12.7}{+2.1}} &
      \tc{\tlds{87.2}{+0.0}}{\tldt{10.1}{+0.2}} &
      \tc{\tlds{69.2}{-0.4}}{\tldt{26.1}{+1.3}} &
      \tc{\tlds{85.7}{+0.2}}{\tldt{11.3}{+0.0}} \\
      \cmidrule(lr){2-11}
  
      NRC & 
      \tc{\tlds{57.7}{-26.2}}{\tldt{19.8}{+9.9}} &
      \tc{\tlds{39.4}{-45.7}}{\tldt{35.5}{+24.8}} &
      \tc{\tlds{87.3}{+0.1}}{\tldt{12.1}{+2.2}} &
      \tc{\tlds{66.6}{-3.0}}{\tldt{55.6}{+30.8}} &
      \tc{\tlds{86.0}{+0.5}}{\tldt{12.2}{+0.9}} \\
      \cmidrule(lr){2-11}
  
      PLUE &  
      \tc{\bestoo \tlds{71.5}{-12.4}}{\bestoo \tldt{52.8}{+42.9}} &
      \tc{\bestoo \tlds{76.1}{-9.0}}{\bestoo \tldt{63.8}{+53.1}} &
      \tc{\tlds{85.5}{-1.8}}{\tldt{20.1}{+10.2}} &
      \tc{\tlds{75.5}{+6.0}}{\tldt{41.1}{+16.3}} &
      \tc{\bestoo \tlds{82.4}{-3.2}}{\bestoo \tldt{43.6}{+32.3}} \\
      \cmidrule(lr){2-11}
  
      \tln{Ada-}{Contrast} & 
      \tc{\tlds{ 9.4}{-74.5}}{\tldt{ 9.8}{-0.1}} &
      \tc{\tlds{ 9.3}{-75.8}}{\tldt{10.0}{-0.7}} &
      \tc{\tlds{86.3}{-1.0}}{\tldt{12.1}{+2.2}} &
      \tc{\tlds{64.5}{-5.1}}{\tldt{33.4}{+8.6}} &
      \tc{\tlds{47.2}{-38.3}}{\tldt{11.3}{+0.0}} \\
      \cmidrule(lr){2-11}
  
      DIFO & 
      \tc{\tlds{ 9.2}{-74.7}}{\tldt{ 9.2}{-0.7}} &
      \tc{\tlds{ 9.2}{-75.9}}{\tldt{ 9.2}{-1.5}} &
      \tc{\bestoo \tlds{85.0}{-2.2}}{\bestoo \tldt{42.1}{+32.2}} &
      \tc{\tlds{56.3}{-13.2}}{\tldt{51.3}{+26.5}} &
      \tc{\tlds{48.4}{-37.1}}{\tldt{10.4}{-1.0}} \\
  
      \bottomrule
    \end{tabular}
    \vspace{-3mm}
    \caption{NTL robustness against source-free domain adaptation (SFDA). We show \cods{source-domain accuracy} (\textbf{SA}) (\%), \codt{target-domain accuracy} (\textbf{TA}) (\%), and accuracy drop from the pre-trained model is in ($\cdot$). The most serious threat (highest TA) to each NTL is in\colorbox[HTML]{fee8e4}{ red.}}
    \vspace{-2.5mm}
    \label{tab:atk_tgt_sfda}
  \end{table}
  
  
% \vspace{-1mm}
\section{Applications of NTL}
\label{sec:applications}


NTL supports different applications, depending on which data are used as source and target domain. We introduce two applications in model intellectual property (IP) protection and then the application of harmful fine-tuning defense. 

\paragraph{Ownership verification (OV).} OV is a passive IP protection manner, which aims to verify the ownership of a deep learning model \cite{cheng2021mid,lederer2023identifying}. NTL solves ownership verification by triggering misclassification on data with pre-defined triggers \cite{wang2021non,chen2024mark,guo2024zeromark}. For example, when training, we add a shallow trigger (only known by the model owner) on the original dataset data and see them as the target domain, while the original data without the trigger is regarded as the source domain. Then, target-specified NTL is used to train a model. Therefore, the ownership can be verified via observing the performance difference of a model on the original data and the data with the pre-defined trigger. For SL model, the shallow trigger has minor influence on the model performance, and thus, the model shows similar performance on original data and data with triggers. In contrast, the NTL model specific to this pre-defined trigger has high performance on the original data but random-guess-like performance on data with the trigger. This provides evidence for verifying the model's ownership.
% 
\paragraph{Applicability authorization (AA).} AA is an active IP protection approach that ensures models can only be effective on authorized data \cite{wang2021non,xu2024idea,si2024iclguard}. NTL solves AA by degrading the model generalization outside the authorized domain. Basic solution is to add a pre-defined trigger on original data (seen as source domain), and the original data without the correct triggers is regarded as the target domain. After training by NTL, the model will only perform well on authorized data (i.e., the data with the trigger). Any unauthorized data will be randomly predicted by the NTL model. Thus, AA can be achieved.




\paragraph{Safety alignment and harmful fine-tuning defense.} 
Fine-tuning large language models (LLMs) with user's own data for downstream tasks has recently become a popular online service \cite{huang2024harmful,openai2024finetune}. However, this practice raises concerns about compromising the safety alignment of LLMs \cite{qi2023fine,yang2023shadow,zhan2023removing}, as harmful data may be present in users' datasets, whether intentionally or unintentionally. To address the risks of harmful fine-tuning, various defensive solutions \cite{huang2024booster,rosati2024representation,huang2024vaccine} have been proposed to ensure that fine-tuned LLMs can effectively refuse harmful queries. Specifically, these defense methods aim to limit the transferability of LLMs from harmless queries to harmful ones, which techniques are variants of the objectives of NTL. 
Actually, all existing NTL approaches can be applied to this task by regarding the alignment data as the source domain and the harmful data as the target domain. Then, target-specified NTL can be conducted to defend agaginst harmful fine-tuning attacks.

\section{Related Work}
The landscape of large language model vulnerabilities has been extensively studied in recent literature \cite{crothers2023machinegeneratedtextcomprehensive,shayegani2023surveyvulnerabilitieslargelanguage,Yao_2024,Huang2023ASO}, that propose detailed taxonomies of threats. These works categorize LLM attacks into distinct types, such as adversarial attacks, data poisoning, and specific vulnerabilities related to prompt engineering. Among these, prompt injection attacks have emerged as a significant and distinct category, underscoring their relevance to LLM security.

The following high-level overview of the collected taxonomy of LLM vulnerabilities is defined in \cite{Yao_2024}:
\begin{itemize}
    \item Adversarial Attacks: Data Poisoning, Backdoor Attacks
    \item Inference Attacks: Attribute Inference, Membership Inferences
    \item Extraction Attacks
    \item Bias and Unfairness
Exploitation
    \item Instruction Tuning Attacks: Jailbreaking, Prompt Injection.
\end{itemize}
Prompt injection attacks are further classified in \cite{shayegani2023surveyvulnerabilitieslargelanguage} into the following: Goal hijacking and \textbf{Prompt leakage}.

The reviewed taxonomies underscore the need for comprehensive frameworks to evaluate LLM security. The agentic approach introduced in this paper builds on these insights, automating adversarial testing to address a wide range of scenarios, including those involving prompt leakage and role-specific vulnerabilities.

\subsection{Prompt Injection and Prompt Leakage}

Prompt injection attacks exploit the blending of instructional and data inputs, manipulating LLMs into deviating from their intended behavior. Prompt injection attacks encompass techniques that override initial instructions, expose private prompts, or generate malicious outputs \cite{Huang2023ASO}. A subset of these attacks, known as prompt leakage, aims specifically at extracting sensitive system prompts embedded within LLM configurations. In \cite{shayegani2023surveyvulnerabilitieslargelanguage}, authors differentiate between prompt leakage and related methods such as goal hijacking, further refining the taxonomy of LLM-specific vulnerabilities.

\subsection{Defense Mechanisms}

Various defense mechanisms have been proposed to address LLM vulnerabilities, particularly prompt injection and leakage \cite{shayegani2023surveyvulnerabilitieslargelanguage,Yao_2024}. We focused on cost-effective methods like instruction postprocessing and prompt engineering, which are viable for proprietary models that cannot be retrained. Instruction preprocessing sanitizes inputs, while postprocessing removes harmful outputs, forming a dual-layer defense. Preprocessing methods include perplexity-based filtering \cite{Jain2023BaselineDF,Xu2022ExploringTU} and token-level analysis \cite{Kumar2023CertifyingLS}. Postprocessing employs another set of techniques, such as censorship by LLMs \cite{Helbling2023LLMSD,Inan2023LlamaGL}, and use of canary tokens and pattern matching \cite{vigil-llm,rebuff}, although their fundamental limitations are noted \cite{Glukhov2023LLMCA}. Prompt engineering employs carefully designed instructions \cite{Schulhoff2024ThePR} and advanced techniques like spotlighting \cite{Hines2024DefendingAI} to mitigate vulnerabilities, though no method is foolproof \cite{schulhoff-etal-2023-ignore}. Adversarial training, by incorporating adversarial examples into the training process, strengthens models against attacks \cite{Bespalov2024TowardsBA,Shaham2015UnderstandingAT}.

\subsection{Security Testing for Prompt Injection Attacks}

Manual testing, such as red teaming \cite{ganguli2022redteaminglanguagemodels} and handcrafted "Ignore Previous Prompt" attacks \cite{Perez2022IgnorePP}, highlights vulnerabilities but is limited in scale. Automated approaches like PAIR \cite{chao2024jailbreakingblackboxlarge} and GPTFUZZER \cite{Yu2023GPTFUZZERRT} achieve higher success rates by refining prompts iteratively or via automated fuzzing. Red teaming with LLMs \cite{Perez2022RedTL} and reinforcement learning \cite{anonymous2024diverse} uncovers diverse vulnerabilities, including data leakage and offensive outputs. Indirect Prompt Injection (IPI) manipulates external data to compromise applications \cite{Greshake2023NotWY}, adapting techniques like SQL injection to LLMs \cite{Liu2023PromptIA}. Prompt secrecy remains fragile, with studies showing reliable prompt extraction \cite{Zhang2023EffectivePE}. Advanced frameworks like Token Space Projection \cite{Maus2023AdversarialPF} and Weak-to-Strong Jailbreaking Attacks \cite{zhao2024weaktostrongjailbreakinglargelanguage} exploit token-space relationships, achieving high success rates for prompt extraction and jailbreaking.

\subsection{Agentic Frameworks for Evaluating LLM Security}

The development of multi-agent systems leveraging large language models (LLMs) has shown promising results in enhancing task-solving capabilities \cite{Hong2023MetaGPTMP, Wang2023UnleashingTE, Talebirad2023MultiAgentCH, Wu2023AutoGenEN, Du2023ImprovingFA}. A key aspect across various frameworks is the specialization of roles among agents \cite{Hong2023MetaGPTMP, Wu2023AutoGenEN}, which mimics human collaboration and improves task decomposition.

Agentic frameworks and the multi-agent debate approach benefit from agent interaction, where agents engage in conversations or debates to refine outputs and correct errors \cite{Wu2023AutoGenEN}. For example, debate systems improve factual accuracy and reasoning by iteratively refining responses through collaborative reasoning \cite{Du2023ImprovingFA}, while AG2 allows agents to autonomously interact and execute tasks with minimal human input.

These frameworks highlight the viability of agentic systems, showing how specialized roles and collaborative mechanisms lead to improved performance, whether in factuality, reasoning, or task execution. By leveraging the strengths of diverse agents, these systems demonstrate a scalable approach to problem-solving.

Recent research on testing LLMs using other LLMs has shown that this approach can be highly effective \cite{chao2024jailbreakingblackboxlarge, Yu2023GPTFUZZERRT, Perez2022RedTL}. Although the papers do not explicitly employ agentic frameworks they inherently reflect a pattern similar to that of an "attacker" and a "judge". \cite{chao2024jailbreakingblackboxlarge}  This pattern became a focal point for our work, where we put the judge into a more direct dialogue, enabling it to generate attacks based on the tested agent response in an active conversation.

A particularly influential paper in shaping our approach is Jailbreaking Black Box Large Language Models in Twenty Queries \cite{chao2024jailbreakingblackboxlarge}. This paper not only introduced the attacker/judge architecture but also provided the initial system prompts used for a judge.
\section{Discussion of Assumptions}\label{sec:discussion}
In this paper, we have made several assumptions for the sake of clarity and simplicity. In this section, we discuss the rationale behind these assumptions, the extent to which these assumptions hold in practice, and the consequences for our protocol when these assumptions hold.

\subsection{Assumptions on the Demand}

There are two simplifying assumptions we make about the demand. First, we assume the demand at any time is relatively small compared to the channel capacities. Second, we take the demand to be constant over time. We elaborate upon both these points below.

\paragraph{Small demands} The assumption that demands are small relative to channel capacities is made precise in \eqref{eq:large_capacity_assumption}. This assumption simplifies two major aspects of our protocol. First, it largely removes congestion from consideration. In \eqref{eq:primal_problem}, there is no constraint ensuring that total flow in both directions stays below capacity--this is always met. Consequently, there is no Lagrange multiplier for congestion and no congestion pricing; only imbalance penalties apply. In contrast, protocols in \cite{sivaraman2020high, varma2021throughput, wang2024fence} include congestion fees due to explicit congestion constraints. Second, the bound \eqref{eq:large_capacity_assumption} ensures that as long as channels remain balanced, the network can always meet demand, no matter how the demand is routed. Since channels can rebalance when necessary, they never drop transactions. This allows prices and flows to adjust as per the equations in \eqref{eq:algorithm}, which makes it easier to prove the protocol's convergence guarantees. This also preserves the key property that a channel's price remains proportional to net money flow through it.

In practice, payment channel networks are used most often for micro-payments, for which on-chain transactions are prohibitively expensive; large transactions typically take place directly on the blockchain. For example, according to \cite{river2023lightning}, the average channel capacity is roughly $0.1$ BTC ($5,000$ BTC distributed over $50,000$ channels), while the average transaction amount is less than $0.0004$ BTC ($44.7k$ satoshis). Thus, the small demand assumption is not too unrealistic. Additionally, the occasional large transaction can be treated as a sequence of smaller transactions by breaking it into packets and executing each packet serially (as done by \cite{sivaraman2020high}).
Lastly, a good path discovery process that favors large capacity channels over small capacity ones can help ensure that the bound in \eqref{eq:large_capacity_assumption} holds.

\paragraph{Constant demands} 
In this work, we assume that any transacting pair of nodes have a steady transaction demand between them (see Section \ref{sec:transaction_requests}). Making this assumption is necessary to obtain the kind of guarantees that we have presented in this paper. Unless the demand is steady, it is unreasonable to expect that the flows converge to a steady value. Weaker assumptions on the demand lead to weaker guarantees. For example, with the more general setting of stochastic, but i.i.d. demand between any two nodes, \cite{varma2021throughput} shows that the channel queue lengths are bounded in expectation. If the demand can be arbitrary, then it is very hard to get any meaningful performance guarantees; \cite{wang2024fence} shows that even for a single bidirectional channel, the competitive ratio is infinite. Indeed, because a PCN is a decentralized system and decisions must be made based on local information alone, it is difficult for the network to find the optimal detailed balance flow at every time step with a time-varying demand.  With a steady demand, the network can discover the optimal flows in a reasonably short time, as our work shows.

We view the constant demand assumption as an approximation for a more general demand process that could be piece-wise constant, stochastic, or both (see simulations in Figure \ref{fig:five_nodes_variable_demand}).
We believe it should be possible to merge ideas from our work and \cite{varma2021throughput} to provide guarantees in a setting with random demands with arbitrary means. We leave this for future work. In addition, our work suggests that a reasonable method of handling stochastic demands is to queue the transaction requests \textit{at the source node} itself. This queuing action should be viewed in conjunction with flow-control. Indeed, a temporarily high unidirectional demand would raise prices for the sender, incentivizing the sender to stop sending the transactions. If the sender queues the transactions, they can send them later when prices drop. This form of queuing does not require any overhaul of the basic PCN infrastructure and is therefore simpler to implement than per-channel queues as suggested by \cite{sivaraman2020high} and \cite{varma2021throughput}.

\subsection{The Incentive of Channels}
The actions of the channels as prescribed by the DEBT control protocol can be summarized as follows. Channels adjust their prices in proportion to the net flow through them. They rebalance themselves whenever necessary and execute any transaction request that has been made of them. We discuss both these aspects below.

\paragraph{On Prices}
In this work, the exclusive role of channel prices is to ensure that the flows through each channel remains balanced. In practice, it would be important to include other components in a channel's price/fee as well: a congestion price  and an incentive price. The congestion price, as suggested by \cite{varma2021throughput}, would depend on the total flow of transactions through the channel, and would incentivize nodes to balance the load over different paths. The incentive price, which is commonly used in practice \cite{river2023lightning}, is necessary to provide channels with an incentive to serve as an intermediary for different channels. In practice, we expect both these components to be smaller than the imbalance price. Consequently, we expect the behavior of our protocol to be similar to our theoretical results even with these additional prices.

A key aspect of our protocol is that channel fees are allowed to be negative. Although the original Lightning network whitepaper \cite{poon2016bitcoin} suggests that negative channel prices may be a good solution to promote rebalancing, the idea of negative prices in not very popular in the literature. To our knowledge, the only prior work with this feature is \cite{varma2021throughput}. Indeed, in papers such as \cite{van2021merchant} and \cite{wang2024fence}, the price function is explicitly modified such that the channel price is never negative. The results of our paper show the benefits of negative prices. For one, in steady state, equal flows in both directions ensure that a channel doesn't loose any money (the other price components mentioned above ensure that the channel will only gain money). More importantly, negative prices are important to ensure that the protocol selectively stifles acyclic flows while allowing circulations to flow. Indeed, in the example of Section \ref{sec:flow_control_example}, the flows between nodes $A$ and $C$ are left on only because the large positive price over one channel is canceled by the corresponding negative price over the other channel, leading to a net zero price.

Lastly, observe that in the DEBT control protocol, the price charged by a channel does not depend on its capacity. This is a natural consequence of the price being the Lagrange multiplier for the net-zero flow constraint, which also does not depend on the channel capacity. In contrast, in many other works, the imbalance price is normalized by the channel capacity \cite{ren2018optimal, lin2020funds, wang2024fence}; this is shown to work well in practice. The rationale for such a price structure is explained well in \cite{wang2024fence}, where this fee is derived with the aim of always maintaining some balance (liquidity) at each end of every channel. This is a reasonable aim if a channel is to never rebalance itself; the experiments of the aforementioned papers are conducted in such a regime. In this work, however, we allow the channels to rebalance themselves a few times in order to settle on a detailed balance flow. This is because our focus is on the long-term steady state performance of the protocol. This difference in perspective also shows up in how the price depends on the channel imbalance. \cite{lin2020funds} and \cite{wang2024fence} advocate for strictly convex prices whereas this work and \cite{varma2021throughput} propose linear prices.

\paragraph{On Rebalancing} 
Recall that the DEBT control protocol ensures that the flows in the network converge to a detailed balance flow, which can be sustained perpetually without any rebalancing. However, during the transient phase (before convergence), channels may have to perform on-chain rebalancing a few times. Since rebalancing is an expensive operation, it is worthwhile discussing methods by which channels can reduce the extent of rebalancing. One option for the channels to reduce the extent of rebalancing is to increase their capacity; however, this comes at the cost of locking in more capital. Each channel can decide for itself the optimum amount of capital to lock in. Another option, which we discuss in Section \ref{sec:five_node}, is for channels to increase the rate $\gamma$ at which they adjust prices. 

Ultimately, whether or not it is beneficial for a channel to rebalance depends on the time-horizon under consideration. Our protocol is based on the assumption that the demand remains steady for a long period of time. If this is indeed the case, it would be worthwhile for a channel to rebalance itself as it can make up this cost through the incentive fees gained from the flow of transactions through it in steady state. If a channel chooses not to rebalance itself, however, there is a risk of being trapped in a deadlock, which is suboptimal for not only the nodes but also the channel.

\section{Conclusion}
This work presents DEBT control: a protocol for payment channel networks that uses source routing and flow control based on channel prices. The protocol is derived by posing a network utility maximization problem and analyzing its dual minimization. It is shown that under steady demands, the protocol guides the network to an optimal, sustainable point. Simulations show its robustness to demand variations. The work demonstrates that simple protocols with strong theoretical guarantees are possible for PCNs and we hope it inspires further theoretical research in this direction.






%% The file named.bst is a bibliography style file for BibTeX 0.99c
\bibliographystyle{named}
\bibliography{ijcai25}

\end{document}

