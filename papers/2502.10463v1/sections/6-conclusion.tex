\section{Conclusion}

% In conclusion, this paper has explored the interaction among all layers in neural networks by utilizing state space models (SSMs) to retrospectively retrieve information. By treating outputs from different layers as sequential data inputs to an SSM, we have demonstrated that our approach allows for a richer representation of the information derived from the original data. The proposed Selective State Space Layer Aggregation (S6LA) module effectively combines outputs from CNNs and transformers, enhancing the model's ability to leverage information from the input. Our empirical results indicate that the S6LA module is beneficial for classification and detection tasks, showcasing the utility of statistical theory in addressing long sequence modeling challenges. Moving forward, we aim to optimize our approach by reducing the number of parameters and FLOPs while further improving accuracy. We also recognize the potential for integrating additional statistical models into computer science applications, suggesting a promising theoretical convergence between these fields. Additionally, exploring the application of our model within larger architectures and in natural language processing tasks presents an exciting avenue for future research.

In conclusion, we have demonstrated an enhanced representation of information derived from the original data by treating outputs from various layers as sequential data inputs to a state space model (SSM). The proposed Selective State Space Layer Aggregation (S6LA) module uniquely combines layer outputs with a continuous perspective, allowing for a more profound understanding of deep models while employing a selective mechanism. Empirical results indicate that the S6LA module significantly benefits classification and detection tasks, showcasing the utility of statistical theory in addressing long sequence modeling challenges. Looking ahead, we aim to optimize our approach by reducing parameters and FLOPs while enhancing accuracy. Additionally, we see potential for integrating further statistical models into computer science applications, suggesting a promising convergence in these fields.

