
\documentclass{article} % For LaTeX2e
\usepackage{iclr2025_conference,times}

% Optional math commands from https://github.com/goodfeli/dlbook_notation.
%%%%% NEW MATH DEFINITIONS %%%%%

% \usepackage{amsmath,amsfonts,bm}
\usepackage{amsmath,amsfonts}

\usepackage{pifont}


\newcommand{\R}{\mathbb{R}}


\def\va{{\mathbf{a}}}
\def\vg{{\mathbf{g}}}

% Sets
\def\sR{\mathbb{R}}
\def\sC{\mathbb{C}}
\def\sZ{\mathbb{Z}}
\def\sN{\mathbb{N}}
\def\sQ{\mathbb{Q}}

\def\sS{\mathcal{S}}



% Vectors
\def\vzero{{\mathbf{0}}}
\def\vone{{\mathbf{1}}}
\def\vmu{{\mathbf{\mu}}}
\def\vtheta{{\mathbf{\theta}}}
\def\va{{\mathbf{a}}}
\def\vb{{\mathbf{b}}}
\def\vc{{\mathbf{c}}}
\def\vd{{\mathbf{d}}}
\def\ve{{\mathbf{e}}}
\def\vf{{\mathbf{f}}}
\def\vg{{\mathbf{g}}}
\def\vh{{\mathbf{h}}}
\def\vi{{\mathbf{i}}}
\def\vj{{\mathbf{j}}}
\def\vk{{\mathbf{k}}}
\def\vl{{\mathbf{l}}}
\def\vm{{\mathbf{m}}}
\def\vn{{\mathbf{n}}}
\def\vo{{\mathbf{o}}}
\def\vp{{\mathbf{p}}}
\def\vq{{\mathbf{q}}}
\def\vr{{\mathbf{r}}}
\def\vs{{\mathbf{s}}}
\def\vt{{\mathbf{t}}}
\def\vu{{\mathbf{u}}}
\def\vv{{\mathbf{v}}}
\def\vw{{\mathbf{w}}}
\def\vx{{\mathbf{x}}}
\def\vy{{\mathbf{y}}}
\def\vz{{\mathbf{z}}}
\def\vzeta{{\mathbf{\zeta}}}

% Matrix
\def\mA{{\mathbf{A}}}
\def\mB{{\mathbf{B}}}
\def\mC{{\mathbf{C}}}
\def\mD{{\mathbf{D}}}
\def\mE{{\mathbf{E}}}
\def\mF{{\mathbf{F}}}
\def\mG{{\mathbf{G}}}
\def\mH{{\mathbf{H}}}
\def\mI{{\mathbf{I}}}
\def\mJ{{\mathbf{J}}}
\def\mK{{\mathbf{K}}}
\def\mL{{\mathbf{L}}}
\def\mM{{\mathbf{M}}}
\def\mN{{\mathbf{N}}}
\def\mO{{\mathbf{O}}}
\def\mP{{\mathbf{P}}}
\def\mQ{{\mathbf{Q}}}
\def\mR{{\mathbf{R}}}
\def\mS{{\mathbf{S}}}
\def\mT{{\mathbf{T}}}
\def\mU{{\mathbf{U}}}
\def\mV{{\mathbf{V}}}
\def\mW{{\mathbf{W}}}
\def\mX{{\mathbf{X}}}
\def\mY{{\mathbf{Y}}}
\def\mZ{{\mathbf{Z}}}
\def\mBeta{{\mathbf{\beta}}}
\def\mPhi{{\mathbf{\Phi}}}
\def\mLambda{{\mathbf{\Lambda}}}
\def\mSigma{{\mathbf{\Sigma}}}


% Expectation
% \def\eE{\mathop{\mathbb{E}}\limits}
\def\eE{\mathbb{E}}

% Probability
\def\pP{\mathbb{P}}

% Tilde
\def\tf{\tilde{f}}
\def\tS{\tilde{S}}
\def\wtF{\widetilde{\mathcal{F}}}
\def\whR{\widehat{R}}
\def\tvx{\tilde{\mathbf{x}}}
\def\ty{\tilde{y}}


\def\defeq{\overset{\textup{def}}{=}}
% \def\defeq{\overset{.}{=}}
\def\defone{\overset{\text{\ding{172}}}{=}}
\def\deftwo{\overset{\text{\ding{173}}}{=}}
\def\leqone{\overset{\text{\ding{172}}}{\leq}}
\def\leqtwo{\overset{\text{\ding{173}}}{\leq}}
\def\leqthree{\overset{\text{\ding{174}}}{\leq}}
\def\leqfour{\overset{\text{\ding{175}}}{\leq}}
\def\eqone{\overset{\text{\ding{172}}}{=}}
\def\eqtwo{\overset{\text{\ding{173}}}{=}}
\def\eqthree{\overset{\text{\ding{174}}}{=}}
\def\eqfour{\overset{\text{\ding{175}}}{=}}
\def\geqfive{\overset{\text{\ding{176}}}{\geq}}

\usepackage{hyperref}
\usepackage{url}
\usepackage{booktabs}
\usepackage{multirow}
\usepackage{graphicx}
\usepackage{makecell}
\usepackage{mathtools}
\usepackage{algorithmic}
\usepackage{algorithm}

\usepackage{colortbl}
\usepackage{xcolor}

% 定义颜色
\definecolor{c1}{rgb}{0.9, 0.9, 0.9}
\definecolor{citecolor}{HTML}{0071BC}
\definecolor{linkcolor}{HTML}{ED1C24}

\hypersetup{
colorlinks=true,
linkcolor=linkcolor,
citecolor=citecolor,
}

\newcommand{\ylq}[1]{{\color{blue}{[ylq: #1]}}}
\newcommand{\para}[1]{\vspace{.05in}\noindent\textbf{#1}}


% \title{Are state space models effective in interaction layers}
% \title{From Layers to States: Deciphering Deep Neural Network Layer Dynamics from a State Space Model Perspective}

\title{From Layers to States: A State Space Model Perspective to Deep Neural Network Layer Dynamics}
% \title{Rethinking Deep Neural Network Layers Through the Lens of State Space Models}
% \title{A State Space Model Perspective on Layer Interactions in Deep Neural Networks}

% Authors must not appear in the submitted version. They should be hidden
% as long as the \iclrfinalcopy macro remains commented out below.
% Non-anonymous submissions will be rejected without review.

% \author{
% % Qinshuo Liu \thanks{Equal Contribution} , Weiqin Zhao, Wei Huang, Yanwen Fang, Lequan Yu, Guodong Li \thanks{Corresponding Author} \\
% % Department of Statistics \& Actuarial Science, The University of Hong Kong\\
% % Pittsburgh, PA 15213, USA \\
% % \texttt{\{hippo,brain,jen\}@cs.cranberry-lemon.edu} \\
% % \centering
% Antiquus S.~Hippocampus, Natalia Cerebro \& Amelie P. Amygdale \thanks{ Use footnote for providing further information
% about author (webpage, alternative address)---\emph{not} for acknowledging
% funding agencies.  Funding acknowledgements go at the end of the paper.} \\
% Department of Computer Science\\
% Cranberry-Lemon University\\
% Pittsburgh, PA 15213, USA \\
% \texttt{\{hippo,brain,jen\}@cs.cranberry-lemon.edu} \\
% \And
% Ji Q. Ren \& Yevgeny LeNet \\
% Department of Computational Neuroscience \\
% University of the Witwatersrand \\
% Joburg, South Africa \\
% \texttt{\{robot,net\}@wits.ac.za} \\
% \AND
% Coauthor \\
% Affiliation \\
% Address \\
% \texttt{email}
% }

\newcommand*{\affaddr}[1]{#1} % No op here. Customize it for different styles.
\newcommand*{\affmark}[1][*]{\textsuperscript{#1}}
\newcommand*{\email}[1]{\texttt{#1}}

\usepackage{authblk}
\author{%
\textbf{Qinshuo Liu}$^{\ast}$\affmark[1], \textbf{Weiqin Zhao}$^{\ast}$\affmark[1], \textbf{Wei Huang}\affmark[2], \textbf{Yanwen Fang}\affmark[1], \textbf{Lequan Yu}$^{\dagger}$\affmark[1], \textbf{Guodong Li}$^{\dagger}$\affmark[1] 
\\
\affaddr{\affmark[1]School of Computing and Data Science, The University of Hong Kong} \\
\affaddr{\affmark[2]Department of Electrical and Electronic Engineering, The University of Hong Kong} \\
 \email{\{u3008680, wqzhao98, u3545683\}@connect.hku.hk} \\
 \email{aaron.weihuang@gmail.com} \\
 \email{\{lqyu, gdli\}@hku.hk}\\
% \affaddr{\LaTeX\ University}%
}

% The \author macro works with any number of authors. There are two commands
% used to separate the names and addresses of multiple authors: \And and \AND.
%
% Using \And between authors leaves it to \LaTeX{} to determine where to break
% the lines. Using \AND forces a linebreak at that point. So, if \LaTeX{}
% puts 3 of 4 authors names on the first line, and the last on the second
% line, try using \AND instead of \And before the third author name.

\newcommand{\fix}{\marginpar{FIX}}
\newcommand{\new}{\marginpar{NEW}}

\newcommand\nnfootnote[1]{%
  \begin{NoHyper}
  \renewcommand\thefootnote{}\footnote{#1}%
  \addtocounter{footnote}{-1}%
  \end{NoHyper}
}


%\iclrfinalcopy % Uncomment for camera-ready version, but NOT for submission.
\iclrfinalcopy
\begin{document}


\maketitle

\nnfootnote{$\ast$ Authors contributed equally. $\dagger$ Corresponding author.}
\vspace{-0.8cm}
\input{sections/1-abstract_new}
\section{Introduction}
\label{Introduction}

% 深度模型很好
In recent years, the depth of neural network architectures has emerged as a crucial factor influencing performance across various domains, including computer vision, natural language processing, and speech recognition.
The network models are capable of capturing increasingly complex features and representations from data as they become deeper, and various methods have emerged to utilize larger numbers of layers to improve performance. 
For instance, the VGG network \citep{simonyan2015deepconvolutionalnetworkslargescale} achieves higher classification accuracy by increasing the number of layers, although its foundation primarily relies on empirical results rather than systematical analysis. Other significant advancements, such as those demonstrated by CNNs \citep{he2016deep, ren2016faster, tan2020efficientnetrethinkingmodelscaling} and Transformers \citep{brown2020language, dosovitskiy2020image, touvron2021training, liu2021swin, wang2022pvt}, showcase how deeper architectures can enhance accuracy and generalization. 

% 怎么enhance深度模型performance: layer aggregation
Growing evidence indicates that strengthening layer interactions can encourage the information flow of a deep neural network.
%
For CNN-based networks, ResNet \citep{he2016deep} employed skip connections, allowing gradients to flow more easily by connecting non-adjacent layers. 
DenseNet \citep{huang2018denselyconnectedconvolutionalnetworks} extended this concept further by enabling each layer to access all preceding layers within a stage, fostering a rich exchange of information. 
Later, GLOM \citep{hinton2023represent} proposed an intensely interactive architecture that incorporates bottom-up, top-down, and same-level connections to effectively represent part-whole hierarchies. 
%
Recently, some studies have begun to frame layer interactions with recurrent models and attention mechanisms, such as RLA \citep{zhao2021recurrence} and MRLA \citep{fang2023cross}. 
%
All of the above models have been shown by empirical evidence to outperform those without interdependence across layers, and their achievements are obtained by treating the outputs of network layers as discrete states. 

However, the perspective of discrete treatment may not be suitable when a neural network is very deep; say, for example, ResNet-152 has 152 layers.
\cite{sander2022residualneuralnetworksdiscretize} proposed to treat the ResNet as a discretized version of neural ordinary differential equations, i.e. the whole ResNet is considered as a continuous process with the outputs from layers being the corresponding states; see also \cite{liu2020does}.
Moreover, \citet{queiruga2020continuous} argued that deep neural network models can learn to represent continuous dynamical systems by embedding them into continuous perspective. 
%
This motivates us to conduct layer aggregation among numerous layers of a neural network by alternatively assuming a continuous process to the outputs of layers.
%
% However, this may not be suitable mathematically when a neural network is very deep; say, for example, ResNet has 152 layers \citep{he2016deep}.\ylq{change this claim...}


Meanwhile, the State Space Model (SSM), a mathematical framework for continuously updating physical systems, enabled the modeling of dynamic processes and temporal dependencies in deep learning \citep{gu2023mamba,liu2024vmamba}. Then, Mamba, a selective state space model \citep{gu2023mamba}, proposed selective mechanism and hardware-aware algorithm, which was particularly adept at addressing long sequence modeling challenges. The selection mechanism allows the model to filter out irrelevant information and remember relevant information infinitely.

% For the development of state space model, \citet{gu2021efficiently} proposed structured state space sequence models (S4) which belong to a type of sequence models for deep learning related to RNNs. They are inspired by a particular continuous system and give the definition of latent state $h$ and structured state matrices $A$ \citep{gu2021efficiently}, with interval $\Delta$ and matrices $B$ being the coefficient for the model input signal $x$. \ylq{Can delete this sentence: Then Mamba model \citep{gu2023mamba} proposed selective mechanism and hardware-aware algorithm, which is particularly adept at addressing long sequence modeling challenges and hence is more suitable for our scenario. The details of these are provided in Section \ref{SSM_theory}.}\ylq{shorten}
%


The significance of layer aggregation in deep models and the popularity of SSMs lead us to propose a novel perspective: layer dynamics in very deep networks can be viewed as a continuous process with long sequence modeling task solvable by selective state space model (S6). By leveraging interactions between layers, outputs from different layers can be treated as sequential data input for an S6, allowing the model to encapsulate a richer representation of the information derived from the original data. 
%
By conceptualizing neural networks as state space models, we introduce a novel structure that integrates traditional models into sequential architectures. This approach opens new research avenues that connect traditional statistical methods with contemporary deep learning techniques. Our proposed Selective State Space Model Layer Aggregation (S6LA) effectively harnesses the benefits of layer interactions while incorporating statistical modeling into vision tasks, such as those performed by CNNs and Vision Transformers (ViTs). A schematic of our model is illustrated in Figure \ref{fig:overview}, with parameters $(\Delta,A,B)$ indicating the influence of {\color{blue}$\boldsymbol{X}$} on the implicit latent state $h$. Here {\color{blue}$\boldsymbol{X}^{t-1}$} represents the output at the $(t-1)$-th layer, which can be either a hidden layer in a deep CNN or an attention layer in a transformer model.

\begin{figure}[t]
\begin{center}
%\framebox[4.0in]{$\;$}
\includegraphics[width=0.9\linewidth]{overview.pdf}
\end{center}
\caption{Schematic diagram of a Network with Selective State Space Model Layer Aggregation.}
\label{fig:overview}
\end{figure}

The main contributions of our work are given below: (1) For a deep neural network, we treat the outputs from layers as states of a continuous process and attempt to leverage the SSM to design the aggregation of layers. To our best knowledge, this is the first time such a perspective has been presented. (2) This leads to a proposed lightweight module, the Selective State Space Model Layer Aggregation (S6LA) module, and it conceptualizes a neural network as a selective state space model (S6), hence solving the layer interactions by the long sequence modelling selective mechanism. (3) Compared with other SOTA convolutional and transformer-based layer aggregation models, S6LA demonstrates superior performance in classification, detection, and instance segmentation tasks.

% cv在最后一个contribution说就可以,只是一个验证的方式
% \begin{enumerate}
%     \item We propose that very deep neural networks can be formulated as state space models. To our knowledge, this is the first time such a perspective has been presented. Our approach redefines how to utilize state space models to integrate different layers, advancing traditional multi-layer models.
%     \item We introduce the State-Space-Interaction-Layer (SSIL), which conceptualizes neural networks as state space models (SSMs) to enhance cross-layer interactions. This framework integrates traditional CNNs, like ResNet, and transformer-based models, such as Deit and Swin Transformer, into sequential architectures, representing the first systematic investigation of such interactions within SSMs.
%     \item In comparison with convolutional and transformer-based vision models, our method demonstrates superior performance in classification, detection, and instance segmentation tasks.
% \end{enumerate}

% \noindent\textbf{Multi-view 3D Reconstruction.}
% \todo{xiaoyang}

% We do not need this
% \noindent\textbf{Manipulation policy.}
% Recently, a series of algorithms have achieved impressive results in robotics manipulation tasks. Algorithms such as ACT~\citep{zhao2023learning}, Diffusion Policy~\citep{chi2023diffusion}, RVT~\citep{goyal2023rvt}, and 3D Diffusion Policy~\citep{ze20243d} have demonstrated strong performance on in-domain tasks even when trained on small-scale real-world datasets. Simultaneously, algorithms~\citep{li2023vision, brohan2022rt,o2023open, kim2024openvla,wu2023unleashing,cheang2024gr2generativevideolanguageactionmodel, tian2024seer} leverage pre-trained datasets to train models that exhibit generalization across multiple tasks and diverse scenarios. Meanwhile, many algorithms utilizing pre-trained vision language models to finish robotics tasks, such as R3M~\citep{nairr3m}, MVP~\citep{xiao2022mvp,radosavovic2023mvp}, VIP~\citep{ma2023vip}, and VC-1~\citep{majumdar2023vc1} pre-trained the vision models on ego-centric dataset~\citep{grauman2022ego4d}. 

% object-centric human motion dataset
% robot dataset

% \todo{Real-to-Real?}

\section{Related Work}
\noindent\textbf{Sim-to-real.}
Sim-to-real transfer requires techniques to enable policies to successfully adapt from simulation to the real world. The most direct approach is improving simulators \citep{todorov2012mujoco, makoviychuk2021isaac, mittal2023orbit, Xiang_2020_SAPIEN}, which reduces the sim-to-real gap. Other methods, such as domain randomization~\citep{tobin2017domain, mehta2020active, chen2021understanding, tremblay2018training, loquercio2019deep, tobin2018domain} and system identification~\citep{ramaswamy2024adaptation, allevato2020iterative, song2024systemid, pmlr-v100-allevato20a}, also aim to bridge this gap.

\noindent\textbf{Real-to-sim-to-real.}
Many recent works leverage real-world data to enhance simulation models. Reconstruction methods integrated with grasping techniques, such as Evo-NeRF~\citep{kerr2023evo} and LERF-TOGO~\citep{lerftogo2023}, enable the grasping of objects using only RGB images. 
% Prior approaches, such as GraspNerf~\citep{dai2023graspnerf}, introduced a generalizable NeRF that achieves real-time grasping. Other works, including Evo-NeRF~\citep{kerr2023evo} and LERF-TOGO~\citep{lerftogo2023}, utilize depth images rendered by NeRF to generate grasping poses.
% Specifically, LERF-TOGO leverages depth information from multiple views to create a dense point cloud, which is subsequently processed by GraspNet~\citep{fang2020graspnet}.
% Recently, 3D Gaussian splatting~\citep{kerbl3Dgaussians} has gained significant attention in robotics due to its fast rendering speed and explicit representation. GaussianGrasper~\citep{zheng2024gaussiangrasper} uses 3DGS for scene reconstruction and normal-guided grasp generation. Similarly, SplatMover~\citep{shorinwa2024splat} introduces a grasp-splat module integrating affordance and semantics within 3DGS. GraspSplats~\citep{ji2024graspsplats} replaces object representation in earlier grasping networks with 3DGS.
Meanwhile, GaussianGrasper~\citep{zheng2024gaussiangrasper}, SplatMover~\citep{shorinwa2024splat}, and GraspSplats~\citep{ji2024graspsplats}, use 3D Gaussian splatting~\citep{kerbl3Dgaussians} for fast rendering and explicit representation in robotics tasks. Additionally, methods like URDFormer~\citep{urdformer}, Digital Cousins~\citep{acdcdai2024}, and Articulate Anything~\citep{articulateanythingle2024} use a single image to reconstruct the environment directly, allowing for collecting large amounts of data for imitation learning or reinforcement learning. These approaches enhance data by varying articulations and leveraging various articulations from simulation datasets to train models deployable in the real world.

% Another approach involves constructing a simulator with a smaller sim-to-real gap using real2sim methods. This allows for the collection of large amounts of data in the simulator for imitation learning or the use of reinforcement learning to interact with the environment, training an end-to-end model rather than using a scripted policy as the actuator. Specifically, URDFormer~\citep{urdformer}, Digital Cousins~\citep{acdcdai2024}, and Articulate Anything~\citep{articulateanythingle2024} use a single image to reconstruct the environment. 
% URDFormer utilizes URDF as a representation, employing generative models to create a photo-URDF dataset, thereby training models to generate URDFs from images. Digital Cousin and Articulate-Anything retrieve similar meshes from datasets to construct scenes in simulation.
% They enhance their data by using different articulations of similar sizes from simulation datasets, and by collecting a large amount of data similar to real environments in geometry and semantics, they train models that are able to be deployed in the real world. 

With the help of 3DGS works like RoboStudio~\citep{robostudio}, SplatSim~\citep{qureshi2024splatsim}
and RoboGSim~\citep{li2024robogsim} use multi-view images or video for world reconstruction, improving multi-view rendering quality with minimal cost. These methods are especially effective in manipulation tasks involving multiple objects or occlusions. 1) RoboStudio focuses on reconstructing the URDF of a robot, offering a photorealistic rendering result and accurate collision mesh.
2) SplatSim utilizes pre-obtained 3D models of objects and backgrounds to collect trajectories in the physics simulator and then re-render them with 3DGS to reduce the visual gap between simulated and real-world environments, but acquiring the 3D models is hard for many tasks. 
3) RoboGSim compares rendering quality, validates high-quality rendering results for novel pose synthesis, and shows the potential for evaluating various manipulation algorithms. However, its sim-to-real validation remains limited.
Different from them, \our reconstructs both geometric and visual aspects with small gaps, and validates robot policies trained on simulated data through extensive experiments in the real world.



% We also includes a simple and efficient alignment method, allowing the integration of objects from various sources. 


% GRS\citep{zook2024grs}
% \noindent{\textbf{3D Recontruction for Robotic Manipution}}
% simply introduce Gaussian splatting
% Gaussian splatting~\citep{kerbl3Dgaussians}, a powerful approach allowing real-time, high-quality radiance field rendering, has been recently applied in robotics manipulation~\citep{lou2024robo, qiu2024feature, qureshi2024splatsim, lu2024manigaussian, abou2024physically}.  



% \todo{Difference between 3D gaussian splitting for object grasping}

% TODO: Mani gaussian (haven't finished reading)
% For instance, ManiGaussian~\citep{lu2024manigaussian} estimates the propagation of diverse semantic features within the Gaussian embedding space to help improve multi-task robotic manipulation.
% For instance, R2SGrasp~\citep{r2sgrasp} employs a real2sim approach to integrate a repair module and an enhancer as part of the model, enabling it to handle noisy real-world RGBD inputs effectively.

% \begin{table}[t]
% \setlength{\tabcolsep}{0pt}
%     \centering
%     \caption{\textbf{Comparison with Existing Tabletop Manipulation Frameworks in Robotics.} ``S.G.'' represents the scene graph-based evaluation. \todo{looks like gembench table}}
%     \resizebox{\textwidth}{!}{%
%         \begin{tabular}{l|cccc}
%         \toprule

%         \textbf{Simulators}
%         & \begin{tabular}[c]{@{}c@{}}\rotatebox{30}{\textbf{\method}\end{tabular}
%         & \begin{tabular}[c]{@{}c@{}}\rotatebox{30}{\textbf{RialTo}~\citep{ritotorne2024rialto}}}\end{tabular}
%         &
%         &
        
%         \\ \midrule
%         %  \todo{} & Isaac Sim & Mujuco & RLBench & AI2-THOR & PyBullet & PyBullet & Isaac Sim & Ravens & RLBench & RLBench \\
%         High-fidelity rendering & - & - & - & - \\
%         Auto-align & \\
%         % \# Articulated Objects & - & - & - & - & - & - & - & - & - \\
%         no-real demo & - & - & - & - & - & - & - & - & - \\
%         object-level extensible & - & - & - & - & - & - & - & - & - \\
%         \# Common Sense & - & - & - & - & - & - & - & - & - \\
%         \# Long Horizon & - & - & - & - & - & - & - & - & - \\
%         \# AI-generated Scenarios & - & - & - & - & - & - & - & - & - \\
%         \# AI-generated Demo & - & - & - & - & - & - & - & - & - \\
%         Benchmark & \greencheck & \redcross & \redcross & \redcross & \redcross & &  \redcross & \redcross & \redcross & \redcross & \redcross \\
%         RayTracing & \greencheck &  & \redcross & \redcross & \redcross & &  \redcross & \redcross & \redcross & \redcross & \redcross \\
%         Evaluation Method & Scene Graph & - & - & - & - & - & - & - & - & \\
%         \bottomrule
%         \end{tabular}
%     }
%     \label{tab:benchmark_comparison_transposed}
% \end{table}

%

\section{Preliminary and Problem Formulation}
\label{SSM_theory}

% This section gives the main formula for the selective state space model given by \cite{gu2021efficiently} and \cite{gu2023mamba}. 

\subsection{Revisiting State Space Models}
The state space model is defined below, and it maps a $1$-dimensional input signal $x(t)$, a continuous process, to an $N$-dimensional latent state $h(t)$, another continuous process:
\begin{equation}
\label{eq:SSM_con}
h^{\prime}(t) =A h(t)+B x(t),
\end{equation}
where $A \in \mathbb{R}^{N \times N}$ and $B \in \mathbb{R}^{N \times 1}$ are the structured state matrix and weight of influence from input to latent state, respectively. We then can obtain the discretization solution of the above equation:
\begin{equation}
        h^{t} = e^{\Delta A}h^{t-1} + \int_{t-1}^{t} e^{A(t-\tau)} Bx(\tau) d\tau.
\end{equation}
Together with the zero-order hold (ZOH) condition \citep{karafyllis2011nonlinear}, i.e. $x(\tau)$ is a constant at intervals $[t-1,t]$ for all integers $t$, we have
\begin{equation}
    h^{t} = e^{\Delta A}h^{t-1} + (\Delta A)^{-1}(\text{exp}(\Delta A)-I) \cdot \Delta B x^{t}.
\end{equation}
As a result, the continuous process at \Eqref{eq:SSM_con} can be replaced by the following discrete sequence:
\begin{equation}
\label{eq:SSM_dis}
%\begin{aligned}
h^t  =\overline{A} h^{t-1}+\overline{B} x^t \hspace{5mm}\text{with}\hspace{5mm}
\overline{A} = \text{exp}(\Delta A) \quad \text{and}   \quad \overline{B} = (\Delta A)^{-1}(\text{exp}(\Delta A)-I) \cdot \Delta B.
%\end{aligned}
\end{equation}
Following \cite{gu2023mamba}, we refine the approximation of $\overline{B}$ using the first-order Taylor series:
\begin{equation}
    \overline{B} = (\Delta A)^{-1}(\text{exp}(\Delta A)-I) \cdot \Delta B \approx (\Delta A)^{-1}(\Delta A) \cdot \Delta B = \Delta B.
\end{equation}
The formulas $\overline{A}=f_A(\Delta, A)$ and $\overline{B}=f_B(\Delta, A, B)$ are called the discretization rule, where $B = \text{Linear}_N(x)$ is a linear projection of input $x$ into $N$-dimension vector, and $\Delta = \text{Linear}_1(x)$; see \citet{gu2021efficiently,gu2023mamba} for details.

% Then \cite{} proposes the settings about transition matrix $A$ and in this paper, they give an exact solution about how to choose this in state space model. 


%This section gives the overview of recurrent mechanism and recalls the mathematical formulation about layer aggregations.
\subsection{CNN Layer Aggregation}
\label{5.1}
Consider a neural network, and let $\boldsymbol{X}^{t-1}$ be the output from its $t$th layer. We then can mathematically formulate the layer aggregation at the $t$th layer below,
\begin{equation}
\label{eq:CNN_agg}
    A^t =g^t(\boldsymbol{X}^{0},\boldsymbol{X}^{1},\cdots,\boldsymbol{X}^{t-2},\boldsymbol{X}^{t-1}), \quad
    \boldsymbol{X}^t = f^t(A^{t-1},\boldsymbol{X}^{t-1}), 
\end{equation}
where $g^t$ is used to summarize the first $t$ layers, $A^t$ is the aggregated information, and $f^t$ produces the new layer output from the last hidden layer and the given aggregation which contains the previous information. 
The Hierarchical Layer Aggregation proposed \citep{yu2018deep} can be shown to have such similar mechanism which satisfies  \Eqref{eq:CNN_agg}.

This formulation could be generalized to the special case of CNNs.
The traditional CNNs do not contain layer aggregation since the layer output only depends on the last layer output, which overlooks the connection between the several previous layers' influence.
DenseNet \citep{huang2018denselyconnectedconvolutionalnetworks} perhaps is the first one for the layer aggregation, and its output at $t$th layer can be formulated into
\begin{equation}
\label{eq:densenet}
\boldsymbol{X}^t=\text{Conv3}^t[\text{Conv1}^t(\text{Concat}(\boldsymbol{X}^0, \boldsymbol{X}^1, \ldots, \boldsymbol{X}^{t-1}))].
\end{equation}
Let $A^t = \text{Conv1}^t(\text{Concat}(\boldsymbol{X}^0, \boldsymbol{X}^1, \ldots, \boldsymbol{X}^{t-1}))$ and $\boldsymbol{X}^t = \text{Conv3}^t (A^t)$, and then DenseNet can be rewritten into our framework at \Eqref{eq:CNN_agg}.
RLA \citep{zhao2021recurrence} considers a more convenient additive form for the layer aggregation, and it has the form of $A^t = \sum _{i=0}^{t-1} \text{Conv1}^{t+1}_i(\boldsymbol{X}^i)$, where the kernel weights of $\text{Conv1}^t_i$ form a partition of the weights in $\text{Conv1}^t$.
% Consider that we can treat all the $\text{Conv1}^t_i$ as $\text{Conv1}^t$ which means the aggregation function does not depend on the current layer. 
As a result, a lightweight aggregation can be formed:
\begin{equation}
\label{eq:densenet_re}
    \boldsymbol{X}^t=\text{Conv3}^t [ A^{t-1} + \text{Conv1}^{t}_{t-1}(\boldsymbol{X}^{t-1}) ].
\end{equation}

Without loss of generality, ResNets \citep{he2016deep,he2016identity} can also be treated as a layer aggregation. Specifically, we can treat the update of $\boldsymbol{X}^t = \boldsymbol{X}^{t-1} + f^{t-1}(\boldsymbol{X}^{t-1})$ with applying the update recursively as $A^t = \sum_{i=0}^{t-1} f^i(\boldsymbol{X}^i) + \boldsymbol{X}^0$ and $\boldsymbol{X}^t = A^{t-1}+\boldsymbol{X}^{t-1}$.


\subsection{Attention Layers Aggregation}
In this section, we show how to generalize the layer aggregation within a transformer. Consider a simple attention layer with general input $\mathbf{X} \in \mathbb{R}^{L \times D}$ and output $\mathbf{O} \in \mathbb{R}^{L \times D}$.
Its query $\mathbf{Q}$, key $\mathbf{K}$ and value $\mathbf{V}$ are given by linear projections $\boldsymbol{W}_q \in \mathbb{R}^{D \times D}$, $\boldsymbol{W}_k \in \mathbb{R}^{D \times D}$ and $\boldsymbol{W}_v \in \mathbb{R}^{D \times D}$, i.e. $\mathbf{Q}^T = \boldsymbol{W}_q \mathbf{X}$, $\mathbf{K}^T = \boldsymbol{W}_k \mathbf{X}$ and $\mathbf{V}^T = \boldsymbol{W}_v \mathbf{X}$. As a result, the output $\mathbf{O}$ has the following mathematical formulation:
\begin{equation}
\mathbf{O} = \text{Self-Attention}(\boldsymbol{X}) = \text{softmax}(\frac{\mathbf{Q}\mathbf{K}^T}{\sqrt{D}})\mathbf{V}.
\end{equation}
Let $\boldsymbol{X}^t \in \mathbb{R}^{L \times D}$ with $1 \leq t \leq T$ be the output from $t$th layer, where $L$ is the number of tokens, $D$ represents the channel of each token, and $T$ is the number of attention layers. 
A vanilla transformer can then be formulated into:
\begin{equation}
    \label{eq:vanillatransformer}
        A^{t} =\boldsymbol{X}^{t-1}+\text{Self-Attention}(\boldsymbol{X}^{t-1}), \quad
        \boldsymbol{X}^{t} = A^{t} + \text{MLP}(\text{Norm}(A^{t})).
\end{equation}

Note that these simple self-attention layers can only capture the connection between the current layer output and the last output; they are supposed to perform better if the information from previous layers can be considered. To this end, we may leverage the idea given by CNN aggregation to concatenate the previous outputs. Specifically, the vanilla transformer at \Eqref{eq:vanillatransformer} has the form of:
\begin{equation}
    \label{eq:vanillatransformerre}
    \boldsymbol{X}^t=f^t(g^t(\boldsymbol{X}^0,\cdots,\boldsymbol{X}^{t-1})),
\end{equation}
where $g^t$ is the attention layer, and $f^t$ is the Add \& Norm layer for the $t$-th layer. 
Following \citet{zhao2021recurrence} at \Eqref{eq:densenet_re}, we may then use the recurrent mechanism to combine all the outputs given by attention layers, i.e. replacing $A^{t} = g^t(\boldsymbol{X}^0,\cdots,\boldsymbol{X}^{t-1})$ by $A^{t} = A^{t-1} + g^{t-1}(\boldsymbol{X}^{t-1})$.
Layer Aggregation via Selective State Space Model
\subsection{The Formula of S6LA}

Denote a sequence $\mathbf{X} = \{\boldsymbol{X}^1, \cdots , \boldsymbol{X}^T\}$, where $\boldsymbol{X}^t$ is the output from $t$th layer, say Convolutional layers/blocks or Attention layers, of a deep neural network, and $T$ is the number of layers.
In financial statistics, the price of an asset can be treated as a process with discrete time when it is sampled in a low frequency, say weekly data, while it will be treated as a process with continuous time the sampling frequency is high, say one-minute data; see \cite{yuan2023haritomodelshighdimensionalhar}.
Accordingly, we may treat $\mathbf{X}$ as a sequence with discrete time as the number of layers $T$ is small or even moderate, and all existing methods for layer aggregation in the literature follow this way.
%, while we may look it as discretized observations of a continuous process as in \cite{a} and \cite{pavan2007high}. 
%time series models with discrete states are employed for low-frequency data, while the diffusion model with continuous processes is a standard tool for high-frequency data.
%In previous literature \citep{pavan2007high}}, all existing methods for layer aggregation treat  $\boldsymbol{X}^t$'s to be discrete states, and hence they all correspond to time series tools in statistics.
However, for a very deep neural network, it is more like the scenario of high-frequency data, and hence a continuous process is more suitable for the sequence $\mathbf{X}$ \citep{sander2022residualneuralnetworksdiscretize,queiruga2020continuous}.
This section further conducts layer aggregation by considering state space models in Section 3.1; see Figure \ref{fig:overview} for the illustration.

Specifically, we utilize the Mamba model \citep{gu2023mamba} due to its effectiveness in processing long sequences.
This model is based on S6 models and can provide a better interpretation on how to leverage the previous information and then how to store it based on its importance. Moreover, it has been demonstrated to have a better performance than traditional transformers and RNNs.
Following its principle, we propose our selective state space model layer aggregation below:
\begin{equation}
    \label{S6_rec_ori}
        h^t = g^t(h^{t-1},\boldsymbol{X}^{t}),  \quad
        \boldsymbol{X}^t = f^t(h^{t-1},\boldsymbol{X}^{t-1}),
\end{equation}
where $h^t$ is a hidden state similar to $A^t$ in \Eqref{eq:CNN_agg}, and it represents the recurrently aggregated information up to $(t-1)$th layer.
We may consider an additive form, as in \Eqref{eq:densenet_re}, for $h^t$. 
Moreover, $g^t$ is the relation function between the current SSM hidden layer state and previous hidden layer state with input. As a result, similar to \Eqref{eq:SSM_dis}, the update of $h^t$ can be formulated as:
\begin{equation}
    \label{S6_rec_up}
        h^t = \overline{A}h^{t-1} + \overline{B}\boldsymbol{X}^t,  \quad
        \boldsymbol{X}^{t} = f^t(h^{t-1},\boldsymbol{X}^{t-1}).
\end{equation}
The choice of function $f^t$ is different for CNNs and Transformer-based models, and they are detailed in the following two subsections; see Figures \ref{fig:overview_CNN} and  \ref{fig:overview_Transformer} for their illustrations, respectively. 

%先总体介绍一个Network怎么对应到SSM的H,A,B,delta
%接下来,具体给你们展示两个例子:

\subsection{Application to Deep CNNs}
\label{CNN_application}

\begin{figure}[t]
\begin{center}
%\framebox[4.0in]{$\;$}
\includegraphics[width=0.9\linewidth]{overview_CNN.pdf}
\end{center}
Detailed operations in S6LA module with Convolutional Neural Network. The green arrow shows the hidden state connection, while the grey arrow indicates layers communications.
\label{fig:overview_CNN}
\end{figure}

For CNNs backbones, we adopt ResNet \citep{he2016deep} as the baseline architecture. We propose to concatenate the input at each layer, say $\boldsymbol{X}^t \in \mathbb{R}^{H \times W \times D}$, where $H$ and $W$ represent the height and width, and $D$ indicates the embedding dimension. For each CNN layer, the input to each block in ResNet—comprising a combination of 1D and 3D convolutional operations—is formed by concatenating $\boldsymbol{X}^{t}$ with the state $h^{t-1} \in \mathbb{R}^{H \times W \times N}$ from the previous layer, where $N$ is the dimension of latent states. This integration effectively incorporates SSM information into the convolutional layers, enhancing the network's capacity to learn complex representations. 
% Upon obtaining the output from the $t$-th CNN block, 

For the specific implementation of S6LA in CNNs, we initialize the SSM state $h^0$ using Kaiming normal initialization method \citep{he2015delving}. This initialization technique is crucial for ensuring effective gradient flow throughout the network, and we will further clarify this point in ablation studies. Next, we employ a selective mechanism to derive two components, the coefficient $B$ for input and the interval $\Delta$ as specified in \Eqref{eq:SSM_dis}. For transition matrix $A$, the initial setting is same as in Mamba models \citep{gu2023mamba}. Then with \Eqref{eq:SSM_dis}, we can get the next hidden layer $h^{t}$ based on the last $h^{t-1}$ and $\boldsymbol{X}^{t}$ for each layer in CNNs.

Utilizing \Eqref{eq:SSM_dis}, we compute the subsequent latent state $h^{t}$ based on the previous state $h^{t-1}$ and the input $\boldsymbol{X}^{t}$ for each layer within the CNN architecture. This methodological framework facilitates improved information flow and retention across layers, thereby enhancing the model's performance. Therefore, the specifics of leveraging our S6LA method with CNNs backbones can be outlined as follows:

% \begin{itemize}
%     \item Layer ${t-1}$: We begin by merging the input $\boldsymbol{X}^{t-1}$ and the hidden state $h^{t-1}$ through a simple concatenation along the feature dimension. This concatenated representation allows us to generate the output $\boldsymbol{O}^{t-1}$ with the CNNs backbone (such as ResNet).
%     \item Next Step Computation: The output component $\boldsymbol{O}^{t-1}$ from previous step, contributes to the next input $\boldsymbol{X}^{t}$ and the hidden state $h^t$. Here, the dimensions of $\boldsymbol{O}^{t-1}$ are $H \times W \times D$ where $H$ and $W$ correspond to the height and width of the input images, respectively, and $D$ represents the feature dimension.
%     \item State Update: For the input state update, we define $\boldsymbol{X}^t$ as the sum of $\boldsymbol{X}^{t-1}$ and $\boldsymbol{O}^{t-1}$. For the hidden state, $h^t$ is derived as a function of these two components, following the formulation provided in Equations \ref{eq:SSM_dis}. The equations are as follows with two trainable parameters $W_{\Delta}$ and $W_B$ (for the $t-1$ layer):
% \begin{equation}
%     \label{CNN_s6_1}
%     h^t = e^{(\Delta A)} h^{t-1} + \Delta B \boldsymbol{O}^{t-1},
% \end{equation}
% where $\Delta = W_{\Delta} (Conv(\boldsymbol{O}^{t-1})),
%     B = W_B (Conv(\boldsymbol{O}^{t-1}))$.
% \end{itemize}

\para{Input Treatment:} We begin by merging the input $\boldsymbol{X}^{t}$ and the hidden state $h^{t-1}$ through a simple concatenation along the feature dimension. This concatenated representation allows us to generate the output $\boldsymbol{O}^{t}$ with the CNNs backbone (such as ResNet).

\para{Latent State Update:} For the input state update, we define $\boldsymbol{X}^{t+1}$ as the sum of $\boldsymbol{X}^{t}$ and $\boldsymbol{O}^{t}$. For the hidden state, $h^t$ is derived as a function of these two components, following the formulation provided in \Eqref{eq:SSM_dis}. The equations are as follows with two trainable parameters $W_{\Delta}$ and $W_B$ (for the $t-1$ layer):
\begin{equation}
    \label{CNN_s6_1}
    h^t = e^{(\Delta A)} h^{t-1} + \Delta B \boldsymbol{O}^{t},
\end{equation}
where $\Delta = W_{\Delta} (\text{Conv}(\boldsymbol{O}^{t})),
    B = W_B (\text{Conv}(\boldsymbol{O}^{t}))$.

\para{Output Computation:} The output component $\boldsymbol{O}^{t}$ from the input treatment step, contributes to the next input $\boldsymbol{X}^{t+1}$ and the computation is: $\boldsymbol{X}^{t+1} = \boldsymbol{O}^{t} + \boldsymbol{X}^{t}$.

\begin{figure}[t]
\begin{center}
%\framebox[4.0in]{$\;$}
\includegraphics[width=0.9\linewidth]{overview_Transformer.pdf}
\end{center}
Diagram of the S6LA architecture with Transformer. The green arrow shows the hidden state connection, while the grey arrow indicates communication between layers. The input consists of image patch tokens (\( \boldsymbol{X}^{t}_p \)) and a class token (\( \boldsymbol{X}^{t}_c \)), processed through positional embedding and attention. The class token is cloned as $\boldsymbol{X}_c^{t'}$, and parameters \( W_\Delta \) and \( W_B \) update the hidden state (\( h^t \)). Updated patch tokens are combined with the class token to form the next input (\( \boldsymbol{X}^{t+1} \)).
\label{fig:overview_Transformer}
\end{figure}

\subsection{Application to Deep ViTs}

In our exploration of S6LA implementation in deep ViT backbones, we draw parallels between the integration of the state space model (SSM) state and the mechanisms used in convolutional neural networks (CNNs). However, the methods of combining inputs within transformer blocks differ significantly from those in CNNs. Like the treatment of attention mechanism, we utilize multiplication combination instead of simply concatenating to deal with the input $\boldsymbol{X}^{t}$ and $h^{t-1}$ in transformer-based models. This approach enhances the interaction between input features and SSM, enabling richer feature representation. Then the next paragraghs gives the specifics of leveraging our S6LA method with ViTs backbones as follows:

\paragraph{Input Treatment:} We begin by combining the class token and input, alongside the application of positional embeddings. 
% The input dimension for the transformer network is defined as: $\boldsymbol{X}^{t-1}_{input} \in \mathbb{R}^{1 \times D}$ where $D = (N+1) \times C$, $N$ is the number of patches and $C$ is the embedding dimension. 
Then following the attention mechanism, $\boldsymbol{X}^{t}_{\text{input}} \in \mathbb{R}^{(L+1)\times D}$ appeared where $L$ is the number of patches and $D$ is the embedding dimension, and it is split into two components next: image patch tokens $\boldsymbol{X}^{t}_p \in \mathbb{R}^{L\times D}$ and a class token $\boldsymbol{X}^{t}_c \in \mathbb{R}^{1 \times D}$.
\begin{small}
\begin{equation}
\boldsymbol{X}^{t}_{\text{input}} = \text{Add} \& \text{Norm}(\text{MLP}(\text{Add} \& \text{Norm}(\text{Attn}(\boldsymbol{X}^{t}))));   \quad \boldsymbol{X}^{t}_p, \boldsymbol{X}^{t}_c = \text{Split}(\boldsymbol{X}^{t}_{\text{input}}).
\end{equation}
\end{small}
The class token plays a crucial role in assessing the correlation between $\boldsymbol{X}^{t}$ and $h^{t-1}$. Our model setting can effectively bridge the features extracted from the patches with the SSM state by facilitating a better integration into the hidden state since it could be considered as a summary feature of the image in sequential layers. 

\paragraph{Latent State Update:} Given the split class token in last step, the hidden state is updated similar to application in CNNs:
\begin{equation}
    h^t = e^{(\Delta A)} h^{t-1} + \Delta B \boldsymbol{X}^{t}_c,
\end{equation}
where $\Delta$ and $B$ are calculated from class token with selective mechanism:
\begin{equation}
\Delta = W_{\Delta} (\boldsymbol{X}^{t}_c), \quad
    B = W_B (\boldsymbol{X}^{t}_c).
\end{equation}

\paragraph{Output Computation:} At the same time, the new patch tokens $\widehat{\boldsymbol{X}}_p^{t}$ are computed as the sum of the previous patch tokens and the product of the previous patch tokens with $h^t$:
\begin{equation}
    \widehat{\boldsymbol{X}}_p^{t} = \boldsymbol{X}^{t}_p + W\boldsymbol{X}^{t}_p h^t.
\end{equation}
Then the next input, $\boldsymbol{X}^{t+1}$, is derived from the concatenation of the updated patch and class tokens:
\begin{equation}
   \boldsymbol{X}^{t+1} = \text{Concat}(\widehat{\boldsymbol{X}}_p^{t},\boldsymbol{X}^{t}_c). 
\end{equation}





\section{Experiments}

\subsection{Datasets}

\textbf{MSMARCO}.
We utilized the MS MARCO Passage Ranking dataset as the data source to evaluate the ability of our method to improve document rankings under more challenging topic-query tasks. Specifically, we assessed whether our method could significantly enhance the ranking of documents by the retrieval model within a RAG system.

To construct topic-lists for evaluation, we applied a K-means clustering algorithm to group similar queries, forming topics that each contained a series of related queries. To further evaluate the performance of our method under extreme topic-query scenarios, we applied an intra-topic similarity filtering process. Only topics with queries exhibiting high semantic diversity and containing a sufficient number of queries were retained.

This process resulted in 29 topics, with each topic containing an average of 22.28 queries. The average similarity score within each topic was approximately 0.5, indicating sufficient diversity among queries to ensure a rigorous evaluation. This curated dataset enabled us to test the robustness of our method in handling highly diverse and challenging topic-query tasks within a RAG system.

\textbf{PROCON}.
To conduct our experiments, we utilized controversial topic data scraped from the PROCON.ORG website. This dataset includes over 80 topics covering various domains, such as society, health, government, and education. Each topic is discussed from two stance labels \{\textit{PRO (support), CON (oppose)}\}, with passages arguing from these perspectives.

To simulate real-world user interactions with a RAG system, we instructed a large language model (GPT-4o) to act as a user and generate 40 potential sub-queries for each topic. These sub-queries were designed to reflect the diverse questions and concerns users might raise when exploring a specific controversial topic. 

After generating the sub-queries, we applied a similarity filtering process to ensure diversity by retaining only those with a similarity score below approximately 0.85. The filtering step effectively removed redundant queries while preserving a wide range of perspectives. As a result, the final set of topic-queries achieved an average similarity score of approximately 0.7, ensuring that the queries were sufficiently diverse yet semantically relevant. This process provided a robust and balanced sub-queries set for evaluation.


\subsection{Experiment Details}
The specific setting details for the Topic-queries RAG manipulation experiment are as follows:

(1) Black-box RAG. We represent the black-box RAG process as \( \text{RAG}_{\text{black}} \). The RAG framework is Conversational RAG from LangChain. The LLMs adopted in RAG are the open-source models Meta-Llama-3.1-8B-Instruct (Llama3.1), Qwen-2.5-7B-Instruct(Qwen2.5). The system prompt and additional detailed descriptions are provided in Appendix~\ref{exp-detail}.

(2) Retrieval Model Specification. We benchmark three dominant dense retrievers—Contriever \cite{gao2021unsupervised}, DPR \cite{karpukhin-etal-2020-dense}, and ANCE \cite{xiong2020approximate}.By convention, we use dot product between the embedding vectors of questions(queries) and candidate documents as their similarity score \(R\) in the ranking. 


\label{opinion-classfication}
(3) Opinion classification. We use Qwen2.5-Instruct-72B as the opinion classifier. Qwen2.5-Instruct-72B, due to its large parameter size, is capable of accurately identifying and classifying opinions within text.

(4) Experimental parameters. In knowledge-guided attack process, we set the maximum editing distance $\epsilon$ to 0.2, the semantic similarity threshold $\lambda$ to 0.85, and the number of iterations $N$ to 5. For adversarial trigger generation, we used a beam size of 3, top-$k$ of 10, a batch size of 32, a temperature of 1.0, a learning rate of 0.005, and a sequence length of 10. In RAG\textsubscript{black}, $k$ (the number of retrieved documents) is set to 3, with the LLM temperature also fixed at 1.0 to mirror real-world conditions.

(5) Poisoned Target. For the PROCON dataset, to investigate the manipulation performance under more challenging conditions, we performed relevance ranking for the documents with respect to each topic-query set $Q$ and the target stance $S_t$ . From the ranked list, we selected the last five documents (i.e., those with the lowest relevance) as the target poisoned documents.
For the MS MARCO dataset, we utilized the top-1000 relevance-ranked passage list for each topic-query set. From this list, we selected the passage with the lowest average rank as the target passage. This approach ensures that the evaluation focuses on passages that are least relevant to the target queries, thus providing a more rigorous benchmark for the proposed method.

(6) Experimental environment. All our experiments are conducted in Python 3.8 environment and run on a NVIDIA DGX H100 GPU. 

\subsection{Research Questions}

We propose four research questions to evaluate the effectiveness of our method in the topic-queries task, focusing on black-box NRM attacks and opinion manipulation to RAGs.

\textbf{RQ1}: Can Topic-FlipRAG significantly enhance the rankings of target documents in the RAG retriever for topic-queries?

\textbf{RQ2}: To what extent does Topic-FlipRAG affect the answers generated by the target RAG systems?

\textbf{RQ3}: Does topic-oriented opinion manipulation significantly impact users' perceptions of controversial topics?

\textbf{RQ4}: How robust does Topic-FlipRAG against exisiting mitigation mechanism?

\subsection{Baseline Settings}
To assess the effectiveness of our proposed method, we compare it against adversarial attack baselines designed for black-box, topic-oriented RAG scenarios, ensuring minimal modifications to the original documents. We exclude BadRAG\cite{xue2024badrag}, a backdoor RAG attack limited to white-box scenarios, and topic-IR-attack\cite{liu2023topic}, as its incomplete implementation prevents reliable replication.
For the selected baseline methods, we adapted them to meet the requirements of our task while preserving their core components. A brief overview of the baseline methods is provided below, with detailed descriptions available in Appendix~\ref{baselines-details}.

\textbf{PoisonedRAG.}
Zou et al.\cite{zou2024poisonedrag} propose an approach adaptable to both black-box and white-box settings. For our task, we employ its black-box strategy by inserting a randomly chosen query from the topic-queries set \( Q \) at the beginning of each document.

\textbf{PAT.}
This gradient-based adversarial retrieval attack uses a pairwise loss function to generate triggers that meet fluency and coherence constraints. We adapt PAT to produce triggers \( T_{\text{pat}} \) for target documents within the topic-queries set, evaluating their effectiveness under black-box conditions.


\textbf{Collision.}
This method generates adversarial paragraphs (collisions) via gradient-based optimization to produce content semantically aligned with the target query. In a topic-queries context, we create collisions for the entire topic-queries set and examine their transfer performance on black-box RAG retrievers.

These baseline methods provide benchmarks for comparing the efficacy of our approach in a fully black-box, topic-oriented RAG attack scenario.

\subsection{Evaluation Metrics}

For \textbf{RQ1}, we focus on ranking manipulation. We measure the average proportion of target opinions in top-3 rankings before and after manipulation (\(\text{Top3}_{\text{ori}}, \text{Top3}_{\text{att}}\)) and define top3-v as their difference. We also employ the Ranking Attack Success Rate (RASR), reflecting how often target documents are successfully boosted, and Boost Rank (BRank), denoting the average rank improvement for all target documents. Lastly, we report the proportion of target documents in the Top-50 and Top-500 positions to indicate how effectively they are pushed toward higher rankings.

\textbf{top3-v.} Computed by subtracting \(\text{Top3}_{\text{ori}}\) from \(\text{Top3}_{\text{att}}\), top3-v ranges from -1 to 1. A positive value signifies a successful increase of the target opinion in top-3 results, while a negative value indicates a detrimental effect.

\textbf{Ranking Attack Success Rate (RASR).} RASR captures how frequently target documents are successfully boosted in each query’s ranking. Higher values indicate greater attack effectiveness.

\textbf{Boost Rank (BRank).} BRank is the average rank improvement for all target documents under each query. A target document contributes negatively if its rank is unintentionally lowered.

\textbf{Top-50, Top-500.} These metrics represent the percentage of target documents that move into specific ranking thresholds in the MS MARCO Dataset after manipulation. Higher percentages imply more effective promotion of target documents. 


For \textbf{RQ2}, we employ Average Stance Variation (ASV) to assess how significantly our opinion manipulation influences the LLM’s responses in a black-box RAG. To address the natural variability of controversial topics and the inherent instability of large language models, we also propose Real Adjusted ASV (\(\Delta\)-ASV).

\textbf{Average Stance Variation (ASV).}
ASV is defined as the absolute difference between the manipulated opinion score and the original opinion score assigned to an LLM response (0 = opposing, 1 = neutral, 2 = supporting). A higher ASV signifies a more pronounced shift in polarity and hence greater manipulation effectiveness.

\textbf{Real Adjusted ASV ($\Delta$-ASV)}. To account for the inherent variability of controversial topics and the instability of large language models, we measure the baseline ASV in a clean state, denoted as ASV\textsubscript{clean} (calculated without adversarial manipulation). $\Delta$-ASV is then obtained by subtracting ASV\textsubscript{clean} from the manipulated ASV, i.e., \( \text{$\Delta$-ASV} = \text{ASV} - \text{ASV\textsubscript{clean}} \). This adjustment ensures that $\Delta$-ASV reflects the true impact of adversarial manipulation by eliminating the influence of natural stance variation. It reflects the extent to which the polarity of the RAG-system outputs is affected by the manipulation.  A positive $\Delta$-ASV indicates a significant shift in the opinion polarity due to manipulation, with larger values representing greater manipulation effectiveness.

\section{Conclusion}
We introduced \methodname, an effective training framework defending against MIAs for LLMs. The extensive experiments demonstrate its robustness in protecting privacy while maintaining strong language modeling performance across various datasets and architectures. Although our study focuses on fine-tuning due to computational constraints, \methodname can be seamlessly applied to large-scale pretraining, as done in prior selective pretraining work~\cite{lin2024not}. By categorizing tokens and treating them appropriately, \methodname opens a novel pathway for MIA defense. Future work can explore improved token selection strategies and multi-objective training approaches.

\newpage

\bibliography{Ref}
\bibliographystyle{iclr2025_conference}

\newpage


\section{Appendix}
\label{appendix}

\subsection{Survey Questions}
\label{app:survey}

\subsubsection{Scenarios}

Participants were asked about three classes of hiring scenarios: technical coding assessments, resume review, and behavioral interviews (the scenarios are listed by class below). For each scenario, they answered two questions, both on 5-point Likert scales:
\begin{itemize}
    \item How fair does this hiring process seem to you? (``This hiring process seems fair'', 1: Strongly disagree to 5: Strongly agree)
    \item If you were applying for a technology job, would you want to be evaluated this way? (``I want to be evaluated this way'', 1: Strongly disagree to 5: Strongly agree)
\end{itemize}

[Technical Coding Assessments]
\begin{enumerate}
\item An applicant submits a sample of code, which is reviewed by a recruitment team, who determines whether the applicant advances to the next phase.
\item An applicant is given an online coding assessment, which is evaluated by an algorithm. If the applicant reaches a certain score on the autograder, the applicant advances to the next phase. All algorithmic decisions are reviewed by a recruitment team.
\item An applicant is given an online coding assessment, which is evaluated by an algorithm. If the algorithm rejects the applicant, the decision is reviewed by a recruitment team. 
\item An applicant is given an online coding assessment, which is evaluated by an algorithm. If the algorithm advances the applicant to the next phase, the decision is reviewed by a recruitment team. 
\item An applicant is given an online coding assessment, which is evaluated by an algorithm that determines whether an applicant advances to the next phase. 
% \item Why did you select the answers above for the different scenarios related to coding assessments?
\end{enumerate}

[Resume Review]
\begin{enumerate}
\item An applicant submits a resume, which is reviewed by a recruitment team, who determines whether the applicant advances to the next phase.
\item An applicant submits a resume, which is evaluated by an algorithm. The algorithm determines whether the applicant advances to the next phase. All algorithmic decisions are reviewed by a recruitment team. 
\item An applicant submits a resume, which is evaluated by an algorithm. If the algorithm rejects your application, the decision is reviewed by a recruitment team. 
\item An applicant submits a resume, which is evaluated by an algorithm. If the algorithm advances the applicant to the next phase, the decision is reviewed by a recruitment team. 
\item An applicant submits a resume, which is evaluated by an algorithm that determines whether an applicant advances to the next phase. 
% \item Why did you select the answers above for the different scenarios related to resumes?
\end{enumerate}

[Behavioral Interviews]
\begin{enumerate}
\item An applicant has an interview with a member of the recruitment team. The recruitment team determines whether the applicant advances to the next phase.
\item An applicant participates in an automated video interview, where the applicant receives interview questions and records video responses. The video, including the applicant’s speech (fluency, prosody, pronunciation, language usage) and nonverbal behaviors (facial expressions, posture, and eye movements), is evaluated by an algorithm. Whether you advance to the next phase is determined by the algorithm. All algorithmic decisions are reviewed by a recruitment team.
\item An applicant participates in an automated video interview, where the applicant receives interview questions and records video responses. The video, including the applicant’s speech (fluency, prosody, pronunciation, language usage) and nonverbal behaviors (facial expressions, posture, and eye movements), is evaluated by an algorithm. If the algorithm rejects the applicant,  the decision is reviewed by a recruitment team. 
\item An applicant participates in an automated video interview, where the applicant receives interview questions and records video responses. The video, including the applicant’s speech (fluency, prosody, pronunciation, language usage) and nonverbal behaviors (facial expressions, posture, and eye movements), is evaluated by an algorithm. If the algorithm advances the applicant to the next phase, the decision is reviewed by a recruitment team. 
\item An applicant participates in an automated video interview, where the applicant receives interview questions and records video responses. The video, including the applicant’s speech (fluency, prosody, pronunciation, language usage) and nonverbal behaviors (facial expressions, posture, and eye movements), is evaluated by an algorithm that determines whether an applicant advances to the next phase.
% \item Why did you select the answers above for the different scenarios related to interviews?
\end{enumerate}

At the end of each set of Likert questions, participants were also asked an open response question (``Why did you select the answers above for the different scenarios related to [coding assessments/resumes/interviews]?'').

\subsubsection{Awareness of AEDTs}

In this section, participants were asked for each hiring process (online coding assessment, automated resume readers, and automated interviews) to check the box to indicate whether they have experience or knowledge of it:
\begin{itemize}
    \item[$\square$] Yes, I have experienced it
    \item[$\square$] No, but I have heard of it
    \item[$\square$] I'm not sure, but have heard of it
    \item[$\square$] No, I have not heard of or experienced it
\end{itemize}

Participants also responded to ``I know how my data was used in the hiring process'' and ``I received feedback from automated hiring algorithms'' from 1: Strongly disagree to 5: Strongly agree.

\subsubsection{Strategy Use}

Participants were asked the following questions about strategy use:
\begin{itemize}
\item Have you modified your resume specifically for automated resume readers? (Yes/No)
\item Have you added keywords from your job description? (Yes/No)
\item Have you changed the layout? (Yes/No)
\item Have you put it through a resume scanner? (Yes/No)
\item Have you modified your resume in some other way for automated hiring? (please specify)
\item Did you use a tool (LeetCode, HackerRank, etc.) to practice for coding assessments? (Yes/No)
\item Have you used anything else to prepare for automated assessments? (please specify)
\item Have you ever received a job referral? (Yes/No)
\item What proportion of your job applications did you have a referral for? (approximate percentage)
\item Approximately how many companies did you apply to? 
\item How did you learn about the application process? (check all that apply)
    \begin{itemize}
        \item[$\square$] Application materials and descriptions
        \item[$\square$] Online resources
        \item[$\square$] Career services through university 
        \item[$\square$] People who had gone through the application process
        \item[$\square$] Recruiter outside of company
        \item[$\square$] Recruiter through company
        \item[$\square$] Family members who worked at companies 
        \item[$\square$] Friends who worked at companies 
        \item[$\square$] Other people who worked at companies
    \end{itemize}
There was also an option to include additional strategies and an attention check in this stage.
\end{itemize}

\subsubsection{Hiring Outcome}
Participants were also asked about their hiring process and its outcome.
\begin{itemize}
\item Have you completed your hiring process? (Yes/No/Not applying to jobs)
\item I am satisfied with my hiring process so far. (1: Strongly disagree to 5: Strongly agree)
\item What is the outcome of your hiring process so far? 
    \begin{itemize}
        \item[$\square$] Multiple job offers
        \item[$\square$] One job offer
        \item[$\square$] No job offers
        \item[$\square$] Not applying to jobs
    \end{itemize}
\end{itemize}

\subsubsection{Demographic Information}
All questions in this section were optional and asked participants to disclose demographic information.

\begin{itemize}
    \item How would you describe your gender identity? (Select all that apply)
        \begin{itemize}
            \item[$\square$] Woman
            \item[$\square$] Man
            \item[$\square$] Non-binary
            \item[$\square$] Genderqueer
            \item[$\square$] Agender
            \item[$\square$] A gender not listed
        \end{itemize}
    \item What best describes you? (Select all that apply)
        \begin{itemize}
            \item[$\square$] Black or African-American
            \item[$\square$] American Indian or Alaskan Native
            \item[$\square$] Asian American or Asian
            \item[$\square$] Hispanic or Latinx
            \item[$\square$] Middle Eastern or North African
            \item[$\square$] Pacific Islander
            \item[$\square$] White or Caucasian
            \item[$\square$] Some other race, ethnicity, or origin 
        \end{itemize}
    \item What is your family’s approximate household income? 
\end{itemize}

\clearpage 

\subsection{Complete Statistical Results}
\label{app:stats}

\begin{table}[ht]
\begin{tabular}{lrrrrl}
\hline
\textbf{}                                            & \textbf{Estimate} & \textbf{Std. Error} & \textbf{t value} & \textbf{Pr(\textgreater{}|t|)} & \textbf{} \\ \hline
(Intercept)                                       & 2.786  & 0.266 & 10.493 & \textless{}0.01 &   \\
Added job description keywords to resume & 0.139  & -1.468    & 0.144 & 0.121            &   \\
Modified resume layout for resume readers & 0.150         & 0.133           & 1.119            & 0.265                         &           \\
Put resume through a resume scanner               & 0.001  & 0.136 & 0.007  & 0.995           &   \\
Practiced for online coding assessment            & 0.249  & 0.140 & 1.787  & 0.075           &   \\
Used referrals                                    & -0.336 & 0.136 & -2.478 & 0.014           & * \\
Percent of companies applied to with referral   & 0.002         & 0.003           & 0.817            & 0.415                         &           \\
Number of companies applied to                    & 0.001  & -0.516    & 0.606 & 0.405            &   \\
Awareness of online coding assessments            & -0.551 & 0.235 & -2.349 & 0.020           & * \\
Awareness of resume scanners                      & 0.014  & 0.183 & 0.076  & 0.940           &   \\
Awareness of automated video interviews           & 0.354  & 0.170 & 2.113  & 0.036           & * \\
Knowledge of data use                             & 0.055  & 0.047 & 1.162  & 0.247           &   \\
Received feedback in the hiring process           & 0.058  & 0.046 & 1.257  & 0.210           &   \\
Used application materials and descriptions       & -0.176 & 0.114 & -1.539 & 0.125           &   \\
Used online resources                             & 0.288  & 0.133 & 2.160  & 0.032           & * \\
Used career services through university           & 0.063  & 0.108 & 0.588  & 0.557           &   \\
Talked with people who had recently applied       & 0.129  & 0.127 & 1.012  & 0.313           &   \\
Connected with recruiter outside of company       & 0.053  & 0.159 & 0.336  & 0.737           &   \\
Connected with recruiter through company          & 0.124  & -1.346    & 0.180 & 0.191            &   \\
Had family who worked at companies        & 0.044  & 0.144 & 0.306  & 0.760           &   \\
Had friends who worked at companies               & 0.140  & 0.112 & 1.247  & 0.214           &   \\
Connected with other company contacts             & -0.022 & 0.126 & -0.179 & 0.858           &   \\
Race                                              & 0.005  & 0.109 & 0.425  & 0.671           &   \\
Gender                                            & -0.003 & 0.142 & -0.024 & 0.981           &   \\
Income                                            & 0.0000002  & 0.0000003 & 0.569  & 0.570           &   \\ \hline
\end{tabular}
\caption{\label{tab:fairStats} Linear regression model of procedural fairness perceptions for automated processes based on strategy use, awareness of AEDTs, gender, race, and income.}
\end{table}

\begin{table}[ht]
\begin{tabular}{lrrrrl}
\hline
\textbf{}                                            & \textbf{Estimate} & \textbf{Std. Error} & \textbf{t value} & \textbf{Pr(\textgreater{}|t|)} & \textbf{} \\ \hline
(Intercept)                                 & 2.479  & 0.268 & 9.267  & \textless{}0.01 &    \\
Added job description keywords to resume    & 0.140         & -1.374              & 0.171           & 0.210                         &           \\
Modified resume layout for resume readers & 0.169         & 0.135          & 1.257            & 0.210                         &           \\
Put resume through a resume scanner         & 0.038  & 0.137 & 0.273  & 0.785           &    \\
Practiced for online coding assessment      & 0.201  & 0.141 & 1.427  & 0.155           &    \\
Used referrals                              & -0.316 & 0.137 & -2.312 & 0.022           & *  \\
Percent of companies applied to with referral   & 0.002         & 0.003           & 0.670            & 0.504                         &           \\
Number of companies applied to              & 0.0004  & 0.001 & 0.544  & 0.589           &    \\
Awareness of online coding assessments      & -0.557 & 0.237 & -2.356 & 0.019           & *  \\
Awareness of resume scanners                & -0.046 & 0.184 & -0.248 & 0.805           &    \\
Awareness of automated video interviews     & 0.440  & 0.169 & 2.608  & {0.010}           & * \\
Knowledge of data use                       & 0.106  & 0.047 & 2.240  & 0.026           & *  \\
Received feedback in the hiring process     & 0.027  & 0.046 & 0.588  & 0.558           &    \\
Used application materials and descriptions & -0.220 & 0.012 & -1.911 & 0.057           &    \\
Used online resources                       & 0.261  & 0.134 & 1.942  & 0.054           &    \\
Used career services through university     & 0.152  & 0.108 & 1.399  & 0.163           &    \\
Talked with people who had recently applied & 0.172  & 0.128 & 1.344  & 0.181           &    \\
Connected with recruiter outside of company & 0.160  & -0.005    & 0.996 & 0.180          &    \\
Connected with recruiter through company    & 0.125  & -1.392    & 0.165 & 0.968           &    \\
Had family who worked at companies  & -0.006 & 0.145 & -0.040 & 0.968           &    \\
Had friends who worked at companies         & 0.134  & 0.113 & 1.188  & 0.236           &    \\
Connected with other company contacts       & 0.049  & 0.127 & 0.385  & 0.700           &    \\
Race                                        & 0.013  & 0.110 & 0.122  & 0.903           &    \\
Gender                                      & -0.116 & 0.143 & -0.815 & 0.416           &    \\
Income                                      & 0.0000002  & 0.0000003 & 0.623  & 0.534           &    \\ \hline
\end{tabular}
\caption{\label{tab:evalStats} Linear regression model of willingness to be evaluated by automated processes based on strategy use, awareness of AEDTs, gender, race, and income.}
\end{table}

\clearpage

\begin{table}[ht]
\begin{tabular}{lrrrrrl}
\toprule
& \textbf{Estimate}  & \textbf{Std. Error} & \textbf{t value} & \textbf{Pr(\textgreater{}|t|)} &   \\
\hline
(Intercept)                                          & 0.329     & 0.237      & 1.386   & 0.168                &   \\
Added job description keywords to resume    & 0.168     & 0.107      & 1.563   & 0.121                 &   \\
Modified resume layout for resume readers & 0.103     & -0.724     & 0.471   & 0.515                 &   \\
Put resume through a resume scanner                  & 0.020     & 0.101      & 0.201   & 0.841                 &   \\
Practiced for online coding assessment               & -0.201    & 0.133      & -1.513  & 0.133                 &   \\
Used referrals                                       & 0.122     & 0.100      & 1.213   & 0.227                 &   \\
Percent of companies applied to with referral   & 0.004     & 0.002      & 2.063   & 0.041                 & * \\
Number of companies applied to                       & 0.0004    & 0.001     & 0.835   & 0.405                 &   \\
Awareness of online coding assessments               & 0.050     & 0.199      & 0.252   & 0.801                 &   \\
Awareness of resume scanners                         & -0.019    & 0.173      & -0.109  & 0.913                 &   \\
Awareness of automated video interviews              & -0.036    & 0.157      & -0.228  & 0.820                 &   \\
Knowledge of data use                                & 0.039     & 0.004      & 0.984   & 0.327                 &   \\
Received feedback in the hiring process              & 0.011     & 0.004      & 0.302   & 0.763                 &   \\
Used application materials and descriptions          & 0.025     & 0.009      & 0.279   & 0.781                 &   \\
Used online resources                                & -0.174    & 0.115      & -1.518  & 0.132                 &   \\
Used career services through university              & 0.055     & 0.085      & 0.644   & 0.521                 &   \\
Talked with people who had recently applied          & 0.024     & 0.107      & 0.225   & 0.823                 &   \\
Connected with recruiter outside of company          & 0.009     & 0.112      & 0.080   & 0.937                 &   \\
Connected with recruiter through company             & 0.115     & 0.088      & 1.314   & 0.191                 &   \\
Had family who worked at companies           & -0.140    & 0.109      & -1.287  & 0.200                 &   \\
Had friends who worked at companies                  & 0.160     & 0.087      & 1.841   & 0.068                 &   \\
Connected with other company contacts                & -0.101    & 0.093     & -1.089  & 0.278                &   \\
Race                                                 & -0.008    & 0.119      & -0.070  & 0.945                 &   \\
Gender                                               & 0.081     & 0.081     & 0.991   & 0.324                 &   \\
Income                                               & 0.000001 & 0.0000002  & 2.530   & 0.013                 & * \\
\bottomrule
\end{tabular}
\caption{\label{tab:jobStats} Linear regression model of job success based on strategy use, awareness of AEDTs, gender, race, and income.}
\end{table}




\end{document}


