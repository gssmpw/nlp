
\documentclass{article} % For LaTeX2e
\usepackage{iclr2025_conference,times}

% Optional math commands from https://github.com/goodfeli/dlbook_notation.
%%%%% NEW MATH DEFINITIONS %%%%%

\usepackage{amsmath,amsfonts,bm}
\usepackage{derivative}
% Mark sections of captions for referring to divisions of figures
\newcommand{\figleft}{{\em (Left)}}
\newcommand{\figcenter}{{\em (Center)}}
\newcommand{\figright}{{\em (Right)}}
\newcommand{\figtop}{{\em (Top)}}
\newcommand{\figbottom}{{\em (Bottom)}}
\newcommand{\captiona}{{\em (a)}}
\newcommand{\captionb}{{\em (b)}}
\newcommand{\captionc}{{\em (c)}}
\newcommand{\captiond}{{\em (d)}}

% Highlight a newly defined term
\newcommand{\newterm}[1]{{\bf #1}}

% Derivative d 
\newcommand{\deriv}{{\mathrm{d}}}

% Figure reference, lower-case.
\def\figref#1{figure~\ref{#1}}
% Figure reference, capital. For start of sentence
\def\Figref#1{Figure~\ref{#1}}
\def\twofigref#1#2{figures \ref{#1} and \ref{#2}}
\def\quadfigref#1#2#3#4{figures \ref{#1}, \ref{#2}, \ref{#3} and \ref{#4}}
% Section reference, lower-case.
\def\secref#1{section~\ref{#1}}
% Section reference, capital.
\def\Secref#1{Section~\ref{#1}}
% Reference to two sections.
\def\twosecrefs#1#2{sections \ref{#1} and \ref{#2}}
% Reference to three sections.
\def\secrefs#1#2#3{sections \ref{#1}, \ref{#2} and \ref{#3}}
% Reference to an equation, lower-case.
\def\eqref#1{equation~\ref{#1}}
% Reference to an equation, upper case
\def\Eqref#1{Equation~\ref{#1}}
% A raw reference to an equation---avoid using if possible
\def\plaineqref#1{\ref{#1}}
% Reference to a chapter, lower-case.
\def\chapref#1{chapter~\ref{#1}}
% Reference to an equation, upper case.
\def\Chapref#1{Chapter~\ref{#1}}
% Reference to a range of chapters
\def\rangechapref#1#2{chapters\ref{#1}--\ref{#2}}
% Reference to an algorithm, lower-case.
\def\algref#1{algorithm~\ref{#1}}
% Reference to an algorithm, upper case.
\def\Algref#1{Algorithm~\ref{#1}}
\def\twoalgref#1#2{algorithms \ref{#1} and \ref{#2}}
\def\Twoalgref#1#2{Algorithms \ref{#1} and \ref{#2}}
% Reference to a part, lower case
\def\partref#1{part~\ref{#1}}
% Reference to a part, upper case
\def\Partref#1{Part~\ref{#1}}
\def\twopartref#1#2{parts \ref{#1} and \ref{#2}}

\def\ceil#1{\lceil #1 \rceil}
\def\floor#1{\lfloor #1 \rfloor}
\def\1{\bm{1}}
\newcommand{\train}{\mathcal{D}}
\newcommand{\valid}{\mathcal{D_{\mathrm{valid}}}}
\newcommand{\test}{\mathcal{D_{\mathrm{test}}}}

\def\eps{{\epsilon}}


% Random variables
\def\reta{{\textnormal{$\eta$}}}
\def\ra{{\textnormal{a}}}
\def\rb{{\textnormal{b}}}
\def\rc{{\textnormal{c}}}
\def\rd{{\textnormal{d}}}
\def\re{{\textnormal{e}}}
\def\rf{{\textnormal{f}}}
\def\rg{{\textnormal{g}}}
\def\rh{{\textnormal{h}}}
\def\ri{{\textnormal{i}}}
\def\rj{{\textnormal{j}}}
\def\rk{{\textnormal{k}}}
\def\rl{{\textnormal{l}}}
% rm is already a command, just don't name any random variables m
\def\rn{{\textnormal{n}}}
\def\ro{{\textnormal{o}}}
\def\rp{{\textnormal{p}}}
\def\rq{{\textnormal{q}}}
\def\rr{{\textnormal{r}}}
\def\rs{{\textnormal{s}}}
\def\rt{{\textnormal{t}}}
\def\ru{{\textnormal{u}}}
\def\rv{{\textnormal{v}}}
\def\rw{{\textnormal{w}}}
\def\rx{{\textnormal{x}}}
\def\ry{{\textnormal{y}}}
\def\rz{{\textnormal{z}}}

% Random vectors
\def\rvepsilon{{\mathbf{\epsilon}}}
\def\rvphi{{\mathbf{\phi}}}
\def\rvtheta{{\mathbf{\theta}}}
\def\rva{{\mathbf{a}}}
\def\rvb{{\mathbf{b}}}
\def\rvc{{\mathbf{c}}}
\def\rvd{{\mathbf{d}}}
\def\rve{{\mathbf{e}}}
\def\rvf{{\mathbf{f}}}
\def\rvg{{\mathbf{g}}}
\def\rvh{{\mathbf{h}}}
\def\rvu{{\mathbf{i}}}
\def\rvj{{\mathbf{j}}}
\def\rvk{{\mathbf{k}}}
\def\rvl{{\mathbf{l}}}
\def\rvm{{\mathbf{m}}}
\def\rvn{{\mathbf{n}}}
\def\rvo{{\mathbf{o}}}
\def\rvp{{\mathbf{p}}}
\def\rvq{{\mathbf{q}}}
\def\rvr{{\mathbf{r}}}
\def\rvs{{\mathbf{s}}}
\def\rvt{{\mathbf{t}}}
\def\rvu{{\mathbf{u}}}
\def\rvv{{\mathbf{v}}}
\def\rvw{{\mathbf{w}}}
\def\rvx{{\mathbf{x}}}
\def\rvy{{\mathbf{y}}}
\def\rvz{{\mathbf{z}}}

% Elements of random vectors
\def\erva{{\textnormal{a}}}
\def\ervb{{\textnormal{b}}}
\def\ervc{{\textnormal{c}}}
\def\ervd{{\textnormal{d}}}
\def\erve{{\textnormal{e}}}
\def\ervf{{\textnormal{f}}}
\def\ervg{{\textnormal{g}}}
\def\ervh{{\textnormal{h}}}
\def\ervi{{\textnormal{i}}}
\def\ervj{{\textnormal{j}}}
\def\ervk{{\textnormal{k}}}
\def\ervl{{\textnormal{l}}}
\def\ervm{{\textnormal{m}}}
\def\ervn{{\textnormal{n}}}
\def\ervo{{\textnormal{o}}}
\def\ervp{{\textnormal{p}}}
\def\ervq{{\textnormal{q}}}
\def\ervr{{\textnormal{r}}}
\def\ervs{{\textnormal{s}}}
\def\ervt{{\textnormal{t}}}
\def\ervu{{\textnormal{u}}}
\def\ervv{{\textnormal{v}}}
\def\ervw{{\textnormal{w}}}
\def\ervx{{\textnormal{x}}}
\def\ervy{{\textnormal{y}}}
\def\ervz{{\textnormal{z}}}

% Random matrices
\def\rmA{{\mathbf{A}}}
\def\rmB{{\mathbf{B}}}
\def\rmC{{\mathbf{C}}}
\def\rmD{{\mathbf{D}}}
\def\rmE{{\mathbf{E}}}
\def\rmF{{\mathbf{F}}}
\def\rmG{{\mathbf{G}}}
\def\rmH{{\mathbf{H}}}
\def\rmI{{\mathbf{I}}}
\def\rmJ{{\mathbf{J}}}
\def\rmK{{\mathbf{K}}}
\def\rmL{{\mathbf{L}}}
\def\rmM{{\mathbf{M}}}
\def\rmN{{\mathbf{N}}}
\def\rmO{{\mathbf{O}}}
\def\rmP{{\mathbf{P}}}
\def\rmQ{{\mathbf{Q}}}
\def\rmR{{\mathbf{R}}}
\def\rmS{{\mathbf{S}}}
\def\rmT{{\mathbf{T}}}
\def\rmU{{\mathbf{U}}}
\def\rmV{{\mathbf{V}}}
\def\rmW{{\mathbf{W}}}
\def\rmX{{\mathbf{X}}}
\def\rmY{{\mathbf{Y}}}
\def\rmZ{{\mathbf{Z}}}

% Elements of random matrices
\def\ermA{{\textnormal{A}}}
\def\ermB{{\textnormal{B}}}
\def\ermC{{\textnormal{C}}}
\def\ermD{{\textnormal{D}}}
\def\ermE{{\textnormal{E}}}
\def\ermF{{\textnormal{F}}}
\def\ermG{{\textnormal{G}}}
\def\ermH{{\textnormal{H}}}
\def\ermI{{\textnormal{I}}}
\def\ermJ{{\textnormal{J}}}
\def\ermK{{\textnormal{K}}}
\def\ermL{{\textnormal{L}}}
\def\ermM{{\textnormal{M}}}
\def\ermN{{\textnormal{N}}}
\def\ermO{{\textnormal{O}}}
\def\ermP{{\textnormal{P}}}
\def\ermQ{{\textnormal{Q}}}
\def\ermR{{\textnormal{R}}}
\def\ermS{{\textnormal{S}}}
\def\ermT{{\textnormal{T}}}
\def\ermU{{\textnormal{U}}}
\def\ermV{{\textnormal{V}}}
\def\ermW{{\textnormal{W}}}
\def\ermX{{\textnormal{X}}}
\def\ermY{{\textnormal{Y}}}
\def\ermZ{{\textnormal{Z}}}

% Vectors
\def\vzero{{\bm{0}}}
\def\vone{{\bm{1}}}
\def\vmu{{\bm{\mu}}}
\def\vtheta{{\bm{\theta}}}
\def\vphi{{\bm{\phi}}}
\def\va{{\bm{a}}}
\def\vb{{\bm{b}}}
\def\vc{{\bm{c}}}
\def\vd{{\bm{d}}}
\def\ve{{\bm{e}}}
\def\vf{{\bm{f}}}
\def\vg{{\bm{g}}}
\def\vh{{\bm{h}}}
\def\vi{{\bm{i}}}
\def\vj{{\bm{j}}}
\def\vk{{\bm{k}}}
\def\vl{{\bm{l}}}
\def\vm{{\bm{m}}}
\def\vn{{\bm{n}}}
\def\vo{{\bm{o}}}
\def\vp{{\bm{p}}}
\def\vq{{\bm{q}}}
\def\vr{{\bm{r}}}
\def\vs{{\bm{s}}}
\def\vt{{\bm{t}}}
\def\vu{{\bm{u}}}
\def\vv{{\bm{v}}}
\def\vw{{\bm{w}}}
\def\vx{{\bm{x}}}
\def\vy{{\bm{y}}}
\def\vz{{\bm{z}}}

% Elements of vectors
\def\evalpha{{\alpha}}
\def\evbeta{{\beta}}
\def\evepsilon{{\epsilon}}
\def\evlambda{{\lambda}}
\def\evomega{{\omega}}
\def\evmu{{\mu}}
\def\evpsi{{\psi}}
\def\evsigma{{\sigma}}
\def\evtheta{{\theta}}
\def\eva{{a}}
\def\evb{{b}}
\def\evc{{c}}
\def\evd{{d}}
\def\eve{{e}}
\def\evf{{f}}
\def\evg{{g}}
\def\evh{{h}}
\def\evi{{i}}
\def\evj{{j}}
\def\evk{{k}}
\def\evl{{l}}
\def\evm{{m}}
\def\evn{{n}}
\def\evo{{o}}
\def\evp{{p}}
\def\evq{{q}}
\def\evr{{r}}
\def\evs{{s}}
\def\evt{{t}}
\def\evu{{u}}
\def\evv{{v}}
\def\evw{{w}}
\def\evx{{x}}
\def\evy{{y}}
\def\evz{{z}}

% Matrix
\def\mA{{\bm{A}}}
\def\mB{{\bm{B}}}
\def\mC{{\bm{C}}}
\def\mD{{\bm{D}}}
\def\mE{{\bm{E}}}
\def\mF{{\bm{F}}}
\def\mG{{\bm{G}}}
\def\mH{{\bm{H}}}
\def\mI{{\bm{I}}}
\def\mJ{{\bm{J}}}
\def\mK{{\bm{K}}}
\def\mL{{\bm{L}}}
\def\mM{{\bm{M}}}
\def\mN{{\bm{N}}}
\def\mO{{\bm{O}}}
\def\mP{{\bm{P}}}
\def\mQ{{\bm{Q}}}
\def\mR{{\bm{R}}}
\def\mS{{\bm{S}}}
\def\mT{{\bm{T}}}
\def\mU{{\bm{U}}}
\def\mV{{\bm{V}}}
\def\mW{{\bm{W}}}
\def\mX{{\bm{X}}}
\def\mY{{\bm{Y}}}
\def\mZ{{\bm{Z}}}
\def\mBeta{{\bm{\beta}}}
\def\mPhi{{\bm{\Phi}}}
\def\mLambda{{\bm{\Lambda}}}
\def\mSigma{{\bm{\Sigma}}}

% Tensor
\DeclareMathAlphabet{\mathsfit}{\encodingdefault}{\sfdefault}{m}{sl}
\SetMathAlphabet{\mathsfit}{bold}{\encodingdefault}{\sfdefault}{bx}{n}
\newcommand{\tens}[1]{\bm{\mathsfit{#1}}}
\def\tA{{\tens{A}}}
\def\tB{{\tens{B}}}
\def\tC{{\tens{C}}}
\def\tD{{\tens{D}}}
\def\tE{{\tens{E}}}
\def\tF{{\tens{F}}}
\def\tG{{\tens{G}}}
\def\tH{{\tens{H}}}
\def\tI{{\tens{I}}}
\def\tJ{{\tens{J}}}
\def\tK{{\tens{K}}}
\def\tL{{\tens{L}}}
\def\tM{{\tens{M}}}
\def\tN{{\tens{N}}}
\def\tO{{\tens{O}}}
\def\tP{{\tens{P}}}
\def\tQ{{\tens{Q}}}
\def\tR{{\tens{R}}}
\def\tS{{\tens{S}}}
\def\tT{{\tens{T}}}
\def\tU{{\tens{U}}}
\def\tV{{\tens{V}}}
\def\tW{{\tens{W}}}
\def\tX{{\tens{X}}}
\def\tY{{\tens{Y}}}
\def\tZ{{\tens{Z}}}


% Graph
\def\gA{{\mathcal{A}}}
\def\gB{{\mathcal{B}}}
\def\gC{{\mathcal{C}}}
\def\gD{{\mathcal{D}}}
\def\gE{{\mathcal{E}}}
\def\gF{{\mathcal{F}}}
\def\gG{{\mathcal{G}}}
\def\gH{{\mathcal{H}}}
\def\gI{{\mathcal{I}}}
\def\gJ{{\mathcal{J}}}
\def\gK{{\mathcal{K}}}
\def\gL{{\mathcal{L}}}
\def\gM{{\mathcal{M}}}
\def\gN{{\mathcal{N}}}
\def\gO{{\mathcal{O}}}
\def\gP{{\mathcal{P}}}
\def\gQ{{\mathcal{Q}}}
\def\gR{{\mathcal{R}}}
\def\gS{{\mathcal{S}}}
\def\gT{{\mathcal{T}}}
\def\gU{{\mathcal{U}}}
\def\gV{{\mathcal{V}}}
\def\gW{{\mathcal{W}}}
\def\gX{{\mathcal{X}}}
\def\gY{{\mathcal{Y}}}
\def\gZ{{\mathcal{Z}}}

% Sets
\def\sA{{\mathbb{A}}}
\def\sB{{\mathbb{B}}}
\def\sC{{\mathbb{C}}}
\def\sD{{\mathbb{D}}}
% Don't use a set called E, because this would be the same as our symbol
% for expectation.
\def\sF{{\mathbb{F}}}
\def\sG{{\mathbb{G}}}
\def\sH{{\mathbb{H}}}
\def\sI{{\mathbb{I}}}
\def\sJ{{\mathbb{J}}}
\def\sK{{\mathbb{K}}}
\def\sL{{\mathbb{L}}}
\def\sM{{\mathbb{M}}}
\def\sN{{\mathbb{N}}}
\def\sO{{\mathbb{O}}}
\def\sP{{\mathbb{P}}}
\def\sQ{{\mathbb{Q}}}
\def\sR{{\mathbb{R}}}
\def\sS{{\mathbb{S}}}
\def\sT{{\mathbb{T}}}
\def\sU{{\mathbb{U}}}
\def\sV{{\mathbb{V}}}
\def\sW{{\mathbb{W}}}
\def\sX{{\mathbb{X}}}
\def\sY{{\mathbb{Y}}}
\def\sZ{{\mathbb{Z}}}

% Entries of a matrix
\def\emLambda{{\Lambda}}
\def\emA{{A}}
\def\emB{{B}}
\def\emC{{C}}
\def\emD{{D}}
\def\emE{{E}}
\def\emF{{F}}
\def\emG{{G}}
\def\emH{{H}}
\def\emI{{I}}
\def\emJ{{J}}
\def\emK{{K}}
\def\emL{{L}}
\def\emM{{M}}
\def\emN{{N}}
\def\emO{{O}}
\def\emP{{P}}
\def\emQ{{Q}}
\def\emR{{R}}
\def\emS{{S}}
\def\emT{{T}}
\def\emU{{U}}
\def\emV{{V}}
\def\emW{{W}}
\def\emX{{X}}
\def\emY{{Y}}
\def\emZ{{Z}}
\def\emSigma{{\Sigma}}

% entries of a tensor
% Same font as tensor, without \bm wrapper
\newcommand{\etens}[1]{\mathsfit{#1}}
\def\etLambda{{\etens{\Lambda}}}
\def\etA{{\etens{A}}}
\def\etB{{\etens{B}}}
\def\etC{{\etens{C}}}
\def\etD{{\etens{D}}}
\def\etE{{\etens{E}}}
\def\etF{{\etens{F}}}
\def\etG{{\etens{G}}}
\def\etH{{\etens{H}}}
\def\etI{{\etens{I}}}
\def\etJ{{\etens{J}}}
\def\etK{{\etens{K}}}
\def\etL{{\etens{L}}}
\def\etM{{\etens{M}}}
\def\etN{{\etens{N}}}
\def\etO{{\etens{O}}}
\def\etP{{\etens{P}}}
\def\etQ{{\etens{Q}}}
\def\etR{{\etens{R}}}
\def\etS{{\etens{S}}}
\def\etT{{\etens{T}}}
\def\etU{{\etens{U}}}
\def\etV{{\etens{V}}}
\def\etW{{\etens{W}}}
\def\etX{{\etens{X}}}
\def\etY{{\etens{Y}}}
\def\etZ{{\etens{Z}}}

% The true underlying data generating distribution
\newcommand{\pdata}{p_{\rm{data}}}
\newcommand{\ptarget}{p_{\rm{target}}}
\newcommand{\pprior}{p_{\rm{prior}}}
\newcommand{\pbase}{p_{\rm{base}}}
\newcommand{\pref}{p_{\rm{ref}}}

% The empirical distribution defined by the training set
\newcommand{\ptrain}{\hat{p}_{\rm{data}}}
\newcommand{\Ptrain}{\hat{P}_{\rm{data}}}
% The model distribution
\newcommand{\pmodel}{p_{\rm{model}}}
\newcommand{\Pmodel}{P_{\rm{model}}}
\newcommand{\ptildemodel}{\tilde{p}_{\rm{model}}}
% Stochastic autoencoder distributions
\newcommand{\pencode}{p_{\rm{encoder}}}
\newcommand{\pdecode}{p_{\rm{decoder}}}
\newcommand{\precons}{p_{\rm{reconstruct}}}

\newcommand{\laplace}{\mathrm{Laplace}} % Laplace distribution

\newcommand{\E}{\mathbb{E}}
\newcommand{\Ls}{\mathcal{L}}
\newcommand{\R}{\mathbb{R}}
\newcommand{\emp}{\tilde{p}}
\newcommand{\lr}{\alpha}
\newcommand{\reg}{\lambda}
\newcommand{\rect}{\mathrm{rectifier}}
\newcommand{\softmax}{\mathrm{softmax}}
\newcommand{\sigmoid}{\sigma}
\newcommand{\softplus}{\zeta}
\newcommand{\KL}{D_{\mathrm{KL}}}
\newcommand{\Var}{\mathrm{Var}}
\newcommand{\standarderror}{\mathrm{SE}}
\newcommand{\Cov}{\mathrm{Cov}}
% Wolfram Mathworld says $L^2$ is for function spaces and $\ell^2$ is for vectors
% But then they seem to use $L^2$ for vectors throughout the site, and so does
% wikipedia.
\newcommand{\normlzero}{L^0}
\newcommand{\normlone}{L^1}
\newcommand{\normltwo}{L^2}
\newcommand{\normlp}{L^p}
\newcommand{\normmax}{L^\infty}

\newcommand{\parents}{Pa} % See usage in notation.tex. Chosen to match Daphne's book.

\DeclareMathOperator*{\argmax}{arg\,max}
\DeclareMathOperator*{\argmin}{arg\,min}

\DeclareMathOperator{\sign}{sign}
\DeclareMathOperator{\Tr}{Tr}
\let\ab\allowbreak


\usepackage{hyperref}
\usepackage{url}
\usepackage{booktabs}
\usepackage{multirow}
\usepackage{graphicx}
\usepackage{makecell}
\usepackage{mathtools}
\usepackage{algorithmic}
\usepackage{algorithm}

\usepackage{colortbl}
\usepackage{xcolor}

% 定义颜色
\definecolor{c1}{rgb}{0.9, 0.9, 0.9}
\definecolor{citecolor}{HTML}{0071BC}
\definecolor{linkcolor}{HTML}{ED1C24}

\hypersetup{
colorlinks=true,
linkcolor=linkcolor,
citecolor=citecolor,
}

\newcommand{\ylq}[1]{{\color{blue}{[ylq: #1]}}}
\newcommand{\para}[1]{\vspace{.05in}\noindent\textbf{#1}}


% \title{Are state space models effective in interaction layers}
% \title{From Layers to States: Deciphering Deep Neural Network Layer Dynamics from a State Space Model Perspective}

\title{From Layers to States: A State Space Model Perspective to Deep Neural Network Layer Dynamics}
% \title{Rethinking Deep Neural Network Layers Through the Lens of State Space Models}
% \title{A State Space Model Perspective on Layer Interactions in Deep Neural Networks}

% Authors must not appear in the submitted version. They should be hidden
% as long as the \iclrfinalcopy macro remains commented out below.
% Non-anonymous submissions will be rejected without review.

% \author{
% % Qinshuo Liu \thanks{Equal Contribution} , Weiqin Zhao, Wei Huang, Yanwen Fang, Lequan Yu, Guodong Li \thanks{Corresponding Author} \\
% % Department of Statistics \& Actuarial Science, The University of Hong Kong\\
% % Pittsburgh, PA 15213, USA \\
% % \texttt{\{hippo,brain,jen\}@cs.cranberry-lemon.edu} \\
% % \centering
% Antiquus S.~Hippocampus, Natalia Cerebro \& Amelie P. Amygdale \thanks{ Use footnote for providing further information
% about author (webpage, alternative address)---\emph{not} for acknowledging
% funding agencies.  Funding acknowledgements go at the end of the paper.} \\
% Department of Computer Science\\
% Cranberry-Lemon University\\
% Pittsburgh, PA 15213, USA \\
% \texttt{\{hippo,brain,jen\}@cs.cranberry-lemon.edu} \\
% \And
% Ji Q. Ren \& Yevgeny LeNet \\
% Department of Computational Neuroscience \\
% University of the Witwatersrand \\
% Joburg, South Africa \\
% \texttt{\{robot,net\}@wits.ac.za} \\
% \AND
% Coauthor \\
% Affiliation \\
% Address \\
% \texttt{email}
% }

\newcommand*{\affaddr}[1]{#1} % No op here. Customize it for different styles.
\newcommand*{\affmark}[1][*]{\textsuperscript{#1}}
\newcommand*{\email}[1]{\texttt{#1}}

\usepackage{authblk}
\author{%
\textbf{Qinshuo Liu}$^{\ast}$\affmark[1], \textbf{Weiqin Zhao}$^{\ast}$\affmark[1], \textbf{Wei Huang}\affmark[2], \textbf{Yanwen Fang}\affmark[1], \textbf{Lequan Yu}$^{\dagger}$\affmark[1], \textbf{Guodong Li}$^{\dagger}$\affmark[1] 
\\
\affaddr{\affmark[1]School of Computing and Data Science, The University of Hong Kong} \\
\affaddr{\affmark[2]Department of Electrical and Electronic Engineering, The University of Hong Kong} \\
 \email{\{u3008680, wqzhao98, u3545683\}@connect.hku.hk} \\
 \email{aaron.weihuang@gmail.com} \\
 \email{\{lqyu, gdli\}@hku.hk}\\
% \affaddr{\LaTeX\ University}%
}

% The \author macro works with any number of authors. There are two commands
% used to separate the names and addresses of multiple authors: \And and \AND.
%
% Using \And between authors leaves it to \LaTeX{} to determine where to break
% the lines. Using \AND forces a linebreak at that point. So, if \LaTeX{}
% puts 3 of 4 authors names on the first line, and the last on the second
% line, try using \AND instead of \And before the third author name.

\newcommand{\fix}{\marginpar{FIX}}
\newcommand{\new}{\marginpar{NEW}}

\newcommand\nnfootnote[1]{%
  \begin{NoHyper}
  \renewcommand\thefootnote{}\footnote{#1}%
  \addtocounter{footnote}{-1}%
  \end{NoHyper}
}


%\iclrfinalcopy % Uncomment for camera-ready version, but NOT for submission.
\iclrfinalcopy
\begin{document}


\maketitle

\nnfootnote{$\ast$ Authors contributed equally. $\dagger$ Corresponding author.}
\vspace{-0.8cm}
\begin{abstract}
%
The depth of neural networks is a critical factor for their capability, with deeper models often demonstrating superior performance. 
%
Motivated by this, significant efforts have been made to enhance layer aggregation - reusing information from previous layers to better extract features at the current layer, to improve the representational power of deep neural networks. 
%
However, previous works have primarily addressed this problem from a discrete-state perspective which is not suitable as the number of network layers grows.
%
This paper novelly treats the outputs from layers as states of a continuous process and considers leveraging the state space model (SSM) to design the aggregation of layers in very deep neural networks.
%
Moreover, inspired by its advancements in modeling long sequences,  the Selective State Space Models (S6) is employed to design a new module called Selective State Space Model Layer Aggregation (S6LA). This module aims to combine traditional CNN or transformer architectures within a sequential framework, enhancing the representational capabilities of state-of-the-art vision networks.
%
Extensive experiments show that S6LA delivers substantial improvements in both image classification and detection tasks, highlighting the potential of integrating SSMs with contemporary deep learning techniques.
%
\end{abstract}

\section{Introduction}
\label{Introduction}

% 深度模型很好
In recent years, the depth of neural network architectures has emerged as a crucial factor influencing performance across various domains, including computer vision, natural language processing, and speech recognition.
The network models are capable of capturing increasingly complex features and representations from data as they become deeper, and various methods have emerged to utilize larger numbers of layers to improve performance. 
For instance, the VGG network \citep{simonyan2015deepconvolutionalnetworkslargescale} achieves higher classification accuracy by increasing the number of layers, although its foundation primarily relies on empirical results rather than systematical analysis. Other significant advancements, such as those demonstrated by CNNs \citep{he2016deep, ren2016faster, tan2020efficientnetrethinkingmodelscaling} and Transformers \citep{brown2020language, dosovitskiy2020image, touvron2021training, liu2021swin, wang2022pvt}, showcase how deeper architectures can enhance accuracy and generalization. 

% 怎么enhance深度模型performance: layer aggregation
Growing evidence indicates that strengthening layer interactions can encourage the information flow of a deep neural network.
%
For CNN-based networks, ResNet \citep{he2016deep} employed skip connections, allowing gradients to flow more easily by connecting non-adjacent layers. 
DenseNet \citep{huang2018denselyconnectedconvolutionalnetworks} extended this concept further by enabling each layer to access all preceding layers within a stage, fostering a rich exchange of information. 
Later, GLOM \citep{hinton2023represent} proposed an intensely interactive architecture that incorporates bottom-up, top-down, and same-level connections to effectively represent part-whole hierarchies. 
%
Recently, some studies have begun to frame layer interactions with recurrent models and attention mechanisms, such as RLA \citep{zhao2021recurrence} and MRLA \citep{fang2023cross}. 
%
All of the above models have been shown by empirical evidence to outperform those without interdependence across layers, and their achievements are obtained by treating the outputs of network layers as discrete states. 

However, the perspective of discrete treatment may not be suitable when a neural network is very deep; say, for example, ResNet-152 has 152 layers.
\cite{sander2022residualneuralnetworksdiscretize} proposed to treat the ResNet as a discretized version of neural ordinary differential equations, i.e. the whole ResNet is considered as a continuous process with the outputs from layers being the corresponding states; see also \cite{liu2020does}.
Moreover, \citet{queiruga2020continuous} argued that deep neural network models can learn to represent continuous dynamical systems by embedding them into continuous perspective. 
%
This motivates us to conduct layer aggregation among numerous layers of a neural network by alternatively assuming a continuous process to the outputs of layers.
%
% However, this may not be suitable mathematically when a neural network is very deep; say, for example, ResNet has 152 layers \citep{he2016deep}.\ylq{change this claim...}


Meanwhile, the State Space Model (SSM), a mathematical framework for continuously updating physical systems, enabled the modeling of dynamic processes and temporal dependencies in deep learning \citep{gu2023mamba,liu2024vmamba}. Then, Mamba, a selective state space model \citep{gu2023mamba}, proposed selective mechanism and hardware-aware algorithm, which was particularly adept at addressing long sequence modeling challenges. The selection mechanism allows the model to filter out irrelevant information and remember relevant information infinitely.

% For the development of state space model, \citet{gu2021efficiently} proposed structured state space sequence models (S4) which belong to a type of sequence models for deep learning related to RNNs. They are inspired by a particular continuous system and give the definition of latent state $h$ and structured state matrices $A$ \citep{gu2021efficiently}, with interval $\Delta$ and matrices $B$ being the coefficient for the model input signal $x$. \ylq{Can delete this sentence: Then Mamba model \citep{gu2023mamba} proposed selective mechanism and hardware-aware algorithm, which is particularly adept at addressing long sequence modeling challenges and hence is more suitable for our scenario. The details of these are provided in Section \ref{SSM_theory}.}\ylq{shorten}
%


The significance of layer aggregation in deep models and the popularity of SSMs lead us to propose a novel perspective: layer dynamics in very deep networks can be viewed as a continuous process with long sequence modeling task solvable by selective state space model (S6). By leveraging interactions between layers, outputs from different layers can be treated as sequential data input for an S6, allowing the model to encapsulate a richer representation of the information derived from the original data. 
%
By conceptualizing neural networks as state space models, we introduce a novel structure that integrates traditional models into sequential architectures. This approach opens new research avenues that connect traditional statistical methods with contemporary deep learning techniques. Our proposed Selective State Space Model Layer Aggregation (S6LA) effectively harnesses the benefits of layer interactions while incorporating statistical modeling into vision tasks, such as those performed by CNNs and Vision Transformers (ViTs). A schematic of our model is illustrated in Figure \ref{fig:overview}, with parameters $(\Delta,A,B)$ indicating the influence of {\color{blue}$\boldsymbol{X}$} on the implicit latent state $h$. Here {\color{blue}$\boldsymbol{X}^{t-1}$} represents the output at the $(t-1)$-th layer, which can be either a hidden layer in a deep CNN or an attention layer in a transformer model.

\begin{figure}[t]
\begin{center}
%\framebox[4.0in]{$\;$}
\includegraphics[width=0.9\linewidth]{overview.pdf}
\end{center}
\caption{Schematic diagram of a Network with Selective State Space Model Layer Aggregation.}
\label{fig:overview}
\end{figure}

The main contributions of our work are given below: (1) For a deep neural network, we treat the outputs from layers as states of a continuous process and attempt to leverage the SSM to design the aggregation of layers. To our best knowledge, this is the first time such a perspective has been presented. (2) This leads to a proposed lightweight module, the Selective State Space Model Layer Aggregation (S6LA) module, and it conceptualizes a neural network as a selective state space model (S6), hence solving the layer interactions by the long sequence modelling selective mechanism. (3) Compared with other SOTA convolutional and transformer-based layer aggregation models, S6LA demonstrates superior performance in classification, detection, and instance segmentation tasks.

% cv在最后一个contribution说就可以,只是一个验证的方式
% \begin{enumerate}
%     \item We propose that very deep neural networks can be formulated as state space models. To our knowledge, this is the first time such a perspective has been presented. Our approach redefines how to utilize state space models to integrate different layers, advancing traditional multi-layer models.
%     \item We introduce the State-Space-Interaction-Layer (SSIL), which conceptualizes neural networks as state space models (SSMs) to enhance cross-layer interactions. This framework integrates traditional CNNs, like ResNet, and transformer-based models, such as Deit and Swin Transformer, into sequential architectures, representing the first systematic investigation of such interactions within SSMs.
%     \item In comparison with convolutional and transformer-based vision models, our method demonstrates superior performance in classification, detection, and instance segmentation tasks.
% \end{enumerate}

\section{Related work}
\label{RelatedWork}


\paragraph{State Space Models.} In the realm of state space models, considerable efforts have been directed toward developing statistical theories. These models are characterized by equations that map a $1$-dimensional input signal $x(t)$ to an $N$-dimensional latent state $h(t)$, with the details provided in \Eqref{eq:SSM_con}. Inspired by continuous state space models in control systems and combined with HiPPO initialization \citep{gu2020hippo}, LSSL \citep{gu2021combiningrecurrentconvolutionalcontinuoustime} showcased the potential of state space models in addressing long-range dependency problems. 
However, due to limitations in memory and computation, their adoption was not widespread until the introduction of structured state space models (S4) \citep{gu2021efficiently}, which proposed normalizing parameters into a diagonal structure. S4 represents a class of recent sequence models in deep learning, broadly related to RNNs, CNNs and classical state space models. Subsequently, \cite{gu2023mamba} introduced the selective structured state space model (S6), which builds upon S4 and demonstrates superior performance compared to transformer backbones in various deep learning tasks, including natural language processing and time series forecasting. 
More recently, VMamba \citep{liu2024vmamba} was developed, leveraging the S6 model to replace the transformer mechanism and employing a scanning approach to convert images into patch sequences. Additionally, Graph-Mamba \citep{wang2024graph} represented a pioneering effort to enhance long-range context modeling in graph networks by integrating a Mamba block with an input-dependent node selection mechanism. These advancements indicate that state space models have also been successfully applied to complex tasks across various domains.

\paragraph{Layer Interaction.} The depth of neural network architectures has emerged as a crucial factor influencing performance. And Figure \ref{fig_corr_layer} of Appendix illustrates the enhanced performance of deeper neural networks. To effectively address the challenges posed by deeper models, increasing efforts have been directed toward improving layer interactions in both CNN and transformer-based architectures. Some studies \citep{hu2018squeeze,woo2018cbam,dosovitskiy2020image} lay much emphasis on amplifying interactions within a layer. DIANet \citep{huang2018denselyconnectedconvolutionalnetworks} employed a parameter-sharing LSTM throughout the network's depth to capture cross-channel relationships by utilizing information from preceding layers. In RLANet \citep{zhao2021recurrence}, a recurrent neural network module was used to iteratively aggregate information from different layers. For attention mechanism, \cite{fang2023cross} proposed to strengthen cross-layer interactions by retrieving query-related information from previous layers. Additionally, RealFormer \citep{he2020realformer} and EA-Transformer \citep{wang2021evolving} both incorporated attention scores from previous layers into the current layer, establishing connections through residual attention. However, these methods face significant memory challenges due to the need to retain features from all encoders, especially when dealing with high-dimensional data and they may lack robust theoretical supports.



%

\section{Preliminary and Problem Formulation}
\label{SSM_theory}

% This section gives the main formula for the selective state space model given by \cite{gu2021efficiently} and \cite{gu2023mamba}. 

\subsection{Revisiting State Space Models}
The state space model is defined below, and it maps a $1$-dimensional input signal $x(t)$, a continuous process, to an $N$-dimensional latent state $h(t)$, another continuous process:
\begin{equation}
\label{eq:SSM_con}
h^{\prime}(t) =A h(t)+B x(t),
\end{equation}
where $A \in \mathbb{R}^{N \times N}$ and $B \in \mathbb{R}^{N \times 1}$ are the structured state matrix and weight of influence from input to latent state, respectively. We then can obtain the discretization solution of the above equation:
\begin{equation}
        h^{t} = e^{\Delta A}h^{t-1} + \int_{t-1}^{t} e^{A(t-\tau)} Bx(\tau) d\tau.
\end{equation}
Together with the zero-order hold (ZOH) condition \citep{karafyllis2011nonlinear}, i.e. $x(\tau)$ is a constant at intervals $[t-1,t]$ for all integers $t$, we have
\begin{equation}
    h^{t} = e^{\Delta A}h^{t-1} + (\Delta A)^{-1}(\text{exp}(\Delta A)-I) \cdot \Delta B x^{t}.
\end{equation}
As a result, the continuous process at \Eqref{eq:SSM_con} can be replaced by the following discrete sequence:
\begin{equation}
\label{eq:SSM_dis}
%\begin{aligned}
h^t  =\overline{A} h^{t-1}+\overline{B} x^t \hspace{5mm}\text{with}\hspace{5mm}
\overline{A} = \text{exp}(\Delta A) \quad \text{and}   \quad \overline{B} = (\Delta A)^{-1}(\text{exp}(\Delta A)-I) \cdot \Delta B.
%\end{aligned}
\end{equation}
Following \cite{gu2023mamba}, we refine the approximation of $\overline{B}$ using the first-order Taylor series:
\begin{equation}
    \overline{B} = (\Delta A)^{-1}(\text{exp}(\Delta A)-I) \cdot \Delta B \approx (\Delta A)^{-1}(\Delta A) \cdot \Delta B = \Delta B.
\end{equation}
The formulas $\overline{A}=f_A(\Delta, A)$ and $\overline{B}=f_B(\Delta, A, B)$ are called the discretization rule, where $B = \text{Linear}_N(x)$ is a linear projection of input $x$ into $N$-dimension vector, and $\Delta = \text{Linear}_1(x)$; see \citet{gu2021efficiently,gu2023mamba} for details.

% Then \cite{} proposes the settings about transition matrix $A$ and in this paper, they give an exact solution about how to choose this in state space model. 


%This section gives the overview of recurrent mechanism and recalls the mathematical formulation about layer aggregations.
\subsection{CNN Layer Aggregation}
\label{5.1}
Consider a neural network, and let $\boldsymbol{X}^{t-1}$ be the output from its $t$th layer. We then can mathematically formulate the layer aggregation at the $t$th layer below,
\begin{equation}
\label{eq:CNN_agg}
    A^t =g^t(\boldsymbol{X}^{0},\boldsymbol{X}^{1},\cdots,\boldsymbol{X}^{t-2},\boldsymbol{X}^{t-1}), \quad
    \boldsymbol{X}^t = f^t(A^{t-1},\boldsymbol{X}^{t-1}), 
\end{equation}
where $g^t$ is used to summarize the first $t$ layers, $A^t$ is the aggregated information, and $f^t$ produces the new layer output from the last hidden layer and the given aggregation which contains the previous information. 
The Hierarchical Layer Aggregation proposed \citep{yu2018deep} can be shown to have such similar mechanism which satisfies  \Eqref{eq:CNN_agg}.

This formulation could be generalized to the special case of CNNs.
The traditional CNNs do not contain layer aggregation since the layer output only depends on the last layer output, which overlooks the connection between the several previous layers' influence.
DenseNet \citep{huang2018denselyconnectedconvolutionalnetworks} perhaps is the first one for the layer aggregation, and its output at $t$th layer can be formulated into
\begin{equation}
\label{eq:densenet}
\boldsymbol{X}^t=\text{Conv3}^t[\text{Conv1}^t(\text{Concat}(\boldsymbol{X}^0, \boldsymbol{X}^1, \ldots, \boldsymbol{X}^{t-1}))].
\end{equation}
Let $A^t = \text{Conv1}^t(\text{Concat}(\boldsymbol{X}^0, \boldsymbol{X}^1, \ldots, \boldsymbol{X}^{t-1}))$ and $\boldsymbol{X}^t = \text{Conv3}^t (A^t)$, and then DenseNet can be rewritten into our framework at \Eqref{eq:CNN_agg}.
RLA \citep{zhao2021recurrence} considers a more convenient additive form for the layer aggregation, and it has the form of $A^t = \sum _{i=0}^{t-1} \text{Conv1}^{t+1}_i(\boldsymbol{X}^i)$, where the kernel weights of $\text{Conv1}^t_i$ form a partition of the weights in $\text{Conv1}^t$.
% Consider that we can treat all the $\text{Conv1}^t_i$ as $\text{Conv1}^t$ which means the aggregation function does not depend on the current layer. 
As a result, a lightweight aggregation can be formed:
\begin{equation}
\label{eq:densenet_re}
    \boldsymbol{X}^t=\text{Conv3}^t [ A^{t-1} + \text{Conv1}^{t}_{t-1}(\boldsymbol{X}^{t-1}) ].
\end{equation}

Without loss of generality, ResNets \citep{he2016deep,he2016identity} can also be treated as a layer aggregation. Specifically, we can treat the update of $\boldsymbol{X}^t = \boldsymbol{X}^{t-1} + f^{t-1}(\boldsymbol{X}^{t-1})$ with applying the update recursively as $A^t = \sum_{i=0}^{t-1} f^i(\boldsymbol{X}^i) + \boldsymbol{X}^0$ and $\boldsymbol{X}^t = A^{t-1}+\boldsymbol{X}^{t-1}$.


\subsection{Attention Layers Aggregation}
In this section, we show how to generalize the layer aggregation within a transformer. Consider a simple attention layer with general input $\mathbf{X} \in \mathbb{R}^{L \times D}$ and output $\mathbf{O} \in \mathbb{R}^{L \times D}$.
Its query $\mathbf{Q}$, key $\mathbf{K}$ and value $\mathbf{V}$ are given by linear projections $\boldsymbol{W}_q \in \mathbb{R}^{D \times D}$, $\boldsymbol{W}_k \in \mathbb{R}^{D \times D}$ and $\boldsymbol{W}_v \in \mathbb{R}^{D \times D}$, i.e. $\mathbf{Q}^T = \boldsymbol{W}_q \mathbf{X}$, $\mathbf{K}^T = \boldsymbol{W}_k \mathbf{X}$ and $\mathbf{V}^T = \boldsymbol{W}_v \mathbf{X}$. As a result, the output $\mathbf{O}$ has the following mathematical formulation:
\begin{equation}
\mathbf{O} = \text{Self-Attention}(\boldsymbol{X}) = \text{softmax}(\frac{\mathbf{Q}\mathbf{K}^T}{\sqrt{D}})\mathbf{V}.
\end{equation}
Let $\boldsymbol{X}^t \in \mathbb{R}^{L \times D}$ with $1 \leq t \leq T$ be the output from $t$th layer, where $L$ is the number of tokens, $D$ represents the channel of each token, and $T$ is the number of attention layers. 
A vanilla transformer can then be formulated into:
\begin{equation}
    \label{eq:vanillatransformer}
        A^{t} =\boldsymbol{X}^{t-1}+\text{Self-Attention}(\boldsymbol{X}^{t-1}), \quad
        \boldsymbol{X}^{t} = A^{t} + \text{MLP}(\text{Norm}(A^{t})).
\end{equation}

Note that these simple self-attention layers can only capture the connection between the current layer output and the last output; they are supposed to perform better if the information from previous layers can be considered. To this end, we may leverage the idea given by CNN aggregation to concatenate the previous outputs. Specifically, the vanilla transformer at \Eqref{eq:vanillatransformer} has the form of:
\begin{equation}
    \label{eq:vanillatransformerre}
    \boldsymbol{X}^t=f^t(g^t(\boldsymbol{X}^0,\cdots,\boldsymbol{X}^{t-1})),
\end{equation}
where $g^t$ is the attention layer, and $f^t$ is the Add \& Norm layer for the $t$-th layer. 
Following \citet{zhao2021recurrence} at \Eqref{eq:densenet_re}, we may then use the recurrent mechanism to combine all the outputs given by attention layers, i.e. replacing $A^{t} = g^t(\boldsymbol{X}^0,\cdots,\boldsymbol{X}^{t-1})$ by $A^{t} = A^{t-1} + g^{t-1}(\boldsymbol{X}^{t-1})$.
Layer Aggregation via Selective State Space Model
\subsection{The Formula of S6LA}

Denote a sequence $\mathbf{X} = \{\boldsymbol{X}^1, \cdots , \boldsymbol{X}^T\}$, where $\boldsymbol{X}^t$ is the output from $t$th layer, say Convolutional layers/blocks or Attention layers, of a deep neural network, and $T$ is the number of layers.
In financial statistics, the price of an asset can be treated as a process with discrete time when it is sampled in a low frequency, say weekly data, while it will be treated as a process with continuous time the sampling frequency is high, say one-minute data; see \cite{yuan2023haritomodelshighdimensionalhar}.
Accordingly, we may treat $\mathbf{X}$ as a sequence with discrete time as the number of layers $T$ is small or even moderate, and all existing methods for layer aggregation in the literature follow this way.
%, while we may look it as discretized observations of a continuous process as in \cite{a} and \cite{pavan2007high}. 
%time series models with discrete states are employed for low-frequency data, while the diffusion model with continuous processes is a standard tool for high-frequency data.
%In previous literature \citep{pavan2007high}}, all existing methods for layer aggregation treat  $\boldsymbol{X}^t$'s to be discrete states, and hence they all correspond to time series tools in statistics.
However, for a very deep neural network, it is more like the scenario of high-frequency data, and hence a continuous process is more suitable for the sequence $\mathbf{X}$ \citep{sander2022residualneuralnetworksdiscretize,queiruga2020continuous}.
This section further conducts layer aggregation by considering state space models in Section 3.1; see Figure \ref{fig:overview} for the illustration.

Specifically, we utilize the Mamba model \citep{gu2023mamba} due to its effectiveness in processing long sequences.
This model is based on S6 models and can provide a better interpretation on how to leverage the previous information and then how to store it based on its importance. Moreover, it has been demonstrated to have a better performance than traditional transformers and RNNs.
Following its principle, we propose our selective state space model layer aggregation below:
\begin{equation}
    \label{S6_rec_ori}
        h^t = g^t(h^{t-1},\boldsymbol{X}^{t}),  \quad
        \boldsymbol{X}^t = f^t(h^{t-1},\boldsymbol{X}^{t-1}),
\end{equation}
where $h^t$ is a hidden state similar to $A^t$ in \Eqref{eq:CNN_agg}, and it represents the recurrently aggregated information up to $(t-1)$th layer.
We may consider an additive form, as in \Eqref{eq:densenet_re}, for $h^t$. 
Moreover, $g^t$ is the relation function between the current SSM hidden layer state and previous hidden layer state with input. As a result, similar to \Eqref{eq:SSM_dis}, the update of $h^t$ can be formulated as:
\begin{equation}
    \label{S6_rec_up}
        h^t = \overline{A}h^{t-1} + \overline{B}\boldsymbol{X}^t,  \quad
        \boldsymbol{X}^{t} = f^t(h^{t-1},\boldsymbol{X}^{t-1}).
\end{equation}
The choice of function $f^t$ is different for CNNs and Transformer-based models, and they are detailed in the following two subsections; see Figures \ref{fig:overview_CNN} and  \ref{fig:overview_Transformer} for their illustrations, respectively. 

%先总体介绍一个Network怎么对应到SSM的H,A,B,delta
%接下来,具体给你们展示两个例子:

\subsection{Application to Deep CNNs}
\label{CNN_application}

\begin{figure}[t]
\begin{center}
%\framebox[4.0in]{$\;$}
\includegraphics[width=0.9\linewidth]{overview_CNN.pdf}
\end{center}
Detailed operations in S6LA module with Convolutional Neural Network. The green arrow shows the hidden state connection, while the grey arrow indicates layers communications.
\label{fig:overview_CNN}
\end{figure}

For CNNs backbones, we adopt ResNet \citep{he2016deep} as the baseline architecture. We propose to concatenate the input at each layer, say $\boldsymbol{X}^t \in \mathbb{R}^{H \times W \times D}$, where $H$ and $W$ represent the height and width, and $D$ indicates the embedding dimension. For each CNN layer, the input to each block in ResNet—comprising a combination of 1D and 3D convolutional operations—is formed by concatenating $\boldsymbol{X}^{t}$ with the state $h^{t-1} \in \mathbb{R}^{H \times W \times N}$ from the previous layer, where $N$ is the dimension of latent states. This integration effectively incorporates SSM information into the convolutional layers, enhancing the network's capacity to learn complex representations. 
% Upon obtaining the output from the $t$-th CNN block, 

For the specific implementation of S6LA in CNNs, we initialize the SSM state $h^0$ using Kaiming normal initialization method \citep{he2015delving}. This initialization technique is crucial for ensuring effective gradient flow throughout the network, and we will further clarify this point in ablation studies. Next, we employ a selective mechanism to derive two components, the coefficient $B$ for input and the interval $\Delta$ as specified in \Eqref{eq:SSM_dis}. For transition matrix $A$, the initial setting is same as in Mamba models \citep{gu2023mamba}. Then with \Eqref{eq:SSM_dis}, we can get the next hidden layer $h^{t}$ based on the last $h^{t-1}$ and $\boldsymbol{X}^{t}$ for each layer in CNNs.

Utilizing \Eqref{eq:SSM_dis}, we compute the subsequent latent state $h^{t}$ based on the previous state $h^{t-1}$ and the input $\boldsymbol{X}^{t}$ for each layer within the CNN architecture. This methodological framework facilitates improved information flow and retention across layers, thereby enhancing the model's performance. Therefore, the specifics of leveraging our S6LA method with CNNs backbones can be outlined as follows:

% \begin{itemize}
%     \item Layer ${t-1}$: We begin by merging the input $\boldsymbol{X}^{t-1}$ and the hidden state $h^{t-1}$ through a simple concatenation along the feature dimension. This concatenated representation allows us to generate the output $\boldsymbol{O}^{t-1}$ with the CNNs backbone (such as ResNet).
%     \item Next Step Computation: The output component $\boldsymbol{O}^{t-1}$ from previous step, contributes to the next input $\boldsymbol{X}^{t}$ and the hidden state $h^t$. Here, the dimensions of $\boldsymbol{O}^{t-1}$ are $H \times W \times D$ where $H$ and $W$ correspond to the height and width of the input images, respectively, and $D$ represents the feature dimension.
%     \item State Update: For the input state update, we define $\boldsymbol{X}^t$ as the sum of $\boldsymbol{X}^{t-1}$ and $\boldsymbol{O}^{t-1}$. For the hidden state, $h^t$ is derived as a function of these two components, following the formulation provided in Equations \ref{eq:SSM_dis}. The equations are as follows with two trainable parameters $W_{\Delta}$ and $W_B$ (for the $t-1$ layer):
% \begin{equation}
%     \label{CNN_s6_1}
%     h^t = e^{(\Delta A)} h^{t-1} + \Delta B \boldsymbol{O}^{t-1},
% \end{equation}
% where $\Delta = W_{\Delta} (Conv(\boldsymbol{O}^{t-1})),
%     B = W_B (Conv(\boldsymbol{O}^{t-1}))$.
% \end{itemize}

\para{Input Treatment:} We begin by merging the input $\boldsymbol{X}^{t}$ and the hidden state $h^{t-1}$ through a simple concatenation along the feature dimension. This concatenated representation allows us to generate the output $\boldsymbol{O}^{t}$ with the CNNs backbone (such as ResNet).

\para{Latent State Update:} For the input state update, we define $\boldsymbol{X}^{t+1}$ as the sum of $\boldsymbol{X}^{t}$ and $\boldsymbol{O}^{t}$. For the hidden state, $h^t$ is derived as a function of these two components, following the formulation provided in \Eqref{eq:SSM_dis}. The equations are as follows with two trainable parameters $W_{\Delta}$ and $W_B$ (for the $t-1$ layer):
\begin{equation}
    \label{CNN_s6_1}
    h^t = e^{(\Delta A)} h^{t-1} + \Delta B \boldsymbol{O}^{t},
\end{equation}
where $\Delta = W_{\Delta} (\text{Conv}(\boldsymbol{O}^{t})),
    B = W_B (\text{Conv}(\boldsymbol{O}^{t}))$.

\para{Output Computation:} The output component $\boldsymbol{O}^{t}$ from the input treatment step, contributes to the next input $\boldsymbol{X}^{t+1}$ and the computation is: $\boldsymbol{X}^{t+1} = \boldsymbol{O}^{t} + \boldsymbol{X}^{t}$.

\begin{figure}[t]
\begin{center}
%\framebox[4.0in]{$\;$}
\includegraphics[width=0.9\linewidth]{overview_Transformer.pdf}
\end{center}
Diagram of the S6LA architecture with Transformer. The green arrow shows the hidden state connection, while the grey arrow indicates communication between layers. The input consists of image patch tokens (\( \boldsymbol{X}^{t}_p \)) and a class token (\( \boldsymbol{X}^{t}_c \)), processed through positional embedding and attention. The class token is cloned as $\boldsymbol{X}_c^{t'}$, and parameters \( W_\Delta \) and \( W_B \) update the hidden state (\( h^t \)). Updated patch tokens are combined with the class token to form the next input (\( \boldsymbol{X}^{t+1} \)).
\label{fig:overview_Transformer}
\end{figure}

\subsection{Application to Deep ViTs}

In our exploration of S6LA implementation in deep ViT backbones, we draw parallels between the integration of the state space model (SSM) state and the mechanisms used in convolutional neural networks (CNNs). However, the methods of combining inputs within transformer blocks differ significantly from those in CNNs. Like the treatment of attention mechanism, we utilize multiplication combination instead of simply concatenating to deal with the input $\boldsymbol{X}^{t}$ and $h^{t-1}$ in transformer-based models. This approach enhances the interaction between input features and SSM, enabling richer feature representation. Then the next paragraghs gives the specifics of leveraging our S6LA method with ViTs backbones as follows:

\paragraph{Input Treatment:} We begin by combining the class token and input, alongside the application of positional embeddings. 
% The input dimension for the transformer network is defined as: $\boldsymbol{X}^{t-1}_{input} \in \mathbb{R}^{1 \times D}$ where $D = (N+1) \times C$, $N$ is the number of patches and $C$ is the embedding dimension. 
Then following the attention mechanism, $\boldsymbol{X}^{t}_{\text{input}} \in \mathbb{R}^{(L+1)\times D}$ appeared where $L$ is the number of patches and $D$ is the embedding dimension, and it is split into two components next: image patch tokens $\boldsymbol{X}^{t}_p \in \mathbb{R}^{L\times D}$ and a class token $\boldsymbol{X}^{t}_c \in \mathbb{R}^{1 \times D}$.
\begin{small}
\begin{equation}
\boldsymbol{X}^{t}_{\text{input}} = \text{Add} \& \text{Norm}(\text{MLP}(\text{Add} \& \text{Norm}(\text{Attn}(\boldsymbol{X}^{t}))));   \quad \boldsymbol{X}^{t}_p, \boldsymbol{X}^{t}_c = \text{Split}(\boldsymbol{X}^{t}_{\text{input}}).
\end{equation}
\end{small}
The class token plays a crucial role in assessing the correlation between $\boldsymbol{X}^{t}$ and $h^{t-1}$. Our model setting can effectively bridge the features extracted from the patches with the SSM state by facilitating a better integration into the hidden state since it could be considered as a summary feature of the image in sequential layers. 

\paragraph{Latent State Update:} Given the split class token in last step, the hidden state is updated similar to application in CNNs:
\begin{equation}
    h^t = e^{(\Delta A)} h^{t-1} + \Delta B \boldsymbol{X}^{t}_c,
\end{equation}
where $\Delta$ and $B$ are calculated from class token with selective mechanism:
\begin{equation}
\Delta = W_{\Delta} (\boldsymbol{X}^{t}_c), \quad
    B = W_B (\boldsymbol{X}^{t}_c).
\end{equation}

\paragraph{Output Computation:} At the same time, the new patch tokens $\widehat{\boldsymbol{X}}_p^{t}$ are computed as the sum of the previous patch tokens and the product of the previous patch tokens with $h^t$:
\begin{equation}
    \widehat{\boldsymbol{X}}_p^{t} = \boldsymbol{X}^{t}_p + W\boldsymbol{X}^{t}_p h^t.
\end{equation}
Then the next input, $\boldsymbol{X}^{t+1}$, is derived from the concatenation of the updated patch and class tokens:
\begin{equation}
   \boldsymbol{X}^{t+1} = \text{Concat}(\widehat{\boldsymbol{X}}_p^{t},\boldsymbol{X}^{t}_c). 
\end{equation}





\section{Experiments}

\subsection{Datasets}

\textbf{MSMARCO}.
We utilized the MS MARCO Passage Ranking dataset as the data source to evaluate the ability of our method to improve document rankings under more challenging topic-query tasks. Specifically, we assessed whether our method could significantly enhance the ranking of documents by the retrieval model within a RAG system.

To construct topic-lists for evaluation, we applied a K-means clustering algorithm to group similar queries, forming topics that each contained a series of related queries. To further evaluate the performance of our method under extreme topic-query scenarios, we applied an intra-topic similarity filtering process. Only topics with queries exhibiting high semantic diversity and containing a sufficient number of queries were retained.

This process resulted in 29 topics, with each topic containing an average of 22.28 queries. The average similarity score within each topic was approximately 0.5, indicating sufficient diversity among queries to ensure a rigorous evaluation. This curated dataset enabled us to test the robustness of our method in handling highly diverse and challenging topic-query tasks within a RAG system.

\textbf{PROCON}.
To conduct our experiments, we utilized controversial topic data scraped from the PROCON.ORG website. This dataset includes over 80 topics covering various domains, such as society, health, government, and education. Each topic is discussed from two stance labels \{\textit{PRO (support), CON (oppose)}\}, with passages arguing from these perspectives.

To simulate real-world user interactions with a RAG system, we instructed a large language model (GPT-4o) to act as a user and generate 40 potential sub-queries for each topic. These sub-queries were designed to reflect the diverse questions and concerns users might raise when exploring a specific controversial topic. 

After generating the sub-queries, we applied a similarity filtering process to ensure diversity by retaining only those with a similarity score below approximately 0.85. The filtering step effectively removed redundant queries while preserving a wide range of perspectives. As a result, the final set of topic-queries achieved an average similarity score of approximately 0.7, ensuring that the queries were sufficiently diverse yet semantically relevant. This process provided a robust and balanced sub-queries set for evaluation.


\subsection{Experiment Details}
The specific setting details for the Topic-queries RAG manipulation experiment are as follows:

(1) Black-box RAG. We represent the black-box RAG process as \( \text{RAG}_{\text{black}} \). The RAG framework is Conversational RAG from LangChain. The LLMs adopted in RAG are the open-source models Meta-Llama-3.1-8B-Instruct (Llama3.1), Qwen-2.5-7B-Instruct(Qwen2.5). The system prompt and additional detailed descriptions are provided in Appendix~\ref{exp-detail}.

(2) Retrieval Model Specification. We benchmark three dominant dense retrievers—Contriever \cite{gao2021unsupervised}, DPR \cite{karpukhin-etal-2020-dense}, and ANCE \cite{xiong2020approximate}.By convention, we use dot product between the embedding vectors of questions(queries) and candidate documents as their similarity score \(R\) in the ranking. 


\label{opinion-classfication}
(3) Opinion classification. We use Qwen2.5-Instruct-72B as the opinion classifier. Qwen2.5-Instruct-72B, due to its large parameter size, is capable of accurately identifying and classifying opinions within text.

(4) Experimental parameters. In knowledge-guided attack process, we set the maximum editing distance $\epsilon$ to 0.2, the semantic similarity threshold $\lambda$ to 0.85, and the number of iterations $N$ to 5. For adversarial trigger generation, we used a beam size of 3, top-$k$ of 10, a batch size of 32, a temperature of 1.0, a learning rate of 0.005, and a sequence length of 10. In RAG\textsubscript{black}, $k$ (the number of retrieved documents) is set to 3, with the LLM temperature also fixed at 1.0 to mirror real-world conditions.

(5) Poisoned Target. For the PROCON dataset, to investigate the manipulation performance under more challenging conditions, we performed relevance ranking for the documents with respect to each topic-query set $Q$ and the target stance $S_t$ . From the ranked list, we selected the last five documents (i.e., those with the lowest relevance) as the target poisoned documents.
For the MS MARCO dataset, we utilized the top-1000 relevance-ranked passage list for each topic-query set. From this list, we selected the passage with the lowest average rank as the target passage. This approach ensures that the evaluation focuses on passages that are least relevant to the target queries, thus providing a more rigorous benchmark for the proposed method.

(6) Experimental environment. All our experiments are conducted in Python 3.8 environment and run on a NVIDIA DGX H100 GPU. 

\subsection{Research Questions}

We propose four research questions to evaluate the effectiveness of our method in the topic-queries task, focusing on black-box NRM attacks and opinion manipulation to RAGs.

\textbf{RQ1}: Can Topic-FlipRAG significantly enhance the rankings of target documents in the RAG retriever for topic-queries?

\textbf{RQ2}: To what extent does Topic-FlipRAG affect the answers generated by the target RAG systems?

\textbf{RQ3}: Does topic-oriented opinion manipulation significantly impact users' perceptions of controversial topics?

\textbf{RQ4}: How robust does Topic-FlipRAG against exisiting mitigation mechanism?

\subsection{Baseline Settings}
To assess the effectiveness of our proposed method, we compare it against adversarial attack baselines designed for black-box, topic-oriented RAG scenarios, ensuring minimal modifications to the original documents. We exclude BadRAG\cite{xue2024badrag}, a backdoor RAG attack limited to white-box scenarios, and topic-IR-attack\cite{liu2023topic}, as its incomplete implementation prevents reliable replication.
For the selected baseline methods, we adapted them to meet the requirements of our task while preserving their core components. A brief overview of the baseline methods is provided below, with detailed descriptions available in Appendix~\ref{baselines-details}.

\textbf{PoisonedRAG.}
Zou et al.\cite{zou2024poisonedrag} propose an approach adaptable to both black-box and white-box settings. For our task, we employ its black-box strategy by inserting a randomly chosen query from the topic-queries set \( Q \) at the beginning of each document.

\textbf{PAT.}
This gradient-based adversarial retrieval attack uses a pairwise loss function to generate triggers that meet fluency and coherence constraints. We adapt PAT to produce triggers \( T_{\text{pat}} \) for target documents within the topic-queries set, evaluating their effectiveness under black-box conditions.


\textbf{Collision.}
This method generates adversarial paragraphs (collisions) via gradient-based optimization to produce content semantically aligned with the target query. In a topic-queries context, we create collisions for the entire topic-queries set and examine their transfer performance on black-box RAG retrievers.

These baseline methods provide benchmarks for comparing the efficacy of our approach in a fully black-box, topic-oriented RAG attack scenario.

\subsection{Evaluation Metrics}

For \textbf{RQ1}, we focus on ranking manipulation. We measure the average proportion of target opinions in top-3 rankings before and after manipulation (\(\text{Top3}_{\text{ori}}, \text{Top3}_{\text{att}}\)) and define top3-v as their difference. We also employ the Ranking Attack Success Rate (RASR), reflecting how often target documents are successfully boosted, and Boost Rank (BRank), denoting the average rank improvement for all target documents. Lastly, we report the proportion of target documents in the Top-50 and Top-500 positions to indicate how effectively they are pushed toward higher rankings.

\textbf{top3-v.} Computed by subtracting \(\text{Top3}_{\text{ori}}\) from \(\text{Top3}_{\text{att}}\), top3-v ranges from -1 to 1. A positive value signifies a successful increase of the target opinion in top-3 results, while a negative value indicates a detrimental effect.

\textbf{Ranking Attack Success Rate (RASR).} RASR captures how frequently target documents are successfully boosted in each query’s ranking. Higher values indicate greater attack effectiveness.

\textbf{Boost Rank (BRank).} BRank is the average rank improvement for all target documents under each query. A target document contributes negatively if its rank is unintentionally lowered.

\textbf{Top-50, Top-500.} These metrics represent the percentage of target documents that move into specific ranking thresholds in the MS MARCO Dataset after manipulation. Higher percentages imply more effective promotion of target documents. 


For \textbf{RQ2}, we employ Average Stance Variation (ASV) to assess how significantly our opinion manipulation influences the LLM’s responses in a black-box RAG. To address the natural variability of controversial topics and the inherent instability of large language models, we also propose Real Adjusted ASV (\(\Delta\)-ASV).

\textbf{Average Stance Variation (ASV).}
ASV is defined as the absolute difference between the manipulated opinion score and the original opinion score assigned to an LLM response (0 = opposing, 1 = neutral, 2 = supporting). A higher ASV signifies a more pronounced shift in polarity and hence greater manipulation effectiveness.

\textbf{Real Adjusted ASV ($\Delta$-ASV)}. To account for the inherent variability of controversial topics and the instability of large language models, we measure the baseline ASV in a clean state, denoted as ASV\textsubscript{clean} (calculated without adversarial manipulation). $\Delta$-ASV is then obtained by subtracting ASV\textsubscript{clean} from the manipulated ASV, i.e., \( \text{$\Delta$-ASV} = \text{ASV} - \text{ASV\textsubscript{clean}} \). This adjustment ensures that $\Delta$-ASV reflects the true impact of adversarial manipulation by eliminating the influence of natural stance variation. It reflects the extent to which the polarity of the RAG-system outputs is affected by the manipulation.  A positive $\Delta$-ASV indicates a significant shift in the opinion polarity due to manipulation, with larger values representing greater manipulation effectiveness.

\section{Conclusion }
This paper introduces the Latent Radiance Field (LRF), which to our knowledge, is the first work to construct radiance field representations directly in the 2D latent space for 3D reconstruction. We present a novel framework for incorporating 3D awareness into 2D representation learning, featuring a correspondence-aware autoencoding method and a VAE-Radiance Field (VAE-RF) alignment strategy to bridge the domain gap between the 2D latent space and the natural 3D space, thereby significantly enhancing the visual quality of our LRF.
Future work will focus on incorporating our method with more compact 3D representations, efficient NVS, few-shot NVS in latent space, as well as exploring its application with potential 3D latent diffusion models.


\newpage

\bibliography{Ref}
\bibliographystyle{iclr2025_conference}

\newpage


\section{Additional Related Works}
\label{sec:app-add-rel-works}
\subsection{Training Data Selection}

\begin{figure*}[!ht]
    \centering
    \includegraphics[width=\textwidth]{figs/per-token-loss-diff.pdf}
    \caption{Histograms of MIA signal of tokens. Each figure depicts a sample. Blue means the member samples while orange represents the non-member samples. We limited the y-axis range to -3 to 3 for better visibility, so it can result in missing several non-significant outliers.}
    \label{fig:add-per-token-loss}
\end{figure*}

Training data selection are methods that filter high-quality data from noisy big data \textit{before training} to improve the model utility and training efficiency. There are several works leveraging reference models~\cite{Coleman2020Selection, xie2023doremi}, prompting LLMs~\cite{li-etal-2024-one}, deduplication~\cite{lee2022deduplicating, kandpal2022deduplicating}, and distribution matching~\cite{kang2024get}. However, we do not aim to cover this data selection approach, as it is orthogonal and can be combined with ours.


\subsection{Selective Training}
Selective training refers to methods that \textit{dynamically choose} specific samples or tokens \textit{during training}. Selective training methods are the most relevant to our work. Generally, sample selection has been widely studied in the context of traditional classification models via online batch selection~\cite{loshchilov2016o, Angelosonl, pmlr-v108-kawaguchi20a}. These batch selection methods replace the naive random mini-batch sampling with mechanisms that consider the importance of each sample mainly via their loss values. ~\citet{2022PrioritizedTraining} indeed choose highly important samples from regular random batches by utilizing a reference model. However, due to the sequential nature of LLMs, which makes the training significantly different from the traditional classification ML, sample-level selection is not effective for language modeling~\cite{kaddour2023no}. \citet{lin2024not} extend the reference model-based framework to select meaningful tokens within batches. All of the previous methods for selective training aim to improve the training performance and compute efficiency. Our work is the first looking at this aspect for defending against MIAs.

\section{Token-level membership inference risk analysis}
Figures~\ref{fig:add-per-token-loss} and~\ref{fig:add-per-token-dynamics} present the analysis for additional samples. Generally, the trends are consistent with the one presented in Section~\ref{sec:analysis}.

\begin{figure*}[!ht]
    \centering
    \includegraphics[width=0.28\textwidth]{figs/mia-ranking_1.png}
    \includegraphics[width=0.28\textwidth]{figs/mia-ranking_2.png}
    \includegraphics[width=0.3\textwidth]{figs/mia-ranking_3.png}    
    \caption{MIA signal ranking of tokens during training. Each figure illustrates a sample.}
    \label{fig:add-per-token-dynamics}
\end{figure*}

\label{sec:app-analysis}

\section{Experiment settings}
\subsection{Implementation details}
\label{sec:app-implementation}
$\bullet$ \textbf{FT}. We implement the conventional fine tuning using Huggingface Trainer. We manually tune the learning rate to make sure no significant underfitting or overfitting. The batch size is selected appropriately to fit the physical memory and comparable with the other methods'.

\noindent $\bullet$ \textbf{Goldfish}. Goldfish is also implemented with Huggingface Trainer, where we custom the \texttt{compute\_loss} function. We implement the deterministic masking version rather than the random masking to make sure the same tokens are masked over epochs, potentially leading to better preventing memorization. The learning rate is also manually tuned, we noticed that the optimal Goldfish learning rate is usually slightly greater than FT's. This can be the gradients of two methods are almost similar, Goldfish just removes some tokens' contribution to the loss calculation. The batch size of FT can set as the same as FT, as Goldfish does not have significant overhead on memory.

\noindent $\bullet$ \textbf{DPSGD}. DPSGD is implemented by FastDP~\cite{bu2023zero}. We implement DPSGD with fastDP~\cite{bu2023zero} which offers state-of-the-art efficiency in terms of memory and training speed. We also use automatic clipping~\cite{bu2023automatic} and a mixed optimization strategy~\cite{mixclipping} between per-layer and per-sample clipping for robust performance and stability.

\noindent $\bullet$ \textbf{\methodname}. We implement \methodname using Huggingface Trainer, same as FT and Goldfish. The learning is reused from FT. The batch size of \methodname is usually smaller than FT and Goldfish when the model becomes large such as Pythia and Llama 2 due to the reference model, which consumes some memory.

For a fair comparison, we aim to implement the same batch size for all methods if feasible. In case of OOM (out of memory), we perform gradient accumulation, so all the methods can have comparable batch sizes. We provide the hyper-parameters of method for GPT2 in Table~\ref{tab:hyperparameter}. For Pythia and Llama 2, the learning rate, batch size, and number of epochs are tuned again, but the hyper-parameters regarding the privacy mechanisms remain the same. To make sure there is no naive overfitting, we evaluate the methods by selecting the best models on a validation set. Moreover, the testing and attack datasets remains identical for evaluating all methods. Additionally, we balance the number of member and non-member samples for MIA evaluation. It is worth noting that for the ablation study and analysis, if not state, the default model architecture and dataset are GPT2 and CC-news.

\begin{table*}[!ht]
    \centering
    \begin{tabular}{c|clc}
    \textbf{LLM} & \textbf{Method} & \textbf{Hyper-parameter} & \textbf{Value}  \\ \hline
     \multirow{22}{*}{\textbf{GPT2}}  &  \multirow{4}{*}{FT} &  Learning rate & 1.75e-5 \\ 
     & & Batch size & 96 \\
     & & Gradient accumulation steps & 1 \\
     & & Number of epochs & 20 \\ \cline{2-4}
       &  \multirow{5}{*}{Goldfish} &  Learning rate & 2e-5 \\ 
     & & Batch size & 96 \\
     & & Grad accumulation steps & 1 \\
     & & Number of epochs & 20 \\
     & & Masking Rate & 25\% \\ \cline{2-4}
           &  \multirow{6}{*}{DPSGD} &  Learning rate & 1.5e-3 \\ 
     & & Batch size & 96 \\
     & & Grad accumulation steps & 1 \\
     & & Number of epochs & 10 \\
     & & Clipping & automatic clipping \\ 
     & & Privacy budget & (8, 1e-5)-DP \\ \cline{2-4}
           &  \multirow{6}{*}{DuoLearn} &  Learning rate & 1.75e-3 \\ 
     & & Batch size & 96 \\
     & & Grad accumulation steps & 1 \\
     & & Number of epochs & 20 \\
     & & $K_h$ & 60\% \\ 
     & & $K_m$ & 20\% \\
     & & $\tau$ & 0 \\
     & & $\alpha$ & 0.8 \\ \hline
    \end{tabular}
    \caption{Hyper-parameters of the methods for GPT2.}
    \label{tab:hyperparameter}
\end{table*}


\section{Additional Results}
\label{sec:app-add-res}

\begin{figure}[!ht]
    \centering
    \includegraphics[width=0.8\linewidth]{figs/add_loss_vs_steps_ft_duolearn.pdf}
    \caption{Breakdown to the cross entropy loss values of FT on the testing set and \methodname on the training set during training.}
    \label{fig:add-overlap-breakdown}
\end{figure}

\subsection{Overall Evaluation}
% \begin{table*}[htp]
%     \centering
%     \begin{tabular}{cl|ccccc|ccccc}
%      \multirow{3}{*}{\textbf{LLM}}  & \multirow{3}{*}{\textbf{Method}} &  \multicolumn{5}{c|}{\textbf{CCNews}} & \multicolumn{5}{c}{\textbf{Wikipedia}} \\ \cmidrule(lr){3-7}  \cmidrule(lr){8-12}
%       &  & PPL & Loss & Ref & min-k & \multicolumn{1}{c|}{zlib} & PPL & Loss & Ref & min-k & zlib \\ \midrule
%       \multirow{4}{*}{GPT2} & \textit{Base} & \textit{29.442} & \textit{0.018} & \textit{0.002} & \textit{0.022} & \textit{0.006} & \textit{34.429} & \textit{0.002} & \textit{0.014} & \textit{0.010} & \textit{0.002} \\ 
%       \multirow{4}{*}{124M} & FT & \textbf{21.861} & 0.030 & 0.026 & 0.016 & 0.016 & \textbf{12.729} & 0.018 & 0.574 & 0.016 & 0.014 \\
%       & Goldfish & 21.902 & 0.030 & 0.024 & 0.028 & 0.016 & 12.853 & 0.018 & 0.632 & 0.016 & 0.010 \\
%       & DPSGD & 26.022 & \textbf{0.018} & \textbf{0.004} & \textbf{0.018} & 0.008 & 18.523 & \textbf{0.004} & 0.036 & 0.018 & 0.006 \\
%       & \methodname & 23.733 & 0.030 & 0.022 & 0.026 & \textbf{0.006} & 13.628 & 0.014 & \textbf{0.010} & \textbf{0.014} & \textbf{0.004} \\ \midrule
      
%       \multirow{4}{*}{Pythia} & \textit{Base} & \textit{13.973} & \textit{0.002} & \textit{0.008} & \textit{0.020} & \textit{0.014} & \textit{10.287} & \textit{0.002} & \textit{0.014} & \textit{0.006} & \textit{0.008} \\ 
%       \multirow{4}{*}{1.4B} & FT & 11.922 & 0.014 & 0.008 & 0.022 & 0.020 & \textbf{6.439} & 0.020 & 0.440 & 0.010 & 0.020 \\
%       & Goldfish & \textbf{11.903} & 0.014 & 0.008 & 0.024 & 0.018 & 6.465 & 0.016 & 0.412 & 0.010 & 0.020 \\
%       & DPSGD & 13.286 & \textbf{0.002} & \textbf{0.004} & \textbf{0.018} & \textbf{0.014} & 7.751 & \textbf{0.004} & \textbf{0.016} & {0.010} & \textbf{0.004} \\
%       & \methodname & 12.670 & 0.004 & 0.020 & \textbf{0.018} & 0.016 & 6.553 & 0.008 & 0.030 & \textbf{0.006} & 0.006 \\ \midrule
      
%       \multirow{4}{*}{Llama-2} & \textit{Base} & \textit{9.364} & \textit{0.006} & \textit{0.006} & \textit{0.024} & \textit{0.006} & \textit{7.014} & \textit{0.006} & \textit{0.016} & \textit{0.016} & \textit{0.010} \\ 
%       \multirow{4}{*}{7B} & FT & \textbf{6.261} & 0.002 & 0.018 & 0.002 & 0.002 & \textbf{3.830} & 0.028 & 0.170 & 0.030 & 0.028 \\
%       & Goldfish & 6.280 & 0.002 & 0.018 & 0.002 & 0.006 & 3.839 & 0.028 & 0.198 & 0.028 & 0.028 \\
%       & DPSGD & 6.777 & 0.008 & 0.026 & 0.016 & 0.010 & 4.490 & \textbf{0.006} & 0.014 & \textbf{0.020} & \textbf{0.010} \\
%       & \methodname & 6.395 & \textbf{0.002} & \textbf{0.020} & \textbf{0.004} & \textbf{0.002} & 4.006 & 0.010 & \textbf{0.002} & 0.028 & 0.012 \\ 
%     \end{tabular}
%     \caption{TPR at FPR of 1\% \textcolor{red}{TODO: check consistency with the main table of MIA AUC scores}}
%     \label{tab:tpr}
% \end{table*}


\begin{table*}[!ht]
  \centering
  \resizebox{\textwidth}{!}{\begin{tabular}{cl|ccccc|ccccc}
   \multirow{3}{*}{\textbf{LLM}}  & \multirow{3}{*}{\textbf{Method}} &  \multicolumn{5}{c|}{\textbf{Wikipedia}} & \multicolumn{5}{c}{\textbf{CC-news}} \\ \cmidrule(lr){3-7}  \cmidrule(lr){8-12}
    &  & PPL & Loss & Ref & min-k & \multicolumn{1}{c|}{zlib} & PPL & Loss & Ref & min-k & zlib \\ \midrule
    \multirow{4}{*}{GPT2} & \textit{Base} & \textit{34.429} & \textit{0.002} & \textit{0.014} & \textit{0.010} & \textit{0.002} & \textit{29.442} & \textit{0.018} & \textit{0.002} & \textit{0.022} & \textit{0.006} \\ 
    \multirow{4}{*}{124M} & FT & \textbf{12.729} & 0.018 & 0.574 & 0.016 & 0.014 & \textbf{21.861} & 0.030 & 0.026 & 0.016 & 0.016 \\
    & Goldfish & 12.853 & 0.018 & 0.632 & 0.016 & 0.010 & 21.902 & 0.030 & 0.024 & 0.028 & 0.016 \\
    & DPSGD & 18.523 & \textbf{0.004} & 0.036 & 0.018 & 0.006 & 26.022 & \textbf{0.018} & \textbf{0.004} & \textbf{0.018} & 0.008 \\
    & \methodname & 13.628 & 0.014 & \textbf{0.010} & \textbf{0.014} & \textbf{0.004} & 23.733 & 0.030 & 0.022 & 0.026 & \textbf{0.006} \\ \midrule
    
    \multirow{4}{*}{Pythia} & \textit{Base} & \textit{10.287} & \textit{0.002} & \textit{0.014} & \textit{0.006} & \textit{0.008} & \textit{13.973} & \textit{0.002} & \textit{0.008} & \textit{0.020} & \textit{0.014} \\ 
    \multirow{4}{*}{1.4B} & FT & \textbf{6.439} & 0.020 & 0.440 & 0.010 & 0.020 & 11.922 & 0.014 & 0.008 & 0.022 & 0.020 \\
    & Goldfish & 6.465 & 0.016 & 0.412 & 0.010 & 0.020 & \textbf{11.903} & 0.014 & 0.008 & 0.024 & 0.018 \\
    & DPSGD & 7.751 & \textbf{0.004} & \textbf{0.016} & {0.010} & \textbf{0.004} & 13.286 & \textbf{0.002} & \textbf{0.004} & \textbf{0.018} & \textbf{0.014} \\
    & \methodname & 6.553 & 0.008 & 0.030 & \textbf{0.006} & 0.006 & 12.670 & 0.004 & 0.020 & \textbf{0.018} & 0.016 \\ \midrule
    
    \multirow{4}{*}{Llama-2} & \textit{Base} & \textit{7.014} & \textit{0.006} & \textit{0.016} & \textit{0.016} & \textit{0.010} & \textit{9.364} & \textit{0.006} & \textit{0.006} & \textit{0.024} & \textit{0.006} \\ 
    \multirow{4}{*}{7B} & FT & \textbf{3.830} & 0.028 & 0.170 & 0.030 & 0.028 & \textbf{6.261} & 0.002 & 0.018 & 0.002 & 0.002 \\
    & Goldfish & 3.839 & 0.028 & 0.198 & 0.028 & 0.028 & 6.280 & 0.002 & 0.018 & 0.002 & 0.006 \\
    & DPSGD & 4.490 & \textbf{0.006} & 0.014 & \textbf{0.020} & \textbf{0.010} & 6.777 & 0.008 & 0.026 & 0.016 & 0.010 \\
    & \methodname & 4.006 & 0.010 & \textbf{0.002} & 0.028 & 0.012 & 6.395 & \textbf{0.002} & \textbf{0.020} & \textbf{0.004} & \textbf{0.002} \\ 
  \end{tabular}}
  \caption{Overall Evaluation: Perplexity (PPL) and TPR at FPR of 1\% scores of the MIAs with different signals (Loss/Ref/Min-k/Zlib). For all metrics, the lower the value, the better the result.}
  \label{tab:tpr}
\end{table*}
Table~\ref{tab:tpr} provides the True Positive Rate (TPR) at low False Positive Rate (FPR) of the overall evaluation. Generally, compared to CC-news, Wikipedia poses a significant higher risk at low FPR. For example, the reference-based attack can achieve a score of 0.57~ on GPT2 if no protection. In general, Goldfish fails to mitigate the risk in this scenario, while both DPSGD and \methodname offer robust protection.

\subsection{Auxiliary dataset}
We investigate the size of the auxiliary dataset which is disjoint with the training data of the target model and the attack model. In this experiment, the other methods are trained with 3K samples. Figure~\ref{fig:aux_size} presents the language modeling performance while varying the auxiliary dataset's size. The result demonstrates that the better reference model, the better language modeling performance. It is worth noting that even with a very small number of samples, \methodname can still outperform DPSGD. Additionally, there is only a little benefit when increasing from 1000 to 3000, this indicates that the reference model is not needed to be perfect, as it just serves as a calibration factor. This phenomena is consistent with previous selective training works~\cite{lin2024not, 2022PrioritizedTraining}.
\begin{figure}
    \centering
    \includegraphics[width=0.8\linewidth]{figs/auxiliary_size.pdf}
    \caption{Language modeling performance while varying the auxiliary dataset's size. Note that the results of FT and Goldfish are significantly overlapping.}
    \label{fig:aux_size}
\end{figure}

\subsection{Training time}
We report the training time for full fine-tuning Pythia 1.4B. We manually increase the batch size that could fit into the GPU's physical memory. As a results, FT and Goldfish can run with a batch size of 48, while DPSGD and \methodname can reach the batch size of 32. We also implement gradient accumulation, so all the methods can have the same virtual batch size.

\begin{table}[!ht]
    \centering
    \begin{tabular}{c|c}
        \textbf{Training Time} & \textbf{\textbf{1 epoch}} (in minutes) \\ \hline
        {FT} & 2.10 \\ 
        {Goldfish} & 2.10 \\
        % {RelaxLoss} & 2.10 \\        
        {DPSGD} & 3.19 \\ 
        {DuoLearn} & 2.85 
    \end{tabular}
    \caption{Training time for one epoch of (full) Pythia 1.4B on a single H100 GPU}
    \label{tab:training-time}
\end{table}

Table~\ref{tab:training-time} presents the training time for one epoch. Goldfish has little to zero overhead compared to FT. DPSGD and \methodname have a slightly higher training time due to the additional computation of the privacy mechanism. In particular, DPSGD has the highest overhead due to the clipping and noise addition mechanisms. Meanwhile, \methodname requires an additional forward pass on the reference model to select the learning and unlearning tokens. \methodname is also feasible to work at scale that has been demonstrated in the pretraining settings of the previous work~\cite{lin2024not}.

\section{Limitations}
The main limitation of our work is the small-scale experiment setting due to the limited computing resources. However, we believe \methodname can be directly applied to large-scale pretraining without requiring any modifications, as done in previous selective pretraining work~\cite{lin2024not}. Another limitation is the reference model, which may be restrictive in highly sensitive or domain-limited settings~\cite{tramr2024position}. From a technical perspective, while we show that \methodname performs well across different datasets and architectures, there is room for improvement. The current approach selects a fixed number of tokens, which may not be optimal since selected tokens contribute unequally. Future work could explore adaptive selection or weighted tokens' contribution. At a high-level, compared to DPSGD, \methodname has not been supported by theoretical guarantees. Future work can investigate the convergence and overfitting analysis.


\end{document}


