\section{Related work}
\label{sec:RelatedWork}

While a thorough review of the by now rich literature on DOA estimation from quantized samples (____) is beyond the scope of this paper, let us describe some existing approaches. In ____, DOA estimation from one-bit samples is performed by one-bit covariance estimation via the arcsine law. Due to near-linearity of the arcsine function close to the origin, in ____ it is argued that for high SNRs the sample covariance matrix of the one-bit samples can be used without further adaption. The authors of ____ propose to first recover the unquantized measurements and then apply MUSIC. In ____, a maximum likelihood-based approach is examined.
On the one hand, the simplicity of the sign-quantizer is appealing from a hardware perspective. On the other hand, it comes with strong limitations like scaling invariance. A second, more recent line of works (____) mitigates these shortcomings by considering dithered sign-quantizers of the form \eqref{eq:QuantizedSnapShots} below. Finally, in ____ the authors of ____ extend their ideas from one-bit to multi-bit quantizers.

Dithering, i.e., adding well-designed noise to a signal before quantization, has a long history in signal processing ____. In particular, the last decade showed substantial progress in deriving rigorous non-asymptotic performance guarantees for signal reconstruction from one-bit measurements, see e.g. ____ and the references therein. Concrete examples encompass one-bit matrix completion (e.g., ____), reconstructing a signal in an unlimited sampling framework with one-bit quantization (e.g., ____), and one-bit quadratic sensing problems such as phase retrieval (e.g., ____). Most relevant to this work is that only recently the first non-asymptotic (and near-minimax optimal) guarantees for estimating covariance matrices from one-bit samples with uniform dither have been derived ____. In ____, the results of ____ have been generalized to the complex domain and applied to massive MIMO; in ____ a data-adaptive variant of the estimator in ____ has been developed. A recent work ____ modified the strategy of ____ to cover heavy-tailed distributions by using truncation before quantizing. The work ____ introduced the idea of using triangular rather than uniform dithering in the context of covariance estimation. 

Almost all previously mentioned studies on DOA estimation from coarsely quantized samples are of empirical nature, proposing algorithmic approaches to the problem and evaluating their performance in simulations. The only exceptions are ____, ____, and ____. The authors of ____ focus on one-bit DOA estimation via Sparse Linear Arrays. They provide conditions under which the identifiability of source DOAs from unquantized data is equivalent to the one of one-bit data. Furthermore, they provide a Cramér-Rao bound analysis and a MUSIC-based reconstruction approach with asymptotic error guarantees. The authors of ____ build upon the idea of ____ in which DOA estimation from undithered one-bit samples is performed by Learned Iterative Soft-Thresholding (LISTA), i.e., unfolding and training ISTA as a network. By adding a uniform dither to the quantization model and relying on the theoretical analysis of the corresponding quantized covariance estimator in ____, they can derive performance guarantees for the one-bit LISTA approach from the results in ____. The authors of ____ used ESPRIT for the quantized single-snapshot DOA problem, where information is only collected at one time instance. This setting cannot be treated using statistical methods. Instead, they exploited analytic properties of the Fourier transform by using a two-bit beta-quantization method, which is very different from dithered quantization.