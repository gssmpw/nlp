\section{Related work}
\label{sec:Related_work}

Several studies available in the literature have analyzed the power consumption of cloud infrastructures and data centers, but it is not clear how this knowledge can be translated to the telco world and especially to edge 5GC deployments running on COTS hardware.
One of the most exhaustive and up-to-date survey on the topic is \cite{Depasquale2023}, where authors summarize the dynamics and the results of recent research efforts in power consumption measurement and power models of virtual entities in the telco cloud sector.
Study in \cite{Jiang2019} presents a comprehensive examination of the power and energy usage of four of the most used hypervisors and one container engine by testing them on six different hardware platforms, including a variety of rack server structures, one desktop server, and one laptop. Power measurements are gathered during different levels of computation-intensive, memory-intensive, and mixed Web server-database workloads. The findings highlight the different characteristics of each hypervisor and their better suitability for specific workloads or platform with none significantly surpassing the others in performance.
A similar comparison is presented in \cite{Morabito2015} with the aim of characterizing the power consumption of different virtualization technologies in the idle state and under CPU, memory, and networking intensive workloads.
Based on the experimental measurements, the authors conclude that, for the analyzed solutions, significant differences are present only during the networking stress test where container solutions perform better than hypervisors.
Further confirmation of these results is presented in \cite{Shea2014}, where hardware virtualization and paravirtualization solutions proved to be up to 40\% more energy demanding than a standard bare-metal machine performing the same networking tasks. On the other hand, the performance and power consumption of container
virtualization systems is comparable to the bare-metal.
While these works present very useful insight, they do not consider the specific needs and scenarios typical of a 5GC deployment and of an edge environment.
A similar analysis but more targeted towards 5GC deployments is presented in \cite{Behravesh2019}, in which the authors benchmark instances of an Apache HTTP server and of a Redis server running on VM, containers and unikernels. The work done in \cite{Behravesh2019} was expanded in \cite{Aggarwal2020} with the addition of Kata containers, a virtualization solution that combines benefits of both containers and VMs.
The common conclusion is that the overhead of the virtualization can be quite significant in terms of resource usage, nevertheless each virtualization environment has some advantages making it the preferred solution in some specific use-cases.
Additional considerations on this subject are presented in \cite{Adoga2022} which shows the performance gains from deploying VNFs on heterogeneous frameworks.
Only recently have studies presented performance analysis of actual 5GC instances. A comparison of CPU usage, latency, connection time and other metrics between open-source 5GC solutions are presented in \cite{Reddy2023} and \cite{Lando2023}.

Our previous study \cite{Bellin2023} covers some of the early results and a preliminary assessment of the realistic power consumption of a 5GC deployed in a network edge environment. The main limitation of the study is that the proposed power monitoring system only considers hardware-based power meters and was validated on a single 5GC implementation and a single connected UE.

\subsection{Lessons learned from the literature and motivation of the work}
\label{subsec:Motivation}

Based on the aforementioned literature, we infer that virtualization typically adversely affects performance and energy consumption. However, certain frameworks may offer advantages in terms of agility, isolation, and ease of management for particular use cases and workloads.
The increasing amount of research that has been carried out in recent years prove the relevance of the topic, not only in light of the ever growing environmental concerns regarding the ICT sector but also the desire of mobile network operators to reduce their operational expenditure.
Nonetheless, power consumption monitoring and optimization in mobile networks is still very much an open topic for research with some significant challenges that still have to be tackled.

In addition to previous efforts, our study includes the following novelties:
\textcolor{black}{
\begin{itemize}
    \item Design and implementation of an experimental testbed for the simulation of an edge computing environment using COTS hardware and open-source software.
    \item Acquisition of power measurements using real 5GC instances instead of benchmarking softwares.
    \item Analysis of the usage of software-based and hardware-based power meters.
    \item Providing insights on the consumption of the single VNFs composing the 5GC based on actual testbed measurements.
\end{itemize}
}