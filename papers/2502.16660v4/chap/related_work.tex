\section{Related Work}

\textbf{Biological Scientific Question Answering} Previous studies have explored the potential of language models in the biological scientific domain \citep{lu2022learn, vilares2019head, jin2021disease, pal2022medmcqa}, such as biological scientific reading comprehension \citep{welbl2018constructing, jin2019pubmedqa} and question-answering \citep{krithara2023bioasq}. A few studies have examined language models' ability to complete biological pathways \citep{li2023chatpathway, park2023comparative, azam2024comprehensive}. Different from previous tasks, this work introduces biological pathway reasoning in realistic research scenarios. See Appendix \ref{appendix related work} for a more detailed comparison.

\textbf{Graph-augmented Language Model} Several studies have explored augmenting LLMs with graph data. In particular, some works enhance LLMs by encoding graph data as text \citep{ye2023natural, wang2024can, fatemi2023talk}, or tuning LLMs specifically for graph-based tasks \citep{liu2023one, tang2024graphgpt, he2024g, zhao2023gimlet, he2024unigraph}. 
% By augmenting LLMs with graph data, they have been applied to knowledge-based QA \citep{sun2023think, he2024g, li2023chain, jin2024graph, cheng2024call, edge2024local}, and to graph-oriented tasks like graph property prediction \citep{wang2024can, he2023harnessing}. A few other studies leverage graph structures during LLM reasoning to tackle complex tasks \citep{jiang2023structgpt, besta2024graph}. Most large graph databases rely on retrieval mechanisms \citep{he2024g, li2023chain} to access data, while some studies use LLMs as interactive agents for database navigation \citep{sun2023think, jin2024graph, li2024graphreader}. In this work, we present a more efficient agent-based approach using subgraph navigation combined with reasoning for improved pathway database exploration.
Augmented LLMs have been applied to knowledge-based QA \citep{sun2023think, he2024g, li2023chain, jin2024graph, cheng2024call, edge2024local} and graph tasks like property prediction \citep{wang2024can, he2023harnessing}. Some studies leverage graph structures for complex reasoning tasks \citep{jiang2023structgpt, besta2024graph}. Most large graph databases use retrieval mechanisms \citep{he2024g, li2023chain}, while others employ LLMs as interactive agents for navigation \citep{sun2023think, jin2024graph, li2024graphreader}. This work introduces a more efficient agent-based approach using subgraph navigation and reasoning to improve pathway database exploration.

% Unlike tasks in previous works, this study addresses whether reasoning in biological systems can be enhanced by pathway graphs, which act as a \textit{structured blueprint} for reasoning about the system’s states. It is not sufficient to simply identify the correct paths in the pathway graph to find the answer. Instead, it is necessary to perform deductive reasoning about the events that occur when the system is intervened upon under specific conditions and to predict the resulting states and mechanisms of the intervened system.

% For large graph databases, most works enable LLMs to access graph data through retrieval mechanisms \citep{he2024g, li2023chain}, while a few studies have explored using LLMs as interactive agents \citep{yao2023react, shinn2023reflexion, zhao2024empowering} to navigate and explore vast graph databases \citep{sun2023think, jin2024graph}. In this work, we introduce an agent-based interactive graph exploration approach using subgraph navigation-based browsing, which is more efficient and offers enhanced navigation capabilities for the pathway database.

% focusing on knowledge acquisition rather than multi-step reasoning. Our research shows that these methods face limitations when applied to biological system reasoning. For such tasks, it is essential for LLMs to not only understand biological pathways but also to reason effectively within those systems.

% \subsection{Retriever Agent}

% Please add the following required packages to your document preamble:
% \usepackage[table,xcdraw]{xcolor}
% Beamer presentation requires \usepackage{colortbl} instead of \usepackage[table,xcdraw]{xcolor}
% Please add the following required packages to your document preamble:
% \usepackage[table,xcdraw]{xcolor}
% Beamer presentation requires \usepackage{colortbl} instead of \usepackage[table,xcdraw]{xcolor}
% Please add the following required packages to your document preamble:
% \usepackage[table,xcdraw]{xcolor}
% Beamer presentation requires \usepackage{colortbl} instead of \usepackage[table,xcdraw]{xcolor}
