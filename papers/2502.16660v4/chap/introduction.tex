\section{Introduction}
% Scientific resaerch represents the pinnacle of intelligence, characterized by a rigorous methodology that includes building hypothesis, connecting new problems with prior knowledge, the design of experiments and the prediction of outcomes. One of the most demanding facets of this process is the capacity for abductive reasoning, which involves formulating plausible explanations for denovo problems based on incomplete knowledge~\citep{perez1983model, paul1993approaches}. This type of reasoning is critical in developing initial theories that can later be tested through more definitive empirical methods. 
Large Language Models (LLMs) have recently demonstrated remarkable performance across scientific domains, including mathematics~\citep{yu2023metamath}, chemistry~\citep{liu2023multi, zhu2022torchdrug}, biology~\citep{hayes2024simulating, madani2020progen, ma2023retrieved}, and materials science~\citep{zheng2023shaping, park2024multi}. In biology, LLMs have shown promise in addressing complex tasks such as protein design~\citep{valentini2023promises, hosseini2024text2protein}, drug discovery~\citep{m2024augmenting, liu2023chatgpt}, clinical trial analysis~\citep{singhal2023large, jin2023matching}, and experiment design~\citep{ai4science2023impact, roohani2024biodiscoveryagent}.


While LLMs have been explored in various biological applications, their emerging capability for multi-step reasoning—extensively studied in fields like mathematics and programming~\citep{wei2022chain, wang2022self, kojima2022large}—remains underexplored in the context of complex, real-world biological problems. In particular, little research has focused on how LLMs can understand and reason through the intricate, multi-step processes inherent to complex biological systems.



% Although LLMs are increasingly capable of addressing more complex, real-world problems within the biological sciences, their fundamental understanding, reasoning, and metacognitive abilities  toward these scenarios—specifically in comprehending and reasoning through the intricate, multi-step processes involved in biological systems—have yet to be thoroughly explored.

% \textcolor{blue}{Change to: While many previous works focus on the sequential language form reasoning for math and code, the reasoning towards complex structured real systems in science study, such as pathways, is not well addressed.}

% Most notably, LLMs possess strong logical thinking ability and show emergent ability to perform  abductive reasoning~\citep{bhagavatula2019abductive, he2024causejudger} with careful designed instructions and step-by-step generative reasoning~\citep{wei2022chain}. Recently, the advancements of LLM agents~\citep{wang2023voyager, zhao2024empowering, ma2024agentboard} further strengthen abductive reasoning abilities of LLMs, as agents ground reasoning with obtained real-world knowledge. This paradigm closely reflects the reasoning process for scientific research and has proved effective in various scientific tasks~\citep{trinh2024solving, baek2024researchagent, bran2023chemcrow}. In this work, we aim at measuring the scientific abductive ability of LLM as well as different LLM reasoning algorithms. We focus mainly on one challenging reasoning task in the biological domain: Pathway Reasoning.





 % moving beyond basic applications like classification and regression

% The rich knowledge of LLMs enables them to perform various domain-specific tasks. The intuition behind is the utilization of reasoning and planning capacity in LLMs, which addresses complex tasks by decomposing them into basic steps that can be solved via existing knowledge. 
% % Chain-of-Thought (CoT) \citep{wei2022chain,wang2022self,kojima2022large} like reasoning has become the mainstream basic method for many LLM applications. 
% Along this research line, several techniques have been proposed to further augment the reasoning capability of LLMs, including Chain-of-Thought (CoT) \citep{wei2022chain,wang2022self,kojima2022large}.
% CoT decomposes the task into basic steps and accomplishes them sequentially, enabling LLMs to deal with tasks with compositionally increasing complexity. 
% However, the exploration of such augmented LLMs' reasoning ability to solve challenging biological tasks has been lagging behind.

Biological reasoning is challenging due to the intricate, long causal chains naturally and widely present in biological systems. These systems are composed of complex networks called pathways, which involves genes, enzymes, substrates, and signaling molecules. Intervention in a single component—such as mutations or infections—can trigger multi-step cascades affecting other components within the organism. Pathway reasoning is essential for explaining phenomena, forming hypotheses, designing experiments, and interpreting results. For example, blocking muscarinic M3 receptors in taste cells reduces calcium mobilization in Type II cells, weakens CALMH1-mediated ATP release, and diminishes taste responses in sensory fibers (Figure \ref{mainfig}). Such insights aid in toxicity analysis, experimental design, and treatments for taste disorders.

% Biological reasoning is inherently persistent and especially challenging as they often involves intricate, long causal chains. Biological systems are composed of complex networks called pathways, which function as interconnected units involving various components, such as genes, enzymes, substrates, and signaling molecules. Intervention in a single component—such as mutations, or pathogen infections—can influence other components within the organism via intricate, multi-step intermediate processes that constitute its complexity.

% These components interact in a highly coordinated manner, enabling the integration of multiple signals and precise regulation of system responses. 

% As a result, intervention in a single component of a pathway—such as mutations, inhibitions, or pathogen infections—can influence other components within the organism via intricate, multi-step intermediate processes. 
% This includes understanding how a specific element within the system impacts the behavior of other components and overall system function, as well as how external interventions, such as mutations, experimental treatments, or drugs, influence the components and function of the system. 

% A wide range of biological phenomena can be explained and predicted by biological pathways reasoning, essential in formulating hypotheses, designing experiments, and predicting and interpreting results in biological research. For example, blocking muscarinic M3 receptors in taste cells triggers a sequence of events, including reduced calcium mobilization in Type II taste cells, a weakened role of CALMH1 in ATP release, and diminished taste-evoked responses in taste sensory fibers, as shown in Figure \ref{mainfig}. These insights can be useful for toxicity analysis, designing experiment groups with induced taste suppression, and developing treatments for decreased sense of taste.

% \chang{A pathway example could be mentioned here. I think what is pathway and why the reasoning on pathway is unique could be emphasized.}
% To enhance the application of large language models (LLMs) in biological research, it is crucial to enable these models to comprehend and predict biological phenomena through reasoning from the perspective of biological systems. 
% \textbf{In this study, we explore this question: Can LLMs understand and perform reasoning in biological pathways?}
% The study of LLMs' understanding of biological systems is currently under-researched.
% spans multiple biological domains and is categorized along three dimensions, focusing on predicting the effects and mechanisms of natural and synthetic interventions—such as mutations, infections, or treatments—on various downstream targets under different conditions. This requires an understanding of complex intermediate processes and interactions within the system.
% There could be more than one treatment introduced to the biological system, so the interaction is also necessary to consider.
% Despite the importance of reasoning about biological pathways, the study of LLMs' capacity to reason about such pathways remains underexplored. 
% Given the complexity of these biological systems and the importance of understanding pathway interactions, the application of LLMs to analyze and predict their behavior presents both opportunities and challenges. While LLMs have shown promise in various biological applications, the exploration of their reasoning abilities in the context of such complex biological systems remains underdeveloped. 
% In this paper, we address this gap by introducing a benchmark, \benchname, to evaluate the reasoning capabilities of LLMs in the context of biological pathways. \benchname compiles biological pathway phenomena from the literature and generates questions with corresponding answers based on these phenomena. These questions span multiple biological domains and are categorized along three dimensions, focusing on predicting the effects and mechanisms of natural and synthetic interventions—such as mutations, infections, or treatments—on various downstream targets under different conditions via complex intermediate pathway processes. The targets may include individual components, interactions between components, their roles in biological processes, or larger-scale biological functions. The goal of \benchname is to assess the ability of language models to comprehend and reason about realistic biological pathway phenomena, which is crucial for advancing biological research.

% While LLMs have shown promise in various biological applications, their reasoning abilities in complex biological contexts remain underexplored. 

% Given the complexity of biological systems and the importance of understanding pathway interactions, the application of LLMs to analyze and predict their behavior presents both opportunities and challenges. 

% In this paper, we first introduce \benchname, a benchmark that serves as a crucial starting point for assessing LLMs' ability to comprehend and reason about realistic biological pathway phenomena. \benchname compiles biological pathway phenomena from literature and generating corresponding questions and answers. These questions span multiple biological domains, focusing on predicting the effects and mechanisms of natural and synthetic interventions on various targets under different conditions through complex intermediate processes. Targets may include individual components, component interactions, their roles in biological processes, or larger-scale functions.

% including functional understanding (effects of a component), dynamic changes (organism responses to component changes), regulation (achieving downstream changes by altering a component), and intervention (hypothetical impacts of modifying a component).

In this work, we evaluate the understanding and reasoning abilities of LLMs for biology tasks through the lens of biological pathways. We explore their potential applications in key pathway research areas, including functional understanding, dynamic changes, regulation, and intervention. To support these investigations, we introduce a pathway benchmark, \benchname, which comprises 5.1K high-quality, complex biological pathway problems derived directly from real research literature, such as PubMed~\citep{lu2011pubmed}. Problems are meticulously curated and checked by experts to cover biological pathway research contexts, including natural dynamic changes, disturbances and interventions, additional intervention conditions, and multi-scale research targets such as single factors, interaction processes, and macro-level functions.

Based on \benchname, we compare various methods using LLMs for pathway reasoning, including chain-of-thought (CoT) and graph-augmented reasoning approaches ~\citep{li2023chain, sun2023think, he2024g}. The results demonstrate that LLMs struggle with pathway reasoning, particularly when interventions and perturbations are introduced into the system. This challenge persists across all LLMs, from LLaMA 8B to GPT-4, and affects both CoT and graph-augmented reasoning approaches. Our further analysis reveals that the difficulty in reasoning increases due to factors such as long reasoning steps, faulty steps, and omissions.

% This highlights the limitations of LLMs' causal reasoning abilities in the context of biological pathways. 


% Graph-augmented reasoning presents an intuitive solution for improving pathway reasoning, as it encourages LLMs to become more globally structure-aware. However, we find that most previous graph-based approaches, which have been effective for other reasoning tasks, fail to perform well in pathway reasoning.

% This approach strikes a balance between expanding exploration across different reasoning chains and guiding the LLM through the complex components of a pathway toward the most relevant subjects. Interestingly, this approach resonates with the reasoning processes employed by scientific experts when analyzing new biological pathways~\citep{chindelevitch2012causal}. Experts often follow a diffusion-like procedure, systematically examining upstream linkages within a pathway and evaluating their potential effects on downstream components.

To address these challenges, we propose a novel approach called \modelname, an LLM agent designed to emulate the way scientists reason using biological pathways. It interactively explores biological pathways in the form of subgraphs during inference. This interactive process allows for a mutually reinforcing relationship between inference and pathway browsing. By utilizing an efficient global-local subgraph navigation method, \modelname enhances the ability to leverage pathway databases for reasoning. By simulating this interactive reasoning process, \modelname provides a more robust and scientifically grounded approach to pathway reasoning, addressing challenges such as interventions and perturbations, as well as long reasoning chains and errors.

% The core idea of PathSeeker is similar to how scientists study biological pathways: using an LLM agent to interactively browse biological pathways in the form of subgraphs during inference. This approach allows for mutual enhancement between inference and pathway browsing. By leveraging an efficient subgraph navigation method, it strengthens the ability to utilize pathway databases, thereby promoting the ability of large language models to leverage biological pathways in inference.




% Based on the \benchname benchmark, we compare different methods using LLMs for pathway reasoning, such as chain-of-thought (CoT) and graph-augmented reasoning \citep{li2023chain, sun2023think, he2024g}. The results show that LLMs struggle in pathway understanding and reasoning, especially when perturbations are introduced into the system. This reveals that LLMs' causal reasoning abilities for biological pathways are limited. Graph-augmented reasoning could be an intuitive solution to improving the robustness to interventional reasoning, as it guides LLM to be more structured aware and more robust to longer reasoning chains. However, we find that most previous graph-based approaches proven effective on other reasoning tasks fails under pathway reasoning. To address these challenges, we then propose a novel approach, \modelname, an LLM agent that interactively reasons through subgraph-based navigation. This method can effectively leverage pathway graph information as blueprints in reasoning, achieving the best performance on the \benchname, especially in the case of interventions.




% In this work, we specifically study the understanding and reasoning abilities of large language models for biology tasks through the lens of biological pathways. we explore the potential and applications of large language models in core pathway research questions, such as functional understanding, dynamic changes, regulation, and intervention. we first introduce \benchname, a benchmark that serves as a crucial starting point for assessing LLMs' ability to comprehend and reason about biological pathway. \benchname constructs 5.1K high-quality complex biological pathway problems in real scientific research scenarios, including various biological pathways research contexts such as natural dynamic changes, disturbances and interventions, additional intervention conditions, and multi-scale research targets such as single factors, interactions processes, and macro functions.

% While \benchname alone cannot fully resolve the challenge, it represents an important step towards addressing this gap in our understanding.




% Based on the \benchname benchmark, this work explores the potential applications of large language models in biological pathway research. We compare different methods, such as chain-of-thought (CoT) and graph-augmented reasoning \citep{li2023chain, sun2023think, he2024g}. The results show that LLMs struggle in pathway understanding and reasoning, especially when perturbations are introduced into the system. This reveals that LLMs' causal reasoning abilities for biological pathways are limited. To address these challenges, we then propose a novel approach, \modelname, an LLM agent that interactively reasons through subgraph-based navigation. This method can effectively leverage pathway graph information as blueprints in reasoning, achieving the best performance on the \benchname, especially in the case of interventions.

% We propose a novel proxy method for graph reasoning. This method allows the model to autonomously and efficiently browse pathway graphs, 

% We conducted extensive evaluations of LLMs using the \benchname benchmark, incorporating reasoning methods such as Chain-of-Thought (CoT) and pathway graph-augmented approaches \citep{li2023chain, sun2023think, he2024g}. The results show that while LLMs demonstrate an understanding of mechanisms within natural organisms, they struggle to predict phenomena and grasp mechanisms when perturbations are introduced into the system—such as during interventions or when organisms are in altered conditions. This reveals that LLMs' causal reasoning abilities for biological pathways are limited. To address these challenges, we then propose a novel approach, \modelname, an LLM agent that interactively reasons through subgraph-based navigation
% while exploring the pathway graph. This method enhances LLMs' performance in complex biological reasoning tasks by effectively leveraging pathway graph information as blueprints in reasoning, especially in the case of interventions.
% In summary, our contributions are as follows:
% \begin{itemize}
% % \item We introduce \benchname, a comprehensive benchmark designed to assess LLMs' ability to understand and reason about biological pathways. \benchname focuses on evaluating the models' capacity to predict the effects and elucidate the mechanisms of both natural and synthetic interventions—such as mutations and infections—on various downstream targets under diverse conditions through complex intermediate pathway processes. The benchmark spans multiple biological domains and is structured along three dimensions: interventions, conditions, and target types.

% \item We introduce \benchname, containing 5.1K high-quality complex biological pathway problems in real scientific research scenarios, including various biological pathways research contexts such as natural dynamic changes, disturbances and interventions, additional intervention conditions, and multi-scale research targets such as single factors, interactions processes, and macro functions.

% % \item We conduct extensive evaluations of LLMs using \benchname, incorporating advanced reasoning methods such as CoT and pathway graph-augmented approaches. Our results reveal that while LLMs demonstrate proficiency in understanding mechanisms within natural organisms, they encounter significant challenges when predicting phenomena and comprehending mechanisms in perturbed systems. These findings highlight critical limitations in LLMs' reasoning capabilities within the domain of biological pathways.

% \item We explore the potential applications of LLMs in biological pathway research, incorporating reasoning methods such as CoT and graph-augmented reasoning. The results show that LLMs encounter challenges in pathway understanding and reasoning, especially in perturbed systems. These findings highlight critical limitations in LLMs' reasoning capabilities within the domain of biological pathways.

% \item We propose \modelname, a novel LLM agent approach that employs interactive, subgraph-based exploration to navigate pathway databases during reasoning. This method enhances LLMs' reasoning in biological pathways by leveraging pathway graphs as structured blueprints, especially for the case with interventions, potentially bridging the gap between LLMs' current capabilities and the complexities of biological systems.

% \end{itemize}

% We evaluate various reasoning methods based on large language models (LLMs), including Chain-of-Thought (CoT) and graph-augmented approaches. The results demonstrate that while LLMs can handle straightforward reasoning tasks in biology systems, such as natural source tasks and tasks without conditions, they continue to struggle with more complex challenges, such as tasks involving unnatural conditions or conditional settings. Surprisingly, the main difficulties arise from reasoning within complex network structures, particularly in planning reasoning trajectories and deducing events within system networks. A lack of knowledge also contributes to these failures. To address these issues, we propose an agent-based method, \modelname, designed to enhance LLMs' comprehension of biological systems. \modelname is capable of interactively exploring biological pathways, improving both the reasoning capabilities and task-specific knowledge. Experimental results show that \modelname more effectively addresses the challenges of reasoning in biological systems.


\begin{figure*}[!t]
    \centering    
    \renewcommand{\thesubfigure}{} % Hide subfigure labels
    \subfigure[]{\includegraphics[width=0.9\linewidth]{fig/biopathway.pdf}}\\
    \vspace{-2mm}
    \caption{Illustration of \benchname task and reasoning method with or without additional biological pathway graph data guidance. The task of \benchname focuses on reasoning about the effects and mechanisms of natural components or synthetic interventions on various downstream targets under different conditions through complex intermediate pathway processes.}
    \label{mainfig}
    \vspace{-2mm} 
\end{figure*}